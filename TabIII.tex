\documentclass[10pt,landscape]{article}


\usepackage[margin=1cm,a3paper]{geometry}

\def\pgfsysdriver{pgfsys-pdftex.def}
\usepackage{pgfplots}
\usepgfplotslibrary{external} 
\usetikzlibrary{pgfplots.external}
\usetikzlibrary{calc}
\usetikzlibrary{shapes}
\usetikzlibrary{patterns}
\usetikzlibrary{plotmarks}
\usepgflibrary{arrows}
\usepgfplotslibrary{polar}
\tikzexternalize

\usepackage{times,latexsym,amssymb}
\usepackage{eulervm} % math font
\pagestyle{empty}

\usepackage{amsmath} 
\usepackage{nicefrac} 
%
% This is used because from 1.11 onwards the use of \pgfdeclareplotmark will
% introduce a shift of the stellar markers.
%
\pgfplotsset{compat=1.10}
%
% Use to increase spacing for constellation labels
\usepackage[letterspace=400]{microtype}
%
\usepackage{pagecolor}
\usepackage{natbib}
%


%\pagecolor{black}
\pagecolor{white}

%%%%%%%%%%%%%%%%%%%%%%%%%%%%%%%%%%%%%%%%%%%%%%%%%%%%%%%%%%%%%%%%%%%%%%
\begin{document}

\input{./input/styles_eq.tex}
\center

\tikzsetnextfilename{TabulaIII}

%
% Coordinate limit of tab to plot
\pgfplotsset{xmin=0,xmax=110,ymin=-40,ymax=40}
\begin{tikzpicture}
%\
% Empty plot to establish anchor points
\begin{axis}[name=base,axis lines=none]

%\node[stars] at (axis cs:{68.980},{16.509}) {\tikz\pgfuseplotmark{GAL};};
%\draw[color=red] (axis cs:{68.980},{16.509}) circle (2pt) ;
\end{axis}

% The Milky Way first, as it should not obstruct coordinate systems 


\begin{axis}[name=milkyWay,axis lines=none]

\addplot[MW1] table[x index=0,y index=1] {./MW/III_MWbase.dat}  -- cycle ;

\addplot[MW2] table[x index=0,y index=1] {./MW/III_ol2_100.dat}  -- cycle ;
\addplot[MW2] table[x index=0,y index=1] {./MW/III_ol2_16.dat}  -- cycle ;
\addplot[MW2] table[x index=0,y index=1] {./MW/III_ol2_19.dat}  -- cycle ;
\addplot[MW2] table[x index=0,y index=1] {./MW/III_ol2_21.dat}  -- cycle ;
\addplot[MW2] table[x index=0,y index=1] {./MW/III_ol2_30.dat}  -- cycle ;
\addplot[MW2] table[x index=0,y index=1] {./MW/III_ol2_32.dat}  -- cycle ;
\addplot[MW2] table[x index=0,y index=1] {./MW/III_ol2_40.dat}  -- cycle ;
\addplot[MW2] table[x index=0,y index=1] {./MW/III_ol2_50.dat}  -- cycle ;
\addplot[MW2] table[x index=0,y index=1] {./MW/III_ol2_55.dat}  -- cycle ;
\addplot[MW2] table[x index=0,y index=1] {./MW/III_ol2_59.dat}  -- cycle ;
\addplot[MW2] table[x index=0,y index=1] {./MW/III_ol2_66.dat}  -- cycle ;
\addplot[MW2] table[x index=0,y index=1] {./MW/III_ol2_69.dat}  -- cycle ;
\addplot[MW2] table[x index=0,y index=1] {./MW/III_ol2_6.dat}  -- cycle ;
\addplot[MW2] table[x index=0,y index=1] {./MW/III_ol2_76.dat}  -- cycle ;
\addplot[MW2] table[x index=0,y index=1] {./MW/III_ol2_82.dat}  -- cycle ;
\addplot[MW2] table[x index=0,y index=1] {./MW/III_ol2_93.dat}  -- cycle ;
\addplot[MW2] table[x index=0,y index=1] {./MW/III_ol2_98.dat}  -- cycle ;

\end{axis}


% The Galactic coordinate system
\input{./input/MW.tex}
% Ecliptic coordinate system
\input{./input/Ecliptics.tex}
% Equatorial coordinate system
%
% Some are commented out because they are surrounded by already plotted borders
% The x-coordinate is in fractional hours, so must be times 15
%

\begin{axis}[name=constellations,at=(base.center),anchor=center,axis lines=none]


\draw[Equator-full] (axis cs:0.00000E+00,1.00000E-01) -- (axis cs:1.00000E+00,1.00000E-01) -- (axis cs:1.00000E+00,-1.00000E-01) -- (axis cs:0.00000E+00,-1.00000E-01) -- cycle ; 
\draw[Equator-full] (axis cs:2.00000E+00,1.00000E-01) -- (axis cs:3.00000E+00,1.00000E-01) -- (axis cs:3.00000E+00,-1.00000E-01) -- (axis cs:2.00000E+00,-1.00000E-01) -- cycle ; 
\draw[Equator-full] (axis cs:4.00000E+00,1.00000E-01) -- (axis cs:5.00000E+00,1.00000E-01) -- (axis cs:5.00000E+00,-1.00000E-01) -- (axis cs:4.00000E+00,-1.00000E-01) -- cycle ; 
\draw[Equator-full] (axis cs:6.00000E+00,1.00000E-01) -- (axis cs:7.00000E+00,1.00000E-01) -- (axis cs:7.00000E+00,-1.00000E-01) -- (axis cs:6.00000E+00,-1.00000E-01) -- cycle ; 
\draw[Equator-full] (axis cs:8.00000E+00,1.00000E-01) -- (axis cs:9.00000E+00,1.00000E-01) -- (axis cs:9.00000E+00,-1.00000E-01) -- (axis cs:8.00000E+00,-1.00000E-01) -- cycle ; 
\draw[Equator-full] (axis cs:1.00000E+01,1.00000E-01) -- (axis cs:1.10000E+01,1.00000E-01) -- (axis cs:1.10000E+01,-1.00000E-01) -- (axis cs:1.00000E+01,-1.00000E-01) -- cycle ; 
\draw[Equator-full] (axis cs:1.20000E+01,1.00000E-01) -- (axis cs:1.30000E+01,1.00000E-01) -- (axis cs:1.30000E+01,-1.00000E-01) -- (axis cs:1.20000E+01,-1.00000E-01) -- cycle ; 
\draw[Equator-full] (axis cs:1.40000E+01,1.00000E-01) -- (axis cs:1.50000E+01,1.00000E-01) -- (axis cs:1.50000E+01,-1.00000E-01) -- (axis cs:1.40000E+01,-1.00000E-01) -- cycle ; 
\draw[Equator-full] (axis cs:1.60000E+01,1.00000E-01) -- (axis cs:1.70000E+01,1.00000E-01) -- (axis cs:1.70000E+01,-1.00000E-01) -- (axis cs:1.60000E+01,-1.00000E-01) -- cycle ; 
\draw[Equator-full] (axis cs:1.80000E+01,1.00000E-01) -- (axis cs:1.90000E+01,1.00000E-01) -- (axis cs:1.90000E+01,-1.00000E-01) -- (axis cs:1.80000E+01,-1.00000E-01) -- cycle ; 
\draw[Equator-full] (axis cs:2.00000E+01,1.00000E-01) -- (axis cs:2.10000E+01,1.00000E-01) -- (axis cs:2.10000E+01,-1.00000E-01) -- (axis cs:2.00000E+01,-1.00000E-01) -- cycle ; 
\draw[Equator-full] (axis cs:2.20000E+01,1.00000E-01) -- (axis cs:2.30000E+01,1.00000E-01) -- (axis cs:2.30000E+01,-1.00000E-01) -- (axis cs:2.20000E+01,-1.00000E-01) -- cycle ; 
\draw[Equator-full] (axis cs:2.40000E+01,1.00000E-01) -- (axis cs:2.50000E+01,1.00000E-01) -- (axis cs:2.50000E+01,-1.00000E-01) -- (axis cs:2.40000E+01,-1.00000E-01) -- cycle ; 
\draw[Equator-full] (axis cs:2.60000E+01,1.00000E-01) -- (axis cs:2.70000E+01,1.00000E-01) -- (axis cs:2.70000E+01,-1.00000E-01) -- (axis cs:2.60000E+01,-1.00000E-01) -- cycle ; 
\draw[Equator-full] (axis cs:2.80000E+01,1.00000E-01) -- (axis cs:2.90000E+01,1.00000E-01) -- (axis cs:2.90000E+01,-1.00000E-01) -- (axis cs:2.80000E+01,-1.00000E-01) -- cycle ; 
\draw[Equator-full] (axis cs:3.00000E+01,1.00000E-01) -- (axis cs:3.10000E+01,1.00000E-01) -- (axis cs:3.10000E+01,-1.00000E-01) -- (axis cs:3.00000E+01,-1.00000E-01) -- cycle ; 
\draw[Equator-full] (axis cs:3.20000E+01,1.00000E-01) -- (axis cs:3.30000E+01,1.00000E-01) -- (axis cs:3.30000E+01,-1.00000E-01) -- (axis cs:3.20000E+01,-1.00000E-01) -- cycle ; 
\draw[Equator-full] (axis cs:3.40000E+01,1.00000E-01) -- (axis cs:3.50000E+01,1.00000E-01) -- (axis cs:3.50000E+01,-1.00000E-01) -- (axis cs:3.40000E+01,-1.00000E-01) -- cycle ; 
\draw[Equator-full] (axis cs:3.60000E+01,1.00000E-01) -- (axis cs:3.70000E+01,1.00000E-01) -- (axis cs:3.70000E+01,-1.00000E-01) -- (axis cs:3.60000E+01,-1.00000E-01) -- cycle ; 
\draw[Equator-full] (axis cs:3.80000E+01,1.00000E-01) -- (axis cs:3.90000E+01,1.00000E-01) -- (axis cs:3.90000E+01,-1.00000E-01) -- (axis cs:3.80000E+01,-1.00000E-01) -- cycle ; 
\draw[Equator-full] (axis cs:4.00000E+01,1.00000E-01) -- (axis cs:4.10000E+01,1.00000E-01) -- (axis cs:4.10000E+01,-1.00000E-01) -- (axis cs:4.00000E+01,-1.00000E-01) -- cycle ; 
\draw[Equator-full] (axis cs:4.20000E+01,1.00000E-01) -- (axis cs:4.30000E+01,1.00000E-01) -- (axis cs:4.30000E+01,-1.00000E-01) -- (axis cs:4.20000E+01,-1.00000E-01) -- cycle ; 
\draw[Equator-full] (axis cs:4.40000E+01,1.00000E-01) -- (axis cs:4.50000E+01,1.00000E-01) -- (axis cs:4.50000E+01,-1.00000E-01) -- (axis cs:4.40000E+01,-1.00000E-01) -- cycle ; 
\draw[Equator-full] (axis cs:4.60000E+01,1.00000E-01) -- (axis cs:4.70000E+01,1.00000E-01) -- (axis cs:4.70000E+01,-1.00000E-01) -- (axis cs:4.60000E+01,-1.00000E-01) -- cycle ; 
\draw[Equator-full] (axis cs:4.80000E+01,1.00000E-01) -- (axis cs:4.90000E+01,1.00000E-01) -- (axis cs:4.90000E+01,-1.00000E-01) -- (axis cs:4.80000E+01,-1.00000E-01) -- cycle ; 
\draw[Equator-full] (axis cs:5.00000E+01,1.00000E-01) -- (axis cs:5.10000E+01,1.00000E-01) -- (axis cs:5.10000E+01,-1.00000E-01) -- (axis cs:5.00000E+01,-1.00000E-01) -- cycle ; 
\draw[Equator-full] (axis cs:5.20000E+01,1.00000E-01) -- (axis cs:5.30000E+01,1.00000E-01) -- (axis cs:5.30000E+01,-1.00000E-01) -- (axis cs:5.20000E+01,-1.00000E-01) -- cycle ; 
\draw[Equator-full] (axis cs:5.40000E+01,1.00000E-01) -- (axis cs:5.50000E+01,1.00000E-01) -- (axis cs:5.50000E+01,-1.00000E-01) -- (axis cs:5.40000E+01,-1.00000E-01) -- cycle ; 
\draw[Equator-full] (axis cs:5.60000E+01,1.00000E-01) -- (axis cs:5.70000E+01,1.00000E-01) -- (axis cs:5.70000E+01,-1.00000E-01) -- (axis cs:5.60000E+01,-1.00000E-01) -- cycle ; 
\draw[Equator-full] (axis cs:5.80000E+01,1.00000E-01) -- (axis cs:5.90000E+01,1.00000E-01) -- (axis cs:5.90000E+01,-1.00000E-01) -- (axis cs:5.80000E+01,-1.00000E-01) -- cycle ; 
\draw[Equator-full] (axis cs:6.00000E+01,1.00000E-01) -- (axis cs:6.10000E+01,1.00000E-01) -- (axis cs:6.10000E+01,-1.00000E-01) -- (axis cs:6.00000E+01,-1.00000E-01) -- cycle ; 
\draw[Equator-full] (axis cs:6.20000E+01,1.00000E-01) -- (axis cs:6.30000E+01,1.00000E-01) -- (axis cs:6.30000E+01,-1.00000E-01) -- (axis cs:6.20000E+01,-1.00000E-01) -- cycle ; 
\draw[Equator-full] (axis cs:6.40000E+01,1.00000E-01) -- (axis cs:6.50000E+01,1.00000E-01) -- (axis cs:6.50000E+01,-1.00000E-01) -- (axis cs:6.40000E+01,-1.00000E-01) -- cycle ; 
\draw[Equator-full] (axis cs:6.60000E+01,1.00000E-01) -- (axis cs:6.70000E+01,1.00000E-01) -- (axis cs:6.70000E+01,-1.00000E-01) -- (axis cs:6.60000E+01,-1.00000E-01) -- cycle ; 
\draw[Equator-full] (axis cs:6.80000E+01,1.00000E-01) -- (axis cs:6.90000E+01,1.00000E-01) -- (axis cs:6.90000E+01,-1.00000E-01) -- (axis cs:6.80000E+01,-1.00000E-01) -- cycle ; 
\draw[Equator-full] (axis cs:7.00000E+01,1.00000E-01) -- (axis cs:7.10000E+01,1.00000E-01) -- (axis cs:7.10000E+01,-1.00000E-01) -- (axis cs:7.00000E+01,-1.00000E-01) -- cycle ; 
\draw[Equator-full] (axis cs:7.20000E+01,1.00000E-01) -- (axis cs:7.30000E+01,1.00000E-01) -- (axis cs:7.30000E+01,-1.00000E-01) -- (axis cs:7.20000E+01,-1.00000E-01) -- cycle ; 
\draw[Equator-full] (axis cs:7.40000E+01,1.00000E-01) -- (axis cs:7.50000E+01,1.00000E-01) -- (axis cs:7.50000E+01,-1.00000E-01) -- (axis cs:7.40000E+01,-1.00000E-01) -- cycle ; 
\draw[Equator-full] (axis cs:7.60000E+01,1.00000E-01) -- (axis cs:7.70000E+01,1.00000E-01) -- (axis cs:7.70000E+01,-1.00000E-01) -- (axis cs:7.60000E+01,-1.00000E-01) -- cycle ; 
\draw[Equator-full] (axis cs:7.80000E+01,1.00000E-01) -- (axis cs:7.90000E+01,1.00000E-01) -- (axis cs:7.90000E+01,-1.00000E-01) -- (axis cs:7.80000E+01,-1.00000E-01) -- cycle ; 
\draw[Equator-full] (axis cs:8.00000E+01,1.00000E-01) -- (axis cs:8.10000E+01,1.00000E-01) -- (axis cs:8.10000E+01,-1.00000E-01) -- (axis cs:8.00000E+01,-1.00000E-01) -- cycle ; 
\draw[Equator-full] (axis cs:8.20000E+01,1.00000E-01) -- (axis cs:8.30000E+01,1.00000E-01) -- (axis cs:8.30000E+01,-1.00000E-01) -- (axis cs:8.20000E+01,-1.00000E-01) -- cycle ; 
\draw[Equator-full] (axis cs:8.40000E+01,1.00000E-01) -- (axis cs:8.50000E+01,1.00000E-01) -- (axis cs:8.50000E+01,-1.00000E-01) -- (axis cs:8.40000E+01,-1.00000E-01) -- cycle ; 
\draw[Equator-full] (axis cs:8.60000E+01,1.00000E-01) -- (axis cs:8.70000E+01,1.00000E-01) -- (axis cs:8.70000E+01,-1.00000E-01) -- (axis cs:8.60000E+01,-1.00000E-01) -- cycle ; 
\draw[Equator-full] (axis cs:8.80000E+01,1.00000E-01) -- (axis cs:8.90000E+01,1.00000E-01) -- (axis cs:8.90000E+01,-1.00000E-01) -- (axis cs:8.80000E+01,-1.00000E-01) -- cycle ; 
\draw[Equator-full] (axis cs:9.00000E+01,1.00000E-01) -- (axis cs:9.10000E+01,1.00000E-01) -- (axis cs:9.10000E+01,-1.00000E-01) -- (axis cs:9.00000E+01,-1.00000E-01) -- cycle ; 
\draw[Equator-full] (axis cs:9.20000E+01,1.00000E-01) -- (axis cs:9.30000E+01,1.00000E-01) -- (axis cs:9.30000E+01,-1.00000E-01) -- (axis cs:9.20000E+01,-1.00000E-01) -- cycle ; 
\draw[Equator-full] (axis cs:9.40000E+01,1.00000E-01) -- (axis cs:9.50000E+01,1.00000E-01) -- (axis cs:9.50000E+01,-1.00000E-01) -- (axis cs:9.40000E+01,-1.00000E-01) -- cycle ; 
\draw[Equator-full] (axis cs:9.60000E+01,1.00000E-01) -- (axis cs:9.70000E+01,1.00000E-01) -- (axis cs:9.70000E+01,-1.00000E-01) -- (axis cs:9.60000E+01,-1.00000E-01) -- cycle ; 
\draw[Equator-full] (axis cs:9.80000E+01,1.00000E-01) -- (axis cs:9.90000E+01,1.00000E-01) -- (axis cs:9.90000E+01,-1.00000E-01) -- (axis cs:9.80000E+01,-1.00000E-01) -- cycle ; 
\draw[Equator-full] (axis cs:1.00000E+02,1.00000E-01) -- (axis cs:1.01000E+02,1.00000E-01) -- (axis cs:1.01000E+02,-1.00000E-01) -- (axis cs:1.00000E+02,-1.00000E-01) -- cycle ; 
\draw[Equator-full] (axis cs:1.02000E+02,1.00000E-01) -- (axis cs:1.03000E+02,1.00000E-01) -- (axis cs:1.03000E+02,-1.00000E-01) -- (axis cs:1.02000E+02,-1.00000E-01) -- cycle ; 
\draw[Equator-full] (axis cs:1.04000E+02,1.00000E-01) -- (axis cs:1.05000E+02,1.00000E-01) -- (axis cs:1.05000E+02,-1.00000E-01) -- (axis cs:1.04000E+02,-1.00000E-01) -- cycle ; 
\draw[Equator-full] (axis cs:1.06000E+02,1.00000E-01) -- (axis cs:1.07000E+02,1.00000E-01) -- (axis cs:1.07000E+02,-1.00000E-01) -- (axis cs:1.06000E+02,-1.00000E-01) -- cycle ; 
\draw[Equator-full] (axis cs:1.08000E+02,1.00000E-01) -- (axis cs:1.09000E+02,1.00000E-01) -- (axis cs:1.09000E+02,-1.00000E-01) -- (axis cs:1.08000E+02,-1.00000E-01) -- cycle ; 
\draw[Equator-full] (axis cs:1.10000E+02,1.00000E-01) -- (axis cs:1.11000E+02,1.00000E-01) -- (axis cs:1.11000E+02,-1.00000E-01) -- (axis cs:1.10000E+02,-1.00000E-01) -- cycle ; 
\draw[Equator-full] (axis cs:1.12000E+02,1.00000E-01) -- (axis cs:1.13000E+02,1.00000E-01) -- (axis cs:1.13000E+02,-1.00000E-01) -- (axis cs:1.12000E+02,-1.00000E-01) -- cycle ; 
\draw[Equator-full] (axis cs:1.14000E+02,1.00000E-01) -- (axis cs:1.15000E+02,1.00000E-01) -- (axis cs:1.15000E+02,-1.00000E-01) -- (axis cs:1.14000E+02,-1.00000E-01) -- cycle ; 
\draw[Equator-full] (axis cs:1.16000E+02,1.00000E-01) -- (axis cs:1.17000E+02,1.00000E-01) -- (axis cs:1.17000E+02,-1.00000E-01) -- (axis cs:1.16000E+02,-1.00000E-01) -- cycle ; 
\draw[Equator-full] (axis cs:1.18000E+02,1.00000E-01) -- (axis cs:1.19000E+02,1.00000E-01) -- (axis cs:1.19000E+02,-1.00000E-01) -- (axis cs:1.18000E+02,-1.00000E-01) -- cycle ; 
\draw[Equator-full] (axis cs:1.20000E+02,1.00000E-01) -- (axis cs:1.21000E+02,1.00000E-01) -- (axis cs:1.21000E+02,-1.00000E-01) -- (axis cs:1.20000E+02,-1.00000E-01) -- cycle ; 
\draw[Equator-full] (axis cs:1.22000E+02,1.00000E-01) -- (axis cs:1.23000E+02,1.00000E-01) -- (axis cs:1.23000E+02,-1.00000E-01) -- (axis cs:1.22000E+02,-1.00000E-01) -- cycle ; 
\draw[Equator-full] (axis cs:1.24000E+02,1.00000E-01) -- (axis cs:1.25000E+02,1.00000E-01) -- (axis cs:1.25000E+02,-1.00000E-01) -- (axis cs:1.24000E+02,-1.00000E-01) -- cycle ; 
\draw[Equator-full] (axis cs:1.26000E+02,1.00000E-01) -- (axis cs:1.27000E+02,1.00000E-01) -- (axis cs:1.27000E+02,-1.00000E-01) -- (axis cs:1.26000E+02,-1.00000E-01) -- cycle ; 
\draw[Equator-full] (axis cs:1.28000E+02,1.00000E-01) -- (axis cs:1.29000E+02,1.00000E-01) -- (axis cs:1.29000E+02,-1.00000E-01) -- (axis cs:1.28000E+02,-1.00000E-01) -- cycle ; 
\draw[Equator-full] (axis cs:1.30000E+02,1.00000E-01) -- (axis cs:1.31000E+02,1.00000E-01) -- (axis cs:1.31000E+02,-1.00000E-01) -- (axis cs:1.30000E+02,-1.00000E-01) -- cycle ; 
\draw[Equator-full] (axis cs:1.32000E+02,1.00000E-01) -- (axis cs:1.33000E+02,1.00000E-01) -- (axis cs:1.33000E+02,-1.00000E-01) -- (axis cs:1.32000E+02,-1.00000E-01) -- cycle ; 
\draw[Equator-full] (axis cs:1.34000E+02,1.00000E-01) -- (axis cs:1.35000E+02,1.00000E-01) -- (axis cs:1.35000E+02,-1.00000E-01) -- (axis cs:1.34000E+02,-1.00000E-01) -- cycle ; 
\draw[Equator-full] (axis cs:1.36000E+02,1.00000E-01) -- (axis cs:1.37000E+02,1.00000E-01) -- (axis cs:1.37000E+02,-1.00000E-01) -- (axis cs:1.36000E+02,-1.00000E-01) -- cycle ; 
\draw[Equator-full] (axis cs:1.38000E+02,1.00000E-01) -- (axis cs:1.39000E+02,1.00000E-01) -- (axis cs:1.39000E+02,-1.00000E-01) -- (axis cs:1.38000E+02,-1.00000E-01) -- cycle ; 
\draw[Equator-full] (axis cs:1.40000E+02,1.00000E-01) -- (axis cs:1.41000E+02,1.00000E-01) -- (axis cs:1.41000E+02,-1.00000E-01) -- (axis cs:1.40000E+02,-1.00000E-01) -- cycle ; 
\draw[Equator-full] (axis cs:1.42000E+02,1.00000E-01) -- (axis cs:1.43000E+02,1.00000E-01) -- (axis cs:1.43000E+02,-1.00000E-01) -- (axis cs:1.42000E+02,-1.00000E-01) -- cycle ; 
\draw[Equator-full] (axis cs:1.44000E+02,1.00000E-01) -- (axis cs:1.45000E+02,1.00000E-01) -- (axis cs:1.45000E+02,-1.00000E-01) -- (axis cs:1.44000E+02,-1.00000E-01) -- cycle ; 
\draw[Equator-full] (axis cs:1.46000E+02,1.00000E-01) -- (axis cs:1.47000E+02,1.00000E-01) -- (axis cs:1.47000E+02,-1.00000E-01) -- (axis cs:1.46000E+02,-1.00000E-01) -- cycle ; 
\draw[Equator-full] (axis cs:1.48000E+02,1.00000E-01) -- (axis cs:1.49000E+02,1.00000E-01) -- (axis cs:1.49000E+02,-1.00000E-01) -- (axis cs:1.48000E+02,-1.00000E-01) -- cycle ; 
\draw[Equator-full] (axis cs:1.50000E+02,1.00000E-01) -- (axis cs:1.51000E+02,1.00000E-01) -- (axis cs:1.51000E+02,-1.00000E-01) -- (axis cs:1.50000E+02,-1.00000E-01) -- cycle ; 
\draw[Equator-full] (axis cs:1.52000E+02,1.00000E-01) -- (axis cs:1.53000E+02,1.00000E-01) -- (axis cs:1.53000E+02,-1.00000E-01) -- (axis cs:1.52000E+02,-1.00000E-01) -- cycle ; 
\draw[Equator-full] (axis cs:1.54000E+02,1.00000E-01) -- (axis cs:1.55000E+02,1.00000E-01) -- (axis cs:1.55000E+02,-1.00000E-01) -- (axis cs:1.54000E+02,-1.00000E-01) -- cycle ; 
\draw[Equator-full] (axis cs:1.56000E+02,1.00000E-01) -- (axis cs:1.57000E+02,1.00000E-01) -- (axis cs:1.57000E+02,-1.00000E-01) -- (axis cs:1.56000E+02,-1.00000E-01) -- cycle ; 
\draw[Equator-full] (axis cs:1.58000E+02,1.00000E-01) -- (axis cs:1.59000E+02,1.00000E-01) -- (axis cs:1.59000E+02,-1.00000E-01) -- (axis cs:1.58000E+02,-1.00000E-01) -- cycle ; 
\draw[Equator-full] (axis cs:1.60000E+02,1.00000E-01) -- (axis cs:1.61000E+02,1.00000E-01) -- (axis cs:1.61000E+02,-1.00000E-01) -- (axis cs:1.60000E+02,-1.00000E-01) -- cycle ; 
\draw[Equator-full] (axis cs:1.62000E+02,1.00000E-01) -- (axis cs:1.63000E+02,1.00000E-01) -- (axis cs:1.63000E+02,-1.00000E-01) -- (axis cs:1.62000E+02,-1.00000E-01) -- cycle ; 
\draw[Equator-full] (axis cs:1.64000E+02,1.00000E-01) -- (axis cs:1.65000E+02,1.00000E-01) -- (axis cs:1.65000E+02,-1.00000E-01) -- (axis cs:1.64000E+02,-1.00000E-01) -- cycle ; 
\draw[Equator-full] (axis cs:1.66000E+02,1.00000E-01) -- (axis cs:1.67000E+02,1.00000E-01) -- (axis cs:1.67000E+02,-1.00000E-01) -- (axis cs:1.66000E+02,-1.00000E-01) -- cycle ; 
\draw[Equator-full] (axis cs:1.68000E+02,1.00000E-01) -- (axis cs:1.69000E+02,1.00000E-01) -- (axis cs:1.69000E+02,-1.00000E-01) -- (axis cs:1.68000E+02,-1.00000E-01) -- cycle ; 
\draw[Equator-full] (axis cs:1.70000E+02,1.00000E-01) -- (axis cs:1.71000E+02,1.00000E-01) -- (axis cs:1.71000E+02,-1.00000E-01) -- (axis cs:1.70000E+02,-1.00000E-01) -- cycle ; 
\draw[Equator-full] (axis cs:1.72000E+02,1.00000E-01) -- (axis cs:1.73000E+02,1.00000E-01) -- (axis cs:1.73000E+02,-1.00000E-01) -- (axis cs:1.72000E+02,-1.00000E-01) -- cycle ; 
\draw[Equator-full] (axis cs:1.74000E+02,1.00000E-01) -- (axis cs:1.75000E+02,1.00000E-01) -- (axis cs:1.75000E+02,-1.00000E-01) -- (axis cs:1.74000E+02,-1.00000E-01) -- cycle ; 
\draw[Equator-full] (axis cs:1.76000E+02,1.00000E-01) -- (axis cs:1.77000E+02,1.00000E-01) -- (axis cs:1.77000E+02,-1.00000E-01) -- (axis cs:1.76000E+02,-1.00000E-01) -- cycle ; 
\draw[Equator-full] (axis cs:1.78000E+02,1.00000E-01) -- (axis cs:1.79000E+02,1.00000E-01) -- (axis cs:1.79000E+02,-1.00000E-01) -- (axis cs:1.78000E+02,-1.00000E-01) -- cycle ; 
\draw[Equator-full] (axis cs:1.80000E+02,1.00000E-01) -- (axis cs:1.81000E+02,1.00000E-01) -- (axis cs:1.81000E+02,-1.00000E-01) -- (axis cs:1.80000E+02,-1.00000E-01) -- cycle ; 
\draw[Equator-full] (axis cs:1.82000E+02,1.00000E-01) -- (axis cs:1.83000E+02,1.00000E-01) -- (axis cs:1.83000E+02,-1.00000E-01) -- (axis cs:1.82000E+02,-1.00000E-01) -- cycle ; 
\draw[Equator-full] (axis cs:1.84000E+02,1.00000E-01) -- (axis cs:1.85000E+02,1.00000E-01) -- (axis cs:1.85000E+02,-1.00000E-01) -- (axis cs:1.84000E+02,-1.00000E-01) -- cycle ; 
\draw[Equator-full] (axis cs:1.86000E+02,1.00000E-01) -- (axis cs:1.87000E+02,1.00000E-01) -- (axis cs:1.87000E+02,-1.00000E-01) -- (axis cs:1.86000E+02,-1.00000E-01) -- cycle ; 
\draw[Equator-full] (axis cs:1.88000E+02,1.00000E-01) -- (axis cs:1.89000E+02,1.00000E-01) -- (axis cs:1.89000E+02,-1.00000E-01) -- (axis cs:1.88000E+02,-1.00000E-01) -- cycle ; 
\draw[Equator-full] (axis cs:1.90000E+02,1.00000E-01) -- (axis cs:1.91000E+02,1.00000E-01) -- (axis cs:1.91000E+02,-1.00000E-01) -- (axis cs:1.90000E+02,-1.00000E-01) -- cycle ; 
\draw[Equator-full] (axis cs:1.92000E+02,1.00000E-01) -- (axis cs:1.93000E+02,1.00000E-01) -- (axis cs:1.93000E+02,-1.00000E-01) -- (axis cs:1.92000E+02,-1.00000E-01) -- cycle ; 
\draw[Equator-full] (axis cs:1.94000E+02,1.00000E-01) -- (axis cs:1.95000E+02,1.00000E-01) -- (axis cs:1.95000E+02,-1.00000E-01) -- (axis cs:1.94000E+02,-1.00000E-01) -- cycle ; 
\draw[Equator-full] (axis cs:1.96000E+02,1.00000E-01) -- (axis cs:1.97000E+02,1.00000E-01) -- (axis cs:1.97000E+02,-1.00000E-01) -- (axis cs:1.96000E+02,-1.00000E-01) -- cycle ; 
\draw[Equator-full] (axis cs:1.98000E+02,1.00000E-01) -- (axis cs:1.99000E+02,1.00000E-01) -- (axis cs:1.99000E+02,-1.00000E-01) -- (axis cs:1.98000E+02,-1.00000E-01) -- cycle ; 
\draw[Equator-full] (axis cs:2.00000E+02,1.00000E-01) -- (axis cs:2.01000E+02,1.00000E-01) -- (axis cs:2.01000E+02,-1.00000E-01) -- (axis cs:2.00000E+02,-1.00000E-01) -- cycle ; 
\draw[Equator-full] (axis cs:2.02000E+02,1.00000E-01) -- (axis cs:2.03000E+02,1.00000E-01) -- (axis cs:2.03000E+02,-1.00000E-01) -- (axis cs:2.02000E+02,-1.00000E-01) -- cycle ; 
\draw[Equator-full] (axis cs:2.04000E+02,1.00000E-01) -- (axis cs:2.05000E+02,1.00000E-01) -- (axis cs:2.05000E+02,-1.00000E-01) -- (axis cs:2.04000E+02,-1.00000E-01) -- cycle ; 
\draw[Equator-full] (axis cs:2.06000E+02,1.00000E-01) -- (axis cs:2.07000E+02,1.00000E-01) -- (axis cs:2.07000E+02,-1.00000E-01) -- (axis cs:2.06000E+02,-1.00000E-01) -- cycle ; 
\draw[Equator-full] (axis cs:2.08000E+02,1.00000E-01) -- (axis cs:2.09000E+02,1.00000E-01) -- (axis cs:2.09000E+02,-1.00000E-01) -- (axis cs:2.08000E+02,-1.00000E-01) -- cycle ; 
\draw[Equator-full] (axis cs:2.10000E+02,1.00000E-01) -- (axis cs:2.11000E+02,1.00000E-01) -- (axis cs:2.11000E+02,-1.00000E-01) -- (axis cs:2.10000E+02,-1.00000E-01) -- cycle ; 
\draw[Equator-full] (axis cs:2.12000E+02,1.00000E-01) -- (axis cs:2.13000E+02,1.00000E-01) -- (axis cs:2.13000E+02,-1.00000E-01) -- (axis cs:2.12000E+02,-1.00000E-01) -- cycle ; 
\draw[Equator-full] (axis cs:2.14000E+02,1.00000E-01) -- (axis cs:2.15000E+02,1.00000E-01) -- (axis cs:2.15000E+02,-1.00000E-01) -- (axis cs:2.14000E+02,-1.00000E-01) -- cycle ; 
\draw[Equator-full] (axis cs:2.16000E+02,1.00000E-01) -- (axis cs:2.17000E+02,1.00000E-01) -- (axis cs:2.17000E+02,-1.00000E-01) -- (axis cs:2.16000E+02,-1.00000E-01) -- cycle ; 
\draw[Equator-full] (axis cs:2.18000E+02,1.00000E-01) -- (axis cs:2.19000E+02,1.00000E-01) -- (axis cs:2.19000E+02,-1.00000E-01) -- (axis cs:2.18000E+02,-1.00000E-01) -- cycle ; 
\draw[Equator-full] (axis cs:2.20000E+02,1.00000E-01) -- (axis cs:2.21000E+02,1.00000E-01) -- (axis cs:2.21000E+02,-1.00000E-01) -- (axis cs:2.20000E+02,-1.00000E-01) -- cycle ; 
\draw[Equator-full] (axis cs:2.22000E+02,1.00000E-01) -- (axis cs:2.23000E+02,1.00000E-01) -- (axis cs:2.23000E+02,-1.00000E-01) -- (axis cs:2.22000E+02,-1.00000E-01) -- cycle ; 
\draw[Equator-full] (axis cs:2.24000E+02,1.00000E-01) -- (axis cs:2.25000E+02,1.00000E-01) -- (axis cs:2.25000E+02,-1.00000E-01) -- (axis cs:2.24000E+02,-1.00000E-01) -- cycle ; 
\draw[Equator-full] (axis cs:2.26000E+02,1.00000E-01) -- (axis cs:2.27000E+02,1.00000E-01) -- (axis cs:2.27000E+02,-1.00000E-01) -- (axis cs:2.26000E+02,-1.00000E-01) -- cycle ; 
\draw[Equator-full] (axis cs:2.28000E+02,1.00000E-01) -- (axis cs:2.29000E+02,1.00000E-01) -- (axis cs:2.29000E+02,-1.00000E-01) -- (axis cs:2.28000E+02,-1.00000E-01) -- cycle ; 
\draw[Equator-full] (axis cs:2.30000E+02,1.00000E-01) -- (axis cs:2.31000E+02,1.00000E-01) -- (axis cs:2.31000E+02,-1.00000E-01) -- (axis cs:2.30000E+02,-1.00000E-01) -- cycle ; 
\draw[Equator-full] (axis cs:2.32000E+02,1.00000E-01) -- (axis cs:2.33000E+02,1.00000E-01) -- (axis cs:2.33000E+02,-1.00000E-01) -- (axis cs:2.32000E+02,-1.00000E-01) -- cycle ; 
\draw[Equator-full] (axis cs:2.34000E+02,1.00000E-01) -- (axis cs:2.35000E+02,1.00000E-01) -- (axis cs:2.35000E+02,-1.00000E-01) -- (axis cs:2.34000E+02,-1.00000E-01) -- cycle ; 
\draw[Equator-full] (axis cs:2.36000E+02,1.00000E-01) -- (axis cs:2.37000E+02,1.00000E-01) -- (axis cs:2.37000E+02,-1.00000E-01) -- (axis cs:2.36000E+02,-1.00000E-01) -- cycle ; 
\draw[Equator-full] (axis cs:2.38000E+02,1.00000E-01) -- (axis cs:2.39000E+02,1.00000E-01) -- (axis cs:2.39000E+02,-1.00000E-01) -- (axis cs:2.38000E+02,-1.00000E-01) -- cycle ; 
\draw[Equator-full] (axis cs:2.40000E+02,1.00000E-01) -- (axis cs:2.41000E+02,1.00000E-01) -- (axis cs:2.41000E+02,-1.00000E-01) -- (axis cs:2.40000E+02,-1.00000E-01) -- cycle ; 
\draw[Equator-full] (axis cs:2.42000E+02,1.00000E-01) -- (axis cs:2.43000E+02,1.00000E-01) -- (axis cs:2.43000E+02,-1.00000E-01) -- (axis cs:2.42000E+02,-1.00000E-01) -- cycle ; 
\draw[Equator-full] (axis cs:2.44000E+02,1.00000E-01) -- (axis cs:2.45000E+02,1.00000E-01) -- (axis cs:2.45000E+02,-1.00000E-01) -- (axis cs:2.44000E+02,-1.00000E-01) -- cycle ; 
\draw[Equator-full] (axis cs:2.46000E+02,1.00000E-01) -- (axis cs:2.47000E+02,1.00000E-01) -- (axis cs:2.47000E+02,-1.00000E-01) -- (axis cs:2.46000E+02,-1.00000E-01) -- cycle ; 
\draw[Equator-full] (axis cs:2.48000E+02,1.00000E-01) -- (axis cs:2.49000E+02,1.00000E-01) -- (axis cs:2.49000E+02,-1.00000E-01) -- (axis cs:2.48000E+02,-1.00000E-01) -- cycle ; 
\draw[Equator-full] (axis cs:2.50000E+02,1.00000E-01) -- (axis cs:2.51000E+02,1.00000E-01) -- (axis cs:2.51000E+02,-1.00000E-01) -- (axis cs:2.50000E+02,-1.00000E-01) -- cycle ; 
\draw[Equator-full] (axis cs:2.52000E+02,1.00000E-01) -- (axis cs:2.53000E+02,1.00000E-01) -- (axis cs:2.53000E+02,-1.00000E-01) -- (axis cs:2.52000E+02,-1.00000E-01) -- cycle ; 
\draw[Equator-full] (axis cs:2.54000E+02,1.00000E-01) -- (axis cs:2.55000E+02,1.00000E-01) -- (axis cs:2.55000E+02,-1.00000E-01) -- (axis cs:2.54000E+02,-1.00000E-01) -- cycle ; 
\draw[Equator-full] (axis cs:2.56000E+02,1.00000E-01) -- (axis cs:2.57000E+02,1.00000E-01) -- (axis cs:2.57000E+02,-1.00000E-01) -- (axis cs:2.56000E+02,-1.00000E-01) -- cycle ; 
\draw[Equator-full] (axis cs:2.58000E+02,1.00000E-01) -- (axis cs:2.59000E+02,1.00000E-01) -- (axis cs:2.59000E+02,-1.00000E-01) -- (axis cs:2.58000E+02,-1.00000E-01) -- cycle ; 
\draw[Equator-full] (axis cs:2.60000E+02,1.00000E-01) -- (axis cs:2.61000E+02,1.00000E-01) -- (axis cs:2.61000E+02,-1.00000E-01) -- (axis cs:2.60000E+02,-1.00000E-01) -- cycle ; 
\draw[Equator-full] (axis cs:2.62000E+02,1.00000E-01) -- (axis cs:2.63000E+02,1.00000E-01) -- (axis cs:2.63000E+02,-1.00000E-01) -- (axis cs:2.62000E+02,-1.00000E-01) -- cycle ; 
\draw[Equator-full] (axis cs:2.64000E+02,1.00000E-01) -- (axis cs:2.65000E+02,1.00000E-01) -- (axis cs:2.65000E+02,-1.00000E-01) -- (axis cs:2.64000E+02,-1.00000E-01) -- cycle ; 
\draw[Equator-full] (axis cs:2.66000E+02,1.00000E-01) -- (axis cs:2.67000E+02,1.00000E-01) -- (axis cs:2.67000E+02,-1.00000E-01) -- (axis cs:2.66000E+02,-1.00000E-01) -- cycle ; 
\draw[Equator-full] (axis cs:2.68000E+02,1.00000E-01) -- (axis cs:2.69000E+02,1.00000E-01) -- (axis cs:2.69000E+02,-1.00000E-01) -- (axis cs:2.68000E+02,-1.00000E-01) -- cycle ; 
\draw[Equator-full] (axis cs:2.70000E+02,1.00000E-01) -- (axis cs:2.71000E+02,1.00000E-01) -- (axis cs:2.71000E+02,-1.00000E-01) -- (axis cs:2.70000E+02,-1.00000E-01) -- cycle ; 
\draw[Equator-full] (axis cs:2.72000E+02,1.00000E-01) -- (axis cs:2.73000E+02,1.00000E-01) -- (axis cs:2.73000E+02,-1.00000E-01) -- (axis cs:2.72000E+02,-1.00000E-01) -- cycle ; 
\draw[Equator-full] (axis cs:2.74000E+02,1.00000E-01) -- (axis cs:2.75000E+02,1.00000E-01) -- (axis cs:2.75000E+02,-1.00000E-01) -- (axis cs:2.74000E+02,-1.00000E-01) -- cycle ; 
\draw[Equator-full] (axis cs:2.76000E+02,1.00000E-01) -- (axis cs:2.77000E+02,1.00000E-01) -- (axis cs:2.77000E+02,-1.00000E-01) -- (axis cs:2.76000E+02,-1.00000E-01) -- cycle ; 
\draw[Equator-full] (axis cs:2.78000E+02,1.00000E-01) -- (axis cs:2.79000E+02,1.00000E-01) -- (axis cs:2.79000E+02,-1.00000E-01) -- (axis cs:2.78000E+02,-1.00000E-01) -- cycle ; 
\draw[Equator-full] (axis cs:2.80000E+02,1.00000E-01) -- (axis cs:2.81000E+02,1.00000E-01) -- (axis cs:2.81000E+02,-1.00000E-01) -- (axis cs:2.80000E+02,-1.00000E-01) -- cycle ; 
\draw[Equator-full] (axis cs:2.82000E+02,1.00000E-01) -- (axis cs:2.83000E+02,1.00000E-01) -- (axis cs:2.83000E+02,-1.00000E-01) -- (axis cs:2.82000E+02,-1.00000E-01) -- cycle ; 
\draw[Equator-full] (axis cs:2.84000E+02,1.00000E-01) -- (axis cs:2.85000E+02,1.00000E-01) -- (axis cs:2.85000E+02,-1.00000E-01) -- (axis cs:2.84000E+02,-1.00000E-01) -- cycle ; 
\draw[Equator-full] (axis cs:2.86000E+02,1.00000E-01) -- (axis cs:2.87000E+02,1.00000E-01) -- (axis cs:2.87000E+02,-1.00000E-01) -- (axis cs:2.86000E+02,-1.00000E-01) -- cycle ; 
\draw[Equator-full] (axis cs:2.88000E+02,1.00000E-01) -- (axis cs:2.89000E+02,1.00000E-01) -- (axis cs:2.89000E+02,-1.00000E-01) -- (axis cs:2.88000E+02,-1.00000E-01) -- cycle ; 
\draw[Equator-full] (axis cs:2.90000E+02,1.00000E-01) -- (axis cs:2.91000E+02,1.00000E-01) -- (axis cs:2.91000E+02,-1.00000E-01) -- (axis cs:2.90000E+02,-1.00000E-01) -- cycle ; 
\draw[Equator-full] (axis cs:2.92000E+02,1.00000E-01) -- (axis cs:2.93000E+02,1.00000E-01) -- (axis cs:2.93000E+02,-1.00000E-01) -- (axis cs:2.92000E+02,-1.00000E-01) -- cycle ; 
\draw[Equator-full] (axis cs:2.94000E+02,1.00000E-01) -- (axis cs:2.95000E+02,1.00000E-01) -- (axis cs:2.95000E+02,-1.00000E-01) -- (axis cs:2.94000E+02,-1.00000E-01) -- cycle ; 
\draw[Equator-full] (axis cs:2.96000E+02,1.00000E-01) -- (axis cs:2.97000E+02,1.00000E-01) -- (axis cs:2.97000E+02,-1.00000E-01) -- (axis cs:2.96000E+02,-1.00000E-01) -- cycle ; 
\draw[Equator-full] (axis cs:2.98000E+02,1.00000E-01) -- (axis cs:2.99000E+02,1.00000E-01) -- (axis cs:2.99000E+02,-1.00000E-01) -- (axis cs:2.98000E+02,-1.00000E-01) -- cycle ; 
\draw[Equator-full] (axis cs:3.00000E+02,1.00000E-01) -- (axis cs:3.01000E+02,1.00000E-01) -- (axis cs:3.01000E+02,-1.00000E-01) -- (axis cs:3.00000E+02,-1.00000E-01) -- cycle ; 
\draw[Equator-full] (axis cs:3.02000E+02,1.00000E-01) -- (axis cs:3.03000E+02,1.00000E-01) -- (axis cs:3.03000E+02,-1.00000E-01) -- (axis cs:3.02000E+02,-1.00000E-01) -- cycle ; 
\draw[Equator-full] (axis cs:3.04000E+02,1.00000E-01) -- (axis cs:3.05000E+02,1.00000E-01) -- (axis cs:3.05000E+02,-1.00000E-01) -- (axis cs:3.04000E+02,-1.00000E-01) -- cycle ; 
\draw[Equator-full] (axis cs:3.06000E+02,1.00000E-01) -- (axis cs:3.07000E+02,1.00000E-01) -- (axis cs:3.07000E+02,-1.00000E-01) -- (axis cs:3.06000E+02,-1.00000E-01) -- cycle ; 
\draw[Equator-full] (axis cs:3.08000E+02,1.00000E-01) -- (axis cs:3.09000E+02,1.00000E-01) -- (axis cs:3.09000E+02,-1.00000E-01) -- (axis cs:3.08000E+02,-1.00000E-01) -- cycle ; 
\draw[Equator-full] (axis cs:3.10000E+02,1.00000E-01) -- (axis cs:3.11000E+02,1.00000E-01) -- (axis cs:3.11000E+02,-1.00000E-01) -- (axis cs:3.10000E+02,-1.00000E-01) -- cycle ; 
\draw[Equator-full] (axis cs:3.12000E+02,1.00000E-01) -- (axis cs:3.13000E+02,1.00000E-01) -- (axis cs:3.13000E+02,-1.00000E-01) -- (axis cs:3.12000E+02,-1.00000E-01) -- cycle ; 
\draw[Equator-full] (axis cs:3.14000E+02,1.00000E-01) -- (axis cs:3.15000E+02,1.00000E-01) -- (axis cs:3.15000E+02,-1.00000E-01) -- (axis cs:3.14000E+02,-1.00000E-01) -- cycle ; 
\draw[Equator-full] (axis cs:3.16000E+02,1.00000E-01) -- (axis cs:3.17000E+02,1.00000E-01) -- (axis cs:3.17000E+02,-1.00000E-01) -- (axis cs:3.16000E+02,-1.00000E-01) -- cycle ; 
\draw[Equator-full] (axis cs:3.18000E+02,1.00000E-01) -- (axis cs:3.19000E+02,1.00000E-01) -- (axis cs:3.19000E+02,-1.00000E-01) -- (axis cs:3.18000E+02,-1.00000E-01) -- cycle ; 
\draw[Equator-full] (axis cs:3.20000E+02,1.00000E-01) -- (axis cs:3.21000E+02,1.00000E-01) -- (axis cs:3.21000E+02,-1.00000E-01) -- (axis cs:3.20000E+02,-1.00000E-01) -- cycle ; 
\draw[Equator-full] (axis cs:3.22000E+02,1.00000E-01) -- (axis cs:3.23000E+02,1.00000E-01) -- (axis cs:3.23000E+02,-1.00000E-01) -- (axis cs:3.22000E+02,-1.00000E-01) -- cycle ; 
\draw[Equator-full] (axis cs:3.24000E+02,1.00000E-01) -- (axis cs:3.25000E+02,1.00000E-01) -- (axis cs:3.25000E+02,-1.00000E-01) -- (axis cs:3.24000E+02,-1.00000E-01) -- cycle ; 
\draw[Equator-full] (axis cs:3.26000E+02,1.00000E-01) -- (axis cs:3.27000E+02,1.00000E-01) -- (axis cs:3.27000E+02,-1.00000E-01) -- (axis cs:3.26000E+02,-1.00000E-01) -- cycle ; 
\draw[Equator-full] (axis cs:3.28000E+02,1.00000E-01) -- (axis cs:3.29000E+02,1.00000E-01) -- (axis cs:3.29000E+02,-1.00000E-01) -- (axis cs:3.28000E+02,-1.00000E-01) -- cycle ; 
\draw[Equator-full] (axis cs:3.30000E+02,1.00000E-01) -- (axis cs:3.31000E+02,1.00000E-01) -- (axis cs:3.31000E+02,-1.00000E-01) -- (axis cs:3.30000E+02,-1.00000E-01) -- cycle ; 
\draw[Equator-full] (axis cs:3.32000E+02,1.00000E-01) -- (axis cs:3.33000E+02,1.00000E-01) -- (axis cs:3.33000E+02,-1.00000E-01) -- (axis cs:3.32000E+02,-1.00000E-01) -- cycle ; 
\draw[Equator-full] (axis cs:3.34000E+02,1.00000E-01) -- (axis cs:3.35000E+02,1.00000E-01) -- (axis cs:3.35000E+02,-1.00000E-01) -- (axis cs:3.34000E+02,-1.00000E-01) -- cycle ; 
\draw[Equator-full] (axis cs:3.36000E+02,1.00000E-01) -- (axis cs:3.37000E+02,1.00000E-01) -- (axis cs:3.37000E+02,-1.00000E-01) -- (axis cs:3.36000E+02,-1.00000E-01) -- cycle ; 
\draw[Equator-full] (axis cs:3.38000E+02,1.00000E-01) -- (axis cs:3.39000E+02,1.00000E-01) -- (axis cs:3.39000E+02,-1.00000E-01) -- (axis cs:3.38000E+02,-1.00000E-01) -- cycle ; 
\draw[Equator-full] (axis cs:3.40000E+02,1.00000E-01) -- (axis cs:3.41000E+02,1.00000E-01) -- (axis cs:3.41000E+02,-1.00000E-01) -- (axis cs:3.40000E+02,-1.00000E-01) -- cycle ; 
\draw[Equator-full] (axis cs:3.42000E+02,1.00000E-01) -- (axis cs:3.43000E+02,1.00000E-01) -- (axis cs:3.43000E+02,-1.00000E-01) -- (axis cs:3.42000E+02,-1.00000E-01) -- cycle ; 
\draw[Equator-full] (axis cs:3.44000E+02,1.00000E-01) -- (axis cs:3.45000E+02,1.00000E-01) -- (axis cs:3.45000E+02,-1.00000E-01) -- (axis cs:3.44000E+02,-1.00000E-01) -- cycle ; 
\draw[Equator-full] (axis cs:3.46000E+02,1.00000E-01) -- (axis cs:3.47000E+02,1.00000E-01) -- (axis cs:3.47000E+02,-1.00000E-01) -- (axis cs:3.46000E+02,-1.00000E-01) -- cycle ; 
\draw[Equator-full] (axis cs:3.48000E+02,1.00000E-01) -- (axis cs:3.49000E+02,1.00000E-01) -- (axis cs:3.49000E+02,-1.00000E-01) -- (axis cs:3.48000E+02,-1.00000E-01) -- cycle ; 
\draw[Equator-full] (axis cs:3.50000E+02,1.00000E-01) -- (axis cs:3.51000E+02,1.00000E-01) -- (axis cs:3.51000E+02,-1.00000E-01) -- (axis cs:3.50000E+02,-1.00000E-01) -- cycle ; 
\draw[Equator-full] (axis cs:3.52000E+02,1.00000E-01) -- (axis cs:3.53000E+02,1.00000E-01) -- (axis cs:3.53000E+02,-1.00000E-01) -- (axis cs:3.52000E+02,-1.00000E-01) -- cycle ; 
\draw[Equator-full] (axis cs:3.54000E+02,1.00000E-01) -- (axis cs:3.55000E+02,1.00000E-01) -- (axis cs:3.55000E+02,-1.00000E-01) -- (axis cs:3.54000E+02,-1.00000E-01) -- cycle ; 
\draw[Equator-full] (axis cs:3.56000E+02,1.00000E-01) -- (axis cs:3.57000E+02,1.00000E-01) -- (axis cs:3.57000E+02,-1.00000E-01) -- (axis cs:3.56000E+02,-1.00000E-01) -- cycle ; 
\draw[Equator-full] (axis cs:3.58000E+02,1.00000E-01) -- (axis cs:3.59000E+02,1.00000E-01) -- (axis cs:3.59000E+02,-1.00000E-01) -- (axis cs:3.58000E+02,-1.00000E-01) -- cycle ; 
\draw[Equator-empty] (axis cs:1.00000E+00,1.00000E-01) -- (axis cs:2.00000E+00,1.00000E-01) -- (axis cs:2.00000E+00,-1.00000E-01) -- (axis cs:1.00000E+00,-1.00000E-01) -- cycle ; 
\draw[Equator-empty] (axis cs:3.00000E+00,1.00000E-01) -- (axis cs:4.00000E+00,1.00000E-01) -- (axis cs:4.00000E+00,-1.00000E-01) -- (axis cs:3.00000E+00,-1.00000E-01) -- cycle ; 
\draw[Equator-empty] (axis cs:5.00000E+00,1.00000E-01) -- (axis cs:6.00000E+00,1.00000E-01) -- (axis cs:6.00000E+00,-1.00000E-01) -- (axis cs:5.00000E+00,-1.00000E-01) -- cycle ; 
\draw[Equator-empty] (axis cs:7.00000E+00,1.00000E-01) -- (axis cs:8.00000E+00,1.00000E-01) -- (axis cs:8.00000E+00,-1.00000E-01) -- (axis cs:7.00000E+00,-1.00000E-01) -- cycle ; 
\draw[Equator-empty] (axis cs:9.00000E+00,1.00000E-01) -- (axis cs:1.00000E+01,1.00000E-01) -- (axis cs:1.00000E+01,-1.00000E-01) -- (axis cs:9.00000E+00,-1.00000E-01) -- cycle ; 
\draw[Equator-empty] (axis cs:1.10000E+01,1.00000E-01) -- (axis cs:1.20000E+01,1.00000E-01) -- (axis cs:1.20000E+01,-1.00000E-01) -- (axis cs:1.10000E+01,-1.00000E-01) -- cycle ; 
\draw[Equator-empty] (axis cs:1.30000E+01,1.00000E-01) -- (axis cs:1.40000E+01,1.00000E-01) -- (axis cs:1.40000E+01,-1.00000E-01) -- (axis cs:1.30000E+01,-1.00000E-01) -- cycle ; 
\draw[Equator-empty] (axis cs:1.50000E+01,1.00000E-01) -- (axis cs:1.60000E+01,1.00000E-01) -- (axis cs:1.60000E+01,-1.00000E-01) -- (axis cs:1.50000E+01,-1.00000E-01) -- cycle ; 
\draw[Equator-empty] (axis cs:1.70000E+01,1.00000E-01) -- (axis cs:1.80000E+01,1.00000E-01) -- (axis cs:1.80000E+01,-1.00000E-01) -- (axis cs:1.70000E+01,-1.00000E-01) -- cycle ; 
\draw[Equator-empty] (axis cs:1.90000E+01,1.00000E-01) -- (axis cs:2.00000E+01,1.00000E-01) -- (axis cs:2.00000E+01,-1.00000E-01) -- (axis cs:1.90000E+01,-1.00000E-01) -- cycle ; 
\draw[Equator-empty] (axis cs:2.10000E+01,1.00000E-01) -- (axis cs:2.20000E+01,1.00000E-01) -- (axis cs:2.20000E+01,-1.00000E-01) -- (axis cs:2.10000E+01,-1.00000E-01) -- cycle ; 
\draw[Equator-empty] (axis cs:2.30000E+01,1.00000E-01) -- (axis cs:2.40000E+01,1.00000E-01) -- (axis cs:2.40000E+01,-1.00000E-01) -- (axis cs:2.30000E+01,-1.00000E-01) -- cycle ; 
\draw[Equator-empty] (axis cs:2.50000E+01,1.00000E-01) -- (axis cs:2.60000E+01,1.00000E-01) -- (axis cs:2.60000E+01,-1.00000E-01) -- (axis cs:2.50000E+01,-1.00000E-01) -- cycle ; 
\draw[Equator-empty] (axis cs:2.70000E+01,1.00000E-01) -- (axis cs:2.80000E+01,1.00000E-01) -- (axis cs:2.80000E+01,-1.00000E-01) -- (axis cs:2.70000E+01,-1.00000E-01) -- cycle ; 
\draw[Equator-empty] (axis cs:2.90000E+01,1.00000E-01) -- (axis cs:3.00000E+01,1.00000E-01) -- (axis cs:3.00000E+01,-1.00000E-01) -- (axis cs:2.90000E+01,-1.00000E-01) -- cycle ; 
\draw[Equator-empty] (axis cs:3.10000E+01,1.00000E-01) -- (axis cs:3.20000E+01,1.00000E-01) -- (axis cs:3.20000E+01,-1.00000E-01) -- (axis cs:3.10000E+01,-1.00000E-01) -- cycle ; 
\draw[Equator-empty] (axis cs:3.30000E+01,1.00000E-01) -- (axis cs:3.40000E+01,1.00000E-01) -- (axis cs:3.40000E+01,-1.00000E-01) -- (axis cs:3.30000E+01,-1.00000E-01) -- cycle ; 
\draw[Equator-empty] (axis cs:3.50000E+01,1.00000E-01) -- (axis cs:3.60000E+01,1.00000E-01) -- (axis cs:3.60000E+01,-1.00000E-01) -- (axis cs:3.50000E+01,-1.00000E-01) -- cycle ; 
\draw[Equator-empty] (axis cs:3.70000E+01,1.00000E-01) -- (axis cs:3.80000E+01,1.00000E-01) -- (axis cs:3.80000E+01,-1.00000E-01) -- (axis cs:3.70000E+01,-1.00000E-01) -- cycle ; 
\draw[Equator-empty] (axis cs:3.90000E+01,1.00000E-01) -- (axis cs:4.00000E+01,1.00000E-01) -- (axis cs:4.00000E+01,-1.00000E-01) -- (axis cs:3.90000E+01,-1.00000E-01) -- cycle ; 
\draw[Equator-empty] (axis cs:4.10000E+01,1.00000E-01) -- (axis cs:4.20000E+01,1.00000E-01) -- (axis cs:4.20000E+01,-1.00000E-01) -- (axis cs:4.10000E+01,-1.00000E-01) -- cycle ; 
\draw[Equator-empty] (axis cs:4.30000E+01,1.00000E-01) -- (axis cs:4.40000E+01,1.00000E-01) -- (axis cs:4.40000E+01,-1.00000E-01) -- (axis cs:4.30000E+01,-1.00000E-01) -- cycle ; 
\draw[Equator-empty] (axis cs:4.50000E+01,1.00000E-01) -- (axis cs:4.60000E+01,1.00000E-01) -- (axis cs:4.60000E+01,-1.00000E-01) -- (axis cs:4.50000E+01,-1.00000E-01) -- cycle ; 
\draw[Equator-empty] (axis cs:4.70000E+01,1.00000E-01) -- (axis cs:4.80000E+01,1.00000E-01) -- (axis cs:4.80000E+01,-1.00000E-01) -- (axis cs:4.70000E+01,-1.00000E-01) -- cycle ; 
\draw[Equator-empty] (axis cs:4.90000E+01,1.00000E-01) -- (axis cs:5.00000E+01,1.00000E-01) -- (axis cs:5.00000E+01,-1.00000E-01) -- (axis cs:4.90000E+01,-1.00000E-01) -- cycle ; 
\draw[Equator-empty] (axis cs:5.10000E+01,1.00000E-01) -- (axis cs:5.20000E+01,1.00000E-01) -- (axis cs:5.20000E+01,-1.00000E-01) -- (axis cs:5.10000E+01,-1.00000E-01) -- cycle ; 
\draw[Equator-empty] (axis cs:5.30000E+01,1.00000E-01) -- (axis cs:5.40000E+01,1.00000E-01) -- (axis cs:5.40000E+01,-1.00000E-01) -- (axis cs:5.30000E+01,-1.00000E-01) -- cycle ; 
\draw[Equator-empty] (axis cs:5.50000E+01,1.00000E-01) -- (axis cs:5.60000E+01,1.00000E-01) -- (axis cs:5.60000E+01,-1.00000E-01) -- (axis cs:5.50000E+01,-1.00000E-01) -- cycle ; 
\draw[Equator-empty] (axis cs:5.70000E+01,1.00000E-01) -- (axis cs:5.80000E+01,1.00000E-01) -- (axis cs:5.80000E+01,-1.00000E-01) -- (axis cs:5.70000E+01,-1.00000E-01) -- cycle ; 
\draw[Equator-empty] (axis cs:5.90000E+01,1.00000E-01) -- (axis cs:6.00000E+01,1.00000E-01) -- (axis cs:6.00000E+01,-1.00000E-01) -- (axis cs:5.90000E+01,-1.00000E-01) -- cycle ; 
\draw[Equator-empty] (axis cs:6.10000E+01,1.00000E-01) -- (axis cs:6.20000E+01,1.00000E-01) -- (axis cs:6.20000E+01,-1.00000E-01) -- (axis cs:6.10000E+01,-1.00000E-01) -- cycle ; 
\draw[Equator-empty] (axis cs:6.30000E+01,1.00000E-01) -- (axis cs:6.40000E+01,1.00000E-01) -- (axis cs:6.40000E+01,-1.00000E-01) -- (axis cs:6.30000E+01,-1.00000E-01) -- cycle ; 
\draw[Equator-empty] (axis cs:6.50000E+01,1.00000E-01) -- (axis cs:6.60000E+01,1.00000E-01) -- (axis cs:6.60000E+01,-1.00000E-01) -- (axis cs:6.50000E+01,-1.00000E-01) -- cycle ; 
\draw[Equator-empty] (axis cs:6.70000E+01,1.00000E-01) -- (axis cs:6.80000E+01,1.00000E-01) -- (axis cs:6.80000E+01,-1.00000E-01) -- (axis cs:6.70000E+01,-1.00000E-01) -- cycle ; 
\draw[Equator-empty] (axis cs:6.90000E+01,1.00000E-01) -- (axis cs:7.00000E+01,1.00000E-01) -- (axis cs:7.00000E+01,-1.00000E-01) -- (axis cs:6.90000E+01,-1.00000E-01) -- cycle ; 
\draw[Equator-empty] (axis cs:7.10000E+01,1.00000E-01) -- (axis cs:7.20000E+01,1.00000E-01) -- (axis cs:7.20000E+01,-1.00000E-01) -- (axis cs:7.10000E+01,-1.00000E-01) -- cycle ; 
\draw[Equator-empty] (axis cs:7.30000E+01,1.00000E-01) -- (axis cs:7.40000E+01,1.00000E-01) -- (axis cs:7.40000E+01,-1.00000E-01) -- (axis cs:7.30000E+01,-1.00000E-01) -- cycle ; 
\draw[Equator-empty] (axis cs:7.50000E+01,1.00000E-01) -- (axis cs:7.60000E+01,1.00000E-01) -- (axis cs:7.60000E+01,-1.00000E-01) -- (axis cs:7.50000E+01,-1.00000E-01) -- cycle ; 
\draw[Equator-empty] (axis cs:7.70000E+01,1.00000E-01) -- (axis cs:7.80000E+01,1.00000E-01) -- (axis cs:7.80000E+01,-1.00000E-01) -- (axis cs:7.70000E+01,-1.00000E-01) -- cycle ; 
\draw[Equator-empty] (axis cs:7.90000E+01,1.00000E-01) -- (axis cs:8.00000E+01,1.00000E-01) -- (axis cs:8.00000E+01,-1.00000E-01) -- (axis cs:7.90000E+01,-1.00000E-01) -- cycle ; 
\draw[Equator-empty] (axis cs:8.10000E+01,1.00000E-01) -- (axis cs:8.20000E+01,1.00000E-01) -- (axis cs:8.20000E+01,-1.00000E-01) -- (axis cs:8.10000E+01,-1.00000E-01) -- cycle ; 
\draw[Equator-empty] (axis cs:8.30000E+01,1.00000E-01) -- (axis cs:8.40000E+01,1.00000E-01) -- (axis cs:8.40000E+01,-1.00000E-01) -- (axis cs:8.30000E+01,-1.00000E-01) -- cycle ; 
\draw[Equator-empty] (axis cs:8.50000E+01,1.00000E-01) -- (axis cs:8.60000E+01,1.00000E-01) -- (axis cs:8.60000E+01,-1.00000E-01) -- (axis cs:8.50000E+01,-1.00000E-01) -- cycle ; 
\draw[Equator-empty] (axis cs:8.70000E+01,1.00000E-01) -- (axis cs:8.80000E+01,1.00000E-01) -- (axis cs:8.80000E+01,-1.00000E-01) -- (axis cs:8.70000E+01,-1.00000E-01) -- cycle ; 
\draw[Equator-empty] (axis cs:8.90000E+01,1.00000E-01) -- (axis cs:9.00000E+01,1.00000E-01) -- (axis cs:9.00000E+01,-1.00000E-01) -- (axis cs:8.90000E+01,-1.00000E-01) -- cycle ; 
\draw[Equator-empty] (axis cs:9.10000E+01,1.00000E-01) -- (axis cs:9.20000E+01,1.00000E-01) -- (axis cs:9.20000E+01,-1.00000E-01) -- (axis cs:9.10000E+01,-1.00000E-01) -- cycle ; 
\draw[Equator-empty] (axis cs:9.30000E+01,1.00000E-01) -- (axis cs:9.40000E+01,1.00000E-01) -- (axis cs:9.40000E+01,-1.00000E-01) -- (axis cs:9.30000E+01,-1.00000E-01) -- cycle ; 
\draw[Equator-empty] (axis cs:9.50000E+01,1.00000E-01) -- (axis cs:9.60000E+01,1.00000E-01) -- (axis cs:9.60000E+01,-1.00000E-01) -- (axis cs:9.50000E+01,-1.00000E-01) -- cycle ; 
\draw[Equator-empty] (axis cs:9.70000E+01,1.00000E-01) -- (axis cs:9.80000E+01,1.00000E-01) -- (axis cs:9.80000E+01,-1.00000E-01) -- (axis cs:9.70000E+01,-1.00000E-01) -- cycle ; 
\draw[Equator-empty] (axis cs:9.90000E+01,1.00000E-01) -- (axis cs:1.00000E+02,1.00000E-01) -- (axis cs:1.00000E+02,-1.00000E-01) -- (axis cs:9.90000E+01,-1.00000E-01) -- cycle ; 
\draw[Equator-empty] (axis cs:1.01000E+02,1.00000E-01) -- (axis cs:1.02000E+02,1.00000E-01) -- (axis cs:1.02000E+02,-1.00000E-01) -- (axis cs:1.01000E+02,-1.00000E-01) -- cycle ; 
\draw[Equator-empty] (axis cs:1.03000E+02,1.00000E-01) -- (axis cs:1.04000E+02,1.00000E-01) -- (axis cs:1.04000E+02,-1.00000E-01) -- (axis cs:1.03000E+02,-1.00000E-01) -- cycle ; 
\draw[Equator-empty] (axis cs:1.05000E+02,1.00000E-01) -- (axis cs:1.06000E+02,1.00000E-01) -- (axis cs:1.06000E+02,-1.00000E-01) -- (axis cs:1.05000E+02,-1.00000E-01) -- cycle ; 
\draw[Equator-empty] (axis cs:1.07000E+02,1.00000E-01) -- (axis cs:1.08000E+02,1.00000E-01) -- (axis cs:1.08000E+02,-1.00000E-01) -- (axis cs:1.07000E+02,-1.00000E-01) -- cycle ; 
\draw[Equator-empty] (axis cs:1.09000E+02,1.00000E-01) -- (axis cs:1.10000E+02,1.00000E-01) -- (axis cs:1.10000E+02,-1.00000E-01) -- (axis cs:1.09000E+02,-1.00000E-01) -- cycle ; 
\draw[Equator-empty] (axis cs:1.11000E+02,1.00000E-01) -- (axis cs:1.12000E+02,1.00000E-01) -- (axis cs:1.12000E+02,-1.00000E-01) -- (axis cs:1.11000E+02,-1.00000E-01) -- cycle ; 
\draw[Equator-empty] (axis cs:1.13000E+02,1.00000E-01) -- (axis cs:1.14000E+02,1.00000E-01) -- (axis cs:1.14000E+02,-1.00000E-01) -- (axis cs:1.13000E+02,-1.00000E-01) -- cycle ; 
\draw[Equator-empty] (axis cs:1.15000E+02,1.00000E-01) -- (axis cs:1.16000E+02,1.00000E-01) -- (axis cs:1.16000E+02,-1.00000E-01) -- (axis cs:1.15000E+02,-1.00000E-01) -- cycle ; 
\draw[Equator-empty] (axis cs:1.17000E+02,1.00000E-01) -- (axis cs:1.18000E+02,1.00000E-01) -- (axis cs:1.18000E+02,-1.00000E-01) -- (axis cs:1.17000E+02,-1.00000E-01) -- cycle ; 
\draw[Equator-empty] (axis cs:1.19000E+02,1.00000E-01) -- (axis cs:1.20000E+02,1.00000E-01) -- (axis cs:1.20000E+02,-1.00000E-01) -- (axis cs:1.19000E+02,-1.00000E-01) -- cycle ; 
\draw[Equator-empty] (axis cs:1.21000E+02,1.00000E-01) -- (axis cs:1.22000E+02,1.00000E-01) -- (axis cs:1.22000E+02,-1.00000E-01) -- (axis cs:1.21000E+02,-1.00000E-01) -- cycle ; 
\draw[Equator-empty] (axis cs:1.23000E+02,1.00000E-01) -- (axis cs:1.24000E+02,1.00000E-01) -- (axis cs:1.24000E+02,-1.00000E-01) -- (axis cs:1.23000E+02,-1.00000E-01) -- cycle ; 
\draw[Equator-empty] (axis cs:1.25000E+02,1.00000E-01) -- (axis cs:1.26000E+02,1.00000E-01) -- (axis cs:1.26000E+02,-1.00000E-01) -- (axis cs:1.25000E+02,-1.00000E-01) -- cycle ; 
\draw[Equator-empty] (axis cs:1.27000E+02,1.00000E-01) -- (axis cs:1.28000E+02,1.00000E-01) -- (axis cs:1.28000E+02,-1.00000E-01) -- (axis cs:1.27000E+02,-1.00000E-01) -- cycle ; 
\draw[Equator-empty] (axis cs:1.29000E+02,1.00000E-01) -- (axis cs:1.30000E+02,1.00000E-01) -- (axis cs:1.30000E+02,-1.00000E-01) -- (axis cs:1.29000E+02,-1.00000E-01) -- cycle ; 
\draw[Equator-empty] (axis cs:1.31000E+02,1.00000E-01) -- (axis cs:1.32000E+02,1.00000E-01) -- (axis cs:1.32000E+02,-1.00000E-01) -- (axis cs:1.31000E+02,-1.00000E-01) -- cycle ; 
\draw[Equator-empty] (axis cs:1.33000E+02,1.00000E-01) -- (axis cs:1.34000E+02,1.00000E-01) -- (axis cs:1.34000E+02,-1.00000E-01) -- (axis cs:1.33000E+02,-1.00000E-01) -- cycle ; 
\draw[Equator-empty] (axis cs:1.35000E+02,1.00000E-01) -- (axis cs:1.36000E+02,1.00000E-01) -- (axis cs:1.36000E+02,-1.00000E-01) -- (axis cs:1.35000E+02,-1.00000E-01) -- cycle ; 
\draw[Equator-empty] (axis cs:1.37000E+02,1.00000E-01) -- (axis cs:1.38000E+02,1.00000E-01) -- (axis cs:1.38000E+02,-1.00000E-01) -- (axis cs:1.37000E+02,-1.00000E-01) -- cycle ; 
\draw[Equator-empty] (axis cs:1.39000E+02,1.00000E-01) -- (axis cs:1.40000E+02,1.00000E-01) -- (axis cs:1.40000E+02,-1.00000E-01) -- (axis cs:1.39000E+02,-1.00000E-01) -- cycle ; 
\draw[Equator-empty] (axis cs:1.41000E+02,1.00000E-01) -- (axis cs:1.42000E+02,1.00000E-01) -- (axis cs:1.42000E+02,-1.00000E-01) -- (axis cs:1.41000E+02,-1.00000E-01) -- cycle ; 
\draw[Equator-empty] (axis cs:1.43000E+02,1.00000E-01) -- (axis cs:1.44000E+02,1.00000E-01) -- (axis cs:1.44000E+02,-1.00000E-01) -- (axis cs:1.43000E+02,-1.00000E-01) -- cycle ; 
\draw[Equator-empty] (axis cs:1.45000E+02,1.00000E-01) -- (axis cs:1.46000E+02,1.00000E-01) -- (axis cs:1.46000E+02,-1.00000E-01) -- (axis cs:1.45000E+02,-1.00000E-01) -- cycle ; 
\draw[Equator-empty] (axis cs:1.47000E+02,1.00000E-01) -- (axis cs:1.48000E+02,1.00000E-01) -- (axis cs:1.48000E+02,-1.00000E-01) -- (axis cs:1.47000E+02,-1.00000E-01) -- cycle ; 
\draw[Equator-empty] (axis cs:1.49000E+02,1.00000E-01) -- (axis cs:1.50000E+02,1.00000E-01) -- (axis cs:1.50000E+02,-1.00000E-01) -- (axis cs:1.49000E+02,-1.00000E-01) -- cycle ; 
\draw[Equator-empty] (axis cs:1.51000E+02,1.00000E-01) -- (axis cs:1.52000E+02,1.00000E-01) -- (axis cs:1.52000E+02,-1.00000E-01) -- (axis cs:1.51000E+02,-1.00000E-01) -- cycle ; 
\draw[Equator-empty] (axis cs:1.53000E+02,1.00000E-01) -- (axis cs:1.54000E+02,1.00000E-01) -- (axis cs:1.54000E+02,-1.00000E-01) -- (axis cs:1.53000E+02,-1.00000E-01) -- cycle ; 
\draw[Equator-empty] (axis cs:1.55000E+02,1.00000E-01) -- (axis cs:1.56000E+02,1.00000E-01) -- (axis cs:1.56000E+02,-1.00000E-01) -- (axis cs:1.55000E+02,-1.00000E-01) -- cycle ; 
\draw[Equator-empty] (axis cs:1.57000E+02,1.00000E-01) -- (axis cs:1.58000E+02,1.00000E-01) -- (axis cs:1.58000E+02,-1.00000E-01) -- (axis cs:1.57000E+02,-1.00000E-01) -- cycle ; 
\draw[Equator-empty] (axis cs:1.59000E+02,1.00000E-01) -- (axis cs:1.60000E+02,1.00000E-01) -- (axis cs:1.60000E+02,-1.00000E-01) -- (axis cs:1.59000E+02,-1.00000E-01) -- cycle ; 
\draw[Equator-empty] (axis cs:1.61000E+02,1.00000E-01) -- (axis cs:1.62000E+02,1.00000E-01) -- (axis cs:1.62000E+02,-1.00000E-01) -- (axis cs:1.61000E+02,-1.00000E-01) -- cycle ; 
\draw[Equator-empty] (axis cs:1.63000E+02,1.00000E-01) -- (axis cs:1.64000E+02,1.00000E-01) -- (axis cs:1.64000E+02,-1.00000E-01) -- (axis cs:1.63000E+02,-1.00000E-01) -- cycle ; 
\draw[Equator-empty] (axis cs:1.65000E+02,1.00000E-01) -- (axis cs:1.66000E+02,1.00000E-01) -- (axis cs:1.66000E+02,-1.00000E-01) -- (axis cs:1.65000E+02,-1.00000E-01) -- cycle ; 
\draw[Equator-empty] (axis cs:1.67000E+02,1.00000E-01) -- (axis cs:1.68000E+02,1.00000E-01) -- (axis cs:1.68000E+02,-1.00000E-01) -- (axis cs:1.67000E+02,-1.00000E-01) -- cycle ; 
\draw[Equator-empty] (axis cs:1.69000E+02,1.00000E-01) -- (axis cs:1.70000E+02,1.00000E-01) -- (axis cs:1.70000E+02,-1.00000E-01) -- (axis cs:1.69000E+02,-1.00000E-01) -- cycle ; 
\draw[Equator-empty] (axis cs:1.71000E+02,1.00000E-01) -- (axis cs:1.72000E+02,1.00000E-01) -- (axis cs:1.72000E+02,-1.00000E-01) -- (axis cs:1.71000E+02,-1.00000E-01) -- cycle ; 
\draw[Equator-empty] (axis cs:1.73000E+02,1.00000E-01) -- (axis cs:1.74000E+02,1.00000E-01) -- (axis cs:1.74000E+02,-1.00000E-01) -- (axis cs:1.73000E+02,-1.00000E-01) -- cycle ; 
\draw[Equator-empty] (axis cs:1.75000E+02,1.00000E-01) -- (axis cs:1.76000E+02,1.00000E-01) -- (axis cs:1.76000E+02,-1.00000E-01) -- (axis cs:1.75000E+02,-1.00000E-01) -- cycle ; 
\draw[Equator-empty] (axis cs:1.77000E+02,1.00000E-01) -- (axis cs:1.78000E+02,1.00000E-01) -- (axis cs:1.78000E+02,-1.00000E-01) -- (axis cs:1.77000E+02,-1.00000E-01) -- cycle ; 
\draw[Equator-empty] (axis cs:1.79000E+02,1.00000E-01) -- (axis cs:1.80000E+02,1.00000E-01) -- (axis cs:1.80000E+02,-1.00000E-01) -- (axis cs:1.79000E+02,-1.00000E-01) -- cycle ; 
\draw[Equator-empty] (axis cs:1.81000E+02,1.00000E-01) -- (axis cs:1.82000E+02,1.00000E-01) -- (axis cs:1.82000E+02,-1.00000E-01) -- (axis cs:1.81000E+02,-1.00000E-01) -- cycle ; 
\draw[Equator-empty] (axis cs:1.83000E+02,1.00000E-01) -- (axis cs:1.84000E+02,1.00000E-01) -- (axis cs:1.84000E+02,-1.00000E-01) -- (axis cs:1.83000E+02,-1.00000E-01) -- cycle ; 
\draw[Equator-empty] (axis cs:1.85000E+02,1.00000E-01) -- (axis cs:1.86000E+02,1.00000E-01) -- (axis cs:1.86000E+02,-1.00000E-01) -- (axis cs:1.85000E+02,-1.00000E-01) -- cycle ; 
\draw[Equator-empty] (axis cs:1.87000E+02,1.00000E-01) -- (axis cs:1.88000E+02,1.00000E-01) -- (axis cs:1.88000E+02,-1.00000E-01) -- (axis cs:1.87000E+02,-1.00000E-01) -- cycle ; 
\draw[Equator-empty] (axis cs:1.89000E+02,1.00000E-01) -- (axis cs:1.90000E+02,1.00000E-01) -- (axis cs:1.90000E+02,-1.00000E-01) -- (axis cs:1.89000E+02,-1.00000E-01) -- cycle ; 
\draw[Equator-empty] (axis cs:1.91000E+02,1.00000E-01) -- (axis cs:1.92000E+02,1.00000E-01) -- (axis cs:1.92000E+02,-1.00000E-01) -- (axis cs:1.91000E+02,-1.00000E-01) -- cycle ; 
\draw[Equator-empty] (axis cs:1.93000E+02,1.00000E-01) -- (axis cs:1.94000E+02,1.00000E-01) -- (axis cs:1.94000E+02,-1.00000E-01) -- (axis cs:1.93000E+02,-1.00000E-01) -- cycle ; 
\draw[Equator-empty] (axis cs:1.95000E+02,1.00000E-01) -- (axis cs:1.96000E+02,1.00000E-01) -- (axis cs:1.96000E+02,-1.00000E-01) -- (axis cs:1.95000E+02,-1.00000E-01) -- cycle ; 
\draw[Equator-empty] (axis cs:1.97000E+02,1.00000E-01) -- (axis cs:1.98000E+02,1.00000E-01) -- (axis cs:1.98000E+02,-1.00000E-01) -- (axis cs:1.97000E+02,-1.00000E-01) -- cycle ; 
\draw[Equator-empty] (axis cs:1.99000E+02,1.00000E-01) -- (axis cs:2.00000E+02,1.00000E-01) -- (axis cs:2.00000E+02,-1.00000E-01) -- (axis cs:1.99000E+02,-1.00000E-01) -- cycle ; 
\draw[Equator-empty] (axis cs:2.01000E+02,1.00000E-01) -- (axis cs:2.02000E+02,1.00000E-01) -- (axis cs:2.02000E+02,-1.00000E-01) -- (axis cs:2.01000E+02,-1.00000E-01) -- cycle ; 
\draw[Equator-empty] (axis cs:2.03000E+02,1.00000E-01) -- (axis cs:2.04000E+02,1.00000E-01) -- (axis cs:2.04000E+02,-1.00000E-01) -- (axis cs:2.03000E+02,-1.00000E-01) -- cycle ; 
\draw[Equator-empty] (axis cs:2.05000E+02,1.00000E-01) -- (axis cs:2.06000E+02,1.00000E-01) -- (axis cs:2.06000E+02,-1.00000E-01) -- (axis cs:2.05000E+02,-1.00000E-01) -- cycle ; 
\draw[Equator-empty] (axis cs:2.07000E+02,1.00000E-01) -- (axis cs:2.08000E+02,1.00000E-01) -- (axis cs:2.08000E+02,-1.00000E-01) -- (axis cs:2.07000E+02,-1.00000E-01) -- cycle ; 
\draw[Equator-empty] (axis cs:2.09000E+02,1.00000E-01) -- (axis cs:2.10000E+02,1.00000E-01) -- (axis cs:2.10000E+02,-1.00000E-01) -- (axis cs:2.09000E+02,-1.00000E-01) -- cycle ; 
\draw[Equator-empty] (axis cs:2.11000E+02,1.00000E-01) -- (axis cs:2.12000E+02,1.00000E-01) -- (axis cs:2.12000E+02,-1.00000E-01) -- (axis cs:2.11000E+02,-1.00000E-01) -- cycle ; 
\draw[Equator-empty] (axis cs:2.13000E+02,1.00000E-01) -- (axis cs:2.14000E+02,1.00000E-01) -- (axis cs:2.14000E+02,-1.00000E-01) -- (axis cs:2.13000E+02,-1.00000E-01) -- cycle ; 
\draw[Equator-empty] (axis cs:2.15000E+02,1.00000E-01) -- (axis cs:2.16000E+02,1.00000E-01) -- (axis cs:2.16000E+02,-1.00000E-01) -- (axis cs:2.15000E+02,-1.00000E-01) -- cycle ; 
\draw[Equator-empty] (axis cs:2.17000E+02,1.00000E-01) -- (axis cs:2.18000E+02,1.00000E-01) -- (axis cs:2.18000E+02,-1.00000E-01) -- (axis cs:2.17000E+02,-1.00000E-01) -- cycle ; 
\draw[Equator-empty] (axis cs:2.19000E+02,1.00000E-01) -- (axis cs:2.20000E+02,1.00000E-01) -- (axis cs:2.20000E+02,-1.00000E-01) -- (axis cs:2.19000E+02,-1.00000E-01) -- cycle ; 
\draw[Equator-empty] (axis cs:2.21000E+02,1.00000E-01) -- (axis cs:2.22000E+02,1.00000E-01) -- (axis cs:2.22000E+02,-1.00000E-01) -- (axis cs:2.21000E+02,-1.00000E-01) -- cycle ; 
\draw[Equator-empty] (axis cs:2.23000E+02,1.00000E-01) -- (axis cs:2.24000E+02,1.00000E-01) -- (axis cs:2.24000E+02,-1.00000E-01) -- (axis cs:2.23000E+02,-1.00000E-01) -- cycle ; 
\draw[Equator-empty] (axis cs:2.25000E+02,1.00000E-01) -- (axis cs:2.26000E+02,1.00000E-01) -- (axis cs:2.26000E+02,-1.00000E-01) -- (axis cs:2.25000E+02,-1.00000E-01) -- cycle ; 
\draw[Equator-empty] (axis cs:2.27000E+02,1.00000E-01) -- (axis cs:2.28000E+02,1.00000E-01) -- (axis cs:2.28000E+02,-1.00000E-01) -- (axis cs:2.27000E+02,-1.00000E-01) -- cycle ; 
\draw[Equator-empty] (axis cs:2.29000E+02,1.00000E-01) -- (axis cs:2.30000E+02,1.00000E-01) -- (axis cs:2.30000E+02,-1.00000E-01) -- (axis cs:2.29000E+02,-1.00000E-01) -- cycle ; 
\draw[Equator-empty] (axis cs:2.31000E+02,1.00000E-01) -- (axis cs:2.32000E+02,1.00000E-01) -- (axis cs:2.32000E+02,-1.00000E-01) -- (axis cs:2.31000E+02,-1.00000E-01) -- cycle ; 
\draw[Equator-empty] (axis cs:2.33000E+02,1.00000E-01) -- (axis cs:2.34000E+02,1.00000E-01) -- (axis cs:2.34000E+02,-1.00000E-01) -- (axis cs:2.33000E+02,-1.00000E-01) -- cycle ; 
\draw[Equator-empty] (axis cs:2.35000E+02,1.00000E-01) -- (axis cs:2.36000E+02,1.00000E-01) -- (axis cs:2.36000E+02,-1.00000E-01) -- (axis cs:2.35000E+02,-1.00000E-01) -- cycle ; 
\draw[Equator-empty] (axis cs:2.37000E+02,1.00000E-01) -- (axis cs:2.38000E+02,1.00000E-01) -- (axis cs:2.38000E+02,-1.00000E-01) -- (axis cs:2.37000E+02,-1.00000E-01) -- cycle ; 
\draw[Equator-empty] (axis cs:2.39000E+02,1.00000E-01) -- (axis cs:2.40000E+02,1.00000E-01) -- (axis cs:2.40000E+02,-1.00000E-01) -- (axis cs:2.39000E+02,-1.00000E-01) -- cycle ; 
\draw[Equator-empty] (axis cs:2.41000E+02,1.00000E-01) -- (axis cs:2.42000E+02,1.00000E-01) -- (axis cs:2.42000E+02,-1.00000E-01) -- (axis cs:2.41000E+02,-1.00000E-01) -- cycle ; 
\draw[Equator-empty] (axis cs:2.43000E+02,1.00000E-01) -- (axis cs:2.44000E+02,1.00000E-01) -- (axis cs:2.44000E+02,-1.00000E-01) -- (axis cs:2.43000E+02,-1.00000E-01) -- cycle ; 
\draw[Equator-empty] (axis cs:2.45000E+02,1.00000E-01) -- (axis cs:2.46000E+02,1.00000E-01) -- (axis cs:2.46000E+02,-1.00000E-01) -- (axis cs:2.45000E+02,-1.00000E-01) -- cycle ; 
\draw[Equator-empty] (axis cs:2.47000E+02,1.00000E-01) -- (axis cs:2.48000E+02,1.00000E-01) -- (axis cs:2.48000E+02,-1.00000E-01) -- (axis cs:2.47000E+02,-1.00000E-01) -- cycle ; 
\draw[Equator-empty] (axis cs:2.49000E+02,1.00000E-01) -- (axis cs:2.50000E+02,1.00000E-01) -- (axis cs:2.50000E+02,-1.00000E-01) -- (axis cs:2.49000E+02,-1.00000E-01) -- cycle ; 
\draw[Equator-empty] (axis cs:2.51000E+02,1.00000E-01) -- (axis cs:2.52000E+02,1.00000E-01) -- (axis cs:2.52000E+02,-1.00000E-01) -- (axis cs:2.51000E+02,-1.00000E-01) -- cycle ; 
\draw[Equator-empty] (axis cs:2.53000E+02,1.00000E-01) -- (axis cs:2.54000E+02,1.00000E-01) -- (axis cs:2.54000E+02,-1.00000E-01) -- (axis cs:2.53000E+02,-1.00000E-01) -- cycle ; 
\draw[Equator-empty] (axis cs:2.55000E+02,1.00000E-01) -- (axis cs:2.56000E+02,1.00000E-01) -- (axis cs:2.56000E+02,-1.00000E-01) -- (axis cs:2.55000E+02,-1.00000E-01) -- cycle ; 
\draw[Equator-empty] (axis cs:2.57000E+02,1.00000E-01) -- (axis cs:2.58000E+02,1.00000E-01) -- (axis cs:2.58000E+02,-1.00000E-01) -- (axis cs:2.57000E+02,-1.00000E-01) -- cycle ; 
\draw[Equator-empty] (axis cs:2.59000E+02,1.00000E-01) -- (axis cs:2.60000E+02,1.00000E-01) -- (axis cs:2.60000E+02,-1.00000E-01) -- (axis cs:2.59000E+02,-1.00000E-01) -- cycle ; 
\draw[Equator-empty] (axis cs:2.61000E+02,1.00000E-01) -- (axis cs:2.62000E+02,1.00000E-01) -- (axis cs:2.62000E+02,-1.00000E-01) -- (axis cs:2.61000E+02,-1.00000E-01) -- cycle ; 
\draw[Equator-empty] (axis cs:2.63000E+02,1.00000E-01) -- (axis cs:2.64000E+02,1.00000E-01) -- (axis cs:2.64000E+02,-1.00000E-01) -- (axis cs:2.63000E+02,-1.00000E-01) -- cycle ; 
\draw[Equator-empty] (axis cs:2.65000E+02,1.00000E-01) -- (axis cs:2.66000E+02,1.00000E-01) -- (axis cs:2.66000E+02,-1.00000E-01) -- (axis cs:2.65000E+02,-1.00000E-01) -- cycle ; 
\draw[Equator-empty] (axis cs:2.67000E+02,1.00000E-01) -- (axis cs:2.68000E+02,1.00000E-01) -- (axis cs:2.68000E+02,-1.00000E-01) -- (axis cs:2.67000E+02,-1.00000E-01) -- cycle ; 
\draw[Equator-empty] (axis cs:2.69000E+02,1.00000E-01) -- (axis cs:2.70000E+02,1.00000E-01) -- (axis cs:2.70000E+02,-1.00000E-01) -- (axis cs:2.69000E+02,-1.00000E-01) -- cycle ; 
\draw[Equator-empty] (axis cs:2.71000E+02,1.00000E-01) -- (axis cs:2.72000E+02,1.00000E-01) -- (axis cs:2.72000E+02,-1.00000E-01) -- (axis cs:2.71000E+02,-1.00000E-01) -- cycle ; 
\draw[Equator-empty] (axis cs:2.73000E+02,1.00000E-01) -- (axis cs:2.74000E+02,1.00000E-01) -- (axis cs:2.74000E+02,-1.00000E-01) -- (axis cs:2.73000E+02,-1.00000E-01) -- cycle ; 
\draw[Equator-empty] (axis cs:2.75000E+02,1.00000E-01) -- (axis cs:2.76000E+02,1.00000E-01) -- (axis cs:2.76000E+02,-1.00000E-01) -- (axis cs:2.75000E+02,-1.00000E-01) -- cycle ; 
\draw[Equator-empty] (axis cs:2.77000E+02,1.00000E-01) -- (axis cs:2.78000E+02,1.00000E-01) -- (axis cs:2.78000E+02,-1.00000E-01) -- (axis cs:2.77000E+02,-1.00000E-01) -- cycle ; 
\draw[Equator-empty] (axis cs:2.79000E+02,1.00000E-01) -- (axis cs:2.80000E+02,1.00000E-01) -- (axis cs:2.80000E+02,-1.00000E-01) -- (axis cs:2.79000E+02,-1.00000E-01) -- cycle ; 
\draw[Equator-empty] (axis cs:2.81000E+02,1.00000E-01) -- (axis cs:2.82000E+02,1.00000E-01) -- (axis cs:2.82000E+02,-1.00000E-01) -- (axis cs:2.81000E+02,-1.00000E-01) -- cycle ; 
\draw[Equator-empty] (axis cs:2.83000E+02,1.00000E-01) -- (axis cs:2.84000E+02,1.00000E-01) -- (axis cs:2.84000E+02,-1.00000E-01) -- (axis cs:2.83000E+02,-1.00000E-01) -- cycle ; 
\draw[Equator-empty] (axis cs:2.85000E+02,1.00000E-01) -- (axis cs:2.86000E+02,1.00000E-01) -- (axis cs:2.86000E+02,-1.00000E-01) -- (axis cs:2.85000E+02,-1.00000E-01) -- cycle ; 
\draw[Equator-empty] (axis cs:2.87000E+02,1.00000E-01) -- (axis cs:2.88000E+02,1.00000E-01) -- (axis cs:2.88000E+02,-1.00000E-01) -- (axis cs:2.87000E+02,-1.00000E-01) -- cycle ; 
\draw[Equator-empty] (axis cs:2.89000E+02,1.00000E-01) -- (axis cs:2.90000E+02,1.00000E-01) -- (axis cs:2.90000E+02,-1.00000E-01) -- (axis cs:2.89000E+02,-1.00000E-01) -- cycle ; 
\draw[Equator-empty] (axis cs:2.91000E+02,1.00000E-01) -- (axis cs:2.92000E+02,1.00000E-01) -- (axis cs:2.92000E+02,-1.00000E-01) -- (axis cs:2.91000E+02,-1.00000E-01) -- cycle ; 
\draw[Equator-empty] (axis cs:2.93000E+02,1.00000E-01) -- (axis cs:2.94000E+02,1.00000E-01) -- (axis cs:2.94000E+02,-1.00000E-01) -- (axis cs:2.93000E+02,-1.00000E-01) -- cycle ; 
\draw[Equator-empty] (axis cs:2.95000E+02,1.00000E-01) -- (axis cs:2.96000E+02,1.00000E-01) -- (axis cs:2.96000E+02,-1.00000E-01) -- (axis cs:2.95000E+02,-1.00000E-01) -- cycle ; 
\draw[Equator-empty] (axis cs:2.97000E+02,1.00000E-01) -- (axis cs:2.98000E+02,1.00000E-01) -- (axis cs:2.98000E+02,-1.00000E-01) -- (axis cs:2.97000E+02,-1.00000E-01) -- cycle ; 
\draw[Equator-empty] (axis cs:2.99000E+02,1.00000E-01) -- (axis cs:3.00000E+02,1.00000E-01) -- (axis cs:3.00000E+02,-1.00000E-01) -- (axis cs:2.99000E+02,-1.00000E-01) -- cycle ; 
\draw[Equator-empty] (axis cs:3.01000E+02,1.00000E-01) -- (axis cs:3.02000E+02,1.00000E-01) -- (axis cs:3.02000E+02,-1.00000E-01) -- (axis cs:3.01000E+02,-1.00000E-01) -- cycle ; 
\draw[Equator-empty] (axis cs:3.03000E+02,1.00000E-01) -- (axis cs:3.04000E+02,1.00000E-01) -- (axis cs:3.04000E+02,-1.00000E-01) -- (axis cs:3.03000E+02,-1.00000E-01) -- cycle ; 
\draw[Equator-empty] (axis cs:3.05000E+02,1.00000E-01) -- (axis cs:3.06000E+02,1.00000E-01) -- (axis cs:3.06000E+02,-1.00000E-01) -- (axis cs:3.05000E+02,-1.00000E-01) -- cycle ; 
\draw[Equator-empty] (axis cs:3.07000E+02,1.00000E-01) -- (axis cs:3.08000E+02,1.00000E-01) -- (axis cs:3.08000E+02,-1.00000E-01) -- (axis cs:3.07000E+02,-1.00000E-01) -- cycle ; 
\draw[Equator-empty] (axis cs:3.09000E+02,1.00000E-01) -- (axis cs:3.10000E+02,1.00000E-01) -- (axis cs:3.10000E+02,-1.00000E-01) -- (axis cs:3.09000E+02,-1.00000E-01) -- cycle ; 
\draw[Equator-empty] (axis cs:3.11000E+02,1.00000E-01) -- (axis cs:3.12000E+02,1.00000E-01) -- (axis cs:3.12000E+02,-1.00000E-01) -- (axis cs:3.11000E+02,-1.00000E-01) -- cycle ; 
\draw[Equator-empty] (axis cs:3.13000E+02,1.00000E-01) -- (axis cs:3.14000E+02,1.00000E-01) -- (axis cs:3.14000E+02,-1.00000E-01) -- (axis cs:3.13000E+02,-1.00000E-01) -- cycle ; 
\draw[Equator-empty] (axis cs:3.15000E+02,1.00000E-01) -- (axis cs:3.16000E+02,1.00000E-01) -- (axis cs:3.16000E+02,-1.00000E-01) -- (axis cs:3.15000E+02,-1.00000E-01) -- cycle ; 
\draw[Equator-empty] (axis cs:3.17000E+02,1.00000E-01) -- (axis cs:3.18000E+02,1.00000E-01) -- (axis cs:3.18000E+02,-1.00000E-01) -- (axis cs:3.17000E+02,-1.00000E-01) -- cycle ; 
\draw[Equator-empty] (axis cs:3.19000E+02,1.00000E-01) -- (axis cs:3.20000E+02,1.00000E-01) -- (axis cs:3.20000E+02,-1.00000E-01) -- (axis cs:3.19000E+02,-1.00000E-01) -- cycle ; 
\draw[Equator-empty] (axis cs:3.21000E+02,1.00000E-01) -- (axis cs:3.22000E+02,1.00000E-01) -- (axis cs:3.22000E+02,-1.00000E-01) -- (axis cs:3.21000E+02,-1.00000E-01) -- cycle ; 
\draw[Equator-empty] (axis cs:3.23000E+02,1.00000E-01) -- (axis cs:3.24000E+02,1.00000E-01) -- (axis cs:3.24000E+02,-1.00000E-01) -- (axis cs:3.23000E+02,-1.00000E-01) -- cycle ; 
\draw[Equator-empty] (axis cs:3.25000E+02,1.00000E-01) -- (axis cs:3.26000E+02,1.00000E-01) -- (axis cs:3.26000E+02,-1.00000E-01) -- (axis cs:3.25000E+02,-1.00000E-01) -- cycle ; 
\draw[Equator-empty] (axis cs:3.27000E+02,1.00000E-01) -- (axis cs:3.28000E+02,1.00000E-01) -- (axis cs:3.28000E+02,-1.00000E-01) -- (axis cs:3.27000E+02,-1.00000E-01) -- cycle ; 
\draw[Equator-empty] (axis cs:3.29000E+02,1.00000E-01) -- (axis cs:3.30000E+02,1.00000E-01) -- (axis cs:3.30000E+02,-1.00000E-01) -- (axis cs:3.29000E+02,-1.00000E-01) -- cycle ; 
\draw[Equator-empty] (axis cs:3.31000E+02,1.00000E-01) -- (axis cs:3.32000E+02,1.00000E-01) -- (axis cs:3.32000E+02,-1.00000E-01) -- (axis cs:3.31000E+02,-1.00000E-01) -- cycle ; 
\draw[Equator-empty] (axis cs:3.33000E+02,1.00000E-01) -- (axis cs:3.34000E+02,1.00000E-01) -- (axis cs:3.34000E+02,-1.00000E-01) -- (axis cs:3.33000E+02,-1.00000E-01) -- cycle ; 
\draw[Equator-empty] (axis cs:3.35000E+02,1.00000E-01) -- (axis cs:3.36000E+02,1.00000E-01) -- (axis cs:3.36000E+02,-1.00000E-01) -- (axis cs:3.35000E+02,-1.00000E-01) -- cycle ; 
\draw[Equator-empty] (axis cs:3.37000E+02,1.00000E-01) -- (axis cs:3.38000E+02,1.00000E-01) -- (axis cs:3.38000E+02,-1.00000E-01) -- (axis cs:3.37000E+02,-1.00000E-01) -- cycle ; 
\draw[Equator-empty] (axis cs:3.39000E+02,1.00000E-01) -- (axis cs:3.40000E+02,1.00000E-01) -- (axis cs:3.40000E+02,-1.00000E-01) -- (axis cs:3.39000E+02,-1.00000E-01) -- cycle ; 
\draw[Equator-empty] (axis cs:3.41000E+02,1.00000E-01) -- (axis cs:3.42000E+02,1.00000E-01) -- (axis cs:3.42000E+02,-1.00000E-01) -- (axis cs:3.41000E+02,-1.00000E-01) -- cycle ; 
\draw[Equator-empty] (axis cs:3.43000E+02,1.00000E-01) -- (axis cs:3.44000E+02,1.00000E-01) -- (axis cs:3.44000E+02,-1.00000E-01) -- (axis cs:3.43000E+02,-1.00000E-01) -- cycle ; 
\draw[Equator-empty] (axis cs:3.45000E+02,1.00000E-01) -- (axis cs:3.46000E+02,1.00000E-01) -- (axis cs:3.46000E+02,-1.00000E-01) -- (axis cs:3.45000E+02,-1.00000E-01) -- cycle ; 
\draw[Equator-empty] (axis cs:3.47000E+02,1.00000E-01) -- (axis cs:3.48000E+02,1.00000E-01) -- (axis cs:3.48000E+02,-1.00000E-01) -- (axis cs:3.47000E+02,-1.00000E-01) -- cycle ; 
\draw[Equator-empty] (axis cs:3.49000E+02,1.00000E-01) -- (axis cs:3.50000E+02,1.00000E-01) -- (axis cs:3.50000E+02,-1.00000E-01) -- (axis cs:3.49000E+02,-1.00000E-01) -- cycle ; 
\draw[Equator-empty] (axis cs:3.51000E+02,1.00000E-01) -- (axis cs:3.52000E+02,1.00000E-01) -- (axis cs:3.52000E+02,-1.00000E-01) -- (axis cs:3.51000E+02,-1.00000E-01) -- cycle ; 
\draw[Equator-empty] (axis cs:3.53000E+02,1.00000E-01) -- (axis cs:3.54000E+02,1.00000E-01) -- (axis cs:3.54000E+02,-1.00000E-01) -- (axis cs:3.53000E+02,-1.00000E-01) -- cycle ; 
\draw[Equator-empty] (axis cs:3.55000E+02,1.00000E-01) -- (axis cs:3.56000E+02,1.00000E-01) -- (axis cs:3.56000E+02,-1.00000E-01) -- (axis cs:3.55000E+02,-1.00000E-01) -- cycle ; 
\draw[Equator-empty] (axis cs:3.57000E+02,1.00000E-01) -- (axis cs:3.58000E+02,1.00000E-01) -- (axis cs:3.58000E+02,-1.00000E-01) -- (axis cs:3.57000E+02,-1.00000E-01) -- cycle ; 
\draw[Equator-empty] (axis cs:3.59000E+02,1.00000E-01) -- (axis cs:3.60000E+02,1.00000E-01) -- (axis cs:3.60000E+02,-1.00000E-01) -- (axis cs:3.59000E+02,-1.00000E-01) -- cycle ; 

\draw[Equator-full] (axis cs:3.60000E+02,1.00000E-01) -- (axis cs:3.61000E+02,1.00000E-01) -- (axis cs:3.61000E+02,-1.00000E-01) -- (axis cs:3.60000E+02,-1.00000E-01) -- cycle ; 
\draw[Equator-full] (axis cs:3.62000E+02,1.00000E-01) -- (axis cs:3.63000E+02,1.00000E-01) -- (axis cs:3.63000E+02,-1.00000E-01) -- (axis cs:3.62000E+02,-1.00000E-01) -- cycle ; 
\draw[Equator-full] (axis cs:3.64000E+02,1.00000E-01) -- (axis cs:3.65000E+02,1.00000E-01) -- (axis cs:3.65000E+02,-1.00000E-01) -- (axis cs:3.64000E+02,-1.00000E-01) -- cycle ; 
\draw[Equator-full] (axis cs:3.66000E+02,1.00000E-01) -- (axis cs:3.67000E+02,1.00000E-01) -- (axis cs:3.67000E+02,-1.00000E-01) -- (axis cs:3.66000E+02,-1.00000E-01) -- cycle ; 
\draw[Equator-full] (axis cs:3.68000E+02,1.00000E-01) -- (axis cs:3.69000E+02,1.00000E-01) -- (axis cs:3.69000E+02,-1.00000E-01) -- (axis cs:3.68000E+02,-1.00000E-01) -- cycle ; 
\draw[Equator-full] (axis cs:3.70000E+02,1.00000E-01) -- (axis cs:3.71000E+02,1.00000E-01) -- (axis cs:3.71000E+02,-1.00000E-01) -- (axis cs:3.70000E+02,-1.00000E-01) -- cycle ; 
\draw[Equator-full] (axis cs:3.72000E+02,1.00000E-01) -- (axis cs:3.73000E+02,1.00000E-01) -- (axis cs:3.73000E+02,-1.00000E-01) -- (axis cs:3.72000E+02,-1.00000E-01) -- cycle ; 
\draw[Equator-full] (axis cs:3.74000E+02,1.00000E-01) -- (axis cs:3.75000E+02,1.00000E-01) -- (axis cs:3.75000E+02,-1.00000E-01) -- (axis cs:3.74000E+02,-1.00000E-01) -- cycle ; 
\draw[Equator-full] (axis cs:3.76000E+02,1.00000E-01) -- (axis cs:3.77000E+02,1.00000E-01) -- (axis cs:3.77000E+02,-1.00000E-01) -- (axis cs:3.76000E+02,-1.00000E-01) -- cycle ; 
\draw[Equator-full] (axis cs:3.78000E+02,1.00000E-01) -- (axis cs:3.79000E+02,1.00000E-01) -- (axis cs:3.79000E+02,-1.00000E-01) -- (axis cs:3.78000E+02,-1.00000E-01) -- cycle ; 

\draw[Equator-empty] (axis cs:3.61000E+02,1.00000E-01) -- (axis cs:3.62000E+02,1.00000E-01) -- (axis cs:3.62000E+02,-1.00000E-01) -- (axis cs:3.61000E+02,-1.00000E-01) -- cycle ; 
\draw[Equator-empty] (axis cs:3.63000E+02,1.00000E-01) -- (axis cs:3.64000E+02,1.00000E-01) -- (axis cs:3.64000E+02,-1.00000E-01) -- (axis cs:3.63000E+02,-1.00000E-01) -- cycle ; 
\draw[Equator-empty] (axis cs:3.65000E+02,1.00000E-01) -- (axis cs:3.66000E+02,1.00000E-01) -- (axis cs:3.66000E+02,-1.00000E-01) -- (axis cs:3.65000E+02,-1.00000E-01) -- cycle ; 
\draw[Equator-empty] (axis cs:3.67000E+02,1.00000E-01) -- (axis cs:3.68000E+02,1.00000E-01) -- (axis cs:3.68000E+02,-1.00000E-01) -- (axis cs:3.67000E+02,-1.00000E-01) -- cycle ; 
\draw[Equator-empty] (axis cs:3.69000E+02,1.00000E-01) -- (axis cs:3.70000E+02,1.00000E-01) -- (axis cs:3.70000E+02,-1.00000E-01) -- (axis cs:3.69000E+02,-1.00000E-01) -- cycle ; 
\draw[Equator-empty] (axis cs:3.71000E+02,1.00000E-01) -- (axis cs:3.72000E+02,1.00000E-01) -- (axis cs:3.72000E+02,-1.00000E-01) -- (axis cs:3.71000E+02,-1.00000E-01) -- cycle ; 
\draw[Equator-empty] (axis cs:3.73000E+02,1.00000E-01) -- (axis cs:3.74000E+02,1.00000E-01) -- (axis cs:3.74000E+02,-1.00000E-01) -- (axis cs:3.73000E+02,-1.00000E-01) -- cycle ; 
\draw[Equator-empty] (axis cs:3.75000E+02,1.00000E-01) -- (axis cs:3.76000E+02,1.00000E-01) -- (axis cs:3.76000E+02,-1.00000E-01) -- (axis cs:3.75000E+02,-1.00000E-01) -- cycle ; 
\draw[Equator-empty] (axis cs:3.77000E+02,1.00000E-01) -- (axis cs:3.78000E+02,1.00000E-01) -- (axis cs:3.78000E+02,-1.00000E-01) -- (axis cs:3.77000E+02,-1.00000E-01) -- cycle ; 
\draw[Equator-empty] (axis cs:3.79000E+02,1.00000E-01) -- (axis cs:3.80000E+02,1.00000E-01) -- (axis cs:3.80000E+02,-1.00000E-01) -- (axis cs:3.79000E+02,-1.00000E-01) -- cycle ; 


\draw [Equator-empty] (axis cs:0.00000E+00,4.00000E-01) -- (axis cs:0.00000E+00,-4.00000E-01)   ;
\node[pin={[pin distance=-0.4\onedegree,Equator-label]90:{0$^\circ$}}] at (axis cs:0.00000E+00,4.00000E-01) {} ;
\draw [Equator-empty] (axis cs:1.00000E+01,4.00000E-01) -- (axis cs:1.00000E+01,-4.00000E-01)   ;
\node[pin={[pin distance=-0.4\onedegree,Equator-label]90:{10$^\circ$}}] at (axis cs:1.00000E+01,4.00000E-01) {} ;
\draw [Equator-empty] (axis cs:2.00000E+01,4.00000E-01) -- (axis cs:2.00000E+01,-4.00000E-01)   ;
\node[pin={[pin distance=-0.4\onedegree,Equator-label]90:{20$^\circ$}}] at (axis cs:2.00000E+01,4.00000E-01) {} ;
\draw [Equator-empty] (axis cs:3.00000E+01,4.00000E-01) -- (axis cs:3.00000E+01,-4.00000E-01)   ;
\node[pin={[pin distance=-0.4\onedegree,Equator-label]90:{30$^\circ$}}] at (axis cs:3.00000E+01,4.00000E-01) {} ;
\draw [Equator-empty] (axis cs:4.00000E+01,4.00000E-01) -- (axis cs:4.00000E+01,-4.00000E-01)   ;
\node[pin={[pin distance=-0.4\onedegree,Equator-label]90:{40$^\circ$}}] at (axis cs:4.00000E+01,4.00000E-01) {} ;
\draw [Equator-empty] (axis cs:5.00000E+01,4.00000E-01) -- (axis cs:5.00000E+01,-4.00000E-01)   ;
\node[pin={[pin distance=-0.4\onedegree,Equator-label]90:{50$^\circ$}}] at (axis cs:5.00000E+01,4.00000E-01) {} ;
\draw [Equator-empty] (axis cs:6.00000E+01,4.00000E-01) -- (axis cs:6.00000E+01,-4.00000E-01)   ;
\node[pin={[pin distance=-0.4\onedegree,Equator-label]90:{60$^\circ$}}] at (axis cs:6.00000E+01,4.00000E-01) {} ;
\draw [Equator-empty] (axis cs:7.00000E+01,4.00000E-01) -- (axis cs:7.00000E+01,-4.00000E-01)   ;
\node[pin={[pin distance=-0.4\onedegree,Equator-label]90:{70$^\circ$}}] at (axis cs:7.00000E+01,4.00000E-01) {} ;
\draw [Equator-empty] (axis cs:8.00000E+01,4.00000E-01) -- (axis cs:8.00000E+01,-4.00000E-01)   ;
\node[pin={[pin distance=-0.4\onedegree,Equator-label]90:{80$^\circ$}}] at (axis cs:8.00000E+01,4.00000E-01) {} ;
\draw [Equator-empty] (axis cs:9.00000E+01,4.00000E-01) -- (axis cs:9.00000E+01,-4.00000E-01)   ;
\node[pin={[pin distance=-0.4\onedegree,Equator-label]90:{90$^\circ$}}] at (axis cs:9.00000E+01,4.00000E-01) {} ;
\draw [Equator-empty] (axis cs:1.00000E+02,4.00000E-01) -- (axis cs:1.00000E+02,-4.00000E-01)   ;
\node[pin={[pin distance=-0.4\onedegree,Equator-label]090:{100$^\circ$}}] at (axis cs:1.00000E+02,4.00000E-01) {} ;
\draw [Equator-empty] (axis cs:1.10000E+02,4.00000E-01) -- (axis cs:1.10000E+02,-4.00000E-01)   ;
\node[pin={[pin distance=-0.4\onedegree,Equator-label]090:{110$^\circ$}}] at (axis cs:1.10000E+02,4.00000E-01) {} ;
\draw [Equator-empty] (axis cs:1.20000E+02,4.00000E-01) -- (axis cs:1.20000E+02,-4.00000E-01)   ;
\node[pin={[pin distance=-0.4\onedegree,Equator-label]090:{120$^\circ$}}] at (axis cs:1.20000E+02,4.00000E-01) {} ;
\draw [Equator-empty] (axis cs:1.30000E+02,4.00000E-01) -- (axis cs:1.30000E+02,-4.00000E-01)   ;
\node[pin={[pin distance=-0.4\onedegree,Equator-label]090:{130$^\circ$}}] at (axis cs:1.30000E+02,4.00000E-01) {} ;
\draw [Equator-empty] (axis cs:1.40000E+02,4.00000E-01) -- (axis cs:1.40000E+02,-4.00000E-01)   ;
\node[pin={[pin distance=-0.4\onedegree,Equator-label]090:{140$^\circ$}}] at (axis cs:1.40000E+02,4.00000E-01) {} ;
\draw [Equator-empty] (axis cs:1.50000E+02,4.00000E-01) -- (axis cs:1.50000E+02,-4.00000E-01)   ;
\node[pin={[pin distance=-0.4\onedegree,Equator-label]090:{150$^\circ$}}] at (axis cs:1.50000E+02,4.00000E-01) {} ;
\draw [Equator-empty] (axis cs:1.60000E+02,4.00000E-01) -- (axis cs:1.60000E+02,-4.00000E-01)   ;
\node[pin={[pin distance=-0.4\onedegree,Equator-label]090:{160$^\circ$}}] at (axis cs:1.60000E+02,4.00000E-01) {} ;
\draw [Equator-empty] (axis cs:1.70000E+02,4.00000E-01) -- (axis cs:1.70000E+02,-4.00000E-01)   ;
\node[pin={[pin distance=-0.4\onedegree,Equator-label]090:{170$^\circ$}}] at (axis cs:1.70000E+02,4.00000E-01) {} ;
\draw [Equator-empty] (axis cs:1.80000E+02,4.00000E-01) -- (axis cs:1.80000E+02,-4.00000E-01)   ;
\node[pin={[pin distance=-0.4\onedegree,Equator-label]090:{180$^\circ$}}] at (axis cs:1.80000E+02,4.00000E-01) {} ;
\draw [Equator-empty] (axis cs:1.90000E+02,4.00000E-01) -- (axis cs:1.90000E+02,-4.00000E-01)   ;
\node[pin={[pin distance=-0.4\onedegree,Equator-label]090:{190$^\circ$}}] at (axis cs:1.90000E+02,4.00000E-01) {} ;
\draw [Equator-empty] (axis cs:2.00000E+02,4.00000E-01) -- (axis cs:2.00000E+02,-4.00000E-01)   ;
\node[pin={[pin distance=-0.4\onedegree,Equator-label]090:{200$^\circ$}}] at (axis cs:2.00000E+02,4.00000E-01) {} ;
\draw [Equator-empty] (axis cs:2.10000E+02,4.00000E-01) -- (axis cs:2.10000E+02,-4.00000E-01)   ;
\node[pin={[pin distance=-0.4\onedegree,Equator-label]090:{210$^\circ$}}] at (axis cs:2.10000E+02,4.00000E-01) {} ;
\draw [Equator-empty] (axis cs:2.20000E+02,4.00000E-01) -- (axis cs:2.20000E+02,-4.00000E-01)   ;
\node[pin={[pin distance=-0.4\onedegree,Equator-label]090:{220$^\circ$}}] at (axis cs:2.20000E+02,4.00000E-01) {} ;
\draw [Equator-empty] (axis cs:2.30000E+02,4.00000E-01) -- (axis cs:2.30000E+02,-4.00000E-01)   ;
\node[pin={[pin distance=-0.4\onedegree,Equator-label]090:{230$^\circ$}}] at (axis cs:2.30000E+02,4.00000E-01) {} ;
\draw [Equator-empty] (axis cs:2.40000E+02,4.00000E-01) -- (axis cs:2.40000E+02,-4.00000E-01)   ;
\node[pin={[pin distance=-0.4\onedegree,Equator-label]090:{240$^\circ$}}] at (axis cs:2.40000E+02,4.00000E-01) {} ;
\draw [Equator-empty] (axis cs:2.50000E+02,4.00000E-01) -- (axis cs:2.50000E+02,-4.00000E-01)   ;
\node[pin={[pin distance=-0.4\onedegree,Equator-label]090:{250$^\circ$}}] at (axis cs:2.50000E+02,4.00000E-01) {} ;
\draw [Equator-empty] (axis cs:2.60000E+02,4.00000E-01) -- (axis cs:2.60000E+02,-4.00000E-01)   ;
\node[pin={[pin distance=-0.4\onedegree,Equator-label]090:{260$^\circ$}}] at (axis cs:2.60000E+02,4.00000E-01) {} ;
\draw [Equator-empty] (axis cs:2.70000E+02,4.00000E-01) -- (axis cs:2.70000E+02,-4.00000E-01)   ;
\node[pin={[pin distance=-0.4\onedegree,Equator-label]090:{270$^\circ$}}] at (axis cs:2.70000E+02,4.00000E-01) {} ;
\draw [Equator-empty] (axis cs:2.80000E+02,4.00000E-01) -- (axis cs:2.80000E+02,-4.00000E-01)   ;
\node[pin={[pin distance=-0.4\onedegree,Equator-label]090:{280$^\circ$}}] at (axis cs:2.80000E+02,4.00000E-01) {} ;
\draw [Equator-empty] (axis cs:2.90000E+02,4.00000E-01) -- (axis cs:2.90000E+02,-4.00000E-01)   ;
\node[pin={[pin distance=-0.4\onedegree,Equator-label]090:{290$^\circ$}}] at (axis cs:2.90000E+02,4.00000E-01) {} ;
\draw [Equator-empty] (axis cs:3.00000E+02,4.00000E-01) -- (axis cs:3.00000E+02,-4.00000E-01)   ;
\node[pin={[pin distance=-0.4\onedegree,Equator-label]090:{300$^\circ$}}] at (axis cs:3.00000E+02,4.00000E-01) {} ;
\draw [Equator-empty] (axis cs:3.10000E+02,4.00000E-01) -- (axis cs:3.10000E+02,-4.00000E-01)   ;
\node[pin={[pin distance=-0.4\onedegree,Equator-label]090:{310$^\circ$}}] at (axis cs:3.10000E+02,4.00000E-01) {} ;
\draw [Equator-empty] (axis cs:3.20000E+02,4.00000E-01) -- (axis cs:3.20000E+02,-4.00000E-01)   ;
\node[pin={[pin distance=-0.4\onedegree,Equator-label]090:{320$^\circ$}}] at (axis cs:3.20000E+02,4.00000E-01) {} ;
\draw [Equator-empty] (axis cs:3.30000E+02,4.00000E-01) -- (axis cs:3.30000E+02,-4.00000E-01)   ;
\node[pin={[pin distance=-0.4\onedegree,Equator-label]090:{330$^\circ$}}] at (axis cs:3.30000E+02,4.00000E-01) {} ;
\draw [Equator-empty] (axis cs:3.40000E+02,4.00000E-01) -- (axis cs:3.40000E+02,-4.00000E-01)   ;
\node[pin={[pin distance=-0.4\onedegree,Equator-label]090:{340$^\circ$}}] at (axis cs:3.40000E+02,4.00000E-01) {} ;
\draw [Equator-empty] (axis cs:3.50000E+02,4.00000E-01) -- (axis cs:3.50000E+02,-4.00000E-01)   ;
\node[pin={[pin distance=-0.4\onedegree,Equator-label]090:{350$^\circ$}}] at (axis cs:3.50000E+02,4.00000E-01) {} ;

\draw [Equator-empty] (axis cs:3.60000E+02,4.00000E-01) -- (axis cs:3.60000E+02,-4.00000E-01)   ;
\node[pin={[pin distance=-0.4\onedegree,Equator-label]090:{0$^\circ$}}] at (axis cs:3.60000E+02,4.00000E-01) {} ;
\draw [Equator-empty] (axis cs:3.70000E+02,4.00000E-01) -- (axis cs:3.70000E+02,-4.00000E-01)   ;
\node[pin={[pin distance=-0.4\onedegree,Equator-label]090:{10$^\circ$}}] at (axis cs:3.70000E+02,4.00000E-01) {} ;




\end{axis}


% Coordinate axes 


% Left and bottom axis with gridlines
  \begin{axis}[at=(base.center),anchor=center,RA_in_hours,color=cAxes] \end{axis}
% Right and top axis, labelling in RA hours (offset)
  \begin{axis}[at=(base.center),anchor=center,yticklabel pos=right,xticklabel pos=right,RA_in_hours
  ,xticklabel style={yshift=2.5ex, anchor=south},color=cAxes]\end{axis}
% Top axis labelling in RA degree (small offset)
  \begin{axis}[at=(base.center),anchor=center,axis y line=none,xticklabel pos=top,RA_in_deg
     ,xticklabel style={font=\footnotesize},xticklabel style={yshift=0.5ex, anchor=south} ,color=cAxes]
  \end{axis}

% Coordinate grid

\begin{axis}[at=(base.center),anchor=center,coordinategrid,RA_in_deg,
    separate axis lines,
    y axis line style= { draw opacity=0 },
    x axis line style= { draw opacity=0 },
    ,xticklabels={},yticklabels={}
] \end{axis}


% Ornamental frame


  \begin{axis}[name=frame,at={($(base.center)+(0cm,+.7\onedegree)$)},anchor=center,width=11.75\tendegree,height=8.53\tendegree,ticks=none
  ,yticklabels={},xticklabels={},axis line style={line width=1pt},color=cFrame] \end{axis}

  \begin{axis}[name=frame2,at={($(frame.center)+(0cm,+0cm)$)},anchor=center,width=11.85\tendegree,height=8.63\tendegree,ticks=none
  ,yticklabels={},xticklabels={},axis line style={line width=3pt},color=cFrame] \end{axis}
  

% Symbol legend


 
 \begin{axis}[name=legend,at=(frame2.south),anchor=north,axis y line=none,axis x
   line=none,xmin=-55,xmax=55,height=0.5\tendegree   ,ymin=-1.5,ymax=1.5,  ,y dir=normal,x dir=reverse, color=cAxes]

\node at (axis cs:54,+1.0) {\small\it 1};                  \node at (axis cs:54,0.3)  {\tikz{\pgfuseplotmark{m1b};}};           
\node at (axis cs:52,+1.0) {\small\it 1{\nicefrac{1}{3}}}; \node at (axis cs:52,0.3)  {\tikz{\pgfuseplotmark{m1c};}};           
\node at (axis cs:53,-0.5)  {\small First};                  

\node at (axis cs:48,+1.0) {\small\it 1{\nicefrac{2}{3}}}; \node at (axis cs:48,0.3)  {\tikz{\pgfuseplotmark{m2a};}};           
\node at (axis cs:46,+1.0) {\small\it 2};                  \node at (axis cs:46,0.3)  {\tikz{\pgfuseplotmark{m2b};}};           
\node at (axis cs:44,+1.0) {\small\it 2{\nicefrac{1}{3}}}; \node at (axis cs:44,0.3)  {\tikz{\pgfuseplotmark{m2c};}};           
\node at (axis cs:46,-0.5)  {\small Second};

\node at (axis cs:40,+1.0) {\small\it 2{\nicefrac{2}{3}}}; \node at (axis cs:40,0.3)  {\tikz{\pgfuseplotmark{m3a};}};           
\node at (axis cs:38,+1.0) {\small\it 3};                  \node at (axis cs:38,0.3)  {\tikz{\pgfuseplotmark{m3b};}};           
\node at (axis cs:36,+1.0) {\small\it 3{\nicefrac{1}{3}}}; \node at (axis cs:36,0.3)  {\tikz{\pgfuseplotmark{m3c};}};           
\node at (axis cs:38,-0.5)  {\small Third};

\node at (axis cs:32,+1.0) {\small\it 3{\nicefrac{2}{3}}}; \node at (axis cs:32,0.3)  {\tikz{\pgfuseplotmark{m4a};}};           
\node at (axis cs:30,+1.0) {\small\it 4};                  \node at (axis cs:30,0.3)  {\tikz{\pgfuseplotmark{m4b};}};           
\node at (axis cs:28,+1.0) {\small\it 4{\nicefrac{1}{3}}}; \node at (axis cs:28,0.3)  {\tikz{\pgfuseplotmark{m4c};}};           
\node at (axis cs:30,-0.5)  {\small Fourth};                                                                      
                                                                                                                  
\node at (axis cs:24,+1.0) {\small\it 4{\nicefrac{2}{3}}}; \node at (axis cs:24,0.3)  {\tikz{\pgfuseplotmark{m5a};}};           
\node at (axis cs:22,+1.0) {\small\it 5};                  \node at (axis cs:22,0.3)  {\tikz{\pgfuseplotmark{m5b};}};           
\node at (axis cs:20,+1.0) {\small\it 5{\nicefrac{1}{3}}}; \node at (axis cs:20,0.3)  {\tikz{\pgfuseplotmark{m5c};}};           
\node at (axis cs:22,-0.5)  {\small Fifth};

\node at (axis cs:16,+1.0) {\small\it 5{\nicefrac{2}{3}}}; \node at (axis cs:16,0.3)  {\tikz{\pgfuseplotmark{m6a};}};           
\node at (axis cs:14,+1.0) {\small\it 6};                  \node at (axis cs:14,0.3)  {\tikz{\pgfuseplotmark{m6b};}};           
\node at (axis cs:12,+1.0) {\small\it 6{\nicefrac{1}{3}}}; \node at (axis cs:12,0.3)  {\tikz{\pgfuseplotmark{m6c};}};           
\node at (axis cs:14,-0.5)  {\small Sixth};                                                                       
                                                                                                                  
                                                                                                                 
\node at  (axis cs:6,+1.0)   {\small\it Variable};  \node at  (axis cs:6,+0.3)       {\tikz{\pgfuseplotmark{m2av};}};           
\node at  (axis cs:2,+.992)    {\small\it Binary};   \node at  (axis cs:2,+0.3)      {\tikz{\pgfuseplotmark{m2ab};}};           
\node at  (axis cs:-2,+1.0)   {\small\it Var. \& Bin.};  \node at  (axis cs:-2,+0.3) {\tikz{\pgfuseplotmark{m2avb};}};           


\node at  (axis cs:-17,+.992)   {\small\it Planetary};  \node at  (axis cs:-18,+0.3)  {\tikz{\pgfuseplotmark{PN};}};           
\node at  (axis cs:-22,+1.0)   {\small\it Emission};   \node at  (axis cs:-22,+0.3)   {\tikz{\pgfuseplotmark{EN};}};           
\node at  (axis cs:-27,+1.0)   {\small\it w/ Cluster};  \node at  (axis cs:-26,+0.3)  {\tikz{\pgfuseplotmark{CN};}};           
\node at (axis cs:-22,-0.5)  {\small Nebulae};


\node at  (axis cs:-32,+.994)   {\small\it Open};      \node at  (axis cs:-32,+0.3)   {\tikz{\pgfuseplotmark{OC};}};           
\node at  (axis cs:-37,+1.0)   {\small\it Globular};   \node at  (axis cs:-37,+0.3)   {\tikz{\pgfuseplotmark{GC};}};           
\node at (axis cs:-34.5,-0.5)  {\small Cluster};

\node at  (axis cs:-42,+.994)   {\small\it Galaxy};      \node at  (axis cs:-42,+0.3) {\tikz{\pgfuseplotmark{GAL};}};         

%
\end{axis}
         



% Constellations
\input{./input/III_Constellations.tex}

% Nebulae and nebulae names
\input{./input/III_Nebulae.tex}
% Stars and star names
\begin{axis}[name=stars,axis lines=none]

\node[stars] at (axis cs:{0.100},{26.918}) {\tikz\pgfuseplotmark{m6c};};
\node[stars] at (axis cs:{0.456},{-3.027}) {\tikz\pgfuseplotmark{m5b};};
\node[stars] at (axis cs:{0.490},{-6.014}) {\tikz\pgfuseplotmark{m4cv};};
\node[stars] at (axis cs:{0.542},{27.082}) {\tikz\pgfuseplotmark{m6av};};
\node[stars] at (axis cs:{0.583},{-29.720}) {\tikz\pgfuseplotmark{m5b};};
\node[stars] at (axis cs:{0.601},{8.957}) {\tikz\pgfuseplotmark{m6c};};
\node[stars] at (axis cs:{0.624},{8.485}) {\tikz\pgfuseplotmark{m6a};};
\node[stars] at (axis cs:{0.740},{-20.046}) {\tikz\pgfuseplotmark{m6c};};
\node[stars] at (axis cs:{0.783},{-24.145}) {\tikz\pgfuseplotmark{m6c};};
\node[stars] at (axis cs:{0.935},{-17.336}) {\tikz\pgfuseplotmark{m5a};};
\node[stars] at (axis cs:{1.082},{-16.529}) {\tikz\pgfuseplotmark{m6a};};
\node[stars] at (axis cs:{1.085},{-29.269}) {\tikz\pgfuseplotmark{m6c};};
\node[stars] at (axis cs:{1.125},{-10.509}) {\tikz\pgfuseplotmark{m5bv};};
\node[stars] at (axis cs:{1.224},{34.660}) {\tikz\pgfuseplotmark{m6b};};
\node[stars] at (axis cs:{1.233},{26.649}) {\tikz\pgfuseplotmark{m6c};};
\node[stars] at (axis cs:{1.255},{27.675}) {\tikz\pgfuseplotmark{m6c};};
\node[stars] at (axis cs:{1.266},{-0.503}) {\tikz\pgfuseplotmark{m6c};};
\node[stars] at (axis cs:{1.334},{-5.707}) {\tikz\pgfuseplotmark{m5av};};
\node[stars] at (axis cs:{1.425},{13.396}) {\tikz\pgfuseplotmark{m6a};};
\node[stars] at (axis cs:{1.653},{29.021}) {\tikz\pgfuseplotmark{m6bvb};};
\node[stars] at (axis cs:{1.709},{-23.107}) {\tikz\pgfuseplotmark{m6c};};
\node[stars] at (axis cs:{1.826},{-17.387}) {\tikz\pgfuseplotmark{m6c};};
\node[stars] at (axis cs:{1.934},{-2.548}) {\tikz\pgfuseplotmark{m6cv};};
\node[stars] at (axis cs:{1.945},{-22.508}) {\tikz\pgfuseplotmark{m6b};};
\node[stars] at (axis cs:{2.015},{-33.529}) {\tikz\pgfuseplotmark{m6a};};
\node[stars] at (axis cs:{2.050},{-2.447}) {\tikz\pgfuseplotmark{m6cv};};
\node[stars] at (axis cs:{2.073},{-8.824}) {\tikz\pgfuseplotmark{m6b};};
\node[stars] at (axis cs:{2.097},{29.090}) {\tikz\pgfuseplotmark{m2bv};};
\node[stars] at (axis cs:{2.140},{-17.578}) {\tikz\pgfuseplotmark{m6bv};};
\node[stars] at (axis cs:{2.171},{36.627}) {\tikz\pgfuseplotmark{m6c};};
\node[stars] at (axis cs:{2.217},{25.463}) {\tikz\pgfuseplotmark{m6c};};
\node[stars] at (axis cs:{2.251},{28.247}) {\tikz\pgfuseplotmark{m6c};};
\node[stars] at (axis cs:{2.260},{18.212}) {\tikz\pgfuseplotmark{m6a};};
\node[stars] at (axis cs:{2.338},{-27.988}) {\tikz\pgfuseplotmark{m5cb};};
\node[stars] at (axis cs:{2.509},{11.146}) {\tikz\pgfuseplotmark{m6avb};};
\node[stars] at (axis cs:{2.579},{-5.248}) {\tikz\pgfuseplotmark{m6a};};
\node[stars] at (axis cs:{2.678},{-12.580}) {\tikz\pgfuseplotmark{m6b};};
\node[stars] at (axis cs:{2.816},{-15.468}) {\tikz\pgfuseplotmark{m5b};};
\node[stars] at (axis cs:{2.893},{-27.799}) {\tikz\pgfuseplotmark{m5c};};
\node[stars] at (axis cs:{2.933},{-35.133}) {\tikz\pgfuseplotmark{m5c};};
\node[stars] at (axis cs:{3.042},{-17.938}) {\tikz\pgfuseplotmark{m5c};};
\node[stars] at (axis cs:{3.309},{15.184}) {\tikz\pgfuseplotmark{m3avb};};
\node[stars] at (axis cs:{3.350},{26.987}) {\tikz\pgfuseplotmark{m6c};};
\node[stars] at (axis cs:{3.426},{-26.022}) {\tikz\pgfuseplotmark{m6bv};};
\node[stars] at (axis cs:{3.435},{-26.285}) {\tikz\pgfuseplotmark{m6b};};
\node[stars] at (axis cs:{3.510},{33.206}) {\tikz\pgfuseplotmark{m6c};};
\node[stars] at (axis cs:{3.615},{-7.780}) {\tikz\pgfuseplotmark{m5cv};};
\node[stars] at (axis cs:{3.651},{20.207}) {\tikz\pgfuseplotmark{m5av};};
\node[stars] at (axis cs:{3.660},{-18.933}) {\tikz\pgfuseplotmark{m4cv};};
\node[stars] at (axis cs:{3.727},{-9.570}) {\tikz\pgfuseplotmark{m6a};};
\node[stars] at (axis cs:{3.733},{22.284}) {\tikz\pgfuseplotmark{m6cv};};
\node[stars] at (axis cs:{3.743},{-34.904}) {\tikz\pgfuseplotmark{m6c};};
\node[stars] at (axis cs:{3.745},{8.821}) {\tikz\pgfuseplotmark{m6bvb};};
\node[stars] at (axis cs:{3.779},{31.536}) {\tikz\pgfuseplotmark{m6c};};
\node[stars] at (axis cs:{3.794},{27.283}) {\tikz\pgfuseplotmark{m6c};};
\node[stars] at (axis cs:{4.037},{-31.446}) {\tikz\pgfuseplotmark{m6a};};
\node[stars] at (axis cs:{4.142},{8.240}) {\tikz\pgfuseplotmark{m6b};};
\node[stars] at (axis cs:{4.273},{38.681}) {\tikz\pgfuseplotmark{m5av};};
\node[stars] at (axis cs:{4.386},{-19.051}) {\tikz\pgfuseplotmark{m6c};};
\node[stars] at (axis cs:{4.449},{1.689}) {\tikz\pgfuseplotmark{m6c};};
\node[stars] at (axis cs:{4.572},{11.206}) {\tikz\pgfuseplotmark{m6b};};
\node[stars] at (axis cs:{4.582},{36.785}) {\tikz\pgfuseplotmark{m5av};};
\node[stars] at (axis cs:{4.659},{31.517}) {\tikz\pgfuseplotmark{m6b};};
\node[stars] at (axis cs:{4.674},{-8.053}) {\tikz\pgfuseplotmark{m6c};};
\node[stars] at (axis cs:{4.857},{-8.824}) {\tikz\pgfuseplotmark{m4av};};
\node[stars] at (axis cs:{5.102},{30.936}) {\tikz\pgfuseplotmark{m6b};};
\node[stars] at (axis cs:{5.149},{8.190}) {\tikz\pgfuseplotmark{m5c};};
\node[stars] at (axis cs:{5.190},{32.911}) {\tikz\pgfuseplotmark{m6a};};
\node[stars] at (axis cs:{5.280},{37.969}) {\tikz\pgfuseplotmark{m5c};};
\node[stars] at (axis cs:{5.380},{-28.981}) {\tikz\pgfuseplotmark{m5c};};
\node[stars] at (axis cs:{5.443},{-20.058}) {\tikz\pgfuseplotmark{m6av};};
\node[stars] at (axis cs:{5.606},{13.482}) {\tikz\pgfuseplotmark{m6c};};
\node[stars] at (axis cs:{5.716},{-12.209}) {\tikz\pgfuseplotmark{m6cv};};
\node[stars] at (axis cs:{5.768},{-15.942}) {\tikz\pgfuseplotmark{m6c};};
\node[stars] at (axis cs:{6.124},{-2.219}) {\tikz\pgfuseplotmark{m6b};};
\node[stars] at (axis cs:{6.159},{14.315}) {\tikz\pgfuseplotmark{m6c};};
\node[stars] at (axis cs:{6.351},{1.940}) {\tikz\pgfuseplotmark{m6av};};
\node[stars] at (axis cs:{6.591},{-18.693}) {\tikz\pgfuseplotmark{m6cv};};
\node[stars] at (axis cs:{6.656},{-0.049}) {\tikz\pgfuseplotmark{m6c};};
\node[stars] at (axis cs:{6.811},{-25.547}) {\tikz\pgfuseplotmark{m6b};};
\node[stars] at (axis cs:{6.879},{16.025}) {\tikz\pgfuseplotmark{m6cb};};
\node[stars] at (axis cs:{6.982},{-33.007}) {\tikz\pgfuseplotmark{m5bv};};
\node[stars] at (axis cs:{6.994},{19.514}) {\tikz\pgfuseplotmark{m6c};};
\node[stars] at (axis cs:{7.012},{17.893}) {\tikz\pgfuseplotmark{m5bv};};
\node[stars] at (axis cs:{7.053},{16.445}) {\tikz\pgfuseplotmark{m6b};};
\node[stars] at (axis cs:{7.084},{10.190}) {\tikz\pgfuseplotmark{m6b};};
\node[stars] at (axis cs:{7.088},{-20.335}) {\tikz\pgfuseplotmark{m6c};};
\node[stars] at (axis cs:{7.111},{-39.915}) {\tikz\pgfuseplotmark{m5c};};
\node[stars] at (axis cs:{7.236},{36.899}) {\tikz\pgfuseplotmark{m6c};};
\node[stars] at (axis cs:{7.466},{-14.864}) {\tikz\pgfuseplotmark{m6c};};
\node[stars] at (axis cs:{7.510},{-3.957}) {\tikz\pgfuseplotmark{m6a};};
\node[stars] at (axis cs:{7.531},{29.752}) {\tikz\pgfuseplotmark{m5cv};};
\node[stars] at (axis cs:{7.594},{-23.788}) {\tikz\pgfuseplotmark{m5c};};
\node[stars] at (axis cs:{7.857},{33.581}) {\tikz\pgfuseplotmark{m6bb};};
\node[stars] at (axis cs:{8.099},{6.955}) {\tikz\pgfuseplotmark{m6avb};};
\node[stars] at (axis cs:{8.148},{20.294}) {\tikz\pgfuseplotmark{m5c};};
\node[stars] at (axis cs:{8.205},{28.280}) {\tikz\pgfuseplotmark{m6c};};
\node[stars] at (axis cs:{8.421},{-29.558}) {\tikz\pgfuseplotmark{m6a};};
\node[stars] at (axis cs:{8.731},{13.371}) {\tikz\pgfuseplotmark{m6c};};
\node[stars] at (axis cs:{8.812},{-3.593}) {\tikz\pgfuseplotmark{m5cv};};
\node[stars] at (axis cs:{8.887},{-0.506}) {\tikz\pgfuseplotmark{m6b};};
\node[stars] at (axis cs:{8.978},{13.207}) {\tikz\pgfuseplotmark{m6c};};
\node[stars] at (axis cs:{9.013},{-14.973}) {\tikz\pgfuseplotmark{m6c};};
\node[stars] at (axis cs:{9.029},{-22.842}) {\tikz\pgfuseplotmark{m6bv};};
\node[stars] at (axis cs:{9.197},{15.231}) {\tikz\pgfuseplotmark{m6bv};};
\node[stars] at (axis cs:{9.220},{33.719}) {\tikz\pgfuseplotmark{m4cvb};};
\node[stars] at (axis cs:{9.280},{24.014}) {\tikz\pgfuseplotmark{m6c};};
\node[stars] at (axis cs:{9.336},{-24.767}) {\tikz\pgfuseplotmark{m6a};};
\node[stars] at (axis cs:{9.338},{35.399}) {\tikz\pgfuseplotmark{m5c};};
\node[stars] at (axis cs:{9.377},{3.135}) {\tikz\pgfuseplotmark{m6c};};
\node[stars] at (axis cs:{9.639},{29.312}) {\tikz\pgfuseplotmark{m4c};};
\node[stars] at (axis cs:{9.703},{-25.108}) {\tikz\pgfuseplotmark{m6cb};};
\node[stars] at (axis cs:{9.832},{30.861}) {\tikz\pgfuseplotmark{m3c};};
\node[stars] at (axis cs:{9.841},{21.250}) {\tikz\pgfuseplotmark{m6bv};};
\node[stars] at (axis cs:{9.982},{21.438}) {\tikz\pgfuseplotmark{m5cb};};
\node[stars] at (axis cs:{10.119},{-16.517}) {\tikz\pgfuseplotmark{m6cb};};
\node[stars] at (axis cs:{10.137},{-23.805}) {\tikz\pgfuseplotmark{m6c};};
\node[stars] at (axis cs:{10.177},{-4.352}) {\tikz\pgfuseplotmark{m6bb};};
\node[stars] at (axis cs:{10.280},{39.459}) {\tikz\pgfuseplotmark{m5c};};
\node[stars] at (axis cs:{10.400},{24.629}) {\tikz\pgfuseplotmark{m6bv};};
\node[stars] at (axis cs:{10.679},{-38.463}) {\tikz\pgfuseplotmark{m6b};};
\node[stars] at (axis cs:{10.897},{-17.987}) {\tikz\pgfuseplotmark{m2bv};};
\node[stars] at (axis cs:{10.959},{-12.012}) {\tikz\pgfuseplotmark{m6b};};
\node[stars] at (axis cs:{11.048},{-10.609}) {\tikz\pgfuseplotmark{m5av};};
\node[stars] at (axis cs:{11.050},{-38.422}) {\tikz\pgfuseplotmark{m6b};};
\node[stars] at (axis cs:{11.185},{-22.006}) {\tikz\pgfuseplotmark{m5c};};
\node[stars] at (axis cs:{11.351},{-4.629}) {\tikz\pgfuseplotmark{m6c};};
\node[stars] at (axis cs:{11.370},{-12.881}) {\tikz\pgfuseplotmark{m6c};};
\node[stars] at (axis cs:{11.549},{-22.522}) {\tikz\pgfuseplotmark{m5c};};
\node[stars] at (axis cs:{11.637},{15.475}) {\tikz\pgfuseplotmark{m5cv};};
\node[stars] at (axis cs:{11.756},{11.974}) {\tikz\pgfuseplotmark{m6a};};
\node[stars] at (axis cs:{11.807},{19.579}) {\tikz\pgfuseplotmark{m6bv};};
\node[stars] at (axis cs:{11.835},{24.267}) {\tikz\pgfuseplotmark{m4bv};};
\node[stars] at (axis cs:{11.848},{6.741}) {\tikz\pgfuseplotmark{m6b};};
\node[stars] at (axis cs:{11.930},{-18.061}) {\tikz\pgfuseplotmark{m6a};};
\node[stars] at (axis cs:{12.004},{-21.722}) {\tikz\pgfuseplotmark{m6a};};
\node[stars] at (axis cs:{12.073},{7.300}) {\tikz\pgfuseplotmark{m6b};};
\node[stars] at (axis cs:{12.096},{5.280}) {\tikz\pgfuseplotmark{m6ab};};
\node[stars] at (axis cs:{12.171},{7.585}) {\tikz\pgfuseplotmark{m4c};};
\node[stars] at (axis cs:{12.245},{16.941}) {\tikz\pgfuseplotmark{m5b};};
\node[stars] at (axis cs:{12.308},{-24.137}) {\tikz\pgfuseplotmark{m6b};};
\node[stars] at (axis cs:{12.357},{-13.561}) {\tikz\pgfuseplotmark{m6a};};
\node[stars] at (axis cs:{12.389},{-23.362}) {\tikz\pgfuseplotmark{m6c};};
\node[stars] at (axis cs:{12.472},{27.710}) {\tikz\pgfuseplotmark{m6ab};};
\node[stars] at (axis cs:{12.532},{-10.644}) {\tikz\pgfuseplotmark{m5cv};};
\node[stars] at (axis cs:{12.826},{3.385}) {\tikz\pgfuseplotmark{m6c};};
\node[stars] at (axis cs:{13.169},{-24.006}) {\tikz\pgfuseplotmark{m5c};};
\node[stars] at (axis cs:{13.252},{-1.144}) {\tikz\pgfuseplotmark{m5a};};
\node[stars] at (axis cs:{13.368},{37.418}) {\tikz\pgfuseplotmark{m6b};};
\node[stars] at (axis cs:{13.573},{-8.741}) {\tikz\pgfuseplotmark{m6c};};
\node[stars] at (axis cs:{13.647},{19.188}) {\tikz\pgfuseplotmark{m6a};};
\node[stars] at (axis cs:{13.742},{23.628}) {\tikz\pgfuseplotmark{m5cvb};};
\node[stars] at (axis cs:{13.811},{24.557}) {\tikz\pgfuseplotmark{m6cv};};
\node[stars] at (axis cs:{13.927},{-7.347}) {\tikz\pgfuseplotmark{m6b};};
\node[stars] at (axis cs:{13.981},{-27.776}) {\tikz\pgfuseplotmark{m6b};};
\node[stars] at (axis cs:{13.994},{27.209}) {\tikz\pgfuseplotmark{m6b};};
\node[stars] at (axis cs:{14.006},{-11.266}) {\tikz\pgfuseplotmark{m5c};};
\node[stars] at (axis cs:{14.038},{13.952}) {\tikz\pgfuseplotmark{m6c};};
\node[stars] at (axis cs:{14.188},{38.499}) {\tikz\pgfuseplotmark{m4b};};
\node[stars] at (axis cs:{14.302},{23.418}) {\tikz\pgfuseplotmark{m4c};};
\node[stars] at (axis cs:{14.459},{28.992}) {\tikz\pgfuseplotmark{m5c};};
\node[stars] at (axis cs:{14.477},{13.696}) {\tikz\pgfuseplotmark{m6c};};
\node[stars] at (axis cs:{14.559},{33.951}) {\tikz\pgfuseplotmark{m6b};};
\node[stars] at (axis cs:{14.579},{21.404}) {\tikz\pgfuseplotmark{m6c};};
\node[stars] at (axis cs:{14.652},{-29.357}) {\tikz\pgfuseplotmark{m4cv};};
\node[stars] at (axis cs:{14.683},{-11.380}) {\tikz\pgfuseplotmark{m6a};};
\node[stars] at (axis cs:{14.957},{6.483}) {\tikz\pgfuseplotmark{m6cv};};
\node[stars] at (axis cs:{15.326},{-38.916}) {\tikz\pgfuseplotmark{m6a};};
\node[stars] at (axis cs:{15.411},{-16.265}) {\tikz\pgfuseplotmark{m6c};};
\node[stars] at (axis cs:{15.610},{-31.552}) {\tikz\pgfuseplotmark{m6av};};
\node[stars] at (axis cs:{15.705},{31.804}) {\tikz\pgfuseplotmark{m5c};};
\node[stars] at (axis cs:{15.736},{7.890}) {\tikz\pgfuseplotmark{m4c};};
\node[stars] at (axis cs:{15.761},{-4.837}) {\tikz\pgfuseplotmark{m5c};};
\node[stars] at (axis cs:{15.824},{-29.526}) {\tikz\pgfuseplotmark{m6c};};
\node[stars] at (axis cs:{15.954},{1.367}) {\tikz\pgfuseplotmark{m6bb};};
\node[stars] at (axis cs:{16.115},{29.659}) {\tikz\pgfuseplotmark{m6c};};
\node[stars] at (axis cs:{16.136},{-33.533}) {\tikz\pgfuseplotmark{m6cb};};
\node[stars] at (axis cs:{16.219},{5.656}) {\tikz\pgfuseplotmark{m6b};};
\node[stars] at (axis cs:{16.272},{14.946}) {\tikz\pgfuseplotmark{m6a};};
\node[stars] at (axis cs:{16.404},{-9.979}) {\tikz\pgfuseplotmark{m6b};};
\node[stars] at (axis cs:{16.421},{21.473}) {\tikz\pgfuseplotmark{m5cvb};};
\node[stars] at (axis cs:{16.455},{4.908}) {\tikz\pgfuseplotmark{m6cb};};
\node[stars] at (axis cs:{16.521},{-9.839}) {\tikz\pgfuseplotmark{m6a};};
\node[stars] at (axis cs:{16.532},{-23.992}) {\tikz\pgfuseplotmark{m6b};};
\node[stars] at (axis cs:{16.547},{32.181}) {\tikz\pgfuseplotmark{m6c};};
\node[stars] at (axis cs:{16.640},{12.956}) {\tikz\pgfuseplotmark{m6c};};
\node[stars] at (axis cs:{16.804},{-23.996}) {\tikz\pgfuseplotmark{m6c};};
\node[stars] at (axis cs:{16.943},{-9.786}) {\tikz\pgfuseplotmark{m6a};};
\node[stars] at (axis cs:{16.988},{20.739}) {\tikz\pgfuseplotmark{m6a};};
\node[stars] at (axis cs:{16.999},{1.993}) {\tikz\pgfuseplotmark{m6c};};
\node[stars] at (axis cs:{17.006},{32.012}) {\tikz\pgfuseplotmark{m6c};};
\node[stars] at (axis cs:{17.092},{5.650}) {\tikz\pgfuseplotmark{m6av};};
\node[stars] at (axis cs:{17.147},{-10.182}) {\tikz\pgfuseplotmark{m3c};};
\node[stars] at (axis cs:{17.433},{35.620}) {\tikz\pgfuseplotmark{m2bvb};};
\node[stars] at (axis cs:{17.455},{19.658}) {\tikz\pgfuseplotmark{m6a};};
\node[stars] at (axis cs:{17.548},{15.674}) {\tikz\pgfuseplotmark{m6b};};
\node[stars] at (axis cs:{17.550},{-8.906}) {\tikz\pgfuseplotmark{m6c};};
\node[stars] at (axis cs:{17.581},{25.458}) {\tikz\pgfuseplotmark{m6a};};
\node[stars] at (axis cs:{17.640},{2.445}) {\tikz\pgfuseplotmark{m6bv};};
\node[stars] at (axis cs:{17.778},{31.425}) {\tikz\pgfuseplotmark{m5c};};
\node[stars] at (axis cs:{17.793},{37.724}) {\tikz\pgfuseplotmark{m6a};};
\node[stars] at (axis cs:{17.863},{21.035}) {\tikz\pgfuseplotmark{m5a};};
\node[stars] at (axis cs:{17.915},{30.090}) {\tikz\pgfuseplotmark{m5a};};
\node[stars] at (axis cs:{17.931},{-2.251}) {\tikz\pgfuseplotmark{m6b};};
\node[stars] at (axis cs:{18.189},{-37.856}) {\tikz\pgfuseplotmark{m6bv};};
\node[stars] at (axis cs:{18.248},{30.064}) {\tikz\pgfuseplotmark{m6c};};
\node[stars] at (axis cs:{18.433},{7.575}) {\tikz\pgfuseplotmark{m5cb};};
\node[stars] at (axis cs:{18.437},{24.584}) {\tikz\pgfuseplotmark{m5a};};
\node[stars] at (axis cs:{18.520},{28.529}) {\tikz\pgfuseplotmark{m6cv};};
\node[stars] at (axis cs:{18.532},{16.133}) {\tikz\pgfuseplotmark{m6bv};};
\node[stars] at (axis cs:{18.600},{-7.923}) {\tikz\pgfuseplotmark{m5cvb};};
\node[stars] at (axis cs:{18.677},{6.995}) {\tikz\pgfuseplotmark{m6b};};
\node[stars] at (axis cs:{18.705},{-0.974}) {\tikz\pgfuseplotmark{m6a};};
\node[stars] at (axis cs:{19.079},{33.114}) {\tikz\pgfuseplotmark{m6b};};
\node[stars] at (axis cs:{19.151},{-2.500}) {\tikz\pgfuseplotmark{m5cv};};
\node[stars] at (axis cs:{19.350},{31.744}) {\tikz\pgfuseplotmark{m6c};};
\node[stars] at (axis cs:{19.450},{3.614}) {\tikz\pgfuseplotmark{m5c};};
\node[stars] at (axis cs:{19.696},{37.386}) {\tikz\pgfuseplotmark{m6cb};};
\node[stars] at (axis cs:{19.867},{27.264}) {\tikz\pgfuseplotmark{m5a};};
\node[stars] at (axis cs:{19.951},{-0.509}) {\tikz\pgfuseplotmark{m6cv};};
\node[stars] at (axis cs:{20.116},{-11.239}) {\tikz\pgfuseplotmark{m6c};};
\node[stars] at (axis cs:{20.144},{-3.247}) {\tikz\pgfuseplotmark{m6c};};
\node[stars] at (axis cs:{20.281},{28.738}) {\tikz\pgfuseplotmark{m5c};};
\node[stars] at (axis cs:{20.627},{-19.081}) {\tikz\pgfuseplotmark{m6cb};};
\node[stars] at (axis cs:{20.645},{-0.449}) {\tikz\pgfuseplotmark{m6cv};};
\node[stars] at (axis cs:{20.654},{1.726}) {\tikz\pgfuseplotmark{m6c};};
\node[stars] at (axis cs:{20.854},{20.469}) {\tikz\pgfuseplotmark{m6b};};
\node[stars] at (axis cs:{20.879},{-30.945}) {\tikz\pgfuseplotmark{m6a};};
\node[stars] at (axis cs:{20.906},{34.246}) {\tikz\pgfuseplotmark{m6c};};
\node[stars] at (axis cs:{20.919},{37.715}) {\tikz\pgfuseplotmark{m6a};};
\node[stars] at (axis cs:{21.006},{-8.183}) {\tikz\pgfuseplotmark{m4a};};
\node[stars] at (axis cs:{21.011},{-8.007}) {\tikz\pgfuseplotmark{m6cv};};
\node[stars] at (axis cs:{21.085},{-6.914}) {\tikz\pgfuseplotmark{m6b};};
\node[stars] at (axis cs:{21.166},{-15.660}) {\tikz\pgfuseplotmark{m6c};};
\node[stars] at (axis cs:{21.203},{-2.848}) {\tikz\pgfuseplotmark{m6c};};
\node[stars] at (axis cs:{21.399},{23.511}) {\tikz\pgfuseplotmark{m6c};};
\node[stars] at (axis cs:{21.405},{-14.599}) {\tikz\pgfuseplotmark{m5b};};
\node[stars] at (axis cs:{21.468},{-3.927}) {\tikz\pgfuseplotmark{m6cv};};
\node[stars] at (axis cs:{21.537},{34.579}) {\tikz\pgfuseplotmark{m6c};};
\node[stars] at (axis cs:{21.564},{19.172}) {\tikz\pgfuseplotmark{m5cv};};
\node[stars] at (axis cs:{21.598},{20.071}) {\tikz\pgfuseplotmark{m6c};};
\node[stars] at (axis cs:{21.614},{-0.399}) {\tikz\pgfuseplotmark{m6cv};};
\node[stars] at (axis cs:{21.674},{19.240}) {\tikz\pgfuseplotmark{m5c};};
\node[stars] at (axis cs:{21.715},{-13.056}) {\tikz\pgfuseplotmark{m6av};};
\node[stars] at (axis cs:{21.776},{34.377}) {\tikz\pgfuseplotmark{m6c};};
\node[stars] at (axis cs:{21.847},{-22.338}) {\tikz\pgfuseplotmark{m6c};};
\node[stars] at (axis cs:{21.944},{-10.902}) {\tikz\pgfuseplotmark{m6cb};};
\node[stars] at (axis cs:{22.095},{7.961}) {\tikz\pgfuseplotmark{m6cvb};};
\node[stars] at (axis cs:{22.401},{-21.629}) {\tikz\pgfuseplotmark{m5b};};
\node[stars] at (axis cs:{22.428},{-25.618}) {\tikz\pgfuseplotmark{m6c};};
\node[stars] at (axis cs:{22.470},{18.356}) {\tikz\pgfuseplotmark{m6bv};};
\node[stars] at (axis cs:{22.546},{6.144}) {\tikz\pgfuseplotmark{m5ab};};
\node[stars] at (axis cs:{22.595},{-26.208}) {\tikz\pgfuseplotmark{m6b};};
\node[stars] at (axis cs:{22.871},{15.346}) {\tikz\pgfuseplotmark{m4av};};
\node[stars] at (axis cs:{22.930},{-30.283}) {\tikz\pgfuseplotmark{m6a};};
\node[stars] at (axis cs:{23.032},{34.800}) {\tikz\pgfuseplotmark{m6c};};
\node[stars] at (axis cs:{23.234},{-36.865}) {\tikz\pgfuseplotmark{m5c};};
\node[stars] at (axis cs:{23.326},{8.209}) {\tikz\pgfuseplotmark{m6c};};
\node[stars] at (axis cs:{23.428},{-7.025}) {\tikz\pgfuseplotmark{m6a};};
\node[stars] at (axis cs:{23.569},{37.237}) {\tikz\pgfuseplotmark{m6bv};};
\node[stars] at (axis cs:{23.657},{-15.676}) {\tikz\pgfuseplotmark{m6a};};
\node[stars] at (axis cs:{23.704},{18.460}) {\tikz\pgfuseplotmark{m6b};};
\node[stars] at (axis cs:{23.712},{-31.892}) {\tikz\pgfuseplotmark{m6bv};};
\node[stars] at (axis cs:{23.714},{-3.524}) {\tikz\pgfuseplotmark{m6c};};
\node[stars] at (axis cs:{23.715},{-23.699}) {\tikz\pgfuseplotmark{m6c};};
\node[stars] at (axis cs:{23.944},{14.661}) {\tikz\pgfuseplotmark{m6c};};
\node[stars] at (axis cs:{23.961},{-39.947}) {\tikz\pgfuseplotmark{m6c};};
\node[stars] at (axis cs:{23.978},{17.434}) {\tikz\pgfuseplotmark{m6b};};
\node[stars] at (axis cs:{23.996},{-15.400}) {\tikz\pgfuseplotmark{m5c};};
\node[stars] at (axis cs:{24.035},{-29.907}) {\tikz\pgfuseplotmark{m6a};};
\node[stars] at (axis cs:{24.181},{7.831}) {\tikz\pgfuseplotmark{m6cv};};
\node[stars] at (axis cs:{24.275},{12.141}) {\tikz\pgfuseplotmark{m6a};};
\node[stars] at (axis cs:{24.407},{-9.404}) {\tikz\pgfuseplotmark{m6c};};
\node[stars] at (axis cs:{24.467},{-3.441}) {\tikz\pgfuseplotmark{m6c};};
\node[stars] at (axis cs:{24.615},{-36.528}) {\tikz\pgfuseplotmark{m6b};};
\node[stars] at (axis cs:{24.716},{-21.275}) {\tikz\pgfuseplotmark{m6a};};
\node[stars] at (axis cs:{24.920},{16.406}) {\tikz\pgfuseplotmark{m6b};};
\node[stars] at (axis cs:{25.146},{8.761}) {\tikz\pgfuseplotmark{m6c};};
\node[stars] at (axis cs:{25.327},{25.746}) {\tikz\pgfuseplotmark{m6c};};
\node[stars] at (axis cs:{25.358},{5.488}) {\tikz\pgfuseplotmark{m4c};};
\node[stars] at (axis cs:{25.364},{-38.133}) {\tikz\pgfuseplotmark{m6c};};
\node[stars] at (axis cs:{25.413},{30.047}) {\tikz\pgfuseplotmark{m6b};};
\node[stars] at (axis cs:{25.437},{-11.325}) {\tikz\pgfuseplotmark{m6a};};
\node[stars] at (axis cs:{25.512},{-36.832}) {\tikz\pgfuseplotmark{m6a};};
\node[stars] at (axis cs:{25.515},{35.246}) {\tikz\pgfuseplotmark{m6a};};
\node[stars] at (axis cs:{25.536},{-32.327}) {\tikz\pgfuseplotmark{m5c};};
\node[stars] at (axis cs:{25.624},{20.268}) {\tikz\pgfuseplotmark{m5cv};};
\node[stars] at (axis cs:{25.681},{-3.690}) {\tikz\pgfuseplotmark{m5b};};
\node[stars] at (axis cs:{25.959},{32.192}) {\tikz\pgfuseplotmark{m6c};};
\node[stars] at (axis cs:{25.978},{-4.765}) {\tikz\pgfuseplotmark{m6c};};
\node[stars] at (axis cs:{26.017},{-15.937}) {\tikz\pgfuseplotmark{m3c};};
\node[stars] at (axis cs:{26.182},{-6.766}) {\tikz\pgfuseplotmark{m6cb};};
\node[stars] at (axis cs:{26.233},{20.083}) {\tikz\pgfuseplotmark{m6c};};
\node[stars] at (axis cs:{26.348},{9.158}) {\tikz\pgfuseplotmark{m4cv};};
\node[stars] at (axis cs:{26.411},{-25.053}) {\tikz\pgfuseplotmark{m5cb};};
\node[stars] at (axis cs:{26.497},{-5.733}) {\tikz\pgfuseplotmark{m5c};};
\node[stars] at (axis cs:{26.504},{-27.349}) {\tikz\pgfuseplotmark{m6c};};
\node[stars] at (axis cs:{26.949},{-37.159}) {\tikz\pgfuseplotmark{m6c};};
\node[stars] at (axis cs:{27.046},{16.955}) {\tikz\pgfuseplotmark{m6a};};
\node[stars] at (axis cs:{27.108},{3.685}) {\tikz\pgfuseplotmark{m6b};};
\node[stars] at (axis cs:{27.162},{37.953}) {\tikz\pgfuseplotmark{m6b};};
\node[stars] at (axis cs:{27.173},{32.690}) {\tikz\pgfuseplotmark{m6a};};
\node[stars] at (axis cs:{27.331},{-31.072}) {\tikz\pgfuseplotmark{m6cv};};
\node[stars] at (axis cs:{27.396},{-10.686}) {\tikz\pgfuseplotmark{m5ab};};
\node[stars] at (axis cs:{27.454},{-38.403}) {\tikz\pgfuseplotmark{m6c};};
\node[stars] at (axis cs:{27.536},{22.275}) {\tikz\pgfuseplotmark{m6ab};};
\node[stars] at (axis cs:{27.717},{11.043}) {\tikz\pgfuseplotmark{m6bv};};
\node[stars] at (axis cs:{27.860},{-39.836}) {\tikz\pgfuseplotmark{m6c};};
\node[stars] at (axis cs:{27.865},{-10.335}) {\tikz\pgfuseplotmark{m4av};};
\node[stars] at (axis cs:{27.908},{-6.874}) {\tikz\pgfuseplotmark{m6c};};
\node[stars] at (axis cs:{27.942},{-15.647}) {\tikz\pgfuseplotmark{m6c};};
\node[stars] at (axis cs:{28.217},{-16.929}) {\tikz\pgfuseplotmark{m6av};};
\node[stars] at (axis cs:{28.270},{29.579}) {\tikz\pgfuseplotmark{m3cvb};};
\node[stars] at (axis cs:{28.347},{-38.594}) {\tikz\pgfuseplotmark{m6b};};
\node[stars] at (axis cs:{28.383},{19.294}) {\tikz\pgfuseplotmark{m4bvb};};
\node[stars] at (axis cs:{28.389},{3.187}) {\tikz\pgfuseplotmark{m5av};};
\node[stars] at (axis cs:{28.660},{20.808}) {\tikz\pgfuseplotmark{m3av};};
\node[stars] at (axis cs:{28.740},{37.128}) {\tikz\pgfuseplotmark{m6c};};
\node[stars] at (axis cs:{28.963},{23.577}) {\tikz\pgfuseplotmark{m6av};};
\node[stars] at (axis cs:{28.974},{1.850}) {\tikz\pgfuseplotmark{m6b};};
\node[stars] at (axis cs:{29.039},{37.252}) {\tikz\pgfuseplotmark{m6ab};};
\node[stars] at (axis cs:{29.167},{-22.527}) {\tikz\pgfuseplotmark{m5b};};
\node[stars] at (axis cs:{29.290},{-10.242}) {\tikz\pgfuseplotmark{m6c};};
\node[stars] at (axis cs:{29.338},{17.818}) {\tikz\pgfuseplotmark{m5b};};
\node[stars] at (axis cs:{29.432},{27.804}) {\tikz\pgfuseplotmark{m6a};};
\node[stars] at (axis cs:{29.482},{23.596}) {\tikz\pgfuseplotmark{m5ab};};
\node[stars] at (axis cs:{29.611},{-33.067}) {\tikz\pgfuseplotmark{m6c};};
\node[stars] at (axis cs:{29.692},{-11.301}) {\tikz\pgfuseplotmark{m6c};};
\node[stars] at (axis cs:{29.858},{12.295}) {\tikz\pgfuseplotmark{m6b};};
\node[stars] at (axis cs:{29.942},{-20.824}) {\tikz\pgfuseplotmark{m5c};};
\node[stars] at (axis cs:{30.001},{-21.078}) {\tikz\pgfuseplotmark{m4b};};
\node[stars] at (axis cs:{30.038},{3.097}) {\tikz\pgfuseplotmark{m6b};};
\node[stars] at (axis cs:{30.112},{-8.524}) {\tikz\pgfuseplotmark{m6avb};};
\node[stars] at (axis cs:{30.311},{-30.002}) {\tikz\pgfuseplotmark{m5c};};
\node[stars] at (axis cs:{30.512},{2.764}) {\tikz\pgfuseplotmark{m4avb};};
\node[stars] at (axis cs:{30.618},{-29.665}) {\tikz\pgfuseplotmark{m6c};};
\node[stars] at (axis cs:{30.646},{13.477}) {\tikz\pgfuseplotmark{m6bv};};
\node[stars] at (axis cs:{30.716},{-23.887}) {\tikz\pgfuseplotmark{m6c};};
\node[stars] at (axis cs:{30.741},{33.284}) {\tikz\pgfuseplotmark{m6a};};
\node[stars] at (axis cs:{30.744},{-15.306}) {\tikz\pgfuseplotmark{m6a};};
\node[stars] at (axis cs:{30.799},{0.128}) {\tikz\pgfuseplotmark{m5cv};};
\node[stars] at (axis cs:{30.914},{25.935}) {\tikz\pgfuseplotmark{m6ab};};
\node[stars] at (axis cs:{30.919},{-4.103}) {\tikz\pgfuseplotmark{m6a};};
\node[stars] at (axis cs:{30.928},{18.253}) {\tikz\pgfuseplotmark{m6c};};
\node[stars] at (axis cs:{30.951},{-0.340}) {\tikz\pgfuseplotmark{m6bv};};
\node[stars] at (axis cs:{31.052},{-11.860}) {\tikz\pgfuseplotmark{m6cv};};
\node[stars] at (axis cs:{31.123},{-29.297}) {\tikz\pgfuseplotmark{m5av};};
\node[stars] at (axis cs:{31.213},{7.736}) {\tikz\pgfuseplotmark{m6c};};
\node[stars] at (axis cs:{31.551},{8.248}) {\tikz\pgfuseplotmark{m6cv};};
\node[stars] at (axis cs:{31.622},{0.035}) {\tikz\pgfuseplotmark{m6c};};
\node[stars] at (axis cs:{31.641},{22.648}) {\tikz\pgfuseplotmark{m5b};};
\node[stars] at (axis cs:{31.705},{25.704}) {\tikz\pgfuseplotmark{m6b};};
\node[stars] at (axis cs:{31.717},{-19.139}) {\tikz\pgfuseplotmark{m6c};};
\node[stars] at (axis cs:{31.793},{23.462}) {\tikz\pgfuseplotmark{m2bv};};
\node[stars] at (axis cs:{32.122},{37.859}) {\tikz\pgfuseplotmark{m5a};};
\node[stars] at (axis cs:{32.190},{-17.779}) {\tikz\pgfuseplotmark{m6bv};};
\node[stars] at (axis cs:{32.346},{17.224}) {\tikz\pgfuseplotmark{m6c};};
\node[stars] at (axis cs:{32.356},{25.940}) {\tikz\pgfuseplotmark{m5bb};};
\node[stars] at (axis cs:{32.386},{34.987}) {\tikz\pgfuseplotmark{m3b};};
\node[stars] at (axis cs:{32.395},{-24.346}) {\tikz\pgfuseplotmark{m6c};};
\node[stars] at (axis cs:{32.657},{19.500}) {\tikz\pgfuseplotmark{m6av};};
\node[stars] at (axis cs:{32.720},{39.039}) {\tikz\pgfuseplotmark{m6bb};};
\node[stars] at (axis cs:{32.800},{25.937}) {\tikz\pgfuseplotmark{m6b};};
\node[stars] at (axis cs:{32.838},{8.570}) {\tikz\pgfuseplotmark{m6a};};
\node[stars] at (axis cs:{32.843},{-10.052}) {\tikz\pgfuseplotmark{m6b};};
\node[stars] at (axis cs:{32.854},{31.526}) {\tikz\pgfuseplotmark{m6c};};
\node[stars] at (axis cs:{32.899},{-1.825}) {\tikz\pgfuseplotmark{m6b};};
\node[stars] at (axis cs:{33.066},{2.745}) {\tikz\pgfuseplotmark{m6c};};
\node[stars] at (axis cs:{33.093},{30.303}) {\tikz\pgfuseplotmark{m5cvb};};
\node[stars] at (axis cs:{33.156},{24.168}) {\tikz\pgfuseplotmark{m6b};};
\node[stars] at (axis cs:{33.198},{-2.393}) {\tikz\pgfuseplotmark{m6ab};};
\node[stars] at (axis cs:{33.200},{21.211}) {\tikz\pgfuseplotmark{m5c};};
\node[stars] at (axis cs:{33.227},{-30.724}) {\tikz\pgfuseplotmark{m5c};};
\node[stars] at (axis cs:{33.250},{8.846}) {\tikz\pgfuseplotmark{m4cv};};
\node[stars] at (axis cs:{33.254},{-21.000}) {\tikz\pgfuseplotmark{m6b};};
\node[stars] at (axis cs:{33.264},{15.280}) {\tikz\pgfuseplotmark{m6av};};
\node[stars] at (axis cs:{33.473},{-9.064}) {\tikz\pgfuseplotmark{m6c};};
\node[stars] at (axis cs:{33.658},{28.691}) {\tikz\pgfuseplotmark{m6c};};
\node[stars] at (axis cs:{33.928},{25.043}) {\tikz\pgfuseplotmark{m6a};};
\node[stars] at (axis cs:{33.942},{25.783}) {\tikz\pgfuseplotmark{m6a};};
\node[stars] at (axis cs:{33.985},{33.359}) {\tikz\pgfuseplotmark{m5c};};
\node[stars] at (axis cs:{34.246},{-6.422}) {\tikz\pgfuseplotmark{m5c};};
\node[stars] at (axis cs:{34.263},{34.224}) {\tikz\pgfuseplotmark{m5b};};
\node[stars] at (axis cs:{34.329},{33.847}) {\tikz\pgfuseplotmark{m4b};};
\node[stars] at (axis cs:{34.506},{1.758}) {\tikz\pgfuseplotmark{m6a};};
\node[stars] at (axis cs:{34.531},{19.901}) {\tikz\pgfuseplotmark{m6a};};
\node[stars] at (axis cs:{34.737},{28.643}) {\tikz\pgfuseplotmark{m5c};};
\node[stars] at (axis cs:{34.741},{23.168}) {\tikz\pgfuseplotmark{m6c};};
\node[stars] at (axis cs:{34.744},{-25.945}) {\tikz\pgfuseplotmark{m6c};};
\node[stars] at (axis cs:{34.837},{-2.977}) {\tikz\pgfuseplotmark{m5bvb};};
\node[stars] at (axis cs:{35.486},{0.395}) {\tikz\pgfuseplotmark{m5cv};};
\node[stars] at (axis cs:{35.506},{-10.777}) {\tikz\pgfuseplotmark{m5c};};
\node[stars] at (axis cs:{35.521},{-17.662}) {\tikz\pgfuseplotmark{m6b};};
\node[stars] at (axis cs:{35.552},{-0.885}) {\tikz\pgfuseplotmark{m5cv};};
\node[stars] at (axis cs:{35.636},{-23.816}) {\tikz\pgfuseplotmark{m5c};};
\node[stars] at (axis cs:{35.740},{-18.355}) {\tikz\pgfuseplotmark{m6c};};
\node[stars] at (axis cs:{36.084},{-25.847}) {\tikz\pgfuseplotmark{m6c};};
\node[stars] at (axis cs:{36.204},{10.610}) {\tikz\pgfuseplotmark{m5c};};
\node[stars] at (axis cs:{36.243},{-2.780}) {\tikz\pgfuseplotmark{m6c};};
\node[stars] at (axis cs:{36.488},{-12.290}) {\tikz\pgfuseplotmark{m5b};};
\node[stars] at (axis cs:{36.501},{-15.341}) {\tikz\pgfuseplotmark{m6avb};};
\node[stars] at (axis cs:{36.647},{-20.042}) {\tikz\pgfuseplotmark{m6b};};
\node[stars] at (axis cs:{36.780},{27.012}) {\tikz\pgfuseplotmark{m6b};};
\node[stars] at (axis cs:{36.847},{10.198}) {\tikz\pgfuseplotmark{m6c};};
\node[stars] at (axis cs:{36.866},{31.801}) {\tikz\pgfuseplotmark{m6a};};
\node[stars] at (axis cs:{37.000},{1.961}) {\tikz\pgfuseplotmark{m6c};};
\node[stars] at (axis cs:{37.007},{-33.811}) {\tikz\pgfuseplotmark{m5b};};
\node[stars] at (axis cs:{37.040},{8.460}) {\tikz\pgfuseplotmark{m4c};};
\node[stars] at (axis cs:{37.042},{29.669}) {\tikz\pgfuseplotmark{m5c};};
\node[stars] at (axis cs:{37.148},{-31.102}) {\tikz\pgfuseplotmark{m6b};};
\node[stars] at (axis cs:{37.202},{29.932}) {\tikz\pgfuseplotmark{m6b};};
\node[stars] at (axis cs:{37.307},{23.469}) {\tikz\pgfuseplotmark{m6c};};
\node[stars] at (axis cs:{37.397},{9.565}) {\tikz\pgfuseplotmark{m6bb};};
\node[stars] at (axis cs:{37.569},{33.834}) {\tikz\pgfuseplotmark{m6c};};
\node[stars] at (axis cs:{37.635},{25.235}) {\tikz\pgfuseplotmark{m6b};};
\node[stars] at (axis cs:{37.637},{-22.545}) {\tikz\pgfuseplotmark{m6c};};
\node[stars] at (axis cs:{37.660},{19.855}) {\tikz\pgfuseplotmark{m6cv};};
\node[stars] at (axis cs:{37.688},{0.256}) {\tikz\pgfuseplotmark{m6bv};};
\node[stars] at (axis cs:{37.727},{17.704}) {\tikz\pgfuseplotmark{m6c};};
\node[stars] at (axis cs:{37.875},{2.267}) {\tikz\pgfuseplotmark{m5c};};
\node[stars] at (axis cs:{38.022},{-15.244}) {\tikz\pgfuseplotmark{m5a};};
\node[stars] at (axis cs:{38.026},{36.147}) {\tikz\pgfuseplotmark{m5c};};
\node[stars] at (axis cs:{38.039},{-1.035}) {\tikz\pgfuseplotmark{m5c};};
\node[stars] at (axis cs:{38.062},{-36.427}) {\tikz\pgfuseplotmark{m6c};};
\node[stars] at (axis cs:{38.219},{34.542}) {\tikz\pgfuseplotmark{m6a};};
\node[stars] at (axis cs:{38.226},{15.035}) {\tikz\pgfuseplotmark{m6b};};
\node[stars] at (axis cs:{38.279},{-34.650}) {\tikz\pgfuseplotmark{m6b};};
\node[stars] at (axis cs:{38.418},{-20.002}) {\tikz\pgfuseplotmark{m6c};};
\node[stars] at (axis cs:{38.461},{-28.232}) {\tikz\pgfuseplotmark{m5bb};};
\node[stars] at (axis cs:{38.678},{-7.859}) {\tikz\pgfuseplotmark{m6a};};
\node[stars] at (axis cs:{38.767},{7.471}) {\tikz\pgfuseplotmark{m6c};};
\node[stars] at (axis cs:{38.866},{39.664}) {\tikz\pgfuseplotmark{m6c};};
\node[stars] at (axis cs:{38.911},{37.312}) {\tikz\pgfuseplotmark{m6a};};
\node[stars] at (axis cs:{38.945},{34.687}) {\tikz\pgfuseplotmark{m5cvb};};
\node[stars] at (axis cs:{38.969},{5.593}) {\tikz\pgfuseplotmark{m5b};};
\node[stars] at (axis cs:{39.000},{-7.831}) {\tikz\pgfuseplotmark{m6a};};
\node[stars] at (axis cs:{39.020},{6.887}) {\tikz\pgfuseplotmark{m6a};};
\node[stars] at (axis cs:{39.039},{-30.045}) {\tikz\pgfuseplotmark{m6a};};
\node[stars] at (axis cs:{39.146},{7.730}) {\tikz\pgfuseplotmark{m6av};};
\node[stars] at (axis cs:{39.158},{12.447}) {\tikz\pgfuseplotmark{m6a};};
\node[stars] at (axis cs:{39.179},{31.608}) {\tikz\pgfuseplotmark{m6b};};
\node[stars] at (axis cs:{39.238},{38.734}) {\tikz\pgfuseplotmark{m6b};};
\node[stars] at (axis cs:{39.244},{-34.578}) {\tikz\pgfuseplotmark{m6a};};
\node[stars] at (axis cs:{39.252},{24.647}) {\tikz\pgfuseplotmark{m6cb};};
\node[stars] at (axis cs:{39.277},{32.891}) {\tikz\pgfuseplotmark{m6c};};
\node[stars] at (axis cs:{39.424},{-3.396}) {\tikz\pgfuseplotmark{m6a};};
\node[stars] at (axis cs:{39.503},{7.695}) {\tikz\pgfuseplotmark{m6c};};
\node[stars] at (axis cs:{39.574},{37.727}) {\tikz\pgfuseplotmark{m6c};};
\node[stars] at (axis cs:{39.578},{-30.194}) {\tikz\pgfuseplotmark{m6a};};
\node[stars] at (axis cs:{39.603},{-37.990}) {\tikz\pgfuseplotmark{m6c};};
\node[stars] at (axis cs:{39.616},{38.089}) {\tikz\pgfuseplotmark{m6c};};
\node[stars] at (axis cs:{39.653},{3.443}) {\tikz\pgfuseplotmark{m6c};};
\node[stars] at (axis cs:{39.704},{21.961}) {\tikz\pgfuseplotmark{m5c};};
\node[stars] at (axis cs:{39.871},{0.328}) {\tikz\pgfuseplotmark{m4bv};};
\node[stars] at (axis cs:{39.891},{-11.872}) {\tikz\pgfuseplotmark{m5a};};
\node[stars] at (axis cs:{40.052},{-9.453}) {\tikz\pgfuseplotmark{m6a};};
\node[stars] at (axis cs:{40.065},{6.112}) {\tikz\pgfuseplotmark{m6c};};
\node[stars] at (axis cs:{40.167},{-39.855}) {\tikz\pgfuseplotmark{m4b};};
\node[stars] at (axis cs:{40.171},{27.061}) {\tikz\pgfuseplotmark{m5c};};
\node[stars] at (axis cs:{40.308},{-0.695}) {\tikz\pgfuseplotmark{m6avb};};
\node[stars] at (axis cs:{40.392},{-14.549}) {\tikz\pgfuseplotmark{m6b};};
\node[stars] at (axis cs:{40.451},{-3.214}) {\tikz\pgfuseplotmark{m6b};};
\node[stars] at (axis cs:{40.528},{-38.384}) {\tikz\pgfuseplotmark{m6b};};
\node[stars] at (axis cs:{40.591},{20.011}) {\tikz\pgfuseplotmark{m6a};};
\node[stars] at (axis cs:{40.621},{10.742}) {\tikz\pgfuseplotmark{m6cv};};
\node[stars] at (axis cs:{40.825},{3.236}) {\tikz\pgfuseplotmark{m3cb};};
\node[stars] at (axis cs:{40.863},{27.707}) {\tikz\pgfuseplotmark{m5a};};
\node[stars] at (axis cs:{40.870},{-2.531}) {\tikz\pgfuseplotmark{m6c};};
\node[stars] at (axis cs:{40.964},{25.638}) {\tikz\pgfuseplotmark{m6cb};};
\node[stars] at (axis cs:{41.031},{-13.859}) {\tikz\pgfuseplotmark{m4c};};
\node[stars] at (axis cs:{41.080},{17.764}) {\tikz\pgfuseplotmark{m6c};};
\node[stars] at (axis cs:{41.086},{-32.525}) {\tikz\pgfuseplotmark{m6c};};
\node[stars] at (axis cs:{41.137},{15.312}) {\tikz\pgfuseplotmark{m6a};};
\node[stars] at (axis cs:{41.236},{10.114}) {\tikz\pgfuseplotmark{m4cv};};
\node[stars] at (axis cs:{41.240},{12.446}) {\tikz\pgfuseplotmark{m5cv};};
\node[stars] at (axis cs:{41.276},{-18.572}) {\tikz\pgfuseplotmark{m4c};};
\node[stars] at (axis cs:{41.337},{4.711}) {\tikz\pgfuseplotmark{m6b};};
\node[stars] at (axis cs:{41.688},{-21.639}) {\tikz\pgfuseplotmark{m6c};};
\node[stars] at (axis cs:{41.743},{35.984}) {\tikz\pgfuseplotmark{m6c};};
\node[stars] at (axis cs:{41.765},{35.555}) {\tikz\pgfuseplotmark{m6c};};
\node[stars] at (axis cs:{41.796},{-22.486}) {\tikz\pgfuseplotmark{m6c};};
\node[stars] at (axis cs:{41.977},{29.247}) {\tikz\pgfuseplotmark{m5a};};
\node[stars] at (axis cs:{42.134},{18.284}) {\tikz\pgfuseplotmark{m6a};};
\node[stars] at (axis cs:{42.191},{25.188}) {\tikz\pgfuseplotmark{m6bv};};
\node[stars] at (axis cs:{42.273},{-32.406}) {\tikz\pgfuseplotmark{m4c};};
\node[stars] at (axis cs:{42.323},{17.464}) {\tikz\pgfuseplotmark{m5cb};};
\node[stars] at (axis cs:{42.362},{37.326}) {\tikz\pgfuseplotmark{m6c};};
\node[stars] at (axis cs:{42.462},{-24.560}) {\tikz\pgfuseplotmark{m6c};};
\node[stars] at (axis cs:{42.476},{-27.942}) {\tikz\pgfuseplotmark{m5c};};
\node[stars] at (axis cs:{42.496},{27.260}) {\tikz\pgfuseplotmark{m4avb};};
\node[stars] at (axis cs:{42.562},{-35.843}) {\tikz\pgfuseplotmark{m6bb};};
\node[stars] at (axis cs:{42.646},{38.319}) {\tikz\pgfuseplotmark{m4cv};};
\node[stars] at (axis cs:{42.647},{36.948}) {\tikz\pgfuseplotmark{m6c};};
\node[stars] at (axis cs:{42.668},{-35.676}) {\tikz\pgfuseplotmark{m5c};};
\node[stars] at (axis cs:{42.699},{-39.932}) {\tikz\pgfuseplotmark{m6c};};
\node[stars] at (axis cs:{42.760},{-21.004}) {\tikz\pgfuseplotmark{m5a};};
\node[stars] at (axis cs:{42.873},{15.082}) {\tikz\pgfuseplotmark{m6a};};
\node[stars] at (axis cs:{42.878},{35.060}) {\tikz\pgfuseplotmark{m5av};};
\node[stars] at (axis cs:{42.984},{-30.814}) {\tikz\pgfuseplotmark{m6c};};
\node[stars] at (axis cs:{43.134},{-12.770}) {\tikz\pgfuseplotmark{m6bv};};
\node[stars] at (axis cs:{43.211},{-9.441}) {\tikz\pgfuseplotmark{m6c};};
\node[stars] at (axis cs:{43.299},{16.483}) {\tikz\pgfuseplotmark{m6c};};
\node[stars] at (axis cs:{43.393},{-38.437}) {\tikz\pgfuseplotmark{m6b};};
\node[stars] at (axis cs:{43.397},{-22.376}) {\tikz\pgfuseplotmark{m6b};};
\node[stars] at (axis cs:{43.428},{38.337}) {\tikz\pgfuseplotmark{m5cv};};
\node[stars] at (axis cs:{43.952},{18.331}) {\tikz\pgfuseplotmark{m6av};};
\node[stars] at (axis cs:{44.057},{8.381}) {\tikz\pgfuseplotmark{m6b};};
\node[stars] at (axis cs:{44.107},{-8.898}) {\tikz\pgfuseplotmark{m4bv};};
\node[stars] at (axis cs:{44.109},{18.023}) {\tikz\pgfuseplotmark{m6av};};
\node[stars] at (axis cs:{44.156},{-3.712}) {\tikz\pgfuseplotmark{m5c};};
\node[stars] at (axis cs:{44.269},{4.501}) {\tikz\pgfuseplotmark{m6cv};};
\node[stars] at (axis cs:{44.304},{-29.855}) {\tikz\pgfuseplotmark{m6c};};
\node[stars] at (axis cs:{44.322},{31.934}) {\tikz\pgfuseplotmark{m5bv};};
\node[stars] at (axis cs:{44.349},{-23.862}) {\tikz\pgfuseplotmark{m5c};};
\node[stars] at (axis cs:{44.386},{-38.191}) {\tikz\pgfuseplotmark{m6c};};
\node[stars] at (axis cs:{44.510},{38.615}) {\tikz\pgfuseplotmark{m6b};};
\node[stars] at (axis cs:{44.522},{20.669}) {\tikz\pgfuseplotmark{m6av};};
\node[stars] at (axis cs:{44.524},{-23.606}) {\tikz\pgfuseplotmark{m6a};};
\node[stars] at (axis cs:{44.675},{-2.783}) {\tikz\pgfuseplotmark{m5c};};
\node[stars] at (axis cs:{44.690},{39.663}) {\tikz\pgfuseplotmark{m5a};};
\node[stars] at (axis cs:{44.697},{-9.776}) {\tikz\pgfuseplotmark{m6c};};
\node[stars] at (axis cs:{44.765},{35.183}) {\tikz\pgfuseplotmark{m5b};};
\node[stars] at (axis cs:{44.777},{-28.907}) {\tikz\pgfuseplotmark{m6c};};
\node[stars] at (axis cs:{44.803},{21.340}) {\tikz\pgfuseplotmark{m5ab};};
\node[stars] at (axis cs:{44.901},{-25.274}) {\tikz\pgfuseplotmark{m6a};};
\node[stars] at (axis cs:{44.910},{-32.507}) {\tikz\pgfuseplotmark{m6c};};
\node[stars] at (axis cs:{44.921},{-2.465}) {\tikz\pgfuseplotmark{m6a};};
\node[stars] at (axis cs:{44.929},{8.907}) {\tikz\pgfuseplotmark{m5a};};
\node[stars] at (axis cs:{45.050},{38.131}) {\tikz\pgfuseplotmark{m6b};};
\node[stars] at (axis cs:{45.184},{10.870}) {\tikz\pgfuseplotmark{m6b};};
\node[stars] at (axis cs:{45.212},{-2.878}) {\tikz\pgfuseplotmark{m6bv};};
\node[stars] at (axis cs:{45.292},{-7.663}) {\tikz\pgfuseplotmark{m6a};};
\node[stars] at (axis cs:{45.407},{-28.091}) {\tikz\pgfuseplotmark{m6b};};
\node[stars] at (axis cs:{45.468},{5.336}) {\tikz\pgfuseplotmark{m6c};};
\node[stars] at (axis cs:{45.476},{26.462}) {\tikz\pgfuseplotmark{m6bv};};
\node[stars] at (axis cs:{45.484},{-9.961}) {\tikz\pgfuseplotmark{m6a};};
\node[stars] at (axis cs:{45.539},{-6.494}) {\tikz\pgfuseplotmark{m6c};};
\node[stars] at (axis cs:{45.570},{4.090}) {\tikz\pgfuseplotmark{m3av};};
\node[stars] at (axis cs:{45.594},{4.353}) {\tikz\pgfuseplotmark{m6a};};
\node[stars] at (axis cs:{45.598},{-23.624}) {\tikz\pgfuseplotmark{m4b};};
\node[stars] at (axis cs:{45.676},{-7.685}) {\tikz\pgfuseplotmark{m5cb};};
\node[stars] at (axis cs:{45.876},{28.270}) {\tikz\pgfuseplotmark{m6c};};
\node[stars] at (axis cs:{46.069},{-7.601}) {\tikz\pgfuseplotmark{m5c};};
\node[stars] at (axis cs:{46.159},{1.863}) {\tikz\pgfuseplotmark{m6b};};
\node[stars] at (axis cs:{46.294},{38.840}) {\tikz\pgfuseplotmark{m3cv};};
\node[stars] at (axis cs:{46.297},{-8.274}) {\tikz\pgfuseplotmark{m6c};};
\node[stars] at (axis cs:{46.361},{25.255}) {\tikz\pgfuseplotmark{m5c};};
\node[stars] at (axis cs:{46.599},{13.187}) {\tikz\pgfuseplotmark{m6a};};
\node[stars] at (axis cs:{46.640},{-6.088}) {\tikz\pgfuseplotmark{m5cv};};
\node[stars] at (axis cs:{46.857},{17.880}) {\tikz\pgfuseplotmark{m6cv};};
\node[stars] at (axis cs:{46.962},{-27.831}) {\tikz\pgfuseplotmark{m6c};};
\node[stars] at (axis cs:{47.088},{18.795}) {\tikz\pgfuseplotmark{m6cv};};
\node[stars] at (axis cs:{47.162},{8.471}) {\tikz\pgfuseplotmark{m6c};};
\node[stars] at (axis cs:{47.334},{20.761}) {\tikz\pgfuseplotmark{m6c};};
\node[stars] at (axis cs:{47.403},{29.077}) {\tikz\pgfuseplotmark{m6a};};
\node[stars] at (axis cs:{47.537},{27.820}) {\tikz\pgfuseplotmark{m6c};};
\node[stars] at (axis cs:{47.613},{26.896}) {\tikz\pgfuseplotmark{m6b};};
\node[stars] at (axis cs:{47.647},{-23.738}) {\tikz\pgfuseplotmark{m6c};};
\node[stars] at (axis cs:{47.662},{11.873}) {\tikz\pgfuseplotmark{m6b};};
\node[stars] at (axis cs:{47.772},{-13.264}) {\tikz\pgfuseplotmark{m6c};};
\node[stars] at (axis cs:{47.820},{-16.025}) {\tikz\pgfuseplotmark{m6c};};
\node[stars] at (axis cs:{47.822},{39.611}) {\tikz\pgfuseplotmark{m5a};};
\node[stars] at (axis cs:{47.828},{-3.811}) {\tikz\pgfuseplotmark{m6b};};
\node[stars] at (axis cs:{47.841},{13.048}) {\tikz\pgfuseplotmark{m6b};};
\node[stars] at (axis cs:{47.907},{19.727}) {\tikz\pgfuseplotmark{m4cv};};
\node[stars] at (axis cs:{48.019},{-28.988}) {\tikz\pgfuseplotmark{m4ab};};
\node[stars] at (axis cs:{48.059},{27.257}) {\tikz\pgfuseplotmark{m6av};};
\node[stars] at (axis cs:{48.110},{6.661}) {\tikz\pgfuseplotmark{m6av};};
\node[stars] at (axis cs:{48.193},{-1.196}) {\tikz\pgfuseplotmark{m5bb};};
\node[stars] at (axis cs:{48.256},{-35.944}) {\tikz\pgfuseplotmark{m6cv};};
\node[stars] at (axis cs:{48.408},{-29.804}) {\tikz\pgfuseplotmark{m6c};};
\node[stars] at (axis cs:{48.725},{21.044}) {\tikz\pgfuseplotmark{m5b};};
\node[stars] at (axis cs:{48.751},{-26.100}) {\tikz\pgfuseplotmark{m6c};};
\node[stars] at (axis cs:{48.835},{30.557}) {\tikz\pgfuseplotmark{m6a};};
\node[stars] at (axis cs:{48.946},{32.857}) {\tikz\pgfuseplotmark{m6c};};
\node[stars] at (axis cs:{48.958},{-8.820}) {\tikz\pgfuseplotmark{m5av};};
\node[stars] at (axis cs:{49.004},{-5.919}) {\tikz\pgfuseplotmark{m6c};};
\node[stars] at (axis cs:{49.008},{34.688}) {\tikz\pgfuseplotmark{m6cv};};
\node[stars] at (axis cs:{49.091},{-5.730}) {\tikz\pgfuseplotmark{m6cv};};
\node[stars] at (axis cs:{49.147},{32.184}) {\tikz\pgfuseplotmark{m6bv};};
\node[stars] at (axis cs:{49.149},{-9.155}) {\tikz\pgfuseplotmark{m6c};};
\node[stars] at (axis cs:{49.417},{31.127}) {\tikz\pgfuseplotmark{m6c};};
\node[stars] at (axis cs:{49.441},{39.283}) {\tikz\pgfuseplotmark{m6b};};
\node[stars] at (axis cs:{49.511},{-28.797}) {\tikz\pgfuseplotmark{m6b};};
\node[stars] at (axis cs:{49.592},{-22.511}) {\tikz\pgfuseplotmark{m5b};};
\node[stars] at (axis cs:{49.593},{-0.930}) {\tikz\pgfuseplotmark{m5cb};};
\node[stars] at (axis cs:{49.671},{-18.560}) {\tikz\pgfuseplotmark{m6ab};};
\node[stars] at (axis cs:{49.683},{34.223}) {\tikz\pgfuseplotmark{m5a};};
\node[stars] at (axis cs:{49.840},{3.370}) {\tikz\pgfuseplotmark{m5avb};};
\node[stars] at (axis cs:{49.879},{-21.758}) {\tikz\pgfuseplotmark{m4avb};};
\node[stars] at (axis cs:{49.895},{-24.123}) {\tikz\pgfuseplotmark{m6av};};
\node[stars] at (axis cs:{49.982},{27.071}) {\tikz\pgfuseplotmark{m6b};};
\node[stars] at (axis cs:{50.085},{29.048}) {\tikz\pgfuseplotmark{m4c};};
\node[stars] at (axis cs:{50.107},{25.663}) {\tikz\pgfuseplotmark{m6c};};
\node[stars] at (axis cs:{50.188},{-26.606}) {\tikz\pgfuseplotmark{m6c};};
\node[stars] at (axis cs:{50.278},{3.675}) {\tikz\pgfuseplotmark{m6a};};
\node[stars] at (axis cs:{50.307},{21.147}) {\tikz\pgfuseplotmark{m5cv};};
\node[stars] at (axis cs:{50.350},{-23.635}) {\tikz\pgfuseplotmark{m5c};};
\node[stars] at (axis cs:{50.550},{27.607}) {\tikz\pgfuseplotmark{m6a};};
\node[stars] at (axis cs:{50.568},{-25.588}) {\tikz\pgfuseplotmark{m6c};};
\node[stars] at (axis cs:{50.689},{20.742}) {\tikz\pgfuseplotmark{m5b};};
\node[stars] at (axis cs:{50.824},{-7.794}) {\tikz\pgfuseplotmark{m6c};};
\node[stars] at (axis cs:{50.901},{0.910}) {\tikz\pgfuseplotmark{m6c};};
\node[stars] at (axis cs:{50.912},{4.882}) {\tikz\pgfuseplotmark{m6c};};
\node[stars] at (axis cs:{51.042},{12.629}) {\tikz\pgfuseplotmark{m6b};};
\node[stars] at (axis cs:{51.077},{24.724}) {\tikz\pgfuseplotmark{m5c};};
\node[stars] at (axis cs:{51.109},{20.803}) {\tikz\pgfuseplotmark{m6b};};
\node[stars] at (axis cs:{51.124},{33.536}) {\tikz\pgfuseplotmark{m6avb};};
\node[stars] at (axis cs:{51.203},{9.029}) {\tikz\pgfuseplotmark{m4av};};
\node[stars] at (axis cs:{51.483},{-35.921}) {\tikz\pgfuseplotmark{m6c};};
\node[stars] at (axis cs:{51.594},{-27.317}) {\tikz\pgfuseplotmark{m6b};};
\node[stars] at (axis cs:{51.647},{28.715}) {\tikz\pgfuseplotmark{m6cv};};
\node[stars] at (axis cs:{51.792},{9.732}) {\tikz\pgfuseplotmark{m4av};};
\node[stars] at (axis cs:{51.828},{12.735}) {\tikz\pgfuseplotmark{m6c};};
\node[stars] at (axis cs:{51.889},{-35.681}) {\tikz\pgfuseplotmark{m6a};};
\node[stars] at (axis cs:{52.004},{-11.287}) {\tikz\pgfuseplotmark{m6a};};
\node[stars] at (axis cs:{52.086},{33.807}) {\tikz\pgfuseplotmark{m6a};};
\node[stars] at (axis cs:{52.111},{22.804}) {\tikz\pgfuseplotmark{m6b};};
\node[stars] at (axis cs:{52.266},{3.248}) {\tikz\pgfuseplotmark{m6c};};
\node[stars] at (axis cs:{52.400},{-12.675}) {\tikz\pgfuseplotmark{m6a};};
\node[stars] at (axis cs:{52.413},{-6.804}) {\tikz\pgfuseplotmark{m6bv};};
\node[stars] at (axis cs:{52.602},{11.336}) {\tikz\pgfuseplotmark{m5b};};
\node[stars] at (axis cs:{52.654},{-5.075}) {\tikz\pgfuseplotmark{m5a};};
\node[stars] at (axis cs:{52.689},{6.189}) {\tikz\pgfuseplotmark{m6b};};
\node[stars] at (axis cs:{52.718},{12.937}) {\tikz\pgfuseplotmark{m4b};};
\node[stars] at (axis cs:{52.974},{-25.614}) {\tikz\pgfuseplotmark{m6c};};
\node[stars] at (axis cs:{53.150},{9.373}) {\tikz\pgfuseplotmark{m6a};};
\node[stars] at (axis cs:{53.167},{35.462}) {\tikz\pgfuseplotmark{m6b};};
\node[stars] at (axis cs:{53.233},{-9.458}) {\tikz\pgfuseplotmark{m4av};};
\node[stars] at (axis cs:{53.396},{39.899}) {\tikz\pgfuseplotmark{m6av};};
\node[stars] at (axis cs:{53.447},{-21.633}) {\tikz\pgfuseplotmark{m4c};};
\node[stars] at (axis cs:{53.487},{-31.080}) {\tikz\pgfuseplotmark{m6c};};
\node[stars] at (axis cs:{53.535},{17.833}) {\tikz\pgfuseplotmark{m6c};};
\node[stars] at (axis cs:{53.611},{24.464}) {\tikz\pgfuseplotmark{m6bvb};};
\node[stars] at (axis cs:{53.640},{-31.875}) {\tikz\pgfuseplotmark{m6cb};};
\node[stars] at (axis cs:{53.656},{-9.868}) {\tikz\pgfuseplotmark{m6c};};
\node[stars] at (axis cs:{53.705},{6.418}) {\tikz\pgfuseplotmark{m6c};};
\node[stars] at (axis cs:{53.718},{-25.583}) {\tikz\pgfuseplotmark{m6c};};
\node[stars] at (axis cs:{53.990},{-11.194}) {\tikz\pgfuseplotmark{m6a};};
\node[stars] at (axis cs:{54.073},{-17.467}) {\tikz\pgfuseplotmark{m5cv};};
\node[stars] at (axis cs:{54.197},{0.588}) {\tikz\pgfuseplotmark{m6bvb};};
\node[stars] at (axis cs:{54.218},{0.402}) {\tikz\pgfuseplotmark{m4c};};
\node[stars] at (axis cs:{54.449},{15.430}) {\tikz\pgfuseplotmark{m6c};};
\node[stars] at (axis cs:{54.622},{-7.392}) {\tikz\pgfuseplotmark{m6a};};
\node[stars] at (axis cs:{54.699},{-27.943}) {\tikz\pgfuseplotmark{m6b};};
\node[stars] at (axis cs:{54.750},{20.916}) {\tikz\pgfuseplotmark{m6c};};
\node[stars] at (axis cs:{54.755},{-5.626}) {\tikz\pgfuseplotmark{m6b};};
\node[stars] at (axis cs:{54.856},{-10.437}) {\tikz\pgfuseplotmark{m6c};};
\node[stars] at (axis cs:{54.857},{16.537}) {\tikz\pgfuseplotmark{m6c};};
\node[stars] at (axis cs:{54.910},{-3.393}) {\tikz\pgfuseplotmark{m6c};};
\node[stars] at (axis cs:{54.963},{3.057}) {\tikz\pgfuseplotmark{m6a};};
\node[stars] at (axis cs:{54.998},{-1.121}) {\tikz\pgfuseplotmark{m6b};};
\node[stars] at (axis cs:{55.048},{-15.226}) {\tikz\pgfuseplotmark{m6c};};
\node[stars] at (axis cs:{55.160},{-5.211}) {\tikz\pgfuseplotmark{m6av};};
\node[stars] at (axis cs:{55.193},{25.329}) {\tikz\pgfuseplotmark{m6b};};
\node[stars] at (axis cs:{55.283},{37.580}) {\tikz\pgfuseplotmark{m6av};};
\node[stars] at (axis cs:{55.307},{-11.802}) {\tikz\pgfuseplotmark{m6c};};
\node[stars] at (axis cs:{55.562},{-31.938}) {\tikz\pgfuseplotmark{m5b};};
\node[stars] at (axis cs:{55.579},{19.700}) {\tikz\pgfuseplotmark{m6a};};
\node[stars] at (axis cs:{55.594},{33.965}) {\tikz\pgfuseplotmark{m5b};};
\node[stars] at (axis cs:{55.709},{-37.313}) {\tikz\pgfuseplotmark{m5a};};
\node[stars] at (axis cs:{55.812},{-9.763}) {\tikz\pgfuseplotmark{m4av};};
\node[stars] at (axis cs:{55.891},{-10.486}) {\tikz\pgfuseplotmark{m6a};};
\node[stars] at (axis cs:{55.947},{19.665}) {\tikz\pgfuseplotmark{m6c};};
\node[stars] at (axis cs:{56.080},{32.288}) {\tikz\pgfuseplotmark{m4avb};};
\node[stars] at (axis cs:{56.118},{20.929}) {\tikz\pgfuseplotmark{m6b};};
\node[stars] at (axis cs:{56.127},{-1.163}) {\tikz\pgfuseplotmark{m5c};};
\node[stars] at (axis cs:{56.131},{36.460}) {\tikz\pgfuseplotmark{m6a};};
\node[stars] at (axis cs:{56.201},{24.289}) {\tikz\pgfuseplotmark{m5c};};
\node[stars] at (axis cs:{56.219},{24.113}) {\tikz\pgfuseplotmark{m4a};};
\node[stars] at (axis cs:{56.235},{-0.297}) {\tikz\pgfuseplotmark{m6a};};
\node[stars] at (axis cs:{56.291},{24.839}) {\tikz\pgfuseplotmark{m6a};};
\node[stars] at (axis cs:{56.302},{24.467}) {\tikz\pgfuseplotmark{m4c};};
\node[stars] at (axis cs:{56.405},{38.676}) {\tikz\pgfuseplotmark{m6c};};
\node[stars] at (axis cs:{56.419},{6.050}) {\tikz\pgfuseplotmark{m5c};};
\node[stars] at (axis cs:{56.457},{24.368}) {\tikz\pgfuseplotmark{m4b};};
\node[stars] at (axis cs:{56.477},{24.554}) {\tikz\pgfuseplotmark{m6a};};
\node[stars] at (axis cs:{56.512},{24.528}) {\tikz\pgfuseplotmark{m6c};};
\node[stars] at (axis cs:{56.536},{-12.102}) {\tikz\pgfuseplotmark{m4cv};};
\node[stars] at (axis cs:{56.539},{6.803}) {\tikz\pgfuseplotmark{m6b};};
\node[stars] at (axis cs:{56.582},{23.948}) {\tikz\pgfuseplotmark{m4cv};};
\node[stars] at (axis cs:{56.614},{-29.338}) {\tikz\pgfuseplotmark{m6b};};
\node[stars] at (axis cs:{56.712},{-23.249}) {\tikz\pgfuseplotmark{m4c};};
\node[stars] at (axis cs:{56.871},{24.105}) {\tikz\pgfuseplotmark{m3bvb};};
\node[stars] at (axis cs:{56.915},{-23.875}) {\tikz\pgfuseplotmark{m5cv};};
\node[stars] at (axis cs:{56.954},{32.195}) {\tikz\pgfuseplotmark{m6c};};
\node[stars] at (axis cs:{56.957},{-36.106}) {\tikz\pgfuseplotmark{m6c};};
\node[stars] at (axis cs:{56.984},{-30.168}) {\tikz\pgfuseplotmark{m6av};};
\node[stars] at (axis cs:{57.027},{24.988}) {\tikz\pgfuseplotmark{m6c};};
\node[stars] at (axis cs:{57.068},{11.143}) {\tikz\pgfuseplotmark{m5b};};
\node[stars] at (axis cs:{57.087},{23.421}) {\tikz\pgfuseplotmark{m5c};};
\node[stars] at (axis cs:{57.149},{-20.903}) {\tikz\pgfuseplotmark{m6a};};
\node[stars] at (axis cs:{57.149},{-37.620}) {\tikz\pgfuseplotmark{m5ab};};
\node[stars] at (axis cs:{57.162},{0.228}) {\tikz\pgfuseplotmark{m6b};};
\node[stars] at (axis cs:{57.237},{23.857}) {\tikz\pgfuseplotmark{m6c};};
\node[stars] at (axis cs:{57.291},{24.053}) {\tikz\pgfuseplotmark{m4avb};};
\node[stars] at (axis cs:{57.297},{24.137}) {\tikz\pgfuseplotmark{m5bv};};
\node[stars] at (axis cs:{57.364},{-36.200}) {\tikz\pgfuseplotmark{m4c};};
\node[stars] at (axis cs:{57.386},{33.091}) {\tikz\pgfuseplotmark{m5cv};};
\node[stars] at (axis cs:{57.431},{23.712}) {\tikz\pgfuseplotmark{m6cv};};
\node[stars] at (axis cs:{57.479},{22.244}) {\tikz\pgfuseplotmark{m6bv};};
\node[stars] at (axis cs:{57.567},{-1.522}) {\tikz\pgfuseplotmark{m6cb};};
\node[stars] at (axis cs:{57.579},{25.579}) {\tikz\pgfuseplotmark{m5c};};
\node[stars] at (axis cs:{57.816},{13.046}) {\tikz\pgfuseplotmark{m6cv};};
\node[stars] at (axis cs:{57.974},{34.359}) {\tikz\pgfuseplotmark{m6a};};
\node[stars] at (axis cs:{58.001},{6.535}) {\tikz\pgfuseplotmark{m6a};};
\node[stars] at (axis cs:{58.019},{31.169}) {\tikz\pgfuseplotmark{m6c};};
\node[stars] at (axis cs:{58.174},{-5.361}) {\tikz\pgfuseplotmark{m5c};};
\node[stars] at (axis cs:{58.292},{17.327}) {\tikz\pgfuseplotmark{m6bv};};
\node[stars] at (axis cs:{58.304},{-18.434}) {\tikz\pgfuseplotmark{m6c};};
\node[stars] at (axis cs:{58.394},{25.683}) {\tikz\pgfuseplotmark{m6cv};};
\node[stars] at (axis cs:{58.412},{-34.732}) {\tikz\pgfuseplotmark{m5b};};
\node[stars] at (axis cs:{58.428},{-24.612}) {\tikz\pgfuseplotmark{m5av};};
\node[stars] at (axis cs:{58.533},{31.884}) {\tikz\pgfuseplotmark{m3bvb};};
\node[stars] at (axis cs:{58.573},{-2.955}) {\tikz\pgfuseplotmark{m5ab};};
\node[stars] at (axis cs:{58.817},{-12.099}) {\tikz\pgfuseplotmark{m6bv};};
\node[stars] at (axis cs:{59.115},{-13.597}) {\tikz\pgfuseplotmark{m6cv};};
\node[stars] at (axis cs:{59.120},{35.081}) {\tikz\pgfuseplotmark{m5cv};};
\node[stars] at (axis cs:{59.158},{-9.751}) {\tikz\pgfuseplotmark{m6cv};};
\node[stars] at (axis cs:{59.217},{22.478}) {\tikz\pgfuseplotmark{m6a};};
\node[stars] at (axis cs:{59.257},{6.040}) {\tikz\pgfuseplotmark{m6b};};
\node[stars] at (axis cs:{59.266},{23.175}) {\tikz\pgfuseplotmark{m6bv};};
\node[stars] at (axis cs:{59.360},{24.462}) {\tikz\pgfuseplotmark{m6c};};
\node[stars] at (axis cs:{59.507},{-13.508}) {\tikz\pgfuseplotmark{m3bv};};
\node[stars] at (axis cs:{59.622},{38.840}) {\tikz\pgfuseplotmark{m6c};};
\node[stars] at (axis cs:{59.718},{-5.470}) {\tikz\pgfuseplotmark{m6a};};
\node[stars] at (axis cs:{59.741},{35.791}) {\tikz\pgfuseplotmark{m4bv};};
\node[stars] at (axis cs:{59.876},{-12.574}) {\tikz\pgfuseplotmark{m6av};};
\node[stars] at (axis cs:{59.919},{10.330}) {\tikz\pgfuseplotmark{m6c};};
\node[stars] at (axis cs:{59.981},{-24.016}) {\tikz\pgfuseplotmark{m5av};};
\node[stars] at (axis cs:{60.154},{17.296}) {\tikz\pgfuseplotmark{m6c};};
\node[stars] at (axis cs:{60.169},{-30.491}) {\tikz\pgfuseplotmark{m6b};};
\node[stars] at (axis cs:{60.170},{12.490}) {\tikz\pgfuseplotmark{m3cv};};
\node[stars] at (axis cs:{60.203},{18.194}) {\tikz\pgfuseplotmark{m6b};};
\node[stars] at (axis cs:{60.312},{36.989}) {\tikz\pgfuseplotmark{m6c};};
\node[stars] at (axis cs:{60.384},{-1.549}) {\tikz\pgfuseplotmark{m5cv};};
\node[stars] at (axis cs:{60.442},{9.998}) {\tikz\pgfuseplotmark{m6a};};
\node[stars] at (axis cs:{60.653},{-0.269}) {\tikz\pgfuseplotmark{m5c};};
\node[stars] at (axis cs:{60.789},{5.989}) {\tikz\pgfuseplotmark{m4b};};
\node[stars] at (axis cs:{60.853},{-20.144}) {\tikz\pgfuseplotmark{m6c};};
\node[stars] at (axis cs:{60.936},{5.436}) {\tikz\pgfuseplotmark{m5cv};};
\node[stars] at (axis cs:{60.986},{8.197}) {\tikz\pgfuseplotmark{m5c};};
\node[stars] at (axis cs:{61.036},{-16.589}) {\tikz\pgfuseplotmark{m6c};};
\node[stars] at (axis cs:{61.041},{2.827}) {\tikz\pgfuseplotmark{m5c};};
\node[stars] at (axis cs:{61.090},{24.106}) {\tikz\pgfuseplotmark{m6a};};
\node[stars] at (axis cs:{61.095},{-12.792}) {\tikz\pgfuseplotmark{m6av};};
\node[stars] at (axis cs:{61.170},{-20.382}) {\tikz\pgfuseplotmark{m6b};};
\node[stars] at (axis cs:{61.174},{22.082}) {\tikz\pgfuseplotmark{m4c};};
\node[stars] at (axis cs:{61.334},{22.009}) {\tikz\pgfuseplotmark{m6bb};};
\node[stars] at (axis cs:{61.406},{-27.652}) {\tikz\pgfuseplotmark{m6a};};
\node[stars] at (axis cs:{61.445},{-20.512}) {\tikz\pgfuseplotmark{m6c};};
\node[stars] at (axis cs:{61.485},{-8.856}) {\tikz\pgfuseplotmark{m6c};};
\node[stars] at (axis cs:{61.652},{27.600}) {\tikz\pgfuseplotmark{m5cv};};
\node[stars] at (axis cs:{61.752},{29.001}) {\tikz\pgfuseplotmark{m5c};};
\node[stars] at (axis cs:{61.925},{15.163}) {\tikz\pgfuseplotmark{m6bvb};};
\node[stars] at (axis cs:{61.998},{17.340}) {\tikz\pgfuseplotmark{m6bb};};
\node[stars] at (axis cs:{62.064},{37.727}) {\tikz\pgfuseplotmark{m6b};};
\node[stars] at (axis cs:{62.153},{38.040}) {\tikz\pgfuseplotmark{m6av};};
\node[stars] at (axis cs:{62.257},{13.398}) {\tikz\pgfuseplotmark{m6b};};
\node[stars] at (axis cs:{62.292},{19.609}) {\tikz\pgfuseplotmark{m6a};};
\node[stars] at (axis cs:{62.324},{-16.386}) {\tikz\pgfuseplotmark{m5cv};};
\node[stars] at (axis cs:{62.429},{3.323}) {\tikz\pgfuseplotmark{m6c};};
\node[stars] at (axis cs:{62.594},{-6.924}) {\tikz\pgfuseplotmark{m5c};};
\node[stars] at (axis cs:{62.691},{-35.274}) {\tikz\pgfuseplotmark{m6c};};
\node[stars] at (axis cs:{62.699},{-8.820}) {\tikz\pgfuseplotmark{m6a};};
\node[stars] at (axis cs:{62.708},{26.481}) {\tikz\pgfuseplotmark{m5cv};};
\node[stars] at (axis cs:{62.746},{33.587}) {\tikz\pgfuseplotmark{m6a};};
\node[stars] at (axis cs:{62.835},{5.523}) {\tikz\pgfuseplotmark{m6a};};
\node[stars] at (axis cs:{62.901},{-20.356}) {\tikz\pgfuseplotmark{m6av};};
\node[stars] at (axis cs:{62.966},{-6.838}) {\tikz\pgfuseplotmark{m4bv};};
\node[stars] at (axis cs:{62.984},{-8.837}) {\tikz\pgfuseplotmark{m6cb};};
\node[stars] at (axis cs:{63.001},{-17.275}) {\tikz\pgfuseplotmark{m6c};};
\node[stars] at (axis cs:{63.131},{17.277}) {\tikz\pgfuseplotmark{m6b};};
\node[stars] at (axis cs:{63.214},{22.413}) {\tikz\pgfuseplotmark{m6cv};};
\node[stars] at (axis cs:{63.388},{7.716}) {\tikz\pgfuseplotmark{m5c};};
\node[stars] at (axis cs:{63.394},{10.212}) {\tikz\pgfuseplotmark{m6c};};
\node[stars] at (axis cs:{63.409},{-1.149}) {\tikz\pgfuseplotmark{m6cv};};
\node[stars] at (axis cs:{63.458},{12.754}) {\tikz\pgfuseplotmark{m6c};};
\node[stars] at (axis cs:{63.485},{9.264}) {\tikz\pgfuseplotmark{m5bb};};
\node[stars] at (axis cs:{63.499},{37.967}) {\tikz\pgfuseplotmark{m6c};};
\node[stars] at (axis cs:{63.599},{-10.256}) {\tikz\pgfuseplotmark{m5bb};};
\node[stars] at (axis cs:{63.651},{10.011}) {\tikz\pgfuseplotmark{m5c};};
\node[stars] at (axis cs:{63.717},{37.543}) {\tikz\pgfuseplotmark{m6c};};
\node[stars] at (axis cs:{63.818},{-7.653}) {\tikz\pgfuseplotmark{m4cb};};
\node[stars] at (axis cs:{63.870},{6.187}) {\tikz\pgfuseplotmark{m6cvb};};
\node[stars] at (axis cs:{63.884},{8.892}) {\tikz\pgfuseplotmark{m4c};};
\node[stars] at (axis cs:{63.943},{15.401}) {\tikz\pgfuseplotmark{m6cv};};
\node[stars] at (axis cs:{64.315},{20.578}) {\tikz\pgfuseplotmark{m5b};};
\node[stars] at (axis cs:{64.330},{-6.472}) {\tikz\pgfuseplotmark{m6b};};
\node[stars] at (axis cs:{64.474},{-33.798}) {\tikz\pgfuseplotmark{m4a};};
\node[stars] at (axis cs:{64.567},{-20.715}) {\tikz\pgfuseplotmark{m6bv};};
\node[stars] at (axis cs:{64.597},{21.579}) {\tikz\pgfuseplotmark{m6a};};
\node[stars] at (axis cs:{64.656},{-22.970}) {\tikz\pgfuseplotmark{m6b};};
\node[stars] at (axis cs:{64.859},{21.142}) {\tikz\pgfuseplotmark{m5cv};};
\node[stars] at (axis cs:{64.903},{21.773}) {\tikz\pgfuseplotmark{m5cv};};
\node[stars] at (axis cs:{64.906},{10.121}) {\tikz\pgfuseplotmark{m6c};};
\node[stars] at (axis cs:{64.948},{15.627}) {\tikz\pgfuseplotmark{m4av};};
\node[stars] at (axis cs:{64.990},{14.035}) {\tikz\pgfuseplotmark{m6av};};
\node[stars] at (axis cs:{65.041},{31.953}) {\tikz\pgfuseplotmark{m6c};};
\node[stars] at (axis cs:{65.088},{27.351}) {\tikz\pgfuseplotmark{m5bb};};
\node[stars] at (axis cs:{65.103},{34.567}) {\tikz\pgfuseplotmark{m5b};};
\node[stars] at (axis cs:{65.105},{18.742}) {\tikz\pgfuseplotmark{m6b};};
\node[stars] at (axis cs:{65.151},{15.095}) {\tikz\pgfuseplotmark{m5cv};};
\node[stars] at (axis cs:{65.161},{-6.246}) {\tikz\pgfuseplotmark{m6cv};};
\node[stars] at (axis cs:{65.163},{-20.640}) {\tikz\pgfuseplotmark{m5c};};
\node[stars] at (axis cs:{65.172},{6.131}) {\tikz\pgfuseplotmark{m6a};};
\node[stars] at (axis cs:{65.178},{-7.592}) {\tikz\pgfuseplotmark{m6av};};
\node[stars] at (axis cs:{65.220},{13.864}) {\tikz\pgfuseplotmark{m6c};};
\node[stars] at (axis cs:{65.363},{-0.098}) {\tikz\pgfuseplotmark{m6a};};
\node[stars] at (axis cs:{65.380},{-25.728}) {\tikz\pgfuseplotmark{m6bvb};};
\node[stars] at (axis cs:{65.406},{-6.284}) {\tikz\pgfuseplotmark{m6c};};
\node[stars] at (axis cs:{65.515},{14.077}) {\tikz\pgfuseplotmark{m6av};};
\node[stars] at (axis cs:{65.595},{20.821}) {\tikz\pgfuseplotmark{m6bvb};};
\node[stars] at (axis cs:{65.646},{25.629}) {\tikz\pgfuseplotmark{m5cb};};
\node[stars] at (axis cs:{65.734},{17.542}) {\tikz\pgfuseplotmark{m4av};};
\node[stars] at (axis cs:{65.774},{-24.892}) {\tikz\pgfuseplotmark{m6av};};
\node[stars] at (axis cs:{65.782},{-35.545}) {\tikz\pgfuseplotmark{m6c};};
\node[stars] at (axis cs:{65.854},{16.777}) {\tikz\pgfuseplotmark{m6av};};
\node[stars] at (axis cs:{65.885},{20.982}) {\tikz\pgfuseplotmark{m6bv};};
\node[stars] at (axis cs:{65.920},{-3.745}) {\tikz\pgfuseplotmark{m5cv};};
\node[stars] at (axis cs:{65.966},{9.461}) {\tikz\pgfuseplotmark{m5b};};
\node[stars] at (axis cs:{65.999},{24.301}) {\tikz\pgfuseplotmark{m6cb};};
\node[stars] at (axis cs:{66.009},{-34.017}) {\tikz\pgfuseplotmark{m4b};};
\node[stars] at (axis cs:{66.024},{17.444}) {\tikz\pgfuseplotmark{m5a};};
\node[stars] at (axis cs:{66.060},{12.158}) {\tikz\pgfuseplotmark{m6c};};
\node[stars] at (axis cs:{66.121},{34.131}) {\tikz\pgfuseplotmark{m6a};};
\node[stars] at (axis cs:{66.156},{33.960}) {\tikz\pgfuseplotmark{m6ab};};
\node[stars] at (axis cs:{66.238},{19.042}) {\tikz\pgfuseplotmark{m6b};};
\node[stars] at (axis cs:{66.342},{22.294}) {\tikz\pgfuseplotmark{m4cv};};
\node[stars] at (axis cs:{66.354},{22.200}) {\tikz\pgfuseplotmark{m5cv};};
\node[stars] at (axis cs:{66.372},{17.928}) {\tikz\pgfuseplotmark{m4cvb};};
\node[stars] at (axis cs:{66.405},{15.941}) {\tikz\pgfuseplotmark{m6cv};};
\node[stars] at (axis cs:{66.506},{4.373}) {\tikz\pgfuseplotmark{m6c};};
\node[stars] at (axis cs:{66.526},{31.439}) {\tikz\pgfuseplotmark{m5c};};
\node[stars] at (axis cs:{66.577},{22.813}) {\tikz\pgfuseplotmark{m4cv};};
\node[stars] at (axis cs:{66.586},{15.618}) {\tikz\pgfuseplotmark{m4cv};};
\node[stars] at (axis cs:{66.588},{8.590}) {\tikz\pgfuseplotmark{m6bv};};
\node[stars] at (axis cs:{66.652},{14.714}) {\tikz\pgfuseplotmark{m5a};};
\node[stars] at (axis cs:{66.737},{-24.081}) {\tikz\pgfuseplotmark{m6b};};
\node[stars] at (axis cs:{66.753},{2.079}) {\tikz\pgfuseplotmark{m6c};};
\node[stars] at (axis cs:{66.823},{22.996}) {\tikz\pgfuseplotmark{m6a};};
\node[stars] at (axis cs:{66.870},{11.212}) {\tikz\pgfuseplotmark{m6b};};
\node[stars] at (axis cs:{67.003},{21.620}) {\tikz\pgfuseplotmark{m6a};};
\node[stars] at (axis cs:{67.015},{1.858}) {\tikz\pgfuseplotmark{m6c};};
\node[stars] at (axis cs:{67.098},{14.741}) {\tikz\pgfuseplotmark{m6b};};
\node[stars] at (axis cs:{67.110},{16.360}) {\tikz\pgfuseplotmark{m5b};};
\node[stars] at (axis cs:{67.134},{1.381}) {\tikz\pgfuseplotmark{m6a};};
\node[stars] at (axis cs:{67.154},{19.180}) {\tikz\pgfuseplotmark{m4a};};
\node[stars] at (axis cs:{67.163},{-19.459}) {\tikz\pgfuseplotmark{m6b};};
\node[stars] at (axis cs:{67.166},{15.871}) {\tikz\pgfuseplotmark{m3cvb};};
\node[stars] at (axis cs:{67.209},{13.047}) {\tikz\pgfuseplotmark{m5bv};};
\node[stars] at (axis cs:{67.217},{30.361}) {\tikz\pgfuseplotmark{m6cb};};
\node[stars] at (axis cs:{67.279},{-13.048}) {\tikz\pgfuseplotmark{m6av};};
\node[stars] at (axis cs:{67.510},{10.262}) {\tikz\pgfuseplotmark{m6c};};
\node[stars] at (axis cs:{67.536},{15.638}) {\tikz\pgfuseplotmark{m6ab};};
\node[stars] at (axis cs:{67.542},{-13.592}) {\tikz\pgfuseplotmark{m6c};};
\node[stars] at (axis cs:{67.640},{16.194}) {\tikz\pgfuseplotmark{m5avb};};
\node[stars] at (axis cs:{67.656},{13.724}) {\tikz\pgfuseplotmark{m5cb};};
\node[stars] at (axis cs:{67.660},{32.458}) {\tikz\pgfuseplotmark{m6c};};
\node[stars] at (axis cs:{67.662},{15.692}) {\tikz\pgfuseplotmark{m5cvb};};
\node[stars] at (axis cs:{67.668},{-35.653}) {\tikz\pgfuseplotmark{m6b};};
\node[stars] at (axis cs:{67.780},{15.105}) {\tikz\pgfuseplotmark{m6cv};};
\node[stars] at (axis cs:{67.858},{-13.644}) {\tikz\pgfuseplotmark{m6cb};};
\node[stars] at (axis cs:{67.966},{15.852}) {\tikz\pgfuseplotmark{m6b};};
\node[stars] at (axis cs:{67.969},{-0.044}) {\tikz\pgfuseplotmark{m5b};};
\node[stars] at (axis cs:{67.987},{36.743}) {\tikz\pgfuseplotmark{m6cv};};
\node[stars] at (axis cs:{68.020},{5.410}) {\tikz\pgfuseplotmark{m6c};};
\node[stars] at (axis cs:{68.156},{-3.209}) {\tikz\pgfuseplotmark{m6av};};
\node[stars] at (axis cs:{68.342},{-10.785}) {\tikz\pgfuseplotmark{m6b};};
\node[stars] at (axis cs:{68.377},{-29.766}) {\tikz\pgfuseplotmark{m5a};};
\node[stars] at (axis cs:{68.388},{18.017}) {\tikz\pgfuseplotmark{m6cb};};
\node[stars] at (axis cs:{68.451},{9.413}) {\tikz\pgfuseplotmark{m6b};};
\node[stars] at (axis cs:{68.462},{14.844}) {\tikz\pgfuseplotmark{m5av};};
\node[stars] at (axis cs:{68.478},{-6.739}) {\tikz\pgfuseplotmark{m6avb};};
\node[stars] at (axis cs:{68.534},{5.569}) {\tikz\pgfuseplotmark{m6a};};
\node[stars] at (axis cs:{68.548},{-8.231}) {\tikz\pgfuseplotmark{m5cv};};
\node[stars] at (axis cs:{68.549},{-8.970}) {\tikz\pgfuseplotmark{m5c};};
\node[stars] at (axis cs:{68.559},{-6.838}) {\tikz\pgfuseplotmark{m6b};};
\node[stars] at (axis cs:{68.658},{28.961}) {\tikz\pgfuseplotmark{m6b};};
\node[stars] at (axis cs:{68.659},{-24.040}) {\tikz\pgfuseplotmark{m6c};};
\node[stars] at (axis cs:{68.753},{-19.921}) {\tikz\pgfuseplotmark{m6b};};
\node[stars] at (axis cs:{68.888},{-30.562}) {\tikz\pgfuseplotmark{m4a};};
\node[stars] at (axis cs:{68.914},{10.161}) {\tikz\pgfuseplotmark{m4cvb};};
\node[stars] at (axis cs:{68.927},{19.882}) {\tikz\pgfuseplotmark{m6c};};
\node[stars] at (axis cs:{68.980},{16.509}) {\tikz\pgfuseplotmark{m1bvb};};
\node[stars] at (axis cs:{69.007},{-3.611}) {\tikz\pgfuseplotmark{m6cv};};
\node[stars] at (axis cs:{69.080},{-3.352}) {\tikz\pgfuseplotmark{m4bv};};
\node[stars] at (axis cs:{69.121},{23.341}) {\tikz\pgfuseplotmark{m6bv};};
\node[stars] at (axis cs:{69.212},{-30.717}) {\tikz\pgfuseplotmark{m6c};};
\node[stars] at (axis cs:{69.307},{0.998}) {\tikz\pgfuseplotmark{m5cv};};
\node[stars] at (axis cs:{69.312},{18.543}) {\tikz\pgfuseplotmark{m6cv};};
\node[stars] at (axis cs:{69.401},{-2.473}) {\tikz\pgfuseplotmark{m5c};};
\node[stars] at (axis cs:{69.539},{16.033}) {\tikz\pgfuseplotmark{m6a};};
\node[stars] at (axis cs:{69.539},{12.511}) {\tikz\pgfuseplotmark{m4c};};
\node[stars] at (axis cs:{69.545},{-14.304}) {\tikz\pgfuseplotmark{m4bb};};
\node[stars] at (axis cs:{69.566},{20.685}) {\tikz\pgfuseplotmark{m6av};};
\node[stars] at (axis cs:{69.723},{-12.123}) {\tikz\pgfuseplotmark{m5b};};
\node[stars] at (axis cs:{69.776},{7.871}) {\tikz\pgfuseplotmark{m5cb};};
\node[stars] at (axis cs:{69.819},{15.918}) {\tikz\pgfuseplotmark{m5ab};};
\node[stars] at (axis cs:{69.832},{-14.359}) {\tikz\pgfuseplotmark{m5c};};
\node[stars] at (axis cs:{69.846},{25.218}) {\tikz\pgfuseplotmark{m6c};};
\node[stars] at (axis cs:{69.947},{-1.052}) {\tikz\pgfuseplotmark{m6b};};
\node[stars] at (axis cs:{70.014},{12.197}) {\tikz\pgfuseplotmark{m5cv};};
\node[stars] at (axis cs:{70.028},{-24.482}) {\tikz\pgfuseplotmark{m6a};};
\node[stars] at (axis cs:{70.110},{-19.671}) {\tikz\pgfuseplotmark{m4cv};};
\node[stars] at (axis cs:{70.332},{28.615}) {\tikz\pgfuseplotmark{m6a};};
\node[stars] at (axis cs:{70.459},{38.280}) {\tikz\pgfuseplotmark{m6b};};
\node[stars] at (axis cs:{70.515},{-37.144}) {\tikz\pgfuseplotmark{m5b};};
\node[stars] at (axis cs:{70.561},{22.957}) {\tikz\pgfuseplotmark{m4cb};};
\node[stars] at (axis cs:{70.789},{-30.765}) {\tikz\pgfuseplotmark{m6a};};
\node[stars] at (axis cs:{70.807},{24.089}) {\tikz\pgfuseplotmark{m6c};};
\node[stars] at (axis cs:{71.022},{-8.504}) {\tikz\pgfuseplotmark{m6av};};
\node[stars] at (axis cs:{71.033},{-18.666}) {\tikz\pgfuseplotmark{m6a};};
\node[stars] at (axis cs:{71.108},{11.146}) {\tikz\pgfuseplotmark{m5cb};};
\node[stars] at (axis cs:{71.267},{-21.283}) {\tikz\pgfuseplotmark{m6a};};
\node[stars] at (axis cs:{71.376},{-3.255}) {\tikz\pgfuseplotmark{m4b};};
\node[stars] at (axis cs:{71.427},{23.628}) {\tikz\pgfuseplotmark{m6c};};
\node[stars] at (axis cs:{71.481},{-39.357}) {\tikz\pgfuseplotmark{m6b};};
\node[stars] at (axis cs:{71.507},{11.705}) {\tikz\pgfuseplotmark{m5c};};
\node[stars] at (axis cs:{71.570},{18.735}) {\tikz\pgfuseplotmark{m6b};};
\node[stars] at (axis cs:{71.601},{-2.954}) {\tikz\pgfuseplotmark{m6c};};
\node[stars] at (axis cs:{71.607},{-28.087}) {\tikz\pgfuseplotmark{m6cv};};
\node[stars] at (axis cs:{71.901},{-16.934}) {\tikz\pgfuseplotmark{m5cv};};
\node[stars] at (axis cs:{71.957},{-30.020}) {\tikz\pgfuseplotmark{m6c};};
\node[stars] at (axis cs:{72.136},{-16.329}) {\tikz\pgfuseplotmark{m6av};};
\node[stars] at (axis cs:{72.152},{-5.674}) {\tikz\pgfuseplotmark{m6a};};
\node[stars] at (axis cs:{72.186},{3.588}) {\tikz\pgfuseplotmark{m6b};};
\node[stars] at (axis cs:{72.304},{31.437}) {\tikz\pgfuseplotmark{m6a};};
\node[stars] at (axis cs:{72.329},{32.588}) {\tikz\pgfuseplotmark{m6a};};
\node[stars] at (axis cs:{72.426},{-13.770}) {\tikz\pgfuseplotmark{m6c};};
\node[stars] at (axis cs:{72.434},{15.904}) {\tikz\pgfuseplotmark{m6bb};};
\node[stars] at (axis cs:{72.460},{6.961}) {\tikz\pgfuseplotmark{m3cv};};
\node[stars] at (axis cs:{72.478},{37.488}) {\tikz\pgfuseplotmark{m5b};};
\node[stars] at (axis cs:{72.548},{-16.217}) {\tikz\pgfuseplotmark{m5b};};
\node[stars] at (axis cs:{72.653},{8.900}) {\tikz\pgfuseplotmark{m4c};};
\node[stars] at (axis cs:{72.802},{5.605}) {\tikz\pgfuseplotmark{m4av};};
\node[stars] at (axis cs:{72.844},{18.840}) {\tikz\pgfuseplotmark{m5bv};};
\node[stars] at (axis cs:{72.868},{-34.906}) {\tikz\pgfuseplotmark{m6a};};
\node[stars] at (axis cs:{72.931},{9.975}) {\tikz\pgfuseplotmark{m6b};};
\node[stars] at (axis cs:{72.958},{13.655}) {\tikz\pgfuseplotmark{m6c};};
\node[stars] at (axis cs:{73.133},{14.251}) {\tikz\pgfuseplotmark{m5av};};
\node[stars] at (axis cs:{73.158},{36.703}) {\tikz\pgfuseplotmark{m5a};};
\node[stars] at (axis cs:{73.196},{27.897}) {\tikz\pgfuseplotmark{m6b};};
\node[stars] at (axis cs:{73.224},{-5.453}) {\tikz\pgfuseplotmark{m4c};};
\node[stars] at (axis cs:{73.345},{2.508}) {\tikz\pgfuseplotmark{m5cv};};
\node[stars] at (axis cs:{73.563},{2.441}) {\tikz\pgfuseplotmark{m4av};};
\node[stars] at (axis cs:{73.695},{11.426}) {\tikz\pgfuseplotmark{m5c};};
\node[stars] at (axis cs:{73.699},{7.779}) {\tikz\pgfuseplotmark{m5c};};
\node[stars] at (axis cs:{73.711},{0.467}) {\tikz\pgfuseplotmark{m6b};};
\node[stars] at (axis cs:{73.724},{10.151}) {\tikz\pgfuseplotmark{m5a};};
\node[stars] at (axis cs:{73.728},{-39.628}) {\tikz\pgfuseplotmark{m6b};};
\node[stars] at (axis cs:{73.743},{19.485}) {\tikz\pgfuseplotmark{m6c};};
\node[stars] at (axis cs:{73.778},{-16.741}) {\tikz\pgfuseplotmark{m6a};};
\node[stars] at (axis cs:{73.828},{-16.418}) {\tikz\pgfuseplotmark{m6a};};
\node[stars] at (axis cs:{73.959},{15.040}) {\tikz\pgfuseplotmark{m6a};};
\node[stars] at (axis cs:{74.037},{7.905}) {\tikz\pgfuseplotmark{m6c};};
\node[stars] at (axis cs:{74.065},{24.592}) {\tikz\pgfuseplotmark{m6c};};
\node[stars] at (axis cs:{74.084},{36.169}) {\tikz\pgfuseplotmark{m6b};};
\node[stars] at (axis cs:{74.093},{13.514}) {\tikz\pgfuseplotmark{m4b};};
\node[stars] at (axis cs:{74.101},{-5.171}) {\tikz\pgfuseplotmark{m5cb};};
\node[stars] at (axis cs:{74.248},{33.166}) {\tikz\pgfuseplotmark{m3av};};
\node[stars] at (axis cs:{74.322},{-1.067}) {\tikz\pgfuseplotmark{m6c};};
\node[stars] at (axis cs:{74.343},{17.154}) {\tikz\pgfuseplotmark{m5c};};
\node[stars] at (axis cs:{74.436},{-14.232}) {\tikz\pgfuseplotmark{m6c};};
\node[stars] at (axis cs:{74.453},{23.948}) {\tikz\pgfuseplotmark{m6a};};
\node[stars] at (axis cs:{74.539},{25.050}) {\tikz\pgfuseplotmark{m6a};};
\node[stars] at (axis cs:{74.545},{-2.213}) {\tikz\pgfuseplotmark{m6c};};
\node[stars] at (axis cs:{74.637},{1.714}) {\tikz\pgfuseplotmark{m4cv};};
\node[stars] at (axis cs:{74.748},{14.543}) {\tikz\pgfuseplotmark{m6bvb};};
\node[stars] at (axis cs:{74.756},{-16.376}) {\tikz\pgfuseplotmark{m6av};};
\node[stars] at (axis cs:{74.814},{37.890}) {\tikz\pgfuseplotmark{m5bb};};
\node[stars] at (axis cs:{74.960},{-10.263}) {\tikz\pgfuseplotmark{m5c};};
\node[stars] at (axis cs:{74.982},{-12.537}) {\tikz\pgfuseplotmark{m5av};};
\node[stars] at (axis cs:{75.076},{39.395}) {\tikz\pgfuseplotmark{m6bb};};
\node[stars] at (axis cs:{75.097},{39.655}) {\tikz\pgfuseplotmark{m6c};};
\node[stars] at (axis cs:{75.166},{-2.066}) {\tikz\pgfuseplotmark{m6c};};
\node[stars] at (axis cs:{75.204},{-5.754}) {\tikz\pgfuseplotmark{m6cv};};
\node[stars] at (axis cs:{75.357},{-20.052}) {\tikz\pgfuseplotmark{m5b};};
\node[stars] at (axis cs:{75.360},{-7.174}) {\tikz\pgfuseplotmark{m5av};};
\node[stars] at (axis cs:{75.394},{-39.718}) {\tikz\pgfuseplotmark{m6b};};
\node[stars] at (axis cs:{75.460},{0.722}) {\tikz\pgfuseplotmark{m6b};};
\node[stars] at (axis cs:{75.541},{-26.275}) {\tikz\pgfuseplotmark{m5b};};
\node[stars] at (axis cs:{75.595},{-31.771}) {\tikz\pgfuseplotmark{m6b};};
\node[stars] at (axis cs:{75.687},{-22.795}) {\tikz\pgfuseplotmark{m6a};};
\node[stars] at (axis cs:{75.689},{-4.210}) {\tikz\pgfuseplotmark{m6a};};
\node[stars] at (axis cs:{75.774},{21.590}) {\tikz\pgfuseplotmark{m5a};};
\node[stars] at (axis cs:{75.967},{-14.370}) {\tikz\pgfuseplotmark{m6c};};
\node[stars] at (axis cs:{75.972},{-24.388}) {\tikz\pgfuseplotmark{m6a};};
\node[stars] at (axis cs:{76.061},{30.494}) {\tikz\pgfuseplotmark{m6c};};
\node[stars] at (axis cs:{76.090},{21.278}) {\tikz\pgfuseplotmark{m6c};};
\node[stars] at (axis cs:{76.102},{-35.483}) {\tikz\pgfuseplotmark{m5ab};};
\node[stars] at (axis cs:{76.109},{-35.705}) {\tikz\pgfuseplotmark{m6cv};};
\node[stars] at (axis cs:{76.142},{15.404}) {\tikz\pgfuseplotmark{m5av};};
\node[stars] at (axis cs:{76.227},{-3.040}) {\tikz\pgfuseplotmark{m6b};};
\node[stars] at (axis cs:{76.317},{-26.152}) {\tikz\pgfuseplotmark{m6a};};
\node[stars] at (axis cs:{76.349},{1.177}) {\tikz\pgfuseplotmark{m6bv};};
\node[stars] at (axis cs:{76.365},{-22.371}) {\tikz\pgfuseplotmark{m3cv};};
\node[stars] at (axis cs:{76.653},{-13.123}) {\tikz\pgfuseplotmark{m6b};};
\node[stars] at (axis cs:{76.690},{-4.655}) {\tikz\pgfuseplotmark{m5bv};};
\node[stars] at (axis cs:{76.791},{-19.392}) {\tikz\pgfuseplotmark{m6c};};
\node[stars] at (axis cs:{76.854},{-12.491}) {\tikz\pgfuseplotmark{m6bv};};
\node[stars] at (axis cs:{76.863},{18.645}) {\tikz\pgfuseplotmark{m5b};};
\node[stars] at (axis cs:{76.910},{9.472}) {\tikz\pgfuseplotmark{m6cb};};
\node[stars] at (axis cs:{76.952},{20.418}) {\tikz\pgfuseplotmark{m5c};};
\node[stars] at (axis cs:{76.958},{-12.593}) {\tikz\pgfuseplotmark{m6c};};
\node[stars] at (axis cs:{76.962},{-5.086}) {\tikz\pgfuseplotmark{m3av};};
\node[stars] at (axis cs:{76.970},{8.498}) {\tikz\pgfuseplotmark{m5c};};
\node[stars] at (axis cs:{76.981},{21.705}) {\tikz\pgfuseplotmark{m6avb};};
\node[stars] at (axis cs:{77.028},{24.265}) {\tikz\pgfuseplotmark{m6a};};
\node[stars] at (axis cs:{77.055},{-15.095}) {\tikz\pgfuseplotmark{m6c};};
\node[stars] at (axis cs:{77.084},{-8.665}) {\tikz\pgfuseplotmark{m6ab};};
\node[stars] at (axis cs:{77.182},{-4.456}) {\tikz\pgfuseplotmark{m5b};};
\node[stars] at (axis cs:{77.287},{-8.754}) {\tikz\pgfuseplotmark{m4cv};};
\node[stars] at (axis cs:{77.332},{9.829}) {\tikz\pgfuseplotmark{m5cv};};
\node[stars] at (axis cs:{77.425},{15.597}) {\tikz\pgfuseplotmark{m5a};};
\node[stars] at (axis cs:{77.438},{28.030}) {\tikz\pgfuseplotmark{m6bvb};};
\node[stars] at (axis cs:{77.514},{-0.565}) {\tikz\pgfuseplotmark{m6b};};
\node[stars] at (axis cs:{77.578},{37.302}) {\tikz\pgfuseplotmark{m6cb};};
\node[stars] at (axis cs:{77.686},{-25.910}) {\tikz\pgfuseplotmark{m6c};};
\node[stars] at (axis cs:{77.742},{-2.254}) {\tikz\pgfuseplotmark{m6c};};
\node[stars] at (axis cs:{77.830},{-2.491}) {\tikz\pgfuseplotmark{m6b};};
\node[stars] at (axis cs:{77.845},{-11.849}) {\tikz\pgfuseplotmark{m6av};};
\node[stars] at (axis cs:{77.910},{29.904}) {\tikz\pgfuseplotmark{m6c};};
\node[stars] at (axis cs:{77.923},{16.045}) {\tikz\pgfuseplotmark{m5c};};
\node[stars] at (axis cs:{77.939},{1.037}) {\tikz\pgfuseplotmark{m6bb};};
\node[stars] at (axis cs:{78.075},{-11.869}) {\tikz\pgfuseplotmark{m4c};};
\node[stars] at (axis cs:{78.201},{-6.057}) {\tikz\pgfuseplotmark{m6b};};
\node[stars] at (axis cs:{78.233},{-16.205}) {\tikz\pgfuseplotmark{m3cv};};
\node[stars] at (axis cs:{78.308},{-12.941}) {\tikz\pgfuseplotmark{m4cvb};};
\node[stars] at (axis cs:{78.323},{37.337}) {\tikz\pgfuseplotmark{m6cb};};
\node[stars] at (axis cs:{78.323},{2.861}) {\tikz\pgfuseplotmark{m4cvb};};
\node[stars] at (axis cs:{78.357},{38.484}) {\tikz\pgfuseplotmark{m5a};};
\node[stars] at (axis cs:{78.381},{1.968}) {\tikz\pgfuseplotmark{m6b};};
\node[stars] at (axis cs:{78.389},{-8.148}) {\tikz\pgfuseplotmark{m6c};};
\node[stars] at (axis cs:{78.447},{0.560}) {\tikz\pgfuseplotmark{m6cv};};
\node[stars] at (axis cs:{78.500},{-14.607}) {\tikz\pgfuseplotmark{m6c};};
\node[stars] at (axis cs:{78.620},{-35.977}) {\tikz\pgfuseplotmark{m6a};};
\node[stars] at (axis cs:{78.634},{-8.202}) {\tikz\pgfuseplotmark{m1bv};};
\node[stars] at (axis cs:{78.684},{5.156}) {\tikz\pgfuseplotmark{m5cv};};
\node[stars] at (axis cs:{78.827},{-1.409}) {\tikz\pgfuseplotmark{m6c};};
\node[stars] at (axis cs:{78.852},{-26.943}) {\tikz\pgfuseplotmark{m5b};};
\node[stars] at (axis cs:{78.852},{32.688}) {\tikz\pgfuseplotmark{m5bvb};};
\node[stars] at (axis cs:{78.865},{22.285}) {\tikz\pgfuseplotmark{m6c};};
\node[stars] at (axis cs:{79.017},{11.341}) {\tikz\pgfuseplotmark{m6a};};
\node[stars] at (axis cs:{79.044},{-11.353}) {\tikz\pgfuseplotmark{m6c};};
\node[stars] at (axis cs:{79.076},{34.312}) {\tikz\pgfuseplotmark{m6bvb};};
\node[stars] at (axis cs:{79.171},{1.947}) {\tikz\pgfuseplotmark{m6c};};
\node[stars] at (axis cs:{79.371},{-34.895}) {\tikz\pgfuseplotmark{m5a};};
\node[stars] at (axis cs:{79.402},{-6.844}) {\tikz\pgfuseplotmark{m4ab};};
\node[stars] at (axis cs:{79.418},{-13.520}) {\tikz\pgfuseplotmark{m5cv};};
\node[stars] at (axis cs:{79.544},{33.372}) {\tikz\pgfuseplotmark{m5av};};
\node[stars] at (axis cs:{79.579},{33.767}) {\tikz\pgfuseplotmark{m6cv};};
\node[stars] at (axis cs:{79.710},{-18.130}) {\tikz\pgfuseplotmark{m6bb};};
\node[stars] at (axis cs:{79.750},{33.748}) {\tikz\pgfuseplotmark{m5cv};};
\node[stars] at (axis cs:{79.797},{2.596}) {\tikz\pgfuseplotmark{m5c};};
\node[stars] at (axis cs:{79.811},{20.135}) {\tikz\pgfuseplotmark{m6bb};};
\node[stars] at (axis cs:{79.819},{22.096}) {\tikz\pgfuseplotmark{m5b};};
\node[stars] at (axis cs:{79.823},{-18.520}) {\tikz\pgfuseplotmark{m6cvb};};
\node[stars] at (axis cs:{79.849},{-27.369}) {\tikz\pgfuseplotmark{m6b};};
\node[stars] at (axis cs:{79.849},{33.985}) {\tikz\pgfuseplotmark{m6c};};
\node[stars] at (axis cs:{79.894},{-13.177}) {\tikz\pgfuseplotmark{m4c};};
\node[stars] at (axis cs:{79.897},{-1.412}) {\tikz\pgfuseplotmark{m6c};};
\node[stars] at (axis cs:{79.996},{-12.315}) {\tikz\pgfuseplotmark{m5c};};
\node[stars] at (axis cs:{80.004},{33.958}) {\tikz\pgfuseplotmark{m5b};};
\node[stars] at (axis cs:{80.086},{-34.699}) {\tikz\pgfuseplotmark{m6c};};
\node[stars] at (axis cs:{80.110},{-5.368}) {\tikz\pgfuseplotmark{m6cb};};
\node[stars] at (axis cs:{80.112},{-21.240}) {\tikz\pgfuseplotmark{m5avb};};
\node[stars] at (axis cs:{80.182},{2.545}) {\tikz\pgfuseplotmark{m6c};};
\node[stars] at (axis cs:{80.236},{19.814}) {\tikz\pgfuseplotmark{m6c};};
\node[stars] at (axis cs:{80.247},{27.957}) {\tikz\pgfuseplotmark{m6c};};
\node[stars] at (axis cs:{80.303},{29.570}) {\tikz\pgfuseplotmark{m6a};};
\node[stars] at (axis cs:{80.320},{-34.345}) {\tikz\pgfuseplotmark{m6bv};};
\node[stars] at (axis cs:{80.383},{-0.416}) {\tikz\pgfuseplotmark{m6a};};
\node[stars] at (axis cs:{80.432},{8.428}) {\tikz\pgfuseplotmark{m6a};};
\node[stars] at (axis cs:{80.441},{-0.382}) {\tikz\pgfuseplotmark{m5av};};
\node[stars] at (axis cs:{80.443},{-24.773}) {\tikz\pgfuseplotmark{m5bb};};
\node[stars] at (axis cs:{80.480},{-17.604}) {\tikz\pgfuseplotmark{m6c};};
\node[stars] at (axis cs:{80.708},{3.544}) {\tikz\pgfuseplotmark{m5bb};};
\node[stars] at (axis cs:{80.800},{-26.705}) {\tikz\pgfuseplotmark{m6c};};
\node[stars] at (axis cs:{80.827},{-8.415}) {\tikz\pgfuseplotmark{m6cb};};
\node[stars] at (axis cs:{80.845},{28.937}) {\tikz\pgfuseplotmark{m6c};};
\node[stars] at (axis cs:{80.850},{-39.678}) {\tikz\pgfuseplotmark{m6av};};
\node[stars] at (axis cs:{80.876},{-13.927}) {\tikz\pgfuseplotmark{m5c};};
\node[stars] at (axis cs:{80.879},{5.323}) {\tikz\pgfuseplotmark{m6c};};
\node[stars] at (axis cs:{80.907},{16.699}) {\tikz\pgfuseplotmark{m6b};};
\node[stars] at (axis cs:{80.926},{-0.160}) {\tikz\pgfuseplotmark{m6av};};
\node[stars] at (axis cs:{80.964},{-0.866}) {\tikz\pgfuseplotmark{m6bb};};
\node[stars] at (axis cs:{80.987},{-7.808}) {\tikz\pgfuseplotmark{m4c};};
\node[stars] at (axis cs:{81.106},{17.384}) {\tikz\pgfuseplotmark{m5bvb};};
\node[stars] at (axis cs:{81.119},{-16.976}) {\tikz\pgfuseplotmark{m6a};};
\node[stars] at (axis cs:{81.119},{-2.397}) {\tikz\pgfuseplotmark{m3cvb};};
\node[stars] at (axis cs:{81.120},{-0.891}) {\tikz\pgfuseplotmark{m5b};};
\node[stars] at (axis cs:{81.150},{2.353}) {\tikz\pgfuseplotmark{m6c};};
\node[stars] at (axis cs:{81.160},{31.224}) {\tikz\pgfuseplotmark{m6c};};
\node[stars] at (axis cs:{81.160},{31.143}) {\tikz\pgfuseplotmark{m6b};};
\node[stars] at (axis cs:{81.163},{37.385}) {\tikz\pgfuseplotmark{m5bb};};
\node[stars] at (axis cs:{81.187},{36.200}) {\tikz\pgfuseplotmark{m6c};};
\node[stars] at (axis cs:{81.187},{1.846}) {\tikz\pgfuseplotmark{m5bv};};
\node[stars] at (axis cs:{81.257},{-10.329}) {\tikz\pgfuseplotmark{m6a};};
\node[stars] at (axis cs:{81.283},{6.349}) {\tikz\pgfuseplotmark{m2av};};
\node[stars] at (axis cs:{81.402},{-1.491}) {\tikz\pgfuseplotmark{m6c};};
\node[stars] at (axis cs:{81.446},{0.520}) {\tikz\pgfuseplotmark{m6cv};};
\node[stars] at (axis cs:{81.499},{-19.695}) {\tikz\pgfuseplotmark{m6ab};};
\node[stars] at (axis cs:{81.510},{-5.518}) {\tikz\pgfuseplotmark{m6c};};
\node[stars] at (axis cs:{81.524},{16.700}) {\tikz\pgfuseplotmark{m6cv};};
\node[stars] at (axis cs:{81.573},{28.607}) {\tikz\pgfuseplotmark{m2ab};};
\node[stars] at (axis cs:{81.662},{6.869}) {\tikz\pgfuseplotmark{m6c};};
\node[stars] at (axis cs:{81.703},{34.392}) {\tikz\pgfuseplotmark{m6b};};
\node[stars] at (axis cs:{81.709},{3.095}) {\tikz\pgfuseplotmark{m5av};};
\node[stars] at (axis cs:{81.714},{33.262}) {\tikz\pgfuseplotmark{m6c};};
\node[stars] at (axis cs:{81.726},{35.457}) {\tikz\pgfuseplotmark{m6c};};
\node[stars] at (axis cs:{81.770},{-11.901}) {\tikz\pgfuseplotmark{m6cv};};
\node[stars] at (axis cs:{81.784},{30.208}) {\tikz\pgfuseplotmark{m6a};};
\node[stars] at (axis cs:{81.792},{17.962}) {\tikz\pgfuseplotmark{m5cb};};
\node[stars] at (axis cs:{81.808},{15.258}) {\tikz\pgfuseplotmark{m6c};};
\node[stars] at (axis cs:{81.814},{2.341}) {\tikz\pgfuseplotmark{m6c};};
\node[stars] at (axis cs:{81.902},{-21.376}) {\tikz\pgfuseplotmark{m6b};};
\node[stars] at (axis cs:{81.909},{21.937}) {\tikz\pgfuseplotmark{m5b};};
\node[stars] at (axis cs:{81.912},{34.476}) {\tikz\pgfuseplotmark{m5bb};};
\node[stars] at (axis cs:{81.940},{15.874}) {\tikz\pgfuseplotmark{m6a};};
\node[stars] at (axis cs:{82.004},{33.763}) {\tikz\pgfuseplotmark{m6cb};};
\node[stars] at (axis cs:{82.006},{1.298}) {\tikz\pgfuseplotmark{m6c};};
\node[stars] at (axis cs:{82.007},{17.239}) {\tikz\pgfuseplotmark{m6a};};
\node[stars] at (axis cs:{82.061},{-20.759}) {\tikz\pgfuseplotmark{m3a};};
\node[stars] at (axis cs:{82.064},{-37.231}) {\tikz\pgfuseplotmark{m6a};};
\node[stars] at (axis cs:{82.145},{13.679}) {\tikz\pgfuseplotmark{m6c};};
\node[stars] at (axis cs:{82.237},{-3.307}) {\tikz\pgfuseplotmark{m6c};};
\node[stars] at (axis cs:{82.319},{25.150}) {\tikz\pgfuseplotmark{m6ab};};
\node[stars] at (axis cs:{82.349},{-3.446}) {\tikz\pgfuseplotmark{m6av};};
\node[stars] at (axis cs:{82.419},{29.186}) {\tikz\pgfuseplotmark{m6c};};
\node[stars] at (axis cs:{82.433},{-1.092}) {\tikz\pgfuseplotmark{m5a};};
\node[stars] at (axis cs:{82.478},{1.789}) {\tikz\pgfuseplotmark{m6a};};
\node[stars] at (axis cs:{82.583},{4.205}) {\tikz\pgfuseplotmark{m6cv};};
\node[stars] at (axis cs:{82.586},{-7.435}) {\tikz\pgfuseplotmark{m6c};};
\node[stars] at (axis cs:{82.609},{15.360}) {\tikz\pgfuseplotmark{m6b};};
\node[stars] at (axis cs:{82.681},{22.462}) {\tikz\pgfuseplotmark{m6c};};
\node[stars] at (axis cs:{82.688},{39.826}) {\tikz\pgfuseplotmark{m6cb};};
\node[stars] at (axis cs:{82.696},{5.948}) {\tikz\pgfuseplotmark{m4cb};};
\node[stars] at (axis cs:{82.782},{-20.863}) {\tikz\pgfuseplotmark{m6a};};
\node[stars] at (axis cs:{82.803},{-35.470}) {\tikz\pgfuseplotmark{m4b};};
\node[stars] at (axis cs:{82.811},{3.292}) {\tikz\pgfuseplotmark{m5cb};};
\node[stars] at (axis cs:{82.837},{-6.708}) {\tikz\pgfuseplotmark{m6c};};
\node[stars] at (axis cs:{82.983},{-7.302}) {\tikz\pgfuseplotmark{m5av};};
\node[stars] at (axis cs:{83.002},{-0.299}) {\tikz\pgfuseplotmark{m2cv};};
\node[stars] at (axis cs:{83.053},{18.594}) {\tikz\pgfuseplotmark{m4cv};};
\node[stars] at (axis cs:{83.059},{17.058}) {\tikz\pgfuseplotmark{m6bvb};};
\node[stars] at (axis cs:{83.158},{0.012}) {\tikz\pgfuseplotmark{m6c};};
\node[stars] at (axis cs:{83.172},{-1.592}) {\tikz\pgfuseplotmark{m5c};};
\node[stars] at (axis cs:{83.182},{32.192}) {\tikz\pgfuseplotmark{m5a};};
\node[stars] at (axis cs:{83.183},{-17.822}) {\tikz\pgfuseplotmark{m3a};};
\node[stars] at (axis cs:{83.214},{-38.513}) {\tikz\pgfuseplotmark{m5c};};
\node[stars] at (axis cs:{83.281},{-35.139}) {\tikz\pgfuseplotmark{m6a};};
\node[stars] at (axis cs:{83.364},{32.801}) {\tikz\pgfuseplotmark{m6c};};
\node[stars] at (axis cs:{83.381},{-1.156}) {\tikz\pgfuseplotmark{m5cv};};
\node[stars] at (axis cs:{83.382},{18.540}) {\tikz\pgfuseplotmark{m6av};};
\node[stars] at (axis cs:{83.409},{34.725}) {\tikz\pgfuseplotmark{m6c};};
\node[stars] at (axis cs:{83.412},{20.474}) {\tikz\pgfuseplotmark{m6c};};
\node[stars] at (axis cs:{83.476},{14.305}) {\tikz\pgfuseplotmark{m6a};};
\node[stars] at (axis cs:{83.510},{-7.024}) {\tikz\pgfuseplotmark{m6c};};
\node[stars] at (axis cs:{83.516},{-1.036}) {\tikz\pgfuseplotmark{m6c};};
\node[stars] at (axis cs:{83.517},{-1.470}) {\tikz\pgfuseplotmark{m6b};};
\node[stars] at (axis cs:{83.570},{3.767}) {\tikz\pgfuseplotmark{m5c};};
\node[stars] at (axis cs:{83.622},{-0.012}) {\tikz\pgfuseplotmark{m6c};};
\node[stars] at (axis cs:{83.705},{9.489}) {\tikz\pgfuseplotmark{m4c};};
\node[stars] at (axis cs:{83.761},{-6.002}) {\tikz\pgfuseplotmark{m5ab};};
\node[stars] at (axis cs:{83.784},{9.934}) {\tikz\pgfuseplotmark{m3cb};};
\node[stars] at (axis cs:{83.805},{10.240}) {\tikz\pgfuseplotmark{m6a};};
\node[stars] at (axis cs:{83.814},{-33.080}) {\tikz\pgfuseplotmark{m6a};};
\node[stars] at (axis cs:{83.843},{-4.424}) {\tikz\pgfuseplotmark{m6c};};
\node[stars] at (axis cs:{83.845},{-5.416}) {\tikz\pgfuseplotmark{m5bb};};
\node[stars] at (axis cs:{83.847},{-4.838}) {\tikz\pgfuseplotmark{m5ab};};
\node[stars] at (axis cs:{83.858},{-5.910}) {\tikz\pgfuseplotmark{m3ab};};
\node[stars] at (axis cs:{83.863},{24.039}) {\tikz\pgfuseplotmark{m5c};};
\node[stars] at (axis cs:{83.879},{-4.364}) {\tikz\pgfuseplotmark{m6cb};};
\node[stars] at (axis cs:{83.900},{-3.253}) {\tikz\pgfuseplotmark{m6cv};};
\node[stars] at (axis cs:{83.915},{-4.856}) {\tikz\pgfuseplotmark{m5c};};
\node[stars] at (axis cs:{83.981},{27.662}) {\tikz\pgfuseplotmark{m6c};};
\node[stars] at (axis cs:{84.043},{-28.708}) {\tikz\pgfuseplotmark{m6c};};
\node[stars] at (axis cs:{84.053},{-1.202}) {\tikz\pgfuseplotmark{m2av};};
\node[stars] at (axis cs:{84.149},{-6.065}) {\tikz\pgfuseplotmark{m6ab};};
\node[stars] at (axis cs:{84.227},{9.291}) {\tikz\pgfuseplotmark{m4b};};
\node[stars] at (axis cs:{84.266},{17.040}) {\tikz\pgfuseplotmark{m6a};};
\node[stars] at (axis cs:{84.268},{11.035}) {\tikz\pgfuseplotmark{m6b};};
\node[stars] at (axis cs:{84.287},{-11.775}) {\tikz\pgfuseplotmark{m6b};};
\node[stars] at (axis cs:{84.287},{26.924}) {\tikz\pgfuseplotmark{m6bb};};
\node[stars] at (axis cs:{84.319},{-27.871}) {\tikz\pgfuseplotmark{m6c};};
\node[stars] at (axis cs:{84.330},{8.952}) {\tikz\pgfuseplotmark{m6b};};
\node[stars] at (axis cs:{84.364},{-5.938}) {\tikz\pgfuseplotmark{m6bv};};
\node[stars] at (axis cs:{84.411},{21.142}) {\tikz\pgfuseplotmark{m3bv};};
\node[stars] at (axis cs:{84.436},{-28.690}) {\tikz\pgfuseplotmark{m5c};};
\node[stars] at (axis cs:{84.441},{33.559}) {\tikz\pgfuseplotmark{m6c};};
\node[stars] at (axis cs:{84.472},{-4.814}) {\tikz\pgfuseplotmark{m6cv};};
\node[stars] at (axis cs:{84.505},{7.541}) {\tikz\pgfuseplotmark{m6b};};
\node[stars] at (axis cs:{84.658},{-6.574}) {\tikz\pgfuseplotmark{m6bv};};
\node[stars] at (axis cs:{84.659},{30.492}) {\tikz\pgfuseplotmark{m5cvb};};
\node[stars] at (axis cs:{84.687},{-2.600}) {\tikz\pgfuseplotmark{m4avb};};
\node[stars] at (axis cs:{84.721},{-7.213}) {\tikz\pgfuseplotmark{m5a};};
\node[stars] at (axis cs:{84.739},{26.618}) {\tikz\pgfuseplotmark{m6c};};
\node[stars] at (axis cs:{84.796},{4.121}) {\tikz\pgfuseplotmark{m5av};};
\node[stars] at (axis cs:{84.818},{-17.849}) {\tikz\pgfuseplotmark{m6cb};};
\node[stars] at (axis cs:{84.826},{29.215}) {\tikz\pgfuseplotmark{m6bv};};
\node[stars] at (axis cs:{84.863},{21.763}) {\tikz\pgfuseplotmark{m6c};};
\node[stars] at (axis cs:{84.879},{-9.707}) {\tikz\pgfuseplotmark{m6c};};
\node[stars] at (axis cs:{84.880},{-3.564}) {\tikz\pgfuseplotmark{m6b};};
\node[stars] at (axis cs:{84.912},{-34.074}) {\tikz\pgfuseplotmark{m3av};};
\node[stars] at (axis cs:{84.934},{25.897}) {\tikz\pgfuseplotmark{m5c};};
\node[stars] at (axis cs:{84.958},{-32.629}) {\tikz\pgfuseplotmark{m5c};};
\node[stars] at (axis cs:{85.150},{31.358}) {\tikz\pgfuseplotmark{m6bv};};
\node[stars] at (axis cs:{85.155},{-2.825}) {\tikz\pgfuseplotmark{m6c};};
\node[stars] at (axis cs:{85.175},{31.921}) {\tikz\pgfuseplotmark{m6cv};};
\node[stars] at (axis cs:{85.190},{-1.942}) {\tikz\pgfuseplotmark{m2avb};};
\node[stars] at (axis cs:{85.192},{-10.409}) {\tikz\pgfuseplotmark{m6cv};};
\node[stars] at (axis cs:{85.211},{-1.129}) {\tikz\pgfuseplotmark{m5bv};};
\node[stars] at (axis cs:{85.273},{0.338}) {\tikz\pgfuseplotmark{m6b};};
\node[stars] at (axis cs:{85.324},{16.534}) {\tikz\pgfuseplotmark{m5av};};
\node[stars] at (axis cs:{85.337},{29.487}) {\tikz\pgfuseplotmark{m6cb};};
\node[stars] at (axis cs:{85.362},{-33.401}) {\tikz\pgfuseplotmark{m6c};};
\node[stars] at (axis cs:{85.418},{-2.896}) {\tikz\pgfuseplotmark{m6cv};};
\node[stars] at (axis cs:{85.423},{-16.726}) {\tikz\pgfuseplotmark{m6c};};
\node[stars] at (axis cs:{85.517},{22.660}) {\tikz\pgfuseplotmark{m6c};};
\node[stars] at (axis cs:{85.548},{-30.535}) {\tikz\pgfuseplotmark{m6c};};
\node[stars] at (axis cs:{85.558},{-22.374}) {\tikz\pgfuseplotmark{m6b};};
\node[stars] at (axis cs:{85.560},{-17.530}) {\tikz\pgfuseplotmark{m6c};};
\node[stars] at (axis cs:{85.563},{-34.668}) {\tikz\pgfuseplotmark{m5cv};};
\node[stars] at (axis cs:{85.619},{1.474}) {\tikz\pgfuseplotmark{m5b};};
\node[stars] at (axis cs:{85.725},{-6.796}) {\tikz\pgfuseplotmark{m6b};};
\node[stars] at (axis cs:{85.789},{-1.613}) {\tikz\pgfuseplotmark{m6cv};};
\node[stars] at (axis cs:{85.831},{23.204}) {\tikz\pgfuseplotmark{m6cv};};
\node[stars] at (axis cs:{85.840},{-18.557}) {\tikz\pgfuseplotmark{m6a};};
\node[stars] at (axis cs:{85.876},{-39.407}) {\tikz\pgfuseplotmark{m6c};};
\node[stars] at (axis cs:{86.116},{-22.448}) {\tikz\pgfuseplotmark{m4ab};};
\node[stars] at (axis cs:{86.118},{-20.126}) {\tikz\pgfuseplotmark{m6c};};
\node[stars] at (axis cs:{86.255},{12.888}) {\tikz\pgfuseplotmark{m6c};};
\node[stars] at (axis cs:{86.257},{4.008}) {\tikz\pgfuseplotmark{m6bb};};
\node[stars] at (axis cs:{86.500},{-32.306}) {\tikz\pgfuseplotmark{m5c};};
\node[stars] at (axis cs:{86.512},{-4.268}) {\tikz\pgfuseplotmark{m6cb};};
\node[stars] at (axis cs:{86.645},{1.168}) {\tikz\pgfuseplotmark{m6b};};
\node[stars] at (axis cs:{86.690},{15.822}) {\tikz\pgfuseplotmark{m6b};};
\node[stars] at (axis cs:{86.717},{9.522}) {\tikz\pgfuseplotmark{m6a};};
\node[stars] at (axis cs:{86.739},{-14.822}) {\tikz\pgfuseplotmark{m4a};};
\node[stars] at (axis cs:{86.769},{-28.639}) {\tikz\pgfuseplotmark{m6c};};
\node[stars] at (axis cs:{86.783},{-16.238}) {\tikz\pgfuseplotmark{m6c};};
\node[stars] at (axis cs:{86.805},{14.488}) {\tikz\pgfuseplotmark{m6av};};
\node[stars] at (axis cs:{86.828},{-35.674}) {\tikz\pgfuseplotmark{m6c};};
\node[stars] at (axis cs:{86.859},{17.729}) {\tikz\pgfuseplotmark{m5c};};
\node[stars] at (axis cs:{86.862},{-10.533}) {\tikz\pgfuseplotmark{m6bv};};
\node[stars] at (axis cs:{86.894},{25.569}) {\tikz\pgfuseplotmark{m6c};};
\node[stars] at (axis cs:{86.929},{13.899}) {\tikz\pgfuseplotmark{m5c};};
\node[stars] at (axis cs:{86.939},{-9.670}) {\tikz\pgfuseplotmark{m2bv};};
\node[stars] at (axis cs:{87.001},{6.454}) {\tikz\pgfuseplotmark{m5c};};
\node[stars] at (axis cs:{87.093},{20.869}) {\tikz\pgfuseplotmark{m6bb};};
\node[stars] at (axis cs:{87.146},{-4.094}) {\tikz\pgfuseplotmark{m6c};};
\node[stars] at (axis cs:{87.254},{24.568}) {\tikz\pgfuseplotmark{m5b};};
\node[stars] at (axis cs:{87.293},{39.181}) {\tikz\pgfuseplotmark{m5a};};
\node[stars] at (axis cs:{87.387},{12.651}) {\tikz\pgfuseplotmark{m5b};};
\node[stars] at (axis cs:{87.402},{-14.484}) {\tikz\pgfuseplotmark{m5cb};};
\node[stars] at (axis cs:{87.473},{-22.972}) {\tikz\pgfuseplotmark{m6b};};
\node[stars] at (axis cs:{87.511},{9.871}) {\tikz\pgfuseplotmark{m6a};};
\node[stars] at (axis cs:{87.554},{4.423}) {\tikz\pgfuseplotmark{m6b};};
\node[stars] at (axis cs:{87.620},{14.306}) {\tikz\pgfuseplotmark{m6c};};
\node[stars] at (axis cs:{87.625},{2.025}) {\tikz\pgfuseplotmark{m6b};};
\node[stars] at (axis cs:{87.740},{-35.768}) {\tikz\pgfuseplotmark{m3b};};
\node[stars] at (axis cs:{87.742},{27.968}) {\tikz\pgfuseplotmark{m6a};};
\node[stars] at (axis cs:{87.760},{37.305}) {\tikz\pgfuseplotmark{m5av};};
\node[stars] at (axis cs:{87.830},{-20.879}) {\tikz\pgfuseplotmark{m4a};};
\node[stars] at (axis cs:{87.842},{-7.518}) {\tikz\pgfuseplotmark{m5cv};};
\node[stars] at (axis cs:{87.857},{32.125}) {\tikz\pgfuseplotmark{m6cv};};
\node[stars] at (axis cs:{87.869},{-22.927}) {\tikz\pgfuseplotmark{m6c};};
\node[stars] at (axis cs:{87.872},{39.148}) {\tikz\pgfuseplotmark{m4b};};
\node[stars] at (axis cs:{87.998},{-29.449}) {\tikz\pgfuseplotmark{m6c};};
\node[stars] at (axis cs:{88.032},{-9.042}) {\tikz\pgfuseplotmark{m6b};};
\node[stars] at (axis cs:{88.093},{14.172}) {\tikz\pgfuseplotmark{m6av};};
\node[stars] at (axis cs:{88.098},{19.868}) {\tikz\pgfuseplotmark{m6bv};};
\node[stars] at (axis cs:{88.110},{1.855}) {\tikz\pgfuseplotmark{m5av};};
\node[stars] at (axis cs:{88.138},{-37.631}) {\tikz\pgfuseplotmark{m6a};};
\node[stars] at (axis cs:{88.165},{39.575}) {\tikz\pgfuseplotmark{m6c};};
\node[stars] at (axis cs:{88.167},{33.917}) {\tikz\pgfuseplotmark{m6bv};};
\node[stars] at (axis cs:{88.279},{-33.801}) {\tikz\pgfuseplotmark{m5bv};};
\node[stars] at (axis cs:{88.332},{27.612}) {\tikz\pgfuseplotmark{m5av};};
\node[stars] at (axis cs:{88.556},{10.586}) {\tikz\pgfuseplotmark{m6b};};
\node[stars] at (axis cs:{88.558},{-29.147}) {\tikz\pgfuseplotmark{m6c};};
\node[stars] at (axis cs:{88.566},{3.225}) {\tikz\pgfuseplotmark{m6c};};
\node[stars] at (axis cs:{88.596},{20.276}) {\tikz\pgfuseplotmark{m4cv};};
\node[stars] at (axis cs:{88.682},{-11.774}) {\tikz\pgfuseplotmark{m6a};};
\node[stars] at (axis cs:{88.683},{0.969}) {\tikz\pgfuseplotmark{m6b};};
\node[stars] at (axis cs:{88.719},{-39.958}) {\tikz\pgfuseplotmark{m6a};};
\node[stars] at (axis cs:{88.736},{19.749}) {\tikz\pgfuseplotmark{m6bv};};
\node[stars] at (axis cs:{88.746},{31.701}) {\tikz\pgfuseplotmark{m6b};};
\node[stars] at (axis cs:{88.793},{7.407}) {\tikz\pgfuseplotmark{m1bv};};
\node[stars] at (axis cs:{88.875},{-37.121}) {\tikz\pgfuseplotmark{m5b};};
\node[stars] at (axis cs:{88.876},{-4.616}) {\tikz\pgfuseplotmark{m6b};};
\node[stars] at (axis cs:{88.897},{-4.789}) {\tikz\pgfuseplotmark{m6c};};
\node[stars] at (axis cs:{88.955},{20.175}) {\tikz\pgfuseplotmark{m6bv};};
\node[stars] at (axis cs:{89.059},{-22.840}) {\tikz\pgfuseplotmark{m6b};};
\node[stars] at (axis cs:{89.087},{-31.382}) {\tikz\pgfuseplotmark{m6a};};
\node[stars] at (axis cs:{89.101},{-14.168}) {\tikz\pgfuseplotmark{m4a};};
\node[stars] at (axis cs:{89.117},{9.509}) {\tikz\pgfuseplotmark{m6b};};
\node[stars] at (axis cs:{89.141},{28.942}) {\tikz\pgfuseplotmark{m6c};};
\node[stars] at (axis cs:{89.143},{-23.215}) {\tikz\pgfuseplotmark{m6c};};
\node[stars] at (axis cs:{89.204},{-31.976}) {\tikz\pgfuseplotmark{m6c};};
\node[stars] at (axis cs:{89.206},{11.521}) {\tikz\pgfuseplotmark{m6b};};
\node[stars] at (axis cs:{89.234},{24.249}) {\tikz\pgfuseplotmark{m6b};};
\node[stars] at (axis cs:{89.384},{-35.283}) {\tikz\pgfuseplotmark{m4cv};};
\node[stars] at (axis cs:{89.477},{1.224}) {\tikz\pgfuseplotmark{m6c};};
\node[stars] at (axis cs:{89.499},{25.954}) {\tikz\pgfuseplotmark{m5a};};
\node[stars] at (axis cs:{89.549},{-0.994}) {\tikz\pgfuseplotmark{m6cv};};
\node[stars] at (axis cs:{89.602},{1.837}) {\tikz\pgfuseplotmark{m6bvb};};
\node[stars] at (axis cs:{89.707},{0.553}) {\tikz\pgfuseplotmark{m5cv};};
\node[stars] at (axis cs:{89.722},{12.808}) {\tikz\pgfuseplotmark{m6a};};
\node[stars] at (axis cs:{89.754},{-9.382}) {\tikz\pgfuseplotmark{m6cv};};
\node[stars] at (axis cs:{89.768},{-9.558}) {\tikz\pgfuseplotmark{m5b};};
\node[stars] at (axis cs:{89.930},{37.212}) {\tikz\pgfuseplotmark{m3av};};
\node[stars] at (axis cs:{90.014},{-3.074}) {\tikz\pgfuseplotmark{m5av};};
\node[stars] at (axis cs:{90.074},{-12.900}) {\tikz\pgfuseplotmark{m6cv};};
\node[stars] at (axis cs:{90.252},{27.572}) {\tikz\pgfuseplotmark{m6b};};
\node[stars] at (axis cs:{90.292},{31.034}) {\tikz\pgfuseplotmark{m6c};};
\node[stars] at (axis cs:{90.305},{-25.418}) {\tikz\pgfuseplotmark{m6b};};
\node[stars] at (axis cs:{90.318},{-33.912}) {\tikz\pgfuseplotmark{m6a};};
\node[stars] at (axis cs:{90.423},{22.401}) {\tikz\pgfuseplotmark{m6c};};
\node[stars] at (axis cs:{90.460},{-10.598}) {\tikz\pgfuseplotmark{m5bb};};
\node[stars] at (axis cs:{90.596},{9.647}) {\tikz\pgfuseplotmark{m4bv};};
\node[stars] at (axis cs:{90.641},{-14.497}) {\tikz\pgfuseplotmark{m6c};};
\node[stars] at (axis cs:{90.730},{32.636}) {\tikz\pgfuseplotmark{m6c};};
\node[stars] at (axis cs:{90.815},{-26.285}) {\tikz\pgfuseplotmark{m5b};};
\node[stars] at (axis cs:{90.853},{11.681}) {\tikz\pgfuseplotmark{m6b};};
\node[stars] at (axis cs:{90.864},{19.690}) {\tikz\pgfuseplotmark{m5cv};};
\node[stars] at (axis cs:{90.980},{20.138}) {\tikz\pgfuseplotmark{m5av};};
\node[stars] at (axis cs:{91.030},{23.263}) {\tikz\pgfuseplotmark{m4c};};
\node[stars] at (axis cs:{91.056},{-6.709}) {\tikz\pgfuseplotmark{m5cv};};
\node[stars] at (axis cs:{91.084},{-32.172}) {\tikz\pgfuseplotmark{m6av};};
\node[stars] at (axis cs:{91.242},{5.420}) {\tikz\pgfuseplotmark{m6a};};
\node[stars] at (axis cs:{91.243},{4.159}) {\tikz\pgfuseplotmark{m6a};};
\node[stars] at (axis cs:{91.246},{-16.484}) {\tikz\pgfuseplotmark{m5bv};};
\node[stars] at (axis cs:{91.261},{37.964}) {\tikz\pgfuseplotmark{m6c};};
\node[stars] at (axis cs:{91.363},{-10.242}) {\tikz\pgfuseplotmark{m6bv};};
\node[stars] at (axis cs:{91.363},{-35.513}) {\tikz\pgfuseplotmark{m6a};};
\node[stars] at (axis cs:{91.391},{33.599}) {\tikz\pgfuseplotmark{m6c};};
\node[stars] at (axis cs:{91.523},{-29.758}) {\tikz\pgfuseplotmark{m6a};};
\node[stars] at (axis cs:{91.536},{35.388}) {\tikz\pgfuseplotmark{m6bb};};
\node[stars] at (axis cs:{91.539},{-14.935}) {\tikz\pgfuseplotmark{m5a};};
\node[stars] at (axis cs:{91.594},{29.512}) {\tikz\pgfuseplotmark{m6bv};};
\node[stars] at (axis cs:{91.634},{-23.111}) {\tikz\pgfuseplotmark{m5c};};
\node[stars] at (axis cs:{91.646},{38.482}) {\tikz\pgfuseplotmark{m5cv};};
\node[stars] at (axis cs:{91.661},{-4.194}) {\tikz\pgfuseplotmark{m5c};};
\node[stars] at (axis cs:{91.740},{-21.812}) {\tikz\pgfuseplotmark{m6av};};
\node[stars] at (axis cs:{91.765},{-34.312}) {\tikz\pgfuseplotmark{m6av};};
\node[stars] at (axis cs:{91.882},{-37.253}) {\tikz\pgfuseplotmark{m5b};};
\node[stars] at (axis cs:{91.893},{14.768}) {\tikz\pgfuseplotmark{m4c};};
\node[stars] at (axis cs:{91.923},{-19.166}) {\tikz\pgfuseplotmark{m5cv};};
\node[stars] at (axis cs:{92.199},{-6.821}) {\tikz\pgfuseplotmark{m6c};};
\node[stars] at (axis cs:{92.241},{-22.427}) {\tikz\pgfuseplotmark{m5c};};
\node[stars] at (axis cs:{92.241},{2.499}) {\tikz\pgfuseplotmark{m6ab};};
\node[stars] at (axis cs:{92.334},{-18.126}) {\tikz\pgfuseplotmark{m6c};};
\node[stars] at (axis cs:{92.385},{22.190}) {\tikz\pgfuseplotmark{m6bv};};
\node[stars] at (axis cs:{92.394},{-14.585}) {\tikz\pgfuseplotmark{m6a};};
\node[stars] at (axis cs:{92.401},{-5.711}) {\tikz\pgfuseplotmark{m6c};};
\node[stars] at (axis cs:{92.433},{23.113}) {\tikz\pgfuseplotmark{m6av};};
\node[stars] at (axis cs:{92.446},{-26.701}) {\tikz\pgfuseplotmark{m6c};};
\node[stars] at (axis cs:{92.450},{-22.774}) {\tikz\pgfuseplotmark{m6a};};
\node[stars] at (axis cs:{92.645},{-27.154}) {\tikz\pgfuseplotmark{m6a};};
\node[stars] at (axis cs:{92.755},{-6.754}) {\tikz\pgfuseplotmark{m6cv};};
\node[stars] at (axis cs:{92.757},{18.130}) {\tikz\pgfuseplotmark{m6cv};};
\node[stars] at (axis cs:{92.807},{-26.482}) {\tikz\pgfuseplotmark{m6b};};
\node[stars] at (axis cs:{92.866},{13.638}) {\tikz\pgfuseplotmark{m6b};};
\node[stars] at (axis cs:{92.885},{24.420}) {\tikz\pgfuseplotmark{m6a};};
\node[stars] at (axis cs:{92.932},{-4.665}) {\tikz\pgfuseplotmark{m6cv};};
\node[stars] at (axis cs:{92.966},{-6.550}) {\tikz\pgfuseplotmark{m5b};};
\node[stars] at (axis cs:{92.985},{14.209}) {\tikz\pgfuseplotmark{m4cb};};
\node[stars] at (axis cs:{93.006},{19.790}) {\tikz\pgfuseplotmark{m6ab};};
\node[stars] at (axis cs:{93.014},{16.130}) {\tikz\pgfuseplotmark{m5b};};
\node[stars] at (axis cs:{93.080},{22.908}) {\tikz\pgfuseplotmark{m6cv};};
\node[stars] at (axis cs:{93.084},{32.693}) {\tikz\pgfuseplotmark{m6a};};
\node[stars] at (axis cs:{93.248},{6.016}) {\tikz\pgfuseplotmark{m6cv};};
\node[stars] at (axis cs:{93.302},{10.627}) {\tikz\pgfuseplotmark{m6cv};};
\node[stars] at (axis cs:{93.439},{-23.862}) {\tikz\pgfuseplotmark{m6c};};
\node[stars] at (axis cs:{93.476},{-3.741}) {\tikz\pgfuseplotmark{m6a};};
\node[stars] at (axis cs:{93.619},{17.906}) {\tikz\pgfuseplotmark{m6b};};
\node[stars] at (axis cs:{93.653},{-4.568}) {\tikz\pgfuseplotmark{m6a};};
\node[stars] at (axis cs:{93.712},{19.156}) {\tikz\pgfuseplotmark{m5cb};};
\node[stars] at (axis cs:{93.714},{-6.275}) {\tikz\pgfuseplotmark{m4b};};
\node[stars] at (axis cs:{93.719},{22.507}) {\tikz\pgfuseplotmark{m3cvb};};
\node[stars] at (axis cs:{93.785},{-20.272}) {\tikz\pgfuseplotmark{m6b};};
\node[stars] at (axis cs:{93.785},{13.851}) {\tikz\pgfuseplotmark{m6bv};};
\node[stars] at (axis cs:{93.824},{-18.477}) {\tikz\pgfuseplotmark{m6b};};
\node[stars] at (axis cs:{93.845},{29.498}) {\tikz\pgfuseplotmark{m4c};};
\node[stars] at (axis cs:{93.855},{16.143}) {\tikz\pgfuseplotmark{m5c};};
\node[stars] at (axis cs:{93.859},{-9.036}) {\tikz\pgfuseplotmark{m6b};};
\node[stars] at (axis cs:{93.874},{-4.914}) {\tikz\pgfuseplotmark{m6bb};};
\node[stars] at (axis cs:{93.893},{-0.512}) {\tikz\pgfuseplotmark{m6a};};
\node[stars] at (axis cs:{93.917},{6.066}) {\tikz\pgfuseplotmark{m6b};};
\node[stars] at (axis cs:{93.937},{-13.718}) {\tikz\pgfuseplotmark{m5b};};
\node[stars] at (axis cs:{93.937},{12.551}) {\tikz\pgfuseplotmark{m5c};};
\node[stars] at (axis cs:{93.975},{1.169}) {\tikz\pgfuseplotmark{m6c};};
\node[stars] at (axis cs:{94.032},{-16.618}) {\tikz\pgfuseplotmark{m6bv};};
\node[stars] at (axis cs:{94.079},{23.970}) {\tikz\pgfuseplotmark{m6bv};};
\node[stars] at (axis cs:{94.099},{17.181}) {\tikz\pgfuseplotmark{m6c};};
\node[stars] at (axis cs:{94.111},{12.272}) {\tikz\pgfuseplotmark{m5bb};};
\node[stars] at (axis cs:{94.138},{-35.140}) {\tikz\pgfuseplotmark{m4cv};};
\node[stars] at (axis cs:{94.148},{-39.264}) {\tikz\pgfuseplotmark{m6b};};
\node[stars] at (axis cs:{94.245},{23.741}) {\tikz\pgfuseplotmark{m6cv};};
\node[stars] at (axis cs:{94.255},{-37.737}) {\tikz\pgfuseplotmark{m6a};};
\node[stars] at (axis cs:{94.265},{-22.715}) {\tikz\pgfuseplotmark{m6b};};
\node[stars] at (axis cs:{94.278},{9.942}) {\tikz\pgfuseplotmark{m5cb};};
\node[stars] at (axis cs:{94.290},{-37.253}) {\tikz\pgfuseplotmark{m6bv};};
\node[stars] at (axis cs:{94.317},{5.100}) {\tikz\pgfuseplotmark{m6a};};
\node[stars] at (axis cs:{94.424},{-16.816}) {\tikz\pgfuseplotmark{m5c};};
\node[stars] at (axis cs:{94.523},{14.383}) {\tikz\pgfuseplotmark{m6b};};
\node[stars] at (axis cs:{94.557},{-19.967}) {\tikz\pgfuseplotmark{m6a};};
\node[stars] at (axis cs:{94.668},{9.047}) {\tikz\pgfuseplotmark{m6c};};
\node[stars] at (axis cs:{94.703},{-15.025}) {\tikz\pgfuseplotmark{m6bv};};
\node[stars] at (axis cs:{94.711},{-9.390}) {\tikz\pgfuseplotmark{m5c};};
\node[stars] at (axis cs:{94.746},{-20.925}) {\tikz\pgfuseplotmark{m6av};};
\node[stars] at (axis cs:{94.758},{17.325}) {\tikz\pgfuseplotmark{m6cv};};
\node[stars] at (axis cs:{94.783},{-8.586}) {\tikz\pgfuseplotmark{m6c};};
\node[stars] at (axis cs:{94.909},{-22.103}) {\tikz\pgfuseplotmark{m6c};};
\node[stars] at (axis cs:{94.921},{-34.396}) {\tikz\pgfuseplotmark{m6a};};
\node[stars] at (axis cs:{94.928},{-7.823}) {\tikz\pgfuseplotmark{m5c};};
\node[stars] at (axis cs:{94.998},{-2.944}) {\tikz\pgfuseplotmark{m5b};};
\node[stars] at (axis cs:{95.018},{14.651}) {\tikz\pgfuseplotmark{m6av};};
\node[stars] at (axis cs:{95.078},{-30.063}) {\tikz\pgfuseplotmark{m3bv};};
\node[stars] at (axis cs:{95.146},{-23.638}) {\tikz\pgfuseplotmark{m6c};};
\node[stars] at (axis cs:{95.151},{-34.144}) {\tikz\pgfuseplotmark{m6av};};
\node[stars] at (axis cs:{95.273},{-11.817}) {\tikz\pgfuseplotmark{m6c};};
\node[stars] at (axis cs:{95.300},{29.541}) {\tikz\pgfuseplotmark{m6c};};
\node[stars] at (axis cs:{95.353},{-11.773}) {\tikz\pgfuseplotmark{m6av};};
\node[stars] at (axis cs:{95.357},{2.269}) {\tikz\pgfuseplotmark{m6cv};};
\node[stars] at (axis cs:{95.358},{17.763}) {\tikz\pgfuseplotmark{m6c};};
\node[stars] at (axis cs:{95.528},{-33.436}) {\tikz\pgfuseplotmark{m4a};};
\node[stars] at (axis cs:{95.652},{12.570}) {\tikz\pgfuseplotmark{m6b};};
\node[stars] at (axis cs:{95.675},{-17.956}) {\tikz\pgfuseplotmark{m2bv};};
\node[stars] at (axis cs:{95.740},{22.513}) {\tikz\pgfuseplotmark{m3bv};};
\node[stars] at (axis cs:{95.810},{-31.790}) {\tikz\pgfuseplotmark{m6c};};
\node[stars] at (axis cs:{95.827},{3.764}) {\tikz\pgfuseplotmark{m6c};};
\node[stars] at (axis cs:{95.900},{-9.875}) {\tikz\pgfuseplotmark{m6cv};};
\node[stars] at (axis cs:{95.942},{4.593}) {\tikz\pgfuseplotmark{m4cb};};
\node[stars] at (axis cs:{95.942},{-15.071}) {\tikz\pgfuseplotmark{m6c};};
\node[stars] at (axis cs:{95.983},{-25.577}) {\tikz\pgfuseplotmark{m6a};};
\node[stars] at (axis cs:{96.004},{-36.708}) {\tikz\pgfuseplotmark{m6ab};};
\node[stars] at (axis cs:{96.010},{8.885}) {\tikz\pgfuseplotmark{m6c};};
\node[stars] at (axis cs:{96.043},{-11.530}) {\tikz\pgfuseplotmark{m5c};};
\node[stars] at (axis cs:{96.086},{-12.962}) {\tikz\pgfuseplotmark{m6bv};};
\node[stars] at (axis cs:{96.182},{25.048}) {\tikz\pgfuseplotmark{m6c};};
\node[stars] at (axis cs:{96.183},{-28.780}) {\tikz\pgfuseplotmark{m6c};};
\node[stars] at (axis cs:{96.220},{16.057}) {\tikz\pgfuseplotmark{m6c};};
\node[stars] at (axis cs:{96.304},{7.086}) {\tikz\pgfuseplotmark{m6cv};};
\node[stars] at (axis cs:{96.319},{-0.946}) {\tikz\pgfuseplotmark{m6b};};
\node[stars] at (axis cs:{96.367},{14.722}) {\tikz\pgfuseplotmark{m6cv};};
\node[stars] at (axis cs:{96.375},{-35.064}) {\tikz\pgfuseplotmark{m6cb};};
\node[stars] at (axis cs:{96.387},{23.327}) {\tikz\pgfuseplotmark{m6b};};
\node[stars] at (axis cs:{96.446},{-3.889}) {\tikz\pgfuseplotmark{m6c};};
\node[stars] at (axis cs:{96.468},{11.126}) {\tikz\pgfuseplotmark{m6c};};
\node[stars] at (axis cs:{96.495},{-7.895}) {\tikz\pgfuseplotmark{m6c};};
\node[stars] at (axis cs:{96.540},{-3.515}) {\tikz\pgfuseplotmark{m6c};};
\node[stars] at (axis cs:{96.644},{-4.597}) {\tikz\pgfuseplotmark{m6c};};
\node[stars] at (axis cs:{96.665},{-1.507}) {\tikz\pgfuseplotmark{m6b};};
\node[stars] at (axis cs:{96.678},{-14.603}) {\tikz\pgfuseplotmark{m6c};};
\node[stars] at (axis cs:{96.687},{-7.512}) {\tikz\pgfuseplotmark{m6cb};};
\node[stars] at (axis cs:{96.782},{-37.895}) {\tikz\pgfuseplotmark{m6c};};
\node[stars] at (axis cs:{96.797},{-25.856}) {\tikz\pgfuseplotmark{m6b};};
\node[stars] at (axis cs:{96.807},{0.299}) {\tikz\pgfuseplotmark{m5c};};
\node[stars] at (axis cs:{96.815},{-0.276}) {\tikz\pgfuseplotmark{m6a};};
\node[stars] at (axis cs:{96.835},{2.908}) {\tikz\pgfuseplotmark{m6a};};
\node[stars] at (axis cs:{96.898},{32.563}) {\tikz\pgfuseplotmark{m6c};};
\node[stars] at (axis cs:{96.986},{20.496}) {\tikz\pgfuseplotmark{m6c};};
\node[stars] at (axis cs:{96.990},{-4.762}) {\tikz\pgfuseplotmark{m5b};};
\node[stars] at (axis cs:{97.043},{-32.580}) {\tikz\pgfuseplotmark{m4c};};
\node[stars] at (axis cs:{97.070},{1.912}) {\tikz\pgfuseplotmark{m6c};};
\node[stars] at (axis cs:{97.078},{10.304}) {\tikz\pgfuseplotmark{m6c};};
\node[stars] at (axis cs:{97.117},{16.238}) {\tikz\pgfuseplotmark{m6c};};
\node[stars] at (axis cs:{97.142},{30.493}) {\tikz\pgfuseplotmark{m6av};};
\node[stars] at (axis cs:{97.156},{-17.466}) {\tikz\pgfuseplotmark{m6a};};
\node[stars] at (axis cs:{97.164},{-32.371}) {\tikz\pgfuseplotmark{m6avb};};
\node[stars] at (axis cs:{97.204},{-7.033}) {\tikz\pgfuseplotmark{m5avb};};
\node[stars] at (axis cs:{97.241},{20.212}) {\tikz\pgfuseplotmark{m4cb};};
\node[stars] at (axis cs:{97.312},{2.646}) {\tikz\pgfuseplotmark{m6c};};
\node[stars] at (axis cs:{97.547},{-10.081}) {\tikz\pgfuseplotmark{m6b};};
\node[stars] at (axis cs:{97.549},{-19.215}) {\tikz\pgfuseplotmark{m6c};};
\node[stars] at (axis cs:{97.645},{-13.148}) {\tikz\pgfuseplotmark{m6cv};};
\node[stars] at (axis cs:{97.693},{-27.770}) {\tikz\pgfuseplotmark{m6b};};
\node[stars] at (axis cs:{97.790},{11.251}) {\tikz\pgfuseplotmark{m6cb};};
\node[stars] at (axis cs:{97.792},{16.938}) {\tikz\pgfuseplotmark{m6cb};};
\node[stars] at (axis cs:{97.805},{-35.259}) {\tikz\pgfuseplotmark{m6a};};
\node[stars] at (axis cs:{97.846},{-12.392}) {\tikz\pgfuseplotmark{m5c};};
\node[stars] at (axis cs:{97.850},{-32.869}) {\tikz\pgfuseplotmark{m6cv};};
\node[stars] at (axis cs:{97.896},{-36.940}) {\tikz\pgfuseplotmark{m6cv};};
\node[stars] at (axis cs:{97.906},{15.903}) {\tikz\pgfuseplotmark{m6c};};
\node[stars] at (axis cs:{97.951},{11.544}) {\tikz\pgfuseplotmark{m5c};};
\node[stars] at (axis cs:{97.959},{-8.158}) {\tikz\pgfuseplotmark{m5c};};
\node[stars] at (axis cs:{97.964},{-23.418}) {\tikz\pgfuseplotmark{m4cv};};
\node[stars] at (axis cs:{98.077},{17.784}) {\tikz\pgfuseplotmark{m6cb};};
\node[stars] at (axis cs:{98.080},{4.856}) {\tikz\pgfuseplotmark{m6av};};
\node[stars] at (axis cs:{98.089},{-37.696}) {\tikz\pgfuseplotmark{m5c};};
\node[stars] at (axis cs:{98.096},{-5.869}) {\tikz\pgfuseplotmark{m6a};};
\node[stars] at (axis cs:{98.097},{11.673}) {\tikz\pgfuseplotmark{m6bb};};
\node[stars] at (axis cs:{98.113},{32.455}) {\tikz\pgfuseplotmark{m6av};};
\node[stars] at (axis cs:{98.162},{-32.030}) {\tikz\pgfuseplotmark{m6ab};};
\node[stars] at (axis cs:{98.195},{-11.166}) {\tikz\pgfuseplotmark{m6cv};};
\node[stars] at (axis cs:{98.226},{7.333}) {\tikz\pgfuseplotmark{m5av};};
\node[stars] at (axis cs:{98.293},{-38.625}) {\tikz\pgfuseplotmark{m6c};};
\node[stars] at (axis cs:{98.361},{-20.924}) {\tikz\pgfuseplotmark{m6c};};
\node[stars] at (axis cs:{98.401},{14.155}) {\tikz\pgfuseplotmark{m6a};};
\node[stars] at (axis cs:{98.408},{-1.220}) {\tikz\pgfuseplotmark{m5b};};
\node[stars] at (axis cs:{98.456},{-36.232}) {\tikz\pgfuseplotmark{m5c};};
\node[stars] at (axis cs:{98.647},{-32.716}) {\tikz\pgfuseplotmark{m6a};};
\node[stars] at (axis cs:{98.693},{7.572}) {\tikz\pgfuseplotmark{m6cv};};
\node[stars] at (axis cs:{98.764},{-22.965}) {\tikz\pgfuseplotmark{m5a};};
\node[stars] at (axis cs:{98.800},{28.022}) {\tikz\pgfuseplotmark{m5cv};};
\node[stars] at (axis cs:{98.816},{0.890}) {\tikz\pgfuseplotmark{m6a};};
\node[stars] at (axis cs:{98.823},{9.988}) {\tikz\pgfuseplotmark{m6b};};
\node[stars] at (axis cs:{98.851},{-36.780}) {\tikz\pgfuseplotmark{m6ab};};
\node[stars] at (axis cs:{98.878},{-22.109}) {\tikz\pgfuseplotmark{m6c};};
\node[stars] at (axis cs:{98.975},{-36.089}) {\tikz\pgfuseplotmark{m6cb};};
\node[stars] at (axis cs:{99.095},{-18.660}) {\tikz\pgfuseplotmark{m6ab};};
\node[stars] at (axis cs:{99.137},{38.445}) {\tikz\pgfuseplotmark{m5cv};};
\node[stars] at (axis cs:{99.147},{-5.211}) {\tikz\pgfuseplotmark{m6a};};
\node[stars] at (axis cs:{99.171},{-19.256}) {\tikz\pgfuseplotmark{m4bv};};
\node[stars] at (axis cs:{99.171},{-22.614}) {\tikz\pgfuseplotmark{m6cvb};};
\node[stars] at (axis cs:{99.194},{-13.321}) {\tikz\pgfuseplotmark{m6bv};};
\node[stars] at (axis cs:{99.258},{-38.146}) {\tikz\pgfuseplotmark{m6b};};
\node[stars] at (axis cs:{99.308},{-36.991}) {\tikz\pgfuseplotmark{m6a};};
\node[stars] at (axis cs:{99.350},{6.135}) {\tikz\pgfuseplotmark{m6bv};};
\node[stars] at (axis cs:{99.364},{24.591}) {\tikz\pgfuseplotmark{m6c};};
\node[stars] at (axis cs:{99.404},{10.853}) {\tikz\pgfuseplotmark{m6c};};
\node[stars] at (axis cs:{99.418},{2.704}) {\tikz\pgfuseplotmark{m6c};};
\node[stars] at (axis cs:{99.420},{-12.985}) {\tikz\pgfuseplotmark{m6b};};
\node[stars] at (axis cs:{99.428},{16.399}) {\tikz\pgfuseplotmark{m2b};};
\node[stars] at (axis cs:{99.448},{-32.340}) {\tikz\pgfuseplotmark{m5c};};
\node[stars] at (axis cs:{99.470},{4.957}) {\tikz\pgfuseplotmark{m6cv};};
\node[stars] at (axis cs:{99.473},{-18.237}) {\tikz\pgfuseplotmark{m4c};};
\node[stars] at (axis cs:{99.585},{-2.544}) {\tikz\pgfuseplotmark{m6b};};
\node[stars] at (axis cs:{99.595},{-23.580}) {\tikz\pgfuseplotmark{m6c};};
\node[stars] at (axis cs:{99.596},{28.984}) {\tikz\pgfuseplotmark{m6a};};
\node[stars] at (axis cs:{99.648},{-16.873}) {\tikz\pgfuseplotmark{m6b};};
\node[stars] at (axis cs:{99.659},{1.613}) {\tikz\pgfuseplotmark{m6cv};};
\node[stars] at (axis cs:{99.665},{39.391}) {\tikz\pgfuseplotmark{m6a};};
\node[stars] at (axis cs:{99.705},{39.903}) {\tikz\pgfuseplotmark{m5c};};
\node[stars] at (axis cs:{99.772},{22.031}) {\tikz\pgfuseplotmark{m6b};};
\node[stars] at (axis cs:{99.820},{-14.146}) {\tikz\pgfuseplotmark{m5a};};
\node[stars] at (axis cs:{99.881},{24.600}) {\tikz\pgfuseplotmark{m6c};};
\node[stars] at (axis cs:{99.888},{28.263}) {\tikz\pgfuseplotmark{m6bv};};
\node[stars] at (axis cs:{99.901},{-23.695}) {\tikz\pgfuseplotmark{m6b};};
\node[stars] at (axis cs:{99.928},{-30.470}) {\tikz\pgfuseplotmark{m6a};};
\node[stars] at (axis cs:{99.949},{12.983}) {\tikz\pgfuseplotmark{m6b};};
\node[stars] at (axis cs:{100.244},{9.896}) {\tikz\pgfuseplotmark{m5avb};};
\node[stars] at (axis cs:{100.273},{0.495}) {\tikz\pgfuseplotmark{m6a};};
\node[stars] at (axis cs:{100.322},{11.003}) {\tikz\pgfuseplotmark{m6cv};};
\node[stars] at (axis cs:{100.337},{28.196}) {\tikz\pgfuseplotmark{m6c};};
\node[stars] at (axis cs:{100.341},{16.397}) {\tikz\pgfuseplotmark{m6c};};
\node[stars] at (axis cs:{100.407},{35.932}) {\tikz\pgfuseplotmark{m6c};};
\node[stars] at (axis cs:{100.485},{-9.168}) {\tikz\pgfuseplotmark{m5c};};
\node[stars] at (axis cs:{100.497},{6.345}) {\tikz\pgfuseplotmark{m6c};};
\node[stars] at (axis cs:{100.601},{17.645}) {\tikz\pgfuseplotmark{m5c};};
\node[stars] at (axis cs:{100.692},{-22.448}) {\tikz\pgfuseplotmark{m6cvb};};
\node[stars] at (axis cs:{100.778},{3.034}) {\tikz\pgfuseplotmark{m6c};};
\node[stars] at (axis cs:{100.807},{37.147}) {\tikz\pgfuseplotmark{m6c};};
\node[stars] at (axis cs:{100.847},{-39.193}) {\tikz\pgfuseplotmark{m6c};};
\node[stars] at (axis cs:{100.911},{3.932}) {\tikz\pgfuseplotmark{m6bv};};
\node[stars] at (axis cs:{100.983},{25.131}) {\tikz\pgfuseplotmark{m3bv};};
\node[stars] at (axis cs:{100.997},{13.228}) {\tikz\pgfuseplotmark{m4c};};
\node[stars] at (axis cs:{101.052},{36.109}) {\tikz\pgfuseplotmark{m6c};};
\node[stars] at (axis cs:{101.119},{-31.070}) {\tikz\pgfuseplotmark{m5cv};};
\node[stars] at (axis cs:{101.189},{28.971}) {\tikz\pgfuseplotmark{m5c};};
\node[stars] at (axis cs:{101.216},{-27.342}) {\tikz\pgfuseplotmark{m6c};};
\node[stars] at (axis cs:{101.287},{-16.716}) {\tikz\pgfuseplotmark{m1b};};
\node[stars] at (axis cs:{101.322},{12.895}) {\tikz\pgfuseplotmark{m3cv};};
\node[stars] at (axis cs:{101.346},{-31.793}) {\tikz\pgfuseplotmark{m6b};};
\node[stars] at (axis cs:{101.347},{-23.462}) {\tikz\pgfuseplotmark{m6b};};
\node[stars] at (axis cs:{101.380},{-30.949}) {\tikz\pgfuseplotmark{m6avb};};
\node[stars] at (axis cs:{101.476},{12.693}) {\tikz\pgfuseplotmark{m6c};};
\node[stars] at (axis cs:{101.497},{-14.796}) {\tikz\pgfuseplotmark{m5c};};
\node[stars] at (axis cs:{101.551},{-37.775}) {\tikz\pgfuseplotmark{m6cv};};
\node[stars] at (axis cs:{101.635},{8.587}) {\tikz\pgfuseplotmark{m6b};};
\node[stars] at (axis cs:{101.663},{-10.107}) {\tikz\pgfuseplotmark{m6a};};
\node[stars] at (axis cs:{101.713},{-14.426}) {\tikz\pgfuseplotmark{m5c};};
\node[stars] at (axis cs:{101.756},{-21.015}) {\tikz\pgfuseplotmark{m6bv};};
\node[stars] at (axis cs:{101.833},{8.037}) {\tikz\pgfuseplotmark{m5a};};
\node[stars] at (axis cs:{101.839},{-37.929}) {\tikz\pgfuseplotmark{m5c};};
\node[stars] at (axis cs:{101.848},{18.193}) {\tikz\pgfuseplotmark{m6c};};
\node[stars] at (axis cs:{101.905},{-8.998}) {\tikz\pgfuseplotmark{m5bv};};
\node[stars] at (axis cs:{101.965},{2.412}) {\tikz\pgfuseplotmark{m4c};};
\node[stars] at (axis cs:{102.079},{-1.319}) {\tikz\pgfuseplotmark{m6a};};
\node[stars] at (axis cs:{102.241},{-15.145}) {\tikz\pgfuseplotmark{m5cv};};
\node[stars] at (axis cs:{102.265},{1.002}) {\tikz\pgfuseplotmark{m6cv};};
\node[stars] at (axis cs:{102.318},{-2.272}) {\tikz\pgfuseplotmark{m6a};};
\node[stars] at (axis cs:{102.422},{32.607}) {\tikz\pgfuseplotmark{m6av};};
\node[stars] at (axis cs:{102.458},{16.203}) {\tikz\pgfuseplotmark{m6bv};};
\node[stars] at (axis cs:{102.460},{-32.508}) {\tikz\pgfuseplotmark{m4av};};
\node[stars] at (axis cs:{102.591},{-17.085}) {\tikz\pgfuseplotmark{m6a};};
\node[stars] at (axis cs:{102.597},{-31.706}) {\tikz\pgfuseplotmark{m6avb};};
\node[stars] at (axis cs:{102.606},{13.413}) {\tikz\pgfuseplotmark{m6a};};
\node[stars] at (axis cs:{102.654},{-25.778}) {\tikz\pgfuseplotmark{m6c};};
\node[stars] at (axis cs:{102.676},{-8.041}) {\tikz\pgfuseplotmark{m6cv};};
\node[stars] at (axis cs:{102.708},{-0.541}) {\tikz\pgfuseplotmark{m6a};};
\node[stars] at (axis cs:{102.718},{-34.367}) {\tikz\pgfuseplotmark{m5b};};
\node[stars] at (axis cs:{102.888},{21.761}) {\tikz\pgfuseplotmark{m5c};};
\node[stars] at (axis cs:{102.914},{3.042}) {\tikz\pgfuseplotmark{m6c};};
\node[stars] at (axis cs:{102.927},{-36.230}) {\tikz\pgfuseplotmark{m6b};};
\node[stars] at (axis cs:{103.000},{23.602}) {\tikz\pgfuseplotmark{m6a};};
\node[stars] at (axis cs:{103.095},{-5.316}) {\tikz\pgfuseplotmark{m6c};};
\node[stars] at (axis cs:{103.197},{33.961}) {\tikz\pgfuseplotmark{m4a};};
\node[stars] at (axis cs:{103.206},{8.380}) {\tikz\pgfuseplotmark{m6a};};
\node[stars] at (axis cs:{103.251},{-26.958}) {\tikz\pgfuseplotmark{m6cv};};
\node[stars] at (axis cs:{103.256},{38.869}) {\tikz\pgfuseplotmark{m6bvb};};
\node[stars] at (axis cs:{103.265},{35.788}) {\tikz\pgfuseplotmark{m6b};};
\node[stars] at (axis cs:{103.306},{38.438}) {\tikz\pgfuseplotmark{m6c};};
\node[stars] at (axis cs:{103.328},{-19.032}) {\tikz\pgfuseplotmark{m6a};};
\node[stars] at (axis cs:{103.341},{-18.933}) {\tikz\pgfuseplotmark{m6b};};
\node[stars] at (axis cs:{103.343},{10.996}) {\tikz\pgfuseplotmark{m6c};};
\node[stars] at (axis cs:{103.387},{-20.224}) {\tikz\pgfuseplotmark{m5av};};
\node[stars] at (axis cs:{103.391},{-28.540}) {\tikz\pgfuseplotmark{m6b};};
\node[stars] at (axis cs:{103.481},{-24.539}) {\tikz\pgfuseplotmark{m6c};};
\node[stars] at (axis cs:{103.488},{38.505}) {\tikz\pgfuseplotmark{m6c};};
\node[stars] at (axis cs:{103.533},{-24.184}) {\tikz\pgfuseplotmark{m4av};};
\node[stars] at (axis cs:{103.536},{-5.852}) {\tikz\pgfuseplotmark{m6cb};};
\node[stars] at (axis cs:{103.547},{-12.039}) {\tikz\pgfuseplotmark{m4b};};
\node[stars] at (axis cs:{103.603},{-1.127}) {\tikz\pgfuseplotmark{m5c};};
\node[stars] at (axis cs:{103.661},{13.178}) {\tikz\pgfuseplotmark{m5ab};};
\node[stars] at (axis cs:{103.675},{-1.756}) {\tikz\pgfuseplotmark{m6cv};};
\node[stars] at (axis cs:{103.745},{-2.804}) {\tikz\pgfuseplotmark{m6b};};
\node[stars] at (axis cs:{103.761},{-20.405}) {\tikz\pgfuseplotmark{m6ab};};
\node[stars] at (axis cs:{103.828},{25.376}) {\tikz\pgfuseplotmark{m6av};};
\node[stars] at (axis cs:{103.894},{8.324}) {\tikz\pgfuseplotmark{m6c};};
\node[stars] at (axis cs:{103.906},{-20.136}) {\tikz\pgfuseplotmark{m5av};};
\node[stars] at (axis cs:{103.946},{-22.941}) {\tikz\pgfuseplotmark{m5cv};};
\node[stars] at (axis cs:{103.978},{-31.790}) {\tikz\pgfuseplotmark{m6cv};};
\node[stars] at (axis cs:{104.028},{-14.043}) {\tikz\pgfuseplotmark{m5bb};};
\node[stars] at (axis cs:{104.034},{-17.054}) {\tikz\pgfuseplotmark{m4cv};};
\node[stars] at (axis cs:{104.108},{9.956}) {\tikz\pgfuseplotmark{m6b};};
\node[stars] at (axis cs:{104.190},{-35.341}) {\tikz\pgfuseplotmark{m6c};};
\node[stars] at (axis cs:{104.250},{-8.179}) {\tikz\pgfuseplotmark{m6c};};
\node[stars] at (axis cs:{104.252},{33.681}) {\tikz\pgfuseplotmark{m6b};};
\node[stars] at (axis cs:{104.323},{-35.507}) {\tikz\pgfuseplotmark{m6cv};};
\node[stars] at (axis cs:{104.357},{11.907}) {\tikz\pgfuseplotmark{m6c};};
\node[stars] at (axis cs:{104.391},{-24.631}) {\tikz\pgfuseplotmark{m5c};};
\node[stars] at (axis cs:{104.428},{-27.537}) {\tikz\pgfuseplotmark{m6cv};};
\node[stars] at (axis cs:{104.531},{-27.164}) {\tikz\pgfuseplotmark{m6cv};};
\node[stars] at (axis cs:{104.605},{-34.112}) {\tikz\pgfuseplotmark{m5b};};
\node[stars] at (axis cs:{104.650},{-25.414}) {\tikz\pgfuseplotmark{m6a};};
\node[stars] at (axis cs:{104.656},{-28.972}) {\tikz\pgfuseplotmark{m2a};};
\node[stars] at (axis cs:{104.662},{7.622}) {\tikz\pgfuseplotmark{m6c};};
\node[stars] at (axis cs:{104.682},{-30.998}) {\tikz\pgfuseplotmark{m6cb};};
\node[stars] at (axis cs:{104.698},{26.081}) {\tikz\pgfuseplotmark{m6cv};};
\node[stars] at (axis cs:{104.738},{3.602}) {\tikz\pgfuseplotmark{m6b};};
\node[stars] at (axis cs:{104.762},{38.052}) {\tikz\pgfuseplotmark{m6b};};
\node[stars] at (axis cs:{104.834},{7.317}) {\tikz\pgfuseplotmark{m6c};};
\node[stars] at (axis cs:{104.866},{25.914}) {\tikz\pgfuseplotmark{m6c};};
\node[stars] at (axis cs:{104.914},{-21.603}) {\tikz\pgfuseplotmark{m6c};};
\node[stars] at (axis cs:{105.035},{-20.158}) {\tikz\pgfuseplotmark{m6c};};
\node[stars] at (axis cs:{105.066},{16.079}) {\tikz\pgfuseplotmark{m6av};};
\node[stars] at (axis cs:{105.075},{-5.367}) {\tikz\pgfuseplotmark{m6c};};
\node[stars] at (axis cs:{105.099},{-8.407}) {\tikz\pgfuseplotmark{m6b};};
\node[stars] at (axis cs:{105.177},{-28.489}) {\tikz\pgfuseplotmark{m6c};};
\node[stars] at (axis cs:{105.207},{-33.465}) {\tikz\pgfuseplotmark{m6c};};
\node[stars] at (axis cs:{105.275},{-25.215}) {\tikz\pgfuseplotmark{m6av};};
\node[stars] at (axis cs:{105.430},{-27.935}) {\tikz\pgfuseplotmark{m3cv};};
\node[stars] at (axis cs:{105.470},{-1.346}) {\tikz\pgfuseplotmark{m6cb};};
\node[stars] at (axis cs:{105.485},{-5.722}) {\tikz\pgfuseplotmark{m5cv};};
\node[stars] at (axis cs:{105.573},{15.336}) {\tikz\pgfuseplotmark{m6a};};
\node[stars] at (axis cs:{105.603},{24.215}) {\tikz\pgfuseplotmark{m5cv};};
\node[stars] at (axis cs:{105.606},{17.755}) {\tikz\pgfuseplotmark{m6bv};};
\node[stars] at (axis cs:{105.639},{16.674}) {\tikz\pgfuseplotmark{m6av};};
\node[stars] at (axis cs:{105.728},{-4.239}) {\tikz\pgfuseplotmark{m5bv};};
\node[stars] at (axis cs:{105.756},{-23.833}) {\tikz\pgfuseplotmark{m3bv};};
\node[stars] at (axis cs:{105.825},{9.138}) {\tikz\pgfuseplotmark{m6b};};
\node[stars] at (axis cs:{105.877},{29.337}) {\tikz\pgfuseplotmark{m6b};};
\node[stars] at (axis cs:{105.909},{10.952}) {\tikz\pgfuseplotmark{m5b};};
\node[stars] at (axis cs:{105.940},{-15.633}) {\tikz\pgfuseplotmark{m4b};};
\node[stars] at (axis cs:{105.965},{12.594}) {\tikz\pgfuseplotmark{m6b};};
\node[stars] at (axis cs:{105.989},{-10.124}) {\tikz\pgfuseplotmark{m6c};};
\node[stars] at (axis cs:{106.022},{-5.324}) {\tikz\pgfuseplotmark{m6a};};
\node[stars] at (axis cs:{106.027},{20.570}) {\tikz\pgfuseplotmark{m4bvb};};
\node[stars] at (axis cs:{106.196},{-22.031}) {\tikz\pgfuseplotmark{m6b};};
\node[stars] at (axis cs:{106.327},{22.637}) {\tikz\pgfuseplotmark{m6b};};
\node[stars] at (axis cs:{106.383},{-34.778}) {\tikz\pgfuseplotmark{m6cb};};
\node[stars] at (axis cs:{106.413},{9.186}) {\tikz\pgfuseplotmark{m6a};};
\node[stars] at (axis cs:{106.502},{-30.656}) {\tikz\pgfuseplotmark{m6cv};};
\node[stars] at (axis cs:{106.509},{-38.383}) {\tikz\pgfuseplotmark{m6b};};
\node[stars] at (axis cs:{106.548},{34.474}) {\tikz\pgfuseplotmark{m6a};};
\node[stars] at (axis cs:{106.650},{-12.394}) {\tikz\pgfuseplotmark{m6c};};
\node[stars] at (axis cs:{106.670},{-11.294}) {\tikz\pgfuseplotmark{m5cvb};};
\node[stars] at (axis cs:{106.719},{-24.960}) {\tikz\pgfuseplotmark{m6b};};
\node[stars] at (axis cs:{106.777},{4.910}) {\tikz\pgfuseplotmark{m6b};};
\node[stars] at (axis cs:{106.843},{34.009}) {\tikz\pgfuseplotmark{m6b};};
\node[stars] at (axis cs:{106.844},{-23.841}) {\tikz\pgfuseplotmark{m6av};};
\node[stars] at (axis cs:{106.854},{28.177}) {\tikz\pgfuseplotmark{m6c};};
\node[stars] at (axis cs:{106.956},{7.471}) {\tikz\pgfuseplotmark{m6a};};
\node[stars] at (axis cs:{107.055},{33.832}) {\tikz\pgfuseplotmark{m6c};};
\node[stars] at (axis cs:{107.092},{15.930}) {\tikz\pgfuseplotmark{m5c};};
\node[stars] at (axis cs:{107.098},{-26.393}) {\tikz\pgfuseplotmark{m2av};};
\node[stars] at (axis cs:{107.151},{37.445}) {\tikz\pgfuseplotmark{m6c};};
\node[stars] at (axis cs:{107.213},{-39.656}) {\tikz\pgfuseplotmark{m5av};};
\node[stars] at (axis cs:{107.334},{-10.346}) {\tikz\pgfuseplotmark{m6c};};
\node[stars] at (axis cs:{107.389},{-16.234}) {\tikz\pgfuseplotmark{m6bv};};
\node[stars] at (axis cs:{107.429},{-25.231}) {\tikz\pgfuseplotmark{m6av};};
\node[stars] at (axis cs:{107.528},{21.247}) {\tikz\pgfuseplotmark{m6c};};
\node[stars] at (axis cs:{107.539},{-18.686}) {\tikz\pgfuseplotmark{m6c};};
\node[stars] at (axis cs:{107.557},{-4.237}) {\tikz\pgfuseplotmark{m5bb};};
\node[stars] at (axis cs:{107.581},{-27.491}) {\tikz\pgfuseplotmark{m5c};};
\node[stars] at (axis cs:{107.763},{-3.898}) {\tikz\pgfuseplotmark{m6c};};
\node[stars] at (axis cs:{107.785},{30.245}) {\tikz\pgfuseplotmark{m4cv};};
\node[stars] at (axis cs:{107.846},{26.856}) {\tikz\pgfuseplotmark{m6av};};
\node[stars] at (axis cs:{107.848},{-0.302}) {\tikz\pgfuseplotmark{m5cv};};
\node[stars] at (axis cs:{107.914},{39.320}) {\tikz\pgfuseplotmark{m5b};};
\node[stars] at (axis cs:{107.923},{-20.883}) {\tikz\pgfuseplotmark{m6a};};
\node[stars] at (axis cs:{107.964},{5.655}) {\tikz\pgfuseplotmark{m6b};};
\node[stars] at (axis cs:{107.966},{-0.493}) {\tikz\pgfuseplotmark{m4c};};
\node[stars] at (axis cs:{108.017},{-30.821}) {\tikz\pgfuseplotmark{m6b};};
\node[stars] at (axis cs:{108.031},{5.474}) {\tikz\pgfuseplotmark{m6c};};
\node[stars] at (axis cs:{108.051},{-25.942}) {\tikz\pgfuseplotmark{m6bv};};
\node[stars] at (axis cs:{108.108},{-36.544}) {\tikz\pgfuseplotmark{m6bvb};};
\node[stars] at (axis cs:{108.110},{24.128}) {\tikz\pgfuseplotmark{m6a};};
\node[stars] at (axis cs:{108.280},{-11.251}) {\tikz\pgfuseplotmark{m6a};};
\node[stars] at (axis cs:{108.343},{16.159}) {\tikz\pgfuseplotmark{m5bv};};
\node[stars] at (axis cs:{108.350},{-22.674}) {\tikz\pgfuseplotmark{m6bv};};
\node[stars] at (axis cs:{108.402},{-27.356}) {\tikz\pgfuseplotmark{m6bv};};
\node[stars] at (axis cs:{108.452},{-22.906}) {\tikz\pgfuseplotmark{m6cb};};
\node[stars] at (axis cs:{108.489},{-30.340}) {\tikz\pgfuseplotmark{m6c};};
\node[stars] at (axis cs:{108.545},{-3.902}) {\tikz\pgfuseplotmark{m6b};};
\node[stars] at (axis cs:{108.563},{-26.352}) {\tikz\pgfuseplotmark{m4cv};};
\node[stars] at (axis cs:{108.565},{-9.947}) {\tikz\pgfuseplotmark{m6b};};
\node[stars] at (axis cs:{108.584},{3.111}) {\tikz\pgfuseplotmark{m5c};};
\node[stars] at (axis cs:{108.618},{-10.316}) {\tikz\pgfuseplotmark{m6b};};
\node[stars] at (axis cs:{108.636},{12.116}) {\tikz\pgfuseplotmark{m6a};};
\node[stars] at (axis cs:{108.675},{24.885}) {\tikz\pgfuseplotmark{m6av};};
\node[stars] at (axis cs:{108.703},{-26.773}) {\tikz\pgfuseplotmark{m4bv};};
\node[stars] at (axis cs:{108.713},{-27.038}) {\tikz\pgfuseplotmark{m6a};};
\node[stars] at (axis cs:{108.831},{-0.161}) {\tikz\pgfuseplotmark{m6c};};
\node[stars] at (axis cs:{108.838},{-30.686}) {\tikz\pgfuseplotmark{m5cv};};
\node[stars] at (axis cs:{108.914},{7.978}) {\tikz\pgfuseplotmark{m6av};};
\node[stars] at (axis cs:{108.930},{-10.584}) {\tikz\pgfuseplotmark{m6b};};
\node[stars] at (axis cs:{108.947},{-23.740}) {\tikz\pgfuseplotmark{m6c};};
\node[stars] at (axis cs:{108.988},{27.897}) {\tikz\pgfuseplotmark{m6av};};
\node[stars] at (axis cs:{109.061},{-15.586}) {\tikz\pgfuseplotmark{m5cv};};
\node[stars] at (axis cs:{109.133},{-38.319}) {\tikz\pgfuseplotmark{m6a};};
\node[stars] at (axis cs:{109.146},{-27.881}) {\tikz\pgfuseplotmark{m5av};};
\node[stars] at (axis cs:{109.153},{-23.315}) {\tikz\pgfuseplotmark{m5avb};};
\node[stars] at (axis cs:{109.206},{-36.592}) {\tikz\pgfuseplotmark{m5bv};};
\node[stars] at (axis cs:{109.238},{-30.897}) {\tikz\pgfuseplotmark{m6cb};};
\node[stars] at (axis cs:{109.264},{26.689}) {\tikz\pgfuseplotmark{m6c};};
\node[stars] at (axis cs:{109.286},{-37.097}) {\tikz\pgfuseplotmark{m3av};};
\node[stars] at (axis cs:{109.382},{-6.680}) {\tikz\pgfuseplotmark{m6c};};
\node[stars] at (axis cs:{109.428},{38.876}) {\tikz\pgfuseplotmark{m6c};};
\node[stars] at (axis cs:{109.450},{-26.797}) {\tikz\pgfuseplotmark{m6c};};
\node[stars] at (axis cs:{109.517},{30.956}) {\tikz\pgfuseplotmark{m6c};};
\node[stars] at (axis cs:{109.523},{16.540}) {\tikz\pgfuseplotmark{m4av};};
\node[stars] at (axis cs:{109.577},{-36.734}) {\tikz\pgfuseplotmark{m5avb};};
\node[stars] at (axis cs:{109.640},{-39.210}) {\tikz\pgfuseplotmark{m5c};};
\node[stars] at (axis cs:{109.668},{-24.559}) {\tikz\pgfuseplotmark{m5bv};};
\node[stars] at (axis cs:{109.677},{-24.954}) {\tikz\pgfuseplotmark{m4cvb};};
\node[stars] at (axis cs:{109.714},{-26.586}) {\tikz\pgfuseplotmark{m5cv};};
\node[stars] at (axis cs:{109.758},{-19.280}) {\tikz\pgfuseplotmark{m6b};};
\node[stars] at (axis cs:{109.807},{-33.727}) {\tikz\pgfuseplotmark{m6cv};};
\node[stars] at (axis cs:{109.843},{2.741}) {\tikz\pgfuseplotmark{m6b};};
\node[stars] at (axis cs:{109.867},{-16.395}) {\tikz\pgfuseplotmark{m6av};};
\node[stars] at (axis cs:{109.949},{7.143}) {\tikz\pgfuseplotmark{m6b};};
\end{axis}




\begin{axis}[name=base,axis lines=none]
 \boldmath

\node[designation-label,anchor=south east]  at (axis cs:{04*15+35/4},{+16+30/ 60  })  {Aldebaran};    

\node[designation-label,anchor=south east]  at (axis cs:{05*15+36/4},{-01-12/ 60  })  {Alnilam};   
\node[designation-label,anchor=south east]  at (axis cs:{05*15+40/4},{-01-56/ 60  })  {Alnitak};   
\node[designation-label,anchor=south east]  at (axis cs:{05*15+55/4},{+07+24/ 60  })  {Betelgeuse};  

\node[designation-label,anchor=south west]  at (axis cs:{05*15+14/4},{- 8-12/ 60  })  {Rigel};   
\node[designation-label,anchor=south west]  at (axis cs:{05*15+25/4},{+06+20/ 60  })  {Bellatrix};   
\node[designation-label,anchor=south west]  at (axis cs:{05*15+26/4},{+28+36/ 60  })  {Alnath};   
\node[designation-label,anchor=south west]  at (axis cs:{06*15+22/4},{-17-57/ 60  })  {Mirzam};   
\node[designation-label,anchor=south west]  at (axis cs:{06*15+37/4},{+16+23/ 60  })  {Alhena};   
\node[designation-label,anchor=south west]  at (axis cs:{06*15+45/4},{-16-42/ 60  })  {Sirius};   
\node[designation-label,anchor=south west]  at (axis cs:{06*15+58/4},{-28-58/ 60  })  {Adara};   
\node[designation-label,anchor=south west]  at (axis cs:{07*15+ 8/4},{-26-23/ 60  })  {Wezen}; 

\node[interest,pin={[pin distance=-0.4\onedegree,interest-label]90:{V838}}] at (axis cs:7*15+4.08/4,-5+50.7/60) {\pgfuseplotmark{+}} ; % V838 Mon
\node[interest,pin={[pin distance=-0.4\onedegree,interest-label]90:{v.\,Maanen's star}}] at (axis cs:12.29,5.39) {\pgfuseplotmark{+}} ;
\node[interest,pin={[pin distance=-0.4\onedegree,interest-label]0:{T}}] at (axis cs:65.498,19.535) {\pgfuseplotmark{+}} ; % T Tau

\node[stars,pin={[pin distance=-0.6\onedegree,Flaamsted]0:{AE}}] at (axis cs:79.06,34.31) {} ; % AE Aur

\node[stars,pin={[pin distance=-0.6\onedegree,Flaamsted]0:{UU}}] at (axis cs:099.1368212654100,+38.4455052837800) {} ; % UU Aur
\node[stars,pin={[pin distance=-0.6\onedegree,Flaamsted]180:{WW}}] at (axis cs:098.1132698606000,+32.4548978969700) {} ; % WW Aur

\node[stars,pin={[pin distance=-0.6\onedegree,Flaamsted]0:{R}}] at (axis cs:6.008,38.577)  {\tikz\pgfuseplotmark{m6av};} ; % R And
\node[stars,pin={[pin distance=-0.6\onedegree,Flaamsted]180:{R}}] at (axis cs:39.26,34.264) {\tikz\pgfuseplotmark{m6cv};} ; % R Tri
\node[stars,pin={[pin distance=-0.6\onedegree,Flaamsted]-90:{X}}] at (axis cs:058.8461572733700,+31.0458443989300)  {\tikz\pgfuseplotmark{m6cv};} ; % X Per
\node[stars,pin={[pin distance=-0.6\onedegree,Flaamsted]270:{T}}] at (axis cs:005.4428049075,-20.0580235628) {} ; % T Cet

\node[stars,pin={[pin distance=-0.6\onedegree,Flaamsted]90:{R}}] at (axis cs:73.8277409664600,-16.4177706596900) {} ; % R Eri
\node[stars,pin={[pin distance=-0.6\onedegree,Flaamsted]0:{S}}] at (axis cs:091.4397903266700,-24.1955597292800)  {\tikz\pgfuseplotmark{m6cv};} ; % S Lep
\node[stars,pin={[pin distance=-0.6\onedegree,Flaamsted]0:{R}}] at (axis cs:74.9014528838000,-14.8062506886500)  {\tikz\pgfuseplotmark{m6cv};} ; % R Lep
\node[stars,pin={[pin distance=-0.6\onedegree,Flaamsted]0:{T}}] at (axis cs:096.3041653170500,+07.0857107174700) {} ; % T Mon



\node[pin={[pin distance=-0.2\onedegree,Bayer]90:{$\boldsymbol\alpha$}}] at (axis cs:{2.097},{29.090}) {}; % And,  2.06 
\node[pin={[pin distance=-0.2\onedegree,Bayer]00:{$\boldsymbol\beta$}}] at (axis cs:{17.433},{35.620}) {}; % And,  2.08 
\node[pin={[pin distance=-0.4\onedegree,Bayer]00:{$\boldsymbol\delta$}}] at (axis cs:{9.832},{30.861}) {}; % And,  3.27 
\node[pin={[pin distance=-0.4\onedegree,Bayer]00:{$\boldsymbol\epsilon$}}] at (axis cs:{9.639},{29.312}) {}; % And,  4.35 
\node[pin={[pin distance=-0.4\onedegree,Bayer]305:{$\boldsymbol\eta$}}] at (axis cs:{14.302},{23.418}) {}; % And,  4.40 
\node[pin={[pin distance=-0.4\onedegree,Bayer]00:{$\boldsymbol\mu$}}] at (axis cs:{14.188},{38.499}) {}; % And,  3.87 
\node[pin={[pin distance=-0.4\onedegree,Bayer]00:{$\boldsymbol\pi$}}] at (axis cs:{9.220},{33.719}) {}; % And,  4.35 
\node[pin={[pin distance=-0.4\onedegree,Bayer]00:{$\boldsymbol\rho$}}] at (axis cs:{5.280},{37.969}) {}; % And,  5.16 
\node[pin={[pin distance=-0.4\onedegree,Bayer]270:{$\boldsymbol\sigma$}}] at (axis cs:{4.582},{36.785}) {}; % And,  4.51 
\node[pin={[pin distance=-0.4\onedegree,Bayer]00:{$\boldsymbol\vartheta$}}] at (axis cs:{4.273},{38.681}) {}; % And,  4.62 
\node[pin={[pin distance=-0.4\onedegree,Bayer]00:{$\boldsymbol\zeta$}}] at (axis cs:{11.835},{24.267}) {}; % And,  4.09 
\node[pin={[pin distance=-0.4\onedegree,Flaamsted]00:{28}}] at (axis cs:{7.531},{29.752}) {}; % And,  5.22 
\node[pin={[pin distance=-0.4\onedegree,Flaamsted]00:{32}}] at (axis cs:{10.280},{39.459}) {}; % And,  5.31 
\node[pin={[pin distance=-0.4\onedegree,Flaamsted]00:{36}}] at (axis cs:{13.742},{23.628}) {}; % And,  5.50 
\node[pin={[pin distance=-0.4\onedegree,Flaamsted]00:{45}}] at (axis cs:{17.793},{37.724}) {}; % And,  5.79 
\node[pin={[pin distance=-0.4\onedegree,Flaamsted]00:{47}}] at (axis cs:{20.919},{37.715}) {}; % And,  5.59 
\node[pin={[pin distance=-0.4\onedegree,Flaamsted]00:{56}}] at (axis cs:{29.039},{37.252}) {}; % And,  5.69 
\node[pin={[pin distance=-0.4\onedegree,Flaamsted]00:{58}}] at (axis cs:{32.122},{37.859}) {}; % And,  4.79 
\node[pin={[pin distance=-0.2\onedegree,Bayer]00:{$\boldsymbol\alpha$}}] at (axis cs:{31.793},{23.462}) {}; % Ari,  2.02 
\node[pin={[pin distance=-0.4\onedegree,Bayer]00:{$\boldsymbol\beta$}}] at (axis cs:{28.660},{20.808}) {}; % Ari,  2.66 
\node[pin={[pin distance=-0.4\onedegree,Bayer]00:{$\boldsymbol\delta$}}] at (axis cs:{47.907},{19.727}) {}; % Ari,  4.35 
\node[pin={[pin distance=-0.4\onedegree,Bayer]00:{$\boldsymbol\epsilon$}}] at (axis cs:{44.803},{21.340}) {}; % Ari,  4.63 
\node[pin={[pin distance=-0.4\onedegree,Bayer]00:{$\boldsymbol\epsilon$}}] at (axis cs:{44.803},{21.340}) {}; % Ari,  4.63 
\node[pin={[pin distance=-0.4\onedegree,Bayer]00:{$\boldsymbol\eta$}}] at (axis cs:{33.200},{21.211}) {}; % Ari,  5.23 
\node[pin={[pin distance=-0.4\onedegree,Bayer]00:{$\boldsymbol\gamma$}}] at (axis cs:{28.383},{19.296}) {}; % Ari,  4.64 
\node[pin={[pin distance=-0.4\onedegree,Bayer]00:{$\boldsymbol\iota$}}] at (axis cs:{29.338},{17.818}) {}; % Ari,  5.10 
\node[pin={[pin distance=-0.4\onedegree,Bayer]00:{$\boldsymbol\kappa$}}] at (axis cs:{31.641},{22.648}) {}; % Ari,  5.03 
\node[pin={[pin distance=-0.4\onedegree,Bayer]90:{$\boldsymbol\lambda$}}] at (axis cs:{29.482},{23.596}) {}; % Ari,  4.77 
\node[pin={[pin distance=-0.4\onedegree,Bayer]00:{$\boldsymbol\mu$}}] at (axis cs:{40.591},{20.011}) {}; % Ari,  5.74 
\node[pin={[pin distance=-0.4\onedegree,Bayer]00:{$\boldsymbol\nu$}}] at (axis cs:{39.704},{21.961}) {}; % Ari,  5.45 
\node[pin={[pin distance=-0.4\onedegree,Bayer]00:{$\boldsymbol\omicron$}}] at (axis cs:{41.137},{15.312}) {}; % Ari,  5.78 
\node[pin={[pin distance=-0.4\onedegree,Bayer]00:{$\boldsymbol\pi$}}] at (axis cs:{42.323},{17.464}) {}; % Ari,  5.32 
\node[pin={[pin distance=-0.4\onedegree,Bayer]00:{$\boldsymbol\rho^2$}}] at (axis cs:{43.952},{18.331}) {}; % Ari,  5.82 
\node[pin={[pin distance=-0.4\onedegree,Bayer]180:{$\boldsymbol\rho^3$}}] at (axis cs:{44.109},{18.023}) {}; % Ari,  5.59 
\node[pin={[pin distance=-0.4\onedegree,Bayer]00:{$\boldsymbol\sigma$}}] at (axis cs:{42.873},{15.082}) {}; % Ari,  5.52 
\node[pin={[pin distance=-0.4\onedegree,Bayer]90:{$\boldsymbol\tau^1$}}] at (axis cs:{50.307},{21.147}) {}; % Ari,  5.31 
\node[pin={[pin distance=-0.4\onedegree,Bayer]00:{$\boldsymbol\tau^2$}}] at (axis cs:{50.689},{20.742}) {}; % Ari,  5.08 
\node[pin={[pin distance=-0.4\onedegree,Bayer]00:{$\boldsymbol\vartheta$}}] at (axis cs:{34.531},{19.901}) {}; % Ari,  5.58 
\node[pin={[pin distance=-0.4\onedegree,Bayer]90:{$\boldsymbol\xi$}}] at (axis cs:{36.204},{10.610}) {}; % Ari,  5.48 
\node[pin={[pin distance=-0.4\onedegree,Bayer]00:{$\boldsymbol\zeta$}}] at (axis cs:{48.725},{21.044}) {}; % Ari,  4.88 
\node[pin={[pin distance=-0.4\onedegree,Flaamsted]90:{10}}] at (axis cs:{30.914},{25.935}) {}; % Ari,  5.69 
\node[pin={[pin distance=-0.4\onedegree,Flaamsted]90:{14}}] at (axis cs:{32.356},{25.940}) {}; % Ari,  4.98 
\node[pin={[pin distance=-0.4\onedegree,Flaamsted]00:{15}}] at (axis cs:{32.657},{19.500}) {}; % Ari,  5.71 
\node[pin={[pin distance=-0.4\onedegree,Flaamsted]00:{19}}] at (axis cs:{33.264},{15.280}) {}; % Ari,  5.71 
\node[pin={[pin distance=-0.4\onedegree,Flaamsted]00:{ 1}}] at (axis cs:{27.536},{22.275}) {}; % Ari,  5.83 
\node[pin={[pin distance=-0.4\onedegree,Flaamsted]00:{20}}] at (axis cs:{33.942},{25.783}) {}; % Ari,  5.79 
\node[pin={[pin distance=-0.4\onedegree,Flaamsted]00:{21}}] at (axis cs:{33.928},{25.043}) {}; % Ari,  5.58 
\node[pin={[pin distance=-0.4\onedegree,Flaamsted]00:{31}}] at (axis cs:{39.158},{12.447}) {}; % Ari,  5.65 
\node[pin={[pin distance=-0.4\onedegree,Flaamsted]00:{33}}] at (axis cs:{40.171},{27.061}) {}; % Ari,  5.30 
\node[pin={[pin distance=-0.4\onedegree,Flaamsted]00:{35}}] at (axis cs:{40.863},{27.707}) {}; % Ari,  4.65 
\node[pin={[pin distance=-0.4\onedegree,Flaamsted]00:{38}}] at (axis cs:{41.240},{12.446}) {}; % Ari,  5.19 
\node[pin={[pin distance=-0.4\onedegree,Flaamsted]00:{39}}] at (axis cs:{41.977},{29.247}) {}; % Ari,  4.52 
\node[pin={[pin distance=-0.4\onedegree,Flaamsted]00:{40}}] at (axis cs:{42.134},{18.284}) {}; % Ari,  5.83 
\node[pin={[pin distance=-0.4\onedegree,Flaamsted]00:{41}}] at (axis cs:{42.496},{27.260}) {}; % Ari,  3.62 
\node[pin={[pin distance=-0.4\onedegree,Flaamsted]00:{47}}] at (axis cs:{44.522},{20.669}) {}; % Ari,  5.80 
\node[pin={[pin distance=-0.4\onedegree,Flaamsted]00:{49}}] at (axis cs:{45.476},{26.462}) {}; % Ari,  5.92 
\node[pin={[pin distance=-0.4\onedegree,Flaamsted]180:{ 4}}] at (axis cs:{27.046},{16.955}) {}; % Ari,  5.86 
\node[pin={[pin distance=-0.4\onedegree,Flaamsted]00:{52}}] at (axis cs:{46.361},{25.255}) {}; % Ari,  5.46 
\node[pin={[pin distance=-0.4\onedegree,Flaamsted]00:{55}}] at (axis cs:{47.403},{29.077}) {}; % Ari,  5.75 
\node[pin={[pin distance=-0.4\onedegree,Flaamsted]00:{56}}] at (axis cs:{48.059},{27.257}) {}; % Ari,  5.78 
\node[pin={[pin distance=-0.4\onedegree,Flaamsted]00:{59}}] at (axis cs:{49.982},{27.071}) {}; % Ari,  5.92 
\node[pin={[pin distance=-0.4\onedegree,Flaamsted]00:{62}}] at (axis cs:{50.550},{27.607}) {}; % Ari,  5.55 
\node[pin={[pin distance=-0.4\onedegree,Flaamsted]00:{64}}] at (axis cs:{51.077},{24.724}) {}; % Ari,  5.50 
\node[pin={[pin distance=-0.4\onedegree,Flaamsted]00:{ 7}}] at (axis cs:{28.963},{23.577}) {}; % Ari,  5.75 
\node[pin={[pin distance=-0.4\onedegree,Bayer]00:{$\boldsymbol\chi$}}] at (axis cs:{83.182},{32.192}) {}; % Aur,  4.74 
\node[pin={[pin distance=-0.4\onedegree,Bayer]00:{$\boldsymbol\iota$}}] at (axis cs:{74.248},{33.166}) {}; % Aur,  2.68 
\node[pin={[pin distance=-0.4\onedegree,Bayer]00:{$\boldsymbol\kappa$}}] at (axis cs:{93.845},{29.498}) {}; % Aur,  4.33 
\node[pin={[pin distance=-0.4\onedegree,Bayer]00:{$\boldsymbol\mu$}}] at (axis cs:{78.357},{38.484}) {}; % Aur,  4.83 
\node[pin={[pin distance=-0.4\onedegree,Bayer]-90:{$\boldsymbol\nu$}}] at (axis cs:{87.872},{39.148}) {}; % Aur,  3.97 
\node[pin={[pin distance=-0.4\onedegree,Bayer]00:{$\boldsymbol\omega$}}] at (axis cs:{74.814},{37.890}) {}; % Aur,  4.99 
\node[pin={[pin distance=-0.4\onedegree,Bayer]-135:{$\boldsymbol\psi^3$}}] at (axis cs:{99.705},{39.903}) {}; % Aur,  5.35 
\node[pin={[pin distance=-0.4\onedegree,Bayer]00:{$\boldsymbol\sigma$}}] at (axis cs:{81.163},{37.385}) {}; % Aur,  4.99 
\node[pin={[pin distance=-0.4\onedegree,Bayer]00:{$\boldsymbol\tau$}}] at (axis cs:{87.293},{39.181}) {}; % Aur,  4.52 
\node[pin={[pin distance=-0.4\onedegree,Bayer]00:{$\boldsymbol\upsilon$}}] at (axis cs:{87.760},{37.305}) {}; % Aur,  4.73 
\node[pin={[pin distance=-0.4\onedegree,Bayer]00:{$\boldsymbol\varphi$}}] at (axis cs:{81.912},{34.476}) {}; % Aur,  5.06 
\node[pin={[pin distance=-0.4\onedegree,Bayer]00:{$\boldsymbol\vartheta$}}] at (axis cs:{89.930},{37.212}) {}; % Aur,  2.65 
\node[pin={[pin distance=-0.4\onedegree,Flaamsted]00:{14}}] at (axis cs:{78.852},{32.688}) {}; % Aur,  4.99 
\node[pin={[pin distance=-0.4\onedegree,Flaamsted]00:{16}}] at (axis cs:{79.544},{33.372}) {}; % Aur,  4.54 
\node[pin={[pin distance=-0.4\onedegree,Flaamsted]90:{19}}] at (axis cs:{80.004},{33.958}) {}; % Aur,  5.04 
\node[pin={[pin distance=-0.4\onedegree,Flaamsted]00:{26}}] at (axis cs:{84.659},{30.492}) {}; % Aur,  5.41 
\node[pin={[pin distance=-0.4\onedegree,Flaamsted]00:{ 2}}] at (axis cs:{73.158},{36.703}) {}; % Aur,  4.78 
\node[pin={[pin distance=-0.4\onedegree,Flaamsted]00:{40}}] at (axis cs:{91.646},{38.482}) {}; % Aur,  5.35 
\node[pin={[pin distance=-0.4\onedegree,Flaamsted]00:{RT}}] at (axis cs:{97.142},{30.493}) {}; % Aur,  5.75 
%\node[pin={[pin distance=-0.4\onedegree,Flaamsted]180:{48}}] at (axis cs:{97.142},{30.493}) {}; % Aur,  5.75 = RT
\node[pin={[pin distance=-0.4\onedegree,Flaamsted]90:{49}}] at (axis cs:{98.800},{28.022}) {}; % Aur,  5.27 
\node[pin={[pin distance=-0.4\onedegree,Flaamsted]00:{51}}] at (axis cs:{99.665},{39.391}) {}; % Aur,  5.70 
\node[pin={[pin distance=-0.4\onedegree,Flaamsted]90:{53}}] at (axis cs:{99.596},{28.984}) {}; % Aur,  5.75 
\node[pin={[pin distance=-0.4\onedegree,Flaamsted]00:{ 5}}] at (axis cs:{75.076},{39.395}) {}; % Aur,  5.98 
\node[pin={[pin distance=-0.4\onedegree,Flaamsted]00:{63}}] at (axis cs:{107.914},{39.320}) {}; % Aur,  4.89 
\node[pin={[pin distance=-0.4\onedegree,Bayer]00:{$\boldsymbol\beta$}}] at (axis cs:{70.515},{-37.144}) {}; % Cae,  5.04 
\node[pin={[pin distance=-0.4\onedegree,Bayer]00:{$\boldsymbol\gamma^1$}}] at (axis cs:{76.102},{-35.483}) {}; % Cae,  4.57 


\node[pin={[pin distance=-0.2\onedegree,Bayer]00:{$\boldsymbol\beta$}}] at (axis cs:{10.897},{-17.987}) {}; % Cet,  2.05 
\node[pin={[pin distance=-0.4\onedegree,Bayer]00:{$\boldsymbol\alpha$}}] at (axis cs:{45.570},{4.090}) {}; % Cet,  2.55 
\node[pin={[pin distance=-0.4\onedegree,Bayer]00:{$\boldsymbol\chi$}}] at (axis cs:{27.396},{-10.686}) {}; % Cet,  4.66 
\node[pin={[pin distance=-0.4\onedegree,Bayer]00:{$\boldsymbol\delta$}}] at (axis cs:{39.871},{0.328}) {}; % Cet,  4.07 
\node[pin={[pin distance=-0.4\onedegree,Bayer]00:{$\boldsymbol\epsilon$}}] at (axis cs:{39.891},{-11.872}) {}; % Cet,  4.83 
\node[pin={[pin distance=-0.4\onedegree,Bayer]-90:{$\boldsymbol\eta$}}] at (axis cs:{17.147},{-10.182}) {}; % Cet,  3.46 
\node[pin={[pin distance=-0.4\onedegree,Bayer]00:{$\boldsymbol\gamma$}}] at (axis cs:{40.825},{3.236}) {}; % Cet,  3.47 
\node[pin={[pin distance=-0.4\onedegree,Bayer]00:{$\boldsymbol\iota$}}] at (axis cs:{4.857},{-8.824}) {}; % Cet,  3.55 
\node[pin={[pin distance=-0.4\onedegree,Bayer]00:{$\boldsymbol\kappa^1$}}] at (axis cs:{49.840},{3.370}) {}; % Cet,  4.85 
\node[pin={[pin distance=-0.4\onedegree,Bayer]90:{$\boldsymbol\kappa^2$}}] at (axis cs:{50.278},{3.675}) {}; % Cet,  5.68 
\node[pin={[pin distance=-0.4\onedegree,Bayer]00:{$\boldsymbol\lambda$}}] at (axis cs:{44.929},{8.907}) {}; % Cet,  4.71 
\node[pin={[pin distance=-0.4\onedegree,Bayer]00:{$\boldsymbol\mu$}}] at (axis cs:{41.236},{10.114}) {}; % Cet,  4.27 
\node[pin={[pin distance=-0.4\onedegree,Bayer]00:{$\boldsymbol\nu$}}] at (axis cs:{38.969},{5.593}) {}; % Cet,  4.88 
\node[pin={[pin distance=-0.4\onedegree,Bayer]00:{$\boldsymbol\omicron$}}] at (axis cs:{34.837},{-2.977}) {}; % Cet,  4.95 
\node[pin={[pin distance=-0.4\onedegree,interest-label]-90:{Mira}}] at (axis cs:{34.837},{-2.977}) {}; % Cet,  4.95 
\node[pin={[pin distance=-0.4\onedegree,Bayer]00:{$\boldsymbol\pi$}}] at (axis cs:{41.031},{-13.859}) {}; % Cet,  4.25 
\node[pin={[pin distance=-0.4\onedegree,Bayer]00:{$\boldsymbol\rho$}}] at (axis cs:{36.488},{-12.290}) {}; % Cet,  4.88 
\node[pin={[pin distance=-0.4\onedegree,Bayer]00:{$\boldsymbol\sigma$}}] at (axis cs:{38.022},{-15.244}) {}; % Cet,  4.74 
\node[pin={[pin distance=-0.4\onedegree,Bayer]00:{$\boldsymbol\tau$}}] at (axis cs:{26.017},{-15.937}) {}; % Cet,  3.49 
\node[pin={[pin distance=-0.4\onedegree,Bayer]00:{$\boldsymbol\upsilon$}}] at (axis cs:{30.001},{-21.078}) {}; % Cet,  4.00 
\node[pin={[pin distance=-0.4\onedegree,Bayer]00:{$\boldsymbol\varphi^1$}}] at (axis cs:{11.048},{-10.609}) {}; % Cet,  4.77 
\node[pin={[pin distance=-0.4\onedegree,Bayer]90:{$\boldsymbol\varphi^2$}}] at (axis cs:{12.532},{-10.644}) {}; % Cet,  5.17 
\node[pin={[pin distance=-0.4\onedegree,Bayer]00:{$\boldsymbol\varphi^3$}}] at (axis cs:{14.006},{-11.266}) {}; % Cet,  5.34 
\node[pin={[pin distance=-0.4\onedegree,Bayer]-90:{$\boldsymbol\varphi^4$}}] at (axis cs:{14.683},{-11.380}) {}; % Cet,  5.62 
\node[pin={[pin distance=-0.4\onedegree,Bayer]00:{$\boldsymbol\vartheta$}}] at (axis cs:{21.006},{-8.183}) {}; % Cet,  3.61 
\node[pin={[pin distance=-0.4\onedegree,Bayer]180:{$\boldsymbol\xi^1$}}] at (axis cs:{33.250},{8.846}) {}; % Cet,  4.37 
\node[pin={[pin distance=-0.4\onedegree,Bayer]00:{$\boldsymbol\xi^2$}}] at (axis cs:{37.040},{8.460}) {}; % Cet,  4.28 
\node[pin={[pin distance=-0.4\onedegree,Bayer]90:{$\boldsymbol\zeta$}}] at (axis cs:{27.865},{-10.335}) {}; % Cet,  3.73 
\node[pin={[pin distance=-0.4\onedegree,Flaamsted]-90:{12}}] at (axis cs:{7.510},{-3.957}) {}; % Cet,  5.73 
\node[pin={[pin distance=-0.4\onedegree,Flaamsted]00:{13}}] at (axis cs:{8.812},{-3.593}) {}; % Cet,  5.20 
\node[pin={[pin distance=-0.4\onedegree,Flaamsted]00:{14}}] at (axis cs:{8.887},{-0.506}) {}; % Cet,  5.94 
\node[pin={[pin distance=-0.4\onedegree,Flaamsted]00:{20}}] at (axis cs:{13.252},{-1.144}) {}; % Cet,  4.76 
\node[pin={[pin distance=-0.4\onedegree,Flaamsted]00:{25}}] at (axis cs:{15.761},{-4.837}) {}; % Cet,  5.40 
\node[pin={[pin distance=-0.4\onedegree,Flaamsted]00:{28}}] at (axis cs:{16.521},{-9.839}) {}; % Cet,  5.58 
\node[pin={[pin distance=-0.4\onedegree,Flaamsted]00:{ 2}}] at (axis cs:{0.935},{-17.336}) {}; % Cet,  4.55 
\node[pin={[pin distance=-0.4\onedegree,Flaamsted]180:{30}}] at (axis cs:{16.943},{-9.786}) {}; % Cet,  5.71 
\node[pin={[pin distance=-0.4\onedegree,Flaamsted]00:{33}}] at (axis cs:{17.640},{2.445}) {}; % Cet,  5.95 
\node[pin={[pin distance=-0.4\onedegree,Flaamsted]00:{34}}] at (axis cs:{17.931},{-2.251}) {}; % Cet,  5.93 
\node[pin={[pin distance=-0.4\onedegree,Flaamsted]90:{37}}] at (axis cs:{18.600},{-7.923}) {}; % Cet,  5.15 
\node[pin={[pin distance=-0.4\onedegree,Flaamsted]00:{38}}] at (axis cs:{18.705},{-0.974}) {}; % Cet,  5.70 
\node[pin={[pin distance=-0.4\onedegree,Flaamsted]00:{39}}] at (axis cs:{19.151},{-2.500}) {}; % Cet,  5.43 
\node[pin={[pin distance=-0.4\onedegree,Flaamsted]00:{ 3}}] at (axis cs:{1.125},{-10.509}) {}; % Cet,  4.94 
\node[pin={[pin distance=-0.4\onedegree,Flaamsted]00:{46}}] at (axis cs:{21.405},{-14.599}) {}; % Cet,  4.91 
\node[pin={[pin distance=-0.4\onedegree,Flaamsted]00:{47}}] at (axis cs:{21.715},{-13.056}) {}; % Cet,  5.51 
\node[pin={[pin distance=-0.4\onedegree,Flaamsted]00:{48}}] at (axis cs:{22.401},{-21.629}) {}; % Cet,  5.11 
\node[pin={[pin distance=-0.4\onedegree,Flaamsted]00:{49}}] at (axis cs:{23.657},{-15.676}) {}; % Cet,  5.62 
\node[pin={[pin distance=-0.4\onedegree,Flaamsted]90:{50}}] at (axis cs:{23.996},{-15.400}) {}; % Cet,  5.41 
\node[pin={[pin distance=-0.4\onedegree,Flaamsted]00:{56}}] at (axis cs:{29.167},{-22.527}) {}; % Cet,  4.90 
\node[pin={[pin distance=-0.4\onedegree,Flaamsted]90:{57}}] at (axis cs:{29.942},{-20.824}) {}; % Cet,  5.44 
\node[pin={[pin distance=-0.4\onedegree,Flaamsted]90:{60}}] at (axis cs:{30.799},{0.128}) {}; % Cet,  5.43 
\node[pin={[pin distance=-0.4\onedegree,Flaamsted]00:{61}}] at (axis cs:{30.951},{-0.340}) {}; % Cet,  5.95 
\node[pin={[pin distance=-0.4\onedegree,Flaamsted]00:{63}}] at (axis cs:{32.899},{-1.825}) {}; % Cet,  5.94 
\node[pin={[pin distance=-0.4\onedegree,Flaamsted]00:{64}}] at (axis cs:{32.838},{8.570}) {}; % Cet,  5.64 
\node[pin={[pin distance=-0.4\onedegree,Flaamsted]00:{66}}] at (axis cs:{33.198},{-2.393}) {}; % Cet,  5.65 
\node[pin={[pin distance=-0.4\onedegree,Flaamsted]00:{67}}] at (axis cs:{34.246},{-6.422}) {}; % Cet,  5.50 
\node[pin={[pin distance=-0.4\onedegree,Flaamsted]00:{69}}] at (axis cs:{35.486},{0.395}) {}; % Cet,  5.30 
\node[pin={[pin distance=-0.4\onedegree,Flaamsted]00:{ 6}}] at (axis cs:{2.816},{-15.468}) {}; % Cet,  4.90 
\node[pin={[pin distance=-0.4\onedegree,Flaamsted]00:{70}}] at (axis cs:{35.552},{-0.885}) {}; % Cet,  5.43 
\node[pin={[pin distance=-0.4\onedegree,Flaamsted]180:{75}}] at (axis cs:{38.039},{-1.035}) {}; % Cet,  5.36 
\node[pin={[pin distance=-0.4\onedegree,Flaamsted]00:{77}}] at (axis cs:{38.678},{-7.859}) {}; % Cet,  5.73 
\node[pin={[pin distance=-0.4\onedegree,Flaamsted]00:{ 7}}] at (axis cs:{3.660},{-18.933}) {}; % Cet,  4.46 
\node[pin={[pin distance=-0.4\onedegree,Flaamsted]90:{80}}] at (axis cs:{39.000},{-7.831}) {}; % Cet,  5.53 
\node[pin={[pin distance=-0.4\onedegree,Flaamsted]00:{81}}] at (axis cs:{39.424},{-3.396}) {}; % Cet,  5.65 
\node[pin={[pin distance=-0.4\onedegree,Flaamsted]00:{84}}] at (axis cs:{40.308},{-0.695}) {}; % Cet,  5.76 
\node[pin={[pin distance=-0.4\onedegree,Flaamsted]90:{93}}] at (axis cs:{45.594},{4.353}) {}; % Cet,  5.62 
\node[pin={[pin distance=-0.4\onedegree,Flaamsted]90:{94}}] at (axis cs:{48.193},{-1.196}) {}; % Cet,  5.07 
\node[pin={[pin distance=-0.4\onedegree,Flaamsted]00:{95}}] at (axis cs:{49.593},{-0.930}) {}; % Cet,  5.42 

\node[pin={[pin distance=-0.2\onedegree,Bayer]00:{$\boldsymbol\alpha$}}] at (axis cs:{101.287},{-16.716}) {}; % CMa, -1.44 
\node[pin={[pin distance=-0.2\onedegree,Bayer]00:{$\boldsymbol\beta$}}] at (axis cs:{95.675},{-17.956}) {}; % CMa,  1.96 
\node[pin={[pin distance=-0.2\onedegree,Bayer]00:{$\boldsymbol\delta$}}] at (axis cs:{107.098},{-26.393}) {}; % CMa,  1.84 
\node[pin={[pin distance=-0.2\onedegree,Bayer]00:{$\boldsymbol\epsilon$}}] at (axis cs:{104.656},{-28.972}) {}; % CMa,  1.53 
\node[pin={[pin distance=-0.4\onedegree,Bayer]00:{$\boldsymbol\gamma$}}] at (axis cs:{105.940},{-15.633}) {}; % CMa,  4.11 
\node[pin={[pin distance=-0.4\onedegree,Bayer]00:{$\boldsymbol\iota$}}] at (axis cs:{104.034},{-17.054}) {}; % CMa,  4.39 
\node[pin={[pin distance=-0.4\onedegree,Bayer]00:{$\boldsymbol\kappa$}}] at (axis cs:{102.460},{-32.508}) {}; % CMa,  3.53 
\node[pin={[pin distance=-0.4\onedegree,Bayer]00:{$\boldsymbol\lambda$}}] at (axis cs:{97.043},{-32.580}) {}; % CMa,  4.47 
\node[pin={[pin distance=-0.4\onedegree,Bayer]00:{$\boldsymbol\mu$}}] at (axis cs:{104.028},{-14.043}) {}; % CMa,  4.97 
\node[pin={[pin distance=-0.4\onedegree,Bayer]00:{$\boldsymbol\nu^1$}}] at (axis cs:{99.095},{-18.660}) {}; % CMa,  5.70 
\node[pin={[pin distance=-0.4\onedegree,Bayer]00:{$\boldsymbol\nu^2$}}] at (axis cs:{99.171},{-19.256}) {}; % CMa,  3.95 
\node[pin={[pin distance=-0.8\onedegree,Bayer]20:{$\boldsymbol\nu^3$}}] at (axis cs:{99.473},{-18.237}) {}; % CMa,  4.43 
\node[pin={[pin distance=-0.4\onedegree,Bayer]00:{$\boldsymbol\omega$}}] at (axis cs:{108.703},{-26.773}) {}; % CMa,  4.04 
\node[pin={[pin distance=-0.4\onedegree,Bayer]00:{$\boldsymbol\omicron^1$}}] at (axis cs:{103.533},{-24.184}) {}; % CMa,  3.84 
\node[pin={[pin distance=-0.4\onedegree,Bayer]00:{$\boldsymbol\omicron^2$}}] at (axis cs:{105.756},{-23.833}) {}; % CMa,  3.04 
\node[pin={[pin distance=-0.4\onedegree,Bayer]180:{$\boldsymbol\pi$}}] at (axis cs:{103.906},{-20.136}) {}; % CMa,  4.68 
\node[pin={[pin distance=-0.4\onedegree,Bayer]00:{$\boldsymbol\sigma$}}] at (axis cs:{105.430},{-27.935}) {}; % CMa,  3.47 
\node[pin={[pin distance=-0.4\onedegree,Bayer]-90:{$\boldsymbol\tau$}}] at (axis cs:{109.677},{-24.954}) {}; % CMa,  4.40 
\node[pin={[pin distance=-0.4\onedegree,Bayer]00:{$\boldsymbol\vartheta$}}] at (axis cs:{103.547},{-12.039}) {}; % CMa,  4.07 
\node[pin={[pin distance=-0.4\onedegree,Bayer]00:{$\boldsymbol\xi^1$}}] at (axis cs:{97.964},{-23.418}) {}; % CMa,  4.33 
\node[pin={[pin distance=-0.8\onedegree,Bayer]20:{$\boldsymbol\xi^2$}}] at (axis cs:{98.764},{-22.965}) {}; % CMa,  4.53 
\node[pin={[pin distance=-0.4\onedegree,Bayer]00:{$\boldsymbol\zeta$}}] at (axis cs:{95.078},{-30.063}) {}; % CMa,  3.02 
\node[pin={[pin distance=-0.4\onedegree,Flaamsted]00:{10}}] at (axis cs:{101.119},{-31.070}) {}; % CMa,  5.23 
\node[pin={[pin distance=-0.4\onedegree,Flaamsted]00:{11}}] at (axis cs:{101.713},{-14.426}) {}; % CMa,  5.28 
\node[pin={[pin distance=-0.4\onedegree,Flaamsted]00:{15}}] at (axis cs:{103.387},{-20.224}) {}; % CMa,  4.82 
\node[pin={[pin distance=-0.4\onedegree,Flaamsted]-90:{17}}] at (axis cs:{103.761},{-20.405}) {}; % CMa,  5.80 
%\node[pin={[pin distance=-0.4\onedegree,Flaamsted]00:{26}}] at (axis cs:{108.051},{-25.942}) {}; % CMa,  5.91 
\node[pin={[pin distance=-0.8\onedegree,Flaamsted]105:{27}}] at (axis cs:{108.563},{-26.352}) {}; % CMa,  4.45 
\node[pin={[pin distance=-0.8\onedegree,Flaamsted]90:{29}}] at (axis cs:{109.668},{-24.559}) {}; % CMa,  4.92 

\node[pin={[pin distance=-0.4\onedegree,Bayer]00:{$\boldsymbol\alpha$}}] at (axis cs:{84.912},{-34.074}) {}; % Col,  2.66 
\node[pin={[pin distance=-0.4\onedegree,Bayer]90:{$\boldsymbol\beta$}}] at (axis cs:{87.740},{-35.768}) {}; % Col,  3.11 
\node[pin={[pin distance=-0.4\onedegree,Bayer]00:{$\boldsymbol\delta$}}] at (axis cs:{95.528},{-33.436}) {}; % Col,  3.85 
\node[pin={[pin distance=-0.4\onedegree,Bayer]90:{$\boldsymbol\epsilon$}}] at (axis cs:{82.803},{-35.470}) {}; % Col,  3.87 
\node[pin={[pin distance=-0.4\onedegree,Bayer]00:{$\boldsymbol\gamma$}}] at (axis cs:{89.384},{-35.283}) {}; % Col,  4.36 
\node[pin={[pin distance=-0.4\onedegree,Bayer]00:{$\boldsymbol\kappa$}}] at (axis cs:{94.138},{-35.140}) {}; % Col,  4.37 
\node[pin={[pin distance=-0.4\onedegree,Bayer]00:{$\boldsymbol\lambda$}}] at (axis cs:{88.279},{-33.801}) {}; % Col,  4.87 
\node[pin={[pin distance=-0.4\onedegree,Bayer]00:{$\boldsymbol\mu$}}] at (axis cs:{86.500},{-32.306}) {}; % Col,  5.16 
\node[pin={[pin distance=-0.4\onedegree,Bayer]00:{$\boldsymbol\nu^2$}}] at (axis cs:{84.436},{-28.690}) {}; % Col,  5.28 
\node[pin={[pin distance=-0.4\onedegree,Bayer]00:{$\boldsymbol\omicron$}}] at (axis cs:{79.371},{-34.895}) {}; % Col,  4.82 
\node[pin={[pin distance=-0.4\onedegree,Bayer]00:{$\boldsymbol\sigma$}}] at (axis cs:{89.087},{-31.382}) {}; % Col,  5.52 
\node[pin={[pin distance=-0.4\onedegree,Bayer]00:{$\boldsymbol\vartheta$}}] at (axis cs:{91.882},{-37.253}) {}; % Col,  5.00 
\node[pin={[pin distance=-0.4\onedegree,Bayer]-90:{$\boldsymbol\xi$}}] at (axis cs:{88.875},{-37.121}) {}; % Col,  4.96 

\node[pin={[pin distance=-0.4\onedegree,Bayer]-90:{$\boldsymbol\beta$}}] at (axis cs:{76.962},{-5.086}) {}; % Eri,  2.79 
\node[pin={[pin distance=-0.4\onedegree,Bayer]00:{$\boldsymbol\delta$}}] at (axis cs:{55.812},{-9.763}) {}; % Eri,  3.53 
\node[pin={[pin distance=-0.4\onedegree,Bayer]00:{$\boldsymbol\epsilon$}}] at (axis cs:{53.233},{-9.458}) {}; % Eri,  3.73 
\node[pin={[pin distance=-0.4\onedegree,Bayer]00:{$\boldsymbol\eta$}}] at (axis cs:{44.107},{-8.898}) {}; % Eri,  3.90 
\node[pin={[pin distance=-0.4\onedegree,Bayer]00:{$\boldsymbol\gamma$}}] at (axis cs:{59.507},{-13.508}) {}; % Eri,  2.96 
\node[pin={[pin distance=-0.4\onedegree,Bayer]00:{$\boldsymbol\iota$}}] at (axis cs:{40.167},{-39.855}) {}; % Eri,  4.12 
\node[pin={[pin distance=-0.4\onedegree,Bayer]00:{$\boldsymbol\lambda$}}] at (axis cs:{77.287},{-8.754}) {}; % Eri,  4.26 
\node[pin={[pin distance=-0.4\onedegree,Bayer]00:{$\boldsymbol\mu$}}] at (axis cs:{71.376},{-3.255}) {}; % Eri,  4.02 
\node[pin={[pin distance=-0.4\onedegree,Bayer]-90:{$\boldsymbol\nu$}}] at (axis cs:{69.080},{-3.352}) {}; % Eri,  3.93 
\node[pin={[pin distance=-0.4\onedegree,Bayer]00:{$\boldsymbol\omega$}}] at (axis cs:{73.224},{-5.453}) {}; % Eri,  4.38 
\node[pin={[pin distance=-0.4\onedegree,Bayer]90:{$\boldsymbol\omicron^1$}}] at (axis cs:{62.966},{-6.838}) {}; % Eri,  4.04 
\node[pin={[pin distance=-0.4\onedegree,Bayer]00:{$\boldsymbol\omicron^2$}}] at (axis cs:{63.818},{-7.653}) {}; % Eri,  4.42 
\node[pin={[pin distance=-0.4\onedegree,Bayer]00:{$\boldsymbol\pi$}}] at (axis cs:{56.536},{-12.102}) {}; % Eri,  4.43 
\node[pin={[pin distance=-0.4\onedegree,Bayer]00:{$\boldsymbol\rho^1$}}] at (axis cs:{45.292},{-7.663}) {}; % Eri,  5.75 
\node[pin={[pin distance=-0.4\onedegree,Bayer]-90:{$\boldsymbol\rho^2$}}] at (axis cs:{45.676},{-7.685}) {}; % Eri,  5.33 
\node[pin={[pin distance=-0.4\onedegree,Bayer]90:{$\boldsymbol\rho^3$}}] at (axis cs:{46.069},{-7.601}) {}; % Eri,  5.26 
\node[pin={[pin distance=-0.4\onedegree,Bayer]00:{$\boldsymbol\psi$}}] at (axis cs:{75.360},{-7.174}) {}; % Eri,  4.79 
\node[pin={[pin distance=-0.4\onedegree,Bayer]180:{$\boldsymbol\tau^1$}}] at (axis cs:{41.276},{-18.572}) {}; % Eri,  4.47 
\node[pin={[pin distance=-0.4\onedegree,Bayer]00:{$\boldsymbol\tau^2$}}] at (axis cs:{42.760},{-21.004}) {}; % Eri,  4.77 
\node[pin={[pin distance=-0.4\onedegree,Bayer]90:{$\boldsymbol\tau^3$}}] at (axis cs:{45.598},{-23.624}) {}; % Eri,  4.09 
\node[pin={[pin distance=-0.4\onedegree,Bayer]00:{$\boldsymbol\tau^4$}}] at (axis cs:{49.879},{-21.758}) {}; % Eri,  3.74 
\node[pin={[pin distance=-0.4\onedegree,Bayer]00:{$\boldsymbol\tau^5$}}] at (axis cs:{53.447},{-21.633}) {}; % Eri,  4.27 
\node[pin={[pin distance=-0.4\onedegree,Bayer]00:{$\boldsymbol\tau^6$}}] at (axis cs:{56.712},{-23.249}) {}; % Eri,  4.22 
\node[pin={[pin distance=-0.4\onedegree,Bayer]180:{$\boldsymbol\tau^7$}}] at (axis cs:{56.915},{-23.875}) {}; % Eri,  5.24 
\node[pin={[pin distance=-0.4\onedegree,Bayer]-90:{$\boldsymbol\tau^8$}}] at (axis cs:{58.428},{-24.612}) {}; % Eri,  4.64 
\node[pin={[pin distance=-0.4\onedegree,Bayer]00:{$\boldsymbol\tau^9$}}] at (axis cs:{59.981},{-24.016}) {}; % Eri,  4.63 
\node[pin={[pin distance=-0.4\onedegree,Bayer]90:{$\boldsymbol\upsilon^1$}}] at (axis cs:{68.377},{-29.766}) {}; % Eri,  4.51 
\node[pin={[pin distance=-0.4\onedegree,Bayer]00:{$\boldsymbol\upsilon^2$}}] at (axis cs:{68.888},{-30.562}) {}; % Eri,  3.81 
\node[pin={[pin distance=-0.4\onedegree,Bayer]00:{$\boldsymbol\upsilon^4$}}] at (axis cs:{64.474},{-33.798}) {}; % Eri,  3.56 
\node[pin={[pin distance=-0.4\onedegree,Bayer]00:{$\boldsymbol\xi$}}] at (axis cs:{65.920},{-3.745}) {}; % Eri,  5.17 
\node[pin={[pin distance=-0.4\onedegree,Bayer]00:{$\boldsymbol\zeta$}}] at (axis cs:{48.958},{-8.820}) {}; % Eri,  4.81 
\node[pin={[pin distance=-0.4\onedegree,Flaamsted]00:{15}}] at (axis cs:{49.592},{-22.511}) {}; % Eri,  4.87 
\node[pin={[pin distance=-0.4\onedegree,Flaamsted]00:{17}}] at (axis cs:{52.654},{-5.075}) {}; % Eri,  4.74 
\node[pin={[pin distance=-0.4\onedegree,Flaamsted]00:{20}}] at (axis cs:{54.073},{-17.467}) {}; % Eri,  5.25 
\node[pin={[pin distance=-0.4\onedegree,Flaamsted]00:{21}}] at (axis cs:{54.755},{-5.626}) {}; % Eri,  5.97 
\node[pin={[pin distance=-0.4\onedegree,Flaamsted]00:{22}}] at (axis cs:{55.160},{-5.211}) {}; % Eri,  5.53 
\node[pin={[pin distance=-0.4\onedegree,Flaamsted]180:{24}}] at (axis cs:{56.127},{-1.163}) {}; % Eri,  5.24 
\node[pin={[pin distance=-0.4\onedegree,Flaamsted]00:{25}}] at (axis cs:{56.235},{-0.297}) {}; % Eri,  5.56 
\node[pin={[pin distance=-0.4\onedegree,Flaamsted]00:{30}}] at (axis cs:{58.174},{-5.361}) {}; % Eri,  5.49 
\node[pin={[pin distance=-0.4\onedegree,Flaamsted]00:{32}}] at (axis cs:{58.573},{-2.955}) {}; % Eri,  4.72 
\node[pin={[pin distance=-0.4\onedegree,Flaamsted]00:{35}}] at (axis cs:{60.384},{-1.549}) {}; % Eri,  5.28 
\node[pin={[pin distance=-0.4\onedegree,Flaamsted]00:{37}}] at (axis cs:{62.594},{-6.924}) {}; % Eri,  5.44 
\node[pin={[pin distance=-0.4\onedegree,Flaamsted]00:{39}}] at (axis cs:{63.599},{-10.256}) {}; % Eri,  4.90 
\node[pin={[pin distance=-0.4\onedegree,Flaamsted]00:{43}}] at (axis cs:{66.009},{-34.017}) {}; % Eri,  3.96 
\node[pin={[pin distance=-0.4\onedegree,Flaamsted]-90:{45}}] at (axis cs:{67.969},{-0.044}) {}; % Eri,  4.90 
\node[pin={[pin distance=-0.4\onedegree,Flaamsted]00:{46}}] at (axis cs:{68.478},{-6.739}) {}; % Eri,  5.73 
\node[pin={[pin distance=-0.4\onedegree,Flaamsted]00:{47}}] at (axis cs:{68.548},{-8.231}) {}; % Eri,  5.20 
\node[pin={[pin distance=-0.4\onedegree,Flaamsted]00:{ 4}}] at (axis cs:{44.349},{-23.862}) {}; % Eri,  5.44 
\node[pin={[pin distance=-0.4\onedegree,Flaamsted]00:{51}}] at (axis cs:{69.401},{-2.473}) {}; % Eri,  5.22 
\node[pin={[pin distance=-0.4\onedegree,Flaamsted]00:{53}}] at (axis cs:{69.545},{-14.304}) {}; % Eri,  3.87 
\node[pin={[pin distance=-0.4\onedegree,Flaamsted]00:{54}}] at (axis cs:{70.110},{-19.671}) {}; % Eri,  4.34 
\node[pin={[pin distance=-0.4\onedegree,Flaamsted]00:{56}}] at (axis cs:{71.022},{-8.504}) {}; % Eri,  5.78 
\node[pin={[pin distance=-0.4\onedegree,Flaamsted]00:{58}}] at (axis cs:{71.901},{-16.934}) {}; % Eri,  5.49 
\node[pin={[pin distance=-0.4\onedegree,Flaamsted]00:{59}}] at (axis cs:{72.136},{-16.329}) {}; % Eri,  5.76 
\node[pin={[pin distance=-0.4\onedegree,Flaamsted]00:{ 5}}] at (axis cs:{44.921},{-2.465}) {}; % Eri,  5.55 
\node[pin={[pin distance=-0.4\onedegree,Flaamsted]90:{60}}] at (axis cs:{72.548},{-16.217}) {}; % Eri,  5.03 
\node[pin={[pin distance=-0.4\onedegree,Flaamsted]90:{62}}] at (axis cs:{74.101},{-5.171}) {}; % Eri,  5.50 
\node[pin={[pin distance=-0.4\onedegree,Flaamsted]00:{63}}] at (axis cs:{74.960},{-10.263}) {}; % Eri,  5.39 
\node[pin={[pin distance=-0.4\onedegree,Flaamsted]00:{64}}] at (axis cs:{74.982},{-12.537}) {}; % Eri,  4.78 
\node[pin={[pin distance=-0.4\onedegree,Flaamsted]-90:{S}}] at (axis cs:{74.982},{-12.537}) {}; % Eri,  4.78 
\node[pin={[pin distance=-0.4\onedegree,Flaamsted]90:{66}}] at (axis cs:{76.690},{-4.655}) {}; % Eri,  5.10 
\node[pin={[pin distance=-0.4\onedegree,Flaamsted]90:{68}}] at (axis cs:{77.182},{-4.456}) {}; % Eri,  5.12 
\node[pin={[pin distance=-0.4\onedegree,Flaamsted]90:{ 6}}] at (axis cs:{44.524},{-23.606}) {}; % Eri,  5.81 
\node[pin={[pin distance=-0.4\onedegree,Bayer]00:{$\boldsymbol\alpha$}}] at (axis cs:{48.019},{-28.988}) {}; % For,  3.80 
\node[pin={[pin distance=-0.4\onedegree,Bayer]00:{$\boldsymbol\beta$}}] at (axis cs:{42.273},{-32.406}) {}; % For,  4.46 
\node[pin={[pin distance=-0.4\onedegree,Bayer]00:{$\boldsymbol\chi^2$}}] at (axis cs:{51.889},{-35.681}) {}; % For,  5.70 
\node[pin={[pin distance=-0.4\onedegree,Bayer]00:{$\boldsymbol\delta$}}] at (axis cs:{55.562},{-31.938}) {}; % For,  4.98 
\node[pin={[pin distance=-0.4\onedegree,Bayer]00:{$\boldsymbol\epsilon$}}] at (axis cs:{45.407},{-28.091}) {}; % For,  5.88 
\node[pin={[pin distance=-0.4\onedegree,Bayer]00:{$\boldsymbol\eta^2$}}] at (axis cs:{42.562},{-35.843}) {}; % For,  5.93 
\node[pin={[pin distance=-0.4\onedegree,Bayer]180:{$\boldsymbol\eta^3$}}] at (axis cs:{42.668},{-35.676}) {}; % For,  5.48 
\node[pin={[pin distance=-0.4\onedegree,Bayer]00:{$\boldsymbol\gamma^2$}}] at (axis cs:{42.476},{-27.942}) {}; % For,  5.39 
\node[pin={[pin distance=-0.4\onedegree,Bayer]00:{$\boldsymbol\iota^1$}}] at (axis cs:{39.039},{-30.045}) {}; % For,  5.73 
\node[pin={[pin distance=-0.4\onedegree,Bayer]-90:{$\boldsymbol\iota^2$}}] at (axis cs:{39.578},{-30.194}) {}; % For,  5.84 
\node[pin={[pin distance=-0.4\onedegree,Bayer]-90:{$\boldsymbol\kappa$}}] at (axis cs:{35.636},{-23.816}) {}; % For,  5.20 
\node[pin={[pin distance=-0.4\onedegree,Bayer]00:{$\boldsymbol\lambda^1$}}] at (axis cs:{38.279},{-34.650}) {}; % For,  5.90 
\node[pin={[pin distance=-0.4\onedegree,Bayer]-90:{$\boldsymbol\lambda^2$}}] at (axis cs:{39.244},{-34.578}) {}; % For,  5.78 
\node[pin={[pin distance=-0.4\onedegree,Bayer]00:{$\boldsymbol\mu$}}] at (axis cs:{33.227},{-30.724}) {}; % For,  5.27 
\node[pin={[pin distance=-0.4\onedegree,Bayer]00:{$\boldsymbol\nu$}}] at (axis cs:{31.123},{-29.297}) {}; % For,  4.70 
\node[pin={[pin distance=-0.4\onedegree,Bayer]00:{$\boldsymbol\omega$}}] at (axis cs:{38.461},{-28.232}) {}; % For,  4.97 
\node[pin={[pin distance=-0.4\onedegree,Bayer]00:{$\boldsymbol\pi$}}] at (axis cs:{30.311},{-30.002}) {}; % For,  5.36 
\node[pin={[pin distance=-0.4\onedegree,Bayer]00:{$\boldsymbol\rho$}}] at (axis cs:{56.984},{-30.168}) {}; % For,  5.53 
\node[pin={[pin distance=-0.4\onedegree,Bayer]00:{$\boldsymbol\psi$}}] at (axis cs:{43.393},{-38.437}) {}; % For,  5.93 
\node[pin={[pin distance=-0.4\onedegree,Bayer]00:{$\boldsymbol\sigma$}}] at (axis cs:{56.614},{-29.338}) {}; % For,  5.92 
\node[pin={[pin distance=-0.4\onedegree,Bayer]00:{$\boldsymbol\varphi$}}] at (axis cs:{37.007},{-33.811}) {}; % For,  5.13 
\node[pin={[pin distance=-0.4\onedegree,Bayer]00:{$\boldsymbol\zeta$}}] at (axis cs:{44.901},{-25.274}) {}; % For,  5.70 
\node[pin={[pin distance=-0.2\onedegree,Bayer]00:{$\boldsymbol\gamma$}}] at (axis cs:{99.428},{16.399}) {}; % Gem,  2.02 
\node[pin={[pin distance=-0.4\onedegree,Bayer]90:{$\boldsymbol\epsilon$}}] at (axis cs:{100.983},{25.131}) {}; % Gem,  3.01 
\node[pin={[pin distance=-0.4\onedegree,Bayer]00:{$\boldsymbol\eta$}}] at (axis cs:{93.719},{22.507}) {}; % Gem,  3.30 
\node[pin={[pin distance=-0.4\onedegree,Bayer]00:{$\boldsymbol\lambda$}}] at (axis cs:{109.523},{16.540}) {}; % Gem,  3.58 
\node[pin={[pin distance=-0.4\onedegree,Bayer]00:{$\boldsymbol\mu$}}] at (axis cs:{95.740},{22.513}) {}; % Gem,  2.90 
\node[pin={[pin distance=-0.4\onedegree,Bayer]00:{$\boldsymbol\nu$}}] at (axis cs:{97.241},{20.212}) {}; % Gem,  4.14 
\node[pin={[pin distance=-0.4\onedegree,Bayer]-90:{$\boldsymbol\omega$}}] at (axis cs:{105.603},{24.215}) {}; % Gem,  5.18 
\node[pin={[pin distance=-0.4\onedegree,Bayer]00:{$\boldsymbol\tau$}}] at (axis cs:{107.785},{30.245}) {}; % Gem,  4.39 
\node[pin={[pin distance=-0.4\onedegree,Bayer]00:{$\boldsymbol\vartheta$}}] at (axis cs:{103.197},{33.961}) {}; % Gem,  3.61 
\node[pin={[pin distance=-0.4\onedegree,Bayer]00:{$\boldsymbol\xi$}}] at (axis cs:{101.322},{12.895}) {}; % Gem,  3.34 
\node[pin={[pin distance=-0.4\onedegree,Bayer]00:{$\boldsymbol\zeta$}}] at (axis cs:{106.027},{20.570}) {}; % Gem,  4.01 
\node[pin={[pin distance=-0.4\onedegree,Flaamsted]-90:{ 1}}] at (axis cs:{91.030},{23.263}) {}; % Gem,  4.18 
\node[pin={[pin distance=-0.4\onedegree,Flaamsted]00:{26}}] at (axis cs:{100.601},{17.645}) {}; % Gem,  5.29 
\node[pin={[pin distance=-0.4\onedegree,Flaamsted]00:{28}}] at (axis cs:{101.189},{28.971}) {}; % Gem,  5.42 
\node[pin={[pin distance=-0.4\onedegree,Flaamsted]90:{30}}] at (axis cs:{100.997},{13.228}) {}; % Gem,  4.49 
\node[pin={[pin distance=-0.4\onedegree,Flaamsted]00:{33}}] at (axis cs:{102.458},{16.203}) {}; % Gem,  5.87 
\node[pin={[pin distance=-0.4\onedegree,Flaamsted]00:{35}}] at (axis cs:{102.606},{13.413}) {}; % Gem,  5.68 
\node[pin={[pin distance=-0.4\onedegree,Flaamsted]00:{36}}] at (axis cs:{102.888},{21.761}) {}; % Gem,  5.27 
\node[pin={[pin distance=-0.4\onedegree,Flaamsted]90:{37}}] at (axis cs:{103.828},{25.376}) {}; % Gem,  5.75 
\node[pin={[pin distance=-0.4\onedegree,Flaamsted]00:{38}}] at (axis cs:{103.661},{13.178}) {}; % Gem,  4.70 
\node[pin={[pin distance=-0.4\onedegree,Flaamsted]00:{ 3}}] at (axis cs:{92.433},{23.113}) {}; % Gem,  5.76 
\node[pin={[pin distance=-0.4\onedegree,Flaamsted]00:{41}}] at (axis cs:{105.066},{16.079}) {}; % Gem,  5.72 
\node[pin={[pin distance=-0.4\onedegree,Flaamsted]00:{45}}] at (axis cs:{107.092},{15.930}) {}; % Gem,  5.49 
\node[pin={[pin distance=-0.4\onedegree,Flaamsted]00:{47}}] at (axis cs:{107.846},{26.856}) {}; % Gem,  5.76 
\node[pin={[pin distance=-0.4\onedegree,Flaamsted]00:{48}}] at (axis cs:{108.110},{24.128}) {}; % Gem,  5.85 
\node[pin={[pin distance=-0.4\onedegree,Flaamsted]00:{51}}] at (axis cs:{108.343},{16.159}) {}; % Gem,  5.06 
\node[pin={[pin distance=-0.4\onedegree,Flaamsted]00:{52}}] at (axis cs:{108.675},{24.885}) {}; % Gem,  5.84 
\node[pin={[pin distance=-0.4\onedegree,Flaamsted]00:{53}}] at (axis cs:{108.988},{27.897}) {}; % Gem,  5.75 
\node[pin={[pin distance=-0.4\onedegree,Flaamsted]180:{ 5}}] at (axis cs:{92.885},{24.420}) {}; % Gem,  5.83 
\node[pin={[pin distance=-0.4\onedegree,Bayer]00:{$\boldsymbol\alpha$}}] at (axis cs:{83.183},{-17.822}) {}; % Lep,  2.59 
\node[pin={[pin distance=-0.4\onedegree,Bayer]00:{$\boldsymbol\beta$}}] at (axis cs:{82.061},{-20.759}) {}; % Lep,  2.84 
\node[pin={[pin distance=-0.4\onedegree,Bayer]00:{$\boldsymbol\delta$}}] at (axis cs:{87.830},{-20.879}) {}; % Lep,  3.78 
\node[pin={[pin distance=-0.4\onedegree,Bayer]00:{$\boldsymbol\epsilon$}}] at (axis cs:{76.365},{-22.371}) {}; % Lep,  3.18 
\node[pin={[pin distance=-0.4\onedegree,Bayer]00:{$\boldsymbol\eta$}}] at (axis cs:{89.101},{-14.168}) {}; % Lep,  3.72 
\node[pin={[pin distance=-0.4\onedegree,Bayer]-90:{$\boldsymbol\gamma$}}] at (axis cs:{86.116},{-22.448}) {}; % Lep,  3.60 
\node[pin={[pin distance=-0.4\onedegree,Bayer]00:{$\boldsymbol\iota$}}] at (axis cs:{78.075},{-11.869}) {}; % Lep,  4.47 
\node[pin={[pin distance=-0.4\onedegree,Bayer]00:{$\boldsymbol\kappa$}}] at (axis cs:{78.308},{-12.941}) {}; % Lep,  4.43 
\node[pin={[pin distance=-0.4\onedegree,Bayer]180:{$\boldsymbol\lambda$}}] at (axis cs:{79.894},{-13.177}) {}; % Lep,  4.28 
\node[pin={[pin distance=-0.4\onedegree,Bayer]00:{$\boldsymbol\mu$}}] at (axis cs:{78.233},{-16.205}) {}; % Lep,  3.29 
\node[pin={[pin distance=-0.4\onedegree,Bayer]00:{$\boldsymbol\nu$}}] at (axis cs:{79.996},{-12.315}) {}; % Lep,  5.29 
\node[pin={[pin distance=-0.4\onedegree,Bayer]00:{$\boldsymbol\vartheta$}}] at (axis cs:{91.539},{-14.935}) {}; % Lep,  4.67 
\node[pin={[pin distance=-0.4\onedegree,Bayer]00:{$\boldsymbol\zeta$}}] at (axis cs:{86.739},{-14.822}) {}; % Lep,  3.55 
\node[pin={[pin distance=-0.4\onedegree,Flaamsted]180:{10}}] at (axis cs:{82.782},{-20.863}) {}; % Lep,  5.54 
\node[pin={[pin distance=-0.4\onedegree,Flaamsted]00:{12}}] at (axis cs:{85.558},{-22.374}) {}; % Lep,  5.88 
\node[pin={[pin distance=-0.4\onedegree,Flaamsted]00:{17}}] at (axis cs:{91.246},{-16.484}) {}; % Lep,  4.97 
\node[pin={[pin distance=-0.4\onedegree,Flaamsted]00:{19}}] at (axis cs:{91.923},{-19.166}) {}; % Lep,  5.28 
\node[pin={[pin distance=-0.4\onedegree,Flaamsted]00:{ 1}}] at (axis cs:{75.687},{-22.795}) {}; % Lep,  5.74 
\node[pin={[pin distance=-0.4\onedegree,Flaamsted]00:{ 8}}] at (axis cs:{80.876},{-13.927}) {}; % Lep,  5.24 
\node[pin={[pin distance=-0.4\onedegree,Bayer]00:{$\boldsymbol\beta$}}] at (axis cs:{97.204},{-7.033}) {}; % Mon,  4.64 
\node[pin={[pin distance=-0.4\onedegree,Bayer]00:{$\boldsymbol\delta$}}] at (axis cs:{107.966},{-0.493}) {}; % Mon,  4.15 
\node[pin={[pin distance=-0.4\onedegree,Bayer]180:{$\boldsymbol\epsilon$}}] at (axis cs:{95.942},{4.593}) {}; % Mon,  4.41 
\node[pin={[pin distance=-0.4\onedegree,Bayer]-90:{$\boldsymbol\gamma$}}] at (axis cs:{93.714},{-6.275}) {}; % Mon,  3.98 
\node[pin={[pin distance=-0.4\onedegree,Flaamsted]-90:{10}}] at (axis cs:{96.990},{-4.762}) {}; % Mon,  5.05 
\node[pin={[pin distance=-0.4\onedegree,Flaamsted]00:{12}}] at (axis cs:{98.080},{4.856}) {}; % Mon,  5.85 
\node[pin={[pin distance=-0.4\onedegree,Flaamsted]00:{13}}] at (axis cs:{98.226},{7.333}) {}; % Mon,  4.51 
\node[pin={[pin distance=-0.4\onedegree,Flaamsted]-90:{15}}] at (axis cs:{100.244},{9.896}) {}; % Mon,  4.67 
\node[pin={[pin distance=-0.4\onedegree,Flaamsted]0:{S}}] at (axis cs:{100.244},{9.896}) {}; % Mon,  4.67 
\node[pin={[pin distance=-0.4\onedegree,Flaamsted]00:{16}}] at (axis cs:{101.635},{8.587}) {}; % Mon,  5.91 
\node[pin={[pin distance=-0.4\onedegree,Flaamsted]00:{17}}] at (axis cs:{101.833},{8.037}) {}; % Mon,  4.77 
\node[pin={[pin distance=-0.4\onedegree,Flaamsted]00:{18}}] at (axis cs:{101.965},{2.412}) {}; % Mon,  4.47 
\node[pin={[pin distance=-0.4\onedegree,Flaamsted]0:{19}}] at (axis cs:{105.728},{-4.239}) {}; % Mon,  4.98 
\node[pin={[pin distance=-0.4\onedegree,Flaamsted]00:{20}}] at (axis cs:{107.557},{-4.237}) {}; % Mon,  4.92 
\node[pin={[pin distance=-0.4\onedegree,Flaamsted]90:{21}}] at (axis cs:{107.848},{-0.302}) {}; % Mon,  5.43 
\node[pin={[pin distance=-0.4\onedegree,Flaamsted]00:{ 2}}] at (axis cs:{89.768},{-9.558}) {}; % Mon,  5.04 
\node[pin={[pin distance=-0.4\onedegree,Flaamsted]00:{ 3}}] at (axis cs:{90.460},{-10.598}) {}; % Mon,  5.00 
\node[pin={[pin distance=-0.4\onedegree,Flaamsted]00:{ 7}}] at (axis cs:{94.928},{-7.823}) {}; % Mon,  5.26 

\node[pin={[pin distance=-0.2\onedegree,Bayer]00:{$\boldsymbol\alpha$}}] at (axis cs:{88.793},{7.407}) {}; % Ori,  0.57 
\node[pin={[pin distance=-0.2\onedegree,Bayer]00:{$\boldsymbol\beta$}}] at (axis cs:{78.634},{-8.202}) {}; % Ori,  0.28 
\node[pin={[pin distance=-0.2\onedegree,Bayer]00:{$\boldsymbol\delta$}}] at (axis cs:{83.002},{-0.299}) {}; % Ori,  2.23 
\node[pin={[pin distance=-0.8\onedegree,Bayer]45:{$\boldsymbol\epsilon$}}] at (axis cs:{84.053},{-1.202}) {}; % Ori,  1.72 
\node[pin={[pin distance=-0.2\onedegree,Bayer]00:{$\boldsymbol\gamma$}}] at (axis cs:{81.283},{6.349}) {}; % Ori,  1.66 
\node[pin={[pin distance=-0.2\onedegree,Bayer]00:{$\boldsymbol\kappa$}}] at (axis cs:{86.939},{-9.670}) {}; % Ori,  2.06 
\node[pin={[pin distance=-0.2\onedegree,Bayer]00:{$\boldsymbol\zeta$}}] at (axis cs:{85.190},{-1.942}) {}; % Ori,  1.74 
\node[pin={[pin distance=-0.4\onedegree,Bayer]90:{$\boldsymbol\chi^1$}}] at (axis cs:{88.596},{20.276}) {}; % Ori,  4.40 
\node[pin={[pin distance=-0.4\onedegree,Bayer]00:{$\boldsymbol\chi^2$}}] at (axis cs:{90.980},{20.138}) {}; % Ori,  4.64 
\node[pin={[pin distance=-0.4\onedegree,Bayer]00:{$\boldsymbol\eta$}}] at (axis cs:{81.119},{-2.397}) {}; % Ori,  3.39 
\node[pin={[pin distance=-0.2\onedegree,Bayer]-90:{$\boldsymbol\iota$}}] at (axis cs:{83.858},{-5.910}) {}; % Ori,  2.78 
\node[pin={[pin distance=-0.4\onedegree,Bayer]00:{$\boldsymbol\lambda$}}] at (axis cs:{83.784},{9.934}) {}; % Ori,  3.39 
\node[pin={[pin distance=-0.4\onedegree,Bayer]00:{$\boldsymbol\lambda$}}] at (axis cs:{83.785},{9.935}) {}; % Ori,  5.61 
\node[pin={[pin distance=-0.4\onedegree,Bayer]00:{$\boldsymbol\mu$}}] at (axis cs:{90.596},{9.647}) {}; % Ori,  4.13 
\node[pin={[pin distance=-0.4\onedegree,Bayer]00:{$\boldsymbol\nu$}}] at (axis cs:{91.893},{14.768}) {}; % Ori,  4.42 
\node[pin={[pin distance=-0.4\onedegree,Bayer]00:{$\boldsymbol\omega$}}] at (axis cs:{84.796},{4.121}) {}; % Ori,  4.54 
\node[pin={[pin distance=-0.4\onedegree,Bayer]90:{$\boldsymbol\omicron^1$}}] at (axis cs:{73.133},{14.251}) {}; % Ori,  4.74 
\node[pin={[pin distance=-0.4\onedegree,Bayer]00:{$\boldsymbol\omicron^2$}}] at (axis cs:{74.093},{13.514}) {}; % Ori,  4.08 
\node[pin={[pin distance=-0.4\onedegree,Bayer]180:{$\boldsymbol\pi^1$}}] at (axis cs:{73.724},{10.151}) {}; % Ori,  4.66 
\node[pin={[pin distance=-0.4\onedegree,Bayer]00:{$\boldsymbol\pi^2$}}] at (axis cs:{72.653},{8.900}) {}; % Ori,  4.36 
\node[pin={[pin distance=-0.4\onedegree,Bayer]00:{$\boldsymbol\pi^3$}}] at (axis cs:{72.460},{6.961}) {}; % Ori,  3.19 
\node[pin={[pin distance=-0.4\onedegree,Bayer]00:{$\boldsymbol\pi^4$}}] at (axis cs:{72.802},{5.605}) {}; % Ori,  3.68 
\node[pin={[pin distance=-0.4\onedegree,Bayer]00:{$\boldsymbol\pi^5$}}] at (axis cs:{73.563},{2.441}) {}; % Ori,  3.71 
\node[pin={[pin distance=-0.4\onedegree,Bayer]00:{$\boldsymbol\pi^6$}}] at (axis cs:{74.637},{1.714}) {}; % Ori,  4.46 
\node[pin={[pin distance=-0.4\onedegree,Bayer]00:{$\boldsymbol\rho$}}] at (axis cs:{78.323},{2.861}) {}; % Ori,  4.48 
\node[pin={[pin distance=-0.4\onedegree,Bayer]00:{$\boldsymbol\psi^1$}}] at (axis cs:{81.187},{1.846}) {}; % Ori,  4.89 
\node[pin={[pin distance=-0.4\onedegree,Bayer]00:{$\boldsymbol\psi^2$}}] at (axis cs:{81.709},{3.095}) {}; % Ori,  4.61 
\node[pin={[pin distance=-0.4\onedegree,Bayer]00:{$\boldsymbol\sigma$}}] at (axis cs:{84.687},{-2.600}) {}; % Ori,  3.80 
\node[pin={[pin distance=-0.4\onedegree,Bayer]00:{$\boldsymbol\tau$}}] at (axis cs:{79.402},{-6.844}) {}; % Ori,  3.60 
\node[pin={[pin distance=-0.4\onedegree,Bayer]-90:{$\boldsymbol\upsilon$}}] at (axis cs:{82.983},{-7.302}) {}; % Ori,  4.61 
\node[pin={[pin distance=-0.4\onedegree,Bayer]-90:{$\boldsymbol\varphi^1$}}] at (axis cs:{83.705},{9.489}) {}; % Ori,  4.41 
\node[pin={[pin distance=-0.4\onedegree,Bayer]180:{$\boldsymbol\varphi^2$}}] at (axis cs:{84.227},{9.291}) {}; % Ori,  4.09 
\node[pin={[pin distance=-0.4\onedegree,Bayer]00:{$\boldsymbol\vartheta$}}] at (axis cs:{83.819},{-5.390}) {}; % Ori,  5.13 
\node[pin={[pin distance=-0.4\onedegree,Bayer]90:{$\boldsymbol\xi$}}] at (axis cs:{92.985},{14.209}) {}; % Ori,  4.46 
\node[pin={[pin distance=-0.4\onedegree,Flaamsted]00:{11}}] at (axis cs:{76.142},{15.404}) {}; % Ori,  4.67 
\node[pin={[pin distance=-0.4\onedegree,Flaamsted]00:{14}}] at (axis cs:{76.970},{8.498}) {}; % Ori,  5.35 
\node[pin={[pin distance=-0.4\onedegree,Flaamsted]00:{15}}] at (axis cs:{77.425},{15.597}) {}; % Ori,  4.82 
\node[pin={[pin distance=-0.4\onedegree,Flaamsted]00:{16}}] at (axis cs:{77.332},{9.829}) {}; % Ori,  5.43 
\node[pin={[pin distance=-0.4\onedegree,Flaamsted]00:{18}}] at (axis cs:{79.017},{11.341}) {}; % Ori,  5.53 
\node[pin={[pin distance=-0.4\onedegree,Flaamsted]00:{21}}] at (axis cs:{79.797},{2.596}) {}; % Ori,  5.34 
\node[pin={[pin distance=-0.4\onedegree,Flaamsted]00:{22}}] at (axis cs:{80.441},{-0.382}) {}; % Ori,  4.72 
\node[pin={[pin distance=-0.4\onedegree,Flaamsted]90:{23}}] at (axis cs:{80.708},{3.544}) {}; % Ori,  4.98 
\node[pin={[pin distance=-0.4\onedegree,Flaamsted]00:{27}}] at (axis cs:{81.120},{-0.891}) {}; % Ori,  5.07 
\node[pin={[pin distance=-0.4\onedegree,Flaamsted]00:{29}}] at (axis cs:{80.987},{-7.808}) {}; % Ori,  4.14 
\node[pin={[pin distance=-0.4\onedegree,Flaamsted]00:{31}}] at (axis cs:{82.433},{-1.092}) {}; % Ori,  4.70 
\node[pin={[pin distance=-0.4\onedegree,Flaamsted]00:{32}}] at (axis cs:{82.696},{5.948}) {}; % Ori,  4.23 
\node[pin={[pin distance=-0.4\onedegree,Flaamsted]00:{33}}] at (axis cs:{82.811},{3.292}) {}; % Ori,  5.46 
\node[pin={[pin distance=-0.4\onedegree,Flaamsted]00:{35}}] at (axis cs:{83.476},{14.305}) {}; % Ori,  5.60 
\node[pin={[pin distance=-0.4\onedegree,Flaamsted]00:{38}}] at (axis cs:{83.570},{3.767}) {}; % Ori,  5.33 
\node[pin={[pin distance=-0.4\onedegree,Flaamsted]00:{42}}] at (axis cs:{83.847},{-4.838}) {}; % Ori,  4.60 
\node[pin={[pin distance=-1.0\onedegree,Flaamsted]135:{45}}] at (axis cs:{83.915},{-4.856}) {}; % Ori,  5.24 
\node[pin={[pin distance=-0.4\onedegree,Flaamsted]00:{49}}] at (axis cs:{84.721},{-7.213}) {}; % Ori,  4.79 
\node[pin={[pin distance=-0.4\onedegree,Flaamsted]00:{51}}] at (axis cs:{85.619},{1.474}) {}; % Ori,  4.90 
\node[pin={[pin distance=-0.4\onedegree,Flaamsted]00:{52}}] at (axis cs:{87.001},{6.454}) {}; % Ori,  5.30 
\node[pin={[pin distance=-0.4\onedegree,Flaamsted]00:{55}}] at (axis cs:{87.842},{-7.518}) {}; % Ori,  5.35 
\node[pin={[pin distance=-0.4\onedegree,Flaamsted]-90:{56}}] at (axis cs:{88.110},{1.855}) {}; % Ori,  4.75 
\node[pin={[pin distance=-0.4\onedegree,Flaamsted]-90:{57}}] at (axis cs:{88.736},{19.749}) {}; % Ori,  5.91 
\node[pin={[pin distance=-0.4\onedegree,Flaamsted]00:{59}}] at (axis cs:{89.602},{1.837}) {}; % Ori,  5.90 
\node[pin={[pin distance=-0.4\onedegree,Flaamsted]90:{ 5}}] at (axis cs:{73.345},{2.508}) {}; % Ori,  5.33 
\node[pin={[pin distance=-0.4\onedegree,Flaamsted]00:{60}}] at (axis cs:{89.707},{0.553}) {}; % Ori,  5.22 
\node[pin={[pin distance=-0.4\onedegree,Flaamsted]00:{63}}] at (axis cs:{91.242},{5.420}) {}; % Ori,  5.66 
\node[pin={[pin distance=-0.4\onedegree,Flaamsted]-90:{64}}] at (axis cs:{90.864},{19.690}) {}; % Ori,  5.15 
\node[pin={[pin distance=-0.4\onedegree,Flaamsted]00:{66}}] at (axis cs:{91.243},{4.159}) {}; % Ori,  5.63 
\node[pin={[pin distance=-0.4\onedegree,Flaamsted]180:{68}}] at (axis cs:{93.006},{19.790}) {}; % Ori,  5.76 
\node[pin={[pin distance=-0.4\onedegree,Flaamsted]00:{69}}] at (axis cs:{93.014},{16.130}) {}; % Ori,  4.96 
\node[pin={[pin distance=-0.4\onedegree,Flaamsted]00:{ 6}}] at (axis cs:{73.695},{11.426}) {}; % Ori,  5.19 
\node[pin={[pin distance=-0.4\onedegree,Flaamsted]190:{71}}] at (axis cs:{93.712},{19.156}) {}; % Ori,  5.21 
\node[pin={[pin distance=-0.4\onedegree,Flaamsted]-90:{72}}] at (axis cs:{93.855},{16.143}) {}; % Ori,  5.34 
\node[pin={[pin distance=-0.4\onedegree,Flaamsted]180:{73}}] at (axis cs:{93.937},{12.551}) {}; % Ori,  5.48 
\node[pin={[pin distance=-0.4\onedegree,Flaamsted]00:{74}}] at (axis cs:{94.111},{12.272}) {}; % Ori,  5.04 
\node[pin={[pin distance=-0.4\onedegree,Flaamsted]00:{75}}] at (axis cs:{94.278},{9.942}) {}; % Ori,  5.40 

\node[pin={[pin distance=-0.4\onedegree,Bayer]00:{$\boldsymbol\chi$}}] at (axis cs:{3.651},{20.207}) {}; % Peg,  4.80 
\node[pin={[pin distance=-0.4\onedegree,Bayer]00:{$\boldsymbol\gamma$}}] at (axis cs:{3.309},{15.184}) {}; % Peg,  2.83 
\node[pin={[pin distance=-0.4\onedegree,Flaamsted]-90:{85}}] at (axis cs:{0.542},{27.082}) {}; % Peg,  5.80 
\node[pin={[pin distance=-0.4\onedegree,Flaamsted]00:{86}}] at (axis cs:{1.425},{13.396}) {}; % Peg,  5.55 
\node[pin={[pin distance=-0.4\onedegree,Flaamsted]00:{87}}] at (axis cs:{2.260},{18.212}) {}; % Peg,  5.56 
\node[pin={[pin distance=-0.4\onedegree,Bayer]00:{$\boldsymbol\omega$}}] at (axis cs:{47.822},{39.611}) {}; % Per,  4.61 
\node[pin={[pin distance=-0.4\onedegree,Bayer]00:{$\boldsymbol\omicron$}}] at (axis cs:{56.080},{32.288}) {}; % Per,  3.86 
\node[pin={[pin distance=-0.4\onedegree,Bayer]00:{$\boldsymbol\pi$}}] at (axis cs:{44.690},{39.663}) {}; % Per,  4.69 
\node[pin={[pin distance=-0.4\onedegree,Bayer]00:{$\boldsymbol\rho$}}] at (axis cs:{46.294},{38.840}) {}; % Per,  3.41 
\node[pin={[pin distance=-0.4\onedegree,Bayer]00:{$\boldsymbol\xi$}}] at (axis cs:{59.741},{35.791}) {}; % Per,  4.05 
\node[pin={[pin distance=-0.4\onedegree,Bayer]00:{$\boldsymbol\zeta$}}] at (axis cs:{58.533},{31.884}) {}; % Per,  2.88 
\node[pin={[pin distance=-0.4\onedegree,Flaamsted]00:{16}}] at (axis cs:{42.646},{38.319}) {}; % Per,  4.22 
\node[pin={[pin distance=-0.4\onedegree,Flaamsted]00:{17}}] at (axis cs:{42.878},{35.060}) {}; % Per,  4.55 
\node[pin={[pin distance=-0.4\onedegree,Flaamsted]-90:{20}}] at (axis cs:{43.428},{38.337}) {}; % Per,  5.35 
\node[pin={[pin distance=-0.4\onedegree,Flaamsted]00:{21}}] at (axis cs:{44.322},{31.934}) {}; % Per,  5.10 
\node[pin={[pin distance=-0.4\onedegree,Flaamsted]00:{24}}] at (axis cs:{44.765},{35.183}) {}; % Per,  4.94 
\node[pin={[pin distance=-0.4\onedegree,Flaamsted]00:{40}}] at (axis cs:{55.594},{33.965}) {}; % Per,  4.98 
\node[pin={[pin distance=-0.4\onedegree,Flaamsted]180:{42}}] at (axis cs:{57.386},{33.091}) {}; % Per,  5.15 
\node[pin={[pin distance=-0.4\onedegree,Flaamsted]00:{50}}] at (axis cs:{62.153},{38.040}) {}; % Per,  5.52 
\node[pin={[pin distance=-0.4\onedegree,Flaamsted]00:{54}}] at (axis cs:{65.103},{34.567}) {}; % Per,  4.93 
\node[pin={[pin distance=-0.4\onedegree,Flaamsted]90:{55}}] at (axis cs:{66.121},{34.131}) {}; % Per,  5.73 
\node[pin={[pin distance=-0.4\onedegree,Flaamsted]00:{56}}] at (axis cs:{66.156},{33.960}) {}; % Per,  5.79 

\node[pin={[pin distance=-0.4\onedegree,Bayer]00:{$\boldsymbol\alpha$}}] at (axis cs:{30.511},{2.764}) {}; % Psc,  3.82 
\node[pin={[pin distance=-0.4\onedegree,Bayer]00:{$\boldsymbol\alpha$}}] at (axis cs:{30.512},{2.764}) {}; % Psc,  3.82 
\node[pin={[pin distance=-0.4\onedegree,Bayer]180:{$\boldsymbol\chi$}}] at (axis cs:{17.863},{21.035}) {}; % Psc,  4.66 
\node[pin={[pin distance=-0.4\onedegree,Bayer]90:{$\boldsymbol\delta$}}] at (axis cs:{12.171},{7.585}) {}; % Psc,  4.42 
\node[pin={[pin distance=-0.4\onedegree,Bayer]00:{$\boldsymbol\epsilon$}}] at (axis cs:{15.736},{7.890}) {}; % Psc,  4.28 
\node[pin={[pin distance=-0.4\onedegree,Bayer]00:{$\boldsymbol\eta$}}] at (axis cs:{22.871},{15.346}) {}; % Psc,  3.63 
\node[pin={[pin distance=-0.4\onedegree,Bayer]00:{$\boldsymbol\mu$}}] at (axis cs:{22.546},{6.144}) {}; % Psc,  4.84 
\node[pin={[pin distance=-0.4\onedegree,Bayer]00:{$\boldsymbol\nu$}}] at (axis cs:{25.358},{5.488}) {}; % Psc,  4.44 
\node[pin={[pin distance=-0.4\onedegree,Bayer]00:{$\boldsymbol\omicron$}}] at (axis cs:{26.348},{9.158}) {}; % Psc,  4.27 
\node[pin={[pin distance=-0.4\onedegree,Bayer]00:{$\boldsymbol\pi$}}] at (axis cs:{24.275},{12.141}) {}; % Psc,  5.55 
\node[pin={[pin distance=-0.4\onedegree,Bayer]00:{$\boldsymbol\rho$}}] at (axis cs:{21.564},{19.172}) {}; % Psc,  5.35 
\node[pin={[pin distance=-0.4\onedegree,Bayer]90:{$\boldsymbol\psi^1$}}] at (axis cs:{16.424},{21.465}) {}; % Psc,  5.54 
\node[pin={[pin distance=-0.4\onedegree,Bayer]00:{$\boldsymbol\psi^2$}}] at (axis cs:{16.988},{20.739}) {}; % Psc,  5.57 
\node[pin={[pin distance=-0.4\onedegree,Bayer]00:{$\boldsymbol\psi^3$}}] at (axis cs:{17.455},{19.658}) {}; % Psc,  5.56 
\node[pin={[pin distance=-0.4\onedegree,Bayer]00:{$\boldsymbol\sigma$}}] at (axis cs:{15.705},{31.804}) {}; % Psc,  5.50 
\node[pin={[pin distance=-0.4\onedegree,Bayer]00:{$\boldsymbol\tau$}}] at (axis cs:{17.915},{30.090}) {}; % Psc,  4.52 
\node[pin={[pin distance=-0.4\onedegree,Bayer]00:{$\boldsymbol\upsilon$}}] at (axis cs:{19.867},{27.264}) {}; % Psc,  4.75 
\node[pin={[pin distance=-0.4\onedegree,Bayer]00:{$\boldsymbol\varphi$}}] at (axis cs:{18.437},{24.584}) {}; % Psc,  4.67 
\node[pin={[pin distance=-0.4\onedegree,Bayer]0:{$\boldsymbol\xi$}}] at (axis cs:{28.389},{3.187}) {}; % Psc,  4.62 
\node[pin={[pin distance=-0.4\onedegree,Bayer]180:{$\boldsymbol\zeta$}}] at (axis cs:{18.433},{7.575}) {}; % Psc,  5.20 
\node[pin={[pin distance=-0.4\onedegree,Flaamsted]00:{105}}] at (axis cs:{24.920},{16.406}) {}; % Psc,  5.99 
\node[pin={[pin distance=-0.4\onedegree,Flaamsted]00:{107}}] at (axis cs:{25.624},{20.268}) {}; % Psc,  5.24 
\node[pin={[pin distance=-0.4\onedegree,Flaamsted]90:{112}}] at (axis cs:{30.038},{3.097}) {}; % Psc,  5.88 
\node[pin={[pin distance=-0.4\onedegree,Flaamsted]90:{29}}] at (axis cs:{0.456},{-3.027}) {}; % Psc,  5.13 
\node[pin={[pin distance=-0.4\onedegree,Flaamsted]90:{30}}] at (axis cs:{0.490},{-6.014}) {}; % Psc,  4.40 
\node[pin={[pin distance=-0.4\onedegree,Flaamsted]180:{32}}] at (axis cs:{0.624},{8.485}) {}; % Psc,  5.70 
\node[pin={[pin distance=-0.4\onedegree,Flaamsted]180:{33}}] at (axis cs:{1.334},{-5.707}) {}; % Psc,  4.62 
\node[pin={[pin distance=-0.4\onedegree,Flaamsted]180:{34}}] at (axis cs:{2.509},{11.146}) {}; % Psc,  5.54 
\node[pin={[pin distance=-0.4\onedegree,Flaamsted]180:{41}}] at (axis cs:{5.149},{8.190}) {}; % Psc,  5.38 
\node[pin={[pin distance=-0.4\onedegree,Flaamsted]00:{44}}] at (axis cs:{6.351},{1.940}) {}; % Psc,  5.77 
\node[pin={[pin distance=-0.4\onedegree,Flaamsted]00:{47}}] at (axis cs:{7.012},{17.893}) {}; % Psc,  5.06 
\node[pin={[pin distance=-0.4\onedegree,Flaamsted]00:{51}}] at (axis cs:{8.099},{6.955}) {}; % Psc,  5.67 
\node[pin={[pin distance=-0.4\onedegree,Flaamsted]00:{52}}] at (axis cs:{8.148},{20.294}) {}; % Psc,  5.37 
\node[pin={[pin distance=-0.4\onedegree,Flaamsted]00:{53}}] at (axis cs:{9.197},{15.231}) {}; % Psc,  5.88 
\node[pin={[pin distance=-0.4\onedegree,Flaamsted]-90:{54}}] at (axis cs:{9.841},{21.250}) {}; % Psc,  5.88 
\node[pin={[pin distance=-0.4\onedegree,Flaamsted]180:{55}}] at (axis cs:{9.982},{21.438}) {}; % Psc,  5.43 
\node[pin={[pin distance=-0.4\onedegree,Flaamsted]00:{57}}] at (axis cs:{11.637},{15.475}) {}; % Psc,  5.41 
\node[pin={[pin distance=-0.4\onedegree,Flaamsted]00:{58}}] at (axis cs:{11.756},{11.974}) {}; % Psc,  5.51 
\node[pin={[pin distance=-0.4\onedegree,Flaamsted]00:{60}}] at (axis cs:{11.848},{6.741}) {}; % Psc,  5.98 
\node[pin={[pin distance=-0.4\onedegree,Flaamsted]00:{62}}] at (axis cs:{12.073},{7.300}) {}; % Psc,  5.92 
\node[pin={[pin distance=-0.4\onedegree,Flaamsted]00:{64}}] at (axis cs:{12.245},{16.941}) {}; % Psc,  5.07 
\node[pin={[pin distance=-0.4\onedegree,Flaamsted]00:{65}}] at (axis cs:{12.470},{27.711}) {}; % Psc,  5.55 
\node[pin={[pin distance=-0.4\onedegree,Flaamsted]00:{65}}] at (axis cs:{12.472},{27.710}) {}; % Psc,  5.55 
\node[pin={[pin distance=-0.4\onedegree,Flaamsted]00:{66}}] at (axis cs:{13.647},{19.188}) {}; % Psc,  5.80 
\node[pin={[pin distance=-0.4\onedegree,Flaamsted]00:{68}}] at (axis cs:{14.459},{28.992}) {}; % Psc,  5.43 
\node[pin={[pin distance=-0.4\onedegree,Flaamsted]00:{72}}] at (axis cs:{16.272},{14.946}) {}; % Psc,  5.64 
\node[pin={[pin distance=-0.4\onedegree,Flaamsted]180:{80}}] at (axis cs:{17.092},{5.650}) {}; % Psc,  5.51 
\node[pin={[pin distance=-0.4\onedegree,Flaamsted]00:{82}}] at (axis cs:{17.778},{31.425}) {}; % Psc,  5.16 
\node[pin={[pin distance=-0.4\onedegree,Flaamsted]00:{87}}] at (axis cs:{18.532},{16.133}) {}; % Psc,  5.97 
\node[pin={[pin distance=-0.4\onedegree,Flaamsted]00:{89}}] at (axis cs:{19.450},{3.614}) {}; % Psc,  5.15 
\node[pin={[pin distance=-0.4\onedegree,Flaamsted]00:{91}}] at (axis cs:{20.281},{28.738}) {}; % Psc,  5.22 
\node[pin={[pin distance=-0.4\onedegree,Flaamsted]180:{94}}] at (axis cs:{21.674},{19.240}) {}; % Psc,  5.50 

\node[pin={[pin distance=-0.4\onedegree,Bayer]-90:{$\boldsymbol\pi$}}] at (axis cs:{109.286},{-37.097}) {}; % Pup,  2.71 
\node[pin={[pin distance=-0.4\onedegree,Bayer]00:{$\boldsymbol\alpha$}}] at (axis cs:{14.652},{-29.357}) {}; % Scl,  4.31 
\node[pin={[pin distance=-0.4\onedegree,Bayer]00:{$\boldsymbol\epsilon$}}] at (axis cs:{26.411},{-25.053}) {}; % Scl,  5.33 
\node[pin={[pin distance=-0.4\onedegree,Bayer]00:{$\boldsymbol\eta$}}] at (axis cs:{6.982},{-33.007}) {}; % Scl,  4.87 
\node[pin={[pin distance=-0.4\onedegree,Bayer]00:{$\boldsymbol\iota$}}] at (axis cs:{5.380},{-28.981}) {}; % Scl,  5.17 
\node[pin={[pin distance=-0.4\onedegree,Bayer]00:{$\boldsymbol\kappa^1$}}] at (axis cs:{2.338},{-27.988}) {}; % Scl,  5.44 
\node[pin={[pin distance=-0.4\onedegree,Bayer]90:{$\boldsymbol\kappa^2$}}] at (axis cs:{2.893},{-27.799}) {}; % Scl,  5.41 
\node[pin={[pin distance=-0.4\onedegree,Bayer]180:{$\boldsymbol\lambda^2$}}] at (axis cs:{11.050},{-38.422}) {}; % Scl,  5.90 
\node[pin={[pin distance=-0.4\onedegree,Bayer]00:{$\boldsymbol\pi$}}] at (axis cs:{25.536},{-32.327}) {}; % Scl,  5.26 
\node[pin={[pin distance=-0.4\onedegree,Bayer]00:{$\boldsymbol\sigma$}}] at (axis cs:{15.610},{-31.552}) {}; % Scl,  5.51 
\node[pin={[pin distance=-0.4\onedegree,Bayer]00:{$\boldsymbol\tau$}}] at (axis cs:{24.035},{-29.907}) {}; % Scl,  5.70 
\node[pin={[pin distance=-0.4\onedegree,Bayer]00:{$\boldsymbol\vartheta$}}] at (axis cs:{2.933},{-35.133}) {}; % Scl,  5.24 
\node[pin={[pin distance=-0.4\onedegree,Bayer]00:{$\boldsymbol\xi$}}] at (axis cs:{15.326},{-38.916}) {}; % Scl,  5.59 
\node[pin={[pin distance=-0.4\onedegree,Bayer]180:{$\boldsymbol\zeta$}}] at (axis cs:{0.583},{-29.720}) {}; % Scl,  5.04 

\node[pin={[pin distance=-0.4\onedegree,Bayer]45:{$\boldsymbol\alpha$}}] at (axis cs:{68.980},{16.509}) {}; % Tau,  0.99 
\node[pin={[pin distance=-0.2\onedegree,Bayer]-90:{$\boldsymbol\beta$}}] at (axis cs:{81.573},{28.607}) {}; % Tau,  1.68 
\node[pin={[pin distance=-0.4\onedegree,Bayer]00:{$\boldsymbol\chi$}}] at (axis cs:{65.646},{25.629}) {}; % Tau,  5.38 
\node[pin={[pin distance=-0.4\onedegree,Bayer]00:{$\boldsymbol\delta^{1,2,3}$}}] at (axis cs:{65.734},{17.542}) {}; % Tau,  3.76 
%\node[pin={[pin distance=-0.8\onedegree,Bayer]180:{$\boldsymbol\delta^2$}}] at (axis cs:{66.024},{17.444}) {}; % Tau,  4.80 
%\node[pin={[pin distance=-0.8\onedegree,Bayer]45:{$\boldsymbol\delta^3$}}] at (axis cs:{66.372},{17.928}) {}; % Tau,  4.31 
\node[pin={[pin distance=-0.4\onedegree,Bayer]180:{$\boldsymbol\epsilon$}}] at (axis cs:{67.154},{19.180}) {}; % Tau,  3.54 
\node[pin={[pin distance=-0.4\onedegree,Bayer]90:{$\boldsymbol\eta$}}] at (axis cs:{56.871},{24.105}) {}; % Tau,  2.87 
\node[pin={[pin distance=-0.4\onedegree,Bayer]00:{$\boldsymbol\gamma$}}] at (axis cs:{64.948},{15.627}) {}; % Tau,  3.65 
\node[pin={[pin distance=-0.4\onedegree,Bayer]00:{$\boldsymbol\iota$}}] at (axis cs:{75.774},{21.590}) {}; % Tau,  4.62 
\node[pin={[pin distance=-0.4\onedegree,Bayer]00:{$\boldsymbol\kappa$}}] at (axis cs:{66.342},{22.294}) {}; % Tau,  4.21 
%\node[pin={[pin distance=-0.4\onedegree,Bayer]00:{$\boldsymbol\kappa^2$}}] at (axis cs:{66.354},{22.200}) {}; % Tau,  5.27 
\node[pin={[pin distance=-0.4\onedegree,Bayer]00:{$\boldsymbol\lambda$}}] at (axis cs:{60.170},{12.490}) {}; % Tau,  3.42 
\node[pin={[pin distance=-0.4\onedegree,Bayer]-90:{$\boldsymbol\mu$}}] at (axis cs:{63.884},{8.892}) {}; % Tau,  4.29 
\node[pin={[pin distance=-0.4\onedegree,Bayer]00:{$\boldsymbol\nu$}}] at (axis cs:{60.789},{5.989}) {}; % Tau,  3.90 
\node[pin={[pin distance=-0.4\onedegree,Bayer]00:{$\boldsymbol\omega^1$}}] at (axis cs:{62.292},{19.609}) {}; % Tau,  5.51 
\node[pin={[pin distance=-0.4\onedegree,Bayer]00:{$\boldsymbol\omega^2$}}] at (axis cs:{64.315},{20.578}) {}; % Tau,  4.92 
\node[pin={[pin distance=-0.4\onedegree,Bayer]180:{$\boldsymbol\omicron$}}] at (axis cs:{51.203},{9.029}) {}; % Tau,  3.61 
\node[pin={[pin distance=-0.4\onedegree,Bayer]00:{$\boldsymbol\pi$}}] at (axis cs:{66.652},{14.714}) {}; % Tau,  4.69 
\node[pin={[pin distance=-0.4\onedegree,Bayer]-90:{$\boldsymbol\rho$}}] at (axis cs:{68.462},{14.844}) {}; % Tau,  4.65 
\node[pin={[pin distance=-0.4\onedegree,Bayer]180:{$\boldsymbol\psi$}}] at (axis cs:{61.752},{29.001}) {}; % Tau,  5.22 
\node[pin={[pin distance=-0.4\onedegree,Bayer]-90:{$\boldsymbol\sigma$}}] at (axis cs:{69.788},{15.800}) {}; % Tau,  5.09 
%\node[pin={[pin distance=-0.4\onedegree,Bayer]180:{$\boldsymbol\sigma^2$}}] at (axis cs:{69.819},{15.918}) {}; % Tau,  4.67 
\node[pin={[pin distance=-0.4\onedegree,Bayer]00:{$\boldsymbol\tau$}}] at (axis cs:{70.561},{22.957}) {}; % Tau,  4.27 
\node[pin={[pin distance=-0.4\onedegree,Bayer]00:{$\boldsymbol\upsilon$}}] at (axis cs:{66.577},{22.813}) {}; % Tau,  4.29 
\node[pin={[pin distance=-0.4\onedegree,Bayer]00:{$\boldsymbol\varphi$}}] at (axis cs:{65.088},{27.351}) {}; % Tau,  4.95 
\node[pin={[pin distance=-0.8\onedegree,Bayer]12:{$\boldsymbol\vartheta$}}] at (axis cs:{67.0},{15.45}) {}; % Tau,  3.41 
\node[pin={[pin distance=-0.4\onedegree,Bayer]90:{$\boldsymbol\xi$}}] at (axis cs:{51.792},{9.732}) {}; % Tau,  3.73 
\node[pin={[pin distance=-0.4\onedegree,Bayer]00:{$\boldsymbol\zeta$}}] at (axis cs:{84.411},{21.142}) {}; % Tau,  3.00 
\node[pin={[pin distance=-0.4\onedegree,Flaamsted]180:{103}}] at (axis cs:{77.028},{24.265}) {}; % Tau,  5.51 
\node[pin={[pin distance=-0.4\onedegree,Flaamsted]00:{104}}] at (axis cs:{76.863},{18.645}) {}; % Tau,  4.92 
\node[pin={[pin distance=-0.4\onedegree,Flaamsted]180:{105}}] at (axis cs:{76.981},{21.705}) {}; % Tau,  5.84 
\node[pin={[pin distance=-0.4\onedegree,Flaamsted]00:{106}}] at (axis cs:{76.952},{20.418}) {}; % Tau,  5.29 
\node[pin={[pin distance=-0.4\onedegree,Flaamsted]-90:{109}}] at (axis cs:{79.819},{22.096}) {}; % Tau,  4.95 
\node[pin={[pin distance=-0.4\onedegree,Flaamsted]00:{10}}] at (axis cs:{54.218},{0.402}) {}; % Tau,  4.30 
\node[pin={[pin distance=-0.4\onedegree,Flaamsted]00:{111}}] at (axis cs:{81.106},{17.384}) {}; % Tau,  5.00 
\node[pin={[pin distance=-0.4\onedegree,Flaamsted]00:{114}}] at (axis cs:{81.909},{21.937}) {}; % Tau,  4.88 
\node[pin={[pin distance=-0.4\onedegree,Flaamsted]00:{115}}] at (axis cs:{81.792},{17.962}) {}; % Tau,  5.41 
\node[pin={[pin distance=-0.4\onedegree,Flaamsted]180:{116}}] at (axis cs:{81.940},{15.874}) {}; % Tau,  5.52 
%\node[pin={[pin distance=-0.4\onedegree,Flaamsted]180:{117}}] at (axis cs:{82.007},{17.239}) {}; % Tau,  5.77 
\node[pin={[pin distance=-0.4\onedegree,Flaamsted]00:{118}}] at (axis cs:{82.319},{25.150}) {}; % Tau,  5.85 
\node[pin={[pin distance=-0.4\onedegree,Flaamsted]00:{119}}] at (axis cs:{83.053},{18.594}) {}; % Tau,  4.36 
\node[pin={[pin distance=-0.4\onedegree,Flaamsted]180:{120}}] at (axis cs:{83.382},{18.540}) {}; % Tau,  5.67 
\node[pin={[pin distance=-0.4\onedegree,Flaamsted]00:{121}}] at (axis cs:{83.863},{24.039}) {}; % Tau,  5.38 
\node[pin={[pin distance=-0.4\onedegree,Flaamsted]90:{122}}] at (axis cs:{84.266},{17.040}) {}; % Tau,  5.53 
\node[pin={[pin distance=-0.4\onedegree,Flaamsted]00:{125}}] at (axis cs:{84.934},{25.897}) {}; % Tau,  5.17 
\node[pin={[pin distance=-0.4\onedegree,Flaamsted]00:{126}}] at (axis cs:{85.324},{16.534}) {}; % Tau,  4.84 
\node[pin={[pin distance=-0.4\onedegree,Flaamsted]00:{12}}] at (axis cs:{54.963},{3.057}) {}; % Tau,  5.55 
\node[pin={[pin distance=-0.4\onedegree,Flaamsted]00:{130}}] at (axis cs:{86.859},{17.729}) {}; % Tau,  5.48 
\node[pin={[pin distance=-0.4\onedegree,Flaamsted]90:{131}}] at (axis cs:{86.805},{14.488}) {}; % Tau,  5.73 
\node[pin={[pin distance=-0.4\onedegree,Flaamsted]00:{132}}] at (axis cs:{87.254},{24.568}) {}; % Tau,  4.89 
\node[pin={[pin distance=-0.4\onedegree,Flaamsted]00:{133}}] at (axis cs:{86.929},{13.899}) {}; % Tau,  5.28 
\node[pin={[pin distance=-0.4\onedegree,Flaamsted]90:{134}}] at (axis cs:{87.387},{12.651}) {}; % Tau,  4.88 
\node[pin={[pin distance=-0.4\onedegree,Flaamsted]180:{136}}] at (axis cs:{88.332},{27.612}) {}; % Tau,  4.56 
\node[pin={[pin distance=-0.4\onedegree,Flaamsted]-90:{137}}] at (axis cs:{88.093},{14.172}) {}; % Tau,  5.60 
\node[pin={[pin distance=-0.4\onedegree,Flaamsted]90:{139}}] at (axis cs:{89.499},{25.954}) {}; % Tau,  4.83 
\node[pin={[pin distance=-0.4\onedegree,Flaamsted]00:{13}}] at (axis cs:{55.579},{19.700}) {}; % Tau,  5.69 
\node[pin={[pin distance=-0.4\onedegree,Flaamsted]00:{16}}] at (axis cs:{56.201},{24.289}) {}; % Tau,  5.46 
\node[pin={[pin distance=-0.8\onedegree,Flaamsted]-45:{17}}] at (axis cs:{56.219},{24.113}) {}; % Tau,  3.71 
\node[pin={[pin distance=-0.4\onedegree,Flaamsted]90:{18}}] at (axis cs:{56.291},{24.839}) {}; % Tau,  5.66 
\node[pin={[pin distance=-0.8\onedegree,Flaamsted]45:{19}}] at (axis cs:{56.302},{24.467}) {}; % Tau,  4.30 
%\node[pin={[pin distance=-0.4\onedegree,Flaamsted]00:{20}}] at (axis cs:{56.457},{24.368}) {}; % Tau,  3.88 
%\node[pin={[pin distance=-0.4\onedegree,Flaamsted]00:{21}}] at (axis cs:{56.477},{24.554}) {}; % Tau,  5.77 
\node[pin={[pin distance=-0.4\onedegree,Flaamsted]-90:{23}}] at (axis cs:{56.582},{23.948}) {}; % Tau,  4.17 
\node[pin={[pin distance=-0.4\onedegree,Flaamsted]180:{27}}] at (axis cs:{57.291},{24.053}) {}; % Tau,  3.63 
%\node[pin={[pin distance=-0.8\onedegree,Flaamsted]-135:{28}}] at (axis cs:{57.297},{24.137}) {}; % Tau,  5.06 
\node[pin={[pin distance=-0.4\onedegree,Flaamsted]00:{29}}] at (axis cs:{56.419},{6.050}) {}; % Tau,  5.34 
\node[pin={[pin distance=-0.4\onedegree,Flaamsted]00:{30}}] at (axis cs:{57.068},{11.143}) {}; % Tau,  5.08 
\node[pin={[pin distance=-0.4\onedegree,Flaamsted]00:{31}}] at (axis cs:{58.001},{6.535}) {}; % Tau,  5.69 
\node[pin={[pin distance=-0.4\onedegree,Flaamsted]00:{32}}] at (axis cs:{59.217},{22.478}) {}; % Tau,  5.62 
\node[pin={[pin distance=-0.4\onedegree,Flaamsted]00:{36}}] at (axis cs:{61.090},{24.106}) {}; % Tau,  5.51 
\node[pin={[pin distance=-0.4\onedegree,Flaamsted]00:{37}}] at (axis cs:{61.174},{22.082}) {}; % Tau,  4.35 
\node[pin={[pin distance=-0.4\onedegree,Flaamsted]-90:{39}}] at (axis cs:{61.334},{22.009}) {}; % Tau,  5.89 
\node[pin={[pin distance=-0.4\onedegree,Flaamsted]00:{40}}] at (axis cs:{60.936},{5.436}) {}; % Tau,  5.32 
\node[pin={[pin distance=-0.4\onedegree,Flaamsted]00:{41}}] at (axis cs:{61.652},{27.600}) {}; % Tau,  5.18 
\node[pin={[pin distance=-0.4\onedegree,Flaamsted]00:{44}}] at (axis cs:{62.708},{26.481}) {}; % Tau,  5.40 
\node[pin={[pin distance=-0.4\onedegree,Flaamsted]00:{45}}] at (axis cs:{62.835},{5.523}) {}; % Tau,  5.71 
\node[pin={[pin distance=-0.4\onedegree,Flaamsted]00:{46}}] at (axis cs:{63.388},{7.716}) {}; % Tau,  5.29 
\node[pin={[pin distance=-0.4\onedegree,Flaamsted]00:{47}}] at (axis cs:{63.485},{9.264}) {}; % Tau,  4.89 
\node[pin={[pin distance=-0.4\onedegree,Flaamsted]00:{ 4}}] at (axis cs:{52.602},{11.336}) {}; % Tau,  5.13 
\node[pin={[pin distance=-0.4\onedegree,Flaamsted]00:{51}}] at (axis cs:{64.597},{21.579}) {}; % Tau,  5.64 
\node[pin={[pin distance=-0.4\onedegree,Flaamsted]00:{53}}] at (axis cs:{64.859},{21.142}) {}; % Tau,  5.50 
\node[pin={[pin distance=-0.4\onedegree,Flaamsted]180:{56}}] at (axis cs:{64.903},{21.773}) {}; % Tau,  5.36 
\node[pin={[pin distance=-0.4\onedegree,Flaamsted]-90:{57}}] at (axis cs:{64.990},{14.035}) {}; % Tau,  5.58 
\node[pin={[pin distance=-0.4\onedegree,Flaamsted]00:{58}}] at (axis cs:{65.151},{15.095}) {}; % Tau,  5.26 
\node[pin={[pin distance=-0.4\onedegree,Flaamsted]-90:{ 5}}] at (axis cs:{52.718},{12.937}) {}; % Tau,  4.13 
\node[pin={[pin distance=-0.4\onedegree,Flaamsted]180:{60}}] at (axis cs:{65.515},{14.077}) {}; % Tau,  5.72 
\node[pin={[pin distance=-0.4\onedegree,Flaamsted]0:{63}}] at (axis cs:{65.854},{16.777}) {}; % Tau,  5.63 
\node[pin={[pin distance=-0.4\onedegree,Flaamsted]00:{66}}] at (axis cs:{65.966},{9.461}) {}; % Tau,  5.11 
\node[pin={[pin distance=-0.4\onedegree,Flaamsted]180:{ 6}}] at (axis cs:{53.150},{9.373}) {}; % Tau,  5.76 
\node[pin={[pin distance=-0.4\onedegree,Flaamsted]0:{71}}] at (axis cs:{66.586},{15.618}) {}; % Tau,  4.49 
\node[pin={[pin distance=-0.4\onedegree,Flaamsted]90:{72}}] at (axis cs:{66.823},{22.996}) {}; % Tau,  5.52 
\node[pin={[pin distance=-0.4\onedegree,Flaamsted]90:{75}}] at (axis cs:{67.110},{16.360}) {}; % Tau,  4.97 
\node[pin={[pin distance=-0.4\onedegree,Flaamsted]-90:{76}}] at (axis cs:{67.098},{14.741}) {}; % Tau,  5.91 
\node[pin={[pin distance=-0.4\onedegree,Flaamsted]00:{79}}] at (axis cs:{67.209},{13.047}) {}; % Tau,  5.02 
\node[pin={[pin distance=-0.4\onedegree,Flaamsted]00:{ 7}}] at (axis cs:{53.611},{24.464}) {}; % Tau,  5.99 
\node[pin={[pin distance=-0.4\onedegree,Flaamsted]-90:{80}}] at (axis cs:{67.536},{15.638}) {}; % Tau,  5.64 
\node[pin={[pin distance=-0.4\onedegree,Flaamsted]180:{81}}] at (axis cs:{67.662},{15.692}) {}; % Tau,  5.46 
\node[pin={[pin distance=-0.4\onedegree,Flaamsted]00:{83}}] at (axis cs:{67.656},{13.724}) {}; % Tau,  5.40 
\node[pin={[pin distance=-0.4\onedegree,Flaamsted]00:{88}}] at (axis cs:{68.914},{10.161}) {}; % Tau,  4.25 
%\node[pin={[pin distance=-0.8\onedegree,Flaamsted]-45:{89}}] at (axis cs:{69.539},{16.033}) {}; % Tau,  5.78 
\node[pin={[pin distance=-0.4\onedegree,Flaamsted]00:{90}}] at (axis cs:{69.539},{12.511}) {}; % Tau,  4.27 
\node[pin={[pin distance=-0.4\onedegree,Flaamsted]-90:{93}}] at (axis cs:{70.014},{12.197}) {}; % Tau,  5.46 
\node[pin={[pin distance=-0.4\onedegree,Flaamsted]00:{97}}] at (axis cs:{72.844},{18.840}) {}; % Tau,  5.09 
\node[pin={[pin distance=-0.4\onedegree,Flaamsted]00:{98}}] at (axis cs:{74.539},{25.050}) {}; % Tau,  5.79 
\node[pin={[pin distance=-0.4\onedegree,Flaamsted]00:{99}}] at (axis cs:{74.453},{23.948}) {}; % Tau,  5.81 
\node[pin={[pin distance=-0.4\onedegree,Bayer]00:{$\boldsymbol\alpha$}}] at (axis cs:{28.270},{29.579}) {}; % Tri,  3.42 
\node[pin={[pin distance=-0.4\onedegree,Bayer]00:{$\boldsymbol\beta$}}] at (axis cs:{32.386},{34.987}) {}; % Tri,  3.02 
\node[pin={[pin distance=-0.4\onedegree,Bayer]90:{$\boldsymbol\delta$}}] at (axis cs:{34.263},{34.224}) {}; % Tri,  4.87 
\node[pin={[pin distance=-0.4\onedegree,Bayer]90:{$\boldsymbol\epsilon$}}] at (axis cs:{30.741},{33.284}) {}; % Tri,  5.52 
\node[pin={[pin distance=-0.4\onedegree,Bayer]00:{$\boldsymbol\gamma$}}] at (axis cs:{34.329},{33.847}) {}; % Tri,  4.01 
\node[pin={[pin distance=-0.4\onedegree,Flaamsted]00:{10}}] at (axis cs:{34.737},{28.643}) {}; % Tri,  5.30 
\node[pin={[pin distance=-0.4\onedegree,Flaamsted]00:{11}}] at (axis cs:{36.866},{31.801}) {}; % Tri,  5.56 
\node[pin={[pin distance=-0.4\onedegree,Flaamsted]00:{12}}] at (axis cs:{37.042},{29.669}) {}; % Tri,  5.29 
\node[pin={[pin distance=-0.4\onedegree,Flaamsted]90:{13}}] at (axis cs:{37.202},{29.932}) {}; % Tri,  5.90 
\node[pin={[pin distance=-0.4\onedegree,Flaamsted]00:{14}}] at (axis cs:{38.026},{36.147}) {}; % Tri,  5.15 
\node[pin={[pin distance=-0.4\onedegree,Flaamsted]90:{15}}] at (axis cs:{38.945},{34.687}) {}; % Tri,  5.39 
\node[pin={[pin distance=-0.4\onedegree,Flaamsted]00:{ 6}}] at (axis cs:{33.093},{30.303}) {}; % Tri,  5.15 
\node[pin={[pin distance=-0.4\onedegree,Flaamsted]00:{ 7}}] at (axis cs:{33.985},{33.359}) {}; % Tri,  5.25 
\end{axis}


%
% Comets, own objects etc.
\begin{axis}[name=transients,axis lines=none]
% Format for label:
%\node[transient,pin={[pin distance=-0.8\onedegree,transient-label]90:{LABEL}}] at (axis cs:{ *15 + /4 + /240 },{  + /60 + /3600 })  {+};
%
% Note for Southern positions, if in sexagesimal, decl. signs need to be inverted: 
%   \(axis cs:{ *15 + /4 + /240 },{   - /60 - /3600 })  {+};
%
% Note "08" is interpreted as Octal 8, and triggers a problem in computation

%%%%%%%%%%%%%%%%%%%%%%%%%%%%%%%%%%%%%%%%%%%%%%%%%%%%%%%%%%%%%%%%%%%%%%%%%%%%%%%%%%%%%%%%
% Comet C/2023 A3 (Tsuchinshan-ATLAS) in 2024, Apr-Dec\
% Coords taken from https://in-the-sky.org/ephemeris.php?objtxt=CK23A030
%%%%%%%%%%%%%%%%%%%%%%%%%%%%%%%%%%%%%%%%%%%%%%%%%%%%%%%%%%%%%%%%%%%%%%%%%%%%%%%%%%%%%%%%
%
% \node[transient,pin={[pin distance=-0.8\onedegree,transient-label]90:{Apr\,01}}] at (axis cs:{14 *15 + 32/4 +  30/240 },{ -04  - 59/60 -  6/3600 })  {+};
% \node[transient,pin={[pin distance=-0.8\onedegree,transient-label]90:{Apr\,08}}] at (axis cs:{14 *15 + 19/4 +  22/240 },{ -04  -  8/60 - 35/3600 })  {+};
% \node[transient,pin={[pin distance=-0.8\onedegree,transient-label]90:{Apr\,15}}] at (axis cs:{14 *15 +  4/4 +  04/240 },{ -03  - 11/60 - 56/3600 })  {+};
% \node[transient,pin={[pin distance=-0.8\onedegree,transient-label]90:{Apr\,22}}] at (axis cs:{13 *15 + 46/4 +  53/240 },{ -02  - 10/60 - 50/3600 })  {+};
% \node[transient,pin={[pin distance=-0.8\onedegree,transient-label]90:{May\,01}}] at (axis cs:{13 *15 + 22/4 +  47/240 },{ -00  - 50/60 -  5/3600 })  {+};
% \node[transient,pin={[pin distance=-0.8\onedegree,transient-label]90:{May\,08}}] at (axis cs:{13 *15 +  3/4 +  21/240 },{ +00  + 10/60 +  8/3600 })  {+};
% \node[transient,pin={[pin distance=-0.8\onedegree,transient-label]90:{May\,15}}] at (axis cs:{12 *15 + 44/4 +   8/240 },{ +01  +  4/60 + 17/3600 })  {+};
% \node[transient,pin={[pin distance=-0.8\onedegree,transient-label]90:{May\,22}}] at (axis cs:{12 *15 + 25/4 +  54/240 },{ +01  + 49/60 + 32/3600 })  {+};
% \node[transient,pin={[pin distance=-0.8\onedegree,transient-label]90:{Jun\,01}}] at (axis cs:{12 *15 +  2/4 +  36/240 },{ +02  + 35/60 + 35/3600 })  {+};
% \node[transient,pin={[pin distance=-0.8\onedegree,transient-label]90:{Jun\,08}}] at (axis cs:{11 *15 + 48/4 +  37/240 },{ +02  + 54/60 + 16/3600 })  {+};
% \node[transient,pin={[pin distance=-0.8\onedegree,transient-label]90:{Jun\,15}}] at (axis cs:{11 *15 + 36/4 +  38/240 },{ +03  +  2/60 + 21/3600 })  {+};
% \node[transient,pin={[pin distance=-0.8\onedegree,transient-label]90:{Jun\,22}}] at (axis cs:{11 *15 + 26/4 +  34/240 },{ +03  +  0/60 + 47/3600 })  {+};
% \node[transient,pin={[pin distance=-0.8\onedegree,transient-label]90:{Jul\,01}}] at (axis cs:{11 *15 + 16/4 +  10/240 },{ +02  + 46/60 + 21/3600 })  {+};
% \node[transient,pin={[pin distance=-0.8\onedegree,transient-label]90:{Jul\,08}}] at (axis cs:{11 *15 +  9/4 +  44/240 },{ +02  + 26/60 + 41/3600 })  {+};
% \node[transient,pin={[pin distance=-0.8\onedegree,transient-label]90:{Jul\,15}}] at (axis cs:{11 *15 +  4/4 +  32/240 },{ +02  +  0/60 + 32/3600 })  {+};
% \node[transient,pin={[pin distance=-0.8\onedegree,transient-label]90:{Jul\,22}}] at (axis cs:{11 *15 +  0/4 +  19/240 },{ +01  + 28/60 + 34/3600 })  {+};
% \node[transient,pin={[pin distance=-0.8\onedegree,transient-label]90:{Aug\,01}}] at (axis cs:{10 *15 + 55/4 +  30/240 },{ +00  + 33/60 + 43/3600 })  {+};
% \node[transient,pin={[pin distance=-0.8\onedegree,transient-label]90:{Aug\,08}}] at (axis cs:{10 *15 + 52/4 +  42/240 },{ -00  - 10/60 - 40/3600 })  {+};
% \node[transient,pin={[pin distance=-0.8\onedegree,transient-label]90:{Aug\,15}}] at (axis cs:{10 *15 + 50/4 +  05/240 },{ -00  - 59/60 - 45/3600 })  {+};
% \node[transient,pin={[pin distance=-0.8\onedegree,transient-label]90:{Aug\,22}}] at (axis cs:{10 *15 + 47/4 +  26/240 },{ -01  - 53/60 - 19/3600 })  {+};
% \node[transient,pin={[pin distance=-0.8\onedegree,transient-label]90:{Sep\,01}}] at (axis cs:{10 *15 + 43/4 +  06/240 },{ -03  - 16/60 - 41/3600 })  {+};
% \node[transient,pin={[pin distance=-0.8\onedegree,transient-label]90:{Sep\,08}}] at (axis cs:{10 *15 + 39/4 +  30/240 },{ -04  - 17/60 - 50/3600 })  {+};
% \node[transient,pin={[pin distance=-0.8\onedegree,transient-label]90:{Sep\,15}}] at (axis cs:{10 *15 + 35/4 +  57/240 },{ -05  - 16/60 -  5/3600 })  {+};
% \node[transient,pin={[pin distance=-0.8\onedegree,transient-label]90:{Sep\,22}}] at (axis cs:{10 *15 + 35/4 +  49/240 },{ -05  - 58/60 -  7/3600 })  {+};
% \node[transient,pin={[pin distance=-0.8\onedegree,transient-label]90:{Oct\,01}}] at (axis cs:{11 *15 +  4/4 +  03/240 },{ -05  - 41/60 - 36/3600 })  {+};
% \node[transient,pin={[pin distance=-0.8\onedegree,transient-label]90:{Oct\,08}}] at (axis cs:{12 *15 + 30/4 +  27/240 },{ -03  - 36/60 - 16/3600 })  {+};
% \node[transient,pin={[pin distance=-0.8\onedegree,transient-label]90:{Oct\,15}}] at (axis cs:{14 *15 + 59/4 +  40/240 },{ +00  + 23/60 + 14/3600 })  {+};
% \node[transient,pin={[pin distance=-0.8\onedegree,transient-label]90:{Oct\,22}}] at (axis cs:{16 *15 + 49/4 +  25/240 },{ +02  + 50/60 + 11/3600 })  {+};
% \node[transient,pin={[pin distance=-0.8\onedegree,transient-label]90:{Nov\,01}}] at (axis cs:{17 *15 + 59/4 +  40/240 },{ +03  + 45/60 + 12/3600 })  {+};
% \node[transient,pin={[pin distance=-0.8\onedegree,transient-label]90:{Nov\,08}}] at (axis cs:{18 *15 + 24/4 +  28/240 },{ +03  + 55/60 + 51/3600 })  {+};
% \node[transient,pin={[pin distance=-0.8\onedegree,transient-label]90:{Nov\,15}}] at (axis cs:{18 *15 + 41/4 +  24/240 },{ +04  +  3/60 + 27/3600 })  {+};
% \node[transient,pin={[pin distance=-0.8\onedegree,transient-label]90:{Nov\,22}}] at (axis cs:{18 *15 + 54/4 +  18/240 },{ +04  + 12/60 + 40/3600 })  {+};
% \node[transient,pin={[pin distance=-0.8\onedegree,transient-label]90:{Dec\,01}}] at (axis cs:{19 *15 +  7/4 +  39/240 },{ +04  + 29/60 + 30/3600 })  {+};
% \node[transient,pin={[pin distance=-0.8\onedegree,transient-label]90:{Dec\,08}}] at (axis cs:{19 *15 + 16/4 +  32/240 },{ +04  + 47/60 + 12/3600 })  {+};
% \node[transient,pin={[pin distance=-0.8\onedegree,transient-label]90:{Dec\,15}}] at (axis cs:{19 *15 + 24/4 +  32/240 },{ +05  +  9/60 +  7/3600 })  {+};
% \node[transient,pin={[pin distance=-0.8\onedegree,transient-label]90:{Dec\,22}}] at (axis cs:{19 *15 + 31/4 +  54/240 },{ +05  + 35/60 + 17/3600 })  {+};
% 
% 
% \draw[transient]
% (axis cs:{14 *15 + 32/4 + 30/240 },{ -04  - 59/60 -  6/3600 }) --
% (axis cs:{14 *15 + 19/4 + 22/240 },{ -04  -  8/60 - 35/3600 }) --
% (axis cs:{14 *15 +  4/4 + 04/240 },{ -03  - 11/60 - 56/3600 }) --
% (axis cs:{13 *15 + 46/4 + 53/240 },{ -02  - 10/60 - 50/3600 }) --
% (axis cs:{13 *15 + 22/4 + 47/240 },{ -00  - 50/60 -  5/3600 }) --
% (axis cs:{13 *15 +  3/4 + 21/240 },{ +00  + 10/60 +  8/3600 }) --
% (axis cs:{12 *15 + 44/4 +  8/240 },{ +01  +  4/60 + 17/3600 }) --
% (axis cs:{12 *15 + 25/4 + 54/240 },{ +01  + 49/60 + 32/3600 }) --
% (axis cs:{12 *15 +  2/4 + 36/240 },{ +02  + 35/60 + 35/3600 }) --
% (axis cs:{11 *15 + 48/4 + 37/240 },{ +02  + 54/60 + 16/3600 }) --
% (axis cs:{11 *15 + 36/4 + 38/240 },{ +03  +  2/60 + 21/3600 }) --
% (axis cs:{11 *15 + 26/4 + 34/240 },{ +03  +  0/60 + 47/3600 }) --
% (axis cs:{11 *15 + 16/4 + 10/240 },{ +02  + 46/60 + 21/3600 }) --
% (axis cs:{11 *15 +  9/4 + 44/240 },{ +02  + 26/60 + 41/3600 }) --
% (axis cs:{11 *15 +  4/4 + 32/240 },{ +02  +  0/60 + 32/3600 }) --
% (axis cs:{11 *15 +  0/4 + 19/240 },{ +01  + 28/60 + 34/3600 }) --
% (axis cs:{10 *15 + 55/4 + 30/240 },{ +00  + 33/60 + 43/3600 }) --
% (axis cs:{10 *15 + 52/4 + 42/240 },{ -00  - 10/60 - 40/3600 }) --
% (axis cs:{10 *15 + 50/4 + 05/240 },{ -00  - 59/60 - 45/3600 }) --
% (axis cs:{10 *15 + 47/4 + 26/240 },{ -01  - 53/60 - 19/3600 }) --
% (axis cs:{10 *15 + 43/4 + 06/240 },{ -03  - 16/60 - 41/3600 }) --
% (axis cs:{10 *15 + 39/4 + 30/240 },{ -04  - 17/60 - 50/3600 }) --
% (axis cs:{10 *15 + 35/4 + 57/240 },{ -05  - 16/60 -  5/3600 }) --
% (axis cs:{10 *15 + 35/4 + 49/240 },{ -05  - 58/60 -  7/3600 }) --
% (axis cs:{11 *15 +  4/4 + 03/240 },{ -05  - 41/60 - 36/3600 }) --
% (axis cs:{12 *15 + 30/4 + 27/240 },{ -03  - 36/60 - 16/3600 }) --
% (axis cs:{14 *15 + 59/4 + 40/240 },{ +00  + 23/60 + 14/3600 }) --
% (axis cs:{16 *15 + 49/4 + 25/240 },{ +02  + 50/60 + 11/3600 }) --
% (axis cs:{17 *15 + 59/4 + 40/240 },{ +03  + 45/60 + 12/3600 }) --
% (axis cs:{18 *15 + 24/4 + 28/240 },{ +03  + 55/60 + 51/3600 }) --
% (axis cs:{18 *15 + 41/4 + 24/240 },{ +04  +  3/60 + 27/3600 }) --
% (axis cs:{18 *15 + 54/4 + 18/240 },{ +04  + 12/60 + 40/3600 }) --
% (axis cs:{19 *15 +  7/4 + 39/240 },{ +04  + 29/60 + 30/3600 }) --
% (axis cs:{19 *15 + 16/4 + 32/240 },{ +04  + 47/60 + 12/3600 }) --
% (axis cs:{19 *15 + 24/4 + 32/240 },{ +05  +  9/60 +  7/3600 }) --
% (axis cs:{19 *15 + 31/4 + 54/240 },{ +05  + 35/60 + 17/3600 }) ;
% 
% 

\end{axis}

%

\end{tikzpicture}

\end{document}
