%%%%%%%%%%%%%%%%%%%%%%%%%%%%%%%%%%%%%%%%%%%%%%%%%%%%%%%%%%%%%%%%%%%%%%
%
% rm TabulaVII.pdf ; pdflatex -shell-escape  TabVII.tex 
%
%%%%%%%%%%%%%%%%%%%%%%%%%%%%%%%%%%%%%%%%%%%%%%%%%%%%%%%%%%%%%%%%%%%%%%
\documentclass[10pt,landscape]{article}
\usepackage[margin=1cm,a3paper]{geometry}


\usepackage[margin=1cm,a3paper]{geometry}

\def\pgfsysdriver{pgfsys-pdftex.def}
\usepackage{pgfplots}
\usepgfplotslibrary{external} 
\usetikzlibrary{pgfplots.external}
\usetikzlibrary{calc}
\usetikzlibrary{shapes}
\usetikzlibrary{patterns}
\usetikzlibrary{plotmarks}
\usepgflibrary{arrows}
\usepgfplotslibrary{polar}
\tikzexternalize

\usepackage{times,latexsym,amssymb}
\usepackage{eulervm} % math font
\pagestyle{empty}

\usepackage{amsmath} 
\usepackage{nicefrac} 
%
% This is used because from 1.11 onwards the use of \pgfdeclareplotmark will
% introduce a shift of the stellar markers.
%
\pgfplotsset{compat=1.10}
%
% Use to increase spacing for constellation labels
\usepackage[letterspace=400]{microtype}
%
\usepackage{pagecolor}
\usepackage{natbib}
%


%\pagecolor{black}
\pagecolor{white}


\begin{document} 


%\input{./input/styles_dark.tex}
\input{./input/styles_po.tex}
\center

\pgfplotsset{xmin=0,xmax=360,ymin=-90,ymax=-2.5
,y coord trafo/.code=\pgfmathparse{-90-#1}
,y coord inv trafo/.code=\pgfmathparse{-90-#1}
,x coord trafo/.code=\pgfmathparse{0-#1}
,x coord inv trafo/.code=\pgfmathparse{0-#1}
}

\tikzsetnextfilename{TabulaVII}


\begin{tikzpicture}

\begin{polaraxis}[rotate=90,name=base,axis y line=none,axis x line=none,clip=false]\end{polaraxis}


% The Milky Way first, as it should not obstruct coordinate systems 
\input{./input/VII_MW.tex}

\input{./input/VII_Grid.tex}

\begin{polaraxis}[rotate=270,name=axis1,at=(base.center),anchor=center,color=cAxes,axis
    lines=none,ymin=-90,ymax=-8.8  
    ,width=13\tendegree,height=13\tendegree
  ,x dir = reverse]


%\draw (axis cs:90,0) arc (90:270:6\tendegree) ;


\draw (6\tendegree,0\tendegree) arc (0:90:6\tendegree) -- 
(-2\tendegree,6\tendegree) -- (-2\tendegree,-6\tendegree) -- (0\tendegree,-6\tendegree)
--  (0\tendegree,-6\tendegree) arc (270:359.9999:6\tendegree) -- cycle ;


\node[rotate=-90]  at (axis cs:90,-28.5)  {18\fh{\small 00}\fm};
\node[rotate=-80]  at (axis cs:100,-28.5) {18\fh{\small 40}\fm};
\node[rotate=-70]  at (axis cs:110,-28.5) {19\fh{\small 20}\fm};
\node[rotate=-60]  at (axis cs:120,-28.5) {20\fh{\small 00}\fm};
\node[rotate=-50]  at (axis cs:130,-28.5) {20\fh{\small 40}\fm};
\node[rotate=-40]  at (axis cs:140,-28.5) {21\fh{\small 20}\fm};
\node[rotate=-30]  at (axis cs:150,-28.5) {22\fh{\small 00}\fm};
\node[rotate=-20]  at (axis cs:160,-28.5) {22\fh{\small 40}\fm};
\node[rotate=-10]  at (axis cs:170,-28.5) {23\fh{\small 20}\fm};
\node[rotate=-00]  at (axis cs:180,-28.5) {0\fh{\small 00}\fm};
\node[rotate=10] at (axis cs:190,-28.5) {0\fh{\small 40}\fm};
\node[rotate=20] at (axis cs:200,-28.5) {1\fh{\small 20}\fm};
\node[rotate=30] at (axis cs:210,-28.5) {2\fh{\small 00}\fm};
\node[rotate=40] at (axis cs:220,-28.5) {2\fh{\small 40}\fm};
\node[rotate=50] at (axis cs:230,-28.5) {3\fh{\small 20}\fm};
\node[rotate=60] at (axis cs:240,-28.5) {4\fh{\small 00}\fm};
\node[rotate=70] at (axis cs:250,-28.5) {4\fh{\small 40}\fm};
\node[rotate=80] at (axis cs:260,-28.5) {5\fh{\small 20}\fm};
\node[rotate=90] at (axis cs:270,-28.5) {6\fh{\small 00}\fm};

\node[rotate=-90]  at (axis cs:90,-29.4)  {\footnotesize$270^\circ$};
\node[rotate=-80]  at (axis cs:100,-29.4) {\footnotesize$280^\circ$};
\node[rotate=-70]  at (axis cs:110,-29.4) {\footnotesize$290^\circ$};
\node[rotate=-60]  at (axis cs:120,-29.4) {\footnotesize$300^\circ$};
\node[rotate=-50]  at (axis cs:130,-29.4) {\footnotesize$310^\circ$};
\node[rotate=-40]  at (axis cs:140,-29.4) {\footnotesize$320^\circ$};
\node[rotate=-30]  at (axis cs:150,-29.4) {\footnotesize$330^\circ$};
\node[rotate=-20]  at (axis cs:160,-29.4) {\footnotesize$340^\circ$};
\node[rotate=-10]  at (axis cs:170,-29.4) {\footnotesize$350^\circ$};
\node[rotate=00]  at (axis cs:180,-29.4) {\footnotesize$0^\circ$};
\node[rotate=10] at (axis cs:190,-29.4) {\footnotesize$10^\circ$};
\node[rotate=20] at (axis cs:200,-29.4) {\footnotesize$20^\circ$};
\node[rotate=30] at (axis cs:210,-29.4) {\footnotesize$30^\circ$};
\node[rotate=40] at (axis cs:220,-29.4) {\footnotesize$40^\circ$};
\node[rotate=50] at (axis cs:230,-29.4) {\footnotesize$50^\circ$};
\node[rotate=60] at (axis cs:240,-29.4) {\footnotesize$60^\circ$};
\node[rotate=70] at (axis cs:250,-29.4) {\footnotesize$70^\circ$};
\node[rotate=80] at (axis cs:260,-29.4) {\footnotesize$80^\circ$};
\node[rotate=90] at (axis cs:270,-29.4) {\footnotesize$90^\circ$};

\node[rotate=90] at (axis cs:280,-28.5) {\footnotesize$100^\circ$};
\node[rotate=90] at (axis cs:280,-27.6) {6\fh{\small 40}\fm};

\node[rotate=-90] at (axis cs:80,-28.5) {\footnotesize$260^\circ$};
\node[rotate=-90] at (axis cs:80,-27.6) {17\fh{\small 20}\fm};




\node at (-2.18\tendegree,5.494957\tendegree)  {7\fh{\small 20}\fm};
\node at (-2.18\tendegree,3.464103\tendegree)  {8\fh{\small 00}\fm};
\node at (-2.18\tendegree,2.383508\tendegree)  {8\fh{\small 40}\fm};
\node at (-2.18\tendegree,1.678200\tendegree)  {9\fh{\small 20}\fm};
\node at (-2.18\tendegree,1.154701\tendegree)  {10\fh{\small 00}\fm};
\node at (-2.18\tendegree,0.7279406\tendegree) {10\fh{\small 40}\fm};
\node at (-2.18\tendegree,0.352\tendegree)     {11\fh{\small 20}\fm};
\node at (-2.18\tendegree,0.0\tendegree)       {12\fh{\small 00}\fm};
\node at (-2.18\tendegree,-0.352\tendegree)    {12\fh{\small 40}\fm};
\node at (-2.18\tendegree,-0.7279406\tendegree){13\fh{\small 20}\fm};
\node at (-2.18\tendegree,-1.154701\tendegree) {14\fh{\small 00}\fm};
\node at (-2.18\tendegree,-1.678200\tendegree) {14\fh{\small 40}\fm};
\node at (-2.18\tendegree,-2.383508\tendegree) {15\fh{\small 20}\fm};
\node at (-2.18\tendegree,-3.464103\tendegree) {16\fh{\small 00}\fm};
\node at (-2.18\tendegree,-5.494957\tendegree) {16\fh{\small 40}\fm};

\node at (-2.06\tendegree,5.494957\tendegree)   {\footnotesize$110^\circ$};
\node at (-2.06\tendegree,3.464103\tendegree)   {\footnotesize$120^\circ$};
\node at (-2.06\tendegree,2.383508\tendegree)  {\footnotesize$130^\circ$}; 
\node at (-2.06\tendegree,1.678200\tendegree)   {\footnotesize$140^\circ$};
\node at (-2.06\tendegree,1.154701\tendegree)   {\footnotesize$150^\circ$};
\node at (-2.06\tendegree,0.7279406\tendegree){\footnotesize$160^\circ$}; 
\node at (-2.06\tendegree,0.352\tendegree)      {\footnotesize$170^\circ$};
\node at (-2.06\tendegree,0.0\tendegree) {\footnotesize$180^\circ$};
\node at (-2.06\tendegree,-0.352\tendegree)    {\footnotesize$190^\circ$};
\node at (-2.06\tendegree,-0.7279406\tendegree) {\footnotesize$200^\circ$};
\node at (-2.06\tendegree,-1.154701\tendegree) {\footnotesize$210^\circ$};
\node at (-2.06\tendegree,-1.678200\tendegree) {\footnotesize$220^\circ$};
\node at (-2.06\tendegree,-2.383508\tendegree)  {\footnotesize$230^\circ$};
\node at (-2.06\tendegree,-3.464103\tendegree) {\footnotesize$240^\circ$};
\node at (-2.06\tendegree,-5.494957\tendegree) {\footnotesize$250^\circ$};



\end{polaraxis}


% Left and bottom axis with gridlines
%  \begin{axis}[at=(base.center),anchor=center,RA_in_hours,color=cAxes] \end{axis}
% Right and top axis, labelling in RA hours (offset)
%  \begin{axis}[at=(base.center),anchor=center,yticklabel pos=right,xticklabel pos=right,RA_in_hours,xticklabel style={yshift=2.5ex, anchor=south},color=cAxes]\end{axis}
% Top axis labelling in RA degree (small offset)
%  \begin{axis}[at=(base.center),anchor=center,axis y line=none,xticklabel pos=top,RA_in_deg     ,xticklabel style={font=\footnotesize},xticklabel style={yshift=0.5ex, anchor=south} ,color=cAxes]\end{axis}



  \begin{axis}[name=frame,at={($(base.center)+(-.376cm,-19.5\onedegree)$)},anchor=center,width=12.43\tendegree,height=8.53\tendegree,ticks=none
  ,yticklabels={},xticklabels={},axis line style={line width=1pt},color=cFrame] \end{axis}

  \begin{axis}[name=frame2,at={($(frame.center)+(0cm,+0cm)$)},anchor=center,width=12.53\tendegree,height=8.63\tendegree,ticks=none
  ,yticklabels={},xticklabels={},axis line style={line width=3pt},color=cFrame] \end{axis}
  


\begin{polaraxis}[rotate=270,at=(base.center),anchor=center,y axis line style= { draw opacity=0 },
    axis x line=none,y tick label style= { opacity=0 },y tick style= {opacity=0 }
,grid=none,width=13\tendegree,height=13\tendegree]

\clip (-0\tendegree,-8\tendegree) --  (8\tendegree,-8\tendegree) --
(8\tendegree,8\tendegree) --  (-0\tendegree,8\tendegree) -- cycle ;

\draw[line width=0.6\onedegree,color=white] (0,0) circle (6.245\tendegree);

\draw[line width=1pt] (0,0) circle (6.215\tendegree);
\draw[line width=3pt] (0,0) circle (6.265\tendegree);

  \end{polaraxis}

\input{./input/VIIaVIII_Legend.tex}

% The Galactic coordinate system
\input{./input/VII_MWcoord.tex}


% Constellations
%
%
% Some are commented out because they are surrounded by already plotted borders
% The x-coordinate is in fractional hours, so must be times 15
%

\begin{polaraxis}[rotate=90,name=constellations,at={($(base.center)+(-0.8cm+0.75pt,0pt)$)},anchor=center,axis lines=none,clip=false]

%\begin{scope}

\clip  (0\tendegree,-6\tendegree) arc (270:179.999:6\tendegree)  -- (-6\tendegree,0\tendegree) arc (180:90:6\tendegree)  -- (2\tendegree,6\tendegree)  -- (2\tendegree,-6\tendegree) -- cycle ;

%\draw[constellation-boundary]   ++ (0,0);
\draw[constellation-boundary]  (axis cs:180.599,-35.696092) --  (axis cs:181.608,-35.696028) --  (axis cs:182.617,-35.695749) --  (axis cs:183.625,-35.695258) --  (axis cs:184.634,-35.694552) --  (axis cs:185.39,-35.693883) ;
  \draw[constellation-boundary]  (axis cs:209.111,-83.120071) --  (axis cs:208.939,-82.620525) --  (axis cs:208.657,-81.621273) --  (axis cs:208.434,-80.621862) --  (axis cs:208.254,-79.622338) --  (axis cs:208.105,-78.622731) --  (axis cs:207.98,-77.623062) --  (axis cs:207.873,-76.623344) --  (axis cs:207.781,-75.623587) --  (axis cs:207.701,-74.623800) --  (axis cs:207.63,-73.623987) --  (axis cs:207.568,-72.624153) --  (axis cs:207.511,-71.624301) --  (axis cs:207.461,-70.624435) --  (axis cs:207.415,-69.624556) --  (axis cs:207.373,-68.624666) --  (axis cs:207.335,-67.624767) --  (axis cs:207.3,-66.624859) --  (axis cs:207.268,-65.624945) ;
  \draw[constellation-boundary]  (axis cs:276.866,-82.458275) --  (axis cs:276.535,-81.960289) --  (axis cs:275.982,-80.963653) --  (axis cs:275.538,-79.966353) --  (axis cs:275.174,-78.968569) --  (axis cs:274.869,-77.970420) --  (axis cs:274.611,-76.971992) --  (axis cs:274.388,-75.973343) --  (axis cs:274.195,-74.974518) --  (axis cs:274.025,-73.975549) --  (axis cs:273.875,-72.976463) --  (axis cs:273.741,-71.977278) --  (axis cs:273.62,-70.978010) --  (axis cs:273.512,-69.978671) --  (axis cs:273.413,-68.979272) --  (axis cs:273.322,-67.979821) --  (axis cs:273.28,-67.480078) ;
  \draw[constellation-boundary]  (axis cs:274.195,-74.974518) --  (axis cs:275.192,-74.962397) --  (axis cs:276.188,-74.950293) --  (axis cs:277.183,-74.938208) --  (axis cs:278.178,-74.926147) --  (axis cs:279.172,-74.914113) --  (axis cs:280.164,-74.902109) --  (axis cs:281.157,-74.890140) --  (axis cs:282.148,-74.878208) --  (axis cs:283.139,-74.866318) --  (axis cs:284.129,-74.854473) --  (axis cs:285.118,-74.842676) --  (axis cs:286.106,-74.830931) --  (axis cs:287.094,-74.819241) --  (axis cs:288.081,-74.807610) --  (axis cs:289.067,-74.796042) --  (axis cs:290.053,-74.784538) --  (axis cs:291.038,-74.773104) --  (axis cs:292.022,-74.761742) --  (axis cs:293.005,-74.750455) --  (axis cs:293.988,-74.739247) --  (axis cs:294.97,-74.728121) --  (axis cs:295.952,-74.717080) --  (axis cs:296.932,-74.706128) --  (axis cs:297.912,-74.695267) --  (axis cs:298.892,-74.684501) --  (axis cs:299.87,-74.673833) --  (axis cs:300.848,-74.663265) --  (axis cs:301.826,-74.652801) --  (axis cs:302.803,-74.642444) --  (axis cs:303.779,-74.632197) --  (axis cs:304.754,-74.622063) --  (axis cs:305.729,-74.612043) --  (axis cs:306.704,-74.602143) --  (axis cs:307.678,-74.592363) --  (axis cs:308.651,-74.582707) --  (axis cs:309.623,-74.573178) --  (axis cs:310.595,-74.563778) --  (axis cs:311.567,-74.554510) --  (axis cs:312.538,-74.545376) --  (axis cs:313.508,-74.536379) --  (axis cs:314.478,-74.527522) --  (axis cs:315.447,-74.518807) --  (axis cs:316.416,-74.510236) --  (axis cs:317.385,-74.501813) --  (axis cs:318.352,-74.493538) --  (axis cs:319.32,-74.485414) --  (axis cs:320.287,-74.477444) --  (axis cs:321.253,-74.469630) --  (axis cs:322.219,-74.461974) --  (axis cs:323.185,-74.454478) --  (axis cs:324.15,-74.447144) --  (axis cs:325.115,-74.439975) --  (axis cs:326.079,-74.432971) --  (axis cs:327.043,-74.426136) --  (axis cs:328.006,-74.419470) --  (axis cs:328.969,-74.412976) --  (axis cs:329.932,-74.406655) --  (axis cs:330.894,-74.400510) --  (axis cs:331.856,-74.394541) --  (axis cs:332.818,-74.388751) --  (axis cs:333.78,-74.383141) --  (axis cs:334.741,-74.377713) --  (axis cs:335.701,-74.372468) --  (axis cs:336.662,-74.367407) --  (axis cs:337.622,-74.362532) --  (axis cs:338.582,-74.357844) --  (axis cs:339.541,-74.353344) --  (axis cs:340.501,-74.349034) --  (axis cs:341.46,-74.344915) --  (axis cs:342.419,-74.340987) --  (axis cs:343.378,-74.337253) --  (axis cs:344.336,-74.333712) --  (axis cs:345.294,-74.330367) --  (axis cs:346.252,-74.327217) --  (axis cs:347.21,-74.324263) --  (axis cs:348.168,-74.321508) --  (axis cs:349.126,-74.318950) --  (axis cs:350.083,-74.316591) --  (axis cs:351.041,-74.314432) --  (axis cs:351.998,-74.312472) --  (axis cs:352.955,-74.310713) --  (axis cs:353.912,-74.309156) --  (axis cs:354.869,-74.307799) --  (axis cs:355.826,-74.306645) --  (axis cs:356.783,-74.305692) --  (axis cs:357.739,-74.304942) --  (axis cs:358.696,-74.304395) --  (axis cs:359.653,-74.304050) --  (axis cs:360.61,-74.303908) --  (axis cs:361.566,-74.303969) --  (axis cs:362.523,-74.304233) --  (axis cs:363.48,-74.304699) --  (axis cs:364.436,-74.305368) ;
  \draw[constellation-boundary]  (axis cs:265.776,-67.571107) --  (axis cs:265.735,-67.071352) --  (axis cs:265.659,-66.071812) --  (axis cs:265.588,-65.072238) --  (axis cs:265.523,-64.072633) --  (axis cs:265.462,-63.073001) --  (axis cs:265.405,-62.073344) --  (axis cs:265.352,-61.073666) --  (axis cs:265.302,-60.073967) --  (axis cs:265.255,-59.074251) --  (axis cs:265.21,-58.074519) --  (axis cs:265.168,-57.074772) ;
  \draw[constellation-boundary]  (axis cs:258.471,-70.159744) --  (axis cs:258.373,-69.160319) --  (axis cs:258.284,-68.160843) --  (axis cs:258.242,-67.661088) ;
  \draw[constellation-boundary]  (axis cs:224.166,-70.511542) --  (axis cs:224.097,-69.511825) --  (axis cs:224.033,-68.512083) --  (axis cs:224.004,-68.012204) ;
  \draw[constellation-boundary]  (axis cs:207.461,-70.624435) --  (axis cs:208.491,-70.618886) --  (axis cs:209.522,-70.613144) --  (axis cs:210.552,-70.607211) --  (axis cs:211.582,-70.601088) --  (axis cs:212.611,-70.594777) --  (axis cs:213.64,-70.588281) --  (axis cs:214.669,-70.581601) --  (axis cs:215.697,-70.574741) --  (axis cs:216.725,-70.567701) --  (axis cs:217.753,-70.560486) --  (axis cs:218.78,-70.553096) --  (axis cs:219.807,-70.545534) --  (axis cs:220.833,-70.537803) --  (axis cs:221.859,-70.529906) --  (axis cs:222.885,-70.521845) --  (axis cs:223.91,-70.513622) --  (axis cs:224.935,-70.505241) --  (axis cs:225.959,-70.496703) --  (axis cs:226.983,-70.488013) --  (axis cs:228.007,-70.479172) --  (axis cs:229.03,-70.470184) --  (axis cs:230.052,-70.461052) --  (axis cs:231.074,-70.451778) --  (axis cs:232.096,-70.442365) --  (axis cs:233.117,-70.432818) --  (axis cs:234.137,-70.423137) --  (axis cs:235.157,-70.413328) --  (axis cs:236.177,-70.403393) --  (axis cs:237.196,-70.393334) --  (axis cs:238.215,-70.383157) --  (axis cs:239.233,-70.372863) --  (axis cs:240.25,-70.362456) --  (axis cs:241.267,-70.351939) --  (axis cs:242.283,-70.341316) --  (axis cs:243.299,-70.330590) --  (axis cs:244.315,-70.319765) --  (axis cs:245.33,-70.308844) --  (axis cs:246.344,-70.297830) --  (axis cs:247.357,-70.286728) --  (axis cs:248.371,-70.275539) --  (axis cs:249.383,-70.264269) --  (axis cs:250.395,-70.252921) --  (axis cs:251.407,-70.241497) --  (axis cs:252.417,-70.230003) --  (axis cs:253.428,-70.218441) --  (axis cs:254.438,-70.206815) --  (axis cs:255.447,-70.195128) --  (axis cs:256.455,-70.183385) --  (axis cs:257.463,-70.171590) --  (axis cs:258.471,-70.159744) ;
  \draw[constellation-boundary]  (axis cs:248.571,-45.767049) --  (axis cs:249.261,-45.759299) --  (axis cs:250.266,-45.747960) --  (axis cs:251.27,-45.736546) --  (axis cs:252.274,-45.725061) --  (axis cs:253.278,-45.713509) --  (axis cs:254.281,-45.701892) --  (axis cs:255.285,-45.690215) --  (axis cs:256.288,-45.678481) --  (axis cs:257.291,-45.666694) --  (axis cs:258.293,-45.654857) --  (axis cs:259.296,-45.642974) --  (axis cs:260.298,-45.631049) --  (axis cs:261.3,-45.619085) --  (axis cs:262.302,-45.607085) --  (axis cs:263.304,-45.595055) --  (axis cs:264.305,-45.582997) --  (axis cs:265.306,-45.570914) --  (axis cs:266.307,-45.558812) --  (axis cs:267.308,-45.546692) --  (axis cs:268.309,-45.534560) --  (axis cs:269.309,-45.522419) --  (axis cs:270.309,-45.510272) --  (axis cs:271.309,-45.498123) --  (axis cs:272.309,-45.485976) --  (axis cs:273.308,-45.473835) --  (axis cs:274.308,-45.461703) --  (axis cs:275.307,-45.449584) --  (axis cs:276.306,-45.437482) --  (axis cs:277.304,-45.425400) --  (axis cs:278.303,-45.413342) --  (axis cs:279.301,-45.401311) --  (axis cs:280.299,-45.389312) --  (axis cs:281.297,-45.377348) --  (axis cs:282.294,-45.365422) --  (axis cs:283.292,-45.353539) --  (axis cs:284.289,-45.341701) --  (axis cs:285.286,-45.329912) --  (axis cs:286.283,-45.318177) --  (axis cs:287.279,-45.306497) --  (axis cs:288.275,-45.294878) --  (axis cs:289.272,-45.283322) --  (axis cs:290.268,-45.271833) --  (axis cs:291.263,-45.260414) --  (axis cs:292.259,-45.249070) --  (axis cs:293.254,-45.237802) --  (axis cs:294.249,-45.226615) --  (axis cs:295.244,-45.215512) --  (axis cs:296.239,-45.204496) --  (axis cs:297.234,-45.193571) --  (axis cs:298.228,-45.182740) --  (axis cs:299.222,-45.172006) --  (axis cs:300.216,-45.161372) --  (axis cs:301.21,-45.150841) --  (axis cs:302.204,-45.140418) --  (axis cs:303.197,-45.130104) --  (axis cs:304.19,-45.119903) --  (axis cs:305.184,-45.109817) --  (axis cs:306.177,-45.099851) --  (axis cs:307.169,-45.090007) --  (axis cs:308.162,-45.080287) --  (axis cs:309.154,-45.070695) --  (axis cs:310.147,-45.061234) --  (axis cs:311.139,-45.051906) --  (axis cs:312.131,-45.042714) --  (axis cs:313.123,-45.033661) --  (axis cs:314.114,-45.024750) --  (axis cs:315.106,-45.015983) --  (axis cs:316.097,-45.007363) --  (axis cs:317.088,-44.998892) --  (axis cs:318.079,-44.990573) --  (axis cs:319.07,-44.982408) --  (axis cs:320.061,-44.974400) --  (axis cs:321.052,-44.966552) --  (axis cs:322.042,-44.958865) ;
  \draw[constellation-boundary]  (axis cs:269.809,-45.516346) --  (axis cs:269.797,-45.016420) --  (axis cs:269.773,-44.016565) --  (axis cs:269.75,-43.016706) --  (axis cs:269.728,-42.016841) --  (axis cs:269.706,-41.016973) --  (axis cs:269.685,-40.017101) --  (axis cs:269.665,-39.017224) --  (axis cs:269.645,-38.017345) --  (axis cs:269.625,-37.017462) --  (axis cs:269.607,-36.017576) --  (axis cs:269.588,-35.017688) --  (axis cs:269.57,-34.017796) --  (axis cs:269.553,-33.017902) --  (axis cs:269.536,-32.018006) --  (axis cs:269.519,-31.018107) --  (axis cs:269.503,-30.018207) ;
  \draw[constellation-boundary]  (axis cs:272.672,-56.983770) --  (axis cs:272.632,-55.984012) --  (axis cs:272.595,-54.984242) --  (axis cs:272.559,-53.984461) --  (axis cs:272.524,-52.984669) --  (axis cs:272.491,-51.984868) --  (axis cs:272.46,-50.985058) --  (axis cs:272.43,-49.985241) --  (axis cs:272.401,-48.985415) --  (axis cs:272.374,-47.985583) --  (axis cs:272.347,-46.985745) --  (axis cs:272.321,-45.985901) --  (axis cs:272.309,-45.485976) ;
  \draw[constellation-boundary]  (axis cs:265.168,-57.074772) --  (axis cs:265.669,-57.068725) --  (axis cs:266.67,-57.056616) --  (axis cs:267.672,-57.044491) --  (axis cs:268.672,-57.032355) --  (axis cs:269.673,-57.020212) --  (axis cs:270.673,-57.008064) --  (axis cs:271.673,-56.995915) --  (axis cs:272.672,-56.983770) --  (axis cs:273.671,-56.971632) --  (axis cs:274.67,-56.959504) --  (axis cs:275.669,-56.947391) --  (axis cs:276.667,-56.935295) --  (axis cs:277.665,-56.923221) --  (axis cs:278.662,-56.911173) --  (axis cs:279.66,-56.899153) --  (axis cs:280.656,-56.887166) --  (axis cs:281.653,-56.875215) --  (axis cs:282.649,-56.863304) --  (axis cs:283.645,-56.851436) --  (axis cs:284.641,-56.839615) --  (axis cs:285.636,-56.827845) --  (axis cs:286.631,-56.816129) --  (axis cs:287.626,-56.804470) --  (axis cs:288.62,-56.792872) --  (axis cs:289.614,-56.781339) --  (axis cs:290.608,-56.769874) --  (axis cs:291.602,-56.758480) --  (axis cs:292.595,-56.747161) --  (axis cs:293.588,-56.735920) --  (axis cs:294.58,-56.724761) --  (axis cs:295.573,-56.713686) --  (axis cs:296.565,-56.702700) --  (axis cs:297.556,-56.691805) --  (axis cs:298.548,-56.681005) --  (axis cs:299.539,-56.670303) --  (axis cs:300.53,-56.659701) --  (axis cs:301.521,-56.649204) --  (axis cs:302.511,-56.638814) --  (axis cs:303.501,-56.628535) --  (axis cs:304.491,-56.618368) --  (axis cs:305.48,-56.608319) --  (axis cs:306.47,-56.598388) --  (axis cs:307.459,-56.588580) ;
  \draw[constellation-boundary]  (axis cs:255.725,-67.690579) --  (axis cs:256.229,-67.684707) --  (axis cs:257.236,-67.672923) --  (axis cs:258.242,-67.661088) --  (axis cs:259.249,-67.649208) --  (axis cs:260.254,-67.637284) --  (axis cs:261.259,-67.625322) --  (axis cs:262.264,-67.613324) --  (axis cs:263.268,-67.601294) --  (axis cs:264.271,-67.589237) --  (axis cs:265.274,-67.577155) --  (axis cs:266.277,-67.565053) --  (axis cs:267.279,-67.552935) --  (axis cs:268.28,-67.540803) --  (axis cs:269.281,-67.528662) --  (axis cs:270.282,-67.516515) --  (axis cs:271.282,-67.504366) --  (axis cs:272.281,-67.492219) --  (axis cs:273.28,-67.480078) ;
  \draw[constellation-boundary]  (axis cs:255.725,-67.690579) --  (axis cs:255.685,-67.190809) --  (axis cs:255.611,-66.191242) --  (axis cs:255.542,-65.191642) ;
  \draw[constellation-boundary]  (axis cs:254.283,-65.206248) --  (axis cs:255.039,-65.197495) --  (axis cs:255.542,-65.191642) ;
  \draw[constellation-boundary]  (axis cs:254.283,-65.206248) --  (axis cs:254.22,-64.206614) --  (axis cs:254.195,-63.790092) ;
  \draw[constellation-boundary]  (axis cs:251.676,-63.818997) --  (axis cs:251.928,-63.816125) --  (axis cs:252.936,-63.804596) --  (axis cs:253.943,-63.793000) --  (axis cs:254.195,-63.790092) ;
  \draw[constellation-boundary]  (axis cs:251.676,-63.818997) --  (axis cs:251.643,-63.235854) --  (axis cs:251.589,-62.236163) --  (axis cs:251.538,-61.236452) ;
  \draw[constellation-boundary]  (axis cs:249.082,-61.264189) --  (axis cs:249.775,-61.256411) --  (axis cs:250.782,-61.245034) --  (axis cs:251.538,-61.236452) ;
  \draw[constellation-boundary]  (axis cs:249.082,-61.264189) --  (axis cs:249.035,-60.264452) --  (axis cs:248.991,-59.264699) --  (axis cs:248.949,-58.264932) --  (axis cs:248.91,-57.265152) --  (axis cs:248.872,-56.265361) --  (axis cs:248.837,-55.265559) --  (axis cs:248.803,-54.265747) --  (axis cs:248.771,-53.265926) --  (axis cs:248.741,-52.266097) --  (axis cs:248.711,-51.266261) --  (axis cs:248.683,-50.266418) --  (axis cs:248.657,-49.266568) --  (axis cs:248.631,-48.266712) --  (axis cs:248.606,-47.266851) --  (axis cs:248.582,-46.266984) --  (axis cs:248.559,-45.267113) --  (axis cs:248.537,-44.267237) --  (axis cs:248.516,-43.267357) --  (axis cs:248.495,-42.267474) ;
  \draw[constellation-boundary]  (axis cs:301.916,-27.641919) --  (axis cs:301.903,-26.641988) --  (axis cs:301.89,-25.642055) ;
  \draw[constellation-boundary]  (axis cs:301.916,-27.641919) --  (axis cs:302.913,-27.631573) --  (axis cs:303.909,-27.621340) --  (axis cs:304.905,-27.611221) --  (axis cs:305.902,-27.601222) --  (axis cs:306.898,-27.591344) --  (axis cs:307.894,-27.581590) --  (axis cs:308.89,-27.571963) --  (axis cs:309.886,-27.562467) --  (axis cs:310.882,-27.553104) --  (axis cs:311.878,-27.543877) --  (axis cs:312.874,-27.534789) --  (axis cs:313.869,-27.525842) --  (axis cs:314.865,-27.517039) --  (axis cs:315.86,-27.508384) --  (axis cs:316.856,-27.499877) --  (axis cs:317.851,-27.491523) --  (axis cs:318.846,-27.483323) --  (axis cs:319.841,-27.475279) --  (axis cs:320.837,-27.467396) --  (axis cs:321.832,-27.459674) ;
  \draw[constellation-boundary]  (axis cs:180.596,-64.696092) --  (axis cs:181.621,-64.696026) --  (axis cs:182.647,-64.695744) --  (axis cs:183.673,-64.695244) --  (axis cs:184.698,-64.694526) --  (axis cs:185.724,-64.693592) --  (axis cs:186.749,-64.692441) --  (axis cs:187.775,-64.691074) --  (axis cs:188.8,-64.689491) --  (axis cs:189.825,-64.687693) --  (axis cs:190.851,-64.685680) --  (axis cs:191.876,-64.683453) --  (axis cs:192.901,-64.681013) --  (axis cs:193.926,-64.678360) --  (axis cs:194.951,-64.675495) --  (axis cs:195.976,-64.672420) --  (axis cs:197,-64.669135) --  (axis cs:198.025,-64.665642) --  (axis cs:199.049,-64.661941) --  (axis cs:200.073,-64.658034) --  (axis cs:201.097,-64.653922) --  (axis cs:202.121,-64.649606) --  (axis cs:203.145,-64.645088) --  (axis cs:204.169,-64.640369) --  (axis cs:205.192,-64.635451) --  (axis cs:206.215,-64.630335) --  (axis cs:207.238,-64.625024) --  (axis cs:208.261,-64.619518) --  (axis cs:209.283,-64.613820) --  (axis cs:210.306,-64.607931) --  (axis cs:211.328,-64.601854) --  (axis cs:212.35,-64.595590) --  (axis cs:213.371,-64.589141) --  (axis cs:214.393,-64.582510) --  (axis cs:215.414,-64.575699) --  (axis cs:216.434,-64.568709) --  (axis cs:217.455,-64.561544) --  (axis cs:218.475,-64.554205) --  (axis cs:219.495,-64.546694) --  (axis cs:220.515,-64.539015) ;
  \draw[constellation-boundary]  (axis cs:180.597,-55.696092) --  (axis cs:181.615,-55.696027) --  (axis cs:182.633,-55.695746) --  (axis cs:183.651,-55.695250) --  (axis cs:184.668,-55.694538) --  (axis cs:185.686,-55.693611) --  (axis cs:186.704,-55.692469) --  (axis cs:187.721,-55.691113) --  (axis cs:188.739,-55.689542) --  (axis cs:189.756,-55.687757) --  (axis cs:190.774,-55.685759) --  (axis cs:191.791,-55.683549) --  (axis cs:192.809,-55.681127) --  (axis cs:193.826,-55.678494) --  (axis cs:194.335,-55.677099) ;
  \draw[constellation-boundary]  (axis cs:194.438,-64.676954) --  (axis cs:194.424,-63.676975) --  (axis cs:194.41,-62.676994) --  (axis cs:194.397,-61.677012) --  (axis cs:194.385,-60.677028) --  (axis cs:194.374,-59.677044) --  (axis cs:194.363,-58.677059) --  (axis cs:194.353,-57.677073) --  (axis cs:194.344,-56.677086) --  (axis cs:194.335,-55.677099) ;
  \draw[constellation-boundary]  (axis cs:220.515,-64.539015) --  (axis cs:220.475,-63.539166) --  (axis cs:220.438,-62.539307) --  (axis cs:220.404,-61.539439) --  (axis cs:220.371,-60.539563) --  (axis cs:220.341,-59.539679) --  (axis cs:220.312,-58.539789) --  (axis cs:220.285,-57.539892) --  (axis cs:220.259,-56.539990) --  (axis cs:220.234,-55.540083) ;
  \draw[constellation-boundary]  (axis cs:214.657,-55.579949) --  (axis cs:214.636,-54.580017) --  (axis cs:214.617,-53.580081) --  (axis cs:214.599,-52.580143) --  (axis cs:214.581,-51.580202) --  (axis cs:214.564,-50.580258) --  (axis cs:214.548,-49.580312) --  (axis cs:214.532,-48.580364) --  (axis cs:214.517,-47.580414) --  (axis cs:214.503,-46.580462) --  (axis cs:214.489,-45.580508) --  (axis cs:214.476,-44.580553) --  (axis cs:214.463,-43.580596) --  (axis cs:214.45,-42.580638) ;
  \draw[constellation-boundary]  (axis cs:214.657,-55.579949) --  (axis cs:215.164,-55.576542) --  (axis cs:216.179,-55.569597) --  (axis cs:217.193,-55.562475) --  (axis cs:218.207,-55.555181) --  (axis cs:219.221,-55.547716) --  (axis cs:220.234,-55.540083) --  (axis cs:221.248,-55.532284) --  (axis cs:222.261,-55.524321) --  (axis cs:223.274,-55.516197) --  (axis cs:224.287,-55.507914) --  (axis cs:225.3,-55.499476) --  (axis cs:226.312,-55.490885) --  (axis cs:227.324,-55.482143) --  (axis cs:228.336,-55.473254) --  (axis cs:229.348,-55.464220) --  (axis cs:230.36,-55.455044) --  (axis cs:231.371,-55.445728) --  (axis cs:232.382,-55.436277) ;
  \draw[constellation-boundary]  (axis cs:214.45,-42.580638) --  (axis cs:214.955,-42.577249) --  (axis cs:215.964,-42.570341) --  (axis cs:216.973,-42.563257) --  (axis cs:217.982,-42.556000) --  (axis cs:218.991,-42.548574) --  (axis cs:219.999,-42.540979) --  (axis cs:221.008,-42.533218) --  (axis cs:222.016,-42.525294) --  (axis cs:223.024,-42.517209) --  (axis cs:224.032,-42.508966) --  (axis cs:225.04,-42.500567) --  (axis cs:225.796,-42.494168) ;
  \draw[constellation-boundary]  (axis cs:225.796,-42.494168) --  (axis cs:225.781,-41.494233) --  (axis cs:225.766,-40.494297) --  (axis cs:225.751,-39.494359) --  (axis cs:225.737,-38.494419) --  (axis cs:225.723,-37.494478) --  (axis cs:225.71,-36.494535) --  (axis cs:225.697,-35.494590) --  (axis cs:225.684,-34.494644) --  (axis cs:225.672,-33.494697) --  (axis cs:225.66,-32.494749) --  (axis cs:225.648,-31.494799) --  (axis cs:225.636,-30.494849) --  (axis cs:225.625,-29.494897) --  (axis cs:225.614,-28.494945) --  (axis cs:225.603,-27.494991) --  (axis cs:225.592,-26.495037) --  (axis cs:225.582,-25.495082) ;
  \draw[constellation-boundary]  (axis cs:190.417,-30.186382) --  (axis cs:190.669,-30.185868) --  (axis cs:191.676,-30.183680) --  (axis cs:192.683,-30.181284) --  (axis cs:193.69,-30.178678) --  (axis cs:194.697,-30.175864) --  (axis cs:195.703,-30.172843) --  (axis cs:196.71,-30.169615) --  (axis cs:197.717,-30.166183) --  (axis cs:198.723,-30.162546) --  (axis cs:199.73,-30.158706) --  (axis cs:200.737,-30.154664) --  (axis cs:201.743,-30.150421) --  (axis cs:202.75,-30.145980) --  (axis cs:203.756,-30.141340) --  (axis cs:204.762,-30.136504) --  (axis cs:205.769,-30.131472) --  (axis cs:206.775,-30.126248) --  (axis cs:207.781,-30.120831) --  (axis cs:208.788,-30.115225) --  (axis cs:209.794,-30.109430) --  (axis cs:210.8,-30.103448) --  (axis cs:211.806,-30.097282) --  (axis cs:212.812,-30.090933) --  (axis cs:213.817,-30.084403) --  (axis cs:214.823,-30.077694) --  (axis cs:215.829,-30.070808) --  (axis cs:216.835,-30.063748) --  (axis cs:217.84,-30.056516) --  (axis cs:218.846,-30.049113) --  (axis cs:219.851,-30.041542) --  (axis cs:220.856,-30.033806) --  (axis cs:221.862,-30.025907) --  (axis cs:222.867,-30.017846) --  (axis cs:223.872,-30.009628) --  (axis cs:224.877,-30.001254) --  (axis cs:225.882,-29.992727) --  (axis cs:226.887,-29.984050) --  (axis cs:227.892,-29.975224) --  (axis cs:228.896,-29.966254) --  (axis cs:229.901,-29.957141) --  (axis cs:230.905,-29.947889) --  (axis cs:231.91,-29.938501) --  (axis cs:232.914,-29.928978) --  (axis cs:233.918,-29.919325) --  (axis cs:234.922,-29.909544) --  (axis cs:235.926,-29.899639) --  (axis cs:236.93,-29.889612) --  (axis cs:237.934,-29.879466) --  (axis cs:238.937,-29.869204) --  (axis cs:239.941,-29.858830) --  (axis cs:240.944,-29.848347) --  (axis cs:241.948,-29.837759) ;
  \draw[constellation-boundary]  (axis cs:190.427,-33.686372) --  (axis cs:190.424,-32.686375) --  (axis cs:190.422,-31.686377) --  (axis cs:190.419,-30.686380) --  (axis cs:190.417,-30.186382) ;
  \draw[constellation-boundary]  (axis cs:185.387,-33.693884) --  (axis cs:185.639,-33.693635) --  (axis cs:186.647,-33.692504) --  (axis cs:187.655,-33.691160) --  (axis cs:188.663,-33.689604) --  (axis cs:189.671,-33.687836) --  (axis cs:190.427,-33.686372) ;
  \draw[constellation-boundary]  (axis cs:185.39,-35.693883) --  (axis cs:185.389,-34.693884) --  (axis cs:185.387,-33.693884) ;
  \draw[constellation-boundary]  (axis cs:180.581,-83.196091) --  (axis cs:181.682,-83.196021) --  (axis cs:182.784,-83.195717) --  (axis cs:183.886,-83.195180) --  (axis cs:184.987,-83.194410) --  (axis cs:186.089,-83.193407) --  (axis cs:187.19,-83.192171) --  (axis cs:188.291,-83.190703) --  (axis cs:189.391,-83.189004) --  (axis cs:190.492,-83.187074) --  (axis cs:191.592,-83.184914) --  (axis cs:192.691,-83.182525) --  (axis cs:193.79,-83.179908) --  (axis cs:194.889,-83.177064) --  (axis cs:195.987,-83.173994) --  (axis cs:197.085,-83.170699) --  (axis cs:198.182,-83.167181) --  (axis cs:199.278,-83.163441) --  (axis cs:200.374,-83.159481) --  (axis cs:201.469,-83.155301) --  (axis cs:202.563,-83.150905) --  (axis cs:203.657,-83.146293) --  (axis cs:204.749,-83.141468) --  (axis cs:205.841,-83.136431) --  (axis cs:206.932,-83.131185) --  (axis cs:208.022,-83.125731) --  (axis cs:209.111,-83.120071) --  (axis cs:210.199,-83.114208) --  (axis cs:211.286,-83.108144) --  (axis cs:212.372,-83.101882) --  (axis cs:213.457,-83.095424) --  (axis cs:214.541,-83.088772) --  (axis cs:215.624,-83.081929) --  (axis cs:216.705,-83.074898) --  (axis cs:217.786,-83.067681) --  (axis cs:218.865,-83.060281) --  (axis cs:219.943,-83.052701) --  (axis cs:221.019,-83.044944) --  (axis cs:222.094,-83.037013) --  (axis cs:223.168,-83.028911) --  (axis cs:224.241,-83.020640) --  (axis cs:225.312,-83.012205) --  (axis cs:226.382,-83.003607) --  (axis cs:227.451,-82.994851) --  (axis cs:228.518,-82.985940) --  (axis cs:229.584,-82.976876) --  (axis cs:230.648,-82.967663) --  (axis cs:231.71,-82.958306) --  (axis cs:232.772,-82.948806) --  (axis cs:233.831,-82.939168) --  (axis cs:234.889,-82.929394) --  (axis cs:235.946,-82.919489) --  (axis cs:237.001,-82.909457) --  (axis cs:238.055,-82.899299) --  (axis cs:239.106,-82.889021) --  (axis cs:240.157,-82.878626) --  (axis cs:241.205,-82.868118) --  (axis cs:242.252,-82.857500) --  (axis cs:243.298,-82.846776) --  (axis cs:244.342,-82.835949) --  (axis cs:245.384,-82.825024) --  (axis cs:246.424,-82.814005) --  (axis cs:247.463,-82.802894) --  (axis cs:248.5,-82.791696) --  (axis cs:249.536,-82.780415) --  (axis cs:250.57,-82.769054) --  (axis cs:251.602,-82.757617) --  (axis cs:252.633,-82.746109) --  (axis cs:253.661,-82.734532) --  (axis cs:254.689,-82.722891) --  (axis cs:255.714,-82.711190) --  (axis cs:256.738,-82.699432) --  (axis cs:257.76,-82.687622) --  (axis cs:258.781,-82.675762) --  (axis cs:259.799,-82.663858) --  (axis cs:260.816,-82.651913) --  (axis cs:261.832,-82.639930) --  (axis cs:262.846,-82.627913) --  (axis cs:263.858,-82.615867) --  (axis cs:264.868,-82.603795) --  (axis cs:265.877,-82.591701) --  (axis cs:266.884,-82.579588) --  (axis cs:267.889,-82.567461) --  (axis cs:268.893,-82.555323) --  (axis cs:269.896,-82.543178) --  (axis cs:270.896,-82.531029) --  (axis cs:271.895,-82.518881) --  (axis cs:272.892,-82.506737) --  (axis cs:273.888,-82.494601) --  (axis cs:274.882,-82.482476) --  (axis cs:275.875,-82.470366) --  (axis cs:276.866,-82.458275) ;
  \draw[constellation-boundary]  (axis cs:204.707,-65.637869) --  (axis cs:204.68,-64.637935) ;
  \draw[constellation-boundary]  (axis cs:204.707,-65.637869) --  (axis cs:205.22,-65.635383) --  (axis cs:206.244,-65.630262) --  (axis cs:207.268,-65.624945) ;
  \draw[constellation-boundary]  (axis cs:224.004,-68.012204) --  (axis cs:224.77,-68.005922) --  (axis cs:225.791,-67.997410) --  (axis cs:226.557,-67.990925) ;
  \draw[constellation-boundary]  (axis cs:226.557,-67.990925) --  (axis cs:226.527,-67.491052) --  (axis cs:226.472,-66.491290) --  (axis cs:226.42,-65.491509) --  (axis cs:226.372,-64.491713) --  (axis cs:226.353,-64.075126) ;
  \draw[constellation-boundary]  (axis cs:226.353,-64.075126) --  (axis cs:226.608,-64.072953) --  (axis cs:227.625,-64.064168) --  (axis cs:228.642,-64.055235) --  (axis cs:229.658,-64.046157) --  (axis cs:230.166,-64.041565) ;
  \draw[constellation-boundary]  (axis cs:230.166,-64.041565) --  (axis cs:230.139,-63.458354) --  (axis cs:230.096,-62.458554) --  (axis cs:230.054,-61.458740) ;
  \draw[constellation-boundary]  (axis cs:230.054,-61.458740) --  (axis cs:230.562,-61.454120) --  (axis cs:231.576,-61.444777) --  (axis cs:232.59,-61.435298) ;
  \draw[constellation-boundary]  (axis cs:232.59,-61.435298) --  (axis cs:232.55,-60.435486) --  (axis cs:232.512,-59.435663) --  (axis cs:232.477,-58.435829) --  (axis cs:232.444,-57.435987) --  (axis cs:232.412,-56.436136) --  (axis cs:232.382,-55.436277) ;
  \draw[constellation-boundary]  (axis cs:232.55,-60.435486) --  (axis cs:233.563,-60.425880) --  (axis cs:234.576,-60.416143) --  (axis cs:235.588,-60.406280) --  (axis cs:236.6,-60.396292) --  (axis cs:237.612,-60.386185) --  (axis cs:238.623,-60.375959) --  (axis cs:239.635,-60.365620) --  (axis cs:240.645,-60.355170) --  (axis cs:241.656,-60.344612) --  (axis cs:242.666,-60.333949) --  (axis cs:243.676,-60.323186) --  (axis cs:244.685,-60.312325) --  (axis cs:245.694,-60.301370) --  (axis cs:246.703,-60.290324) --  (axis cs:247.712,-60.279191) --  (axis cs:248.72,-60.267974) --  (axis cs:249.035,-60.264452) ;
  \draw[constellation-boundary]  (axis cs:228.083,-55.475490) --  (axis cs:228.057,-54.475608) ;
  \draw[constellation-boundary]  (axis cs:269.625,-37.017462) --  (axis cs:270.126,-37.011388) --  (axis cs:271.125,-36.999240) --  (axis cs:272.125,-36.987092) --  (axis cs:273.125,-36.974950) --  (axis cs:274.124,-36.962816) --  (axis cs:275.124,-36.950694) --  (axis cs:276.123,-36.938588) --  (axis cs:277.122,-36.926502) --  (axis cs:278.121,-36.914440) --  (axis cs:279.119,-36.902404) --  (axis cs:280.118,-36.890399) --  (axis cs:281.116,-36.878428) --  (axis cs:282.115,-36.866495) --  (axis cs:283.113,-36.854604) --  (axis cs:284.11,-36.842757) --  (axis cs:285.108,-36.830960) --  (axis cs:286.106,-36.819214) --  (axis cs:287.103,-36.807525) --  (axis cs:288.101,-36.795895) --  (axis cs:289.098,-36.784328) --  (axis cs:289.596,-36.778569) ;
  \draw[constellation-boundary]  (axis cs:289.77,-45.277569) --  (axis cs:289.758,-44.777636) --  (axis cs:289.736,-43.777765) --  (axis cs:289.714,-42.777891) --  (axis cs:289.693,-41.778012) --  (axis cs:289.672,-40.778130) --  (axis cs:289.652,-39.778245) --  (axis cs:289.633,-38.778356) --  (axis cs:289.614,-37.778464) --  (axis cs:289.596,-36.778569) ;
  \draw[constellation-boundary]  (axis cs:190.405,-25.186395) ;
  \draw[constellation-boundary]  (axis cs:180.6,-25.196092) --  (axis cs:181.605,-25.196028) --  (axis cs:182.611,-25.195750) --  (axis cs:183.617,-25.195260) --  (axis cs:184.622,-25.194557) --  (axis cs:185.628,-25.193641) --  (axis cs:186.634,-25.192512) --  (axis cs:187.639,-25.191171) --  (axis cs:188.645,-25.189619) --  (axis cs:189.65,-25.187856) --  (axis cs:190.405,-25.186395) ;
  \draw[constellation-boundary]  (axis cs:321.928,-36.459303) --  (axis cs:322.921,-36.451761) --  (axis cs:323.914,-36.444385) --  (axis cs:324.906,-36.437177) --  (axis cs:325.899,-36.430140) --  (axis cs:326.892,-36.423276) --  (axis cs:327.884,-36.416586) --  (axis cs:328.876,-36.410073) --  (axis cs:329.869,-36.403738) --  (axis cs:330.861,-36.397584) --  (axis cs:331.853,-36.391612) --  (axis cs:332.845,-36.385824) --  (axis cs:333.837,-36.380222) --  (axis cs:334.829,-36.374807) --  (axis cs:335.821,-36.369582) --  (axis cs:336.813,-36.364547) --  (axis cs:337.804,-36.359704) --  (axis cs:338.796,-36.355055) --  (axis cs:339.788,-36.350601) --  (axis cs:340.779,-36.346343) --  (axis cs:341.771,-36.342282) --  (axis cs:342.762,-36.338421) --  (axis cs:343.754,-36.334759) --  (axis cs:344.745,-36.331298) --  (axis cs:345.736,-36.328039) --  (axis cs:346.727,-36.324983) --  (axis cs:347.719,-36.322131) --  (axis cs:348.71,-36.319484) --  (axis cs:349.701,-36.317042) --  (axis cs:350.692,-36.314806) --  (axis cs:351.683,-36.312778) ;

\draw[constellation-boundary]  (axis cs:322.117,-49.458577) --  (axis cs:322.1,-48.458645) --  (axis cs:322.082,-47.458711) --  (axis cs:322.066,-46.458774) --  (axis cs:322.05,-45.458835) --  (axis cs:322.035,-44.458894) --  (axis cs:322.02,-43.458951) --  (axis cs:322.006,-42.459006) --  (axis cs:321.992,-41.459059) --  (axis cs:321.978,-40.459111) --  (axis cs:321.965,-39.459161) --  (axis cs:321.952,-38.459210) --  (axis cs:321.94,-37.459257) --  (axis cs:321.928,-36.459303) --  (axis cs:321.916,-35.459348) --  (axis cs:321.905,-34.459392) --  (axis cs:321.894,-33.459435) --  (axis cs:321.883,-32.459477) --  (axis cs:321.872,-31.459518) --  (axis cs:321.862,-30.459558) --  (axis cs:321.852,-29.459597) --  (axis cs:321.841,-28.459636) --  (axis cs:321.832,-27.459674) --  (axis cs:321.822,-26.459711) --  (axis cs:321.812,-25.459747) ;


\draw[constellation-boundary]  (axis cs:322.117,-49.458577) --  (axis cs:323.106,-49.451066) --  (axis cs:324.095,-49.443720) --  (axis cs:325.083,-49.436543) --  (axis cs:326.072,-49.429536) --  (axis cs:327.06,-49.422702) --  (axis cs:328.048,-49.416041) --  (axis cs:329.036,-49.409557) --  (axis cs:330.024,-49.403251) --  (axis cs:331.011,-49.397124) --  (axis cs:331.999,-49.391180) ;
  \draw[constellation-boundary] (axis cs:332.114, -56.390839) -- (axis
  cs:333.098, -56.385099) -- (axis cs:334.081, -56.379544) -- (axis
  cs:335.065, -56.374175) -- (axis cs:336.048, -56.368994) -- (axis
  cs:337.031, -56.364002) -- (axis cs:338.015, -56.359201) -- (axis
  cs:338.998, -56.354592) -- (axis cs:339.981, -56.350177) -- (axis
  cs:340.963, -56.345956) -- (axis cs:341.946, -56.341932) -- (axis
  cs:342.929, -56.338105) -- (axis cs:343.911, -56.334476) -- (axis
  cs:344.894, -56.331046) -- (axis cs:345.876, -56.327816) -- (axis
  cs:346.858, -56.324788) -- (axis cs:347.84, -56.321962) -- (axis cs:348.823,
  -56.319339) -- (axis cs:349.805, -56.316920) -- (axis cs:350.787,
  -56.314705) -- (axis cs:351.769, -56.312695) ; \draw[constellation-boundary]
  (axis cs:351.693, -39.312768) -- (axis cs:352.683, -39.310949) -- (axis
  cs:353.673, -39.309338) -- (axis cs:354.663, -39.307934) -- (axis
  cs:355.653, -39.306740) -- (axis cs:356.643, -39.305754) -- (axis
  cs:357.633, -39.304978) -- (axis cs:358.623, -39.304412) -- (axis
  cs:359.613, -39.304055) -- (axis cs:360.603, -39.303908) -- (axis
  cs:361.593, -39.303971) -- (axis cs:362.583, -39.304244) -- (axis
  cs:363.573, -39.304727) -- (axis cs:364.563, -39.305419)
  ; \draw[constellation-boundary] (axis cs:346.727, -36.324983) -- (axis
  cs:346.723, -35.324990) -- (axis cs:346.719, -34.324996) -- (axis
  cs:346.714, -33.325003) -- (axis cs:346.71, -32.325009) -- (axis cs:346.706,
  -31.325015) -- (axis cs:346.702, -30.325021) -- (axis cs:346.698,
  -29.325027) -- (axis cs:346.694, -28.325033) -- (axis cs:346.69, -27.325039)
  -- (axis cs:346.686, -26.325044) -- (axis cs:346.683, -25.325050)
  ; \draw[constellation-boundary] (axis cs:215.534, -25.072697) -- (axis
  cs:215.785, -25.070961) -- (axis cs:216.789, -25.063909) -- (axis
  cs:217.794, -25.056684) -- (axis cs:218.798, -25.049289) -- (axis
  cs:219.803, -25.041727) -- (axis cs:220.807, -25.033998) -- (axis
  cs:221.811, -25.026107) -- (axis cs:222.815, -25.018055) -- (axis
  cs:223.819, -25.009845) -- (axis cs:224.824, -25.001479)
  ; \draw[constellation-boundary] (axis cs:215.534, -25.072697)
  ; \draw[constellation-boundary] (axis cs:361.533, -81.803966) -- (axis
  cs:362.449, -81.804219) -- (axis cs:363.365, -81.804665) -- (axis cs:364.28,
  -81.805305) ; \draw[constellation-boundary] (axis cs:361.533, -81.803966) --
  (axis cs:361.537, -81.303966) -- (axis cs:361.544, -80.303967) -- (axis
  cs:361.549, -79.303968) -- (axis cs:361.554, -78.303968) -- (axis
  cs:361.558, -77.303968) -- (axis cs:361.561, -76.303968) -- (axis
  cs:361.564, -75.303969) -- (axis cs:361.566, -74.303969)
  ; \draw[constellation-boundary] (axis cs:323.185, -74.454478) -- (axis
  cs:323.084, -73.454865) -- (axis cs:322.995, -72.455208) -- (axis
  cs:322.915, -71.455516) -- (axis cs:322.842, -70.455793) -- (axis
  cs:322.777, -69.456044) -- (axis cs:322.717, -68.456273) -- (axis
  cs:322.663, -67.456482) -- (axis cs:322.613, -66.456674) -- (axis
  cs:322.567, -65.456852) -- (axis cs:322.524, -64.457017) -- (axis
  cs:322.484, -63.457170) -- (axis cs:322.447, -62.457312) -- (axis
  cs:322.412, -61.457446) -- (axis cs:322.379, -60.457571) -- (axis
  cs:322.349, -59.457689) ; \draw[constellation-boundary] (axis cs:351.998,
  -74.312472) -- (axis cs:351.973, -73.312496) -- (axis cs:351.951,
  -72.312518) -- (axis cs:351.931, -71.312537) -- (axis cs:351.913,
  -70.312555) -- (axis cs:351.896, -69.312571) -- (axis cs:351.881,
  -68.312585) -- (axis cs:351.868, -67.312598) -- (axis cs:351.861,
  -66.812605) ; \draw[constellation-boundary] (axis cs:332.399, -66.889995) --
  (axis cs:332.38, -66.390051) -- (axis cs:332.344, -65.390156) -- (axis
  cs:332.311, -64.390254) -- (axis cs:332.281, -63.390345) -- (axis
  cs:332.252, -62.390429) -- (axis cs:332.225, -61.390508) -- (axis cs:332.2,
  -60.390583) -- (axis cs:332.177, -59.390652) -- (axis cs:332.155,
  -58.390718) -- (axis cs:332.134, -57.390780) -- (axis cs:332.114,
  -56.390839) -- (axis cs:332.095, -55.390895) -- (axis cs:332.077,
  -54.390948) -- (axis cs:332.06, -53.390999) -- (axis cs:332.044, -52.391047)
  -- (axis cs:332.028, -51.391093) -- (axis cs:332.013, -50.391137) -- (axis
  cs:331.999, -49.391180) ; \draw[constellation-boundary] (axis cs:332.399,
  -66.889995) -- (axis cs:333.373, -66.884308) -- (axis cs:334.348,
  -66.878804) -- (axis cs:335.322, -66.873485) -- (axis cs:336.296,
  -66.868352) -- (axis cs:337.27, -66.863407) -- (axis cs:338.244, -66.858652)
  -- (axis cs:339.218, -66.854087) -- (axis cs:340.191, -66.849714) -- (axis
  cs:341.164, -66.845535) -- (axis cs:342.137, -66.841550) -- (axis cs:343.11,
  -66.837760) -- (axis cs:344.083, -66.834167) -- (axis cs:345.056,
  -66.830771) -- (axis cs:346.028, -66.827574) -- (axis cs:347.001,
  -66.824576) -- (axis cs:347.973, -66.821778) -- (axis cs:348.945,
  -66.819181) -- (axis cs:349.917, -66.816787) -- (axis cs:350.889,
  -66.814594) -- (axis cs:351.861, -66.812605) ; \draw[constellation-boundary]
  (axis cs:307.565, -59.588058) -- (axis cs:307.528, -58.588242) -- (axis
  cs:307.492, -57.588416) -- (axis cs:307.459, -56.588580) -- (axis
  cs:307.427, -55.588736) -- (axis cs:307.397, -54.588884) -- (axis
  cs:307.368, -53.589026) -- (axis cs:307.341, -52.589160) -- (axis
  cs:307.315, -51.589289) -- (axis cs:307.29, -50.589412) -- (axis cs:307.266,
  -49.589530) -- (axis cs:307.243, -48.589643) -- (axis cs:307.221,
  -47.589752) -- (axis cs:307.2, -46.589857) -- (axis cs:307.179, -45.589958)
  -- (axis cs:307.16, -44.590055) -- (axis cs:307.14, -43.590149) -- (axis
  cs:307.122, -42.590240) -- (axis cs:307.104, -41.590328) -- (axis
  cs:307.087, -40.590414) -- (axis cs:307.07, -39.590497) -- (axis cs:307.054,
  -38.590577) -- (axis cs:307.038, -37.590656) -- (axis cs:307.022,
  -36.590732) -- (axis cs:307.007, -35.590806) -- (axis cs:306.992,
  -34.590879) -- (axis cs:306.978, -33.590950) -- (axis cs:306.964,
  -32.591019) -- (axis cs:306.95, -31.591086) -- (axis cs:306.937, -30.591153)
  -- (axis cs:306.924, -29.591218) -- (axis cs:306.911, -28.591281) -- (axis
  cs:306.898, -27.591344) ; \draw[constellation-boundary] (axis cs:307.565,
  -59.588058) -- (axis cs:308.552, -59.578389) -- (axis cs:309.539,
  -59.568847) -- (axis cs:310.526, -59.559437) -- (axis cs:311.513,
  -59.550160) -- (axis cs:312.499, -59.541020) -- (axis cs:313.485,
  -59.532019) -- (axis cs:314.471, -59.523160) -- (axis cs:315.457,
  -59.514445) -- (axis cs:316.442, -59.505877) -- (axis cs:317.427,
  -59.497458) -- (axis cs:318.412, -59.489191) -- (axis cs:319.396,
  -59.481078) -- (axis cs:320.381, -59.473122) -- (axis cs:321.365,
  -59.465325) -- (axis cs:322.349, -59.457689) ; \draw[constellation-boundary]
  (axis cs:236.93, -29.889612) -- (axis cs:236.923, -29.389645) -- (axis
  cs:236.91, -28.389711) -- (axis cs:236.897, -27.389775) -- (axis cs:236.885,
  -26.389839) -- (axis cs:236.872, -25.389901) ; \draw[constellation-boundary]
  (axis cs:228.057, -54.475608) -- (axis cs:228.31, -54.473373) -- (axis
  cs:229.321, -54.464343) -- (axis cs:230.332, -54.455170) -- (axis
  cs:231.343, -54.445859) -- (axis cs:232.353, -54.436411)
  ; \draw[constellation-boundary] (axis cs:232.353, -54.436411) -- (axis
  cs:232.326, -53.436539) -- (axis cs:232.3, -52.436661) -- (axis cs:232.276,
  -51.436778) -- (axis cs:232.252, -50.436890) -- (axis cs:232.229,
  -49.436997) -- (axis cs:232.207, -48.437099) ; \draw[constellation-boundary]
  (axis cs:232.207, -48.437099) -- (axis cs:233.216, -48.427538) -- (axis
  cs:234.224, -48.417846) -- (axis cs:235.232, -48.408026) -- (axis cs:236.24,
  -48.398082) -- (axis cs:237.247, -48.388017) ; \draw[constellation-boundary]
  (axis cs:237.247, -48.388017) -- (axis cs:237.225, -47.388130) -- (axis
  cs:237.203, -46.388238) -- (axis cs:237.183, -45.388342) -- (axis
  cs:237.163, -44.388443) -- (axis cs:237.143, -43.388540) -- (axis
  cs:237.124, -42.388634) ; \draw[constellation-boundary] (axis cs:237.124,
  -42.388634) -- (axis cs:238.13, -42.378465) -- (axis cs:239.136, -42.368181)
  -- (axis cs:240.142, -42.357785) -- (axis cs:241.147, -42.347281) -- (axis
  cs:242.153, -42.336671) -- (axis cs:243.158, -42.325959) -- (axis
  cs:244.163, -42.315148) -- (axis cs:245.168, -42.304241) -- (axis
  cs:246.172, -42.293242) -- (axis cs:247.177, -42.282155) -- (axis
  cs:248.181, -42.270982) -- (axis cs:248.495, -42.267474)
  ; \draw[constellation-boundary] (axis cs:242.153, -42.336671) -- (axis
  cs:242.134, -41.336772) -- (axis cs:242.115, -40.336870) -- (axis
  cs:242.097, -39.336966) -- (axis cs:242.08, -38.337059) -- (axis cs:242.063,
  -37.337149) -- (axis cs:242.046, -36.337237) -- (axis cs:242.03, -35.337322)
  -- (axis cs:242.014, -34.337406) -- (axis cs:241.999, -33.337487) -- (axis
  cs:241.984, -32.337567) -- (axis cs:241.969, -31.337645) -- (axis
  cs:241.955, -30.337721) -- (axis cs:241.948, -29.837759)
  ; \draw[constellation-boundary] (axis cs:180.591, -75.696092) -- (axis
  cs:181.639, -75.696025) -- (axis cs:182.687, -75.695736) -- (axis
  cs:183.734, -75.695225) -- (axis cs:184.782, -75.694493) -- (axis
  cs:185.829, -75.693539) -- (axis cs:186.876, -75.692363) -- (axis
  cs:187.924, -75.690967) -- (axis cs:188.971, -75.689350) -- (axis
  cs:190.018, -75.687514) -- (axis cs:191.065, -75.685459) -- (axis
  cs:192.111, -75.683185) -- (axis cs:193.158, -75.680694) -- (axis
  cs:194.204, -75.677985) -- (axis cs:195.25, -75.675062) -- (axis cs:196.296,
  -75.671923) -- (axis cs:197.342, -75.668570) -- (axis cs:198.387,
  -75.665006) -- (axis cs:199.432, -75.661229) -- (axis cs:200.477,
  -75.657243) -- (axis cs:201.522, -75.653049) -- (axis cs:202.566,
  -75.648647) -- (axis cs:203.61, -75.644040) -- (axis cs:204.653, -75.639228)
  -- (axis cs:205.696, -75.634215) -- (axis cs:206.739, -75.629000) -- (axis
  cs:207.781, -75.623587) ; \draw[constellation-boundary] (axis cs:266.002,
  -30.060661) -- (axis cs:265.986, -29.060758) -- (axis cs:265.97, -28.060853)
  -- (axis cs:265.955, -27.060946) -- (axis cs:265.94, -26.061037) -- (axis
  cs:265.925, -25.061127) ; \draw[constellation-boundary] (axis cs:253.235,
  -30.212305) -- (axis cs:253.22, -29.212394) -- (axis cs:253.205, -28.212481)
  -- (axis cs:253.19, -27.212566) -- (axis cs:253.176, -26.212650) -- (axis
  cs:253.162, -25.212732) ; \draw[constellation-boundary] (axis cs:253.235,
  -30.212305) -- (axis cs:253.987, -30.203601) -- (axis cs:254.989,
  -30.191941) -- (axis cs:255.991, -30.180224) -- (axis cs:256.992,
  -30.168452) -- (axis cs:257.994, -30.156629) -- (axis cs:258.995,
  -30.144760) -- (axis cs:259.997, -30.132847) -- (axis cs:260.998,
  -30.120894) -- (axis cs:261.999, -30.108905) -- (axis cs:263, -30.096883) --
  (axis cs:264.001, -30.084833) -- (axis cs:265.001, -30.072758) -- (axis
  cs:266.002, -30.060661) -- (axis cs:267.002, -30.048547) -- (axis
  cs:268.003, -30.036418) -- (axis cs:269.003, -30.024279) -- (axis
  cs:269.503, -30.018207) ; \draw[constellation-boundary] (axis cs:351.778,
  -57.812685) -- (axis cs:352.759, -57.810883) -- (axis cs:353.74, -57.809286)
  -- (axis cs:354.721, -57.807896) -- (axis cs:355.702, -57.806713) -- (axis
  cs:356.682, -57.805737) -- (axis cs:357.663, -57.804968) -- (axis
  cs:358.644, -57.804407) -- (axis cs:359.624, -57.804054) -- (axis
  cs:360.605, -57.803908) -- (axis cs:361.585, -57.803971) -- (axis
  cs:362.566, -57.804241) -- (axis cs:363.547, -57.804719) -- (axis
  cs:364.527, -57.805405) ; \draw[constellation-boundary] (axis cs:351.778,
  -57.812685) -- (axis cs:351.775, -57.312689) -- (axis cs:351.769,
  -56.312695) -- (axis cs:351.762, -55.312701) -- (axis cs:351.757,
  -54.312706) -- (axis cs:351.751, -53.312712) -- (axis cs:351.746,
  -52.312717) -- (axis cs:351.741, -51.312722) -- (axis cs:351.736,
  -50.312727) -- (axis cs:351.731, -49.312731) -- (axis cs:351.727,
  -48.312736) -- (axis cs:351.722, -47.312740) -- (axis cs:351.718,
  -46.312744) -- (axis cs:351.714, -45.312748) -- (axis cs:351.71, -44.312751)
  -- (axis cs:351.707, -43.312755) -- (axis cs:351.703, -42.312759) -- (axis
  cs:351.699, -41.312762) -- (axis cs:351.696, -40.312765) -- (axis
  cs:351.693, -39.312768) -- (axis cs:351.69, -38.312772) -- (axis cs:351.686,
  -37.312775) -- (axis cs:351.683, -36.312778) ; \draw[constellation-boundary]
  (axis cs:141.734, -40.291866) -- (axis cs:141.744, -39.541903) -- (axis
  cs:141.756, -38.541950) -- (axis cs:141.769, -37.541997) -- (axis cs:141.78,
  -36.542042) -- (axis cs:141.792, -35.542086) -- (axis cs:141.803,
  -34.542129) -- (axis cs:141.814, -33.542171) -- (axis cs:141.825,
  -32.542212) -- (axis cs:141.836, -31.542253) -- (axis cs:141.846,
  -30.542292) -- (axis cs:141.856, -29.542330) -- (axis cs:141.866,
  -28.542368) -- (axis cs:141.876, -27.542405) -- (axis cs:141.886,
  -26.542442) -- (axis cs:141.895, -25.542477) ; \draw[constellation-boundary]
  (axis cs:141.734, -40.291866) -- (axis cs:142.238, -40.295673) -- (axis
  cs:143.246, -40.303162) -- (axis cs:144.254, -40.310482) -- (axis
  cs:145.262, -40.317630) -- (axis cs:146.271, -40.324605) -- (axis
  cs:147.279, -40.331403) -- (axis cs:148.288, -40.338022) -- (axis
  cs:149.297, -40.344462) -- (axis cs:150.305, -40.350719) -- (axis
  cs:151.314, -40.356791) -- (axis cs:152.323, -40.362678) -- (axis
  cs:153.332, -40.368376) -- (axis cs:154.341, -40.373885) -- (axis
  cs:155.351, -40.379202) -- (axis cs:156.36, -40.384326) -- (axis cs:157.369,
  -40.389254) -- (axis cs:158.379, -40.393987) -- (axis cs:159.388,
  -40.398522) -- (axis cs:160.398, -40.402857) -- (axis cs:161.407,
  -40.406992) -- (axis cs:162.417, -40.410925) -- (axis cs:163.427,
  -40.414655) -- (axis cs:164.437, -40.418180) -- (axis cs:165.447,
  -40.421500) -- (axis cs:166.456, -40.424613) ; \draw[constellation-boundary]
  (axis cs:163.959, -35.666490) -- (axis cs:163.964, -34.666499) -- (axis
  cs:163.969, -33.666508) -- (axis cs:163.974, -32.666516) -- (axis
  cs:163.978, -31.833190) ; \draw[constellation-boundary] (axis cs:163.959,
  -35.666490) -- (axis cs:164.463, -35.668224) -- (axis cs:165.471,
  -35.671539) -- (axis cs:166.479, -35.674647) -- (axis cs:167.488,
  -35.677549) -- (axis cs:168.496, -35.680242) -- (axis cs:169.505,
  -35.682726) -- (axis cs:170.513, -35.685001) -- (axis cs:171.522,
  -35.687065) -- (axis cs:172.53, -35.688919) -- (axis cs:173.539, -35.690560)
  -- (axis cs:174.547, -35.691990) -- (axis cs:175.556, -35.693207) -- (axis
  cs:176.565, -35.694211) -- (axis cs:177.573, -35.695001) -- (axis
  cs:178.582, -35.695579) -- (axis cs:179.591, -35.695942)
  ; \draw[constellation-boundary] (axis cs:160.201, -31.818578) -- (axis
  cs:160.453, -31.819640) -- (axis cs:161.46, -31.823764) -- (axis cs:162.467,
  -31.827687) -- (axis cs:163.474, -31.831407) -- (axis cs:163.978,
  -31.833190) ; \draw[constellation-boundary] (axis cs:160.201, -31.818578) --
  (axis cs:160.202, -31.651913) -- (axis cs:160.208, -30.651926) -- (axis
  cs:160.213, -29.818602) ; \draw[constellation-boundary] (axis cs:155.181,
  -29.794774) -- (axis cs:155.433, -29.796081) -- (axis cs:156.439,
  -29.801190) -- (axis cs:157.445, -29.806104) -- (axis cs:158.452,
  -29.810822) -- (axis cs:159.458, -29.815342) -- (axis cs:160.213,
  -29.818602) ; \draw[constellation-boundary] (axis cs:155.181, -29.794774) --
  (axis cs:155.182, -29.628110) -- (axis cs:155.189, -28.628128) -- (axis
  cs:155.196, -27.628146) -- (axis cs:155.199, -27.128155)
  ; \draw[constellation-boundary] (axis cs:147.659, -27.083497) -- (axis
  cs:148.413, -27.088428) -- (axis cs:149.418, -27.094845) -- (axis
  cs:150.424, -27.101080) -- (axis cs:151.429, -27.107131) -- (axis
  cs:152.434, -27.112997) -- (axis cs:153.44, -27.118675) -- (axis cs:154.445,
  -27.124163) -- (axis cs:155.199, -27.128155) ; \draw[constellation-boundary]
  (axis cs:147.659, -27.083497) -- (axis cs:147.663, -26.583510) -- (axis
  cs:147.672, -25.583537) ; \draw[constellation-boundary] (axis cs:-4.174,
  -74.306645) -- (axis cs:-3.217, -74.305692) -- (axis cs:-2.261, -74.304942)
  -- (axis cs:-1.304, -74.304395) -- (axis cs:-0.347, -74.304050) -- (axis
  cs:0.609585, -74.303908) -- (axis cs:1.56628, -74.303969) -- (axis
  cs:2.52298, -74.304233) -- (axis cs:3.47971, -74.304699) -- (axis
  cs:4.43646, -74.305368) -- (axis cs:5.39327, -74.306240) -- (axis
  cs:6.35012, -74.307314) -- (axis cs:7.30704, -74.308590) -- (axis
  cs:8.26404, -74.310067) -- (axis cs:9.22113, -74.311746) -- (axis
  cs:10.1783, -74.313625) -- (axis cs:11.1356, -74.315704) -- (axis cs:12.093,
  -74.317984) -- (axis cs:12.3324, -74.318585) ; \draw[constellation-boundary]
  (axis cs:64.8823, -48.699667) -- (axis cs:64.9071, -47.699803) -- (axis
  cs:64.9309, -46.699934) -- (axis cs:64.9538, -45.700060) -- (axis cs:64.976,
  -44.700182) -- (axis cs:64.9974, -43.700299) -- (axis cs:65.0181,
  -42.700413) -- (axis cs:65.0381, -41.700523) -- (axis cs:65.0575,
  -40.700630) -- (axis cs:65.0764, -39.700734) -- (axis cs:65.0947,
  -38.700834) -- (axis cs:65.1125, -37.700932) -- (axis cs:65.1298,
  -36.701027) ; \draw[constellation-boundary] (axis cs:68.4241, -46.238801) --
  (axis cs:68.9217, -46.244428) -- (axis cs:69.917, -46.255741) -- (axis
  cs:70.9125, -46.267128) -- (axis cs:71.9082, -46.278588) -- (axis
  cs:72.9041, -46.290116) -- (axis cs:73.4021, -46.295905)
  ; \draw[constellation-boundary] (axis cs:73.4021, -46.295905) -- (axis
  cs:73.4142, -45.795975) -- (axis cs:73.4377, -44.796112) -- (axis
  cs:73.4604, -43.796244) -- (axis cs:73.4824, -42.796372)
  ; \draw[constellation-boundary] (axis cs:73.4824, -42.796372) -- (axis
  cs:73.9807, -42.802179) -- (axis cs:74.9775, -42.813838) -- (axis
  cs:75.9744, -42.825554) -- (axis cs:76.9716, -42.837325) -- (axis
  cs:77.9689, -42.849147) -- (axis cs:78.9664, -42.861015) -- (axis
  cs:79.9641, -42.872927) -- (axis cs:80.962, -42.884878) -- (axis cs:81.9601,
  -42.896866) -- (axis cs:82.9584, -42.908886) -- (axis cs:83.9569,
  -42.920935) -- (axis cs:84.9556, -42.933010) -- (axis cs:85.9544,
  -42.945105) -- (axis cs:86.9535, -42.957219) -- (axis cs:87.9528,
  -42.969347) -- (axis cs:88.9522, -42.981486) -- (axis cs:89.9519,
  -42.993631) -- (axis cs:90.9518, -43.005780) -- (axis cs:91.9518,
  -43.017927) -- (axis cs:92.9521, -43.030071) -- (axis cs:93.9525,
  -43.042207) -- (axis cs:94.9532, -43.054331) -- (axis cs:95.954, -43.066440)
  -- (axis cs:96.9551, -43.078529) -- (axis cs:97.9563, -43.090596) -- (axis
  cs:98.9577, -43.102636) -- (axis cs:99.709, -43.111647)
  ; \draw[constellation-boundary] (axis cs:75.9744, -42.825554) -- (axis
  cs:75.996, -41.825681) -- (axis cs:76.0169, -40.825804) -- (axis cs:76.0371,
  -39.825924) -- (axis cs:76.0568, -38.826040) -- (axis cs:76.076, -37.826152)
  -- (axis cs:76.0946, -36.826262) -- (axis cs:76.1127, -35.826369) -- (axis
  cs:76.1304, -34.826473) -- (axis cs:76.1477, -33.826575) -- (axis
  cs:76.1646, -32.826674) -- (axis cs:76.1811, -31.826771) -- (axis
  cs:76.1972, -30.826866) -- (axis cs:76.213, -29.826960) -- (axis cs:76.2285,
  -28.827051) -- (axis cs:76.2437, -27.827140) -- (axis cs:76.2549,
  -27.077206) ; \draw[constellation-boundary] (axis cs:73.7593, -27.047983) --
  (axis cs:73.763, -26.798004) -- (axis cs:73.7775, -25.798089)
  ; \draw[constellation-boundary] (axis cs:71.7633, -27.024881) -- (axis
  cs:72.2623, -27.030632) -- (axis cs:73.2603, -27.042183) -- (axis
  cs:74.2584, -27.053798) -- (axis cs:75.2566, -27.065474) -- (axis
  cs:76.2549, -27.077206) -- (axis cs:77.2533, -27.088992) -- (axis
  cs:78.2519, -27.100827) -- (axis cs:79.2505, -27.112708) -- (axis
  cs:80.2492, -27.124631) -- (axis cs:81.2481, -27.136594) -- (axis cs:82.247,
  -27.148591) -- (axis cs:83.2461, -27.160620) -- (axis cs:84.2452,
  -27.172677) -- (axis cs:85.2445, -27.184757) -- (axis cs:86.2439,
  -27.196859) -- (axis cs:87.2434, -27.208977) -- (axis cs:88.2429,
  -27.221109) -- (axis cs:89.2426, -27.233249) -- (axis cs:90.2425,
  -27.245396) -- (axis cs:91.2424, -27.257545) -- (axis cs:92.2424,
  -27.269692) -- (axis cs:92.9925, -27.278799) ; \draw[constellation-boundary]
  (axis cs:71.7224, -29.774645) -- (axis cs:71.7375, -28.774732) -- (axis
  cs:71.7524, -27.774818) -- (axis cs:71.7633, -27.024881)
  ; \draw[constellation-boundary] (axis cs:69.9767, -29.754662) -- (axis
  cs:70.2261, -29.757503) -- (axis cs:71.2236, -29.768913) -- (axis
  cs:71.7224, -29.774645) ; \draw[constellation-boundary] (axis cs:69.8624,
  -36.754011) -- (axis cs:69.8799, -35.754111) -- (axis cs:69.897, -34.754208)
  -- (axis cs:69.9137, -33.754303) -- (axis cs:69.9299, -32.754396) -- (axis
  cs:69.9459, -31.754486) -- (axis cs:69.9614, -30.754575) -- (axis
  cs:69.9767, -29.754662) ; \draw[constellation-boundary] (axis cs:65.1298,
  -36.701027) -- (axis cs:66.1259, -36.712022) -- (axis cs:67.1221,
  -36.723105) -- (axis cs:68.1184, -36.734273) -- (axis cs:69.1149,
  -36.745523) -- (axis cs:69.8624, -36.754011) ; \draw[constellation-boundary]
  (axis cs:111.677, -33.250465) -- (axis cs:111.693, -32.250556) -- (axis
  cs:111.709, -31.250644) -- (axis cs:111.724, -30.250731) -- (axis
  cs:111.739, -29.250816) -- (axis cs:111.754, -28.250899) -- (axis
  cs:111.768, -27.250981) -- (axis cs:111.783, -26.251062) -- (axis
  cs:111.796, -25.251141) ; \draw[constellation-boundary] (axis cs:92.899,
  -33.028232) -- (axis cs:92.9161, -32.028335) -- (axis cs:92.9328,
  -31.028437) -- (axis cs:92.9491, -30.028536) -- (axis cs:92.9652,
  -29.028633) -- (axis cs:92.9809, -28.028729) -- (axis cs:92.9963,
  -27.028822) -- (axis cs:93.0115, -26.028914) -- (axis cs:93.0264,
  -25.029005) ; \draw[constellation-boundary] (axis cs:92.899, -33.028232) --
  (axis cs:93.1491, -33.031266) -- (axis cs:94.1494, -33.043400) -- (axis
  cs:95.1498, -33.055521) -- (axis cs:96.1504, -33.067627) -- (axis
  cs:97.1512, -33.079712) -- (axis cs:98.152, -33.091774) -- (axis cs:99.153,
  -33.103809) -- (axis cs:100.154, -33.115813) -- (axis cs:101.155,
  -33.127782) -- (axis cs:102.157, -33.139714) -- (axis cs:103.158,
  -33.151603) -- (axis cs:104.16, -33.163447) -- (axis cs:105.162, -33.175242)
  -- (axis cs:106.164, -33.186984) -- (axis cs:107.166, -33.198670) -- (axis
  cs:108.168, -33.210297) -- (axis cs:109.171, -33.221859) -- (axis
  cs:110.173, -33.233355) -- (axis cs:111.176, -33.244780) -- (axis
  cs:111.677, -33.250465) ; \draw[constellation-boundary] (axis cs:166.337,
  -57.174436) -- (axis cs:166.342, -56.674443) -- (axis cs:166.352,
  -55.674457) -- (axis cs:166.361, -54.674471) -- (axis cs:166.37, -53.674484)
  -- (axis cs:166.378, -52.674496) -- (axis cs:166.386, -51.674508) -- (axis
  cs:166.394, -50.674519) -- (axis cs:166.401, -49.674530) -- (axis
  cs:166.408, -48.674541) -- (axis cs:166.414, -47.674551) -- (axis
  cs:166.421, -46.674560) -- (axis cs:166.427, -45.674569) -- (axis
  cs:166.433, -44.674578) -- (axis cs:166.439, -43.674587) -- (axis
  cs:166.445, -42.674595) -- (axis cs:166.45, -41.674603) -- (axis cs:166.455,
  -40.674611) -- (axis cs:166.46, -39.674619) -- (axis cs:166.465, -38.674626)
  -- (axis cs:166.47, -37.674633) -- (axis cs:166.475, -36.674640) -- (axis
  cs:166.479, -35.674647) ; \draw[constellation-boundary] (axis cs:133.324,
  -56.973965) -- (axis cs:133.83, -56.978424) -- (axis cs:134.843, -56.987231)
  -- (axis cs:135.856, -56.995889) -- (axis cs:136.869, -57.004394) -- (axis
  cs:137.883, -57.012746) -- (axis cs:138.896, -57.020939) -- (axis cs:139.91,
  -57.028973) -- (axis cs:140.925, -57.036844) -- (axis cs:141.939,
  -57.044550) -- (axis cs:142.954, -57.052089) -- (axis cs:143.968,
  -57.059457) -- (axis cs:144.983, -57.066654) -- (axis cs:145.999,
  -57.073675) -- (axis cs:147.014, -57.080521) -- (axis cs:148.03, -57.087187)
  -- (axis cs:149.046, -57.093672) -- (axis cs:150.062, -57.099974) -- (axis
  cs:151.078, -57.106090) -- (axis cs:152.094, -57.112020) -- (axis cs:153.11,
  -57.117761) -- (axis cs:154.127, -57.123310) -- (axis cs:155.144,
  -57.128667) -- (axis cs:156.161, -57.133830) -- (axis cs:157.178,
  -57.138797) -- (axis cs:158.195, -57.143566) -- (axis cs:159.213,
  -57.148136) -- (axis cs:160.23, -57.152505) -- (axis cs:161.248, -57.156673)
  -- (axis cs:162.265, -57.160637) -- (axis cs:163.283, -57.164396) -- (axis
  cs:164.301, -57.167950) -- (axis cs:165.319, -57.171297) -- (axis
  cs:166.337, -57.174436) -- (axis cs:167.355, -57.177365) -- (axis
  cs:168.374, -57.180085) -- (axis cs:169.392, -57.182594) -- (axis
  cs:170.156, -57.184336) ; \draw[constellation-boundary] (axis cs:133.324,
  -56.973965) -- (axis cs:133.338, -56.474030) -- (axis cs:133.367,
  -55.474155) -- (axis cs:133.38, -54.974215) ; \draw[constellation-boundary]
  (axis cs:127.567, -54.920469) -- (axis cs:127.82, -54.922898) -- (axis
  cs:128.83, -54.932533) -- (axis cs:129.841, -54.942036) -- (axis cs:130.852,
  -54.951405) -- (axis cs:131.863, -54.960635) -- (axis cs:132.874,
  -54.969724) -- (axis cs:133.38, -54.974215) ; \draw[constellation-boundary]
  (axis cs:127.567, -54.920469) -- (axis cs:127.581, -54.420539) -- (axis
  cs:127.609, -53.420672) ; \draw[constellation-boundary] (axis cs:123.32,
  -53.378214) -- (axis cs:123.824, -53.383322) -- (axis cs:124.833,
  -53.393449) -- (axis cs:125.843, -53.403457) -- (axis cs:126.852,
  -53.413341) -- (axis cs:127.609, -53.420672) ; \draw[constellation-boundary]
  (axis cs:123.32, -53.378214) -- (axis cs:123.348, -52.378356) -- (axis
  cs:123.375, -51.378491) -- (axis cs:123.381, -51.128524)
  ; \draw[constellation-boundary] (axis cs:120.862, -51.102580) -- (axis
  cs:120.881, -50.352682) -- (axis cs:120.906, -49.352813) -- (axis cs:120.93,
  -48.352939) -- (axis cs:120.954, -47.353060) -- (axis cs:120.976,
  -46.353176) -- (axis cs:120.997, -45.353288) -- (axis cs:121.018,
  -44.353397) -- (axis cs:121.038, -43.353501) ; \draw[constellation-boundary]
  (axis cs:90.7489, -50.754547) -- (axis cs:91.7489, -50.766696) -- (axis
  cs:92.7492, -50.778840) -- (axis cs:93.7498, -50.790978) -- (axis
  cs:94.7506, -50.803105) -- (axis cs:95.7517, -50.815217) -- (axis cs:96.753,
  -50.827311) -- (axis cs:97.7546, -50.839382) -- (axis cs:98.7565,
  -50.851428) -- (axis cs:99.7586, -50.863445) -- (axis cs:100.761,
  -50.875428) -- (axis cs:101.764, -50.887375) -- (axis cs:102.767,
  -50.899282) -- (axis cs:103.77, -50.911144) -- (axis cs:104.773, -50.922958)
  -- (axis cs:105.777, -50.934721) -- (axis cs:106.781, -50.946430) -- (axis
  cs:107.785, -50.958079) -- (axis cs:108.789, -50.969666) -- (axis
  cs:109.794, -50.981188) -- (axis cs:110.799, -50.992640) -- (axis
  cs:111.804, -51.004019) -- (axis cs:112.81, -51.015322) -- (axis cs:113.815,
  -51.026545) -- (axis cs:114.821, -51.037685) -- (axis cs:115.827,
  -51.048738) -- (axis cs:116.834, -51.059700) -- (axis cs:117.84, -51.070569)
  -- (axis cs:118.847, -51.081341) -- (axis cs:119.854, -51.092012) -- (axis
  cs:120.862, -51.102580) -- (axis cs:121.869, -51.113041) -- (axis
  cs:122.877, -51.123391) -- (axis cs:123.381, -51.128524)
  ; \draw[constellation-boundary] (axis cs:90.6937, -52.504212) -- (axis
  cs:91.6937, -52.516360) -- (axis cs:92.694, -52.528506) -- (axis cs:93.1943,
  -52.534576) ; \draw[constellation-boundary] (axis cs:90.6937, -52.504212) --
  (axis cs:90.7099, -52.004310) -- (axis cs:90.7413, -51.004501) -- (axis
  cs:90.7713, -50.004683) -- (axis cs:90.8001, -49.004858) -- (axis
  cs:90.8278, -48.005026) -- (axis cs:90.8544, -47.005188) -- (axis cs:90.88,
  -46.005344) -- (axis cs:90.9048, -45.005494) -- (axis cs:90.9287,
  -44.005639) -- (axis cs:90.9518, -43.005780) ; \draw[constellation-boundary]
  (axis cs:93.1073, -55.034048) -- (axis cs:93.1434, -54.034267) -- (axis
  cs:93.1777, -53.034475) -- (axis cs:93.1943, -52.534576)
  ; \draw[constellation-boundary] (axis cs:93.1073, -55.034048) -- (axis
  cs:93.6077, -55.040117) -- (axis cs:94.6086, -55.052245) -- (axis
  cs:95.6099, -55.064360) -- (axis cs:96.6114, -55.076456) -- (axis
  cs:97.6132, -55.088531) -- (axis cs:98.1143, -55.094560)
  ; \draw[constellation-boundary] (axis cs:97.9951, -58.093843) -- (axis
  cs:98.0369, -57.094095) -- (axis cs:98.0766, -56.094333) -- (axis
  cs:98.1143, -55.094560) ; \draw[constellation-boundary] (axis cs:97.9951,
  -58.093843) -- (axis cs:98.4963, -58.099866) -- (axis cs:99.499, -58.111891)
  -- (axis cs:100.502, -58.123883) -- (axis cs:101.505, -58.135840) -- (axis
  cs:102.509, -58.147757) -- (axis cs:103.011, -58.153699)
  ; \draw[constellation-boundary] (axis cs:102.703, -64.151880) -- (axis
  cs:102.763, -63.152235) -- (axis cs:102.819, -62.152566) -- (axis
  cs:102.872, -61.152876) -- (axis cs:102.921, -60.153167) -- (axis
  cs:102.967, -59.153441) -- (axis cs:103.011, -58.153699)
  ; \draw[constellation-boundary] (axis cs:135.244, -75.495465) -- (axis
  cs:135.369, -74.495991) -- (axis cs:135.48, -73.496455) -- (axis cs:135.578,
  -72.496868) -- (axis cs:135.666, -71.497237) -- (axis cs:135.745,
  -70.497569) -- (axis cs:135.817, -69.497871) -- (axis cs:135.882,
  -68.498145) -- (axis cs:135.942, -67.498396) -- (axis cs:135.997,
  -66.498627) -- (axis cs:136.048, -65.498840) -- (axis cs:136.095,
  -64.499037) ; \draw[constellation-boundary] (axis cs:169.857, -75.684007) --
  (axis cs:169.891, -74.684044) -- (axis cs:169.92, -73.684077) -- (axis
  cs:169.947, -72.684106) -- (axis cs:169.97, -71.684132) -- (axis cs:169.991,
  -70.684155) -- (axis cs:170.011, -69.684176) -- (axis cs:170.028,
  -68.684196) -- (axis cs:170.044, -67.684213) -- (axis cs:170.059,
  -66.684229) -- (axis cs:170.072, -65.684244) -- (axis cs:170.085,
  -64.684258) -- (axis cs:170.096, -63.684271) -- (axis cs:170.107,
  -62.684283) -- (axis cs:170.117, -61.684294) -- (axis cs:170.127,
  -60.684304) -- (axis cs:170.136, -59.684314) -- (axis cs:170.144,
  -58.684323) -- (axis cs:170.152, -57.684332) -- (axis cs:170.156,
  -57.184336) ; \draw[constellation-boundary] (axis cs:170.085, -64.684258) --
  (axis cs:170.341, -64.684816) -- (axis cs:171.366, -64.686915) -- (axis
  cs:172.392, -64.688799) -- (axis cs:173.417, -64.690468) -- (axis
  cs:174.442, -64.691921) -- (axis cs:175.468, -64.693158) -- (axis
  cs:176.493, -64.694179) -- (axis cs:177.519, -64.694983) -- (axis
  cs:178.545, -64.695570) -- (axis cs:179.57, -64.695940)
  ; \draw[constellation-boundary] (axis cs:179.057, -64.695782) -- (axis
  cs:179.059, -63.695782) -- (axis cs:179.061, -62.695783) -- (axis
  cs:179.063, -61.695783) -- (axis cs:179.064, -60.695783) -- (axis
  cs:179.066, -59.695783) -- (axis cs:179.067, -58.695784) -- (axis
  cs:179.068, -57.695784) -- (axis cs:179.07, -56.695784) -- (axis cs:179.071,
  -55.695784) ; \draw[constellation-boundary] (axis cs:179.071, -55.695784) --
  (axis cs:179.58, -55.695941) ; \draw[constellation-boundary] (axis
  cs:111.652, -82.775886) -- (axis cs:112.687, -82.787200) -- (axis
  cs:113.724, -82.798432) -- (axis cs:114.762, -82.809578) -- (axis
  cs:115.802, -82.820634) -- (axis cs:116.843, -82.831597) -- (axis
  cs:117.886, -82.842464) -- (axis cs:118.931, -82.853229) -- (axis
  cs:119.977, -82.863890) -- (axis cs:121.026, -82.874443) -- (axis
  cs:122.075, -82.884883) -- (axis cs:123.126, -82.895209) -- (axis
  cs:124.179, -82.905414) -- (axis cs:125.234, -82.915497) -- (axis cs:126.29,
  -82.925454) -- (axis cs:127.347, -82.935280) -- (axis cs:128.406,
  -82.944973) -- (axis cs:129.467, -82.954528) -- (axis cs:130.529,
  -82.963943) -- (axis cs:131.593, -82.973214) -- (axis cs:132.658,
  -82.982338) -- (axis cs:133.724, -82.991310) -- (axis cs:134.792,
  -83.000129) -- (axis cs:135.861, -83.008790) -- (axis cs:136.932,
  -83.017291) -- (axis cs:138.004, -83.025627) -- (axis cs:139.078,
  -83.033797) -- (axis cs:140.152, -83.041797) -- (axis cs:141.229,
  -83.049624) -- (axis cs:142.306, -83.057275) -- (axis cs:143.385,
  -83.064747) -- (axis cs:144.464, -83.072037) -- (axis cs:145.546,
  -83.079143) -- (axis cs:146.628, -83.086062) -- (axis cs:147.711,
  -83.092790) -- (axis cs:148.796, -83.099326) -- (axis cs:149.881,
  -83.105667) -- (axis cs:150.968, -83.111810) -- (axis cs:152.056,
  -83.117754) -- (axis cs:153.144, -83.123495) -- (axis cs:154.234,
  -83.129031) -- (axis cs:155.324, -83.134361) -- (axis cs:156.416,
  -83.139482) -- (axis cs:157.508, -83.144392) -- (axis cs:158.601,
  -83.149089) -- (axis cs:159.695, -83.153572) -- (axis cs:160.79, -83.157838)
  -- (axis cs:161.885, -83.161886) -- (axis cs:162.982, -83.165714) -- (axis
  cs:164.078, -83.169320) -- (axis cs:165.176, -83.172704) -- (axis
  cs:166.274, -83.175864) -- (axis cs:167.372, -83.178799) -- (axis
  cs:168.471, -83.181507) -- (axis cs:169.57, -83.183987) -- (axis cs:170.67,
  -83.186238) -- (axis cs:171.77, -83.188260) -- (axis cs:172.871, -83.190052)
  -- (axis cs:173.972, -83.191612) -- (axis cs:175.073, -83.192941) -- (axis
  cs:176.174, -83.194037) -- (axis cs:177.276, -83.194901) -- (axis
  cs:178.377, -83.195531) -- (axis cs:179.479, -83.195928)
  ; \draw[constellation-boundary] (axis cs:99.709, -43.111647) -- (axis
  cs:99.7311, -42.111780) -- (axis cs:99.7525, -41.111908) -- (axis
  cs:99.7733, -40.112033) -- (axis cs:99.7935, -39.112154) -- (axis
  cs:99.8131, -38.112271) -- (axis cs:99.8322, -37.112386) -- (axis
  cs:99.8508, -36.112497) -- (axis cs:99.8689, -35.112606) -- (axis
  cs:99.8866, -34.112711) -- (axis cs:99.9039, -33.112815)
  ; \draw[constellation-boundary] (axis cs:179.091, -25.195788)
  ; \draw[constellation-boundary] (axis cs:164.008, -25.166576) -- (axis
  cs:164.511, -25.168306) -- (axis cs:165.516, -25.171611) -- (axis
  cs:166.522, -25.174710) -- (axis cs:167.527, -25.177603) -- (axis
  cs:168.533, -25.180289) -- (axis cs:169.538, -25.182766) -- (axis
  cs:170.544, -25.185034) -- (axis cs:171.549, -25.187092) -- (axis
  cs:172.555, -25.188940) -- (axis cs:173.56, -25.190577) -- (axis cs:174.566,
  -25.192002) -- (axis cs:175.572, -25.193215) -- (axis cs:176.577,
  -25.194216) -- (axis cs:177.583, -25.195005) -- (axis cs:178.589,
  -25.195580) -- (axis cs:179.594, -25.195943) ; \draw[constellation-boundary]
  (axis cs:164.008, -25.166576) ; \draw[constellation-boundary] (axis
  cs:58.3188, -52.796845) -- (axis cs:58.3235, -52.630203) -- (axis cs:58.351,
  -51.630345) -- (axis cs:58.3772, -50.630481) ; \draw[constellation-boundary]
  (axis cs:60.6929, -56.155588) -- (axis cs:60.7098, -55.655677) -- (axis
  cs:60.7423, -54.655850) -- (axis cs:60.7732, -53.656014) -- (axis
  cs:60.7979, -52.822811) ; \draw[constellation-boundary] (axis cs:60.6929,
  -56.155588) -- (axis cs:61.6839, -56.166150) -- (axis cs:62.6751,
  -56.176814) -- (axis cs:63.6666, -56.187577) -- (axis cs:64.6584,
  -56.198436) -- (axis cs:65.6505, -56.209388) ; \draw[constellation-boundary]
  (axis cs:65.5546, -58.708857) -- (axis cs:65.5946, -57.709078) -- (axis
  cs:65.6324, -56.709288) -- (axis cs:65.6505, -56.209388)
  ; \draw[constellation-boundary] (axis cs:65.5546, -58.708857) -- (axis
  cs:66.5462, -58.719890) -- (axis cs:67.5381, -58.731009) -- (axis
  cs:68.5302, -58.742211) -- (axis cs:69.2746, -58.750665)
  ; \draw[constellation-boundary] (axis cs:68.5816, -69.746720) -- (axis
  cs:68.8285, -69.749534) -- (axis cs:69.8163, -69.760839) -- (axis
  cs:70.8047, -69.772220) -- (axis cs:71.7936, -69.783671) -- (axis
  cs:72.7831, -69.795191) -- (axis cs:73.7731, -69.806776) -- (axis
  cs:74.7636, -69.818423) -- (axis cs:75.7547, -69.830127) -- (axis
  cs:76.7463, -69.841886) -- (axis cs:77.7384, -69.853697) -- (axis
  cs:78.7312, -69.865554) -- (axis cs:79.7244, -69.877456) -- (axis
  cs:80.7183, -69.889398) -- (axis cs:81.7127, -69.901377) -- (axis
  cs:82.7076, -69.913390) -- (axis cs:83.7032, -69.925432) -- (axis
  cs:84.6993, -69.937500) -- (axis cs:85.6959, -69.949591) -- (axis
  cs:86.6932, -69.961700) -- (axis cs:87.691, -69.973825) -- (axis cs:88.6894,
  -69.985961) -- (axis cs:89.6884, -69.998105) -- (axis cs:90.688, -70.010253)
  -- (axis cs:91.6881, -70.022401) -- (axis cs:92.6889, -70.034547) -- (axis
  cs:93.6902, -70.046685) -- (axis cs:94.6921, -70.058812) -- (axis
  cs:95.6946, -70.070925) -- (axis cs:96.6977, -70.083020) -- (axis
  cs:97.7013, -70.095093) -- (axis cs:98.4545, -70.104131)
  ; \draw[constellation-boundary] (axis cs:90.1736, -64.001053) -- (axis
  cs:91.1735, -64.013201) -- (axis cs:92.1738, -64.025349) -- (axis
  cs:93.1746, -64.037491) -- (axis cs:94.1757, -64.049624) -- (axis
  cs:95.1774, -64.061745) -- (axis cs:96.1794, -64.073850) -- (axis
  cs:97.1819, -64.085935) -- (axis cs:98.1848, -64.097996) -- (axis
  cs:99.1882, -64.110030) -- (axis cs:100.192, -64.122033) -- (axis
  cs:101.196, -64.134001) -- (axis cs:102.201, -64.145930) -- (axis
  cs:103.206, -64.157818) -- (axis cs:104.211, -64.169660) -- (axis
  cs:105.217, -64.181452) -- (axis cs:106.224, -64.193191) -- (axis
  cs:107.231, -64.204873) -- (axis cs:108.238, -64.216495) -- (axis
  cs:109.246, -64.228053) -- (axis cs:110.254, -64.239544) -- (axis
  cs:111.262, -64.250963) -- (axis cs:112.271, -64.262307) -- (axis
  cs:113.281, -64.273573) -- (axis cs:114.29, -64.284757) -- (axis cs:115.3,
  -64.295856) -- (axis cs:116.311, -64.306866) -- (axis cs:117.322,
  -64.317784) -- (axis cs:118.333, -64.328606) -- (axis cs:119.345,
  -64.339329) -- (axis cs:120.358, -64.349949) -- (axis cs:121.37, -64.360463)
  -- (axis cs:122.383, -64.370869) -- (axis cs:123.396, -64.381161) -- (axis
  cs:124.41, -64.391338) -- (axis cs:125.424, -64.401396) -- (axis cs:126.439,
  -64.411332) -- (axis cs:127.454, -64.421142) -- (axis cs:128.469,
  -64.430824) -- (axis cs:129.485, -64.440374) -- (axis cs:130.501,
  -64.449790) -- (axis cs:131.517, -64.459068) -- (axis cs:132.534,
  -64.468205) -- (axis cs:133.551, -64.477199) -- (axis cs:134.568,
  -64.486047) -- (axis cs:135.586, -64.494745) -- (axis cs:136.095,
  -64.499037) ; \draw[constellation-boundary] (axis cs:90.1736, -64.001053) --
  (axis cs:90.2347, -63.001424) -- (axis cs:90.2917, -62.001770) -- (axis
  cs:90.3451, -61.002094) ; \draw[constellation-boundary] (axis cs:82.8576,
  -60.911287) -- (axis cs:83.3561, -60.917307) -- (axis cs:84.3534,
  -60.929366) -- (axis cs:85.3511, -60.941450) -- (axis cs:86.3491,
  -60.953553) -- (axis cs:87.3475, -60.965673) -- (axis cs:88.3463,
  -60.977805) -- (axis cs:89.3455, -60.989947) -- (axis cs:90.3451,
  -61.002094) ; \draw[constellation-boundary] (axis cs:82.8576, -60.911287) --
  (axis cs:82.9071, -59.911586) -- (axis cs:82.9537, -58.911867) -- (axis
  cs:82.9977, -57.912132) -- (axis cs:83.0188, -57.412260)
  ; \draw[constellation-boundary] (axis cs:75.5477, -57.323042) -- (axis
  cs:76.5428, -57.334790) -- (axis cs:77.5383, -57.346590) -- (axis cs:78.534,
  -57.358438) -- (axis cs:79.5301, -57.370331) -- (axis cs:80.5264,
  -57.382266) -- (axis cs:81.5231, -57.394238) -- (axis cs:82.5202,
  -57.406245) -- (axis cs:83.0188, -57.412260) ; \draw[constellation-boundary]
  (axis cs:75.5477, -57.323042) -- (axis cs:75.5677, -56.823159) -- (axis
  cs:75.606, -55.823385) -- (axis cs:75.6424, -54.823599) -- (axis cs:75.677,
  -53.823803) ; \draw[constellation-boundary] (axis cs:68.2177, -53.737636) --
  (axis cs:68.2493, -52.737814) -- (axis cs:68.2794, -51.737984) -- (axis
  cs:68.3082, -50.738147) -- (axis cs:68.3358, -49.738303) -- (axis
  cs:68.3622, -48.738452) -- (axis cs:68.3877, -47.738596) -- (axis
  cs:68.4122, -46.738734) -- (axis cs:68.4241, -46.238801)
  ; \draw[constellation-boundary] (axis cs:68.2177, -53.737636) -- (axis
  cs:68.7146, -53.743255) -- (axis cs:69.7084, -53.754552) -- (axis
  cs:70.7025, -53.765924) -- (axis cs:71.6968, -53.777368) -- (axis
  cs:72.6915, -53.788882) -- (axis cs:73.6864, -53.800462) -- (axis
  cs:74.6816, -53.812103) -- (axis cs:75.677, -53.823803)
  ; \draw[constellation-boundary] (axis cs:62.15, -48.669978) -- (axis
  cs:62.895, -48.678002) -- (axis cs:63.8886, -48.688787) -- (axis cs:64.8823,
  -48.699667) -- (axis cs:65.8763, -48.710640) -- (axis cs:66.8705,
  -48.721701) -- (axis cs:67.8649, -48.732848) -- (axis cs:68.3622,
  -48.738452) ; \draw[constellation-boundary] (axis cs:62.0987, -50.669703) --
  (axis cs:62.1248, -49.669843) -- (axis cs:62.15, -48.669978)
  ; \draw[constellation-boundary] (axis cs:58.3772, -50.630481) -- (axis
  cs:58.8732, -50.635622) -- (axis cs:59.8654, -50.645988) -- (axis
  cs:60.8579, -50.656462) -- (axis cs:61.8505, -50.667042) -- (axis
  cs:62.0987, -50.669703) ; \draw[constellation-boundary] (axis cs:59.1058,
  -39.636831) -- (axis cs:60.1005, -39.647223) -- (axis cs:61.0954,
  -39.657722) -- (axis cs:62.0904, -39.668326) -- (axis cs:63.0856,
  -39.679032) -- (axis cs:64.0809, -39.689835) -- (axis cs:65.0764,
  -39.700734) ; \draw[constellation-boundary] (axis cs:59.0313, -43.636443) --
  (axis cs:59.0508, -42.636545) -- (axis cs:59.0697, -41.636643) -- (axis
  cs:59.088, -40.636738) -- (axis cs:59.1058, -39.636831)
  ; \draw[constellation-boundary] (axis cs:52.3267, -43.569408) -- (axis
  cs:53.0713, -43.576578) -- (axis cs:54.0642, -43.586250) -- (axis
  cs:55.0573, -43.596049) -- (axis cs:56.0506, -43.605971) -- (axis cs:57.044,
  -43.616012) -- (axis cs:58.0376, -43.626171) -- (axis cs:59.0313,
  -43.636443) ; \draw[constellation-boundary] (axis cs:52.2889, -45.569227) --
  (axis cs:52.3082, -44.569319) -- (axis cs:52.3267, -43.569408)
  ; \draw[constellation-boundary] (axis cs:46.0908, -45.512484) -- (axis
  cs:47.0821, -45.521191) -- (axis cs:48.0735, -45.530044) -- (axis
  cs:49.0651, -45.539039) -- (axis cs:50.0569, -45.548175) -- (axis
  cs:51.0488, -45.557447) -- (axis cs:52.0409, -45.566854) -- (axis
  cs:52.2889, -45.569227) ; \draw[constellation-boundary] (axis cs:46.0345,
  -48.512239) -- (axis cs:46.054, -47.512324) -- (axis cs:46.0727, -46.512405)
  -- (axis cs:46.0908, -45.512484) ; \draw[constellation-boundary] (axis
  cs:41.0852, -48.471011) -- (axis cs:42.0748, -48.478946) -- (axis
  cs:43.0645, -48.487039) -- (axis cs:44.0543, -48.495287) -- (axis
  cs:45.0443, -48.503688) -- (axis cs:46.0345, -48.512239)
  ; \draw[constellation-boundary] (axis cs:41.0476, -50.470862) -- (axis
  cs:41.0668, -49.470938) -- (axis cs:41.0852, -48.471011)
  ; \draw[constellation-boundary] (axis cs:37.3411, -50.442573) -- (axis
  cs:38.0823, -50.448045) -- (axis cs:39.0706, -50.455487) -- (axis cs:40.059,
  -50.463093) -- (axis cs:41.0476, -50.470862) ; \draw[constellation-boundary]
  (axis cs:37.2833, -53.442361) -- (axis cs:37.3034, -52.442435) -- (axis
  cs:37.3227, -51.442505) -- (axis cs:37.3411, -50.442573)
  ; \draw[constellation-boundary] (axis cs:33.5841, -53.416474) -- (axis
  cs:34.0772, -53.419785) -- (axis cs:35.0635, -53.426538) -- (axis cs:36.05,
  -53.433464) -- (axis cs:37.0366, -53.440560) -- (axis cs:37.2833,
  -53.442361) ; \draw[constellation-boundary] (axis cs:21.2733, -52.848565) --
  (axis cs:22.2583, -52.852912) -- (axis cs:23.2433, -52.857453) -- (axis
  cs:24.2285, -52.862187) -- (axis cs:24.9674, -52.865864)
  ; \draw[constellation-boundary] (axis cs:24.9674, -52.865864) -- (axis
  cs:24.9742, -52.365881) -- (axis cs:24.9875, -51.365915) -- (axis
  cs:24.9939, -50.865931) ; \draw[constellation-boundary] (axis cs:24.9939,
  -50.865931) -- (axis cs:25.2405, -50.867181) -- (axis cs:26.2268,
  -50.872303) -- (axis cs:27.2133, -50.877613) -- (axis cs:28.1999,
  -50.883111) -- (axis cs:28.6933, -50.885929) ; \draw[constellation-boundary]
  (axis cs:28.6933, -50.885929) -- (axis cs:28.7005, -50.385950) -- (axis
  cs:28.7143, -49.385990) -- (axis cs:28.7277, -48.386028) -- (axis
  cs:28.7384, -47.552726) ; \draw[constellation-boundary] (axis cs:28.7384,
  -47.552726) -- (axis cs:29.2325, -47.555595) -- (axis cs:30.2208,
  -47.561471) -- (axis cs:31.2092, -47.567529) -- (axis cs:32.1977,
  -47.573769) -- (axis cs:33.1863, -47.580187) -- (axis cs:34.1751,
  -47.586782) -- (axis cs:35.164, -47.593553) -- (axis cs:36.153, -47.600496)
  ; \draw[constellation-boundary] (axis cs:36.153, -47.600496) -- (axis
  cs:36.1556, -47.433839) -- (axis cs:36.1708, -46.433893) -- (axis
  cs:36.1855, -45.433945) -- (axis cs:36.1997, -44.433996) -- (axis
  cs:36.2134, -43.434044) -- (axis cs:36.2267, -42.434092) -- (axis
  cs:36.2395, -41.434137) -- (axis cs:36.252, -40.434181) -- (axis cs:36.2641,
  -39.434224) ; \draw[constellation-boundary] (axis cs:46.1872, -39.512904) --
  (axis cs:46.1933, -39.096264) ; \draw[constellation-boundary] (axis
  cs:46.1933, -39.096264) -- (axis cs:47.1864, -39.104987) -- (axis
  cs:48.1795, -39.113854) -- (axis cs:49.1728, -39.122865) -- (axis
  cs:50.1663, -39.132016) -- (axis cs:51.1598, -39.141303) -- (axis
  cs:52.1535, -39.150725) -- (axis cs:53.1473, -39.160279) -- (axis
  cs:53.6443, -39.165104) ; \draw[constellation-boundary] (axis cs:53.6443,
  -39.165104) -- (axis cs:53.6536, -38.581817) -- (axis cs:53.6694,
  -37.581893) -- (axis cs:53.6847, -36.581968) -- (axis cs:53.6996,
  -35.582040) ; \draw[constellation-boundary] (axis cs:53.6996, -35.582040) --
  (axis cs:54.1969, -35.586901) -- (axis cs:55.1917, -35.596716) -- (axis
  cs:56.1867, -35.606654) -- (axis cs:57.1817, -35.616712) -- (axis
  cs:57.4305, -35.619245) ; \draw[constellation-boundary] (axis cs:57.4305,
  -35.619245) -- (axis cs:57.4457, -34.619323) -- (axis cs:57.4606,
  -33.619399) -- (axis cs:57.4751, -32.619473) -- (axis cs:57.4894,
  -31.619545) -- (axis cs:57.5033, -30.619616) -- (axis cs:57.5169,
  -29.619686) -- (axis cs:57.5303, -28.619754) -- (axis cs:57.5434,
  -27.619820) -- (axis cs:57.5562, -26.619886) -- (axis cs:57.5689,
  -25.619951) ; \draw[constellation-boundary] (axis cs:26.3507, -39.372630) --
  (axis cs:26.3594, -38.372653) -- (axis cs:26.3679, -37.372676) -- (axis
  cs:26.3761, -36.372697) -- (axis cs:26.3842, -35.372719) -- (axis cs:26.392,
  -34.372740) -- (axis cs:26.3997, -33.372760) -- (axis cs:26.4072,
  -32.372780) -- (axis cs:26.4145, -31.372799) -- (axis cs:26.4217,
  -30.372818) -- (axis cs:26.4288, -29.372837) -- (axis cs:26.4357,
  -28.372855) -- (axis cs:26.4424, -27.372873) -- (axis cs:26.4491,
  -26.372890) -- (axis cs:26.4556, -25.372908) ; \draw[constellation-boundary]
  (axis cs:-4.347, -39.306740) -- (axis cs:-3.357, -39.305754) -- (axis
  cs:-2.367, -39.304978) -- (axis cs:-1.377, -39.304412) -- (axis cs:-0.387,
  -39.304055) -- (axis cs:0.602937, -39.303908) -- (axis cs:1.59292,
  -39.303971) -- (axis cs:2.5829, -39.304244) -- (axis cs:3.57289, -39.304727)
  -- (axis cs:4.56289, -39.305419) -- (axis cs:5.55289, -39.306321) -- (axis
  cs:6.54292, -39.307432) -- (axis cs:7.53296, -39.308752) -- (axis
  cs:8.52301, -39.310280) -- (axis cs:9.51309, -39.312017) -- (axis
  cs:10.5032, -39.313961) -- (axis cs:11.4933, -39.316112) -- (axis
  cs:12.4835, -39.318469) -- (axis cs:13.4737, -39.321032) -- (axis
  cs:14.4639, -39.323799) -- (axis cs:15.4542, -39.326771) -- (axis
  cs:16.4445, -39.329946) -- (axis cs:17.4349, -39.333323) -- (axis
  cs:18.4253, -39.336901) -- (axis cs:19.4158, -39.340679) -- (axis
  cs:20.4063, -39.344656) -- (axis cs:21.3969, -39.348832) -- (axis
  cs:22.3875, -39.353203) -- (axis cs:23.3782, -39.357771) -- (axis
  cs:24.3689, -39.362532) -- (axis cs:25.3598, -39.367485) -- (axis
  cs:26.3507, -39.372630) -- (axis cs:27.3416, -39.377964) -- (axis
  cs:28.3327, -39.383487) -- (axis cs:29.3238, -39.389195) -- (axis cs:30.315,
  -39.395089) -- (axis cs:31.3063, -39.401165) -- (axis cs:32.2977,
  -39.407422) -- (axis cs:33.2891, -39.413859) -- (axis cs:34.2807,
  -39.420473) -- (axis cs:35.2723, -39.427262) -- (axis cs:36.2641,
  -39.434224) -- (axis cs:37.2559, -39.441358) -- (axis cs:38.2478,
  -39.448661) -- (axis cs:39.2399, -39.456131) -- (axis cs:40.232, -39.463766)
  -- (axis cs:41.2243, -39.471563) -- (axis cs:42.2166, -39.479521) -- (axis
  cs:43.2091, -39.487636) -- (axis cs:44.2017, -39.495907) -- (axis
  cs:45.1944, -39.504330) -- (axis cs:46.1872, -39.512904)
  ; \draw[constellation-boundary] (axis cs:53.365, -52.747077) -- (axis
  cs:53.8601, -52.751917) -- (axis cs:54.8506, -52.761689) -- (axis
  cs:55.8412, -52.771585) -- (axis cs:56.8321, -52.781602) -- (axis
  cs:57.8232, -52.791736) -- (axis cs:58.8145, -52.801984) -- (axis
  cs:59.8061, -52.812343) -- (axis cs:60.7979, -52.822811)
  ; \draw[constellation-boundary] (axis cs:53.2368, -57.079786) -- (axis
  cs:53.2531, -56.579866) -- (axis cs:53.2844, -55.580018) -- (axis
  cs:53.3141, -54.580163) -- (axis cs:53.3424, -53.580301) -- (axis cs:53.365,
  -52.747077) ; \draw[constellation-boundary] (axis cs:48.7911, -57.037787) --
  (axis cs:49.7786, -57.046884) -- (axis cs:50.7664, -57.056118) -- (axis
  cs:51.7544, -57.065487) -- (axis cs:52.7426, -57.074987) -- (axis
  cs:53.2368, -57.079786) ; \draw[constellation-boundary] (axis cs:48.3627,
  -67.035829) -- (axis cs:48.3921, -66.535963) -- (axis cs:48.4475,
  -65.536217) -- (axis cs:48.4989, -64.536451) -- (axis cs:48.5466,
  -63.536669) -- (axis cs:48.5911, -62.536873) -- (axis cs:48.6327,
  -61.537063) -- (axis cs:48.6716, -60.537241) -- (axis cs:48.7083,
  -59.537408) -- (axis cs:48.7428, -58.537566) -- (axis cs:48.7755,
  -57.537715) -- (axis cs:48.7911, -57.037787) ; \draw[constellation-boundary]
  (axis cs:33.2023, -66.915201) -- (axis cs:33.6903, -66.918477) -- (axis
  cs:34.6664, -66.925161) -- (axis cs:35.6428, -66.932017) -- (axis
  cs:36.6194, -66.939042) -- (axis cs:37.5963, -66.946236) -- (axis
  cs:38.5735, -66.953595) -- (axis cs:39.551, -66.961118) -- (axis cs:40.5288,
  -66.968803) -- (axis cs:41.5069, -66.976647) -- (axis cs:42.4854,
  -66.984649) -- (axis cs:43.4641, -66.992805) -- (axis cs:44.4431,
  -67.001114) -- (axis cs:45.4225, -67.009574) -- (axis cs:46.4022,
  -67.018181) -- (axis cs:47.3823, -67.026933) -- (axis cs:48.3627,
  -67.035829) -- (axis cs:49.3434, -67.044864) -- (axis cs:50.3246,
  -67.054038) -- (axis cs:51.306, -67.063346) -- (axis cs:52.2879, -67.072787)
  -- (axis cs:53.2701, -67.082357) -- (axis cs:54.2527, -67.092055) -- (axis
  cs:55.2357, -67.101876) -- (axis cs:56.2191, -67.111819) -- (axis
  cs:57.2029, -67.121880) -- (axis cs:58.1871, -67.132057) -- (axis
  cs:59.1717, -67.142346) -- (axis cs:60.1568, -67.152744) -- (axis
  cs:61.1422, -67.163249) -- (axis cs:62.1281, -67.173858) -- (axis
  cs:63.1143, -67.184567) -- (axis cs:64.1011, -67.195373) -- (axis
  cs:65.0882, -67.206273) -- (axis cs:66.0758, -67.217263) -- (axis
  cs:67.0639, -67.228342) -- (axis cs:68.0524, -67.239504) -- (axis
  cs:68.7941, -67.247930) ; \draw[constellation-boundary] (axis cs:33.2023,
  -66.915201) -- (axis cs:33.2236, -66.415271) -- (axis cs:33.2636,
  -65.415405) -- (axis cs:33.3008, -64.415529) -- (axis cs:33.3353,
  -63.415644) -- (axis cs:33.3675, -62.415751) -- (axis cs:33.3976,
  -61.415852) -- (axis cs:33.4258, -60.415946) -- (axis cs:33.4523,
  -59.416035) -- (axis cs:33.4774, -58.416118) -- (axis cs:33.501, -57.416197)
  -- (axis cs:33.5234, -56.416272) -- (axis cs:33.5447, -55.416342) -- (axis
  cs:33.5648, -54.416410) -- (axis cs:33.5841, -53.416474)
  ; \draw[constellation-boundary] (axis cs:67.9575, -74.743167) -- (axis
  cs:68.1121, -73.744047) -- (axis cs:68.2492, -72.744828) -- (axis
  cs:68.3717, -71.745525) -- (axis cs:68.4819, -70.746152) -- (axis
  cs:68.5816, -69.746720) -- (axis cs:68.6723, -68.747237) -- (axis
  cs:68.7552, -67.747709) -- (axis cs:68.8313, -66.748142) -- (axis
  cs:68.9015, -65.748542) -- (axis cs:68.9665, -64.748912) -- (axis
  cs:69.0268, -63.749255) -- (axis cs:69.083, -62.749575) -- (axis cs:69.1356,
  -61.749874) -- (axis cs:69.1848, -60.750154) -- (axis cs:69.231, -59.750417)
  -- (axis cs:69.2746, -58.750665) ; \draw[constellation-boundary] (axis
  cs:52.0758, -74.574131) -- (axis cs:52.562, -74.578886) -- (axis cs:53.535,
  -74.588491) -- (axis cs:54.5085, -74.598222) -- (axis cs:55.4827,
  -74.608075) -- (axis cs:56.4574, -74.618047) -- (axis cs:57.4327,
  -74.628136) -- (axis cs:58.4086, -74.638338) -- (axis cs:59.3852,
  -74.648652) -- (axis cs:60.3623, -74.659073) -- (axis cs:61.3401,
  -74.669600) -- (axis cs:62.3186, -74.680228) -- (axis cs:63.2977,
  -74.690956) -- (axis cs:64.2775, -74.701779) -- (axis cs:65.2579,
  -74.712695) -- (axis cs:66.239, -74.723701) -- (axis cs:67.2208, -74.734793)
  -- (axis cs:67.9575, -74.743167) ; \draw[constellation-boundary] (axis
  cs:1.53337, -81.803966) -- (axis cs:2.44897, -81.804219) -- (axis cs:3.3646,
  -81.804665) -- (axis cs:4.28029, -81.805305) -- (axis cs:5.19605,
  -81.806140) -- (axis cs:6.1119, -81.807168) -- (axis cs:7.02787, -81.808389)
  -- (axis cs:7.94396, -81.809803) -- (axis cs:8.86022, -81.811410) -- (axis
  cs:9.77664, -81.813210) -- (axis cs:10.6933, -81.815201) -- (axis
  cs:11.6101, -81.817384) -- (axis cs:12.5271, -81.819758) -- (axis
  cs:13.4444, -81.822322) -- (axis cs:14.362, -81.825076) -- (axis cs:15.2799,
  -81.828019) -- (axis cs:16.198, -81.831151) -- (axis cs:17.1165, -81.834470)
  -- (axis cs:18.0354, -81.837977) -- (axis cs:18.9546, -81.841669) -- (axis
  cs:19.8742, -81.845547) -- (axis cs:20.7942, -81.849609) -- (axis
  cs:21.7146, -81.853854) -- (axis cs:22.6355, -81.858282) -- (axis
  cs:23.5568, -81.862891) -- (axis cs:24.4786, -81.867680) -- (axis cs:25.401,
  -81.872648) -- (axis cs:26.3238, -81.877795) -- (axis cs:27.2472,
  -81.883117) -- (axis cs:28.1711, -81.888615) -- (axis cs:29.0956,
  -81.894288) -- (axis cs:30.0208, -81.900132) -- (axis cs:30.9465,
  -81.906148) -- (axis cs:31.8729, -81.912334) -- (axis cs:32.7999,
  -81.918688) -- (axis cs:33.7276, -81.925209) -- (axis cs:34.656, -81.931895)
  -- (axis cs:35.5851, -81.938745) -- (axis cs:36.515, -81.945757) -- (axis
  cs:37.4456, -81.952928) -- (axis cs:38.3769, -81.960258) -- (axis
  cs:39.3091, -81.967745) -- (axis cs:40.2421, -81.975386) -- (axis
  cs:41.1758, -81.983180) -- (axis cs:42.1105, -81.991125) -- (axis cs:43.046,
  -81.999219) -- (axis cs:43.9824, -82.007460) -- (axis cs:44.9197,
  -82.015846) -- (axis cs:45.8579, -82.024374) -- (axis cs:46.797, -82.033043)
  -- (axis cs:47.7371, -82.041851) -- (axis cs:48.6782, -82.050794) -- (axis
  cs:49.6203, -82.059872) -- (axis cs:50.0917, -82.064460)
  ; \draw[constellation-boundary] (axis cs:1.53337, -81.803966) -- (axis
  cs:1.53731, -81.303966) -- (axis cs:1.54398, -80.303967) -- (axis cs:1.5494,
  -79.303968) -- (axis cs:1.55391, -78.303968) -- (axis cs:1.55772,
  -77.303968) -- (axis cs:1.56098, -76.303968) -- (axis cs:1.56381,
  -75.303969) -- (axis cs:1.56628, -74.303969) ; \draw[constellation-boundary]
  (axis cs:12.2954, -75.318538) -- (axis cs:12.3324, -74.318585)
  ; \draw[constellation-boundary] (axis cs:12.2954, -75.318538) -- (axis
  cs:13.0113, -75.320409) -- (axis cs:13.966, -75.323078) -- (axis cs:14.9208,
  -75.325943) -- (axis cs:15.8759, -75.329005) -- (axis cs:16.8311,
  -75.332262) -- (axis cs:17.7865, -75.335714) -- (axis cs:18.7421,
  -75.339360) -- (axis cs:19.698, -75.343199) -- (axis cs:20.6541, -75.347229)
  ; \draw[constellation-boundary] (axis cs:20.6541, -75.347229) -- (axis
  cs:20.7173, -74.347366) -- (axis cs:20.7731, -73.347486) -- (axis
  cs:20.8227, -72.347593) -- (axis cs:20.8672, -71.347689) -- (axis
  cs:20.9073, -70.347776) -- (axis cs:20.9437, -69.347854) -- (axis
  cs:20.9768, -68.347926) -- (axis cs:21.0072, -67.347991) -- (axis
  cs:21.0351, -66.348051) -- (axis cs:21.0609, -65.348107) -- (axis
  cs:21.0848, -64.348158) -- (axis cs:21.107, -63.348206) -- (axis cs:21.1277,
  -62.348251) -- (axis cs:21.1471, -61.348293) -- (axis cs:21.1653,
  -60.348332) -- (axis cs:21.1824, -59.348369) -- (axis cs:21.1985,
  -58.348404) -- (axis cs:21.2138, -57.348437) -- (axis cs:21.2282,
  -56.348468) -- (axis cs:21.2419, -55.348497) -- (axis cs:21.2549,
  -54.348525) -- (axis cs:21.2673, -53.348552) -- (axis cs:21.2733,
  -52.848565) ; \draw[constellation-boundary] (axis cs:109.02, -85.261446) --
  (axis cs:110.346, -84.268724) -- (axis cs:111.281, -83.273852) -- (axis
  cs:111.976, -82.277660) -- (axis cs:112.513, -81.280600) -- (axis cs:112.94,
  -80.282941) -- (axis cs:113.289, -79.284849) -- (axis cs:113.578,
  -78.286435) -- (axis cs:113.823, -77.287775) -- (axis cs:114.033,
  -76.288922) -- (axis cs:114.215, -75.289916) ; \draw[constellation-boundary]
  (axis cs:48.233, -84.555390) -- (axis cs:49.1482, -83.559858) -- (axis
  cs:49.8191, -82.563131) -- (axis cs:50.3322, -81.565633) -- (axis
  cs:50.7374, -80.567608) -- (axis cs:51.0657, -79.569209) -- (axis
  cs:51.3372, -78.570532) -- (axis cs:51.5656, -77.571645) -- (axis
  cs:51.7604, -76.572594) -- (axis cs:51.9288, -75.573415) -- (axis
  cs:52.0758, -74.574131) ; \draw[constellation-boundary] (axis cs:48.233,
  -84.555390) -- (axis cs:48.69, -84.559880) -- (axis cs:49.6051, -84.568958)
  -- (axis cs:50.5215, -84.578163) -- (axis cs:51.4394, -84.587494) -- (axis
  cs:52.3587, -84.596948) -- (axis cs:53.2795, -84.606524) -- (axis
  cs:54.2019, -84.616219) -- (axis cs:55.1257, -84.626030) -- (axis
  cs:56.0511, -84.635956) -- (axis cs:56.9781, -84.645994) -- (axis
  cs:57.9066, -84.656141) -- (axis cs:58.8368, -84.666395) -- (axis
  cs:59.7687, -84.676754) -- (axis cs:60.7023, -84.687214) -- (axis
  cs:61.6375, -84.697774) -- (axis cs:62.5746, -84.708431) -- (axis
  cs:63.5133, -84.719182) -- (axis cs:64.4539, -84.730024) -- (axis
  cs:65.3963, -84.740955) -- (axis cs:66.3405, -84.751971) -- (axis
  cs:67.2866, -84.763071) -- (axis cs:68.2346, -84.774251) -- (axis
  cs:69.1845, -84.785507) -- (axis cs:70.1364, -84.796839) -- (axis
  cs:71.0902, -84.808241) -- (axis cs:72.0461, -84.819712) -- (axis
  cs:73.0039, -84.831248) -- (axis cs:73.9638, -84.842846) -- (axis
  cs:74.9258, -84.854503) -- (axis cs:75.8899, -84.866216) -- (axis
  cs:76.8561, -84.877981) -- (axis cs:77.8244, -84.889796) -- (axis
  cs:78.7949, -84.901657) -- (axis cs:79.7676, -84.913561) -- (axis
  cs:80.7425, -84.925505) -- (axis cs:81.7197, -84.937485) -- (axis
  cs:82.6991, -84.949497) -- (axis cs:83.6808, -84.961539) -- (axis
  cs:84.6648, -84.973606) -- (axis cs:85.6511, -84.985696) -- (axis
  cs:86.6398, -84.997805) -- (axis cs:87.6309, -85.009928) -- (axis
  cs:88.6243, -85.022064) -- (axis cs:89.6201, -85.034207) -- (axis
  cs:90.6184, -85.046355) -- (axis cs:91.6191, -85.058504) -- (axis
  cs:92.6223, -85.070649) -- (axis cs:93.6279, -85.082788) -- (axis cs:94.636,
  -85.094916) -- (axis cs:95.6467, -85.107030) -- (axis cs:96.6599,
  -85.119126) -- (axis cs:97.6756, -85.131200) -- (axis cs:98.6939,
  -85.143248) -- (axis cs:99.7147, -85.155266) -- (axis cs:100.738,
  -85.167251) -- (axis cs:101.764, -85.179198) -- (axis cs:102.793,
  -85.191104) -- (axis cs:103.824, -85.202964) -- (axis cs:104.858,
  -85.214775) -- (axis cs:105.894, -85.226533) -- (axis cs:106.934,
  -85.238233) -- (axis cs:107.975, -85.249872) -- (axis cs:109.02, -85.261446)
  ; \draw[constellation-boundary] (axis cs:97.7708, -75.100034) -- (axis
  cs:98.0222, -75.103048) -- (axis cs:99.0282, -75.115086) -- (axis
  cs:100.035, -75.127094) -- (axis cs:101.043, -75.139068) -- (axis
  cs:102.051, -75.151004) -- (axis cs:103.06, -75.162898) -- (axis cs:104.071,
  -75.174746) -- (axis cs:105.081, -75.186545) -- (axis cs:106.093,
  -75.198292) -- (axis cs:107.106, -75.209981) -- (axis cs:108.119,
  -75.221611) -- (axis cs:109.133, -75.233176) -- (axis cs:110.148,
  -75.244674) -- (axis cs:111.163, -75.256101) -- (axis cs:112.18, -75.267452)
  -- (axis cs:113.197, -75.278726) -- (axis cs:114.215, -75.289916) -- (axis
  cs:115.233, -75.301021) -- (axis cs:116.253, -75.312037) -- (axis
  cs:117.273, -75.322960) -- (axis cs:118.294, -75.333786) -- (axis
  cs:119.316, -75.344512) -- (axis cs:120.338, -75.355135) -- (axis
  cs:121.361, -75.365651) -- (axis cs:122.385, -75.376056) -- (axis cs:123.41,
  -75.386348) -- (axis cs:124.435, -75.396523) -- (axis cs:125.462,
  -75.406577) -- (axis cs:126.488, -75.416507) -- (axis cs:127.516,
  -75.426311) -- (axis cs:128.544, -75.435984) -- (axis cs:129.573,
  -75.445523) -- (axis cs:130.603, -75.454926) -- (axis cs:131.633,
  -75.464189) -- (axis cs:132.664, -75.473309) -- (axis cs:133.695,
  -75.482284) -- (axis cs:134.727, -75.491109) -- (axis cs:135.76, -75.499783)
  -- (axis cs:136.794, -75.508301) -- (axis cs:137.828, -75.516662) -- (axis
  cs:138.862, -75.524863) -- (axis cs:139.898, -75.532900) -- (axis
  cs:140.933, -75.540772) -- (axis cs:141.97, -75.548475) -- (axis cs:143.007,
  -75.556006) -- (axis cs:144.044, -75.563364) -- (axis cs:145.082,
  -75.570546) -- (axis cs:146.121, -75.577548) -- (axis cs:147.16, -75.584370)
  -- (axis cs:148.199, -75.591008) -- (axis cs:149.239, -75.597461) -- (axis
  cs:150.28, -75.603725) -- (axis cs:151.321, -75.609800) -- (axis cs:152.362,
  -75.615682) -- (axis cs:153.404, -75.621370) -- (axis cs:154.446,
  -75.626862) -- (axis cs:155.489, -75.632156) -- (axis cs:156.532,
  -75.637250) -- (axis cs:157.575, -75.642142) -- (axis cs:158.619,
  -75.646831) -- (axis cs:159.663, -75.651315) -- (axis cs:160.707,
  -75.655593) -- (axis cs:161.752, -75.659663) -- (axis cs:162.797,
  -75.663523) -- (axis cs:163.842, -75.667172) -- (axis cs:164.888,
  -75.670609) -- (axis cs:165.933, -75.673834) -- (axis cs:166.979,
  -75.676843) -- (axis cs:168.026, -75.679638) -- (axis cs:169.072,
  -75.682216) -- (axis cs:170.119, -75.684577) -- (axis cs:171.165,
  -75.686719) -- (axis cs:172.212, -75.688643) -- (axis cs:173.259,
  -75.690348) -- (axis cs:174.307, -75.691832) -- (axis cs:175.354,
  -75.693096) -- (axis cs:176.401, -75.694138) -- (axis cs:177.449,
  -75.694959) -- (axis cs:178.496, -75.695559) -- (axis cs:179.544,
  -75.695936) ; \draw[constellation-boundary] (axis cs:97.7708, -75.100034) --
  (axis cs:97.9408, -74.101053) -- (axis cs:98.0913, -73.101955) -- (axis
  cs:98.2254, -72.102759) -- (axis cs:98.3458, -71.103480) -- (axis
  cs:98.4545, -70.104131) -- (axis cs:98.5532, -69.104723) -- (axis
  cs:98.6432, -68.105262) -- (axis cs:98.7258, -67.105757) -- (axis
  cs:98.8019, -66.106213) -- (axis cs:98.8721, -65.106634) -- (axis
  cs:98.9373, -64.107024) ; \draw[constellation-boundary] (axis cs:-4.298,
  -57.806713) -- (axis cs:-3.318, -57.805737) -- (axis cs:-2.337, -57.804968)
  -- (axis cs:-1.356, -57.804407) -- (axis cs:-0.376, -57.804054) -- (axis
  cs:0.604805, -57.803908) -- (axis cs:1.58544, -57.803971) -- (axis
  cs:2.56607, -57.804241) -- (axis cs:3.54672, -57.804719) -- (axis
  cs:4.52738, -57.805405) -- (axis cs:5.50806, -57.806298) -- (axis
  cs:6.48877, -57.807399) -- (axis cs:7.46951, -57.808706) -- (axis
  cs:8.45029, -57.810221) -- (axis cs:9.43111, -57.811941) -- (axis cs:10.412,
  -57.813866) -- (axis cs:11.3929, -57.815997) -- (axis cs:12.3739,
  -57.818333) -- (axis cs:13.3549, -57.820872) -- (axis cs:14.336, -57.823614)
  -- (axis cs:15.3172, -57.826558) -- (axis cs:16.2985, -57.829704) -- (axis
  cs:17.2799, -57.833050) -- (axis cs:18.2613, -57.836596) -- (axis
  cs:19.2429, -57.840341) -- (axis cs:20.2245, -57.844282) -- (axis
  cs:21.2063, -57.848420) -- (axis cs:22.1881, -57.852754) -- (axis
  cs:23.1701, -57.857281) -- (axis cs:24.1522, -57.862000) -- (axis
  cs:25.1345, -57.866911) -- (axis cs:26.1168, -57.872012) -- (axis
  cs:27.0993, -57.877301) -- (axis cs:28.082, -57.882777) -- (axis cs:29.0648,
  -57.888437) -- (axis cs:30.0477, -57.894282) -- (axis cs:31.0309,
  -57.900308) -- (axis cs:32.0141, -57.906515) -- (axis cs:32.9976,
  -57.912899) -- (axis cs:33.4894, -57.916158) ; \draw[constellation-boundary]
  (axis cs:121.038, -43.353501) -- (axis cs:122.044, -43.363943) -- (axis
  cs:123.05, -43.374274) -- (axis cs:124.056, -43.384491) -- (axis cs:125.062,
  -43.394592) -- (axis cs:126.069, -43.404572) -- (axis cs:126.572,
  -43.409516) ; \draw[constellation-boundary] (axis cs:126.572, -43.409516) --
  (axis cs:126.59, -42.409605) -- (axis cs:126.608, -41.409692) -- (axis
  cs:126.625, -40.409776) -- (axis cs:126.642, -39.409857) -- (axis
  cs:126.658, -38.409936) -- (axis cs:126.674, -37.410013) -- (axis
  cs:126.689, -36.410088) -- (axis cs:126.704, -35.410161) -- (axis
  cs:126.719, -34.410233) -- (axis cs:126.733, -33.410302) -- (axis
  cs:126.747, -32.410370) -- (axis cs:126.76, -31.410437) -- (axis cs:126.774,
  -30.410502) -- (axis cs:126.787, -29.410566) -- (axis cs:126.799,
  -28.410628) -- (axis cs:126.812, -27.410690) -- (axis cs:126.824,
  -26.410750) -- (axis cs:126.836, -25.410809) ; \draw[constellation-boundary]
  (axis cs:126.678, -37.160032) -- (axis cs:127.18, -37.164939) -- (axis
  cs:128.186, -37.174656) -- (axis cs:129.191, -37.184244) -- (axis
  cs:130.197, -37.193699) -- (axis cs:131.203, -37.203020) -- (axis
  cs:132.209, -37.212201) -- (axis cs:133.215, -37.221242) -- (axis
  cs:134.221, -37.230139) -- (axis cs:135.228, -37.238889) -- (axis
  cs:136.234, -37.247490) -- (axis cs:137.241, -37.255939) -- (axis
  cs:138.247, -37.264233) -- (axis cs:139.254, -37.272370) -- (axis
  cs:140.261, -37.280348) -- (axis cs:141.268, -37.288162) -- (axis
  cs:141.772, -37.292008) ;
  




% Labels




\node[constellation-label] at (axis cs:{246,-53}) {\large\bfseries\textls{Norma}};

\node[constellation-label] at (axis cs:{234,-65}) {\normalsize\bfseries\textls{Triangulum}};
\node[constellation-label] at (axis cs:{245,-65}) {\normalsize\bfseries\textls{australe}};

\node[constellation-label] at (axis cs:{218,-67}) {\normalsize\bfseries\textls{Circinus}};

\node[constellation-label] at (axis cs:{120,-70}) {\large\bfseries\textls{Volans}};
\node[constellation-label] at (axis cs:{190,-74}) {\normalsize\bfseries\textls{Musca}};
\node[constellation-label] at (axis cs:{150,-73}) {\normalsize\bfseries\textls{Car}};
\node[constellation-label] at (axis cs:{110,-58}) {\Large\bfseries\textls{Carina}};
\node[constellation-label] at (axis cs:{105,-44}) {\Large\bfseries\textls{Puppis}};

\node[constellation-label,rotate=-20] at (axis cs:{160,-78}) {\normalsize\bfseries\textls{Chamaeleon}};
\node[constellation-label,rotate=60] at (axis cs:{240,-75}) {\Large\bfseries\textls{Apus}};
\node[constellation-label,rotate=103] at (axis cs:{103,-31}) {\normalsize\bfseries\textls{Canis Major}};
\node[constellation-label,rotate=85] at (axis cs:{85,-37}) {\large\bfseries\textls{Columba}};
\node[constellation-label] at (axis cs:{85,-53}) {\large\bfseries\textls{Pictor}};
\node[constellation-label,rotate=-45] at (axis cs:{78,-62}) {\large\bfseries\textls{Dorado}};
\node[constellation-label,rotate=0] at (axis cs:{89,-74}) {\large\bfseries\textls{Mensa}};
\node[constellation-label,rotate=-17] at (axis cs:{73,-39}) {\normalsize\bfseries\textls{Caelum}};
\node[constellation-label,rotate=-32] at (axis cs:{57,-59}) {\normalsize\bfseries\textls{Reticulum}};
\node[constellation-label,rotate=45] at (axis cs:{45,-73}) {\large\bfseries\textls{Hydrus}};
\node[constellation-label,rotate=0] at (axis cs:{50,-51}) {\large\bfseries\textls{Horologium}};
\node[constellation-label,rotate=5] at (axis cs:{45,-45}) {\LARGE\bfseries\textls{Eridanus}};


\node[constellation-label,rotate=42] at (axis cs:{42,-37}) {\Large\bfseries\textls{Fornax}};
\node[constellation-label,rotate=5] at (axis cs:{5,-37}) {\Large\bfseries\textls{Sculptor}};
\node[constellation-label,rotate=7] at (axis cs:{7,-47}) {\Large\bfseries\textls{Phoenix}};
\node[constellation-label,rotate=-5] at (axis cs:{355,-63}) {\Large\bfseries\textls{Tucana}};

\node[constellation-label] at (axis cs:{310,-79}) {\Large\bfseries\textls{Octans}};
\node[constellation-label,rotate=-17] at (axis cs:{343,-45}) {\Large\bfseries\textls{Grus}};

\node[constellation-label,rotate=-25] at (axis cs:{335,-35}) {\normalsize\bfseries\textls{Piscis Austrinus}};

\node[constellation-label,rotate=-45] at (axis cs:{315,-38}) {\normalsize\bfseries\textls{Micros-}};
\node[constellation-label,rotate=-45] at (axis cs:{315,-36}) {\normalsize\bfseries\textls{copium}};

\node[constellation-label,rotate=-40] at (axis cs:{320,-57}) {\large\bfseries\textls{Indus}};

\node[constellation-label,rotate=-65] at (axis cs:{295,-65}) {\Large\bfseries\textls{Pavo}};
\node[constellation-label,rotate=0] at (axis cs:{258,-55}) {\Large\bfseries\textls{Ara}};

\node[constellation-label,rotate=-70] at (axis cs:{290,-52}) {\normalsize\bfseries\textls{Telescopium}};

\node[constellation-label,rotate=-80] at (axis cs:{280,-41}) {\normalsize\bfseries\textls{Corona australis}};
\node[constellation-label,rotate=-70] at (axis cs:{290,-34}) {\Large\bfseries\textls{Sagittarius}};
\node[constellation-label,rotate=-0] at (axis cs:{260,-38}) {\LARGE\bfseries\textls{Scorpius}};
\end{polaraxis}


% Nebulae and nebulae names
\input{./input/VII_Nebulae.tex}
% Stars and star names
\input{./input/VII_Stars.tex}

\begin{polaraxis}[rotate=90,name=constellations,at={($(base.center)+(-.8cm+0.75pt,0pt)$)},anchor=center,axis lines=none,clip=false]

  \clip (0\tendegree,-6\tendegree) arc (270:90:6\tendegree)
  -- (2\tendegree,6\tendegree)  -- (2\tendegree,-6\tendegree)
   -- cycle ;


\node[ecliptics-empty,pin={[pin distance=-0.4\onedegree,interest-label]-90:{NEP}}] at (axis cs:{18*15},{+66+33/ 60  }) {\pgfuseplotmark{+}} ; %  ecliptic pole
\node[ecliptics-empty,pin={[pin distance=-0.4\onedegree,interest-label]-90:{SEP}}] at (axis cs:{6*15},{-66-33/ 60  }) {\pgfuseplotmark{+}} ; %  ecliptic pole

\node[designation-label,anchor=south west]  at (axis cs:{01*15-37/4},{-57+14/ 60  })  {Achernar};
\node[designation-label,anchor=south west]  at (axis cs:{06*15-23/4},{-52+41/ 60  })  {Canopus};
\node[designation-label,anchor=north west]  at (axis cs:{ 8*15-22/4},{-59+30/ 60  })  {Avior};
\node[designation-label,anchor=north west]  at (axis cs:{ 9*15-13/4},{-69+43/ 60  })  {Miaplacidus};
\node[designation-label,anchor=south east]  at (axis cs:{17*15-33/4},{-37+06/ 60  })  {Shaula};
\node[designation-label,anchor=south east]  at (axis cs:{17*15-37/4},{-42+59/ 60  })  {Sargas};
\node[designation-label,anchor=north west]  at (axis cs:{18*15-24/4},{-34+23/ 60  })  {Kaus Australis};
\node[designation-label,anchor=north east]  at (axis cs:{20*15-25/4},{-56+44/ 60  })  {Peacock};
\node[designation-label,anchor=north east]  at (axis cs:{22*15- 8/4},{-46+57/ 60  })  {Alnair};

% add S Dor, R Dor??

\node[pin={[pin distance=-0.6\onedegree,Bayer]180:{$\alpha$}}] at (axis cs:{221.965},{-79.045}) {}; % Aps,  3.81 
\node[pin={[pin distance=-0.4\onedegree,Bayer]-90:{$\beta$}}] at (axis cs:{250.769},{-77.517}) {}; % Aps,  4.23 
\node[pin={[pin distance=-0.8\onedegree,Bayer]135:{$\delta^{1,2}$}}] at (axis cs:{245.087},{-78.696}) {}; % Aps,  4.74 
%\node[pin={[pin distance=-0.6\onedegree,Bayer]00:{$\delta^2$}}] at (axis cs:{245.112},{-78.667}) {}; % Aps,  5.26 
\node[pin={[pin distance=-0.6\onedegree,Bayer]00:{$\epsilon$}}] at (axis cs:{215.596},{-80.109}) {}; % Aps,  5.06 
\node[pin={[pin distance=-0.4\onedegree,Bayer]-90:{$\eta$}}] at (axis cs:{214.558},{-81.008}) {}; % Aps,  4.90 
\node[pin={[pin distance=-0.6\onedegree,Bayer]00:{$\gamma$}}] at (axis cs:{248.363},{-78.897}) {}; % Aps,  3.87 
\node[pin={[pin distance=-0.4\onedegree,Bayer]-90:{$\iota$}}] at (axis cs:{260.524},{-70.123}) {}; % Aps,  5.40 
\node[pin={[pin distance=-0.6\onedegree,Bayer]00:{$\kappa^1$}}] at (axis cs:{232.878},{-73.389}) {}; % Aps,  5.40 
\node[pin={[pin distance=-0.6\onedegree,Bayer]180:{$\kappa^2$}}] at (axis cs:{235.089},{-73.446}) {}; % Aps,  5.65 
\node[pin={[pin distance=-0.6\onedegree,Bayer]180:{$\vartheta$}}] at (axis cs:{211.333},{-76.797}) {}; % Aps,  5.69 
\node[pin={[pin distance=-0.6\onedegree,Bayer]00:{$\zeta$}}] at (axis cs:{260.498},{-67.771}) {}; % Aps,  4.76

\node[pin={[pin distance=-0.3\onedegree,Bayer]00:{$\alpha$}}] at (axis cs:{262.960},{-49.876}) {}; % Ara,  2.85 
\node[pin={[pin distance=-0.3\onedegree,Bayer]180:{$\beta$}}] at (axis cs:{261.325},{-55.530}) {}; % Ara,  2.82 
\node[pin={[pin distance=-0.4\onedegree,Bayer]00:{$\delta$}}] at (axis cs:{262.775},{-60.684}) {}; % Ara,  3.60 
\node[pin={[pin distance=-0.6\onedegree,Bayer]00:{$\epsilon^1$}}] at (axis cs:{254.896},{-53.160}) {}; % Ara,  4.05 
\node[pin={[pin distance=-0.6\onedegree,Bayer]180:{$\epsilon^2$}}] at (axis cs:{255.786},{-53.237}) {}; % Ara,  5.28 
\node[pin={[pin distance=-0.4\onedegree,Bayer]00:{$\eta$}}] at (axis cs:{252.446},{-59.041}) {}; % Ara,  3.76 
\node[pin={[pin distance=-0.4\onedegree,Bayer]00:{$\gamma$}}] at (axis cs:{261.349},{-56.378}) {}; % Ara,  3.32 
\node[pin={[pin distance=-0.6\onedegree,Bayer]00:{$\iota$}}] at (axis cs:{260.817},{-47.468}) {}; % Ara,  5.26 
\node[pin={[pin distance=-0.6\onedegree,Bayer]00:{$\kappa$}}] at (axis cs:{261.500},{-50.634}) {}; % Ara,  5.20 
\node[pin={[pin distance=-0.6\onedegree,Bayer]180:{$\lambda$}}] at (axis cs:{265.099},{-49.415}) {}; % Ara,  4.76 
\node[pin={[pin distance=-0.6\onedegree,Bayer]00:{$\mu$}}] at (axis cs:{266.036},{-51.834}) {}; % Ara,  5.12 
\node[pin={[pin distance=-0.6\onedegree,Bayer]00:{$\pi$}}] at (axis cs:{264.523},{-54.500}) {}; % Ara,  5.25 
\node[pin={[pin distance=-0.6\onedegree,Bayer]00:{$\sigma$}}] at (axis cs:{263.915},{-46.506}) {}; % Ara,  4.58 
\node[pin={[pin distance=-0.6\onedegree,Bayer]00:{$\vartheta$}}] at (axis cs:{271.658},{-50.091}) {}; % Ara,  3.67 
\node[pin={[pin distance=-0.4\onedegree,Bayer]00:{$\zeta$}}] at (axis cs:{254.655},{-55.990}) {}; % Ara,  3.11

\node[pin={[pin distance=-0.6\onedegree,Bayer]00:{$\alpha$}}] at (axis cs:{70.140},{-41.864}) {}; % Cae,  4.45 
\node[pin={[pin distance=-0.6\onedegree,Bayer]00:{$\beta$}}] at (axis cs:{70.515},{-37.144}) {}; % Cae,  5.04 
\node[pin={[pin distance=-0.6\onedegree,Bayer]00:{$\delta$}}] at (axis cs:{67.709},{-44.954}) {}; % Cae,  5.07 
\node[pin={[pin distance=-0.6\onedegree,Bayer]-90:{$\gamma^{1,2}$}}] at (axis cs:{76.102},{-35.483}) {}; % Cae,  4.57 
%\node[pin={[pin distance=-0.6\onedegree,Bayer]00:{$\gamma^2$}}] at (axis cs:{76.109},{-35.705}) {}; % Cae,  6.33 
\node[pin={[pin distance=-0.6\onedegree,Bayer]00:{$\zeta$}}] at (axis cs:{71.957},{-30.020}) {}; % Cae,  6.35

\node[pin={[pin distance=-0.\onedegree,Bayer]00:{$\alpha$}}] at (axis cs:{95.988},{-52.695}) {}; % Car, -0.63 
\node[pin={[pin distance=-0.2\onedegree,Bayer]00:{$\beta$}}] at (axis cs:{138.300},{-69.717}) {}; % Car,  1.67 
\node[pin={[pin distance=-0.4\onedegree,Bayer]00:{$\epsilon$}}] at (axis cs:{125.628},{-59.509}) {}; % Car,  1.95 
\node[pin={[pin distance=-0.4\onedegree,Bayer]00:{$\iota$}}] at (axis cs:{139.273},{-59.275}) {}; % Car,  2.25 
\node[pin={[pin distance=-0.4\onedegree,Bayer]00:{$\chi$}}] at (axis cs:{119.195},{-52.982}) {}; % Car,  3.46 
\node[pin={[pin distance=-0.6\onedegree,Bayer]00:{$\eta$}}] at (axis cs:{161.265},{-59.684}) {}; % Car,  6.21 
\node[pin={[pin distance=-0.4\onedegree,Bayer]00:{$\omega$}}] at (axis cs:{153.434},{-70.038}) {}; % Car,  3.30 
\node[pin={[pin distance=-0.6\onedegree,Bayer]00:{$\upsilon$}}] at (axis cs:{146.776},{-65.072}) {}; % Car,  3.01 
\node[pin={[pin distance=-0.6\onedegree,Bayer]00:{$\vartheta$}}] at (axis cs:{160.739},{-64.394}) {}; % Car,  2.74

\node[pin={[pin distance=-0.4\onedegree,Bayer]00:{$\alpha^1$}}] at (axis cs:{219.902},{-60.834}) {}; % Cen, -0.01 
\node[pin={[pin distance=-0.4\onedegree,Bayer]00:{$\beta$}}] at (axis cs:{210.956},{-60.373}) {}; % Cen,  0.64 
\node[pin={[pin distance=-0.4\onedegree,Bayer]00:{$\epsilon$}}] at (axis cs:{204.972},{-53.466}) {}; % Cen,  2.28 
\node[pin={[pin distance=-0.4\onedegree,Bayer]00:{$\eta$}}] at (axis cs:{218.877},{-42.158}) {}; % Cen,  2.34 
\node[pin={[pin distance=-0.4\onedegree,Bayer]00:{$\gamma$}}] at (axis cs:{190.379},{-48.960}) {}; % Cen,  2.18 
\node[pin={[pin distance=-0.4\onedegree,Bayer]00:{$\vartheta$}}] at (axis cs:{211.671},{-36.370}) {}; % Cen,  2.08 
\node[pin={[pin distance=-0.6\onedegree,Bayer]00:{$\chi$}}] at (axis cs:{211.512},{-41.179}) {}; % Cen,  4.36 
\node[pin={[pin distance=-0.6\onedegree,Bayer]00:{$\delta$}}] at (axis cs:{182.090},{-50.722}) {}; % Cen,  2.58 
\node[pin={[pin distance=-0.6\onedegree,Bayer]00:{$\iota$}}] at (axis cs:{200.149},{-36.712}) {}; % Cen,  2.76 
\node[pin={[pin distance=-0.6\onedegree,Bayer]00:{$\kappa$}}] at (axis cs:{224.790},{-42.104}) {}; % Cen,  3.13 
\node[pin={[pin distance=-0.6\onedegree,Bayer]00:{$\lambda$}}] at (axis cs:{173.945},{-63.020}) {}; % Cen,  3.12 
\node[pin={[pin distance=-0.6\onedegree,Bayer]00:{$\mu$}}] at (axis cs:{207.404},{-42.474}) {}; % Cen,  3.47 
\node[pin={[pin distance=-0.6\onedegree,Bayer]00:{$\nu$}}] at (axis cs:{207.376},{-41.688}) {}; % Cen,  3.40 
\node[pin={[pin distance=-0.6\onedegree,Bayer]00:{$\omicron^1$}}] at (axis cs:{172.942},{-59.442}) {}; % Cen,  5.09 
\node[pin={[pin distance=-0.6\onedegree,Bayer]00:{$\omicron^2$}}] at (axis cs:{172.953},{-59.515}) {}; % Cen,  5.19 
\node[pin={[pin distance=-0.6\onedegree,Bayer]00:{$\pi$}}] at (axis cs:{170.252},{-54.491}) {}; % Cen,  3.89 
\node[pin={[pin distance=-0.6\onedegree,Bayer]00:{$\psi$}}] at (axis cs:{215.139},{-37.885}) {}; % Cen,  4.05 
\node[pin={[pin distance=-0.6\onedegree,Bayer]00:{$\rho$}}] at (axis cs:{182.913},{-52.368}) {}; % Cen,  3.96 
\node[pin={[pin distance=-0.6\onedegree,Bayer]00:{$\sigma$}}] at (axis cs:{187.010},{-50.230}) {}; % Cen,  3.91 
\node[pin={[pin distance=-0.6\onedegree,Bayer]00:{$\tau$}}] at (axis cs:{189.426},{-48.541}) {}; % Cen,  3.85 
\node[pin={[pin distance=-0.6\onedegree,Bayer]00:{$\upsilon^1$}}] at (axis cs:{209.670},{-44.804}) {}; % Cen,  3.86 
\node[pin={[pin distance=-0.6\onedegree,Bayer]00:{$\upsilon^2$}}] at (axis cs:{210.431},{-45.603}) {}; % Cen,  4.34 
\node[pin={[pin distance=-0.6\onedegree,Bayer]00:{$\varphi$}}] at (axis cs:{209.568},{-42.101}) {}; % Cen,  3.82 
\node[pin={[pin distance=-0.6\onedegree,Bayer]00:{$\xi^1$}}] at (axis cs:{195.889},{-49.527}) {}; % Cen,  4.83 
\node[pin={[pin distance=-0.6\onedegree,Bayer]00:{$\xi^2$}}] at (axis cs:{196.728},{-49.906}) {}; % Cen,  4.28 
\node[pin={[pin distance=-0.6\onedegree,Bayer]00:{$\zeta$}}] at (axis cs:{208.885},{-47.288}) {}; % Cen,  2.53 
\node[pin={[pin distance=-0.6\onedegree,Flaamsted]00:{$  1$}}] at (axis cs:{206.422},{-33.044}) {}; % Cen,  4.23 
\node[pin={[pin distance=-0.6\onedegree,Flaamsted]00:{$  2$}}] at (axis cs:{207.361},{-34.451}) {}; % Cen,  4.29 
\node[pin={[pin distance=-0.6\onedegree,Flaamsted]00:{$  3$}}] at (axis cs:{207.957},{-32.994}) {}; % Cen,  4.55 
\node[pin={[pin distance=-0.6\onedegree,Flaamsted]00:{$  4$}}] at (axis cs:{208.302},{-31.927}) {}; % Cen,  4.74

\node[pin={[pin distance=-0.6\onedegree,Bayer]00:{$\alpha$}}] at (axis cs:{124.631},{-76.920}) {}; % Cha,  4.06 
\node[pin={[pin distance=-0.6\onedegree,Bayer]00:{$\beta$}}] at (axis cs:{184.587},{-79.312}) {}; % Cha,  4.25 
\node[pin={[pin distance=-0.6\onedegree,Bayer]-90:{$\delta^{1,2}$}}] at (axis cs:{161.318},{-80.470}) {}; % Cha,  5.50 
%\node[pin={[pin distance=-0.6\onedegree,Bayer]00:{$\delta^2$}}] at (axis cs:{161.446},{-80.540}) {}; % Cha,  4.45 
\node[pin={[pin distance=-0.6\onedegree,Bayer]180:{$\epsilon$}}] at (axis cs:{179.907},{-78.222}) {}; % Cha,  4.91 
\node[pin={[pin distance=-0.6\onedegree,Bayer]90:{$\eta$}}] at (axis cs:{130.331},{-78.963}) {}; % Cha,  5.46 
\node[pin={[pin distance=-0.3\onedegree,Bayer]-90:{$\gamma$}}] at (axis cs:{158.867},{-78.608}) {}; % Cha,  4.10 
\node[pin={[pin distance=-0.6\onedegree,Bayer]00:{$\iota$}}] at (axis cs:{141.038},{-80.787}) {}; % Cha,  5.34 
\node[pin={[pin distance=-0.6\onedegree,Bayer]180:{$\kappa$}}] at (axis cs:{181.194},{-76.519}) {}; % Cha,  5.03 
\node[pin={[pin distance=-0.6\onedegree,Bayer]00:{$\mu^1$}}] at (axis cs:{150.182},{-82.214}) {}; % Cha,  5.52 
\node[pin={[pin distance=-0.6\onedegree,Bayer]00:{$\nu$}}] at (axis cs:{146.586},{-76.776}) {}; % Cha,  5.44 
\node[pin={[pin distance=-0.6\onedegree,Bayer]-90:{$\pi$}}] at (axis cs:{174.315},{-75.896}) {}; % Cha,  5.65 
\node[pin={[pin distance=-0.3\onedegree,Bayer]-90:{$\vartheta$}}] at (axis cs:{125.161},{-77.484}) {}; % Cha,  4.34 
\node[pin={[pin distance=-0.6\onedegree,Bayer]90:{$\zeta$}}] at (axis cs:{143.472},{-80.941}) {}; % Cha,  5.07

\node[pin={[pin distance=-0.4\onedegree,Bayer]00:{$\alpha$}}] at (axis cs:{220.627},{-64.975}) {}; % Cir,  3.18 
\node[pin={[pin distance=-0.6\onedegree,Bayer]00:{$\beta$}}] at (axis cs:{229.379},{-58.801}) {}; % Cir,  4.07 
\node[pin={[pin distance=-0.5\onedegree,Bayer]90:{$\delta$}}] at (axis cs:{229.237},{-60.957}) {}; % Cir,  5.08 
\node[pin={[pin distance=-0.5\onedegree,Bayer]90:{$\epsilon$}}] at (axis cs:{229.412},{-63.610}) {}; % Cir,  4.85 
\node[pin={[pin distance=-0.6\onedegree,Bayer]00:{$\eta$}}] at (axis cs:{226.201},{-64.031}) {}; % Cir,  5.17 
\node[pin={[pin distance=-0.6\onedegree,Bayer]180:{$\gamma$}}] at (axis cs:{230.844},{-59.321}) {}; % Cir,  4.51 
\node[pin={[pin distance=-0.6\onedegree,Bayer]00:{$\vartheta$}}] at (axis cs:{224.183},{-62.781}) {}; % Cir,  5.08 
\node[pin={[pin distance=-0.6\onedegree,Bayer]-90:{$\zeta$}}] at (axis cs:{223.677},{-65.991}) {}; % Cir,  6.08

\node[pin={[pin distance=-0.3\onedegree,Bayer]-90:{$\alpha$}}] at (axis cs:{84.912},{-34.074}) {}; % Col,  2.66 
\node[pin={[pin distance=-0.4\onedegree,Bayer]00:{$\beta$}}] at (axis cs:{87.740},{-35.768}) {}; % Col,  3.11 
\node[pin={[pin distance=-0.6\onedegree,Bayer]90:{$\delta$}}] at (axis cs:{95.528},{-33.436}) {}; % Col,  3.85 
\node[pin={[pin distance=-0.6\onedegree,Bayer]00:{$\epsilon$}}] at (axis cs:{82.803},{-35.470}) {}; % Col,  3.87 
\node[pin={[pin distance=-0.4\onedegree,Bayer]00:{$\eta$}}] at (axis cs:{89.787},{-42.815}) {}; % Col,  3.94 
\node[pin={[pin distance=-0.6\onedegree,Bayer]00:{$\gamma$}}] at (axis cs:{89.384},{-35.283}) {}; % Col,  4.36 
\node[pin={[pin distance=-0.6\onedegree,Bayer]00:{$\kappa$}}] at (axis cs:{94.138},{-35.140}) {}; % Col,  4.37 
\node[pin={[pin distance=-0.6\onedegree,Bayer]00:{$\lambda$}}] at (axis cs:{88.279},{-33.801}) {}; % Col,  4.87 
\node[pin={[pin distance=-0.6\onedegree,Bayer]00:{$\mu$}}] at (axis cs:{86.500},{-32.306}) {}; % Col,  5.16 
\node[pin={[pin distance=-0.6\onedegree,Bayer]00:{$\nu^1$}}] at (axis cs:{84.319},{-27.871}) {}; % Col,  6.15 
\node[pin={[pin distance=-0.6\onedegree,Bayer]00:{$\nu^2$}}] at (axis cs:{84.436},{-28.690}) {}; % Col,  5.28 
\node[pin={[pin distance=-0.6\onedegree,Bayer]00:{$\omicron$}}] at (axis cs:{79.371},{-34.895}) {}; % Col,  4.82 
%\node[pin={[pin distance=-0.6\onedegree,Bayer]00:{$\pi^1$}}] at (axis cs:{91.671},{-42.299}) {}; % Col,  6.16 
\node[pin={[pin distance=-0.6\onedegree,Bayer]00:{$\pi^{1,2}$}}] at (axis cs:{91.970},{-42.154}) {}; % Col,  5.50 
\node[pin={[pin distance=-0.6\onedegree,Bayer]00:{$\sigma$}}] at (axis cs:{89.087},{-31.382}) {}; % Col,  5.52 
\node[pin={[pin distance=-0.6\onedegree,Bayer]00:{$\vartheta$}}] at (axis cs:{91.882},{-37.253}) {}; % Col,  5.00 
\node[pin={[pin distance=-0.6\onedegree,Bayer]180:{$\xi$}}] at (axis cs:{88.875},{-37.121}) {}; % Col,  4.96

\node[pin={[pin distance=-0.6\onedegree,Bayer]00:{$\alpha$}}] at (axis cs:{287.368},{-37.904}) {}; % CrA,  4.10 
\node[pin={[pin distance=-0.6\onedegree,Bayer]-90:{$\beta$}}] at (axis cs:{287.507},{-39.341}) {}; % CrA,  4.11 
\node[pin={[pin distance=-0.6\onedegree,Bayer]00:{$\delta$}}] at (axis cs:{287.087},{-40.496}) {}; % CrA,  4.58 
\node[pin={[pin distance=-0.6\onedegree,Bayer]00:{$\epsilon$}}] at (axis cs:{284.681},{-37.107}) {}; % CrA,  4.83 
\node[pin={[pin distance=-0.6\onedegree,Bayer]00:{$\eta^1$}}] at (axis cs:{282.210},{-43.680}) {}; % CrA,  5.46 
\node[pin={[pin distance=-0.6\onedegree,Bayer]180:{$\eta^2$}}] at (axis cs:{282.396},{-43.434}) {}; % CrA,  5.60 
\node[pin={[pin distance=-0.6\onedegree,Bayer]00:{$\gamma$}}] at (axis cs:{286.605},{-37.063}) {}; % CrA,  4.23 
\node[pin={[pin distance=-0.6\onedegree,Bayer]-90:{$\kappa^2$}}] at (axis cs:{278.346},{-38.726}) {}; % CrA,  5.61 
\node[pin={[pin distance=-0.6\onedegree,Bayer]00:{$\lambda$}}] at (axis cs:{280.946},{-38.323}) {}; % CrA,  5.12 
\node[pin={[pin distance=-0.6\onedegree,Bayer]-90:{$\mu$}}] at (axis cs:{281.936},{-40.406}) {}; % CrA,  5.21 
\node[pin={[pin distance=-0.6\onedegree,Bayer]00:{$\vartheta$}}] at (axis cs:{278.376},{-42.312}) {}; % CrA,  4.62 
\node[pin={[pin distance=-0.6\onedegree,Bayer]00:{$\zeta$}}] at (axis cs:{285.779},{-42.095}) {}; % CrA,  4.74


\node[pin={[pin distance=-0.4\onedegree,Bayer]00:{$\alpha$}}] at (axis cs:{68.499},{-55.045}) {}; % Dor,  3.26 
\node[pin={[pin distance=-0.6\onedegree,Bayer]00:{$\beta$}}] at (axis cs:{83.406},{-62.490}) {}; % Dor,  3.76 
\node[pin={[pin distance=-0.6\onedegree,Bayer]00:{$\delta$}}] at (axis cs:{86.193},{-65.735}) {}; % Dor,  4.34 
\node[pin={[pin distance=-0.6\onedegree,Bayer]00:{$\epsilon$}}] at (axis cs:{87.473},{-66.901}) {}; % Dor,  5.10 
\node[pin={[pin distance=-0.6\onedegree,Bayer]180:{$\eta^1$}}] at (axis cs:{91.539},{-66.040}) {}; % Dor,  5.70 
\node[pin={[pin distance=-0.6\onedegree,Bayer]00:{$\eta^2$}}] at (axis cs:{92.812},{-65.589}) {}; % Dor,  5.02 
\node[pin={[pin distance=-0.6\onedegree,Bayer]00:{$\gamma$}}] at (axis cs:{64.007},{-51.487}) {}; % Dor,  4.26 
\node[pin={[pin distance=-0.6\onedegree,Bayer]00:{$\kappa$}}] at (axis cs:{71.088},{-59.732}) {}; % Dor,  5.28 
\node[pin={[pin distance=-0.6\onedegree,Bayer]00:{$\lambda$}}] at (axis cs:{81.580},{-58.912}) {}; % Dor,  5.13 
\node[pin={[pin distance=-0.6\onedegree,Bayer]00:{$\nu$}}] at (axis cs:{92.184},{-68.843}) {}; % Dor,  5.05 
%\node[pin={[pin distance=-0.6\onedegree,Bayer]00:{$\pi^1$}}] at (axis cs:{95.659},{-69.984}) {}; % Dor,  5.55 
\node[pin={[pin distance=-0.6\onedegree,Bayer]00:{$\pi^{1,2}$}}] at (axis cs:{96.369},{-69.690}) {}; % Dor,  5.38 
\node[pin={[pin distance=-0.6\onedegree,Bayer]00:{$\vartheta$}}] at (axis cs:{78.439},{-67.185}) {}; % Dor,  4.80 
\node[pin={[pin distance=-0.6\onedegree,Bayer]180:{$\zeta$}}] at (axis cs:{76.378},{-57.473}) {}; % Dor,  4.71

\node[pin={[pin distance=-0.\onedegree,Bayer]00:{$\alpha$}}] at (axis cs:{24.429},{-57.237}) {}; % Eri,  0.54 
\node[pin={[pin distance=-0.4\onedegree,Bayer]180:{$\chi$}}] at (axis cs:{28.989},{-51.609}) {}; % Eri,  3.72 
\node[pin={[pin distance=-0.4\onedegree,Bayer]90:{$\iota$}}] at (axis cs:{40.167},{-39.855}) {}; % Eri,  4.12 
\node[pin={[pin distance=-0.6\onedegree,Bayer]00:{$\kappa$}}] at (axis cs:{36.746},{-47.704}) {}; % Eri,  4.25 
%\node[pin={[pin distance=-0.6\onedegree,Bayer]00:{$\upsilon^1$}}] at (axis cs:{68.377},{-29.766}) {}; % Eri,  4.51 
\node[pin={[pin distance=-0.4\onedegree,Bayer]180:{$\upsilon^2$}}] at (axis cs:{68.888},{-30.562}) {}; % Eri,  3.81 
\node[pin={[pin distance=-0.4\onedegree,Bayer]00:{$\upsilon^4$}}] at (axis cs:{64.474},{-33.798}) {}; % Eri,  3.56 
\node[pin={[pin distance=-0.4\onedegree,Bayer]00:{$\varphi$}}] at (axis cs:{34.127},{-51.512}) {}; % Eri,  3.55 
\node[pin={[pin distance=-0.4\onedegree,Bayer]90:{$\vartheta^1$}}] at (axis cs:{44.565},{-40.305}) {}; % Eri,  3.22 
\node[pin={[pin distance=-0.4\onedegree,Flaamsted]00:{$ 43$}}] at (axis cs:{66.009},{-34.017}) {}; % Eri,  3.96

\node[pin={[pin distance=-0.6\onedegree,Bayer]00:{$\alpha$}}] at (axis cs:{48.019},{-28.988}) {}; % For,  3.80 
\node[pin={[pin distance=-0.6\onedegree,Bayer]00:{$\beta$}}] at (axis cs:{42.273},{-32.406}) {}; % For,  4.46 
%\node[pin={[pin distance=-0.6\onedegree,Bayer]00:{$\chi^1$}}] at (axis cs:{51.483},{-35.921}) {}; % For,  6.39 
\node[pin={[pin distance=-0.6\onedegree,Bayer]0:{$\chi^{1,2}$}}] at (axis cs:{51.889},{-35.681}) {}; % For,  5.70 
\node[pin={[pin distance=-0.6\onedegree,Bayer]00:{$\delta$}}] at (axis cs:{55.562},{-31.938}) {}; % For,  4.98 
\node[pin={[pin distance=-0.6\onedegree,Bayer]00:{$\epsilon$}}] at (axis cs:{45.407},{-28.091}) {}; % For,  5.88 
%\node[pin={[pin distance=-0.6\onedegree,Bayer]00:{$\eta^{2,3}$}}] at (axis cs:{42.562},{-35.843}) {}; % For,  5.93 
\node[pin={[pin distance=-0.6\onedegree,Bayer]00:{$\eta^{2,3}$}}] at (axis cs:{42.668},{-35.676}) {}; % For,  5.48 
\node[pin={[pin distance=-0.6\onedegree,Bayer]00:{$\gamma^2$}}] at (axis cs:{42.476},{-27.942}) {}; % For,  5.39 
\node[pin={[pin distance=-0.6\onedegree,Bayer]00:{$\iota^1$}}] at (axis cs:{39.039},{-30.045}) {}; % For,  5.73 
\node[pin={[pin distance=-0.6\onedegree,Bayer]00:{$\iota^2$}}] at (axis cs:{39.578},{-30.194}) {}; % For,  5.84 
\node[pin={[pin distance=-0.6\onedegree,Bayer]180:{$\lambda^1$}}] at (axis cs:{38.279},{-34.650}) {}; % For,  5.90 
\node[pin={[pin distance=-0.6\onedegree,Bayer]00:{$\lambda^2$}}] at (axis cs:{39.244},{-34.578}) {}; % For,  5.78 
\node[pin={[pin distance=-0.6\onedegree,Bayer]00:{$\mu$}}] at (axis cs:{33.227},{-30.724}) {}; % For,  5.27 
\node[pin={[pin distance=-0.6\onedegree,Bayer]00:{$\nu$}}] at (axis cs:{31.123},{-29.297}) {}; % For,  4.70 
\node[pin={[pin distance=-0.6\onedegree,Bayer]00:{$\omega$}}] at (axis cs:{38.461},{-28.232}) {}; % For,  4.97 
\node[pin={[pin distance=-0.6\onedegree,Bayer]00:{$\pi$}}] at (axis cs:{30.311},{-30.002}) {}; % For,  5.36 
\node[pin={[pin distance=-0.6\onedegree,Bayer]00:{$\psi$}}] at (axis cs:{43.393},{-38.437}) {}; % For,  5.93 
\node[pin={[pin distance=-0.6\onedegree,Bayer]00:{$\rho$}}] at (axis cs:{56.984},{-30.168}) {}; % For,  5.53 
\node[pin={[pin distance=-0.6\onedegree,Bayer]00:{$\sigma$}}] at (axis cs:{56.614},{-29.338}) {}; % For,  5.92 
\node[pin={[pin distance=-0.6\onedegree,Bayer]00:{$\tau$}}] at (axis cs:{54.699},{-27.943}) {}; % For,  6.01 
\node[pin={[pin distance=-0.6\onedegree,Bayer]00:{$\varphi$}}] at (axis cs:{37.007},{-33.811}) {}; % For,  5.13

\node[pin={[pin distance=-0.2\onedegree,Bayer]00:{$\alpha$}}] at (axis cs:{332.058},{-46.961}) {}; % Gru,  1.77 
\node[pin={[pin distance=-0.3\onedegree,Bayer]00:{$\beta$}}] at (axis cs:{340.667},{-46.885}) {}; % Gru,  2.12 
\node[pin={[pin distance=-0.6\onedegree,Bayer]180:{$\delta^1$}}] at (axis cs:{337.317},{-43.495}) {}; % Gru,  3.96 
\node[pin={[pin distance=-0.6\onedegree,Bayer]00:{$\delta^2$}}] at (axis cs:{337.439},{-43.749}) {}; % Gru,  4.15 
\node[pin={[pin distance=-0.4\onedegree,Bayer]00:{$\epsilon$}}] at (axis cs:{342.139},{-51.317}) {}; % Gru,  3.49 
\node[pin={[pin distance=-0.6\onedegree,Bayer]00:{$\eta$}}] at (axis cs:{341.408},{-53.500}) {}; % Gru,  4.85 
\node[pin={[pin distance=-0.4\onedegree,Bayer]180:{$\gamma$}}] at (axis cs:{328.482},{-37.365}) {}; % Gru,  3.01 
\node[pin={[pin distance=-0.6\onedegree,Bayer]00:{$\iota$}}] at (axis cs:{347.590},{-45.246}) {}; % Gru,  3.88 
\node[pin={[pin distance=-0.6\onedegree,Bayer]00:{$\kappa$}}] at (axis cs:{346.165},{-53.965}) {}; % Gru,  5.37 
\node[pin={[pin distance=-0.6\onedegree,Bayer]00:{$\lambda$}}] at (axis cs:{331.529},{-39.543}) {}; % Gru,  4.46 
\node[pin={[pin distance=-0.6\onedegree,Bayer]180:{$\mu^1$}}] at (axis cs:{333.904},{-41.346}) {}; % Gru,  4.81 
\node[pin={[pin distance=-0.6\onedegree,Bayer]00:{$\mu^2$}}] at (axis cs:{334.111},{-41.627}) {}; % Gru,  5.11 
\node[pin={[pin distance=-0.6\onedegree,Bayer]00:{$\nu$}}] at (axis cs:{337.163},{-39.132}) {}; % Gru,  5.47 
\node[pin={[pin distance=-0.6\onedegree,Bayer]180:{$\omicron$}}] at (axis cs:{351.652},{-52.722}) {}; % Gru,  5.53 
\node[pin={[pin distance=-0.6\onedegree,Bayer]00:{$\pi^{1,2}$}}] at (axis cs:{335.684},{-45.948}) {}; % Gru,  6.13 
%\node[pin={[pin distance=-0.6\onedegree,Bayer]00:{$\pi^2$}}] at (axis cs:{335.783},{-45.928}) {}; % Gru,  5.63 
\node[pin={[pin distance=-0.6\onedegree,Bayer]00:{$\rho$}}] at (axis cs:{340.875},{-41.414}) {}; % Gru,  4.84 
\node[pin={[pin distance=-0.6\onedegree,Bayer]90:{$\sigma^{1,2}$}}] at (axis cs:{339.122},{-40.582}) {}; % Gru,  6.28 
%\node[pin={[pin distance=-0.6\onedegree,Bayer]00:{$\sigma^2$}}] at (axis cs:{339.245},{-40.591}) {}; % Gru,  5.88 
\node[pin={[pin distance=-0.6\onedegree,Bayer]00:{$\tau^1$}}] at (axis cs:{343.408},{-48.598}) {}; % Gru,  6.03 
\node[pin={[pin distance=-0.6\onedegree,Bayer]00:{$\tau^3$}}] at (axis cs:{344.199},{-47.969}) {}; % Gru,  5.71 
\node[pin={[pin distance=-0.6\onedegree,Bayer]00:{$\upsilon$}}] at (axis cs:{346.723},{-38.892}) {}; % Gru,  5.63 
\node[pin={[pin distance=-0.6\onedegree,Bayer]00:{$\varphi$}}] at (axis cs:{349.541},{-40.824}) {}; % Gru,  5.54 
\node[pin={[pin distance=-0.6\onedegree,Bayer]00:{$\vartheta$}}] at (axis cs:{346.720},{-43.520}) {}; % Gru,  4.33 
\node[pin={[pin distance=-0.6\onedegree,Bayer]00:{$\xi$}}] at (axis cs:{323.024},{-41.179}) {}; % Gru,  5.29 
\node[pin={[pin distance=-0.6\onedegree,Bayer]00:{$\zeta$}}] at (axis cs:{345.220},{-52.754}) {}; % Gru,  4.12

\node[pin={[pin distance=-0.4\onedegree,Bayer]00:{$\alpha$}}] at (axis cs:{63.500},{-42.294}) {}; % Hor,  3.86 
\node[pin={[pin distance=-0.6\onedegree,Bayer]00:{$\beta$}}] at (axis cs:{44.699},{-64.071}) {}; % Hor,  4.98 
\node[pin={[pin distance=-0.6\onedegree,Bayer]00:{$\delta$}}] at (axis cs:{62.711},{-41.993}) {}; % Hor,  4.93 
\node[pin={[pin distance=-0.6\onedegree,Bayer]180:{$\eta$}}] at (axis cs:{39.352},{-52.543}) {}; % Hor,  5.30 
\node[pin={[pin distance=-0.6\onedegree,Bayer]90:{$\gamma$}}] at (axis cs:{41.364},{-63.704}) {}; % Hor,  5.75 
\node[pin={[pin distance=-0.4\onedegree,Bayer]90:{$\iota$}}] at (axis cs:{40.639},{-50.800}) {}; % Hor,  5.40 
\node[pin={[pin distance=-0.6\onedegree,Bayer]90:{$\lambda$}}] at (axis cs:{36.225},{-60.312}) {}; % Hor,  5.37 
\node[pin={[pin distance=-0.6\onedegree,Bayer]00:{$\mu$}}] at (axis cs:{45.903},{-59.738}) {}; % Hor,  5.12 
\node[pin={[pin distance=-0.6\onedegree,Bayer]-90:{$\nu$}}] at (axis cs:{42.256},{-62.806}) {}; % Hor,  5.26
\node[pin={[pin distance=-0.6\onedegree,Bayer]00:{$\zeta$}}] at (axis cs:{40.165},{-54.550}) {}; % Hor,  5.22

\node[pin={[pin distance=-0.6\onedegree,Bayer]00:{$\beta$}}] at (axis cs:{178.227},{-33.908}) {}; % Hya,  4.29 
\node[pin={[pin distance=-0.6\onedegree,Bayer]00:{$\chi^1$}}] at (axis cs:{166.333},{-27.293}) {}; % Hya,  4.93 
\node[pin={[pin distance=-0.6\onedegree,Bayer]00:{$\chi^2$}}] at (axis cs:{166.490},{-27.288}) {}; % Hya,  5.71 
\node[pin={[pin distance=-0.6\onedegree,Bayer]00:{$\omicron$}}] at (axis cs:{175.053},{-34.744}) {}; % Hya,  4.70 
\node[pin={[pin distance=-0.6\onedegree,Bayer]00:{$\xi$}}] at (axis cs:{173.250},{-31.858}) {}; % Hya,  3.54 
\node[pin={[pin distance=-0.6\onedegree,Flaamsted]00:{$ 50$}}] at (axis cs:{213.192},{-27.261}) {}; % Hya,  5.08 
\node[pin={[pin distance=-0.6\onedegree,Flaamsted]00:{$ 51$}}] at (axis cs:{215.774},{-27.754}) {}; % Hya,  4.79 
\node[pin={[pin distance=-0.6\onedegree,Flaamsted]00:{$ 52$}}] at (axis cs:{217.043},{-29.492}) {}; % Hya,  4.98 
\node[pin={[pin distance=-0.6\onedegree,Flaamsted]00:{$ 58$}}] at (axis cs:{222.572},{-27.960}) {}; % Hya,  4.42 
\node[pin={[pin distance=-0.6\onedegree,Flaamsted]00:{$ 59$}}] at (axis cs:{224.664},{-27.657}) {}; % Hya,  5.67 
\node[pin={[pin distance=-0.6\onedegree,Flaamsted]00:{$ 60$}}] at (axis cs:{225.527},{-28.060}) {}; % Hya,  5.83

\node[pin={[pin distance=-0.4\onedegree,Bayer]00:{$\alpha$}}] at (axis cs:{29.692},{-61.570}) {}; % Hyi,  2.86 
\node[pin={[pin distance=-0.4\onedegree,Bayer]00:{$\beta$}}] at (axis cs:{6.438},{-77.254}) {}; % Hyi,  2.82 
\node[pin={[pin distance=-0.5\onedegree,Bayer]00:{$\delta$}}] at (axis cs:{35.437},{-68.659}) {}; % Hyi,  4.08 
\node[pin={[pin distance=-0.5\onedegree,Bayer]180:{$\epsilon$}}] at (axis cs:{39.897},{-68.267}) {}; % Hyi,  4.11 
\node[pin={[pin distance=-0.6\onedegree,Bayer]180:{$\eta^2$}}] at (axis cs:{28.734},{-67.647}) {}; % Hyi,  4.69 
\node[pin={[pin distance=-0.4\onedegree,Bayer]00:{$\gamma$}}] at (axis cs:{56.810},{-74.239}) {}; % Hyi,  3.26 
\node[pin={[pin distance=-0.6\onedegree,Bayer]180:{$\iota$}}] at (axis cs:{48.990},{-77.388}) {}; % Hyi,  5.51 
\node[pin={[pin distance=-0.6\onedegree,Bayer]90:{$\kappa$}}] at (axis cs:{35.718},{-73.646}) {}; % Hyi,  5.99 
\node[pin={[pin distance=-0.6\onedegree,Bayer]90:{$\lambda$}}] at (axis cs:{12.148},{-74.923}) {}; % Hyi,  5.08 
\node[pin={[pin distance=-0.6\onedegree,Bayer]-90:{$\mu$}}] at (axis cs:{37.919},{-79.109}) {}; % Hyi,  5.27 
\node[pin={[pin distance=-0.6\onedegree,Bayer]00:{$\nu$}}] at (axis cs:{42.619},{-75.067}) {}; % Hyi,  4.74 
\node[pin={[pin distance=-0.6\onedegree,Bayer]00:{$\pi^{1,2}$}}] at (axis cs:{33.561},{-67.841}) {}; % Hyi,  5.57 
%\node[pin={[pin distance=-0.6\onedegree,Bayer]00:{$\pi^2$}}] at (axis cs:{33.869},{-67.746}) {}; % Hyi,  5.67 
\node[pin={[pin distance=-0.6\onedegree,Bayer]-90:{$\sigma$}}] at (axis cs:{28.961},{-78.348}) {}; % Hyi,  6.15 
\node[pin={[pin distance=-0.6\onedegree,Bayer]90:{$\tau^1$}}] at (axis cs:{25.339},{-79.148}) {}; % Hyi,  6.35 
\node[pin={[pin distance=-0.6\onedegree,Bayer]180:{$\tau^2$}}] at (axis cs:{26.944},{-80.176}) {}; % Hyi,  6.05 
\node[pin={[pin distance=-0.6\onedegree,Bayer]00:{$\vartheta$}}] at (axis cs:{45.564},{-71.902}) {}; % Hyi,  5.51 
\node[pin={[pin distance=-0.6\onedegree,Bayer]00:{$\zeta$}}] at (axis cs:{41.386},{-67.616}) {}; % Hyi,  4.84 

\node[pin={[pin distance=-0.4\onedegree,Bayer]00:{$\alpha$}}] at (axis cs:{309.392},{-47.291}) {}; % Ind,  3.11 
\node[pin={[pin distance=-0.5\onedegree,Bayer]00:{$\beta$}}] at (axis cs:{313.703},{-58.454}) {}; % Ind,  3.65 
\node[pin={[pin distance=-0.6\onedegree,Bayer]180:{$\delta$}}] at (axis cs:{329.479},{-54.992}) {}; % Ind,  4.41 
\node[pin={[pin distance=-0.6\onedegree,Bayer]00:{$\epsilon$}}] at (axis cs:{330.840},{-56.786}) {}; % Ind,  4.72 
\node[pin={[pin distance=-0.6\onedegree,Bayer]00:{$\eta$}}] at (axis cs:{311.010},{-51.921}) {}; % Ind,  4.52 
\node[pin={[pin distance=-0.6\onedegree,Bayer]00:{$\gamma$}}] at (axis cs:{321.564},{-54.660}) {}; % Ind,  6.09 
\node[pin={[pin distance=-0.6\onedegree,Bayer]00:{$\iota$}}] at (axis cs:{312.875},{-51.608}) {}; % Ind,  5.06 
\node[pin={[pin distance=-0.6\onedegree,Bayer]180:{$\kappa^1$}}] at (axis cs:{329.625},{-59.012}) {}; % Ind,  6.14 
\node[pin={[pin distance=-0.6\onedegree,Bayer]90:{$\kappa^2$}}] at (axis cs:{331.463},{-59.636}) {}; % Ind,  5.62 
\node[pin={[pin distance=-0.6\onedegree,Bayer]00:{$\mu$}}] at (axis cs:{316.309},{-54.727}) {}; % Ind,  5.17 
\node[pin={[pin distance=-0.6\onedegree,Bayer]00:{$\nu$}}] at (axis cs:{336.154},{-72.255}) {}; % Ind,  5.29 
\node[pin={[pin distance=-0.6\onedegree,Bayer]180:{$\omicron$}}] at (axis cs:{327.697},{-69.629}) {}; % Ind,  5.51 
\node[pin={[pin distance=-0.6\onedegree,Bayer]180:{$\pi$}}] at (axis cs:{329.059},{-57.899}) {}; % Ind,  6.18 
\node[pin={[pin distance=-0.4\onedegree,Bayer]90:{$\rho$}}] at (axis cs:{343.665},{-70.074}) {}; % Ind,  6.06 
\node[pin={[pin distance=-0.5\onedegree,Bayer]00:{$\vartheta$}}] at (axis cs:{319.967},{-53.449}) {}; % Ind,  4.50 
\node[pin={[pin distance=-0.6\onedegree,Bayer]00:{$\zeta$}}] at (axis cs:{312.371},{-46.227}) {}; % Ind,  4.89

\node[pin={[pin distance=-0.6\onedegree,Bayer]00:{$\tau$}}] at (axis cs:{234.664},{-29.778}) {}; % Lib,  3.66 
\node[pin={[pin distance=-0.6\onedegree,Bayer]00:{$\upsilon$}}] at (axis cs:{234.256},{-28.135}) {}; % Lib,  3.60 
\node[pin={[pin distance=-0.6\onedegree,Flaamsted]00:{$ 36$}}] at (axis cs:{233.656},{-28.047}) {}; % Lib,  5.13 
\node[pin={[pin distance=-0.4\onedegree,Bayer]00:{$\alpha$}}] at (axis cs:{220.482},{-47.388}) {}; % Lup,  2.29 
\node[pin={[pin distance=-0.6\onedegree,Bayer]00:{$\beta$}}] at (axis cs:{224.633},{-43.134}) {}; % Lup,  2.68 
\node[pin={[pin distance=-0.6\onedegree,Bayer]00:{$\chi$}}] at (axis cs:{237.740},{-33.627}) {}; % Lup,  3.96 
\node[pin={[pin distance=-0.6\onedegree,Bayer]00:{$\delta$}}] at (axis cs:{230.343},{-40.647}) {}; % Lup,  3.22 
\node[pin={[pin distance=-0.6\onedegree,Bayer]00:{$\epsilon$}}] at (axis cs:{230.670},{-44.690}) {}; % Lup,  3.38 
\node[pin={[pin distance=-0.6\onedegree,Bayer]00:{$\eta$}}] at (axis cs:{240.031},{-38.396}) {}; % Lup,  3.43 
\node[pin={[pin distance=-0.6\onedegree,Bayer]00:{$\gamma$}}] at (axis cs:{233.785},{-41.167}) {}; % Lup,  2.78 
\node[pin={[pin distance=-0.6\onedegree,Bayer]00:{$\iota$}}] at (axis cs:{214.851},{-46.058}) {}; % Lup,  3.55 
\node[pin={[pin distance=-0.6\onedegree,Bayer]00:{$\kappa^1$}}] at (axis cs:{227.984},{-48.738}) {}; % Lup,  3.86 
\node[pin={[pin distance=-0.6\onedegree,Bayer]00:{$\lambda$}}] at (axis cs:{227.211},{-45.280}) {}; % Lup,  4.06 
\node[pin={[pin distance=-0.6\onedegree,Bayer]00:{$\mu$}}] at (axis cs:{229.633},{-47.875}) {}; % Lup,  4.28 
\node[pin={[pin distance=-0.6\onedegree,Bayer]00:{$\nu^1$}}] at (axis cs:{230.534},{-47.928}) {}; % Lup,  4.99 
\node[pin={[pin distance=-0.6\onedegree,Bayer]00:{$\nu^2$}}] at (axis cs:{230.451},{-48.318}) {}; % Lup,  5.66 
\node[pin={[pin distance=-0.6\onedegree,Bayer]00:{$\omega$}}] at (axis cs:{234.513},{-42.567}) {}; % Lup,  4.34 
\node[pin={[pin distance=-0.6\onedegree,Bayer]00:{$\omicron$}}] at (axis cs:{222.910},{-43.575}) {}; % Lup,  4.33 
\node[pin={[pin distance=-0.6\onedegree,Bayer]00:{$\pi$}}] at (axis cs:{226.280},{-47.051}) {}; % Lup,  3.91 
\node[pin={[pin distance=-0.6\onedegree,Bayer]00:{$\psi^1$}}] at (axis cs:{234.942},{-34.412}) {}; % Lup,  4.66 
\node[pin={[pin distance=-0.6\onedegree,Bayer]00:{$\psi^2$}}] at (axis cs:{235.671},{-34.710}) {}; % Lup,  4.73 
\node[pin={[pin distance=-0.6\onedegree,Bayer]00:{$\rho$}}] at (axis cs:{219.472},{-49.426}) {}; % Lup,  4.05 
\node[pin={[pin distance=-0.6\onedegree,Bayer]00:{$\sigma$}}] at (axis cs:{218.154},{-50.457}) {}; % Lup,  4.43 
\node[pin={[pin distance=-0.6\onedegree,Bayer]00:{$\tau^1$}}] at (axis cs:{216.534},{-45.221}) {}; % Lup,  4.56 
\node[pin={[pin distance=-0.6\onedegree,Bayer]00:{$\tau^2$}}] at (axis cs:{216.545},{-45.379}) {}; % Lup,  4.35 
\node[pin={[pin distance=-0.6\onedegree,Bayer]00:{$\upsilon$}}] at (axis cs:{231.188},{-39.710}) {}; % Lup,  5.38 
\node[pin={[pin distance=-0.6\onedegree,Bayer]00:{$\varphi^1$}}] at (axis cs:{230.452},{-36.261}) {}; % Lup,  3.56 
\node[pin={[pin distance=-0.6\onedegree,Bayer]00:{$\varphi^2$}}] at (axis cs:{230.789},{-36.858}) {}; % Lup,  4.53 
\node[pin={[pin distance=-0.6\onedegree,Bayer]00:{$\vartheta$}}] at (axis cs:{241.648},{-36.802}) {}; % Lup,  4.22 
\node[pin={[pin distance=-0.6\onedegree,Bayer]00:{$\xi^1$}}] at (axis cs:{239.223},{-33.966}) {}; % Lup,  5.09 
\node[pin={[pin distance=-0.6\onedegree,Bayer]00:{$\zeta$}}] at (axis cs:{228.071},{-52.099}) {}; % Lup,  3.41 
\node[pin={[pin distance=-0.6\onedegree,Flaamsted]00:{$  1$}}] at (axis cs:{228.655},{-31.519}) {}; % Lup,  4.92 
\node[pin={[pin distance=-0.6\onedegree,Flaamsted]00:{$  2$}}] at (axis cs:{229.458},{-30.148}) {}; % Lup,  4.34 

\node[pin={[pin distance=-0.6\onedegree,Bayer]00:{$\alpha$}}] at (axis cs:{92.560},{-74.753}) {}; % Men,  5.07 
\node[pin={[pin distance=-0.6\onedegree,Bayer]90:{$\beta$}}] at (axis cs:{75.679},{-71.314}) {}; % Men,  5.30 
\node[pin={[pin distance=-0.6\onedegree,Bayer]00:{$\delta$}}] at (axis cs:{64.497},{-80.214}) {}; % Men,  5.69 
\node[pin={[pin distance=-0.6\onedegree,Bayer]00:{$\epsilon$}}] at (axis cs:{111.409},{-79.094}) {}; % Men,  5.53 
\node[pin={[pin distance=-0.6\onedegree,Bayer]00:{$\eta$}}] at (axis cs:{73.797},{-74.937}) {}; % Men,  5.46 
\node[pin={[pin distance=-0.6\onedegree,Bayer]00:{$\gamma$}}] at (axis cs:{82.971},{-76.341}) {}; % Men,  5.18 
\node[pin={[pin distance=-0.6\onedegree,Bayer]00:{$\iota$}}] at (axis cs:{83.901},{-78.821}) {}; % Men,  6.05 
\node[pin={[pin distance=-0.6\onedegree,Bayer]00:{$\kappa$}}] at (axis cs:{87.570},{-79.361}) {}; % Men,  5.46 
\node[pin={[pin distance=-0.6\onedegree,Bayer]90:{$\mu$}}] at (axis cs:{70.767},{-70.931}) {}; % Men,  5.53 
\node[pin={[pin distance=-0.6\onedegree,Bayer]00:{$\nu$}}] at (axis cs:{65.242},{-81.580}) {}; % Men,  5.77 
\node[pin={[pin distance=-0.6\onedegree,Bayer]00:{$\pi$}}] at (axis cs:{84.291},{-80.469}) {}; % Men,  5.65 
\node[pin={[pin distance=-0.6\onedegree,Bayer]00:{$\vartheta$}}] at (axis cs:{104.144},{-79.420}) {}; % Men,  5.46 
\node[pin={[pin distance=-0.6\onedegree,Bayer]00:{$\xi$}}] at (axis cs:{74.712},{-82.470}) {}; % Men,  5.84 
\node[pin={[pin distance=-0.6\onedegree,Bayer]00:{$\zeta$}}] at (axis cs:{100.012},{-80.813}) {}; % Men,  5.61 

\node[pin={[pin distance=-0.6\onedegree,Bayer]00:{$\alpha$}}] at (axis cs:{312.492},{-33.780}) {}; % Mic,  4.90 
\node[pin={[pin distance=-0.6\onedegree,Bayer]00:{$\beta$}}] at (axis cs:{312.995},{-33.178}) {}; % Mic,  6.06 
\node[pin={[pin distance=-0.6\onedegree,Bayer]00:{$\delta$}}] at (axis cs:{316.505},{-30.125}) {}; % Mic,  5.70 
\node[pin={[pin distance=-0.6\onedegree,Bayer]180:{$\epsilon$}}] at (axis cs:{319.485},{-32.172}) {}; % Mic,  4.72 
\node[pin={[pin distance=-0.6\onedegree,Bayer]00:{$\eta$}}] at (axis cs:{316.606},{-41.386}) {}; % Mic,  5.52 
\node[pin={[pin distance=-0.6\onedegree,Bayer]00:{$\gamma$}}] at (axis cs:{315.323},{-32.258}) {}; % Mic,  4.67 
\node[pin={[pin distance=-0.6\onedegree,Bayer]00:{$\iota$}}] at (axis cs:{312.121},{-43.988}) {}; % Mic,  5.11 
\node[pin={[pin distance=-0.6\onedegree,Bayer]180:{$\nu$}}] at (axis cs:{308.479},{-44.516}) {}; % Mic,  5.12 
\node[pin={[pin distance=-0.6\onedegree,Bayer]180:{$\vartheta^1$}}] at (axis cs:{320.190},{-40.809}) {}; % Mic,  4.81 
\node[pin={[pin distance=-0.6\onedegree,Bayer]-90:{$\vartheta^2$}}] at (axis cs:{321.103},{-41.007}) {}; % Mic,  5.78 
\node[pin={[pin distance=-0.6\onedegree,Bayer]00:{$\zeta$}}] at (axis cs:{315.741},{-38.631}) {}; % Mic,  5.33

\node[pin={[pin distance=-0.6\onedegree,Bayer]00:{$\alpha$}}] at (axis cs:{189.296},{-69.136}) {}; % Mus,  2.69 
\node[pin={[pin distance=-0.6\onedegree,Bayer]00:{$\beta$}}] at (axis cs:{191.570},{-68.108}) {}; % Mus,  3.08 
\node[pin={[pin distance=-0.4\onedegree,Bayer]00:{$\delta$}}] at (axis cs:{195.568},{-71.549}) {}; % Mus,  3.61 
\node[pin={[pin distance=-0.6\onedegree,Bayer]00:{$\epsilon$}}] at (axis cs:{184.393},{-67.961}) {}; % Mus,  4.06 
\node[pin={[pin distance=-0.6\onedegree,Bayer]00:{$\eta$}}] at (axis cs:{198.812},{-67.894}) {}; % Mus,  4.78 
\node[pin={[pin distance=-0.4\onedegree,Bayer]90:{$\gamma$}}] at (axis cs:{188.117},{-72.133}) {}; % Mus,  3.84 
\node[pin={[pin distance=-0.6\onedegree,Bayer]180:{$\iota^1$}}] at (axis cs:{201.280},{-74.888}) {}; % Mus,  5.04 
\node[pin={[pin distance=-0.6\onedegree,Bayer]00:{$\lambda$}}] at (axis cs:{176.402},{-66.729}) {}; % Mus,  3.63 
\node[pin={[pin distance=-0.6\onedegree,Bayer]00:{$\mu$}}] at (axis cs:{177.061},{-66.815}) {}; % Mus,  4.74 
\node[pin={[pin distance=-0.6\onedegree,Bayer]00:{$\vartheta$}}] at (axis cs:{197.030},{-65.306}) {}; % Mus,  5.66 
\node[pin={[pin distance=-0.6\onedegree,Bayer]00:{$\zeta^1$}}] at (axis cs:{185.550},{-68.307}) {}; % Mus,  5.74 
\node[pin={[pin distance=-0.6\onedegree,Bayer]00:{$\zeta^2$}}] at (axis cs:{185.531},{-67.522}) {}; % Mus,  5.16

%\node[pin={[pin distance=-0.6\onedegree,Bayer]00:{$\delta$}}] at (axis cs:{241.623},{-45.173}) {}; % Nor,  4.72 
%\node[pin={[pin distance=-0.6\onedegree,Bayer]00:{$\epsilon$}}] at (axis cs:{246.796},{-47.555}) {}; % Nor,  4.53 
\node[pin={[pin distance=-0.4\onedegree,Bayer]-90:{$\eta$}}] at (axis cs:{240.804},{-49.229}) {}; % Nor,  4.64 
\node[pin={[pin distance=-0.6\onedegree,Bayer]90:{$\gamma^1$}}] at (axis cs:{244.254},{-50.068}) {}; % Nor,  4.98 
\node[pin={[pin distance=-0.6\onedegree,Bayer]-90:{$\gamma^2$}}] at (axis cs:{244.960},{-50.155}) {}; % Nor,  4.01 
\node[pin={[pin distance=-0.6\onedegree,Bayer]180:{$\iota^1$}}] at (axis cs:{240.884},{-57.775}) {}; % Nor,  4.66 
\node[pin={[pin distance=-0.6\onedegree,Bayer]00:{$\iota^2$}}] at (axis cs:{242.327},{-57.934}) {}; % Nor,  5.58 
\node[pin={[pin distance=-0.6\onedegree,Bayer]00:{$\kappa$}}] at (axis cs:{243.370},{-54.630}) {}; % Nor,  4.95 
%\node[pin={[pin distance=-0.6\onedegree,Bayer]00:{$\lambda$}}] at (axis cs:{244.823},{-42.674}) {}; % Nor,  5.44 
\node[pin={[pin distance=-0.6\onedegree,Bayer]00:{$\mu$}}] at (axis cs:{248.521},{-44.045}) {}; % Nor,  4.92 
\node[pin={[pin distance=-0.6\onedegree,Bayer]00:{$\vartheta$}}] at (axis cs:{243.814},{-47.372}) {}; % Nor,  5.13 
\node[pin={[pin distance=-0.6\onedegree,Bayer]00:{$\zeta$}}] at (axis cs:{243.345},{-55.541}) {}; % Nor,  5.79

\node[pin={[pin distance=-0.6\onedegree,Bayer]00:{$\alpha$}}] at (axis cs:{316.179},{-77.024}) {}; % Oct,  5.13 
\node[pin={[pin distance=-0.6\onedegree,Bayer]90:{$\beta$}}] at (axis cs:{341.515},{-81.381}) {}; % Oct,  4.14 
\node[pin={[pin distance=-0.6\onedegree,Bayer]00:{$\chi$}}] at (axis cs:{283.696},{-87.606}) {}; % Oct,  5.28 
\node[pin={[pin distance=-0.6\onedegree,Bayer]00:{$\delta$}}] at (axis cs:{216.730},{-83.668}) {}; % Oct,  4.31 
\node[pin={[pin distance=-0.6\onedegree,Bayer]00:{$\epsilon$}}] at (axis cs:{335.007},{-80.440}) {}; % Oct,  5.18 
\node[pin={[pin distance=-0.6\onedegree,Bayer]00:{$\eta$}}] at (axis cs:{164.807},{-84.594}) {}; % Oct,  6.19 
\node[pin={[pin distance=-0.6\onedegree,Bayer]180:{$\gamma^{1,2}$}}] at (axis cs:{358.027},{-82.019}) {}; % Oct,  5.11 
%\node[pin={[pin distance=-0.6\onedegree,Bayer]00:{$\gamma^2$}}] at (axis cs:{359.387},{-82.170}) {}; % Oct,  5.73 
\node[pin={[pin distance=-0.6\onedegree,Bayer]90:{$\gamma^3$}}] at (axis cs:{2.509},{-82.224}) {}; % Oct,  5.28 
\node[pin={[pin distance=-0.6\onedegree,Bayer]00:{$\iota$}}] at (axis cs:{193.745},{-85.123}) {}; % Oct,  5.46 
\node[pin={[pin distance=-0.6\onedegree,Bayer]00:{$\kappa$}}] at (axis cs:{205.231},{-85.786}) {}; % Oct,  5.56 
\node[pin={[pin distance=-0.6\onedegree,Bayer]90:{$\lambda$}}] at (axis cs:{327.727},{-82.719}) {}; % Oct,  5.29 
\node[pin={[pin distance=-0.6\onedegree,Bayer]00:{$\mu^1$}}] at (axis cs:{310.512},{-76.180}) {}; % Oct,  5.99 
%\node[pin={[pin distance=-0.6\onedegree,Bayer]00:{$\mu^2$}}] at (axis cs:{310.434},{-75.351}) {}; % Oct,  6.44 
\node[pin={[pin distance=-0.6\onedegree,Bayer]00:{$\nu$}}] at (axis cs:{325.369},{-77.390}) {}; % Oct,  3.74 
\node[pin={[pin distance=-0.6\onedegree,Bayer]00:{$\omega$}}] at (axis cs:{227.787},{-84.788}) {}; % Oct,  5.88 
\node[pin={[pin distance=-0.6\onedegree,Bayer]-90:{$\pi^{1,2}$}}] at (axis cs:{225.462},{-83.227}) {}; % Oct,  5.65 
%\node[pin={[pin distance=-0.6\onedegree,Bayer]00:{$\pi^2$}}] at (axis cs:{226.196},{-83.038}) {}; % Oct,  5.64 
\node[pin={[pin distance=-0.6\onedegree,Bayer]00:{$\psi$}}] at (axis cs:{334.461},{-77.511}) {}; % Oct,  5.48 
\node[pin={[pin distance=-0.6\onedegree,Bayer]00:{$\rho$}}] at (axis cs:{235.821},{-84.465}) {}; % Oct,  5.57 
\node[pin={[pin distance=-0.6\onedegree,Bayer]00:{$\sigma$}}] at (axis cs:{317.195},{-88.956}) {}; % Oct,  5.45 
\node[pin={[pin distance=-0.6\onedegree,Bayer]180:{$\tau$}}] at (axis cs:{352.016},{-87.482}) {}; % Oct,  5.50 
\node[pin={[pin distance=-0.6\onedegree,Bayer]00:{$\upsilon$}}] at (axis cs:{337.906},{-85.967}) {}; % Oct,  5.76 
\node[pin={[pin distance=-0.6\onedegree,Bayer]00:{$\varphi$}}] at (axis cs:{275.902},{-75.044}) {}; % Oct,  5.47 
\node[pin={[pin distance=-0.6\onedegree,Bayer]180:{$\vartheta$}}] at (axis cs:{0.399},{-77.065}) {}; % Oct,  4.79
\node[pin={[pin distance=-0.6\onedegree,Bayer]00:{$\xi$}}] at (axis cs:{342.595},{-80.124}) {}; % Oct,  5.32 
\node[pin={[pin distance=-0.6\onedegree,Bayer]00:{$\zeta$}}] at (axis cs:{134.171},{-85.663}) {}; % Oct,  5.43

\node[pin={[pin distance=-0.2\onedegree,Bayer]90:{$\alpha$}}] at (axis cs:{306.412},{-56.735}) {}; % Pav,  1.92 
\node[pin={[pin distance=-0.5\onedegree,Bayer]-90:{$\beta$}}] at (axis cs:{311.240},{-66.203}) {}; % Pav,  3.42 
\node[pin={[pin distance=-0.5\onedegree,Bayer]-90:{$\delta$}}] at (axis cs:{302.182},{-66.182}) {}; % Pav,  3.55 
\node[pin={[pin distance=-0.6\onedegree,Bayer]00:{$\epsilon$}}] at (axis cs:{300.148},{-72.910}) {}; % Pav,  3.95 
\node[pin={[pin distance=-0.4\onedegree,Bayer]-90:{$\eta$}}] at (axis cs:{266.433},{-64.724}) {}; % Pav,  3.59 
\node[pin={[pin distance=-0.8\onedegree,Bayer]-135:{$\gamma$}}] at (axis cs:{321.611},{-65.366}) {}; % Pav,  4.23 
\node[pin={[pin distance=-0.6\onedegree,Bayer]00:{$\iota$}}] at (axis cs:{272.609},{-62.002}) {}; % Pav,  5.48 
\node[pin={[pin distance=-0.6\onedegree,Bayer]00:{$\kappa$}}] at (axis cs:{284.238},{-67.233}) {}; % Pav,  4.40 
\node[pin={[pin distance=-0.6\onedegree,Bayer]00:{$\lambda$}}] at (axis cs:{283.054},{-62.188}) {}; % Pav,  4.22 
\node[pin={[pin distance=-0.6\onedegree,Bayer]90:{$\mu^{1,2}$}}] at (axis cs:{300.096},{-66.949}) {}; % Pav,  5.76 
%\node[pin={[pin distance=-0.6\onedegree,Bayer]00:{$\mu^2$}}] at (axis cs:{300.469},{-66.944}) {}; % Pav,  5.33 
\node[pin={[pin distance=-0.6\onedegree,Bayer]00:{$\nu$}}] at (axis cs:{277.843},{-62.278}) {}; % Pav,  4.62 
\node[pin={[pin distance=-0.6\onedegree,Bayer]00:{$\omega$}}] at (axis cs:{284.652},{-60.201}) {}; % Pav,  5.13 
\node[pin={[pin distance=-0.6\onedegree,Bayer]00:{$\omicron$}}] at (axis cs:{318.335},{-70.126}) {}; % Pav,  5.08 
\node[pin={[pin distance=-0.6\onedegree,Bayer]00:{$\pi$}}] at (axis cs:{272.145},{-63.669}) {}; % Pav,  4.33 
\node[pin={[pin distance=-0.6\onedegree,Bayer]00:{$\rho$}}] at (axis cs:{309.397},{-61.530}) {}; % Pav,  4.86 
\node[pin={[pin distance=-0.6\onedegree,Bayer]90:{$\sigma$}}] at (axis cs:{312.326},{-68.776}) {}; % Pav,  5.41 
\node[pin={[pin distance=-0.6\onedegree,Bayer]00:{$\tau$}}] at (axis cs:{289.119},{-69.191}) {}; % Pav,  6.26 
\node[pin={[pin distance=-0.6\onedegree,Bayer]00:{$\upsilon$}}] at (axis cs:{310.488},{-66.761}) {}; % Pav,  5.15 
\node[pin={[pin distance=-0.5\onedegree,Bayer]90:{$\varphi^1$}}] at (axis cs:{308.895},{-60.582}) {}; % Pav,  4.76 
\node[pin={[pin distance=-0.6\onedegree,Bayer]00:{$\varphi^2$}}] at (axis cs:{310.011},{-60.549}) {}; % Pav,  5.12 
\node[pin={[pin distance=-0.6\onedegree,Bayer]-90:{$\vartheta$}}] at (axis cs:{282.158},{-65.077}) {}; % Pav,  5.71 
\node[pin={[pin distance=-0.6\onedegree,Bayer]00:{$\xi$}}] at (axis cs:{275.807},{-61.494}) {}; % Pav,  4.36 
\node[pin={[pin distance=-0.6\onedegree,Bayer]-90:{$\zeta$}}] at (axis cs:{280.759},{-71.428}) {}; % Pav,  4.01

\node[pin={[pin distance=-0.4\onedegree,Bayer]00:{$\alpha$}}] at (axis cs:{6.571},{-42.306}) {}; % Phe,  2.40 
\node[pin={[pin distance=-0.4\onedegree,Bayer]00:{$\beta$}}] at (axis cs:{16.521},{-46.718}) {}; % Phe,  3.32 
\node[pin={[pin distance=-0.6\onedegree,Bayer]00:{$\chi$}}] at (axis cs:{30.427},{-44.713}) {}; % Phe,  5.14 
\node[pin={[pin distance=-0.4\onedegree,Bayer]00:{$\delta$}}] at (axis cs:{22.813},{-49.073}) {}; % Phe,  3.94 
\node[pin={[pin distance=-0.6\onedegree,Bayer]00:{$\epsilon$}}] at (axis cs:{2.353},{-45.747}) {}; % Phe,  3.88 
\node[pin={[pin distance=-0.6\onedegree,Bayer]00:{$\eta$}}] at (axis cs:{10.838},{-57.463}) {}; % Phe,  4.37 
\node[pin={[pin distance=-0.6\onedegree,Bayer]00:{$\gamma$}}] at (axis cs:{22.091},{-43.318}) {}; % Phe,  3.44 
\node[pin={[pin distance=-0.6\onedegree,Bayer]00:{$\iota$}}] at (axis cs:{353.769},{-42.615}) {}; % Phe,  4.71 
\node[pin={[pin distance=-0.6\onedegree,Bayer]00:{$\kappa$}}] at (axis cs:{6.551},{-43.680}) {}; % Phe,  3.95 
\node[pin={[pin distance=-0.6\onedegree,Bayer]00:{$\lambda^1$}}] at (axis cs:{7.854},{-48.803}) {}; % Phe,  4.76 
\node[pin={[pin distance=-0.6\onedegree,Bayer]00:{$\lambda^2$}}] at (axis cs:{8.922},{-48.001}) {}; % Phe,  5.52 
\node[pin={[pin distance=-0.6\onedegree,Bayer]00:{$\mu$}}] at (axis cs:{10.331},{-46.085}) {}; % Phe,  4.60 
\node[pin={[pin distance=-0.6\onedegree,Bayer]00:{$\nu$}}] at (axis cs:{18.796},{-45.531}) {}; % Phe,  4.97 
\node[pin={[pin distance=-0.6\onedegree,Bayer]00:{$\omega$}}] at (axis cs:{15.508},{-57.002}) {}; % Phe,  6.10 
\node[pin={[pin distance=-0.6\onedegree,Bayer]00:{$\pi$}}] at (axis cs:{359.732},{-52.746}) {}; % Phe,  5.13 
\node[pin={[pin distance=-0.6\onedegree,Bayer]00:{$\psi$}}] at (axis cs:{28.411},{-46.303}) {}; % Phe,  4.43 
\node[pin={[pin distance=-0.6\onedegree,Bayer]00:{$\rho$}}] at (axis cs:{12.672},{-50.987}) {}; % Phe,  5.23 
\node[pin={[pin distance=-0.6\onedegree,Bayer]00:{$\sigma$}}] at (axis cs:{356.817},{-50.226}) {}; % Phe,  5.17 
\node[pin={[pin distance=-0.6\onedegree,Bayer]00:{$\tau$}}] at (axis cs:{0.269},{-48.810}) {}; % Phe,  5.71 
\node[pin={[pin distance=-0.6\onedegree,Bayer]00:{$\upsilon$}}] at (axis cs:{16.949},{-41.487}) {}; % Phe,  5.21 
\node[pin={[pin distance=-0.6\onedegree,Bayer]00:{$\varphi$}}] at (axis cs:{28.592},{-42.497}) {}; % Phe,  5.12 
\node[pin={[pin distance=-0.6\onedegree,Bayer]00:{$\vartheta$}}] at (axis cs:{354.866},{-46.638}) {}; % Phe,  6.42 
\node[pin={[pin distance=-0.6\onedegree,Bayer]00:{$\xi$}}] at (axis cs:{10.443},{-56.501}) {}; % Phe,  5.71 
\node[pin={[pin distance=-0.6\onedegree,Bayer]00:{$\zeta$}}] at (axis cs:{17.096},{-55.246}) {}; % Phe,  3.98

\node[pin={[pin distance=-0.3\onedegree,Bayer]00:{$\alpha$}}] at (axis cs:{102.048},{-61.941}) {}; % Pic,  3.25 
\node[pin={[pin distance=-0.4\onedegree,Bayer]00:{$\beta$}}] at (axis cs:{86.821},{-51.066}) {}; % Pic,  3.86 
\node[pin={[pin distance=-0.6\onedegree,Bayer]-90:{$\delta$}}] at (axis cs:{92.575},{-54.969}) {}; % Pic,  4.72 
\node[pin={[pin distance=-0.6\onedegree,Bayer]-90:{$\eta^1$}}] at (axis cs:{75.703},{-49.151}) {}; % Pic,  5.38 
\node[pin={[pin distance=-0.6\onedegree,Bayer]00:{$\eta^2$}}] at (axis cs:{76.242},{-49.578}) {}; % Pic,  5.04 
\node[pin={[pin distance=-0.6\onedegree,Bayer]00:{$\gamma$}}] at (axis cs:{87.457},{-56.166}) {}; % Pic,  4.50 
\node[pin={[pin distance=-0.6\onedegree,Bayer]00:{$\iota$}}] at (axis cs:{72.730},{-53.461}) {}; % Pic,  5.62 
\node[pin={[pin distance=-0.6\onedegree,Bayer]00:{$\kappa$}}] at (axis cs:{80.592},{-56.134}) {}; % Pic,  6.10 
\node[pin={[pin distance=-0.6\onedegree,Bayer]90:{$\lambda$}}] at (axis cs:{70.693},{-50.481}) {}; % Pic,  5.30 
\node[pin={[pin distance=-0.6\onedegree,Bayer]180:{$\mu$}}] at (axis cs:{97.993},{-58.754}) {}; % Pic,  5.72 
\node[pin={[pin distance=-0.6\onedegree,Bayer]00:{$\nu$}}] at (axis cs:{95.733},{-56.370}) {}; % Pic,  5.61 
\node[pin={[pin distance=-0.6\onedegree,Bayer]00:{$\vartheta$}}] at (axis cs:{81.193},{-52.316}) {}; % Pic,  6.24 
\node[pin={[pin distance=-0.6\onedegree,Bayer]-90:{$\zeta$}}] at (axis cs:{79.842},{-50.606}) {}; % Pic,  5.44

\node[pin={[pin distance=-0.\onedegree,Bayer]90:{$\alpha$}}] at (axis cs:{344.413},{-29.622}) {}; % PsA,  1.23 
\node[pin={[pin distance=-0.6\onedegree,Bayer]00:{$\beta$}}] at (axis cs:{337.876},{-32.346}) {}; % PsA,  4.29 
\node[pin={[pin distance=-0.6\onedegree,Bayer]00:{$\delta$}}] at (axis cs:{343.987},{-32.540}) {}; % PsA,  4.22 
\node[pin={[pin distance=-0.6\onedegree,Bayer]00:{$\epsilon$}}] at (axis cs:{340.164},{-27.043}) {}; % PsA,  4.19 
\node[pin={[pin distance=-0.6\onedegree,Bayer]00:{$\eta$}}] at (axis cs:{330.209},{-28.454}) {}; % PsA,  5.43 
\node[pin={[pin distance=-0.4\onedegree,Bayer]90:{$\gamma$}}] at (axis cs:{343.131},{-32.875}) {}; % PsA,  4.51 
\node[pin={[pin distance=-0.6\onedegree,Bayer]00:{$\iota$}}] at (axis cs:{326.237},{-33.026}) {}; % PsA,  4.34 
\node[pin={[pin distance=-0.6\onedegree,Bayer]00:{$\lambda$}}] at (axis cs:{333.578},{-27.767}) {}; % PsA,  5.44 
\node[pin={[pin distance=-0.6\onedegree,Bayer]180:{$\mu$}}] at (axis cs:{332.096},{-32.988}) {}; % PsA,  4.50 
\node[pin={[pin distance=-0.6\onedegree,Bayer]180:{$\pi$}}] at (axis cs:{345.874},{-34.749}) {}; % PsA,  5.13 
\node[pin={[pin distance=-0.6\onedegree,Bayer]00:{$\tau$}}] at (axis cs:{332.537},{-32.548}) {}; % PsA,  4.94 
\node[pin={[pin distance=-0.6\onedegree,Bayer]180:{$\upsilon$}}] at (axis cs:{332.108},{-34.044}) {}; % PsA,  4.97 
\node[pin={[pin distance=-0.6\onedegree,Bayer]00:{$\vartheta$}}] at (axis cs:{326.934},{-30.898}) {}; % PsA,  5.02 
\node[pin={[pin distance=-0.6\onedegree,Flaamsted]00:{$ 13$}}] at (axis cs:{331.099},{-29.916}) {}; % PsA,  6.45 
\node[pin={[pin distance=-0.6\onedegree,Flaamsted]00:{$ 19$}}] at (axis cs:{340.592},{-29.361}) {}; % PsA,  6.17 
\node[pin={[pin distance=-0.6\onedegree,Flaamsted]00:{$ 21$}}] at (axis cs:{342.837},{-29.536}) {}; % PsA,  5.99 
\node[pin={[pin distance=-0.6\onedegree,Flaamsted]00:{$  6$}}] at (axis cs:{323.061},{-33.944}) {}; % PsA,  5.97 
\node[pin={[pin distance=-0.6\onedegree,Flaamsted]00:{$  7$}}] at (axis cs:{324.203},{-33.048}) {}; % PsA,  6.11

\node[pin={[pin distance=-0.4\onedegree,Bayer]00:{$\zeta$}}] at (axis cs:{120.896},{-40.003}) {}; % Pup,  2.22 
\node[pin={[pin distance=-0.4\onedegree,Bayer]180:{$\nu$}}] at (axis cs:{99.440},{-43.196}) {}; % Pup,  3.18 
\node[pin={[pin distance=-0.4\onedegree,Bayer]-90:{$\pi$}}] at (axis cs:{109.286},{-37.097}) {}; % Pup,  2.71 
\node[pin={[pin distance=-0.4\onedegree,Bayer]00:{$\sigma$}}] at (axis cs:{112.308},{-43.301}) {}; % Pup,  3.26 
\node[pin={[pin distance=-0.4\onedegree,Bayer]00:{$\tau$}}] at (axis cs:{102.484},{-50.614}) {}; % Pup,  2.94 

\node[pin={[pin distance=-0.6\onedegree,Bayer]00:{$\alpha$}}] at (axis cs:{130.898},{-33.186}) {}; % Pyx,  3.69 
\node[pin={[pin distance=-0.6\onedegree,Bayer]00:{$\beta$}}] at (axis cs:{130.026},{-35.308}) {}; % Pyx,  3.97 
\node[pin={[pin distance=-0.6\onedegree,Bayer]00:{$\delta$}}] at (axis cs:{133.882},{-27.682}) {}; % Pyx,  4.88 
\node[pin={[pin distance=-0.6\onedegree,Bayer]00:{$\epsilon$}}] at (axis cs:{137.485},{-30.365}) {}; % Pyx,  5.60 
\node[pin={[pin distance=-0.6\onedegree,Bayer]00:{$\gamma$}}] at (axis cs:{132.633},{-27.710}) {}; % Pyx,  4.02 
\node[pin={[pin distance=-0.6\onedegree,Bayer]00:{$\lambda$}}] at (axis cs:{140.801},{-28.834}) {}; % Pyx,  4.72 
\node[pin={[pin distance=-0.6\onedegree,Bayer]00:{$\zeta$}}] at (axis cs:{129.927},{-29.561}) {}; % Pyx,  4.88

\node[pin={[pin distance=-0.4\onedegree,Bayer]-90:{$\alpha$}}] at (axis cs:{63.606},{-62.474}) {}; % Ret,  3.34 
\node[pin={[pin distance=-0.4\onedegree,Bayer]180:{$\beta$}}] at (axis cs:{56.050},{-64.807}) {}; % Ret,  3.84 
\node[pin={[pin distance=-0.6\onedegree,Bayer]180:{$\delta$}}] at (axis cs:{59.686},{-61.400}) {}; % Ret,  4.56 
\node[pin={[pin distance=-0.6\onedegree,Bayer]00:{$\epsilon$}}] at (axis cs:{64.121},{-59.302}) {}; % Ret,  4.44 
\node[pin={[pin distance=-0.6\onedegree,Bayer]90:{$\eta$}}] at (axis cs:{65.472},{-63.386}) {}; % Ret,  5.24 
\node[pin={[pin distance=-0.6\onedegree,Bayer]180:{$\gamma$}}] at (axis cs:{60.224},{-62.159}) {}; % Ret,  4.51 
\node[pin={[pin distance=-0.6\onedegree,Bayer]00:{$\iota$}}] at (axis cs:{60.326},{-61.079}) {}; % Ret,  4.96 
\node[pin={[pin distance=-0.6\onedegree,Bayer]00:{$\kappa$}}] at (axis cs:{52.344},{-62.937}) {}; % Ret,  4.71 
\node[pin={[pin distance=-0.6\onedegree,Bayer]180:{$\vartheta$}}] at (axis cs:{64.418},{-63.255}) {}; % Ret,  6.05 
\node[pin={[pin distance=-0.6\onedegree,Bayer]00:{$\zeta^2$}}] at (axis cs:{49.553},{-62.506}) {}; % Ret,  5.24

\node[pin={[pin distance=-0.6\onedegree,Bayer]00:{$\alpha$}}] at (axis cs:{14.652},{-29.357}) {}; % Scl,  4.31 
\node[pin={[pin distance=-0.6\onedegree,Bayer]00:{$\beta$}}] at (axis cs:{353.243},{-37.818}) {}; % Scl,  4.38 
\node[pin={[pin distance=-0.6\onedegree,Bayer]00:{$\delta$}}] at (axis cs:{357.231},{-28.130}) {}; % Scl,  4.58 
\node[pin={[pin distance=-0.6\onedegree,Bayer]-90:{$\eta$}}] at (axis cs:{6.982},{-33.007}) {}; % Scl,  4.87 
\node[pin={[pin distance=-0.6\onedegree,Bayer]180:{$\gamma$}}] at (axis cs:{349.706},{-32.532}) {}; % Scl,  4.41 
\node[pin={[pin distance=-0.6\onedegree,Bayer]00:{$\iota$}}] at (axis cs:{5.380},{-28.981}) {}; % Scl,  5.17 
\node[pin={[pin distance=-0.6\onedegree,Bayer]00:{$\kappa^1$}}] at (axis cs:{2.338},{-27.988}) {}; % Scl,  5.44 
\node[pin={[pin distance=-0.6\onedegree,Bayer]00:{$\kappa^2$}}] at (axis cs:{2.893},{-27.799}) {}; % Scl,  5.41 
\node[pin={[pin distance=-0.6\onedegree,Bayer]-90:{$\lambda^1$}}] at (axis cs:{10.679},{-38.463}) {}; % Scl,  6.08 
\node[pin={[pin distance=-0.6\onedegree,Bayer]00:{$\lambda^2$}}] at (axis cs:{11.050},{-38.422}) {}; % Scl,  5.90 
\node[pin={[pin distance=-0.6\onedegree,Bayer]00:{$\mu$}}] at (axis cs:{355.159},{-32.073}) {}; % Scl,  5.31 
\node[pin={[pin distance=-0.6\onedegree,Bayer]00:{$\pi$}}] at (axis cs:{25.536},{-32.327}) {}; % Scl,  5.26 
\node[pin={[pin distance=-0.6\onedegree,Bayer]00:{$\sigma$}}] at (axis cs:{15.610},{-31.552}) {}; % Scl,  5.51 
\node[pin={[pin distance=-0.6\onedegree,Bayer]00:{$\tau$}}] at (axis cs:{24.035},{-29.907}) {}; % Scl,  5.70 
\node[pin={[pin distance=-0.6\onedegree,Bayer]00:{$\vartheta$}}] at (axis cs:{2.933},{-35.133}) {}; % Scl,  5.24 
\node[pin={[pin distance=-0.6\onedegree,Bayer]00:{$\xi$}}] at (axis cs:{15.326},{-38.916}) {}; % Scl,  5.59 
\node[pin={[pin distance=-0.6\onedegree,Bayer]00:{$\zeta$}}] at (axis cs:{0.583},{-29.720}) {}; % Scl,  5.04

\node[pin={[pin distance=-0.4\onedegree,Bayer]00:{$\epsilon$}}] at (axis cs:{252.541},{-34.293}) {}; % Sco,  2.29 
\node[pin={[pin distance=-0.4\onedegree,Bayer]00:{$\kappa$}}] at (axis cs:{265.622},{-39.030}) {}; % Sco,  2.39 
\node[pin={[pin distance=-0.2\onedegree,Bayer]00:{$\lambda$}}] at (axis cs:{263.402},{-37.104}) {}; % Sco,  1.63 
\node[pin={[pin distance=-0.2\onedegree,Bayer]-90:{$\vartheta$}}] at (axis cs:{264.330},{-42.998}) {}; % Sco,  1.86 
\node[pin={[pin distance=-0.6\onedegree,Bayer]90:{$\eta$}}] at (axis cs:{258.038},{-43.239}) {}; % Sco,  3.32 
\node[pin={[pin distance=-0.4\onedegree,Bayer]00:{$\iota^1$}}] at (axis cs:{266.896},{-40.127}) {}; % Sco,  3.01 
\node[pin={[pin distance=-0.6\onedegree,Bayer]180:{$\iota^2$}}] at (axis cs:{267.546},{-40.090}) {}; % Sco,  4.81 
\node[pin={[pin distance=-0.4\onedegree,Bayer]90:{$\mu^{1,2}$}}] at (axis cs:{252.968},{-38.047}) {}; % Sco,  3.00 
%\node[pin={[pin distance=-0.6\onedegree,Bayer]00:{$\mu^2$}}] at (axis cs:{253.084},{-38.018}) {}; % Sco,  3.56 
\node[pin={[pin distance=-0.6\onedegree,Bayer]00:{$\rho$}}] at (axis cs:{239.221},{-29.214}) {}; % Sco,  3.88 
\node[pin={[pin distance=-0.6\onedegree,Bayer]00:{$\tau$}}] at (axis cs:{248.971},{-28.216}) {}; % Sco,  2.83 
\node[pin={[pin distance=-0.4\onedegree,Bayer]00:{$\upsilon$}}] at (axis cs:{262.691},{-37.296}) {}; % Sco,  2.68 
\node[pin={[pin distance=-0.8\onedegree,Bayer]60:{$\zeta^{1,2}$}}] at (axis cs:{253.499},{-42.362}) {}; % Sco,  4.77 
%\node[pin={[pin distance=-0.6\onedegree,Bayer]00:{$\zeta^2$}}] at (axis cs:{253.646},{-42.361}) {}; % Sco,  3.61 
\node[pin={[pin distance=-0.6\onedegree,Flaamsted]00:{$ 12$}}] at (axis cs:{243.067},{-28.417}) {}; % Sco,  5.67 
\node[pin={[pin distance=-0.6\onedegree,Flaamsted]00:{$ 13$}}] at (axis cs:{243.076},{-27.926}) {}; % Sco,  4.58 
\node[pin={[pin distance=-0.6\onedegree,Flaamsted]-90:{$ 27$}}] at (axis cs:{254.297},{-33.259}) {}; % Sco,  5.48


\node[pin={[pin distance=-0.2\onedegree,Bayer]00:{$\epsilon$}}] at (axis cs:{276.043},{-34.385}) {}; % Sgr,  1.81 
\node[pin={[pin distance=-0.4\onedegree,Bayer]00:{$\alpha$}}] at (axis cs:{290.972},{-40.616}) {}; % Sgr,  3.96 
\node[pin={[pin distance=-0.6\onedegree,Bayer]180:{$\beta^1$}}] at (axis cs:{290.660},{-44.459}) {}; % Sgr,  4.01 
\node[pin={[pin distance=-0.6\onedegree,Bayer]-90:{$\beta^2$}}] at (axis cs:{290.805},{-44.800}) {}; % Sgr,  4.28 
\node[pin={[pin distance=-0.4\onedegree,Bayer]00:{$\delta$}}] at (axis cs:{275.249},{-29.828}) {}; % Sgr,  2.70 
\node[pin={[pin distance=-0.4\onedegree,Bayer]180:{$\eta$}}] at (axis cs:{274.407},{-36.761}) {}; % Sgr,  3.13 
%\node[pin={[pin distance=-0.6\onedegree,Bayer]00:{$\gamma^1$}}] at (axis cs:{271.255},{-29.580}) {}; % Sgr,  4.65 
\node[pin={[pin distance=-0.6\onedegree,Bayer]00:{$\gamma^2$}}] at (axis cs:{271.452},{-30.424}) {}; % Sgr,  3.63 
\node[pin={[pin distance=-0.4\onedegree,Bayer]00:{$\iota$}}] at (axis cs:{298.815},{-41.868}) {}; % Sgr,  4.12 
\node[pin={[pin distance=-0.6\onedegree,Bayer]180:{$\kappa^1$}}] at (axis cs:{305.615},{-42.049}) {}; % Sgr,  5.58 
\node[pin={[pin distance=-0.6\onedegree,Bayer]00:{$\kappa^2$}}] at (axis cs:{305.972},{-42.423}) {}; % Sgr,  5.64 
\node[pin={[pin distance=-0.6\onedegree,Bayer]00:{$\tau$}}] at (axis cs:{286.735},{-27.670}) {}; % Sgr,  3.32 
\node[pin={[pin distance=-1\onedegree,Bayer]80:{$\vartheta^1$}}] at (axis cs:{299.934},{-35.276}) {}; % Sgr,  4.37 
\node[pin={[pin distance=-1\onedegree,Bayer]100:{$\vartheta^2$}}] at (axis cs:{299.964},{-34.698}) {}; % Sgr,  5.31 
\node[pin={[pin distance=-0.4\onedegree,Bayer]00:{$\zeta$}}] at (axis cs:{285.653},{-29.880}) {}; % Sgr,  2.61 
\node[pin={[pin distance=-0.6\onedegree,Flaamsted]-90:{$ 18$}}] at (axis cs:{276.256},{-30.756}) {}; % Sgr,  5.58 
\node[pin={[pin distance=-0.6\onedegree,Flaamsted]00:{$  3$}}] at (axis cs:{266.890},{-27.831}) {}; % Sgr,  4.56 
\node[pin={[pin distance=-0.6\onedegree,Flaamsted]00:{$ 59$}}] at (axis cs:{299.237},{-27.170}) {}; % Sgr,  4.53 
\node[pin={[pin distance=-0.6\onedegree,Flaamsted]00:{$ 62$}}] at (axis cs:{300.665},{-27.710}) {}; % Sgr,  4.49

\node[pin={[pin distance=-0.4\onedegree,Bayer]00:{$\alpha$}}] at (axis cs:{276.743},{-45.968}) {}; % Tel,  3.49 
\node[pin={[pin distance=-0.6\onedegree,Bayer]00:{$\delta^{1,2}$}}] at (axis cs:{277.939},{-45.915}) {}; % Tel,  4.93 
%\node[pin={[pin distance=-0.6\onedegree,Bayer]00:{$\delta^2$}}] at (axis cs:{278.008},{-45.757}) {}; % Tel,  5.08 
\node[pin={[pin distance=-0.6\onedegree,Bayer]00:{$\epsilon$}}] at (axis cs:{272.807},{-45.954}) {}; % Tel,  4.52 
\node[pin={[pin distance=-0.6\onedegree,Bayer]00:{$\eta$}}] at (axis cs:{290.713},{-54.424}) {}; % Tel,  5.03 
\node[pin={[pin distance=-0.6\onedegree,Bayer]00:{$\iota$}}] at (axis cs:{293.804},{-48.099}) {}; % Tel,  4.89 
\node[pin={[pin distance=-0.6\onedegree,Bayer]00:{$\kappa$}}] at (axis cs:{283.165},{-52.107}) {}; % Tel,  5.18 
\node[pin={[pin distance=-0.6\onedegree,Bayer]00:{$\lambda$}}] at (axis cs:{284.616},{-52.939}) {}; % Tel,  4.85 
\node[pin={[pin distance=-0.6\onedegree,Bayer]00:{$\mu$}}] at (axis cs:{292.644},{-55.110}) {}; % Tel,  6.30 
\node[pin={[pin distance=-0.6\onedegree,Bayer]180:{$\nu$}}] at (axis cs:{297.005},{-56.362}) {}; % Tel,  5.34 
\node[pin={[pin distance=-0.6\onedegree,Bayer]00:{$\rho$}}] at (axis cs:{286.583},{-52.341}) {}; % Tel,  5.18 
\node[pin={[pin distance=-0.6\onedegree,Bayer]00:{$\xi$}}] at (axis cs:{301.846},{-52.881}) {}; % Tel,  4.92 
\node[pin={[pin distance=-0.6\onedegree,Bayer]180:{$\zeta$}}] at (axis cs:{277.208},{-49.071}) {}; % Tel,  4.11

\node[pin={[pin distance=-0.3\onedegree,Bayer]180:{$\alpha$}}] at (axis cs:{252.166},{-69.028}) {}; % TrA,  1.91 
\node[pin={[pin distance=-0.4\onedegree,Bayer]00:{$\beta$}}] at (axis cs:{238.786},{-63.430}) {}; % TrA,  2.84 
\node[pin={[pin distance=-0.4\onedegree,Bayer]00:{$\delta$}}] at (axis cs:{243.859},{-63.686}) {}; % TrA,  3.85 
\node[pin={[pin distance=-0.6\onedegree,Bayer]00:{$\epsilon$}}] at (axis cs:{234.180},{-66.317}) {}; % TrA,  4.11 
%\node[pin={[pin distance=-0.5\onedegree,Bayer]90:{$\eta^1$}}] at (axis cs:{250.346},{-68.296}) {}; % TrA,  5.89 
\node[pin={[pin distance=-0.4\onedegree,Bayer]00:{$\gamma$}}] at (axis cs:{229.727},{-68.679}) {}; % TrA,  2.88 
\node[pin={[pin distance=-0.4\onedegree,Bayer]-90:{$\iota$}}] at (axis cs:{246.989},{-64.058}) {}; % TrA,  5.29 
\node[pin={[pin distance=-0.6\onedegree,Bayer]00:{$\kappa$}}] at (axis cs:{238.873},{-68.603}) {}; % TrA,  5.10 
\node[pin={[pin distance=-0.6\onedegree,Bayer]00:{$\vartheta$}}] at (axis cs:{248.937},{-65.495}) {}; % TrA,  5.50 
\node[pin={[pin distance=-0.6\onedegree,Bayer]180:{$\zeta$}}] at (axis cs:{247.117},{-70.084}) {}; % TrA,  4.91

\node[pin={[pin distance=-0.4\onedegree,Bayer]00:{$\alpha$}}] at (axis cs:{334.625},{-60.259}) {}; % Tuc,  2.86 
\node[pin={[pin distance=-0.6\onedegree,Bayer]180:{$\beta^1$}}] at (axis cs:{7.886},{-62.958}) {}; % Tuc,  4.34 
\node[pin={[pin distance=-0.6\onedegree,Bayer]90:{$\beta^3$}}] at (axis cs:{8.183},{-63.031}) {}; % Tuc,  5.07 
\node[pin={[pin distance=-0.6\onedegree,Bayer]00:{$\delta$}}] at (axis cs:{336.833},{-64.966}) {}; % Tuc,  4.50 
\node[pin={[pin distance=-0.6\onedegree,Bayer]00:{$\epsilon$}}] at (axis cs:{359.979},{-65.577}) {}; % Tuc,  4.50 
\node[pin={[pin distance=-0.6\onedegree,Bayer]180:{$\eta$}}] at (axis cs:{359.396},{-64.298}) {}; % Tuc,  5.00 
\node[pin={[pin distance=-0.6\onedegree,Bayer]00:{$\gamma$}}] at (axis cs:{349.357},{-58.236}) {}; % Tuc,  4.00 
\node[pin={[pin distance=-0.6\onedegree,Bayer]00:{$\iota$}}] at (axis cs:{16.828},{-61.775}) {}; % Tuc,  5.35 
\node[pin={[pin distance=-0.4\onedegree,Bayer]-90:{$\kappa$}}] at (axis cs:{18.942},{-68.876}) {}; % Tuc,  4.25 
\node[pin={[pin distance=-0.6\onedegree,Bayer]00:{$\lambda^2$}}] at (axis cs:{13.751},{-69.527}) {}; % Tuc,  5.46 
\node[pin={[pin distance=-0.6\onedegree,Bayer]00:{$\nu$}}] at (axis cs:{338.250},{-61.982}) {}; % Tuc,  4.95 
\node[pin={[pin distance=-0.6\onedegree,Bayer]00:{$\pi$}}] at (axis cs:{5.163},{-69.625}) {}; % Tuc,  5.50 
\node[pin={[pin distance=-0.6\onedegree,Bayer]00:{$\rho$}}] at (axis cs:{10.618},{-65.468}) {}; % Tuc,  5.39 
\node[pin={[pin distance=-0.6\onedegree,Bayer]-90:{$\vartheta$}}] at (axis cs:{8.347},{-71.266}) {}; % Tuc,  6.12 
\node[pin={[pin distance=-0.6\onedegree,Bayer]00:{$\zeta$}}] at (axis cs:{5.018},{-64.875}) {}; % Tuc,  4.23

\node[pin={[pin distance=-0.4\onedegree,Bayer]00:{$\delta$}}] at (axis cs:{131.176},{-54.709}) {}; % Vel,  1.94 
\node[pin={[pin distance=-0.4\onedegree,Bayer]00:{$\gamma^2$}}] at (axis cs:{122.383},{-47.337}) {}; % Vel,  1.79 
\node[pin={[pin distance=-0.4\onedegree,Bayer]00:{$\kappa$}}] at (axis cs:{140.528},{-55.010}) {}; % Vel,  2.48 
\node[pin={[pin distance=-0.4\onedegree,Bayer]00:{$\lambda$}}] at (axis cs:{136.999},{-43.432}) {}; % Vel,  2.21 
\node[pin={[pin distance=-0.6\onedegree,Bayer]00:{$\mu$}}] at (axis cs:{161.692},{-49.420}) {}; % Vel,  2.72 
\node[pin={[pin distance=-0.6\onedegree,Bayer]00:{$\psi$}}] at (axis cs:{142.675},{-40.467}) {}; % Vel,  3.59 
\node[pin={[pin distance=-0.6\onedegree,Bayer]00:{$\varphi$}}] at (axis cs:{149.216},{-54.568}) {}; % Vel,  3.54

\node[pin={[pin distance=-0.4\onedegree,Bayer]00:{$\alpha$}}] at (axis cs:{135.612},{-66.396}) {}; % Vol,  4.00 
\node[pin={[pin distance=-0.4\onedegree,Bayer]90:{$\beta$}}] at (axis cs:{126.434},{-66.137}) {}; % Vol,  3.77 
\node[pin={[pin distance=-0.4\onedegree,Bayer]00:{$\delta$}}] at (axis cs:{109.208},{-67.957}) {}; % Vol,  3.97 
\node[pin={[pin distance=-0.4\onedegree,Bayer]00:{$\epsilon$}}] at (axis cs:{121.982},{-68.617}) {}; % Vol,  4.40 
\node[pin={[pin distance=-0.4\onedegree,Bayer]-90:{$\eta$}}] at (axis cs:{125.519},{-73.400}) {}; % Vol,  5.29 
\node[pin={[pin distance=-0.4\onedegree,Bayer]180:{$\gamma^2$}}] at (axis cs:{107.187},{-70.499}) {}; % Vol,  3.76 
\node[pin={[pin distance=-0.6\onedegree,Bayer]00:{$\iota$}}] at (axis cs:{102.862},{-70.963}) {}; % Vol,  5.40 
\node[pin={[pin distance=-0.4\onedegree,Bayer]90:{$\kappa^1$}}] at (axis cs:{124.954},{-71.515}) {}; % Vol,  5.34 
\node[pin={[pin distance=-0.4\onedegree,Bayer]180:{$\vartheta$}}] at (axis cs:{129.772},{-70.387}) {}; % Vol,  5.19 
\node[pin={[pin distance=-0.4\onedegree,Bayer]00:{$\zeta$}}] at (axis cs:{115.455},{-72.606}) {}; % Vol,  3.95 




\end{polaraxis}


%
\end{tikzpicture}

\end{document}
