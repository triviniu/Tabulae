
\documentclass[10pt,landscape]{article}


\usepackage[margin=1cm,a3paper]{geometry}

\def\pgfsysdriver{pgfsys-pdftex.def}
\usepackage{pgfplots}
\usepgfplotslibrary{external} 
\usetikzlibrary{pgfplots.external}
\usetikzlibrary{calc}
\usetikzlibrary{shapes}
\usetikzlibrary{patterns}
\usetikzlibrary{plotmarks}
\usepgflibrary{arrows}
\usepgfplotslibrary{polar}
\tikzexternalize

\usepackage{times,latexsym,amssymb}
\usepackage{eulervm} % math font
\pagestyle{empty}

\usepackage{amsmath} 
\usepackage{nicefrac} 
%
% This is used because from 1.11 onwards the use of \pgfdeclareplotmark will
% introduce a shift of the stellar markers.
%
\pgfplotsset{compat=1.10}
%
% Use to increase spacing for constellation labels
\usepackage[letterspace=400]{microtype}
%
\usepackage{pagecolor}
\usepackage{natbib}
%


%\pagecolor{black}
\pagecolor{white}

%%%%%%%%%%%%%%%%%%%%%%%%%%%%%%%%%%%%%%%%%%%%%%%%%%%%%%%%%%%%%%%%%%%%%%
\begin{document}




%%%%%%%%%%%%%%%%%%%%%%%%%%%%%%%%%%%%%%%%%%%%%%%%%%%%%%%%%%%%%%%%%%%%%%%%%%%%%%%%%%%%%%%%%%
%%%%%%%%%%%%%%%%%%%%%%%%%%%%%%%%%%%%%%%%%%%%%%%%%%%%%%%%%%%%%%%%%%%%%%%%%%%%%%%%%%%%%%%%%%
%   Magnitude markers
%%%%%%%%%%%%%%%%%%%%%%%%%%%%%%%%%%%%%%%%%%%%%%%%%%%%%%%%%%%%%%%%%%%%%%%%%%%%%%%%%%%%%%%%%%
%%%%%%%%%%%%%%%%%%%%%%%%%%%%%%%%%%%%%%%%%%%%%%%%%%%%%%%%%%%%%%%%%%%%%%%%%%%%%%%%%%%%%%%%%%

\pgfdeclareplotmark{m1b}{%
\node[scale=\mOnescale*\AtoBscale*\symscale] at (0,0) {\tikz {%
\draw[fill=black,line width=.3pt,even odd rule,rotate=90] 
(0.*25.71428:10pt)  -- (0.5*25.71428:5pt)   -- 
(1.*25.71428:10pt)  -- (1.5*25.71428:5pt)   -- 
(2.*25.71428:10pt)  -- (2.5*25.71428:5pt)   -- 
(3.*25.71428:10pt)  -- (3.5*25.71428:5pt)   -- 
(4.*25.71428:10pt)  -- (4.5*25.71428:5pt)   -- 
(5.*25.71428:10pt)  -- (5.5*25.71428:5pt)   -- 
(6.*25.71428:10pt)  -- (6.5*25.71428:5pt)   -- 
(7.*25.71428:10pt)  -- (7.5*25.71428:5pt)   -- 
(8.*25.71428:10pt)  -- (8.5*25.71428:5pt)   -- 
(9.*25.71428:10pt)  -- (9.5*25.71428:5pt)   -- 
(10.*25.71428:10pt) -- (10.5*25.71428:5pt)  -- 
(11.*25.71428:10pt) -- (11.5*25.71428:5pt)  -- 
(12.*25.71428:10pt) -- (12.5*25.71428:5pt)  -- 
(13.*25.71428:10pt) -- (13.5*25.71428:5pt)  --  cycle;
}}}


\pgfdeclareplotmark{m1bv}{%
\node[scale=\mOnescale*\AtoBscale*\symscale] at (0,0) {\tikz {%
\draw[fill=black,line width=.3pt,even odd rule,rotate=90] 
(0.*25.71428:9pt)  -- (0.5*25.71428:4.5pt)   -- 
(1.*25.71428:9pt)  -- (1.5*25.71428:4.5pt)   -- 
(2.*25.71428:9pt)  -- (2.5*25.71428:4.5pt)   -- 
(3.*25.71428:9pt)  -- (3.5*25.71428:4.5pt)   -- 
(4.*25.71428:9pt)  -- (4.5*25.71428:4.5pt)   -- 
(5.*25.71428:9pt)  -- (5.5*25.71428:4.5pt)   -- 
(6.*25.71428:9pt)  -- (6.5*25.71428:4.5pt)   -- 
(7.*25.71428:9pt)  -- (7.5*25.71428:4.5pt)   -- 
(8.*25.71428:9pt)  -- (8.5*25.71428:4.5pt)   -- 
(9.*25.71428:9pt)  -- (9.5*25.71428:4.5pt)   -- 
(10.*25.71428:9pt) -- (10.5*25.71428:4.5pt)  -- 
(11.*25.71428:9pt) -- (11.5*25.71428:4.5pt)  -- 
(12.*25.71428:9pt) -- (12.5*25.71428:4.5pt)  -- 
(13.*25.71428:9pt) -- (13.5*25.71428:4.5pt)  --cycle;
\draw (0,0) circle (9pt);
%\draw[line width=.2pt,even odd rule,rotate=90] 
%(0.*25.71428:11pt)  -- (0.5*25.71428:6.5pt)   -- 
%(1.*25.71428:11pt)  -- (1.5*25.71428:6.5pt)   -- 
%(2.*25.71428:11pt)  -- (2.5*25.71428:6.5pt)   -- 
%(3.*25.71428:11pt)  -- (3.5*25.71428:6.5pt)   -- 
%(4.*25.71428:11pt)  -- (4.5*25.71428:6.5pt)   -- 
%(5.*25.71428:11pt)  -- (5.5*25.71428:6.5pt)   -- 
%(6.*25.71428:11pt)  -- (6.5*25.71428:6.5pt)   -- 
%(7.*25.71428:11pt)  -- (7.5*25.71428:6.5pt)   -- 
%(8.*25.71428:11pt)  -- (8.5*25.71428:6.5pt)   -- 
%(9.*25.71428:11pt)  -- (9.5*25.71428:6.5pt)   -- 
%(10.*25.71428:11pt) -- (10.5*25.71428:6.5pt)  -- 
%(11.*25.71428:11pt) -- (11.5*25.71428:6.5pt)  -- 
%(12.*25.71428:11pt) -- (12.5*25.71428:6.5pt)  -- 
%(13.*25.71428:11pt) -- (13.5*25.71428:6.5pt)  --  cycle;
}}}



\pgfdeclareplotmark{m1bb}{%
\node[scale=\mOnescale*\AtoBscale*\symscale] at (0,0) {\tikz {%
\draw[fill=black,line width=.3pt,even odd rule,rotate=90] 
(0.*25.71428:10pt)  -- (0.5*25.71428:5pt)   -- 
(1.*25.71428:10pt)  -- (1.5*25.71428:5pt)   -- 
(2.*25.71428:10pt)  -- (2.5*25.71428:5pt)   -- 
(3.*25.71428:10pt)  -- (3.5*25.71428:5pt)   -- 
(4.*25.71428:10pt)  -- (4.5*25.71428:5pt)   -- 
(5.*25.71428:10pt)  -- (5.5*25.71428:5pt)   -- 
(6.*25.71428:10pt)  -- (6.5*25.71428:5pt)   -- 
(7.*25.71428:10pt)  -- (7.5*25.71428:5pt)   -- 
(8.*25.71428:10pt)  -- (8.5*25.71428:5pt)   -- 
(9.*25.71428:10pt)  -- (9.5*25.71428:5pt)   -- 
(10.*25.71428:10pt) -- (10.5*25.71428:5pt)  -- 
(11.*25.71428:10pt) -- (11.5*25.71428:5pt)  -- 
(12.*25.71428:10pt) -- (12.5*25.71428:5pt)  -- 
(13.*25.71428:10pt) -- (13.5*25.71428:5pt)  --   cycle;
\draw[line width=.4pt,even odd rule,rotate=90] 
(0.5*25.71428:9pt)  -- (0.5*25.71428:5pt)   
(1.5*25.71428:9pt)  -- (1.5*25.71428:5pt)   
(2.5*25.71428:9pt)  -- (2.5*25.71428:5pt)   
(3.5*25.71428:9pt)  -- (3.5*25.71428:5pt)   
(4.5*25.71428:9pt)  -- (4.5*25.71428:5pt)   
(5.5*25.71428:9pt)  -- (5.5*25.71428:5pt)   
(6.5*25.71428:9pt)  -- (6.5*25.71428:5pt)   
(7.5*25.71428:9pt)  -- (7.5*25.71428:5pt)   
(8.5*25.71428:9pt)  -- (8.5*25.71428:5pt)   
(9.5*25.71428:9pt)  -- (9.5*25.71428:5pt)   
(10.5*25.71428:9pt) -- (10.5*25.71428:5pt)  
(11.5*25.71428:9pt) -- (11.5*25.71428:5pt)  
(12.5*25.71428:9pt) -- (12.5*25.71428:5pt)  
(13.5*25.71428:9pt) -- (13.5*25.71428:5pt)  ;
 }}}

\pgfdeclareplotmark{m1bvb}{%
\node[scale=\mOnescale*\AtoBscale*\symscale] at (0,0) {\tikz {%
\draw[fill=black,line width=.3pt,even odd rule,rotate=90] 
(0.*25.71428:9pt)  -- (0.5*25.71428:4.5pt)   -- 
(1.*25.71428:9pt)  -- (1.5*25.71428:4.5pt)   -- 
(2.*25.71428:9pt)  -- (2.5*25.71428:4.5pt)   -- 
(3.*25.71428:9pt)  -- (3.5*25.71428:4.5pt)   -- 
(4.*25.71428:9pt)  -- (4.5*25.71428:4.5pt)   -- 
(5.*25.71428:9pt)  -- (5.5*25.71428:4.5pt)   -- 
(6.*25.71428:9pt)  -- (6.5*25.71428:4.5pt)   -- 
(7.*25.71428:9pt)  -- (7.5*25.71428:4.5pt)   -- 
(8.*25.71428:9pt)  -- (8.5*25.71428:4.5pt)   -- 
(9.*25.71428:9pt)  -- (9.5*25.71428:4.5pt)   -- 
(10.*25.71428:9pt) -- (10.5*25.71428:4.5pt)  -- 
(11.*25.71428:9pt) -- (11.5*25.71428:4.5pt)  -- 
(12.*25.71428:9pt) -- (12.5*25.71428:4.5pt)  -- 
(13.*25.71428:9pt) -- (13.5*25.71428:4.5pt)  --  cycle;
\draw (0,0) circle (9pt);
%\draw[line width=.2pt,even odd rule,rotate=90] 
%(0.*25.71428:11pt)  -- (0.5*25.71428:6.5pt)   -- 
%(1.*25.71428:11pt)  -- (1.5*25.71428:6.5pt)   -- 
%(2.*25.71428:11pt)  -- (2.5*25.71428:6.5pt)   -- 
%(3.*25.71428:11pt)  -- (3.5*25.71428:6.5pt)   -- 
%(4.*25.71428:11pt)  -- (4.5*25.71428:6.5pt)   -- 
%(5.*25.71428:11pt)  -- (5.5*25.71428:6.5pt)   -- 
%(6.*25.71428:11pt)  -- (6.5*25.71428:6.5pt)   -- 
%(7.*25.71428:11pt)  -- (7.5*25.71428:6.5pt)   -- 
%(8.*25.71428:11pt)  -- (8.5*25.71428:6.5pt)   -- 
%(9.*25.71428:11pt)  -- (9.5*25.71428:6.5pt)   -- 
%(10.*25.71428:11pt) -- (10.5*25.71428:6.5pt)  -- 
%(11.*25.71428:11pt) -- (11.5*25.71428:6.5pt)  -- 
%(12.*25.71428:11pt) -- (12.5*25.71428:6.5pt)  -- 
%(13.*25.71428:11pt) -- (13.5*25.71428:6.5pt)  --  cycle;
\draw[line width=.4pt,even odd rule,rotate=90] 
(0.5*25.71428:10pt)  -- (0.5*25.71428:6.5pt)   
(1.5*25.71428:10pt)  -- (1.5*25.71428:6.5pt)   
(2.5*25.71428:10pt)  -- (2.5*25.71428:6.5pt)   
(3.5*25.71428:10pt)  -- (3.5*25.71428:6.5pt)   
(4.5*25.71428:10pt)  -- (4.5*25.71428:6.5pt)   
(5.5*25.71428:10pt)  -- (5.5*25.71428:6.5pt)   
(6.5*25.71428:10pt)  -- (6.5*25.71428:6.5pt)   
(7.5*25.71428:10pt)  -- (7.5*25.71428:6.5pt)   
(8.5*25.71428:10pt)  -- (8.5*25.71428:6.5pt)   
(9.5*25.71428:10pt)  -- (9.5*25.71428:6.5pt)   
(10.5*25.71428:10pt) -- (10.5*25.71428:6.5pt)  
(11.5*25.71428:10pt) -- (11.5*25.71428:6.5pt)  
(12.5*25.71428:10pt) -- (12.5*25.71428:6.5pt)  
(13.5*25.71428:10pt) -- (13.5*25.71428:6.5pt)  ;
 }}}



\pgfdeclareplotmark{m1c}{%
\node[scale=\mOnescale*\AtoCscale*\symscale] at (0,0) {\tikz {%
\draw[fill=black,line width=.3pt,even odd rule,rotate=90] 
(0.*25.71428:10pt)  -- (0.5*25.71428:5pt)   -- 
(1.*25.71428:10pt)  -- (1.5*25.71428:5pt)   -- 
(2.*25.71428:10pt)  -- (2.5*25.71428:5pt)   -- 
(3.*25.71428:10pt)  -- (3.5*25.71428:5pt)   -- 
(4.*25.71428:10pt)  -- (4.5*25.71428:5pt)   -- 
(5.*25.71428:10pt)  -- (5.5*25.71428:5pt)   -- 
(6.*25.71428:10pt)  -- (6.5*25.71428:5pt)   -- 
(7.*25.71428:10pt)  -- (7.5*25.71428:5pt)   -- 
(8.*25.71428:10pt)  -- (8.5*25.71428:5pt)   -- 
(9.*25.71428:10pt)  -- (9.5*25.71428:5pt)   -- 
(10.*25.71428:10pt) -- (10.5*25.71428:5pt)  -- 
(11.*25.71428:10pt) -- (11.5*25.71428:5pt)  -- 
(12.*25.71428:10pt) -- (12.5*25.71428:5pt)  -- 
(13.*25.71428:10pt) -- (13.5*25.71428:5pt)  --  cycle
(0,0) circle (2pt);
 }}}


\pgfdeclareplotmark{m1cv}{%
\node[scale=\mOnescale*\AtoCscale*\symscale] at (0,0) {\tikz {%
\draw[fill=black,line width=.3pt,even odd rule,rotate=90] 
(0.*25.71428:9pt)  -- (0.5*25.71428:4.5pt)   -- 
(1.*25.71428:9pt)  -- (1.5*25.71428:4.5pt)   -- 
(2.*25.71428:9pt)  -- (2.5*25.71428:4.5pt)   -- 
(3.*25.71428:9pt)  -- (3.5*25.71428:4.5pt)   -- 
(4.*25.71428:9pt)  -- (4.5*25.71428:4.5pt)   -- 
(5.*25.71428:9pt)  -- (5.5*25.71428:4.5pt)   -- 
(6.*25.71428:9pt)  -- (6.5*25.71428:4.5pt)   -- 
(7.*25.71428:9pt)  -- (7.5*25.71428:4.5pt)   -- 
(8.*25.71428:9pt)  -- (8.5*25.71428:4.5pt)   -- 
(9.*25.71428:9pt)  -- (9.5*25.71428:4.5pt)   -- 
(10.*25.71428:9pt) -- (10.5*25.71428:4.5pt)  -- 
(11.*25.71428:9pt) -- (11.5*25.71428:4.5pt)  -- 
(12.*25.71428:9pt) -- (12.5*25.71428:4.5pt)  -- 
(13.*25.71428:9pt) -- (13.5*25.71428:4.5pt)  -- cycle
(0,0) circle (2pt);
\draw (0,0) circle (9pt);
%\draw[line width=.2pt,even odd rule,rotate=90] 
%(0.*25.71428:11pt)  -- (0.5*25.71428:6.5pt)   -- 
%(1.*25.71428:11pt)  -- (1.5*25.71428:6.5pt)   -- 
%(2.*25.71428:11pt)  -- (2.5*25.71428:6.5pt)   -- 
%(3.*25.71428:11pt)  -- (3.5*25.71428:6.5pt)   -- 
%(4.*25.71428:11pt)  -- (4.5*25.71428:6.5pt)   -- 
%(5.*25.71428:11pt)  -- (5.5*25.71428:6.5pt)   -- 
%(6.*25.71428:11pt)  -- (6.5*25.71428:6.5pt)   -- 
%(7.*25.71428:11pt)  -- (7.5*25.71428:6.5pt)   -- 
%(8.*25.71428:11pt)  -- (8.5*25.71428:6.5pt)   -- 
%(9.*25.71428:11pt)  -- (9.5*25.71428:6.5pt)   -- 
%(10.*25.71428:11pt) -- (10.5*25.71428:6.5pt)  -- 
%(11.*25.71428:11pt) -- (11.5*25.71428:6.5pt)  -- 
%(12.*25.71428:11pt) -- (12.5*25.71428:6.5pt)  -- 
%(13.*25.71428:11pt) -- (13.5*25.71428:6.5pt)  --  cycle;
 }}}



\pgfdeclareplotmark{m1cb}{%
\node[scale=\mOnescale*\AtoCscale*\symscale] at (0,0) {\tikz {%
\draw[fill=black,line width=.3pt,even odd rule,rotate=90] 
(0.*25.71428:10pt)  -- (0.5*25.71428:5pt)   -- 
(1.*25.71428:10pt)  -- (1.5*25.71428:5pt)   -- 
(2.*25.71428:10pt)  -- (2.5*25.71428:5pt)   -- 
(3.*25.71428:10pt)  -- (3.5*25.71428:5pt)   -- 
(4.*25.71428:10pt)  -- (4.5*25.71428:5pt)   -- 
(5.*25.71428:10pt)  -- (5.5*25.71428:5pt)   -- 
(6.*25.71428:10pt)  -- (6.5*25.71428:5pt)   -- 
(7.*25.71428:10pt)  -- (7.5*25.71428:5pt)   -- 
(8.*25.71428:10pt)  -- (8.5*25.71428:5pt)   -- 
(9.*25.71428:10pt)  -- (9.5*25.71428:5pt)   -- 
(10.*25.71428:10pt) -- (10.5*25.71428:5pt)  -- 
(11.*25.71428:10pt) -- (11.5*25.71428:5pt)  -- 
(12.*25.71428:10pt) -- (12.5*25.71428:5pt)  -- 
(13.*25.71428:10pt) -- (13.5*25.71428:5pt)  --  cycle
(0,0) circle (2pt);
\draw[line width=.4pt,even odd rule,rotate=90] 
(0.5*25.71428:9pt)  -- (0.5*25.71428:5pt)   
(1.5*25.71428:9pt)  -- (1.5*25.71428:5pt)   
(2.5*25.71428:9pt)  -- (2.5*25.71428:5pt)   
(3.5*25.71428:9pt)  -- (3.5*25.71428:5pt)   
(4.5*25.71428:9pt)  -- (4.5*25.71428:5pt)   
(5.5*25.71428:9pt)  -- (5.5*25.71428:5pt)   
(6.5*25.71428:9pt)  -- (6.5*25.71428:5pt)   
(7.5*25.71428:9pt)  -- (7.5*25.71428:5pt)   
(8.5*25.71428:9pt)  -- (8.5*25.71428:5pt)   
(9.5*25.71428:9pt)  -- (9.5*25.71428:5pt)   
(10.5*25.71428:9pt) -- (10.5*25.71428:5pt)  
(11.5*25.71428:9pt) -- (11.5*25.71428:5pt)  
(12.5*25.71428:9pt) -- (12.5*25.71428:5pt)  
(13.5*25.71428:9pt) -- (13.5*25.71428:5pt)  ;
 }}}

\pgfdeclareplotmark{m1cvb}{%
\node[scale=\mOnescale*\AtoCscale*\symscale] at (0,0) {\tikz {%
\draw[fill=black,line width=.3pt,even odd rule,rotate=90] 
(0.*25.71428:9pt)  -- (0.5*25.71428:4.5pt)   -- 
(1.*25.71428:9pt)  -- (1.5*25.71428:4.5pt)   -- 
(2.*25.71428:9pt)  -- (2.5*25.71428:4.5pt)   -- 
(3.*25.71428:9pt)  -- (3.5*25.71428:4.5pt)   -- 
(4.*25.71428:9pt)  -- (4.5*25.71428:4.5pt)   -- 
(5.*25.71428:9pt)  -- (5.5*25.71428:4.5pt)   -- 
(6.*25.71428:9pt)  -- (6.5*25.71428:4.5pt)   -- 
(7.*25.71428:9pt)  -- (7.5*25.71428:4.5pt)   -- 
(8.*25.71428:9pt)  -- (8.5*25.71428:4.5pt)   -- 
(9.*25.71428:9pt)  -- (9.5*25.71428:4.5pt)   -- 
(10.*25.71428:9pt) -- (10.5*25.71428:4.5pt)  -- 
(11.*25.71428:9pt) -- (11.5*25.71428:4.5pt)  -- 
(12.*25.71428:9pt) -- (12.5*25.71428:4.5pt)  -- 
(13.*25.71428:9pt) -- (13.5*25.71428:4.5pt)  --  cycle
(0,0) circle (2pt);
\draw (0,0) circle (9pt);
%\draw[line width=.2pt,even odd rule,rotate=90] 
%(0.*25.71428:11pt)  -- (0.5*25.71428:6.5pt)   -- 
%(1.*25.71428:11pt)  -- (1.5*25.71428:6.5pt)   -- 
%(2.*25.71428:11pt)  -- (2.5*25.71428:6.5pt)   -- 
%(3.*25.71428:11pt)  -- (3.5*25.71428:6.5pt)   -- 
%(4.*25.71428:11pt)  -- (4.5*25.71428:6.5pt)   -- 
%(5.*25.71428:11pt)  -- (5.5*25.71428:6.5pt)   -- 
%(6.*25.71428:11pt)  -- (6.5*25.71428:6.5pt)   -- 
%(7.*25.71428:11pt)  -- (7.5*25.71428:6.5pt)   -- 
%(8.*25.71428:11pt)  -- (8.5*25.71428:6.5pt)   -- 
%(9.*25.71428:11pt)  -- (9.5*25.71428:6.5pt)   -- 
%(10.*25.71428:11pt) -- (10.5*25.71428:6.5pt)  -- 
%(11.*25.71428:11pt) -- (11.5*25.71428:6.5pt)  -- 
%(12.*25.71428:11pt) -- (12.5*25.71428:6.5pt)  -- 
%(13.*25.71428:11pt) -- (13.5*25.71428:6.5pt)  --  cycle;
\draw[line width=.4pt,even odd rule,rotate=90] 
(0.5*25.71428:10pt)  -- (0.5*25.71428:6.5pt)   
(1.5*25.71428:10pt)  -- (1.5*25.71428:6.5pt)   
(2.5*25.71428:10pt)  -- (2.5*25.71428:6.5pt)   
(3.5*25.71428:10pt)  -- (3.5*25.71428:6.5pt)   
(4.5*25.71428:10pt)  -- (4.5*25.71428:6.5pt)   
(5.5*25.71428:10pt)  -- (5.5*25.71428:6.5pt)   
(6.5*25.71428:10pt)  -- (6.5*25.71428:6.5pt)   
(7.5*25.71428:10pt)  -- (7.5*25.71428:6.5pt)   
(8.5*25.71428:10pt)  -- (8.5*25.71428:6.5pt)   
(9.5*25.71428:10pt)  -- (9.5*25.71428:6.5pt)   
(10.5*25.71428:10pt) -- (10.5*25.71428:6.5pt)  
(11.5*25.71428:10pt) -- (11.5*25.71428:6.5pt)  
(12.5*25.71428:10pt) -- (12.5*25.71428:6.5pt)  
(13.5*25.71428:10pt) -- (13.5*25.71428:6.5pt)  ;
 }}}



\pgfdeclareplotmark{m2a}{%
\node[scale=\mTwoscale*\symscale] at (0,0) {\tikz {%
\draw[fill=black,line width=.3pt,even odd rule,rotate=90] 
(0.*30.0:10pt)  -- (0.5*30.0:5pt)   -- 
(1.*30.0:10pt)  -- (1.5*30.0:5pt)   -- 
(2.*30.0:10pt)  -- (2.5*30.0:5pt)   -- 
(3.*30.0:10pt)  -- (3.5*30.0:5pt)   -- 
(4.*30.0:10pt)  -- (4.5*30.0:5pt)   -- 
(5.*30.0:10pt)  -- (5.5*30.0:5pt)   -- 
(6.*30.0:10pt)  -- (6.5*30.0:5pt)   -- 
(7.*30.0:10pt)  -- (7.5*30.0:5pt)   -- 
(8.*30.0:10pt)  -- (8.5*30.0:5pt)   -- 
(9.*30.0:10pt)  -- (9.5*30.0:5pt)   -- 
(10.*30.0:10pt) -- (10.5*30.0:5pt)  -- 
(11.*30.0:10pt) -- (11.5*30.0:5pt)  --  cycle;
 }}}


\pgfdeclareplotmark{m2av}{%
\node[scale=\mTwoscale*\symscale] at (0,0) {\tikz {%
\draw[fill=black,line width=.3pt,even odd rule,rotate=90] 
(0.*30.0:9pt)  -- (0.5*30.0:4.5pt)   -- 
(1.*30.0:9pt)  -- (1.5*30.0:4.5pt)   -- 
(2.*30.0:9pt)  -- (2.5*30.0:4.5pt)   -- 
(3.*30.0:9pt)  -- (3.5*30.0:4.5pt)   -- 
(4.*30.0:9pt)  -- (4.5*30.0:4.5pt)   -- 
(5.*30.0:9pt)  -- (5.5*30.0:4.5pt)   -- 
(6.*30.0:9pt)  -- (6.5*30.0:4.5pt)   -- 
(7.*30.0:9pt)  -- (7.5*30.0:4.5pt)   -- 
(8.*30.0:9pt)  -- (8.5*30.0:4.5pt)   -- 
(9.*30.0:9pt)  -- (9.5*30.0:4.5pt)   -- 
(10.*30.0:9pt) -- (10.5*30.0:4.5pt)  -- 
(11.*30.0:9pt) -- (11.5*30.0:4.5pt)  -- cycle;
\draw (0,0) circle (9pt);
%\draw[line width=.2pt,even odd rule,rotate=90] 
%(0.*30.0:11pt)  -- (0.5*30.0:6.5pt)   -- 
%(1.*30.0:11pt)  -- (1.5*30.0:6.5pt)   -- 
%(2.*30.0:11pt)  -- (2.5*30.0:6.5pt)   -- 
%(3.*30.0:11pt)  -- (3.5*30.0:6.5pt)   -- 
%(4.*30.0:11pt)  -- (4.5*30.0:6.5pt)   -- 
%(5.*30.0:11pt)  -- (5.5*30.0:6.5pt)   -- 
%(6.*30.0:11pt)  -- (6.5*30.0:6.5pt)   -- 
%(7.*30.0:11pt)  -- (7.5*30.0:6.5pt)   -- 
%(8.*30.0:11pt)  -- (8.5*30.0:6.5pt)   -- 
%(9.*30.0:11pt)  -- (9.5*30.0:6.5pt)   -- 
%(10.*30.0:11pt) -- (10.5*30.0:6.5pt)  -- 
%(11.*30.0:11pt) -- (11.5*30.0:6.5pt)  --  cycle;
 }}}



\pgfdeclareplotmark{m2ab}{%
\node[scale=\mTwoscale*\symscale] at (0,0) {\tikz {%
\draw[fill=black,line width=.3pt,even odd rule,rotate=90] 
(0.*30.0:10pt)  -- (0.5*30.0:5pt)   -- 
(1.*30.0:10pt)  -- (1.5*30.0:5pt)   -- 
(2.*30.0:10pt)  -- (2.5*30.0:5pt)   -- 
(3.*30.0:10pt)  -- (3.5*30.0:5pt)   -- 
(4.*30.0:10pt)  -- (4.5*30.0:5pt)   -- 
(5.*30.0:10pt)  -- (5.5*30.0:5pt)   -- 
(6.*30.0:10pt)  -- (6.5*30.0:5pt)   -- 
(7.*30.0:10pt)  -- (7.5*30.0:5pt)   -- 
(8.*30.0:10pt)  -- (8.5*30.0:5pt)   -- 
(9.*30.0:10pt)  -- (9.5*30.0:5pt)   -- 
(10.*30.0:10pt) -- (10.5*30.0:5pt)  -- 
(11.*30.0:10pt) -- (11.5*30.0:5pt)  --  cycle;
\draw[line width=.4pt,even odd rule,rotate=90] 
(0.5*30.0:9pt)  -- (0.5*30.0:5pt)   
(1.5*30.0:9pt)  -- (1.5*30.0:5pt)   
(2.5*30.0:9pt)  -- (2.5*30.0:5pt)   
(3.5*30.0:9pt)  -- (3.5*30.0:5pt)   
(4.5*30.0:9pt)  -- (4.5*30.0:5pt)   
(5.5*30.0:9pt)  -- (5.5*30.0:5pt)   
(6.5*30.0:9pt)  -- (6.5*30.0:5pt)   
(7.5*30.0:9pt)  -- (7.5*30.0:5pt)   
(8.5*30.0:9pt)  -- (8.5*30.0:5pt)   
(9.5*30.0:9pt)  -- (9.5*30.0:5pt)   
(10.5*30.0:9pt) -- (10.5*30.0:5pt)  
(11.5*30.0:9pt) -- (11.5*30.0:5pt) ;
 }}}

\pgfdeclareplotmark{m2avb}{%
\node[scale=\mTwoscale*\symscale] at (0,0) {\tikz {%
\draw[fill=black,line width=.3pt,even odd rule,rotate=90] 
(0.*30.0:9pt)  -- (0.5*30.0:4.5pt)   -- 
(1.*30.0:9pt)  -- (1.5*30.0:4.5pt)   -- 
(2.*30.0:9pt)  -- (2.5*30.0:4.5pt)   -- 
(3.*30.0:9pt)  -- (3.5*30.0:4.5pt)   -- 
(4.*30.0:9pt)  -- (4.5*30.0:4.5pt)   -- 
(5.*30.0:9pt)  -- (5.5*30.0:4.5pt)   -- 
(6.*30.0:9pt)  -- (6.5*30.0:4.5pt)   -- 
(7.*30.0:9pt)  -- (7.5*30.0:4.5pt)   -- 
(8.*30.0:9pt)  -- (8.5*30.0:4.5pt)   -- 
(9.*30.0:9pt)  -- (9.5*30.0:4.5pt)   -- 
(10.*30.0:9pt) -- (10.5*30.0:4.5pt)  -- 
(11.*30.0:9pt) -- (11.5*30.0:4.5pt)  --  cycle;
\draw (0,0) circle (9pt);
%\draw[line width=.2pt,even odd rule,rotate=90] 
%(0.*30.0:11pt)  -- (0.5*30.0:6.5pt)   -- 
%(1.*30.0:11pt)  -- (1.5*30.0:6.5pt)   -- 
%(2.*30.0:11pt)  -- (2.5*30.0:6.5pt)   -- 
%(3.*30.0:11pt)  -- (3.5*30.0:6.5pt)   -- 
%(4.*30.0:11pt)  -- (4.5*30.0:6.5pt)   -- 
%(5.*30.0:11pt)  -- (5.5*30.0:6.5pt)   -- 
%(6.*30.0:11pt)  -- (6.5*30.0:6.5pt)   -- 
%(7.*30.0:11pt)  -- (7.5*30.0:6.5pt)   -- 
%(8.*30.0:11pt)  -- (8.5*30.0:6.5pt)   -- 
%(9.*30.0:11pt)  -- (9.5*30.0:6.5pt)   -- 
%(10.*30.0:11pt) -- (10.5*30.0:6.5pt)  -- 
%(11.*30.0:11pt) -- (11.5*30.0:6.5pt)  -- cycle;
\draw[line width=.4pt,even odd rule,rotate=90] 
(0.5*30.0:10pt)  -- (0.5*30.0:6.5pt)   
(1.5*30.0:10pt)  -- (1.5*30.0:6.5pt)   
(2.5*30.0:10pt)  -- (2.5*30.0:6.5pt)   
(3.5*30.0:10pt)  -- (3.5*30.0:6.5pt)   
(4.5*30.0:10pt)  -- (4.5*30.0:6.5pt)   
(5.5*30.0:10pt)  -- (5.5*30.0:6.5pt)   
(6.5*30.0:10pt)  -- (6.5*30.0:6.5pt)   
(7.5*30.0:10pt)  -- (7.5*30.0:6.5pt)   
(8.5*30.0:10pt)  -- (8.5*30.0:6.5pt)   
(9.5*30.0:10pt)  -- (9.5*30.0:6.5pt)   
(10.5*30.0:10pt) -- (10.5*30.0:6.5pt)  
(11.5*30.0:10pt) -- (11.5*30.0:6.5pt)  ;
 }}}



\pgfdeclareplotmark{m2b}{%
\node[scale=\mTwoscale*\AtoBscale*\symscale] at (0,0) {\tikz {%
\draw[fill=black,line width=.3pt,even odd rule,rotate=90] 
(0.*30.0:10pt)  -- (0.5*30.0:5pt)   -- 
(1.*30.0:10pt)  -- (1.5*30.0:5pt)   -- 
(2.*30.0:10pt)  -- (2.5*30.0:5pt)   -- 
(3.*30.0:10pt)  -- (3.5*30.0:5pt)   -- 
(4.*30.0:10pt)  -- (4.5*30.0:5pt)   -- 
(5.*30.0:10pt)  -- (5.5*30.0:5pt)   -- 
(6.*30.0:10pt)  -- (6.5*30.0:5pt)   -- 
(7.*30.0:10pt)  -- (7.5*30.0:5pt)   -- 
(8.*30.0:10pt)  -- (8.5*30.0:5pt)   -- 
(9.*30.0:10pt)  -- (9.5*30.0:5pt)   -- 
(10.*30.0:10pt) -- (10.5*30.0:5pt)  -- 
(11.*30.0:10pt) -- (11.5*30.0:5pt)  -- cycle
(0,0) circle (2pt);
 }}}


\pgfdeclareplotmark{m2bv}{%
\node[scale=\mTwoscale*\AtoBscale*\symscale] at (0,0) {\tikz {%
\draw[fill=black,line width=.3pt,even odd rule,rotate=90] 
(0.*30.0:9pt)  -- (0.5*30.0:4.5pt)   -- 
(1.*30.0:9pt)  -- (1.5*30.0:4.5pt)   -- 
(2.*30.0:9pt)  -- (2.5*30.0:4.5pt)   -- 
(3.*30.0:9pt)  -- (3.5*30.0:4.5pt)   -- 
(4.*30.0:9pt)  -- (4.5*30.0:4.5pt)   -- 
(5.*30.0:9pt)  -- (5.5*30.0:4.5pt)   -- 
(6.*30.0:9pt)  -- (6.5*30.0:4.5pt)   -- 
(7.*30.0:9pt)  -- (7.5*30.0:4.5pt)   -- 
(8.*30.0:9pt)  -- (8.5*30.0:4.5pt)   -- 
(9.*30.0:9pt)  -- (9.5*30.0:4.5pt)   -- 
(10.*30.0:9pt) -- (10.5*30.0:4.5pt)  -- 
(11.*30.0:9pt) -- (11.5*30.0:4.5pt)  --  cycle
(0,0) circle (2pt);
\draw (0,0) circle (9pt);
%\draw[line width=.2pt,even odd rule,rotate=90] 
%(0.*30.0:11pt)  -- (0.5*30.0:6.5pt)   -- 
%(1.*30.0:11pt)  -- (1.5*30.0:6.5pt)   -- 
%(2.*30.0:11pt)  -- (2.5*30.0:6.5pt)   -- 
%(3.*30.0:11pt)  -- (3.5*30.0:6.5pt)   -- 
%(4.*30.0:11pt)  -- (4.5*30.0:6.5pt)   -- 
%(5.*30.0:11pt)  -- (5.5*30.0:6.5pt)   -- 
%(6.*30.0:11pt)  -- (6.5*30.0:6.5pt)   -- 
%(7.*30.0:11pt)  -- (7.5*30.0:6.5pt)   -- 
%(8.*30.0:11pt)  -- (8.5*30.0:6.5pt)   -- 
%(9.*30.0:11pt)  -- (9.5*30.0:6.5pt)   -- 
%(10.*30.0:11pt) -- (10.5*30.0:6.5pt)  -- 
%(11.*30.0:11pt) -- (11.5*30.0:6.5pt)  --  cycle;
 }}}



\pgfdeclareplotmark{m2bb}{%
\node[scale=\mTwoscale*\AtoBscale*\symscale] at (0,0) {\tikz {%
\draw[fill=black,line width=.3pt,even odd rule,rotate=90] 
(0.*30.0:10pt)  -- (0.5*30.0:5pt)   -- 
(1.*30.0:10pt)  -- (1.5*30.0:5pt)   -- 
(2.*30.0:10pt)  -- (2.5*30.0:5pt)   -- 
(3.*30.0:10pt)  -- (3.5*30.0:5pt)   -- 
(4.*30.0:10pt)  -- (4.5*30.0:5pt)   -- 
(5.*30.0:10pt)  -- (5.5*30.0:5pt)   -- 
(6.*30.0:10pt)  -- (6.5*30.0:5pt)   -- 
(7.*30.0:10pt)  -- (7.5*30.0:5pt)   -- 
(8.*30.0:10pt)  -- (8.5*30.0:5pt)   -- 
(9.*30.0:10pt)  -- (9.5*30.0:5pt)   -- 
(10.*30.0:10pt) -- (10.5*30.0:5pt)  -- 
(11.*30.0:10pt) -- (11.5*30.0:5pt)  --  cycle
(0,0) circle (2pt);
\draw[line width=.4pt,even odd rule,rotate=90] 
(0.5*30.0:9pt)  -- (0.5*30.0:5pt)   
(1.5*30.0:9pt)  -- (1.5*30.0:5pt)   
(2.5*30.0:9pt)  -- (2.5*30.0:5pt)   
(3.5*30.0:9pt)  -- (3.5*30.0:5pt)   
(4.5*30.0:9pt)  -- (4.5*30.0:5pt)   
(5.5*30.0:9pt)  -- (5.5*30.0:5pt)   
(6.5*30.0:9pt)  -- (6.5*30.0:5pt)   
(7.5*30.0:9pt)  -- (7.5*30.0:5pt)   
(8.5*30.0:9pt)  -- (8.5*30.0:5pt)   
(9.5*30.0:9pt)  -- (9.5*30.0:5pt)   
(10.5*30.0:9pt) -- (10.5*30.0:5pt)  
(11.5*30.0:9pt) -- (11.5*30.0:5pt)  ;
 }}}

\pgfdeclareplotmark{m2bvb}{%
\node[scale=\mTwoscale*\AtoBscale*\symscale] at (0,0) {\tikz {%
\draw[fill=black,line width=.3pt,even odd rule,rotate=90] 
(0.*30.0:9pt)  -- (0.5*30.0:4.5pt)   -- 
(1.*30.0:9pt)  -- (1.5*30.0:4.5pt)   -- 
(2.*30.0:9pt)  -- (2.5*30.0:4.5pt)   -- 
(3.*30.0:9pt)  -- (3.5*30.0:4.5pt)   -- 
(4.*30.0:9pt)  -- (4.5*30.0:4.5pt)   -- 
(5.*30.0:9pt)  -- (5.5*30.0:4.5pt)   -- 
(6.*30.0:9pt)  -- (6.5*30.0:4.5pt)   -- 
(7.*30.0:9pt)  -- (7.5*30.0:4.5pt)   -- 
(8.*30.0:9pt)  -- (8.5*30.0:4.5pt)   -- 
(9.*30.0:9pt)  -- (9.5*30.0:4.5pt)   -- 
(10.*30.0:9pt) -- (10.5*30.0:4.5pt)  -- 
(11.*30.0:9pt) -- (11.5*30.0:4.5pt)  --  cycle
(0,0) circle (2pt);
\draw (0,0) circle (9pt);
%\draw[line width=.2pt,even odd rule,rotate=90] 
%(0.*30.0:11pt)  -- (0.5*30.0:6.5pt)   -- 
%(1.*30.0:11pt)  -- (1.5*30.0:6.5pt)   -- 
%(2.*30.0:11pt)  -- (2.5*30.0:6.5pt)   -- 
%(3.*30.0:11pt)  -- (3.5*30.0:6.5pt)   -- 
%(4.*30.0:11pt)  -- (4.5*30.0:6.5pt)   -- 
%(5.*30.0:11pt)  -- (5.5*30.0:6.5pt)   -- 
%(6.*30.0:11pt)  -- (6.5*30.0:6.5pt)   -- 
%(7.*30.0:11pt)  -- (7.5*30.0:6.5pt)   -- 
%(8.*30.0:11pt)  -- (8.5*30.0:6.5pt)   -- 
%(9.*30.0:11pt)  -- (9.5*30.0:6.5pt)   -- 
%(10.*30.0:11pt) -- (10.5*30.0:6.5pt)  -- 
%(11.*30.0:11pt) -- (11.5*30.0:6.5pt)  --  cycle;
\draw[line width=.4pt,even odd rule,rotate=90] 
(0.5*30.0:10pt)  -- (0.5*30.0:6.5pt)   
(1.5*30.0:10pt)  -- (1.5*30.0:6.5pt)   
(2.5*30.0:10pt)  -- (2.5*30.0:6.5pt)   
(3.5*30.0:10pt)  -- (3.5*30.0:6.5pt)   
(4.5*30.0:10pt)  -- (4.5*30.0:6.5pt)   
(5.5*30.0:10pt)  -- (5.5*30.0:6.5pt)   
(6.5*30.0:10pt)  -- (6.5*30.0:6.5pt)   
(7.5*30.0:10pt)  -- (7.5*30.0:6.5pt)   
(8.5*30.0:10pt)  -- (8.5*30.0:6.5pt)   
(9.5*30.0:10pt)  -- (9.5*30.0:6.5pt)   
(10.5*30.0:10pt) -- (10.5*30.0:6.5pt)  
(11.5*30.0:10pt) -- (11.5*30.0:6.5pt)  ;
 }}}



\pgfdeclareplotmark{m2c}{%
\node[scale=\mTwoscale*\AtoCscale*\symscale] at (0,0) {\tikz {%
\draw[fill=black,line width=.3pt,even odd rule,rotate=90] 
(0.*30.0:10pt)  -- (0.5*30.0:5pt)   -- 
(1.*30.0:10pt)  -- (1.5*30.0:5pt)   -- 
(2.*30.0:10pt)  -- (2.5*30.0:5pt)   -- 
(3.*30.0:10pt)  -- (3.5*30.0:5pt)   -- 
(4.*30.0:10pt)  -- (4.5*30.0:5pt)   -- 
(5.*30.0:10pt)  -- (5.5*30.0:5pt)   -- 
(6.*30.0:10pt)  -- (6.5*30.0:5pt)   -- 
(7.*30.0:10pt)  -- (7.5*30.0:5pt)   -- 
(8.*30.0:10pt)  -- (8.5*30.0:5pt)   -- 
(9.*30.0:10pt)  -- (9.5*30.0:5pt)   -- 
(10.*30.0:10pt) -- (10.5*30.0:5pt)  -- 
(11.*30.0:10pt) -- (11.5*30.0:5pt)  --cycle
(0,0) circle (3pt);
 }}}


\pgfdeclareplotmark{m2cv}{%
\node[scale=\mTwoscale*\AtoCscale*\symscale] at (0,0) {\tikz {%
\draw[fill=black,line width=.3pt,even odd rule,rotate=90] 
(0.*30.0:9pt)  -- (0.5*30.0:4.5pt)   -- 
(1.*30.0:9pt)  -- (1.5*30.0:4.5pt)   -- 
(2.*30.0:9pt)  -- (2.5*30.0:4.5pt)   -- 
(3.*30.0:9pt)  -- (3.5*30.0:4.5pt)   -- 
(4.*30.0:9pt)  -- (4.5*30.0:4.5pt)   -- 
(5.*30.0:9pt)  -- (5.5*30.0:4.5pt)   -- 
(6.*30.0:9pt)  -- (6.5*30.0:4.5pt)   -- 
(7.*30.0:9pt)  -- (7.5*30.0:4.5pt)   -- 
(8.*30.0:9pt)  -- (8.5*30.0:4.5pt)   -- 
(9.*30.0:9pt)  -- (9.5*30.0:4.5pt)   -- 
(10.*30.0:9pt) -- (10.5*30.0:4.5pt)  -- 
(11.*30.0:9pt) -- (11.5*30.0:4.5pt)  --  cycle
(0,0) circle (3pt);
\draw (0,0) circle (9pt);
%\draw[line width=.2pt,even odd rule,rotate=90] 
%(0.*30.0:11pt)  -- (0.5*30.0:6.5pt)   -- 
%(1.*30.0:11pt)  -- (1.5*30.0:6.5pt)   -- 
%(2.*30.0:11pt)  -- (2.5*30.0:6.5pt)   -- 
%(3.*30.0:11pt)  -- (3.5*30.0:6.5pt)   -- 
%(4.*30.0:11pt)  -- (4.5*30.0:6.5pt)   -- 
%(5.*30.0:11pt)  -- (5.5*30.0:6.5pt)   -- 
%(6.*30.0:11pt)  -- (6.5*30.0:6.5pt)   -- 
%(7.*30.0:11pt)  -- (7.5*30.0:6.5pt)   -- 
%(8.*30.0:11pt)  -- (8.5*30.0:6.5pt)   -- 
%(9.*30.0:11pt)  -- (9.5*30.0:6.5pt)   -- 
%(10.*30.0:11pt) -- (10.5*30.0:6.5pt)  -- 
%(11.*30.0:11pt) -- (11.5*30.0:6.5pt)  --  cycle;
 }}}



\pgfdeclareplotmark{m2cb}{%
\node[scale=\mTwoscale*\AtoCscale*\symscale] at (0,0) {\tikz {%
\draw[fill=black,line width=.3pt,even odd rule,rotate=90] 
(0.*30.0:10pt)  -- (0.5*30.0:5pt)   -- 
(1.*30.0:10pt)  -- (1.5*30.0:5pt)   -- 
(2.*30.0:10pt)  -- (2.5*30.0:5pt)   -- 
(3.*30.0:10pt)  -- (3.5*30.0:5pt)   -- 
(4.*30.0:10pt)  -- (4.5*30.0:5pt)   -- 
(5.*30.0:10pt)  -- (5.5*30.0:5pt)   -- 
(6.*30.0:10pt)  -- (6.5*30.0:5pt)   -- 
(7.*30.0:10pt)  -- (7.5*30.0:5pt)   -- 
(8.*30.0:10pt)  -- (8.5*30.0:5pt)   -- 
(9.*30.0:10pt)  -- (9.5*30.0:5pt)   -- 
(10.*30.0:10pt) -- (10.5*30.0:5pt)  -- 
(11.*30.0:10pt) -- (11.5*30.0:5pt)  --  cycle
(0,0) circle (3pt);
\draw[line width=.4pt,even odd rule,rotate=90] 
(0.5*30.0:9pt)  -- (0.5*30.0:5pt)   
(1.5*30.0:9pt)  -- (1.5*30.0:5pt)   
(2.5*30.0:9pt)  -- (2.5*30.0:5pt)   
(3.5*30.0:9pt)  -- (3.5*30.0:5pt)   
(4.5*30.0:9pt)  -- (4.5*30.0:5pt)   
(5.5*30.0:9pt)  -- (5.5*30.0:5pt)   
(6.5*30.0:9pt)  -- (6.5*30.0:5pt)   
(7.5*30.0:9pt)  -- (7.5*30.0:5pt)   
(8.5*30.0:9pt)  -- (8.5*30.0:5pt)   
(9.5*30.0:9pt)  -- (9.5*30.0:5pt)   
(10.5*30.0:9pt) -- (10.5*30.0:5pt)  
(11.5*30.0:9pt) -- (11.5*30.0:5pt)  ;
 }}}

\pgfdeclareplotmark{m2cvb}{%
\node[scale=\mTwoscale*\AtoCscale*\symscale] at (0,0) {\tikz {%
\draw[fill=black,line width=.3pt,even odd rule,rotate=90] 
(0.*30.0:9pt)  -- (0.5*30.0:4.5pt)   -- 
(1.*30.0:9pt)  -- (1.5*30.0:4.5pt)   -- 
(2.*30.0:9pt)  -- (2.5*30.0:4.5pt)   -- 
(3.*30.0:9pt)  -- (3.5*30.0:4.5pt)   -- 
(4.*30.0:9pt)  -- (4.5*30.0:4.5pt)   -- 
(5.*30.0:9pt)  -- (5.5*30.0:4.5pt)   -- 
(6.*30.0:9pt)  -- (6.5*30.0:4.5pt)   -- 
(7.*30.0:9pt)  -- (7.5*30.0:4.5pt)   -- 
(8.*30.0:9pt)  -- (8.5*30.0:4.5pt)   -- 
(9.*30.0:9pt)  -- (9.5*30.0:4.5pt)   -- 
(10.*30.0:9pt) -- (10.5*30.0:4.5pt)  -- 
(11.*30.0:9pt) -- (11.5*30.0:4.5pt)  --  cycle
(0,0) circle (3pt);
\draw (0,0) circle (9pt);
%\draw[line width=.2pt,even odd rule,rotate=90] 
%(0.*30.0:11pt)  -- (0.5*30.0:6.5pt)   -- 
%(1.*30.0:11pt)  -- (1.5*30.0:6.5pt)   -- 
%(2.*30.0:11pt)  -- (2.5*30.0:6.5pt)   -- 
%(3.*30.0:11pt)  -- (3.5*30.0:6.5pt)   -- 
%(4.*30.0:11pt)  -- (4.5*30.0:6.5pt)   -- 
%(5.*30.0:11pt)  -- (5.5*30.0:6.5pt)   -- 
%(6.*30.0:11pt)  -- (6.5*30.0:6.5pt)   -- 
%(7.*30.0:11pt)  -- (7.5*30.0:6.5pt)   -- 
%(8.*30.0:11pt)  -- (8.5*30.0:6.5pt)   -- 
%(9.*30.0:11pt)  -- (9.5*30.0:6.5pt)   -- 
%(10.*30.0:11pt) -- (10.5*30.0:6.5pt)  -- 
%(11.*30.0:11pt) -- (11.5*30.0:6.5pt)  --  cycle;
\draw[line width=.4pt,even odd rule,rotate=90] 
(0.5*30.0:10pt)  -- (0.5*30.0:6.5pt)   
(1.5*30.0:10pt)  -- (1.5*30.0:6.5pt)   
(2.5*30.0:10pt)  -- (2.5*30.0:6.5pt)   
(3.5*30.0:10pt)  -- (3.5*30.0:6.5pt)   
(4.5*30.0:10pt)  -- (4.5*30.0:6.5pt)   
(5.5*30.0:10pt)  -- (5.5*30.0:6.5pt)   
(6.5*30.0:10pt)  -- (6.5*30.0:6.5pt)   
(7.5*30.0:10pt)  -- (7.5*30.0:6.5pt)   
(8.5*30.0:10pt)  -- (8.5*30.0:6.5pt)   
(9.5*30.0:10pt)  -- (9.5*30.0:6.5pt)   
(10.5*30.0:10pt) -- (10.5*30.0:6.5pt)  
(11.5*30.0:10pt) -- (11.5*30.0:6.5pt)  ;
 }}}



\pgfdeclareplotmark{m3a}{%
\node[scale=\mThreescale*\symscale] at (0,0) {\tikz {%
\draw[fill=black,line width=.3pt,even odd rule,rotate=90] 
(0.*36.0:10pt)  -- (0.5*36.0:5pt)   -- 
(1.*36.0:10pt)  -- (1.5*36.0:5pt)   -- 
(2.*36.0:10pt)  -- (2.5*36.0:5pt)   -- 
(3.*36.0:10pt)  -- (3.5*36.0:5pt)   -- 
(4.*36.0:10pt)  -- (4.5*36.0:5pt)   -- 
(5.*36.0:10pt)  -- (5.5*36.0:5pt)   -- 
(6.*36.0:10pt)  -- (6.5*36.0:5pt)   -- 
(7.*36.0:10pt)  -- (7.5*36.0:5pt)   -- 
(8.*36.0:10pt)  -- (8.5*36.0:5pt)   -- 
(9.*36.0:10pt)  -- (9.5*36.0:5pt)   --  cycle;
 }}}


\pgfdeclareplotmark{m3av}{%
\node[scale=\mThreescale*\symscale] at (0,0) {\tikz {%
\draw[fill=black,line width=.3pt,even odd rule,rotate=90] 
(0.*36.0:9pt)  -- (0.5*36.0:4.5pt)   -- 
(1.*36.0:9pt)  -- (1.5*36.0:4.5pt)   -- 
(2.*36.0:9pt)  -- (2.5*36.0:4.5pt)   -- 
(3.*36.0:9pt)  -- (3.5*36.0:4.5pt)   -- 
(4.*36.0:9pt)  -- (4.5*36.0:4.5pt)   -- 
(5.*36.0:9pt)  -- (5.5*36.0:4.5pt)   -- 
(6.*36.0:9pt)  -- (6.5*36.0:4.5pt)   -- 
(7.*36.0:9pt)  -- (7.5*36.0:4.5pt)   -- 
(8.*36.0:9pt)  -- (8.5*36.0:4.5pt)   -- 
(9.*36.0:9pt)  -- (9.5*36.0:4.5pt)   -- cycle;
\draw (0,0) circle (9pt);
%\draw[line width=.2pt,even odd rule,rotate=90] 
%(0.*36.0:11pt)  -- (0.5*36.0:6.5pt)   -- 
%(1.*36.0:11pt)  -- (1.5*36.0:6.5pt)   -- 
%(2.*36.0:11pt)  -- (2.5*36.0:6.5pt)   -- 
%(3.*36.0:11pt)  -- (3.5*36.0:6.5pt)   -- 
%(4.*36.0:11pt)  -- (4.5*36.0:6.5pt)   -- 
%(5.*36.0:11pt)  -- (5.5*36.0:6.5pt)   -- 
%(6.*36.0:11pt)  -- (6.5*36.0:6.5pt)   -- 
%(7.*36.0:11pt)  -- (7.5*36.0:6.5pt)   -- 
%(8.*36.0:11pt)  -- (8.5*36.0:6.5pt)   -- 
%(9.*36.0:11pt)  -- (9.5*36.0:6.5pt)   --  cycle;
 }}}



\pgfdeclareplotmark{m3ab}{%
\node[scale=\mThreescale*\symscale] at (0,0) {\tikz {%
\draw[fill=black,line width=.3pt,even odd rule,rotate=90] 
(0.*36.0:10pt)  -- (0.5*36.0:5pt)   -- 
(1.*36.0:10pt)  -- (1.5*36.0:5pt)   -- 
(2.*36.0:10pt)  -- (2.5*36.0:5pt)   -- 
(3.*36.0:10pt)  -- (3.5*36.0:5pt)   -- 
(4.*36.0:10pt)  -- (4.5*36.0:5pt)   -- 
(5.*36.0:10pt)  -- (5.5*36.0:5pt)   -- 
(6.*36.0:10pt)  -- (6.5*36.0:5pt)   -- 
(7.*36.0:10pt)  -- (7.5*36.0:5pt)   -- 
(8.*36.0:10pt)  -- (8.5*36.0:5pt)   -- 
(9.*36.0:10pt)  -- (9.5*36.0:5pt)   --  cycle;
\draw[line width=.4pt,even odd rule,rotate=90] 
(0.5*36.0:9pt)  -- (0.5*36.0:5pt)   
(1.5*36.0:9pt)  -- (1.5*36.0:5pt)   
(2.5*36.0:9pt)  -- (2.5*36.0:5pt)   
(3.5*36.0:9pt)  -- (3.5*36.0:5pt)   
(4.5*36.0:9pt)  -- (4.5*36.0:5pt)   
(5.5*36.0:9pt)  -- (5.5*36.0:5pt)   
(6.5*36.0:9pt)  -- (6.5*36.0:5pt)   
(7.5*36.0:9pt)  -- (7.5*36.0:5pt)   
(8.5*36.0:9pt)  -- (8.5*36.0:5pt)   
(9.5*36.0:9pt)  -- (9.5*36.0:5pt)  ;
 }}}

\pgfdeclareplotmark{m3avb}{%
\node[scale=\mThreescale*\symscale] at (0,0) {\tikz {%
\draw[fill=black,line width=.3pt,even odd rule,rotate=90] 
(0.*36.0:9pt)  -- (0.5*36.0:4.5pt)   -- 
(1.*36.0:9pt)  -- (1.5*36.0:4.5pt)   -- 
(2.*36.0:9pt)  -- (2.5*36.0:4.5pt)   -- 
(3.*36.0:9pt)  -- (3.5*36.0:4.5pt)   -- 
(4.*36.0:9pt)  -- (4.5*36.0:4.5pt)   -- 
(5.*36.0:9pt)  -- (5.5*36.0:4.5pt)   -- 
(6.*36.0:9pt)  -- (6.5*36.0:4.5pt)   -- 
(7.*36.0:9pt)  -- (7.5*36.0:4.5pt)   -- 
(8.*36.0:9pt)  -- (8.5*36.0:4.5pt)   -- 
(9.*36.0:9pt)  -- (9.5*36.0:4.5pt)   --  cycle;
\draw (0,0) circle (9pt);
%\draw[line width=.2pt,even odd rule,rotate=90] 
%(0.*36.0:11pt)  -- (0.5*36.0:6.5pt)   -- 
%(1.*36.0:11pt)  -- (1.5*36.0:6.5pt)   -- 
%(2.*36.0:11pt)  -- (2.5*36.0:6.5pt)   -- 
%(3.*36.0:11pt)  -- (3.5*36.0:6.5pt)   -- 
%(4.*36.0:11pt)  -- (4.5*36.0:6.5pt)   -- 
%(5.*36.0:11pt)  -- (5.5*36.0:6.5pt)   -- 
%(6.*36.0:11pt)  -- (6.5*36.0:6.5pt)   -- 
%(7.*36.0:11pt)  -- (7.5*36.0:6.5pt)   -- 
%(8.*36.0:11pt)  -- (8.5*36.0:6.5pt)   -- 
%(9.*36.0:11pt)  -- (9.5*36.0:6.5pt)   --  cycle;
\draw[line width=.4pt,even odd rule,rotate=90] 
(0.5*36.0:10pt)  -- (0.5*36.0:6.5pt)   
(1.5*36.0:10pt)  -- (1.5*36.0:6.5pt)   
(2.5*36.0:10pt)  -- (2.5*36.0:6.5pt)   
(3.5*36.0:10pt)  -- (3.5*36.0:6.5pt)   
(4.5*36.0:10pt)  -- (4.5*36.0:6.5pt)   
(5.5*36.0:10pt)  -- (5.5*36.0:6.5pt)   
(6.5*36.0:10pt)  -- (6.5*36.0:6.5pt)   
(7.5*36.0:10pt)  -- (7.5*36.0:6.5pt)   
(8.5*36.0:10pt)  -- (8.5*36.0:6.5pt)   
(9.5*36.0:10pt)  -- (9.5*36.0:6.5pt)   ;
 }}}



\pgfdeclareplotmark{m3b}{%
\node[scale=\mThreescale*\AtoBscale*\symscale] at (0,0) {\tikz {%
\draw[fill=black,line width=.3pt,even odd rule,rotate=90] 
(0.*36.0:10pt)  -- (0.5*36.0:5pt)   -- 
(1.*36.0:10pt)  -- (1.5*36.0:5pt)   -- 
(2.*36.0:10pt)  -- (2.5*36.0:5pt)   -- 
(3.*36.0:10pt)  -- (3.5*36.0:5pt)   -- 
(4.*36.0:10pt)  -- (4.5*36.0:5pt)   -- 
(5.*36.0:10pt)  -- (5.5*36.0:5pt)   -- 
(6.*36.0:10pt)  -- (6.5*36.0:5pt)   -- 
(7.*36.0:10pt)  -- (7.5*36.0:5pt)   -- 
(8.*36.0:10pt)  -- (8.5*36.0:5pt)   -- 
(9.*36.0:10pt)  -- (9.5*36.0:5pt)   -- cycle
(0,0) circle (2pt);
 }}}


\pgfdeclareplotmark{m3bv}{%
\node[scale=\mThreescale*\AtoBscale*\symscale] at (0,0) {\tikz {%
\draw[fill=black,line width=.3pt,even odd rule,rotate=90] 
(0.*36.0:9pt)  -- (0.5*36.0:4.5pt)   -- 
(1.*36.0:9pt)  -- (1.5*36.0:4.5pt)   -- 
(2.*36.0:9pt)  -- (2.5*36.0:4.5pt)   -- 
(3.*36.0:9pt)  -- (3.5*36.0:4.5pt)   -- 
(4.*36.0:9pt)  -- (4.5*36.0:4.5pt)   -- 
(5.*36.0:9pt)  -- (5.5*36.0:4.5pt)   -- 
(6.*36.0:9pt)  -- (6.5*36.0:4.5pt)   -- 
(7.*36.0:9pt)  -- (7.5*36.0:4.5pt)   -- 
(8.*36.0:9pt)  -- (8.5*36.0:4.5pt)   -- 
(9.*36.0:9pt)  -- (9.5*36.0:4.5pt)   -- cycle
(0,0) circle (2pt);
\draw (0,0) circle (9pt);
%\draw[line width=.2pt,even odd rule,rotate=90] 
%(0.*36.0:11pt)  -- (0.5*36.0:6.5pt)   -- 
%(1.*36.0:11pt)  -- (1.5*36.0:6.5pt)   -- 
%(2.*36.0:11pt)  -- (2.5*36.0:6.5pt)   -- 
%(3.*36.0:11pt)  -- (3.5*36.0:6.5pt)   -- 
%(4.*36.0:11pt)  -- (4.5*36.0:6.5pt)   -- 
%(5.*36.0:11pt)  -- (5.5*36.0:6.5pt)   -- 
%(6.*36.0:11pt)  -- (6.5*36.0:6.5pt)   -- 
%(7.*36.0:11pt)  -- (7.5*36.0:6.5pt)   -- 
%(8.*36.0:11pt)  -- (8.5*36.0:6.5pt)   -- 
%(9.*36.0:11pt)  -- (9.5*36.0:6.5pt)   -- cycle;
 }}}



\pgfdeclareplotmark{m3bb}{%
\node[scale=\mThreescale*\AtoBscale*\symscale] at (0,0) {\tikz {%
\draw[fill=black,line width=.3pt,even odd rule,rotate=90] 
(0.*36.0:10pt)  -- (0.5*36.0:5pt)   -- 
(1.*36.0:10pt)  -- (1.5*36.0:5pt)   -- 
(2.*36.0:10pt)  -- (2.5*36.0:5pt)   -- 
(3.*36.0:10pt)  -- (3.5*36.0:5pt)   -- 
(4.*36.0:10pt)  -- (4.5*36.0:5pt)   -- 
(5.*36.0:10pt)  -- (5.5*36.0:5pt)   -- 
(6.*36.0:10pt)  -- (6.5*36.0:5pt)   -- 
(7.*36.0:10pt)  -- (7.5*36.0:5pt)   -- 
(8.*36.0:10pt)  -- (8.5*36.0:5pt)   -- 
(9.*36.0:10pt)  -- (9.5*36.0:5pt)   --  cycle
(0,0) circle (2pt);
\draw[line width=.4pt,even odd rule,rotate=90] 
(0.5*36.0:9pt)  -- (0.5*36.0:5pt)   
(1.5*36.0:9pt)  -- (1.5*36.0:5pt)   
(2.5*36.0:9pt)  -- (2.5*36.0:5pt)   
(3.5*36.0:9pt)  -- (3.5*36.0:5pt)   
(4.5*36.0:9pt)  -- (4.5*36.0:5pt)   
(5.5*36.0:9pt)  -- (5.5*36.0:5pt)   
(6.5*36.0:9pt)  -- (6.5*36.0:5pt)   
(7.5*36.0:9pt)  -- (7.5*36.0:5pt)   
(8.5*36.0:9pt)  -- (8.5*36.0:5pt)   
(9.5*36.0:9pt)  -- (9.5*36.0:5pt)  ;
 }}}

\pgfdeclareplotmark{m3bvb}{%
\node[scale=\mThreescale*\AtoBscale*\symscale] at (0,0) {\tikz {%
\draw[fill=black,line width=.3pt,even odd rule,rotate=90] 
(0.*36.0:9pt)  -- (0.5*36.0:4.5pt)   -- 
(1.*36.0:9pt)  -- (1.5*36.0:4.5pt)   -- 
(2.*36.0:9pt)  -- (2.5*36.0:4.5pt)   -- 
(3.*36.0:9pt)  -- (3.5*36.0:4.5pt)   -- 
(4.*36.0:9pt)  -- (4.5*36.0:4.5pt)   -- 
(5.*36.0:9pt)  -- (5.5*36.0:4.5pt)   -- 
(6.*36.0:9pt)  -- (6.5*36.0:4.5pt)   -- 
(7.*36.0:9pt)  -- (7.5*36.0:4.5pt)   -- 
(8.*36.0:9pt)  -- (8.5*36.0:4.5pt)   -- 
(9.*36.0:9pt)  -- (9.5*36.0:4.5pt)   --  cycle
(0,0) circle (2pt);
\draw (0,0) circle (9pt);
%\draw[line width=.2pt,even odd rule,rotate=90] 
%(0.*36.0:11pt)  -- (0.5*36.0:6.5pt)   -- 
%(1.*36.0:11pt)  -- (1.5*36.0:6.5pt)   -- 
%(2.*36.0:11pt)  -- (2.5*36.0:6.5pt)   -- 
%(3.*36.0:11pt)  -- (3.5*36.0:6.5pt)   -- 
%(4.*36.0:11pt)  -- (4.5*36.0:6.5pt)   -- 
%(5.*36.0:11pt)  -- (5.5*36.0:6.5pt)   -- 
%(6.*36.0:11pt)  -- (6.5*36.0:6.5pt)   -- 
%(7.*36.0:11pt)  -- (7.5*36.0:6.5pt)   -- 
%(8.*36.0:11pt)  -- (8.5*36.0:6.5pt)   -- 
%(9.*36.0:11pt)  -- (9.5*36.0:6.5pt)   -- cycle;
\draw[line width=.4pt,even odd rule,rotate=90] 
(0.5*36.0:10pt)  -- (0.5*36.0:6.5pt)   
(1.5*36.0:10pt)  -- (1.5*36.0:6.5pt)   
(2.5*36.0:10pt)  -- (2.5*36.0:6.5pt)   
(3.5*36.0:10pt)  -- (3.5*36.0:6.5pt)   
(4.5*36.0:10pt)  -- (4.5*36.0:6.5pt)   
(5.5*36.0:10pt)  -- (5.5*36.0:6.5pt)   
(6.5*36.0:10pt)  -- (6.5*36.0:6.5pt)   
(7.5*36.0:10pt)  -- (7.5*36.0:6.5pt)   
(8.5*36.0:10pt)  -- (8.5*36.0:6.5pt)   
(9.5*36.0:10pt)  -- (9.5*36.0:6.5pt)  ;
 }}}



\pgfdeclareplotmark{m3c}{%
\node[scale=\mThreescale*\AtoCscale*\symscale] at (0,0) {\tikz {%
\draw[fill=black,line width=.3pt,even odd rule,rotate=90] 
(0.*36.0:10pt)  -- (0.5*36.0:5pt)   -- 
(1.*36.0:10pt)  -- (1.5*36.0:5pt)   -- 
(2.*36.0:10pt)  -- (2.5*36.0:5pt)   -- 
(3.*36.0:10pt)  -- (3.5*36.0:5pt)   -- 
(4.*36.0:10pt)  -- (4.5*36.0:5pt)   -- 
(5.*36.0:10pt)  -- (5.5*36.0:5pt)   -- 
(6.*36.0:10pt)  -- (6.5*36.0:5pt)   -- 
(7.*36.0:10pt)  -- (7.5*36.0:5pt)   -- 
(8.*36.0:10pt)  -- (8.5*36.0:5pt)   -- 
(9.*36.0:10pt)  -- (9.5*36.0:5pt)   -- cycle
(0,0) circle (3pt);
 }}}


\pgfdeclareplotmark{m3cv}{%
\node[scale=\mThreescale*\AtoCscale*\symscale] at (0,0) {\tikz {%
\draw[fill=black,line width=.3pt,even odd rule,rotate=90] 
(0.*36.0:9pt)  -- (0.5*36.0:4.5pt)   -- 
(1.*36.0:9pt)  -- (1.5*36.0:4.5pt)   -- 
(2.*36.0:9pt)  -- (2.5*36.0:4.5pt)   -- 
(3.*36.0:9pt)  -- (3.5*36.0:4.5pt)   -- 
(4.*36.0:9pt)  -- (4.5*36.0:4.5pt)   -- 
(5.*36.0:9pt)  -- (5.5*36.0:4.5pt)   -- 
(6.*36.0:9pt)  -- (6.5*36.0:4.5pt)   -- 
(7.*36.0:9pt)  -- (7.5*36.0:4.5pt)   -- 
(8.*36.0:9pt)  -- (8.5*36.0:4.5pt)   -- 
(9.*36.0:9pt)  -- (9.5*36.0:4.5pt)   --  cycle
(0,0) circle (3pt);
\draw (0,0) circle (9pt);
%\draw[line width=.2pt,even odd rule,rotate=90] 
%(0.*36.0:11pt)  -- (0.5*36.0:6.5pt)   -- 
%(1.*36.0:11pt)  -- (1.5*36.0:6.5pt)   -- 
%(2.*36.0:11pt)  -- (2.5*36.0:6.5pt)   -- 
%(3.*36.0:11pt)  -- (3.5*36.0:6.5pt)   -- 
%(4.*36.0:11pt)  -- (4.5*36.0:6.5pt)   -- 
%(5.*36.0:11pt)  -- (5.5*36.0:6.5pt)   -- 
%(6.*36.0:11pt)  -- (6.5*36.0:6.5pt)   -- 
%(7.*36.0:11pt)  -- (7.5*36.0:6.5pt)   -- 
%(8.*36.0:11pt)  -- (8.5*36.0:6.5pt)   -- 
%(9.*36.0:11pt)  -- (9.5*36.0:6.5pt)   -- cycle;
 }}}



\pgfdeclareplotmark{m3cb}{%
\node[scale=\mThreescale*\AtoCscale*\symscale] at (0,0) {\tikz {%
\draw[fill=black,line width=.3pt,even odd rule,rotate=90] 
(0.*36.0:10pt)  -- (0.5*36.0:5pt)   -- 
(1.*36.0:10pt)  -- (1.5*36.0:5pt)   -- 
(2.*36.0:10pt)  -- (2.5*36.0:5pt)   -- 
(3.*36.0:10pt)  -- (3.5*36.0:5pt)   -- 
(4.*36.0:10pt)  -- (4.5*36.0:5pt)   -- 
(5.*36.0:10pt)  -- (5.5*36.0:5pt)   -- 
(6.*36.0:10pt)  -- (6.5*36.0:5pt)   -- 
(7.*36.0:10pt)  -- (7.5*36.0:5pt)   -- 
(8.*36.0:10pt)  -- (8.5*36.0:5pt)   -- 
(9.*36.0:10pt)  -- (9.5*36.0:5pt)   -- cycle
(0,0) circle (3pt);
\draw[line width=.4pt,even odd rule,rotate=90] 
(0.5*36.0:9pt)  -- (0.5*36.0:5pt)   
(1.5*36.0:9pt)  -- (1.5*36.0:5pt)   
(2.5*36.0:9pt)  -- (2.5*36.0:5pt)   
(3.5*36.0:9pt)  -- (3.5*36.0:5pt)   
(4.5*36.0:9pt)  -- (4.5*36.0:5pt)   
(5.5*36.0:9pt)  -- (5.5*36.0:5pt)   
(6.5*36.0:9pt)  -- (6.5*36.0:5pt)   
(7.5*36.0:9pt)  -- (7.5*36.0:5pt)   
(8.5*36.0:9pt)  -- (8.5*36.0:5pt)   
(9.5*36.0:9pt)  -- (9.5*36.0:5pt)   ;
 }}}

\pgfdeclareplotmark{m3cvb}{%
\node[scale=\mThreescale*\AtoCscale*\symscale] at (0,0) {\tikz {%
\draw[fill=black,line width=.3pt,even odd rule,rotate=90] 
(0.*36.0:9pt)  -- (0.5*36.0:4.5pt)   -- 
(1.*36.0:9pt)  -- (1.5*36.0:4.5pt)   -- 
(2.*36.0:9pt)  -- (2.5*36.0:4.5pt)   -- 
(3.*36.0:9pt)  -- (3.5*36.0:4.5pt)   -- 
(4.*36.0:9pt)  -- (4.5*36.0:4.5pt)   -- 
(5.*36.0:9pt)  -- (5.5*36.0:4.5pt)   -- 
(6.*36.0:9pt)  -- (6.5*36.0:4.5pt)   -- 
(7.*36.0:9pt)  -- (7.5*36.0:4.5pt)   -- 
(8.*36.0:9pt)  -- (8.5*36.0:4.5pt)   -- 
(9.*36.0:9pt)  -- (9.5*36.0:4.5pt)   -- cycle
(0,0) circle (3pt);
\draw (0,0) circle (9pt);
%\draw[line width=.2pt,even odd rule,rotate=90] 
%(0.*36.0:11pt)  -- (0.5*36.0:6.5pt)   -- 
%(1.*36.0:11pt)  -- (1.5*36.0:6.5pt)   -- 
%(2.*36.0:11pt)  -- (2.5*36.0:6.5pt)   -- 
%(3.*36.0:11pt)  -- (3.5*36.0:6.5pt)   -- 
%(4.*36.0:11pt)  -- (4.5*36.0:6.5pt)   -- 
%(5.*36.0:11pt)  -- (5.5*36.0:6.5pt)   -- 
%(6.*36.0:11pt)  -- (6.5*36.0:6.5pt)   -- 
%(7.*36.0:11pt)  -- (7.5*36.0:6.5pt)   -- 
%(8.*36.0:11pt)  -- (8.5*36.0:6.5pt)   -- 
%(9.*36.0:11pt)  -- (9.5*36.0:6.5pt)   --  cycle;
\draw[line width=.4pt,even odd rule,rotate=90] 
(0.5*36.0:10pt)  -- (0.5*36.0:6.5pt)   
(1.5*36.0:10pt)  -- (1.5*36.0:6.5pt)   
(2.5*36.0:10pt)  -- (2.5*36.0:6.5pt)   
(3.5*36.0:10pt)  -- (3.5*36.0:6.5pt)   
(4.5*36.0:10pt)  -- (4.5*36.0:6.5pt)   
(5.5*36.0:10pt)  -- (5.5*36.0:6.5pt)   
(6.5*36.0:10pt)  -- (6.5*36.0:6.5pt)   
(7.5*36.0:10pt)  -- (7.5*36.0:6.5pt)   
(8.5*36.0:10pt)  -- (8.5*36.0:6.5pt)   
(9.5*36.0:10pt)  -- (9.5*36.0:6.5pt)   ;
 }}}



\pgfdeclareplotmark{m4a}{%
\node[scale=\mFourscale*\symscale] at (0,0) {\tikz {%
\draw[fill=black,line width=.3pt,even odd rule,rotate=90] 
(0.*45.0:10pt)  -- (0.5*45.0:5pt)   -- 
(1.*45.0:10pt)  -- (1.5*45.0:5pt)   -- 
(2.*45.0:10pt)  -- (2.5*45.0:5pt)   -- 
(3.*45.0:10pt)  -- (3.5*45.0:5pt)   -- 
(4.*45.0:10pt)  -- (4.5*45.0:5pt)   -- 
(5.*45.0:10pt)  -- (5.5*45.0:5pt)   -- 
(6.*45.0:10pt)  -- (6.5*45.0:5pt)   -- 
(7.*45.0:10pt)  -- (7.5*45.0:5pt)   --  cycle;
 }}}


\pgfdeclareplotmark{m4av}{%
\node[scale=\mFourscale*\symscale] at (0,0) {\tikz {%
\draw[fill=black,line width=.3pt,even odd rule,rotate=90] 
(0.*45.0:9pt)  -- (0.5*45.0:4.5pt)   -- 
(1.*45.0:9pt)  -- (1.5*45.0:4.5pt)   -- 
(2.*45.0:9pt)  -- (2.5*45.0:4.5pt)   -- 
(3.*45.0:9pt)  -- (3.5*45.0:4.5pt)   -- 
(4.*45.0:9pt)  -- (4.5*45.0:4.5pt)   -- 
(5.*45.0:9pt)  -- (5.5*45.0:4.5pt)   -- 
(6.*45.0:9pt)  -- (6.5*45.0:4.5pt)   -- 
(7.*45.0:9pt)  -- (7.5*45.0:4.5pt)   -- cycle;
\draw (0,0) circle (9pt);
%\draw[line width=.3pt,even odd rule,rotate=90] 
%(0.*45.0:11pt)  -- (0.5*45.0:6.5pt)   -- 
%(1.*45.0:11pt)  -- (1.5*45.0:6.5pt)   -- 
%(2.*45.0:11pt)  -- (2.5*45.0:6.5pt)   -- 
%(3.*45.0:11pt)  -- (3.5*45.0:6.5pt)   -- 
%(4.*45.0:11pt)  -- (4.5*45.0:6.5pt)   -- 
%(5.*45.0:11pt)  -- (5.5*45.0:6.5pt)   -- 
%(6.*45.0:11pt)  -- (6.5*45.0:6.5pt)   -- 
%(7.*45.0:11pt)  -- (7.5*45.0:6.5pt)   --  cycle;
 }}}



\pgfdeclareplotmark{m4ab}{%
\node[scale=\mFourscale*\symscale] at (0,0) {\tikz {%
\draw[fill=black,line width=.3pt,even odd rule,rotate=90] 
(0.*45.0:10pt)  -- (0.5*45.0:5pt)   -- 
(1.*45.0:10pt)  -- (1.5*45.0:5pt)   -- 
(2.*45.0:10pt)  -- (2.5*45.0:5pt)   -- 
(3.*45.0:10pt)  -- (3.5*45.0:5pt)   -- 
(4.*45.0:10pt)  -- (4.5*45.0:5pt)   -- 
(5.*45.0:10pt)  -- (5.5*45.0:5pt)   -- 
(6.*45.0:10pt)  -- (6.5*45.0:5pt)   -- 
(7.*45.0:10pt)  -- (7.5*45.0:5pt)   --  cycle;
\draw[line width=.4pt,even odd rule,rotate=90] 
(0.5*45.0:9pt)  -- (0.5*45.0:5pt)   
(1.5*45.0:9pt)  -- (1.5*45.0:5pt)   
(2.5*45.0:9pt)  -- (2.5*45.0:5pt)   
(3.5*45.0:9pt)  -- (3.5*45.0:5pt)   
(4.5*45.0:9pt)  -- (4.5*45.0:5pt)   
(5.5*45.0:9pt)  -- (5.5*45.0:5pt)   
(6.5*45.0:9pt)  -- (6.5*45.0:5pt)   
(7.5*45.0:9pt)  -- (7.5*45.0:5pt)  ;
 }}}

\pgfdeclareplotmark{m4avb}{%
\node[scale=\mFourscale*\symscale] at (0,0) {\tikz {%
\draw[fill=black,line width=.3pt,even odd rule,rotate=90] 
(0.*45.0:9pt)  -- (0.5*45.0:4.5pt)   -- 
(1.*45.0:9pt)  -- (1.5*45.0:4.5pt)   -- 
(2.*45.0:9pt)  -- (2.5*45.0:4.5pt)   -- 
(3.*45.0:9pt)  -- (3.5*45.0:4.5pt)   -- 
(4.*45.0:9pt)  -- (4.5*45.0:4.5pt)   -- 
(5.*45.0:9pt)  -- (5.5*45.0:4.5pt)   -- 
(6.*45.0:9pt)  -- (6.5*45.0:4.5pt)   -- 
(7.*45.0:9pt)  -- (7.5*45.0:4.5pt)   --  cycle;
\draw (0,0) circle (9pt);
%\draw[line width=.3pt,even odd rule,rotate=90] 
%(0.*45.0:11pt)  -- (0.5*45.0:6.5pt)   -- 
%(1.*45.0:11pt)  -- (1.5*45.0:6.5pt)   -- 
%(2.*45.0:11pt)  -- (2.5*45.0:6.5pt)   -- 
%(3.*45.0:11pt)  -- (3.5*45.0:6.5pt)   -- 
%(4.*45.0:11pt)  -- (4.5*45.0:6.5pt)   -- 
%(5.*45.0:11pt)  -- (5.5*45.0:6.5pt)   -- 
%(6.*45.0:11pt)  -- (6.5*45.0:6.5pt)   -- 
%(7.*45.0:11pt)  -- (7.5*45.0:6.5pt)   --  cycle;
\draw[line width=.4pt,even odd rule,rotate=90] 
(0.5*45.0:10pt)  -- (0.5*45.0:6.5pt)   
(1.5*45.0:10pt)  -- (1.5*45.0:6.5pt)   
(2.5*45.0:10pt)  -- (2.5*45.0:6.5pt)   
(3.5*45.0:10pt)  -- (3.5*45.0:6.5pt)   
(4.5*45.0:10pt)  -- (4.5*45.0:6.5pt)   
(5.5*45.0:10pt)  -- (5.5*45.0:6.5pt)   
(6.5*45.0:10pt)  -- (6.5*45.0:6.5pt)   
(7.5*45.0:10pt)  -- (7.5*45.0:6.5pt)   ;
 }}}



\pgfdeclareplotmark{m4b}{%
\node[scale=\mFourscale*\AtoBscale*\symscale] at (0,0) {\tikz {%
\draw[fill=black,line width=.3pt,even odd rule,rotate=90] 
(0.*45.0:10pt)  -- (0.5*45.0:5pt)   -- 
(1.*45.0:10pt)  -- (1.5*45.0:5pt)   -- 
(2.*45.0:10pt)  -- (2.5*45.0:5pt)   -- 
(3.*45.0:10pt)  -- (3.5*45.0:5pt)   -- 
(4.*45.0:10pt)  -- (4.5*45.0:5pt)   -- 
(5.*45.0:10pt)  -- (5.5*45.0:5pt)   -- 
(6.*45.0:10pt)  -- (6.5*45.0:5pt)   -- 
(7.*45.0:10pt)  -- (7.5*45.0:5pt)   -- cycle
(0,0) circle (2pt);
 }}}


\pgfdeclareplotmark{m4bv}{%
\node[scale=\mFourscale*\AtoBscale*\symscale] at (0,0) {\tikz {%
\draw[fill=black,line width=.3pt,even odd rule,rotate=90] 
(0.*45.0:9pt)  -- (0.5*45.0:4.5pt)   -- 
(1.*45.0:9pt)  -- (1.5*45.0:4.5pt)   -- 
(2.*45.0:9pt)  -- (2.5*45.0:4.5pt)   -- 
(3.*45.0:9pt)  -- (3.5*45.0:4.5pt)   -- 
(4.*45.0:9pt)  -- (4.5*45.0:4.5pt)   -- 
(5.*45.0:9pt)  -- (5.5*45.0:4.5pt)   -- 
(6.*45.0:9pt)  -- (6.5*45.0:4.5pt)   -- 
(7.*45.0:9pt)  -- (7.5*45.0:4.5pt)   -- cycle
(0,0) circle (2pt);
\draw (0,0) circle (9pt);
%\draw[line width=.3pt,even odd rule,rotate=90] 
%(0.*45.0:11pt)  -- (0.5*45.0:6.5pt)   -- 
%(1.*45.0:11pt)  -- (1.5*45.0:6.5pt)   -- 
%(2.*45.0:11pt)  -- (2.5*45.0:6.5pt)   -- 
%(3.*45.0:11pt)  -- (3.5*45.0:6.5pt)   -- 
%(4.*45.0:11pt)  -- (4.5*45.0:6.5pt)   -- 
%(5.*45.0:11pt)  -- (5.5*45.0:6.5pt)   -- 
%(6.*45.0:11pt)  -- (6.5*45.0:6.5pt)   -- 
%(7.*45.0:11pt)  -- (7.5*45.0:6.5pt)   --cycle;
 }}}



\pgfdeclareplotmark{m4bb}{%
\node[scale=\mFourscale*\AtoBscale*\symscale] at (0,0) {\tikz {%
\draw[fill=black,line width=.3pt,even odd rule,rotate=90] 
(0.*45.0:10pt)  -- (0.5*45.0:5pt)   -- 
(1.*45.0:10pt)  -- (1.5*45.0:5pt)   -- 
(2.*45.0:10pt)  -- (2.5*45.0:5pt)   -- 
(3.*45.0:10pt)  -- (3.5*45.0:5pt)   -- 
(4.*45.0:10pt)  -- (4.5*45.0:5pt)   -- 
(5.*45.0:10pt)  -- (5.5*45.0:5pt)   -- 
(6.*45.0:10pt)  -- (6.5*45.0:5pt)   -- 
(7.*45.0:10pt)  -- (7.5*45.0:5pt)   --  cycle
(0,0) circle (2pt);
\draw[line width=.4pt,even odd rule,rotate=90] 
(0.5*45.0:9pt)  -- (0.5*45.0:5pt)   
(1.5*45.0:9pt)  -- (1.5*45.0:5pt)   
(2.5*45.0:9pt)  -- (2.5*45.0:5pt)   
(3.5*45.0:9pt)  -- (3.5*45.0:5pt)   
(4.5*45.0:9pt)  -- (4.5*45.0:5pt)   
(5.5*45.0:9pt)  -- (5.5*45.0:5pt)   
(6.5*45.0:9pt)  -- (6.5*45.0:5pt)   
(7.5*45.0:9pt)  -- (7.5*45.0:5pt)  ;
 }}}

\pgfdeclareplotmark{m4bvb}{%
\node[scale=\mFourscale*\AtoBscale*\symscale] at (0,0) {\tikz {%
\draw[fill=black,line width=.3pt,even odd rule,rotate=90] 
(0.*45.0:9pt)  -- (0.5*45.0:4.5pt)   -- 
(1.*45.0:9pt)  -- (1.5*45.0:4.5pt)   -- 
(2.*45.0:9pt)  -- (2.5*45.0:4.5pt)   -- 
(3.*45.0:9pt)  -- (3.5*45.0:4.5pt)   -- 
(4.*45.0:9pt)  -- (4.5*45.0:4.5pt)   -- 
(5.*45.0:9pt)  -- (5.5*45.0:4.5pt)   -- 
(6.*45.0:9pt)  -- (6.5*45.0:4.5pt)   -- 
(7.*45.0:9pt)  -- (7.5*45.0:4.5pt)   --  cycle
(0,0) circle (2pt);
\draw (0,0) circle (9pt);
%\draw[line width=.3pt,even odd rule,rotate=90] 
%(0.*45.0:11pt)  -- (0.5*45.0:6.5pt)   -- 
%(1.*45.0:11pt)  -- (1.5*45.0:6.5pt)   -- 
%(2.*45.0:11pt)  -- (2.5*45.0:6.5pt)   -- 
%(3.*45.0:11pt)  -- (3.5*45.0:6.5pt)   -- 
%(4.*45.0:11pt)  -- (4.5*45.0:6.5pt)   -- 
%(5.*45.0:11pt)  -- (5.5*45.0:6.5pt)   -- 
%(6.*45.0:11pt)  -- (6.5*45.0:6.5pt)   -- 
%(7.*45.0:11pt)  -- (7.5*45.0:6.5pt)   -- cycle;
\draw[line width=.4pt,even odd rule,rotate=90] 
(0.5*45.0:10pt)  -- (0.5*45.0:6.5pt)   
(1.5*45.0:10pt)  -- (1.5*45.0:6.5pt)   
(2.5*45.0:10pt)  -- (2.5*45.0:6.5pt)   
(3.5*45.0:10pt)  -- (3.5*45.0:6.5pt)   
(4.5*45.0:10pt)  -- (4.5*45.0:6.5pt)   
(5.5*45.0:10pt)  -- (5.5*45.0:6.5pt)   
(6.5*45.0:10pt)  -- (6.5*45.0:6.5pt)   
(7.5*45.0:10pt)  -- (7.5*45.0:6.5pt)  ;
 }}}



\pgfdeclareplotmark{m4c}{%
\node[scale=\mFourscale*\AtoCscale*\symscale] at (0,0) {\tikz {%
\draw[fill=black,line width=.3pt,even odd rule,rotate=90] 
(0.*45.0:10pt)  -- (0.5*45.0:5pt)   -- 
(1.*45.0:10pt)  -- (1.5*45.0:5pt)   -- 
(2.*45.0:10pt)  -- (2.5*45.0:5pt)   -- 
(3.*45.0:10pt)  -- (3.5*45.0:5pt)   -- 
(4.*45.0:10pt)  -- (4.5*45.0:5pt)   -- 
(5.*45.0:10pt)  -- (5.5*45.0:5pt)   -- 
(6.*45.0:10pt)  -- (6.5*45.0:5pt)   -- 
(7.*45.0:10pt)  -- (7.5*45.0:5pt)   -- cycle
(0,0) circle (3pt);
 }}}


\pgfdeclareplotmark{m4cv}{%
\node[scale=\mFourscale*\AtoCscale*\symscale] at (0,0) {\tikz {%
\draw[fill=black,line width=.3pt,even odd rule,rotate=90] 
(0.*45.0:9pt)  -- (0.5*45.0:4.5pt)   -- 
(1.*45.0:9pt)  -- (1.5*45.0:4.5pt)   -- 
(2.*45.0:9pt)  -- (2.5*45.0:4.5pt)   -- 
(3.*45.0:9pt)  -- (3.5*45.0:4.5pt)   -- 
(4.*45.0:9pt)  -- (4.5*45.0:4.5pt)   -- 
(5.*45.0:9pt)  -- (5.5*45.0:4.5pt)   -- 
(6.*45.0:9pt)  -- (6.5*45.0:4.5pt)   -- 
(7.*45.0:9pt)  -- (7.5*45.0:4.5pt)   --  cycle
(0,0) circle (3pt);
\draw (0,0) circle (9pt);
%\draw[line width=.3pt,even odd rule,rotate=90] 
%(0.*45.0:11pt)  -- (0.5*45.0:6.5pt)   -- 
%(1.*45.0:11pt)  -- (1.5*45.0:6.5pt)   -- 
%(2.*45.0:11pt)  -- (2.5*45.0:6.5pt)   -- 
%(3.*45.0:11pt)  -- (3.5*45.0:6.5pt)   -- 
%(4.*45.0:11pt)  -- (4.5*45.0:6.5pt)   -- 
%(5.*45.0:11pt)  -- (5.5*45.0:6.5pt)   -- 
%(6.*45.0:11pt)  -- (6.5*45.0:6.5pt)   -- 
%(7.*45.0:11pt)  -- (7.5*45.0:6.5pt)   -- cycle;
 }}}



\pgfdeclareplotmark{m4cb}{%
\node[scale=\mFourscale*\AtoCscale*\symscale] at (0,0) {\tikz {%
\draw[fill=black,line width=.3pt,even odd rule,rotate=90] 
(0.*45.0:10pt)  -- (0.5*45.0:5pt)   -- 
(1.*45.0:10pt)  -- (1.5*45.0:5pt)   -- 
(2.*45.0:10pt)  -- (2.5*45.0:5pt)   -- 
(3.*45.0:10pt)  -- (3.5*45.0:5pt)   -- 
(4.*45.0:10pt)  -- (4.5*45.0:5pt)   -- 
(5.*45.0:10pt)  -- (5.5*45.0:5pt)   -- 
(6.*45.0:10pt)  -- (6.5*45.0:5pt)   -- 
(7.*45.0:10pt)  -- (7.5*45.0:5pt)   -- cycle
(0,0) circle (3pt);
\draw[line width=.4pt,even odd rule,rotate=90] 
(0.5*45.0:9pt)  -- (0.5*45.0:5pt)   
(1.5*45.0:9pt)  -- (1.5*45.0:5pt)   
(2.5*45.0:9pt)  -- (2.5*45.0:5pt)   
(3.5*45.0:9pt)  -- (3.5*45.0:5pt)   
(4.5*45.0:9pt)  -- (4.5*45.0:5pt)   
(5.5*45.0:9pt)  -- (5.5*45.0:5pt)   
(6.5*45.0:9pt)  -- (6.5*45.0:5pt)   
(7.5*45.0:9pt)  -- (7.5*45.0:5pt)  ;
 }}}

\pgfdeclareplotmark{m4cvb}{%
\node[scale=\mFourscale*\AtoCscale*\symscale] at (0,0) {\tikz {%
\draw[fill=black,line width=.3pt,even odd rule,rotate=90] 
(0.*45.0:9pt)  -- (0.5*45.0:4.5pt)   -- 
(1.*45.0:9pt)  -- (1.5*45.0:4.5pt)   -- 
(2.*45.0:9pt)  -- (2.5*45.0:4.5pt)   -- 
(3.*45.0:9pt)  -- (3.5*45.0:4.5pt)   -- 
(4.*45.0:9pt)  -- (4.5*45.0:4.5pt)   -- 
(5.*45.0:9pt)  -- (5.5*45.0:4.5pt)   -- 
(6.*45.0:9pt)  -- (6.5*45.0:4.5pt)   -- 
(7.*45.0:9pt)  -- (7.5*45.0:4.5pt)   --  cycle
(0,0) circle (3pt);
\draw (0,0) circle (9pt);
%\draw[line width=.3pt,even odd rule,rotate=90] 
%(0.*45.0:11pt)  -- (0.5*45.0:6.5pt)   -- 
%(1.*45.0:11pt)  -- (1.5*45.0:6.5pt)   -- 
%(2.*45.0:11pt)  -- (2.5*45.0:6.5pt)   -- 
%(3.*45.0:11pt)  -- (3.5*45.0:6.5pt)   -- 
%(4.*45.0:11pt)  -- (4.5*45.0:6.5pt)   -- 
%(5.*45.0:11pt)  -- (5.5*45.0:6.5pt)   -- 
%(6.*45.0:11pt)  -- (6.5*45.0:6.5pt)   -- 
%(7.*45.0:11pt)  -- (7.5*45.0:6.5pt)   --  cycle;
\draw[line width=.4pt,even odd rule,rotate=90] 
(0.5*45.0:10pt)  -- (0.5*45.0:6.5pt)   
(1.5*45.0:10pt)  -- (1.5*45.0:6.5pt)   
(2.5*45.0:10pt)  -- (2.5*45.0:6.5pt)   
(3.5*45.0:10pt)  -- (3.5*45.0:6.5pt)   
(4.5*45.0:10pt)  -- (4.5*45.0:6.5pt)   
(5.5*45.0:10pt)  -- (5.5*45.0:6.5pt)   
(6.5*45.0:10pt)  -- (6.5*45.0:6.5pt)   
(7.5*45.0:10pt)  -- (7.5*45.0:6.5pt)   ;
 }}}



\pgfdeclareplotmark{m5a}{%
\node[scale=\mFivescale*\symscale] at (0,0) {\tikz {%
\draw[fill=black,line width=.3pt,even odd rule,rotate=90] 
(0.*60.0:10pt)  -- (0.5*60.0:1.14*5.0pt)   -- 
(1.*60.0:10pt)  -- (1.5*60.0:1.14*5.0pt)   -- 
(2.*60.0:10pt)  -- (2.5*60.0:1.14*5.0pt)   -- 
(3.*60.0:10pt)  -- (3.5*60.0:1.14*5.0pt)   -- 
(4.*60.0:10pt)  -- (4.5*60.0:1.14*5.0pt)   -- 
(5.*60.0:10pt)  -- (5.5*60.0:1.14*5.0pt)   --  cycle;
 }}}


\pgfdeclareplotmark{m5av}{%
\node[scale=\mFivescale*\symscale] at (0,0) {\tikz {%
\draw[fill=black,line width=.3pt,even odd rule,rotate=90] 
(0.*60.0:9pt)  -- (0.5*60.0:1.14*4.5pt)   -- 
(1.*60.0:9pt)  -- (1.5*60.0:1.14*4.5pt)   -- 
(2.*60.0:9pt)  -- (2.5*60.0:1.14*4.5pt)   -- 
(3.*60.0:9pt)  -- (3.5*60.0:1.14*4.5pt)   -- 
(4.*60.0:9pt)  -- (4.5*60.0:1.14*4.5pt)   -- 
(5.*60.0:9pt)  -- (5.5*60.0:1.14*4.5pt)   -- cycle;
\draw (0,0) circle (9pt);
%\draw[line width=.35pt,even odd rule,rotate=90] 
%(0.*60.0:11pt)  -- (0.5*60.0:1.14*1.00*6.0pt)   -- 
%(1.*60.0:11pt)  -- (1.5*60.0:1.14*1.00*6.0pt)   -- 
%(2.*60.0:11pt)  -- (2.5*60.0:1.14*1.00*6.0pt)   -- 
%(3.*60.0:11pt)  -- (3.5*60.0:1.14*1.00*6.0pt)   -- 
%(4.*60.0:11pt)  -- (4.5*60.0:1.14*1.00*6.0pt)   -- 
%(5.*60.0:11pt)  -- (5.5*60.0:1.14*1.00*6.0pt)   --  cycle;
 }}}



\pgfdeclareplotmark{m5ab}{%
\node[scale=\mFivescale*\symscale] at (0,0) {\tikz {%
\draw[fill=black,line width=.3pt,even odd rule,rotate=90] 
(0.*60.0:10pt)  -- (0.5*60.0:1.14*5.0pt)   -- 
(1.*60.0:10pt)  -- (1.5*60.0:1.14*5.0pt)   -- 
(2.*60.0:10pt)  -- (2.5*60.0:1.14*5.0pt)   -- 
(3.*60.0:10pt)  -- (3.5*60.0:1.14*5.0pt)   -- 
(4.*60.0:10pt)  -- (4.5*60.0:1.14*5.0pt)   -- 
(5.*60.0:10pt)  -- (5.5*60.0:1.14*5.0pt)   --  cycle;
\draw[line width=0.6pt,even odd rule,rotate=90] 
(0.5*60.0:9pt)  -- (0.5*60.0:1.14*5.0pt)   
(1.5*60.0:9pt)  -- (1.5*60.0:1.14*5.0pt)   
(2.5*60.0:9pt)  -- (2.5*60.0:1.14*5.0pt)   
(3.5*60.0:9pt)  -- (3.5*60.0:1.14*5.0pt)   
(4.5*60.0:9pt)  -- (4.5*60.0:1.14*5.0pt)   
(5.5*60.0:9pt)  -- (5.5*60.0:1.14*5.0pt)  ;
 }}}

\pgfdeclareplotmark{m5avb}{%
\node[scale=\mFivescale*\symscale] at (0,0) {\tikz {%
\draw[fill=black,line width=.3pt,even odd rule,rotate=90] 
(0.*60.0:9pt)  -- (0.5*60.0:1.14*4.5pt)   -- 
(1.*60.0:9pt)  -- (1.5*60.0:1.14*4.5pt)   -- 
(2.*60.0:9pt)  -- (2.5*60.0:1.14*4.5pt)   -- 
(3.*60.0:9pt)  -- (3.5*60.0:1.14*4.5pt)   -- 
(4.*60.0:9pt)  -- (4.5*60.0:1.14*4.5pt)   -- 
(5.*60.0:9pt)  -- (5.5*60.0:1.14*4.5pt)   --  cycle;
\draw (0,0) circle (9pt);
%\draw[line width=.35pt,even odd rule,rotate=90] 
%(0.*60.0:11pt)  -- (0.5*60.0:1.14*1.00*6.0pt)   -- 
%(1.*60.0:11pt)  -- (1.5*60.0:1.14*1.00*6.0pt)   -- 
%(2.*60.0:11pt)  -- (2.5*60.0:1.14*1.00*6.0pt)   -- 
%(3.*60.0:11pt)  -- (3.5*60.0:1.14*1.00*6.0pt)   -- 
%(4.*60.0:11pt)  -- (4.5*60.0:1.14*1.00*6.0pt)   -- 
%(5.*60.0:11pt)  -- (5.5*60.0:1.14*1.00*6.0pt)   --  cycle;
\draw[line width=0.6pt,even odd rule,rotate=90] 
(0.5*60.0:10pt)  -- (0.5*60.0:1.14*1.00*6.0pt)   
(1.5*60.0:10pt)  -- (1.5*60.0:1.14*1.00*6.0pt)   
(2.5*60.0:10pt)  -- (2.5*60.0:1.14*1.00*6.0pt)   
(3.5*60.0:10pt)  -- (3.5*60.0:1.14*1.00*6.0pt)   
(4.5*60.0:10pt)  -- (4.5*60.0:1.14*1.00*6.0pt)   
(5.5*60.0:10pt)  -- (5.5*60.0:1.14*1.00*6.0pt)   ;
 }}}



\pgfdeclareplotmark{m5b}{%
\node[scale=\mFivescale*\AtoBscale*\symscale] at (0,0) {\tikz {%
\draw[fill=black,line width=.3pt,even odd rule,rotate=90] 
(0.*60.0:10pt)  -- (0.5*60.0:1.14*5.0pt)   -- 
(1.*60.0:10pt)  -- (1.5*60.0:1.14*5.0pt)   -- 
(2.*60.0:10pt)  -- (2.5*60.0:1.14*5.0pt)   -- 
(3.*60.0:10pt)  -- (3.5*60.0:1.14*5.0pt)   -- 
(4.*60.0:10pt)  -- (4.5*60.0:1.14*5.0pt)   -- 
(5.*60.0:10pt)  -- (5.5*60.0:1.14*5.0pt)   -- cycle
(0,0) circle (2pt);
 }}}


\pgfdeclareplotmark{m5bv}{%
\node[scale=\mFivescale*\AtoBscale*\symscale] at (0,0) {\tikz {%
\draw[fill=black,line width=.3pt,even odd rule,rotate=90] 
(0.*60.0:9pt)  -- (0.5*60.0:1.14*4.5pt)   -- 
(1.*60.0:9pt)  -- (1.5*60.0:1.14*4.5pt)   -- 
(2.*60.0:9pt)  -- (2.5*60.0:1.14*4.5pt)   -- 
(3.*60.0:9pt)  -- (3.5*60.0:1.14*4.5pt)   -- 
(4.*60.0:9pt)  -- (4.5*60.0:1.14*4.5pt)   -- 
(5.*60.0:9pt)  -- (5.5*60.0:1.14*4.5pt)   -- cycle
(0,0) circle (2pt);
\draw (0,0) circle (9pt);
%\draw[line width=.35pt,even odd rule,rotate=90] 
%(0.*60.0:11pt)  -- (0.5*60.0:1.14*1.00*6.0pt)   -- 
%(1.*60.0:11pt)  -- (1.5*60.0:1.14*1.00*6.0pt)   -- 
%(2.*60.0:11pt)  -- (2.5*60.0:1.14*1.00*6.0pt)   -- 
%(3.*60.0:11pt)  -- (3.5*60.0:1.14*1.00*6.0pt)   -- 
%(4.*60.0:11pt)  -- (4.5*60.0:1.14*1.00*6.0pt)   -- 
%(5.*60.0:11pt)  -- (5.5*60.0:1.14*1.00*6.0pt)   --  cycle;
 }}}



\pgfdeclareplotmark{m5bb}{%
\node[scale=\mFivescale*\AtoBscale*\symscale] at (0,0) {\tikz {%
\draw[fill=black,line width=.3pt,even odd rule,rotate=90] 
(0.*60.0:10pt)  -- (0.5*60.0:1.14*5.0pt)   -- 
(1.*60.0:10pt)  -- (1.5*60.0:1.14*5.0pt)   -- 
(2.*60.0:10pt)  -- (2.5*60.0:1.14*5.0pt)   -- 
(3.*60.0:10pt)  -- (3.5*60.0:1.14*5.0pt)   -- 
(4.*60.0:10pt)  -- (4.5*60.0:1.14*5.0pt)   -- 
(5.*60.0:10pt)  -- (5.5*60.0:1.14*5.0pt)   --  cycle
(0,0) circle (2pt);
\draw[line width=0.6pt,even odd rule,rotate=90] 
(0.5*60.0:9pt)  -- (0.5*60.0:1.14*5.0pt)   
(1.5*60.0:9pt)  -- (1.5*60.0:1.14*5.0pt)   
(2.5*60.0:9pt)  -- (2.5*60.0:1.14*5.0pt)   
(3.5*60.0:9pt)  -- (3.5*60.0:1.14*5.0pt)   
(4.5*60.0:9pt)  -- (4.5*60.0:1.14*5.0pt)   
(5.5*60.0:9pt)  -- (5.5*60.0:1.14*5.0pt)  ;
 }}}

\pgfdeclareplotmark{m5bvb}{%
\node[scale=\mFivescale*\AtoBscale*\symscale] at (0,0) {\tikz {%
\draw[fill=black,line width=.3pt,even odd rule,rotate=90] 
(0.*60.0:9pt)  -- (0.5*60.0:1.14*4.5pt)   -- 
(1.*60.0:9pt)  -- (1.5*60.0:1.14*4.5pt)   -- 
(2.*60.0:9pt)  -- (2.5*60.0:1.14*4.5pt)   -- 
(3.*60.0:9pt)  -- (3.5*60.0:1.14*4.5pt)   -- 
(4.*60.0:9pt)  -- (4.5*60.0:1.14*4.5pt)   -- 
(5.*60.0:9pt)  -- (5.5*60.0:1.14*4.5pt)   --  cycle
(0,0) circle (2pt);
\draw (0,0) circle (9pt);
%\draw[line width=.35pt,even odd rule,rotate=90] 
%(0.*60.0:11pt)  -- (0.5*60.0:1.14*1.00*6.0pt)   -- 
%(1.*60.0:11pt)  -- (1.5*60.0:1.14*1.00*6.0pt)   -- 
%(2.*60.0:11pt)  -- (2.5*60.0:1.14*1.00*6.0pt)   -- 
%(3.*60.0:11pt)  -- (3.5*60.0:1.14*1.00*6.0pt)   -- 
%(4.*60.0:11pt)  -- (4.5*60.0:1.14*1.00*6.0pt)   -- 
%(5.*60.0:11pt)  -- (5.5*60.0:1.14*1.00*6.0pt)   -- cycle;
\draw[line width=0.6pt,even odd rule,rotate=90] 
(0.5*60.0:10pt)  -- (0.5*60.0:1.14*1.00*6.0pt)   
(1.5*60.0:10pt)  -- (1.5*60.0:1.14*1.00*6.0pt)   
(2.5*60.0:10pt)  -- (2.5*60.0:1.14*1.00*6.0pt)   
(3.5*60.0:10pt)  -- (3.5*60.0:1.14*1.00*6.0pt)   
(4.5*60.0:10pt)  -- (4.5*60.0:1.14*1.00*6.0pt)   
(5.5*60.0:10pt)  -- (5.5*60.0:1.14*1.00*6.0pt)  ;
 }}}



\pgfdeclareplotmark{m5c}{%
\node[scale=\mFivescale*\AtoCscale*\symscale] at (0,0) {\tikz {%
\draw[fill=black,line width=.3pt,even odd rule,rotate=90] 
(0.*60.0:10pt)  -- (0.5*60.0:1.14*5.0pt)   -- 
(1.*60.0:10pt)  -- (1.5*60.0:1.14*5.0pt)   -- 
(2.*60.0:10pt)  -- (2.5*60.0:1.14*5.0pt)   -- 
(3.*60.0:10pt)  -- (3.5*60.0:1.14*5.0pt)   -- 
(4.*60.0:10pt)  -- (4.5*60.0:1.14*5.0pt)   -- 
(5.*60.0:10pt)  -- (5.5*60.0:1.14*5.0pt)   -- cycle
(0,0) circle (3.25pt);
 }}}


\pgfdeclareplotmark{m5cv}{%
\node[scale=\mFivescale*\AtoCscale*\symscale] at (0,0) {\tikz {%
\draw[fill=black,line width=.3pt,even odd rule,rotate=90] 
(0.*60.0:9pt)  -- (0.5*60.0:1.14*4.5pt)   -- 
(1.*60.0:9pt)  -- (1.5*60.0:1.14*4.5pt)   -- 
(2.*60.0:9pt)  -- (2.5*60.0:1.14*4.5pt)   -- 
(3.*60.0:9pt)  -- (3.5*60.0:1.14*4.5pt)   -- 
(4.*60.0:9pt)  -- (4.5*60.0:1.14*4.5pt)   -- 
(5.*60.0:9pt)  -- (5.5*60.0:1.14*4.5pt)   --  cycle
(0,0) circle (3.25pt);
\draw (0,0) circle (9pt);
%\draw[line width=.35pt,even odd rule,rotate=90] 
%(0.*60.0:11pt)  -- (0.5*60.0:1.14*1.00*6.0pt)   -- 
%(1.*60.0:11pt)  -- (1.5*60.0:1.14*1.00*6.0pt)   -- 
%(2.*60.0:11pt)  -- (2.5*60.0:1.14*1.00*6.0pt)   -- 
%(3.*60.0:11pt)  -- (3.5*60.0:1.14*1.00*6.0pt)   -- 
%(4.*60.0:11pt)  -- (4.5*60.0:1.14*1.00*6.0pt)   -- 
%(5.*60.0:11pt)  -- (5.5*60.0:1.14*1.00*6.0pt)   -- cycle;
 }}}



\pgfdeclareplotmark{m5cb}{%
\node[scale=\mFivescale*\AtoCscale*\symscale] at (0,0) {\tikz {%
\draw[fill=black,line width=.3pt,even odd rule,rotate=90] 
(0.*60.0:10pt)  -- (0.5*60.0:1.14*5.0pt)   -- 
(1.*60.0:10pt)  -- (1.5*60.0:1.14*5.0pt)   -- 
(2.*60.0:10pt)  -- (2.5*60.0:1.14*5.0pt)   -- 
(3.*60.0:10pt)  -- (3.5*60.0:1.14*5.0pt)   -- 
(4.*60.0:10pt)  -- (4.5*60.0:1.14*5.0pt)   -- 
(5.*60.0:10pt)  -- (5.5*60.0:1.14*5.0pt)   -- cycle
(0,0) circle (3.25pt);
\draw[line width=0.6pt,even odd rule,rotate=90] 
(0.5*60.0:9pt)  -- (0.5*60.0:1.14*5.0pt)   
(1.5*60.0:9pt)  -- (1.5*60.0:1.14*5.0pt)   
(2.5*60.0:9pt)  -- (2.5*60.0:1.14*5.0pt)   
(3.5*60.0:9pt)  -- (3.5*60.0:1.14*5.0pt)   
(4.5*60.0:9pt)  -- (4.5*60.0:1.14*5.0pt)   
(5.5*60.0:9pt)  -- (5.5*60.0:1.14*5.0pt)  ;
 }}}

\pgfdeclareplotmark{m5cvb}{%
\node[scale=\mFivescale*\AtoCscale*\symscale] at (0,0) {\tikz {%
\draw[fill=black,line width=.3pt,even odd rule,rotate=90] 
(0.*60.0:9pt)  -- (0.5*60.0:1.14*4.5pt)   -- 
(1.*60.0:9pt)  -- (1.5*60.0:1.14*4.5pt)   -- 
(2.*60.0:9pt)  -- (2.5*60.0:1.14*4.5pt)   -- 
(3.*60.0:9pt)  -- (3.5*60.0:1.14*4.5pt)   -- 
(4.*60.0:9pt)  -- (4.5*60.0:1.14*4.5pt)   -- 
(5.*60.0:9pt)  -- (5.5*60.0:1.14*4.5pt)   --  cycle
(0,0) circle (3.25pt);
\draw (0,0) circle (9pt);
%\draw[line width=.35pt,even odd rule,rotate=90] 
%(0.*60.0:11pt)  -- (0.5*60.0:1.14*1.00*6.0pt)   -- 
%(1.*60.0:11pt)  -- (1.5*60.0:1.14*1.00*6.0pt)   -- 
%(2.*60.0:11pt)  -- (2.5*60.0:1.14*1.00*6.0pt)   -- 
%(3.*60.0:11pt)  -- (3.5*60.0:1.14*1.00*6.0pt)   -- 
%(4.*60.0:11pt)  -- (4.5*60.0:1.14*1.00*6.0pt)   -- 
%(5.*60.0:11pt)  -- (5.5*60.0:1.14*1.00*6.0pt)   -- cycle;
\draw[line width=0.6pt,even odd rule,rotate=90] 
(0.5*60.0:10pt)  -- (0.5*60.0:1.14*1.00*6.0pt)   
(1.5*60.0:10pt)  -- (1.5*60.0:1.14*1.00*6.0pt)   
(2.5*60.0:10pt)  -- (2.5*60.0:1.14*1.00*6.0pt)   
(3.5*60.0:10pt)  -- (3.5*60.0:1.14*1.00*6.0pt)   
(4.5*60.0:10pt)  -- (4.5*60.0:1.14*1.00*6.0pt)   
(5.5*60.0:10pt)  -- (5.5*60.0:1.14*1.00*6.0pt)   ;
 }}}

\pgfdeclareplotmark{m6a}{%
\node[scale=\mSixscale*\symscale] at (0,0) {\tikz {%
\draw[fill,line width=0.2pt,even odd rule,rotate=90] %
(0:10pt) -- (36:3.8pt)    -- 
(72:10pt) -- (108:3.8pt)  -- 
(144:10pt) -- (180:3.8pt) -- 
(216:10pt) -- (252:3.8pt) --
(288:10pt) -- (324:3.8pt) -- cycle
(0pt,0pt) circle (2.5pt);
% The following is needed to center the darn thing.
\draw[opacity=0] (0pt,0pt) circle (13pt) ;
 }}}


\pgfdeclareplotmark{m6av}{%
\node[scale=\mSixscale*\symscale] at (0,0) {\tikz {%
\draw[fill,line width=0.2pt,even odd rule,rotate=90] %
(0:0.9*10pt) -- (36:0.9*3.8pt)    -- 
(72:0.9*10pt) -- (108:0.9*3.8pt)  -- 
(144:0.9*10pt) -- (180:0.9*3.8pt) -- 
(216:0.9*10pt) -- (252:0.9*3.8pt) --
(288:0.9*10pt) -- (324:0.9*3.8pt) -- cycle
(0pt,0pt) circle (2.5pt);
\draw (0,0) circle (10pt);
%\draw[line width=0.6pt,even odd rule,rotate=90] %
%(0:0.9*13pt) --  (36:0.9*6.0pt)    -- 
%(72:0.9*13pt) -- (108:0.9*6.0pt)  -- 
%(144:0.9*13pt) -- (180:0.9*6.0pt) -- 
%(216:0.9*13pt) -- (252:0.9*6.0pt) --
%(288:0.9*13pt) -- (324:0.9*6.0pt) -- cycle ;
% The following is needed to center the darn thing.
\draw[opacity=0] (0pt,0pt) circle (13pt) ;
 }}}

\pgfdeclareplotmark{m6ab}{%
\node[scale=\mSixscale*\symscale] at (0,0) {\tikz {%
\draw[fill,line width=0.2pt,even odd rule,rotate=90] %
(0:10pt) -- (36:3.8pt)    -- 
(72:10pt) -- (108:3.8pt)  -- 
(144:10pt) -- (180:3.8pt) -- 
(216:10pt) -- (252:3.8pt) --
(288:10pt) -- (324:3.8pt) -- cycle
(0pt,0pt) circle (2.5pt);
\draw[line width=0.9pt,even odd rule,rotate=90] %
(36:9pt) -- (36:3.8pt)    
(108:9pt) -- (108:3.8pt)  
(180:9pt) -- (180:3.8pt) 
(252:9pt) -- (252:3.8pt) 
(324:9pt) -- (324:3.8pt) 
(0pt,0pt) circle (2.5pt);
% The following is needed to center the darn thing.
\draw[opacity=0] (0pt,0pt) circle (13pt) ;
 }}}

\pgfdeclareplotmark{m6avb}{%
\node[scale=\mSixscale*\symscale] at (0,0) {\tikz {%
\draw[fill,line width=0.2pt,even odd rule,rotate=90] %
(0:10pt) -- (36:3.8pt)    -- 
(72:10pt) -- (108:3.8pt)  -- 
(144:10pt) -- (180:3.8pt) -- 
(216:10pt) -- (252:3.8pt) --
(288:10pt) -- (324:3.8pt) -- cycle
(0pt,0pt) circle (2.5pt);
\draw (0,0) circle (10pt);
%\draw[line width=0.6pt,even odd rule,rotate=90] %
%(0:0.9*13pt) --  (36:0.9*6.0pt)    -- 
%(72:0.9*13pt) -- (108:0.9*6.0pt)  -- 
%(144:0.9*13pt) -- (180:0.9*6.0pt) -- 
%(216:0.9*13pt) -- (252:0.9*6.0pt) --
%(288:0.9*13pt) -- (324:0.9*6.0pt) -- cycle ;
\draw[line width=0.9pt,even odd rule,rotate=90] %
(36:12pt) -- (36:6.0pt)    
(108:12pt) -- (108:6.0pt)  
(180:12pt) -- (180:6.0pt) 
(252:12pt) -- (252:6.0pt) 
(324:12pt) -- (324:6.0pt) 
(0pt,0pt) circle (2.5pt);
% The following is needed to center the darn thing.
\draw[opacity=0] (0pt,0pt) circle (13pt) ;
 }}}



\pgfdeclareplotmark{m6b}{%
\node[scale=\mSixscale*\AtoBscale*\symscale] at (0,0) {\tikz {%
\draw[fill,line width=0.2pt,even odd rule,rotate=45] %
(0:10pt)   --  (45:5pt) -- 
(90:10pt)  -- (135:5pt) -- 
(180:10pt) -- (225:5pt) --
(270:10pt) -- (315:5pt) -- cycle 
(0pt,0pt) circle (3pt) ;
 }}}


\pgfdeclareplotmark{m6bv}{%
\node[scale=\mSixscale*\AtoBscale*\symscale] at (0,0) {\tikz {%
\draw[fill,line width=0.2pt,even odd rule,rotate=45] %
(0:10pt)   --  (45:5pt) -- 
(90:10pt)  -- (135:5pt) -- 
(180:10pt) -- (225:5pt) --
(270:10pt) -- (315:5pt) -- cycle 
(0pt,0pt) circle (3pt) ;
\draw (0,0) circle (10pt);
%\draw[line width=0.6pt,even odd rule,rotate=45] %
%(0:13pt)   --  (45:7pt) -- 
%(90:13pt)  -- (135:7pt) -- 
%(180:13pt) -- (225:7pt) --
%(270:13pt) -- (315:7pt) -- cycle ;
 }}}

\pgfdeclareplotmark{m6bb}{%
\node[scale=\mSixscale*\AtoBscale*\symscale] at (0,0) {\tikz {%
\draw[fill,line width=0.2pt,even odd rule,rotate=45] %
(0:10pt)   --  (45:5pt) -- 
(90:10pt)  -- (135:5pt) -- 
(180:10pt) -- (225:5pt) --
(270:10pt) -- (315:5pt) -- cycle 
(0pt,0pt) circle (3pt) ;
\draw[line width=1.2pt,even odd rule,rotate=45] %
(45:9pt)   --  (45:5pt) 
(135:9pt)  -- (135:5pt) 
(225:9pt) -- (225:5pt) 
(315:9pt) -- (315:5pt) ;
 }}}

\pgfdeclareplotmark{m6bvb}{%
\node[scale=\mSixscale*\AtoBscale*\symscale] at (0,0) {\tikz {%
\draw[fill,line width=0.2pt,even odd rule,rotate=45] %
(0:10pt)   --  (45:5pt) -- 
(90:10pt)  -- (135:5pt) -- 
(180:10pt) -- (225:5pt) --
(270:10pt) -- (315:5pt) -- cycle 
(0pt,0pt) circle (3pt) ;
\draw (0,0) circle (10pt);
%\draw[line width=0.6pt,even odd rule,rotate=45] %
%(0:13pt)   --  (45:7pt) -- 
%(90:13pt)  -- (135:7pt) -- 
%(180:13pt) -- (225:7pt) --
%(270:13pt) -- (315:7pt) -- cycle ;
\draw[line width=1.2pt,even odd rule,rotate=45] %
(45:11pt)   --  (45:7pt) 
(135:11pt)  -- (135:7pt) 
(225:11pt)  -- (225:7pt) 
(315:11pt)  -- (315:7pt) ;
 }}}


\pgfdeclareplotmark{m6c}{%
\node[scale=\mSixscale*\AtoCscale*\symscale] at (0,0) {\tikz {%
\draw[fill,line width=0.2pt,even odd rule,rotate=90] %
(0pt,0pt) circle (5pt) ;
 }}}


\pgfdeclareplotmark{m6cv}{%
\node[scale=\mSixscale*\AtoCscale*\symscale] at (0,0) {\tikz {%
\draw[fill,line width=0.2pt,even odd rule,rotate=90] %
(0pt,0pt) circle (5pt) ;
\draw[line width=0.6pt,even odd rule,rotate=90] %
(0pt,0pt) circle (7pt) ;
 }}}

\pgfdeclareplotmark{m6cb}{%
\node[scale=\mSixscale*\AtoCscale*\symscale] at (0,0) {\tikz {%
\draw[fill,line width=0.2pt,even odd rule,rotate=90] %
(0pt,0pt) circle (5pt) ;
\draw[line width=2pt,even odd rule] %
(-8pt,0pt) -- (8pt,0pt) ;
 }}}

\pgfdeclareplotmark{m6cvb}{%
\node[scale=\mSixscale*\AtoCscale*\symscale] at (0,0) {\tikz {%
\draw[fill,line width=0.2pt,even odd rule,rotate=90] %
(0pt,0pt) circle (5pt) ;
\draw[line width=0.6pt,even odd rule,rotate=90] %
(0pt,0pt) circle (7pt) ;
\draw[line width=2pt,even odd rule] %
(-9pt,0pt) -- (-7pt,0pt) 
(9pt,0pt) -- (7pt,0pt) ;
 }}}




%%%%%%%%%%%%%%%%%%%%%%%%%%%%%%%%%%%%%%%%%%%%%%%%%%%%%%%%%%%%%%%%%%%%%%%%%%%%%%%%%%%%%%%%%%
%%%%%%%%%%%%%%%%%%%%%%%%%%%%%%%%%%%%%%%%%%%%%%%%%%%%%%%%%%%%%%%%%%%%%%%%%%%%%%%%%%%%%%%%%%
%   Nebulae markers
%%%%%%%%%%%%%%%%%%%%%%%%%%%%%%%%%%%%%%%%%%%%%%%%%%%%%%%%%%%%%%%%%%%%%%%%%%%%%%%%%%%%%%%%%%
%%%%%%%%%%%%%%%%%%%%%%%%%%%%%%%%%%%%%%%%%%%%%%%%%%%%%%%%%%%%%%%%%%%%%%%%%%%%%%%%%%%%%%%%%%


\pgfdeclarepatternformonly{dense crosshatch dots}{\pgfqpoint{-1pt}{-1pt}}{\pgfqpoint{.6pt}{.6pt}}{\pgfqpoint{.8pt}{.8pt}}%
{
    \pgfpathcircle{\pgfqpoint{0pt}{0pt}}{.3pt}
    \pgfpathcircle{\pgfqpoint{1pt}{1pt}}{.3pt}
    \pgfusepath{fill}
}

% Open Cluster
%\pgfdeclareplotmark{OC}{%
%\node at (0,0) {\tikz { 
%\draw[scale=3,draw opacity=0,line width=.2pt,pattern=dense crosshatch dots] %
%(0:1pt) -- (30:0.75pt)  -- (60:1pt) -- (90:0.75pt)  -- (120:1pt) --  (150:0.75pt)  -- (180:1pt) --
%(210:0.75pt) -- (240:1pt)  -- (270:0.75pt) -- (300:1pt) -- (330:0.75pt)  -- (0:1pt) ;
%};};   
%}

\pgfdeclareplotmark{OC}{%
\node at (0,0) {\tikz {%
%\clip (0:1pt) -- (30:0.75pt)  -- (60:1pt) -- (90:0.75pt)  -- (120:1pt) --  (150:0.75pt)  -- (180:1pt) --
%(210:0.75pt) -- (240:1pt)  -- (270:0.75pt) -- (300:1pt) -- (330:0.75pt)  -- (0:1pt) ;
\filldraw[scale=3,draw opacity=1,line width=.2pt]
(0,0) circle (.1pt)
(0:.4pt) circle (.1pt) (60:.4pt) circle (.1pt) (120:.4pt) circle (.1pt)
(180:.4pt) circle (.1pt) (240:.4pt) circle (.1pt) (300:.4pt) circle (.1pt)
%
(30:.75pt) circle (.1pt) (90:.75pt) circle (.1pt) (150:.75pt) circle (.1pt)
(210:.75pt) circle (.1pt) (270:.75pt) circle (.1pt) (330:.75pt) circle (.1pt)
;
}}}

% Globular Cluster
\pgfdeclareplotmark{GC}{%
\node at (0,0) {\tikz { 
\filldraw[scale=3,draw opacity=1,line width=.2pt]
(0,0) circle (.1pt)
(0:.4pt) circle (.1pt) (60:.4pt) circle (.1pt) (120:.4pt) circle (.1pt)
(180:.4pt) circle (.1pt) (240:.4pt) circle (.1pt) (300:.4pt) circle (.1pt)
%
(30:.75pt) circle (.1pt) (90:.75pt) circle (.1pt) (150:.75pt) circle (.1pt)
(210:.75pt) circle (.1pt) (270:.75pt) circle (.1pt) (330:.75pt) circle (.1pt)
;
\draw[scale=3,line width=.3pt] %
circle (1pt) ;
}}}

% Emission Nebula
\pgfdeclareplotmark{EN}{%
\node[opacity=0] at (0,0) {\tikz { 
\clip (0,0) circle (2.5pt);
\draw[opacity=1,line width=.3pt] %
(-4pt,-1pt) -- (4pt,7pt) 
(-4pt,-2.5pt) -- (4pt,5.5pt)  
(-4pt,-4pt) -- (4pt,4pt) 
(-4pt,-5.5pt) -- (4pt,2.5pt) 
(-4pt,-7pt) -- (4pt,1pt) 
;
}}}

% Nebuluous Cluster
\pgfdeclareplotmark{CN}{%
\node[opacity=0] at (0,0) {\tikz { 
\clip (0,0) circle (2.5pt);
\draw[opacity=1,line width=.3pt] %
(-4pt,-1pt) -- (4pt,7pt) 
(-4pt,-2.5pt) -- (4pt,5.5pt)  
(-4pt,-4pt) -- (4pt,4pt) 
(-4pt,-5.5pt) -- (4pt,2.5pt) 
(-4pt,-7pt) -- (4pt,1pt) ;
}}  ;
\node at (0,0) {\tikz { 
\filldraw[scale=3,draw opacity=1,line width=.2pt]
(0,0) circle (.1pt)
(0:.4pt) circle (.1pt) (60:.4pt) circle (.1pt) (120:.4pt) circle (.1pt)
(180:.4pt) circle (.1pt) (240:.4pt) circle (.1pt) (300:.4pt) circle (.1pt)
%
(30:.75pt) circle (.1pt) (90:.75pt) circle (.1pt) (150:.75pt) circle (.1pt)
(210:.75pt) circle (.1pt) (270:.75pt) circle (.1pt) (330:.75pt) circle (.1pt)
;
}}}


% Planetary Nebula
\pgfdeclareplotmark{PN}{%
\node at (0,0) {\tikz { 
\draw[line width=.3pt] %
circle (2.7pt);
\clip (0,0) circle (2.7pt);
\draw[opacity=1,line width=.3pt] %
(-4pt,-1pt) -- (4pt,7pt) 
(-4pt,-2.5pt) -- (4pt,5.5pt)  
(-4pt,-4pt) -- (4pt,4pt) 
(-4pt,-5.5pt) -- (4pt,2.5pt) 
(-4pt,-7pt) -- (4pt,1pt) 
;
\draw[line width=.3pt,fill=white] %
circle (1.2pt) ;
}}}

% Galaxies
\pgfdeclareplotmark{GAL}{%
\node [rotate=-15] at (0,0) {\tikz {
\draw (0,0) ellipse (4pt and 1.5pt);
\clip (0,0) ellipse (4pt and 1.5pt);
\draw[opacity=1,line width=.3pt] %
(-4pt,-1pt) -- (4pt,7pt) 
(-4pt,-2.5pt) -- (4pt,5.5pt)  
(-4pt,-4pt) -- (4pt,4pt) 
(-4pt,-5.5pt) -- (4pt,2.5pt) 
(-4pt,-7pt) -- (4pt,1pt) 
;
}}}  ;


 


% 
% Colors
\definecolor{Burgundy}{rgb}{.75,.25,.25}
\definecolor{IndianRed4}{HTML}{8B3A3A}
\definecolor{IndianRed3}{HTML}{CD5555}
\definecolor{IndianRed2}{HTML}{EE6363}
\definecolor{IndianRed1}{HTML}{FF6A6A}
\definecolor{GoldenRod1}{HTML}{FFC125}
\definecolor{Orchid}{rgb}{0.85, 0.44, 0.84}
\definecolor{Mauve}{HTML}{B784A7}
\colorlet{cConstellation}{IndianRed4}
%\colorlet{cConstellation}{burgundy}
\definecolor{ForestGreen}{rgb}{0.13, 0.55, 0.13}

\colorlet{cFrame}{black}
\colorlet{cStars}{black}
\colorlet{cAxes}{black}


\colorlet{cProper}{IndianRed2}
\colorlet{cBayer}{IndianRed3}
\colorlet{cFlamsteed}{IndianRed4}

\colorlet{cTransient}{ForestGreen}

\colorlet{cNGC}{Orchid}
\colorlet{cMessier}{Orchid}

%\colorlet{cNGC}{Mauve}
%\colorlet{cMessier}{Mauve}

%\colorlet{cNGC}{IndianRed4}
%\colorlet{cMessier}{IndianRed4}

%\colorlet{cNGC}{green!80!black}
%\colorlet{cMessier}{blue}

\colorlet{cEcliptic}{GoldenRod1!80!black}
\colorlet{cEquator}{gray}

\colorlet{cGrid}{black}

\definecolor{DeepSkyBlue}{HTML}{00BFFF}
% Milky Way  color
\colorlet{cMW}{DeepSkyBlue}
\colorlet{cMW0}{white}
\colorlet{cMW1}{DeepSkyBlue!10}
\colorlet{cMW2}{DeepSkyBlue!14}
\colorlet{cMW3}{DeepSkyBlue!18}
\colorlet{cMW4}{DeepSkyBlue!22}
\colorlet{cMW5}{DeepSkyBlue!26}

%\colorlet{cMW1}{DeepSkyBlue!15}
%\colorlet{cMW2}{DeepSkyBlue!30}
%\colorlet{cMW3}{DeepSkyBlue!45}
%\colorlet{cMW4}{DeepSkyBlue!60}
%\colorlet{cMW5}{DeepSkyBlue!75}

%\colorlet{cMW6}{DeepSkyBlue!40}
%\colorlet{cMW7}{DeepSkyBlue!50}
%\colorlet{cMW8}{DeepSkyBlue!60}
\colorlet{cMW9}{DeepSkyBlue!70}

\def\symscale {1.5}

\def\AtoBscale {0.9}
\def\AtoCscale {0.8}

\def\mOnescale   {0.48}
\def\mTwoscale   {0.38}
\def\mThreescale {0.3}
\def\mFourscale  {0.23}
\def\mFivescale  {0.15}
\def\mSixscale   {0.1}


% Styles:
\pgfplotsset{MW0/.style={mark=,color=cMW0,fill=cMW0,opacity=1}}
\pgfplotsset{MW1/.style={mark=,color=cMW1,fill=cMW1,opacity=1}}
\pgfplotsset{MW2/.style={mark=,color=cMW2,fill=cMW2,opacity=1}}
\pgfplotsset{MW3/.style={mark=,color=cMW3,fill=cMW3,opacity=1}}
\pgfplotsset{MW4/.style={mark=,color=cMW4,fill=cMW4,opacity=1}}
\pgfplotsset{MW5/.style={mark=,color=cMW5,fill=cMW5,opacity=1}}

%\pgfplotsset{MW6/.style={mark=,color=cMW6,opacity=1}}
%\pgfplotsset{MW7/.style={mark=,color=cMW7,opacity=1}}
%\pgfplotsset{MW8/.style={mark=,color=cMW8,opacity=1}}
%\pgfplotsset{MW9/.style={mark=,color=cMW9,opacity=1}}
\pgfplotsset{MWX/.style={mark=,opacity=1}}
\pgfplotsset{MWR/.style={mark=,color=red,opacity=1}}


% Coordinate grid overlay
\pgfplotsset{coordinategrid/.style={grid=major,major grid style={thin,color=cGrid}}}
\tikzset{coordinategrid/.style={thin,color=cGrid}}
% Ecliptics
\tikzset{ecliptics-full/.style={color=cEcliptic,line width=.4pt,mark=,fill=cEcliptic,opacity=1}}
\tikzset{ecliptics-empty/.style={color=cEcliptic,line width=.4pt,mark=,fill=,fill opacity=0}}
\tikzset{ecliptics-label/.style={color=black,font=\scriptsize}}
% Galactic Equator
\tikzset{MWE-full/.style={color=cMW,line width=.4pt,mark=,fill=cMW,opacity=1}}
\tikzset{MWE-empty/.style={color=cMW,line width=.4pt,mark=,fill=,fill opacity=0}}
\tikzset{MWE-label/.style={color=black,font=\scriptsize}}
% Equator
\tikzset{Equator-full/.style={color=cEquator,line width=.4pt,mark=,fill=cEquator,opacity=1}}
\tikzset{Equator-empty/.style={color=cEquator,line width=.4pt,mark=,fill=,fill opacity=0}}
\tikzset{Equator-label/.style={color=black,font=\scriptsize}}
% Transients
\tikzset{transient-label/.style={color=cTransient,font=\bfseries\scriptsize}}
\tikzset{transient/.style={color=cTransient,line width=1pt,mark=,opacity=1}}

% To enable consistent input scripting:
\newcommand\omicron{o}

% Scale of panels:
\newlength{\tendegree}
\setlength{\tendegree}{3cm}
\newlength{\onedegree}
\setlength{\onedegree}{0.1\tendegree}

% General style definitions
\pgfkeys{/pgf/number format/.cd,fixed,precision=4}


% For RA tickmark labeling
\newcommand{\fh}{$^{\rm h}$}
\newcommand{\fm}{$^{\rm m}$}



% Style for constellation boundaries and names:
\pgfplotsset{constellation/.style={thin,mark=,color=cConstellation,dashed}}
\tikzset{constellation-label/.style={color=cConstellation!80,font=\bfseries}}
\tikzset{constellation-boundary/.style={thin,mark=,color=cConstellation,dashed}}


\tikzset{designation-label/.style={color=cProper,font=\bfseries\scriptsize}}
 
% Style for Star markers and names
\tikzset{stars/.style={color=black}}

\tikzset{Proper/.style={color=cProper,font=\bfseries\footnotesize},pin edge={draw opacity=0}}
\tikzset{Bayer/.style={color=cBayer,font=\bfseries\scriptsize},pin edge={draw opacity=0}}
\tikzset{Flaamsted/.style={color=cFlamsteed,font=\bfseries\tiny},pin edge={draw opacity=0}}
% Intersting objects label
\tikzset{interest/.style={color=cProper}}
\tikzset{interest-label/.style={color=cProper,font=\bfseries\tiny}}

% Style for Nebulae markers and names
\tikzset{NGC/.style={color=cNGC,pattern color=cNGC}}
\tikzset{NGC-label/.style={color=cNGC,font=\bfseries\tiny},pin edge={draw opacity=0}}

\tikzset{Caldwell/.style={color=cMessier,pattern color=cMessier}}
\tikzset{Caldwell-label/.style={color=cMessier,font=\bfseries\scriptsize},pin edge={draw opacity=0}}

\tikzset{Messier/.style={color=cMessier,pattern color=cMessier}}
\tikzset{Messier-label/.style={color=cMessier,font=\bfseries\scriptsize},pin edge={draw opacity=0}}
% An extra one for the Messier galaxies in the Virgo-group 
\tikzset{Messier-label-crowded/.style={pin edge={draw opacity=1,color=cMessier,thin},color=cMessier,font=\bfseries\footnotesize}}



\pgfplotsset{tick label style={font=\normalsize}, label style={font=\large}, legend style={font=\large}
  ,every axis/.append style={scale only axis},scaled ticks=false
  % A3paper has size 297 × 420mm
  ,width=11\tendegree,height=8\tendegree
  ,/pgf/number format/set thousands separator={\,}
%
  ,x dir=reverse  ,ytick={-90,-80,...,90},minor y tick num=0
  ,yticklabels={$-90^\circ$,$-80^\circ$,$-70^\circ$,$-60^\circ$,$-50^\circ$,$-40^\circ$,$-30^\circ$,$-20^\circ$,$-10^\circ$,$0^\circ$,
    $+10^\circ$,$+20^\circ$,$+30^\circ$,$+40^\circ$,$+50^\circ$,$+60^\circ$,$+70^\circ$,$+80^\circ$,$+90^\circ$}
%  
}

% RA tickmark labels in hours
\pgfplotsset{RA_in_hours/.style={,xtick={0,10,...,390},minor x tick num=0
      ,xticklabels={0\fh{\small 00}\fm,0\fh{\small 40}\fm,1\fh{\small 20}\fm,2\fh{\small
      00}\fm,2\fh{\small 40}\fm,3\fh{\small 20}\fm,4\fh{\small 00}\fm,4\fh{\small
      40}\fm,5\fh{\small 20}\fm,6\fh{\small 00}\fm,6\fh{\small 40}\fm,7\fh{\small
      20}\fm,8\fh{\small 00}\fm,8\fh{\small 40}\fm,9\fh{\small 20}\fm,10\fh{\small
      00}\fm,10\fh{\small 40}\fm,11\fh{\small 20}\fm,12\fh{\small 00}\fm,12\fh{\small
      40}\fm,13\fh{\small 20}\fm,14\fh{\small 00}\fm,14\fh{\small 40}\fm,15\fh{\small
      20}\fm,16\fh{\small 00}\fm,16\fh{\small 40}\fm,17\fh{\small 20}\fm,18\fh{\small
      00}\fm,18\fh{\small 40}\fm,19\fh{\small 20}\fm,20\fh{\small 00}\fm,20\fh{\small
      40}\fm,21\fh{\small 20}\fm,22\fh{\small 00}\fm,22\fh{\small 40}\fm,23\fh{\small
      20}\fm,0\fh{\small 00}\fm,0\fh{\small 40}\fm,1\fh{\small 20}\fm}}}

% RA tickmark labels in degree
\pgfplotsset{RA_in_deg/.style={,xtick={0,10,...,390},minor x tick num=0,x dir=reverse,
      tick label style={font=\footnotesize},xticklabels={$0^\circ$,$10^\circ$,$20^\circ$,$30^\circ$, 
      $40^\circ$,$50^\circ$,$60^\circ$,$70^\circ$,$80^\circ$,$90^\circ$,$100^\circ$,$110^\circ$,$120^\circ$,
      $130^\circ$,$140^\circ$,$150^\circ$,$160^\circ$,$170^\circ$,$180^\circ$,$190^\circ$,$200^\circ$,
      $210^\circ$,$220^\circ$,$230^\circ$,$240^\circ$,$250^\circ$,$260^\circ$,$270^\circ$,$280^\circ$,
      $290^\circ$,$300^\circ$,$310^\circ$,$320^\circ$,$330^\circ$,$340^\circ$,$350^\circ$,
      $0^\circ$,$10^\circ$,$20^\circ$}  
}} 


\center

\tikzsetnextfilename{TabulaIV}
%
% Coordinate limit of tab to plot
\pgfplotsset{xmin=90,xmax=200,ymin=-40,ymax=40}

\begin{tikzpicture}
%
% Empty plot to establish anchor points
\begin{axis}[name=base,axis lines=none]\end{axis}

% The Milky Way first, as it should not obstruct coordinate grid 


\begin{axis}[name=milkyWay,axis lines=none]

\addplot[MW1] table[x index=0,y index=1] {./MW/IV_MWbase.dat}  -- cycle ;

\addplot[MW2] table[x index=0,y index=1] {./MW/IV_ol2_101.dat}  -- cycle ;
\addplot[MW2] table[x index=0,y index=1] {./MW/IV_ol2_102.dat}  -- cycle ;
\addplot[MW2] table[x index=0,y index=1] {./MW/IV_ol2_106.dat}  -- cycle ;
\addplot[MW2] table[x index=0,y index=1] {./MW/IV_ol2_107.dat}  -- cycle ;
\addplot[MW2] table[x index=0,y index=1] {./MW/IV_ol2_110.dat}  -- cycle ;
\addplot[MW2] table[x index=0,y index=1] {./MW/IV_ol2_15.dat}  -- cycle ;
\addplot[MW2] table[x index=0,y index=1] {./MW/IV_ol2_17.dat}  -- cycle ;
\addplot[MW2] table[x index=0,y index=1] {./MW/IV_ol2_21.dat}  -- cycle ;
\addplot[MW2] table[x index=0,y index=1] {./MW/IV_ol2_23.dat}  -- cycle ;
\addplot[MW2] table[x index=0,y index=1] {./MW/IV_ol2_26.dat}  -- cycle ;
\addplot[MW2] table[x index=0,y index=1] {./MW/IV_ol2_33.dat}  -- cycle ;
\addplot[MW2] table[x index=0,y index=1] {./MW/IV_ol2_34.dat}  -- cycle ;
\addplot[MW2] table[x index=0,y index=1] {./MW/IV_ol2_48.dat}  -- cycle ;
\addplot[MW2] table[x index=0,y index=1] {./MW/IV_ol2_50.dat}  -- cycle ;
\addplot[MW2] table[x index=0,y index=1] {./MW/IV_ol2_52.dat}  -- cycle ;
\addplot[MW2] table[x index=0,y index=1] {./MW/IV_ol2_55.dat}  -- cycle ;
\addplot[MW2] table[x index=0,y index=1] {./MW/IV_ol2_59.dat}  -- cycle ;
\addplot[MW2] table[x index=0,y index=1] {./MW/IV_ol2_5.dat}  -- cycle ;
\addplot[MW2] table[x index=0,y index=1] {./MW/IV_ol2_61.dat}  -- cycle ;
\addplot[MW2] table[x index=0,y index=1] {./MW/IV_ol2_6.dat}  -- cycle ;
\addplot[MW2] table[x index=0,y index=1] {./MW/IV_ol2_71.dat}  -- cycle ;
\addplot[MW2] table[x index=0,y index=1] {./MW/IV_ol2_83.dat}  -- cycle ;
\addplot[MW2] table[x index=0,y index=1] {./MW/IV_ol2_84.dat}  -- cycle ;
\addplot[MW2] table[x index=0,y index=1] {./MW/IV_ol2_85.dat}  -- cycle ;
\addplot[MW2] table[x index=0,y index=1] {./MW/IV_ol2_90.dat}  -- cycle ;
\addplot[MW2] table[x index=0,y index=1] {./MW/IV_ol2_9.dat}  -- cycle ;

\addplot[MW2] table[x index=0,y index=1] {./MW/IV_ol2_37.dat}  -- cycle ;
\addplot[MW2] table[x index=0,y index=1] {./MW/IV_ol2_103.dat}  -- cycle ;
\addplot[MW2] table[x index=0,y index=1] {./MW/IV_ol2_86.dat}  -- cycle ;

\addplot[MW1] table[x index=0,y index=1] {./MW/IV_ol2_92.dat}  -- cycle ;
\addplot[MW1] table[x index=0,y index=1] {./MW/IV_ol2_62.dat}  -- cycle ;

\addplot[MW3] table[x index=0,y index=1] {./MW/IV_ol3_13.dat}  -- cycle ;
\addplot[MW3] table[x index=0,y index=1] {./MW/IV_ol3_14.dat}  -- cycle ;
\addplot[MW3] table[x index=0,y index=1] {./MW/IV_ol3_15.dat}  -- cycle ;
\addplot[MW3] table[x index=0,y index=1] {./MW/IV_ol3_22.dat}  -- cycle ;
\addplot[MW3] table[x index=0,y index=1] {./MW/IV_ol3_24.dat}  -- cycle ;
\addplot[MW3] table[x index=0,y index=1] {./MW/IV_ol3_29.dat}  -- cycle ;
\addplot[MW3] table[x index=0,y index=1] {./MW/IV_ol3_30.dat}  -- cycle ;
\addplot[MW3] table[x index=0,y index=1] {./MW/IV_ol3_35.dat}  -- cycle ;
\addplot[MW3] table[x index=0,y index=1] {./MW/IV_ol3_38.dat}  -- cycle ;
\addplot[MW3] table[x index=0,y index=1] {./MW/IV_ol3_40.dat}  -- cycle ;
\addplot[MW3] table[x index=0,y index=1] {./MW/IV_ol3_5.dat}  -- cycle ;

\end{axis}


% The Galactic coordinate system
%
% Some are commented out because they are surrounded by already plotted borders
% The x-coordinate is in fractional hours, so must be times 15
%

\begin{axis}[name=constellations,at=(base.center),anchor=center,axis lines=none]

\draw[MWE-full] (axis cs:2.66308E+02,-2.88819E+01) -- (axis cs:2.66899E+02,-2.80275E+01) -- (axis cs:2.67093E+02,-2.81309E+01) -- (axis cs:2.66503E+02,-2.89861E+01) -- cycle ; 
\draw[MWE-full] (axis cs:2.67481E+02,-2.71706E+01) -- (axis cs:2.68054E+02,-2.63113E+01) -- (axis cs:2.68246E+02,-2.64131E+01) -- (axis cs:2.67674E+02,-2.72732E+01) -- cycle ; 
\draw[MWE-full] (axis cs:2.68618E+02,-2.54498E+01) -- (axis cs:2.69174E+02,-2.45861E+01) -- (axis cs:2.69365E+02,-2.46865E+01) -- (axis cs:2.68809E+02,-2.55508E+01) -- cycle ; 
\draw[MWE-full] (axis cs:2.69723E+02,-2.37204E+01) -- (axis cs:2.70264E+02,-2.28528E+01) -- (axis cs:2.70453E+02,-2.29518E+01) -- (axis cs:2.69912E+02,-2.38201E+01) -- cycle ; 
\draw[MWE-full] (axis cs:2.70799E+02,-2.19834E+01) -- (axis cs:2.71327E+02,-2.11122E+01) -- (axis cs:2.71514E+02,-2.12100E+01) -- (axis cs:2.70986E+02,-2.20817E+01) -- cycle ; 
\draw[MWE-full] (axis cs:2.71848E+02,-2.02394E+01) -- (axis cs:2.72364E+02,-1.93650E+01) -- (axis cs:2.72550E+02,-1.94617E+01) -- (axis cs:2.72035E+02,-2.03366E+01) -- cycle ; 
\draw[MWE-full] (axis cs:2.72874E+02,-1.84892E+01) -- (axis cs:2.73379E+02,-1.76121E+01) -- (axis cs:2.73564E+02,-1.77078E+01) -- (axis cs:2.73059E+02,-1.85854E+01) -- cycle ; 
\draw[MWE-full] (axis cs:2.73879E+02,-1.67336E+01) -- (axis cs:2.74375E+02,-1.58540E+01) -- (axis cs:2.74558E+02,-1.59488E+01) -- (axis cs:2.74063E+02,-1.68289E+01) -- cycle ; 
\draw[MWE-full] (axis cs:2.74866E+02,-1.49732E+01) -- (axis cs:2.75353E+02,-1.40913E+01) -- (axis cs:2.75535E+02,-1.41854E+01) -- (axis cs:2.75049E+02,-1.50676E+01) -- cycle ; 
\draw[MWE-full] (axis cs:2.75837E+02,-1.32085E+01) -- (axis cs:2.76316E+02,-1.23248E+01) -- (axis cs:2.76498E+02,-1.24181E+01) -- (axis cs:2.76018E+02,-1.33022E+01) -- cycle ; 
\draw[MWE-full] (axis cs:2.76793E+02,-1.14402E+01) -- (axis cs:2.77267E+02,-1.05548E+01) -- (axis cs:2.77447E+02,-1.06476E+01) -- (axis cs:2.76974E+02,-1.15333E+01) -- cycle ; 
\draw[MWE-full] (axis cs:2.77738E+02,-9.66873E+00) -- (axis cs:2.78206E+02,-8.78200E+00) -- (axis cs:2.78385E+02,-8.87430E+00) -- (axis cs:2.77917E+02,-9.76126E+00) -- cycle ; 
\draw[MWE-full] (axis cs:2.78672E+02,-7.89468E+00) -- (axis cs:2.79136E+02,-7.00682E+00) -- (axis cs:2.79315E+02,-7.09872E+00) -- (axis cs:2.78851E+02,-7.98676E+00) -- cycle ; 
\draw[MWE-full] (axis cs:2.79598E+02,-6.11851E+00) -- (axis cs:2.80059E+02,-5.22978E+00) -- (axis cs:2.80237E+02,-5.32137E+00) -- (axis cs:2.79777E+02,-6.21024E+00) -- cycle ; 
\draw[MWE-full] (axis cs:2.80518E+02,-4.34070E+00) -- (axis cs:2.80977E+02,-3.45134E+00) -- (axis cs:2.81155E+02,-3.54271E+00) -- (axis cs:2.80697E+02,-4.43217E+00) -- cycle ; 
\draw[MWE-full] (axis cs:2.81434E+02,-2.56174E+00) -- (axis cs:2.81891E+02,-1.67196E+00) -- (axis cs:2.82069E+02,-1.76320E+00) -- (axis cs:2.81612E+02,-2.65303E+00) -- cycle ; 
\draw[MWE-full] (axis cs:2.82347E+02,-7.82062E-01) -- (axis cs:2.82917E+02,1.07899E-01) -- (axis cs:2.82981E+02,1.66967E-02) -- (axis cs:2.82525E+02,-8.73274E-01) -- cycle ; 
\draw[MWE-full] (axis cs:2.83259E+02,9.97867E-01) -- (axis cs:2.83715E+02,1.88779E+00) -- (axis cs:2.83893E+02,1.79654E+00) -- (axis cs:2.83437E+02,9.06652E-01) -- cycle ; 
\draw[MWE-full] (axis cs:2.84172E+02,2.77760E+00) -- (axis cs:2.84629E+02,3.66725E+00) -- (axis cs:2.84808E+02,3.57586E+00) -- (axis cs:2.84350E+02,2.68629E+00) -- cycle ; 
\draw[MWE-full] (axis cs:2.85088E+02,4.55668E+00) -- (axis cs:2.85547E+02,5.44583E+00) -- (axis cs:2.85725E+02,5.35422E+00) -- (axis cs:2.85266E+02,4.46519E+00) -- cycle ; 
\draw[MWE-full] (axis cs:2.86008E+02,6.33465E+00) -- (axis cs:2.86470E+02,7.22308E+00) -- (axis cs:2.86649E+02,7.13115E+00) -- (axis cs:2.86186E+02,6.24290E+00) -- cycle ; 
\draw[MWE-full] (axis cs:2.86934E+02,8.11105E+00) -- (axis cs:2.87400E+02,8.99851E+00) -- (axis cs:2.87580E+02,8.90618E+00) -- (axis cs:2.87113E+02,8.01893E+00) -- cycle ; 
\draw[MWE-full] (axis cs:2.87868E+02,9.88539E+00) -- (axis cs:2.88339E+02,1.07716E+01) -- (axis cs:2.88519E+02,1.06788E+01) -- (axis cs:2.88048E+02,9.79283E+00) -- cycle ; 
\draw[MWE-full] (axis cs:2.88813E+02,1.16572E+01) -- (axis cs:2.89289E+02,1.25420E+01) -- (axis cs:2.89470E+02,1.24486E+01) -- (axis cs:2.88993E+02,1.15641E+01) -- cycle ; 
\draw[MWE-full] (axis cs:2.89769E+02,1.34259E+01) -- (axis cs:2.90253E+02,1.43090E+01) -- (axis cs:2.90435E+02,1.42149E+01) -- (axis cs:2.89951E+02,1.33322E+01) -- cycle ; 
\draw[MWE-full] (axis cs:2.90740E+02,1.51910E+01) -- (axis cs:2.91231E+02,1.60721E+01) -- (axis cs:2.91414E+02,1.59772E+01) -- (axis cs:2.90922E+02,1.50966E+01) -- cycle ; 
\draw[MWE-full] (axis cs:2.91727E+02,1.69520E+01) -- (axis cs:2.92227E+02,1.78307E+01) -- (axis cs:2.92411E+02,1.77350E+01) -- (axis cs:2.91910E+02,1.68567E+01) -- cycle ; 
\draw[MWE-full] (axis cs:2.92732E+02,1.87082E+01) -- (axis cs:2.93242E+02,1.95843E+01) -- (axis cs:2.93428E+02,1.94875E+01) -- (axis cs:2.92917E+02,1.86119E+01) -- cycle ; 
\draw[MWE-full] (axis cs:2.93758E+02,2.04590E+01) -- (axis cs:2.94280E+02,2.13321E+01) -- (axis cs:2.94467E+02,2.12342E+01) -- (axis cs:2.93944E+02,2.03617E+01) -- cycle ; 
\draw[MWE-full] (axis cs:2.94808E+02,2.22037E+01) -- (axis cs:2.95342E+02,2.30735E+01) -- (axis cs:2.95531E+02,2.29744E+01) -- (axis cs:2.94996E+02,2.21052E+01) -- cycle ; 
\draw[MWE-full] (axis cs:2.95884E+02,2.39415E+01) -- (axis cs:2.96433E+02,2.48076E+01) -- (axis cs:2.96623E+02,2.47072E+01) -- (axis cs:2.96073E+02,2.38417E+01) -- cycle ; 
\draw[MWE-full] (axis cs:2.96989E+02,2.56717E+01) -- (axis cs:2.97554E+02,2.65337E+01) -- (axis cs:2.97746E+02,2.64318E+01) -- (axis cs:2.97180E+02,2.55705E+01) -- cycle ; 
\draw[MWE-full] (axis cs:2.98127E+02,2.73934E+01) -- (axis cs:2.98709E+02,2.82508E+01) -- (axis cs:2.98903E+02,2.81473E+01) -- (axis cs:2.98320E+02,2.72907E+01) -- cycle ; 
\draw[MWE-full] (axis cs:2.99300E+02,2.91056E+01) -- (axis cs:2.99901E+02,2.99579E+01) -- (axis cs:3.00098E+02,2.98527E+01) -- (axis cs:2.99495E+02,2.90013E+01) -- cycle ; 
\draw[MWE-full] (axis cs:3.00513E+02,3.08074E+01) -- (axis cs:3.01135E+02,3.16540E+01) -- (axis cs:3.01334E+02,3.15469E+01) -- (axis cs:3.00710E+02,3.07013E+01) -- cycle ; 
\draw[MWE-full] (axis cs:3.01769E+02,3.24975E+01) -- (axis cs:3.02415E+02,3.33378E+01) -- (axis cs:3.02616E+02,3.32287E+01) -- (axis cs:3.01969E+02,3.23894E+01) -- cycle ; 
\draw[MWE-full] (axis cs:3.03073E+02,3.41747E+01) -- (axis cs:3.03745E+02,3.50081E+01) -- (axis cs:3.03948E+02,3.48968E+01) -- (axis cs:3.03275E+02,3.40646E+01) -- cycle ; 
\draw[MWE-full] (axis cs:3.04430E+02,3.58377E+01) -- (axis cs:3.05130E+02,3.66633E+01) -- (axis cs:3.05335E+02,3.65497E+01) -- (axis cs:3.04634E+02,3.57252E+01) -- cycle ; 
\draw[MWE-full] (axis cs:3.05844E+02,3.74848E+01) -- (axis cs:3.06575E+02,3.83019E+01) -- (axis cs:3.06782E+02,3.81857E+01) -- (axis cs:3.06051E+02,3.73699E+01) -- cycle ; 
\draw[MWE-full] (axis cs:3.07322E+02,3.91143E+01) -- (axis cs:3.08087E+02,3.99220E+01) -- (axis cs:3.08296E+02,3.98031E+01) -- (axis cs:3.07531E+02,3.89969E+01) -- cycle ; 
\draw[MWE-full] (axis cs:3.08869E+02,4.07244E+01) -- (axis cs:3.09671E+02,4.15215E+01) -- (axis cs:3.09883E+02,4.13998E+01) -- (axis cs:3.09080E+02,4.06042E+01) -- cycle ; 
\draw[MWE-full] (axis cs:3.10492E+02,4.23130E+01) -- (axis cs:3.11335E+02,4.30984E+01) -- (axis cs:3.11548E+02,4.29736E+01) -- (axis cs:3.10705E+02,4.21897E+01) -- cycle ; 
\draw[MWE-full] (axis cs:3.12199E+02,4.38775E+01) -- (axis cs:3.13086E+02,4.46499E+01) -- (axis cs:3.13301E+02,4.45218E+01) -- (axis cs:3.12414E+02,4.37511E+01) -- cycle ; 
\draw[MWE-full] (axis cs:3.13996E+02,4.54153E+01) -- (axis cs:3.14932E+02,4.61733E+01) -- (axis cs:3.15149E+02,4.60418E+01) -- (axis cs:3.14213E+02,4.52855E+01) -- cycle ; 
\draw[MWE-full] (axis cs:3.15893E+02,4.69235E+01) -- (axis cs:3.16882E+02,4.76654E+01) -- (axis cs:3.17100E+02,4.75301E+01) -- (axis cs:3.16111E+02,4.67901E+01) -- cycle ; 
\draw[MWE-full] (axis cs:3.17899E+02,4.83986E+01) -- (axis cs:3.18945E+02,4.91226E+01) -- (axis cs:3.19165E+02,4.89834E+01) -- (axis cs:3.18118E+02,4.82614E+01) -- cycle ; 
\draw[MWE-full] (axis cs:3.20023E+02,4.98368E+01) -- (axis cs:3.21132E+02,5.05409E+01) -- (axis cs:3.21351E+02,5.03976E+01) -- (axis cs:3.20242E+02,4.96956E+01) -- cycle ; 
\draw[MWE-full] (axis cs:3.22274E+02,5.12341E+01) -- (axis cs:3.23452E+02,5.19158E+01) -- (axis cs:3.23670E+02,5.17682E+01) -- (axis cs:3.22493E+02,5.10886E+01) -- cycle ; 
\draw[MWE-full] (axis cs:3.24665E+02,5.25855E+01) -- (axis cs:3.25915E+02,5.32424E+01) -- (axis cs:3.26132E+02,5.30902E+01) -- (axis cs:3.24882E+02,5.24356E+01) -- cycle ; 
\draw[MWE-full] (axis cs:3.27204E+02,5.38858E+01) -- (axis cs:3.28533E+02,5.45151E+01) -- (axis cs:3.28746E+02,5.43582E+01) -- (axis cs:3.27419E+02,5.37314E+01) -- cycle ; 
\draw[MWE-full] (axis cs:3.29903E+02,5.51293E+01) -- (axis cs:3.31315E+02,5.57277E+01) -- (axis cs:3.31523E+02,5.55661E+01) -- (axis cs:3.30114E+02,5.49701E+01) -- cycle ; 
\draw[MWE-full] (axis cs:3.32770E+02,5.63095E+01) -- (axis cs:3.34269E+02,5.68737E+01) -- (axis cs:3.34472E+02,5.67072E+01) -- (axis cs:3.32976E+02,5.61454E+01) -- cycle ; 
\draw[MWE-full] (axis cs:3.35814E+02,5.74194E+01) -- (axis cs:3.37405E+02,5.79457E+01) -- (axis cs:3.37598E+02,5.77743E+01) -- (axis cs:3.36012E+02,5.72505E+01) -- cycle ; 
\draw[MWE-full] (axis cs:3.39042E+02,5.84516E+01) -- (axis cs:3.40725E+02,5.89360E+01) -- (axis cs:3.40908E+02,5.87597E+01) -- (axis cs:3.39230E+02,5.82777E+01) -- cycle ; 
\draw[MWE-full] (axis cs:3.42456E+02,5.93980E+01) -- (axis cs:3.44233E+02,5.98365E+01) -- (axis cs:3.44402E+02,5.96555E+01) -- (axis cs:3.42632E+02,5.92193E+01) -- cycle ; 
\draw[MWE-full] (axis cs:3.46056E+02,6.02505E+01) -- (axis cs:3.47924E+02,6.06388E+01) -- (axis cs:3.48077E+02,6.04533E+01) -- (axis cs:3.46217E+02,6.00672E+01) -- cycle ; 
\draw[MWE-full] (axis cs:3.49837E+02,6.10005E+01) -- (axis cs:3.51791E+02,6.13346E+01) -- (axis cs:3.51924E+02,6.11451E+01) -- (axis cs:3.49979E+02,6.08130E+01) -- cycle ; 
\draw[MWE-full] (axis cs:3.53786E+02,6.16401E+01) -- (axis cs:3.55819E+02,6.19160E+01) -- (axis cs:3.55929E+02,6.17229E+01) -- (axis cs:3.53908E+02,6.14487E+01) -- cycle ; 
\draw[MWE-full] (axis cs:3.57886E+02,6.21614E+01) -- (axis cs:3.59986E+02,6.23756E+01) -- (axis cs:7.05566E-02,6.21796E+01) -- (axis cs:3.57984E+02,6.19668E+01) -- cycle ; 
\draw[MWE-full] (axis cs:2.11273E+00,6.25578E+01) -- (axis cs:4.26373E+00,6.27073E+01) -- (axis cs:4.32138E+00,6.25091E+01) -- (axis cs:2.18430E+00,6.23606E+01) -- cycle ; 
\draw[MWE-full] (axis cs:6.43417E+00,6.28237E+01) -- (axis cs:8.61932E+00,6.29064E+01) -- (axis cs:8.64801E+00,6.27068E+01) -- (axis cs:6.47748E+00,6.26247E+01) -- cycle ; 
\draw[MWE-full] (axis cs:1.08143E+01,6.29552E+01) -- (axis cs:1.30140E+01,6.29699E+01) -- (axis cs:1.30130E+01,6.27699E+01) -- (axis cs:1.08282E+01,6.27553E+01) -- cycle ; 
\draw[MWE-full] (axis cs:1.52134E+01,6.29504E+01) -- (axis cs:1.74073E+01,6.28969E+01) -- (axis cs:1.73765E+01,6.26974E+01) -- (axis cs:1.51974E+01,6.27506E+01) -- cycle ; 
\draw[MWE-full] (axis cs:1.95907E+01,6.28094E+01) -- (axis cs:2.17586E+01,6.26884E+01) -- (axis cs:2.16990E+01,6.24903E+01) -- (axis cs:1.95453E+01,6.26105E+01) -- cycle ; 
\draw[MWE-full] (axis cs:2.39066E+01,6.25342E+01) -- (axis cs:2.60301E+01,6.23475E+01) -- (axis cs:2.59432E+01,6.21517E+01) -- (axis cs:2.38331E+01,6.23372E+01) -- cycle ; 
\draw[MWE-full] (axis cs:2.81251E+01,6.21289E+01) -- (axis cs:3.01880E+01,6.18792E+01) -- (axis cs:3.00762E+01,6.16863E+01) -- (axis cs:2.80255E+01,6.19345E+01) -- cycle ; 
\draw[MWE-full] (axis cs:3.22156E+01,6.15991E+01) -- (axis cs:3.42049E+01,6.12895E+01) -- (axis cs:3.40709E+01,6.11003E+01) -- (axis cs:3.20923E+01,6.14079E+01) -- cycle ; 
\draw[MWE-full] (axis cs:3.61537E+01,6.09515E+01) -- (axis cs:3.80599E+01,6.05860E+01) -- (axis cs:3.79064E+01,6.04008E+01) -- (axis cs:3.60096E+01,6.07642E+01) -- cycle ; 
\draw[MWE-full] (axis cs:3.99219E+01,6.01940E+01) -- (axis cs:4.17386E+01,5.97766E+01) -- (axis cs:4.15687E+01,5.95959E+01) -- (axis cs:3.97599E+01,6.00110E+01) -- cycle ; 
\draw[MWE-full] (axis cs:4.35091E+01,5.93347E+01) -- (axis cs:4.52331E+01,5.88695E+01) -- (axis cs:4.50497E+01,5.86936E+01) -- (axis cs:4.33322E+01,5.91564E+01) -- cycle ; 
\draw[MWE-full] (axis cs:4.69102E+01,5.83820E+01) -- (axis cs:4.85407E+01,5.78732E+01) -- (axis cs:4.83464E+01,5.77022E+01) -- (axis cs:4.67210E+01,5.82085E+01) -- cycle ; 
\draw[MWE-full] (axis cs:5.01248E+01,5.73442E+01) -- (axis cs:5.16632E+01,5.67958E+01) -- (axis cs:5.14604E+01,5.66296E+01) -- (axis cs:4.99260E+01,5.71755E+01) -- cycle ; 
\draw[MWE-full] (axis cs:5.31565E+01,5.62291E+01) -- (axis cs:5.46057E+01,5.56449E+01) -- (axis cs:5.43965E+01,5.54837E+01) -- (axis cs:5.29503E+01,5.60654E+01) -- cycle ; 
\draw[MWE-full] (axis cs:5.60117E+01,5.50442E+01) -- (axis cs:5.73757E+01,5.44279E+01) -- (axis cs:5.71620E+01,5.42714E+01) -- (axis cs:5.58001E+01,5.48854E+01) -- cycle ; 
\draw[MWE-full] (axis cs:5.86988E+01,5.37966E+01) -- (axis cs:5.99822E+01,5.31512E+01) -- (axis cs:5.97655E+01,5.29994E+01) -- (axis cs:5.84834E+01,5.36425E+01) -- cycle ; 
\draw[MWE-full] (axis cs:6.12273E+01,5.24925E+01) -- (axis cs:6.24353E+01,5.18211E+01) -- (axis cs:6.22168E+01,5.16738E+01) -- (axis cs:6.10096E+01,5.23429E+01) -- cycle ; 
\draw[MWE-full] (axis cs:6.36076E+01,5.11377E+01) -- (axis cs:6.47454E+01,5.04429E+01) -- (axis cs:6.45262E+01,5.02999E+01) -- (axis cs:6.33886E+01,5.09926E+01) -- cycle ; 
\draw[MWE-full] (axis cs:6.58501E+01,4.97374E+01) -- (axis cs:6.69230E+01,4.90218E+01) -- (axis cs:6.67038E+01,4.88829E+01) -- (axis cs:6.56308E+01,4.95965E+01) -- cycle ; 
\draw[MWE-full] (axis cs:6.79652E+01,4.82965E+01) -- (axis cs:6.89781E+01,4.75620E+01) -- (axis cs:6.87597E+01,4.74270E+01) -- (axis cs:6.77464E+01,4.81596E+01) -- cycle ; 
\draw[MWE-full] (axis cs:6.99629E+01,4.68189E+01) -- (axis cs:7.09207E+01,4.60676E+01) -- (axis cs:7.07035E+01,4.59363E+01) -- (axis cs:6.97450E+01,4.66858E+01) -- cycle ; 
\draw[MWE-full] (axis cs:7.18527E+01,4.53086E+01) -- (axis cs:7.27599E+01,4.45422E+01) -- (axis cs:7.25443E+01,4.44144E+01) -- (axis cs:7.16362E+01,4.51791E+01) -- cycle ; 
\draw[MWE-full] (axis cs:7.36436E+01,4.37688E+01) -- (axis cs:7.45045E+01,4.29888E+01) -- (axis cs:7.42908E+01,4.28642E+01) -- (axis cs:7.34289E+01,4.36426E+01) -- cycle ; 
\draw[MWE-full] (axis cs:7.53439E+01,4.22025E+01) -- (axis cs:7.61626E+01,4.14103E+01) -- (axis cs:7.59510E+01,4.12888E+01) -- (axis cs:7.51312E+01,4.20795E+01) -- cycle ; 
\draw[MWE-full] (axis cs:7.69615E+01,4.06124E+01) -- (axis cs:7.77415E+01,3.98092E+01) -- (axis cs:7.75321E+01,3.96906E+01) -- (axis cs:7.67510E+01,4.04924E+01) -- cycle ; 
\draw[MWE-full] (axis cs:7.85035E+01,3.90009E+01) -- (axis cs:7.92483E+01,3.81877E+01) -- (axis cs:7.90411E+01,3.80718E+01) -- (axis cs:7.82952E+01,3.88836E+01) -- cycle ; 
\draw[MWE-full] (axis cs:7.99766E+01,3.73700E+01) -- (axis cs:8.06892E+01,3.65480E+01) -- (axis cs:8.04843E+01,3.64345E+01) -- (axis cs:7.97705E+01,3.72554E+01) -- cycle ; 
\draw[MWE-full] (axis cs:8.13867E+01,3.57218E+01) -- (axis cs:8.20700E+01,3.48916E+01) -- (axis cs:8.18674E+01,3.47805E+01) -- (axis cs:8.11830E+01,3.56095E+01) -- cycle ; 
\draw[MWE-full] (axis cs:8.27396E+01,3.40578E+01) -- (axis cs:8.33961E+01,3.32204E+01) -- (axis cs:8.31957E+01,3.31114E+01) -- (axis cs:8.25381E+01,3.39478E+01) -- cycle ; 
\draw[MWE-full] (axis cs:8.40402E+01,3.23796E+01) -- (axis cs:8.46724E+01,3.15356E+01) -- (axis cs:8.44742E+01,3.14287E+01) -- (axis cs:8.38409E+01,3.22717E+01) -- cycle ; 
\draw[MWE-full] (axis cs:8.52933E+01,3.06886E+01) -- (axis cs:8.59034E+01,2.98387E+01) -- (axis cs:8.57074E+01,2.97337E+01) -- (axis cs:8.50962E+01,3.05826E+01) -- cycle ; 
\draw[MWE-full] (axis cs:8.65033E+01,2.89861E+01) -- (axis cs:8.70933E+01,2.81309E+01) -- (axis cs:8.68993E+01,2.80275E+01) -- (axis cs:8.63082E+01,2.88819E+01) -- cycle ; 
\draw[MWE-full] (axis cs:8.76740E+01,2.72732E+01) -- (axis cs:8.82459E+01,2.64131E+01) -- (axis cs:8.80537E+01,2.63113E+01) -- (axis cs:8.74810E+01,2.71706E+01) -- cycle ; 
\draw[MWE-full] (axis cs:8.88093E+01,2.55508E+01) -- (axis cs:8.93646E+01,2.46865E+01) -- (axis cs:8.91743E+01,2.45861E+01) -- (axis cs:8.86180E+01,2.54498E+01) -- cycle ; 
\draw[MWE-full] (axis cs:8.99124E+01,2.38201E+01) -- (axis cs:9.04529E+01,2.29518E+01) -- (axis cs:9.02642E+01,2.28528E+01) -- (axis cs:8.97229E+01,2.37204E+01) -- cycle ; 
\draw[MWE-full] (axis cs:9.09865E+01,2.20817E+01) -- (axis cs:9.15135E+01,2.12100E+01) -- (axis cs:9.13265E+01,2.11122E+01) -- (axis cs:9.07986E+01,2.19834E+01) -- cycle ; 
\draw[MWE-full] (axis cs:9.20345E+01,2.03366E+01) -- (axis cs:9.25496E+01,1.94617E+01) -- (axis cs:9.23640E+01,1.93650E+01) -- (axis cs:9.18481E+01,2.02394E+01) -- cycle ; 
\draw[MWE-full] (axis cs:9.30592E+01,1.85854E+01) -- (axis cs:9.35636E+01,1.77078E+01) -- (axis cs:9.33793E+01,1.76121E+01) -- (axis cs:9.28742E+01,1.84892E+01) -- cycle ; 
\draw[MWE-full] (axis cs:9.40631E+01,1.68289E+01) -- (axis cs:9.45580E+01,1.59488E+01) -- (axis cs:9.43750E+01,1.58540E+01) -- (axis cs:9.38795E+01,1.67336E+01) -- cycle ; 
\draw[MWE-full] (axis cs:9.50487E+01,1.50676E+01) -- (axis cs:9.55353E+01,1.41854E+01) -- (axis cs:9.53533E+01,1.40913E+01) -- (axis cs:9.48661E+01,1.49732E+01) -- cycle ; 
\draw[MWE-full] (axis cs:9.60182E+01,1.33022E+01) -- (axis cs:9.64975E+01,1.24181E+01) -- (axis cs:9.63165E+01,1.23248E+01) -- (axis cs:9.58366E+01,1.32085E+01) -- cycle ; 
\draw[MWE-full] (axis cs:9.69737E+01,1.15333E+01) -- (axis cs:9.74469E+01,1.06476E+01) -- (axis cs:9.72666E+01,1.05548E+01) -- (axis cs:9.67931E+01,1.14402E+01) -- cycle ; 
\draw[MWE-full] (axis cs:9.79174E+01,9.76126E+00) -- (axis cs:9.83854E+01,8.87430E+00) -- (axis cs:9.82058E+01,8.78200E+00) -- (axis cs:9.77375E+01,9.66874E+00) -- cycle ; 
\draw[MWE-full] (axis cs:9.88512E+01,7.98676E+00) -- (axis cs:9.93149E+01,7.09872E+00) -- (axis cs:9.91360E+01,7.00683E+00) -- (axis cs:9.86719E+01,7.89468E+00) -- cycle ; 
\draw[MWE-full] (axis cs:9.97769E+01,6.21024E+00) -- (axis cs:1.00237E+02,5.32137E+00) -- (axis cs:1.00059E+02,5.22978E+00) -- (axis cs:9.95982E+01,6.11851E+00) -- cycle ; 
\draw[MWE-full] (axis cs:1.00697E+02,4.43217E+00) -- (axis cs:1.01155E+02,3.54271E+00) -- (axis cs:1.00977E+02,3.45134E+00) -- (axis cs:1.00518E+02,4.34070E+00) -- cycle ; 
\draw[MWE-full] (axis cs:1.01612E+02,2.65303E+00) -- (axis cs:1.02069E+02,1.76320E+00) -- (axis cs:1.01891E+02,1.67196E+00) -- (axis cs:1.01434E+02,2.56174E+00) -- cycle ; 
\draw[MWE-full] (axis cs:1.02525E+02,8.73274E-01) -- (axis cs:1.02981E+02,-1.66966E-02) -- (axis cs:1.02917E+02,-1.07899E-01) -- (axis cs:1.02347E+02,7.82062E-01) -- cycle ; 
\draw[MWE-full] (axis cs:1.03437E+02,-9.06652E-01) -- (axis cs:1.03893E+02,-1.79654E+00) -- (axis cs:1.03715E+02,-1.88779E+00) -- (axis cs:1.03259E+02,-9.97867E-01) -- cycle ; 
\draw[MWE-full] (axis cs:1.04350E+02,-2.68629E+00) -- (axis cs:1.04808E+02,-3.57586E+00) -- (axis cs:1.04629E+02,-3.66725E+00) -- (axis cs:1.04172E+02,-2.77760E+00) -- cycle ; 
\draw[MWE-full] (axis cs:1.05266E+02,-4.46519E+00) -- (axis cs:1.05726E+02,-5.35422E+00) -- (axis cs:1.05547E+02,-5.44583E+00) -- (axis cs:1.05088E+02,-4.55668E+00) -- cycle ; 
\draw[MWE-full] (axis cs:1.06186E+02,-6.24290E+00) -- (axis cs:1.06649E+02,-7.13115E+00) -- (axis cs:1.06470E+02,-7.22308E+00) -- (axis cs:1.06008E+02,-6.33465E+00) -- cycle ; 
\draw[MWE-full] (axis cs:1.07113E+02,-8.01893E+00) -- (axis cs:1.07580E+02,-8.90618E+00) -- (axis cs:1.07400E+02,-8.99851E+00) -- (axis cs:1.06934E+02,-8.11105E+00) -- cycle ; 
\draw[MWE-full] (axis cs:1.08048E+02,-9.79283E+00) -- (axis cs:1.08520E+02,-1.06788E+01) -- (axis cs:1.08339E+02,-1.07716E+01) -- (axis cs:1.07868E+02,-9.88539E+00) -- cycle ; 
\draw[MWE-full] (axis cs:1.08993E+02,-1.15641E+01) -- (axis cs:1.09471E+02,-1.24486E+01) -- (axis cs:1.09289E+02,-1.25420E+01) -- (axis cs:1.08813E+02,-1.16572E+01) -- cycle ; 
\draw[MWE-full] (axis cs:1.09951E+02,-1.33322E+01) -- (axis cs:1.10435E+02,-1.42149E+01) -- (axis cs:1.10253E+02,-1.43090E+01) -- (axis cs:1.09769E+02,-1.34259E+01) -- cycle ; 
\draw[MWE-full] (axis cs:1.10922E+02,-1.50966E+01) -- (axis cs:1.11414E+02,-1.59772E+01) -- (axis cs:1.11231E+02,-1.60721E+01) -- (axis cs:1.10740E+02,-1.51910E+01) -- cycle ; 
\draw[MWE-full] (axis cs:1.11910E+02,-1.68567E+01) -- (axis cs:1.12411E+02,-1.77350E+01) -- (axis cs:1.12227E+02,-1.78307E+01) -- (axis cs:1.11727E+02,-1.69520E+01) -- cycle ; 
\draw[MWE-full] (axis cs:1.12917E+02,-1.86119E+01) -- (axis cs:1.13428E+02,-1.94875E+01) -- (axis cs:1.13242E+02,-1.95843E+01) -- (axis cs:1.12732E+02,-1.87082E+01) -- cycle ; 
\draw[MWE-full] (axis cs:1.13945E+02,-2.03617E+01) -- (axis cs:1.14467E+02,-2.12342E+01) -- (axis cs:1.14280E+02,-2.13321E+01) -- (axis cs:1.13758E+02,-2.04590E+01) -- cycle ; 
\draw[MWE-full] (axis cs:1.14996E+02,-2.21052E+01) -- (axis cs:1.15531E+02,-2.29744E+01) -- (axis cs:1.15342E+02,-2.30735E+01) -- (axis cs:1.14808E+02,-2.22037E+01) -- cycle ; 
\draw[MWE-full] (axis cs:1.16073E+02,-2.38417E+01) -- (axis cs:1.16623E+02,-2.47072E+01) -- (axis cs:1.16433E+02,-2.48076E+01) -- (axis cs:1.15884E+02,-2.39415E+01) -- cycle ; 
\draw[MWE-full] (axis cs:1.17180E+02,-2.55705E+01) -- (axis cs:1.17746E+02,-2.64318E+01) -- (axis cs:1.17554E+02,-2.65337E+01) -- (axis cs:1.16989E+02,-2.56717E+01) -- cycle ; 
\draw[MWE-full] (axis cs:1.18320E+02,-2.72907E+01) -- (axis cs:1.18903E+02,-2.81473E+01) -- (axis cs:1.18709E+02,-2.82508E+01) -- (axis cs:1.18127E+02,-2.73934E+01) -- cycle ; 
\draw[MWE-full] (axis cs:1.19495E+02,-2.90013E+01) -- (axis cs:1.20098E+02,-2.98527E+01) -- (axis cs:1.19901E+02,-2.99579E+01) -- (axis cs:1.19300E+02,-2.91056E+01) -- cycle ; 
\draw[MWE-full] (axis cs:1.20710E+02,-3.07013E+01) -- (axis cs:1.21334E+02,-3.15469E+01) -- (axis cs:1.21135E+02,-3.16540E+01) -- (axis cs:1.20513E+02,-3.08074E+01) -- cycle ; 
\draw[MWE-full] (axis cs:1.21969E+02,-3.23894E+01) -- (axis cs:1.22616E+02,-3.32287E+01) -- (axis cs:1.22415E+02,-3.33378E+01) -- (axis cs:1.21769E+02,-3.24975E+01) -- cycle ; 
\draw[MWE-full] (axis cs:1.23275E+02,-3.40646E+01) -- (axis cs:1.23948E+02,-3.48968E+01) -- (axis cs:1.23745E+02,-3.50081E+01) -- (axis cs:1.23073E+02,-3.41747E+01) -- cycle ; 
\draw[MWE-full] (axis cs:1.24634E+02,-3.57252E+01) -- (axis cs:1.25335E+02,-3.65497E+01) -- (axis cs:1.25130E+02,-3.66633E+01) -- (axis cs:1.24430E+02,-3.58377E+01) -- cycle ; 
\draw[MWE-full] (axis cs:1.26051E+02,-3.73699E+01) -- (axis cs:1.26782E+02,-3.81857E+01) -- (axis cs:1.26575E+02,-3.83019E+01) -- (axis cs:1.25844E+02,-3.74848E+01) -- cycle ; 
\draw[MWE-full] (axis cs:1.27531E+02,-3.89969E+01) -- (axis cs:1.28296E+02,-3.98031E+01) -- (axis cs:1.28087E+02,-3.99220E+01) -- (axis cs:1.27322E+02,-3.91143E+01) -- cycle ; 
\draw[MWE-full] (axis cs:1.29080E+02,-4.06042E+01) -- (axis cs:1.29883E+02,-4.13998E+01) -- (axis cs:1.29671E+02,-4.15215E+01) -- (axis cs:1.28869E+02,-4.07244E+01) -- cycle ; 
\draw[MWE-full] (axis cs:1.30705E+02,-4.21897E+01) -- (axis cs:1.31549E+02,-4.29736E+01) -- (axis cs:1.31335E+02,-4.30984E+01) -- (axis cs:1.30492E+02,-4.23130E+01) -- cycle ; 
\draw[MWE-full] (axis cs:1.32414E+02,-4.37511E+01) -- (axis cs:1.33301E+02,-4.45218E+01) -- (axis cs:1.33086E+02,-4.46499E+01) -- (axis cs:1.32199E+02,-4.38775E+01) -- cycle ; 
\draw[MWE-full] (axis cs:1.34213E+02,-4.52855E+01) -- (axis cs:1.35149E+02,-4.60418E+01) -- (axis cs:1.34932E+02,-4.61733E+01) -- (axis cs:1.33996E+02,-4.54153E+01) -- cycle ; 
\draw[MWE-full] (axis cs:1.36111E+02,-4.67901E+01) -- (axis cs:1.37100E+02,-4.75301E+01) -- (axis cs:1.36882E+02,-4.76654E+01) -- (axis cs:1.35893E+02,-4.69235E+01) -- cycle ; 
\draw[MWE-full] (axis cs:1.38118E+02,-4.82614E+01) -- (axis cs:1.39165E+02,-4.89834E+01) -- (axis cs:1.38945E+02,-4.91226E+01) -- (axis cs:1.37899E+02,-4.83986E+01) -- cycle ; 
\draw[MWE-full] (axis cs:1.40242E+02,-4.96956E+01) -- (axis cs:1.41351E+02,-5.03976E+01) -- (axis cs:1.41132E+02,-5.05409E+01) -- (axis cs:1.40023E+02,-4.98368E+01) -- cycle ; 
\draw[MWE-full] (axis cs:1.42493E+02,-5.10886E+01) -- (axis cs:1.43670E+02,-5.17682E+01) -- (axis cs:1.43452E+02,-5.19158E+01) -- (axis cs:1.42274E+02,-5.12341E+01) -- cycle ; 
\draw[MWE-full] (axis cs:1.44882E+02,-5.24356E+01) -- (axis cs:1.46132E+02,-5.30902E+01) -- (axis cs:1.45915E+02,-5.32424E+01) -- (axis cs:1.44665E+02,-5.25855E+01) -- cycle ; 
\draw[MWE-full] (axis cs:1.47419E+02,-5.37314E+01) -- (axis cs:1.48746E+02,-5.43582E+01) -- (axis cs:1.48533E+02,-5.45151E+01) -- (axis cs:1.47204E+02,-5.38858E+01) -- cycle ; 
\draw[MWE-full] (axis cs:1.50114E+02,-5.49701E+01) -- (axis cs:1.51523E+02,-5.55661E+01) -- (axis cs:1.51315E+02,-5.57277E+01) -- (axis cs:1.49903E+02,-5.51293E+01) -- cycle ; 
\draw[MWE-full] (axis cs:1.52976E+02,-5.61454E+01) -- (axis cs:1.54472E+02,-5.67072E+01) -- (axis cs:1.54269E+02,-5.68737E+01) -- (axis cs:1.52770E+02,-5.63095E+01) -- cycle ; 
\draw[MWE-full] (axis cs:1.56012E+02,-5.72505E+01) -- (axis cs:1.57598E+02,-5.77743E+01) -- (axis cs:1.57405E+02,-5.79457E+01) -- (axis cs:1.55814E+02,-5.74194E+01) -- cycle ; 
\draw[MWE-full] (axis cs:1.59230E+02,-5.82777E+01) -- (axis cs:1.60908E+02,-5.87597E+01) -- (axis cs:1.60725E+02,-5.89360E+01) -- (axis cs:1.59042E+02,-5.84516E+01) -- cycle ; 
\draw[MWE-full] (axis cs:1.62632E+02,-5.92193E+01) -- (axis cs:1.64402E+02,-5.96555E+01) -- (axis cs:1.64233E+02,-5.98365E+01) -- (axis cs:1.62456E+02,-5.93980E+01) -- cycle ; 
\draw[MWE-full] (axis cs:1.66217E+02,-6.00672E+01) -- (axis cs:1.68077E+02,-6.04533E+01) -- (axis cs:1.67924E+02,-6.06388E+01) -- (axis cs:1.66056E+02,-6.02505E+01) -- cycle ; 
\draw[MWE-full] (axis cs:1.69979E+02,-6.08130E+01) -- (axis cs:1.71924E+02,-6.11451E+01) -- (axis cs:1.71791E+02,-6.13346E+01) -- (axis cs:1.69837E+02,-6.10005E+01) -- cycle ; 
\draw[MWE-full] (axis cs:1.73908E+02,-6.14487E+01) -- (axis cs:1.75929E+02,-6.17229E+01) -- (axis cs:1.75819E+02,-6.19160E+01) -- (axis cs:1.73786E+02,-6.16401E+01) -- cycle ; 
\draw[MWE-full] (axis cs:1.77984E+02,-6.19668E+01) -- (axis cs:1.80071E+02,-6.21796E+01) -- (axis cs:1.79986E+02,-6.23756E+01) -- (axis cs:1.77886E+02,-6.21614E+01) -- cycle ; 
\draw[MWE-full] (axis cs:1.82184E+02,-6.23606E+01) -- (axis cs:1.84321E+02,-6.25091E+01) -- (axis cs:1.84264E+02,-6.27073E+01) -- (axis cs:1.82113E+02,-6.25578E+01) -- cycle ; 
\draw[MWE-full] (axis cs:1.86478E+02,-6.26247E+01) -- (axis cs:1.88648E+02,-6.27068E+01) -- (axis cs:1.88619E+02,-6.29064E+01) -- (axis cs:1.86434E+02,-6.28237E+01) -- cycle ; 
\draw[MWE-full] (axis cs:1.90828E+02,-6.27553E+01) -- (axis cs:1.93013E+02,-6.27699E+01) -- (axis cs:1.93014E+02,-6.29699E+01) -- (axis cs:1.90814E+02,-6.29552E+01) -- cycle ; 
\draw[MWE-full] (axis cs:1.95197E+02,-6.27506E+01) -- (axis cs:1.97377E+02,-6.26974E+01) -- (axis cs:1.97407E+02,-6.28969E+01) -- (axis cs:1.95213E+02,-6.29504E+01) -- cycle ; 
\draw[MWE-full] (axis cs:1.99545E+02,-6.26105E+01) -- (axis cs:2.01699E+02,-6.24903E+01) -- (axis cs:2.01759E+02,-6.26884E+01) -- (axis cs:1.99591E+02,-6.28094E+01) -- cycle ; 
\draw[MWE-full] (axis cs:2.03833E+02,-6.23372E+01) -- (axis cs:2.05943E+02,-6.21517E+01) -- (axis cs:2.06030E+02,-6.23475E+01) -- (axis cs:2.03907E+02,-6.25342E+01) -- cycle ; 
\draw[MWE-full] (axis cs:2.08026E+02,-6.19345E+01) -- (axis cs:2.10076E+02,-6.16863E+01) -- (axis cs:2.10188E+02,-6.18792E+01) -- (axis cs:2.08125E+02,-6.21289E+01) -- cycle ; 
\draw[MWE-full] (axis cs:2.12092E+02,-6.14079E+01) -- (axis cs:2.14071E+02,-6.11003E+01) -- (axis cs:2.14205E+02,-6.12895E+01) -- (axis cs:2.12216E+02,-6.15991E+01) -- cycle ; 
\draw[MWE-full] (axis cs:2.16010E+02,-6.07642E+01) -- (axis cs:2.17906E+02,-6.04008E+01) -- (axis cs:2.18060E+02,-6.05860E+01) -- (axis cs:2.16154E+02,-6.09515E+01) -- cycle ; 
\draw[MWE-full] (axis cs:2.19760E+02,-6.00110E+01) -- (axis cs:2.21569E+02,-5.95959E+01) -- (axis cs:2.21739E+02,-5.97766E+01) -- (axis cs:2.19922E+02,-6.01940E+01) -- cycle ; 
\draw[MWE-full] (axis cs:2.23332E+02,-5.91564E+01) -- (axis cs:2.25050E+02,-5.86936E+01) -- (axis cs:2.25233E+02,-5.88695E+01) -- (axis cs:2.23509E+02,-5.93347E+01) -- cycle ; 
\draw[MWE-full] (axis cs:2.26721E+02,-5.82085E+01) -- (axis cs:2.28346E+02,-5.77022E+01) -- (axis cs:2.28541E+02,-5.78732E+01) -- (axis cs:2.26910E+02,-5.83820E+01) -- cycle ; 
\draw[MWE-full] (axis cs:2.29926E+02,-5.71755E+01) -- (axis cs:2.31460E+02,-5.66296E+01) -- (axis cs:2.31663E+02,-5.67958E+01) -- (axis cs:2.30125E+02,-5.73442E+01) -- cycle ; 
\draw[MWE-full] (axis cs:2.32950E+02,-5.60654E+01) -- (axis cs:2.34397E+02,-5.54837E+01) -- (axis cs:2.34606E+02,-5.56449E+01) -- (axis cs:2.33157E+02,-5.62291E+01) -- cycle ; 
\draw[MWE-full] (axis cs:2.35800E+02,-5.48854E+01) -- (axis cs:2.37162E+02,-5.42714E+01) -- (axis cs:2.37376E+02,-5.44279E+01) -- (axis cs:2.36012E+02,-5.50442E+01) -- cycle ; 
\draw[MWE-full] (axis cs:2.38483E+02,-5.36425E+01) -- (axis cs:2.39766E+02,-5.29994E+01) -- (axis cs:2.39982E+02,-5.31512E+01) -- (axis cs:2.38699E+02,-5.37966E+01) -- cycle ; 
\draw[MWE-full] (axis cs:2.41010E+02,-5.23429E+01) -- (axis cs:2.42217E+02,-5.16738E+01) -- (axis cs:2.42435E+02,-5.18211E+01) -- (axis cs:2.41227E+02,-5.24925E+01) -- cycle ; 
\draw[MWE-full] (axis cs:2.43389E+02,-5.09926E+01) -- (axis cs:2.44526E+02,-5.02999E+01) -- (axis cs:2.44745E+02,-5.04429E+01) -- (axis cs:2.43608E+02,-5.11377E+01) -- cycle ; 
\draw[MWE-full] (axis cs:2.45631E+02,-4.95965E+01) -- (axis cs:2.46704E+02,-4.88829E+01) -- (axis cs:2.46923E+02,-4.90218E+01) -- (axis cs:2.45850E+02,-4.97374E+01) -- cycle ; 
\draw[MWE-full] (axis cs:2.47746E+02,-4.81596E+01) -- (axis cs:2.48760E+02,-4.74270E+01) -- (axis cs:2.48978E+02,-4.75620E+01) -- (axis cs:2.47965E+02,-4.82965E+01) -- cycle ; 
\draw[MWE-full] (axis cs:2.49745E+02,-4.66858E+01) -- (axis cs:2.50704E+02,-4.59363E+01) -- (axis cs:2.50921E+02,-4.60676E+01) -- (axis cs:2.49963E+02,-4.68189E+01) -- cycle ; 
\draw[MWE-full] (axis cs:2.51636E+02,-4.51791E+01) -- (axis cs:2.52544E+02,-4.44144E+01) -- (axis cs:2.52760E+02,-4.45422E+01) -- (axis cs:2.51853E+02,-4.53086E+01) -- cycle ; 
\draw[MWE-full] (axis cs:2.53429E+02,-4.36426E+01) -- (axis cs:2.54291E+02,-4.28642E+01) -- (axis cs:2.54505E+02,-4.29888E+01) -- (axis cs:2.53644E+02,-4.37688E+01) -- cycle ; 
\draw[MWE-full] (axis cs:2.55131E+02,-4.20795E+01) -- (axis cs:2.55951E+02,-4.12888E+01) -- (axis cs:2.56163E+02,-4.14103E+01) -- (axis cs:2.55344E+02,-4.22025E+01) -- cycle ; 
\draw[MWE-full] (axis cs:2.56751E+02,-4.04924E+01) -- (axis cs:2.57532E+02,-3.96906E+01) -- (axis cs:2.57742E+02,-3.98092E+01) -- (axis cs:2.56962E+02,-4.06124E+01) -- cycle ; 
\draw[MWE-full] (axis cs:2.58295E+02,-3.88836E+01) -- (axis cs:2.59041E+02,-3.80718E+01) -- (axis cs:2.59248E+02,-3.81877E+01) -- (axis cs:2.58504E+02,-3.90009E+01) -- cycle ; 
\draw[MWE-full] (axis cs:2.59771E+02,-3.72554E+01) -- (axis cs:2.60484E+02,-3.64345E+01) -- (axis cs:2.60689E+02,-3.65480E+01) -- (axis cs:2.59977E+02,-3.73700E+01) -- cycle ; 
\draw[MWE-full] (axis cs:2.61183E+02,-3.56095E+01) -- (axis cs:2.61867E+02,-3.47805E+01) -- (axis cs:2.62070E+02,-3.48916E+01) -- (axis cs:2.61387E+02,-3.57218E+01) -- cycle ; 
\draw[MWE-full] (axis cs:2.62538E+02,-3.39478E+01) -- (axis cs:2.63196E+02,-3.31114E+01) -- (axis cs:2.63396E+02,-3.32204E+01) -- (axis cs:2.62740E+02,-3.40578E+01) -- cycle ; 
\draw[MWE-full] (axis cs:2.63841E+02,-3.22717E+01) -- (axis cs:2.64474E+02,-3.14287E+01) -- (axis cs:2.64672E+02,-3.15356E+01) -- (axis cs:2.64040E+02,-3.23796E+01) -- cycle ; 
\draw[MWE-full] (axis cs:2.65096E+02,-3.05826E+01) -- (axis cs:2.65707E+02,-2.97337E+01) -- (axis cs:2.65903E+02,-2.98387E+01) -- (axis cs:2.65293E+02,-3.06886E+01) -- cycle ; 
\draw[MWE-full] (axis cs:2.66308E+02,-2.88819E+01) -- (axis cs:2.66899E+02,-2.80275E+01) -- (axis cs:2.67093E+02,-2.81309E+01) -- (axis cs:2.66503E+02,-2.89861E+01) -- cycle ; 
\draw[MWE-empty] (axis cs:2.66899E+02,-2.80275E+01) -- (axis cs:2.67481E+02,-2.71706E+01) -- (axis cs:2.67674E+02,-2.72732E+01) -- (axis cs:2.67093E+02,-2.81309E+01) -- cycle ; 
\draw[MWE-empty] (axis cs:2.68054E+02,-2.63113E+01) -- (axis cs:2.68618E+02,-2.54498E+01) -- (axis cs:2.68809E+02,-2.55508E+01) -- (axis cs:2.68246E+02,-2.64131E+01) -- cycle ; 
\draw[MWE-empty] (axis cs:2.69174E+02,-2.45861E+01) -- (axis cs:2.69723E+02,-2.37204E+01) -- (axis cs:2.69912E+02,-2.38201E+01) -- (axis cs:2.69365E+02,-2.46865E+01) -- cycle ; 
\draw[MWE-empty] (axis cs:2.70264E+02,-2.28528E+01) -- (axis cs:2.70799E+02,-2.19834E+01) -- (axis cs:2.70986E+02,-2.20817E+01) -- (axis cs:2.70453E+02,-2.29518E+01) -- cycle ; 
\draw[MWE-empty] (axis cs:2.71327E+02,-2.11122E+01) -- (axis cs:2.71848E+02,-2.02394E+01) -- (axis cs:2.72035E+02,-2.03366E+01) -- (axis cs:2.71514E+02,-2.12100E+01) -- cycle ; 
\draw[MWE-empty] (axis cs:2.72364E+02,-1.93650E+01) -- (axis cs:2.72874E+02,-1.84892E+01) -- (axis cs:2.73059E+02,-1.85854E+01) -- (axis cs:2.72550E+02,-1.94617E+01) -- cycle ; 
\draw[MWE-empty] (axis cs:2.73379E+02,-1.76121E+01) -- (axis cs:2.73879E+02,-1.67336E+01) -- (axis cs:2.74063E+02,-1.68289E+01) -- (axis cs:2.73564E+02,-1.77078E+01) -- cycle ; 
\draw[MWE-empty] (axis cs:2.74375E+02,-1.58540E+01) -- (axis cs:2.74866E+02,-1.49732E+01) -- (axis cs:2.75049E+02,-1.50676E+01) -- (axis cs:2.74558E+02,-1.59488E+01) -- cycle ; 
\draw[MWE-empty] (axis cs:2.75353E+02,-1.40913E+01) -- (axis cs:2.75837E+02,-1.32085E+01) -- (axis cs:2.76018E+02,-1.33022E+01) -- (axis cs:2.75535E+02,-1.41854E+01) -- cycle ; 
\draw[MWE-empty] (axis cs:2.76316E+02,-1.23248E+01) -- (axis cs:2.76793E+02,-1.14402E+01) -- (axis cs:2.76974E+02,-1.15333E+01) -- (axis cs:2.76498E+02,-1.24181E+01) -- cycle ; 
\draw[MWE-empty] (axis cs:2.77267E+02,-1.05548E+01) -- (axis cs:2.77738E+02,-9.66873E+00) -- (axis cs:2.77917E+02,-9.76126E+00) -- (axis cs:2.77447E+02,-1.06476E+01) -- cycle ; 
\draw[MWE-empty] (axis cs:2.78206E+02,-8.78200E+00) -- (axis cs:2.78672E+02,-7.89468E+00) -- (axis cs:2.78851E+02,-7.98676E+00) -- (axis cs:2.78385E+02,-8.87430E+00) -- cycle ; 
\draw[MWE-empty] (axis cs:2.79136E+02,-7.00682E+00) -- (axis cs:2.79598E+02,-6.11851E+00) -- (axis cs:2.79777E+02,-6.21024E+00) -- (axis cs:2.79315E+02,-7.09872E+00) -- cycle ; 
\draw[MWE-empty] (axis cs:2.80059E+02,-5.22978E+00) -- (axis cs:2.80518E+02,-4.34070E+00) -- (axis cs:2.80697E+02,-4.43217E+00) -- (axis cs:2.80237E+02,-5.32137E+00) -- cycle ; 
\draw[MWE-empty] (axis cs:2.80977E+02,-3.45134E+00) -- (axis cs:2.81434E+02,-2.56174E+00) -- (axis cs:2.81612E+02,-2.65303E+00) -- (axis cs:2.81155E+02,-3.54271E+00) -- cycle ; 
\draw[MWE-empty] (axis cs:2.81891E+02,-1.67196E+00) -- (axis cs:2.82347E+02,-7.82062E-01) -- (axis cs:2.82525E+02,-8.73274E-01) -- (axis cs:2.82069E+02,-1.76320E+00) -- cycle ; 
\draw[MWE-empty] (axis cs:2.82917E+02,1.07899E-01) -- (axis cs:2.83259E+02,9.97867E-01) -- (axis cs:2.83437E+02,9.06652E-01) -- (axis cs:2.82981E+02,1.66967E-02) -- cycle ; 
\draw[MWE-empty] (axis cs:2.83715E+02,1.88779E+00) -- (axis cs:2.84172E+02,2.77760E+00) -- (axis cs:2.84350E+02,2.68629E+00) -- (axis cs:2.83893E+02,1.79654E+00) -- cycle ; 
\draw[MWE-empty] (axis cs:2.84629E+02,3.66725E+00) -- (axis cs:2.85088E+02,4.55668E+00) -- (axis cs:2.85266E+02,4.46519E+00) -- (axis cs:2.84808E+02,3.57586E+00) -- cycle ; 
\draw[MWE-empty] (axis cs:2.85547E+02,5.44583E+00) -- (axis cs:2.86008E+02,6.33465E+00) -- (axis cs:2.86186E+02,6.24290E+00) -- (axis cs:2.85725E+02,5.35422E+00) -- cycle ; 
\draw[MWE-empty] (axis cs:2.86470E+02,7.22308E+00) -- (axis cs:2.86934E+02,8.11105E+00) -- (axis cs:2.87113E+02,8.01893E+00) -- (axis cs:2.86649E+02,7.13115E+00) -- cycle ; 
\draw[MWE-empty] (axis cs:2.87400E+02,8.99851E+00) -- (axis cs:2.87868E+02,9.88539E+00) -- (axis cs:2.88048E+02,9.79283E+00) -- (axis cs:2.87580E+02,8.90618E+00) -- cycle ; 
\draw[MWE-empty] (axis cs:2.88339E+02,1.07716E+01) -- (axis cs:2.88813E+02,1.16572E+01) -- (axis cs:2.88993E+02,1.15641E+01) -- (axis cs:2.88519E+02,1.06788E+01) -- cycle ; 
\draw[MWE-empty] (axis cs:2.89289E+02,1.25420E+01) -- (axis cs:2.89769E+02,1.34259E+01) -- (axis cs:2.89951E+02,1.33322E+01) -- (axis cs:2.89470E+02,1.24486E+01) -- cycle ; 
\draw[MWE-empty] (axis cs:2.90253E+02,1.43090E+01) -- (axis cs:2.90740E+02,1.51910E+01) -- (axis cs:2.90922E+02,1.50966E+01) -- (axis cs:2.90435E+02,1.42149E+01) -- cycle ; 
\draw[MWE-empty] (axis cs:2.91231E+02,1.60721E+01) -- (axis cs:2.91727E+02,1.69520E+01) -- (axis cs:2.91910E+02,1.68567E+01) -- (axis cs:2.91414E+02,1.59772E+01) -- cycle ; 
\draw[MWE-empty] (axis cs:2.92227E+02,1.78307E+01) -- (axis cs:2.92732E+02,1.87082E+01) -- (axis cs:2.92917E+02,1.86119E+01) -- (axis cs:2.92411E+02,1.77350E+01) -- cycle ; 
\draw[MWE-empty] (axis cs:2.93242E+02,1.95843E+01) -- (axis cs:2.93758E+02,2.04590E+01) -- (axis cs:2.93944E+02,2.03617E+01) -- (axis cs:2.93428E+02,1.94875E+01) -- cycle ; 
\draw[MWE-empty] (axis cs:2.94280E+02,2.13321E+01) -- (axis cs:2.94808E+02,2.22037E+01) -- (axis cs:2.94996E+02,2.21052E+01) -- (axis cs:2.94467E+02,2.12342E+01) -- cycle ; 
\draw[MWE-empty] (axis cs:2.95342E+02,2.30735E+01) -- (axis cs:2.95884E+02,2.39415E+01) -- (axis cs:2.96073E+02,2.38417E+01) -- (axis cs:2.95531E+02,2.29744E+01) -- cycle ; 
\draw[MWE-empty] (axis cs:2.96433E+02,2.48076E+01) -- (axis cs:2.96989E+02,2.56717E+01) -- (axis cs:2.97180E+02,2.55705E+01) -- (axis cs:2.96623E+02,2.47072E+01) -- cycle ; 
\draw[MWE-empty] (axis cs:2.97554E+02,2.65337E+01) -- (axis cs:2.98127E+02,2.73934E+01) -- (axis cs:2.98320E+02,2.72907E+01) -- (axis cs:2.97746E+02,2.64318E+01) -- cycle ; 
\draw[MWE-empty] (axis cs:2.98709E+02,2.82508E+01) -- (axis cs:2.99300E+02,2.91056E+01) -- (axis cs:2.99495E+02,2.90013E+01) -- (axis cs:2.98903E+02,2.81473E+01) -- cycle ; 
\draw[MWE-empty] (axis cs:2.99901E+02,2.99579E+01) -- (axis cs:3.00513E+02,3.08074E+01) -- (axis cs:3.00710E+02,3.07013E+01) -- (axis cs:3.00098E+02,2.98527E+01) -- cycle ; 
\draw[MWE-empty] (axis cs:3.01135E+02,3.16540E+01) -- (axis cs:3.01769E+02,3.24975E+01) -- (axis cs:3.01969E+02,3.23894E+01) -- (axis cs:3.01334E+02,3.15469E+01) -- cycle ; 
\draw[MWE-empty] (axis cs:3.02415E+02,3.33378E+01) -- (axis cs:3.03073E+02,3.41747E+01) -- (axis cs:3.03275E+02,3.40646E+01) -- (axis cs:3.02616E+02,3.32287E+01) -- cycle ; 
\draw[MWE-empty] (axis cs:3.03745E+02,3.50081E+01) -- (axis cs:3.04430E+02,3.58377E+01) -- (axis cs:3.04634E+02,3.57252E+01) -- (axis cs:3.03948E+02,3.48968E+01) -- cycle ; 
\draw[MWE-empty] (axis cs:3.05130E+02,3.66633E+01) -- (axis cs:3.05844E+02,3.74848E+01) -- (axis cs:3.06051E+02,3.73699E+01) -- (axis cs:3.05335E+02,3.65497E+01) -- cycle ; 
\draw[MWE-empty] (axis cs:3.06575E+02,3.83019E+01) -- (axis cs:3.07322E+02,3.91143E+01) -- (axis cs:3.07531E+02,3.89969E+01) -- (axis cs:3.06782E+02,3.81857E+01) -- cycle ; 
\draw[MWE-empty] (axis cs:3.08087E+02,3.99220E+01) -- (axis cs:3.08869E+02,4.07244E+01) -- (axis cs:3.09080E+02,4.06042E+01) -- (axis cs:3.08296E+02,3.98031E+01) -- cycle ; 
\draw[MWE-empty] (axis cs:3.09671E+02,4.15215E+01) -- (axis cs:3.10492E+02,4.23130E+01) -- (axis cs:3.10705E+02,4.21897E+01) -- (axis cs:3.09883E+02,4.13998E+01) -- cycle ; 
\draw[MWE-empty] (axis cs:3.11335E+02,4.30984E+01) -- (axis cs:3.12199E+02,4.38775E+01) -- (axis cs:3.12414E+02,4.37511E+01) -- (axis cs:3.11548E+02,4.29736E+01) -- cycle ; 
\draw[MWE-empty] (axis cs:3.13086E+02,4.46499E+01) -- (axis cs:3.13996E+02,4.54153E+01) -- (axis cs:3.14213E+02,4.52855E+01) -- (axis cs:3.13301E+02,4.45218E+01) -- cycle ; 
\draw[MWE-empty] (axis cs:3.14932E+02,4.61733E+01) -- (axis cs:3.15893E+02,4.69235E+01) -- (axis cs:3.16111E+02,4.67901E+01) -- (axis cs:3.15149E+02,4.60418E+01) -- cycle ; 
\draw[MWE-empty] (axis cs:3.16882E+02,4.76654E+01) -- (axis cs:3.17899E+02,4.83986E+01) -- (axis cs:3.18118E+02,4.82614E+01) -- (axis cs:3.17100E+02,4.75301E+01) -- cycle ; 
\draw[MWE-empty] (axis cs:3.18945E+02,4.91226E+01) -- (axis cs:3.20023E+02,4.98368E+01) -- (axis cs:3.20242E+02,4.96956E+01) -- (axis cs:3.19165E+02,4.89834E+01) -- cycle ; 
\draw[MWE-empty] (axis cs:3.21132E+02,5.05409E+01) -- (axis cs:3.22274E+02,5.12341E+01) -- (axis cs:3.22493E+02,5.10886E+01) -- (axis cs:3.21351E+02,5.03976E+01) -- cycle ; 
\draw[MWE-empty] (axis cs:3.23452E+02,5.19158E+01) -- (axis cs:3.24665E+02,5.25855E+01) -- (axis cs:3.24882E+02,5.24356E+01) -- (axis cs:3.23670E+02,5.17682E+01) -- cycle ; 
\draw[MWE-empty] (axis cs:3.25915E+02,5.32424E+01) -- (axis cs:3.27204E+02,5.38858E+01) -- (axis cs:3.27419E+02,5.37314E+01) -- (axis cs:3.26132E+02,5.30902E+01) -- cycle ; 
\draw[MWE-empty] (axis cs:3.28533E+02,5.45151E+01) -- (axis cs:3.29903E+02,5.51293E+01) -- (axis cs:3.30114E+02,5.49701E+01) -- (axis cs:3.28746E+02,5.43582E+01) -- cycle ; 
\draw[MWE-empty] (axis cs:3.31315E+02,5.57277E+01) -- (axis cs:3.32770E+02,5.63095E+01) -- (axis cs:3.32976E+02,5.61454E+01) -- (axis cs:3.31523E+02,5.55661E+01) -- cycle ; 
\draw[MWE-empty] (axis cs:3.34269E+02,5.68737E+01) -- (axis cs:3.35814E+02,5.74194E+01) -- (axis cs:3.36012E+02,5.72505E+01) -- (axis cs:3.34472E+02,5.67072E+01) -- cycle ; 
\draw[MWE-empty] (axis cs:3.37405E+02,5.79457E+01) -- (axis cs:3.39042E+02,5.84516E+01) -- (axis cs:3.39230E+02,5.82777E+01) -- (axis cs:3.37598E+02,5.77743E+01) -- cycle ; 
\draw[MWE-empty] (axis cs:3.40725E+02,5.89360E+01) -- (axis cs:3.42456E+02,5.93980E+01) -- (axis cs:3.42632E+02,5.92193E+01) -- (axis cs:3.40908E+02,5.87597E+01) -- cycle ; 
\draw[MWE-empty] (axis cs:3.44233E+02,5.98365E+01) -- (axis cs:3.46056E+02,6.02505E+01) -- (axis cs:3.46217E+02,6.00672E+01) -- (axis cs:3.44402E+02,5.96555E+01) -- cycle ; 
\draw[MWE-empty] (axis cs:3.47924E+02,6.06388E+01) -- (axis cs:3.49837E+02,6.10005E+01) -- (axis cs:3.49979E+02,6.08130E+01) -- (axis cs:3.48077E+02,6.04533E+01) -- cycle ; 
\draw[MWE-empty] (axis cs:3.51791E+02,6.13346E+01) -- (axis cs:3.53786E+02,6.16401E+01) -- (axis cs:3.53908E+02,6.14487E+01) -- (axis cs:3.51924E+02,6.11451E+01) -- cycle ; 
\draw[MWE-empty] (axis cs:3.55819E+02,6.19160E+01) -- (axis cs:3.57886E+02,6.21614E+01) -- (axis cs:3.57984E+02,6.19668E+01) -- (axis cs:3.55929E+02,6.17229E+01) -- cycle ; 
\draw[MWE-empty] (axis cs:3.59986E+02,6.23756E+01) -- (axis cs:2.11273E+00,6.25578E+01) -- (axis cs:2.18430E+00,6.23606E+01) -- (axis cs:7.05566E-02,6.21796E+01) -- cycle ; 
\draw[MWE-empty] (axis cs:4.26373E+00,6.27073E+01) -- (axis cs:6.43417E+00,6.28237E+01) -- (axis cs:6.47748E+00,6.26247E+01) -- (axis cs:4.32138E+00,6.25091E+01) -- cycle ; 
\draw[MWE-empty] (axis cs:8.61932E+00,6.29064E+01) -- (axis cs:1.08143E+01,6.29552E+01) -- (axis cs:1.08282E+01,6.27553E+01) -- (axis cs:8.64801E+00,6.27068E+01) -- cycle ; 
\draw[MWE-empty] (axis cs:1.30140E+01,6.29699E+01) -- (axis cs:1.52134E+01,6.29504E+01) -- (axis cs:1.51974E+01,6.27506E+01) -- (axis cs:1.30130E+01,6.27699E+01) -- cycle ; 
\draw[MWE-empty] (axis cs:1.74073E+01,6.28969E+01) -- (axis cs:1.95907E+01,6.28094E+01) -- (axis cs:1.95453E+01,6.26105E+01) -- (axis cs:1.73765E+01,6.26974E+01) -- cycle ; 
\draw[MWE-empty] (axis cs:2.17586E+01,6.26884E+01) -- (axis cs:2.39066E+01,6.25342E+01) -- (axis cs:2.38331E+01,6.23372E+01) -- (axis cs:2.16990E+01,6.24903E+01) -- cycle ; 
\draw[MWE-empty] (axis cs:2.60301E+01,6.23475E+01) -- (axis cs:2.81251E+01,6.21289E+01) -- (axis cs:2.80255E+01,6.19345E+01) -- (axis cs:2.59432E+01,6.21517E+01) -- cycle ; 
\draw[MWE-empty] (axis cs:3.01880E+01,6.18792E+01) -- (axis cs:3.22156E+01,6.15991E+01) -- (axis cs:3.20923E+01,6.14079E+01) -- (axis cs:3.00762E+01,6.16863E+01) -- cycle ; 
\draw[MWE-empty] (axis cs:3.42049E+01,6.12895E+01) -- (axis cs:3.61537E+01,6.09515E+01) -- (axis cs:3.60096E+01,6.07642E+01) -- (axis cs:3.40709E+01,6.11003E+01) -- cycle ; 
\draw[MWE-empty] (axis cs:3.80599E+01,6.05860E+01) -- (axis cs:3.99219E+01,6.01940E+01) -- (axis cs:3.97599E+01,6.00110E+01) -- (axis cs:3.79064E+01,6.04008E+01) -- cycle ; 
\draw[MWE-empty] (axis cs:4.17386E+01,5.97766E+01) -- (axis cs:4.35091E+01,5.93347E+01) -- (axis cs:4.33322E+01,5.91564E+01) -- (axis cs:4.15687E+01,5.95959E+01) -- cycle ; 
\draw[MWE-empty] (axis cs:4.52331E+01,5.88695E+01) -- (axis cs:4.69102E+01,5.83820E+01) -- (axis cs:4.67210E+01,5.82085E+01) -- (axis cs:4.50497E+01,5.86936E+01) -- cycle ; 
\draw[MWE-empty] (axis cs:4.85407E+01,5.78732E+01) -- (axis cs:5.01248E+01,5.73442E+01) -- (axis cs:4.99260E+01,5.71755E+01) -- (axis cs:4.83464E+01,5.77022E+01) -- cycle ; 
\draw[MWE-empty] (axis cs:5.16632E+01,5.67958E+01) -- (axis cs:5.31565E+01,5.62291E+01) -- (axis cs:5.29503E+01,5.60654E+01) -- (axis cs:5.14604E+01,5.66296E+01) -- cycle ; 
\draw[MWE-empty] (axis cs:5.46057E+01,5.56449E+01) -- (axis cs:5.60117E+01,5.50442E+01) -- (axis cs:5.58001E+01,5.48854E+01) -- (axis cs:5.43965E+01,5.54837E+01) -- cycle ; 
\draw[MWE-empty] (axis cs:5.73757E+01,5.44279E+01) -- (axis cs:5.86988E+01,5.37966E+01) -- (axis cs:5.84834E+01,5.36425E+01) -- (axis cs:5.71620E+01,5.42714E+01) -- cycle ; 
\draw[MWE-empty] (axis cs:5.99822E+01,5.31512E+01) -- (axis cs:6.12273E+01,5.24925E+01) -- (axis cs:6.10096E+01,5.23429E+01) -- (axis cs:5.97655E+01,5.29994E+01) -- cycle ; 
\draw[MWE-empty] (axis cs:6.24353E+01,5.18211E+01) -- (axis cs:6.36076E+01,5.11377E+01) -- (axis cs:6.33886E+01,5.09926E+01) -- (axis cs:6.22168E+01,5.16738E+01) -- cycle ; 
\draw[MWE-empty] (axis cs:6.47454E+01,5.04429E+01) -- (axis cs:6.58501E+01,4.97374E+01) -- (axis cs:6.56308E+01,4.95965E+01) -- (axis cs:6.45262E+01,5.02999E+01) -- cycle ; 
\draw[MWE-empty] (axis cs:6.69230E+01,4.90218E+01) -- (axis cs:6.79652E+01,4.82965E+01) -- (axis cs:6.77464E+01,4.81596E+01) -- (axis cs:6.67038E+01,4.88829E+01) -- cycle ; 
\draw[MWE-empty] (axis cs:6.89781E+01,4.75620E+01) -- (axis cs:6.99629E+01,4.68189E+01) -- (axis cs:6.97450E+01,4.66858E+01) -- (axis cs:6.87597E+01,4.74270E+01) -- cycle ; 
\draw[MWE-empty] (axis cs:7.09207E+01,4.60676E+01) -- (axis cs:7.18527E+01,4.53086E+01) -- (axis cs:7.16362E+01,4.51791E+01) -- (axis cs:7.07035E+01,4.59363E+01) -- cycle ; 
\draw[MWE-empty] (axis cs:7.27599E+01,4.45422E+01) -- (axis cs:7.36436E+01,4.37688E+01) -- (axis cs:7.34289E+01,4.36426E+01) -- (axis cs:7.25443E+01,4.44144E+01) -- cycle ; 
\draw[MWE-empty] (axis cs:7.45045E+01,4.29888E+01) -- (axis cs:7.53439E+01,4.22025E+01) -- (axis cs:7.51312E+01,4.20795E+01) -- (axis cs:7.42908E+01,4.28642E+01) -- cycle ; 
\draw[MWE-empty] (axis cs:7.61626E+01,4.14103E+01) -- (axis cs:7.69615E+01,4.06124E+01) -- (axis cs:7.67510E+01,4.04924E+01) -- (axis cs:7.59510E+01,4.12888E+01) -- cycle ; 
\draw[MWE-empty] (axis cs:7.77415E+01,3.98092E+01) -- (axis cs:7.85035E+01,3.90009E+01) -- (axis cs:7.82952E+01,3.88836E+01) -- (axis cs:7.75321E+01,3.96906E+01) -- cycle ; 
\draw[MWE-empty] (axis cs:7.92483E+01,3.81877E+01) -- (axis cs:7.99766E+01,3.73700E+01) -- (axis cs:7.97705E+01,3.72554E+01) -- (axis cs:7.90411E+01,3.80718E+01) -- cycle ; 
\draw[MWE-empty] (axis cs:8.06892E+01,3.65480E+01) -- (axis cs:8.13867E+01,3.57218E+01) -- (axis cs:8.11830E+01,3.56095E+01) -- (axis cs:8.04843E+01,3.64345E+01) -- cycle ; 
\draw[MWE-empty] (axis cs:8.20700E+01,3.48916E+01) -- (axis cs:8.27396E+01,3.40578E+01) -- (axis cs:8.25381E+01,3.39478E+01) -- (axis cs:8.18674E+01,3.47805E+01) -- cycle ; 
\draw[MWE-empty] (axis cs:8.33961E+01,3.32204E+01) -- (axis cs:8.40402E+01,3.23796E+01) -- (axis cs:8.38409E+01,3.22717E+01) -- (axis cs:8.31957E+01,3.31114E+01) -- cycle ; 
\draw[MWE-empty] (axis cs:8.46724E+01,3.15356E+01) -- (axis cs:8.52933E+01,3.06886E+01) -- (axis cs:8.50962E+01,3.05826E+01) -- (axis cs:8.44742E+01,3.14287E+01) -- cycle ; 
\draw[MWE-empty] (axis cs:8.59034E+01,2.98387E+01) -- (axis cs:8.65033E+01,2.89861E+01) -- (axis cs:8.63082E+01,2.88819E+01) -- (axis cs:8.57074E+01,2.97337E+01) -- cycle ; 
\draw[MWE-empty] (axis cs:8.70933E+01,2.81309E+01) -- (axis cs:8.76740E+01,2.72732E+01) -- (axis cs:8.74810E+01,2.71706E+01) -- (axis cs:8.68993E+01,2.80275E+01) -- cycle ; 
\draw[MWE-empty] (axis cs:8.82459E+01,2.64131E+01) -- (axis cs:8.88093E+01,2.55508E+01) -- (axis cs:8.86180E+01,2.54498E+01) -- (axis cs:8.80537E+01,2.63113E+01) -- cycle ; 
\draw[MWE-empty] (axis cs:8.93646E+01,2.46865E+01) -- (axis cs:8.99124E+01,2.38201E+01) -- (axis cs:8.97229E+01,2.37204E+01) -- (axis cs:8.91743E+01,2.45861E+01) -- cycle ; 
\draw[MWE-empty] (axis cs:9.04529E+01,2.29518E+01) -- (axis cs:9.09865E+01,2.20817E+01) -- (axis cs:9.07986E+01,2.19834E+01) -- (axis cs:9.02642E+01,2.28528E+01) -- cycle ; 
\draw[MWE-empty] (axis cs:9.15135E+01,2.12100E+01) -- (axis cs:9.20345E+01,2.03366E+01) -- (axis cs:9.18481E+01,2.02394E+01) -- (axis cs:9.13265E+01,2.11122E+01) -- cycle ; 
\draw[MWE-empty] (axis cs:9.25496E+01,1.94617E+01) -- (axis cs:9.30592E+01,1.85854E+01) -- (axis cs:9.28742E+01,1.84892E+01) -- (axis cs:9.23640E+01,1.93650E+01) -- cycle ; 
\draw[MWE-empty] (axis cs:9.35636E+01,1.77078E+01) -- (axis cs:9.40631E+01,1.68289E+01) -- (axis cs:9.38795E+01,1.67336E+01) -- (axis cs:9.33793E+01,1.76121E+01) -- cycle ; 
\draw[MWE-empty] (axis cs:9.45580E+01,1.59488E+01) -- (axis cs:9.50487E+01,1.50676E+01) -- (axis cs:9.48661E+01,1.49732E+01) -- (axis cs:9.43750E+01,1.58540E+01) -- cycle ; 
\draw[MWE-empty] (axis cs:9.55353E+01,1.41854E+01) -- (axis cs:9.60182E+01,1.33022E+01) -- (axis cs:9.58366E+01,1.32085E+01) -- (axis cs:9.53533E+01,1.40913E+01) -- cycle ; 
\draw[MWE-empty] (axis cs:9.64975E+01,1.24181E+01) -- (axis cs:9.69737E+01,1.15333E+01) -- (axis cs:9.67931E+01,1.14402E+01) -- (axis cs:9.63165E+01,1.23248E+01) -- cycle ; 
\draw[MWE-empty] (axis cs:9.74469E+01,1.06476E+01) -- (axis cs:9.79174E+01,9.76126E+00) -- (axis cs:9.77375E+01,9.66874E+00) -- (axis cs:9.72666E+01,1.05548E+01) -- cycle ; 
\draw[MWE-empty] (axis cs:9.83854E+01,8.87430E+00) -- (axis cs:9.88512E+01,7.98676E+00) -- (axis cs:9.86719E+01,7.89468E+00) -- (axis cs:9.82058E+01,8.78200E+00) -- cycle ; 
\draw[MWE-empty] (axis cs:9.93149E+01,7.09872E+00) -- (axis cs:9.97769E+01,6.21024E+00) -- (axis cs:9.95982E+01,6.11851E+00) -- (axis cs:9.91360E+01,7.00683E+00) -- cycle ; 
\draw[MWE-empty] (axis cs:1.00237E+02,5.32137E+00) -- (axis cs:1.00697E+02,4.43217E+00) -- (axis cs:1.00518E+02,4.34070E+00) -- (axis cs:1.00059E+02,5.22978E+00) -- cycle ; 
\draw[MWE-empty] (axis cs:1.01155E+02,3.54271E+00) -- (axis cs:1.01612E+02,2.65303E+00) -- (axis cs:1.01434E+02,2.56174E+00) -- (axis cs:1.00977E+02,3.45134E+00) -- cycle ; 
\draw[MWE-empty] (axis cs:1.02069E+02,1.76320E+00) -- (axis cs:1.02525E+02,8.73274E-01) -- (axis cs:1.02347E+02,7.82062E-01) -- (axis cs:1.01891E+02,1.67196E+00) -- cycle ; 
\draw[MWE-empty] (axis cs:1.02981E+02,-1.66966E-02) -- (axis cs:1.03437E+02,-9.06652E-01) -- (axis cs:1.03259E+02,-9.97867E-01) -- (axis cs:1.02917E+02,-1.07899E-01) -- cycle ; 
\draw[MWE-empty] (axis cs:1.03893E+02,-1.79654E+00) -- (axis cs:1.04350E+02,-2.68629E+00) -- (axis cs:1.04172E+02,-2.77760E+00) -- (axis cs:1.03715E+02,-1.88779E+00) -- cycle ; 
\draw[MWE-empty] (axis cs:1.04808E+02,-3.57586E+00) -- (axis cs:1.05266E+02,-4.46519E+00) -- (axis cs:1.05088E+02,-4.55668E+00) -- (axis cs:1.04629E+02,-3.66725E+00) -- cycle ; 
\draw[MWE-empty] (axis cs:1.05726E+02,-5.35422E+00) -- (axis cs:1.06186E+02,-6.24290E+00) -- (axis cs:1.06008E+02,-6.33465E+00) -- (axis cs:1.05547E+02,-5.44583E+00) -- cycle ; 
\draw[MWE-empty] (axis cs:1.06649E+02,-7.13115E+00) -- (axis cs:1.07113E+02,-8.01893E+00) -- (axis cs:1.06934E+02,-8.11105E+00) -- (axis cs:1.06470E+02,-7.22308E+00) -- cycle ; 
\draw[MWE-empty] (axis cs:1.07580E+02,-8.90618E+00) -- (axis cs:1.08048E+02,-9.79283E+00) -- (axis cs:1.07868E+02,-9.88539E+00) -- (axis cs:1.07400E+02,-8.99851E+00) -- cycle ; 
\draw[MWE-empty] (axis cs:1.08520E+02,-1.06788E+01) -- (axis cs:1.08993E+02,-1.15641E+01) -- (axis cs:1.08813E+02,-1.16572E+01) -- (axis cs:1.08339E+02,-1.07716E+01) -- cycle ; 
\draw[MWE-empty] (axis cs:1.09471E+02,-1.24486E+01) -- (axis cs:1.09951E+02,-1.33322E+01) -- (axis cs:1.09769E+02,-1.34259E+01) -- (axis cs:1.09289E+02,-1.25420E+01) -- cycle ; 
\draw[MWE-empty] (axis cs:1.10435E+02,-1.42149E+01) -- (axis cs:1.10922E+02,-1.50966E+01) -- (axis cs:1.10740E+02,-1.51910E+01) -- (axis cs:1.10253E+02,-1.43090E+01) -- cycle ; 
\draw[MWE-empty] (axis cs:1.11414E+02,-1.59772E+01) -- (axis cs:1.11910E+02,-1.68567E+01) -- (axis cs:1.11727E+02,-1.69520E+01) -- (axis cs:1.11231E+02,-1.60721E+01) -- cycle ; 
\draw[MWE-empty] (axis cs:1.12411E+02,-1.77350E+01) -- (axis cs:1.12917E+02,-1.86119E+01) -- (axis cs:1.12732E+02,-1.87082E+01) -- (axis cs:1.12227E+02,-1.78307E+01) -- cycle ; 
\draw[MWE-empty] (axis cs:1.13428E+02,-1.94875E+01) -- (axis cs:1.13945E+02,-2.03617E+01) -- (axis cs:1.13758E+02,-2.04590E+01) -- (axis cs:1.13242E+02,-1.95843E+01) -- cycle ; 
\draw[MWE-empty] (axis cs:1.14467E+02,-2.12342E+01) -- (axis cs:1.14996E+02,-2.21052E+01) -- (axis cs:1.14808E+02,-2.22037E+01) -- (axis cs:1.14280E+02,-2.13321E+01) -- cycle ; 
\draw[MWE-empty] (axis cs:1.15531E+02,-2.29744E+01) -- (axis cs:1.16073E+02,-2.38417E+01) -- (axis cs:1.15884E+02,-2.39415E+01) -- (axis cs:1.15342E+02,-2.30735E+01) -- cycle ; 
\draw[MWE-empty] (axis cs:1.16623E+02,-2.47072E+01) -- (axis cs:1.17180E+02,-2.55705E+01) -- (axis cs:1.16989E+02,-2.56717E+01) -- (axis cs:1.16433E+02,-2.48076E+01) -- cycle ; 
\draw[MWE-empty] (axis cs:1.17746E+02,-2.64318E+01) -- (axis cs:1.18320E+02,-2.72907E+01) -- (axis cs:1.18127E+02,-2.73934E+01) -- (axis cs:1.17554E+02,-2.65337E+01) -- cycle ; 
\draw[MWE-empty] (axis cs:1.18903E+02,-2.81473E+01) -- (axis cs:1.19495E+02,-2.90013E+01) -- (axis cs:1.19300E+02,-2.91056E+01) -- (axis cs:1.18709E+02,-2.82508E+01) -- cycle ; 
\draw[MWE-empty] (axis cs:1.20098E+02,-2.98527E+01) -- (axis cs:1.20710E+02,-3.07013E+01) -- (axis cs:1.20513E+02,-3.08074E+01) -- (axis cs:1.19901E+02,-2.99579E+01) -- cycle ; 
\draw[MWE-empty] (axis cs:1.21334E+02,-3.15469E+01) -- (axis cs:1.21969E+02,-3.23894E+01) -- (axis cs:1.21769E+02,-3.24975E+01) -- (axis cs:1.21135E+02,-3.16540E+01) -- cycle ; 
\draw[MWE-empty] (axis cs:1.22616E+02,-3.32287E+01) -- (axis cs:1.23275E+02,-3.40646E+01) -- (axis cs:1.23073E+02,-3.41747E+01) -- (axis cs:1.22415E+02,-3.33378E+01) -- cycle ; 
\draw[MWE-empty] (axis cs:1.23948E+02,-3.48968E+01) -- (axis cs:1.24634E+02,-3.57252E+01) -- (axis cs:1.24430E+02,-3.58377E+01) -- (axis cs:1.23745E+02,-3.50081E+01) -- cycle ; 
\draw[MWE-empty] (axis cs:1.25335E+02,-3.65497E+01) -- (axis cs:1.26051E+02,-3.73699E+01) -- (axis cs:1.25844E+02,-3.74848E+01) -- (axis cs:1.25130E+02,-3.66633E+01) -- cycle ; 
\draw[MWE-empty] (axis cs:1.26782E+02,-3.81857E+01) -- (axis cs:1.27531E+02,-3.89969E+01) -- (axis cs:1.27322E+02,-3.91143E+01) -- (axis cs:1.26575E+02,-3.83019E+01) -- cycle ; 
\draw[MWE-empty] (axis cs:1.28296E+02,-3.98031E+01) -- (axis cs:1.29080E+02,-4.06042E+01) -- (axis cs:1.28869E+02,-4.07244E+01) -- (axis cs:1.28087E+02,-3.99220E+01) -- cycle ; 
\draw[MWE-empty] (axis cs:1.29883E+02,-4.13998E+01) -- (axis cs:1.30705E+02,-4.21897E+01) -- (axis cs:1.30492E+02,-4.23130E+01) -- (axis cs:1.29671E+02,-4.15215E+01) -- cycle ; 
\draw[MWE-empty] (axis cs:1.31549E+02,-4.29736E+01) -- (axis cs:1.32414E+02,-4.37511E+01) -- (axis cs:1.32199E+02,-4.38775E+01) -- (axis cs:1.31335E+02,-4.30984E+01) -- cycle ; 
\draw[MWE-empty] (axis cs:1.33301E+02,-4.45218E+01) -- (axis cs:1.34213E+02,-4.52855E+01) -- (axis cs:1.33996E+02,-4.54153E+01) -- (axis cs:1.33086E+02,-4.46499E+01) -- cycle ; 
\draw[MWE-empty] (axis cs:1.35149E+02,-4.60418E+01) -- (axis cs:1.36111E+02,-4.67901E+01) -- (axis cs:1.35893E+02,-4.69235E+01) -- (axis cs:1.34932E+02,-4.61733E+01) -- cycle ; 
\draw[MWE-empty] (axis cs:1.37100E+02,-4.75301E+01) -- (axis cs:1.38118E+02,-4.82614E+01) -- (axis cs:1.37899E+02,-4.83986E+01) -- (axis cs:1.36882E+02,-4.76654E+01) -- cycle ; 
\draw[MWE-empty] (axis cs:1.39165E+02,-4.89834E+01) -- (axis cs:1.40242E+02,-4.96956E+01) -- (axis cs:1.40023E+02,-4.98368E+01) -- (axis cs:1.38945E+02,-4.91226E+01) -- cycle ; 
\draw[MWE-empty] (axis cs:1.41351E+02,-5.03976E+01) -- (axis cs:1.42493E+02,-5.10886E+01) -- (axis cs:1.42274E+02,-5.12341E+01) -- (axis cs:1.41132E+02,-5.05409E+01) -- cycle ; 
\draw[MWE-empty] (axis cs:1.43670E+02,-5.17682E+01) -- (axis cs:1.44882E+02,-5.24356E+01) -- (axis cs:1.44665E+02,-5.25855E+01) -- (axis cs:1.43452E+02,-5.19158E+01) -- cycle ; 
\draw[MWE-empty] (axis cs:1.46132E+02,-5.30902E+01) -- (axis cs:1.47419E+02,-5.37314E+01) -- (axis cs:1.47204E+02,-5.38858E+01) -- (axis cs:1.45915E+02,-5.32424E+01) -- cycle ; 
\draw[MWE-empty] (axis cs:1.48746E+02,-5.43582E+01) -- (axis cs:1.50114E+02,-5.49701E+01) -- (axis cs:1.49903E+02,-5.51293E+01) -- (axis cs:1.48533E+02,-5.45151E+01) -- cycle ; 
\draw[MWE-empty] (axis cs:1.51523E+02,-5.55661E+01) -- (axis cs:1.52976E+02,-5.61454E+01) -- (axis cs:1.52770E+02,-5.63095E+01) -- (axis cs:1.51315E+02,-5.57277E+01) -- cycle ; 
\draw[MWE-empty] (axis cs:1.54472E+02,-5.67072E+01) -- (axis cs:1.56012E+02,-5.72505E+01) -- (axis cs:1.55814E+02,-5.74194E+01) -- (axis cs:1.54269E+02,-5.68737E+01) -- cycle ; 
\draw[MWE-empty] (axis cs:1.57598E+02,-5.77743E+01) -- (axis cs:1.59230E+02,-5.82777E+01) -- (axis cs:1.59042E+02,-5.84516E+01) -- (axis cs:1.57405E+02,-5.79457E+01) -- cycle ; 
\draw[MWE-empty] (axis cs:1.60908E+02,-5.87597E+01) -- (axis cs:1.62632E+02,-5.92193E+01) -- (axis cs:1.62456E+02,-5.93980E+01) -- (axis cs:1.60725E+02,-5.89360E+01) -- cycle ; 
\draw[MWE-empty] (axis cs:1.64402E+02,-5.96555E+01) -- (axis cs:1.66217E+02,-6.00672E+01) -- (axis cs:1.66056E+02,-6.02505E+01) -- (axis cs:1.64233E+02,-5.98365E+01) -- cycle ; 
\draw[MWE-empty] (axis cs:1.68077E+02,-6.04533E+01) -- (axis cs:1.69979E+02,-6.08130E+01) -- (axis cs:1.69837E+02,-6.10005E+01) -- (axis cs:1.67924E+02,-6.06388E+01) -- cycle ; 
\draw[MWE-empty] (axis cs:1.71924E+02,-6.11451E+01) -- (axis cs:1.73908E+02,-6.14487E+01) -- (axis cs:1.73786E+02,-6.16401E+01) -- (axis cs:1.71791E+02,-6.13346E+01) -- cycle ; 
\draw[MWE-empty] (axis cs:1.75929E+02,-6.17229E+01) -- (axis cs:1.77984E+02,-6.19668E+01) -- (axis cs:1.77886E+02,-6.21614E+01) -- (axis cs:1.75819E+02,-6.19160E+01) -- cycle ; 
\draw[MWE-empty] (axis cs:1.80071E+02,-6.21796E+01) -- (axis cs:1.82184E+02,-6.23606E+01) -- (axis cs:1.82113E+02,-6.25578E+01) -- (axis cs:1.79986E+02,-6.23756E+01) -- cycle ; 
\draw[MWE-empty] (axis cs:1.84321E+02,-6.25091E+01) -- (axis cs:1.86478E+02,-6.26247E+01) -- (axis cs:1.86434E+02,-6.28237E+01) -- (axis cs:1.84264E+02,-6.27073E+01) -- cycle ; 
\draw[MWE-empty] (axis cs:1.88648E+02,-6.27068E+01) -- (axis cs:1.90828E+02,-6.27553E+01) -- (axis cs:1.90814E+02,-6.29552E+01) -- (axis cs:1.88619E+02,-6.29064E+01) -- cycle ; 
\draw[MWE-empty] (axis cs:1.93013E+02,-6.27699E+01) -- (axis cs:1.95197E+02,-6.27506E+01) -- (axis cs:1.95213E+02,-6.29504E+01) -- (axis cs:1.93014E+02,-6.29699E+01) -- cycle ; 
\draw[MWE-empty] (axis cs:1.97377E+02,-6.26974E+01) -- (axis cs:1.99545E+02,-6.26105E+01) -- (axis cs:1.99591E+02,-6.28094E+01) -- (axis cs:1.97407E+02,-6.28969E+01) -- cycle ; 
\draw[MWE-empty] (axis cs:2.01699E+02,-6.24903E+01) -- (axis cs:2.03833E+02,-6.23372E+01) -- (axis cs:2.03907E+02,-6.25342E+01) -- (axis cs:2.01759E+02,-6.26884E+01) -- cycle ; 
\draw[MWE-empty] (axis cs:2.05943E+02,-6.21517E+01) -- (axis cs:2.08026E+02,-6.19345E+01) -- (axis cs:2.08125E+02,-6.21289E+01) -- (axis cs:2.06030E+02,-6.23475E+01) -- cycle ; 
\draw[MWE-empty] (axis cs:2.10076E+02,-6.16863E+01) -- (axis cs:2.12092E+02,-6.14079E+01) -- (axis cs:2.12216E+02,-6.15991E+01) -- (axis cs:2.10188E+02,-6.18792E+01) -- cycle ; 
\draw[MWE-empty] (axis cs:2.14071E+02,-6.11003E+01) -- (axis cs:2.16010E+02,-6.07642E+01) -- (axis cs:2.16154E+02,-6.09515E+01) -- (axis cs:2.14205E+02,-6.12895E+01) -- cycle ; 
\draw[MWE-empty] (axis cs:2.17906E+02,-6.04008E+01) -- (axis cs:2.19760E+02,-6.00110E+01) -- (axis cs:2.19922E+02,-6.01940E+01) -- (axis cs:2.18060E+02,-6.05860E+01) -- cycle ; 
\draw[MWE-empty] (axis cs:2.21569E+02,-5.95959E+01) -- (axis cs:2.23332E+02,-5.91564E+01) -- (axis cs:2.23509E+02,-5.93347E+01) -- (axis cs:2.21739E+02,-5.97766E+01) -- cycle ; 
\draw[MWE-empty] (axis cs:2.25050E+02,-5.86936E+01) -- (axis cs:2.26721E+02,-5.82085E+01) -- (axis cs:2.26910E+02,-5.83820E+01) -- (axis cs:2.25233E+02,-5.88695E+01) -- cycle ; 
\draw[MWE-empty] (axis cs:2.28346E+02,-5.77022E+01) -- (axis cs:2.29926E+02,-5.71755E+01) -- (axis cs:2.30125E+02,-5.73442E+01) -- (axis cs:2.28541E+02,-5.78732E+01) -- cycle ; 
\draw[MWE-empty] (axis cs:2.31460E+02,-5.66296E+01) -- (axis cs:2.32950E+02,-5.60654E+01) -- (axis cs:2.33157E+02,-5.62291E+01) -- (axis cs:2.31663E+02,-5.67958E+01) -- cycle ; 
\draw[MWE-empty] (axis cs:2.34397E+02,-5.54837E+01) -- (axis cs:2.35800E+02,-5.48854E+01) -- (axis cs:2.36012E+02,-5.50442E+01) -- (axis cs:2.34606E+02,-5.56449E+01) -- cycle ; 
\draw[MWE-empty] (axis cs:2.37162E+02,-5.42714E+01) -- (axis cs:2.38483E+02,-5.36425E+01) -- (axis cs:2.38699E+02,-5.37966E+01) -- (axis cs:2.37376E+02,-5.44279E+01) -- cycle ; 
\draw[MWE-empty] (axis cs:2.39766E+02,-5.29994E+01) -- (axis cs:2.41010E+02,-5.23429E+01) -- (axis cs:2.41227E+02,-5.24925E+01) -- (axis cs:2.39982E+02,-5.31512E+01) -- cycle ; 
\draw[MWE-empty] (axis cs:2.42217E+02,-5.16738E+01) -- (axis cs:2.43389E+02,-5.09926E+01) -- (axis cs:2.43608E+02,-5.11377E+01) -- (axis cs:2.42435E+02,-5.18211E+01) -- cycle ; 
\draw[MWE-empty] (axis cs:2.44526E+02,-5.02999E+01) -- (axis cs:2.45631E+02,-4.95965E+01) -- (axis cs:2.45850E+02,-4.97374E+01) -- (axis cs:2.44745E+02,-5.04429E+01) -- cycle ; 
\draw[MWE-empty] (axis cs:2.46704E+02,-4.88829E+01) -- (axis cs:2.47746E+02,-4.81596E+01) -- (axis cs:2.47965E+02,-4.82965E+01) -- (axis cs:2.46923E+02,-4.90218E+01) -- cycle ; 
\draw[MWE-empty] (axis cs:2.48760E+02,-4.74270E+01) -- (axis cs:2.49745E+02,-4.66858E+01) -- (axis cs:2.49963E+02,-4.68189E+01) -- (axis cs:2.48978E+02,-4.75620E+01) -- cycle ; 
\draw[MWE-empty] (axis cs:2.50704E+02,-4.59363E+01) -- (axis cs:2.51636E+02,-4.51791E+01) -- (axis cs:2.51853E+02,-4.53086E+01) -- (axis cs:2.50921E+02,-4.60676E+01) -- cycle ; 
\draw[MWE-empty] (axis cs:2.52544E+02,-4.44144E+01) -- (axis cs:2.53429E+02,-4.36426E+01) -- (axis cs:2.53644E+02,-4.37688E+01) -- (axis cs:2.52760E+02,-4.45422E+01) -- cycle ; 
\draw[MWE-empty] (axis cs:2.54291E+02,-4.28642E+01) -- (axis cs:2.55131E+02,-4.20795E+01) -- (axis cs:2.55344E+02,-4.22025E+01) -- (axis cs:2.54505E+02,-4.29888E+01) -- cycle ; 
\draw[MWE-empty] (axis cs:2.55951E+02,-4.12888E+01) -- (axis cs:2.56751E+02,-4.04924E+01) -- (axis cs:2.56962E+02,-4.06124E+01) -- (axis cs:2.56163E+02,-4.14103E+01) -- cycle ; 
\draw[MWE-empty] (axis cs:2.57532E+02,-3.96906E+01) -- (axis cs:2.58295E+02,-3.88836E+01) -- (axis cs:2.58504E+02,-3.90009E+01) -- (axis cs:2.57742E+02,-3.98092E+01) -- cycle ; 
\draw[MWE-empty] (axis cs:2.59041E+02,-3.80718E+01) -- (axis cs:2.59771E+02,-3.72554E+01) -- (axis cs:2.59977E+02,-3.73700E+01) -- (axis cs:2.59248E+02,-3.81877E+01) -- cycle ; 
\draw[MWE-empty] (axis cs:2.60484E+02,-3.64345E+01) -- (axis cs:2.61183E+02,-3.56095E+01) -- (axis cs:2.61387E+02,-3.57218E+01) -- (axis cs:2.60689E+02,-3.65480E+01) -- cycle ; 
\draw[MWE-empty] (axis cs:2.61867E+02,-3.47805E+01) -- (axis cs:2.62538E+02,-3.39478E+01) -- (axis cs:2.62740E+02,-3.40578E+01) -- (axis cs:2.62070E+02,-3.48916E+01) -- cycle ; 
\draw[MWE-empty] (axis cs:2.63196E+02,-3.31114E+01) -- (axis cs:2.63841E+02,-3.22717E+01) -- (axis cs:2.64040E+02,-3.23796E+01) -- (axis cs:2.63396E+02,-3.32204E+01) -- cycle ; 
\draw[MWE-empty] (axis cs:2.64474E+02,-3.14287E+01) -- (axis cs:2.65096E+02,-3.05826E+01) -- (axis cs:2.65293E+02,-3.06886E+01) -- (axis cs:2.64672E+02,-3.15356E+01) -- cycle ; 
\draw[MWE-empty] (axis cs:2.65707E+02,-2.97337E+01) -- (axis cs:2.66308E+02,-2.88819E+01) -- (axis cs:2.66503E+02,-2.89861E+01) -- (axis cs:2.65903E+02,-2.98387E+01) -- cycle ; 

 
\draw [MWE-empty] (axis cs:2.66016E+02,-2.87251E+01) -- (axis cs:2.66797E+02,-2.91419E+01)   ;
\node[pin={[pin distance=-0.4\onedegree,MWE-label]0:{0$^\circ$}}] at (axis cs:2.66016E+02,-2.87251E+01) {} ;
\draw [MWE-empty] (axis cs:2.71569E+02,-2.00932E+01) -- (axis cs:2.72314E+02,-2.04821E+01)   ;
\node[pin={[pin distance=-0.4\onedegree,MWE-label]00:{10$^\circ$}}] at (axis cs:2.71569E+02,-2.00932E+01) {} ;
\draw [MWE-empty] (axis cs:2.76522E+02,-1.13004E+01) -- (axis cs:2.77245E+02,-1.16726E+01)   ;
\node[pin={[pin distance=-0.4\onedegree,MWE-label]00:{20$^\circ$}}] at (axis cs:2.76522E+02,-1.13004E+01) {} ;
\draw [MWE-empty] (axis cs:2.81167E+02,-2.42475E+00) -- (axis cs:2.81879E+02,-2.78993E+00)   ;
\node[pin={[pin distance=-0.4\onedegree,MWE-label]00:{30$^\circ$}}] at (axis cs:2.81167E+02,-2.42475E+00) {} ;
\draw [MWE-empty] (axis cs:2.85739E+02,6.47217E+00) -- (axis cs:2.86454E+02,6.10515E+00)   ;
\node[pin={[pin distance=-0.4\onedegree,MWE-label]00:{40$^\circ$}}] at (axis cs:2.85739E+02,6.47217E+00) {} ;
\draw [MWE-empty] (axis cs:2.90465E+02,1.53325E+01) -- (axis cs:2.91196E+02,1.49546E+01)   ;
\node[pin={[pin distance=-0.4\onedegree,MWE-label]00:{50$^\circ$}}] at (axis cs:2.90465E+02,1.53325E+01) {} ;
\draw [MWE-empty] (axis cs:2.95599E+02,2.40907E+01) -- (axis cs:2.96357E+02,2.36917E+01)   ;
\node[pin={[pin distance=-0.4\onedegree,MWE-label]00:{60$^\circ$}}] at (axis cs:2.95599E+02,2.40907E+01) {} ;
\draw [MWE-empty] (axis cs:3.01469E+02,3.26590E+01) -- (axis cs:3.02267E+02,3.22267E+01)   ;
\node[pin={[pin distance=-0.4\onedegree,MWE-label]00:{70$^\circ$}}] at (axis cs:3.01469E+02,3.26590E+01) {} ;
\draw [MWE-empty] (axis cs:3.08552E+02,4.09041E+01) -- (axis cs:3.09394E+02,4.04231E+01)   ;
\node[pin={[pin distance=-0.4\onedegree,MWE-label]00:{80$^\circ$}}] at (axis cs:3.08552E+02,4.09041E+01) {} ;
\draw [MWE-empty] (axis cs:3.17568E+02,4.86036E+01) -- (axis cs:3.18444E+02,4.80548E+01)   ;
\node[pin={[pin distance=-0.4\onedegree,MWE-label]90:{90$^\circ$}}] at (axis cs:3.17568E+02,4.86036E+01) {} ;
\draw [MWE-empty] (axis cs:3.29582E+02,5.53674E+01) -- (axis cs:3.30428E+02,5.47306E+01)   ;
\node[pin={[pin distance=-0.4\onedegree,MWE-label]090:{100$^\circ$}}] at (axis cs:3.29582E+02,5.53674E+01) {} ;
\draw [MWE-empty] (axis cs:3.45811E+02,6.05250E+01) -- (axis cs:3.46455E+02,5.97919E+01)   ;
\node[pin={[pin distance=-0.4\onedegree,MWE-label]090:{110$^\circ$}}] at (axis cs:3.45811E+02,6.05250E+01) {} ;
\draw [MWE-empty] (axis cs:6.36807E+00,6.31222E+01) -- (axis cs:6.54138E+00,6.23261E+01)   ;
\node[pin={[pin distance=-0.4\onedegree,MWE-label]090:{120$^\circ$}}] at (axis cs:6.36807E+00,6.31222E+01) {} ;
\draw [MWE-empty] (axis cs:2.82769E+01,6.24205E+01) -- (axis cs:2.78784E+01,6.16426E+01)   ;
\node[pin={[pin distance=-0.4\onedegree,MWE-label]090:{130$^\circ$}}] at (axis cs:2.82769E+01,6.24205E+01) {} ;
\draw [MWE-empty] (axis cs:4.71975E+01,5.86418E+01) -- (axis cs:4.64407E+01,5.79477E+01)   ;
\node[pin={[pin distance=-0.4\onedegree,MWE-label]090:{140$^\circ$}}] at (axis cs:4.71975E+01,5.86418E+01) {} ;
\draw [MWE-empty] (axis cs:6.15567E+01,5.27160E+01) -- (axis cs:6.06857E+01,5.21179E+01)   ;
\node[pin={[pin distance=-0.4\onedegree,MWE-label]090:{150$^\circ$}}] at (axis cs:6.15567E+01,5.27160E+01) {} ;
\draw [MWE-empty] (axis cs:7.21792E+01,4.55021E+01) -- (axis cs:7.13134E+01,4.49840E+01)   ;
\node[pin={[pin distance=-0.4\onedegree,MWE-label]090:{160$^\circ$}}] at (axis cs:7.21792E+01,4.55021E+01) {} ;
\draw [MWE-empty] (axis cs:8.02868E+01,3.75414E+01) -- (axis cs:7.94627E+01,3.70827E+01)   ;
\node[pin={[pin distance=-0.4\onedegree,MWE-label]090:{170$^\circ$}}] at (axis cs:8.02868E+01,3.75414E+01) {} ;
\draw [MWE-empty] (axis cs:8.67966E+01,2.91419E+01) -- (axis cs:8.60164E+01,2.87251E+01)   ;
\node[pin={[pin distance=-0.4\onedegree,MWE-label]090:{180$^\circ$}}] at (axis cs:8.67966E+01,2.91419E+01) {} ;
\draw [MWE-empty] (axis cs:9.23144E+01,2.04821E+01) -- (axis cs:9.15691E+01,2.00932E+01)   ;
\node[pin={[pin distance=-0.4\onedegree,MWE-label]180:{190$^\circ$}}] at (axis cs:9.23144E+01,2.04821E+01) {} ;
\draw [MWE-empty] (axis cs:9.72449E+01,1.16726E+01) -- (axis cs:9.65223E+01,1.13004E+01)   ;
\node[pin={[pin distance=-0.4\onedegree,MWE-label]180:{200$^\circ$}}] at (axis cs:9.72449E+01,1.16726E+01) {} ;
\draw [MWE-empty] (axis cs:1.01879E+02,2.78993E+00) -- (axis cs:1.01167E+02,2.42475E+00)   ;
\node[pin={[pin distance=-0.4\onedegree,MWE-label]180:{210$^\circ$}}] at (axis cs:1.01879E+02,2.78993E+00) {} ;
\draw [MWE-empty] (axis cs:1.06454E+02,-6.10515E+00) -- (axis cs:1.05739E+02,-6.47217E+00)   ;
\node[pin={[pin distance=-0.4\onedegree,MWE-label]180:{220$^\circ$}}] at (axis cs:1.06454E+02,-6.10515E+00) {} ;
\draw [MWE-empty] (axis cs:1.11196E+02,-1.49546E+01) -- (axis cs:1.10466E+02,-1.53325E+01)   ;
\node[pin={[pin distance=-0.4\onedegree,MWE-label]180:{230$^\circ$}}] at (axis cs:1.11196E+02,-1.49546E+01) {} ;
\draw [MWE-empty] (axis cs:1.16357E+02,-2.36917E+01) -- (axis cs:1.15599E+02,-2.40907E+01)   ;
\node[pin={[pin distance=-0.4\onedegree,MWE-label]180:{240$^\circ$}}] at (axis cs:1.16357E+02,-2.36917E+01) {} ;
\draw [MWE-empty] (axis cs:1.22267E+02,-3.22267E+01) -- (axis cs:1.21469E+02,-3.26590E+01)   ;
\node[pin={[pin distance=-0.4\onedegree,MWE-label]180:{250$^\circ$}}] at (axis cs:1.22267E+02,-3.22267E+01) {} ;
\draw [MWE-empty] (axis cs:1.29395E+02,-4.04231E+01) -- (axis cs:1.28552E+02,-4.09041E+01)   ;
\node[pin={[pin distance=-0.4\onedegree,MWE-label]180:{260$^\circ$}}] at (axis cs:1.29395E+02,-4.04231E+01) {} ;
\draw [MWE-empty] (axis cs:1.38444E+02,-4.80548E+01) -- (axis cs:1.37568E+02,-4.86036E+01)   ;
\node[pin={[pin distance=-0.4\onedegree,MWE-label]090:{270$^\circ$}}] at (axis cs:1.38444E+02,-4.80548E+01) {} ;
\draw [MWE-empty] (axis cs:1.50428E+02,-5.47306E+01) -- (axis cs:1.49582E+02,-5.53674E+01)   ;
\node[pin={[pin distance=-0.4\onedegree,MWE-label]090:{280$^\circ$}}] at (axis cs:1.50428E+02,-5.47306E+01) {} ;
\draw [MWE-empty] (axis cs:1.66455E+02,-5.97919E+01) -- (axis cs:1.65811E+02,-6.05250E+01)   ;
\node[pin={[pin distance=-0.4\onedegree,MWE-label]090:{290$^\circ$}}] at (axis cs:1.66455E+02,-5.97919E+01) {} ;
\draw [MWE-empty] (axis cs:1.86541E+02,-6.23261E+01) -- (axis cs:1.86368E+02,-6.31222E+01)   ;
\node[pin={[pin distance=-0.4\onedegree,MWE-label]090:{300$^\circ$}}] at (axis cs:1.86541E+02,-6.23261E+01) {} ;
\draw [MWE-empty] (axis cs:2.07878E+02,-6.16426E+01) -- (axis cs:2.08277E+02,-6.24205E+01)   ;
\node[pin={[pin distance=-0.4\onedegree,MWE-label]090:{310$^\circ$}}] at (axis cs:2.07878E+02,-6.16426E+01) {} ;
\draw [MWE-empty] (axis cs:2.26441E+02,-5.79477E+01) -- (axis cs:2.27198E+02,-5.86418E+01)   ;
\node[pin={[pin distance=-0.4\onedegree,MWE-label]090:{320$^\circ$}}] at (axis cs:2.26441E+02,-5.79477E+01) {} ;
\draw [MWE-empty] (axis cs:2.40686E+02,-5.21179E+01) -- (axis cs:2.41557E+02,-5.27160E+01)   ;
\node[pin={[pin distance=-0.4\onedegree,MWE-label]090:{330$^\circ$}}] at (axis cs:2.40686E+02,-5.21179E+01) {} ;
\draw [MWE-empty] (axis cs:2.51313E+02,-4.49840E+01) -- (axis cs:2.52179E+02,-4.55021E+01)   ;
\node[pin={[pin distance=-0.4\onedegree,MWE-label]090:{340$^\circ$}}] at (axis cs:2.51313E+02,-4.49840E+01) {} ;
\draw [MWE-empty] (axis cs:2.59463E+02,-3.70827E+01) -- (axis cs:2.60287E+02,-3.75414E+01)   ;
\node[pin={[pin distance=-0.4\onedegree,MWE-label]090:{350$^\circ$}}] at (axis cs:2.59463E+02,-3.70827E+01) {} ;


\end{axis}

% Ecliptic coordinate system


\begin{axis}[name=ecliptics,axis lines=none,at=(base.center),anchor=center]

% This simply plots the line
%\addplot[ecliptics,domain=0:390,samples=390] {23.433*sin(x)};

\draw [ecliptics-full] (axis cs:3.59960E+02,9.17485E-02) -- (axis cs:360.0398,-9.17483E-02)  -- (axis cs:3.59122E+02,-4.89506E-01) -- (axis cs:3.59043E+02,-3.06005E-01) -- cycle  ;

\draw [ecliptics-full] (axis cs:{8.77725E-01},{4.89506E-01}) -- (axis cs:{9.57272E-01},{3.06005E-01})  -- (axis cs:{1.87485E+00},{7.03652E-01}) -- (axis cs:{1.79532E+00},{8.87166E-01}) -- cycle  ;
\draw [ecliptics-full] (axis cs:2.71311E+00,1.28463E+00) -- (axis cs:2.79259E+00,1.10109E+00)  -- (axis cs:3.71059E+00,1.49822E+00) -- (axis cs:3.63117E+00,1.68179E+00) -- cycle  ;
\draw [ecliptics-full] (axis cs:4.54959E+00,2.07854E+00) -- (axis cs:4.62893E+00,1.89493E+00)  -- (axis cs:5.54771E+00,2.29113E+00) -- (axis cs:5.46845E+00,2.47479E+00) -- cycle  ;
\draw [ecliptics-full] (axis cs:6.38786E+00,2.87043E+00) -- (axis cs:6.46701E+00,2.68672E+00)  -- (axis cs:7.38691E+00,3.08158E+00) -- (axis cs:7.30789E+00,3.26536E+00) -- cycle  ;
\draw [ecliptics-full] (axis cs:8.22863E+00,3.65948E+00) -- (axis cs:8.30751E+00,3.47563E+00)  -- (axis cs:9.22889E+00,3.86875E+00) -- (axis cs:9.15017E+00,4.05268E+00) -- cycle  ;
\draw [ecliptics-full] (axis cs:1.00726E+01,4.44487E+00) -- (axis cs:1.01511E+01,4.26084E+00)  -- (axis cs:1.10743E+01,4.65180E+00) -- (axis cs:1.09960E+01,4.83593E+00) -- cycle  ;
\draw [ecliptics-full] (axis cs:1.19204E+01,5.22577E+00) -- (axis cs:1.19986E+01,5.04154E+00)  -- (axis cs:1.29239E+01,5.42993E+00) -- (axis cs:1.28460E+01,5.61429E+00) -- cycle  ;
\draw [ecliptics-full] (axis cs:1.37728E+01,6.00137E+00) -- (axis cs:1.38505E+01,5.81689E+00)  -- (axis cs:1.47783E+01,6.20231E+00) -- (axis cs:1.47009E+01,6.38692E+00) -- cycle  ;
\draw [ecliptics-full] (axis cs:1.56304E+01,6.77084E+00) -- (axis cs:1.57075E+01,6.58609E+00)  -- (axis cs:1.66382E+01,6.96811E+00) -- (axis cs:1.65614E+01,7.15301E+00) -- cycle  ;
\draw [ecliptics-full] (axis cs:1.74939E+01,7.53334E+00) -- (axis cs:1.75704E+01,7.34829E+00)  -- (axis cs:1.85041E+01,7.72651E+00) -- (axis cs:1.84280E+01,7.91173E+00) -- cycle  ;
\draw [ecliptics-full] (axis cs:1.93638E+01,8.28806E+00) -- (axis cs:1.94396E+01,8.10267E+00)  -- (axis cs:2.03768E+01,8.47667E+00) -- (axis cs:2.03014E+01,8.66224E+00) -- cycle  ;
\draw [ecliptics-full] (axis cs:2.12408E+01,9.03415E+00) -- (axis cs:2.13159E+01,8.84840E+00)  -- (axis cs:2.22568E+01,9.21776E+00) -- (axis cs:2.21822E+01,9.40370E+00) -- cycle  ;
\draw [ecliptics-full] (axis cs:2.31256E+01,9.77078E+00) -- (axis cs:2.31998E+01,9.58464E+00)  -- (axis cs:2.41448E+01,9.94894E+00) -- (axis cs:2.40710E+01,1.01353E+01) -- cycle  ;
\draw [ecliptics-full] (axis cs:2.50186E+01,1.04971E+01) -- (axis cs:2.50918E+01,1.03105E+01)  -- (axis cs:2.60411E+01,1.06694E+01) -- (axis cs:2.59683E+01,1.08562E+01) -- cycle  ;
\draw [ecliptics-full] (axis cs:2.69204E+01,1.12123E+01) -- (axis cs:2.69926E+01,1.10253E+01)  -- (axis cs:2.79465E+01,1.13782E+01) -- (axis cs:2.78747E+01,1.15655E+01) -- cycle  ;
\draw [ecliptics-full] (axis cs:2.88315E+01,1.19155E+01) -- (axis cs:2.89026E+01,1.17280E+01)  -- (axis cs:2.98613E+01,1.20746E+01) -- (axis cs:2.97907E+01,1.22623E+01) -- cycle  ;
\draw [ecliptics-full] (axis cs:3.07524E+01,1.26059E+01) -- (axis cs:3.08224E+01,1.24179E+01)  -- (axis cs:3.17860E+01,1.27578E+01) -- (axis cs:3.17166E+01,1.29460E+01) -- cycle  ;
\draw [ecliptics-full] (axis cs:3.26835E+01,1.32826E+01) -- (axis cs:3.27522E+01,1.30941E+01)  -- (axis cs:3.37211E+01,1.34268E+01) -- (axis cs:3.36530E+01,1.36155E+01) -- cycle  ;
\draw [ecliptics-full] (axis cs:3.46253E+01,1.39448E+01) -- (axis cs:3.46927E+01,1.37558E+01)  -- (axis cs:3.56669E+01,1.40809E+01) -- (axis cs:3.56002E+01,1.42701E+01) -- cycle  ;
\draw [ecliptics-full] (axis cs:3.65780E+01,1.45916E+01) -- (axis cs:3.66440E+01,1.44021E+01)  -- (axis cs:3.76238E+01,1.47192E+01) -- (axis cs:3.75586E+01,1.49090E+01) -- cycle  ;
\draw [ecliptics-full] (axis cs:3.85421E+01,1.52222E+01) -- (axis cs:3.86065E+01,1.50321E+01)  -- (axis cs:3.95921E+01,1.53408E+01) -- (axis cs:3.95285E+01,1.55312E+01) -- cycle  ;
\draw [ecliptics-full] (axis cs:4.05178E+01,1.58358E+01) -- (axis cs:4.05806E+01,1.56451E+01)  -- (axis cs:4.15720E+01,1.59450E+01) -- (axis cs:4.15101E+01,1.61359E+01) -- cycle  ;
\draw [ecliptics-full] (axis cs:4.25053E+01,1.64315E+01) -- (axis cs:4.25664E+01,1.62403E+01)  -- (axis cs:4.35638E+01,1.65309E+01) -- (axis cs:4.35036E+01,1.67224E+01) -- cycle  ;
\draw [ecliptics-full] (axis cs:4.45048E+01,1.70085E+01) -- (axis cs:4.45641E+01,1.68167E+01)  -- (axis cs:4.55675E+01,1.70976E+01) -- (axis cs:4.55091E+01,1.72897E+01) -- cycle  ;
\draw [ecliptics-full] (axis cs:4.65165E+01,1.75660E+01) -- (axis cs:4.65738E+01,1.73736E+01)  -- (axis cs:4.75832E+01,1.76445E+01) -- (axis cs:4.75268E+01,1.78371E+01) -- cycle  ;
\draw [ecliptics-full] (axis cs:4.85402E+01,1.81031E+01) -- (axis cs:4.85956E+01,1.79102E+01)  -- (axis cs:4.96110E+01,1.81706E+01) -- (axis cs:4.95567E+01,1.83638E+01) -- cycle  ;
\draw [ecliptics-full] (axis cs:5.05762E+01,1.86192E+01) -- (axis cs:5.06294E+01,1.84256E+01)  -- (axis cs:5.16509E+01,1.86752E+01) -- (axis cs:5.15987E+01,1.88690E+01) -- cycle  ;
\draw [ecliptics-full] (axis cs:5.26242E+01,1.91133E+01) -- (axis cs:5.26753E+01,1.89192E+01)  -- (axis cs:5.37027E+01,1.91575E+01) -- (axis cs:5.36527E+01,1.93519E+01) -- cycle  ;
\draw [ecliptics-full] (axis cs:5.46842E+01,1.95848E+01) -- (axis cs:5.47330E+01,1.93901E+01)  -- (axis cs:5.57662E+01,1.96168E+01) -- (axis cs:5.57187E+01,1.98118E+01) -- cycle  ;
\draw [ecliptics-full] (axis cs:5.67560E+01,2.00328E+01) -- (axis cs:5.68024E+01,1.98376E+01)  -- (axis cs:5.78414E+01,2.00524E+01) -- (axis cs:5.77962E+01,2.02479E+01) -- cycle  ;
\draw [ecliptics-full] (axis cs:5.88393E+01,2.04568E+01) -- (axis cs:5.88831E+01,2.02610E+01)  -- (axis cs:5.99277E+01,2.04635E+01) -- (axis cs:5.98851E+01,2.06595E+01) -- cycle  ;
\draw [ecliptics-full] (axis cs:6.09337E+01,2.08559E+01) -- (axis cs:6.09750E+01,2.06597E+01)  -- (axis cs:6.20250E+01,2.08495E+01) -- (axis cs:6.19850E+01,2.10459E+01) -- cycle  ;
\draw [ecliptics-full] (axis cs:6.30389E+01,2.12295E+01) -- (axis cs:6.30775E+01,2.10328E+01)  -- (axis cs:6.41327E+01,2.12096E+01) -- (axis cs:6.40954E+01,2.14066E+01) -- cycle  ;
\draw [ecliptics-full] (axis cs:6.51544E+01,2.15771E+01) -- (axis cs:6.51903E+01,2.13799E+01)  -- (axis cs:6.62503E+01,2.15434E+01) -- (axis cs:6.62158E+01,2.17408E+01) -- cycle  ;
\draw [ecliptics-full] (axis cs:6.72796E+01,2.18979E+01) -- (axis cs:6.73127E+01,2.17002E+01)  -- (axis cs:6.83773E+01,2.18502E+01) -- (axis cs:6.83457E+01,2.20481E+01) -- cycle  ;
\draw [ecliptics-full] (axis cs:6.94140E+01,2.21914E+01) -- (axis cs:6.94442E+01,2.19934E+01)  -- (axis cs:7.05131E+01,2.21296E+01) -- (axis cs:7.04845E+01,2.23278E+01) -- cycle  ;
\draw [ecliptics-full] (axis cs:7.15569E+01,2.24572E+01) -- (axis cs:7.15841E+01,2.22587E+01)  -- (axis cs:7.26570E+01,2.23809E+01) -- (axis cs:7.26313E+01,2.25795E+01) -- cycle  ;
\draw [ecliptics-full] (axis cs:7.37076E+01,2.26946E+01) -- (axis cs:7.37317E+01,2.24959E+01)  -- (axis cs:7.48082E+01,2.26037E+01) -- (axis cs:7.47856E+01,2.28026E+01) -- cycle  ;
\draw [ecliptics-full] (axis cs:7.58653E+01,2.29034E+01) -- (axis cs:7.58863E+01,2.27043E+01)  -- (axis cs:7.69660E+01,2.27977E+01) -- (axis cs:7.69465E+01,2.29969E+01) -- cycle  ;
\draw [ecliptics-full] (axis cs:7.80291E+01,2.30830E+01) -- (axis cs:7.80470E+01,2.28837E+01)  -- (axis cs:7.91294E+01,2.29624E+01) -- (axis cs:7.91131E+01,2.31619E+01) -- cycle  ;
\draw [ecliptics-full] (axis cs:8.01982E+01,2.32333E+01) -- (axis cs:8.02130E+01,2.30338E+01)  -- (axis cs:8.12976E+01,2.30977E+01) -- (axis cs:8.12845E+01,2.32973E+01) -- cycle  ;
\draw [ecliptics-full] (axis cs:8.23718E+01,2.33538E+01) -- (axis cs:8.23833E+01,2.31541E+01)  -- (axis cs:8.34698E+01,2.32031E+01) -- (axis cs:8.34599E+01,2.34029E+01) -- cycle  ;
\draw [ecliptics-full] (axis cs:8.45488E+01,2.34444E+01) -- (axis cs:8.45570E+01,2.32446E+01)  -- (axis cs:8.56449E+01,2.32785E+01) -- (axis cs:8.56383E+01,2.34785E+01) -- cycle  ;
\draw [ecliptics-full] (axis cs:8.67283E+01,2.35049E+01) -- (axis cs:8.67332E+01,2.33050E+01)  -- (axis cs:8.78219E+01,2.33239E+01) -- (axis cs:8.78186E+01,2.35239E+01) -- cycle  ;
\draw [ecliptics-full] (axis cs:8.89093E+01,2.35352E+01) -- (axis cs:8.89109E+01,2.33352E+01)  -- (axis cs:9.00000E+01,2.33390E+01) -- (axis cs:9.00000E+01,2.35390E+01) -- cycle  ;
\draw [ecliptics-full] (axis cs:9.10907E+01,2.35352E+01) -- (axis cs:9.10891E+01,2.33352E+01)  -- (axis cs:9.21781E+01,2.33239E+01) -- (axis cs:9.21814E+01,2.35239E+01) -- cycle  ;
\draw [ecliptics-full] (axis cs:9.32717E+01,2.35049E+01) -- (axis cs:9.32668E+01,2.33050E+01)  -- (axis cs:9.43551E+01,2.32785E+01) -- (axis cs:9.43617E+01,2.34785E+01) -- cycle  ;
\draw [ecliptics-full] (axis cs:9.54512E+01,2.34444E+01) -- (axis cs:9.54430E+01,2.32446E+01)  -- (axis cs:9.65302E+01,2.32031E+01) -- (axis cs:9.65401E+01,2.34029E+01) -- cycle  ;
\draw [ecliptics-full] (axis cs:9.76282E+01,2.33538E+01) -- (axis cs:9.76167E+01,2.31541E+01)  -- (axis cs:9.87024E+01,2.30977E+01) -- (axis cs:9.87155E+01,2.32973E+01) -- cycle  ;
\draw [ecliptics-full] (axis cs:9.98018E+01,2.32333E+01) -- (axis cs:9.97870E+01,2.30338E+01)  -- (axis cs:1.00871E+02,2.29624E+01) -- (axis cs:1.00887E+02,2.31619E+01) -- cycle  ;
\draw [ecliptics-full] (axis cs:1.01971E+02,2.30830E+01) -- (axis cs:1.01953E+02,2.28837E+01)  -- (axis cs:1.03034E+02,2.27977E+01) -- (axis cs:1.03054E+02,2.29969E+01) -- cycle  ;
\draw [ecliptics-full] (axis cs:1.04135E+02,2.29034E+01) -- (axis cs:1.04114E+02,2.27043E+01)  -- (axis cs:1.05192E+02,2.26037E+01) -- (axis cs:1.05214E+02,2.28026E+01) -- cycle  ;
\draw [ecliptics-full] (axis cs:1.06292E+02,2.26946E+01) -- (axis cs:1.06268E+02,2.24959E+01)  -- (axis cs:1.07343E+02,2.23809E+01) -- (axis cs:1.07369E+02,2.25795E+01) -- cycle  ;
\draw [ecliptics-full] (axis cs:1.08443E+02,2.24572E+01) -- (axis cs:1.08416E+02,2.22587E+01)  -- (axis cs:1.09487E+02,2.21296E+01) -- (axis cs:1.09516E+02,2.23278E+01) -- cycle  ;
\draw [ecliptics-full] (axis cs:1.10586E+02,2.21914E+01) -- (axis cs:1.10556E+02,2.19934E+01)  -- (axis cs:1.11623E+02,2.18502E+01) -- (axis cs:1.11654E+02,2.20481E+01) -- cycle  ;
\draw [ecliptics-full] (axis cs:1.12720E+02,2.18979E+01) -- (axis cs:1.12687E+02,2.17002E+01)  -- (axis cs:1.13750E+02,2.15434E+01) -- (axis cs:1.13784E+02,2.17408E+01) -- cycle  ;
\draw [ecliptics-full] (axis cs:1.14846E+02,2.15771E+01) -- (axis cs:1.14810E+02,2.13799E+01)  -- (axis cs:1.15867E+02,2.12096E+01) -- (axis cs:1.15905E+02,2.14066E+01) -- cycle  ;
\draw [ecliptics-full] (axis cs:1.16961E+02,2.12295E+01) -- (axis cs:1.16922E+02,2.10328E+01)  -- (axis cs:1.17975E+02,2.08495E+01) -- (axis cs:1.18015E+02,2.10459E+01) -- cycle  ;
\draw [ecliptics-full] (axis cs:1.19066E+02,2.08559E+01) -- (axis cs:1.19025E+02,2.06597E+01)  -- (axis cs:1.20072E+02,2.04635E+01) -- (axis cs:1.20115E+02,2.06595E+01) -- cycle  ;
\draw [ecliptics-full] (axis cs:1.21161E+02,2.04568E+01) -- (axis cs:1.21117E+02,2.02610E+01)  -- (axis cs:1.22159E+02,2.00524E+01) -- (axis cs:1.22204E+02,2.02479E+01) -- cycle  ;
\draw [ecliptics-full] (axis cs:1.23244E+02,2.00328E+01) -- (axis cs:1.23198E+02,1.98376E+01)  -- (axis cs:1.24234E+02,1.96168E+01) -- (axis cs:1.24281E+02,1.98118E+01) -- cycle  ;
\draw [ecliptics-full] (axis cs:1.25316E+02,1.95848E+01) -- (axis cs:1.25267E+02,1.93901E+01)  -- (axis cs:1.26297E+02,1.91575E+01) -- (axis cs:1.26347E+02,1.93519E+01) -- cycle  ;
\draw [ecliptics-full] (axis cs:1.27376E+02,1.91133E+01) -- (axis cs:1.27325E+02,1.89192E+01)  -- (axis cs:1.28349E+02,1.86752E+01) -- (axis cs:1.28401E+02,1.88690E+01) -- cycle  ;
\draw [ecliptics-full] (axis cs:1.29424E+02,1.86192E+01) -- (axis cs:1.29371E+02,1.84256E+01)  -- (axis cs:1.30389E+02,1.81706E+01) -- (axis cs:1.30443E+02,1.83638E+01) -- cycle  ;
\draw [ecliptics-full] (axis cs:1.31460E+02,1.81031E+01) -- (axis cs:1.31404E+02,1.79102E+01)  -- (axis cs:1.32417E+02,1.76445E+01) -- (axis cs:1.32473E+02,1.78371E+01) -- cycle  ;
\draw [ecliptics-full] (axis cs:1.33484E+02,1.75660E+01) -- (axis cs:1.33426E+02,1.73736E+01)  -- (axis cs:1.34433E+02,1.70976E+01) -- (axis cs:1.34491E+02,1.72897E+01) -- cycle  ;
\draw [ecliptics-full] (axis cs:1.35495E+02,1.70085E+01) -- (axis cs:1.35436E+02,1.68167E+01)  -- (axis cs:1.36436E+02,1.65309E+01) -- (axis cs:1.36496E+02,1.67224E+01) -- cycle  ;
\draw [ecliptics-full] (axis cs:1.37495E+02,1.64315E+01) -- (axis cs:1.37434E+02,1.62403E+01)  -- (axis cs:1.38428E+02,1.59450E+01) -- (axis cs:1.38490E+02,1.61359E+01) -- cycle  ;
\draw [ecliptics-full] (axis cs:1.39482E+02,1.58358E+01) -- (axis cs:1.39419E+02,1.56451E+01)  -- (axis cs:1.40408E+02,1.53408E+01) -- (axis cs:1.40471E+02,1.55312E+01) -- cycle  ;
\draw [ecliptics-full] (axis cs:1.41458E+02,1.52222E+01) -- (axis cs:1.41393E+02,1.50321E+01)  -- (axis cs:1.42376E+02,1.47192E+01) -- (axis cs:1.42441E+02,1.49090E+01) -- cycle  ;
\draw [ecliptics-full] (axis cs:1.43422E+02,1.45916E+01) -- (axis cs:1.43356E+02,1.44021E+01)  -- (axis cs:1.44333E+02,1.40809E+01) -- (axis cs:1.44400E+02,1.42701E+01) -- cycle  ;
\draw [ecliptics-full] (axis cs:1.45375E+02,1.39448E+01) -- (axis cs:1.45307E+02,1.37558E+01)  -- (axis cs:1.46279E+02,1.34268E+01) -- (axis cs:1.46347E+02,1.36155E+01) -- cycle  ;
\draw [ecliptics-full] (axis cs:1.47317E+02,1.32826E+01) -- (axis cs:1.47248E+02,1.30941E+01)  -- (axis cs:1.48214E+02,1.27578E+01) -- (axis cs:1.48283E+02,1.29460E+01) -- cycle  ;
\draw [ecliptics-full] (axis cs:1.49248E+02,1.26059E+01) -- (axis cs:1.49178E+02,1.24179E+01)  -- (axis cs:1.50139E+02,1.20746E+01) -- (axis cs:1.50209E+02,1.22623E+01) -- cycle  ;
\draw [ecliptics-full] (axis cs:1.51169E+02,1.19155E+01) -- (axis cs:1.51097E+02,1.17280E+01)  -- (axis cs:1.52054E+02,1.13782E+01) -- (axis cs:1.52125E+02,1.15655E+01) -- cycle  ;
\draw [ecliptics-full] (axis cs:1.53080E+02,1.12123E+01) -- (axis cs:1.53007E+02,1.10253E+01)  -- (axis cs:1.53959E+02,1.06694E+01) -- (axis cs:1.54032E+02,1.08562E+01) -- cycle  ;
\draw [ecliptics-full] (axis cs:1.54981E+02,1.04971E+01) -- (axis cs:1.54908E+02,1.03105E+01)  -- (axis cs:1.55855E+02,9.94894E+00) -- (axis cs:1.55929E+02,1.01353E+01) -- cycle  ;
\draw [ecliptics-full] (axis cs:1.56874E+02,9.77078E+00) -- (axis cs:1.56800E+02,9.58464E+00)  -- (axis cs:1.57743E+02,9.21776E+00) -- (axis cs:1.57818E+02,9.40370E+00) -- cycle  ;
\draw [ecliptics-full] (axis cs:1.58759E+02,9.03415E+00) -- (axis cs:1.58684E+02,8.84840E+00)  -- (axis cs:1.59623E+02,8.47667E+00) -- (axis cs:1.59699E+02,8.66224E+00) -- cycle  ;
\draw [ecliptics-full] (axis cs:1.60636E+02,8.28806E+00) -- (axis cs:1.60560E+02,8.10267E+00)  -- (axis cs:1.61496E+02,7.72651E+00) -- (axis cs:1.61572E+02,7.91173E+00) -- cycle  ;
\draw [ecliptics-full] (axis cs:1.62506E+02,7.53334E+00) -- (axis cs:1.62430E+02,7.34829E+00)  -- (axis cs:1.63362E+02,6.96811E+00) -- (axis cs:1.63439E+02,7.15301E+00) -- cycle  ;
\draw [ecliptics-full] (axis cs:1.64370E+02,6.77084E+00) -- (axis cs:1.64292E+02,6.58609E+00)  -- (axis cs:1.65222E+02,6.20231E+00) -- (axis cs:1.65299E+02,6.38692E+00) -- cycle  ;
\draw [ecliptics-full] (axis cs:1.66227E+02,6.00137E+00) -- (axis cs:1.66150E+02,5.81689E+00)  -- (axis cs:1.67076E+02,5.42993E+00) -- (axis cs:1.67154E+02,5.61429E+00) -- cycle  ;
\draw [ecliptics-full] (axis cs:1.68080E+02,5.22577E+00) -- (axis cs:1.68001E+02,5.04154E+00)  -- (axis cs:1.68926E+02,4.65180E+00) -- (axis cs:1.69004E+02,4.83593E+00) -- cycle  ;
\draw [ecliptics-full] (axis cs:1.69927E+02,4.44487E+00) -- (axis cs:1.69849E+02,4.26084E+00)  -- (axis cs:1.70771E+02,3.86875E+00) -- (axis cs:1.70850E+02,4.05268E+00) -- cycle  ;
\draw [ecliptics-full] (axis cs:1.71771E+02,3.65948E+00) -- (axis cs:1.71692E+02,3.47563E+00)  -- (axis cs:1.72613E+02,3.08158E+00) -- (axis cs:1.72692E+02,3.26536E+00) -- cycle  ;
\draw [ecliptics-full] (axis cs:1.73612E+02,2.87043E+00) -- (axis cs:1.73533E+02,2.68672E+00)  -- (axis cs:1.74452E+02,2.29113E+00) -- (axis cs:1.74532E+02,2.47479E+00) -- cycle  ;
\draw [ecliptics-full] (axis cs:1.75450E+02,2.07854E+00) -- (axis cs:1.75371E+02,1.89493E+00)  -- (axis cs:1.76289E+02,1.49822E+00) -- (axis cs:1.76369E+02,1.68179E+00) -- cycle  ;
\draw [ecliptics-full] (axis cs:1.77287E+02,1.28463E+00) -- (axis cs:1.77207E+02,1.10109E+00)  -- (axis cs:1.78125E+02,7.03652E-01) -- (axis cs:1.78205E+02,8.87166E-01) -- cycle  ;
\draw [ecliptics-full] (axis cs:1.79122E+02,4.89506E-01) -- (axis cs:1.79043E+02,3.06005E-01)  -- (axis cs:1.79960E+02,-9.17483E-02) -- (axis cs:1.80040E+02,9.17485E-02) -- cycle ;




\draw [ecliptics-full] (axis cs:3.58125E+02,-7.03651E-01) -- (axis cs:3.58205E+02,-8.87166E-01)  -- (axis cs:3.57287E+02,-1.28463E+00) -- (axis cs:3.57207E+02,-1.10109E+00) -- cycle  ;
\draw [ecliptics-full] (axis cs:3.56289E+02,-1.49822E+00) -- (axis cs:3.56369E+02,-1.68179E+00)  -- (axis cs:3.55450E+02,-2.07854E+00) -- (axis cs:3.55371E+02,-1.89493E+00) -- cycle  ;
\draw [ecliptics-full] (axis cs:3.54452E+02,-2.29113E+00) -- (axis cs:3.54532E+02,-2.47479E+00)  -- (axis cs:3.53612E+02,-2.87043E+00) -- (axis cs:3.53533E+02,-2.68672E+00) -- cycle  ;
\draw [ecliptics-full] (axis cs:3.52613E+02,-3.08158E+00) -- (axis cs:3.52692E+02,-3.26536E+00)  -- (axis cs:3.51771E+02,-3.65948E+00) -- (axis cs:3.51693E+02,-3.47563E+00) -- cycle  ;
\draw [ecliptics-full] (axis cs:3.50771E+02,-3.86875E+00) -- (axis cs:3.50850E+02,-4.05268E+00)  -- (axis cs:3.49927E+02,-4.44487E+00) -- (axis cs:3.49849E+02,-4.26084E+00) -- cycle  ;
\draw [ecliptics-full] (axis cs:3.48926E+02,-4.65180E+00) -- (axis cs:3.49004E+02,-4.83593E+00)  -- (axis cs:3.48080E+02,-5.22577E+00) -- (axis cs:3.48001E+02,-5.04154E+00) -- cycle  ;
\draw [ecliptics-full] (axis cs:3.47076E+02,-5.42993E+00) -- (axis cs:3.47154E+02,-5.61429E+00)  -- (axis cs:3.46227E+02,-6.00137E+00) -- (axis cs:3.46150E+02,-5.81689E+00) -- cycle  ;
\draw [ecliptics-full] (axis cs:3.45222E+02,-6.20231E+00) -- (axis cs:3.45299E+02,-6.38692E+00)  -- (axis cs:3.44370E+02,-6.77084E+00) -- (axis cs:3.44292E+02,-6.58609E+00) -- cycle  ;
\draw [ecliptics-full] (axis cs:3.43362E+02,-6.96811E+00) -- (axis cs:3.43439E+02,-7.15301E+00)  -- (axis cs:3.42506E+02,-7.53334E+00) -- (axis cs:3.42430E+02,-7.34829E+00) -- cycle  ;
\draw [ecliptics-full] (axis cs:3.41496E+02,-7.72651E+00) -- (axis cs:3.41572E+02,-7.91173E+00)  -- (axis cs:3.40636E+02,-8.28806E+00) -- (axis cs:3.40560E+02,-8.10267E+00) -- cycle  ;
\draw [ecliptics-full] (axis cs:3.39623E+02,-8.47667E+00) -- (axis cs:3.39699E+02,-8.66224E+00)  -- (axis cs:3.38759E+02,-9.03415E+00) -- (axis cs:3.38684E+02,-8.84840E+00) -- cycle  ;
\draw [ecliptics-full] (axis cs:3.37743E+02,-9.21776E+00) -- (axis cs:3.37818E+02,-9.40370E+00)  -- (axis cs:3.36874E+02,-9.77078E+00) -- (axis cs:3.36800E+02,-9.58464E+00) -- cycle  ;
\draw [ecliptics-full] (axis cs:3.35855E+02,-9.94894E+00) -- (axis cs:3.35929E+02,-1.01353E+01)  -- (axis cs:3.34981E+02,-1.04971E+01) -- (axis cs:3.34908E+02,-1.03105E+01) -- cycle  ;
\draw [ecliptics-full] (axis cs:3.33959E+02,-1.06694E+01) -- (axis cs:3.34032E+02,-1.08562E+01)  -- (axis cs:3.33080E+02,-1.12123E+01) -- (axis cs:3.33007E+02,-1.10253E+01) -- cycle  ;
\draw [ecliptics-full] (axis cs:3.32054E+02,-1.13782E+01) -- (axis cs:3.32125E+02,-1.15655E+01)  -- (axis cs:3.31169E+02,-1.19155E+01) -- (axis cs:3.31097E+02,-1.17280E+01) -- cycle  ;
\draw [ecliptics-full] (axis cs:3.30139E+02,-1.20746E+01) -- (axis cs:3.30209E+02,-1.22623E+01)  -- (axis cs:3.29248E+02,-1.26059E+01) -- (axis cs:3.29178E+02,-1.24179E+01) -- cycle  ;
\draw [ecliptics-full] (axis cs:3.28214E+02,-1.27578E+01) -- (axis cs:3.28283E+02,-1.29460E+01)  -- (axis cs:3.27316E+02,-1.32826E+01) -- (axis cs:3.27248E+02,-1.30941E+01) -- cycle  ;
\draw [ecliptics-full] (axis cs:3.26279E+02,-1.34268E+01) -- (axis cs:3.26347E+02,-1.36155E+01)  -- (axis cs:3.25375E+02,-1.39448E+01) -- (axis cs:3.25307E+02,-1.37558E+01) -- cycle  ;
\draw [ecliptics-full] (axis cs:3.24333E+02,-1.40809E+01) -- (axis cs:3.24400E+02,-1.42701E+01)  -- (axis cs:3.23422E+02,-1.45916E+01) -- (axis cs:3.23356E+02,-1.44021E+01) -- cycle  ;
\draw [ecliptics-full] (axis cs:3.22376E+02,-1.47192E+01) -- (axis cs:3.22441E+02,-1.49090E+01)  -- (axis cs:3.21458E+02,-1.52222E+01) -- (axis cs:3.21393E+02,-1.50321E+01) -- cycle  ;
\draw [ecliptics-full] (axis cs:3.20408E+02,-1.53408E+01) -- (axis cs:3.20471E+02,-1.55312E+01)  -- (axis cs:3.19482E+02,-1.58358E+01) -- (axis cs:3.19419E+02,-1.56451E+01) -- cycle  ;
\draw [ecliptics-full] (axis cs:3.18428E+02,-1.59450E+01) -- (axis cs:3.18490E+02,-1.61359E+01)  -- (axis cs:3.17495E+02,-1.64315E+01) -- (axis cs:3.17434E+02,-1.62403E+01) -- cycle  ;
\draw [ecliptics-full] (axis cs:3.16436E+02,-1.65309E+01) -- (axis cs:3.16496E+02,-1.67224E+01)  -- (axis cs:3.15495E+02,-1.70085E+01) -- (axis cs:3.15436E+02,-1.68167E+01) -- cycle  ;
\draw [ecliptics-full] (axis cs:3.14433E+02,-1.70976E+01) -- (axis cs:3.14491E+02,-1.72897E+01)  -- (axis cs:3.13484E+02,-1.75660E+01) -- (axis cs:3.13426E+02,-1.73736E+01) -- cycle  ;
\draw [ecliptics-full] (axis cs:3.12417E+02,-1.76445E+01) -- (axis cs:3.12473E+02,-1.78371E+01)  -- (axis cs:3.11460E+02,-1.81031E+01) -- (axis cs:3.11404E+02,-1.79102E+01) -- cycle  ;
\draw [ecliptics-full] (axis cs:3.10389E+02,-1.81706E+01) -- (axis cs:3.10443E+02,-1.83638E+01)  -- (axis cs:3.09424E+02,-1.86192E+01) -- (axis cs:3.09371E+02,-1.84256E+01) -- cycle  ;
\draw [ecliptics-full] (axis cs:3.08349E+02,-1.86752E+01) -- (axis cs:3.08401E+02,-1.88690E+01)  -- (axis cs:3.07376E+02,-1.91133E+01) -- (axis cs:3.07325E+02,-1.89192E+01) -- cycle  ;
\draw [ecliptics-full] (axis cs:3.06297E+02,-1.91575E+01) -- (axis cs:3.06347E+02,-1.93519E+01)  -- (axis cs:3.05316E+02,-1.95848E+01) -- (axis cs:3.05267E+02,-1.93901E+01) -- cycle  ;
\draw [ecliptics-full] (axis cs:3.04234E+02,-1.96168E+01) -- (axis cs:3.04281E+02,-1.98118E+01)  -- (axis cs:3.03244E+02,-2.00328E+01) -- (axis cs:3.03198E+02,-1.98376E+01) -- cycle  ;
\draw [ecliptics-full] (axis cs:3.02159E+02,-2.00524E+01) -- (axis cs:3.02204E+02,-2.02479E+01)  -- (axis cs:3.01161E+02,-2.04568E+01) -- (axis cs:3.01117E+02,-2.02610E+01) -- cycle  ;
\draw [ecliptics-full] (axis cs:3.00072E+02,-2.04635E+01) -- (axis cs:3.00115E+02,-2.06595E+01)  -- (axis cs:2.99066E+02,-2.08559E+01) -- (axis cs:2.99025E+02,-2.06597E+01) -- cycle  ;
\draw [ecliptics-full] (axis cs:2.97975E+02,-2.08495E+01) -- (axis cs:2.98015E+02,-2.10459E+01)  -- (axis cs:2.96961E+02,-2.12295E+01) -- (axis cs:2.96922E+02,-2.10328E+01) -- cycle  ;
\draw [ecliptics-full] (axis cs:2.95867E+02,-2.12096E+01) -- (axis cs:2.95905E+02,-2.14066E+01)  -- (axis cs:2.94846E+02,-2.15771E+01) -- (axis cs:2.94810E+02,-2.13799E+01) -- cycle  ;
\draw [ecliptics-full] (axis cs:2.93750E+02,-2.15434E+01) -- (axis cs:2.93784E+02,-2.17408E+01)  -- (axis cs:2.92720E+02,-2.18979E+01) -- (axis cs:2.92687E+02,-2.17002E+01) -- cycle  ;
\draw [ecliptics-full] (axis cs:2.91623E+02,-2.18502E+01) -- (axis cs:2.91654E+02,-2.20481E+01)  -- (axis cs:2.90586E+02,-2.21914E+01) -- (axis cs:2.90556E+02,-2.19934E+01) -- cycle  ;
\draw [ecliptics-full] (axis cs:2.89487E+02,-2.21296E+01) -- (axis cs:2.89516E+02,-2.23278E+01)  -- (axis cs:2.88443E+02,-2.24572E+01) -- (axis cs:2.88416E+02,-2.22587E+01) -- cycle  ;
\draw [ecliptics-full] (axis cs:2.87343E+02,-2.23809E+01) -- (axis cs:2.87369E+02,-2.25795E+01)  -- (axis cs:2.86292E+02,-2.26946E+01) -- (axis cs:2.86268E+02,-2.24959E+01) -- cycle  ;
\draw [ecliptics-full] (axis cs:2.85192E+02,-2.26037E+01) -- (axis cs:2.85214E+02,-2.28026E+01)  -- (axis cs:2.84135E+02,-2.29034E+01) -- (axis cs:2.84114E+02,-2.27043E+01) -- cycle  ;
\draw [ecliptics-full] (axis cs:2.83034E+02,-2.27977E+01) -- (axis cs:2.83054E+02,-2.29969E+01)  -- (axis cs:2.81971E+02,-2.30830E+01) -- (axis cs:2.81953E+02,-2.28837E+01) -- cycle  ;
\draw [ecliptics-full] (axis cs:2.80871E+02,-2.29624E+01) -- (axis cs:2.80887E+02,-2.31619E+01)  -- (axis cs:2.79802E+02,-2.32333E+01) -- (axis cs:2.79787E+02,-2.30338E+01) -- cycle  ;
\draw [ecliptics-full] (axis cs:2.78702E+02,-2.30977E+01) -- (axis cs:2.78715E+02,-2.32973E+01)  -- (axis cs:2.77628E+02,-2.33538E+01) -- (axis cs:2.77617E+02,-2.31541E+01) -- cycle  ;
\draw [ecliptics-full] (axis cs:2.76530E+02,-2.32031E+01) -- (axis cs:2.76540E+02,-2.34029E+01)  -- (axis cs:2.75451E+02,-2.34444E+01) -- (axis cs:2.75443E+02,-2.32446E+01) -- cycle  ;
\draw [ecliptics-full] (axis cs:2.74355E+02,-2.32785E+01) -- (axis cs:2.74362E+02,-2.34785E+01)  -- (axis cs:2.73272E+02,-2.35049E+01) -- (axis cs:2.73267E+02,-2.33050E+01) -- cycle  ;
\draw [ecliptics-full] (axis cs:2.72178E+02,-2.33239E+01) -- (axis cs:2.72181E+02,-2.35239E+01)  -- (axis cs:2.71091E+02,-2.35352E+01) -- (axis cs:2.71089E+02,-2.33352E+01) -- cycle  ;
\draw [ecliptics-full] (axis cs:2.70000E+02,-2.33390E+01) -- (axis cs:2.70000E+02,-2.35390E+01)  -- (axis cs:2.68909E+02,-2.35352E+01) -- (axis cs:2.68911E+02,-2.33352E+01) -- cycle  ;
\draw [ecliptics-full] (axis cs:2.67822E+02,-2.33239E+01) -- (axis cs:2.67819E+02,-2.35239E+01)  -- (axis cs:2.66728E+02,-2.35049E+01) -- (axis cs:2.66733E+02,-2.33050E+01) -- cycle  ;
\draw [ecliptics-full] (axis cs:2.65645E+02,-2.32785E+01) -- (axis cs:2.65638E+02,-2.34785E+01)  -- (axis cs:2.64549E+02,-2.34444E+01) -- (axis cs:2.64557E+02,-2.32446E+01) -- cycle  ;
\draw [ecliptics-full] (axis cs:2.63470E+02,-2.32031E+01) -- (axis cs:2.63460E+02,-2.34029E+01)  -- (axis cs:2.62372E+02,-2.33538E+01) -- (axis cs:2.62383E+02,-2.31541E+01) -- cycle  ;
\draw [ecliptics-full] (axis cs:2.61298E+02,-2.30977E+01) -- (axis cs:2.61285E+02,-2.32973E+01)  -- (axis cs:2.60198E+02,-2.32333E+01) -- (axis cs:2.60213E+02,-2.30338E+01) -- cycle  ;
\draw [ecliptics-full] (axis cs:2.59129E+02,-2.29624E+01) -- (axis cs:2.59113E+02,-2.31619E+01)  -- (axis cs:2.58029E+02,-2.30830E+01) -- (axis cs:2.58047E+02,-2.28837E+01) -- cycle  ;
\draw [ecliptics-full] (axis cs:2.56966E+02,-2.27977E+01) -- (axis cs:2.56946E+02,-2.29969E+01)  -- (axis cs:2.55865E+02,-2.29034E+01) -- (axis cs:2.55886E+02,-2.27043E+01) -- cycle  ;
\draw [ecliptics-full] (axis cs:2.54808E+02,-2.26037E+01) -- (axis cs:2.54786E+02,-2.28026E+01)  -- (axis cs:2.53708E+02,-2.26946E+01) -- (axis cs:2.53732E+02,-2.24959E+01) -- cycle  ;
\draw [ecliptics-full] (axis cs:2.52657E+02,-2.23809E+01) -- (axis cs:2.52631E+02,-2.25795E+01)  -- (axis cs:2.51557E+02,-2.24572E+01) -- (axis cs:2.51584E+02,-2.22587E+01) -- cycle  ;
\draw [ecliptics-full] (axis cs:2.50513E+02,-2.21296E+01) -- (axis cs:2.50484E+02,-2.23278E+01)  -- (axis cs:2.49414E+02,-2.21914E+01) -- (axis cs:2.49444E+02,-2.19934E+01) -- cycle  ;
\draw [ecliptics-full] (axis cs:2.48377E+02,-2.18502E+01) -- (axis cs:2.48346E+02,-2.20481E+01)  -- (axis cs:2.47280E+02,-2.18979E+01) -- (axis cs:2.47313E+02,-2.17002E+01) -- cycle  ;
\draw [ecliptics-full] (axis cs:2.46250E+02,-2.15434E+01) -- (axis cs:2.46216E+02,-2.17408E+01)  -- (axis cs:2.45154E+02,-2.15771E+01) -- (axis cs:2.45190E+02,-2.13799E+01) -- cycle  ;
\draw [ecliptics-full] (axis cs:2.44133E+02,-2.12096E+01) -- (axis cs:2.44095E+02,-2.14066E+01)  -- (axis cs:2.43039E+02,-2.12295E+01) -- (axis cs:2.43078E+02,-2.10328E+01) -- cycle  ;
\draw [ecliptics-full] (axis cs:2.42025E+02,-2.08495E+01) -- (axis cs:2.41985E+02,-2.10459E+01)  -- (axis cs:2.40934E+02,-2.08559E+01) -- (axis cs:2.40975E+02,-2.06597E+01) -- cycle  ;
\draw [ecliptics-full] (axis cs:2.39928E+02,-2.04635E+01) -- (axis cs:2.39885E+02,-2.06595E+01)  -- (axis cs:2.38839E+02,-2.04568E+01) -- (axis cs:2.38883E+02,-2.02610E+01) -- cycle  ;
\draw [ecliptics-full] (axis cs:2.37841E+02,-2.00524E+01) -- (axis cs:2.37796E+02,-2.02479E+01)  -- (axis cs:2.36756E+02,-2.00328E+01) -- (axis cs:2.36802E+02,-1.98376E+01) -- cycle  ;
\draw [ecliptics-full] (axis cs:2.35766E+02,-1.96168E+01) -- (axis cs:2.35719E+02,-1.98118E+01)  -- (axis cs:2.34684E+02,-1.95848E+01) -- (axis cs:2.34733E+02,-1.93901E+01) -- cycle  ;
\draw [ecliptics-full] (axis cs:2.33703E+02,-1.91575E+01) -- (axis cs:2.33653E+02,-1.93519E+01)  -- (axis cs:2.32624E+02,-1.91133E+01) -- (axis cs:2.32675E+02,-1.89192E+01) -- cycle  ;
\draw [ecliptics-full] (axis cs:2.31651E+02,-1.86752E+01) -- (axis cs:2.31599E+02,-1.88690E+01)  -- (axis cs:2.30576E+02,-1.86192E+01) -- (axis cs:2.30629E+02,-1.84256E+01) -- cycle  ;
\draw [ecliptics-full] (axis cs:2.29611E+02,-1.81706E+01) -- (axis cs:2.29557E+02,-1.83638E+01)  -- (axis cs:2.28540E+02,-1.81031E+01) -- (axis cs:2.28596E+02,-1.79102E+01) -- cycle  ;
\draw [ecliptics-full] (axis cs:2.27583E+02,-1.76445E+01) -- (axis cs:2.27527E+02,-1.78371E+01)  -- (axis cs:2.26516E+02,-1.75660E+01) -- (axis cs:2.26574E+02,-1.73736E+01) -- cycle  ;
\draw [ecliptics-full] (axis cs:2.25567E+02,-1.70976E+01) -- (axis cs:2.25509E+02,-1.72897E+01)  -- (axis cs:2.24505E+02,-1.70085E+01) -- (axis cs:2.24564E+02,-1.68167E+01) -- cycle  ;
\draw [ecliptics-full] (axis cs:2.23564E+02,-1.65309E+01) -- (axis cs:2.23504E+02,-1.67224E+01)  -- (axis cs:2.22505E+02,-1.64315E+01) -- (axis cs:2.22566E+02,-1.62403E+01) -- cycle  ;
\draw [ecliptics-full] (axis cs:2.21572E+02,-1.59450E+01) -- (axis cs:2.21510E+02,-1.61359E+01)  -- (axis cs:2.20518E+02,-1.58358E+01) -- (axis cs:2.20581E+02,-1.56451E+01) -- cycle  ;
\draw [ecliptics-full] (axis cs:2.19592E+02,-1.53408E+01) -- (axis cs:2.19529E+02,-1.55312E+01)  -- (axis cs:2.18542E+02,-1.52222E+01) -- (axis cs:2.18607E+02,-1.50321E+01) -- cycle  ;
\draw [ecliptics-full] (axis cs:2.17624E+02,-1.47192E+01) -- (axis cs:2.17559E+02,-1.49090E+01)  -- (axis cs:2.16578E+02,-1.45916E+01) -- (axis cs:2.16644E+02,-1.44021E+01) -- cycle  ;
\draw [ecliptics-full] (axis cs:2.15667E+02,-1.40809E+01) -- (axis cs:2.15600E+02,-1.42701E+01)  -- (axis cs:2.14625E+02,-1.39448E+01) -- (axis cs:2.14693E+02,-1.37558E+01) -- cycle  ;
\draw [ecliptics-full] (axis cs:2.13721E+02,-1.34268E+01) -- (axis cs:2.13653E+02,-1.36155E+01)  -- (axis cs:2.12683E+02,-1.32826E+01) -- (axis cs:2.12752E+02,-1.30941E+01) -- cycle  ;
\draw [ecliptics-full] (axis cs:2.11786E+02,-1.27578E+01) -- (axis cs:2.11717E+02,-1.29460E+01)  -- (axis cs:2.10752E+02,-1.26059E+01) -- (axis cs:2.10822E+02,-1.24179E+01) -- cycle  ;
\draw [ecliptics-full] (axis cs:2.09861E+02,-1.20746E+01) -- (axis cs:2.09791E+02,-1.22623E+01)  -- (axis cs:2.08831E+02,-1.19155E+01) -- (axis cs:2.08903E+02,-1.17280E+01) -- cycle  ;
\draw [ecliptics-full] (axis cs:2.07946E+02,-1.13782E+01) -- (axis cs:2.07875E+02,-1.15655E+01)  -- (axis cs:2.06920E+02,-1.12123E+01) -- (axis cs:2.06993E+02,-1.10253E+01) -- cycle  ;
\draw [ecliptics-full] (axis cs:2.06041E+02,-1.06694E+01) -- (axis cs:2.05968E+02,-1.08562E+01)  -- (axis cs:2.05019E+02,-1.04971E+01) -- (axis cs:2.05092E+02,-1.03105E+01) -- cycle  ;
\draw [ecliptics-full] (axis cs:2.04145E+02,-9.94894E+00) -- (axis cs:2.04071E+02,-1.01353E+01)  -- (axis cs:2.03126E+02,-9.77078E+00) -- (axis cs:2.03200E+02,-9.58464E+00) -- cycle  ;
\draw [ecliptics-full] (axis cs:2.02257E+02,-9.21776E+00) -- (axis cs:2.02182E+02,-9.40370E+00)  -- (axis cs:2.01241E+02,-9.03415E+00) -- (axis cs:2.01316E+02,-8.84840E+00) -- cycle  ;
\draw [ecliptics-full] (axis cs:2.00377E+02,-8.47667E+00) -- (axis cs:2.00301E+02,-8.66224E+00)  -- (axis cs:1.99364E+02,-8.28806E+00) -- (axis cs:1.99440E+02,-8.10267E+00) -- cycle  ;
\draw [ecliptics-full] (axis cs:1.98504E+02,-7.72651E+00) -- (axis cs:1.98428E+02,-7.91173E+00)  -- (axis cs:1.97494E+02,-7.53334E+00) -- (axis cs:1.97570E+02,-7.34829E+00) -- cycle  ;
\draw [ecliptics-full] (axis cs:1.96638E+02,-6.96811E+00) -- (axis cs:1.96561E+02,-7.15301E+00)  -- (axis cs:1.95630E+02,-6.77084E+00) -- (axis cs:1.95708E+02,-6.58609E+00) -- cycle  ;
\draw [ecliptics-full] (axis cs:1.94778E+02,-6.20231E+00) -- (axis cs:1.94701E+02,-6.38692E+00)  -- (axis cs:1.93773E+02,-6.00137E+00) -- (axis cs:1.93850E+02,-5.81689E+00) -- cycle  ;
\draw [ecliptics-full] (axis cs:1.92924E+02,-5.42993E+00) -- (axis cs:1.92846E+02,-5.61429E+00)  -- (axis cs:1.91920E+02,-5.22577E+00) -- (axis cs:1.91999E+02,-5.04154E+00) -- cycle  ;
\draw [ecliptics-full] (axis cs:1.91074E+02,-4.65180E+00) -- (axis cs:1.90996E+02,-4.83593E+00)  -- (axis cs:1.90073E+02,-4.44487E+00) -- (axis cs:1.90151E+02,-4.26084E+00) -- cycle  ;
\draw [ecliptics-full] (axis cs:1.89229E+02,-3.86875E+00) -- (axis cs:1.89150E+02,-4.05268E+00)  -- (axis cs:1.88229E+02,-3.65948E+00) -- (axis cs:1.88308E+02,-3.47563E+00) -- cycle  ;
\draw [ecliptics-full] (axis cs:1.87387E+02,-3.08158E+00) -- (axis cs:1.87308E+02,-3.26536E+00)  -- (axis cs:1.86388E+02,-2.87043E+00) -- (axis cs:1.86467E+02,-2.68672E+00) -- cycle  ;
\draw [ecliptics-full] (axis cs:1.85548E+02,-2.29113E+00) -- (axis cs:1.85468E+02,-2.47479E+00)  -- (axis cs:1.84550E+02,-2.07854E+00) -- (axis cs:1.84629E+02,-1.89493E+00) -- cycle  ;
\draw [ecliptics-full] (axis cs:1.83711E+02,-1.49822E+00) -- (axis cs:1.83631E+02,-1.68179E+00)  -- (axis cs:1.82713E+02,-1.28463E+00) -- (axis cs:1.82793E+02,-1.10109E+00) -- cycle  ;
\draw [ecliptics-full] (axis cs:1.81875E+02,-7.03652E-01) -- (axis cs:1.81795E+02,-8.87166E-01)  -- (axis cs:1.80878E+02,-4.89506E-01) -- (axis cs:1.80957E+02,-3.06005E-01) -- cycle  ;

\draw [ecliptics-empty] (axis cs:3.59043E+02,-3.06005E-01) -- (axis cs:3.59122E+02,-4.89506E-01)  -- (axis cs:3.58205E+02,-8.87166E-01) -- (axis cs:3.58125E+02,-7.03651E-01) -- cycle  ;
\draw [ecliptics-empty] (axis cs:3.57207E+02,-1.10109E+00) -- (axis cs:3.57287E+02,-1.28463E+00)  -- (axis cs:3.56369E+02,-1.68179E+00) -- (axis cs:3.56289E+02,-1.49822E+00) -- cycle  ;
\draw [ecliptics-empty] (axis cs:3.55371E+02,-1.89493E+00) -- (axis cs:3.55450E+02,-2.07854E+00)  -- (axis cs:3.54532E+02,-2.47479E+00) -- (axis cs:3.54452E+02,-2.29113E+00) -- cycle  ;
\draw [ecliptics-empty] (axis cs:3.53533E+02,-2.68672E+00) -- (axis cs:3.53612E+02,-2.87043E+00)  -- (axis cs:3.52692E+02,-3.26536E+00) -- (axis cs:3.52613E+02,-3.08158E+00) -- cycle  ;
\draw [ecliptics-empty] (axis cs:3.51693E+02,-3.47563E+00) -- (axis cs:3.51771E+02,-3.65948E+00)  -- (axis cs:3.50850E+02,-4.05268E+00) -- (axis cs:3.50771E+02,-3.86875E+00) -- cycle  ;
\draw [ecliptics-empty] (axis cs:3.49849E+02,-4.26084E+00) -- (axis cs:3.49927E+02,-4.44487E+00)  -- (axis cs:3.49004E+02,-4.83593E+00) -- (axis cs:3.48926E+02,-4.65180E+00) -- cycle  ;
\draw [ecliptics-empty] (axis cs:3.48001E+02,-5.04154E+00) -- (axis cs:3.48080E+02,-5.22577E+00)  -- (axis cs:3.47154E+02,-5.61429E+00) -- (axis cs:3.47076E+02,-5.42993E+00) -- cycle  ;
\draw [ecliptics-empty] (axis cs:3.46150E+02,-5.81689E+00) -- (axis cs:3.46227E+02,-6.00137E+00)  -- (axis cs:3.45299E+02,-6.38692E+00) -- (axis cs:3.45222E+02,-6.20231E+00) -- cycle  ;
\draw [ecliptics-empty] (axis cs:3.44292E+02,-6.58609E+00) -- (axis cs:3.44370E+02,-6.77084E+00)  -- (axis cs:3.43439E+02,-7.15301E+00) -- (axis cs:3.43362E+02,-6.96811E+00) -- cycle  ;
\draw [ecliptics-empty] (axis cs:3.42430E+02,-7.34829E+00) -- (axis cs:3.42506E+02,-7.53334E+00)  -- (axis cs:3.41572E+02,-7.91173E+00) -- (axis cs:3.41496E+02,-7.72651E+00) -- cycle  ;
\draw [ecliptics-empty] (axis cs:3.40560E+02,-8.10267E+00) -- (axis cs:3.40636E+02,-8.28806E+00)  -- (axis cs:3.39699E+02,-8.66224E+00) -- (axis cs:3.39623E+02,-8.47667E+00) -- cycle  ;
\draw [ecliptics-empty] (axis cs:3.38684E+02,-8.84840E+00) -- (axis cs:3.38759E+02,-9.03415E+00)  -- (axis cs:3.37818E+02,-9.40370E+00) -- (axis cs:3.37743E+02,-9.21776E+00) -- cycle  ;
\draw [ecliptics-empty] (axis cs:3.36800E+02,-9.58464E+00) -- (axis cs:3.36874E+02,-9.77078E+00)  -- (axis cs:3.35929E+02,-1.01353E+01) -- (axis cs:3.35855E+02,-9.94894E+00) -- cycle  ;
\draw [ecliptics-empty] (axis cs:3.34908E+02,-1.03105E+01) -- (axis cs:3.34981E+02,-1.04971E+01)  -- (axis cs:3.34032E+02,-1.08562E+01) -- (axis cs:3.33959E+02,-1.06694E+01) -- cycle  ;
\draw [ecliptics-empty] (axis cs:3.33007E+02,-1.10253E+01) -- (axis cs:3.33080E+02,-1.12123E+01)  -- (axis cs:3.32125E+02,-1.15655E+01) -- (axis cs:3.32054E+02,-1.13782E+01) -- cycle  ;
\draw [ecliptics-empty] (axis cs:3.31097E+02,-1.17280E+01) -- (axis cs:3.31169E+02,-1.19155E+01)  -- (axis cs:3.30209E+02,-1.22623E+01) -- (axis cs:3.30139E+02,-1.20746E+01) -- cycle  ;
\draw [ecliptics-empty] (axis cs:3.29178E+02,-1.24179E+01) -- (axis cs:3.29248E+02,-1.26059E+01)  -- (axis cs:3.28283E+02,-1.29460E+01) -- (axis cs:3.28214E+02,-1.27578E+01) -- cycle  ;
\draw [ecliptics-empty] (axis cs:3.27248E+02,-1.30941E+01) -- (axis cs:3.27316E+02,-1.32826E+01)  -- (axis cs:3.26347E+02,-1.36155E+01) -- (axis cs:3.26279E+02,-1.34268E+01) -- cycle  ;
\draw [ecliptics-empty] (axis cs:3.25307E+02,-1.37558E+01) -- (axis cs:3.25375E+02,-1.39448E+01)  -- (axis cs:3.24400E+02,-1.42701E+01) -- (axis cs:3.24333E+02,-1.40809E+01) -- cycle  ;
\draw [ecliptics-empty] (axis cs:3.23356E+02,-1.44021E+01) -- (axis cs:3.23422E+02,-1.45916E+01)  -- (axis cs:3.22441E+02,-1.49090E+01) -- (axis cs:3.22376E+02,-1.47192E+01) -- cycle  ;
\draw [ecliptics-empty] (axis cs:3.21393E+02,-1.50321E+01) -- (axis cs:3.21458E+02,-1.52222E+01)  -- (axis cs:3.20471E+02,-1.55312E+01) -- (axis cs:3.20408E+02,-1.53408E+01) -- cycle  ;
\draw [ecliptics-empty] (axis cs:3.19419E+02,-1.56451E+01) -- (axis cs:3.19482E+02,-1.58358E+01)  -- (axis cs:3.18490E+02,-1.61359E+01) -- (axis cs:3.18428E+02,-1.59450E+01) -- cycle  ;
\draw [ecliptics-empty] (axis cs:3.17434E+02,-1.62403E+01) -- (axis cs:3.17495E+02,-1.64315E+01)  -- (axis cs:3.16496E+02,-1.67224E+01) -- (axis cs:3.16436E+02,-1.65309E+01) -- cycle  ;
\draw [ecliptics-empty] (axis cs:3.15436E+02,-1.68167E+01) -- (axis cs:3.15495E+02,-1.70085E+01)  -- (axis cs:3.14491E+02,-1.72897E+01) -- (axis cs:3.14433E+02,-1.70976E+01) -- cycle  ;
\draw [ecliptics-empty] (axis cs:3.13426E+02,-1.73736E+01) -- (axis cs:3.13484E+02,-1.75660E+01)  -- (axis cs:3.12473E+02,-1.78371E+01) -- (axis cs:3.12417E+02,-1.76445E+01) -- cycle  ;
\draw [ecliptics-empty] (axis cs:3.11404E+02,-1.79102E+01) -- (axis cs:3.11460E+02,-1.81031E+01)  -- (axis cs:3.10443E+02,-1.83638E+01) -- (axis cs:3.10389E+02,-1.81706E+01) -- cycle  ;
\draw [ecliptics-empty] (axis cs:3.09371E+02,-1.84256E+01) -- (axis cs:3.09424E+02,-1.86192E+01)  -- (axis cs:3.08401E+02,-1.88690E+01) -- (axis cs:3.08349E+02,-1.86752E+01) -- cycle  ;
\draw [ecliptics-empty] (axis cs:3.07325E+02,-1.89192E+01) -- (axis cs:3.07376E+02,-1.91133E+01)  -- (axis cs:3.06347E+02,-1.93519E+01) -- (axis cs:3.06297E+02,-1.91575E+01) -- cycle  ;
\draw [ecliptics-empty] (axis cs:3.05267E+02,-1.93901E+01) -- (axis cs:3.05316E+02,-1.95848E+01)  -- (axis cs:3.04281E+02,-1.98118E+01) -- (axis cs:3.04234E+02,-1.96168E+01) -- cycle  ;
\draw [ecliptics-empty] (axis cs:3.03198E+02,-1.98376E+01) -- (axis cs:3.03244E+02,-2.00328E+01)  -- (axis cs:3.02204E+02,-2.02479E+01) -- (axis cs:3.02159E+02,-2.00524E+01) -- cycle  ;
\draw [ecliptics-empty] (axis cs:3.01117E+02,-2.02610E+01) -- (axis cs:3.01161E+02,-2.04568E+01)  -- (axis cs:3.00115E+02,-2.06595E+01) -- (axis cs:3.00072E+02,-2.04635E+01) -- cycle  ;
\draw [ecliptics-empty] (axis cs:2.99025E+02,-2.06597E+01) -- (axis cs:2.99066E+02,-2.08559E+01)  -- (axis cs:2.98015E+02,-2.10459E+01) -- (axis cs:2.97975E+02,-2.08495E+01) -- cycle  ;
\draw [ecliptics-empty] (axis cs:2.96922E+02,-2.10328E+01) -- (axis cs:2.96961E+02,-2.12295E+01)  -- (axis cs:2.95905E+02,-2.14066E+01) -- (axis cs:2.95867E+02,-2.12096E+01) -- cycle  ;
\draw [ecliptics-empty] (axis cs:2.94810E+02,-2.13799E+01) -- (axis cs:2.94846E+02,-2.15771E+01)  -- (axis cs:2.93784E+02,-2.17408E+01) -- (axis cs:2.93750E+02,-2.15434E+01) -- cycle  ;
\draw [ecliptics-empty] (axis cs:2.92687E+02,-2.17002E+01) -- (axis cs:2.92720E+02,-2.18979E+01)  -- (axis cs:2.91654E+02,-2.20481E+01) -- (axis cs:2.91623E+02,-2.18502E+01) -- cycle  ;
\draw [ecliptics-empty] (axis cs:2.90556E+02,-2.19934E+01) -- (axis cs:2.90586E+02,-2.21914E+01)  -- (axis cs:2.89516E+02,-2.23278E+01) -- (axis cs:2.89487E+02,-2.21296E+01) -- cycle  ;
\draw [ecliptics-empty] (axis cs:2.88416E+02,-2.22587E+01) -- (axis cs:2.88443E+02,-2.24572E+01)  -- (axis cs:2.87369E+02,-2.25795E+01) -- (axis cs:2.87343E+02,-2.23809E+01) -- cycle  ;
\draw [ecliptics-empty] (axis cs:2.86268E+02,-2.24959E+01) -- (axis cs:2.86292E+02,-2.26946E+01)  -- (axis cs:2.85214E+02,-2.28026E+01) -- (axis cs:2.85192E+02,-2.26037E+01) -- cycle  ;
\draw [ecliptics-empty] (axis cs:2.84114E+02,-2.27043E+01) -- (axis cs:2.84135E+02,-2.29034E+01)  -- (axis cs:2.83054E+02,-2.29969E+01) -- (axis cs:2.83034E+02,-2.27977E+01) -- cycle  ;
\draw [ecliptics-empty] (axis cs:2.81953E+02,-2.28837E+01) -- (axis cs:2.81971E+02,-2.30830E+01)  -- (axis cs:2.80887E+02,-2.31619E+01) -- (axis cs:2.80871E+02,-2.29624E+01) -- cycle  ;
\draw [ecliptics-empty] (axis cs:2.79787E+02,-2.30338E+01) -- (axis cs:2.79802E+02,-2.32333E+01)  -- (axis cs:2.78715E+02,-2.32973E+01) -- (axis cs:2.78702E+02,-2.30977E+01) -- cycle  ;
\draw [ecliptics-empty] (axis cs:2.77617E+02,-2.31541E+01) -- (axis cs:2.77628E+02,-2.33538E+01)  -- (axis cs:2.76540E+02,-2.34029E+01) -- (axis cs:2.76530E+02,-2.32031E+01) -- cycle  ;
\draw [ecliptics-empty] (axis cs:2.75443E+02,-2.32446E+01) -- (axis cs:2.75451E+02,-2.34444E+01)  -- (axis cs:2.74362E+02,-2.34785E+01) -- (axis cs:2.74355E+02,-2.32785E+01) -- cycle  ;
\draw [ecliptics-empty] (axis cs:2.73267E+02,-2.33050E+01) -- (axis cs:2.73272E+02,-2.35049E+01)  -- (axis cs:2.72181E+02,-2.35239E+01) -- (axis cs:2.72178E+02,-2.33239E+01) -- cycle  ;
\draw [ecliptics-empty] (axis cs:2.71089E+02,-2.33352E+01) -- (axis cs:2.71091E+02,-2.35352E+01)  -- (axis cs:2.70000E+02,-2.35390E+01) -- (axis cs:2.70000E+02,-2.33390E+01) -- cycle  ;
\draw [ecliptics-empty] (axis cs:2.68911E+02,-2.33352E+01) -- (axis cs:2.68909E+02,-2.35352E+01)  -- (axis cs:2.67819E+02,-2.35239E+01) -- (axis cs:2.67822E+02,-2.33239E+01) -- cycle  ;
\draw [ecliptics-empty] (axis cs:2.66733E+02,-2.33050E+01) -- (axis cs:2.66728E+02,-2.35049E+01)  -- (axis cs:2.65638E+02,-2.34785E+01) -- (axis cs:2.65645E+02,-2.32785E+01) -- cycle  ;
\draw [ecliptics-empty] (axis cs:2.64557E+02,-2.32446E+01) -- (axis cs:2.64549E+02,-2.34444E+01)  -- (axis cs:2.63460E+02,-2.34029E+01) -- (axis cs:2.63470E+02,-2.32031E+01) -- cycle  ;
\draw [ecliptics-empty] (axis cs:2.62383E+02,-2.31541E+01) -- (axis cs:2.62372E+02,-2.33538E+01)  -- (axis cs:2.61285E+02,-2.32973E+01) -- (axis cs:2.61298E+02,-2.30977E+01) -- cycle  ;
\draw [ecliptics-empty] (axis cs:2.60213E+02,-2.30338E+01) -- (axis cs:2.60198E+02,-2.32333E+01)  -- (axis cs:2.59113E+02,-2.31619E+01) -- (axis cs:2.59129E+02,-2.29624E+01) -- cycle  ;
\draw [ecliptics-empty] (axis cs:2.58047E+02,-2.28837E+01) -- (axis cs:2.58029E+02,-2.30830E+01)  -- (axis cs:2.56946E+02,-2.29969E+01) -- (axis cs:2.56966E+02,-2.27977E+01) -- cycle  ;
\draw [ecliptics-empty] (axis cs:2.55886E+02,-2.27043E+01) -- (axis cs:2.55865E+02,-2.29034E+01)  -- (axis cs:2.54786E+02,-2.28026E+01) -- (axis cs:2.54808E+02,-2.26037E+01) -- cycle  ;
\draw [ecliptics-empty] (axis cs:2.53732E+02,-2.24959E+01) -- (axis cs:2.53708E+02,-2.26946E+01)  -- (axis cs:2.52631E+02,-2.25795E+01) -- (axis cs:2.52657E+02,-2.23809E+01) -- cycle  ;
\draw [ecliptics-empty] (axis cs:2.51584E+02,-2.22587E+01) -- (axis cs:2.51557E+02,-2.24572E+01)  -- (axis cs:2.50484E+02,-2.23278E+01) -- (axis cs:2.50513E+02,-2.21296E+01) -- cycle  ;
\draw [ecliptics-empty] (axis cs:2.49444E+02,-2.19934E+01) -- (axis cs:2.49414E+02,-2.21914E+01)  -- (axis cs:2.48346E+02,-2.20481E+01) -- (axis cs:2.48377E+02,-2.18502E+01) -- cycle  ;
\draw [ecliptics-empty] (axis cs:2.47313E+02,-2.17002E+01) -- (axis cs:2.47280E+02,-2.18979E+01)  -- (axis cs:2.46216E+02,-2.17408E+01) -- (axis cs:2.46250E+02,-2.15434E+01) -- cycle  ;
\draw [ecliptics-empty] (axis cs:2.45190E+02,-2.13799E+01) -- (axis cs:2.45154E+02,-2.15771E+01)  -- (axis cs:2.44095E+02,-2.14066E+01) -- (axis cs:2.44133E+02,-2.12096E+01) -- cycle  ;
\draw [ecliptics-empty] (axis cs:2.43078E+02,-2.10328E+01) -- (axis cs:2.43039E+02,-2.12295E+01)  -- (axis cs:2.41985E+02,-2.10459E+01) -- (axis cs:2.42025E+02,-2.08495E+01) -- cycle  ;
\draw [ecliptics-empty] (axis cs:2.40975E+02,-2.06597E+01) -- (axis cs:2.40934E+02,-2.08559E+01)  -- (axis cs:2.39885E+02,-2.06595E+01) -- (axis cs:2.39928E+02,-2.04635E+01) -- cycle  ;
\draw [ecliptics-empty] (axis cs:2.38883E+02,-2.02610E+01) -- (axis cs:2.38839E+02,-2.04568E+01)  -- (axis cs:2.37796E+02,-2.02479E+01) -- (axis cs:2.37841E+02,-2.00524E+01) -- cycle  ;
\draw [ecliptics-empty] (axis cs:2.36802E+02,-1.98376E+01) -- (axis cs:2.36756E+02,-2.00328E+01)  -- (axis cs:2.35719E+02,-1.98118E+01) -- (axis cs:2.35766E+02,-1.96168E+01) -- cycle  ;
\draw [ecliptics-empty] (axis cs:2.34733E+02,-1.93901E+01) -- (axis cs:2.34684E+02,-1.95848E+01)  -- (axis cs:2.33653E+02,-1.93519E+01) -- (axis cs:2.33703E+02,-1.91575E+01) -- cycle  ;
\draw [ecliptics-empty] (axis cs:2.32675E+02,-1.89192E+01) -- (axis cs:2.32624E+02,-1.91133E+01)  -- (axis cs:2.31599E+02,-1.88690E+01) -- (axis cs:2.31651E+02,-1.86752E+01) -- cycle  ;
\draw [ecliptics-empty] (axis cs:2.30629E+02,-1.84256E+01) -- (axis cs:2.30576E+02,-1.86192E+01)  -- (axis cs:2.29557E+02,-1.83638E+01) -- (axis cs:2.29611E+02,-1.81706E+01) -- cycle  ;
\draw [ecliptics-empty] (axis cs:2.28596E+02,-1.79102E+01) -- (axis cs:2.28540E+02,-1.81031E+01)  -- (axis cs:2.27527E+02,-1.78371E+01) -- (axis cs:2.27583E+02,-1.76445E+01) -- cycle  ;
\draw [ecliptics-empty] (axis cs:2.26574E+02,-1.73736E+01) -- (axis cs:2.26516E+02,-1.75660E+01)  -- (axis cs:2.25509E+02,-1.72897E+01) -- (axis cs:2.25567E+02,-1.70976E+01) -- cycle  ;
\draw [ecliptics-empty] (axis cs:2.24564E+02,-1.68167E+01) -- (axis cs:2.24505E+02,-1.70085E+01)  -- (axis cs:2.23504E+02,-1.67224E+01) -- (axis cs:2.23564E+02,-1.65309E+01) -- cycle  ;
\draw [ecliptics-empty] (axis cs:2.22566E+02,-1.62403E+01) -- (axis cs:2.22505E+02,-1.64315E+01)  -- (axis cs:2.21510E+02,-1.61359E+01) -- (axis cs:2.21572E+02,-1.59450E+01) -- cycle  ;
\draw [ecliptics-empty] (axis cs:2.20581E+02,-1.56451E+01) -- (axis cs:2.20518E+02,-1.58358E+01)  -- (axis cs:2.19529E+02,-1.55312E+01) -- (axis cs:2.19592E+02,-1.53408E+01) -- cycle  ;
\draw [ecliptics-empty] (axis cs:2.18607E+02,-1.50321E+01) -- (axis cs:2.18542E+02,-1.52222E+01)  -- (axis cs:2.17559E+02,-1.49090E+01) -- (axis cs:2.17624E+02,-1.47192E+01) -- cycle  ;
\draw [ecliptics-empty] (axis cs:2.16644E+02,-1.44021E+01) -- (axis cs:2.16578E+02,-1.45916E+01)  -- (axis cs:2.15600E+02,-1.42701E+01) -- (axis cs:2.15667E+02,-1.40809E+01) -- cycle  ;
\draw [ecliptics-empty] (axis cs:2.14693E+02,-1.37558E+01) -- (axis cs:2.14625E+02,-1.39448E+01)  -- (axis cs:2.13653E+02,-1.36155E+01) -- (axis cs:2.13721E+02,-1.34268E+01) -- cycle  ;
\draw [ecliptics-empty] (axis cs:2.12752E+02,-1.30941E+01) -- (axis cs:2.12683E+02,-1.32826E+01)  -- (axis cs:2.11717E+02,-1.29460E+01) -- (axis cs:2.11786E+02,-1.27578E+01) -- cycle  ;
\draw [ecliptics-empty] (axis cs:2.10822E+02,-1.24179E+01) -- (axis cs:2.10752E+02,-1.26059E+01)  -- (axis cs:2.09791E+02,-1.22623E+01) -- (axis cs:2.09861E+02,-1.20746E+01) -- cycle  ;
\draw [ecliptics-empty] (axis cs:2.08903E+02,-1.17280E+01) -- (axis cs:2.08831E+02,-1.19155E+01)  -- (axis cs:2.07875E+02,-1.15655E+01) -- (axis cs:2.07946E+02,-1.13782E+01) -- cycle  ;
\draw [ecliptics-empty] (axis cs:2.06993E+02,-1.10253E+01) -- (axis cs:2.06920E+02,-1.12123E+01)  -- (axis cs:2.05968E+02,-1.08562E+01) -- (axis cs:2.06041E+02,-1.06694E+01) -- cycle  ;
\draw [ecliptics-empty] (axis cs:2.05092E+02,-1.03105E+01) -- (axis cs:2.05019E+02,-1.04971E+01)  -- (axis cs:2.04071E+02,-1.01353E+01) -- (axis cs:2.04145E+02,-9.94894E+00) -- cycle  ;
\draw [ecliptics-empty] (axis cs:2.03200E+02,-9.58464E+00) -- (axis cs:2.03126E+02,-9.77078E+00)  -- (axis cs:2.02182E+02,-9.40370E+00) -- (axis cs:2.02257E+02,-9.21776E+00) -- cycle  ;
\draw [ecliptics-empty] (axis cs:2.01316E+02,-8.84840E+00) -- (axis cs:2.01241E+02,-9.03415E+00)  -- (axis cs:2.00301E+02,-8.66224E+00) -- (axis cs:2.00377E+02,-8.47667E+00) -- cycle  ;
\draw [ecliptics-empty] (axis cs:1.99440E+02,-8.10267E+00) -- (axis cs:1.99364E+02,-8.28806E+00)  -- (axis cs:1.98428E+02,-7.91173E+00) -- (axis cs:1.98504E+02,-7.72651E+00) -- cycle  ;
\draw [ecliptics-empty] (axis cs:1.97570E+02,-7.34829E+00) -- (axis cs:1.97494E+02,-7.53334E+00)  -- (axis cs:1.96561E+02,-7.15301E+00) -- (axis cs:1.96638E+02,-6.96811E+00) -- cycle  ;
\draw [ecliptics-empty] (axis cs:1.95708E+02,-6.58609E+00) -- (axis cs:1.95630E+02,-6.77084E+00)  -- (axis cs:1.94701E+02,-6.38692E+00) -- (axis cs:1.94778E+02,-6.20231E+00) -- cycle  ;
\draw [ecliptics-empty] (axis cs:1.93850E+02,-5.81689E+00) -- (axis cs:1.93773E+02,-6.00137E+00)  -- (axis cs:1.92846E+02,-5.61429E+00) -- (axis cs:1.92924E+02,-5.42993E+00) -- cycle  ;
\draw [ecliptics-empty] (axis cs:1.91999E+02,-5.04154E+00) -- (axis cs:1.91920E+02,-5.22577E+00)  -- (axis cs:1.90996E+02,-4.83593E+00) -- (axis cs:1.91074E+02,-4.65180E+00) -- cycle  ;
\draw [ecliptics-empty] (axis cs:1.90151E+02,-4.26084E+00) -- (axis cs:1.90073E+02,-4.44487E+00)  -- (axis cs:1.89150E+02,-4.05268E+00) -- (axis cs:1.89229E+02,-3.86875E+00) -- cycle  ;
\draw [ecliptics-empty] (axis cs:1.88308E+02,-3.47563E+00) -- (axis cs:1.88229E+02,-3.65948E+00)  -- (axis cs:1.87308E+02,-3.26536E+00) -- (axis cs:1.87387E+02,-3.08158E+00) -- cycle  ;
\draw [ecliptics-empty] (axis cs:1.86467E+02,-2.68672E+00) -- (axis cs:1.86388E+02,-2.87043E+00)  -- (axis cs:1.85468E+02,-2.47479E+00) -- (axis cs:1.85548E+02,-2.29113E+00) -- cycle  ;
\draw [ecliptics-empty] (axis cs:1.84629E+02,-1.89493E+00) -- (axis cs:1.84550E+02,-2.07854E+00)  -- (axis cs:1.83631E+02,-1.68179E+00) -- (axis cs:1.83711E+02,-1.49822E+00) -- cycle  ;
\draw [ecliptics-empty] (axis cs:1.82793E+02,-1.10109E+00) -- (axis cs:1.82713E+02,-1.28463E+00)  -- (axis cs:1.81795E+02,-8.87166E-01) -- (axis cs:1.81875E+02,-7.03652E-01) -- cycle  ;
\draw [ecliptics-empty] (axis cs:1.80957E+02,-3.06005E-01) -- (axis cs:1.80878E+02,-4.89506E-01)  -- (axis cs:1.79960E+02,-9.17483E-02) -- (axis cs:1.80040E+02,9.17485E-02) -- cycle  ;



\draw [ecliptics-empty] (axis cs:-0.4000854E-01,9.17484E-02) -- (axis cs:0.0398,-9.17484E-02)  -- (axis cs:9.57272E-01,3.06005E-01) -- (axis cs:8.77725E-01,4.89506E-01) -- cycle  ;
\draw [ecliptics-empty] (axis cs:1.79532E+00,8.87166E-01) -- (axis cs:1.87485E+00,7.03652E-01)  -- (axis cs:2.79259E+00,1.10109E+00) -- (axis cs:2.71311E+00,1.28463E+00) -- cycle  ;
\draw [ecliptics-empty] (axis cs:3.63117E+00,1.68179E+00) -- (axis cs:3.71059E+00,1.49822E+00)  -- (axis cs:4.62893E+00,1.89493E+00) -- (axis cs:4.54959E+00,2.07854E+00) -- cycle  ;
\draw [ecliptics-empty] (axis cs:5.46845E+00,2.47479E+00) -- (axis cs:5.54771E+00,2.29113E+00)  -- (axis cs:6.46701E+00,2.68672E+00) -- (axis cs:6.38786E+00,2.87043E+00) -- cycle  ;
\draw [ecliptics-empty] (axis cs:7.30789E+00,3.26536E+00) -- (axis cs:7.38691E+00,3.08158E+00)  -- (axis cs:8.30751E+00,3.47563E+00) -- (axis cs:8.22863E+00,3.65948E+00) -- cycle  ;
\draw [ecliptics-empty] (axis cs:9.15017E+00,4.05268E+00) -- (axis cs:9.22889E+00,3.86875E+00)  -- (axis cs:1.01511E+01,4.26084E+00) -- (axis cs:1.00726E+01,4.44487E+00) -- cycle  ;
\draw [ecliptics-empty] (axis cs:1.09960E+01,4.83593E+00) -- (axis cs:1.10743E+01,4.65180E+00)  -- (axis cs:1.19986E+01,5.04154E+00) -- (axis cs:1.19204E+01,5.22577E+00) -- cycle  ;
\draw [ecliptics-empty] (axis cs:1.28460E+01,5.61429E+00) -- (axis cs:1.29239E+01,5.42993E+00)  -- (axis cs:1.38505E+01,5.81689E+00) -- (axis cs:1.37728E+01,6.00137E+00) -- cycle  ;
\draw [ecliptics-empty] (axis cs:1.47009E+01,6.38692E+00) -- (axis cs:1.47783E+01,6.20231E+00)  -- (axis cs:1.57075E+01,6.58609E+00) -- (axis cs:1.56304E+01,6.77084E+00) -- cycle  ;
\draw [ecliptics-empty] (axis cs:1.65614E+01,7.15301E+00) -- (axis cs:1.66382E+01,6.96811E+00)  -- (axis cs:1.75704E+01,7.34829E+00) -- (axis cs:1.74939E+01,7.53334E+00) -- cycle  ;
\draw [ecliptics-empty] (axis cs:1.84280E+01,7.91173E+00) -- (axis cs:1.85041E+01,7.72651E+00)  -- (axis cs:1.94396E+01,8.10267E+00) -- (axis cs:1.93638E+01,8.28806E+00) -- cycle  ;
\draw [ecliptics-empty] (axis cs:2.03014E+01,8.66224E+00) -- (axis cs:2.03768E+01,8.47667E+00)  -- (axis cs:2.13159E+01,8.84840E+00) -- (axis cs:2.12408E+01,9.03415E+00) -- cycle  ;
\draw [ecliptics-empty] (axis cs:2.21822E+01,9.40370E+00) -- (axis cs:2.22568E+01,9.21776E+00)  -- (axis cs:2.31998E+01,9.58464E+00) -- (axis cs:2.31256E+01,9.77078E+00) -- cycle  ;
\draw [ecliptics-empty] (axis cs:2.40710E+01,1.01353E+01) -- (axis cs:2.41448E+01,9.94894E+00)  -- (axis cs:2.50918E+01,1.03105E+01) -- (axis cs:2.50186E+01,1.04971E+01) -- cycle  ;
\draw [ecliptics-empty] (axis cs:2.59683E+01,1.08562E+01) -- (axis cs:2.60411E+01,1.06694E+01)  -- (axis cs:2.69926E+01,1.10253E+01) -- (axis cs:2.69204E+01,1.12123E+01) -- cycle  ;
\draw [ecliptics-empty] (axis cs:2.78747E+01,1.15655E+01) -- (axis cs:2.79465E+01,1.13782E+01)  -- (axis cs:2.89026E+01,1.17280E+01) -- (axis cs:2.88315E+01,1.19155E+01) -- cycle  ;
\draw [ecliptics-empty] (axis cs:2.97907E+01,1.22623E+01) -- (axis cs:2.98613E+01,1.20746E+01)  -- (axis cs:3.08224E+01,1.24179E+01) -- (axis cs:3.07524E+01,1.26059E+01) -- cycle  ;
\draw [ecliptics-empty] (axis cs:3.17166E+01,1.29460E+01) -- (axis cs:3.17860E+01,1.27578E+01)  -- (axis cs:3.27522E+01,1.30941E+01) -- (axis cs:3.26835E+01,1.32826E+01) -- cycle  ;
\draw [ecliptics-empty] (axis cs:3.36530E+01,1.36155E+01) -- (axis cs:3.37211E+01,1.34268E+01)  -- (axis cs:3.46927E+01,1.37558E+01) -- (axis cs:3.46253E+01,1.39448E+01) -- cycle  ;
\draw [ecliptics-empty] (axis cs:3.56002E+01,1.42701E+01) -- (axis cs:3.56669E+01,1.40809E+01)  -- (axis cs:3.66440E+01,1.44021E+01) -- (axis cs:3.65780E+01,1.45916E+01) -- cycle  ;
\draw [ecliptics-empty] (axis cs:3.75586E+01,1.49090E+01) -- (axis cs:3.76238E+01,1.47192E+01)  -- (axis cs:3.86065E+01,1.50321E+01) -- (axis cs:3.85421E+01,1.52222E+01) -- cycle  ;
\draw [ecliptics-empty] (axis cs:3.95285E+01,1.55312E+01) -- (axis cs:3.95921E+01,1.53408E+01)  -- (axis cs:4.05806E+01,1.56451E+01) -- (axis cs:4.05178E+01,1.58358E+01) -- cycle  ;
\draw [ecliptics-empty] (axis cs:4.15101E+01,1.61359E+01) -- (axis cs:4.15720E+01,1.59450E+01)  -- (axis cs:4.25664E+01,1.62403E+01) -- (axis cs:4.25053E+01,1.64315E+01) -- cycle  ;
\draw [ecliptics-empty] (axis cs:4.35036E+01,1.67224E+01) -- (axis cs:4.35638E+01,1.65309E+01)  -- (axis cs:4.45641E+01,1.68167E+01) -- (axis cs:4.45048E+01,1.70085E+01) -- cycle  ;
\draw [ecliptics-empty] (axis cs:4.55091E+01,1.72897E+01) -- (axis cs:4.55675E+01,1.70976E+01)  -- (axis cs:4.65738E+01,1.73736E+01) -- (axis cs:4.65165E+01,1.75660E+01) -- cycle  ;
\draw [ecliptics-empty] (axis cs:4.75268E+01,1.78371E+01) -- (axis cs:4.75832E+01,1.76445E+01)  -- (axis cs:4.85956E+01,1.79102E+01) -- (axis cs:4.85402E+01,1.81031E+01) -- cycle  ;
\draw [ecliptics-empty] (axis cs:4.95567E+01,1.83638E+01) -- (axis cs:4.96110E+01,1.81706E+01)  -- (axis cs:5.06294E+01,1.84256E+01) -- (axis cs:5.05762E+01,1.86192E+01) -- cycle  ;
\draw [ecliptics-empty] (axis cs:5.15987E+01,1.88690E+01) -- (axis cs:5.16509E+01,1.86752E+01)  -- (axis cs:5.26753E+01,1.89192E+01) -- (axis cs:5.26242E+01,1.91133E+01) -- cycle  ;
\draw [ecliptics-empty] (axis cs:5.36527E+01,1.93519E+01) -- (axis cs:5.37027E+01,1.91575E+01)  -- (axis cs:5.47330E+01,1.93901E+01) -- (axis cs:5.46842E+01,1.95848E+01) -- cycle  ;
\draw [ecliptics-empty] (axis cs:5.57187E+01,1.98118E+01) -- (axis cs:5.57662E+01,1.96168E+01)  -- (axis cs:5.68024E+01,1.98376E+01) -- (axis cs:5.67560E+01,2.00328E+01) -- cycle  ;
\draw [ecliptics-empty] (axis cs:5.77962E+01,2.02479E+01) -- (axis cs:5.78414E+01,2.00524E+01)  -- (axis cs:5.88831E+01,2.02610E+01) -- (axis cs:5.88393E+01,2.04568E+01) -- cycle  ;
\draw [ecliptics-empty] (axis cs:5.98851E+01,2.06595E+01) -- (axis cs:5.99277E+01,2.04635E+01)  -- (axis cs:6.09750E+01,2.06597E+01) -- (axis cs:6.09337E+01,2.08559E+01) -- cycle  ;
\draw [ecliptics-empty] (axis cs:6.19850E+01,2.10459E+01) -- (axis cs:6.20250E+01,2.08495E+01)  -- (axis cs:6.30775E+01,2.10328E+01) -- (axis cs:6.30389E+01,2.12295E+01) -- cycle  ;
\draw [ecliptics-empty] (axis cs:6.40954E+01,2.14066E+01) -- (axis cs:6.41327E+01,2.12096E+01)  -- (axis cs:6.51903E+01,2.13799E+01) -- (axis cs:6.51544E+01,2.15771E+01) -- cycle  ;
\draw [ecliptics-empty] (axis cs:6.62158E+01,2.17408E+01) -- (axis cs:6.62503E+01,2.15434E+01)  -- (axis cs:6.73127E+01,2.17002E+01) -- (axis cs:6.72796E+01,2.18979E+01) -- cycle  ;
\draw [ecliptics-empty] (axis cs:6.83457E+01,2.20481E+01) -- (axis cs:6.83773E+01,2.18502E+01)  -- (axis cs:6.94442E+01,2.19934E+01) -- (axis cs:6.94140E+01,2.21914E+01) -- cycle  ;
\draw [ecliptics-empty] (axis cs:7.04845E+01,2.23278E+01) -- (axis cs:7.05131E+01,2.21296E+01)  -- (axis cs:7.15841E+01,2.22587E+01) -- (axis cs:7.15569E+01,2.24572E+01) -- cycle  ;
\draw [ecliptics-empty] (axis cs:7.26313E+01,2.25795E+01) -- (axis cs:7.26570E+01,2.23809E+01)  -- (axis cs:7.37317E+01,2.24959E+01) -- (axis cs:7.37076E+01,2.26946E+01) -- cycle  ;
\draw [ecliptics-empty] (axis cs:7.47856E+01,2.28026E+01) -- (axis cs:7.48082E+01,2.26037E+01)  -- (axis cs:7.58863E+01,2.27043E+01) -- (axis cs:7.58653E+01,2.29034E+01) -- cycle  ;
\draw [ecliptics-empty] (axis cs:7.69465E+01,2.29969E+01) -- (axis cs:7.69660E+01,2.27977E+01)  -- (axis cs:7.80470E+01,2.28837E+01) -- (axis cs:7.80291E+01,2.30830E+01) -- cycle  ;
\draw [ecliptics-empty] (axis cs:7.91131E+01,2.31619E+01) -- (axis cs:7.91294E+01,2.29624E+01)  -- (axis cs:8.02130E+01,2.30338E+01) -- (axis cs:8.01982E+01,2.32333E+01) -- cycle  ;
\draw [ecliptics-empty] (axis cs:8.12845E+01,2.32973E+01) -- (axis cs:8.12976E+01,2.30977E+01)  -- (axis cs:8.23833E+01,2.31541E+01) -- (axis cs:8.23718E+01,2.33538E+01) -- cycle  ;
\draw [ecliptics-empty] (axis cs:8.34599E+01,2.34029E+01) -- (axis cs:8.34698E+01,2.32031E+01)  -- (axis cs:8.45570E+01,2.32446E+01) -- (axis cs:8.45488E+01,2.34444E+01) -- cycle  ;
\draw [ecliptics-empty] (axis cs:8.56383E+01,2.34785E+01) -- (axis cs:8.56449E+01,2.32785E+01)  -- (axis cs:8.67332E+01,2.33050E+01) -- (axis cs:8.67283E+01,2.35049E+01) -- cycle  ;
\draw [ecliptics-empty] (axis cs:8.78186E+01,2.35239E+01) -- (axis cs:8.78219E+01,2.33239E+01)  -- (axis cs:8.89109E+01,2.33352E+01) -- (axis cs:8.89093E+01,2.35352E+01) -- cycle  ;
\draw [ecliptics-empty] (axis cs:9.00000E+01,2.35390E+01) -- (axis cs:9.00000E+01,2.33390E+01)  -- (axis cs:9.10891E+01,2.33352E+01) -- (axis cs:9.10907E+01,2.35352E+01) -- cycle  ;
\draw [ecliptics-empty] (axis cs:9.21814E+01,2.35239E+01) -- (axis cs:9.21781E+01,2.33239E+01)  -- (axis cs:9.32668E+01,2.33050E+01) -- (axis cs:9.32717E+01,2.35049E+01) -- cycle  ;
\draw [ecliptics-empty] (axis cs:9.43617E+01,2.34785E+01) -- (axis cs:9.43551E+01,2.32785E+01)  -- (axis cs:9.54430E+01,2.32446E+01) -- (axis cs:9.54512E+01,2.34444E+01) -- cycle  ;
\draw [ecliptics-empty] (axis cs:9.65401E+01,2.34029E+01) -- (axis cs:9.65302E+01,2.32031E+01)  -- (axis cs:9.76167E+01,2.31541E+01) -- (axis cs:9.76282E+01,2.33538E+01) -- cycle  ;
\draw [ecliptics-empty] (axis cs:9.87155E+01,2.32973E+01) -- (axis cs:9.87024E+01,2.30977E+01)  -- (axis cs:9.97870E+01,2.30338E+01) -- (axis cs:9.98018E+01,2.32333E+01) -- cycle  ;
\draw [ecliptics-empty] (axis cs:1.00887E+02,2.31619E+01) -- (axis cs:1.00871E+02,2.29624E+01)  -- (axis cs:1.01953E+02,2.28837E+01) -- (axis cs:1.01971E+02,2.30830E+01) -- cycle  ;
\draw [ecliptics-empty] (axis cs:1.03054E+02,2.29969E+01) -- (axis cs:1.03034E+02,2.27977E+01)  -- (axis cs:1.04114E+02,2.27043E+01) -- (axis cs:1.04135E+02,2.29034E+01) -- cycle  ;
\draw [ecliptics-empty] (axis cs:1.05214E+02,2.28026E+01) -- (axis cs:1.05192E+02,2.26037E+01)  -- (axis cs:1.06268E+02,2.24959E+01) -- (axis cs:1.06292E+02,2.26946E+01) -- cycle  ;
\draw [ecliptics-empty] (axis cs:1.07369E+02,2.25795E+01) -- (axis cs:1.07343E+02,2.23809E+01)  -- (axis cs:1.08416E+02,2.22587E+01) -- (axis cs:1.08443E+02,2.24572E+01) -- cycle  ;
\draw [ecliptics-empty] (axis cs:1.09516E+02,2.23278E+01) -- (axis cs:1.09487E+02,2.21296E+01)  -- (axis cs:1.10556E+02,2.19934E+01) -- (axis cs:1.10586E+02,2.21914E+01) -- cycle  ;
\draw [ecliptics-empty] (axis cs:1.11654E+02,2.20481E+01) -- (axis cs:1.11623E+02,2.18502E+01)  -- (axis cs:1.12687E+02,2.17002E+01) -- (axis cs:1.12720E+02,2.18979E+01) -- cycle  ;
\draw [ecliptics-empty] (axis cs:1.13784E+02,2.17408E+01) -- (axis cs:1.13750E+02,2.15434E+01)  -- (axis cs:1.14810E+02,2.13799E+01) -- (axis cs:1.14846E+02,2.15771E+01) -- cycle  ;
\draw [ecliptics-empty] (axis cs:1.15905E+02,2.14066E+01) -- (axis cs:1.15867E+02,2.12096E+01)  -- (axis cs:1.16922E+02,2.10328E+01) -- (axis cs:1.16961E+02,2.12295E+01) -- cycle  ;
\draw [ecliptics-empty] (axis cs:1.18015E+02,2.10459E+01) -- (axis cs:1.17975E+02,2.08495E+01)  -- (axis cs:1.19025E+02,2.06597E+01) -- (axis cs:1.19066E+02,2.08559E+01) -- cycle  ;
\draw [ecliptics-empty] (axis cs:1.20115E+02,2.06595E+01) -- (axis cs:1.20072E+02,2.04635E+01)  -- (axis cs:1.21117E+02,2.02610E+01) -- (axis cs:1.21161E+02,2.04568E+01) -- cycle  ;
\draw [ecliptics-empty] (axis cs:1.22204E+02,2.02479E+01) -- (axis cs:1.22159E+02,2.00524E+01)  -- (axis cs:1.23198E+02,1.98376E+01) -- (axis cs:1.23244E+02,2.00328E+01) -- cycle  ;
\draw [ecliptics-empty] (axis cs:1.24281E+02,1.98118E+01) -- (axis cs:1.24234E+02,1.96168E+01)  -- (axis cs:1.25267E+02,1.93901E+01) -- (axis cs:1.25316E+02,1.95848E+01) -- cycle  ;
\draw [ecliptics-empty] (axis cs:1.26347E+02,1.93519E+01) -- (axis cs:1.26297E+02,1.91575E+01)  -- (axis cs:1.27325E+02,1.89192E+01) -- (axis cs:1.27376E+02,1.91133E+01) -- cycle  ;
\draw [ecliptics-empty] (axis cs:1.28401E+02,1.88690E+01) -- (axis cs:1.28349E+02,1.86752E+01)  -- (axis cs:1.29371E+02,1.84256E+01) -- (axis cs:1.29424E+02,1.86192E+01) -- cycle  ;
\draw [ecliptics-empty] (axis cs:1.30443E+02,1.83638E+01) -- (axis cs:1.30389E+02,1.81706E+01)  -- (axis cs:1.31404E+02,1.79102E+01) -- (axis cs:1.31460E+02,1.81031E+01) -- cycle  ;
\draw [ecliptics-empty] (axis cs:1.32473E+02,1.78371E+01) -- (axis cs:1.32417E+02,1.76445E+01)  -- (axis cs:1.33426E+02,1.73736E+01) -- (axis cs:1.33484E+02,1.75660E+01) -- cycle  ;
\draw [ecliptics-empty] (axis cs:1.34491E+02,1.72897E+01) -- (axis cs:1.34433E+02,1.70976E+01)  -- (axis cs:1.35436E+02,1.68167E+01) -- (axis cs:1.35495E+02,1.70085E+01) -- cycle  ;
\draw [ecliptics-empty] (axis cs:1.36496E+02,1.67224E+01) -- (axis cs:1.36436E+02,1.65309E+01)  -- (axis cs:1.37434E+02,1.62403E+01) -- (axis cs:1.37495E+02,1.64315E+01) -- cycle  ;
\draw [ecliptics-empty] (axis cs:1.38490E+02,1.61359E+01) -- (axis cs:1.38428E+02,1.59450E+01)  -- (axis cs:1.39419E+02,1.56451E+01) -- (axis cs:1.39482E+02,1.58358E+01) -- cycle  ;
\draw [ecliptics-empty] (axis cs:1.40471E+02,1.55312E+01) -- (axis cs:1.40408E+02,1.53408E+01)  -- (axis cs:1.41393E+02,1.50321E+01) -- (axis cs:1.41458E+02,1.52222E+01) -- cycle  ;
\draw [ecliptics-empty] (axis cs:1.42441E+02,1.49090E+01) -- (axis cs:1.42376E+02,1.47192E+01)  -- (axis cs:1.43356E+02,1.44021E+01) -- (axis cs:1.43422E+02,1.45916E+01) -- cycle  ;
\draw [ecliptics-empty] (axis cs:1.44400E+02,1.42701E+01) -- (axis cs:1.44333E+02,1.40809E+01)  -- (axis cs:1.45307E+02,1.37558E+01) -- (axis cs:1.45375E+02,1.39448E+01) -- cycle  ;
\draw [ecliptics-empty] (axis cs:1.46347E+02,1.36155E+01) -- (axis cs:1.46279E+02,1.34268E+01)  -- (axis cs:1.47248E+02,1.30941E+01) -- (axis cs:1.47317E+02,1.32826E+01) -- cycle  ;
\draw [ecliptics-empty] (axis cs:1.48283E+02,1.29460E+01) -- (axis cs:1.48214E+02,1.27578E+01)  -- (axis cs:1.49178E+02,1.24179E+01) -- (axis cs:1.49248E+02,1.26059E+01) -- cycle  ;
\draw [ecliptics-empty] (axis cs:1.50209E+02,1.22623E+01) -- (axis cs:1.50139E+02,1.20746E+01)  -- (axis cs:1.51097E+02,1.17280E+01) -- (axis cs:1.51169E+02,1.19155E+01) -- cycle  ;
\draw [ecliptics-empty] (axis cs:1.52125E+02,1.15655E+01) -- (axis cs:1.52054E+02,1.13782E+01)  -- (axis cs:1.53007E+02,1.10253E+01) -- (axis cs:1.53080E+02,1.12123E+01) -- cycle  ;
\draw [ecliptics-empty] (axis cs:1.54032E+02,1.08562E+01) -- (axis cs:1.53959E+02,1.06694E+01)  -- (axis cs:1.54908E+02,1.03105E+01) -- (axis cs:1.54981E+02,1.04971E+01) -- cycle  ;
\draw [ecliptics-empty] (axis cs:1.55929E+02,1.01353E+01) -- (axis cs:1.55855E+02,9.94894E+00)  -- (axis cs:1.56800E+02,9.58464E+00) -- (axis cs:1.56874E+02,9.77078E+00) -- cycle  ;
\draw [ecliptics-empty] (axis cs:1.57818E+02,9.40370E+00) -- (axis cs:1.57743E+02,9.21776E+00)  -- (axis cs:1.58684E+02,8.84840E+00) -- (axis cs:1.58759E+02,9.03415E+00) -- cycle  ;
\draw [ecliptics-empty] (axis cs:1.59699E+02,8.66224E+00) -- (axis cs:1.59623E+02,8.47667E+00)  -- (axis cs:1.60560E+02,8.10267E+00) -- (axis cs:1.60636E+02,8.28806E+00) -- cycle  ;
\draw [ecliptics-empty] (axis cs:1.61572E+02,7.91173E+00) -- (axis cs:1.61496E+02,7.72651E+00)  -- (axis cs:1.62430E+02,7.34829E+00) -- (axis cs:1.62506E+02,7.53334E+00) -- cycle  ;
\draw [ecliptics-empty] (axis cs:1.63439E+02,7.15301E+00) -- (axis cs:1.63362E+02,6.96811E+00)  -- (axis cs:1.64292E+02,6.58609E+00) -- (axis cs:1.64370E+02,6.77084E+00) -- cycle  ;
\draw [ecliptics-empty] (axis cs:1.65299E+02,6.38692E+00) -- (axis cs:1.65222E+02,6.20231E+00)  -- (axis cs:1.66150E+02,5.81689E+00) -- (axis cs:1.66227E+02,6.00137E+00) -- cycle  ;
\draw [ecliptics-empty] (axis cs:1.67154E+02,5.61429E+00) -- (axis cs:1.67076E+02,5.42993E+00)  -- (axis cs:1.68001E+02,5.04154E+00) -- (axis cs:1.68080E+02,5.22577E+00) -- cycle  ;
\draw [ecliptics-empty] (axis cs:1.69004E+02,4.83593E+00) -- (axis cs:1.68926E+02,4.65180E+00)  -- (axis cs:1.69849E+02,4.26084E+00) -- (axis cs:1.69927E+02,4.44487E+00) -- cycle  ;
\draw [ecliptics-empty] (axis cs:1.70850E+02,4.05268E+00) -- (axis cs:1.70771E+02,3.86875E+00)  -- (axis cs:1.71692E+02,3.47563E+00) -- (axis cs:1.71771E+02,3.65948E+00) -- cycle  ;
\draw [ecliptics-empty] (axis cs:1.72692E+02,3.26536E+00) -- (axis cs:1.72613E+02,3.08158E+00)  -- (axis cs:1.73533E+02,2.68672E+00) -- (axis cs:1.73612E+02,2.87043E+00) -- cycle  ;
\draw [ecliptics-empty] (axis cs:1.74532E+02,2.47479E+00) -- (axis cs:1.74452E+02,2.29113E+00)  -- (axis cs:1.75371E+02,1.89493E+00) -- (axis cs:1.75450E+02,2.07854E+00) -- cycle  ;
\draw [ecliptics-empty] (axis cs:1.76369E+02,1.68179E+00) -- (axis cs:1.76289E+02,1.49822E+00)  -- (axis cs:1.77207E+02,1.10109E+00) -- (axis cs:1.77287E+02,1.28463E+00) -- cycle  ;
\draw [ecliptics-empty] (axis cs:1.78205E+02,8.87166E-01) -- (axis cs:1.78125E+02,7.03652E-01)  -- (axis cs:1.79043E+02,3.06005E-01) -- (axis cs:1.79122E+02,4.89506E-01) -- cycle  ;




\draw [ecliptics-empty] (axis cs:-0.4000854E-01+360,9.17484E-02) -- (axis cs:0.0398+360,-9.17484E-02)  -- (axis cs:9.57272E-01+360,3.06005E-01) -- (axis cs:8.77725E-01+360,4.89506E-01) -- cycle  ;
\draw [ecliptics-empty] (axis cs:1.79532E+00+360,8.87166E-01) -- (axis cs:1.87485E+00+360,7.03652E-01)  -- (axis cs:2.79259E+00+360,1.10109E+00) -- (axis cs:2.71311E+00+360,1.28463E+00) -- cycle  ;
\draw [ecliptics-empty] (axis cs:3.63117E+00+360,1.68179E+00) -- (axis cs:3.71059E+00+360,1.49822E+00)  -- (axis cs:4.62893E+00+360,1.89493E+00) -- (axis cs:4.54959E+00+360,2.07854E+00) -- cycle  ;
\draw [ecliptics-empty] (axis cs:5.46845E+00+360,2.47479E+00) -- (axis cs:5.54771E+00+360,2.29113E+00)  -- (axis cs:6.46701E+00+360,2.68672E+00) -- (axis cs:6.38786E+00+360,2.87043E+00) -- cycle  ;
\draw [ecliptics-empty] (axis cs:7.30789E+00+360,3.26536E+00) -- (axis cs:7.38691E+00+360,3.08158E+00)  -- (axis cs:8.30751E+00+360,3.47563E+00) -- (axis cs:8.22863E+00+360,3.65948E+00) -- cycle  ;
\draw [ecliptics-empty] (axis cs:9.15017E+00+360,4.05268E+00) -- (axis cs:9.22889E+00+360,3.86875E+00)  -- (axis cs:1.01511E+01+360,4.26084E+00) -- (axis cs:1.00726E+01+360,4.44487E+00) -- cycle  ;
\draw [ecliptics-empty] (axis cs:1.09960E+01+360,4.83593E+00) -- (axis cs:1.10743E+01+360,4.65180E+00)  -- (axis cs:1.19986E+01+360,5.04154E+00) -- (axis cs:1.19204E+01+360,5.22577E+00) -- cycle  ;
\draw [ecliptics-empty] (axis cs:1.28460E+01+360,5.61429E+00) -- (axis cs:1.29239E+01+360,5.42993E+00)  -- (axis cs:1.38505E+01+360,5.81689E+00) -- (axis cs:1.37728E+01+360,6.00137E+00) -- cycle  ;
\draw [ecliptics-empty] (axis cs:1.47009E+01+360,6.38692E+00) -- (axis cs:1.47783E+01+360,6.20231E+00)  -- (axis cs:1.57075E+01+360,6.58609E+00) -- (axis cs:1.56304E+01+360,6.77084E+00) -- cycle  ;
\draw [ecliptics-empty] (axis cs:1.65614E+01+360,7.15301E+00) -- (axis cs:1.66382E+01+360,6.96811E+00)  -- (axis cs:1.75704E+01+360,7.34829E+00) -- (axis cs:1.74939E+01+360,7.53334E+00) -- cycle  ;
\draw [ecliptics-empty] (axis cs:1.84280E+01+360,7.91173E+00) -- (axis cs:1.85041E+01+360,7.72651E+00)  -- (axis cs:1.94396E+01+360,8.10267E+00) -- (axis cs:1.93638E+01+360,8.28806E+00) -- cycle  ;
\draw [ecliptics-empty] (axis cs:2.03014E+01+360,8.66224E+00) -- (axis cs:2.03768E+01+360,8.47667E+00)  -- (axis cs:2.13159E+01+360,8.84840E+00) -- (axis cs:2.12408E+01+360,9.03415E+00) -- cycle  ;


\draw [ecliptics-full] (axis cs:8.77725E-01+360,{4.89506E-01}) -- (axis cs:9.57272E-01+360,{3.06005E-01})  -- (axis cs:1.87485E+00+360,{7.03652E-01}) -- (axis cs:1.79532E+00+360,{8.87166E-01}) -- cycle  ;
\draw [ecliptics-full] (axis cs:2.71311E+00+360,1.28463E+00) -- (axis cs:2.79259E+00+360,1.10109E+00)  -- (axis cs:3.71059E+00+360,1.49822E+00) -- (axis cs:3.63117E+00+360,1.68179E+00) -- cycle  ;
\draw [ecliptics-full] (axis cs:4.54959E+00+360,2.07854E+00) -- (axis cs:4.62893E+00+360,1.89493E+00)  -- (axis cs:5.54771E+00+360,2.29113E+00) -- (axis cs:5.46845E+00+360,2.47479E+00) -- cycle  ;
\draw [ecliptics-full] (axis cs:6.38786E+00+360,2.87043E+00) -- (axis cs:6.46701E+00+360,2.68672E+00)  -- (axis cs:7.38691E+00+360,3.08158E+00) -- (axis cs:7.30789E+00+360,3.26536E+00) -- cycle  ;
\draw [ecliptics-full] (axis cs:8.22863E+00+360,3.65948E+00) -- (axis cs:8.30751E+00+360,3.47563E+00)  -- (axis cs:9.22889E+00+360,3.86875E+00) -- (axis cs:9.15017E+00+360,4.05268E+00) -- cycle  ;
\draw [ecliptics-full] (axis cs:1.00726E+01+360,4.44487E+00) -- (axis cs:1.01511E+01+360,4.26084E+00)  -- (axis cs:1.10743E+01+360,4.65180E+00) -- (axis cs:1.09960E+01+360,4.83593E+00) -- cycle  ;
\draw [ecliptics-full] (axis cs:1.19204E+01+360,5.22577E+00) -- (axis cs:1.19986E+01+360,5.04154E+00)  -- (axis cs:1.29239E+01+360,5.42993E+00) -- (axis cs:1.28460E+01+360,5.61429E+00) -- cycle  ;
\draw [ecliptics-full] (axis cs:1.37728E+01+360,6.00137E+00) -- (axis cs:1.38505E+01+360,5.81689E+00)  -- (axis cs:1.47783E+01+360,6.20231E+00) -- (axis cs:1.47009E+01+360,6.38692E+00) -- cycle  ;
\draw [ecliptics-full] (axis cs:1.56304E+01+360,6.77084E+00) -- (axis cs:1.57075E+01+360,6.58609E+00)  -- (axis cs:1.66382E+01+360,6.96811E+00) -- (axis cs:1.65614E+01+360,7.15301E+00) -- cycle  ;
\draw [ecliptics-full] (axis cs:1.74939E+01+360,7.53334E+00) -- (axis cs:1.75704E+01+360,7.34829E+00)  -- (axis cs:1.85041E+01+360,7.72651E+00) -- (axis cs:1.84280E+01+360,7.91173E+00) -- cycle  ;
\draw [ecliptics-full] (axis cs:1.93638E+01+360,8.28806E+00) -- (axis cs:1.94396E+01+360,8.10267E+00)  -- (axis cs:2.03768E+01+360,8.47667E+00) -- (axis cs:2.03014E+01+360,8.66224E+00) -- cycle  ;


\draw [ecliptics-empty] (axis cs:-1.59111E-01,3.66993E-01) -- (axis cs:1.59111E-01,-3.66993E-01)   ;
\node[pin={[pin distance=-0.4\onedegree,ecliptics-label]-90:{0$^\circ$}}] at (axis cs:1.59111E-01,3.66993E-01) {} ;
\draw [ecliptics-empty] (axis cs:9.03202E+00,4.32857E+00) -- (axis cs:9.34691E+00,3.59283E+00)   ;
\node[pin={[pin distance=-0.4\onedegree,ecliptics-label]90:{10$^\circ$}}] at (axis cs:9.03202E+00,4.32857E+00) {} ;
\draw [ecliptics-empty] (axis cs:1.83136E+01,8.18953E+00) -- (axis cs:1.86183E+01,7.44866E+00)   ;
\node[pin={[pin distance=-0.4\onedegree,ecliptics-label]90:{20$^\circ$}}] at (axis cs:1.83136E+01,8.18953E+00) {} ;
\draw [ecliptics-empty] (axis cs:2.77669E+01,1.18463E+01) -- (axis cs:2.80539E+01,1.10973E+01)   ;
\node[pin={[pin distance=-0.4\onedegree,ecliptics-label]90:{30$^\circ$}}] at (axis cs:2.77669E+01,1.18463E+01) {} ;
\draw [ecliptics-empty] (axis cs:3.74606E+01,1.51936E+01) -- (axis cs:3.77214E+01,1.44344E+01)   ;
\node[pin={[pin distance=-0.4\onedegree,ecliptics-label]90:{40$^\circ$}}] at (axis cs:3.74606E+01,1.51936E+01) {} ;
\draw [ecliptics-empty] (axis cs:4.74420E+01,1.81261E+01) -- (axis cs:4.76675E+01,1.73555E+01)   ;
\node[pin={[pin distance=-0.4\onedegree,ecliptics-label]90:{50$^\circ$}}] at (axis cs:4.74420E+01,1.81261E+01) {} ;
\draw [ecliptics-empty] (axis cs:5.77283E+01,2.05410E+01) -- (axis cs:5.79088E+01,1.97592E+01)   ;
\node[pin={[pin distance=-0.4\onedegree,ecliptics-label]90:{60$^\circ$}}] at (axis cs:5.77283E+01,2.05410E+01) {} ;
\draw [ecliptics-empty] (axis cs:6.82981E+01,2.23448E+01) -- (axis cs:6.84246E+01,2.15535E+01)   ;
\node[pin={[pin distance=-0.4\onedegree,ecliptics-label]90:{70$^\circ$}}] at (axis cs:6.82981E+01,2.23448E+01) {} ;
\draw [ecliptics-empty] (axis cs:7.90885E+01,2.34610E+01) -- (axis cs:7.91538E+01,2.26633E+01)   ;
\node[pin={[pin distance=-0.4\onedegree,ecliptics-label]90:{80$^\circ$}}] at (axis cs:7.90885E+01,2.34610E+01) {} ;
\draw [ecliptics-empty] (axis cs:9.00000E+01,2.38390E+01) -- (axis cs:9.00000E+01,2.30390E+01)   ;
\node[pin={[pin distance=-0.4\onedegree,ecliptics-label]90:{90$^\circ$}}] at (axis cs:9.00000E+01,2.38390E+01) {} ;
\draw [ecliptics-empty] (axis cs:1.00911E+02,2.34610E+01) -- (axis cs:1.00846E+02,2.26633E+01)   ;
\node[pin={[pin distance=-0.4\onedegree,ecliptics-label]90:{100$^\circ$}}] at (axis cs:1.00911E+02,2.34610E+01) {} ;
\draw [ecliptics-empty] (axis cs:1.11702E+02,2.23448E+01) -- (axis cs:1.11575E+02,2.15535E+01)   ;
\node[pin={[pin distance=-0.4\onedegree,ecliptics-label]90:{110$^\circ$}}] at (axis cs:1.11702E+02,2.23448E+01) {} ;
\draw [ecliptics-empty] (axis cs:1.22272E+02,2.05410E+01) -- (axis cs:1.22091E+02,1.97592E+01)   ;
\node[pin={[pin distance=-0.4\onedegree,ecliptics-label]90:{120$^\circ$}}] at (axis cs:1.22272E+02,2.05410E+01) {} ;
\draw [ecliptics-empty] (axis cs:1.32558E+02,1.81261E+01) -- (axis cs:1.32332E+02,1.73555E+01)   ;
\node[pin={[pin distance=-0.4\onedegree,ecliptics-label]90:{130$^\circ$}}] at (axis cs:1.32558E+02,1.81261E+01) {} ;
\draw [ecliptics-empty] (axis cs:1.42539E+02,1.51936E+01) -- (axis cs:1.42279E+02,1.44344E+01)   ;
\node[pin={[pin distance=-0.4\onedegree,ecliptics-label]90:{140$^\circ$}}] at (axis cs:1.42539E+02,1.51936E+01) {} ;
\draw [ecliptics-empty] (axis cs:1.52233E+02,1.18463E+01) -- (axis cs:1.51946E+02,1.10973E+01)   ;
\node[pin={[pin distance=0\onedegree,ecliptics-label]-45:{150$^\circ$}}] at (axis cs:1.52233E+02,1.18463E+01) {} ;
\draw [ecliptics-empty] (axis cs:1.61686E+02,8.18953E+00) -- (axis cs:1.61382E+02,7.44866E+00)   ;
\node[pin={[pin distance=-0.4\onedegree,ecliptics-label]90:{160$^\circ$}}] at (axis cs:1.61686E+02,8.18953E+00) {} ;
\draw [ecliptics-empty] (axis cs:1.70968E+02,4.32857E+00) -- (axis cs:1.70653E+02,3.59283E+00)   ;
\node[pin={[pin distance=-0.4\onedegree,ecliptics-label]90:{170$^\circ$}}] at (axis cs:1.70968E+02,4.32857E+00) {} ;
\draw [ecliptics-empty] (axis cs:360-1.79841E+02,3.66993E-01) -- (axis cs:1.79841E+02,-3.66993E-01)   ;
%\node[pin={[pin distance=0.4\onedegree,ecliptics-label]-90:{180$^\circ$}}] at (axis cs:1.79841E+02,3.66993E-01) {} ;

\draw [ecliptics-empty] (axis cs:3.50653E+02,-3.59283E+00) -- (axis cs:3.50968E+02,-4.32857E+00)   ;
\node[pin={[pin distance=-0.4\onedegree,ecliptics-label]90:{350$^\circ$}}] at (axis cs:3.50653E+02,-3.59283E+00) {} ;
\draw [ecliptics-empty] (axis cs:3.41382E+02,-7.44866E+00) -- (axis cs:3.41686E+02,-8.18953E+00)   ;
\node[pin={[pin distance=-0.4\onedegree,ecliptics-label]90:{340$^\circ$}}] at (axis cs:3.41382E+02,-7.44866E+00) {} ;
\draw [ecliptics-empty] (axis cs:3.31946E+02,-1.10973E+01) -- (axis cs:3.32233E+02,-1.18463E+01)   ;
\node[pin={[pin distance=-0.4\onedegree,ecliptics-label]90:{330$^\circ$}}] at (axis cs:3.31946E+02,-1.10973E+01) {} ;
\draw [ecliptics-empty] (axis cs:3.22279E+02,-1.44344E+01) -- (axis cs:3.22539E+02,-1.51936E+01)   ;
\node[pin={[pin distance=-0.4\onedegree,ecliptics-label]90:{320$^\circ$}}] at (axis cs:3.22279E+02,-1.44344E+01) {} ;
\draw [ecliptics-empty] (axis cs:3.12332E+02,-1.73555E+01) -- (axis cs:3.12558E+02,-1.81261E+01)   ;
\node[pin={[pin distance=-0.4\onedegree,ecliptics-label]90:{310$^\circ$}}] at (axis cs:3.12332E+02,-1.73555E+01) {} ;
\draw [ecliptics-empty] (axis cs:3.02091E+02,-1.97592E+01) -- (axis cs:3.02272E+02,-2.05410E+01)   ;
\node[pin={[pin distance=-0.4\onedegree,ecliptics-label]90:{300$^\circ$}}] at (axis cs:3.02091E+02,-1.97592E+01) {} ;
\draw [ecliptics-empty] (axis cs:2.91575E+02,-2.15535E+01) -- (axis cs:2.91702E+02,-2.23448E+01)   ;
\node[pin={[pin distance=-0.4\onedegree,ecliptics-label]90:{290$^\circ$}}] at (axis cs:2.91575E+02,-2.15535E+01) {} ;
\draw [ecliptics-empty] (axis cs:2.80846E+02,-2.26633E+01) -- (axis cs:2.80911E+02,-2.34610E+01)   ;
\node[pin={[pin distance=-0.4\onedegree,ecliptics-label]90:{280$^\circ$}}] at (axis cs:2.80846E+02,-2.26633E+01) {} ;
\draw [ecliptics-empty] (axis cs:2.70000E+02,-2.30390E+01) -- (axis cs:2.70000E+02,-2.38390E+01)   ;
\node[pin={[pin distance=-0.4\onedegree,ecliptics-label]90:{270$^\circ$}}] at (axis cs:2.70000E+02,-2.30390E+01) {} ;
\draw [ecliptics-empty] (axis cs:2.59154E+02,-2.26633E+01) -- (axis cs:2.59089E+02,-2.34610E+01)   ;
\node[pin={[pin distance=-0.4\onedegree,ecliptics-label]90:{260$^\circ$}}] at (axis cs:2.59154E+02,-2.26633E+01) {} ;
\draw [ecliptics-empty] (axis cs:2.48425E+02,-2.15535E+01) -- (axis cs:2.48298E+02,-2.23448E+01)   ;
\node[pin={[pin distance=-0.4\onedegree,ecliptics-label]90:{250$^\circ$}}] at (axis cs:2.48425E+02,-2.15535E+01) {} ;
\draw [ecliptics-empty] (axis cs:2.37909E+02,-1.97592E+01) -- (axis cs:2.37728E+02,-2.05410E+01)   ;
\node[pin={[pin distance=-0.4\onedegree,ecliptics-label]90:{240$^\circ$}}] at (axis cs:2.37909E+02,-1.97592E+01) {} ;
\draw [ecliptics-empty] (axis cs:2.27668E+02,-1.73555E+01) -- (axis cs:2.27442E+02,-1.81261E+01)   ;
\node[pin={[pin distance=-0.4\onedegree,ecliptics-label]90:{230$^\circ$}}] at (axis cs:2.27668E+02,-1.73555E+01) {} ;
\draw [ecliptics-empty] (axis cs:2.17721E+02,-1.44344E+01) -- (axis cs:2.17461E+02,-1.51936E+01)   ;
\node[pin={[pin distance=-0.4\onedegree,ecliptics-label]90:{220$^\circ$}}] at (axis cs:2.17721E+02,-1.44344E+01) {} ;
\draw [ecliptics-empty] (axis cs:2.08054E+02,-1.10973E+01) -- (axis cs:2.07767E+02,-1.18463E+01)   ;
\node[pin={[pin distance=-0.4\onedegree,ecliptics-label]90:{210$^\circ$}}] at (axis cs:2.08054E+02,-1.10973E+01) {} ;
\draw [ecliptics-empty] (axis cs:1.98618E+02,-7.44866E+00) -- (axis cs:1.98314E+02,-8.18953E+00)   ;
\node[pin={[pin distance=-0.4\onedegree,ecliptics-label]90:{200$^\circ$}}] at (axis cs:1.98618E+02,-7.44866E+00) {} ;
\draw [ecliptics-empty] (axis cs:1.89347E+02,-3.59283E+00) -- (axis cs:1.89032E+02,-4.32857E+00)   ;
\node[pin={[pin distance=-0.4\onedegree,ecliptics-label]90:{190$^\circ$}}] at (axis cs:1.89347E+02,-3.59283E+00) {} ;

%\draw [ecliptics-empty] (axis cs:-1.59111E-01+360,3.66993E-01) -- (axis cs:1.59111E-01+360,-3.66993E-01)   ;
%\node[pin={[pin distance=-0.4\onedegree,ecliptics-label]-90:{0$^\circ$}}] at (axis cs:1.59111E-01+360,3.66993E-01) {} ;
\draw [ecliptics-empty] (axis cs:9.03202E+00+360,4.32857E+00) -- (axis cs:9.34691E+00+360,3.59283E+00)   ;
\node[pin={[pin distance=-0.4\onedegree,ecliptics-label]90:{10$^\circ$}}] at (axis cs:9.03202E+00+360,4.32857E+00) {} ;
\draw [ecliptics-empty] (axis cs:1.83136E+01+360,8.18953E+00) -- (axis cs:1.86183E+01+360,7.44866E+00)   ;
\node[pin={[pin distance=-0.4\onedegree,ecliptics-label]90:{20$^\circ$}}] at (axis cs:1.83136E+01+360,8.18953E+00) {} ;



\end{axis}

% Equatorial coordinate system
%
% Some are commented out because they are surrounded by already plotted borders
% The x-coordinate is in fractional hours, so must be times 15
%

\begin{axis}[name=constellations,at=(base.center),anchor=center,axis lines=none]


\draw[Equator-full] (axis cs:0.00000E+00,1.00000E-01) -- (axis cs:1.00000E+00,1.00000E-01) -- (axis cs:1.00000E+00,-1.00000E-01) -- (axis cs:0.00000E+00,-1.00000E-01) -- cycle ; 
\draw[Equator-full] (axis cs:2.00000E+00,1.00000E-01) -- (axis cs:3.00000E+00,1.00000E-01) -- (axis cs:3.00000E+00,-1.00000E-01) -- (axis cs:2.00000E+00,-1.00000E-01) -- cycle ; 
\draw[Equator-full] (axis cs:4.00000E+00,1.00000E-01) -- (axis cs:5.00000E+00,1.00000E-01) -- (axis cs:5.00000E+00,-1.00000E-01) -- (axis cs:4.00000E+00,-1.00000E-01) -- cycle ; 
\draw[Equator-full] (axis cs:6.00000E+00,1.00000E-01) -- (axis cs:7.00000E+00,1.00000E-01) -- (axis cs:7.00000E+00,-1.00000E-01) -- (axis cs:6.00000E+00,-1.00000E-01) -- cycle ; 
\draw[Equator-full] (axis cs:8.00000E+00,1.00000E-01) -- (axis cs:9.00000E+00,1.00000E-01) -- (axis cs:9.00000E+00,-1.00000E-01) -- (axis cs:8.00000E+00,-1.00000E-01) -- cycle ; 
\draw[Equator-full] (axis cs:1.00000E+01,1.00000E-01) -- (axis cs:1.10000E+01,1.00000E-01) -- (axis cs:1.10000E+01,-1.00000E-01) -- (axis cs:1.00000E+01,-1.00000E-01) -- cycle ; 
\draw[Equator-full] (axis cs:1.20000E+01,1.00000E-01) -- (axis cs:1.30000E+01,1.00000E-01) -- (axis cs:1.30000E+01,-1.00000E-01) -- (axis cs:1.20000E+01,-1.00000E-01) -- cycle ; 
\draw[Equator-full] (axis cs:1.40000E+01,1.00000E-01) -- (axis cs:1.50000E+01,1.00000E-01) -- (axis cs:1.50000E+01,-1.00000E-01) -- (axis cs:1.40000E+01,-1.00000E-01) -- cycle ; 
\draw[Equator-full] (axis cs:1.60000E+01,1.00000E-01) -- (axis cs:1.70000E+01,1.00000E-01) -- (axis cs:1.70000E+01,-1.00000E-01) -- (axis cs:1.60000E+01,-1.00000E-01) -- cycle ; 
\draw[Equator-full] (axis cs:1.80000E+01,1.00000E-01) -- (axis cs:1.90000E+01,1.00000E-01) -- (axis cs:1.90000E+01,-1.00000E-01) -- (axis cs:1.80000E+01,-1.00000E-01) -- cycle ; 
\draw[Equator-full] (axis cs:2.00000E+01,1.00000E-01) -- (axis cs:2.10000E+01,1.00000E-01) -- (axis cs:2.10000E+01,-1.00000E-01) -- (axis cs:2.00000E+01,-1.00000E-01) -- cycle ; 
\draw[Equator-full] (axis cs:2.20000E+01,1.00000E-01) -- (axis cs:2.30000E+01,1.00000E-01) -- (axis cs:2.30000E+01,-1.00000E-01) -- (axis cs:2.20000E+01,-1.00000E-01) -- cycle ; 
\draw[Equator-full] (axis cs:2.40000E+01,1.00000E-01) -- (axis cs:2.50000E+01,1.00000E-01) -- (axis cs:2.50000E+01,-1.00000E-01) -- (axis cs:2.40000E+01,-1.00000E-01) -- cycle ; 
\draw[Equator-full] (axis cs:2.60000E+01,1.00000E-01) -- (axis cs:2.70000E+01,1.00000E-01) -- (axis cs:2.70000E+01,-1.00000E-01) -- (axis cs:2.60000E+01,-1.00000E-01) -- cycle ; 
\draw[Equator-full] (axis cs:2.80000E+01,1.00000E-01) -- (axis cs:2.90000E+01,1.00000E-01) -- (axis cs:2.90000E+01,-1.00000E-01) -- (axis cs:2.80000E+01,-1.00000E-01) -- cycle ; 
\draw[Equator-full] (axis cs:3.00000E+01,1.00000E-01) -- (axis cs:3.10000E+01,1.00000E-01) -- (axis cs:3.10000E+01,-1.00000E-01) -- (axis cs:3.00000E+01,-1.00000E-01) -- cycle ; 
\draw[Equator-full] (axis cs:3.20000E+01,1.00000E-01) -- (axis cs:3.30000E+01,1.00000E-01) -- (axis cs:3.30000E+01,-1.00000E-01) -- (axis cs:3.20000E+01,-1.00000E-01) -- cycle ; 
\draw[Equator-full] (axis cs:3.40000E+01,1.00000E-01) -- (axis cs:3.50000E+01,1.00000E-01) -- (axis cs:3.50000E+01,-1.00000E-01) -- (axis cs:3.40000E+01,-1.00000E-01) -- cycle ; 
\draw[Equator-full] (axis cs:3.60000E+01,1.00000E-01) -- (axis cs:3.70000E+01,1.00000E-01) -- (axis cs:3.70000E+01,-1.00000E-01) -- (axis cs:3.60000E+01,-1.00000E-01) -- cycle ; 
\draw[Equator-full] (axis cs:3.80000E+01,1.00000E-01) -- (axis cs:3.90000E+01,1.00000E-01) -- (axis cs:3.90000E+01,-1.00000E-01) -- (axis cs:3.80000E+01,-1.00000E-01) -- cycle ; 
\draw[Equator-full] (axis cs:4.00000E+01,1.00000E-01) -- (axis cs:4.10000E+01,1.00000E-01) -- (axis cs:4.10000E+01,-1.00000E-01) -- (axis cs:4.00000E+01,-1.00000E-01) -- cycle ; 
\draw[Equator-full] (axis cs:4.20000E+01,1.00000E-01) -- (axis cs:4.30000E+01,1.00000E-01) -- (axis cs:4.30000E+01,-1.00000E-01) -- (axis cs:4.20000E+01,-1.00000E-01) -- cycle ; 
\draw[Equator-full] (axis cs:4.40000E+01,1.00000E-01) -- (axis cs:4.50000E+01,1.00000E-01) -- (axis cs:4.50000E+01,-1.00000E-01) -- (axis cs:4.40000E+01,-1.00000E-01) -- cycle ; 
\draw[Equator-full] (axis cs:4.60000E+01,1.00000E-01) -- (axis cs:4.70000E+01,1.00000E-01) -- (axis cs:4.70000E+01,-1.00000E-01) -- (axis cs:4.60000E+01,-1.00000E-01) -- cycle ; 
\draw[Equator-full] (axis cs:4.80000E+01,1.00000E-01) -- (axis cs:4.90000E+01,1.00000E-01) -- (axis cs:4.90000E+01,-1.00000E-01) -- (axis cs:4.80000E+01,-1.00000E-01) -- cycle ; 
\draw[Equator-full] (axis cs:5.00000E+01,1.00000E-01) -- (axis cs:5.10000E+01,1.00000E-01) -- (axis cs:5.10000E+01,-1.00000E-01) -- (axis cs:5.00000E+01,-1.00000E-01) -- cycle ; 
\draw[Equator-full] (axis cs:5.20000E+01,1.00000E-01) -- (axis cs:5.30000E+01,1.00000E-01) -- (axis cs:5.30000E+01,-1.00000E-01) -- (axis cs:5.20000E+01,-1.00000E-01) -- cycle ; 
\draw[Equator-full] (axis cs:5.40000E+01,1.00000E-01) -- (axis cs:5.50000E+01,1.00000E-01) -- (axis cs:5.50000E+01,-1.00000E-01) -- (axis cs:5.40000E+01,-1.00000E-01) -- cycle ; 
\draw[Equator-full] (axis cs:5.60000E+01,1.00000E-01) -- (axis cs:5.70000E+01,1.00000E-01) -- (axis cs:5.70000E+01,-1.00000E-01) -- (axis cs:5.60000E+01,-1.00000E-01) -- cycle ; 
\draw[Equator-full] (axis cs:5.80000E+01,1.00000E-01) -- (axis cs:5.90000E+01,1.00000E-01) -- (axis cs:5.90000E+01,-1.00000E-01) -- (axis cs:5.80000E+01,-1.00000E-01) -- cycle ; 
\draw[Equator-full] (axis cs:6.00000E+01,1.00000E-01) -- (axis cs:6.10000E+01,1.00000E-01) -- (axis cs:6.10000E+01,-1.00000E-01) -- (axis cs:6.00000E+01,-1.00000E-01) -- cycle ; 
\draw[Equator-full] (axis cs:6.20000E+01,1.00000E-01) -- (axis cs:6.30000E+01,1.00000E-01) -- (axis cs:6.30000E+01,-1.00000E-01) -- (axis cs:6.20000E+01,-1.00000E-01) -- cycle ; 
\draw[Equator-full] (axis cs:6.40000E+01,1.00000E-01) -- (axis cs:6.50000E+01,1.00000E-01) -- (axis cs:6.50000E+01,-1.00000E-01) -- (axis cs:6.40000E+01,-1.00000E-01) -- cycle ; 
\draw[Equator-full] (axis cs:6.60000E+01,1.00000E-01) -- (axis cs:6.70000E+01,1.00000E-01) -- (axis cs:6.70000E+01,-1.00000E-01) -- (axis cs:6.60000E+01,-1.00000E-01) -- cycle ; 
\draw[Equator-full] (axis cs:6.80000E+01,1.00000E-01) -- (axis cs:6.90000E+01,1.00000E-01) -- (axis cs:6.90000E+01,-1.00000E-01) -- (axis cs:6.80000E+01,-1.00000E-01) -- cycle ; 
\draw[Equator-full] (axis cs:7.00000E+01,1.00000E-01) -- (axis cs:7.10000E+01,1.00000E-01) -- (axis cs:7.10000E+01,-1.00000E-01) -- (axis cs:7.00000E+01,-1.00000E-01) -- cycle ; 
\draw[Equator-full] (axis cs:7.20000E+01,1.00000E-01) -- (axis cs:7.30000E+01,1.00000E-01) -- (axis cs:7.30000E+01,-1.00000E-01) -- (axis cs:7.20000E+01,-1.00000E-01) -- cycle ; 
\draw[Equator-full] (axis cs:7.40000E+01,1.00000E-01) -- (axis cs:7.50000E+01,1.00000E-01) -- (axis cs:7.50000E+01,-1.00000E-01) -- (axis cs:7.40000E+01,-1.00000E-01) -- cycle ; 
\draw[Equator-full] (axis cs:7.60000E+01,1.00000E-01) -- (axis cs:7.70000E+01,1.00000E-01) -- (axis cs:7.70000E+01,-1.00000E-01) -- (axis cs:7.60000E+01,-1.00000E-01) -- cycle ; 
\draw[Equator-full] (axis cs:7.80000E+01,1.00000E-01) -- (axis cs:7.90000E+01,1.00000E-01) -- (axis cs:7.90000E+01,-1.00000E-01) -- (axis cs:7.80000E+01,-1.00000E-01) -- cycle ; 
\draw[Equator-full] (axis cs:8.00000E+01,1.00000E-01) -- (axis cs:8.10000E+01,1.00000E-01) -- (axis cs:8.10000E+01,-1.00000E-01) -- (axis cs:8.00000E+01,-1.00000E-01) -- cycle ; 
\draw[Equator-full] (axis cs:8.20000E+01,1.00000E-01) -- (axis cs:8.30000E+01,1.00000E-01) -- (axis cs:8.30000E+01,-1.00000E-01) -- (axis cs:8.20000E+01,-1.00000E-01) -- cycle ; 
\draw[Equator-full] (axis cs:8.40000E+01,1.00000E-01) -- (axis cs:8.50000E+01,1.00000E-01) -- (axis cs:8.50000E+01,-1.00000E-01) -- (axis cs:8.40000E+01,-1.00000E-01) -- cycle ; 
\draw[Equator-full] (axis cs:8.60000E+01,1.00000E-01) -- (axis cs:8.70000E+01,1.00000E-01) -- (axis cs:8.70000E+01,-1.00000E-01) -- (axis cs:8.60000E+01,-1.00000E-01) -- cycle ; 
\draw[Equator-full] (axis cs:8.80000E+01,1.00000E-01) -- (axis cs:8.90000E+01,1.00000E-01) -- (axis cs:8.90000E+01,-1.00000E-01) -- (axis cs:8.80000E+01,-1.00000E-01) -- cycle ; 
\draw[Equator-full] (axis cs:9.00000E+01,1.00000E-01) -- (axis cs:9.10000E+01,1.00000E-01) -- (axis cs:9.10000E+01,-1.00000E-01) -- (axis cs:9.00000E+01,-1.00000E-01) -- cycle ; 
\draw[Equator-full] (axis cs:9.20000E+01,1.00000E-01) -- (axis cs:9.30000E+01,1.00000E-01) -- (axis cs:9.30000E+01,-1.00000E-01) -- (axis cs:9.20000E+01,-1.00000E-01) -- cycle ; 
\draw[Equator-full] (axis cs:9.40000E+01,1.00000E-01) -- (axis cs:9.50000E+01,1.00000E-01) -- (axis cs:9.50000E+01,-1.00000E-01) -- (axis cs:9.40000E+01,-1.00000E-01) -- cycle ; 
\draw[Equator-full] (axis cs:9.60000E+01,1.00000E-01) -- (axis cs:9.70000E+01,1.00000E-01) -- (axis cs:9.70000E+01,-1.00000E-01) -- (axis cs:9.60000E+01,-1.00000E-01) -- cycle ; 
\draw[Equator-full] (axis cs:9.80000E+01,1.00000E-01) -- (axis cs:9.90000E+01,1.00000E-01) -- (axis cs:9.90000E+01,-1.00000E-01) -- (axis cs:9.80000E+01,-1.00000E-01) -- cycle ; 
\draw[Equator-full] (axis cs:1.00000E+02,1.00000E-01) -- (axis cs:1.01000E+02,1.00000E-01) -- (axis cs:1.01000E+02,-1.00000E-01) -- (axis cs:1.00000E+02,-1.00000E-01) -- cycle ; 
\draw[Equator-full] (axis cs:1.02000E+02,1.00000E-01) -- (axis cs:1.03000E+02,1.00000E-01) -- (axis cs:1.03000E+02,-1.00000E-01) -- (axis cs:1.02000E+02,-1.00000E-01) -- cycle ; 
\draw[Equator-full] (axis cs:1.04000E+02,1.00000E-01) -- (axis cs:1.05000E+02,1.00000E-01) -- (axis cs:1.05000E+02,-1.00000E-01) -- (axis cs:1.04000E+02,-1.00000E-01) -- cycle ; 
\draw[Equator-full] (axis cs:1.06000E+02,1.00000E-01) -- (axis cs:1.07000E+02,1.00000E-01) -- (axis cs:1.07000E+02,-1.00000E-01) -- (axis cs:1.06000E+02,-1.00000E-01) -- cycle ; 
\draw[Equator-full] (axis cs:1.08000E+02,1.00000E-01) -- (axis cs:1.09000E+02,1.00000E-01) -- (axis cs:1.09000E+02,-1.00000E-01) -- (axis cs:1.08000E+02,-1.00000E-01) -- cycle ; 
\draw[Equator-full] (axis cs:1.10000E+02,1.00000E-01) -- (axis cs:1.11000E+02,1.00000E-01) -- (axis cs:1.11000E+02,-1.00000E-01) -- (axis cs:1.10000E+02,-1.00000E-01) -- cycle ; 
\draw[Equator-full] (axis cs:1.12000E+02,1.00000E-01) -- (axis cs:1.13000E+02,1.00000E-01) -- (axis cs:1.13000E+02,-1.00000E-01) -- (axis cs:1.12000E+02,-1.00000E-01) -- cycle ; 
\draw[Equator-full] (axis cs:1.14000E+02,1.00000E-01) -- (axis cs:1.15000E+02,1.00000E-01) -- (axis cs:1.15000E+02,-1.00000E-01) -- (axis cs:1.14000E+02,-1.00000E-01) -- cycle ; 
\draw[Equator-full] (axis cs:1.16000E+02,1.00000E-01) -- (axis cs:1.17000E+02,1.00000E-01) -- (axis cs:1.17000E+02,-1.00000E-01) -- (axis cs:1.16000E+02,-1.00000E-01) -- cycle ; 
\draw[Equator-full] (axis cs:1.18000E+02,1.00000E-01) -- (axis cs:1.19000E+02,1.00000E-01) -- (axis cs:1.19000E+02,-1.00000E-01) -- (axis cs:1.18000E+02,-1.00000E-01) -- cycle ; 
\draw[Equator-full] (axis cs:1.20000E+02,1.00000E-01) -- (axis cs:1.21000E+02,1.00000E-01) -- (axis cs:1.21000E+02,-1.00000E-01) -- (axis cs:1.20000E+02,-1.00000E-01) -- cycle ; 
\draw[Equator-full] (axis cs:1.22000E+02,1.00000E-01) -- (axis cs:1.23000E+02,1.00000E-01) -- (axis cs:1.23000E+02,-1.00000E-01) -- (axis cs:1.22000E+02,-1.00000E-01) -- cycle ; 
\draw[Equator-full] (axis cs:1.24000E+02,1.00000E-01) -- (axis cs:1.25000E+02,1.00000E-01) -- (axis cs:1.25000E+02,-1.00000E-01) -- (axis cs:1.24000E+02,-1.00000E-01) -- cycle ; 
\draw[Equator-full] (axis cs:1.26000E+02,1.00000E-01) -- (axis cs:1.27000E+02,1.00000E-01) -- (axis cs:1.27000E+02,-1.00000E-01) -- (axis cs:1.26000E+02,-1.00000E-01) -- cycle ; 
\draw[Equator-full] (axis cs:1.28000E+02,1.00000E-01) -- (axis cs:1.29000E+02,1.00000E-01) -- (axis cs:1.29000E+02,-1.00000E-01) -- (axis cs:1.28000E+02,-1.00000E-01) -- cycle ; 
\draw[Equator-full] (axis cs:1.30000E+02,1.00000E-01) -- (axis cs:1.31000E+02,1.00000E-01) -- (axis cs:1.31000E+02,-1.00000E-01) -- (axis cs:1.30000E+02,-1.00000E-01) -- cycle ; 
\draw[Equator-full] (axis cs:1.32000E+02,1.00000E-01) -- (axis cs:1.33000E+02,1.00000E-01) -- (axis cs:1.33000E+02,-1.00000E-01) -- (axis cs:1.32000E+02,-1.00000E-01) -- cycle ; 
\draw[Equator-full] (axis cs:1.34000E+02,1.00000E-01) -- (axis cs:1.35000E+02,1.00000E-01) -- (axis cs:1.35000E+02,-1.00000E-01) -- (axis cs:1.34000E+02,-1.00000E-01) -- cycle ; 
\draw[Equator-full] (axis cs:1.36000E+02,1.00000E-01) -- (axis cs:1.37000E+02,1.00000E-01) -- (axis cs:1.37000E+02,-1.00000E-01) -- (axis cs:1.36000E+02,-1.00000E-01) -- cycle ; 
\draw[Equator-full] (axis cs:1.38000E+02,1.00000E-01) -- (axis cs:1.39000E+02,1.00000E-01) -- (axis cs:1.39000E+02,-1.00000E-01) -- (axis cs:1.38000E+02,-1.00000E-01) -- cycle ; 
\draw[Equator-full] (axis cs:1.40000E+02,1.00000E-01) -- (axis cs:1.41000E+02,1.00000E-01) -- (axis cs:1.41000E+02,-1.00000E-01) -- (axis cs:1.40000E+02,-1.00000E-01) -- cycle ; 
\draw[Equator-full] (axis cs:1.42000E+02,1.00000E-01) -- (axis cs:1.43000E+02,1.00000E-01) -- (axis cs:1.43000E+02,-1.00000E-01) -- (axis cs:1.42000E+02,-1.00000E-01) -- cycle ; 
\draw[Equator-full] (axis cs:1.44000E+02,1.00000E-01) -- (axis cs:1.45000E+02,1.00000E-01) -- (axis cs:1.45000E+02,-1.00000E-01) -- (axis cs:1.44000E+02,-1.00000E-01) -- cycle ; 
\draw[Equator-full] (axis cs:1.46000E+02,1.00000E-01) -- (axis cs:1.47000E+02,1.00000E-01) -- (axis cs:1.47000E+02,-1.00000E-01) -- (axis cs:1.46000E+02,-1.00000E-01) -- cycle ; 
\draw[Equator-full] (axis cs:1.48000E+02,1.00000E-01) -- (axis cs:1.49000E+02,1.00000E-01) -- (axis cs:1.49000E+02,-1.00000E-01) -- (axis cs:1.48000E+02,-1.00000E-01) -- cycle ; 
\draw[Equator-full] (axis cs:1.50000E+02,1.00000E-01) -- (axis cs:1.51000E+02,1.00000E-01) -- (axis cs:1.51000E+02,-1.00000E-01) -- (axis cs:1.50000E+02,-1.00000E-01) -- cycle ; 
\draw[Equator-full] (axis cs:1.52000E+02,1.00000E-01) -- (axis cs:1.53000E+02,1.00000E-01) -- (axis cs:1.53000E+02,-1.00000E-01) -- (axis cs:1.52000E+02,-1.00000E-01) -- cycle ; 
\draw[Equator-full] (axis cs:1.54000E+02,1.00000E-01) -- (axis cs:1.55000E+02,1.00000E-01) -- (axis cs:1.55000E+02,-1.00000E-01) -- (axis cs:1.54000E+02,-1.00000E-01) -- cycle ; 
\draw[Equator-full] (axis cs:1.56000E+02,1.00000E-01) -- (axis cs:1.57000E+02,1.00000E-01) -- (axis cs:1.57000E+02,-1.00000E-01) -- (axis cs:1.56000E+02,-1.00000E-01) -- cycle ; 
\draw[Equator-full] (axis cs:1.58000E+02,1.00000E-01) -- (axis cs:1.59000E+02,1.00000E-01) -- (axis cs:1.59000E+02,-1.00000E-01) -- (axis cs:1.58000E+02,-1.00000E-01) -- cycle ; 
\draw[Equator-full] (axis cs:1.60000E+02,1.00000E-01) -- (axis cs:1.61000E+02,1.00000E-01) -- (axis cs:1.61000E+02,-1.00000E-01) -- (axis cs:1.60000E+02,-1.00000E-01) -- cycle ; 
\draw[Equator-full] (axis cs:1.62000E+02,1.00000E-01) -- (axis cs:1.63000E+02,1.00000E-01) -- (axis cs:1.63000E+02,-1.00000E-01) -- (axis cs:1.62000E+02,-1.00000E-01) -- cycle ; 
\draw[Equator-full] (axis cs:1.64000E+02,1.00000E-01) -- (axis cs:1.65000E+02,1.00000E-01) -- (axis cs:1.65000E+02,-1.00000E-01) -- (axis cs:1.64000E+02,-1.00000E-01) -- cycle ; 
\draw[Equator-full] (axis cs:1.66000E+02,1.00000E-01) -- (axis cs:1.67000E+02,1.00000E-01) -- (axis cs:1.67000E+02,-1.00000E-01) -- (axis cs:1.66000E+02,-1.00000E-01) -- cycle ; 
\draw[Equator-full] (axis cs:1.68000E+02,1.00000E-01) -- (axis cs:1.69000E+02,1.00000E-01) -- (axis cs:1.69000E+02,-1.00000E-01) -- (axis cs:1.68000E+02,-1.00000E-01) -- cycle ; 
\draw[Equator-full] (axis cs:1.70000E+02,1.00000E-01) -- (axis cs:1.71000E+02,1.00000E-01) -- (axis cs:1.71000E+02,-1.00000E-01) -- (axis cs:1.70000E+02,-1.00000E-01) -- cycle ; 
\draw[Equator-full] (axis cs:1.72000E+02,1.00000E-01) -- (axis cs:1.73000E+02,1.00000E-01) -- (axis cs:1.73000E+02,-1.00000E-01) -- (axis cs:1.72000E+02,-1.00000E-01) -- cycle ; 
\draw[Equator-full] (axis cs:1.74000E+02,1.00000E-01) -- (axis cs:1.75000E+02,1.00000E-01) -- (axis cs:1.75000E+02,-1.00000E-01) -- (axis cs:1.74000E+02,-1.00000E-01) -- cycle ; 
\draw[Equator-full] (axis cs:1.76000E+02,1.00000E-01) -- (axis cs:1.77000E+02,1.00000E-01) -- (axis cs:1.77000E+02,-1.00000E-01) -- (axis cs:1.76000E+02,-1.00000E-01) -- cycle ; 
\draw[Equator-full] (axis cs:1.78000E+02,1.00000E-01) -- (axis cs:1.79000E+02,1.00000E-01) -- (axis cs:1.79000E+02,-1.00000E-01) -- (axis cs:1.78000E+02,-1.00000E-01) -- cycle ; 
\draw[Equator-full] (axis cs:1.80000E+02,1.00000E-01) -- (axis cs:1.81000E+02,1.00000E-01) -- (axis cs:1.81000E+02,-1.00000E-01) -- (axis cs:1.80000E+02,-1.00000E-01) -- cycle ; 
\draw[Equator-full] (axis cs:1.82000E+02,1.00000E-01) -- (axis cs:1.83000E+02,1.00000E-01) -- (axis cs:1.83000E+02,-1.00000E-01) -- (axis cs:1.82000E+02,-1.00000E-01) -- cycle ; 
\draw[Equator-full] (axis cs:1.84000E+02,1.00000E-01) -- (axis cs:1.85000E+02,1.00000E-01) -- (axis cs:1.85000E+02,-1.00000E-01) -- (axis cs:1.84000E+02,-1.00000E-01) -- cycle ; 
\draw[Equator-full] (axis cs:1.86000E+02,1.00000E-01) -- (axis cs:1.87000E+02,1.00000E-01) -- (axis cs:1.87000E+02,-1.00000E-01) -- (axis cs:1.86000E+02,-1.00000E-01) -- cycle ; 
\draw[Equator-full] (axis cs:1.88000E+02,1.00000E-01) -- (axis cs:1.89000E+02,1.00000E-01) -- (axis cs:1.89000E+02,-1.00000E-01) -- (axis cs:1.88000E+02,-1.00000E-01) -- cycle ; 
\draw[Equator-full] (axis cs:1.90000E+02,1.00000E-01) -- (axis cs:1.91000E+02,1.00000E-01) -- (axis cs:1.91000E+02,-1.00000E-01) -- (axis cs:1.90000E+02,-1.00000E-01) -- cycle ; 
\draw[Equator-full] (axis cs:1.92000E+02,1.00000E-01) -- (axis cs:1.93000E+02,1.00000E-01) -- (axis cs:1.93000E+02,-1.00000E-01) -- (axis cs:1.92000E+02,-1.00000E-01) -- cycle ; 
\draw[Equator-full] (axis cs:1.94000E+02,1.00000E-01) -- (axis cs:1.95000E+02,1.00000E-01) -- (axis cs:1.95000E+02,-1.00000E-01) -- (axis cs:1.94000E+02,-1.00000E-01) -- cycle ; 
\draw[Equator-full] (axis cs:1.96000E+02,1.00000E-01) -- (axis cs:1.97000E+02,1.00000E-01) -- (axis cs:1.97000E+02,-1.00000E-01) -- (axis cs:1.96000E+02,-1.00000E-01) -- cycle ; 
\draw[Equator-full] (axis cs:1.98000E+02,1.00000E-01) -- (axis cs:1.99000E+02,1.00000E-01) -- (axis cs:1.99000E+02,-1.00000E-01) -- (axis cs:1.98000E+02,-1.00000E-01) -- cycle ; 
\draw[Equator-full] (axis cs:2.00000E+02,1.00000E-01) -- (axis cs:2.01000E+02,1.00000E-01) -- (axis cs:2.01000E+02,-1.00000E-01) -- (axis cs:2.00000E+02,-1.00000E-01) -- cycle ; 
\draw[Equator-full] (axis cs:2.02000E+02,1.00000E-01) -- (axis cs:2.03000E+02,1.00000E-01) -- (axis cs:2.03000E+02,-1.00000E-01) -- (axis cs:2.02000E+02,-1.00000E-01) -- cycle ; 
\draw[Equator-full] (axis cs:2.04000E+02,1.00000E-01) -- (axis cs:2.05000E+02,1.00000E-01) -- (axis cs:2.05000E+02,-1.00000E-01) -- (axis cs:2.04000E+02,-1.00000E-01) -- cycle ; 
\draw[Equator-full] (axis cs:2.06000E+02,1.00000E-01) -- (axis cs:2.07000E+02,1.00000E-01) -- (axis cs:2.07000E+02,-1.00000E-01) -- (axis cs:2.06000E+02,-1.00000E-01) -- cycle ; 
\draw[Equator-full] (axis cs:2.08000E+02,1.00000E-01) -- (axis cs:2.09000E+02,1.00000E-01) -- (axis cs:2.09000E+02,-1.00000E-01) -- (axis cs:2.08000E+02,-1.00000E-01) -- cycle ; 
\draw[Equator-full] (axis cs:2.10000E+02,1.00000E-01) -- (axis cs:2.11000E+02,1.00000E-01) -- (axis cs:2.11000E+02,-1.00000E-01) -- (axis cs:2.10000E+02,-1.00000E-01) -- cycle ; 
\draw[Equator-full] (axis cs:2.12000E+02,1.00000E-01) -- (axis cs:2.13000E+02,1.00000E-01) -- (axis cs:2.13000E+02,-1.00000E-01) -- (axis cs:2.12000E+02,-1.00000E-01) -- cycle ; 
\draw[Equator-full] (axis cs:2.14000E+02,1.00000E-01) -- (axis cs:2.15000E+02,1.00000E-01) -- (axis cs:2.15000E+02,-1.00000E-01) -- (axis cs:2.14000E+02,-1.00000E-01) -- cycle ; 
\draw[Equator-full] (axis cs:2.16000E+02,1.00000E-01) -- (axis cs:2.17000E+02,1.00000E-01) -- (axis cs:2.17000E+02,-1.00000E-01) -- (axis cs:2.16000E+02,-1.00000E-01) -- cycle ; 
\draw[Equator-full] (axis cs:2.18000E+02,1.00000E-01) -- (axis cs:2.19000E+02,1.00000E-01) -- (axis cs:2.19000E+02,-1.00000E-01) -- (axis cs:2.18000E+02,-1.00000E-01) -- cycle ; 
\draw[Equator-full] (axis cs:2.20000E+02,1.00000E-01) -- (axis cs:2.21000E+02,1.00000E-01) -- (axis cs:2.21000E+02,-1.00000E-01) -- (axis cs:2.20000E+02,-1.00000E-01) -- cycle ; 
\draw[Equator-full] (axis cs:2.22000E+02,1.00000E-01) -- (axis cs:2.23000E+02,1.00000E-01) -- (axis cs:2.23000E+02,-1.00000E-01) -- (axis cs:2.22000E+02,-1.00000E-01) -- cycle ; 
\draw[Equator-full] (axis cs:2.24000E+02,1.00000E-01) -- (axis cs:2.25000E+02,1.00000E-01) -- (axis cs:2.25000E+02,-1.00000E-01) -- (axis cs:2.24000E+02,-1.00000E-01) -- cycle ; 
\draw[Equator-full] (axis cs:2.26000E+02,1.00000E-01) -- (axis cs:2.27000E+02,1.00000E-01) -- (axis cs:2.27000E+02,-1.00000E-01) -- (axis cs:2.26000E+02,-1.00000E-01) -- cycle ; 
\draw[Equator-full] (axis cs:2.28000E+02,1.00000E-01) -- (axis cs:2.29000E+02,1.00000E-01) -- (axis cs:2.29000E+02,-1.00000E-01) -- (axis cs:2.28000E+02,-1.00000E-01) -- cycle ; 
\draw[Equator-full] (axis cs:2.30000E+02,1.00000E-01) -- (axis cs:2.31000E+02,1.00000E-01) -- (axis cs:2.31000E+02,-1.00000E-01) -- (axis cs:2.30000E+02,-1.00000E-01) -- cycle ; 
\draw[Equator-full] (axis cs:2.32000E+02,1.00000E-01) -- (axis cs:2.33000E+02,1.00000E-01) -- (axis cs:2.33000E+02,-1.00000E-01) -- (axis cs:2.32000E+02,-1.00000E-01) -- cycle ; 
\draw[Equator-full] (axis cs:2.34000E+02,1.00000E-01) -- (axis cs:2.35000E+02,1.00000E-01) -- (axis cs:2.35000E+02,-1.00000E-01) -- (axis cs:2.34000E+02,-1.00000E-01) -- cycle ; 
\draw[Equator-full] (axis cs:2.36000E+02,1.00000E-01) -- (axis cs:2.37000E+02,1.00000E-01) -- (axis cs:2.37000E+02,-1.00000E-01) -- (axis cs:2.36000E+02,-1.00000E-01) -- cycle ; 
\draw[Equator-full] (axis cs:2.38000E+02,1.00000E-01) -- (axis cs:2.39000E+02,1.00000E-01) -- (axis cs:2.39000E+02,-1.00000E-01) -- (axis cs:2.38000E+02,-1.00000E-01) -- cycle ; 
\draw[Equator-full] (axis cs:2.40000E+02,1.00000E-01) -- (axis cs:2.41000E+02,1.00000E-01) -- (axis cs:2.41000E+02,-1.00000E-01) -- (axis cs:2.40000E+02,-1.00000E-01) -- cycle ; 
\draw[Equator-full] (axis cs:2.42000E+02,1.00000E-01) -- (axis cs:2.43000E+02,1.00000E-01) -- (axis cs:2.43000E+02,-1.00000E-01) -- (axis cs:2.42000E+02,-1.00000E-01) -- cycle ; 
\draw[Equator-full] (axis cs:2.44000E+02,1.00000E-01) -- (axis cs:2.45000E+02,1.00000E-01) -- (axis cs:2.45000E+02,-1.00000E-01) -- (axis cs:2.44000E+02,-1.00000E-01) -- cycle ; 
\draw[Equator-full] (axis cs:2.46000E+02,1.00000E-01) -- (axis cs:2.47000E+02,1.00000E-01) -- (axis cs:2.47000E+02,-1.00000E-01) -- (axis cs:2.46000E+02,-1.00000E-01) -- cycle ; 
\draw[Equator-full] (axis cs:2.48000E+02,1.00000E-01) -- (axis cs:2.49000E+02,1.00000E-01) -- (axis cs:2.49000E+02,-1.00000E-01) -- (axis cs:2.48000E+02,-1.00000E-01) -- cycle ; 
\draw[Equator-full] (axis cs:2.50000E+02,1.00000E-01) -- (axis cs:2.51000E+02,1.00000E-01) -- (axis cs:2.51000E+02,-1.00000E-01) -- (axis cs:2.50000E+02,-1.00000E-01) -- cycle ; 
\draw[Equator-full] (axis cs:2.52000E+02,1.00000E-01) -- (axis cs:2.53000E+02,1.00000E-01) -- (axis cs:2.53000E+02,-1.00000E-01) -- (axis cs:2.52000E+02,-1.00000E-01) -- cycle ; 
\draw[Equator-full] (axis cs:2.54000E+02,1.00000E-01) -- (axis cs:2.55000E+02,1.00000E-01) -- (axis cs:2.55000E+02,-1.00000E-01) -- (axis cs:2.54000E+02,-1.00000E-01) -- cycle ; 
\draw[Equator-full] (axis cs:2.56000E+02,1.00000E-01) -- (axis cs:2.57000E+02,1.00000E-01) -- (axis cs:2.57000E+02,-1.00000E-01) -- (axis cs:2.56000E+02,-1.00000E-01) -- cycle ; 
\draw[Equator-full] (axis cs:2.58000E+02,1.00000E-01) -- (axis cs:2.59000E+02,1.00000E-01) -- (axis cs:2.59000E+02,-1.00000E-01) -- (axis cs:2.58000E+02,-1.00000E-01) -- cycle ; 
\draw[Equator-full] (axis cs:2.60000E+02,1.00000E-01) -- (axis cs:2.61000E+02,1.00000E-01) -- (axis cs:2.61000E+02,-1.00000E-01) -- (axis cs:2.60000E+02,-1.00000E-01) -- cycle ; 
\draw[Equator-full] (axis cs:2.62000E+02,1.00000E-01) -- (axis cs:2.63000E+02,1.00000E-01) -- (axis cs:2.63000E+02,-1.00000E-01) -- (axis cs:2.62000E+02,-1.00000E-01) -- cycle ; 
\draw[Equator-full] (axis cs:2.64000E+02,1.00000E-01) -- (axis cs:2.65000E+02,1.00000E-01) -- (axis cs:2.65000E+02,-1.00000E-01) -- (axis cs:2.64000E+02,-1.00000E-01) -- cycle ; 
\draw[Equator-full] (axis cs:2.66000E+02,1.00000E-01) -- (axis cs:2.67000E+02,1.00000E-01) -- (axis cs:2.67000E+02,-1.00000E-01) -- (axis cs:2.66000E+02,-1.00000E-01) -- cycle ; 
\draw[Equator-full] (axis cs:2.68000E+02,1.00000E-01) -- (axis cs:2.69000E+02,1.00000E-01) -- (axis cs:2.69000E+02,-1.00000E-01) -- (axis cs:2.68000E+02,-1.00000E-01) -- cycle ; 
\draw[Equator-full] (axis cs:2.70000E+02,1.00000E-01) -- (axis cs:2.71000E+02,1.00000E-01) -- (axis cs:2.71000E+02,-1.00000E-01) -- (axis cs:2.70000E+02,-1.00000E-01) -- cycle ; 
\draw[Equator-full] (axis cs:2.72000E+02,1.00000E-01) -- (axis cs:2.73000E+02,1.00000E-01) -- (axis cs:2.73000E+02,-1.00000E-01) -- (axis cs:2.72000E+02,-1.00000E-01) -- cycle ; 
\draw[Equator-full] (axis cs:2.74000E+02,1.00000E-01) -- (axis cs:2.75000E+02,1.00000E-01) -- (axis cs:2.75000E+02,-1.00000E-01) -- (axis cs:2.74000E+02,-1.00000E-01) -- cycle ; 
\draw[Equator-full] (axis cs:2.76000E+02,1.00000E-01) -- (axis cs:2.77000E+02,1.00000E-01) -- (axis cs:2.77000E+02,-1.00000E-01) -- (axis cs:2.76000E+02,-1.00000E-01) -- cycle ; 
\draw[Equator-full] (axis cs:2.78000E+02,1.00000E-01) -- (axis cs:2.79000E+02,1.00000E-01) -- (axis cs:2.79000E+02,-1.00000E-01) -- (axis cs:2.78000E+02,-1.00000E-01) -- cycle ; 
\draw[Equator-full] (axis cs:2.80000E+02,1.00000E-01) -- (axis cs:2.81000E+02,1.00000E-01) -- (axis cs:2.81000E+02,-1.00000E-01) -- (axis cs:2.80000E+02,-1.00000E-01) -- cycle ; 
\draw[Equator-full] (axis cs:2.82000E+02,1.00000E-01) -- (axis cs:2.83000E+02,1.00000E-01) -- (axis cs:2.83000E+02,-1.00000E-01) -- (axis cs:2.82000E+02,-1.00000E-01) -- cycle ; 
\draw[Equator-full] (axis cs:2.84000E+02,1.00000E-01) -- (axis cs:2.85000E+02,1.00000E-01) -- (axis cs:2.85000E+02,-1.00000E-01) -- (axis cs:2.84000E+02,-1.00000E-01) -- cycle ; 
\draw[Equator-full] (axis cs:2.86000E+02,1.00000E-01) -- (axis cs:2.87000E+02,1.00000E-01) -- (axis cs:2.87000E+02,-1.00000E-01) -- (axis cs:2.86000E+02,-1.00000E-01) -- cycle ; 
\draw[Equator-full] (axis cs:2.88000E+02,1.00000E-01) -- (axis cs:2.89000E+02,1.00000E-01) -- (axis cs:2.89000E+02,-1.00000E-01) -- (axis cs:2.88000E+02,-1.00000E-01) -- cycle ; 
\draw[Equator-full] (axis cs:2.90000E+02,1.00000E-01) -- (axis cs:2.91000E+02,1.00000E-01) -- (axis cs:2.91000E+02,-1.00000E-01) -- (axis cs:2.90000E+02,-1.00000E-01) -- cycle ; 
\draw[Equator-full] (axis cs:2.92000E+02,1.00000E-01) -- (axis cs:2.93000E+02,1.00000E-01) -- (axis cs:2.93000E+02,-1.00000E-01) -- (axis cs:2.92000E+02,-1.00000E-01) -- cycle ; 
\draw[Equator-full] (axis cs:2.94000E+02,1.00000E-01) -- (axis cs:2.95000E+02,1.00000E-01) -- (axis cs:2.95000E+02,-1.00000E-01) -- (axis cs:2.94000E+02,-1.00000E-01) -- cycle ; 
\draw[Equator-full] (axis cs:2.96000E+02,1.00000E-01) -- (axis cs:2.97000E+02,1.00000E-01) -- (axis cs:2.97000E+02,-1.00000E-01) -- (axis cs:2.96000E+02,-1.00000E-01) -- cycle ; 
\draw[Equator-full] (axis cs:2.98000E+02,1.00000E-01) -- (axis cs:2.99000E+02,1.00000E-01) -- (axis cs:2.99000E+02,-1.00000E-01) -- (axis cs:2.98000E+02,-1.00000E-01) -- cycle ; 
\draw[Equator-full] (axis cs:3.00000E+02,1.00000E-01) -- (axis cs:3.01000E+02,1.00000E-01) -- (axis cs:3.01000E+02,-1.00000E-01) -- (axis cs:3.00000E+02,-1.00000E-01) -- cycle ; 
\draw[Equator-full] (axis cs:3.02000E+02,1.00000E-01) -- (axis cs:3.03000E+02,1.00000E-01) -- (axis cs:3.03000E+02,-1.00000E-01) -- (axis cs:3.02000E+02,-1.00000E-01) -- cycle ; 
\draw[Equator-full] (axis cs:3.04000E+02,1.00000E-01) -- (axis cs:3.05000E+02,1.00000E-01) -- (axis cs:3.05000E+02,-1.00000E-01) -- (axis cs:3.04000E+02,-1.00000E-01) -- cycle ; 
\draw[Equator-full] (axis cs:3.06000E+02,1.00000E-01) -- (axis cs:3.07000E+02,1.00000E-01) -- (axis cs:3.07000E+02,-1.00000E-01) -- (axis cs:3.06000E+02,-1.00000E-01) -- cycle ; 
\draw[Equator-full] (axis cs:3.08000E+02,1.00000E-01) -- (axis cs:3.09000E+02,1.00000E-01) -- (axis cs:3.09000E+02,-1.00000E-01) -- (axis cs:3.08000E+02,-1.00000E-01) -- cycle ; 
\draw[Equator-full] (axis cs:3.10000E+02,1.00000E-01) -- (axis cs:3.11000E+02,1.00000E-01) -- (axis cs:3.11000E+02,-1.00000E-01) -- (axis cs:3.10000E+02,-1.00000E-01) -- cycle ; 
\draw[Equator-full] (axis cs:3.12000E+02,1.00000E-01) -- (axis cs:3.13000E+02,1.00000E-01) -- (axis cs:3.13000E+02,-1.00000E-01) -- (axis cs:3.12000E+02,-1.00000E-01) -- cycle ; 
\draw[Equator-full] (axis cs:3.14000E+02,1.00000E-01) -- (axis cs:3.15000E+02,1.00000E-01) -- (axis cs:3.15000E+02,-1.00000E-01) -- (axis cs:3.14000E+02,-1.00000E-01) -- cycle ; 
\draw[Equator-full] (axis cs:3.16000E+02,1.00000E-01) -- (axis cs:3.17000E+02,1.00000E-01) -- (axis cs:3.17000E+02,-1.00000E-01) -- (axis cs:3.16000E+02,-1.00000E-01) -- cycle ; 
\draw[Equator-full] (axis cs:3.18000E+02,1.00000E-01) -- (axis cs:3.19000E+02,1.00000E-01) -- (axis cs:3.19000E+02,-1.00000E-01) -- (axis cs:3.18000E+02,-1.00000E-01) -- cycle ; 
\draw[Equator-full] (axis cs:3.20000E+02,1.00000E-01) -- (axis cs:3.21000E+02,1.00000E-01) -- (axis cs:3.21000E+02,-1.00000E-01) -- (axis cs:3.20000E+02,-1.00000E-01) -- cycle ; 
\draw[Equator-full] (axis cs:3.22000E+02,1.00000E-01) -- (axis cs:3.23000E+02,1.00000E-01) -- (axis cs:3.23000E+02,-1.00000E-01) -- (axis cs:3.22000E+02,-1.00000E-01) -- cycle ; 
\draw[Equator-full] (axis cs:3.24000E+02,1.00000E-01) -- (axis cs:3.25000E+02,1.00000E-01) -- (axis cs:3.25000E+02,-1.00000E-01) -- (axis cs:3.24000E+02,-1.00000E-01) -- cycle ; 
\draw[Equator-full] (axis cs:3.26000E+02,1.00000E-01) -- (axis cs:3.27000E+02,1.00000E-01) -- (axis cs:3.27000E+02,-1.00000E-01) -- (axis cs:3.26000E+02,-1.00000E-01) -- cycle ; 
\draw[Equator-full] (axis cs:3.28000E+02,1.00000E-01) -- (axis cs:3.29000E+02,1.00000E-01) -- (axis cs:3.29000E+02,-1.00000E-01) -- (axis cs:3.28000E+02,-1.00000E-01) -- cycle ; 
\draw[Equator-full] (axis cs:3.30000E+02,1.00000E-01) -- (axis cs:3.31000E+02,1.00000E-01) -- (axis cs:3.31000E+02,-1.00000E-01) -- (axis cs:3.30000E+02,-1.00000E-01) -- cycle ; 
\draw[Equator-full] (axis cs:3.32000E+02,1.00000E-01) -- (axis cs:3.33000E+02,1.00000E-01) -- (axis cs:3.33000E+02,-1.00000E-01) -- (axis cs:3.32000E+02,-1.00000E-01) -- cycle ; 
\draw[Equator-full] (axis cs:3.34000E+02,1.00000E-01) -- (axis cs:3.35000E+02,1.00000E-01) -- (axis cs:3.35000E+02,-1.00000E-01) -- (axis cs:3.34000E+02,-1.00000E-01) -- cycle ; 
\draw[Equator-full] (axis cs:3.36000E+02,1.00000E-01) -- (axis cs:3.37000E+02,1.00000E-01) -- (axis cs:3.37000E+02,-1.00000E-01) -- (axis cs:3.36000E+02,-1.00000E-01) -- cycle ; 
\draw[Equator-full] (axis cs:3.38000E+02,1.00000E-01) -- (axis cs:3.39000E+02,1.00000E-01) -- (axis cs:3.39000E+02,-1.00000E-01) -- (axis cs:3.38000E+02,-1.00000E-01) -- cycle ; 
\draw[Equator-full] (axis cs:3.40000E+02,1.00000E-01) -- (axis cs:3.41000E+02,1.00000E-01) -- (axis cs:3.41000E+02,-1.00000E-01) -- (axis cs:3.40000E+02,-1.00000E-01) -- cycle ; 
\draw[Equator-full] (axis cs:3.42000E+02,1.00000E-01) -- (axis cs:3.43000E+02,1.00000E-01) -- (axis cs:3.43000E+02,-1.00000E-01) -- (axis cs:3.42000E+02,-1.00000E-01) -- cycle ; 
\draw[Equator-full] (axis cs:3.44000E+02,1.00000E-01) -- (axis cs:3.45000E+02,1.00000E-01) -- (axis cs:3.45000E+02,-1.00000E-01) -- (axis cs:3.44000E+02,-1.00000E-01) -- cycle ; 
\draw[Equator-full] (axis cs:3.46000E+02,1.00000E-01) -- (axis cs:3.47000E+02,1.00000E-01) -- (axis cs:3.47000E+02,-1.00000E-01) -- (axis cs:3.46000E+02,-1.00000E-01) -- cycle ; 
\draw[Equator-full] (axis cs:3.48000E+02,1.00000E-01) -- (axis cs:3.49000E+02,1.00000E-01) -- (axis cs:3.49000E+02,-1.00000E-01) -- (axis cs:3.48000E+02,-1.00000E-01) -- cycle ; 
\draw[Equator-full] (axis cs:3.50000E+02,1.00000E-01) -- (axis cs:3.51000E+02,1.00000E-01) -- (axis cs:3.51000E+02,-1.00000E-01) -- (axis cs:3.50000E+02,-1.00000E-01) -- cycle ; 
\draw[Equator-full] (axis cs:3.52000E+02,1.00000E-01) -- (axis cs:3.53000E+02,1.00000E-01) -- (axis cs:3.53000E+02,-1.00000E-01) -- (axis cs:3.52000E+02,-1.00000E-01) -- cycle ; 
\draw[Equator-full] (axis cs:3.54000E+02,1.00000E-01) -- (axis cs:3.55000E+02,1.00000E-01) -- (axis cs:3.55000E+02,-1.00000E-01) -- (axis cs:3.54000E+02,-1.00000E-01) -- cycle ; 
\draw[Equator-full] (axis cs:3.56000E+02,1.00000E-01) -- (axis cs:3.57000E+02,1.00000E-01) -- (axis cs:3.57000E+02,-1.00000E-01) -- (axis cs:3.56000E+02,-1.00000E-01) -- cycle ; 
\draw[Equator-full] (axis cs:3.58000E+02,1.00000E-01) -- (axis cs:3.59000E+02,1.00000E-01) -- (axis cs:3.59000E+02,-1.00000E-01) -- (axis cs:3.58000E+02,-1.00000E-01) -- cycle ; 
\draw[Equator-empty] (axis cs:1.00000E+00,1.00000E-01) -- (axis cs:2.00000E+00,1.00000E-01) -- (axis cs:2.00000E+00,-1.00000E-01) -- (axis cs:1.00000E+00,-1.00000E-01) -- cycle ; 
\draw[Equator-empty] (axis cs:3.00000E+00,1.00000E-01) -- (axis cs:4.00000E+00,1.00000E-01) -- (axis cs:4.00000E+00,-1.00000E-01) -- (axis cs:3.00000E+00,-1.00000E-01) -- cycle ; 
\draw[Equator-empty] (axis cs:5.00000E+00,1.00000E-01) -- (axis cs:6.00000E+00,1.00000E-01) -- (axis cs:6.00000E+00,-1.00000E-01) -- (axis cs:5.00000E+00,-1.00000E-01) -- cycle ; 
\draw[Equator-empty] (axis cs:7.00000E+00,1.00000E-01) -- (axis cs:8.00000E+00,1.00000E-01) -- (axis cs:8.00000E+00,-1.00000E-01) -- (axis cs:7.00000E+00,-1.00000E-01) -- cycle ; 
\draw[Equator-empty] (axis cs:9.00000E+00,1.00000E-01) -- (axis cs:1.00000E+01,1.00000E-01) -- (axis cs:1.00000E+01,-1.00000E-01) -- (axis cs:9.00000E+00,-1.00000E-01) -- cycle ; 
\draw[Equator-empty] (axis cs:1.10000E+01,1.00000E-01) -- (axis cs:1.20000E+01,1.00000E-01) -- (axis cs:1.20000E+01,-1.00000E-01) -- (axis cs:1.10000E+01,-1.00000E-01) -- cycle ; 
\draw[Equator-empty] (axis cs:1.30000E+01,1.00000E-01) -- (axis cs:1.40000E+01,1.00000E-01) -- (axis cs:1.40000E+01,-1.00000E-01) -- (axis cs:1.30000E+01,-1.00000E-01) -- cycle ; 
\draw[Equator-empty] (axis cs:1.50000E+01,1.00000E-01) -- (axis cs:1.60000E+01,1.00000E-01) -- (axis cs:1.60000E+01,-1.00000E-01) -- (axis cs:1.50000E+01,-1.00000E-01) -- cycle ; 
\draw[Equator-empty] (axis cs:1.70000E+01,1.00000E-01) -- (axis cs:1.80000E+01,1.00000E-01) -- (axis cs:1.80000E+01,-1.00000E-01) -- (axis cs:1.70000E+01,-1.00000E-01) -- cycle ; 
\draw[Equator-empty] (axis cs:1.90000E+01,1.00000E-01) -- (axis cs:2.00000E+01,1.00000E-01) -- (axis cs:2.00000E+01,-1.00000E-01) -- (axis cs:1.90000E+01,-1.00000E-01) -- cycle ; 
\draw[Equator-empty] (axis cs:2.10000E+01,1.00000E-01) -- (axis cs:2.20000E+01,1.00000E-01) -- (axis cs:2.20000E+01,-1.00000E-01) -- (axis cs:2.10000E+01,-1.00000E-01) -- cycle ; 
\draw[Equator-empty] (axis cs:2.30000E+01,1.00000E-01) -- (axis cs:2.40000E+01,1.00000E-01) -- (axis cs:2.40000E+01,-1.00000E-01) -- (axis cs:2.30000E+01,-1.00000E-01) -- cycle ; 
\draw[Equator-empty] (axis cs:2.50000E+01,1.00000E-01) -- (axis cs:2.60000E+01,1.00000E-01) -- (axis cs:2.60000E+01,-1.00000E-01) -- (axis cs:2.50000E+01,-1.00000E-01) -- cycle ; 
\draw[Equator-empty] (axis cs:2.70000E+01,1.00000E-01) -- (axis cs:2.80000E+01,1.00000E-01) -- (axis cs:2.80000E+01,-1.00000E-01) -- (axis cs:2.70000E+01,-1.00000E-01) -- cycle ; 
\draw[Equator-empty] (axis cs:2.90000E+01,1.00000E-01) -- (axis cs:3.00000E+01,1.00000E-01) -- (axis cs:3.00000E+01,-1.00000E-01) -- (axis cs:2.90000E+01,-1.00000E-01) -- cycle ; 
\draw[Equator-empty] (axis cs:3.10000E+01,1.00000E-01) -- (axis cs:3.20000E+01,1.00000E-01) -- (axis cs:3.20000E+01,-1.00000E-01) -- (axis cs:3.10000E+01,-1.00000E-01) -- cycle ; 
\draw[Equator-empty] (axis cs:3.30000E+01,1.00000E-01) -- (axis cs:3.40000E+01,1.00000E-01) -- (axis cs:3.40000E+01,-1.00000E-01) -- (axis cs:3.30000E+01,-1.00000E-01) -- cycle ; 
\draw[Equator-empty] (axis cs:3.50000E+01,1.00000E-01) -- (axis cs:3.60000E+01,1.00000E-01) -- (axis cs:3.60000E+01,-1.00000E-01) -- (axis cs:3.50000E+01,-1.00000E-01) -- cycle ; 
\draw[Equator-empty] (axis cs:3.70000E+01,1.00000E-01) -- (axis cs:3.80000E+01,1.00000E-01) -- (axis cs:3.80000E+01,-1.00000E-01) -- (axis cs:3.70000E+01,-1.00000E-01) -- cycle ; 
\draw[Equator-empty] (axis cs:3.90000E+01,1.00000E-01) -- (axis cs:4.00000E+01,1.00000E-01) -- (axis cs:4.00000E+01,-1.00000E-01) -- (axis cs:3.90000E+01,-1.00000E-01) -- cycle ; 
\draw[Equator-empty] (axis cs:4.10000E+01,1.00000E-01) -- (axis cs:4.20000E+01,1.00000E-01) -- (axis cs:4.20000E+01,-1.00000E-01) -- (axis cs:4.10000E+01,-1.00000E-01) -- cycle ; 
\draw[Equator-empty] (axis cs:4.30000E+01,1.00000E-01) -- (axis cs:4.40000E+01,1.00000E-01) -- (axis cs:4.40000E+01,-1.00000E-01) -- (axis cs:4.30000E+01,-1.00000E-01) -- cycle ; 
\draw[Equator-empty] (axis cs:4.50000E+01,1.00000E-01) -- (axis cs:4.60000E+01,1.00000E-01) -- (axis cs:4.60000E+01,-1.00000E-01) -- (axis cs:4.50000E+01,-1.00000E-01) -- cycle ; 
\draw[Equator-empty] (axis cs:4.70000E+01,1.00000E-01) -- (axis cs:4.80000E+01,1.00000E-01) -- (axis cs:4.80000E+01,-1.00000E-01) -- (axis cs:4.70000E+01,-1.00000E-01) -- cycle ; 
\draw[Equator-empty] (axis cs:4.90000E+01,1.00000E-01) -- (axis cs:5.00000E+01,1.00000E-01) -- (axis cs:5.00000E+01,-1.00000E-01) -- (axis cs:4.90000E+01,-1.00000E-01) -- cycle ; 
\draw[Equator-empty] (axis cs:5.10000E+01,1.00000E-01) -- (axis cs:5.20000E+01,1.00000E-01) -- (axis cs:5.20000E+01,-1.00000E-01) -- (axis cs:5.10000E+01,-1.00000E-01) -- cycle ; 
\draw[Equator-empty] (axis cs:5.30000E+01,1.00000E-01) -- (axis cs:5.40000E+01,1.00000E-01) -- (axis cs:5.40000E+01,-1.00000E-01) -- (axis cs:5.30000E+01,-1.00000E-01) -- cycle ; 
\draw[Equator-empty] (axis cs:5.50000E+01,1.00000E-01) -- (axis cs:5.60000E+01,1.00000E-01) -- (axis cs:5.60000E+01,-1.00000E-01) -- (axis cs:5.50000E+01,-1.00000E-01) -- cycle ; 
\draw[Equator-empty] (axis cs:5.70000E+01,1.00000E-01) -- (axis cs:5.80000E+01,1.00000E-01) -- (axis cs:5.80000E+01,-1.00000E-01) -- (axis cs:5.70000E+01,-1.00000E-01) -- cycle ; 
\draw[Equator-empty] (axis cs:5.90000E+01,1.00000E-01) -- (axis cs:6.00000E+01,1.00000E-01) -- (axis cs:6.00000E+01,-1.00000E-01) -- (axis cs:5.90000E+01,-1.00000E-01) -- cycle ; 
\draw[Equator-empty] (axis cs:6.10000E+01,1.00000E-01) -- (axis cs:6.20000E+01,1.00000E-01) -- (axis cs:6.20000E+01,-1.00000E-01) -- (axis cs:6.10000E+01,-1.00000E-01) -- cycle ; 
\draw[Equator-empty] (axis cs:6.30000E+01,1.00000E-01) -- (axis cs:6.40000E+01,1.00000E-01) -- (axis cs:6.40000E+01,-1.00000E-01) -- (axis cs:6.30000E+01,-1.00000E-01) -- cycle ; 
\draw[Equator-empty] (axis cs:6.50000E+01,1.00000E-01) -- (axis cs:6.60000E+01,1.00000E-01) -- (axis cs:6.60000E+01,-1.00000E-01) -- (axis cs:6.50000E+01,-1.00000E-01) -- cycle ; 
\draw[Equator-empty] (axis cs:6.70000E+01,1.00000E-01) -- (axis cs:6.80000E+01,1.00000E-01) -- (axis cs:6.80000E+01,-1.00000E-01) -- (axis cs:6.70000E+01,-1.00000E-01) -- cycle ; 
\draw[Equator-empty] (axis cs:6.90000E+01,1.00000E-01) -- (axis cs:7.00000E+01,1.00000E-01) -- (axis cs:7.00000E+01,-1.00000E-01) -- (axis cs:6.90000E+01,-1.00000E-01) -- cycle ; 
\draw[Equator-empty] (axis cs:7.10000E+01,1.00000E-01) -- (axis cs:7.20000E+01,1.00000E-01) -- (axis cs:7.20000E+01,-1.00000E-01) -- (axis cs:7.10000E+01,-1.00000E-01) -- cycle ; 
\draw[Equator-empty] (axis cs:7.30000E+01,1.00000E-01) -- (axis cs:7.40000E+01,1.00000E-01) -- (axis cs:7.40000E+01,-1.00000E-01) -- (axis cs:7.30000E+01,-1.00000E-01) -- cycle ; 
\draw[Equator-empty] (axis cs:7.50000E+01,1.00000E-01) -- (axis cs:7.60000E+01,1.00000E-01) -- (axis cs:7.60000E+01,-1.00000E-01) -- (axis cs:7.50000E+01,-1.00000E-01) -- cycle ; 
\draw[Equator-empty] (axis cs:7.70000E+01,1.00000E-01) -- (axis cs:7.80000E+01,1.00000E-01) -- (axis cs:7.80000E+01,-1.00000E-01) -- (axis cs:7.70000E+01,-1.00000E-01) -- cycle ; 
\draw[Equator-empty] (axis cs:7.90000E+01,1.00000E-01) -- (axis cs:8.00000E+01,1.00000E-01) -- (axis cs:8.00000E+01,-1.00000E-01) -- (axis cs:7.90000E+01,-1.00000E-01) -- cycle ; 
\draw[Equator-empty] (axis cs:8.10000E+01,1.00000E-01) -- (axis cs:8.20000E+01,1.00000E-01) -- (axis cs:8.20000E+01,-1.00000E-01) -- (axis cs:8.10000E+01,-1.00000E-01) -- cycle ; 
\draw[Equator-empty] (axis cs:8.30000E+01,1.00000E-01) -- (axis cs:8.40000E+01,1.00000E-01) -- (axis cs:8.40000E+01,-1.00000E-01) -- (axis cs:8.30000E+01,-1.00000E-01) -- cycle ; 
\draw[Equator-empty] (axis cs:8.50000E+01,1.00000E-01) -- (axis cs:8.60000E+01,1.00000E-01) -- (axis cs:8.60000E+01,-1.00000E-01) -- (axis cs:8.50000E+01,-1.00000E-01) -- cycle ; 
\draw[Equator-empty] (axis cs:8.70000E+01,1.00000E-01) -- (axis cs:8.80000E+01,1.00000E-01) -- (axis cs:8.80000E+01,-1.00000E-01) -- (axis cs:8.70000E+01,-1.00000E-01) -- cycle ; 
\draw[Equator-empty] (axis cs:8.90000E+01,1.00000E-01) -- (axis cs:9.00000E+01,1.00000E-01) -- (axis cs:9.00000E+01,-1.00000E-01) -- (axis cs:8.90000E+01,-1.00000E-01) -- cycle ; 
\draw[Equator-empty] (axis cs:9.10000E+01,1.00000E-01) -- (axis cs:9.20000E+01,1.00000E-01) -- (axis cs:9.20000E+01,-1.00000E-01) -- (axis cs:9.10000E+01,-1.00000E-01) -- cycle ; 
\draw[Equator-empty] (axis cs:9.30000E+01,1.00000E-01) -- (axis cs:9.40000E+01,1.00000E-01) -- (axis cs:9.40000E+01,-1.00000E-01) -- (axis cs:9.30000E+01,-1.00000E-01) -- cycle ; 
\draw[Equator-empty] (axis cs:9.50000E+01,1.00000E-01) -- (axis cs:9.60000E+01,1.00000E-01) -- (axis cs:9.60000E+01,-1.00000E-01) -- (axis cs:9.50000E+01,-1.00000E-01) -- cycle ; 
\draw[Equator-empty] (axis cs:9.70000E+01,1.00000E-01) -- (axis cs:9.80000E+01,1.00000E-01) -- (axis cs:9.80000E+01,-1.00000E-01) -- (axis cs:9.70000E+01,-1.00000E-01) -- cycle ; 
\draw[Equator-empty] (axis cs:9.90000E+01,1.00000E-01) -- (axis cs:1.00000E+02,1.00000E-01) -- (axis cs:1.00000E+02,-1.00000E-01) -- (axis cs:9.90000E+01,-1.00000E-01) -- cycle ; 
\draw[Equator-empty] (axis cs:1.01000E+02,1.00000E-01) -- (axis cs:1.02000E+02,1.00000E-01) -- (axis cs:1.02000E+02,-1.00000E-01) -- (axis cs:1.01000E+02,-1.00000E-01) -- cycle ; 
\draw[Equator-empty] (axis cs:1.03000E+02,1.00000E-01) -- (axis cs:1.04000E+02,1.00000E-01) -- (axis cs:1.04000E+02,-1.00000E-01) -- (axis cs:1.03000E+02,-1.00000E-01) -- cycle ; 
\draw[Equator-empty] (axis cs:1.05000E+02,1.00000E-01) -- (axis cs:1.06000E+02,1.00000E-01) -- (axis cs:1.06000E+02,-1.00000E-01) -- (axis cs:1.05000E+02,-1.00000E-01) -- cycle ; 
\draw[Equator-empty] (axis cs:1.07000E+02,1.00000E-01) -- (axis cs:1.08000E+02,1.00000E-01) -- (axis cs:1.08000E+02,-1.00000E-01) -- (axis cs:1.07000E+02,-1.00000E-01) -- cycle ; 
\draw[Equator-empty] (axis cs:1.09000E+02,1.00000E-01) -- (axis cs:1.10000E+02,1.00000E-01) -- (axis cs:1.10000E+02,-1.00000E-01) -- (axis cs:1.09000E+02,-1.00000E-01) -- cycle ; 
\draw[Equator-empty] (axis cs:1.11000E+02,1.00000E-01) -- (axis cs:1.12000E+02,1.00000E-01) -- (axis cs:1.12000E+02,-1.00000E-01) -- (axis cs:1.11000E+02,-1.00000E-01) -- cycle ; 
\draw[Equator-empty] (axis cs:1.13000E+02,1.00000E-01) -- (axis cs:1.14000E+02,1.00000E-01) -- (axis cs:1.14000E+02,-1.00000E-01) -- (axis cs:1.13000E+02,-1.00000E-01) -- cycle ; 
\draw[Equator-empty] (axis cs:1.15000E+02,1.00000E-01) -- (axis cs:1.16000E+02,1.00000E-01) -- (axis cs:1.16000E+02,-1.00000E-01) -- (axis cs:1.15000E+02,-1.00000E-01) -- cycle ; 
\draw[Equator-empty] (axis cs:1.17000E+02,1.00000E-01) -- (axis cs:1.18000E+02,1.00000E-01) -- (axis cs:1.18000E+02,-1.00000E-01) -- (axis cs:1.17000E+02,-1.00000E-01) -- cycle ; 
\draw[Equator-empty] (axis cs:1.19000E+02,1.00000E-01) -- (axis cs:1.20000E+02,1.00000E-01) -- (axis cs:1.20000E+02,-1.00000E-01) -- (axis cs:1.19000E+02,-1.00000E-01) -- cycle ; 
\draw[Equator-empty] (axis cs:1.21000E+02,1.00000E-01) -- (axis cs:1.22000E+02,1.00000E-01) -- (axis cs:1.22000E+02,-1.00000E-01) -- (axis cs:1.21000E+02,-1.00000E-01) -- cycle ; 
\draw[Equator-empty] (axis cs:1.23000E+02,1.00000E-01) -- (axis cs:1.24000E+02,1.00000E-01) -- (axis cs:1.24000E+02,-1.00000E-01) -- (axis cs:1.23000E+02,-1.00000E-01) -- cycle ; 
\draw[Equator-empty] (axis cs:1.25000E+02,1.00000E-01) -- (axis cs:1.26000E+02,1.00000E-01) -- (axis cs:1.26000E+02,-1.00000E-01) -- (axis cs:1.25000E+02,-1.00000E-01) -- cycle ; 
\draw[Equator-empty] (axis cs:1.27000E+02,1.00000E-01) -- (axis cs:1.28000E+02,1.00000E-01) -- (axis cs:1.28000E+02,-1.00000E-01) -- (axis cs:1.27000E+02,-1.00000E-01) -- cycle ; 
\draw[Equator-empty] (axis cs:1.29000E+02,1.00000E-01) -- (axis cs:1.30000E+02,1.00000E-01) -- (axis cs:1.30000E+02,-1.00000E-01) -- (axis cs:1.29000E+02,-1.00000E-01) -- cycle ; 
\draw[Equator-empty] (axis cs:1.31000E+02,1.00000E-01) -- (axis cs:1.32000E+02,1.00000E-01) -- (axis cs:1.32000E+02,-1.00000E-01) -- (axis cs:1.31000E+02,-1.00000E-01) -- cycle ; 
\draw[Equator-empty] (axis cs:1.33000E+02,1.00000E-01) -- (axis cs:1.34000E+02,1.00000E-01) -- (axis cs:1.34000E+02,-1.00000E-01) -- (axis cs:1.33000E+02,-1.00000E-01) -- cycle ; 
\draw[Equator-empty] (axis cs:1.35000E+02,1.00000E-01) -- (axis cs:1.36000E+02,1.00000E-01) -- (axis cs:1.36000E+02,-1.00000E-01) -- (axis cs:1.35000E+02,-1.00000E-01) -- cycle ; 
\draw[Equator-empty] (axis cs:1.37000E+02,1.00000E-01) -- (axis cs:1.38000E+02,1.00000E-01) -- (axis cs:1.38000E+02,-1.00000E-01) -- (axis cs:1.37000E+02,-1.00000E-01) -- cycle ; 
\draw[Equator-empty] (axis cs:1.39000E+02,1.00000E-01) -- (axis cs:1.40000E+02,1.00000E-01) -- (axis cs:1.40000E+02,-1.00000E-01) -- (axis cs:1.39000E+02,-1.00000E-01) -- cycle ; 
\draw[Equator-empty] (axis cs:1.41000E+02,1.00000E-01) -- (axis cs:1.42000E+02,1.00000E-01) -- (axis cs:1.42000E+02,-1.00000E-01) -- (axis cs:1.41000E+02,-1.00000E-01) -- cycle ; 
\draw[Equator-empty] (axis cs:1.43000E+02,1.00000E-01) -- (axis cs:1.44000E+02,1.00000E-01) -- (axis cs:1.44000E+02,-1.00000E-01) -- (axis cs:1.43000E+02,-1.00000E-01) -- cycle ; 
\draw[Equator-empty] (axis cs:1.45000E+02,1.00000E-01) -- (axis cs:1.46000E+02,1.00000E-01) -- (axis cs:1.46000E+02,-1.00000E-01) -- (axis cs:1.45000E+02,-1.00000E-01) -- cycle ; 
\draw[Equator-empty] (axis cs:1.47000E+02,1.00000E-01) -- (axis cs:1.48000E+02,1.00000E-01) -- (axis cs:1.48000E+02,-1.00000E-01) -- (axis cs:1.47000E+02,-1.00000E-01) -- cycle ; 
\draw[Equator-empty] (axis cs:1.49000E+02,1.00000E-01) -- (axis cs:1.50000E+02,1.00000E-01) -- (axis cs:1.50000E+02,-1.00000E-01) -- (axis cs:1.49000E+02,-1.00000E-01) -- cycle ; 
\draw[Equator-empty] (axis cs:1.51000E+02,1.00000E-01) -- (axis cs:1.52000E+02,1.00000E-01) -- (axis cs:1.52000E+02,-1.00000E-01) -- (axis cs:1.51000E+02,-1.00000E-01) -- cycle ; 
\draw[Equator-empty] (axis cs:1.53000E+02,1.00000E-01) -- (axis cs:1.54000E+02,1.00000E-01) -- (axis cs:1.54000E+02,-1.00000E-01) -- (axis cs:1.53000E+02,-1.00000E-01) -- cycle ; 
\draw[Equator-empty] (axis cs:1.55000E+02,1.00000E-01) -- (axis cs:1.56000E+02,1.00000E-01) -- (axis cs:1.56000E+02,-1.00000E-01) -- (axis cs:1.55000E+02,-1.00000E-01) -- cycle ; 
\draw[Equator-empty] (axis cs:1.57000E+02,1.00000E-01) -- (axis cs:1.58000E+02,1.00000E-01) -- (axis cs:1.58000E+02,-1.00000E-01) -- (axis cs:1.57000E+02,-1.00000E-01) -- cycle ; 
\draw[Equator-empty] (axis cs:1.59000E+02,1.00000E-01) -- (axis cs:1.60000E+02,1.00000E-01) -- (axis cs:1.60000E+02,-1.00000E-01) -- (axis cs:1.59000E+02,-1.00000E-01) -- cycle ; 
\draw[Equator-empty] (axis cs:1.61000E+02,1.00000E-01) -- (axis cs:1.62000E+02,1.00000E-01) -- (axis cs:1.62000E+02,-1.00000E-01) -- (axis cs:1.61000E+02,-1.00000E-01) -- cycle ; 
\draw[Equator-empty] (axis cs:1.63000E+02,1.00000E-01) -- (axis cs:1.64000E+02,1.00000E-01) -- (axis cs:1.64000E+02,-1.00000E-01) -- (axis cs:1.63000E+02,-1.00000E-01) -- cycle ; 
\draw[Equator-empty] (axis cs:1.65000E+02,1.00000E-01) -- (axis cs:1.66000E+02,1.00000E-01) -- (axis cs:1.66000E+02,-1.00000E-01) -- (axis cs:1.65000E+02,-1.00000E-01) -- cycle ; 
\draw[Equator-empty] (axis cs:1.67000E+02,1.00000E-01) -- (axis cs:1.68000E+02,1.00000E-01) -- (axis cs:1.68000E+02,-1.00000E-01) -- (axis cs:1.67000E+02,-1.00000E-01) -- cycle ; 
\draw[Equator-empty] (axis cs:1.69000E+02,1.00000E-01) -- (axis cs:1.70000E+02,1.00000E-01) -- (axis cs:1.70000E+02,-1.00000E-01) -- (axis cs:1.69000E+02,-1.00000E-01) -- cycle ; 
\draw[Equator-empty] (axis cs:1.71000E+02,1.00000E-01) -- (axis cs:1.72000E+02,1.00000E-01) -- (axis cs:1.72000E+02,-1.00000E-01) -- (axis cs:1.71000E+02,-1.00000E-01) -- cycle ; 
\draw[Equator-empty] (axis cs:1.73000E+02,1.00000E-01) -- (axis cs:1.74000E+02,1.00000E-01) -- (axis cs:1.74000E+02,-1.00000E-01) -- (axis cs:1.73000E+02,-1.00000E-01) -- cycle ; 
\draw[Equator-empty] (axis cs:1.75000E+02,1.00000E-01) -- (axis cs:1.76000E+02,1.00000E-01) -- (axis cs:1.76000E+02,-1.00000E-01) -- (axis cs:1.75000E+02,-1.00000E-01) -- cycle ; 
\draw[Equator-empty] (axis cs:1.77000E+02,1.00000E-01) -- (axis cs:1.78000E+02,1.00000E-01) -- (axis cs:1.78000E+02,-1.00000E-01) -- (axis cs:1.77000E+02,-1.00000E-01) -- cycle ; 
\draw[Equator-empty] (axis cs:1.79000E+02,1.00000E-01) -- (axis cs:1.80000E+02,1.00000E-01) -- (axis cs:1.80000E+02,-1.00000E-01) -- (axis cs:1.79000E+02,-1.00000E-01) -- cycle ; 
\draw[Equator-empty] (axis cs:1.81000E+02,1.00000E-01) -- (axis cs:1.82000E+02,1.00000E-01) -- (axis cs:1.82000E+02,-1.00000E-01) -- (axis cs:1.81000E+02,-1.00000E-01) -- cycle ; 
\draw[Equator-empty] (axis cs:1.83000E+02,1.00000E-01) -- (axis cs:1.84000E+02,1.00000E-01) -- (axis cs:1.84000E+02,-1.00000E-01) -- (axis cs:1.83000E+02,-1.00000E-01) -- cycle ; 
\draw[Equator-empty] (axis cs:1.85000E+02,1.00000E-01) -- (axis cs:1.86000E+02,1.00000E-01) -- (axis cs:1.86000E+02,-1.00000E-01) -- (axis cs:1.85000E+02,-1.00000E-01) -- cycle ; 
\draw[Equator-empty] (axis cs:1.87000E+02,1.00000E-01) -- (axis cs:1.88000E+02,1.00000E-01) -- (axis cs:1.88000E+02,-1.00000E-01) -- (axis cs:1.87000E+02,-1.00000E-01) -- cycle ; 
\draw[Equator-empty] (axis cs:1.89000E+02,1.00000E-01) -- (axis cs:1.90000E+02,1.00000E-01) -- (axis cs:1.90000E+02,-1.00000E-01) -- (axis cs:1.89000E+02,-1.00000E-01) -- cycle ; 
\draw[Equator-empty] (axis cs:1.91000E+02,1.00000E-01) -- (axis cs:1.92000E+02,1.00000E-01) -- (axis cs:1.92000E+02,-1.00000E-01) -- (axis cs:1.91000E+02,-1.00000E-01) -- cycle ; 
\draw[Equator-empty] (axis cs:1.93000E+02,1.00000E-01) -- (axis cs:1.94000E+02,1.00000E-01) -- (axis cs:1.94000E+02,-1.00000E-01) -- (axis cs:1.93000E+02,-1.00000E-01) -- cycle ; 
\draw[Equator-empty] (axis cs:1.95000E+02,1.00000E-01) -- (axis cs:1.96000E+02,1.00000E-01) -- (axis cs:1.96000E+02,-1.00000E-01) -- (axis cs:1.95000E+02,-1.00000E-01) -- cycle ; 
\draw[Equator-empty] (axis cs:1.97000E+02,1.00000E-01) -- (axis cs:1.98000E+02,1.00000E-01) -- (axis cs:1.98000E+02,-1.00000E-01) -- (axis cs:1.97000E+02,-1.00000E-01) -- cycle ; 
\draw[Equator-empty] (axis cs:1.99000E+02,1.00000E-01) -- (axis cs:2.00000E+02,1.00000E-01) -- (axis cs:2.00000E+02,-1.00000E-01) -- (axis cs:1.99000E+02,-1.00000E-01) -- cycle ; 
\draw[Equator-empty] (axis cs:2.01000E+02,1.00000E-01) -- (axis cs:2.02000E+02,1.00000E-01) -- (axis cs:2.02000E+02,-1.00000E-01) -- (axis cs:2.01000E+02,-1.00000E-01) -- cycle ; 
\draw[Equator-empty] (axis cs:2.03000E+02,1.00000E-01) -- (axis cs:2.04000E+02,1.00000E-01) -- (axis cs:2.04000E+02,-1.00000E-01) -- (axis cs:2.03000E+02,-1.00000E-01) -- cycle ; 
\draw[Equator-empty] (axis cs:2.05000E+02,1.00000E-01) -- (axis cs:2.06000E+02,1.00000E-01) -- (axis cs:2.06000E+02,-1.00000E-01) -- (axis cs:2.05000E+02,-1.00000E-01) -- cycle ; 
\draw[Equator-empty] (axis cs:2.07000E+02,1.00000E-01) -- (axis cs:2.08000E+02,1.00000E-01) -- (axis cs:2.08000E+02,-1.00000E-01) -- (axis cs:2.07000E+02,-1.00000E-01) -- cycle ; 
\draw[Equator-empty] (axis cs:2.09000E+02,1.00000E-01) -- (axis cs:2.10000E+02,1.00000E-01) -- (axis cs:2.10000E+02,-1.00000E-01) -- (axis cs:2.09000E+02,-1.00000E-01) -- cycle ; 
\draw[Equator-empty] (axis cs:2.11000E+02,1.00000E-01) -- (axis cs:2.12000E+02,1.00000E-01) -- (axis cs:2.12000E+02,-1.00000E-01) -- (axis cs:2.11000E+02,-1.00000E-01) -- cycle ; 
\draw[Equator-empty] (axis cs:2.13000E+02,1.00000E-01) -- (axis cs:2.14000E+02,1.00000E-01) -- (axis cs:2.14000E+02,-1.00000E-01) -- (axis cs:2.13000E+02,-1.00000E-01) -- cycle ; 
\draw[Equator-empty] (axis cs:2.15000E+02,1.00000E-01) -- (axis cs:2.16000E+02,1.00000E-01) -- (axis cs:2.16000E+02,-1.00000E-01) -- (axis cs:2.15000E+02,-1.00000E-01) -- cycle ; 
\draw[Equator-empty] (axis cs:2.17000E+02,1.00000E-01) -- (axis cs:2.18000E+02,1.00000E-01) -- (axis cs:2.18000E+02,-1.00000E-01) -- (axis cs:2.17000E+02,-1.00000E-01) -- cycle ; 
\draw[Equator-empty] (axis cs:2.19000E+02,1.00000E-01) -- (axis cs:2.20000E+02,1.00000E-01) -- (axis cs:2.20000E+02,-1.00000E-01) -- (axis cs:2.19000E+02,-1.00000E-01) -- cycle ; 
\draw[Equator-empty] (axis cs:2.21000E+02,1.00000E-01) -- (axis cs:2.22000E+02,1.00000E-01) -- (axis cs:2.22000E+02,-1.00000E-01) -- (axis cs:2.21000E+02,-1.00000E-01) -- cycle ; 
\draw[Equator-empty] (axis cs:2.23000E+02,1.00000E-01) -- (axis cs:2.24000E+02,1.00000E-01) -- (axis cs:2.24000E+02,-1.00000E-01) -- (axis cs:2.23000E+02,-1.00000E-01) -- cycle ; 
\draw[Equator-empty] (axis cs:2.25000E+02,1.00000E-01) -- (axis cs:2.26000E+02,1.00000E-01) -- (axis cs:2.26000E+02,-1.00000E-01) -- (axis cs:2.25000E+02,-1.00000E-01) -- cycle ; 
\draw[Equator-empty] (axis cs:2.27000E+02,1.00000E-01) -- (axis cs:2.28000E+02,1.00000E-01) -- (axis cs:2.28000E+02,-1.00000E-01) -- (axis cs:2.27000E+02,-1.00000E-01) -- cycle ; 
\draw[Equator-empty] (axis cs:2.29000E+02,1.00000E-01) -- (axis cs:2.30000E+02,1.00000E-01) -- (axis cs:2.30000E+02,-1.00000E-01) -- (axis cs:2.29000E+02,-1.00000E-01) -- cycle ; 
\draw[Equator-empty] (axis cs:2.31000E+02,1.00000E-01) -- (axis cs:2.32000E+02,1.00000E-01) -- (axis cs:2.32000E+02,-1.00000E-01) -- (axis cs:2.31000E+02,-1.00000E-01) -- cycle ; 
\draw[Equator-empty] (axis cs:2.33000E+02,1.00000E-01) -- (axis cs:2.34000E+02,1.00000E-01) -- (axis cs:2.34000E+02,-1.00000E-01) -- (axis cs:2.33000E+02,-1.00000E-01) -- cycle ; 
\draw[Equator-empty] (axis cs:2.35000E+02,1.00000E-01) -- (axis cs:2.36000E+02,1.00000E-01) -- (axis cs:2.36000E+02,-1.00000E-01) -- (axis cs:2.35000E+02,-1.00000E-01) -- cycle ; 
\draw[Equator-empty] (axis cs:2.37000E+02,1.00000E-01) -- (axis cs:2.38000E+02,1.00000E-01) -- (axis cs:2.38000E+02,-1.00000E-01) -- (axis cs:2.37000E+02,-1.00000E-01) -- cycle ; 
\draw[Equator-empty] (axis cs:2.39000E+02,1.00000E-01) -- (axis cs:2.40000E+02,1.00000E-01) -- (axis cs:2.40000E+02,-1.00000E-01) -- (axis cs:2.39000E+02,-1.00000E-01) -- cycle ; 
\draw[Equator-empty] (axis cs:2.41000E+02,1.00000E-01) -- (axis cs:2.42000E+02,1.00000E-01) -- (axis cs:2.42000E+02,-1.00000E-01) -- (axis cs:2.41000E+02,-1.00000E-01) -- cycle ; 
\draw[Equator-empty] (axis cs:2.43000E+02,1.00000E-01) -- (axis cs:2.44000E+02,1.00000E-01) -- (axis cs:2.44000E+02,-1.00000E-01) -- (axis cs:2.43000E+02,-1.00000E-01) -- cycle ; 
\draw[Equator-empty] (axis cs:2.45000E+02,1.00000E-01) -- (axis cs:2.46000E+02,1.00000E-01) -- (axis cs:2.46000E+02,-1.00000E-01) -- (axis cs:2.45000E+02,-1.00000E-01) -- cycle ; 
\draw[Equator-empty] (axis cs:2.47000E+02,1.00000E-01) -- (axis cs:2.48000E+02,1.00000E-01) -- (axis cs:2.48000E+02,-1.00000E-01) -- (axis cs:2.47000E+02,-1.00000E-01) -- cycle ; 
\draw[Equator-empty] (axis cs:2.49000E+02,1.00000E-01) -- (axis cs:2.50000E+02,1.00000E-01) -- (axis cs:2.50000E+02,-1.00000E-01) -- (axis cs:2.49000E+02,-1.00000E-01) -- cycle ; 
\draw[Equator-empty] (axis cs:2.51000E+02,1.00000E-01) -- (axis cs:2.52000E+02,1.00000E-01) -- (axis cs:2.52000E+02,-1.00000E-01) -- (axis cs:2.51000E+02,-1.00000E-01) -- cycle ; 
\draw[Equator-empty] (axis cs:2.53000E+02,1.00000E-01) -- (axis cs:2.54000E+02,1.00000E-01) -- (axis cs:2.54000E+02,-1.00000E-01) -- (axis cs:2.53000E+02,-1.00000E-01) -- cycle ; 
\draw[Equator-empty] (axis cs:2.55000E+02,1.00000E-01) -- (axis cs:2.56000E+02,1.00000E-01) -- (axis cs:2.56000E+02,-1.00000E-01) -- (axis cs:2.55000E+02,-1.00000E-01) -- cycle ; 
\draw[Equator-empty] (axis cs:2.57000E+02,1.00000E-01) -- (axis cs:2.58000E+02,1.00000E-01) -- (axis cs:2.58000E+02,-1.00000E-01) -- (axis cs:2.57000E+02,-1.00000E-01) -- cycle ; 
\draw[Equator-empty] (axis cs:2.59000E+02,1.00000E-01) -- (axis cs:2.60000E+02,1.00000E-01) -- (axis cs:2.60000E+02,-1.00000E-01) -- (axis cs:2.59000E+02,-1.00000E-01) -- cycle ; 
\draw[Equator-empty] (axis cs:2.61000E+02,1.00000E-01) -- (axis cs:2.62000E+02,1.00000E-01) -- (axis cs:2.62000E+02,-1.00000E-01) -- (axis cs:2.61000E+02,-1.00000E-01) -- cycle ; 
\draw[Equator-empty] (axis cs:2.63000E+02,1.00000E-01) -- (axis cs:2.64000E+02,1.00000E-01) -- (axis cs:2.64000E+02,-1.00000E-01) -- (axis cs:2.63000E+02,-1.00000E-01) -- cycle ; 
\draw[Equator-empty] (axis cs:2.65000E+02,1.00000E-01) -- (axis cs:2.66000E+02,1.00000E-01) -- (axis cs:2.66000E+02,-1.00000E-01) -- (axis cs:2.65000E+02,-1.00000E-01) -- cycle ; 
\draw[Equator-empty] (axis cs:2.67000E+02,1.00000E-01) -- (axis cs:2.68000E+02,1.00000E-01) -- (axis cs:2.68000E+02,-1.00000E-01) -- (axis cs:2.67000E+02,-1.00000E-01) -- cycle ; 
\draw[Equator-empty] (axis cs:2.69000E+02,1.00000E-01) -- (axis cs:2.70000E+02,1.00000E-01) -- (axis cs:2.70000E+02,-1.00000E-01) -- (axis cs:2.69000E+02,-1.00000E-01) -- cycle ; 
\draw[Equator-empty] (axis cs:2.71000E+02,1.00000E-01) -- (axis cs:2.72000E+02,1.00000E-01) -- (axis cs:2.72000E+02,-1.00000E-01) -- (axis cs:2.71000E+02,-1.00000E-01) -- cycle ; 
\draw[Equator-empty] (axis cs:2.73000E+02,1.00000E-01) -- (axis cs:2.74000E+02,1.00000E-01) -- (axis cs:2.74000E+02,-1.00000E-01) -- (axis cs:2.73000E+02,-1.00000E-01) -- cycle ; 
\draw[Equator-empty] (axis cs:2.75000E+02,1.00000E-01) -- (axis cs:2.76000E+02,1.00000E-01) -- (axis cs:2.76000E+02,-1.00000E-01) -- (axis cs:2.75000E+02,-1.00000E-01) -- cycle ; 
\draw[Equator-empty] (axis cs:2.77000E+02,1.00000E-01) -- (axis cs:2.78000E+02,1.00000E-01) -- (axis cs:2.78000E+02,-1.00000E-01) -- (axis cs:2.77000E+02,-1.00000E-01) -- cycle ; 
\draw[Equator-empty] (axis cs:2.79000E+02,1.00000E-01) -- (axis cs:2.80000E+02,1.00000E-01) -- (axis cs:2.80000E+02,-1.00000E-01) -- (axis cs:2.79000E+02,-1.00000E-01) -- cycle ; 
\draw[Equator-empty] (axis cs:2.81000E+02,1.00000E-01) -- (axis cs:2.82000E+02,1.00000E-01) -- (axis cs:2.82000E+02,-1.00000E-01) -- (axis cs:2.81000E+02,-1.00000E-01) -- cycle ; 
\draw[Equator-empty] (axis cs:2.83000E+02,1.00000E-01) -- (axis cs:2.84000E+02,1.00000E-01) -- (axis cs:2.84000E+02,-1.00000E-01) -- (axis cs:2.83000E+02,-1.00000E-01) -- cycle ; 
\draw[Equator-empty] (axis cs:2.85000E+02,1.00000E-01) -- (axis cs:2.86000E+02,1.00000E-01) -- (axis cs:2.86000E+02,-1.00000E-01) -- (axis cs:2.85000E+02,-1.00000E-01) -- cycle ; 
\draw[Equator-empty] (axis cs:2.87000E+02,1.00000E-01) -- (axis cs:2.88000E+02,1.00000E-01) -- (axis cs:2.88000E+02,-1.00000E-01) -- (axis cs:2.87000E+02,-1.00000E-01) -- cycle ; 
\draw[Equator-empty] (axis cs:2.89000E+02,1.00000E-01) -- (axis cs:2.90000E+02,1.00000E-01) -- (axis cs:2.90000E+02,-1.00000E-01) -- (axis cs:2.89000E+02,-1.00000E-01) -- cycle ; 
\draw[Equator-empty] (axis cs:2.91000E+02,1.00000E-01) -- (axis cs:2.92000E+02,1.00000E-01) -- (axis cs:2.92000E+02,-1.00000E-01) -- (axis cs:2.91000E+02,-1.00000E-01) -- cycle ; 
\draw[Equator-empty] (axis cs:2.93000E+02,1.00000E-01) -- (axis cs:2.94000E+02,1.00000E-01) -- (axis cs:2.94000E+02,-1.00000E-01) -- (axis cs:2.93000E+02,-1.00000E-01) -- cycle ; 
\draw[Equator-empty] (axis cs:2.95000E+02,1.00000E-01) -- (axis cs:2.96000E+02,1.00000E-01) -- (axis cs:2.96000E+02,-1.00000E-01) -- (axis cs:2.95000E+02,-1.00000E-01) -- cycle ; 
\draw[Equator-empty] (axis cs:2.97000E+02,1.00000E-01) -- (axis cs:2.98000E+02,1.00000E-01) -- (axis cs:2.98000E+02,-1.00000E-01) -- (axis cs:2.97000E+02,-1.00000E-01) -- cycle ; 
\draw[Equator-empty] (axis cs:2.99000E+02,1.00000E-01) -- (axis cs:3.00000E+02,1.00000E-01) -- (axis cs:3.00000E+02,-1.00000E-01) -- (axis cs:2.99000E+02,-1.00000E-01) -- cycle ; 
\draw[Equator-empty] (axis cs:3.01000E+02,1.00000E-01) -- (axis cs:3.02000E+02,1.00000E-01) -- (axis cs:3.02000E+02,-1.00000E-01) -- (axis cs:3.01000E+02,-1.00000E-01) -- cycle ; 
\draw[Equator-empty] (axis cs:3.03000E+02,1.00000E-01) -- (axis cs:3.04000E+02,1.00000E-01) -- (axis cs:3.04000E+02,-1.00000E-01) -- (axis cs:3.03000E+02,-1.00000E-01) -- cycle ; 
\draw[Equator-empty] (axis cs:3.05000E+02,1.00000E-01) -- (axis cs:3.06000E+02,1.00000E-01) -- (axis cs:3.06000E+02,-1.00000E-01) -- (axis cs:3.05000E+02,-1.00000E-01) -- cycle ; 
\draw[Equator-empty] (axis cs:3.07000E+02,1.00000E-01) -- (axis cs:3.08000E+02,1.00000E-01) -- (axis cs:3.08000E+02,-1.00000E-01) -- (axis cs:3.07000E+02,-1.00000E-01) -- cycle ; 
\draw[Equator-empty] (axis cs:3.09000E+02,1.00000E-01) -- (axis cs:3.10000E+02,1.00000E-01) -- (axis cs:3.10000E+02,-1.00000E-01) -- (axis cs:3.09000E+02,-1.00000E-01) -- cycle ; 
\draw[Equator-empty] (axis cs:3.11000E+02,1.00000E-01) -- (axis cs:3.12000E+02,1.00000E-01) -- (axis cs:3.12000E+02,-1.00000E-01) -- (axis cs:3.11000E+02,-1.00000E-01) -- cycle ; 
\draw[Equator-empty] (axis cs:3.13000E+02,1.00000E-01) -- (axis cs:3.14000E+02,1.00000E-01) -- (axis cs:3.14000E+02,-1.00000E-01) -- (axis cs:3.13000E+02,-1.00000E-01) -- cycle ; 
\draw[Equator-empty] (axis cs:3.15000E+02,1.00000E-01) -- (axis cs:3.16000E+02,1.00000E-01) -- (axis cs:3.16000E+02,-1.00000E-01) -- (axis cs:3.15000E+02,-1.00000E-01) -- cycle ; 
\draw[Equator-empty] (axis cs:3.17000E+02,1.00000E-01) -- (axis cs:3.18000E+02,1.00000E-01) -- (axis cs:3.18000E+02,-1.00000E-01) -- (axis cs:3.17000E+02,-1.00000E-01) -- cycle ; 
\draw[Equator-empty] (axis cs:3.19000E+02,1.00000E-01) -- (axis cs:3.20000E+02,1.00000E-01) -- (axis cs:3.20000E+02,-1.00000E-01) -- (axis cs:3.19000E+02,-1.00000E-01) -- cycle ; 
\draw[Equator-empty] (axis cs:3.21000E+02,1.00000E-01) -- (axis cs:3.22000E+02,1.00000E-01) -- (axis cs:3.22000E+02,-1.00000E-01) -- (axis cs:3.21000E+02,-1.00000E-01) -- cycle ; 
\draw[Equator-empty] (axis cs:3.23000E+02,1.00000E-01) -- (axis cs:3.24000E+02,1.00000E-01) -- (axis cs:3.24000E+02,-1.00000E-01) -- (axis cs:3.23000E+02,-1.00000E-01) -- cycle ; 
\draw[Equator-empty] (axis cs:3.25000E+02,1.00000E-01) -- (axis cs:3.26000E+02,1.00000E-01) -- (axis cs:3.26000E+02,-1.00000E-01) -- (axis cs:3.25000E+02,-1.00000E-01) -- cycle ; 
\draw[Equator-empty] (axis cs:3.27000E+02,1.00000E-01) -- (axis cs:3.28000E+02,1.00000E-01) -- (axis cs:3.28000E+02,-1.00000E-01) -- (axis cs:3.27000E+02,-1.00000E-01) -- cycle ; 
\draw[Equator-empty] (axis cs:3.29000E+02,1.00000E-01) -- (axis cs:3.30000E+02,1.00000E-01) -- (axis cs:3.30000E+02,-1.00000E-01) -- (axis cs:3.29000E+02,-1.00000E-01) -- cycle ; 
\draw[Equator-empty] (axis cs:3.31000E+02,1.00000E-01) -- (axis cs:3.32000E+02,1.00000E-01) -- (axis cs:3.32000E+02,-1.00000E-01) -- (axis cs:3.31000E+02,-1.00000E-01) -- cycle ; 
\draw[Equator-empty] (axis cs:3.33000E+02,1.00000E-01) -- (axis cs:3.34000E+02,1.00000E-01) -- (axis cs:3.34000E+02,-1.00000E-01) -- (axis cs:3.33000E+02,-1.00000E-01) -- cycle ; 
\draw[Equator-empty] (axis cs:3.35000E+02,1.00000E-01) -- (axis cs:3.36000E+02,1.00000E-01) -- (axis cs:3.36000E+02,-1.00000E-01) -- (axis cs:3.35000E+02,-1.00000E-01) -- cycle ; 
\draw[Equator-empty] (axis cs:3.37000E+02,1.00000E-01) -- (axis cs:3.38000E+02,1.00000E-01) -- (axis cs:3.38000E+02,-1.00000E-01) -- (axis cs:3.37000E+02,-1.00000E-01) -- cycle ; 
\draw[Equator-empty] (axis cs:3.39000E+02,1.00000E-01) -- (axis cs:3.40000E+02,1.00000E-01) -- (axis cs:3.40000E+02,-1.00000E-01) -- (axis cs:3.39000E+02,-1.00000E-01) -- cycle ; 
\draw[Equator-empty] (axis cs:3.41000E+02,1.00000E-01) -- (axis cs:3.42000E+02,1.00000E-01) -- (axis cs:3.42000E+02,-1.00000E-01) -- (axis cs:3.41000E+02,-1.00000E-01) -- cycle ; 
\draw[Equator-empty] (axis cs:3.43000E+02,1.00000E-01) -- (axis cs:3.44000E+02,1.00000E-01) -- (axis cs:3.44000E+02,-1.00000E-01) -- (axis cs:3.43000E+02,-1.00000E-01) -- cycle ; 
\draw[Equator-empty] (axis cs:3.45000E+02,1.00000E-01) -- (axis cs:3.46000E+02,1.00000E-01) -- (axis cs:3.46000E+02,-1.00000E-01) -- (axis cs:3.45000E+02,-1.00000E-01) -- cycle ; 
\draw[Equator-empty] (axis cs:3.47000E+02,1.00000E-01) -- (axis cs:3.48000E+02,1.00000E-01) -- (axis cs:3.48000E+02,-1.00000E-01) -- (axis cs:3.47000E+02,-1.00000E-01) -- cycle ; 
\draw[Equator-empty] (axis cs:3.49000E+02,1.00000E-01) -- (axis cs:3.50000E+02,1.00000E-01) -- (axis cs:3.50000E+02,-1.00000E-01) -- (axis cs:3.49000E+02,-1.00000E-01) -- cycle ; 
\draw[Equator-empty] (axis cs:3.51000E+02,1.00000E-01) -- (axis cs:3.52000E+02,1.00000E-01) -- (axis cs:3.52000E+02,-1.00000E-01) -- (axis cs:3.51000E+02,-1.00000E-01) -- cycle ; 
\draw[Equator-empty] (axis cs:3.53000E+02,1.00000E-01) -- (axis cs:3.54000E+02,1.00000E-01) -- (axis cs:3.54000E+02,-1.00000E-01) -- (axis cs:3.53000E+02,-1.00000E-01) -- cycle ; 
\draw[Equator-empty] (axis cs:3.55000E+02,1.00000E-01) -- (axis cs:3.56000E+02,1.00000E-01) -- (axis cs:3.56000E+02,-1.00000E-01) -- (axis cs:3.55000E+02,-1.00000E-01) -- cycle ; 
\draw[Equator-empty] (axis cs:3.57000E+02,1.00000E-01) -- (axis cs:3.58000E+02,1.00000E-01) -- (axis cs:3.58000E+02,-1.00000E-01) -- (axis cs:3.57000E+02,-1.00000E-01) -- cycle ; 
\draw[Equator-empty] (axis cs:3.59000E+02,1.00000E-01) -- (axis cs:3.60000E+02,1.00000E-01) -- (axis cs:3.60000E+02,-1.00000E-01) -- (axis cs:3.59000E+02,-1.00000E-01) -- cycle ; 

\draw[Equator-full] (axis cs:3.60000E+02,1.00000E-01) -- (axis cs:3.61000E+02,1.00000E-01) -- (axis cs:3.61000E+02,-1.00000E-01) -- (axis cs:3.60000E+02,-1.00000E-01) -- cycle ; 
\draw[Equator-full] (axis cs:3.62000E+02,1.00000E-01) -- (axis cs:3.63000E+02,1.00000E-01) -- (axis cs:3.63000E+02,-1.00000E-01) -- (axis cs:3.62000E+02,-1.00000E-01) -- cycle ; 
\draw[Equator-full] (axis cs:3.64000E+02,1.00000E-01) -- (axis cs:3.65000E+02,1.00000E-01) -- (axis cs:3.65000E+02,-1.00000E-01) -- (axis cs:3.64000E+02,-1.00000E-01) -- cycle ; 
\draw[Equator-full] (axis cs:3.66000E+02,1.00000E-01) -- (axis cs:3.67000E+02,1.00000E-01) -- (axis cs:3.67000E+02,-1.00000E-01) -- (axis cs:3.66000E+02,-1.00000E-01) -- cycle ; 
\draw[Equator-full] (axis cs:3.68000E+02,1.00000E-01) -- (axis cs:3.69000E+02,1.00000E-01) -- (axis cs:3.69000E+02,-1.00000E-01) -- (axis cs:3.68000E+02,-1.00000E-01) -- cycle ; 
\draw[Equator-full] (axis cs:3.70000E+02,1.00000E-01) -- (axis cs:3.71000E+02,1.00000E-01) -- (axis cs:3.71000E+02,-1.00000E-01) -- (axis cs:3.70000E+02,-1.00000E-01) -- cycle ; 
\draw[Equator-full] (axis cs:3.72000E+02,1.00000E-01) -- (axis cs:3.73000E+02,1.00000E-01) -- (axis cs:3.73000E+02,-1.00000E-01) -- (axis cs:3.72000E+02,-1.00000E-01) -- cycle ; 
\draw[Equator-full] (axis cs:3.74000E+02,1.00000E-01) -- (axis cs:3.75000E+02,1.00000E-01) -- (axis cs:3.75000E+02,-1.00000E-01) -- (axis cs:3.74000E+02,-1.00000E-01) -- cycle ; 
\draw[Equator-full] (axis cs:3.76000E+02,1.00000E-01) -- (axis cs:3.77000E+02,1.00000E-01) -- (axis cs:3.77000E+02,-1.00000E-01) -- (axis cs:3.76000E+02,-1.00000E-01) -- cycle ; 
\draw[Equator-full] (axis cs:3.78000E+02,1.00000E-01) -- (axis cs:3.79000E+02,1.00000E-01) -- (axis cs:3.79000E+02,-1.00000E-01) -- (axis cs:3.78000E+02,-1.00000E-01) -- cycle ; 

\draw[Equator-empty] (axis cs:3.61000E+02,1.00000E-01) -- (axis cs:3.62000E+02,1.00000E-01) -- (axis cs:3.62000E+02,-1.00000E-01) -- (axis cs:3.61000E+02,-1.00000E-01) -- cycle ; 
\draw[Equator-empty] (axis cs:3.63000E+02,1.00000E-01) -- (axis cs:3.64000E+02,1.00000E-01) -- (axis cs:3.64000E+02,-1.00000E-01) -- (axis cs:3.63000E+02,-1.00000E-01) -- cycle ; 
\draw[Equator-empty] (axis cs:3.65000E+02,1.00000E-01) -- (axis cs:3.66000E+02,1.00000E-01) -- (axis cs:3.66000E+02,-1.00000E-01) -- (axis cs:3.65000E+02,-1.00000E-01) -- cycle ; 
\draw[Equator-empty] (axis cs:3.67000E+02,1.00000E-01) -- (axis cs:3.68000E+02,1.00000E-01) -- (axis cs:3.68000E+02,-1.00000E-01) -- (axis cs:3.67000E+02,-1.00000E-01) -- cycle ; 
\draw[Equator-empty] (axis cs:3.69000E+02,1.00000E-01) -- (axis cs:3.70000E+02,1.00000E-01) -- (axis cs:3.70000E+02,-1.00000E-01) -- (axis cs:3.69000E+02,-1.00000E-01) -- cycle ; 
\draw[Equator-empty] (axis cs:3.71000E+02,1.00000E-01) -- (axis cs:3.72000E+02,1.00000E-01) -- (axis cs:3.72000E+02,-1.00000E-01) -- (axis cs:3.71000E+02,-1.00000E-01) -- cycle ; 
\draw[Equator-empty] (axis cs:3.73000E+02,1.00000E-01) -- (axis cs:3.74000E+02,1.00000E-01) -- (axis cs:3.74000E+02,-1.00000E-01) -- (axis cs:3.73000E+02,-1.00000E-01) -- cycle ; 
\draw[Equator-empty] (axis cs:3.75000E+02,1.00000E-01) -- (axis cs:3.76000E+02,1.00000E-01) -- (axis cs:3.76000E+02,-1.00000E-01) -- (axis cs:3.75000E+02,-1.00000E-01) -- cycle ; 
\draw[Equator-empty] (axis cs:3.77000E+02,1.00000E-01) -- (axis cs:3.78000E+02,1.00000E-01) -- (axis cs:3.78000E+02,-1.00000E-01) -- (axis cs:3.77000E+02,-1.00000E-01) -- cycle ; 
\draw[Equator-empty] (axis cs:3.79000E+02,1.00000E-01) -- (axis cs:3.80000E+02,1.00000E-01) -- (axis cs:3.80000E+02,-1.00000E-01) -- (axis cs:3.79000E+02,-1.00000E-01) -- cycle ; 


\draw [Equator-empty] (axis cs:0.00000E+00,4.00000E-01) -- (axis cs:0.00000E+00,-4.00000E-01)   ;
\node[pin={[pin distance=-0.4\onedegree,Equator-label]90:{0$^\circ$}}] at (axis cs:0.00000E+00,4.00000E-01) {} ;
\draw [Equator-empty] (axis cs:1.00000E+01,4.00000E-01) -- (axis cs:1.00000E+01,-4.00000E-01)   ;
\node[pin={[pin distance=-0.4\onedegree,Equator-label]90:{10$^\circ$}}] at (axis cs:1.00000E+01,4.00000E-01) {} ;
\draw [Equator-empty] (axis cs:2.00000E+01,4.00000E-01) -- (axis cs:2.00000E+01,-4.00000E-01)   ;
\node[pin={[pin distance=-0.4\onedegree,Equator-label]90:{20$^\circ$}}] at (axis cs:2.00000E+01,4.00000E-01) {} ;
\draw [Equator-empty] (axis cs:3.00000E+01,4.00000E-01) -- (axis cs:3.00000E+01,-4.00000E-01)   ;
\node[pin={[pin distance=-0.4\onedegree,Equator-label]90:{30$^\circ$}}] at (axis cs:3.00000E+01,4.00000E-01) {} ;
\draw [Equator-empty] (axis cs:4.00000E+01,4.00000E-01) -- (axis cs:4.00000E+01,-4.00000E-01)   ;
\node[pin={[pin distance=-0.4\onedegree,Equator-label]90:{40$^\circ$}}] at (axis cs:4.00000E+01,4.00000E-01) {} ;
\draw [Equator-empty] (axis cs:5.00000E+01,4.00000E-01) -- (axis cs:5.00000E+01,-4.00000E-01)   ;
\node[pin={[pin distance=-0.4\onedegree,Equator-label]90:{50$^\circ$}}] at (axis cs:5.00000E+01,4.00000E-01) {} ;
\draw [Equator-empty] (axis cs:6.00000E+01,4.00000E-01) -- (axis cs:6.00000E+01,-4.00000E-01)   ;
\node[pin={[pin distance=-0.4\onedegree,Equator-label]90:{60$^\circ$}}] at (axis cs:6.00000E+01,4.00000E-01) {} ;
\draw [Equator-empty] (axis cs:7.00000E+01,4.00000E-01) -- (axis cs:7.00000E+01,-4.00000E-01)   ;
\node[pin={[pin distance=-0.4\onedegree,Equator-label]90:{70$^\circ$}}] at (axis cs:7.00000E+01,4.00000E-01) {} ;
\draw [Equator-empty] (axis cs:8.00000E+01,4.00000E-01) -- (axis cs:8.00000E+01,-4.00000E-01)   ;
\node[pin={[pin distance=-0.4\onedegree,Equator-label]90:{80$^\circ$}}] at (axis cs:8.00000E+01,4.00000E-01) {} ;
\draw [Equator-empty] (axis cs:9.00000E+01,4.00000E-01) -- (axis cs:9.00000E+01,-4.00000E-01)   ;
\node[pin={[pin distance=-0.4\onedegree,Equator-label]90:{90$^\circ$}}] at (axis cs:9.00000E+01,4.00000E-01) {} ;
\draw [Equator-empty] (axis cs:1.00000E+02,4.00000E-01) -- (axis cs:1.00000E+02,-4.00000E-01)   ;
\node[pin={[pin distance=-0.4\onedegree,Equator-label]090:{100$^\circ$}}] at (axis cs:1.00000E+02,4.00000E-01) {} ;
\draw [Equator-empty] (axis cs:1.10000E+02,4.00000E-01) -- (axis cs:1.10000E+02,-4.00000E-01)   ;
\node[pin={[pin distance=-0.4\onedegree,Equator-label]090:{110$^\circ$}}] at (axis cs:1.10000E+02,4.00000E-01) {} ;
\draw [Equator-empty] (axis cs:1.20000E+02,4.00000E-01) -- (axis cs:1.20000E+02,-4.00000E-01)   ;
\node[pin={[pin distance=-0.4\onedegree,Equator-label]090:{120$^\circ$}}] at (axis cs:1.20000E+02,4.00000E-01) {} ;
\draw [Equator-empty] (axis cs:1.30000E+02,4.00000E-01) -- (axis cs:1.30000E+02,-4.00000E-01)   ;
\node[pin={[pin distance=-0.4\onedegree,Equator-label]090:{130$^\circ$}}] at (axis cs:1.30000E+02,4.00000E-01) {} ;
\draw [Equator-empty] (axis cs:1.40000E+02,4.00000E-01) -- (axis cs:1.40000E+02,-4.00000E-01)   ;
\node[pin={[pin distance=-0.4\onedegree,Equator-label]090:{140$^\circ$}}] at (axis cs:1.40000E+02,4.00000E-01) {} ;
\draw [Equator-empty] (axis cs:1.50000E+02,4.00000E-01) -- (axis cs:1.50000E+02,-4.00000E-01)   ;
\node[pin={[pin distance=-0.4\onedegree,Equator-label]090:{150$^\circ$}}] at (axis cs:1.50000E+02,4.00000E-01) {} ;
\draw [Equator-empty] (axis cs:1.60000E+02,4.00000E-01) -- (axis cs:1.60000E+02,-4.00000E-01)   ;
\node[pin={[pin distance=-0.4\onedegree,Equator-label]090:{160$^\circ$}}] at (axis cs:1.60000E+02,4.00000E-01) {} ;
\draw [Equator-empty] (axis cs:1.70000E+02,4.00000E-01) -- (axis cs:1.70000E+02,-4.00000E-01)   ;
\node[pin={[pin distance=-0.4\onedegree,Equator-label]090:{170$^\circ$}}] at (axis cs:1.70000E+02,4.00000E-01) {} ;
\draw [Equator-empty] (axis cs:1.80000E+02,4.00000E-01) -- (axis cs:1.80000E+02,-4.00000E-01)   ;
\node[pin={[pin distance=-0.4\onedegree,Equator-label]090:{180$^\circ$}}] at (axis cs:1.80000E+02,4.00000E-01) {} ;
\draw [Equator-empty] (axis cs:1.90000E+02,4.00000E-01) -- (axis cs:1.90000E+02,-4.00000E-01)   ;
\node[pin={[pin distance=-0.4\onedegree,Equator-label]090:{190$^\circ$}}] at (axis cs:1.90000E+02,4.00000E-01) {} ;
\draw [Equator-empty] (axis cs:2.00000E+02,4.00000E-01) -- (axis cs:2.00000E+02,-4.00000E-01)   ;
\node[pin={[pin distance=-0.4\onedegree,Equator-label]090:{200$^\circ$}}] at (axis cs:2.00000E+02,4.00000E-01) {} ;
\draw [Equator-empty] (axis cs:2.10000E+02,4.00000E-01) -- (axis cs:2.10000E+02,-4.00000E-01)   ;
\node[pin={[pin distance=-0.4\onedegree,Equator-label]090:{210$^\circ$}}] at (axis cs:2.10000E+02,4.00000E-01) {} ;
\draw [Equator-empty] (axis cs:2.20000E+02,4.00000E-01) -- (axis cs:2.20000E+02,-4.00000E-01)   ;
\node[pin={[pin distance=-0.4\onedegree,Equator-label]090:{220$^\circ$}}] at (axis cs:2.20000E+02,4.00000E-01) {} ;
\draw [Equator-empty] (axis cs:2.30000E+02,4.00000E-01) -- (axis cs:2.30000E+02,-4.00000E-01)   ;
\node[pin={[pin distance=-0.4\onedegree,Equator-label]090:{230$^\circ$}}] at (axis cs:2.30000E+02,4.00000E-01) {} ;
\draw [Equator-empty] (axis cs:2.40000E+02,4.00000E-01) -- (axis cs:2.40000E+02,-4.00000E-01)   ;
\node[pin={[pin distance=-0.4\onedegree,Equator-label]090:{240$^\circ$}}] at (axis cs:2.40000E+02,4.00000E-01) {} ;
\draw [Equator-empty] (axis cs:2.50000E+02,4.00000E-01) -- (axis cs:2.50000E+02,-4.00000E-01)   ;
\node[pin={[pin distance=-0.4\onedegree,Equator-label]090:{250$^\circ$}}] at (axis cs:2.50000E+02,4.00000E-01) {} ;
\draw [Equator-empty] (axis cs:2.60000E+02,4.00000E-01) -- (axis cs:2.60000E+02,-4.00000E-01)   ;
\node[pin={[pin distance=-0.4\onedegree,Equator-label]090:{260$^\circ$}}] at (axis cs:2.60000E+02,4.00000E-01) {} ;
\draw [Equator-empty] (axis cs:2.70000E+02,4.00000E-01) -- (axis cs:2.70000E+02,-4.00000E-01)   ;
\node[pin={[pin distance=-0.4\onedegree,Equator-label]090:{270$^\circ$}}] at (axis cs:2.70000E+02,4.00000E-01) {} ;
\draw [Equator-empty] (axis cs:2.80000E+02,4.00000E-01) -- (axis cs:2.80000E+02,-4.00000E-01)   ;
\node[pin={[pin distance=-0.4\onedegree,Equator-label]090:{280$^\circ$}}] at (axis cs:2.80000E+02,4.00000E-01) {} ;
\draw [Equator-empty] (axis cs:2.90000E+02,4.00000E-01) -- (axis cs:2.90000E+02,-4.00000E-01)   ;
\node[pin={[pin distance=-0.4\onedegree,Equator-label]090:{290$^\circ$}}] at (axis cs:2.90000E+02,4.00000E-01) {} ;
\draw [Equator-empty] (axis cs:3.00000E+02,4.00000E-01) -- (axis cs:3.00000E+02,-4.00000E-01)   ;
\node[pin={[pin distance=-0.4\onedegree,Equator-label]090:{300$^\circ$}}] at (axis cs:3.00000E+02,4.00000E-01) {} ;
\draw [Equator-empty] (axis cs:3.10000E+02,4.00000E-01) -- (axis cs:3.10000E+02,-4.00000E-01)   ;
\node[pin={[pin distance=-0.4\onedegree,Equator-label]090:{310$^\circ$}}] at (axis cs:3.10000E+02,4.00000E-01) {} ;
\draw [Equator-empty] (axis cs:3.20000E+02,4.00000E-01) -- (axis cs:3.20000E+02,-4.00000E-01)   ;
\node[pin={[pin distance=-0.4\onedegree,Equator-label]090:{320$^\circ$}}] at (axis cs:3.20000E+02,4.00000E-01) {} ;
\draw [Equator-empty] (axis cs:3.30000E+02,4.00000E-01) -- (axis cs:3.30000E+02,-4.00000E-01)   ;
\node[pin={[pin distance=-0.4\onedegree,Equator-label]090:{330$^\circ$}}] at (axis cs:3.30000E+02,4.00000E-01) {} ;
\draw [Equator-empty] (axis cs:3.40000E+02,4.00000E-01) -- (axis cs:3.40000E+02,-4.00000E-01)   ;
\node[pin={[pin distance=-0.4\onedegree,Equator-label]090:{340$^\circ$}}] at (axis cs:3.40000E+02,4.00000E-01) {} ;
\draw [Equator-empty] (axis cs:3.50000E+02,4.00000E-01) -- (axis cs:3.50000E+02,-4.00000E-01)   ;
\node[pin={[pin distance=-0.4\onedegree,Equator-label]090:{350$^\circ$}}] at (axis cs:3.50000E+02,4.00000E-01) {} ;

\draw [Equator-empty] (axis cs:3.60000E+02,4.00000E-01) -- (axis cs:3.60000E+02,-4.00000E-01)   ;
\node[pin={[pin distance=-0.4\onedegree,Equator-label]090:{0$^\circ$}}] at (axis cs:3.60000E+02,4.00000E-01) {} ;
\draw [Equator-empty] (axis cs:3.70000E+02,4.00000E-01) -- (axis cs:3.70000E+02,-4.00000E-01)   ;
\node[pin={[pin distance=-0.4\onedegree,Equator-label]090:{10$^\circ$}}] at (axis cs:3.70000E+02,4.00000E-01) {} ;




\end{axis}


% Coordinate axes 


% Left and bottom axis with gridlines
  \begin{axis}[at=(base.center),anchor=center,RA_in_hours,color=cAxes] \end{axis}
% Right and top axis, labelling in RA hours (offset)
  \begin{axis}[at=(base.center),anchor=center,yticklabel pos=right,xticklabel pos=right,RA_in_hours
  ,xticklabel style={yshift=2.5ex, anchor=south},color=cAxes]\end{axis}
% Top axis labelling in RA degree (small offset)
  \begin{axis}[at=(base.center),anchor=center,axis y line=none,xticklabel pos=top,RA_in_deg
     ,xticklabel style={font=\footnotesize},xticklabel style={yshift=0.5ex, anchor=south} ,color=cAxes]
  \end{axis}

% Coordinate grid

\begin{axis}[at=(base.center),anchor=center,coordinategrid,RA_in_deg,
    separate axis lines,
    y axis line style= { draw opacity=0 },
    x axis line style= { draw opacity=0 },
    ,xticklabels={},yticklabels={}
] \end{axis}


% Ornamental frame


  \begin{axis}[name=frame,at={($(base.center)+(0cm,+.7\onedegree)$)},anchor=center,width=11.75\tendegree,height=8.53\tendegree,ticks=none
  ,yticklabels={},xticklabels={},axis line style={line width=1pt},color=cFrame] \end{axis}

  \begin{axis}[name=frame2,at={($(frame.center)+(0cm,+0cm)$)},anchor=center,width=11.85\tendegree,height=8.63\tendegree,ticks=none
  ,yticklabels={},xticklabels={},axis line style={line width=3pt},color=cFrame] \end{axis}
  

% Symbol legend


 
 \begin{axis}[name=legend,at=(frame2.south),anchor=north,axis y line=none,axis x
   line=none,xmin=-55,xmax=55,height=0.5\tendegree   ,ymin=-1.5,ymax=1.5,  ,y dir=normal,x dir=reverse, color=cAxes]

\node at (axis cs:54,+1.0) {\small\it 1};                  \node at (axis cs:54,0.3)  {\tikz{\pgfuseplotmark{m1b};}};           
\node at (axis cs:52,+1.0) {\small\it 1{\nicefrac{1}{3}}}; \node at (axis cs:52,0.3)  {\tikz{\pgfuseplotmark{m1c};}};           
\node at (axis cs:53,-0.5)  {\small First};                  

\node at (axis cs:48,+1.0) {\small\it 1{\nicefrac{2}{3}}}; \node at (axis cs:48,0.3)  {\tikz{\pgfuseplotmark{m2a};}};           
\node at (axis cs:46,+1.0) {\small\it 2};                  \node at (axis cs:46,0.3)  {\tikz{\pgfuseplotmark{m2b};}};           
\node at (axis cs:44,+1.0) {\small\it 2{\nicefrac{1}{3}}}; \node at (axis cs:44,0.3)  {\tikz{\pgfuseplotmark{m2c};}};           
\node at (axis cs:46,-0.5)  {\small Second};

\node at (axis cs:40,+1.0) {\small\it 2{\nicefrac{2}{3}}}; \node at (axis cs:40,0.3)  {\tikz{\pgfuseplotmark{m3a};}};           
\node at (axis cs:38,+1.0) {\small\it 3};                  \node at (axis cs:38,0.3)  {\tikz{\pgfuseplotmark{m3b};}};           
\node at (axis cs:36,+1.0) {\small\it 3{\nicefrac{1}{3}}}; \node at (axis cs:36,0.3)  {\tikz{\pgfuseplotmark{m3c};}};           
\node at (axis cs:38,-0.5)  {\small Third};

\node at (axis cs:32,+1.0) {\small\it 3{\nicefrac{2}{3}}}; \node at (axis cs:32,0.3)  {\tikz{\pgfuseplotmark{m4a};}};           
\node at (axis cs:30,+1.0) {\small\it 4};                  \node at (axis cs:30,0.3)  {\tikz{\pgfuseplotmark{m4b};}};           
\node at (axis cs:28,+1.0) {\small\it 4{\nicefrac{1}{3}}}; \node at (axis cs:28,0.3)  {\tikz{\pgfuseplotmark{m4c};}};           
\node at (axis cs:30,-0.5)  {\small Fourth};                                                                      
                                                                                                                  
\node at (axis cs:24,+1.0) {\small\it 4{\nicefrac{2}{3}}}; \node at (axis cs:24,0.3)  {\tikz{\pgfuseplotmark{m5a};}};           
\node at (axis cs:22,+1.0) {\small\it 5};                  \node at (axis cs:22,0.3)  {\tikz{\pgfuseplotmark{m5b};}};           
\node at (axis cs:20,+1.0) {\small\it 5{\nicefrac{1}{3}}}; \node at (axis cs:20,0.3)  {\tikz{\pgfuseplotmark{m5c};}};           
\node at (axis cs:22,-0.5)  {\small Fifth};

\node at (axis cs:16,+1.0) {\small\it 5{\nicefrac{2}{3}}}; \node at (axis cs:16,0.3)  {\tikz{\pgfuseplotmark{m6a};}};           
\node at (axis cs:14,+1.0) {\small\it 6};                  \node at (axis cs:14,0.3)  {\tikz{\pgfuseplotmark{m6b};}};           
\node at (axis cs:12,+1.0) {\small\it 6{\nicefrac{1}{3}}}; \node at (axis cs:12,0.3)  {\tikz{\pgfuseplotmark{m6c};}};           
\node at (axis cs:14,-0.5)  {\small Sixth};                                                                       
                                                                                                                  
                                                                                                                 
\node at  (axis cs:6,+1.0)   {\small\it Variable};  \node at  (axis cs:6,+0.3)       {\tikz{\pgfuseplotmark{m2av};}};           
\node at  (axis cs:2,+.992)    {\small\it Binary};   \node at  (axis cs:2,+0.3)      {\tikz{\pgfuseplotmark{m2ab};}};           
\node at  (axis cs:-2,+1.0)   {\small\it Var. \& Bin.};  \node at  (axis cs:-2,+0.3) {\tikz{\pgfuseplotmark{m2avb};}};           


\node at  (axis cs:-17,+.992)   {\small\it Planetary};  \node at  (axis cs:-18,+0.3)  {\tikz{\pgfuseplotmark{PN};}};           
\node at  (axis cs:-22,+1.0)   {\small\it Emission};   \node at  (axis cs:-22,+0.3)   {\tikz{\pgfuseplotmark{EN};}};           
\node at  (axis cs:-27,+1.0)   {\small\it w/ Cluster};  \node at  (axis cs:-26,+0.3)  {\tikz{\pgfuseplotmark{CN};}};           
\node at (axis cs:-22,-0.5)  {\small Nebulae};


\node at  (axis cs:-32,+.994)   {\small\it Open};      \node at  (axis cs:-32,+0.3)   {\tikz{\pgfuseplotmark{OC};}};           
\node at  (axis cs:-37,+1.0)   {\small\it Globular};   \node at  (axis cs:-37,+0.3)   {\tikz{\pgfuseplotmark{GC};}};           
\node at (axis cs:-34.5,-0.5)  {\small Cluster};

\node at  (axis cs:-42,+.994)   {\small\it Galaxy};      \node at  (axis cs:-42,+0.3) {\tikz{\pgfuseplotmark{GAL};}};         

%
\end{axis}
         


% Constellations
%
% Some are commented out because they are surrounded by already plotted borders
% The x-coordinate is in fractional hours, so must be times 15
%


\begin{axis}[name=constellations,axis lines=none]


\draw[constellation-boundary]   ++ (0,0);  \draw[constellation-boundary]  (axis cs:180.599,    -35.696092) --  (axis cs:181.608,    -35.696028) --  (axis cs:182.617,    -35.695749) --  (axis cs:183.625,    -35.695258) --  (axis cs:184.634,    -35.694552) --  (axis cs:185.39,    -35.693883) ;
  \draw[constellation-boundary]  (axis cs:204.064,    +07.360584) --  (axis cs:204.563,    +07.363008) ;
  \draw[constellation-boundary]  (axis cs:204.064,    +07.360584) --  (axis cs:204.059,    +08.360572) --  (axis cs:204.054,    +09.360561) --  (axis cs:204.049,    +10.360549) --  (axis cs:204.044,    +11.360537) --  (axis cs:204.039,    +12.360525) --  (axis cs:204.034,    +13.360513) --  (axis cs:204.029,    +14.360500) --  (axis cs:204.024,    +15.360488) --  (axis cs:204.019,    +16.360475) --  (axis cs:204.013,    +17.360463) --  (axis cs:204.008,    +18.360450) --  (axis cs:204.003,    +19.360437) --  (axis cs:203.997,    +20.360424) --  (axis cs:203.992,    +21.360411) --  (axis cs:203.986,    +22.360397) --  (axis cs:203.98,    +23.360384) --  (axis cs:203.975,    +24.360370) --  (axis cs:203.969,    +25.360356) --  (axis cs:203.963,    +26.360342) --  (axis cs:203.957,    +27.360327) --  (axis cs:203.954,    +27.860320) ;
  \draw[constellation-boundary]  (axis cs:181.594,    +33.303971) --  (axis cs:182.586,    +33.304245) --  (axis cs:183.578,    +33.304728) --  (axis cs:184.57,    +33.305422) --  (axis cs:185.562,    +33.306326) --  (axis cs:186.554,    +33.307439) ;
  \draw[constellation-boundary]  (axis cs:181.591,    +44.303971) --  (axis cs:182.579,    +44.304243) --  (axis cs:182.826,    +44.304344) ;
  \draw[constellation-boundary]  (axis cs:182.826,    +44.304344) ;
  \draw[constellation-boundary]  (axis cs:200.227,    +27.843782) --  (axis cs:200.475,    +27.844798) --  (axis cs:201.469,    +27.848987) --  (axis cs:202.463,    +27.853374) --  (axis cs:203.457,    +27.857956) --  (axis cs:204.451,    +27.862733) ;
  \draw[constellation-boundary]  (axis cs:200.227,    +27.843782) --  (axis cs:200.224,    +28.343776) --  (axis cs:200.219,    +29.343765) --  (axis cs:200.213,    +30.343754) --  (axis cs:200.208,    +31.343743) ;
  \draw[constellation-boundary]  (axis cs:186.558,    +31.307441) --  (axis cs:187.55,    +31.308765) --  (axis cs:188.543,    +31.310297) --  (axis cs:189.536,    +31.312038) --  (axis cs:190.528,    +31.313987) --  (axis cs:191.521,    +31.316143) --  (axis cs:192.513,    +31.318506) --  (axis cs:193.506,    +31.321075) --  (axis cs:194.499,    +31.323850) --  (axis cs:195.492,    +31.326829) --  (axis cs:196.484,    +31.330012) --  (axis cs:197.477,    +31.333397) --  (axis cs:198.47,    +31.336984) --  (axis cs:199.463,    +31.340772) --  (axis cs:200.208,    +31.343743) ;
  \draw[constellation-boundary]  (axis cs:186.558,    +31.307441) --  (axis cs:186.556,    +32.307440) --  (axis cs:186.554,    +33.307439) ;
  \draw[constellation-boundary]  (axis cs:190.417,    -30.186382) --  (axis cs:190.669,    -30.185868) --  (axis cs:191.676,    -30.183680) --  (axis cs:192.683,    -30.181284) --  (axis cs:193.69,    -30.178678) --  (axis cs:194.697,    -30.175864) --  (axis cs:195.703,    -30.172843) --  (axis cs:196.71,    -30.169615) --  (axis cs:197.717,    -30.166183) --  (axis cs:198.723,    -30.162546) --  (axis cs:199.73,    -30.158706) --  (axis cs:200.737,    -30.154664) --  (axis cs:201.743,    -30.150421) --  (axis cs:202.75,    -30.145980) --  (axis cs:203.756,    -30.141340) --  (axis cs:204.762,    -30.136504) ;
  \draw[constellation-boundary]  (axis cs:190.427,    -33.686372) --  (axis cs:190.424,    -32.686375) --  (axis cs:190.422,    -31.686377) --  (axis cs:190.419,    -30.686380) --  (axis cs:190.417,    -30.186382) ;
  \draw[constellation-boundary]  (axis cs:185.387,    -33.693884) --  (axis cs:185.639,    -33.693635) --  (axis cs:186.647,    -33.692504) --  (axis cs:187.655,    -33.691160) --  (axis cs:188.663,    -33.689604) --  (axis cs:189.671,    -33.687836) --  (axis cs:190.427,    -33.686372) ;
  \draw[constellation-boundary]  (axis cs:185.39,    -35.693883) --  (axis cs:185.389,    -34.693884) --  (axis cs:185.387,    -33.693884) ;
  \draw[constellation-boundary]  (axis cs:180.602,    +13.303908) --  (axis cs:181.599,    +13.303972) --  (axis cs:182.596,    +13.304247) --  (axis cs:183.593,    +13.304733) --  (axis cs:184.59,    +13.305430) --  (axis cs:185.587,    +13.306338) --  (axis cs:186.584,    +13.307457) --  (axis cs:187.581,    +13.308787) --  (axis cs:188.578,    +13.310326) --  (axis cs:189.575,    +13.312075) --  (axis cs:190.572,    +13.314032) --  (axis cs:191.569,    +13.316198) --  (axis cs:192.566,    +13.318572) --  (axis cs:193.564,    +13.321152) --  (axis cs:194.062,    +13.322520) ;
  \draw[constellation-boundary]  (axis cs:181.596,    +28.303971) --  (axis cs:181.595,    +29.303971) --  (axis cs:181.595,    +30.303971) --  (axis cs:181.595,    +31.303971) --  (axis cs:181.595,    +32.303971) --  (axis cs:181.594,    +33.303971) --  (axis cs:181.594,    +34.303971) --  (axis cs:181.594,    +35.303971) --  (axis cs:181.594,    +36.303971) --  (axis cs:181.593,    +37.303971) --  (axis cs:181.593,    +38.303971) --  (axis cs:181.593,    +39.303971) --  (axis cs:181.593,    +40.303971) --  (axis cs:181.592,    +41.303971) --  (axis cs:181.592,    +42.303971) --  (axis cs:181.592,    +43.303971) --  (axis cs:181.591,    +44.303971) ;
  \draw[constellation-boundary]  (axis cs:194.059,    +14.322516) --  (axis cs:194.558,    +14.323935) --  (axis cs:195.554,    +14.326926) --  (axis cs:196.551,    +14.330122) --  (axis cs:197.548,    +14.333522) --  (axis cs:198.545,    +14.337124) --  (axis cs:199.542,    +14.340927) --  (axis cs:200.539,    +14.344930) --  (axis cs:201.536,    +14.349132) --  (axis cs:202.533,    +14.353532) --  (axis cs:203.53,    +14.358129) --  (axis cs:204.029,    +14.360500) ;
  \draw[constellation-boundary]  (axis cs:194.062,    +13.322520) --  (axis cs:194.059,    +14.322516) ;
  \draw[constellation-boundary]  (axis cs:180.6,    -11.696092) --  (axis cs:181.603,    -11.696028) --  (axis cs:182.605,    -11.695752) --  (axis cs:183.608,    -11.695263) --  (axis cs:184.61,    -11.694562) --  (axis cs:185.613,    -11.693649) --  (axis cs:186.615,    -11.692524) --  (axis cs:187.617,    -11.691187) --  (axis cs:188.62,    -11.689640) --  (axis cs:189.622,    -11.687882) --  (axis cs:190.625,    -11.685914) --  (axis cs:191.627,    -11.683736) --  (axis cs:192.63,    -11.681350) --  (axis cs:193.632,    -11.678756) --  (axis cs:194.133,    -11.677381) ;
  \draw[constellation-boundary]  (axis cs:190.405,    -25.186395) --  (axis cs:190.403,    -24.686396) --  (axis cs:190.401,    -23.686398) --  (axis cs:190.399,    -22.686401) ;
  \draw[constellation-boundary]  (axis cs:190.399,    -22.686401) --  (axis cs:190.65,    -22.685888) --  (axis cs:191.655,    -22.683705) --  (axis cs:192.66,    -22.681312) --  (axis cs:193.664,    -22.678712) --  (axis cs:194.669,    -22.675903) --  (axis cs:195.674,    -22.672888) --  (axis cs:196.679,    -22.669667) --  (axis cs:197.684,    -22.666240) --  (axis cs:198.689,    -22.662610) --  (axis cs:199.693,    -22.658778) --  (axis cs:200.698,    -22.654743) --  (axis cs:201.703,    -22.650509) --  (axis cs:202.707,    -22.646075) --  (axis cs:203.712,    -22.641444) --  (axis cs:204.717,    -22.636616) ;
  \draw[constellation-boundary]  (axis cs:194.167,    -22.677333) --  (axis cs:194.164,    -21.677338) --  (axis cs:194.16,    -20.677342) --  (axis cs:194.157,    -19.677347) --  (axis cs:194.154,    -18.677351) --  (axis cs:194.151,    -17.677356) --  (axis cs:194.148,    -16.677360) --  (axis cs:194.145,    -15.677364) --  (axis cs:194.142,    -14.677368) --  (axis cs:194.139,    -13.677372) --  (axis cs:194.136,    -12.677377) --  (axis cs:194.133,    -11.677381) ;
  \draw[constellation-boundary]  (axis cs:180.6,    -25.196092) --  (axis cs:181.605,    -25.196028) --  (axis cs:182.611,    -25.195750) --  (axis cs:183.617,    -25.195260) --  (axis cs:184.622,    -25.194557) --  (axis cs:185.628,    -25.193641) --  (axis cs:186.634,    -25.192512) --  (axis cs:187.639,    -25.191171) --  (axis cs:188.645,    -25.189619) --  (axis cs:189.65,    -25.187856) --  (axis cs:190.405,    -25.186395) ;
  \draw[constellation-boundary]  (axis cs:180.602,    +28.303908) --  (axis cs:181.596,    +28.303971) ;
  \draw[constellation-boundary]  (axis cs:141.734,    -40.291866) --  (axis cs:141.744,    -39.541903) --  (axis cs:141.756,    -38.541950) --  (axis cs:141.769,    -37.541997) --  (axis cs:141.78,    -36.542042) --  (axis cs:141.792,    -35.542086) --  (axis cs:141.803,    -34.542129) --  (axis cs:141.814,    -33.542171) --  (axis cs:141.825,    -32.542212) --  (axis cs:141.836,    -31.542253) --  (axis cs:141.846,    -30.542292) --  (axis cs:141.856,    -29.542330) --  (axis cs:141.866,    -28.542368) --  (axis cs:141.876,    -27.542405) --  (axis cs:141.886,    -26.542442) --  (axis cs:141.895,    -25.542477) --  (axis cs:141.904,    -24.542512) ;
  \draw[constellation-boundary]  (axis cs:141.734,    -40.291866) --  (axis cs:142.238,    -40.295673) --  (axis cs:143.246,    -40.303162) --  (axis cs:144.254,    -40.310482) --  (axis cs:145.262,    -40.317630) --  (axis cs:146.271,    -40.324605) --  (axis cs:147.279,    -40.331403) --  (axis cs:148.288,    -40.338022) --  (axis cs:149.297,    -40.344462) --  (axis cs:150.305,    -40.350719) --  (axis cs:151.314,    -40.356791) --  (axis cs:152.323,    -40.362678) --  (axis cs:153.332,    -40.368376) --  (axis cs:154.341,    -40.373885) --  (axis cs:155.351,    -40.379202) --  (axis cs:156.36,    -40.384326) --  (axis cs:157.369,    -40.389254) --  (axis cs:158.379,    -40.393987) --  (axis cs:159.388,    -40.398522) --  (axis cs:160.398,    -40.402857) --  (axis cs:161.407,    -40.406992) --  (axis cs:162.417,    -40.410925) --  (axis cs:163.427,    -40.414655) --  (axis cs:164.437,    -40.418180) --  (axis cs:165.447,    -40.421500) --  (axis cs:166.456,    -40.424613) ;
  \draw[constellation-boundary]  (axis cs:163.959,    -35.666490) --  (axis cs:163.964,    -34.666499) --  (axis cs:163.969,    -33.666508) --  (axis cs:163.974,    -32.666516) --  (axis cs:163.978,    -31.833190) ;
  \draw[constellation-boundary]  (axis cs:163.959,    -35.666490) --  (axis cs:164.463,    -35.668224) --  (axis cs:165.471,    -35.671539) --  (axis cs:166.479,    -35.674647) --  (axis cs:167.488,    -35.677549) --  (axis cs:168.496,    -35.680242) --  (axis cs:169.505,    -35.682726) --  (axis cs:170.513,    -35.685001) --  (axis cs:171.522,    -35.687065) --  (axis cs:172.53,    -35.688919) --  (axis cs:173.539,    -35.690560) --  (axis cs:174.547,    -35.691990) --  (axis cs:175.556,    -35.693207) --  (axis cs:176.565,    -35.694211) --  (axis cs:177.573,    -35.695001) --  (axis cs:178.582,    -35.695579) --  (axis cs:179.591,    -35.695942) ;
  \draw[constellation-boundary]  (axis cs:160.201,    -31.818578) --  (axis cs:160.453,    -31.819640) --  (axis cs:161.46,    -31.823764) --  (axis cs:162.467,    -31.827687) --  (axis cs:163.474,    -31.831407) --  (axis cs:163.978,    -31.833190) ;
  \draw[constellation-boundary]  (axis cs:160.201,    -31.818578) --  (axis cs:160.202,    -31.651913) --  (axis cs:160.208,    -30.651926) --  (axis cs:160.213,    -29.818602) ;
  \draw[constellation-boundary]  (axis cs:155.181,    -29.794774) --  (axis cs:155.433,    -29.796081) --  (axis cs:156.439,    -29.801190) --  (axis cs:157.445,    -29.806104) --  (axis cs:158.452,    -29.810822) --  (axis cs:159.458,    -29.815342) --  (axis cs:160.213,    -29.818602) ;
  \draw[constellation-boundary]  (axis cs:155.181,    -29.794774) --  (axis cs:155.182,    -29.628110) --  (axis cs:155.189,    -28.628128) --  (axis cs:155.196,    -27.628146) --  (axis cs:155.199,    -27.128155) ;
  \draw[constellation-boundary]  (axis cs:147.659,    -27.083497) --  (axis cs:148.413,    -27.088428) --  (axis cs:149.418,    -27.094845) --  (axis cs:150.424,    -27.101080) --  (axis cs:151.429,    -27.107131) --  (axis cs:152.434,    -27.112997) --  (axis cs:153.44,    -27.118675) --  (axis cs:154.445,    -27.124163) --  (axis cs:155.199,    -27.128155) ;
  \draw[constellation-boundary]  (axis cs:147.659,    -27.083497) --  (axis cs:147.663,    -26.583510) --  (axis cs:147.672,    -25.583537) --  (axis cs:147.68,    -24.583564) ;
  \draw[constellation-boundary]  (axis cs:104.265,    +44.341840) --  (axis cs:104.277,    +44.841771) ;
  \draw[constellation-boundary]  (axis cs:104.265,    +44.341840) --  (axis cs:105.262,    +44.330051) --  (axis cs:106.259,    +44.318314) --  (axis cs:107.256,    +44.306633) --  (axis cs:108.252,    +44.295012) --  (axis cs:109.249,    +44.283455) --  (axis cs:110.245,    +44.271964) --  (axis cs:111.241,    +44.260544) --  (axis cs:112.236,    +44.249198) --  (axis cs:112.734,    +44.243553) ;
  \draw[constellation-boundary]  (axis cs:112.561,    +35.244534) --  (axis cs:112.569,    +35.744486) --  (axis cs:112.587,    +36.744388) --  (axis cs:112.604,    +37.744286) --  (axis cs:112.623,    +38.744183) --  (axis cs:112.642,    +39.744076) --  (axis cs:112.661,    +40.743966) --  (axis cs:112.681,    +41.743853) --  (axis cs:112.702,    +42.743736) --  (axis cs:112.723,    +43.743615) --  (axis cs:112.734,    +44.243553) ;
  \draw[constellation-boundary]  (axis cs:100.09,    +35.390565) --  (axis cs:101.089,    +35.378593) --  (axis cs:102.087,    +35.366659) --  (axis cs:103.085,    +35.354766) --  (axis cs:104.083,    +35.342919) --  (axis cs:105.081,    +35.331120) --  (axis cs:106.079,    +35.319373) --  (axis cs:107.076,    +35.307682) --  (axis cs:108.074,    +35.296050) --  (axis cs:109.071,    +35.284481) --  (axis cs:110.068,    +35.272978) --  (axis cs:111.065,    +35.261545) --  (axis cs:112.062,    +35.250186) --  (axis cs:113.059,    +35.238903) --  (axis cs:114.056,    +35.227700) --  (axis cs:115.052,    +35.216580) --  (axis cs:116.048,    +35.205547) --  (axis cs:117.045,    +35.194605) --  (axis cs:118.041,    +35.183755) --  (axis cs:118.29,    +35.181058) ;
  \draw[constellation-boundary]  (axis cs:99.9657,    +27.891313) --  (axis cs:99.9812,    +28.891220) --  (axis cs:99.997,    +29.891125) --  (axis cs:100.013,    +30.891028) --  (axis cs:100.03,    +31.890929) --  (axis cs:100.046,    +32.890828) --  (axis cs:100.064,    +33.890724) --  (axis cs:100.081,    +34.890618) --  (axis cs:100.09,    +35.390565) ;
  \draw[constellation-boundary]  (axis cs:90.2211,    +28.009289) --  (axis cs:90.9711,    +28.000178) --  (axis cs:91.9709,    +27.988030) --  (axis cs:92.9707,    +27.975886) --  (axis cs:93.9703,    +27.963751) --  (axis cs:94.9698,    +27.951627) --  (axis cs:95.9692,    +27.939519) --  (axis cs:96.9685,    +27.927429) --  (axis cs:97.9677,    +27.915363) --  (axis cs:98.9667,    +27.903323) --  (axis cs:99.9657,    +27.891313) ;
  \draw[constellation-boundary]  (axis cs:85.9779,    +28.560806) --  (axis cs:86.9783,    +28.548691) --  (axis cs:87.9787,    +28.536563) --  (axis cs:88.9788,    +28.524424) --  (axis cs:89.9789,    +28.512279) --  (axis cs:90.2289,    +28.509242) ;
  \draw[constellation-boundary]  (axis cs:85.9544,    -42.945105) --  (axis cs:86.9535,    -42.957219) --  (axis cs:87.9528,    -42.969347) --  (axis cs:88.9522,    -42.981486) --  (axis cs:89.9519,    -42.993631) --  (axis cs:90.9518,    -43.005780) --  (axis cs:91.9518,    -43.017927) --  (axis cs:92.9521,    -43.030071) --  (axis cs:93.9525,    -43.042207) --  (axis cs:94.9532,    -43.054331) --  (axis cs:95.954,    -43.066440) --  (axis cs:96.9551,    -43.078529) --  (axis cs:97.9563,    -43.090596) --  (axis cs:98.9577,    -43.102636) --  (axis cs:99.709,    -43.111647) ;
  \draw[constellation-boundary]  (axis cs:85.2445,    -27.184757) --  (axis cs:86.2439,    -27.196859) --  (axis cs:87.2434,    -27.208977) --  (axis cs:88.2429,    -27.221109) --  (axis cs:89.2426,    -27.233249) --  (axis cs:90.2425,    -27.245396) --  (axis cs:91.2424,    -27.257545) --  (axis cs:92.2424,    -27.269692) --  (axis cs:92.9925,    -27.278799) ;
  \draw[constellation-boundary]  (axis cs:140.404,    +06.470075) --  (axis cs:140.412,    +07.470043) --  (axis cs:140.42,    +08.470012) --  (axis cs:140.428,    +09.469980) --  (axis cs:140.436,    +10.469948) --  (axis cs:140.445,    +11.469916) --  (axis cs:140.453,    +12.469883) --  (axis cs:140.461,    +13.469851) --  (axis cs:140.47,    +14.469818) --  (axis cs:140.478,    +15.469784) --  (axis cs:140.487,    +16.469751) --  (axis cs:140.495,    +17.469717) --  (axis cs:140.504,    +18.469682) --  (axis cs:140.513,    +19.469648) --  (axis cs:140.522,    +20.469612) --  (axis cs:140.531,    +21.469577) --  (axis cs:140.54,    +22.469541) --  (axis cs:140.549,    +23.469504) --  (axis cs:140.559,    +24.469467) --  (axis cs:140.568,    +25.469429) --  (axis cs:140.578,    +26.469391) --  (axis cs:140.588,    +27.469352) --  (axis cs:140.598,    +28.469312) --  (axis cs:140.608,    +29.469271) --  (axis cs:140.619,    +30.469230) --  (axis cs:140.629,    +31.469187) --  (axis cs:140.64,    +32.469144) --  (axis cs:140.652,    +33.469100) --  (axis cs:140.663,    +34.469055) --  (axis cs:140.675,    +35.469009) --  (axis cs:140.687,    +36.468961) --  (axis cs:140.699,    +37.468913) --  (axis cs:140.712,    +38.468863) --  (axis cs:140.722,    +39.218824) ;
  \draw[constellation-boundary]  (axis cs:120.548,    +06.654989) --  (axis cs:120.559,    +07.654932) --  (axis cs:120.57,    +08.654875) --  (axis cs:120.581,    +09.654818) ;
  \draw[constellation-boundary]  (axis cs:120.548,    +06.654989) --  (axis cs:120.673,    +06.653671) --  (axis cs:121.672,    +06.643190) --  (axis cs:122.672,    +06.632817) --  (axis cs:123.671,    +06.622556) --  (axis cs:124.67,    +06.612410) --  (axis cs:125.669,    +06.602382) --  (axis cs:126.668,    +06.592476) --  (axis cs:127.667,    +06.582693) --  (axis cs:128.666,    +06.573038) --  (axis cs:129.665,    +06.563512) --  (axis cs:130.664,    +06.554120) --  (axis cs:131.663,    +06.544863) --  (axis cs:132.662,    +06.535745) --  (axis cs:133.661,    +06.526768) --  (axis cs:134.66,    +06.517935) --  (axis cs:135.659,    +06.509249) --  (axis cs:136.658,    +06.500713) --  (axis cs:137.657,    +06.492328) --  (axis cs:138.656,    +06.484098) --  (axis cs:139.655,    +06.476025) --  (axis cs:140.654,    +06.468112) --  (axis cs:141.653,    +06.460360) --  (axis cs:142.652,    +06.452773) --  (axis cs:143.651,    +06.445352) --  (axis cs:144.649,    +06.438100) --  (axis cs:145.648,    +06.431019) --  (axis cs:146.647,    +06.424111) --  (axis cs:147.646,    +06.417379) --  (axis cs:148.645,    +06.410823) --  (axis cs:149.643,    +06.404447) --  (axis cs:150.642,    +06.398252) --  (axis cs:151.641,    +06.392241) --  (axis cs:152.64,    +06.386414) --  (axis cs:153.638,    +06.380774) --  (axis cs:154.637,    +06.375322) --  (axis cs:155.636,    +06.370061) --  (axis cs:156.634,    +06.364991) --  (axis cs:157.633,    +06.360114) --  (axis cs:158.632,    +06.355432) --  (axis cs:159.63,    +06.350946) --  (axis cs:160.629,    +06.346658) --  (axis cs:161.628,    +06.342568) --  (axis cs:162.626,    +06.338678) --  (axis cs:162.876,    +06.337737) ;
  \draw[constellation-boundary]  (axis cs:118.833,    +09.673429) --  (axis cs:119.707,    +09.664084) --  (axis cs:120.581,    +09.654818) ;
  \draw[constellation-boundary]  (axis cs:118.833,    +09.673429) --  (axis cs:118.844,    +10.673370) --  (axis cs:118.855,    +11.673310) --  (axis cs:118.866,    +12.673249) --  (axis cs:118.877,    +13.673189) --  (axis cs:118.889,    +14.673127) --  (axis cs:118.9,    +15.673066) --  (axis cs:118.912,    +16.673003) --  (axis cs:118.924,    +17.672940) --  (axis cs:118.936,    +18.672876) --  (axis cs:118.948,    +19.672812) ;
  \draw[constellation-boundary]  (axis cs:118.948,    +19.672812) --  (axis cs:119.821,    +19.663477) --  (axis cs:120.07,    +19.660824) ;
  \draw[constellation-boundary]  (axis cs:120.07,    +19.660824) --  (axis cs:120.082,    +20.660760) --  (axis cs:120.094,    +21.660695) --  (axis cs:120.107,    +22.660630) --  (axis cs:120.119,    +23.660563) --  (axis cs:120.132,    +24.660495) --  (axis cs:120.145,    +25.660426) --  (axis cs:120.158,    +26.660356) --  (axis cs:120.172,    +27.660285) ;
  \draw[constellation-boundary]  (axis cs:120.172,    +27.660285) --  (axis cs:120.919,    +27.652374) --  (axis cs:121.916,    +27.641919) ;
  \draw[constellation-boundary]  (axis cs:121.916,    +27.641919) --  (axis cs:121.929,    +28.641849) --  (axis cs:121.943,    +29.641778) --  (axis cs:121.957,    +30.641705) --  (axis cs:121.971,    +31.641630) --  (axis cs:121.986,    +32.641554) --  (axis cs:121.993,    +33.141516) ;
  \draw[constellation-boundary]  (axis cs:111.677,    -33.250465) --  (axis cs:111.693,    -32.250556) --  (axis cs:111.709,    -31.250644) --  (axis cs:111.724,    -30.250731) --  (axis cs:111.739,    -29.250816) --  (axis cs:111.754,    -28.250899) --  (axis cs:111.768,    -27.250981) --  (axis cs:111.783,    -26.251062) --  (axis cs:111.796,    -25.251141) --  (axis cs:111.81,    -24.251218) --  (axis cs:111.824,    -23.251295) --  (axis cs:111.837,    -22.251370) --  (axis cs:111.85,    -21.251444) --  (axis cs:111.863,    -20.251518) --  (axis cs:111.876,    -19.251590) --  (axis cs:111.888,    -18.251661) --  (axis cs:111.901,    -17.251732) --  (axis cs:111.913,    -16.251802) --  (axis cs:111.926,    -15.251871) --  (axis cs:111.938,    -14.251940) --  (axis cs:111.95,    -13.252007) --  (axis cs:111.962,    -12.252075) --  (axis cs:111.973,    -11.252142) ;
  \draw[constellation-boundary]  (axis cs:92.899,    -33.028232) --  (axis cs:92.9161,    -32.028335) --  (axis cs:92.9328,    -31.028437) --  (axis cs:92.9491,    -30.028536) --  (axis cs:92.9652,    -29.028633) --  (axis cs:92.9809,    -28.028729) --  (axis cs:92.9963,    -27.028822) --  (axis cs:93.0115,    -26.028914) --  (axis cs:93.0264,    -25.029005) --  (axis cs:93.0411,    -24.029094) --  (axis cs:93.0555,    -23.029181) --  (axis cs:93.0697,    -22.029268) --  (axis cs:93.0838,    -21.029353) --  (axis cs:93.0976,    -20.029437) --  (axis cs:93.1113,    -19.029520) --  (axis cs:93.1248,    -18.029602) --  (axis cs:93.1381,    -17.029683) --  (axis cs:93.1513,    -16.029763) --  (axis cs:93.1644,    -15.029842) --  (axis cs:93.1773,    -14.029921) --  (axis cs:93.1902,    -13.029999) --  (axis cs:93.2029,    -12.030076) --  (axis cs:93.2156,    -11.030153) ;
  \draw[constellation-boundary]  (axis cs:92.899,    -33.028232) --  (axis cs:93.1491,    -33.031266) --  (axis cs:94.1494,    -33.043400) --  (axis cs:95.1498,    -33.055521) --  (axis cs:96.1504,    -33.067627) --  (axis cs:97.1512,    -33.079712) --  (axis cs:98.152,    -33.091774) --  (axis cs:99.153,    -33.103809) --  (axis cs:100.154,    -33.115813) --  (axis cs:101.155,    -33.127782) --  (axis cs:102.157,    -33.139714) --  (axis cs:103.158,    -33.151603) --  (axis cs:104.16,    -33.163447) --  (axis cs:105.162,    -33.175242) --  (axis cs:106.164,    -33.186984) --  (axis cs:107.166,    -33.198670) --  (axis cs:108.168,    -33.210297) --  (axis cs:109.171,    -33.221859) --  (axis cs:110.173,    -33.233355) --  (axis cs:111.176,    -33.244780) --  (axis cs:111.677,    -33.250465) ;
  \draw[constellation-boundary]  (axis cs:109.6,    -00.224326) --  (axis cs:110.6,    -00.235792) --  (axis cs:111.6,    -00.247187) --  (axis cs:112.599,    -00.258506) --  (axis cs:113.599,    -00.269746) --  (axis cs:114.599,    -00.280904) --  (axis cs:115.599,    -00.291976) --  (axis cs:116.599,    -00.302960) --  (axis cs:117.599,    -00.313851) --  (axis cs:118.599,    -00.324647) --  (axis cs:119.599,    -00.335344) --  (axis cs:120.599,    -00.345938) --  (axis cs:121.599,    -00.356428) --  (axis cs:122.599,    -00.366808) --  (axis cs:122.849,    -00.369386) ;
  \draw[constellation-boundary]  (axis cs:109.6,    -00.224326) --  (axis cs:109.611,    +00.775608) --  (axis cs:109.617,    +01.275574) ;
  \draw[constellation-boundary]  (axis cs:106.867,    +01.307445) --  (axis cs:107.617,    +01.298706) --  (axis cs:108.617,    +01.287108) --  (axis cs:109.617,    +01.275574) ;
  \draw[constellation-boundary]  (axis cs:106.867,    +01.307445) --  (axis cs:106.873,    +01.807411) --  (axis cs:106.885,    +02.807342) --  (axis cs:106.897,    +03.807274) --  (axis cs:106.908,    +04.807205) --  (axis cs:106.914,    +05.307171) ;
  \draw[constellation-boundary]  (axis cs:106.664,    +05.310091) --  (axis cs:106.914,    +05.307171) ;
  \draw[constellation-boundary]  (axis cs:106.664,    +05.310091) --  (axis cs:106.67,    +05.810056) --  (axis cs:106.682,    +06.809987) --  (axis cs:106.694,    +07.809918) --  (axis cs:106.706,    +08.809848) --  (axis cs:106.718,    +09.809778) --  (axis cs:106.73,    +10.809707) --  (axis cs:106.742,    +11.809636) --  (axis cs:106.748,    +12.309600) ;
  \draw[constellation-boundary]  (axis cs:106.748,    +12.309600) --  (axis cs:107.747,    +12.297949) --  (axis cs:108.747,    +12.286359) --  (axis cs:109.746,    +12.274834) --  (axis cs:110.745,    +12.263379) --  (axis cs:111.744,    +12.251995) --  (axis cs:112.743,    +12.240687) --  (axis cs:113.742,    +12.229458) --  (axis cs:114.241,    +12.223875) ;
  \draw[constellation-boundary]  (axis cs:114.241,    +12.223875) --  (axis cs:114.247,    +12.723842) --  (axis cs:114.253,    +13.223809) ;
  \draw[constellation-boundary]  (axis cs:114.253,    +13.223809) --  (axis cs:114.752,    +13.218247) --  (axis cs:115.751,    +13.207188) --  (axis cs:116.75,    +13.196218) --  (axis cs:117.748,    +13.185341) --  (axis cs:118.747,    +13.174560) --  (axis cs:118.872,    +13.173219) ;
  \draw[constellation-boundary]  (axis cs:166.433,    -44.674578) --  (axis cs:166.439,    -43.674587) --  (axis cs:166.445,    -42.674595) --  (axis cs:166.45,    -41.674603) --  (axis cs:166.455,    -40.674611) --  (axis cs:166.46,    -39.674619) --  (axis cs:166.465,    -38.674626) --  (axis cs:166.47,    -37.674633) --  (axis cs:166.475,    -36.674640) --  (axis cs:166.479,    -35.674647) ;
  \draw[constellation-boundary]  (axis cs:121.018,    -44.353397) --  (axis cs:121.038,    -43.353501) ;
  \draw[constellation-boundary]  (axis cs:90.9287,    -44.005639) --  (axis cs:90.9518,    -43.005780) ;
  \draw[constellation-boundary]  (axis cs:99.709,    -43.111647) --  (axis cs:99.7311,    -42.111780) --  (axis cs:99.7525,    -41.111908) --  (axis cs:99.7733,    -40.112033) --  (axis cs:99.7935,    -39.112154) --  (axis cs:99.8131,    -38.112271) --  (axis cs:99.8322,    -37.112386) --  (axis cs:99.8508,    -36.112497) --  (axis cs:99.8689,    -35.112606) --  (axis cs:99.8866,    -34.112711) --  (axis cs:99.9039,    -33.112815) ;
  \draw[constellation-boundary]  (axis cs:179.604,    +13.304056) ;
  \draw[constellation-boundary]  (axis cs:179.097,    -11.695789) --  (axis cs:179.598,    -11.695943) ;
  \draw[constellation-boundary]  (axis cs:179.091,    -25.195788) --  (axis cs:179.092,    -24.695788) --  (axis cs:179.092,    -23.695788) --  (axis cs:179.092,    -22.695788) --  (axis cs:179.093,    -21.695788) --  (axis cs:179.093,    -20.695788) --  (axis cs:179.094,    -19.695788) --  (axis cs:179.094,    -18.695789) --  (axis cs:179.094,    -17.695789) --  (axis cs:179.095,    -16.695789) --  (axis cs:179.095,    -15.695789) --  (axis cs:179.096,    -14.695789) --  (axis cs:179.096,    -13.695789) --  (axis cs:179.096,    -12.695789) --  (axis cs:179.097,    -11.695789) --  (axis cs:179.097,    -10.695789) --  (axis cs:179.098,    -09.695789) --  (axis cs:179.098,    -08.695789) --  (axis cs:179.098,    -07.695789) --  (axis cs:179.099,    -06.695789) ;
  \draw[constellation-boundary]  (axis cs:162.827,    -06.662171) --  (axis cs:163.578,    -06.664926) --  (axis cs:164.579,    -06.668422) --  (axis cs:165.581,    -06.671713) --  (axis cs:166.582,    -06.674800) --  (axis cs:167.583,    -06.677681) --  (axis cs:168.585,    -06.680355) --  (axis cs:169.586,    -06.682822) --  (axis cs:170.587,    -06.685081) --  (axis cs:171.589,    -06.687130) --  (axis cs:172.59,    -06.688970) --  (axis cs:173.591,    -06.690600) --  (axis cs:174.593,    -06.692020) --  (axis cs:175.594,    -06.693228) --  (axis cs:176.595,    -06.694224) --  (axis cs:177.597,    -06.695009) --  (axis cs:178.598,    -06.695582) --  (axis cs:179.099,    -06.695789) ;
  \draw[constellation-boundary]  (axis cs:162.776,    -19.662075) --  (axis cs:163.529,    -19.664837) --  (axis cs:164.031,    -19.666615) ;
  \draw[constellation-boundary]  (axis cs:162.776,    -19.662075) --  (axis cs:162.78,    -18.662083) --  (axis cs:162.784,    -17.662090) --  (axis cs:162.788,    -16.662098) --  (axis cs:162.792,    -15.662106) --  (axis cs:162.796,    -14.662113) --  (axis cs:162.8,    -13.662121) --  (axis cs:162.804,    -12.662128) --  (axis cs:162.808,    -11.662135) --  (axis cs:162.812,    -10.662143) --  (axis cs:162.816,    -09.662150) --  (axis cs:162.819,    -08.662157) --  (axis cs:162.823,    -07.662164) --  (axis cs:162.827,    -06.662171) --  (axis cs:162.831,    -05.662178) --  (axis cs:162.835,    -04.662186) --  (axis cs:162.838,    -03.662193) --  (axis cs:162.842,    -02.662200) --  (axis cs:162.846,    -01.662207) --  (axis cs:162.85,    -00.662214) --  (axis cs:162.853,    +00.337779) --  (axis cs:162.857,    +01.337772) --  (axis cs:162.861,    +02.337765) --  (axis cs:162.865,    +03.337758) --  (axis cs:162.868,    +04.337751) --  (axis cs:162.872,    +05.337744) --  (axis cs:162.876,    +06.337737) ;
  \draw[constellation-boundary]  (axis cs:164.008,    -25.166576) --  (axis cs:164.511,    -25.168306) --  (axis cs:165.516,    -25.171611) --  (axis cs:166.522,    -25.174710) --  (axis cs:167.527,    -25.177603) --  (axis cs:168.533,    -25.180289) --  (axis cs:169.538,    -25.182766) --  (axis cs:170.544,    -25.185034) --  (axis cs:171.549,    -25.187092) --  (axis cs:172.555,    -25.188940) --  (axis cs:173.56,    -25.190577) --  (axis cs:174.566,    -25.192002) --  (axis cs:175.572,    -25.193215) --  (axis cs:176.577,    -25.194216) --  (axis cs:177.583,    -25.195005) --  (axis cs:178.589,    -25.195580) --  (axis cs:179.594,    -25.195943) ;
  \draw[constellation-boundary]  (axis cs:164.008,    -25.166576) --  (axis cs:164.01,    -24.666580) --  (axis cs:164.014,    -23.666587) --  (axis cs:164.019,    -22.666594) --  (axis cs:164.023,    -21.666601) --  (axis cs:164.027,    -20.666608) --  (axis cs:164.031,    -19.666615) ;
  \draw[constellation-boundary]  (axis cs:174.342,    -06.691684) --  (axis cs:174.344,    -05.691685) --  (axis cs:174.345,    -04.691686) --  (axis cs:174.346,    -03.691687) --  (axis cs:174.348,    -02.691688) --  (axis cs:174.349,    -01.691689) --  (axis cs:174.35,    -00.691690) --  (axis cs:174.352,    +00.308309) --  (axis cs:174.353,    +01.308308) --  (axis cs:174.355,    +02.308307) --  (axis cs:174.356,    +03.308306) --  (axis cs:174.357,    +04.308305) --  (axis cs:174.359,    +05.308304) --  (axis cs:174.36,    +06.308303) --  (axis cs:174.361,    +07.308302) --  (axis cs:174.363,    +08.308301) --  (axis cs:174.364,    +09.308301) --  (axis cs:174.366,    +10.308300) ;
  \draw[constellation-boundary]  (axis cs:85.4666,    -10.936101) --  (axis cs:86.4663,    -10.948207) --  (axis cs:87.4661,    -10.960328) --  (axis cs:88.4659,    -10.972462) --  (axis cs:89.4658,    -10.984604) --  (axis cs:90.4656,    -10.996752) --  (axis cs:91.4656,    -11.008901) --  (axis cs:92.4656,    -11.021047) --  (axis cs:93.4656,    -11.033187) --  (axis cs:94.4656,    -11.045317) --  (axis cs:95.4657,    -11.057434) --  (axis cs:96.4659,    -11.069533) --  (axis cs:97.466,    -11.081611) --  (axis cs:98.4663,    -11.093665) --  (axis cs:99.4665,    -11.105691) --  (axis cs:100.467,    -11.117684) --  (axis cs:101.467,    -11.129642) --  (axis cs:102.468,    -11.141561) --  (axis cs:103.468,    -11.153437) --  (axis cs:104.468,    -11.165266) --  (axis cs:105.469,    -11.177045) --  (axis cs:106.47,    -11.188771) --  (axis cs:107.47,    -11.200439) --  (axis cs:108.471,    -11.212046) --  (axis cs:109.471,    -11.223589) --  (axis cs:110.472,    -11.235064) --  (axis cs:111.473,    -11.246468) --  (axis cs:112.474,    -11.257796) --  (axis cs:113.475,    -11.269047) --  (axis cs:114.475,    -11.280215) --  (axis cs:115.476,    -11.291298) --  (axis cs:116.477,    -11.302293) --  (axis cs:117.478,    -11.313196) --  (axis cs:118.479,    -11.324003) --  (axis cs:119.48,    -11.334712) --  (axis cs:120.482,    -11.345319) --  (axis cs:121.483,    -11.355820) --  (axis cs:122.484,    -11.366214) --  (axis cs:123.485,    -11.376496) --  (axis cs:124.486,    -11.386663) --  (axis cs:125.488,    -11.396713) --  (axis cs:126.489,    -11.406642) --  (axis cs:126.99,    -11.411560) ;
  \draw[constellation-boundary]  (axis cs:96.3728,    +11.933297) --  (axis cs:96.7477,    +11.928764) --  (axis cs:97.7473,    +11.916692) --  (axis cs:98.7469,    +11.904646) --  (axis cs:99.7465,    +11.892629) --  (axis cs:100.746,    +11.880645) --  (axis cs:101.745,    +11.868697) --  (axis cs:102.745,    +11.856790) --  (axis cs:103.744,    +11.844927) --  (axis cs:104.744,    +11.833111) --  (axis cs:105.743,    +11.821346) ;
  \draw[constellation-boundary]  (axis cs:95.0695,    +17.449508) --  (axis cs:95.8192,    +17.440427) --  (axis cs:96.4439,    +17.432866) ;
  \draw[constellation-boundary]  (axis cs:95.0695,    +17.449508) --  (axis cs:95.0762,    +17.949468) --  (axis cs:95.0896,    +18.949386) --  (axis cs:95.1033,    +19.949304) --  (axis cs:95.1171,    +20.949220) --  (axis cs:95.1241,    +21.449178) ;
  \draw[constellation-boundary]  (axis cs:90.1252,    +21.509872) --  (axis cs:90.1322,    +22.009829) --  (axis cs:90.1465,    +23.009743) --  (axis cs:90.1609,    +24.009655) --  (axis cs:90.1756,    +25.009566) --  (axis cs:90.1905,    +26.009475) --  (axis cs:90.2057,    +27.009383) --  (axis cs:90.2211,    +28.009289) --  (axis cs:90.2289,    +28.509242) ;
  \draw[constellation-boundary]  (axis cs:90.1252,    +21.509872) --  (axis cs:90.8751,    +21.500761) --  (axis cs:91.875,    +21.488612) --  (axis cs:92.8748,    +21.476468) --  (axis cs:93.8745,    +21.464332) --  (axis cs:94.8742,    +21.452207) --  (axis cs:95.1241,    +21.449178) ;
  \draw[constellation-boundary]  (axis cs:118.258,    +33.181229) --  (axis cs:118.266,    +33.681187) --  (axis cs:118.282,    +34.681101) --  (axis cs:118.29,    +35.181058) ;
  \draw[constellation-boundary]  (axis cs:118.258,    +33.181229) --  (axis cs:119.005,    +33.173171) --  (axis cs:120.001,    +33.162515) --  (axis cs:120.997,    +33.151963) --  (axis cs:121.993,    +33.141516) --  (axis cs:122.989,    +33.131179) --  (axis cs:123.985,    +33.120954) --  (axis cs:124.98,    +33.110844) --  (axis cs:125.976,    +33.100854) --  (axis cs:126.971,    +33.090984) --  (axis cs:127.966,    +33.081240) --  (axis cs:128.961,    +33.071623) --  (axis cs:129.956,    +33.062136) --  (axis cs:130.951,    +33.052782) --  (axis cs:131.946,    +33.043565) --  (axis cs:132.94,    +33.034486) --  (axis cs:133.935,    +33.025549) --  (axis cs:134.93,    +33.016756) --  (axis cs:135.924,    +33.008109) --  (axis cs:136.918,    +32.999612) --  (axis cs:137.912,    +32.991267) --  (axis cs:138.906,    +32.983077) --  (axis cs:139.901,    +32.975043) --  (axis cs:140.894,    +32.967169) --  (axis cs:141.888,    +32.959456) --  (axis cs:142.882,    +32.951907) --  (axis cs:143.876,    +32.944525) --  (axis cs:144.869,    +32.937311) --  (axis cs:145.863,    +32.930267) --  (axis cs:146.856,    +32.923397) --  (axis cs:147.85,    +32.916700) --  (axis cs:148.843,    +32.910181) --  (axis cs:149.836,    +32.903840) --  (axis cs:150.084,    +32.902283) ;
  \draw[constellation-boundary]  (axis cs:105.718,    +09.821489) --  (axis cs:106.718,    +09.809778) ;
  \draw[constellation-boundary]  (axis cs:105.718,    +09.821489) --  (axis cs:105.731,    +10.821418) --  (axis cs:105.743,    +11.821346) ;
  \draw[constellation-boundary]  (axis cs:145.27,    -11.566775) --  (axis cs:145.521,    -11.568535) --  (axis cs:146.523,    -11.575465) --  (axis cs:147.525,    -11.582219) --  (axis cs:148.527,    -11.588796) --  (axis cs:149.529,    -11.595193) --  (axis cs:150.531,    -11.601408) --  (axis cs:151.533,    -11.607440) --  (axis cs:152.535,    -11.613286) --  (axis cs:153.537,    -11.618946) --  (axis cs:154.539,    -11.624416) --  (axis cs:155.542,    -11.629696) --  (axis cs:156.544,    -11.634784) --  (axis cs:157.546,    -11.639678) --  (axis cs:158.548,    -11.644376) --  (axis cs:159.55,    -11.648878) --  (axis cs:160.553,    -11.653182) --  (axis cs:161.555,    -11.657287) --  (axis cs:162.557,    -11.661191) --  (axis cs:162.808,    -11.662135) ;
  \draw[constellation-boundary]  (axis cs:145.27,    -11.566775) --  (axis cs:145.278,    -10.566801) --  (axis cs:145.285,    -09.566827) --  (axis cs:145.292,    -08.566852) --  (axis cs:145.299,    -07.566878) --  (axis cs:145.306,    -06.566903) --  (axis cs:145.314,    -05.566928) --  (axis cs:145.321,    -04.566953) --  (axis cs:145.328,    -03.566978) --  (axis cs:145.335,    -02.567003) --  (axis cs:145.342,    -01.567028) --  (axis cs:145.349,    -00.567052) --  (axis cs:145.356,    +00.432923) --  (axis cs:145.363,    +01.432898) --  (axis cs:145.37,    +02.432873) --  (axis cs:145.377,    +03.432848) --  (axis cs:145.384,    +04.432823) --  (axis cs:145.391,    +05.432798) --  (axis cs:145.398,    +06.432773) ;
  \draw[constellation-boundary]  (axis cs:137.637,    -24.508626) --  (axis cs:138.39,    -24.514815) --  (axis cs:139.394,    -24.522930) --  (axis cs:140.398,    -24.530885) --  (axis cs:141.402,    -24.538678) --  (axis cs:142.406,    -24.546306) --  (axis cs:143.411,    -24.553767) --  (axis cs:144.415,    -24.561060) --  (axis cs:145.419,    -24.568180) --  (axis cs:146.424,    -24.575128) --  (axis cs:147.429,    -24.581899) --  (axis cs:147.68,    -24.583564) ;
  \draw[constellation-boundary]  (axis cs:137.637,    -24.508626) --  (axis cs:137.647,    -23.508667) --  (axis cs:137.657,    -22.508707) --  (axis cs:137.666,    -21.508747) --  (axis cs:137.676,    -20.508786) --  (axis cs:137.685,    -19.508825) ;
  \draw[constellation-boundary]  (axis cs:130.164,    -19.442368) --  (axis cs:130.414,    -19.444712) --  (axis cs:131.417,    -19.454003) --  (axis cs:132.42,    -19.463156) --  (axis cs:133.422,    -19.472167) --  (axis cs:134.425,    -19.481034) --  (axis cs:135.428,    -19.489754) --  (axis cs:136.431,    -19.498325) --  (axis cs:137.434,    -19.506744) --  (axis cs:137.685,    -19.508825) ;
  \draw[constellation-boundary]  (axis cs:130.164,    -19.442368) --  (axis cs:130.174,    -18.442417) --  (axis cs:130.184,    -17.442466) ;
  \draw[constellation-boundary]  (axis cs:126.927,    -17.411253) --  (axis cs:127.428,    -17.416143) --  (axis cs:128.43,    -17.425829) --  (axis cs:129.433,    -17.435386) --  (axis cs:130.184,    -17.442466) ;
  \draw[constellation-boundary]  (axis cs:122.734,    -11.368795) --  (axis cs:122.745,    -10.368850) --  (axis cs:122.755,    -09.368905) --  (axis cs:122.766,    -08.368959) --  (axis cs:122.777,    -07.369013) --  (axis cs:122.787,    -06.369067) --  (axis cs:122.797,    -05.369120) --  (axis cs:122.808,    -04.369174) --  (axis cs:122.818,    -03.369227) --  (axis cs:122.828,    -02.369280) --  (axis cs:122.839,    -01.369333) --  (axis cs:122.849,    -00.369386) --  (axis cs:122.859,    +00.630561) --  (axis cs:122.87,    +01.630508) --  (axis cs:122.88,    +02.630455) --  (axis cs:122.89,    +03.630401) --  (axis cs:122.901,    +04.630348) --  (axis cs:122.911,    +05.630295) --  (axis cs:122.921,    +06.630241) ;
  \draw[constellation-boundary]  (axis cs:150.042,    +27.902415) --  (axis cs:150.046,    +28.402402) --  (axis cs:150.055,    +29.402377) --  (axis cs:150.063,    +30.402351) --  (axis cs:150.071,    +31.402324) --  (axis cs:150.08,    +32.402297) --  (axis cs:150.084,    +32.902283) ;
  \draw[constellation-boundary]  (axis cs:150.042,    +27.902415) --  (axis cs:150.788,    +27.897806) --  (axis cs:151.782,    +27.891821) --  (axis cs:152.777,    +27.886020) --  (axis cs:153.771,    +27.880405) --  (axis cs:154.765,    +27.874979) --  (axis cs:155.759,    +27.869741) --  (axis cs:156.753,    +27.864694) --  (axis cs:157.747,    +27.859840) --  (axis cs:158.741,    +27.855180) --  (axis cs:159.238,    +27.852923) ;
  \draw[constellation-boundary]  (axis cs:159.211,    +22.852985) --  (axis cs:159.214,    +23.352979) --  (axis cs:159.219,    +24.352967) --  (axis cs:159.224,    +25.352955) --  (axis cs:159.23,    +26.352942) --  (axis cs:159.236,    +27.352930) --  (axis cs:159.238,    +27.852923) ;
  \draw[constellation-boundary]  (axis cs:159.211,    +22.852985) --  (axis cs:159.708,    +22.850775) --  (axis cs:160.704,    +22.846501) --  (axis cs:161.699,    +22.842426) --  (axis cs:162.694,    +22.838550) --  (axis cs:162.943,    +22.837613) ;
  \draw[constellation-boundary]  (axis cs:162.943,    +22.837613) --  (axis cs:162.945,    +23.337608) --  (axis cs:162.949,    +24.337600) --  (axis cs:162.951,    +24.837596) ;
  \draw[constellation-boundary]  (axis cs:162.951,    +24.837596) --  (axis cs:163.697,    +24.834860) --  (axis cs:164.692,    +24.831388) --  (axis cs:165.686,    +24.828118) --  (axis cs:166.681,    +24.825052) ;
  \draw[constellation-boundary]  (axis cs:166.681,    +24.825052) --  (axis cs:166.683,    +25.325050) --  (axis cs:166.686,    +26.325044) --  (axis cs:166.69,    +27.325039) --  (axis cs:166.694,    +28.325033) --  (axis cs:166.698,    +29.325027) --  (axis cs:166.702,    +30.325021) --  (axis cs:166.706,    +31.325015) --  (axis cs:166.71,    +32.325009) --  (axis cs:166.714,    +33.325003) ;
  \draw[constellation-boundary]  (axis cs:166.694,    +28.325033) --  (axis cs:167.688,    +28.322174) --  (axis cs:168.681,    +28.319521) --  (axis cs:169.675,    +28.317073) --  (axis cs:170.668,    +28.314832) --  (axis cs:171.662,    +28.312799) --  (axis cs:172.655,    +28.310973) --  (axis cs:173.648,    +28.309356) --  (axis cs:174.642,    +28.307948) --  (axis cs:175.635,    +28.306750) --  (axis cs:176.629,    +28.305761) --  (axis cs:177.622,    +28.304982) --  (axis cs:178.615,    +28.304414) --  (axis cs:179.609,    +28.304056) ;
  \draw[constellation-boundary]  (axis cs:179.604,    +10.304056) --  (axis cs:179.604,    +11.304056) --  (axis cs:179.604,    +12.304056) --  (axis cs:179.604,    +13.304056) --  (axis cs:179.605,    +14.304056) --  (axis cs:179.605,    +15.304056) --  (axis cs:179.605,    +16.304056) --  (axis cs:179.606,    +17.304056) --  (axis cs:179.606,    +18.304056) --  (axis cs:179.606,    +19.304056) --  (axis cs:179.606,    +20.304056) --  (axis cs:179.607,    +21.304056) --  (axis cs:179.607,    +22.304056) --  (axis cs:179.607,    +23.304056) --  (axis cs:179.608,    +24.304056) --  (axis cs:179.608,    +25.304056) --  (axis cs:179.608,    +26.304056) --  (axis cs:179.609,    +27.304056) --  (axis cs:179.609,    +28.304056) ;
  \draw[constellation-boundary]  (axis cs:174.366,    +10.308300) --  (axis cs:174.615,    +10.307966) --  (axis cs:175.613,    +10.306762) --  (axis cs:176.611,    +10.305769) --  (axis cs:177.608,    +10.304987) --  (axis cs:178.606,    +10.304416) --  (axis cs:179.604,    +10.304056) ;
  \draw[constellation-boundary]  (axis cs:140.722,    +39.218824) --  (axis cs:140.97,    +39.216874) --  (axis cs:141.962,    +39.209173) --  (axis cs:142.954,    +39.201637) --  (axis cs:143.946,    +39.194266) --  (axis cs:144.938,    +39.187064) --  (axis cs:145.682,    +39.181774) ;
  \draw[constellation-boundary]  (axis cs:145.682,    +39.181774) --  (axis cs:145.685,    +39.431764) --  (axis cs:145.697,    +40.431722) --  (axis cs:145.709,    +41.431678) ;
  \draw[constellation-boundary]  (axis cs:154.359,    +39.377419) --  (axis cs:154.369,    +40.377394) --  (axis cs:154.378,    +41.377368) ;
  \draw[constellation-boundary]  (axis cs:154.359,    +39.377419) --  (axis cs:154.855,    +39.374738) --  (axis cs:155.846,    +39.369517) --  (axis cs:156.837,    +39.364487) --  (axis cs:157.827,    +39.359649) --  (axis cs:158.818,    +39.355004) --  (axis cs:159.809,    +39.350554) --  (axis cs:160.799,    +39.346300) --  (axis cs:161.79,    +39.342244) --  (axis cs:162.78,    +39.338386) --  (axis cs:163.523,    +39.335623) ;
  \draw[constellation-boundary]  (axis cs:163.489,    +33.335685) --  (axis cs:163.495,    +34.335675) --  (axis cs:163.5,    +35.335665) --  (axis cs:163.506,    +36.335655) --  (axis cs:163.511,    +37.335645) --  (axis cs:163.517,    +38.335634) --  (axis cs:163.523,    +39.335623) ;
  \draw[constellation-boundary]  (axis cs:163.489,    +33.335685) --  (axis cs:163.738,    +33.334787) --  (axis cs:164.73,    +33.331323) --  (axis cs:165.722,    +33.328062) --  (axis cs:166.714,    +33.325003) ;
  \draw[constellation-boundary]  (axis cs:88.9658,    -10.978532) --  (axis cs:88.9784,    -09.978609) --  (axis cs:88.9909,    -08.978684) --  (axis cs:89.0033,    -07.978760) --  (axis cs:89.0156,    -06.978835) --  (axis cs:89.0279,    -05.978910) --  (axis cs:89.0402,    -04.978984) --  (axis cs:89.0524,    -03.979058) ;
  \draw[constellation-boundary]  (axis cs:139.512,    +41.478600) --  (axis cs:140.008,    +41.474612) --  (axis cs:141,    +41.466755) --  (axis cs:141.992,    +41.459059) --  (axis cs:142.983,    +41.451527) --  (axis cs:143.975,    +41.444162) --  (axis cs:144.966,    +41.436964) --  (axis cs:145.957,    +41.429937) --  (axis cs:146.948,    +41.423083) --  (axis cs:147.939,    +41.416403) --  (axis cs:148.93,    +41.409899) --  (axis cs:149.921,    +41.403574) --  (axis cs:150.911,    +41.397429) --  (axis cs:151.902,    +41.391467) --  (axis cs:152.893,    +41.385688) --  (axis cs:153.883,    +41.380094) --  (axis cs:154.378,    +41.377368) ;
  \draw[constellation-boundary]  (axis cs:139.512,    +41.478600) --  (axis cs:139.527,    +42.478541) --  (axis cs:139.542,    +43.478480) --  (axis cs:139.558,    +44.478417) ;
  \draw[constellation-boundary]  (axis cs:95.348,    +09.945550) --  (axis cs:95.7229,    +09.941010) --  (axis cs:96.3477,    +09.933448) ;
  \draw[constellation-boundary]  (axis cs:96.3477,    +09.933448) --  (axis cs:96.3602,    +10.933373) --  (axis cs:96.3728,    +11.933297) --  (axis cs:96.3855,    +12.933220) --  (axis cs:96.3983,    +13.933143) --  (axis cs:96.4112,    +14.933065) --  (axis cs:96.4242,    +15.932986) --  (axis cs:96.4373,    +16.932906) --  (axis cs:96.4439,    +17.432866) ;
  \draw[constellation-boundary]  (axis cs:89.0524,    -03.979058) --  (axis cs:89.5524,    -03.985130) --  (axis cs:90.5523,    -03.997278) --  (axis cs:91.5522,    -04.009427) --  (axis cs:92.5522,    -04.021573) --  (axis cs:93.5521,    -04.033712) --  (axis cs:94.5521,    -04.045841) --  (axis cs:95.1771,    -04.053415) ;
  \draw[constellation-boundary]  (axis cs:95.1771,    -04.053415) --  (axis cs:95.1892,    -03.053489) --  (axis cs:95.2014,    -02.053562) --  (axis cs:95.2135,    -01.053636) --  (axis cs:95.2256,    -00.053709) --  (axis cs:95.2377,    +00.946218) --  (axis cs:95.2498,    +01.946144) --  (axis cs:95.262,    +02.946071) --  (axis cs:95.2741,    +03.945997) --  (axis cs:95.2863,    +04.945923) --  (axis cs:95.2985,    +05.945849) --  (axis cs:95.3108,    +06.945775) --  (axis cs:95.3231,    +07.945700) --  (axis cs:95.3355,    +08.945625) --  (axis cs:95.348,    +09.945550) ;
  \draw[constellation-boundary]  (axis cs:85.7936,    +15.561920) ;
  \draw[constellation-boundary]  (axis cs:85.755,    +12.562154) --  (axis cs:85.7614,    +13.062115) --  (axis cs:85.7742,    +14.062038) --  (axis cs:85.7871,    +15.061960) --  (axis cs:85.7936,    +15.561920) ;
  \draw[constellation-boundary]  (axis cs:85.755,    +12.562154) --  (axis cs:86.7552,    +12.550043) --  (axis cs:87.7553,    +12.537918) --  (axis cs:88.2553,    +12.531851) ;
  \draw[constellation-boundary]  (axis cs:88.2553,    +12.531851) --  (axis cs:88.2617,    +13.031812) --  (axis cs:88.2745,    +14.031734) --  (axis cs:88.2875,    +15.031656) --  (axis cs:88.3005,    +16.031576) --  (axis cs:88.3137,    +17.031496) --  (axis cs:88.3271,    +18.031415) ;
  \draw[constellation-boundary]  (axis cs:87.327,    +18.043547) --  (axis cs:87.827,    +18.037483) --  (axis cs:88.3271,    +18.031415) ;
  \draw[constellation-boundary]  (axis cs:87.327,    +18.043547) --  (axis cs:87.3405,    +19.043466) --  (axis cs:87.3541,    +20.043383) --  (axis cs:87.3679,    +21.043299) --  (axis cs:87.3819,    +22.043214) --  (axis cs:87.3938,    +22.876476) ;
  \draw[constellation-boundary]  (axis cs:87.3938,    +22.876476) --  (axis cs:87.8939,    +22.870411) --  (axis cs:88.894,    +22.858273) --  (axis cs:89.8941,    +22.846128) --  (axis cs:90.1441,    +22.843091) ;
  \draw[constellation-boundary]  (axis cs:121.038,    -43.353501) --  (axis cs:122.044,    -43.363943) --  (axis cs:123.05,    -43.374274) --  (axis cs:124.056,    -43.384491) --  (axis cs:125.062,    -43.394592) --  (axis cs:126.069,    -43.404572) --  (axis cs:126.572,    -43.409516) ;
  \draw[constellation-boundary]  (axis cs:126.572,    -43.409516) --  (axis cs:126.59,    -42.409605) --  (axis cs:126.608,    -41.409692) --  (axis cs:126.625,    -40.409776) --  (axis cs:126.642,    -39.409857) --  (axis cs:126.658,    -38.409936) --  (axis cs:126.674,    -37.410013) --  (axis cs:126.689,    -36.410088) --  (axis cs:126.704,    -35.410161) --  (axis cs:126.719,    -34.410233) --  (axis cs:126.733,    -33.410302) --  (axis cs:126.747,    -32.410370) --  (axis cs:126.76,    -31.410437) --  (axis cs:126.774,    -30.410502) --  (axis cs:126.787,    -29.410566) --  (axis cs:126.799,    -28.410628) --  (axis cs:126.812,    -27.410690) --  (axis cs:126.824,    -26.410750) --  (axis cs:126.836,    -25.410809) --  (axis cs:126.848,    -24.410868) --  (axis cs:126.86,    -23.410925) --  (axis cs:126.872,    -22.410981) --  (axis cs:126.883,    -21.411037) --  (axis cs:126.894,    -20.411092) --  (axis cs:126.905,    -19.411146) --  (axis cs:126.916,    -18.411200) --  (axis cs:126.927,    -17.411253) --  (axis cs:126.938,    -16.411305) --  (axis cs:126.948,    -15.411357) --  (axis cs:126.959,    -14.411408) --  (axis cs:126.969,    -13.411459) --  (axis cs:126.979,    -12.411510) --  (axis cs:126.99,    -11.411560) ;
  \draw[constellation-boundary]  (axis cs:126.678,    -37.160032) --  (axis cs:127.18,    -37.164939) --  (axis cs:128.186,    -37.174656) --  (axis cs:129.191,    -37.184244) --  (axis cs:130.197,    -37.193699) --  (axis cs:131.203,    -37.203020) --  (axis cs:132.209,    -37.212201) --  (axis cs:133.215,    -37.221242) --  (axis cs:134.221,    -37.230139) --  (axis cs:135.228,    -37.238889) --  (axis cs:136.234,    -37.247490) --  (axis cs:137.241,    -37.255939) --  (axis cs:138.247,    -37.264233) --  (axis cs:139.254,    -37.272370) --  (axis cs:140.261,    -37.280348) --  (axis cs:141.268,    -37.288162) --  (axis cs:141.772,    -37.292008) ;
  


% Labels

\node[constellation-label] at (axis cs:{105,38})  {\large \textls{Auriga}};
\node[constellation-label] at (axis cs:{130,36})  {\large \textls{Lynx}};
\node[constellation-label] at (axis cs:{150,37})  {\large \textls{Leo Minor}};
\node[constellation-label] at (axis cs:{171,35})  {\large \textls{Ursa Major}};
\node[constellation-label] at (axis cs:{192,37})  {\large \textls{Canes}}; 
  \node[constellation-label] at (axis cs:{192,35})  {\large \textls{Venatici}};
\node[constellation-label] at (axis cs:{188,23.5})  {\large \textls{Coma}}; 
  \node[constellation-label] at (axis cs:{188,22})  {\large \textls{Berenices}};
\node[constellation-label] at (axis cs:{103,19}) {\huge \textls{Gemini}};
\node[constellation-label] at (axis cs:{112,-2})  {\Large \textls{Monoceros}};
\node[constellation-label] at (axis cs:{103,-18})  {\Large \textls{Canis}};
  \node[constellation-label] at (axis cs:{103,-22})  {\Large \textls{Major}};
\node[constellation-label] at (axis cs:{116,8})  {\Large \textls{Canis}};
  \node[constellation-label] at (axis cs:{116,4})  {\Large \textls{Minor}};
\node[constellation-label] at (axis cs:{155,-2}) {\Large \textls{Sextans}};
\node[constellation-label] at (axis cs:{172,-16}) {\Large \textls{Crater}};
\node[constellation-label] at (axis cs:{186,-20}) {\Large \textls{Corvus}};
\node[constellation-label][rotate=-45] at (axis cs:{115,-35}) {\huge \textls{Puppis}};
\node[constellation-label] at (axis cs:{135,-29}) {\Large \textls{Pyxis}};
\node[constellation-label] at (axis cs:{153,-35}) {\Large \textls{Antlia}};
\node[constellation-label] at (axis cs:{185,-38}) {\large \textls{Centaurus}};
\node[constellation-label] at (axis cs:{130,23}) {\huge \textls{Cancer}};
\node[constellation-label] at (axis cs:{160,16.5}) {\Huge \textls{Leo}};
\node[constellation-label] at (axis cs:{182.5,-5}) {\Huge \textls{Virgo}};
\node[constellation-label][rotate=30] at (axis cs:{155,-23}) {\Huge \textls{Hydra}};
\node[constellation-label,rotate=-90] at (axis cs:{92,8})  {\large \textls{Orion}};
\node[constellation-label,rotate=-90] at (axis cs:{91,-20})  {\large \textls{Lepus}};
\node[constellation-label] at (axis cs:{95,-36}) {\large \textls{Columba}};
\node[constellation-label] at (axis cs:{136,-38.5}) {\normalsize \textls{Vela}};

\end{axis}



% Nebulae and nebulae names
\begin{axis}[name=NGC,axis lines=none]

\node[NGC,pin={[pin distance=-1.2\onedegree,NGC-label]90:{2207}}] at (axis cs:94.100,-21.367) {\tikz\pgfuseplotmark{GAL};}; % 10.7
\node[NGC,pin={[pin distance=-1.2\onedegree,NGC-label]90:{2217}}] at (axis cs:95.425,-27.233) {\tikz\pgfuseplotmark{GAL};}; % 10.4
\node[NGC,pin={[pin distance=-1.2\onedegree,NGC-label]90:{2613}}] at (axis cs:128.350,-22.967) {\tikz\pgfuseplotmark{GAL};}; % 10.4
\node[NGC,pin={[pin distance=-1.2\onedegree,NGC-label]90:{2683}}] at (axis cs:133.175,33.417) {\tikz\pgfuseplotmark{GAL};}; %  9.7
\node[NGC,pin={[pin distance=-1.2\onedegree,NGC-label]90:{2775}}] at (axis cs:137.575,7.033) {\tikz\pgfuseplotmark{GAL};}; % 10.3
\node[NGC,pin={[pin distance=-1.2\onedegree,NGC-label]90:{2784}}] at (axis cs:138.075,-24.167) {\tikz\pgfuseplotmark{GAL};}; % 10.1
\node[NGC,pin={[pin distance=-1.2\onedegree,NGC-label]90:{2859}}] at (axis cs:141.075,34.517) {\tikz\pgfuseplotmark{GAL};}; % 10.7
\node[NGC,pin={[pin distance=-1.2\onedegree,NGC-label]90:{2903/5}}] at (axis cs:143.050,21.500) {\tikz\pgfuseplotmark{GAL};}; %  8.9
%\node[NGC,pin={[pin distance=-1.2\onedegree,NGC-label]90:{2905}}] at (axis cs:143.050,21.517) {\tikz\pgfuseplotmark{GAL};}; % 10. 
\node[NGC,pin={[pin distance=-1.2\onedegree,NGC-label]90:{2974}}] at (axis cs:145.650,-3.700) {\tikz\pgfuseplotmark{GAL};}; % 10.8
\node[NGC,pin={[pin distance=-1.2\onedegree,NGC-label]90:{2986}}] at (axis cs:146.075,-21.283) {\tikz\pgfuseplotmark{GAL};}; % 10.9
\node[NGC,pin={[pin distance=-1.2\onedegree,NGC-label]90:{3109}}] at (axis cs:150.775,-26.150) {\tikz\pgfuseplotmark{GAL};}; % 10. 
\node[NGC,pin={[pin distance=-1.2\onedegree,NGC-label]90:{3115}}] at (axis cs:151.300,-7.717) {\tikz\pgfuseplotmark{GAL};}; %  9.2
\node[NGC,pin={[pin distance=-1.2\onedegree,NGC-label]90:{3166/9}}] at (axis cs:153.450,3.433) {\tikz\pgfuseplotmark{GAL};}; % 10.6
\node[NGC,pin={[pin distance=-1.2\onedegree,NGC-label]90:{}}] at (axis cs:153.550,3.467) {\tikz\pgfuseplotmark{GAL};}; % 10.5
\node[NGC,pin={[pin distance=-1.2\onedegree,NGC-label]90:{3193}}] at (axis cs:154.600,21.900) {\tikz\pgfuseplotmark{GAL};}; % 10.9
\node[NGC,pin={[pin distance=-1.2\onedegree,NGC-label]90:{3227}}] at (axis cs:155.875,19.867) {\tikz\pgfuseplotmark{GAL};}; % 10.8
\node[NGC,pin={[pin distance=-1.2\onedegree,NGC-label]90:{3245}}] at (axis cs:156.825,28.500) {\tikz\pgfuseplotmark{GAL};}; % 10.8
\node[NGC,pin={[pin distance=-1.2\onedegree,NGC-label]0:{3338}}] at (axis cs:160.525,13.750) {\tikz\pgfuseplotmark{GAL};}; % 10.8
\node[NGC,pin={[pin distance=-1.2\onedegree,NGC-label]90:{3344}}] at (axis cs:160.875,24.917) {\tikz\pgfuseplotmark{GAL};}; % 10.0
\node[NGC,pin={[pin distance=-1.2\onedegree,NGC-label]180:{3377}}] at (axis cs:161.925,13.983) {\tikz\pgfuseplotmark{GAL};}; % 10.2
\node[NGC,pin={[pin distance=-1.2\onedegree,NGC-label]180:{3384}}] at (axis cs:162.075,12.633) {\tikz\pgfuseplotmark{GAL};}; % 10.0
\node[NGC,pin={[pin distance=-1.2\onedegree,NGC-label]180:{3412}}] at (axis cs:162.725,13.417) {\tikz\pgfuseplotmark{GAL};}; % 10.6
\node[NGC,pin={[pin distance=-1.2\onedegree,NGC-label]90:{3414}}] at (axis cs:162.825,27.983) {\tikz\pgfuseplotmark{GAL};}; % 10.8
\node[NGC,pin={[pin distance=-1.2\onedegree,NGC-label]90:{3489}}] at (axis cs:165.075,13.900) {\tikz\pgfuseplotmark{GAL};}; % 10.3
\node[NGC,pin={[pin distance=-1.2\onedegree,NGC-label]90:{3486}}] at (axis cs:165.100,28.967) {\tikz\pgfuseplotmark{GAL};}; % 10.3
\node[NGC,pin={[pin distance=-1.2\onedegree,NGC-label]90:{3521}}] at (axis cs:166.450,-0.033) {\tikz\pgfuseplotmark{GAL};}; %  8.9
\node[NGC,pin={[pin distance=-1.2\onedegree,NGC-label]90:{3557}}] at (axis cs:167.500,-37.533) {\tikz\pgfuseplotmark{GAL};}; % 10.4
\node[NGC,pin={[pin distance=-1.2\onedegree,NGC-label]90:{3585}}] at (axis cs:168.325,-26.750) {\tikz\pgfuseplotmark{GAL};}; % 10.0
\node[NGC,pin={[pin distance=-1.2\onedegree,NGC-label]00:{3607}}] at (axis cs:169.225,18.050) {\tikz\pgfuseplotmark{GAL};}; % 10.0
\node[NGC,pin={[pin distance=-1.2\onedegree,NGC-label]90:{3621}}] at (axis cs:169.575,-32.817) {\tikz\pgfuseplotmark{GAL};}; % 10. 
\node[NGC,pin={[pin distance=-1.2\onedegree,NGC-label]90:{3626}}] at (axis cs:170.025,18.350) {\tikz\pgfuseplotmark{GAL};}; % 10.9
\node[NGC,pin={[pin distance=-1.2\onedegree,NGC-label]90:{3628}}] at (axis cs:170.075,13.600) {\tikz\pgfuseplotmark{GAL};}; %  9.5
\node[NGC,pin={[pin distance=-1.2\onedegree,NGC-label]90:{3640}}] at (axis cs:170.275,3.233) {\tikz\pgfuseplotmark{GAL};}; % 10.3
\node[NGC,pin={[pin distance=-1.2\onedegree,NGC-label]-90:{3665}}] at (axis cs:171.175,38.767) {\tikz\pgfuseplotmark{GAL};}; % 10.8
\node[NGC,pin={[pin distance=-1.2\onedegree,NGC-label]90:{3810}}] at (axis cs:175.250,11.467) {\tikz\pgfuseplotmark{GAL};}; % 10.8
\node[NGC,pin={[pin distance=-1.2\onedegree,NGC-label]90:{3923}}] at (axis cs:177.750,-28.800) {\tikz\pgfuseplotmark{GAL};}; % 10.1
\node[NGC,pin={[pin distance=-1.2\onedegree,NGC-label]90:{3962}}] at (axis cs:178.675,-13.967) {\tikz\pgfuseplotmark{GAL};}; % 10.6
\node[NGC,pin={[pin distance=-1.2\onedegree,NGC-label]90:{4038}}] at (axis cs:180.475,-18.867) {\tikz\pgfuseplotmark{GAL};}; % 10.7
\node[NGC,pin={[pin distance=-1.2\onedegree,NGC-label]180:{4151}}] at (axis cs:182.625,39.400) {\tikz\pgfuseplotmark{GAL};}; % 10.4
\node[NGC,pin={[pin distance=-1.2\onedegree,NGC-label]90:{4179}}] at (axis cs:183.225,1.300) {\tikz\pgfuseplotmark{GAL};}; % 10.9
\node[NGC,pin={[pin distance=-1.2\onedegree,NGC-label]90:{4203}}] at (axis cs:183.775,33.200) {\tikz\pgfuseplotmark{GAL};}; % 10.7
\node[NGC,pin={[pin distance=-1.2\onedegree,NGC-label]90:{4214}}] at (axis cs:183.900,36.333) {\tikz\pgfuseplotmark{GAL};}; %  9.7
\node[NGC,pin={[pin distance=-1.2\onedegree,NGC-label]0:{4216}}] at (axis cs:183.975,13.150) {\tikz\pgfuseplotmark{GAL};}; % 10.0
\node[NGC,pin={[pin distance=-1.2\onedegree,NGC-label]90:{4244}}] at (axis cs:184.375,37.817) {\tikz\pgfuseplotmark{GAL};}; % 10.2
\node[NGC,pin={[pin distance=-1.2\onedegree,NGC-label]90:{4261}}] at (axis cs:184.850,5.817) {\tikz\pgfuseplotmark{GAL};}; % 10.3
\node[NGC,pin={[pin distance=-1.2\onedegree,NGC-label]0:{4267}}] at (axis cs:184.950,12.800) {\tikz\pgfuseplotmark{GAL};}; % 10.9
\node[NGC,pin={[pin distance=-1.2\onedegree,NGC-label]0:{4274}}] at (axis cs:184.950,29.617) {\tikz\pgfuseplotmark{GAL};}; % 10.4
\node[NGC,pin={[pin distance=-1.2\onedegree,NGC-label]180:{4278}}] at (axis cs:185.025,29.283) {\tikz\pgfuseplotmark{GAL};}; % 10.2
\node[NGC,pin={[pin distance=-1.2\onedegree,NGC-label]90:{4314}}] at (axis cs:185.650,29.883) {\tikz\pgfuseplotmark{GAL};}; % 10.5
\node[NGC,pin={[pin distance=-1.2\onedegree,NGC-label]90:{4371}}] at (axis cs:186.225,11.700) {\tikz\pgfuseplotmark{GAL};}; % 10.8
\node[NGC,pin={[pin distance=-1.2\onedegree,NGC-label]90:{4395}}] at (axis cs:186.450,33.550) {\tikz\pgfuseplotmark{GAL};}; % 10.2
\node[NGC,pin={[pin distance=-1.2\onedegree,NGC-label]90:{}}] at (axis cs:186.475,18.217) {\tikz\pgfuseplotmark{GAL};}; % 10.9
\node[NGC,pin={[pin distance=-1.2\onedegree,NGC-label]90:{4414}}] at (axis cs:186.600,31.217) {\tikz\pgfuseplotmark{GAL};}; % 10.3
\node[NGC,pin={[pin distance=-1.2\onedegree,NGC-label]-90:{4429}}] at (axis cs:186.850,11.117) {\tikz\pgfuseplotmark{GAL};}; % 10.2
\node[NGC,pin={[pin distance=-1.2\onedegree,NGC-label]90:{}}] at (axis cs:186.925,13.083) {\tikz\pgfuseplotmark{GAL};}; % 10.9
\node[NGC,pin={[pin distance=-1.2\onedegree,NGC-label]90:{}}] at (axis cs:186.950,13.017) {\tikz\pgfuseplotmark{GAL};}; % 10.1
\node[NGC,pin={[pin distance=-1.2\onedegree,NGC-label]-90:{4442}}] at (axis cs:187.025,9.800) {\tikz\pgfuseplotmark{GAL};}; % 10.5
\node[NGC,pin={[pin distance=-1.2\onedegree,NGC-label]90:{4450}}] at (axis cs:187.125,17.083) {\tikz\pgfuseplotmark{GAL};}; % 10.1
\node[NGC,pin={[pin distance=-1.2\onedegree,NGC-label]90:{4457}}] at (axis cs:187.250,3.567) {\tikz\pgfuseplotmark{GAL};}; % 10.8
\node[NGC,pin={[pin distance=-1.2\onedegree,NGC-label]90:{}}] at (axis cs:187.250,13.983) {\tikz\pgfuseplotmark{GAL};}; % 10.4
\node[NGC,pin={[pin distance=-1.2\onedegree,NGC-label]90:{}}] at (axis cs:187.450,13.433) {\tikz\pgfuseplotmark{GAL};}; % 10.2
\node[NGC,pin={[pin distance=-1.2\onedegree,NGC-label]90:{}}] at (axis cs:187.500,13.633) {\tikz\pgfuseplotmark{GAL};}; % 10.4
\node[NGC,pin={[pin distance=-1.2\onedegree,NGC-label]90:{4494}}] at (axis cs:187.850,25.783) {\tikz\pgfuseplotmark{GAL};}; %  9.9
\node[NGC,pin={[pin distance=-1.2\onedegree,NGC-label]90:{4517}}] at (axis cs:188.200,0.117) {\tikz\pgfuseplotmark{GAL};}; % 10.5
\node[NGC,pin={[pin distance=-1.2\onedegree,NGC-label]180:{4526}}] at (axis cs:188.500,7.700) {\tikz\pgfuseplotmark{GAL};}; %  9.6
\node[NGC,pin={[pin distance=-1.2\onedegree,NGC-label]90:{4527}}] at (axis cs:188.525,2.650) {\tikz\pgfuseplotmark{GAL};}; % 10.4
\node[NGC,pin={[pin distance=-1.2\onedegree,NGC-label]180:{4535}}] at (axis cs:188.575,8.200) {\tikz\pgfuseplotmark{GAL};}; %  9.8
\node[NGC,pin={[pin distance=-1.2\onedegree,NGC-label]-90:{4536}}] at (axis cs:188.625,2.183) {\tikz\pgfuseplotmark{GAL};}; % 10.4
\node[NGC,pin={[pin distance=-1.2\onedegree,NGC-label]-90:{4546}}] at (axis cs:188.875,-3.800) {\tikz\pgfuseplotmark{GAL};}; % 10.3
\node[NGC,pin={[pin distance=-1.2\onedegree,NGC-label]90:{4559}}] at (axis cs:189,27.967) {\tikz\pgfuseplotmark{GAL};}; %  9.9
\node[NGC,pin={[pin distance=-1.2\onedegree,NGC-label]90:{4565}}] at (axis cs:189.075,25.983) {\tikz\pgfuseplotmark{GAL};}; %  9.6
\node[NGC,pin={[pin distance=-1.2\onedegree,NGC-label]-90:{4568}}] at (axis cs:189.150,11.233) {\tikz\pgfuseplotmark{GAL};}; % 10.8
\node[NGC,pin={[pin distance=-1.2\onedegree,NGC-label]-90:{4570}}] at (axis cs:189.225,7.250) {\tikz\pgfuseplotmark{GAL};}; % 10.9
\node[NGC,pin={[pin distance=-1.2\onedegree,NGC-label]-90:{4596}}] at (axis cs:189.975,10.183) {\tikz\pgfuseplotmark{GAL};}; % 10.5
\node[NGC,pin={[pin distance=-1.2\onedegree,NGC-label]90:{4631}}] at (axis cs:190.525,32.533) {\tikz\pgfuseplotmark{GAL};}; %  9.3
\node[NGC,pin={[pin distance=-1.2\onedegree,NGC-label]90:{4636}}] at (axis cs:190.700,2.683) {\tikz\pgfuseplotmark{GAL};}; %  9.6
\node[NGC,pin={[pin distance=-1.2\onedegree,NGC-label]180:{4643}}] at (axis cs:190.825,1.983) {\tikz\pgfuseplotmark{GAL};}; % 10.6
\node[NGC,pin={[pin distance=-1.2\onedegree,NGC-label]90:{4651}}] at (axis cs:190.925,16.400) {\tikz\pgfuseplotmark{GAL};}; % 10.7
\node[NGC,pin={[pin distance=-1.2\onedegree,NGC-label]180:{4654}}] at (axis cs:191,13.133) {\tikz\pgfuseplotmark{GAL};}; % 10.5
\node[NGC,pin={[pin distance=-1.2\onedegree,NGC-label]180:{4656}}] at (axis cs:191,32.167) {\tikz\pgfuseplotmark{GAL};}; % 10.4
\node[NGC,pin={[pin distance=-1.2\onedegree,NGC-label]-90:{4666}}] at (axis cs:191.275,-0.467) {\tikz\pgfuseplotmark{GAL};}; % 10.8
\node[NGC,pin={[pin distance=-1.2\onedegree,NGC-label]0:{4689}}] at (axis cs:191.950,13.767) {\tikz\pgfuseplotmark{GAL};}; % 10.9
\node[NGC,pin={[pin distance=-1.2\onedegree,NGC-label]90:{4698}}] at (axis cs:192.100,8.483) {\tikz\pgfuseplotmark{GAL};}; % 10.7
\node[NGC,pin={[pin distance=-1.2\onedegree,NGC-label]-90:{4697}}] at (axis cs:192.150,-5.800) {\tikz\pgfuseplotmark{GAL};}; %  9.3
\node[NGC,pin={[pin distance=-1.2\onedegree,NGC-label]90:{4699}}] at (axis cs:192.250,-8.667) {\tikz\pgfuseplotmark{GAL};}; %  9.6
\node[NGC,pin={[pin distance=-1.2\onedegree,NGC-label]180:{4725}}] at (axis cs:192.600,25.500) {\tikz\pgfuseplotmark{GAL};}; %  9.2
\node[NGC,pin={[pin distance=-1.2\onedegree,NGC-label]90:{4754}}] at (axis cs:193.075,11.317) {\tikz\pgfuseplotmark{GAL};}; % 10.6
\node[NGC,pin={[pin distance=-1.2\onedegree,NGC-label]90:{4753}}] at (axis cs:193.100,-1.200) {\tikz\pgfuseplotmark{GAL};}; %  9.9
\node[NGC,pin={[pin distance=-1.2\onedegree,NGC-label]180:{4762}}] at (axis cs:193.225,11.233) {\tikz\pgfuseplotmark{GAL};}; % 10.2
\node[NGC,pin={[pin distance=-1.2\onedegree,NGC-label]90:{4856}}] at (axis cs:194.825,-15.033) {\tikz\pgfuseplotmark{GAL};}; % 10.4
\node[NGC,pin={[pin distance=-1.2\onedegree,NGC-label]90:{4958}}] at (axis cs:196.450,-8.017) {\tikz\pgfuseplotmark{GAL};}; % 10.5
\node[NGC,pin={[pin distance=-1.2\onedegree,NGC-label]180:{5005}}] at (axis cs:197.725,37.050) {\tikz\pgfuseplotmark{GAL};}; %  9.8
\node[NGC,pin={[pin distance=-1.2\onedegree,NGC-label]90:{5018}}] at (axis cs:198.250,-19.517) {\tikz\pgfuseplotmark{GAL};}; % 10.8
\node[NGC,pin={[pin distance=-1.2\onedegree,NGC-label]-90:{5033}}] at (axis cs:198.350,36.600) {\tikz\pgfuseplotmark{GAL};}; % 10.1
\node[NGC,pin={[pin distance=-1.2\onedegree,NGC-label]-90:{2392}}] at (axis cs:112.300,20.917) {\tikz\pgfuseplotmark{PN};}; % 10. 
\node[NGC,pin={[pin distance=-1.2\onedegree,NGC-label]90:{2438}}] at (axis cs:115.450,-14.733) {\tikz\pgfuseplotmark{PN};}; % 10. 
\node[NGC,pin={[pin distance=-1.2\onedegree,NGC-label]90:{3242}}] at (axis cs:156.200,-18.633) {\tikz\pgfuseplotmark{PN};}; %  9. 
\node[NGC,pin={[pin distance=-1.2\onedegree,NGC-label]90:{4361}}] at (axis cs:186.125,-18.800) {\tikz\pgfuseplotmark{PN};}; % 10. 
\node[NGC,pin={[pin distance=-1.2\onedegree,NGC-label]90:{2264}}] at (axis cs:100.275,9.883) {\tikz\pgfuseplotmark{CN};}; %  3.9
\node[NGC,pin={[pin distance=-1.2\onedegree,NGC-label]180:{2362}}] at (axis cs:109.700,-24.950) {\tikz\pgfuseplotmark{CN};}; %  4.1
\node[NGC,pin={[pin distance=-1.2\onedegree,NGC-label]90:{2467}}] at (axis cs:118.150,-26.383) {\tikz\pgfuseplotmark{CN};}; %  7. 
\node[NGC,pin={[pin distance=-1.2\onedegree,NGC-label]90:{2579}}] at (axis cs:125.275,-36.183) {\tikz\pgfuseplotmark{CN};}; %  7.5

\node[NGC,pin={[pin distance=-1.2\onedegree,NGC-label]90:{2129}}] at (axis cs:90.250,23.300) {\tikz\pgfuseplotmark{OC};}; %  6.7
\node[NGC,pin={[pin distance=-1.2\onedegree,NGC-label]90:{2141}}] at (axis cs:90.775,10.433) {\tikz\pgfuseplotmark{OC};}; %  9.4
\node[NGC,pin={[pin distance=-1.2\onedegree,NGC-label]0:{}}] at (axis cs:91.250,24) {\tikz\pgfuseplotmark{OC};}; %  8.4
\node[NGC,pin={[pin distance=-1.2\onedegree,NGC-label]180:{2158}}] at (axis cs:91.875,24.100) {\tikz\pgfuseplotmark{OC};}; %  8.6
\node[NGC,pin={[pin distance=-1.2\onedegree,NGC-label]0:{2169}}] at (axis cs:92.100,13.950) {\tikz\pgfuseplotmark{OC};}; %  5.9
\node[NGC,pin={[pin distance=-1.2\onedegree,NGC-label]90:{2175}}] at (axis cs:92.450,20.317) {\tikz\pgfuseplotmark{OC};}; %  6.8
\node[NGC,pin={[pin distance=-1.2\onedegree,NGC-label]-90:{2186}}] at (axis cs:93.050,5.450) {\tikz\pgfuseplotmark{OC};}; %  8.7
\node[NGC,pin={[pin distance=-1.2\onedegree,NGC-label]0:{2194}}] at (axis cs:93.450,12.800) {\tikz\pgfuseplotmark{OC};}; %  8.5
\node[NGC,pin={[pin distance=-1.2\onedegree,NGC-label]-90:{2204}}] at (axis cs:93.925,-18.650) {\tikz\pgfuseplotmark{OC};}; %  8.6
\node[NGC,pin={[pin distance=-1.2\onedegree,NGC-label]90:{2215}}] at (axis cs:95.250,-7.283) {\tikz\pgfuseplotmark{OC};}; %  8.4
\node[NGC,pin={[pin distance=-1.2\onedegree,NGC-label]90:{2232}}] at (axis cs:96.650,-4.750) {\tikz\pgfuseplotmark{OC};}; %  3.9
\node[NGC,pin={[pin distance=-1.2\onedegree,NGC-label]-90:{2236}}] at (axis cs:97.425,6.833) {\tikz\pgfuseplotmark{OC};}; %  8.5
\node[NGC,pin={[pin distance=-1.2\onedegree,NGC-label]90:{2243}}] at (axis cs:97.450,-31.283) {\tikz\pgfuseplotmark{OC};}; %  9.4
\node[NGC,pin={[pin distance=-1.2\onedegree,NGC-label]-90:{2244}}] at (axis cs:98.100,4.867) {\tikz\pgfuseplotmark{OC};}; %  4.8
\node[NGC,pin={[pin distance=-1.2\onedegree,NGC-label]90:{2250}}] at (axis cs:98.200,-5.033) {\tikz\pgfuseplotmark{OC};}; %  9. 
\node[NGC,pin={[pin distance=-1.2\onedegree,NGC-label]180:{2251}}] at (axis cs:98.675,8.367) {\tikz\pgfuseplotmark{OC};}; %  7.3
\node[NGC,pin={[pin distance=-1.2\onedegree,NGC-label]90:{2252}}] at (axis cs:98.750,5.383) {\tikz\pgfuseplotmark{OC};}; %  8. 
\node[NGC,pin={[pin distance=-1.2\onedegree,NGC-label]180:{2254}}] at (axis cs:99,7.667) {\tikz\pgfuseplotmark{OC};}; %  9.7
\node[NGC,pin={[pin distance=-1.2\onedegree,NGC-label]90:{2266}}] at (axis cs:100.800,26.967) {\tikz\pgfuseplotmark{OC};}; % 10. 
\node[NGC,pin={[pin distance=-1.2\onedegree,NGC-label]180:{2269}}] at (axis cs:100.975,4.567) {\tikz\pgfuseplotmark{OC};}; % 10.0
\node[NGC,pin={[pin distance=-1.2\onedegree,NGC-label]90:{2286}}] at (axis cs:101.900,-3.167) {\tikz\pgfuseplotmark{OC};}; %  7.5
\node[NGC,pin={[pin distance=-1.2\onedegree,NGC-label]180:{2301}}] at (axis cs:102.950,0.467) {\tikz\pgfuseplotmark{OC};}; %  6.0
\node[NGC,pin={[pin distance=-1.2\onedegree,NGC-label]-90:{2302}}] at (axis cs:102.975,-7.067) {\tikz\pgfuseplotmark{OC};}; %  8.9
\node[NGC,pin={[pin distance=-1.2\onedegree,NGC-label]90:{2304}}] at (axis cs:103.750,18.017) {\tikz\pgfuseplotmark{OC};}; % 10. 
\node[NGC,pin={[pin distance=-1.2\onedegree,NGC-label]0:{2311}}] at (axis cs:104.450,-4.583) {\tikz\pgfuseplotmark{OC};}; % 10. 
\node[NGC,pin={[pin distance=-1.2\onedegree,NGC-label]90:{2324}}] at (axis cs:106.050,1.050) {\tikz\pgfuseplotmark{OC};}; %  8.4
\node[NGC,pin={[pin distance=-1.2\onedegree,NGC-label]90:{2335}}] at (axis cs:106.650,-10.083) {\tikz\pgfuseplotmark{OC};}; %  7.2
\node[NGC,pin={[pin distance=-1.2\onedegree,NGC-label]90:{2331}}] at (axis cs:106.800,27.350) {\tikz\pgfuseplotmark{OC};}; %  9. 
\node[NGC,pin={[pin distance=-1.2\onedegree,NGC-label]0:{2343}}] at (axis cs:107.075,-10.650) {\tikz\pgfuseplotmark{OC};}; %  6.7
\node[NGC,pin={[pin distance=-1.2\onedegree,NGC-label]0:{2345}}] at (axis cs:107.075,-13.167) {\tikz\pgfuseplotmark{OC};}; %  7.7
\node[NGC,pin={[pin distance=-1.2\onedegree,NGC-label]90:{2354}}] at (axis cs:108.575,-25.733) {\tikz\pgfuseplotmark{OC};}; %  6.5
\node[NGC,pin={[pin distance=-0.8\onedegree,NGC-label]90:{2353}}] at (axis cs:108.650,-10.300) {\tikz\pgfuseplotmark{OC};}; %  7.1
\node[NGC,pin={[pin distance=-1.2\onedegree,NGC-label]90:{2355}}] at (axis cs:109.225,13.783) {\tikz\pgfuseplotmark{OC};}; % 10. 
\node[NGC,pin={[pin distance=-1.2\onedegree,NGC-label]90:{2360}}] at (axis cs:109.450,-15.617) {\tikz\pgfuseplotmark{OC};}; %  7.2

\node[NGC,pin={[pin distance=-1.2\onedegree,NGC-label]90:{2367}}] at (axis cs:110.025,-21.933) {\tikz\pgfuseplotmark{OC};}; %  7.9
\node[NGC,pin={[pin distance=-1.2\onedegree,NGC-label]90:{2374}}] at (axis cs:111,-13.267) {\tikz\pgfuseplotmark{OC};}; %  8.0
\node[NGC,pin={[pin distance=-1.2\onedegree,NGC-label]90:{2383/4}}] at (axis cs:111.200,-20.933) {\tikz\pgfuseplotmark{OC};}; %  8.4
%\node[NGC,pin={[pin distance=-1.2\onedegree,NGC-label]180:{2384}}] at (axis cs:111.275,-21.033) {\tikz\pgfuseplotmark{OC};}; %  7.4
\node[NGC,pin={[pin distance=-1.2\onedegree,NGC-label]90:{2395}}] at (axis cs:111.775,13.583) {\tikz\pgfuseplotmark{OC};}; %  8.0
\node[NGC,pin={[pin distance=-1.2\onedegree,NGC-label]90:{2396}}] at (axis cs:112.025,-11.733) {\tikz\pgfuseplotmark{OC};}; %  7. 
\node[NGC,pin={[pin distance=-1.2\onedegree,NGC-label]00:{2414}}] at (axis cs:113.325,-15.450) {\tikz\pgfuseplotmark{OC};}; %  7.9
\node[NGC,pin={[pin distance=-1.2\onedegree,NGC-label]90:{2421}}] at (axis cs:114.075,-20.617) {\tikz\pgfuseplotmark{OC};}; %  8.3
\node[NGC,pin={[pin distance=-1.2\onedegree,NGC-label]90:{2423}}] at (axis cs:114.275,-13.867) {\tikz\pgfuseplotmark{OC};}; %  6.7
\node[NGC,pin={[pin distance=-1.2\onedegree,NGC-label]90:{2420}}] at (axis cs:114.625,21.567) {\tikz\pgfuseplotmark{OC};}; %  8.3
\node[NGC,pin={[pin distance=-1.2\onedegree,NGC-label]90:{2439}}] at (axis cs:115.200,-31.650) {\tikz\pgfuseplotmark{OC};}; %  6.9
\node[NGC,pin={[pin distance=-1.2\onedegree,NGC-label]90:{2432}}] at (axis cs:115.225,-19.083) {\tikz\pgfuseplotmark{OC};}; % 10. 
\node[NGC,pin={[pin distance=-1.2\onedegree,NGC-label]90:{2451}}] at (axis cs:116.350,-37.967) {\tikz\pgfuseplotmark{OC};}; %  2.8
\node[NGC,pin={[pin distance=-1.2\onedegree,NGC-label]90:{2453}}] at (axis cs:116.950,-27.233) {\tikz\pgfuseplotmark{OC};}; %  8.3
\node[NGC,pin={[pin distance=-1.2\onedegree,NGC-label]90:{2455}}] at (axis cs:117.250,-21.300) {\tikz\pgfuseplotmark{OC};}; % 10. 
\node[NGC,pin={[pin distance=-1.2\onedegree,NGC-label]90:{2477}}] at (axis cs:118.075,-38.550) {\tikz\pgfuseplotmark{OC};}; %  5.8
\node[NGC,pin={[pin distance=-1.2\onedegree,NGC-label]90:{2482}}] at (axis cs:118.725,-24.300) {\tikz\pgfuseplotmark{OC};}; %  7.3
\node[NGC,pin={[pin distance=-1.2\onedegree,NGC-label]90:{2479}}] at (axis cs:118.775,-17.717) {\tikz\pgfuseplotmark{OC};}; % 10. 
\node[NGC,pin={[pin distance=-1.2\onedegree,NGC-label]90:{2483}}] at (axis cs:118.975,-27.933) {\tikz\pgfuseplotmark{OC};}; %  7.6
\node[NGC,pin={[pin distance=-1.2\onedegree,NGC-label]90:{2489}}] at (axis cs:119.050,-30.067) {\tikz\pgfuseplotmark{OC};}; %  7.9
\node[NGC,pin={[pin distance=-1.2\onedegree,NGC-label]90:{2506}}] at (axis cs:120.050,-10.783) {\tikz\pgfuseplotmark{OC};}; %  7.6
\node[NGC,pin={[pin distance=-1.2\onedegree,NGC-label]0:{2509}}] at (axis cs:120.175,-19.067) {\tikz\pgfuseplotmark{OC};}; %  9. 
\node[NGC,pin={[pin distance=-1.2\onedegree,NGC-label]90:{2527}}] at (axis cs:121.325,-28.167) {\tikz\pgfuseplotmark{OC};}; %  6.5
\node[NGC,pin={[pin distance=-1.2\onedegree,NGC-label]90:{2533}}] at (axis cs:121.750,-29.900) {\tikz\pgfuseplotmark{OC};}; %  7.6
\node[NGC,pin={[pin distance=-1.2\onedegree,NGC-label]90:{2539}}] at (axis cs:122.675,-12.833) {\tikz\pgfuseplotmark{OC};}; %  6.5
\node[NGC,pin={[pin distance=-1.2\onedegree,NGC-label]90:{2546}}] at (axis cs:123.100,-37.633) {\tikz\pgfuseplotmark{OC};}; %  6.3
\node[NGC,pin={[pin distance=-1.2\onedegree,NGC-label]-90:{2567}}] at (axis cs:124.650,-30.633) {\tikz\pgfuseplotmark{OC};}; %  7.4
\node[NGC,pin={[pin distance=-1.2\onedegree,NGC-label]90:{2571}}] at (axis cs:124.725,-29.733) {\tikz\pgfuseplotmark{OC};}; %  7.0
\node[NGC,pin={[pin distance=-1.2\onedegree,NGC-label]180:{2580}}] at (axis cs:125.400,-30.317) {\tikz\pgfuseplotmark{OC};}; % 10. 
\node[NGC,pin={[pin distance=-1.2\onedegree,NGC-label]90:{2587}}] at (axis cs:125.875,-29.500) {\tikz\pgfuseplotmark{OC};}; %  9. 
\node[NGC,pin={[pin distance=-1.2\onedegree,NGC-label]-90:{2627}}] at (axis cs:129.325,-29.950) {\tikz\pgfuseplotmark{OC};}; %  8. 
\node[NGC,pin={[pin distance=-1.2\onedegree,NGC-label]90:{2658}}] at (axis cs:130.850,-32.650) {\tikz\pgfuseplotmark{OC};}; %  9. 
\node[NGC,pin={[pin distance=-1.2\onedegree,NGC-label]90:{2818}}] at (axis cs:139,-36.617) {\tikz\pgfuseplotmark{OC};}; %  8.2
\node[NGC,pin={[pin distance=-1.2\onedegree,NGC-label]90:{2298}}] at (axis cs:102.250,-36) {\tikz\pgfuseplotmark{GC};}; %  9.4
\node[NGC,pin={[pin distance=-1.2\onedegree,NGC-label]90:{2419}}] at (axis cs:114.525,38.883) {\tikz\pgfuseplotmark{GC};}; % 10.4
\node[NGC,pin={[pin distance=-1.2\onedegree,NGC-label]90:{4147}}] at (axis cs:182.525,18.550) {\tikz\pgfuseplotmark{GC};}; % 10.3
\node[NGC,pin={[pin distance=-1.2\onedegree,NGC-label]90:{5053}}] at (axis cs:199.100,17.700) {\tikz\pgfuseplotmark{GC};}; %  9.8




\node[Messier,pin={[pin distance=-0.8\onedegree,Messier-label]90:{M35}}] at (axis cs:92.225,24.333) {\tikz\pgfuseplotmark{OC};}; %  5.1
\node[Messier,pin={[pin distance=-0.8\onedegree,Messier-label]90:{M64}}] at (axis cs:194.175,21.683) {\tikz\pgfuseplotmark{GAL};}; %  8.5
\node[Messier,pin={[pin distance=-0.8\onedegree,Messier-label]90:{M44}}] at (axis cs:130.025,19.983) {\tikz\pgfuseplotmark{OC};}; %  3.1
\node[Messier,pin={[pin distance=-0.8\onedegree,Messier-label]90:{M53}}] at (axis cs:198.225,18.167) {\tikz\pgfuseplotmark{GC};}; %  7.7
\node[Messier,pin={[pin distance=-0.8\onedegree,Messier-label]90:{M85}}] at (axis cs:186.350,18.183) {\tikz\pgfuseplotmark{GAL};}; %  9.2
\node[Messier,pin={[pin distance=-1.2\onedegree,Messier-label]90:{M100}}] at (axis cs:185.725,15.817) {\tikz\pgfuseplotmark{GAL};}; %  9.4
\node[Messier,pin={[pin distance=-0.8\onedegree,Messier-label]0:{M98}}] at (axis cs:183.450,14.900) {\tikz\pgfuseplotmark{GAL};}; % 10.1
\node[Messier,pin={[pin distance=-0.8\onedegree,Messier-label]180:{M65}}] at (axis cs:169.725,13.083) {\tikz\pgfuseplotmark{GAL};}; %  9.3
\node[Messier,pin={[pin distance=-0.8\onedegree,Messier-label]-90:{M66}}] at (axis cs:170.050,12.983) {\tikz\pgfuseplotmark{GAL};}; %  9.0
\node[Messier,pin={[pin distance=-0.8\onedegree,Messier-label]0:{M105}}] at (axis cs:161.950,12.583) {\tikz\pgfuseplotmark{GAL};}; %  9.3
\node[Messier,pin={[pin distance=-0.8\onedegree,Messier-label]90:{M67}}] at (axis cs:132.600,11.817) {\tikz\pgfuseplotmark{OC};}; %  6.9
\node[Messier,pin={[pin distance=-0.8\onedegree,Messier-label]00:{M95}}] at (axis cs:161,11.700) {\tikz\pgfuseplotmark{GAL};}; %  9.7
\node[Messier,pin={[pin distance=-0.8\onedegree,Messier-label]180:{M96}}] at (axis cs:161.700,11.817) {\tikz\pgfuseplotmark{GAL};}; %  9.2
\node[Messier,pin={[pin distance=-0.8\onedegree,Messier-label]0:{M49}}] at (axis cs:187.450,8) {\tikz\pgfuseplotmark{GAL};}; %  8.4
\node[Messier,pin={[pin distance=-0.8\onedegree,Messier-label]90:{M61}}] at (axis cs:185.475,4.467) {\tikz\pgfuseplotmark{GAL};}; %  9.7
\node[Messier,pin={[pin distance=-0.8\onedegree,Messier-label]90:{M48}}] at (axis cs:123.450,-5.800) {\tikz\pgfuseplotmark{OC};}; %  5.8
\node[Messier,pin={[pin distance=-0.8\onedegree,Messier-label]90:{M50}}] at (axis cs:105.800,-8.333) {\tikz\pgfuseplotmark{OC};}; %  5.9
\node[Messier,pin={[pin distance=-0.8\onedegree,Messier-label]90:{M104}}] at (axis cs:190,-11.617) {\tikz\pgfuseplotmark{GAL};}; %  8.3
\node[Messier,pin={[pin distance=-0.8\onedegree,Messier-label]90:{M41}}] at (axis cs:101.750,-20.733) {\tikz\pgfuseplotmark{OC};}; %  4.5
\node[Messier,pin={[pin distance=-0.8\onedegree,Messier-label]90:{M93}}] at (axis cs:116.150,-23.867) {\tikz\pgfuseplotmark{OC};}; %  6.2
\node[Messier,pin={[pin distance=-0.8\onedegree,Messier-label]90:{M68}}] at (axis cs:189.875,-26.750) {\tikz\pgfuseplotmark{GC};}; %  8.2


\node[Messier] at (axis cs:188,14.417) {\tikz\pgfuseplotmark{GAL};}; %  9.5
\node[Messier] at (axis cs:188.850,14.500) {\tikz\pgfuseplotmark{GAL};}; % 10.2
\node[Messier] at (axis cs:115.450,-14.817) {\tikz\pgfuseplotmark{OC};}; %  6.1
\node[Messier] at (axis cs:114.150,-14.500) {\tikz\pgfuseplotmark{OC};}; %  4.4
\node[Messier] at (axis cs:189.425,11.817) {\tikz\pgfuseplotmark{GAL};}; %  9.8
\node[Messier] at (axis cs:190.500,11.650) {\tikz\pgfuseplotmark{GAL};}; %  9.8
\node[Messier] at (axis cs:190.925,11.550) {\tikz\pgfuseplotmark{GAL};}; %  8.8
\node[Messier] at (axis cs:186.275,12.883) {\tikz\pgfuseplotmark{GAL};}; %  9.3
\node[Messier] at (axis cs:186.550,12.950) {\tikz\pgfuseplotmark{GAL};}; %  9.2
\node[Messier] at (axis cs:187.700,12.400) {\tikz\pgfuseplotmark{GAL};}; %  8.6
\node[Messier] at (axis cs:188.925,12.550) {\tikz\pgfuseplotmark{GAL};}; %  9.8
\node[Messier] at (axis cs:184.700,14.417) {\tikz\pgfuseplotmark{GAL};}; %  9.8
\node[Messier] at (axis cs:189.200,13.167) {\tikz\pgfuseplotmark{GAL};}; %  9.5


\node[pin={[pin distance=1\onedegree,Messier-label-crowded]90:{M88}}] at (axis cs:188,14.417)          [circle,inner sep=0pt,opacity=0]{};
\node[pin={[pin distance=0.75\onedegree,Messier-label-crowded]160:{M91}}] at (axis cs:188.850,14.500)  [circle,inner sep=0pt,opacity=0]{};
\node[pin={[pin distance=1.1\onedegree,Messier-label-crowded]-90:{M46}}] at (axis cs:115.450,-14.817)    [circle,inner sep=0pt,opacity=0]{};
\node[pin={[pin distance=0.65\onedegree,Messier-label-crowded]-90:{M47}}] at (axis cs:114.150,-14.500) [circle,inner sep=0pt,opacity=0]{};
\node[pin={[pin distance=0.\onedegree,Messier-label-crowded]0:{M58}}] at (axis cs:189.425,11.817)      [circle,inner sep=0pt,opacity=0]{};
\node[pin={[pin distance=2\onedegree,Messier-label-crowded]145:{M59}}] at (axis cs:190.500,11.650)     [circle,inner sep=0pt,opacity=0]{};
\node[pin={[pin distance=.5\onedegree,Messier-label-crowded]-105:{M60}}] at (axis cs:190.925,11.550)  [circle,inner sep=0pt,opacity=0]{};
\node[pin={[pin distance=1\onedegree,Messier-label-crowded]-22:{M84}}] at (axis cs:186.275,12.883)  [circle,inner sep=0pt,opacity=0]{};
\node[pin={[pin distance=0.25\onedegree,Messier-label-crowded]45:{M86}}] at (axis cs:186.550,12.950)   [circle,inner sep=0pt,opacity=0]{};
\node[pin={[pin distance=1.4\onedegree,Messier-label-crowded]-45:{M87}}] at (axis cs:187.700,12.400)  [circle,inner sep=0pt,opacity=0]{};
\node[pin={[pin distance=0.25\onedegree,Messier-label-crowded]180:{M89}}] at (axis cs:188.925,12.550)  [circle,inner sep=0pt,opacity=0]{};
\node[pin={[pin distance=0.75\onedegree,Messier-label-crowded]75:{M99}}] at (axis cs:184.700,14.417)   [circle,inner sep=0pt,opacity=0]{};
\node[pin={[pin distance=0.75\onedegree,Messier-label-crowded]100:{M90}}] at (axis cs:189.200,13.167)     [circle,inner sep=0pt,opacity=0]{};



\node[NGC,pin={[pin distance=-1.2\onedegree,NGC-label]-90:{3842}}] at (axis cs:176.008980,+19.949825) {\tikz\pgfuseplotmark{GAL};}; % 10.7
\node[interest,pin={[pin distance=-0.5\onedegree,interest-label]0:{Leo cluster}}] at (axis cs:176,19.75) {\tikz\draw[dashed]  circle (1.5\onedegree) ;};

\node[interest,pin={[pin distance=-1.5\onedegree,interest-label]270:{Coma cluster}}] at (axis cs:195,27.97) {\tikz\draw[dashed]  circle (1.5\onedegree) ;}; 

\node[interest,pin={[pin distance=-1.5\onedegree,interest-label]90:{Virgo cluster}}] at (axis cs:186.6337,+12.7233) {\tikz\draw[dashed]  circle (8\onedegree) ;}; 

\end{axis}

% Stars and star names
\begin{axis}[name=stars,axis lines=none]
\node[stars] at (axis cs:{90.014},{-3.074}) {\tikz\pgfuseplotmark{m5av};};
\node[stars] at (axis cs:{90.074},{-12.900}) {\tikz\pgfuseplotmark{m6cv};};
\node[stars] at (axis cs:{90.252},{27.572}) {\tikz\pgfuseplotmark{m6b};};
\node[stars] at (axis cs:{90.292},{31.034}) {\tikz\pgfuseplotmark{m6c};};
\node[stars] at (axis cs:{90.305},{-25.418}) {\tikz\pgfuseplotmark{m6b};};
\node[stars] at (axis cs:{90.318},{-33.912}) {\tikz\pgfuseplotmark{m6a};};
\node[stars] at (axis cs:{90.423},{22.401}) {\tikz\pgfuseplotmark{m6c};};
\node[stars] at (axis cs:{90.460},{-10.598}) {\tikz\pgfuseplotmark{m5bb};};
\node[stars] at (axis cs:{90.596},{9.647}) {\tikz\pgfuseplotmark{m4bv};};
\node[stars] at (axis cs:{90.641},{-14.497}) {\tikz\pgfuseplotmark{m6c};};
\node[stars] at (axis cs:{90.730},{32.636}) {\tikz\pgfuseplotmark{m6c};};
\node[stars] at (axis cs:{90.815},{-26.285}) {\tikz\pgfuseplotmark{m5b};};
\node[stars] at (axis cs:{90.853},{11.681}) {\tikz\pgfuseplotmark{m6b};};
\node[stars] at (axis cs:{90.864},{19.690}) {\tikz\pgfuseplotmark{m5cv};};
\node[stars] at (axis cs:{90.980},{20.138}) {\tikz\pgfuseplotmark{m5av};};
\node[stars] at (axis cs:{91.030},{23.263}) {\tikz\pgfuseplotmark{m4c};};
\node[stars] at (axis cs:{91.056},{-6.709}) {\tikz\pgfuseplotmark{m5cv};};
\node[stars] at (axis cs:{91.084},{-32.172}) {\tikz\pgfuseplotmark{m6av};};
\node[stars] at (axis cs:{91.242},{5.420}) {\tikz\pgfuseplotmark{m6a};};
\node[stars] at (axis cs:{91.243},{4.159}) {\tikz\pgfuseplotmark{m6a};};
\node[stars] at (axis cs:{91.246},{-16.484}) {\tikz\pgfuseplotmark{m5bv};};
\node[stars] at (axis cs:{91.261},{37.964}) {\tikz\pgfuseplotmark{m6c};};
\node[stars] at (axis cs:{91.363},{-10.242}) {\tikz\pgfuseplotmark{m6bv};};
\node[stars] at (axis cs:{91.363},{-35.513}) {\tikz\pgfuseplotmark{m6a};};
\node[stars] at (axis cs:{91.391},{33.599}) {\tikz\pgfuseplotmark{m6c};};
\node[stars] at (axis cs:{91.523},{-29.758}) {\tikz\pgfuseplotmark{m6a};};
\node[stars] at (axis cs:{91.536},{35.388}) {\tikz\pgfuseplotmark{m6bb};};
\node[stars] at (axis cs:{91.539},{-14.935}) {\tikz\pgfuseplotmark{m5a};};
\node[stars] at (axis cs:{91.594},{29.512}) {\tikz\pgfuseplotmark{m6bv};};
\node[stars] at (axis cs:{91.634},{-23.111}) {\tikz\pgfuseplotmark{m5c};};
\node[stars] at (axis cs:{91.646},{38.482}) {\tikz\pgfuseplotmark{m5cv};};
\node[stars] at (axis cs:{91.661},{-4.194}) {\tikz\pgfuseplotmark{m5c};};
\node[stars] at (axis cs:{91.740},{-21.812}) {\tikz\pgfuseplotmark{m6av};};
\node[stars] at (axis cs:{91.765},{-34.312}) {\tikz\pgfuseplotmark{m6av};};
\node[stars] at (axis cs:{91.882},{-37.253}) {\tikz\pgfuseplotmark{m5b};};
\node[stars] at (axis cs:{91.893},{14.768}) {\tikz\pgfuseplotmark{m4c};};
\node[stars] at (axis cs:{91.923},{-19.166}) {\tikz\pgfuseplotmark{m5cv};};
\node[stars] at (axis cs:{92.199},{-6.821}) {\tikz\pgfuseplotmark{m6c};};
\node[stars] at (axis cs:{92.241},{-22.427}) {\tikz\pgfuseplotmark{m5c};};
\node[stars] at (axis cs:{92.241},{2.499}) {\tikz\pgfuseplotmark{m6ab};};
\node[stars] at (axis cs:{92.334},{-18.126}) {\tikz\pgfuseplotmark{m6c};};
\node[stars] at (axis cs:{92.385},{22.190}) {\tikz\pgfuseplotmark{m6bv};};
\node[stars] at (axis cs:{92.394},{-14.585}) {\tikz\pgfuseplotmark{m6a};};
\node[stars] at (axis cs:{92.401},{-5.711}) {\tikz\pgfuseplotmark{m6c};};
\node[stars] at (axis cs:{92.433},{23.113}) {\tikz\pgfuseplotmark{m6av};};
\node[stars] at (axis cs:{92.446},{-26.701}) {\tikz\pgfuseplotmark{m6c};};
\node[stars] at (axis cs:{92.450},{-22.774}) {\tikz\pgfuseplotmark{m6a};};
\node[stars] at (axis cs:{92.645},{-27.154}) {\tikz\pgfuseplotmark{m6a};};
\node[stars] at (axis cs:{92.755},{-6.754}) {\tikz\pgfuseplotmark{m6cv};};
\node[stars] at (axis cs:{92.757},{18.130}) {\tikz\pgfuseplotmark{m6cv};};
\node[stars] at (axis cs:{92.807},{-26.482}) {\tikz\pgfuseplotmark{m6b};};
\node[stars] at (axis cs:{92.866},{13.638}) {\tikz\pgfuseplotmark{m6b};};
\node[stars] at (axis cs:{92.885},{24.420}) {\tikz\pgfuseplotmark{m6a};};
\node[stars] at (axis cs:{92.932},{-4.665}) {\tikz\pgfuseplotmark{m6cv};};
\node[stars] at (axis cs:{92.966},{-6.550}) {\tikz\pgfuseplotmark{m5b};};
\node[stars] at (axis cs:{92.985},{14.209}) {\tikz\pgfuseplotmark{m4cb};};
\node[stars] at (axis cs:{93.006},{19.790}) {\tikz\pgfuseplotmark{m6ab};};
\node[stars] at (axis cs:{93.014},{16.130}) {\tikz\pgfuseplotmark{m5b};};
\node[stars] at (axis cs:{93.080},{22.908}) {\tikz\pgfuseplotmark{m6cv};};
\node[stars] at (axis cs:{93.084},{32.693}) {\tikz\pgfuseplotmark{m6a};};
\node[stars] at (axis cs:{93.248},{6.016}) {\tikz\pgfuseplotmark{m6cv};};
\node[stars] at (axis cs:{93.302},{10.627}) {\tikz\pgfuseplotmark{m6cv};};
\node[stars] at (axis cs:{93.439},{-23.862}) {\tikz\pgfuseplotmark{m6c};};
\node[stars] at (axis cs:{93.476},{-3.741}) {\tikz\pgfuseplotmark{m6a};};
\node[stars] at (axis cs:{93.619},{17.906}) {\tikz\pgfuseplotmark{m6b};};
\node[stars] at (axis cs:{93.653},{-4.568}) {\tikz\pgfuseplotmark{m6a};};
\node[stars] at (axis cs:{93.712},{19.156}) {\tikz\pgfuseplotmark{m5cb};};
\node[stars] at (axis cs:{93.714},{-6.275}) {\tikz\pgfuseplotmark{m4b};};
\node[stars] at (axis cs:{93.719},{22.507}) {\tikz\pgfuseplotmark{m3cvb};};
\node[stars] at (axis cs:{93.785},{-20.272}) {\tikz\pgfuseplotmark{m6b};};
\node[stars] at (axis cs:{93.785},{13.851}) {\tikz\pgfuseplotmark{m6bv};};
\node[stars] at (axis cs:{93.824},{-18.477}) {\tikz\pgfuseplotmark{m6b};};
\node[stars] at (axis cs:{93.845},{29.498}) {\tikz\pgfuseplotmark{m4c};};
\node[stars] at (axis cs:{93.855},{16.143}) {\tikz\pgfuseplotmark{m5c};};
\node[stars] at (axis cs:{93.859},{-9.036}) {\tikz\pgfuseplotmark{m6b};};
\node[stars] at (axis cs:{93.874},{-4.914}) {\tikz\pgfuseplotmark{m6bb};};
\node[stars] at (axis cs:{93.893},{-0.512}) {\tikz\pgfuseplotmark{m6a};};
\node[stars] at (axis cs:{93.917},{6.066}) {\tikz\pgfuseplotmark{m6b};};
\node[stars] at (axis cs:{93.937},{-13.718}) {\tikz\pgfuseplotmark{m5b};};
\node[stars] at (axis cs:{93.937},{12.551}) {\tikz\pgfuseplotmark{m5c};};
\node[stars] at (axis cs:{93.975},{1.169}) {\tikz\pgfuseplotmark{m6c};};
\node[stars] at (axis cs:{94.032},{-16.618}) {\tikz\pgfuseplotmark{m6bv};};
\node[stars] at (axis cs:{94.079},{23.970}) {\tikz\pgfuseplotmark{m6bv};};
\node[stars] at (axis cs:{94.099},{17.181}) {\tikz\pgfuseplotmark{m6c};};
\node[stars] at (axis cs:{94.111},{12.272}) {\tikz\pgfuseplotmark{m5bb};};
\node[stars] at (axis cs:{94.138},{-35.140}) {\tikz\pgfuseplotmark{m4cv};};
\node[stars] at (axis cs:{94.148},{-39.264}) {\tikz\pgfuseplotmark{m6b};};
\node[stars] at (axis cs:{94.245},{23.741}) {\tikz\pgfuseplotmark{m6cv};};
\node[stars] at (axis cs:{94.255},{-37.737}) {\tikz\pgfuseplotmark{m6a};};
\node[stars] at (axis cs:{94.265},{-22.715}) {\tikz\pgfuseplotmark{m6b};};
\node[stars] at (axis cs:{94.278},{9.942}) {\tikz\pgfuseplotmark{m5cb};};
\node[stars] at (axis cs:{94.290},{-37.253}) {\tikz\pgfuseplotmark{m6bv};};
\node[stars] at (axis cs:{94.317},{5.100}) {\tikz\pgfuseplotmark{m6a};};
\node[stars] at (axis cs:{94.424},{-16.816}) {\tikz\pgfuseplotmark{m5c};};
\node[stars] at (axis cs:{94.523},{14.383}) {\tikz\pgfuseplotmark{m6b};};
\node[stars] at (axis cs:{94.557},{-19.967}) {\tikz\pgfuseplotmark{m6a};};
\node[stars] at (axis cs:{94.668},{9.047}) {\tikz\pgfuseplotmark{m6c};};
\node[stars] at (axis cs:{94.703},{-15.025}) {\tikz\pgfuseplotmark{m6bv};};
\node[stars] at (axis cs:{94.711},{-9.390}) {\tikz\pgfuseplotmark{m5c};};
\node[stars] at (axis cs:{94.746},{-20.925}) {\tikz\pgfuseplotmark{m6av};};
\node[stars] at (axis cs:{94.758},{17.325}) {\tikz\pgfuseplotmark{m6cv};};
\node[stars] at (axis cs:{94.783},{-8.586}) {\tikz\pgfuseplotmark{m6c};};
\node[stars] at (axis cs:{94.909},{-22.103}) {\tikz\pgfuseplotmark{m6c};};
\node[stars] at (axis cs:{94.921},{-34.396}) {\tikz\pgfuseplotmark{m6a};};
\node[stars] at (axis cs:{94.928},{-7.823}) {\tikz\pgfuseplotmark{m5c};};
\node[stars] at (axis cs:{94.998},{-2.944}) {\tikz\pgfuseplotmark{m5b};};
\node[stars] at (axis cs:{95.018},{14.651}) {\tikz\pgfuseplotmark{m6av};};
\node[stars] at (axis cs:{95.078},{-30.063}) {\tikz\pgfuseplotmark{m3bv};};
\node[stars] at (axis cs:{95.146},{-23.638}) {\tikz\pgfuseplotmark{m6c};};
\node[stars] at (axis cs:{95.151},{-34.144}) {\tikz\pgfuseplotmark{m6av};};
\node[stars] at (axis cs:{95.273},{-11.817}) {\tikz\pgfuseplotmark{m6c};};
\node[stars] at (axis cs:{95.300},{29.541}) {\tikz\pgfuseplotmark{m6c};};
\node[stars] at (axis cs:{95.353},{-11.773}) {\tikz\pgfuseplotmark{m6av};};
\node[stars] at (axis cs:{95.357},{2.269}) {\tikz\pgfuseplotmark{m6cv};};
\node[stars] at (axis cs:{95.358},{17.763}) {\tikz\pgfuseplotmark{m6c};};
\node[stars] at (axis cs:{95.528},{-33.436}) {\tikz\pgfuseplotmark{m4a};};
\node[stars] at (axis cs:{95.652},{12.570}) {\tikz\pgfuseplotmark{m6b};};
\node[stars] at (axis cs:{95.675},{-17.956}) {\tikz\pgfuseplotmark{m2bv};};
\node[stars] at (axis cs:{95.740},{22.513}) {\tikz\pgfuseplotmark{m3bv};};
\node[stars] at (axis cs:{95.810},{-31.790}) {\tikz\pgfuseplotmark{m6c};};
\node[stars] at (axis cs:{95.827},{3.764}) {\tikz\pgfuseplotmark{m6c};};
\node[stars] at (axis cs:{95.900},{-9.875}) {\tikz\pgfuseplotmark{m6cv};};
\node[stars] at (axis cs:{95.942},{4.593}) {\tikz\pgfuseplotmark{m4cb};};
\node[stars] at (axis cs:{95.942},{-15.071}) {\tikz\pgfuseplotmark{m6c};};
\node[stars] at (axis cs:{95.983},{-25.577}) {\tikz\pgfuseplotmark{m6a};};
\node[stars] at (axis cs:{96.004},{-36.708}) {\tikz\pgfuseplotmark{m6ab};};
\node[stars] at (axis cs:{96.010},{8.885}) {\tikz\pgfuseplotmark{m6c};};
\node[stars] at (axis cs:{96.043},{-11.530}) {\tikz\pgfuseplotmark{m5c};};
\node[stars] at (axis cs:{96.086},{-12.962}) {\tikz\pgfuseplotmark{m6bv};};
\node[stars] at (axis cs:{96.182},{25.048}) {\tikz\pgfuseplotmark{m6c};};
\node[stars] at (axis cs:{96.183},{-28.780}) {\tikz\pgfuseplotmark{m6c};};
\node[stars] at (axis cs:{96.220},{16.057}) {\tikz\pgfuseplotmark{m6c};};
\node[stars] at (axis cs:{96.304},{7.086}) {\tikz\pgfuseplotmark{m6cv};};
\node[stars] at (axis cs:{96.319},{-0.946}) {\tikz\pgfuseplotmark{m6b};};
\node[stars] at (axis cs:{96.367},{14.722}) {\tikz\pgfuseplotmark{m6cv};};
\node[stars] at (axis cs:{96.375},{-35.064}) {\tikz\pgfuseplotmark{m6cb};};
\node[stars] at (axis cs:{96.387},{23.327}) {\tikz\pgfuseplotmark{m6b};};
\node[stars] at (axis cs:{96.446},{-3.889}) {\tikz\pgfuseplotmark{m6c};};
\node[stars] at (axis cs:{96.468},{11.126}) {\tikz\pgfuseplotmark{m6c};};
\node[stars] at (axis cs:{96.495},{-7.895}) {\tikz\pgfuseplotmark{m6c};};
\node[stars] at (axis cs:{96.540},{-3.515}) {\tikz\pgfuseplotmark{m6c};};
\node[stars] at (axis cs:{96.644},{-4.597}) {\tikz\pgfuseplotmark{m6c};};
\node[stars] at (axis cs:{96.665},{-1.507}) {\tikz\pgfuseplotmark{m6b};};
\node[stars] at (axis cs:{96.678},{-14.603}) {\tikz\pgfuseplotmark{m6c};};
\node[stars] at (axis cs:{96.687},{-7.512}) {\tikz\pgfuseplotmark{m6cb};};
\node[stars] at (axis cs:{96.782},{-37.895}) {\tikz\pgfuseplotmark{m6c};};
\node[stars] at (axis cs:{96.797},{-25.856}) {\tikz\pgfuseplotmark{m6b};};
\node[stars] at (axis cs:{96.807},{0.299}) {\tikz\pgfuseplotmark{m5c};};
\node[stars] at (axis cs:{96.815},{-0.276}) {\tikz\pgfuseplotmark{m6a};};
\node[stars] at (axis cs:{96.835},{2.908}) {\tikz\pgfuseplotmark{m6a};};
\node[stars] at (axis cs:{96.898},{32.563}) {\tikz\pgfuseplotmark{m6c};};
\node[stars] at (axis cs:{96.986},{20.496}) {\tikz\pgfuseplotmark{m6c};};
\node[stars] at (axis cs:{96.990},{-4.762}) {\tikz\pgfuseplotmark{m5b};};
\node[stars] at (axis cs:{97.043},{-32.580}) {\tikz\pgfuseplotmark{m4c};};
\node[stars] at (axis cs:{97.070},{1.912}) {\tikz\pgfuseplotmark{m6c};};
\node[stars] at (axis cs:{97.078},{10.304}) {\tikz\pgfuseplotmark{m6c};};
\node[stars] at (axis cs:{97.117},{16.238}) {\tikz\pgfuseplotmark{m6c};};
\node[stars] at (axis cs:{97.142},{30.493}) {\tikz\pgfuseplotmark{m6av};};
\node[stars] at (axis cs:{97.156},{-17.466}) {\tikz\pgfuseplotmark{m6a};};
\node[stars] at (axis cs:{97.164},{-32.371}) {\tikz\pgfuseplotmark{m6avb};};
\node[stars] at (axis cs:{97.204},{-7.033}) {\tikz\pgfuseplotmark{m5avb};};
\node[stars] at (axis cs:{97.241},{20.212}) {\tikz\pgfuseplotmark{m4cb};};
\node[stars] at (axis cs:{97.312},{2.646}) {\tikz\pgfuseplotmark{m6c};};
\node[stars] at (axis cs:{97.547},{-10.081}) {\tikz\pgfuseplotmark{m6b};};
\node[stars] at (axis cs:{97.549},{-19.215}) {\tikz\pgfuseplotmark{m6c};};
\node[stars] at (axis cs:{97.645},{-13.148}) {\tikz\pgfuseplotmark{m6cv};};
\node[stars] at (axis cs:{97.693},{-27.770}) {\tikz\pgfuseplotmark{m6b};};
\node[stars] at (axis cs:{97.790},{11.251}) {\tikz\pgfuseplotmark{m6cb};};
\node[stars] at (axis cs:{97.792},{16.938}) {\tikz\pgfuseplotmark{m6cb};};
\node[stars] at (axis cs:{97.805},{-35.259}) {\tikz\pgfuseplotmark{m6a};};
\node[stars] at (axis cs:{97.846},{-12.392}) {\tikz\pgfuseplotmark{m5c};};
\node[stars] at (axis cs:{97.850},{-32.869}) {\tikz\pgfuseplotmark{m6cv};};
\node[stars] at (axis cs:{97.896},{-36.940}) {\tikz\pgfuseplotmark{m6cv};};
\node[stars] at (axis cs:{97.906},{15.903}) {\tikz\pgfuseplotmark{m6c};};
\node[stars] at (axis cs:{97.951},{11.544}) {\tikz\pgfuseplotmark{m5c};};
\node[stars] at (axis cs:{97.959},{-8.158}) {\tikz\pgfuseplotmark{m5c};};
\node[stars] at (axis cs:{97.964},{-23.418}) {\tikz\pgfuseplotmark{m4cv};};
\node[stars] at (axis cs:{98.077},{17.784}) {\tikz\pgfuseplotmark{m6cb};};
\node[stars] at (axis cs:{98.080},{4.856}) {\tikz\pgfuseplotmark{m6av};};
\node[stars] at (axis cs:{98.089},{-37.696}) {\tikz\pgfuseplotmark{m5c};};
\node[stars] at (axis cs:{98.096},{-5.869}) {\tikz\pgfuseplotmark{m6a};};
\node[stars] at (axis cs:{98.097},{11.673}) {\tikz\pgfuseplotmark{m6bb};};
\node[stars] at (axis cs:{98.113},{32.455}) {\tikz\pgfuseplotmark{m6av};};
\node[stars] at (axis cs:{98.162},{-32.030}) {\tikz\pgfuseplotmark{m6ab};};
\node[stars] at (axis cs:{98.195},{-11.166}) {\tikz\pgfuseplotmark{m6cv};};
\node[stars] at (axis cs:{98.226},{7.333}) {\tikz\pgfuseplotmark{m5av};};
\node[stars] at (axis cs:{98.293},{-38.625}) {\tikz\pgfuseplotmark{m6c};};
\node[stars] at (axis cs:{98.361},{-20.924}) {\tikz\pgfuseplotmark{m6c};};
\node[stars] at (axis cs:{98.401},{14.155}) {\tikz\pgfuseplotmark{m6a};};
\node[stars] at (axis cs:{98.408},{-1.220}) {\tikz\pgfuseplotmark{m5b};};
\node[stars] at (axis cs:{98.456},{-36.232}) {\tikz\pgfuseplotmark{m5c};};
\node[stars] at (axis cs:{98.647},{-32.716}) {\tikz\pgfuseplotmark{m6a};};
\node[stars] at (axis cs:{98.693},{7.572}) {\tikz\pgfuseplotmark{m6cv};};
\node[stars] at (axis cs:{98.764},{-22.965}) {\tikz\pgfuseplotmark{m5a};};
\node[stars] at (axis cs:{98.800},{28.022}) {\tikz\pgfuseplotmark{m5cv};};
\node[stars] at (axis cs:{98.816},{0.890}) {\tikz\pgfuseplotmark{m6a};};
\node[stars] at (axis cs:{98.823},{9.988}) {\tikz\pgfuseplotmark{m6b};};
\node[stars] at (axis cs:{98.851},{-36.780}) {\tikz\pgfuseplotmark{m6ab};};
\node[stars] at (axis cs:{98.878},{-22.109}) {\tikz\pgfuseplotmark{m6c};};
\node[stars] at (axis cs:{98.975},{-36.089}) {\tikz\pgfuseplotmark{m6cb};};
\node[stars] at (axis cs:{99.095},{-18.660}) {\tikz\pgfuseplotmark{m6ab};};
\node[stars] at (axis cs:{99.137},{38.445}) {\tikz\pgfuseplotmark{m5cv};};
\node[stars] at (axis cs:{99.147},{-5.211}) {\tikz\pgfuseplotmark{m6a};};
\node[stars] at (axis cs:{99.171},{-19.256}) {\tikz\pgfuseplotmark{m4bv};};
\node[stars] at (axis cs:{99.171},{-22.614}) {\tikz\pgfuseplotmark{m6cvb};};
\node[stars] at (axis cs:{99.194},{-13.321}) {\tikz\pgfuseplotmark{m6bv};};
\node[stars] at (axis cs:{99.258},{-38.146}) {\tikz\pgfuseplotmark{m6b};};
\node[stars] at (axis cs:{99.308},{-36.991}) {\tikz\pgfuseplotmark{m6a};};
\node[stars] at (axis cs:{99.350},{6.135}) {\tikz\pgfuseplotmark{m6bv};};
\node[stars] at (axis cs:{99.364},{24.591}) {\tikz\pgfuseplotmark{m6c};};
\node[stars] at (axis cs:{99.404},{10.853}) {\tikz\pgfuseplotmark{m6c};};
\node[stars] at (axis cs:{99.418},{2.704}) {\tikz\pgfuseplotmark{m6c};};
\node[stars] at (axis cs:{99.420},{-12.985}) {\tikz\pgfuseplotmark{m6b};};
\node[stars] at (axis cs:{99.428},{16.399}) {\tikz\pgfuseplotmark{m2b};};
\node[stars] at (axis cs:{99.448},{-32.340}) {\tikz\pgfuseplotmark{m5c};};
\node[stars] at (axis cs:{99.470},{4.957}) {\tikz\pgfuseplotmark{m6cv};};
\node[stars] at (axis cs:{99.473},{-18.237}) {\tikz\pgfuseplotmark{m4c};};
\node[stars] at (axis cs:{99.585},{-2.544}) {\tikz\pgfuseplotmark{m6b};};
\node[stars] at (axis cs:{99.595},{-23.580}) {\tikz\pgfuseplotmark{m6c};};
\node[stars] at (axis cs:{99.596},{28.984}) {\tikz\pgfuseplotmark{m6a};};
\node[stars] at (axis cs:{99.648},{-16.873}) {\tikz\pgfuseplotmark{m6b};};
\node[stars] at (axis cs:{99.659},{1.613}) {\tikz\pgfuseplotmark{m6cv};};
\node[stars] at (axis cs:{99.665},{39.391}) {\tikz\pgfuseplotmark{m6a};};
\node[stars] at (axis cs:{99.705},{39.903}) {\tikz\pgfuseplotmark{m5c};};
\node[stars] at (axis cs:{99.772},{22.031}) {\tikz\pgfuseplotmark{m6b};};
\node[stars] at (axis cs:{99.820},{-14.146}) {\tikz\pgfuseplotmark{m5a};};
\node[stars] at (axis cs:{99.881},{24.600}) {\tikz\pgfuseplotmark{m6c};};
\node[stars] at (axis cs:{99.888},{28.263}) {\tikz\pgfuseplotmark{m6bv};};
\node[stars] at (axis cs:{99.901},{-23.695}) {\tikz\pgfuseplotmark{m6b};};
\node[stars] at (axis cs:{99.928},{-30.470}) {\tikz\pgfuseplotmark{m6a};};
\node[stars] at (axis cs:{99.949},{12.983}) {\tikz\pgfuseplotmark{m6b};};
\node[stars] at (axis cs:{100.244},{9.896}) {\tikz\pgfuseplotmark{m5avb};};
\node[stars] at (axis cs:{100.273},{0.495}) {\tikz\pgfuseplotmark{m6a};};
\node[stars] at (axis cs:{100.322},{11.003}) {\tikz\pgfuseplotmark{m6cv};};
\node[stars] at (axis cs:{100.337},{28.196}) {\tikz\pgfuseplotmark{m6c};};
\node[stars] at (axis cs:{100.341},{16.397}) {\tikz\pgfuseplotmark{m6c};};
\node[stars] at (axis cs:{100.407},{35.932}) {\tikz\pgfuseplotmark{m6c};};
\node[stars] at (axis cs:{100.485},{-9.168}) {\tikz\pgfuseplotmark{m5c};};
\node[stars] at (axis cs:{100.497},{6.345}) {\tikz\pgfuseplotmark{m6c};};
\node[stars] at (axis cs:{100.601},{17.645}) {\tikz\pgfuseplotmark{m5c};};
\node[stars] at (axis cs:{100.692},{-22.448}) {\tikz\pgfuseplotmark{m6cvb};};
\node[stars] at (axis cs:{100.778},{3.034}) {\tikz\pgfuseplotmark{m6c};};
\node[stars] at (axis cs:{100.807},{37.147}) {\tikz\pgfuseplotmark{m6c};};
\node[stars] at (axis cs:{100.847},{-39.193}) {\tikz\pgfuseplotmark{m6c};};
\node[stars] at (axis cs:{100.911},{3.932}) {\tikz\pgfuseplotmark{m6bv};};
\node[stars] at (axis cs:{100.983},{25.131}) {\tikz\pgfuseplotmark{m3bv};};
\node[stars] at (axis cs:{100.997},{13.228}) {\tikz\pgfuseplotmark{m4c};};
\node[stars] at (axis cs:{101.052},{36.109}) {\tikz\pgfuseplotmark{m6c};};
\node[stars] at (axis cs:{101.119},{-31.070}) {\tikz\pgfuseplotmark{m5cv};};
\node[stars] at (axis cs:{101.189},{28.971}) {\tikz\pgfuseplotmark{m5c};};
\node[stars] at (axis cs:{101.216},{-27.342}) {\tikz\pgfuseplotmark{m6c};};
\node[stars] at (axis cs:{101.287},{-16.716}) {\tikz\pgfuseplotmark{m1b};};
\node[stars] at (axis cs:{101.322},{12.895}) {\tikz\pgfuseplotmark{m3cv};};
\node[stars] at (axis cs:{101.346},{-31.793}) {\tikz\pgfuseplotmark{m6b};};
\node[stars] at (axis cs:{101.347},{-23.462}) {\tikz\pgfuseplotmark{m6b};};
\node[stars] at (axis cs:{101.380},{-30.949}) {\tikz\pgfuseplotmark{m6avb};};
\node[stars] at (axis cs:{101.476},{12.693}) {\tikz\pgfuseplotmark{m6c};};
\node[stars] at (axis cs:{101.497},{-14.796}) {\tikz\pgfuseplotmark{m5c};};
\node[stars] at (axis cs:{101.551},{-37.775}) {\tikz\pgfuseplotmark{m6cv};};
\node[stars] at (axis cs:{101.635},{8.587}) {\tikz\pgfuseplotmark{m6b};};
\node[stars] at (axis cs:{101.663},{-10.107}) {\tikz\pgfuseplotmark{m6a};};
\node[stars] at (axis cs:{101.713},{-14.426}) {\tikz\pgfuseplotmark{m5c};};
\node[stars] at (axis cs:{101.756},{-21.015}) {\tikz\pgfuseplotmark{m6bv};};
\node[stars] at (axis cs:{101.833},{8.037}) {\tikz\pgfuseplotmark{m5a};};
\node[stars] at (axis cs:{101.839},{-37.929}) {\tikz\pgfuseplotmark{m5c};};
\node[stars] at (axis cs:{101.848},{18.193}) {\tikz\pgfuseplotmark{m6c};};
\node[stars] at (axis cs:{101.905},{-8.998}) {\tikz\pgfuseplotmark{m5bv};};
\node[stars] at (axis cs:{101.965},{2.412}) {\tikz\pgfuseplotmark{m4c};};
\node[stars] at (axis cs:{102.079},{-1.319}) {\tikz\pgfuseplotmark{m6a};};
\node[stars] at (axis cs:{102.241},{-15.145}) {\tikz\pgfuseplotmark{m5cv};};
\node[stars] at (axis cs:{102.265},{1.002}) {\tikz\pgfuseplotmark{m6cv};};
\node[stars] at (axis cs:{102.318},{-2.272}) {\tikz\pgfuseplotmark{m6a};};
\node[stars] at (axis cs:{102.422},{32.607}) {\tikz\pgfuseplotmark{m6av};};
\node[stars] at (axis cs:{102.458},{16.203}) {\tikz\pgfuseplotmark{m6bv};};
\node[stars] at (axis cs:{102.460},{-32.508}) {\tikz\pgfuseplotmark{m4av};};
\node[stars] at (axis cs:{102.591},{-17.085}) {\tikz\pgfuseplotmark{m6a};};
\node[stars] at (axis cs:{102.597},{-31.706}) {\tikz\pgfuseplotmark{m6avb};};
\node[stars] at (axis cs:{102.606},{13.413}) {\tikz\pgfuseplotmark{m6a};};
\node[stars] at (axis cs:{102.654},{-25.778}) {\tikz\pgfuseplotmark{m6c};};
\node[stars] at (axis cs:{102.676},{-8.041}) {\tikz\pgfuseplotmark{m6cv};};
\node[stars] at (axis cs:{102.708},{-0.541}) {\tikz\pgfuseplotmark{m6a};};
\node[stars] at (axis cs:{102.718},{-34.367}) {\tikz\pgfuseplotmark{m5b};};
\node[stars] at (axis cs:{102.888},{21.761}) {\tikz\pgfuseplotmark{m5c};};
\node[stars] at (axis cs:{102.914},{3.042}) {\tikz\pgfuseplotmark{m6c};};
\node[stars] at (axis cs:{102.927},{-36.230}) {\tikz\pgfuseplotmark{m6b};};
\node[stars] at (axis cs:{103.000},{23.602}) {\tikz\pgfuseplotmark{m6a};};
\node[stars] at (axis cs:{103.095},{-5.316}) {\tikz\pgfuseplotmark{m6c};};
\node[stars] at (axis cs:{103.197},{33.961}) {\tikz\pgfuseplotmark{m4a};};
\node[stars] at (axis cs:{103.206},{8.380}) {\tikz\pgfuseplotmark{m6a};};
\node[stars] at (axis cs:{103.251},{-26.958}) {\tikz\pgfuseplotmark{m6cv};};
\node[stars] at (axis cs:{103.256},{38.869}) {\tikz\pgfuseplotmark{m6bvb};};
\node[stars] at (axis cs:{103.265},{35.788}) {\tikz\pgfuseplotmark{m6b};};
\node[stars] at (axis cs:{103.306},{38.438}) {\tikz\pgfuseplotmark{m6c};};
\node[stars] at (axis cs:{103.328},{-19.032}) {\tikz\pgfuseplotmark{m6a};};
\node[stars] at (axis cs:{103.341},{-18.933}) {\tikz\pgfuseplotmark{m6b};};
\node[stars] at (axis cs:{103.343},{10.996}) {\tikz\pgfuseplotmark{m6c};};
\node[stars] at (axis cs:{103.387},{-20.224}) {\tikz\pgfuseplotmark{m5av};};
\node[stars] at (axis cs:{103.391},{-28.540}) {\tikz\pgfuseplotmark{m6b};};
\node[stars] at (axis cs:{103.481},{-24.539}) {\tikz\pgfuseplotmark{m6c};};
\node[stars] at (axis cs:{103.488},{38.505}) {\tikz\pgfuseplotmark{m6c};};
\node[stars] at (axis cs:{103.533},{-24.184}) {\tikz\pgfuseplotmark{m4av};};
\node[stars] at (axis cs:{103.536},{-5.852}) {\tikz\pgfuseplotmark{m6cb};};
\node[stars] at (axis cs:{103.547},{-12.039}) {\tikz\pgfuseplotmark{m4b};};
\node[stars] at (axis cs:{103.603},{-1.127}) {\tikz\pgfuseplotmark{m5c};};
\node[stars] at (axis cs:{103.661},{13.178}) {\tikz\pgfuseplotmark{m5ab};};
\node[stars] at (axis cs:{103.675},{-1.756}) {\tikz\pgfuseplotmark{m6cv};};
\node[stars] at (axis cs:{103.745},{-2.804}) {\tikz\pgfuseplotmark{m6b};};
\node[stars] at (axis cs:{103.761},{-20.405}) {\tikz\pgfuseplotmark{m6ab};};
\node[stars] at (axis cs:{103.828},{25.376}) {\tikz\pgfuseplotmark{m6av};};
\node[stars] at (axis cs:{103.894},{8.324}) {\tikz\pgfuseplotmark{m6c};};
\node[stars] at (axis cs:{103.906},{-20.136}) {\tikz\pgfuseplotmark{m5av};};
\node[stars] at (axis cs:{103.946},{-22.941}) {\tikz\pgfuseplotmark{m5cv};};
\node[stars] at (axis cs:{103.978},{-31.790}) {\tikz\pgfuseplotmark{m6cv};};
\node[stars] at (axis cs:{104.028},{-14.043}) {\tikz\pgfuseplotmark{m5bb};};
\node[stars] at (axis cs:{104.034},{-17.054}) {\tikz\pgfuseplotmark{m4cv};};
\node[stars] at (axis cs:{104.108},{9.956}) {\tikz\pgfuseplotmark{m6b};};
\node[stars] at (axis cs:{104.190},{-35.341}) {\tikz\pgfuseplotmark{m6c};};
\node[stars] at (axis cs:{104.250},{-8.179}) {\tikz\pgfuseplotmark{m6c};};
\node[stars] at (axis cs:{104.252},{33.681}) {\tikz\pgfuseplotmark{m6b};};
\node[stars] at (axis cs:{104.323},{-35.507}) {\tikz\pgfuseplotmark{m6cv};};
\node[stars] at (axis cs:{104.357},{11.907}) {\tikz\pgfuseplotmark{m6c};};
\node[stars] at (axis cs:{104.391},{-24.631}) {\tikz\pgfuseplotmark{m5c};};
\node[stars] at (axis cs:{104.428},{-27.537}) {\tikz\pgfuseplotmark{m6cv};};
\node[stars] at (axis cs:{104.531},{-27.164}) {\tikz\pgfuseplotmark{m6cv};};
\node[stars] at (axis cs:{104.605},{-34.112}) {\tikz\pgfuseplotmark{m5b};};
\node[stars] at (axis cs:{104.650},{-25.414}) {\tikz\pgfuseplotmark{m6a};};
\node[stars] at (axis cs:{104.656},{-28.972}) {\tikz\pgfuseplotmark{m2a};};
\node[stars] at (axis cs:{104.662},{7.622}) {\tikz\pgfuseplotmark{m6c};};
\node[stars] at (axis cs:{104.682},{-30.998}) {\tikz\pgfuseplotmark{m6cb};};
\node[stars] at (axis cs:{104.698},{26.081}) {\tikz\pgfuseplotmark{m6cv};};
\node[stars] at (axis cs:{104.738},{3.602}) {\tikz\pgfuseplotmark{m6b};};
\node[stars] at (axis cs:{104.762},{38.052}) {\tikz\pgfuseplotmark{m6b};};
\node[stars] at (axis cs:{104.834},{7.317}) {\tikz\pgfuseplotmark{m6c};};
\node[stars] at (axis cs:{104.866},{25.914}) {\tikz\pgfuseplotmark{m6c};};
\node[stars] at (axis cs:{104.914},{-21.603}) {\tikz\pgfuseplotmark{m6c};};
\node[stars] at (axis cs:{105.035},{-20.158}) {\tikz\pgfuseplotmark{m6c};};
\node[stars] at (axis cs:{105.066},{16.079}) {\tikz\pgfuseplotmark{m6av};};
\node[stars] at (axis cs:{105.075},{-5.367}) {\tikz\pgfuseplotmark{m6c};};
\node[stars] at (axis cs:{105.099},{-8.407}) {\tikz\pgfuseplotmark{m6b};};
\node[stars] at (axis cs:{105.177},{-28.489}) {\tikz\pgfuseplotmark{m6c};};
\node[stars] at (axis cs:{105.207},{-33.465}) {\tikz\pgfuseplotmark{m6c};};
\node[stars] at (axis cs:{105.275},{-25.215}) {\tikz\pgfuseplotmark{m6av};};
\node[stars] at (axis cs:{105.430},{-27.935}) {\tikz\pgfuseplotmark{m3cv};};
\node[stars] at (axis cs:{105.470},{-1.346}) {\tikz\pgfuseplotmark{m6cb};};
\node[stars] at (axis cs:{105.485},{-5.722}) {\tikz\pgfuseplotmark{m5cv};};
\node[stars] at (axis cs:{105.573},{15.336}) {\tikz\pgfuseplotmark{m6a};};
\node[stars] at (axis cs:{105.603},{24.215}) {\tikz\pgfuseplotmark{m5cv};};
\node[stars] at (axis cs:{105.606},{17.755}) {\tikz\pgfuseplotmark{m6bv};};
\node[stars] at (axis cs:{105.639},{16.674}) {\tikz\pgfuseplotmark{m6av};};
\node[stars] at (axis cs:{105.728},{-4.239}) {\tikz\pgfuseplotmark{m5bv};};
\node[stars] at (axis cs:{105.756},{-23.833}) {\tikz\pgfuseplotmark{m3bv};};
\node[stars] at (axis cs:{105.825},{9.138}) {\tikz\pgfuseplotmark{m6b};};
\node[stars] at (axis cs:{105.877},{29.337}) {\tikz\pgfuseplotmark{m6b};};
\node[stars] at (axis cs:{105.909},{10.952}) {\tikz\pgfuseplotmark{m5b};};
\node[stars] at (axis cs:{105.940},{-15.633}) {\tikz\pgfuseplotmark{m4b};};
\node[stars] at (axis cs:{105.965},{12.594}) {\tikz\pgfuseplotmark{m6b};};
\node[stars] at (axis cs:{105.989},{-10.124}) {\tikz\pgfuseplotmark{m6c};};
\node[stars] at (axis cs:{106.022},{-5.324}) {\tikz\pgfuseplotmark{m6a};};
\node[stars] at (axis cs:{106.027},{20.570}) {\tikz\pgfuseplotmark{m4bvb};};
\node[stars] at (axis cs:{106.196},{-22.031}) {\tikz\pgfuseplotmark{m6b};};
\node[stars] at (axis cs:{106.327},{22.637}) {\tikz\pgfuseplotmark{m6b};};
\node[stars] at (axis cs:{106.383},{-34.778}) {\tikz\pgfuseplotmark{m6cb};};
\node[stars] at (axis cs:{106.413},{9.186}) {\tikz\pgfuseplotmark{m6a};};
\node[stars] at (axis cs:{106.502},{-30.656}) {\tikz\pgfuseplotmark{m6cv};};
\node[stars] at (axis cs:{106.509},{-38.383}) {\tikz\pgfuseplotmark{m6b};};
\node[stars] at (axis cs:{106.548},{34.474}) {\tikz\pgfuseplotmark{m6a};};
\node[stars] at (axis cs:{106.650},{-12.394}) {\tikz\pgfuseplotmark{m6c};};
\node[stars] at (axis cs:{106.670},{-11.294}) {\tikz\pgfuseplotmark{m5cvb};};
\node[stars] at (axis cs:{106.719},{-24.960}) {\tikz\pgfuseplotmark{m6b};};
\node[stars] at (axis cs:{106.777},{4.910}) {\tikz\pgfuseplotmark{m6b};};
\node[stars] at (axis cs:{106.843},{34.009}) {\tikz\pgfuseplotmark{m6b};};
\node[stars] at (axis cs:{106.844},{-23.841}) {\tikz\pgfuseplotmark{m6av};};
\node[stars] at (axis cs:{106.854},{28.177}) {\tikz\pgfuseplotmark{m6c};};
\node[stars] at (axis cs:{106.956},{7.471}) {\tikz\pgfuseplotmark{m6a};};
\node[stars] at (axis cs:{107.055},{33.832}) {\tikz\pgfuseplotmark{m6c};};
\node[stars] at (axis cs:{107.092},{15.930}) {\tikz\pgfuseplotmark{m5c};};
\node[stars] at (axis cs:{107.098},{-26.393}) {\tikz\pgfuseplotmark{m2av};};
\node[stars] at (axis cs:{107.151},{37.445}) {\tikz\pgfuseplotmark{m6c};};
\node[stars] at (axis cs:{107.213},{-39.656}) {\tikz\pgfuseplotmark{m5av};};
\node[stars] at (axis cs:{107.334},{-10.346}) {\tikz\pgfuseplotmark{m6c};};
\node[stars] at (axis cs:{107.389},{-16.234}) {\tikz\pgfuseplotmark{m6bv};};
\node[stars] at (axis cs:{107.429},{-25.231}) {\tikz\pgfuseplotmark{m6av};};
\node[stars] at (axis cs:{107.528},{21.247}) {\tikz\pgfuseplotmark{m6c};};
\node[stars] at (axis cs:{107.539},{-18.686}) {\tikz\pgfuseplotmark{m6c};};
\node[stars] at (axis cs:{107.557},{-4.237}) {\tikz\pgfuseplotmark{m5bb};};
\node[stars] at (axis cs:{107.581},{-27.491}) {\tikz\pgfuseplotmark{m5c};};
\node[stars] at (axis cs:{107.763},{-3.898}) {\tikz\pgfuseplotmark{m6c};};
\node[stars] at (axis cs:{107.785},{30.245}) {\tikz\pgfuseplotmark{m4cv};};
\node[stars] at (axis cs:{107.846},{26.856}) {\tikz\pgfuseplotmark{m6av};};
\node[stars] at (axis cs:{107.848},{-0.302}) {\tikz\pgfuseplotmark{m5cv};};
\node[stars] at (axis cs:{107.914},{39.320}) {\tikz\pgfuseplotmark{m5b};};
\node[stars] at (axis cs:{107.923},{-20.883}) {\tikz\pgfuseplotmark{m6a};};
\node[stars] at (axis cs:{107.964},{5.655}) {\tikz\pgfuseplotmark{m6b};};
\node[stars] at (axis cs:{107.966},{-0.493}) {\tikz\pgfuseplotmark{m4c};};
\node[stars] at (axis cs:{108.017},{-30.821}) {\tikz\pgfuseplotmark{m6b};};
\node[stars] at (axis cs:{108.031},{5.474}) {\tikz\pgfuseplotmark{m6c};};
\node[stars] at (axis cs:{108.051},{-25.942}) {\tikz\pgfuseplotmark{m6bv};};
\node[stars] at (axis cs:{108.108},{-36.544}) {\tikz\pgfuseplotmark{m6bvb};};
\node[stars] at (axis cs:{108.110},{24.128}) {\tikz\pgfuseplotmark{m6a};};
\node[stars] at (axis cs:{108.280},{-11.251}) {\tikz\pgfuseplotmark{m6a};};
\node[stars] at (axis cs:{108.343},{16.159}) {\tikz\pgfuseplotmark{m5bv};};
\node[stars] at (axis cs:{108.350},{-22.674}) {\tikz\pgfuseplotmark{m6bv};};
\node[stars] at (axis cs:{108.402},{-27.356}) {\tikz\pgfuseplotmark{m6bv};};
\node[stars] at (axis cs:{108.452},{-22.906}) {\tikz\pgfuseplotmark{m6cb};};
\node[stars] at (axis cs:{108.489},{-30.340}) {\tikz\pgfuseplotmark{m6c};};
\node[stars] at (axis cs:{108.545},{-3.902}) {\tikz\pgfuseplotmark{m6b};};
\node[stars] at (axis cs:{108.563},{-26.352}) {\tikz\pgfuseplotmark{m4cv};};
\node[stars] at (axis cs:{108.565},{-9.947}) {\tikz\pgfuseplotmark{m6b};};
\node[stars] at (axis cs:{108.584},{3.111}) {\tikz\pgfuseplotmark{m5c};};
\node[stars] at (axis cs:{108.618},{-10.316}) {\tikz\pgfuseplotmark{m6b};};
\node[stars] at (axis cs:{108.636},{12.116}) {\tikz\pgfuseplotmark{m6a};};
\node[stars] at (axis cs:{108.675},{24.885}) {\tikz\pgfuseplotmark{m6av};};
\node[stars] at (axis cs:{108.703},{-26.773}) {\tikz\pgfuseplotmark{m4bv};};
\node[stars] at (axis cs:{108.713},{-27.038}) {\tikz\pgfuseplotmark{m6a};};
\node[stars] at (axis cs:{108.831},{-0.161}) {\tikz\pgfuseplotmark{m6c};};
\node[stars] at (axis cs:{108.838},{-30.686}) {\tikz\pgfuseplotmark{m5cv};};
\node[stars] at (axis cs:{108.914},{7.978}) {\tikz\pgfuseplotmark{m6av};};
\node[stars] at (axis cs:{108.930},{-10.584}) {\tikz\pgfuseplotmark{m6b};};
\node[stars] at (axis cs:{108.947},{-23.740}) {\tikz\pgfuseplotmark{m6c};};
\node[stars] at (axis cs:{108.988},{27.897}) {\tikz\pgfuseplotmark{m6av};};
\node[stars] at (axis cs:{109.061},{-15.586}) {\tikz\pgfuseplotmark{m5cv};};
\node[stars] at (axis cs:{109.133},{-38.319}) {\tikz\pgfuseplotmark{m6a};};
\node[stars] at (axis cs:{109.146},{-27.881}) {\tikz\pgfuseplotmark{m5av};};
\node[stars] at (axis cs:{109.153},{-23.315}) {\tikz\pgfuseplotmark{m5avb};};
\node[stars] at (axis cs:{109.206},{-36.592}) {\tikz\pgfuseplotmark{m5bv};};
\node[stars] at (axis cs:{109.238},{-30.897}) {\tikz\pgfuseplotmark{m6cb};};
\node[stars] at (axis cs:{109.264},{26.689}) {\tikz\pgfuseplotmark{m6c};};
\node[stars] at (axis cs:{109.286},{-37.097}) {\tikz\pgfuseplotmark{m3av};};
\node[stars] at (axis cs:{109.382},{-6.680}) {\tikz\pgfuseplotmark{m6c};};
\node[stars] at (axis cs:{109.428},{38.876}) {\tikz\pgfuseplotmark{m6c};};
\node[stars] at (axis cs:{109.450},{-26.797}) {\tikz\pgfuseplotmark{m6c};};
\node[stars] at (axis cs:{109.517},{30.956}) {\tikz\pgfuseplotmark{m6c};};
\node[stars] at (axis cs:{109.523},{16.540}) {\tikz\pgfuseplotmark{m4av};};
\node[stars] at (axis cs:{109.577},{-36.734}) {\tikz\pgfuseplotmark{m5avb};};
\node[stars] at (axis cs:{109.640},{-39.210}) {\tikz\pgfuseplotmark{m5c};};
\node[stars] at (axis cs:{109.668},{-24.559}) {\tikz\pgfuseplotmark{m5bv};};
\node[stars] at (axis cs:{109.677},{-24.954}) {\tikz\pgfuseplotmark{m4cvb};};
\node[stars] at (axis cs:{109.714},{-26.586}) {\tikz\pgfuseplotmark{m5cv};};
\node[stars] at (axis cs:{109.758},{-19.280}) {\tikz\pgfuseplotmark{m6b};};
\node[stars] at (axis cs:{109.807},{-33.727}) {\tikz\pgfuseplotmark{m6cv};};
\node[stars] at (axis cs:{109.843},{2.741}) {\tikz\pgfuseplotmark{m6b};};
\node[stars] at (axis cs:{109.867},{-16.395}) {\tikz\pgfuseplotmark{m6av};};
\node[stars] at (axis cs:{109.949},{7.143}) {\tikz\pgfuseplotmark{m6b};};
\node[stars] at (axis cs:{110.029},{15.143}) {\tikz\pgfuseplotmark{m6c};};
\node[stars] at (axis cs:{110.031},{21.982}) {\tikz\pgfuseplotmark{m4a};};
\node[stars] at (axis cs:{110.229},{-26.964}) {\tikz\pgfuseplotmark{m6bv};};
\node[stars] at (axis cs:{110.243},{-14.360}) {\tikz\pgfuseplotmark{m6a};};
\node[stars] at (axis cs:{110.268},{-25.892}) {\tikz\pgfuseplotmark{m6bv};};
\node[stars] at (axis cs:{110.487},{20.443}) {\tikz\pgfuseplotmark{m5bvb};};
\node[stars] at (axis cs:{110.509},{-8.979}) {\tikz\pgfuseplotmark{m6c};};
\node[stars] at (axis cs:{110.511},{36.760}) {\tikz\pgfuseplotmark{m5b};};
\node[stars] at (axis cs:{110.514},{0.177}) {\tikz\pgfuseplotmark{m6b};};
\node[stars] at (axis cs:{110.556},{38.996}) {\tikz\pgfuseplotmark{m6c};};
\node[stars] at (axis cs:{110.556},{-19.016}) {\tikz\pgfuseplotmark{m5b};};
\node[stars] at (axis cs:{110.577},{-2.979}) {\tikz\pgfuseplotmark{m6c};};
\node[stars] at (axis cs:{110.606},{-5.983}) {\tikz\pgfuseplotmark{m6av};};
\node[stars] at (axis cs:{110.618},{0.701}) {\tikz\pgfuseplotmark{m6c};};
\node[stars] at (axis cs:{110.753},{-31.924}) {\tikz\pgfuseplotmark{m5cv};};
\node[stars] at (axis cs:{110.867},{22.945}) {\tikz\pgfuseplotmark{m6c};};
\node[stars] at (axis cs:{110.869},{25.051}) {\tikz\pgfuseplotmark{m5b};};
\node[stars] at (axis cs:{110.871},{-27.834}) {\tikz\pgfuseplotmark{m5c};};
\node[stars] at (axis cs:{110.883},{-32.202}) {\tikz\pgfuseplotmark{m5c};};
\node[stars] at (axis cs:{110.993},{-35.838}) {\tikz\pgfuseplotmark{m6c};};
\node[stars] at (axis cs:{111.024},{-29.303}) {\tikz\pgfuseplotmark{m2cv};};
\node[stars] at (axis cs:{111.072},{-22.913}) {\tikz\pgfuseplotmark{m6c};};
\node[stars] at (axis cs:{111.115},{15.517}) {\tikz\pgfuseplotmark{m6cv};};
\node[stars] at (axis cs:{111.139},{27.638}) {\tikz\pgfuseplotmark{m6a};};
\node[stars] at (axis cs:{111.167},{-16.201}) {\tikz\pgfuseplotmark{m5cv};};
\node[stars] at (axis cs:{111.183},{-31.809}) {\tikz\pgfuseplotmark{m5cvb};};
\node[stars] at (axis cs:{111.212},{-19.012}) {\tikz\pgfuseplotmark{m6c};};
\node[stars] at (axis cs:{111.242},{11.669}) {\tikz\pgfuseplotmark{m5c};};
\node[stars] at (axis cs:{111.285},{-13.752}) {\tikz\pgfuseplotmark{m6a};};
\node[stars] at (axis cs:{111.333},{-21.982}) {\tikz\pgfuseplotmark{m6b};};
\node[stars] at (axis cs:{111.355},{-25.218}) {\tikz\pgfuseplotmark{m6a};};
\node[stars] at (axis cs:{111.412},{9.276}) {\tikz\pgfuseplotmark{m5b};};
\node[stars] at (axis cs:{111.429},{-31.739}) {\tikz\pgfuseplotmark{m6cv};};
\node[stars] at (axis cs:{111.432},{27.798}) {\tikz\pgfuseplotmark{m4a};};
\node[stars] at (axis cs:{111.463},{-5.775}) {\tikz\pgfuseplotmark{m6b};};
\node[stars] at (axis cs:{111.616},{10.608}) {\tikz\pgfuseplotmark{m6c};};
\node[stars] at (axis cs:{111.672},{11.009}) {\tikz\pgfuseplotmark{m6c};};
\node[stars] at (axis cs:{111.677},{-34.141}) {\tikz\pgfuseplotmark{m6b};};
\node[stars] at (axis cs:{111.735},{20.257}) {\tikz\pgfuseplotmark{m6bv};};
\node[stars] at (axis cs:{111.748},{-23.086}) {\tikz\pgfuseplotmark{m6av};};
\node[stars] at (axis cs:{111.783},{-17.865}) {\tikz\pgfuseplotmark{m6avb};};
\node[stars] at (axis cs:{111.788},{8.289}) {\tikz\pgfuseplotmark{m3bv};};
\node[stars] at (axis cs:{111.929},{-22.859}) {\tikz\pgfuseplotmark{m6b};};
\node[stars] at (axis cs:{111.935},{21.445}) {\tikz\pgfuseplotmark{m5cb};};
\node[stars] at (axis cs:{111.965},{-11.557}) {\tikz\pgfuseplotmark{m6ab};};
\node[stars] at (axis cs:{111.997},{-29.156}) {\tikz\pgfuseplotmark{m6a};};
\node[stars] at (axis cs:{112.008},{-26.839}) {\tikz\pgfuseplotmark{m6cb};};
\node[stars] at (axis cs:{112.009},{6.942}) {\tikz\pgfuseplotmark{m5c};};
\node[stars] at (axis cs:{112.041},{8.925}) {\tikz\pgfuseplotmark{m4cvb};};
\node[stars] at (axis cs:{112.166},{27.553}) {\tikz\pgfuseplotmark{m6c};};
\node[stars] at (axis cs:{112.197},{15.110}) {\tikz\pgfuseplotmark{m6c};};
\node[stars] at (axis cs:{112.213},{-31.848}) {\tikz\pgfuseplotmark{m6cb};};
\node[stars] at (axis cs:{112.270},{-31.456}) {\tikz\pgfuseplotmark{m6a};};
\node[stars] at (axis cs:{112.274},{-38.812}) {\tikz\pgfuseplotmark{m5c};};
\node[stars] at (axis cs:{112.278},{31.785}) {\tikz\pgfuseplotmark{m4cvb};};
\node[stars] at (axis cs:{112.328},{-1.905}) {\tikz\pgfuseplotmark{m6a};};
\node[stars] at (axis cs:{112.335},{28.118}) {\tikz\pgfuseplotmark{m5b};};
\node[stars] at (axis cs:{112.341},{-14.998}) {\tikz\pgfuseplotmark{m6bb};};
\node[stars] at (axis cs:{112.342},{-10.326}) {\tikz\pgfuseplotmark{m6a};};
\node[stars] at (axis cs:{112.357},{-7.551}) {\tikz\pgfuseplotmark{m6b};};
\node[stars] at (axis cs:{112.449},{12.006}) {\tikz\pgfuseplotmark{m5av};};
\node[stars] at (axis cs:{112.453},{27.916}) {\tikz\pgfuseplotmark{m5b};};
\node[stars] at (axis cs:{112.464},{-23.024}) {\tikz\pgfuseplotmark{m5a};};
\node[stars] at (axis cs:{112.678},{-30.962}) {\tikz\pgfuseplotmark{m5av};};
\node[stars] at (axis cs:{112.713},{-5.226}) {\tikz\pgfuseplotmark{m6c};};
\node[stars] at (axis cs:{112.787},{11.203}) {\tikz\pgfuseplotmark{m6c};};
\node[stars] at (axis cs:{112.952},{17.086}) {\tikz\pgfuseplotmark{m5c};};
\node[stars] at (axis cs:{113.024},{-8.881}) {\tikz\pgfuseplotmark{m6bvb};};
\node[stars] at (axis cs:{113.025},{1.914}) {\tikz\pgfuseplotmark{m5c};};
\node[stars] at (axis cs:{113.093},{-35.961}) {\tikz\pgfuseplotmark{m6c};};
\node[stars] at (axis cs:{113.291},{-24.711}) {\tikz\pgfuseplotmark{m6a};};
\node[stars] at (axis cs:{113.299},{3.290}) {\tikz\pgfuseplotmark{m6a};};
\node[stars] at (axis cs:{113.332},{-19.412}) {\tikz\pgfuseplotmark{m6a};};
\node[stars] at (axis cs:{113.342},{-14.338}) {\tikz\pgfuseplotmark{m6c};};
\node[stars] at (axis cs:{113.353},{37.188}) {\tikz\pgfuseplotmark{m6c};};
\node[stars] at (axis cs:{113.402},{15.826}) {\tikz\pgfuseplotmark{m5cv};};
\node[stars] at (axis cs:{113.450},{-14.524}) {\tikz\pgfuseplotmark{m5bv};};
\node[stars] at (axis cs:{113.463},{-36.338}) {\tikz\pgfuseplotmark{m5cv};};
\node[stars] at (axis cs:{113.513},{-22.296}) {\tikz\pgfuseplotmark{m4c};};
\node[stars] at (axis cs:{113.521},{10.568}) {\tikz\pgfuseplotmark{m6c};};
\node[stars] at (axis cs:{113.553},{-33.463}) {\tikz\pgfuseplotmark{m6b};};
\node[stars] at (axis cs:{113.566},{3.372}) {\tikz\pgfuseplotmark{m6ab};};
\node[stars] at (axis cs:{113.645},{-27.012}) {\tikz\pgfuseplotmark{m6a};};
\node[stars] at (axis cs:{113.649},{31.888}) {\tikz\pgfuseplotmark{m2ab};};
\node[stars] at (axis cs:{113.787},{30.961}) {\tikz\pgfuseplotmark{m5c};};
\node[stars] at (axis cs:{113.845},{-28.369}) {\tikz\pgfuseplotmark{m5ab};};
\node[stars] at (axis cs:{113.981},{26.896}) {\tikz\pgfuseplotmark{m4bv};};
\node[stars] at (axis cs:{114.016},{-14.493}) {\tikz\pgfuseplotmark{m6av};};
\node[stars] at (axis cs:{114.031},{-22.160}) {\tikz\pgfuseplotmark{m6c};};
\node[stars] at (axis cs:{114.070},{-8.311}) {\tikz\pgfuseplotmark{m6c};};
\node[stars] at (axis cs:{114.145},{5.862}) {\tikz\pgfuseplotmark{m6b};};
\node[stars] at (axis cs:{114.171},{-19.702}) {\tikz\pgfuseplotmark{m6av};};
\node[stars] at (axis cs:{114.320},{-4.111}) {\tikz\pgfuseplotmark{m5cv};};
\node[stars] at (axis cs:{114.321},{-23.775}) {\tikz\pgfuseplotmark{m6c};};
\node[stars] at (axis cs:{114.342},{-34.968}) {\tikz\pgfuseplotmark{m5a};};
\node[stars] at (axis cs:{114.438},{-38.010}) {\tikz\pgfuseplotmark{m6c};};
\node[stars] at (axis cs:{114.560},{24.360}) {\tikz\pgfuseplotmark{m6c};};
\node[stars] at (axis cs:{114.575},{-25.365}) {\tikz\pgfuseplotmark{m5av};};
\node[stars] at (axis cs:{114.636},{-38.781}) {\tikz\pgfuseplotmark{m6c};};
\node[stars] at (axis cs:{114.637},{35.048}) {\tikz\pgfuseplotmark{m6ab};};
\node[stars] at (axis cs:{114.683},{-36.497}) {\tikz\pgfuseplotmark{m6a};};
\node[stars] at (axis cs:{114.708},{-26.804}) {\tikz\pgfuseplotmark{m4cvb};};
\node[stars] at (axis cs:{114.791},{34.584}) {\tikz\pgfuseplotmark{m5b};};
\node[stars] at (axis cs:{114.800},{24.222}) {\tikz\pgfuseplotmark{m6c};};
\node[stars] at (axis cs:{114.825},{5.225}) {\tikz\pgfuseplotmark{m1bvb};};
\node[stars] at (axis cs:{114.851},{-16.847}) {\tikz\pgfuseplotmark{m6c};};
\node[stars] at (axis cs:{114.864},{-38.308}) {\tikz\pgfuseplotmark{m5a};};
\node[stars] at (axis cs:{114.869},{17.674}) {\tikz\pgfuseplotmark{m5bv};};
\node[stars] at (axis cs:{114.933},{-38.139}) {\tikz\pgfuseplotmark{m6ab};};
\node[stars] at (axis cs:{114.949},{-38.260}) {\tikz\pgfuseplotmark{m6a};};
\node[stars] at (axis cs:{114.975},{32.010}) {\tikz\pgfuseplotmark{m6c};};
\node[stars] at (axis cs:{114.992},{-37.579}) {\tikz\pgfuseplotmark{m6bv};};
\node[stars] at (axis cs:{115.029},{5.231}) {\tikz\pgfuseplotmark{m6b};};
\node[stars] at (axis cs:{115.056},{-19.661}) {\tikz\pgfuseplotmark{m6b};};
\node[stars] at (axis cs:{115.061},{38.344}) {\tikz\pgfuseplotmark{m6a};};
\node[stars] at (axis cs:{115.088},{-11.753}) {\tikz\pgfuseplotmark{m6c};};
\node[stars] at (axis cs:{115.097},{-15.264}) {\tikz\pgfuseplotmark{m5b};};
\node[stars] at (axis cs:{115.148},{-8.185}) {\tikz\pgfuseplotmark{m6b};};
\node[stars] at (axis cs:{115.197},{13.771}) {\tikz\pgfuseplotmark{m6c};};
\node[stars] at (axis cs:{115.244},{23.018}) {\tikz\pgfuseplotmark{m6b};};
\node[stars] at (axis cs:{115.312},{-9.551}) {\tikz\pgfuseplotmark{m4b};};
\node[stars] at (axis cs:{115.316},{-38.534}) {\tikz\pgfuseplotmark{m5c};};
\node[stars] at (axis cs:{115.348},{-22.337}) {\tikz\pgfuseplotmark{m6c};};
\node[stars] at (axis cs:{115.396},{3.625}) {\tikz\pgfuseplotmark{m6b};};
\node[stars] at (axis cs:{115.466},{13.480}) {\tikz\pgfuseplotmark{m6a};};
\node[stars] at (axis cs:{115.513},{14.208}) {\tikz\pgfuseplotmark{m6av};};
\node[stars] at (axis cs:{115.681},{34.000}) {\tikz\pgfuseplotmark{m6b};};
\node[stars] at (axis cs:{115.701},{-26.351}) {\tikz\pgfuseplotmark{m6a};};
\node[stars] at (axis cs:{115.773},{0.189}) {\tikz\pgfuseplotmark{m6c};};
\node[stars] at (axis cs:{115.800},{-36.050}) {\tikz\pgfuseplotmark{m6a};};
\node[stars] at (axis cs:{115.828},{28.884}) {\tikz\pgfuseplotmark{m4cv};};
\node[stars] at (axis cs:{115.842},{22.399}) {\tikz\pgfuseplotmark{m6c};};
\node[stars] at (axis cs:{115.885},{-28.411}) {\tikz\pgfuseplotmark{m5ab};};
\node[stars] at (axis cs:{115.929},{-38.202}) {\tikz\pgfuseplotmark{m6c};};
\node[stars] at (axis cs:{115.952},{-28.955}) {\tikz\pgfuseplotmark{m4b};};
\node[stars] at (axis cs:{116.029},{25.784}) {\tikz\pgfuseplotmark{m5cv};};
\node[stars] at (axis cs:{116.032},{2.405}) {\tikz\pgfuseplotmark{m6cv};};
\node[stars] at (axis cs:{116.040},{-36.062}) {\tikz\pgfuseplotmark{m6a};};
\node[stars] at (axis cs:{116.059},{12.860}) {\tikz\pgfuseplotmark{m6c};};
\node[stars] at (axis cs:{116.112},{24.398}) {\tikz\pgfuseplotmark{m4a};};
\node[stars] at (axis cs:{116.142},{-24.674}) {\tikz\pgfuseplotmark{m6av};};
\node[stars] at (axis cs:{116.142},{-37.943}) {\tikz\pgfuseplotmark{m6b};};
\node[stars] at (axis cs:{116.314},{-37.969}) {\tikz\pgfuseplotmark{m4av};};
\node[stars] at (axis cs:{116.329},{28.026}) {\tikz\pgfuseplotmark{m1cvb};};
\node[stars] at (axis cs:{116.371},{-14.690}) {\tikz\pgfuseplotmark{m6bb};};
\node[stars] at (axis cs:{116.396},{-34.172}) {\tikz\pgfuseplotmark{m5c};};
\node[stars] at (axis cs:{116.487},{-14.564}) {\tikz\pgfuseplotmark{m5b};};
\node[stars] at (axis cs:{116.509},{-6.772}) {\tikz\pgfuseplotmark{m5c};};
\node[stars] at (axis cs:{116.531},{18.510}) {\tikz\pgfuseplotmark{m5b};};
\node[stars] at (axis cs:{116.544},{-37.934}) {\tikz\pgfuseplotmark{m6bv};};
\node[stars] at (axis cs:{116.568},{10.768}) {\tikz\pgfuseplotmark{m5c};};
\node[stars] at (axis cs:{116.639},{-37.772}) {\tikz\pgfuseplotmark{m6c};};
\node[stars] at (axis cs:{116.664},{37.517}) {\tikz\pgfuseplotmark{m5cv};};
\node[stars] at (axis cs:{116.686},{-12.675}) {\tikz\pgfuseplotmark{m6c};};
\node[stars] at (axis cs:{116.774},{-39.331}) {\tikz\pgfuseplotmark{m6cv};};
\node[stars] at (axis cs:{116.802},{-22.519}) {\tikz\pgfuseplotmark{m6bv};};
\node[stars] at (axis cs:{116.811},{-36.073}) {\tikz\pgfuseplotmark{m6c};};
\node[stars] at (axis cs:{116.854},{-38.511}) {\tikz\pgfuseplotmark{m5bv};};
\node[stars] at (axis cs:{116.876},{33.415}) {\tikz\pgfuseplotmark{m5cv};};
\node[stars] at (axis cs:{116.911},{-15.991}) {\tikz\pgfuseplotmark{m6cv};};
\node[stars] at (axis cs:{116.938},{-16.014}) {\tikz\pgfuseplotmark{m6cvb};};
\node[stars] at (axis cs:{116.986},{-12.193}) {\tikz\pgfuseplotmark{m6a};};
\node[stars] at (axis cs:{117.022},{-25.937}) {\tikz\pgfuseplotmark{m4cv};};
\node[stars] at (axis cs:{117.140},{23.141}) {\tikz\pgfuseplotmark{m6cv};};
\node[stars] at (axis cs:{117.257},{-24.912}) {\tikz\pgfuseplotmark{m5c};};
\node[stars] at (axis cs:{117.258},{13.372}) {\tikz\pgfuseplotmark{m6b};};
\node[stars] at (axis cs:{117.311},{-35.243}) {\tikz\pgfuseplotmark{m6bv};};
\node[stars] at (axis cs:{117.324},{-24.860}) {\tikz\pgfuseplotmark{m3c};};
\node[stars] at (axis cs:{117.370},{-13.353}) {\tikz\pgfuseplotmark{m6c};};
\node[stars] at (axis cs:{117.385},{12.814}) {\tikz\pgfuseplotmark{m6c};};
\node[stars] at (axis cs:{117.398},{-33.289}) {\tikz\pgfuseplotmark{m6a};};
\node[stars] at (axis cs:{117.417},{-14.085}) {\tikz\pgfuseplotmark{m6cv};};
\node[stars] at (axis cs:{117.422},{-17.228}) {\tikz\pgfuseplotmark{m5c};};
\node[stars] at (axis cs:{117.524},{-19.523}) {\tikz\pgfuseplotmark{m6b};};
\node[stars] at (axis cs:{117.544},{-9.183}) {\tikz\pgfuseplotmark{m6a};};
\node[stars] at (axis cs:{117.697},{3.277}) {\tikz\pgfuseplotmark{m6c};};
\node[stars] at (axis cs:{117.713},{4.459}) {\tikz\pgfuseplotmark{m6c};};
\node[stars] at (axis cs:{117.723},{0.079}) {\tikz\pgfuseplotmark{m6c};};
\node[stars] at (axis cs:{117.730},{-11.128}) {\tikz\pgfuseplotmark{m6c};};
\node[stars] at (axis cs:{117.750},{-24.528}) {\tikz\pgfuseplotmark{m6c};};
\node[stars] at (axis cs:{117.760},{33.234}) {\tikz\pgfuseplotmark{m6b};};
\node[stars] at (axis cs:{117.920},{-12.819}) {\tikz\pgfuseplotmark{m6c};};
\node[stars] at (axis cs:{117.925},{1.767}) {\tikz\pgfuseplotmark{m5b};};
\node[stars] at (axis cs:{117.929},{-21.174}) {\tikz\pgfuseplotmark{m6a};};
\node[stars] at (axis cs:{117.943},{-13.898}) {\tikz\pgfuseplotmark{m5cv};};
\node[stars] at (axis cs:{117.987},{19.325}) {\tikz\pgfuseplotmark{m6b};};
\node[stars] at (axis cs:{118.030},{3.277}) {\tikz\pgfuseplotmark{m6cv};};
\node[stars] at (axis cs:{118.065},{-34.705}) {\tikz\pgfuseplotmark{m5bb};};
\node[stars] at (axis cs:{118.079},{-14.846}) {\tikz\pgfuseplotmark{m6a};};
\node[stars] at (axis cs:{118.161},{-38.863}) {\tikz\pgfuseplotmark{m4cv};};
\node[stars] at (axis cs:{118.199},{-5.428}) {\tikz\pgfuseplotmark{m6av};};
\node[stars] at (axis cs:{118.265},{-36.364}) {\tikz\pgfuseplotmark{m5c};};
\node[stars] at (axis cs:{118.374},{26.766}) {\tikz\pgfuseplotmark{m5b};};
\node[stars] at (axis cs:{118.546},{-35.877}) {\tikz\pgfuseplotmark{m5cv};};
\node[stars] at (axis cs:{118.666},{-34.846}) {\tikz\pgfuseplotmark{m6b};};
\node[stars] at (axis cs:{118.808},{-30.918}) {\tikz\pgfuseplotmark{m6c};};
\node[stars] at (axis cs:{118.881},{8.863}) {\tikz\pgfuseplotmark{m6a};};
\node[stars] at (axis cs:{118.916},{19.884}) {\tikz\pgfuseplotmark{m5c};};
\node[stars] at (axis cs:{118.920},{35.412}) {\tikz\pgfuseplotmark{m6cv};};
\node[stars] at (axis cs:{119.095},{-30.285}) {\tikz\pgfuseplotmark{m6cv};};
\node[stars] at (axis cs:{119.100},{4.486}) {\tikz\pgfuseplotmark{m6c};};
\node[stars] at (axis cs:{119.215},{-22.880}) {\tikz\pgfuseplotmark{m4c};};
\node[stars] at (axis cs:{119.248},{15.790}) {\tikz\pgfuseplotmark{m6a};};
\node[stars] at (axis cs:{119.316},{8.641}) {\tikz\pgfuseplotmark{m6b};};
\node[stars] at (axis cs:{119.318},{1.127}) {\tikz\pgfuseplotmark{m6c};};
\node[stars] at (axis cs:{119.417},{-30.335}) {\tikz\pgfuseplotmark{m5a};};
\node[stars] at (axis cs:{119.525},{7.213}) {\tikz\pgfuseplotmark{m6cv};};
\node[stars] at (axis cs:{119.586},{2.225}) {\tikz\pgfuseplotmark{m5cb};};
\node[stars] at (axis cs:{119.631},{16.519}) {\tikz\pgfuseplotmark{m6bv};};
\node[stars] at (axis cs:{119.774},{-23.310}) {\tikz\pgfuseplotmark{m5b};};
\node[stars] at (axis cs:{119.868},{-39.297}) {\tikz\pgfuseplotmark{m5c};};
\node[stars] at (axis cs:{119.896},{13.242}) {\tikz\pgfuseplotmark{m6b};};
\node[stars] at (axis cs:{119.934},{-3.679}) {\tikz\pgfuseplotmark{m5b};};
\node[stars] at (axis cs:{119.967},{-18.399}) {\tikz\pgfuseplotmark{m5a};};
\node[stars] at (axis cs:{120.184},{-2.882}) {\tikz\pgfuseplotmark{m6cv};};
\node[stars] at (axis cs:{120.197},{17.309}) {\tikz\pgfuseplotmark{m6a};};
\node[stars] at (axis cs:{120.200},{19.816}) {\tikz\pgfuseplotmark{m6c};};
\node[stars] at (axis cs:{120.233},{25.393}) {\tikz\pgfuseplotmark{m6a};};
\node[stars] at (axis cs:{120.253},{23.583}) {\tikz\pgfuseplotmark{m6cb};};
\node[stars] at (axis cs:{120.264},{-6.419}) {\tikz\pgfuseplotmark{m6c};};
\node[stars] at (axis cs:{120.306},{-1.392}) {\tikz\pgfuseplotmark{m5av};};
\node[stars] at (axis cs:{120.308},{4.880}) {\tikz\pgfuseplotmark{m6a};};
\node[stars] at (axis cs:{120.376},{16.455}) {\tikz\pgfuseplotmark{m6b};};
\node[stars] at (axis cs:{120.406},{-37.284}) {\tikz\pgfuseplotmark{m6b};};
\node[stars] at (axis cs:{120.432},{25.089}) {\tikz\pgfuseplotmark{m6c};};
\node[stars] at (axis cs:{120.461},{8.914}) {\tikz\pgfuseplotmark{m6cv};};
\node[stars] at (axis cs:{120.480},{35.413}) {\tikz\pgfuseplotmark{m6c};};
\node[stars] at (axis cs:{120.526},{-37.050}) {\tikz\pgfuseplotmark{m6cv};};
\node[stars] at (axis cs:{120.566},{2.335}) {\tikz\pgfuseplotmark{m4c};};
\node[stars] at (axis cs:{120.608},{-6.337}) {\tikz\pgfuseplotmark{m6c};};
\node[stars] at (axis cs:{120.767},{-32.463}) {\tikz\pgfuseplotmark{m6ab};};
\node[stars] at (axis cs:{120.880},{27.794}) {\tikz\pgfuseplotmark{m5b};};
\node[stars] at (axis cs:{121.067},{-32.675}) {\tikz\pgfuseplotmark{m5cv};};
\node[stars] at (axis cs:{121.173},{-19.728}) {\tikz\pgfuseplotmark{m6b};};
\node[stars] at (axis cs:{121.189},{18.842}) {\tikz\pgfuseplotmark{m6c};};
\node[stars] at (axis cs:{121.269},{13.118}) {\tikz\pgfuseplotmark{m5c};};
\node[stars] at (axis cs:{121.404},{27.530}) {\tikz\pgfuseplotmark{m6cvb};};
\node[stars] at (axis cs:{121.437},{-33.569}) {\tikz\pgfuseplotmark{m6bb};};
\node[stars] at (axis cs:{121.457},{-0.573}) {\tikz\pgfuseplotmark{m6cv};};
\node[stars] at (axis cs:{121.577},{22.635}) {\tikz\pgfuseplotmark{m6bv};};
\node[stars] at (axis cs:{121.614},{-9.244}) {\tikz\pgfuseplotmark{m6cb};};
\node[stars] at (axis cs:{121.825},{-20.554}) {\tikz\pgfuseplotmark{m5c};};
\node[stars] at (axis cs:{121.886},{-24.304}) {\tikz\pgfuseplotmark{m3av};};
\node[stars] at (axis cs:{121.941},{21.582}) {\tikz\pgfuseplotmark{m5c};};
\node[stars] at (axis cs:{122.149},{-2.984}) {\tikz\pgfuseplotmark{m4c};};
\node[stars] at (axis cs:{122.157},{-37.681}) {\tikz\pgfuseplotmark{m6cv};};
\node[stars] at (axis cs:{122.177},{13.641}) {\tikz\pgfuseplotmark{m6c};};
\node[stars] at (axis cs:{122.182},{-20.363}) {\tikz\pgfuseplotmark{m6c};};
\node[stars] at (axis cs:{122.237},{-11.340}) {\tikz\pgfuseplotmark{m6c};};
\node[stars] at (axis cs:{122.257},{-19.245}) {\tikz\pgfuseplotmark{m4c};};
\node[stars] at (axis cs:{122.292},{-35.455}) {\tikz\pgfuseplotmark{m6c};};
\node[stars] at (axis cs:{122.369},{-16.249}) {\tikz\pgfuseplotmark{m6a};};
\node[stars] at (axis cs:{122.397},{29.093}) {\tikz\pgfuseplotmark{m6c};};
\node[stars] at (axis cs:{122.555},{25.844}) {\tikz\pgfuseplotmark{m6cb};};
\node[stars] at (axis cs:{122.613},{25.507}) {\tikz\pgfuseplotmark{m6av};};
\node[stars] at (axis cs:{122.666},{-13.799}) {\tikz\pgfuseplotmark{m6ab};};
\node[stars] at (axis cs:{122.745},{14.629}) {\tikz\pgfuseplotmark{m6c};};
\node[stars] at (axis cs:{122.757},{-37.292}) {\tikz\pgfuseplotmark{m6c};};
\node[stars] at (axis cs:{122.818},{-12.927}) {\tikz\pgfuseplotmark{m5a};};
\node[stars] at (axis cs:{122.819},{9.821}) {\tikz\pgfuseplotmark{m6b};};
\node[stars] at (axis cs:{122.840},{-39.619}) {\tikz\pgfuseplotmark{m4cv};};
\node[stars] at (axis cs:{122.888},{-7.772}) {\tikz\pgfuseplotmark{m5c};};
\node[stars] at (axis cs:{123.053},{17.648}) {\tikz\pgfuseplotmark{m6ab};};
\node[stars] at (axis cs:{123.215},{-37.924}) {\tikz\pgfuseplotmark{m6c};};
\node[stars] at (axis cs:{123.249},{16.514}) {\tikz\pgfuseplotmark{m6b};};
\node[stars] at (axis cs:{123.287},{29.656}) {\tikz\pgfuseplotmark{m6av};};
\node[stars] at (axis cs:{123.333},{-15.788}) {\tikz\pgfuseplotmark{m5bv};};
\node[stars] at (axis cs:{123.373},{-35.899}) {\tikz\pgfuseplotmark{m5av};};
\node[stars] at (axis cs:{123.421},{-33.569}) {\tikz\pgfuseplotmark{m6c};};
\node[stars] at (axis cs:{123.493},{-36.322}) {\tikz\pgfuseplotmark{m5bvb};};
\node[stars] at (axis cs:{123.546},{-32.141}) {\tikz\pgfuseplotmark{m6b};};
\node[stars] at (axis cs:{123.546},{17.676}) {\tikz\pgfuseplotmark{m6cb};};
\node[stars] at (axis cs:{123.555},{-35.490}) {\tikz\pgfuseplotmark{m6a};};
\node[stars] at (axis cs:{123.587},{13.048}) {\tikz\pgfuseplotmark{m6c};};
\node[stars] at (axis cs:{123.969},{-30.926}) {\tikz\pgfuseplotmark{m6cb};};
\node[stars] at (axis cs:{123.995},{-35.903}) {\tikz\pgfuseplotmark{m6c};};
\node[stars] at (axis cs:{124.129},{9.186}) {\tikz\pgfuseplotmark{m4av};};
\node[stars] at (axis cs:{124.346},{-16.285}) {\tikz\pgfuseplotmark{m6cv};};
\node[stars] at (axis cs:{124.382},{8.866}) {\tikz\pgfuseplotmark{m6c};};
\node[stars] at (axis cs:{124.493},{-30.004}) {\tikz\pgfuseplotmark{m6c};};
\node[stars] at (axis cs:{124.560},{15.678}) {\tikz\pgfuseplotmark{m6c};};
\node[stars] at (axis cs:{124.572},{-35.452}) {\tikz\pgfuseplotmark{m6a};};
\node[stars] at (axis cs:{124.600},{-12.632}) {\tikz\pgfuseplotmark{m6b};};
\node[stars] at (axis cs:{124.639},{-36.659}) {\tikz\pgfuseplotmark{m4c};};
\node[stars] at (axis cs:{124.813},{-10.165}) {\tikz\pgfuseplotmark{m6cv};};
\node[stars] at (axis cs:{124.873},{-34.590}) {\tikz\pgfuseplotmark{m6c};};
\node[stars] at (axis cs:{124.958},{3.948}) {\tikz\pgfuseplotmark{m6b};};
\node[stars] at (axis cs:{125.016},{27.218}) {\tikz\pgfuseplotmark{m5bv};};
\node[stars] at (axis cs:{125.054},{-0.909}) {\tikz\pgfuseplotmark{m6c};};
\node[stars] at (axis cs:{125.071},{-5.330}) {\tikz\pgfuseplotmark{m6b};};
\node[stars] at (axis cs:{125.087},{20.747}) {\tikz\pgfuseplotmark{m6a};};
\node[stars] at (axis cs:{125.114},{-22.925}) {\tikz\pgfuseplotmark{m6b};};
\node[stars] at (axis cs:{125.134},{24.022}) {\tikz\pgfuseplotmark{m6b};};
\node[stars] at (axis cs:{125.338},{-36.484}) {\tikz\pgfuseplotmark{m5cv};};
\node[stars] at (axis cs:{125.338},{-20.079}) {\tikz\pgfuseplotmark{m6a};};
\node[stars] at (axis cs:{125.346},{-33.054}) {\tikz\pgfuseplotmark{m5a};};
\node[stars] at (axis cs:{125.351},{-39.621}) {\tikz\pgfuseplotmark{m6c};};
\node[stars] at (axis cs:{125.478},{-17.586}) {\tikz\pgfuseplotmark{m6a};};
\node[stars] at (axis cs:{125.626},{-6.179}) {\tikz\pgfuseplotmark{m6c};};
\node[stars] at (axis cs:{125.695},{-13.055}) {\tikz\pgfuseplotmark{m6bv};};
\node[stars] at (axis cs:{125.708},{-26.348}) {\tikz\pgfuseplotmark{m6b};};
\node[stars] at (axis cs:{125.725},{-7.543}) {\tikz\pgfuseplotmark{m6bv};};
\node[stars] at (axis cs:{125.821},{-38.286}) {\tikz\pgfuseplotmark{m6cv};};
\node[stars] at (axis cs:{125.841},{18.332}) {\tikz\pgfuseplotmark{m6b};};
\node[stars] at (axis cs:{125.980},{10.632}) {\tikz\pgfuseplotmark{m6bvb};};
\node[stars] at (axis cs:{126.146},{-3.751}) {\tikz\pgfuseplotmark{m6a};};
\node[stars] at (axis cs:{126.152},{-4.717}) {\tikz\pgfuseplotmark{m6b};};
\node[stars] at (axis cs:{126.230},{-23.154}) {\tikz\pgfuseplotmark{m6a};};
\node[stars] at (axis cs:{126.266},{-24.046}) {\tikz\pgfuseplotmark{m5c};};
\node[stars] at (axis cs:{126.270},{35.011}) {\tikz\pgfuseplotmark{m6b};};
\node[stars] at (axis cs:{126.329},{-21.046}) {\tikz\pgfuseplotmark{m6b};};
\node[stars] at (axis cs:{126.398},{2.102}) {\tikz\pgfuseplotmark{m6a};};
\node[stars] at (axis cs:{126.414},{-17.439}) {\tikz\pgfuseplotmark{m6c};};
\node[stars] at (axis cs:{126.415},{-3.906}) {\tikz\pgfuseplotmark{m4bv};};
\node[stars] at (axis cs:{126.458},{17.046}) {\tikz\pgfuseplotmark{m6b};};
\node[stars] at (axis cs:{126.478},{7.564}) {\tikz\pgfuseplotmark{m5b};};
\node[stars] at (axis cs:{126.482},{-14.929}) {\tikz\pgfuseplotmark{m6b};};
\node[stars] at (axis cs:{126.613},{-3.987}) {\tikz\pgfuseplotmark{m6av};};
\node[stars] at (axis cs:{126.615},{27.893}) {\tikz\pgfuseplotmark{m6a};};
\node[stars] at (axis cs:{126.675},{-12.535}) {\tikz\pgfuseplotmark{m6a};};
\node[stars] at (axis cs:{126.683},{12.655}) {\tikz\pgfuseplotmark{m6av};};
\node[stars] at (axis cs:{126.696},{26.935}) {\tikz\pgfuseplotmark{m6cvb};};
\node[stars] at (axis cs:{126.818},{-31.673}) {\tikz\pgfuseplotmark{m6c};};
\node[stars] at (axis cs:{126.998},{-35.114}) {\tikz\pgfuseplotmark{m6avb};};
\node[stars] at (axis cs:{127.082},{-8.816}) {\tikz\pgfuseplotmark{m6c};};
\node[stars] at (axis cs:{127.121},{-2.517}) {\tikz\pgfuseplotmark{m6cv};};
\node[stars] at (axis cs:{127.153},{24.145}) {\tikz\pgfuseplotmark{m6bv};};
\node[stars] at (axis cs:{127.156},{14.211}) {\tikz\pgfuseplotmark{m6b};};
\node[stars] at (axis cs:{127.213},{-9.748}) {\tikz\pgfuseplotmark{m6b};};
\node[stars] at (axis cs:{127.619},{-32.159}) {\tikz\pgfuseplotmark{m6av};};
\node[stars] at (axis cs:{127.833},{37.265}) {\tikz\pgfuseplotmark{m6c};};
\node[stars] at (axis cs:{127.853},{-39.064}) {\tikz\pgfuseplotmark{m6cb};};
\node[stars] at (axis cs:{127.877},{24.081}) {\tikz\pgfuseplotmark{m6a};};
\node[stars] at (axis cs:{127.879},{-19.577}) {\tikz\pgfuseplotmark{m5c};};
\node[stars] at (axis cs:{127.899},{18.094}) {\tikz\pgfuseplotmark{m5c};};
\node[stars] at (axis cs:{127.922},{18.988}) {\tikz\pgfuseplotmark{m6c};};
\node[stars] at (axis cs:{128.139},{-15.030}) {\tikz\pgfuseplotmark{m6c};};
\node[stars] at (axis cs:{128.166},{10.066}) {\tikz\pgfuseplotmark{m6cv};};
\node[stars] at (axis cs:{128.177},{20.441}) {\tikz\pgfuseplotmark{m5c};};
\node[stars] at (axis cs:{128.215},{-31.501}) {\tikz\pgfuseplotmark{m6c};};
\node[stars] at (axis cs:{128.229},{38.016}) {\tikz\pgfuseplotmark{m6b};};
\node[stars] at (axis cs:{128.244},{-34.634}) {\tikz\pgfuseplotmark{m6cv};};
\node[stars] at (axis cs:{128.250},{24.085}) {\tikz\pgfuseplotmark{m6c};};
\node[stars] at (axis cs:{128.270},{-24.607}) {\tikz\pgfuseplotmark{m6c};};
\node[stars] at (axis cs:{128.333},{-38.371}) {\tikz\pgfuseplotmark{m6cv};};
\node[stars] at (axis cs:{128.341},{36.436}) {\tikz\pgfuseplotmark{m6c};};
\node[stars] at (axis cs:{128.410},{-38.849}) {\tikz\pgfuseplotmark{m6bv};};
\node[stars] at (axis cs:{128.431},{4.757}) {\tikz\pgfuseplotmark{m6a};};
\node[stars] at (axis cs:{128.438},{13.257}) {\tikz\pgfuseplotmark{m6c};};
\node[stars] at (axis cs:{128.507},{-2.152}) {\tikz\pgfuseplotmark{m6a};};
\node[stars] at (axis cs:{128.556},{8.452}) {\tikz\pgfuseplotmark{m6b};};
\node[stars] at (axis cs:{128.622},{-37.611}) {\tikz\pgfuseplotmark{m6cvb};};
\node[stars] at (axis cs:{128.624},{-27.098}) {\tikz\pgfuseplotmark{m6cv};};
\node[stars] at (axis cs:{128.633},{-32.598}) {\tikz\pgfuseplotmark{m6c};};
\node[stars] at (axis cs:{128.683},{36.420}) {\tikz\pgfuseplotmark{m6a};};
\node[stars] at (axis cs:{128.802},{-39.970}) {\tikz\pgfuseplotmark{m6c};};
\node[stars] at (axis cs:{128.854},{2.743}) {\tikz\pgfuseplotmark{m6c};};
\node[stars] at (axis cs:{128.867},{-7.982}) {\tikz\pgfuseplotmark{m6av};};
\node[stars] at (axis cs:{128.870},{-26.843}) {\tikz\pgfuseplotmark{m6b};};
\node[stars] at (axis cs:{128.962},{6.620}) {\tikz\pgfuseplotmark{m6bb};};
\node[stars] at (axis cs:{129.032},{15.313}) {\tikz\pgfuseplotmark{m6c};};
\node[stars] at (axis cs:{129.274},{9.655}) {\tikz\pgfuseplotmark{m6bv};};
\node[stars] at (axis cs:{129.363},{-4.934}) {\tikz\pgfuseplotmark{m6c};};
\node[stars] at (axis cs:{129.414},{5.704}) {\tikz\pgfuseplotmark{m4c};};
\node[stars] at (axis cs:{129.467},{-26.255}) {\tikz\pgfuseplotmark{m5cv};};
\node[stars] at (axis cs:{129.579},{32.802}) {\tikz\pgfuseplotmark{m6b};};
\node[stars] at (axis cs:{129.668},{-19.737}) {\tikz\pgfuseplotmark{m6cb};};
\node[stars] at (axis cs:{129.689},{3.341}) {\tikz\pgfuseplotmark{m4c};};
\node[stars] at (axis cs:{129.783},{-22.662}) {\tikz\pgfuseplotmark{m5b};};
\node[stars] at (axis cs:{129.842},{-36.607}) {\tikz\pgfuseplotmark{m6b};};
\node[stars] at (axis cs:{129.852},{8.018}) {\tikz\pgfuseplotmark{m6c};};
\node[stars] at (axis cs:{129.927},{-29.561}) {\tikz\pgfuseplotmark{m5b};};
\node[stars] at (axis cs:{130.006},{-12.475}) {\tikz\pgfuseplotmark{m5b};};
\node[stars] at (axis cs:{130.026},{-35.308}) {\tikz\pgfuseplotmark{m4b};};
\node[stars] at (axis cs:{130.027},{20.008}) {\tikz\pgfuseplotmark{m6cvb};};
\node[stars] at (axis cs:{130.076},{31.942}) {\tikz\pgfuseplotmark{m6c};};
\node[stars] at (axis cs:{130.092},{19.670}) {\tikz\pgfuseplotmark{m6cb};};
\node[stars] at (axis cs:{130.113},{19.545}) {\tikz\pgfuseplotmark{m6cb};};
\node[stars] at (axis cs:{130.431},{-15.943}) {\tikz\pgfuseplotmark{m5b};};
\node[stars] at (axis cs:{130.541},{-11.966}) {\tikz\pgfuseplotmark{m6c};};
\node[stars] at (axis cs:{130.737},{-35.943}) {\tikz\pgfuseplotmark{m6c};};
\node[stars] at (axis cs:{130.801},{12.681}) {\tikz\pgfuseplotmark{m6a};};
\node[stars] at (axis cs:{130.806},{3.398}) {\tikz\pgfuseplotmark{m4cv};};
\node[stars] at (axis cs:{130.821},{21.468}) {\tikz\pgfuseplotmark{m5a};};
\node[stars] at (axis cs:{130.898},{-33.186}) {\tikz\pgfuseplotmark{m4av};};
\node[stars] at (axis cs:{130.918},{-7.234}) {\tikz\pgfuseplotmark{m5ab};};
\node[stars] at (axis cs:{130.999},{4.335}) {\tikz\pgfuseplotmark{m6c};};
\node[stars] at (axis cs:{131.171},{18.154}) {\tikz\pgfuseplotmark{m4b};};
\node[stars] at (axis cs:{131.188},{10.081}) {\tikz\pgfuseplotmark{m6av};};
\node[stars] at (axis cs:{131.216},{-37.147}) {\tikz\pgfuseplotmark{m6av};};
\node[stars] at (axis cs:{131.230},{-21.168}) {\tikz\pgfuseplotmark{m6b};};
\node[stars] at (axis cs:{131.255},{5.680}) {\tikz\pgfuseplotmark{m6b};};
\node[stars] at (axis cs:{131.339},{30.698}) {\tikz\pgfuseplotmark{m6b};};
\node[stars] at (axis cs:{131.395},{4.664}) {\tikz\pgfuseplotmark{m6c};};
\node[stars] at (axis cs:{131.455},{-25.388}) {\tikz\pgfuseplotmark{m6b};};
\node[stars] at (axis cs:{131.510},{-2.048}) {\tikz\pgfuseplotmark{m6a};};
\node[stars] at (axis cs:{131.529},{-11.007}) {\tikz\pgfuseplotmark{m6cb};};
\node[stars] at (axis cs:{131.594},{-13.547}) {\tikz\pgfuseplotmark{m4c};};
\node[stars] at (axis cs:{131.674},{28.760}) {\tikz\pgfuseplotmark{m4bb};};
\node[stars] at (axis cs:{131.694},{6.419}) {\tikz\pgfuseplotmark{m3cvb};};
\node[stars] at (axis cs:{131.705},{-34.623}) {\tikz\pgfuseplotmark{m6cv};};
\node[stars] at (axis cs:{131.733},{12.110}) {\tikz\pgfuseplotmark{m6b};};
\node[stars] at (axis cs:{131.812},{-1.897}) {\tikz\pgfuseplotmark{m5cv};};
\node[stars] at (axis cs:{132.020},{-6.559}) {\tikz\pgfuseplotmark{m6b};};
\node[stars] at (axis cs:{132.108},{5.838}) {\tikz\pgfuseplotmark{m4c};};
\node[stars] at (axis cs:{132.156},{-1.045}) {\tikz\pgfuseplotmark{m6c};};
\node[stars] at (axis cs:{132.341},{-3.443}) {\tikz\pgfuseplotmark{m5cv};};
\node[stars] at (axis cs:{132.437},{-21.048}) {\tikz\pgfuseplotmark{m6c};};
\node[stars] at (axis cs:{132.465},{-32.780}) {\tikz\pgfuseplotmark{m5c};};
\node[stars] at (axis cs:{132.468},{-39.141}) {\tikz\pgfuseplotmark{m6cv};};
\node[stars] at (axis cs:{132.509},{-29.463}) {\tikz\pgfuseplotmark{m6b};};
\node[stars] at (axis cs:{132.590},{-28.618}) {\tikz\pgfuseplotmark{m6c};};
\node[stars] at (axis cs:{132.633},{-27.710}) {\tikz\pgfuseplotmark{m4b};};
\node[stars] at (axis cs:{132.634},{33.285}) {\tikz\pgfuseplotmark{m6c};};
\node[stars] at (axis cs:{132.688},{18.832}) {\tikz\pgfuseplotmark{m6c};};
\node[stars] at (axis cs:{132.756},{15.351}) {\tikz\pgfuseplotmark{m6c};};
\node[stars] at (axis cs:{132.893},{-7.177}) {\tikz\pgfuseplotmark{m6ab};};
\node[stars] at (axis cs:{133.101},{5.340}) {\tikz\pgfuseplotmark{m6c};};
\node[stars] at (axis cs:{133.109},{-32.509}) {\tikz\pgfuseplotmark{m6c};};
\node[stars] at (axis cs:{133.119},{28.259}) {\tikz\pgfuseplotmark{m6cvb};};
\node[stars] at (axis cs:{133.128},{-13.233}) {\tikz\pgfuseplotmark{m6b};};
\node[stars] at (axis cs:{133.144},{32.474}) {\tikz\pgfuseplotmark{m6av};};
\node[stars] at (axis cs:{133.149},{28.331}) {\tikz\pgfuseplotmark{m6b};};
\node[stars] at (axis cs:{133.161},{-36.545}) {\tikz\pgfuseplotmark{m6cv};};
\node[stars] at (axis cs:{133.200},{-38.724}) {\tikz\pgfuseplotmark{m6a};};
\node[stars] at (axis cs:{133.272},{-16.952}) {\tikz\pgfuseplotmark{m6c};};
\node[stars] at (axis cs:{133.482},{35.538}) {\tikz\pgfuseplotmark{m6cv};};
\node[stars] at (axis cs:{133.561},{30.579}) {\tikz\pgfuseplotmark{m5cb};};
\node[stars] at (axis cs:{133.575},{-5.434}) {\tikz\pgfuseplotmark{m6bv};};
\node[stars] at (axis cs:{133.802},{-18.241}) {\tikz\pgfuseplotmark{m6ab};};
\node[stars] at (axis cs:{133.845},{17.231}) {\tikz\pgfuseplotmark{m6cv};};
\node[stars] at (axis cs:{133.848},{5.945}) {\tikz\pgfuseplotmark{m3b};};
\node[stars] at (axis cs:{133.882},{-27.682}) {\tikz\pgfuseplotmark{m5b};};
\node[stars] at (axis cs:{133.915},{27.927}) {\tikz\pgfuseplotmark{m5c};};
\node[stars] at (axis cs:{133.977},{-15.439}) {\tikz\pgfuseplotmark{m6c};};
\node[stars] at (axis cs:{133.981},{11.626}) {\tikz\pgfuseplotmark{m5c};};
\node[stars] at (axis cs:{133.983},{-23.818}) {\tikz\pgfuseplotmark{m6c};};
\node[stars] at (axis cs:{134.142},{-16.709}) {\tikz\pgfuseplotmark{m6bv};};
\node[stars] at (axis cs:{134.154},{4.237}) {\tikz\pgfuseplotmark{m6b};};
\node[stars] at (axis cs:{134.236},{32.910}) {\tikz\pgfuseplotmark{m5c};};
\node[stars] at (axis cs:{134.284},{17.144}) {\tikz\pgfuseplotmark{m6c};};
\node[stars] at (axis cs:{134.312},{15.323}) {\tikz\pgfuseplotmark{m5c};};
\node[stars] at (axis cs:{134.397},{15.581}) {\tikz\pgfuseplotmark{m6a};};
\node[stars] at (axis cs:{134.425},{9.388}) {\tikz\pgfuseplotmark{m6c};};
\node[stars] at (axis cs:{134.494},{30.234}) {\tikz\pgfuseplotmark{m6c};};
\node[stars] at (axis cs:{134.622},{11.858}) {\tikz\pgfuseplotmark{m4cv};};
\node[stars] at (axis cs:{134.683},{-16.132}) {\tikz\pgfuseplotmark{m6a};};
\node[stars] at (axis cs:{134.795},{18.135}) {\tikz\pgfuseplotmark{m6cv};};
\node[stars] at (axis cs:{134.816},{-28.806}) {\tikz\pgfuseplotmark{m6c};};
\node[stars] at (axis cs:{134.862},{13.074}) {\tikz\pgfuseplotmark{m6c};};
\node[stars] at (axis cs:{134.886},{32.419}) {\tikz\pgfuseplotmark{m5cvb};};
\node[stars] at (axis cs:{134.916},{-19.208}) {\tikz\pgfuseplotmark{m6c};};
\node[stars] at (axis cs:{135.128},{37.604}) {\tikz\pgfuseplotmark{m6c};};
\node[stars] at (axis cs:{135.298},{-26.663}) {\tikz\pgfuseplotmark{m6c};};
\node[stars] at (axis cs:{135.351},{32.252}) {\tikz\pgfuseplotmark{m6bb};};
\node[stars] at (axis cs:{135.381},{5.641}) {\tikz\pgfuseplotmark{m6b};};
\node[stars] at (axis cs:{135.420},{39.713}) {\tikz\pgfuseplotmark{m6c};};
\node[stars] at (axis cs:{135.453},{27.903}) {\tikz\pgfuseplotmark{m6bvb};};
\node[stars] at (axis cs:{135.492},{-0.482}) {\tikz\pgfuseplotmark{m6a};};
\node[stars] at (axis cs:{135.527},{-39.402}) {\tikz\pgfuseplotmark{m6c};};
\node[stars] at (axis cs:{135.684},{24.453}) {\tikz\pgfuseplotmark{m5cv};};
\node[stars] at (axis cs:{135.687},{7.298}) {\tikz\pgfuseplotmark{m6a};};
\node[stars] at (axis cs:{136.230},{32.377}) {\tikz\pgfuseplotmark{m6c};};
\node[stars] at (axis cs:{136.493},{5.092}) {\tikz\pgfuseplotmark{m5b};};
\node[stars] at (axis cs:{136.632},{38.452}) {\tikz\pgfuseplotmark{m5a};};
\node[stars] at (axis cs:{136.750},{1.462}) {\tikz\pgfuseplotmark{m6cv};};
\node[stars] at (axis cs:{136.937},{10.668}) {\tikz\pgfuseplotmark{m5cv};};
\node[stars] at (axis cs:{137.000},{29.654}) {\tikz\pgfuseplotmark{m5c};};
\node[stars] at (axis cs:{137.012},{-25.858}) {\tikz\pgfuseplotmark{m5av};};
\node[stars] at (axis cs:{137.017},{32.540}) {\tikz\pgfuseplotmark{m6c};};
\node[stars] at (axis cs:{137.176},{-8.589}) {\tikz\pgfuseplotmark{m6ab};};
\node[stars] at (axis cs:{137.181},{-26.768}) {\tikz\pgfuseplotmark{m6c};};
\node[stars] at (axis cs:{137.186},{-16.277}) {\tikz\pgfuseplotmark{m6c};};
\node[stars] at (axis cs:{137.197},{26.629}) {\tikz\pgfuseplotmark{m6bb};};
\node[stars] at (axis cs:{137.213},{33.882}) {\tikz\pgfuseplotmark{m6b};};
\node[stars] at (axis cs:{137.268},{-18.328}) {\tikz\pgfuseplotmark{m6a};};
\node[stars] at (axis cs:{137.298},{-12.358}) {\tikz\pgfuseplotmark{m6a};};
\node[stars] at (axis cs:{137.340},{22.045}) {\tikz\pgfuseplotmark{m5c};};
\node[stars] at (axis cs:{137.398},{-8.788}) {\tikz\pgfuseplotmark{m5c};};
\node[stars] at (axis cs:{137.443},{11.564}) {\tikz\pgfuseplotmark{m6c};};
\node[stars] at (axis cs:{137.485},{-30.365}) {\tikz\pgfuseplotmark{m6ab};};
\node[stars] at (axis cs:{137.587},{21.996}) {\tikz\pgfuseplotmark{m6b};};
\node[stars] at (axis cs:{137.662},{30.963}) {\tikz\pgfuseplotmark{m6bv};};
\node[stars] at (axis cs:{137.921},{-39.259}) {\tikz\pgfuseplotmark{m6b};};
\node[stars] at (axis cs:{137.982},{5.469}) {\tikz\pgfuseplotmark{m6cv};};
\node[stars] at (axis cs:{137.995},{-19.747}) {\tikz\pgfuseplotmark{m6a};};
\node[stars] at (axis cs:{138.054},{3.867}) {\tikz\pgfuseplotmark{m6c};};
\node[stars] at (axis cs:{138.109},{-7.110}) {\tikz\pgfuseplotmark{m6bv};};
\node[stars] at (axis cs:{138.340},{-29.664}) {\tikz\pgfuseplotmark{m6c};};
\node[stars] at (axis cs:{138.358},{-38.616}) {\tikz\pgfuseplotmark{m6c};};
\node[stars] at (axis cs:{138.405},{21.283}) {\tikz\pgfuseplotmark{m6c};};
\node[stars] at (axis cs:{138.513},{-14.695}) {\tikz\pgfuseplotmark{m6c};};
\node[stars] at (axis cs:{138.591},{2.314}) {\tikz\pgfuseplotmark{m4bv};};
\node[stars] at (axis cs:{138.738},{-37.602}) {\tikz\pgfuseplotmark{m6a};};
\node[stars] at (axis cs:{138.808},{14.941}) {\tikz\pgfuseplotmark{m5c};};
\node[stars] at (axis cs:{138.809},{34.634}) {\tikz\pgfuseplotmark{m6b};};
\node[stars] at (axis cs:{138.854},{-15.025}) {\tikz\pgfuseplotmark{m6cv};};
\node[stars] at (axis cs:{138.903},{-38.570}) {\tikz\pgfuseplotmark{m5b};};
\node[stars] at (axis cs:{138.938},{-37.413}) {\tikz\pgfuseplotmark{m5a};};
\node[stars] at (axis cs:{139.172},{-8.745}) {\tikz\pgfuseplotmark{m5c};};
\node[stars] at (axis cs:{139.174},{-6.353}) {\tikz\pgfuseplotmark{m5c};};
\node[stars] at (axis cs:{139.238},{-11.103}) {\tikz\pgfuseplotmark{m6c};};
\node[stars] at (axis cs:{139.238},{-39.402}) {\tikz\pgfuseplotmark{m5c};};
\node[stars] at (axis cs:{139.282},{-14.574}) {\tikz\pgfuseplotmark{m6a};};
\node[stars] at (axis cs:{139.464},{11.501}) {\tikz\pgfuseplotmark{m6c};};
\node[stars] at (axis cs:{139.608},{35.364}) {\tikz\pgfuseplotmark{m6bb};};
\node[stars] at (axis cs:{139.711},{36.803}) {\tikz\pgfuseplotmark{m4ab};};
\node[stars] at (axis cs:{139.888},{-15.835}) {\tikz\pgfuseplotmark{m6a};};
\node[stars] at (axis cs:{139.943},{-11.975}) {\tikz\pgfuseplotmark{m5a};};
\node[stars] at (axis cs:{139.950},{-34.103}) {\tikz\pgfuseplotmark{m6c};};
\node[stars] at (axis cs:{140.121},{-9.556}) {\tikz\pgfuseplotmark{m5ab};};
\node[stars] at (axis cs:{140.124},{-37.581}) {\tikz\pgfuseplotmark{m6b};};
\node[stars] at (axis cs:{140.231},{-15.618}) {\tikz\pgfuseplotmark{m6c};};
\node[stars] at (axis cs:{140.247},{38.188}) {\tikz\pgfuseplotmark{m6bb};};
\node[stars] at (axis cs:{140.264},{34.392}) {\tikz\pgfuseplotmark{m3bv};};
\node[stars] at (axis cs:{140.363},{32.902}) {\tikz\pgfuseplotmark{m6c};};
\node[stars] at (axis cs:{140.373},{-25.965}) {\tikz\pgfuseplotmark{m5av};};
\node[stars] at (axis cs:{140.801},{-28.834}) {\tikz\pgfuseplotmark{m5a};};
\node[stars] at (axis cs:{140.882},{25.183}) {\tikz\pgfuseplotmark{m6c};};
\node[stars] at (axis cs:{140.937},{-37.757}) {\tikz\pgfuseplotmark{m6c};};
\node[stars] at (axis cs:{141.068},{-39.425}) {\tikz\pgfuseplotmark{m6b};};
\node[stars] at (axis cs:{141.164},{26.182}) {\tikz\pgfuseplotmark{m4c};};
\node[stars] at (axis cs:{141.350},{-5.117}) {\tikz\pgfuseplotmark{m6av};};
\node[stars] at (axis cs:{141.385},{16.586}) {\tikz\pgfuseplotmark{m6c};};
\node[stars] at (axis cs:{141.593},{-1.464}) {\tikz\pgfuseplotmark{m6b};};
\node[stars] at (axis cs:{141.687},{-28.788}) {\tikz\pgfuseplotmark{m6bv};};
\node[stars] at (axis cs:{141.827},{-22.344}) {\tikz\pgfuseplotmark{m5a};};
\node[stars] at (axis cs:{141.897},{-8.658}) {\tikz\pgfuseplotmark{m2bvb};};
\node[stars] at (axis cs:{141.945},{-6.071}) {\tikz\pgfuseplotmark{m5cv};};
\node[stars] at (axis cs:{142.114},{9.057}) {\tikz\pgfuseplotmark{m5c};};
\node[stars] at (axis cs:{142.122},{8.188}) {\tikz\pgfuseplotmark{m6a};};
\node[stars] at (axis cs:{142.260},{-1.257}) {\tikz\pgfuseplotmark{m6c};};
\node[stars] at (axis cs:{142.287},{-2.769}) {\tikz\pgfuseplotmark{m5bvb};};
\node[stars] at (axis cs:{142.303},{-20.749}) {\tikz\pgfuseplotmark{m6a};};
\node[stars] at (axis cs:{142.311},{-35.951}) {\tikz\pgfuseplotmark{m4c};};
\node[stars] at (axis cs:{142.317},{-38.403}) {\tikz\pgfuseplotmark{m6c};};
\node[stars] at (axis cs:{142.352},{-2.205}) {\tikz\pgfuseplotmark{m6c};};
\node[stars] at (axis cs:{142.385},{-4.246}) {\tikz\pgfuseplotmark{m6c};};
\node[stars] at (axis cs:{142.458},{-23.345}) {\tikz\pgfuseplotmark{m6c};};
\node[stars] at (axis cs:{142.477},{-26.590}) {\tikz\pgfuseplotmark{m5c};};
\node[stars] at (axis cs:{142.594},{-15.577}) {\tikz\pgfuseplotmark{m6a};};
\node[stars] at (axis cs:{142.680},{33.656}) {\tikz\pgfuseplotmark{m6ab};};
\node[stars] at (axis cs:{142.692},{-31.889}) {\tikz\pgfuseplotmark{m6cb};};
\node[stars] at (axis cs:{142.884},{-31.872}) {\tikz\pgfuseplotmark{m6b};};
\node[stars] at (axis cs:{142.885},{35.103}) {\tikz\pgfuseplotmark{m5cv};};
\node[stars] at (axis cs:{142.888},{-35.715}) {\tikz\pgfuseplotmark{m6ab};};
\node[stars] at (axis cs:{142.912},{-10.552}) {\tikz\pgfuseplotmark{m6c};};
\node[stars] at (axis cs:{142.930},{22.968}) {\tikz\pgfuseplotmark{m4cv};};
\node[stars] at (axis cs:{142.982},{-10.370}) {\tikz\pgfuseplotmark{m6b};};
\node[stars] at (axis cs:{142.986},{11.300}) {\tikz\pgfuseplotmark{m5b};};
\node[stars] at (axis cs:{142.990},{9.716}) {\tikz\pgfuseplotmark{m5b};};
\node[stars] at (axis cs:{142.996},{-1.185}) {\tikz\pgfuseplotmark{m5a};};
\node[stars] at (axis cs:{143.077},{-28.628}) {\tikz\pgfuseplotmark{m6cv};};
\node[stars] at (axis cs:{143.085},{-19.400}) {\tikz\pgfuseplotmark{m6a};};
\node[stars] at (axis cs:{143.172},{1.864}) {\tikz\pgfuseplotmark{m6b};};
\node[stars] at (axis cs:{143.232},{-13.517}) {\tikz\pgfuseplotmark{m6b};};
\node[stars] at (axis cs:{143.258},{-8.505}) {\tikz\pgfuseplotmark{m6b};};
\node[stars] at (axis cs:{143.282},{-39.129}) {\tikz\pgfuseplotmark{m6cb};};
\node[stars] at (axis cs:{143.302},{-21.115}) {\tikz\pgfuseplotmark{m5b};};
\node[stars] at (axis cs:{143.333},{-7.190}) {\tikz\pgfuseplotmark{m6c};};
\node[stars] at (axis cs:{143.359},{-22.864}) {\tikz\pgfuseplotmark{m6b};};
\node[stars] at (axis cs:{143.377},{36.487}) {\tikz\pgfuseplotmark{m6c};};
\node[stars] at (axis cs:{143.496},{23.454}) {\tikz\pgfuseplotmark{m6cv};};
\node[stars] at (axis cs:{143.556},{36.397}) {\tikz\pgfuseplotmark{m5av};};
\node[stars] at (axis cs:{143.636},{-5.915}) {\tikz\pgfuseplotmark{m6a};};
\node[stars] at (axis cs:{143.766},{39.621}) {\tikz\pgfuseplotmark{m5a};};
\node[stars] at (axis cs:{143.799},{-35.824}) {\tikz\pgfuseplotmark{m6c};};
\node[stars] at (axis cs:{143.891},{-19.583}) {\tikz\pgfuseplotmark{m6cb};};
\node[stars] at (axis cs:{143.915},{35.810}) {\tikz\pgfuseplotmark{m5cv};};
\node[stars] at (axis cs:{143.970},{14.379}) {\tikz\pgfuseplotmark{m6cb};};
\node[stars] at (axis cs:{144.179},{31.162}) {\tikz\pgfuseplotmark{m6av};};
\node[stars] at (axis cs:{144.251},{-25.297}) {\tikz\pgfuseplotmark{m6a};};
\node[stars] at (axis cs:{144.261},{16.438}) {\tikz\pgfuseplotmark{m6a};};
\node[stars] at (axis cs:{144.291},{-32.178}) {\tikz\pgfuseplotmark{m6a};};
\node[stars] at (axis cs:{144.303},{6.836}) {\tikz\pgfuseplotmark{m5bv};};
\node[stars] at (axis cs:{144.368},{-36.096}) {\tikz\pgfuseplotmark{m6b};};
\node[stars] at (axis cs:{144.441},{-3.171}) {\tikz\pgfuseplotmark{m6c};};
\node[stars] at (axis cs:{144.465},{-9.424}) {\tikz\pgfuseplotmark{m6c};};
\node[stars] at (axis cs:{144.614},{4.649}) {\tikz\pgfuseplotmark{m5a};};
\node[stars] at (axis cs:{144.948},{-10.570}) {\tikz\pgfuseplotmark{m6cv};};
\node[stars] at (axis cs:{144.964},{-1.143}) {\tikz\pgfuseplotmark{m4bv};};
\node[stars] at (axis cs:{145.077},{-14.332}) {\tikz\pgfuseplotmark{m5bv};};
\node[stars] at (axis cs:{145.084},{-10.769}) {\tikz\pgfuseplotmark{m6c};};
\node[stars] at (axis cs:{145.288},{9.892}) {\tikz\pgfuseplotmark{m4a};};
\node[stars] at (axis cs:{145.321},{-23.591}) {\tikz\pgfuseplotmark{m5av};};
\node[stars] at (axis cs:{145.396},{31.278}) {\tikz\pgfuseplotmark{m6bv};};
\node[stars] at (axis cs:{145.410},{25.913}) {\tikz\pgfuseplotmark{m6c};};
\node[stars] at (axis cs:{145.501},{39.758}) {\tikz\pgfuseplotmark{m6a};};
\node[stars] at (axis cs:{145.560},{-23.915}) {\tikz\pgfuseplotmark{m5b};};
\node[stars] at (axis cs:{145.672},{-35.502}) {\tikz\pgfuseplotmark{m6c};};
\node[stars] at (axis cs:{145.678},{35.093}) {\tikz\pgfuseplotmark{m6bv};};
\node[stars] at (axis cs:{145.889},{29.974}) {\tikz\pgfuseplotmark{m6a};};
\node[stars] at (axis cs:{145.933},{14.022}) {\tikz\pgfuseplotmark{m5cv};};
\node[stars] at (axis cs:{146.050},{-27.769}) {\tikz\pgfuseplotmark{m5a};};
\node[stars] at (axis cs:{146.125},{18.863}) {\tikz\pgfuseplotmark{m6c};};
\node[stars] at (axis cs:{146.342},{-30.203}) {\tikz\pgfuseplotmark{m6cv};};
\node[stars] at (axis cs:{146.463},{23.774}) {\tikz\pgfuseplotmark{m3bv};};
\node[stars] at (axis cs:{146.542},{6.708}) {\tikz\pgfuseplotmark{m6a};};
\node[stars] at (axis cs:{146.597},{11.810}) {\tikz\pgfuseplotmark{m6a};};
\node[stars] at (axis cs:{146.598},{1.786}) {\tikz\pgfuseplotmark{m6a};};
\node[stars] at (axis cs:{146.858},{11.568}) {\tikz\pgfuseplotmark{m6c};};
\node[stars] at (axis cs:{147.367},{-37.187}) {\tikz\pgfuseplotmark{m6b};};
\node[stars] at (axis cs:{147.459},{21.179}) {\tikz\pgfuseplotmark{m6bv};};
\node[stars] at (axis cs:{147.464},{-36.268}) {\tikz\pgfuseplotmark{m6c};};
\node[stars] at (axis cs:{147.625},{4.343}) {\tikz\pgfuseplotmark{m6c};};
\node[stars] at (axis cs:{147.758},{13.066}) {\tikz\pgfuseplotmark{m6cv};};
\node[stars] at (axis cs:{147.808},{-4.243}) {\tikz\pgfuseplotmark{m6b};};
\node[stars] at (axis cs:{147.840},{-6.182}) {\tikz\pgfuseplotmark{m6c};};
\node[stars] at (axis cs:{147.870},{-14.846}) {\tikz\pgfuseplotmark{m4b};};
\node[stars] at (axis cs:{147.971},{24.395}) {\tikz\pgfuseplotmark{m5c};};
\node[stars] at (axis cs:{147.998},{-16.535}) {\tikz\pgfuseplotmark{m6b};};
\node[stars] at (axis cs:{148.050},{0.075}) {\tikz\pgfuseplotmark{m6c};};
\node[stars] at (axis cs:{148.051},{2.454}) {\tikz\pgfuseplotmark{m6bv};};
\node[stars] at (axis cs:{148.127},{-8.105}) {\tikz\pgfuseplotmark{m5b};};
\node[stars] at (axis cs:{148.191},{26.007}) {\tikz\pgfuseplotmark{m4b};};
\node[stars] at (axis cs:{148.242},{-27.332}) {\tikz\pgfuseplotmark{m6c};};
\node[stars] at (axis cs:{148.429},{5.958}) {\tikz\pgfuseplotmark{m6bv};};
\node[stars] at (axis cs:{148.551},{-25.932}) {\tikz\pgfuseplotmark{m5b};};
\node[stars] at (axis cs:{148.633},{-22.487}) {\tikz\pgfuseplotmark{m6c};};
\node[stars] at (axis cs:{148.718},{-19.009}) {\tikz\pgfuseplotmark{m5b};};
\node[stars] at (axis cs:{149.108},{8.933}) {\tikz\pgfuseplotmark{m6a};};
\node[stars] at (axis cs:{149.148},{-33.418}) {\tikz\pgfuseplotmark{m6a};};
\node[stars] at (axis cs:{149.195},{-26.549}) {\tikz\pgfuseplotmark{m6c};};
\node[stars] at (axis cs:{149.225},{-27.475}) {\tikz\pgfuseplotmark{m6c};};
\node[stars] at (axis cs:{149.433},{-1.942}) {\tikz\pgfuseplotmark{m6cb};};
\node[stars] at (axis cs:{149.532},{8.314}) {\tikz\pgfuseplotmark{m6b};};
\node[stars] at (axis cs:{149.556},{12.445}) {\tikz\pgfuseplotmark{m5c};};
\node[stars] at (axis cs:{149.608},{27.759}) {\tikz\pgfuseplotmark{m6c};};
\node[stars] at (axis cs:{149.718},{-35.891}) {\tikz\pgfuseplotmark{m5c};};
\node[stars] at (axis cs:{149.776},{-23.951}) {\tikz\pgfuseplotmark{m6cv};};
\node[stars] at (axis cs:{149.901},{29.645}) {\tikz\pgfuseplotmark{m6a};};
\node[stars] at (axis cs:{150.053},{8.044}) {\tikz\pgfuseplotmark{m5av};};
\node[stars] at (axis cs:{150.253},{31.924}) {\tikz\pgfuseplotmark{m5c};};
\node[stars] at (axis cs:{150.704},{21.949}) {\tikz\pgfuseplotmark{m6av};};
\node[stars] at (axis cs:{150.921},{-9.574}) {\tikz\pgfuseplotmark{m6b};};
\node[stars] at (axis cs:{151.012},{-18.101}) {\tikz\pgfuseplotmark{m6cb};};
\node[stars] at (axis cs:{151.035},{3.201}) {\tikz\pgfuseplotmark{m6c};};
\node[stars] at (axis cs:{151.087},{-24.285}) {\tikz\pgfuseplotmark{m6a};};
\node[stars] at (axis cs:{151.097},{-39.976}) {\tikz\pgfuseplotmark{m6c};};
\node[stars] at (axis cs:{151.281},{-13.064}) {\tikz\pgfuseplotmark{m5av};};
\node[stars] at (axis cs:{151.314},{-36.384}) {\tikz\pgfuseplotmark{m6c};};
\node[stars] at (axis cs:{151.420},{15.757}) {\tikz\pgfuseplotmark{m6c};};
\node[stars] at (axis cs:{151.698},{5.611}) {\tikz\pgfuseplotmark{m6c};};
\node[stars] at (axis cs:{151.790},{-17.142}) {\tikz\pgfuseplotmark{m6a};};
\node[stars] at (axis cs:{151.833},{16.762}) {\tikz\pgfuseplotmark{m4av};};
\node[stars] at (axis cs:{151.857},{35.245}) {\tikz\pgfuseplotmark{m4cv};};
\node[stars] at (axis cs:{151.976},{9.997}) {\tikz\pgfuseplotmark{m4c};};
\node[stars] at (axis cs:{151.985},{-0.372}) {\tikz\pgfuseplotmark{m4c};};
\node[stars] at (axis cs:{152.007},{-37.333}) {\tikz\pgfuseplotmark{m6c};};
\node[stars] at (axis cs:{152.066},{31.604}) {\tikz\pgfuseplotmark{m6c};};
\node[stars] at (axis cs:{152.093},{11.967}) {\tikz\pgfuseplotmark{m1cvb};};
\node[stars] at (axis cs:{152.148},{-15.612}) {\tikz\pgfuseplotmark{m6c};};
\node[stars] at (axis cs:{152.382},{-35.857}) {\tikz\pgfuseplotmark{m6c};};
\node[stars] at (axis cs:{152.485},{-12.095}) {\tikz\pgfuseplotmark{m6c};};
\node[stars] at (axis cs:{152.525},{-12.816}) {\tikz\pgfuseplotmark{m5c};};
\node[stars] at (axis cs:{152.531},{-8.408}) {\tikz\pgfuseplotmark{m6b};};
\node[stars] at (axis cs:{152.647},{-12.354}) {\tikz\pgfuseplotmark{m4a};};
\node[stars] at (axis cs:{152.733},{-8.418}) {\tikz\pgfuseplotmark{m6a};};
\node[stars] at (axis cs:{152.803},{37.402}) {\tikz\pgfuseplotmark{m6av};};
\node[stars] at (axis cs:{152.824},{-7.316}) {\tikz\pgfuseplotmark{m6c};};
\node[stars] at (axis cs:{152.909},{13.355}) {\tikz\pgfuseplotmark{m6c};};
\node[stars] at (axis cs:{153.012},{-28.606}) {\tikz\pgfuseplotmark{m6c};};
\node[stars] at (axis cs:{153.158},{-19.153}) {\tikz\pgfuseplotmark{m6c};};
\node[stars] at (axis cs:{153.202},{4.614}) {\tikz\pgfuseplotmark{m6a};};
\node[stars] at (axis cs:{153.331},{-27.029}) {\tikz\pgfuseplotmark{m6c};};
\node[stars] at (axis cs:{153.353},{-33.031}) {\tikz\pgfuseplotmark{m6c};};
\node[stars] at (axis cs:{153.457},{27.136}) {\tikz\pgfuseplotmark{m6b};};
\node[stars] at (axis cs:{153.537},{-23.814}) {\tikz\pgfuseplotmark{m6c};};
\node[stars] at (axis cs:{153.624},{21.168}) {\tikz\pgfuseplotmark{m6b};};
\node[stars] at (axis cs:{153.776},{31.468}) {\tikz\pgfuseplotmark{m6c};};
\node[stars] at (axis cs:{153.837},{-36.518}) {\tikz\pgfuseplotmark{m6c};};
\node[stars] at (axis cs:{154.037},{-11.203}) {\tikz\pgfuseplotmark{m6b};};
\node[stars] at (axis cs:{154.060},{29.310}) {\tikz\pgfuseplotmark{m5c};};
\node[stars] at (axis cs:{154.117},{28.682}) {\tikz\pgfuseplotmark{m6c};};
\node[stars] at (axis cs:{154.170},{13.728}) {\tikz\pgfuseplotmark{m5c};};
\node[stars] at (axis cs:{154.173},{23.417}) {\tikz\pgfuseplotmark{m3cvb};};
\node[stars] at (axis cs:{154.174},{25.371}) {\tikz\pgfuseplotmark{m6av};};
\node[stars] at (axis cs:{154.311},{23.106}) {\tikz\pgfuseplotmark{m6a};};
\node[stars] at (axis cs:{154.408},{-8.069}) {\tikz\pgfuseplotmark{m5c};};
\node[stars] at (axis cs:{154.532},{-28.992}) {\tikz\pgfuseplotmark{m6av};};
\node[stars] at (axis cs:{154.658},{-36.804}) {\tikz\pgfuseplotmark{m6c};};
\node[stars] at (axis cs:{154.753},{24.712}) {\tikz\pgfuseplotmark{m6c};};
\node[stars] at (axis cs:{154.820},{-12.528}) {\tikz\pgfuseplotmark{m6b};};
\node[stars] at (axis cs:{154.885},{-5.106}) {\tikz\pgfuseplotmark{m6c};};
\node[stars] at (axis cs:{154.934},{19.471}) {\tikz\pgfuseplotmark{m5av};};
\node[stars] at (axis cs:{154.993},{19.841}) {\tikz\pgfuseplotmark{m2cvb};};
\node[stars] at (axis cs:{154.998},{-9.059}) {\tikz\pgfuseplotmark{m6c};};
\node[stars] at (axis cs:{155.369},{-23.711}) {\tikz\pgfuseplotmark{m6c};};
\node[stars] at (axis cs:{155.460},{14.976}) {\tikz\pgfuseplotmark{m6cv};};
\node[stars] at (axis cs:{155.554},{-19.866}) {\tikz\pgfuseplotmark{m6b};};
\node[stars] at (axis cs:{155.752},{6.542}) {\tikz\pgfuseplotmark{m6b};};
\node[stars] at (axis cs:{155.776},{33.908}) {\tikz\pgfuseplotmark{m6b};};
\node[stars] at (axis cs:{155.805},{-30.162}) {\tikz\pgfuseplotmark{m6c};};
\node[stars] at (axis cs:{155.860},{-4.074}) {\tikz\pgfuseplotmark{m6bv};};
\node[stars] at (axis cs:{155.868},{-0.902}) {\tikz\pgfuseplotmark{m6c};};
\node[stars] at (axis cs:{155.872},{-38.010}) {\tikz\pgfuseplotmark{m5cv};};
\node[stars] at (axis cs:{155.886},{-3.643}) {\tikz\pgfuseplotmark{m6cv};};
\node[stars] at (axis cs:{155.924},{29.616}) {\tikz\pgfuseplotmark{m6c};};
\node[stars] at (axis cs:{156.036},{33.719}) {\tikz\pgfuseplotmark{m6a};};
\node[stars] at (axis cs:{156.055},{2.368}) {\tikz\pgfuseplotmark{m6cb};};
\node[stars] at (axis cs:{156.313},{8.785}) {\tikz\pgfuseplotmark{m6av};};
\node[stars] at (axis cs:{156.434},{-7.060}) {\tikz\pgfuseplotmark{m6a};};
\node[stars] at (axis cs:{156.437},{35.425}) {\tikz\pgfuseplotmark{m6c};};
\node[stars] at (axis cs:{156.478},{33.796}) {\tikz\pgfuseplotmark{m5a};};
\node[stars] at (axis cs:{156.523},{-16.836}) {\tikz\pgfuseplotmark{m4a};};
\node[stars] at (axis cs:{156.654},{-0.988}) {\tikz\pgfuseplotmark{m6c};};
\node[stars] at (axis cs:{156.752},{19.364}) {\tikz\pgfuseplotmark{m6c};};
\node[stars] at (axis cs:{156.788},{-31.068}) {\tikz\pgfuseplotmark{m4cv};};
\node[stars] at (axis cs:{156.912},{9.762}) {\tikz\pgfuseplotmark{m6bv};};
\node[stars] at (axis cs:{156.971},{36.707}) {\tikz\pgfuseplotmark{m4c};};
\node[stars] at (axis cs:{157.183},{-3.742}) {\tikz\pgfuseplotmark{m6b};};
\node[stars] at (axis cs:{157.370},{-2.739}) {\tikz\pgfuseplotmark{m5c};};
\node[stars] at (axis cs:{157.371},{-29.664}) {\tikz\pgfuseplotmark{m6av};};
\node[stars] at (axis cs:{157.397},{-30.607}) {\tikz\pgfuseplotmark{m6av};};
\node[stars] at (axis cs:{157.527},{38.925}) {\tikz\pgfuseplotmark{m6a};};
\node[stars] at (axis cs:{157.573},{-0.637}) {\tikz\pgfuseplotmark{m5bv};};
\node[stars] at (axis cs:{157.745},{-7.637}) {\tikz\pgfuseplotmark{m6cb};};
\node[stars] at (axis cs:{157.749},{-13.588}) {\tikz\pgfuseplotmark{m6av};};
\node[stars] at (axis cs:{157.953},{-28.237}) {\tikz\pgfuseplotmark{m6b};};
\node[stars] at (axis cs:{157.964},{32.379}) {\tikz\pgfuseplotmark{m6b};};
\node[stars] at (axis cs:{158.049},{14.137}) {\tikz\pgfuseplotmark{m5cv};};
\node[stars] at (axis cs:{158.203},{9.306}) {\tikz\pgfuseplotmark{m4av};};
\node[stars] at (axis cs:{158.379},{34.989}) {\tikz\pgfuseplotmark{m6a};};
\node[stars] at (axis cs:{158.504},{-23.745}) {\tikz\pgfuseplotmark{m5b};};
\node[stars] at (axis cs:{158.700},{6.954}) {\tikz\pgfuseplotmark{m5b};};
\node[stars] at (axis cs:{158.740},{-23.176}) {\tikz\pgfuseplotmark{m6b};};
\node[stars] at (axis cs:{158.759},{8.650}) {\tikz\pgfuseplotmark{m6avb};};
\node[stars] at (axis cs:{158.804},{-39.562}) {\tikz\pgfuseplotmark{m6av};};
\node[stars] at (axis cs:{158.912},{-18.569}) {\tikz\pgfuseplotmark{m6c};};
\node[stars] at (axis cs:{159.019},{-26.675}) {\tikz\pgfuseplotmark{m6cb};};
\node[stars] at (axis cs:{159.069},{-16.344}) {\tikz\pgfuseplotmark{m6b};};
\node[stars] at (axis cs:{159.089},{36.327}) {\tikz\pgfuseplotmark{m6c};};
\node[stars] at (axis cs:{159.135},{-12.230}) {\tikz\pgfuseplotmark{m6a};};
\node[stars] at (axis cs:{159.307},{-27.412}) {\tikz\pgfuseplotmark{m5b};};
\node[stars] at (axis cs:{159.389},{-13.385}) {\tikz\pgfuseplotmark{m5bv};};
\node[stars] at (axis cs:{159.468},{34.078}) {\tikz\pgfuseplotmark{m6c};};
\node[stars] at (axis cs:{159.646},{-16.876}) {\tikz\pgfuseplotmark{m5b};};
\node[stars] at (axis cs:{159.680},{31.976}) {\tikz\pgfuseplotmark{m5a};};
\node[stars] at (axis cs:{159.710},{-12.443}) {\tikz\pgfuseplotmark{m6b};};
\node[stars] at (axis cs:{159.782},{37.910}) {\tikz\pgfuseplotmark{m6a};};
\node[stars] at (axis cs:{160.215},{-35.742}) {\tikz\pgfuseplotmark{m6c};};
\node[stars] at (axis cs:{160.351},{-1.741}) {\tikz\pgfuseplotmark{m6c};};
\node[stars] at (axis cs:{160.547},{31.697}) {\tikz\pgfuseplotmark{m6bv};};
\node[stars] at (axis cs:{160.631},{-13.974}) {\tikz\pgfuseplotmark{m6c};};
\node[stars] at (axis cs:{160.680},{-32.715}) {\tikz\pgfuseplotmark{m6a};};
\node[stars] at (axis cs:{160.758},{26.325}) {\tikz\pgfuseplotmark{m6a};};
\node[stars] at (axis cs:{160.837},{4.747}) {\tikz\pgfuseplotmark{m6bb};};
\node[stars] at (axis cs:{160.854},{23.188}) {\tikz\pgfuseplotmark{m5b};};
\node[stars] at (axis cs:{161.061},{19.759}) {\tikz\pgfuseplotmark{m6c};};
\node[stars] at (axis cs:{161.289},{2.488}) {\tikz\pgfuseplotmark{m6c};};
\node[stars] at (axis cs:{161.466},{30.682}) {\tikz\pgfuseplotmark{m5cb};};
\node[stars] at (axis cs:{161.524},{6.373}) {\tikz\pgfuseplotmark{m6c};};
\node[stars] at (axis cs:{161.602},{18.891}) {\tikz\pgfuseplotmark{m5c};};
\node[stars] at (axis cs:{161.605},{14.194}) {\tikz\pgfuseplotmark{m5c};};
\node[stars] at (axis cs:{161.717},{-17.297}) {\tikz\pgfuseplotmark{m5c};};
\node[stars] at (axis cs:{162.059},{-31.688}) {\tikz\pgfuseplotmark{m6b};};
\node[stars] at (axis cs:{162.169},{-1.959}) {\tikz\pgfuseplotmark{m6bv};};
\node[stars] at (axis cs:{162.238},{29.416}) {\tikz\pgfuseplotmark{m6c};};
\node[stars] at (axis cs:{162.314},{10.545}) {\tikz\pgfuseplotmark{m5c};};
\node[stars] at (axis cs:{162.406},{-16.193}) {\tikz\pgfuseplotmark{m3b};};
\node[stars] at (axis cs:{162.431},{-9.853}) {\tikz\pgfuseplotmark{m6a};};
\node[stars] at (axis cs:{162.474},{27.974}) {\tikz\pgfuseplotmark{m6b};};
\node[stars] at (axis cs:{162.488},{-34.058}) {\tikz\pgfuseplotmark{m6a};};
\node[stars] at (axis cs:{162.575},{-8.898}) {\tikz\pgfuseplotmark{m6a};};
\node[stars] at (axis cs:{162.773},{-3.093}) {\tikz\pgfuseplotmark{m6b};};
\node[stars] at (axis cs:{163.057},{1.025}) {\tikz\pgfuseplotmark{m6c};};
\node[stars] at (axis cs:{163.150},{-0.202}) {\tikz\pgfuseplotmark{m6c};};
\node[stars] at (axis cs:{163.328},{34.215}) {\tikz\pgfuseplotmark{m4av};};
\node[stars] at (axis cs:{163.354},{-2.255}) {\tikz\pgfuseplotmark{m6bb};};
\node[stars] at (axis cs:{163.373},{-20.139}) {\tikz\pgfuseplotmark{m5cvb};};
\node[stars] at (axis cs:{163.387},{-15.446}) {\tikz\pgfuseplotmark{m6c};};
\node[stars] at (axis cs:{163.432},{-2.129}) {\tikz\pgfuseplotmark{m5c};};
\node[stars] at (axis cs:{163.574},{-13.758}) {\tikz\pgfuseplotmark{m6a};};
\node[stars] at (axis cs:{163.676},{25.491}) {\tikz\pgfuseplotmark{m6cv};};
\node[stars] at (axis cs:{163.742},{34.035}) {\tikz\pgfuseplotmark{m6a};};
\node[stars] at (axis cs:{163.798},{-20.665}) {\tikz\pgfuseplotmark{m6c};};
\node[stars] at (axis cs:{163.903},{24.749}) {\tikz\pgfuseplotmark{m4cb};};
\node[stars] at (axis cs:{163.927},{0.737}) {\tikz\pgfuseplotmark{m6bb};};
\node[stars] at (axis cs:{163.935},{33.507}) {\tikz\pgfuseplotmark{m5b};};
\node[stars] at (axis cs:{164.006},{6.185}) {\tikz\pgfuseplotmark{m6bv};};
\node[stars] at (axis cs:{164.070},{22.352}) {\tikz\pgfuseplotmark{m6c};};
\node[stars] at (axis cs:{164.143},{25.500}) {\tikz\pgfuseplotmark{m6c};};
\node[stars] at (axis cs:{164.179},{-37.138}) {\tikz\pgfuseplotmark{m5a};};
\node[stars] at (axis cs:{164.807},{-33.738}) {\tikz\pgfuseplotmark{m6ab};};
\node[stars] at (axis cs:{164.879},{-16.354}) {\tikz\pgfuseplotmark{m6b};};
\node[stars] at (axis cs:{164.887},{36.093}) {\tikz\pgfuseplotmark{m6bv};};
\node[stars] at (axis cs:{164.944},{-18.299}) {\tikz\pgfuseplotmark{m4b};};
\node[stars] at (axis cs:{165.049},{-14.083}) {\tikz\pgfuseplotmark{m6a};};
\node[stars] at (axis cs:{165.140},{3.617}) {\tikz\pgfuseplotmark{m5a};};
\node[stars] at (axis cs:{165.170},{-31.839}) {\tikz\pgfuseplotmark{m6b};};
\node[stars] at (axis cs:{165.187},{6.101}) {\tikz\pgfuseplotmark{m5b};};
\node[stars] at (axis cs:{165.210},{39.212}) {\tikz\pgfuseplotmark{m5b};};
\node[stars] at (axis cs:{165.239},{-15.793}) {\tikz\pgfuseplotmark{m6c};};
\node[stars] at (axis cs:{165.457},{-2.485}) {\tikz\pgfuseplotmark{m5av};};
\node[stars] at (axis cs:{165.582},{20.180}) {\tikz\pgfuseplotmark{m4cv};};
\node[stars] at (axis cs:{165.602},{-26.831}) {\tikz\pgfuseplotmark{m6c};};
\node[stars] at (axis cs:{165.811},{-0.752}) {\tikz\pgfuseplotmark{m6b};};
\node[stars] at (axis cs:{165.812},{-11.303}) {\tikz\pgfuseplotmark{m6a};};
\node[stars] at (axis cs:{165.817},{-31.961}) {\tikz\pgfuseplotmark{m6c};};
\node[stars] at (axis cs:{165.902},{-13.435}) {\tikz\pgfuseplotmark{m6c};};
\node[stars] at (axis cs:{165.902},{-0.001}) {\tikz\pgfuseplotmark{m6b};};
\node[stars] at (axis cs:{166.130},{38.241}) {\tikz\pgfuseplotmark{m6bb};};
\node[stars] at (axis cs:{166.226},{-35.805}) {\tikz\pgfuseplotmark{m5c};};
\node[stars] at (axis cs:{166.254},{7.336}) {\tikz\pgfuseplotmark{m5a};};
\node[stars] at (axis cs:{166.333},{-27.293}) {\tikz\pgfuseplotmark{m5bb};};
\node[stars] at (axis cs:{166.392},{-11.089}) {\tikz\pgfuseplotmark{m6b};};
\node[stars] at (axis cs:{166.490},{-27.288}) {\tikz\pgfuseplotmark{m6av};};
\node[stars] at (axis cs:{166.683},{17.737}) {\tikz\pgfuseplotmark{m6c};};
\node[stars] at (axis cs:{166.726},{1.955}) {\tikz\pgfuseplotmark{m6av};};
\node[stars] at (axis cs:{166.915},{23.324}) {\tikz\pgfuseplotmark{m6c};};
\node[stars] at (axis cs:{167.066},{-29.972}) {\tikz\pgfuseplotmark{m6c};};
\node[stars] at (axis cs:{167.183},{-28.080}) {\tikz\pgfuseplotmark{m5c};};
\node[stars] at (axis cs:{167.205},{24.658}) {\tikz\pgfuseplotmark{m6a};};
\node[stars] at (axis cs:{167.330},{36.309}) {\tikz\pgfuseplotmark{m6av};};
\node[stars] at (axis cs:{167.472},{-32.368}) {\tikz\pgfuseplotmark{m6a};};
\node[stars] at (axis cs:{167.915},{-22.826}) {\tikz\pgfuseplotmark{m4c};};
\node[stars] at (axis cs:{167.932},{14.400}) {\tikz\pgfuseplotmark{m6c};};
\node[stars] at (axis cs:{167.991},{-26.806}) {\tikz\pgfuseplotmark{m6c};};
\node[stars] at (axis cs:{168.062},{-32.434}) {\tikz\pgfuseplotmark{m6cv};};
\node[stars] at (axis cs:{168.127},{-18.500}) {\tikz\pgfuseplotmark{m6b};};
\node[stars] at (axis cs:{168.135},{35.814}) {\tikz\pgfuseplotmark{m6cb};};
\node[stars] at (axis cs:{168.145},{-21.749}) {\tikz\pgfuseplotmark{m6c};};
\node[stars] at (axis cs:{168.440},{-0.069}) {\tikz\pgfuseplotmark{m5c};};
\node[stars] at (axis cs:{168.508},{8.061}) {\tikz\pgfuseplotmark{m6a};};
\node[stars] at (axis cs:{168.527},{20.524}) {\tikz\pgfuseplotmark{m3av};};
\node[stars] at (axis cs:{168.560},{15.429}) {\tikz\pgfuseplotmark{m3cv};};
\node[stars] at (axis cs:{168.801},{23.095}) {\tikz\pgfuseplotmark{m5av};};
\node[stars] at (axis cs:{168.966},{13.307}) {\tikz\pgfuseplotmark{m5c};};
\node[stars] at (axis cs:{169.165},{-3.652}) {\tikz\pgfuseplotmark{m4c};};
\node[stars] at (axis cs:{169.242},{-7.135}) {\tikz\pgfuseplotmark{m6bvb};};
\node[stars] at (axis cs:{169.300},{-38.014}) {\tikz\pgfuseplotmark{m6c};};
\node[stars] at (axis cs:{169.323},{2.010}) {\tikz\pgfuseplotmark{m5c};};
\node[stars] at (axis cs:{169.412},{-34.737}) {\tikz\pgfuseplotmark{m6c};};
\node[stars] at (axis cs:{169.546},{31.529}) {\tikz\pgfuseplotmark{m4avb};};
\node[stars] at (axis cs:{169.620},{33.094}) {\tikz\pgfuseplotmark{m3c};};
\node[stars] at (axis cs:{169.729},{1.650}) {\tikz\pgfuseplotmark{m6b};};
\node[stars] at (axis cs:{169.783},{38.186}) {\tikz\pgfuseplotmark{m5av};};
\node[stars] at (axis cs:{169.835},{-14.778}) {\tikz\pgfuseplotmark{m4a};};
\node[stars] at (axis cs:{170.284},{6.029}) {\tikz\pgfuseplotmark{m4b};};
\node[stars] at (axis cs:{170.803},{-36.165}) {\tikz\pgfuseplotmark{m5b};};
\node[stars] at (axis cs:{170.825},{0.132}) {\tikz\pgfuseplotmark{m6b};};
\node[stars] at (axis cs:{170.841},{-18.780}) {\tikz\pgfuseplotmark{m5b};};
\node[stars] at (axis cs:{170.981},{10.529}) {\tikz\pgfuseplotmark{m4bvb};};
\node[stars] at (axis cs:{171.010},{1.408}) {\tikz\pgfuseplotmark{m5c};};
\node[stars] at (axis cs:{171.152},{-10.859}) {\tikz\pgfuseplotmark{m5a};};
\node[stars] at (axis cs:{171.221},{-17.684}) {\tikz\pgfuseplotmark{m4bb};};
\node[stars] at (axis cs:{171.246},{11.430}) {\tikz\pgfuseplotmark{m6a};};
\node[stars] at (axis cs:{171.373},{-36.063}) {\tikz\pgfuseplotmark{m5c};};
\node[stars] at (axis cs:{171.388},{-37.747}) {\tikz\pgfuseplotmark{m6b};};
\node[stars] at (axis cs:{171.402},{16.456}) {\tikz\pgfuseplotmark{m6ab};};
\node[stars] at (axis cs:{171.459},{3.860}) {\tikz\pgfuseplotmark{m6c};};
\node[stars] at (axis cs:{171.606},{33.451}) {\tikz\pgfuseplotmark{m6c};};
\node[stars] at (axis cs:{171.689},{3.013}) {\tikz\pgfuseplotmark{m6cb};};
\node[stars] at (axis cs:{171.790},{-12.357}) {\tikz\pgfuseplotmark{m6b};};
\node[stars] at (axis cs:{171.974},{-1.700}) {\tikz\pgfuseplotmark{m6c};};
\node[stars] at (axis cs:{171.984},{2.856}) {\tikz\pgfuseplotmark{m5bb};};
\node[stars] at (axis cs:{171.994},{-35.329}) {\tikz\pgfuseplotmark{m6cv};};
\node[stars] at (axis cs:{172.267},{39.337}) {\tikz\pgfuseplotmark{m5cb};};
\node[stars] at (axis cs:{172.411},{-24.464}) {\tikz\pgfuseplotmark{m6ab};};
\node[stars] at (axis cs:{172.424},{15.413}) {\tikz\pgfuseplotmark{m6a};};
\node[stars] at (axis cs:{172.579},{-3.003}) {\tikz\pgfuseplotmark{m5a};};
\node[stars] at (axis cs:{172.621},{18.410}) {\tikz\pgfuseplotmark{m6a};};
\node[stars] at (axis cs:{172.937},{14.364}) {\tikz\pgfuseplotmark{m6cb};};
\node[stars] at (axis cs:{172.948},{-20.776}) {\tikz\pgfuseplotmark{m6c};};
\node[stars] at (axis cs:{173.068},{-29.261}) {\tikz\pgfuseplotmark{m6ab};};
\node[stars] at (axis cs:{173.097},{-26.747}) {\tikz\pgfuseplotmark{m6c};};
\node[stars] at (axis cs:{173.198},{-7.827}) {\tikz\pgfuseplotmark{m6b};};
\node[stars] at (axis cs:{173.226},{-31.087}) {\tikz\pgfuseplotmark{m5cv};};
\node[stars] at (axis cs:{173.250},{-31.858}) {\tikz\pgfuseplotmark{m4a};};
\node[stars] at (axis cs:{173.312},{-16.280}) {\tikz\pgfuseplotmark{m6b};};
\node[stars] at (axis cs:{173.401},{2.499}) {\tikz\pgfuseplotmark{m6c};};
\node[stars] at (axis cs:{173.485},{36.815}) {\tikz\pgfuseplotmark{m6c};};
\node[stars] at (axis cs:{173.591},{3.060}) {\tikz\pgfuseplotmark{m6a};};
\node[stars] at (axis cs:{173.623},{-32.831}) {\tikz\pgfuseplotmark{m6bv};};
\node[stars] at (axis cs:{173.677},{16.797}) {\tikz\pgfuseplotmark{m6bb};};
\node[stars] at (axis cs:{173.745},{-4.361}) {\tikz\pgfuseplotmark{m6c};};
\node[stars] at (axis cs:{173.766},{20.441}) {\tikz\pgfuseplotmark{m6c};};
\node[stars] at (axis cs:{174.075},{27.781}) {\tikz\pgfuseplotmark{m6a};};
\node[stars] at (axis cs:{174.146},{-33.570}) {\tikz\pgfuseplotmark{m6ab};};
\node[stars] at (axis cs:{174.170},{-37.237}) {\tikz\pgfuseplotmark{m6c};};
\node[stars] at (axis cs:{174.170},{-9.802}) {\tikz\pgfuseplotmark{m5a};};
\node[stars] at (axis cs:{174.237},{-0.824}) {\tikz\pgfuseplotmark{m4c};};
\node[stars] at (axis cs:{174.255},{-32.988}) {\tikz\pgfuseplotmark{m6c};};
\node[stars] at (axis cs:{174.541},{8.884}) {\tikz\pgfuseplotmark{m6cv};};
\node[stars] at (axis cs:{174.600},{-2.436}) {\tikz\pgfuseplotmark{m6cb};};
\node[stars] at (axis cs:{174.615},{8.134}) {\tikz\pgfuseplotmark{m5cv};};
\node[stars] at (axis cs:{174.635},{33.626}) {\tikz\pgfuseplotmark{m6c};};
\node[stars] at (axis cs:{174.667},{-13.202}) {\tikz\pgfuseplotmark{m5c};};
\node[stars] at (axis cs:{174.752},{-24.721}) {\tikz\pgfuseplotmark{m6c};};
\node[stars] at (axis cs:{174.960},{-16.620}) {\tikz\pgfuseplotmark{m6cv};};
\node[stars] at (axis cs:{174.963},{-14.468}) {\tikz\pgfuseplotmark{m6c};};
\node[stars] at (axis cs:{175.053},{-34.744}) {\tikz\pgfuseplotmark{m5a};};
\node[stars] at (axis cs:{175.196},{21.353}) {\tikz\pgfuseplotmark{m5c};};
\node[stars] at (axis cs:{175.263},{34.202}) {\tikz\pgfuseplotmark{m5cv};};
\node[stars] at (axis cs:{175.285},{-29.196}) {\tikz\pgfuseplotmark{m6c};};
\node[stars] at (axis cs:{175.393},{31.746}) {\tikz\pgfuseplotmark{m6ab};};
\node[stars] at (axis cs:{175.433},{-32.499}) {\tikz\pgfuseplotmark{m5cb};};
\node[stars] at (axis cs:{175.515},{-20.294}) {\tikz\pgfuseplotmark{m6c};};
\node[stars] at (axis cs:{175.863},{-37.190}) {\tikz\pgfuseplotmark{m6b};};
\node[stars] at (axis cs:{175.980},{-6.677}) {\tikz\pgfuseplotmark{m6b};};
\node[stars] at (axis cs:{176.055},{25.218}) {\tikz\pgfuseplotmark{m6b};};
\node[stars] at (axis cs:{176.191},{-18.351}) {\tikz\pgfuseplotmark{m5a};};
\node[stars] at (axis cs:{176.321},{8.258}) {\tikz\pgfuseplotmark{m5a};};
\node[stars] at (axis cs:{176.465},{6.529}) {\tikz\pgfuseplotmark{m4bv};};
\node[stars] at (axis cs:{176.779},{-35.907}) {\tikz\pgfuseplotmark{m6c};};
\node[stars] at (axis cs:{176.816},{-30.286}) {\tikz\pgfuseplotmark{m6c};};
\node[stars] at (axis cs:{176.979},{8.246}) {\tikz\pgfuseplotmark{m5c};};
\node[stars] at (axis cs:{176.996},{20.219}) {\tikz\pgfuseplotmark{m5av};};
\node[stars] at (axis cs:{177.098},{-10.313}) {\tikz\pgfuseplotmark{m6cb};};
\node[stars] at (axis cs:{177.161},{14.284}) {\tikz\pgfuseplotmark{m6bv};};
\node[stars] at (axis cs:{177.188},{-26.750}) {\tikz\pgfuseplotmark{m5bv};};
\node[stars] at (axis cs:{177.255},{-0.319}) {\tikz\pgfuseplotmark{m6c};};
\node[stars] at (axis cs:{177.265},{14.572}) {\tikz\pgfuseplotmark{m2bv};};
\node[stars] at (axis cs:{177.312},{16.243}) {\tikz\pgfuseplotmark{m6b};};
\node[stars] at (axis cs:{177.424},{34.932}) {\tikz\pgfuseplotmark{m6a};};
\node[stars] at (axis cs:{177.582},{-15.863}) {\tikz\pgfuseplotmark{m6b};};
\node[stars] at (axis cs:{177.655},{-27.278}) {\tikz\pgfuseplotmark{m6c};};
\node[stars] at (axis cs:{177.674},{1.764}) {\tikz\pgfuseplotmark{m4a};};
\node[stars] at (axis cs:{177.731},{12.279}) {\tikz\pgfuseplotmark{m6c};};
\node[stars] at (axis cs:{177.759},{-5.333}) {\tikz\pgfuseplotmark{m6a};};
\node[stars] at (axis cs:{177.790},{33.375}) {\tikz\pgfuseplotmark{m6cb};};
\node[stars] at (axis cs:{177.841},{-12.188}) {\tikz\pgfuseplotmark{m6c};};
\node[stars] at (axis cs:{177.923},{-30.835}) {\tikz\pgfuseplotmark{m6a};};
\node[stars] at (axis cs:{178.227},{-33.908}) {\tikz\pgfuseplotmark{m4cv};};
\node[stars] at (axis cs:{178.245},{37.719}) {\tikz\pgfuseplotmark{m6cv};};
\node[stars] at (axis cs:{178.362},{-35.066}) {\tikz\pgfuseplotmark{m6c};};
\node[stars] at (axis cs:{178.459},{0.552}) {\tikz\pgfuseplotmark{m6cv};};
\node[stars] at (axis cs:{178.608},{-37.749}) {\tikz\pgfuseplotmark{m6c};};
\node[stars] at (axis cs:{178.677},{-25.714}) {\tikz\pgfuseplotmark{m5c};};
\node[stars] at (axis cs:{178.763},{8.444}) {\tikz\pgfuseplotmark{m6a};};
\node[stars] at (axis cs:{178.809},{36.756}) {\tikz\pgfuseplotmark{m6cv};};
\node[stars] at (axis cs:{178.917},{-28.477}) {\tikz\pgfuseplotmark{m6b};};
\node[stars] at (axis cs:{178.919},{15.647}) {\tikz\pgfuseplotmark{m6ab};};
\node[stars] at (axis cs:{178.978},{-39.689}) {\tikz\pgfuseplotmark{m6b};};
\node[stars] at (axis cs:{179.004},{-17.151}) {\tikz\pgfuseplotmark{m5c};};
\node[stars] at (axis cs:{179.266},{-33.315}) {\tikz\pgfuseplotmark{m6cv};};
\node[stars] at (axis cs:{179.530},{32.274}) {\tikz\pgfuseplotmark{m6cv};};
\node[stars] at (axis cs:{179.727},{-25.909}) {\tikz\pgfuseplotmark{m6c};};
\node[stars] at (axis cs:{179.764},{0.530}) {\tikz\pgfuseplotmark{m6c};};
\node[stars] at (axis cs:{179.789},{-10.476}) {\tikz\pgfuseplotmark{m6c};};
\node[stars] at (axis cs:{179.823},{33.167}) {\tikz\pgfuseplotmark{m6b};};
\node[stars] at (axis cs:{179.987},{3.655}) {\tikz\pgfuseplotmark{m5c};};
\node[stars] at (axis cs:{179.989},{34.035}) {\tikz\pgfuseplotmark{m6c};};
\node[stars] at (axis cs:{180.176},{-21.837}) {\tikz\pgfuseplotmark{m6c};};
\node[stars] at (axis cs:{180.185},{-10.446}) {\tikz\pgfuseplotmark{m6a};};
\node[stars] at (axis cs:{180.213},{-19.659}) {\tikz\pgfuseplotmark{m5cv};};
\node[stars] at (axis cs:{180.218},{6.614}) {\tikz\pgfuseplotmark{m5a};};
\node[stars] at (axis cs:{180.257},{-1.768}) {\tikz\pgfuseplotmark{m6c};};
\node[stars] at (axis cs:{180.414},{36.042}) {\tikz\pgfuseplotmark{m6a};};
\node[stars] at (axis cs:{180.715},{-7.684}) {\tikz\pgfuseplotmark{m6c};};
\node[stars] at (axis cs:{180.935},{5.558}) {\tikz\pgfuseplotmark{m6c};};
\node[stars] at (axis cs:{181.069},{21.459}) {\tikz\pgfuseplotmark{m6bb};};
\node[stars] at (axis cs:{181.302},{8.733}) {\tikz\pgfuseplotmark{m4b};};
\node[stars] at (axis cs:{181.486},{-35.694}) {\tikz\pgfuseplotmark{m6cv};};
\node[stars] at (axis cs:{181.499},{-3.131}) {\tikz\pgfuseplotmark{m6c};};
\node[stars] at (axis cs:{182.103},{-24.729}) {\tikz\pgfuseplotmark{m4b};};
\node[stars] at (axis cs:{182.422},{1.898}) {\tikz\pgfuseplotmark{m6b};};
\node[stars] at (axis cs:{182.511},{-34.705}) {\tikz\pgfuseplotmark{m6cb};};
\node[stars] at (axis cs:{182.514},{5.807}) {\tikz\pgfuseplotmark{m6a};};
\node[stars] at (axis cs:{182.531},{-22.620}) {\tikz\pgfuseplotmark{m3bv};};
\node[stars] at (axis cs:{182.632},{16.809}) {\tikz\pgfuseplotmark{m6cv};};
\node[stars] at (axis cs:{182.641},{-37.870}) {\tikz\pgfuseplotmark{m6b};};
\node[stars] at (axis cs:{182.692},{27.281}) {\tikz\pgfuseplotmark{m6bv};};
\node[stars] at (axis cs:{182.766},{-23.602}) {\tikz\pgfuseplotmark{m5c};};
\node[stars] at (axis cs:{182.963},{25.870}) {\tikz\pgfuseplotmark{m6a};};
\node[stars] at (axis cs:{183.004},{28.536}) {\tikz\pgfuseplotmark{m6c};};
\node[stars] at (axis cs:{183.039},{20.542}) {\tikz\pgfuseplotmark{m6a};};
\node[stars] at (axis cs:{183.304},{-34.125}) {\tikz\pgfuseplotmark{m6cv};};
\node[stars] at (axis cs:{183.354},{-38.929}) {\tikz\pgfuseplotmark{m6a};};
\node[stars] at (axis cs:{183.358},{10.262}) {\tikz\pgfuseplotmark{m6a};};
\node[stars] at (axis cs:{183.403},{-33.793}) {\tikz\pgfuseplotmark{m6cb};};
\node[stars] at (axis cs:{183.748},{-20.844}) {\tikz\pgfuseplotmark{m6a};};
\node[stars] at (axis cs:{183.794},{-10.312}) {\tikz\pgfuseplotmark{m6b};};
\node[stars] at (axis cs:{183.952},{-17.542}) {\tikz\pgfuseplotmark{m3av};};
\node[stars] at (axis cs:{184.001},{14.899}) {\tikz\pgfuseplotmark{m5b};};
\node[stars] at (axis cs:{184.086},{23.945}) {\tikz\pgfuseplotmark{m5b};};
\node[stars] at (axis cs:{184.126},{33.061}) {\tikz\pgfuseplotmark{m5b};};
\node[stars] at (axis cs:{184.264},{-16.694}) {\tikz\pgfuseplotmark{m6bv};};
\node[stars] at (axis cs:{184.377},{28.937}) {\tikz\pgfuseplotmark{m6a};};
\node[stars] at (axis cs:{184.434},{15.145}) {\tikz\pgfuseplotmark{m6c};};
\node[stars] at (axis cs:{184.447},{-36.094}) {\tikz\pgfuseplotmark{m6c};};
\node[stars] at (axis cs:{184.632},{30.249}) {\tikz\pgfuseplotmark{m6c};};
\node[stars] at (axis cs:{184.668},{-0.787}) {\tikz\pgfuseplotmark{m6b};};
\node[stars] at (axis cs:{184.758},{26.008}) {\tikz\pgfuseplotmark{m6cv};};
\node[stars] at (axis cs:{184.830},{23.035}) {\tikz\pgfuseplotmark{m6c};};
\node[stars] at (axis cs:{184.873},{28.157}) {\tikz\pgfuseplotmark{m6c};};
\node[stars] at (axis cs:{184.976},{-0.667}) {\tikz\pgfuseplotmark{m4bv};};
\node[stars] at (axis cs:{185.045},{-22.176}) {\tikz\pgfuseplotmark{m6b};};
\node[stars] at (axis cs:{185.074},{26.002}) {\tikz\pgfuseplotmark{m6cv};};
\node[stars] at (axis cs:{185.082},{26.619}) {\tikz\pgfuseplotmark{m6av};};
\node[stars] at (axis cs:{185.087},{3.312}) {\tikz\pgfuseplotmark{m5bv};};
\node[stars] at (axis cs:{185.140},{-22.216}) {\tikz\pgfuseplotmark{m5cvb};};
\node[stars] at (axis cs:{185.179},{17.793}) {\tikz\pgfuseplotmark{m5a};};
\node[stars] at (axis cs:{185.203},{15.541}) {\tikz\pgfuseplotmark{m6c};};
\node[stars] at (axis cs:{185.232},{-13.565}) {\tikz\pgfuseplotmark{m5cv};};
\node[stars] at (axis cs:{185.545},{24.774}) {\tikz\pgfuseplotmark{m6c};};
\node[stars] at (axis cs:{185.626},{25.846}) {\tikz\pgfuseplotmark{m5av};};
\node[stars] at (axis cs:{185.633},{5.305}) {\tikz\pgfuseplotmark{m6cb};};
\node[stars] at (axis cs:{185.814},{-4.974}) {\tikz\pgfuseplotmark{m6c};};
\node[stars] at (axis cs:{185.829},{-11.812}) {\tikz\pgfuseplotmark{m6cv};};
\node[stars] at (axis cs:{185.840},{-24.841}) {\tikz\pgfuseplotmark{m6a};};
\node[stars] at (axis cs:{185.898},{-35.412}) {\tikz\pgfuseplotmark{m5c};};
\node[stars] at (axis cs:{185.903},{-39.302}) {\tikz\pgfuseplotmark{m6c};};
\node[stars] at (axis cs:{185.936},{-38.911}) {\tikz\pgfuseplotmark{m6a};};
\node[stars] at (axis cs:{185.949},{-30.335}) {\tikz\pgfuseplotmark{m6c};};
\node[stars] at (axis cs:{186.077},{26.098}) {\tikz\pgfuseplotmark{m5cv};};
\node[stars] at (axis cs:{186.299},{-11.610}) {\tikz\pgfuseplotmark{m6b};};
\node[stars] at (axis cs:{186.313},{23.926}) {\tikz\pgfuseplotmark{m6b};};
\node[stars] at (axis cs:{186.327},{-27.749}) {\tikz\pgfuseplotmark{m6b};};
\node[stars] at (axis cs:{186.341},{-35.186}) {\tikz\pgfuseplotmark{m6av};};
\node[stars] at (axis cs:{186.462},{39.019}) {\tikz\pgfuseplotmark{m5b};};
\node[stars] at (axis cs:{186.600},{27.268}) {\tikz\pgfuseplotmark{m5b};};
\node[stars] at (axis cs:{186.715},{-32.830}) {\tikz\pgfuseplotmark{m6a};};
\node[stars] at (axis cs:{186.734},{28.268}) {\tikz\pgfuseplotmark{m4c};};
\node[stars] at (axis cs:{186.747},{26.826}) {\tikz\pgfuseplotmark{m5b};};
\node[stars] at (axis cs:{186.925},{8.610}) {\tikz\pgfuseplotmark{m6c};};
\node[stars] at (axis cs:{186.956},{-16.632}) {\tikz\pgfuseplotmark{m6c};};
\node[stars] at (axis cs:{186.965},{-4.615}) {\tikz\pgfuseplotmark{m6cv};};
\node[stars] at (axis cs:{187.094},{-39.041}) {\tikz\pgfuseplotmark{m5c};};
\node[stars] at (axis cs:{187.228},{25.913}) {\tikz\pgfuseplotmark{m5cvb};};
\node[stars] at (axis cs:{187.363},{24.109}) {\tikz\pgfuseplotmark{m5c};};
\node[stars] at (axis cs:{187.430},{20.896}) {\tikz\pgfuseplotmark{m6a};};
\node[stars] at (axis cs:{187.466},{-16.515}) {\tikz\pgfuseplotmark{m3bv};};
\node[stars] at (axis cs:{187.520},{-13.393}) {\tikz\pgfuseplotmark{m6cb};};
\node[stars] at (axis cs:{187.573},{-23.696}) {\tikz\pgfuseplotmark{m6a};};
\node[stars] at (axis cs:{187.752},{24.567}) {\tikz\pgfuseplotmark{m5cv};};
\node[stars] at (axis cs:{187.839},{7.604}) {\tikz\pgfuseplotmark{m6b};};
\node[stars] at (axis cs:{187.911},{-5.053}) {\tikz\pgfuseplotmark{m6c};};
\node[stars] at (axis cs:{188.018},{-16.196}) {\tikz\pgfuseplotmark{m4cv};};
\node[stars] at (axis cs:{188.018},{-32.534}) {\tikz\pgfuseplotmark{m6c};};
\node[stars] at (axis cs:{188.137},{-21.212}) {\tikz\pgfuseplotmark{m6c};};
\node[stars] at (axis cs:{188.150},{-13.859}) {\tikz\pgfuseplotmark{m6a};};
\node[stars] at (axis cs:{188.262},{10.295}) {\tikz\pgfuseplotmark{m6c};};
\node[stars] at (axis cs:{188.343},{-19.792}) {\tikz\pgfuseplotmark{m6cv};};
\node[stars] at (axis cs:{188.393},{24.283}) {\tikz\pgfuseplotmark{m6c};};
\node[stars] at (axis cs:{188.393},{-12.830}) {\tikz\pgfuseplotmark{m6a};};
\node[stars] at (axis cs:{188.412},{33.247}) {\tikz\pgfuseplotmark{m5c};};
\node[stars] at (axis cs:{188.445},{-9.452}) {\tikz\pgfuseplotmark{m5c};};
\node[stars] at (axis cs:{188.448},{33.385}) {\tikz\pgfuseplotmark{m6c};};
\node[stars] at (axis cs:{188.597},{-23.397}) {\tikz\pgfuseplotmark{m3av};};
\node[stars] at (axis cs:{188.713},{22.629}) {\tikz\pgfuseplotmark{m5a};};
\node[stars] at (axis cs:{188.782},{18.377}) {\tikz\pgfuseplotmark{m5bvb};};
\node[stars] at (axis cs:{188.784},{21.881}) {\tikz\pgfuseplotmark{m6a};};
\node[stars] at (axis cs:{188.995},{-20.527}) {\tikz\pgfuseplotmark{m6cv};};
\node[stars] at (axis cs:{189.004},{-39.869}) {\tikz\pgfuseplotmark{m6a};};
\node[stars] at (axis cs:{189.197},{-5.832}) {\tikz\pgfuseplotmark{m6b};};
\node[stars] at (axis cs:{189.243},{17.089}) {\tikz\pgfuseplotmark{m6a};};
\node[stars] at (axis cs:{189.426},{-27.139}) {\tikz\pgfuseplotmark{m5c};};
\node[stars] at (axis cs:{189.518},{3.282}) {\tikz\pgfuseplotmark{m6c};};
\node[stars] at (axis cs:{189.593},{1.855}) {\tikz\pgfuseplotmark{m6av};};
\node[stars] at (axis cs:{189.686},{-18.250}) {\tikz\pgfuseplotmark{m6b};};
\node[stars] at (axis cs:{189.759},{22.659}) {\tikz\pgfuseplotmark{m6c};};
\node[stars] at (axis cs:{189.764},{-30.422}) {\tikz\pgfuseplotmark{m6a};};
\node[stars] at (axis cs:{189.780},{21.062}) {\tikz\pgfuseplotmark{m5c};};
\node[stars] at (axis cs:{189.812},{-7.995}) {\tikz\pgfuseplotmark{m5ab};};
\node[stars] at (axis cs:{189.820},{35.952}) {\tikz\pgfuseplotmark{m6cv};};
\node[stars] at (axis cs:{189.969},{-39.987}) {\tikz\pgfuseplotmark{m5av};};
\node[stars] at (axis cs:{190.318},{-13.015}) {\tikz\pgfuseplotmark{m6bvb};};
\node[stars] at (axis cs:{190.393},{10.426}) {\tikz\pgfuseplotmark{m6cvb};};
\node[stars] at (axis cs:{190.415},{-1.449}) {\tikz\pgfuseplotmark{m3cvb};};
\node[stars] at (axis cs:{190.456},{-19.759}) {\tikz\pgfuseplotmark{m6b};};
\node[stars] at (axis cs:{190.471},{10.236}) {\tikz\pgfuseplotmark{m5bv};};
\node[stars] at (axis cs:{190.488},{6.807}) {\tikz\pgfuseplotmark{m6av};};
\node[stars] at (axis cs:{190.909},{-1.577}) {\tikz\pgfuseplotmark{m6b};};
\node[stars] at (axis cs:{190.994},{-36.349}) {\tikz\pgfuseplotmark{m6c};};
\node[stars] at (axis cs:{191.002},{-28.324}) {\tikz\pgfuseplotmark{m5c};};
\node[stars] at (axis cs:{191.248},{39.279}) {\tikz\pgfuseplotmark{m6b};};
\node[stars] at (axis cs:{191.404},{7.673}) {\tikz\pgfuseplotmark{m5cv};};
\node[stars] at (axis cs:{191.594},{9.540}) {\tikz\pgfuseplotmark{m6ab};};
\node[stars] at (axis cs:{191.662},{16.578}) {\tikz\pgfuseplotmark{m5b};};
\node[stars] at (axis cs:{191.693},{-33.315}) {\tikz\pgfuseplotmark{m6b};};
\node[stars] at (axis cs:{191.760},{5.950}) {\tikz\pgfuseplotmark{m6cv};};
\node[stars] at (axis cs:{191.807},{11.958}) {\tikz\pgfuseplotmark{m6bb};};
\node[stars] at (axis cs:{191.889},{-6.302}) {\tikz\pgfuseplotmark{m6c};};
\node[stars] at (axis cs:{191.964},{3.573}) {\tikz\pgfuseplotmark{m6cv};};
\node[stars] at (axis cs:{191.974},{-24.852}) {\tikz\pgfuseplotmark{m6c};};
\node[stars] at (axis cs:{192.060},{13.553}) {\tikz\pgfuseplotmark{m6c};};
\node[stars] at (axis cs:{192.109},{-27.597}) {\tikz\pgfuseplotmark{m6a};};
\node[stars] at (axis cs:{192.196},{24.840}) {\tikz\pgfuseplotmark{m6c};};
\node[stars] at (axis cs:{192.226},{14.122}) {\tikz\pgfuseplotmark{m6a};};
\node[stars] at (axis cs:{192.323},{27.552}) {\tikz\pgfuseplotmark{m6a};};
\node[stars] at (axis cs:{192.545},{37.517}) {\tikz\pgfuseplotmark{m6bv};};
\node[stars] at (axis cs:{192.572},{22.863}) {\tikz\pgfuseplotmark{m6c};};
\node[stars] at (axis cs:{192.672},{-33.999}) {\tikz\pgfuseplotmark{m5b};};
\node[stars] at (axis cs:{192.846},{-10.338}) {\tikz\pgfuseplotmark{m6cvb};};
\node[stars] at (axis cs:{192.904},{3.057}) {\tikz\pgfuseplotmark{m6b};};
\node[stars] at (axis cs:{192.925},{27.541}) {\tikz\pgfuseplotmark{m5bv};};
\node[stars] at (axis cs:{192.987},{-39.680}) {\tikz\pgfuseplotmark{m6bv};};
\node[stars] at (axis cs:{192.991},{-26.738}) {\tikz\pgfuseplotmark{m6cv};};
\node[stars] at (axis cs:{193.051},{17.074}) {\tikz\pgfuseplotmark{m6cb};};
\node[stars] at (axis cs:{193.115},{16.122}) {\tikz\pgfuseplotmark{m6c};};
\node[stars] at (axis cs:{193.296},{-3.553}) {\tikz\pgfuseplotmark{m6b};};
\node[stars] at (axis cs:{193.324},{21.245}) {\tikz\pgfuseplotmark{m5bb};};
\node[stars] at (axis cs:{193.384},{19.481}) {\tikz\pgfuseplotmark{m6c};};
\node[stars] at (axis cs:{193.409},{-4.225}) {\tikz\pgfuseplotmark{m6c};};
\node[stars] at (axis cs:{193.457},{12.418}) {\tikz\pgfuseplotmark{m6c};};
\node[stars] at (axis cs:{193.555},{33.534}) {\tikz\pgfuseplotmark{m6c};};
\node[stars] at (axis cs:{193.578},{-11.648}) {\tikz\pgfuseplotmark{m6b};};
\node[stars] at (axis cs:{193.588},{-9.539}) {\tikz\pgfuseplotmark{m5av};};
\node[stars] at (axis cs:{193.901},{3.397}) {\tikz\pgfuseplotmark{m3cv};};
\node[stars] at (axis cs:{193.972},{-15.327}) {\tikz\pgfuseplotmark{m6c};};
\node[stars] at (axis cs:{194.007},{38.318}) {\tikz\pgfuseplotmark{m3bvb};};
\node[stars] at (axis cs:{194.303},{-12.067}) {\tikz\pgfuseplotmark{m6c};};
\node[stars] at (axis cs:{194.388},{-22.753}) {\tikz\pgfuseplotmark{m6c};};
\node[stars] at (axis cs:{194.731},{17.409}) {\tikz\pgfuseplotmark{m5av};};
\node[stars] at (axis cs:{194.915},{-3.812}) {\tikz\pgfuseplotmark{m6a};};
\node[stars] at (axis cs:{195.069},{30.785}) {\tikz\pgfuseplotmark{m5bv};};
\node[stars] at (axis cs:{195.136},{-33.505}) {\tikz\pgfuseplotmark{m6b};};
\node[stars] at (axis cs:{195.150},{-3.368}) {\tikz\pgfuseplotmark{m6bv};};
\node[stars] at (axis cs:{195.161},{18.373}) {\tikz\pgfuseplotmark{m6cb};};
\node[stars] at (axis cs:{195.290},{17.123}) {\tikz\pgfuseplotmark{m6b};};
\node[stars] at (axis cs:{195.544},{10.959}) {\tikz\pgfuseplotmark{m3av};};
\node[stars] at (axis cs:{195.942},{-20.583}) {\tikz\pgfuseplotmark{m6a};};
\node[stars] at (axis cs:{196.435},{35.799}) {\tikz\pgfuseplotmark{m5cv};};
\node[stars] at (axis cs:{196.588},{21.153}) {\tikz\pgfuseplotmark{m6bb};};
\node[stars] at (axis cs:{196.594},{22.616}) {\tikz\pgfuseplotmark{m6av};};
\node[stars] at (axis cs:{196.726},{-35.862}) {\tikz\pgfuseplotmark{m6a};};
\node[stars] at (axis cs:{196.795},{27.625}) {\tikz\pgfuseplotmark{m5a};};
\node[stars] at (axis cs:{196.973},{27.556}) {\tikz\pgfuseplotmark{m6c};};
\node[stars] at (axis cs:{196.974},{-10.740}) {\tikz\pgfuseplotmark{m5c};};
\node[stars] at (axis cs:{197.135},{-8.984}) {\tikz\pgfuseplotmark{m6a};};
\node[stars] at (axis cs:{197.264},{-23.118}) {\tikz\pgfuseplotmark{m5b};};
\node[stars] at (axis cs:{197.302},{10.022}) {\tikz\pgfuseplotmark{m6a};};
\node[stars] at (axis cs:{197.309},{-9.538}) {\tikz\pgfuseplotmark{m6c};};
\node[stars] at (axis cs:{197.411},{37.423}) {\tikz\pgfuseplotmark{m6b};};
\node[stars] at (axis cs:{197.439},{-10.329}) {\tikz\pgfuseplotmark{m6b};};
\node[stars] at (axis cs:{197.449},{16.848}) {\tikz\pgfuseplotmark{m6b};};
\node[stars] at (axis cs:{197.487},{-5.539}) {\tikz\pgfuseplotmark{m4cb};};
\node[stars] at (axis cs:{197.497},{17.529}) {\tikz\pgfuseplotmark{m4cv};};
\node[stars] at (axis cs:{197.513},{38.499}) {\tikz\pgfuseplotmark{m6bvb};};
\node[stars] at (axis cs:{197.913},{-26.552}) {\tikz\pgfuseplotmark{m6c};};
\node[stars] at (axis cs:{197.968},{27.878}) {\tikz\pgfuseplotmark{m4cv};};
\node[stars] at (axis cs:{198.013},{-37.803}) {\tikz\pgfuseplotmark{m5a};};
\node[stars] at (axis cs:{198.015},{-16.198}) {\tikz\pgfuseplotmark{m5bvb};};
\node[stars] at (axis cs:{198.035},{24.258}) {\tikz\pgfuseplotmark{m6c};};
\node[stars] at (axis cs:{198.137},{11.556}) {\tikz\pgfuseplotmark{m6a};};
\node[stars] at (axis cs:{198.302},{18.727}) {\tikz\pgfuseplotmark{m6b};};
\node[stars] at (axis cs:{198.545},{-19.931}) {\tikz\pgfuseplotmark{m5c};};
\node[stars] at (axis cs:{198.630},{11.331}) {\tikz\pgfuseplotmark{m6av};};
\node[stars] at (axis cs:{198.789},{-36.371}) {\tikz\pgfuseplotmark{m6c};};
\node[stars] at (axis cs:{198.995},{-19.943}) {\tikz\pgfuseplotmark{m5c};};
\node[stars] at (axis cs:{199.060},{19.052}) {\tikz\pgfuseplotmark{m6c};};
\node[stars] at (axis cs:{199.134},{19.785}) {\tikz\pgfuseplotmark{m6c};};
\node[stars] at (axis cs:{199.194},{9.424}) {\tikz\pgfuseplotmark{m5c};};
\node[stars] at (axis cs:{199.221},{-31.506}) {\tikz\pgfuseplotmark{m5b};};
\node[stars] at (axis cs:{199.315},{13.676}) {\tikz\pgfuseplotmark{m5c};};
\node[stars] at (axis cs:{199.374},{-0.676}) {\tikz\pgfuseplotmark{m6cv};};
\node[stars] at (axis cs:{199.401},{5.470}) {\tikz\pgfuseplotmark{m5av};};
\node[stars] at (axis cs:{199.601},{-18.311}) {\tikz\pgfuseplotmark{m5a};};
\node[stars] at (axis cs:{199.616},{34.098}) {\tikz\pgfuseplotmark{m6a};};
\node[stars] at (axis cs:{199.730},{-23.171}) {\tikz\pgfuseplotmark{m3bv};};
\end{axis}

\begin{axis}[name=base,axis lines=none]
\boldmath

\node[designation-label,anchor=south east]  at (axis cs:{07*15+45/4},{+28+01/ 60  })  {Pollux};   
\node[designation-label,anchor=south west]  at (axis cs:{06*15+22/4},{-17-57/ 60  })  {Mirzam};   
\node[designation-label,anchor=south west]  at (axis cs:{06*15+37/4},{+16+23/ 60  })  {Alhena};   
\node[designation-label,anchor=south west]  at (axis cs:{06*15+45/4},{-16-42/ 60  })  {Sirius};   
\node[designation-label,anchor=south west]  at (axis cs:{06*15+58/4},{-28-58/ 60  })  {Adara};   
\node[designation-label,anchor=south west]  at (axis cs:{07*15+ 8/4},{-26-23/ 60  })  {Wezen};   
\node[designation-label,anchor=south west]  at (axis cs:{07*15+34/4},{+31+53/ 60  })  {Castor};   
\node[designation-label,anchor=south west]  at (axis cs:{07*15+39/4},{+05+13/ 60  })  {Procyon};   
\node[designation-label,anchor=south west]  at (axis cs:{ 9*15+27/4},{- 8-39/ 60  })  {Alfard};   
\node[designation-label,anchor=south east]  at (axis cs:{10*15+ 8/4},{+11+58/ 60  })  {Regulus};  

\node[interest,pin={[pin distance=-0.4\onedegree,interest-label]90:{3C\,273}}] at (axis cs:187.2779159404900,+02.0523) {\pgfuseplotmark{+}} ; % 

\node[interest,pin={[pin distance=-0.4\onedegree,interest-label]90:{V838}}] at (axis cs:7*15+4.08/4,-5+50.7/60) {\pgfuseplotmark{+}} ; % V838 Mon
\node[stars,pin={[pin distance=-0.6\onedegree,Flaamsted]0:{S}}] at (axis cs:91.4397903266700,-24.1955597292800)  {\tikz\pgfuseplotmark{m6cv};} ; % S Lep
\node[stars,pin={[pin distance=-0.6\onedegree,Flaamsted]0:{T}}] at (axis cs:096.3041653170500,+07.0857107174700) {} ; % T Mon
\node[stars,pin={[pin distance=-0.6\onedegree,Flaamsted]0:{UU}}] at (axis cs:099.1368212654100,+38.4455052837800) {} ; % UU Aur
\node[stars,pin={[pin distance=-0.6\onedegree,Flaamsted]180:{WW}}] at (axis cs:098.1132698606000,+32.4548978969700) {} ; % WW Aur
\node[pin={[pin distance=-0.4\onedegree,Flaamsted]00:{RT}}] at (axis cs:{97.142},{30.493}) {}; % Aur
\node[pin={[pin distance=-0.4\onedegree,Flaamsted]-90:{15}}] at (axis cs:{100.244},{9.896}) {}; % Mon
\node[pin={[pin distance=-0.4\onedegree,Flaamsted]0:{S}}] at (axis cs:{100.244},{9.896}) {}; % Mon
\node[pin={[pin distance=-0.4\onedegree,Flaamsted]0:{U}}] at (axis cs:159.3886373063,-13.3845424786) {}; % Hya
\node[pin={[pin distance=-0.4\onedegree,Flaamsted]0:{V}}] at (axis cs:162.9052566137600,-21.2500927577900) {\tikz\pgfuseplotmark{m6cv};}; % Hya
\node[pin={[pin distance=-0.4\onedegree,Flaamsted]-90:{R}}] at (axis cs:146.8895165870019,+11.4288398010358) {}; % Leo
\node[pin={[pin distance=-0.4\onedegree,Flaamsted]0:{R}}] at (axis cs:146.3928,+34.511) {}; % LMi
\node[pin={[pin distance=-0.4\onedegree,Flaamsted]0:{U}}] at (axis cs:112.6977965189300,-09.7768770770800) {}; % Mon
\node[pin={[pin distance=-0.4\onedegree,Flaamsted]0:{R}}] at (axis cs:124.1409447210600,+11.7262414172200) {\tikz\pgfuseplotmark{m6cv};}; % Cnc
\node[pin={[pin distance=-0.8\onedegree,Flaamsted]180:{R}}] at (axis cs:189.6247241831999,+06.98861) {\tikz\pgfuseplotmark{m6cv};}; % Vir



\node[pin={[pin distance=-0.6\onedegree,Bayer]00:{$\boldsymbol\alpha$}}] at (axis cs:{156.788},{-31.068}) {}; % Ant,  4.27 
\node[pin={[pin distance=-0.6\onedegree,Bayer]180:{$\boldsymbol\delta$}}] at (axis cs:{157.397},{-30.607}) {}; % Ant,  5.58 
\node[pin={[pin distance=-0.6\onedegree,Bayer]180:{$\boldsymbol\epsilon$}}] at (axis cs:{142.311},{-35.951}) {}; % Ant,  4.49 
\node[pin={[pin distance=-0.6\onedegree,Bayer]00:{$\boldsymbol\eta$}}] at (axis cs:{149.718},{-35.891}) {}; % Ant,  5.23 
\node[pin={[pin distance=-0.6\onedegree,Bayer]00:{$\boldsymbol\iota$}}] at (axis cs:{164.179},{-37.138}) {}; % Ant,  4.60 
\node[pin={[pin distance=-0.6\onedegree,Bayer]00:{$\boldsymbol\vartheta$}}] at (axis cs:{146.050},{-27.769}) {}; % Ant,  4.79 
\node[pin={[pin distance=-0.6\onedegree,Bayer]180:{$\boldsymbol\zeta^2$}}] at (axis cs:{142.884},{-31.872}) {}; % Ant,  5.91 
\node[pin={[pin distance=-0.6\onedegree,Bayer]00:{$\boldsymbol\kappa$}}] at (axis cs:{93.845},{29.498}) {}; % Aur,  4.33 
\node[pin={[pin distance=-0.4\onedegree,Bayer]-135:{$\boldsymbol\psi^3$}}] at (axis cs:{99.705},{39.903}) {}; % Aur,  5.35 
\node[pin={[pin distance=-0.6\onedegree,Flaamsted]00:{40}}] at (axis cs:{91.646},{38.482}) {}; % Aur,  5.35 
\node[pin={[pin distance=-0.6\onedegree,Flaamsted]180:{48}}] at (axis cs:{97.142},{30.493}) {}; % Aur,  5.75 
\node[pin={[pin distance=-0.6\onedegree,Flaamsted]00:{49}}] at (axis cs:{98.800},{28.022}) {}; % Aur,  5.27 
\node[pin={[pin distance=-0.6\onedegree,Flaamsted]00:{51}}] at (axis cs:{99.665},{39.391}) {}; % Aur,  5.70 
\node[pin={[pin distance=-0.6\onedegree,Flaamsted]00:{53}}] at (axis cs:{99.596},{28.984}) {}; % Aur,  5.75 
\node[pin={[pin distance=-0.6\onedegree,Flaamsted]00:{63}}] at (axis cs:{107.914},{39.320}) {}; % Aur,  4.89 
\node[pin={[pin distance=-0.6\onedegree,Flaamsted]00:{65}}] at (axis cs:{110.511},{36.760}) {}; % Aur,  5.13 
\node[pin={[pin distance=-0.4\onedegree,Bayer]00:{$\boldsymbol\alpha$}}] at (axis cs:{101.287},{-16.716}) {}; % CMa, -1.44 
\node[pin={[pin distance=-0.4\onedegree,Bayer]00:{$\boldsymbol\beta$}}] at (axis cs:{95.675},{-17.956}) {}; % CMa,  1.96 
\node[pin={[pin distance=-0.4\onedegree,Bayer]00:{$\boldsymbol\delta$}}] at (axis cs:{107.098},{-26.393}) {}; % CMa,  1.84 
\node[pin={[pin distance=-0.4\onedegree,Bayer]00:{$\boldsymbol\epsilon$}}] at (axis cs:{104.656},{-28.972}) {}; % CMa,  1.53 
\node[pin={[pin distance=-0.4\onedegree,Bayer]00:{$\boldsymbol\eta$}}] at (axis cs:{111.024},{-29.303}) {}; % CMa,  2.46 
\node[pin={[pin distance=-0.6\onedegree,Bayer]00:{$\boldsymbol\gamma$}}] at (axis cs:{105.940},{-15.633}) {}; % CMa,  4.11 
\node[pin={[pin distance=-0.6\onedegree,Bayer]00:{$\boldsymbol\iota$}}] at (axis cs:{104.034},{-17.054}) {}; % CMa,  4.39 
\node[pin={[pin distance=-0.6\onedegree,Bayer]00:{$\boldsymbol\kappa$}}] at (axis cs:{102.460},{-32.508}) {}; % CMa,  3.53 
\node[pin={[pin distance=-0.6\onedegree,Bayer]00:{$\boldsymbol\lambda$}}] at (axis cs:{97.043},{-32.580}) {}; % CMa,  4.47 
\node[pin={[pin distance=-0.6\onedegree,Bayer]00:{$\boldsymbol\mu$}}] at (axis cs:{104.028},{-14.043}) {}; % CMa,  4.97 
\node[pin={[pin distance=-0.6\onedegree,Bayer]00:{$\boldsymbol\nu^1$}}] at (axis cs:{99.095},{-18.660}) {}; % CMa,  5.70 
\node[pin={[pin distance=-0.6\onedegree,Bayer]00:{$\boldsymbol\nu^2$}}] at (axis cs:{99.171},{-19.256}) {}; % CMa,  3.95 
\node[pin={[pin distance=-0.6\onedegree,Bayer]00:{$\boldsymbol\nu^3$}}] at (axis cs:{99.473},{-18.237}) {}; % CMa,  4.43 
\node[pin={[pin distance=-0.6\onedegree,Bayer]00:{$\boldsymbol\omega$}}] at (axis cs:{108.703},{-26.773}) {}; % CMa,  4.04 
\node[pin={[pin distance=-0.6\onedegree,Bayer]00:{$\boldsymbol\omicron^1$}}] at (axis cs:{103.533},{-24.184}) {}; % CMa,  3.84 
\node[pin={[pin distance=-0.6\onedegree,Bayer]00:{$\boldsymbol\omicron^2$}}] at (axis cs:{105.756},{-23.833}) {}; % CMa,  3.04 
\node[pin={[pin distance=-0.6\onedegree,Bayer]90:{$\boldsymbol\pi$}}] at (axis cs:{103.906},{-20.136}) {}; % CMa,  4.68 
\node[pin={[pin distance=-0.6\onedegree,Bayer]00:{$\boldsymbol\sigma$}}] at (axis cs:{105.430},{-27.935}) {}; % CMa,  3.47 
\node[pin={[pin distance=-0.6\onedegree,Bayer]00:{$\boldsymbol\tau$}}] at (axis cs:{109.677},{-24.954}) {}; % CMa,  4.40 
\node[pin={[pin distance=-0.6\onedegree,Bayer]00:{$\boldsymbol\vartheta$}}] at (axis cs:{103.547},{-12.039}) {}; % CMa,  4.07 
\node[pin={[pin distance=-0.6\onedegree,Bayer]00:{$\boldsymbol\xi^1$}}] at (axis cs:{97.964},{-23.418}) {}; % CMa,  4.33 
\node[pin={[pin distance=-0.6\onedegree,Bayer]00:{$\boldsymbol\xi^2$}}] at (axis cs:{98.764},{-22.965}) {}; % CMa,  4.53 
\node[pin={[pin distance=-0.6\onedegree,Bayer]00:{$\boldsymbol\zeta$}}] at (axis cs:{95.078},{-30.063}) {}; % CMa,  3.02 
\node[pin={[pin distance=-0.6\onedegree,Flaamsted]00:{10}}] at (axis cs:{101.119},{-31.070}) {}; % CMa,  5.23 
\node[pin={[pin distance=-0.6\onedegree,Flaamsted]00:{11}}] at (axis cs:{101.713},{-14.426}) {}; % CMa,  5.28 
\node[pin={[pin distance=-0.6\onedegree,Flaamsted]00:{15}}] at (axis cs:{103.387},{-20.224}) {}; % CMa,  4.82 
\node[pin={[pin distance=-0.6\onedegree,Flaamsted]-90:{17}}] at (axis cs:{103.761},{-20.405}) {}; % CMa,  5.80 
\node[pin={[pin distance=-0.6\onedegree,Flaamsted]90:{26}}] at (axis cs:{108.051},{-25.942}) {}; % CMa,  5.91 
\node[pin={[pin distance=-0.6\onedegree,Flaamsted]180:{27}}] at (axis cs:{108.563},{-26.352}) {}; % CMa,  4.45 
\node[pin={[pin distance=-0.6\onedegree,Flaamsted]00:{29}}] at (axis cs:{109.668},{-24.559}) {}; % CMa,  4.92 

\node[pin={[pin distance=-0.4\onedegree,Bayer]00:{$\boldsymbol\alpha$}}] at (axis cs:{114.825},{5.225}) {}; % CMi,  0.40 
\node[pin={[pin distance=-0.6\onedegree,Bayer]00:{$\boldsymbol\beta$}}] at (axis cs:{111.788},{8.289}) {}; % CMi,  2.89 
\node[pin={[pin distance=-0.6\onedegree,Bayer]00:{$\boldsymbol\delta^1$}}] at (axis cs:{113.025},{1.914}) {}; % CMi,  5.24 
\node[pin={[pin distance=-0.6\onedegree,Bayer]-90:{$\boldsymbol\delta^{2,3}$}}] at (axis cs:{113.299},{3.290}) {}; % CMi,  5.59 
%\node[pin={[pin distance=-0.6\onedegree,Bayer]00:{$\boldsymbol\delta^3$}}] at (axis cs:{113.566},{3.372}) {}; % CMi,  5.84 
\node[pin={[pin distance=-0.6\onedegree,Bayer]00:{$\boldsymbol\epsilon$}}] at (axis cs:{111.412},{9.276}) {}; % CMi,  4.99 
\node[pin={[pin distance=-0.6\onedegree,Bayer]00:{$\boldsymbol\eta$}}] at (axis cs:{112.009},{6.942}) {}; % CMi,  5.24 
\node[pin={[pin distance=-0.6\onedegree,Bayer]180:{$\boldsymbol\gamma$}}] at (axis cs:{112.041},{8.925}) {}; % CMi,  4.33 
\node[pin={[pin distance=-0.6\onedegree,Bayer]00:{$\boldsymbol\zeta$}}] at (axis cs:{117.925},{1.767}) {}; % CMi,  5.13 
\node[pin={[pin distance=-0.6\onedegree,Flaamsted]00:{11}}] at (axis cs:{116.568},{10.768}) {}; % CMi,  5.25 
\node[pin={[pin distance=-0.6\onedegree,Flaamsted]00:{14}}] at (axis cs:{119.586},{2.225}) {}; % CMi,  5.29 
\node[pin={[pin distance=-0.6\onedegree,Flaamsted]00:{ 1}}] at (axis cs:{111.242},{11.669}) {}; % CMi,  5.37 
\node[pin={[pin distance=-0.6\onedegree,Flaamsted]00:{ 6}}] at (axis cs:{112.449},{12.006}) {}; % CMi,  4.54 

\node[pin={[pin distance=-0.6\onedegree,Bayer]90:{$\boldsymbol\alpha$}}] at (axis cs:{134.622},{11.858}) {}; % Cnc,  4.26 
\node[pin={[pin distance=-0.6\onedegree,Bayer]00:{$\boldsymbol\beta$}}] at (axis cs:{124.129},{9.186}) {}; % Cnc,  3.52 
\node[pin={[pin distance=-0.6\onedegree,Bayer]00:{$\boldsymbol\chi$}}] at (axis cs:{125.016},{27.218}) {}; % Cnc,  5.13 
\node[pin={[pin distance=-0.6\onedegree,Bayer]00:{$\boldsymbol\delta$}}] at (axis cs:{131.171},{18.154}) {}; % Cnc,  3.93 
\node[pin={[pin distance=-0.6\onedegree,Bayer]00:{$\boldsymbol\eta$}}] at (axis cs:{128.177},{20.441}) {}; % Cnc,  5.34 
\node[pin={[pin distance=-0.6\onedegree,Bayer]180:{$\boldsymbol\gamma$}}] at (axis cs:{130.821},{21.468}) {}; % Cnc,  4.67 
\node[pin={[pin distance=-0.6\onedegree,Bayer]00:{$\boldsymbol\iota$}}] at (axis cs:{131.674},{28.760}) {}; % Cnc,  4.02 
\node[pin={[pin distance=-0.6\onedegree,Bayer]00:{$\boldsymbol\kappa$}}] at (axis cs:{136.937},{10.668}) {}; % Cnc,  5.25 
\node[pin={[pin distance=-0.6\onedegree,Bayer]00:{$\boldsymbol\lambda$}}] at (axis cs:{125.134},{24.022}) {}; % Cnc,  5.92 
\node[pin={[pin distance=-0.6\onedegree,Bayer]00:{$\boldsymbol\mu^1$}}] at (axis cs:{121.577},{22.635}) {}; % Cnc,  5.98 
\node[pin={[pin distance=-0.6\onedegree,Bayer]00:{$\boldsymbol\mu^2$}}] at (axis cs:{121.941},{21.582}) {}; % Cnc,  5.30 
\node[pin={[pin distance=-0.6\onedegree,Bayer]00:{$\boldsymbol\nu$}}] at (axis cs:{135.684},{24.453}) {}; % Cnc,  5.46 
\node[pin={[pin distance=-0.6\onedegree,Bayer]180:{$\boldsymbol\omega^1$}}] at (axis cs:{120.233},{25.393}) {}; % Cnc,  5.86 
\node[pin={[pin distance=-0.6\onedegree,Bayer]00:{$\boldsymbol\omicron^1$}}] at (axis cs:{134.312},{15.323}) {}; % Cnc,  5.23 
\node[pin={[pin distance=-0.6\onedegree,Bayer]90:{$\boldsymbol\omicron^2$}}] at (axis cs:{134.397},{15.581}) {}; % Cnc,  5.69 
\node[pin={[pin distance=-0.6\onedegree,Bayer]00:{$\boldsymbol\pi^2$}}] at (axis cs:{138.808},{14.941}) {}; % Cnc,  5.35 
\node[pin={[pin distance=-0.6\onedegree,Bayer]00:{$\boldsymbol\psi$}}] at (axis cs:{122.613},{25.507}) {}; % Cnc,  5.73 
\node[pin={[pin distance=-0.6\onedegree,Bayer]00:{$\boldsymbol\rho^1$}}] at (axis cs:{133.149},{28.331}) {}; % Cnc,  5.95 
\node[pin={[pin distance=-0.6\onedegree,Bayer]-90:{$\boldsymbol\rho^2$}}] at (axis cs:{133.915},{27.927}) {}; % Cnc,  5.23 
\node[pin={[pin distance=-0.6\onedegree,Bayer]00:{$\boldsymbol\sigma^1$}}] at (axis cs:{133.144},{32.474}) {}; % Cnc,  5.67 
\node[pin={[pin distance=-0.6\onedegree,Bayer]00:{$\boldsymbol\sigma^2$}}] at (axis cs:{134.236},{32.910}) {}; % Cnc,  5.44 
\node[pin={[pin distance=-0.6\onedegree,Bayer]-90:{$\boldsymbol\sigma^3$}}] at (axis cs:{134.886},{32.419}) {}; % Cnc,  5.23 
\node[pin={[pin distance=-0.6\onedegree,Bayer]00:{$\boldsymbol\tau$}}] at (axis cs:{137.000},{29.654}) {}; % Cnc,  5.42 
\node[pin={[pin distance=-0.6\onedegree,Bayer]90:{$\boldsymbol\upsilon^1$}}] at (axis cs:{127.877},{24.081}) {}; % Cnc,  5.70 
\node[pin={[pin distance=-0.6\onedegree,Bayer]00:{$\boldsymbol\varphi^1$}}] at (axis cs:{126.615},{27.893}) {}; % Cnc,  5.58 
\node[pin={[pin distance=-0.6\onedegree,Bayer]00:{$\boldsymbol\vartheta$}}] at (axis cs:{127.899},{18.094}) {}; % Cnc,  5.35 
\node[pin={[pin distance=-0.6\onedegree,Bayer]00:{$\boldsymbol\xi$}}] at (axis cs:{137.340},{22.045}) {}; % Cnc,  5.15 
\node[pin={[pin distance=-0.6\onedegree,Bayer]00:{$\boldsymbol\zeta^1$}}] at (axis cs:{123.053},{17.648}) {}; % Cnc,  5.63 
\node[pin={[pin distance=-0.6\onedegree,Flaamsted]00:{15}}] at (axis cs:{123.287},{29.656}) {}; % Cnc,  5.62 
\node[pin={[pin distance=-0.6\onedegree,Flaamsted]00:{ 1}}] at (axis cs:{119.248},{15.790}) {}; % Cnc,  5.81 
\node[pin={[pin distance=-0.6\onedegree,Flaamsted]00:{20}}] at (axis cs:{125.841},{18.332}) {}; % Cnc,  5.94 
\node[pin={[pin distance=-0.6\onedegree,Flaamsted]00:{21}}] at (axis cs:125.98000,+10.6320) {}; % Cnc,  5.94 
\node[pin={[pin distance=-0.6\onedegree,Flaamsted]00:{27}}] at (axis cs:{126.683},{12.655}) {}; % Cnc,  5.58 
\node[pin={[pin distance=-0.6\onedegree,Flaamsted]00:{29}}] at (axis cs:{127.156},{14.211}) {}; % Cnc,  5.96 
\node[pin={[pin distance=-0.6\onedegree,Flaamsted]00:{36}}] at (axis cs:{129.274},{9.655}) {}; % Cnc,  5.92 
\node[pin={[pin distance=-0.6\onedegree,Flaamsted]00:{ 3}}] at (axis cs:{120.197},{17.309}) {}; % Cnc,  5.59 
\node[pin={[pin distance=-0.6\onedegree,Flaamsted]00:{45}}] at (axis cs:{130.801},{12.681}) {}; % Cnc,  5.63 
\node[pin={[pin distance=-0.6\onedegree,Flaamsted]00:{49}}] at (axis cs:{131.188},{10.081}) {}; % Cnc,  5.63 
\node[pin={[pin distance=-0.6\onedegree,Flaamsted]00:{50}}] at (axis cs:{131.733},{12.110}) {}; % Cnc,  5.89 
\node[pin={[pin distance=-0.6\onedegree,Flaamsted]00:{57}}] at (axis cs:{133.561},{30.579}) {}; % Cnc,  5.42 
\node[pin={[pin distance=-0.6\onedegree,Flaamsted]00:{ 5}}] at (axis cs:{120.376},{16.455}) {}; % Cnc,  5.99 
\node[pin={[pin distance=-0.6\onedegree,Flaamsted]00:{60}}] at (axis cs:{133.981},{11.626}) {}; % Cnc,  5.44 
\node[pin={[pin distance=-0.6\onedegree,Flaamsted]180:{66}}] at (axis cs:{135.351},{32.252}) {}; % Cnc,  5.95 
\node[pin={[pin distance=-0.6\onedegree,Flaamsted]00:{75}}] at (axis cs:{137.197},{26.629}) {}; % Cnc,  5.96 
\node[pin={[pin distance=-0.6\onedegree,Flaamsted]00:{ 8}}] at (axis cs:{121.269},{13.118}) {}; % Cnc,  5.15 

\node[pin={[pin distance=-0.6\onedegree,Bayer]00:{$\boldsymbol\delta$}}] at (axis cs:{95.528},{-33.436}) {}; % Col,  3.85 
\node[pin={[pin distance=-0.6\onedegree,Bayer]00:{$\boldsymbol\kappa$}}] at (axis cs:{94.138},{-35.140}) {}; % Col,  4.37 
\node[pin={[pin distance=-0.6\onedegree,Bayer]00:{$\boldsymbol\vartheta$}}] at (axis cs:{91.882},{-37.253}) {}; % Col,  5.00 

\node[pin={[pin distance=-0.6\onedegree,Bayer]00:{$\boldsymbol\alpha$}}] at (axis cs:{197.497},{17.529}) {}; % Com,  4.32 
\node[pin={[pin distance=-0.6\onedegree,Bayer]00:{$\boldsymbol\beta$}}] at (axis cs:{197.968},{27.878}) {}; % Com,  4.25 
\node[pin={[pin distance=-0.6\onedegree,Bayer]00:{$\boldsymbol\gamma$}}] at (axis cs:{186.734},{28.268}) {}; % Com,  4.34 
\node[pin={[pin distance=-0.6\onedegree,Flaamsted]00:{11}}] at (axis cs:{185.179},{17.793}) {}; % Com,  4.73 
\node[pin={[pin distance=-0.6\onedegree,Flaamsted]-90:{12}}] at (axis cs:{185.626},{25.846}) {}; % Com,  4.80 
\node[pin={[pin distance=-1\onedegree,Flaamsted]45:{13}}] at (axis cs:{186.077},{26.098}) {}; % Com,  5.15 
\node[pin={[pin distance=-0.6\onedegree,Flaamsted]00:{14}}] at (axis cs:{186.600},{27.268}) {}; % Com,  4.92 
\node[pin={[pin distance=-0.6\onedegree,Flaamsted]00:{16}}] at (axis cs:{186.747},{26.826}) {}; % Com,  4.96 
\node[pin={[pin distance=-0.6\onedegree,Flaamsted]00:{17}}] at (axis cs:{187.228},{25.913}) {}; % Com,  5.25 
\node[pin={[pin distance=-0.6\onedegree,Flaamsted]00:{18}}] at (axis cs:{187.363},{24.109}) {}; % Com,  5.48 
\node[pin={[pin distance=-0.6\onedegree,Flaamsted]00:{20}}] at (axis cs:{187.430},{20.896}) {}; % Com,  5.68 
\node[pin={[pin distance=-0.6\onedegree,Flaamsted]00:{21}}] at (axis cs:{187.752},{24.567}) {}; % Com,  5.44 
\node[pin={[pin distance=-0.6\onedegree,Flaamsted]00:{23}}] at (axis cs:{188.713},{22.629}) {}; % Com,  4.79 
\node[pin={[pin distance=-0.6\onedegree,Flaamsted]00:{24}}] at (axis cs:{188.782},{18.377}) {}; % Com,  4.99 
\node[pin={[pin distance=-0.6\onedegree,Flaamsted]00:{25}}] at (axis cs:{189.243},{17.089}) {}; % Com,  5.68 
\node[pin={[pin distance=-0.6\onedegree,Flaamsted]00:{26}}] at (axis cs:{189.780},{21.062}) {}; % Com,  5.48 
\node[pin={[pin distance=-0.6\onedegree,Flaamsted]180:{27}}] at (axis cs:{191.662},{16.578}) {}; % Com,  5.11 
\node[pin={[pin distance=-0.6\onedegree,Flaamsted]180:{29}}] at (axis cs:{192.226},{14.122}) {}; % Com,  5.71 
\node[pin={[pin distance=-0.6\onedegree,Flaamsted]00:{ 2}}] at (axis cs:{181.069},{21.459}) {}; % Com,  5.89 
\node[pin={[pin distance=-0.6\onedegree,Flaamsted]-90:{30}}] at (axis cs:{192.323},{27.552}) {}; % Com,  5.76 
\node[pin={[pin distance=-0.6\onedegree,Flaamsted]90:{31}}] at (axis cs:{192.925},{27.541}) {}; % Com,  4.93 
\node[pin={[pin distance=-0.6\onedegree,Flaamsted]00:{35}}] at (axis cs:{193.324},{21.245}) {}; % Com,  4.92 
\node[pin={[pin distance=-0.6\onedegree,Flaamsted]00:{36}}] at (axis cs:{194.731},{17.409}) {}; % Com,  4.75 
\node[pin={[pin distance=-0.6\onedegree,Flaamsted]00:{37}}] at (axis cs:{195.069},{30.785}) {}; % Com,  4.89 
\node[pin={[pin distance=-0.6\onedegree,Flaamsted]-90:{38}}] at (axis cs:{195.290},{17.123}) {}; % Com,  5.97 
\node[pin={[pin distance=-0.6\onedegree,Flaamsted]00:{40}}] at (axis cs:{196.594},{22.616}) {}; % Com,  5.64 
\node[pin={[pin distance=-0.6\onedegree,Flaamsted]00:{41}}] at (axis cs:{196.795},{27.625}) {}; % Com,  4.78 
\node[pin={[pin distance=-0.6\onedegree,Flaamsted]00:{ 4}}] at (axis cs:{182.963},{25.870}) {}; % Com,  5.65 
\node[pin={[pin distance=-0.6\onedegree,Flaamsted]00:{ 5}}] at (axis cs:{183.039},{20.542}) {}; % Com,  5.60 
\node[pin={[pin distance=-0.6\onedegree,Flaamsted]-90:{ 6}}] at (axis cs:{184.001},{14.899}) {}; % Com,  5.10 
\node[pin={[pin distance=-0.6\onedegree,Flaamsted]00:{ 7}}] at (axis cs:{184.086},{23.945}) {}; % Com,  4.93 

\node[pin={[pin distance=-0.6\onedegree,Bayer]00:{$\boldsymbol\alpha$}}] at (axis cs:{164.944},{-18.299}) {}; % Crt,  4.08 
\node[pin={[pin distance=-0.6\onedegree,Bayer]00:{$\boldsymbol\beta$}}] at (axis cs:{167.915},{-22.826}) {}; % Crt,  4.46 
\node[pin={[pin distance=-0.6\onedegree,Bayer]00:{$\boldsymbol\delta$}}] at (axis cs:{169.835},{-14.778}) {}; % Crt,  3.57 
\node[pin={[pin distance=-0.6\onedegree,Bayer]00:{$\boldsymbol\epsilon$}}] at (axis cs:{171.152},{-10.859}) {}; % Crt,  4.81 
\node[pin={[pin distance=-0.6\onedegree,Bayer]00:{$\boldsymbol\eta$}}] at (axis cs:{179.004},{-17.151}) {}; % Crt,  5.17 
\node[pin={[pin distance=-0.6\onedegree,Bayer]00:{$\boldsymbol\gamma$}}] at (axis cs:{171.221},{-17.684}) {}; % Crt,  4.08 
\node[pin={[pin distance=-0.6\onedegree,Bayer]00:{$\boldsymbol\iota$}}] at (axis cs:{174.667},{-13.202}) {}; % Crt,  5.49 
\node[pin={[pin distance=-0.6\onedegree,Bayer]00:{$\boldsymbol\kappa$}}] at (axis cs:{171.790},{-12.357}) {}; % Crt,  5.93 
\node[pin={[pin distance=-0.6\onedegree,Bayer]00:{$\boldsymbol\lambda$}}] at (axis cs:{170.841},{-18.780}) {}; % Crt,  5.08 
\node[pin={[pin distance=-0.6\onedegree,Bayer]00:{$\boldsymbol\vartheta$}}] at (axis cs:{174.170},{-9.802}) {}; % Crt,  4.69 
\node[pin={[pin distance=-0.6\onedegree,Bayer]00:{$\boldsymbol\zeta$}}] at (axis cs:{176.191},{-18.351}) {}; % Crt,  4.71 
\node[pin={[pin distance=-0.6\onedegree,Bayer]00:{$\boldsymbol\alpha$}}] at (axis cs:{182.103},{-24.729}) {}; % Crv,  4.04 
\node[pin={[pin distance=-0.4\onedegree,Bayer]00:{$\boldsymbol\beta$}}] at (axis cs:{188.597},{-23.397}) {}; % Crv,  2.66 
\node[pin={[pin distance=-0.4\onedegree,Bayer]-90:{$\boldsymbol\delta$}}] at (axis cs:{187.466},{-16.515}) {}; % Crv,  2.97 
\node[pin={[pin distance=-0.4\onedegree,Bayer]00:{$\boldsymbol\epsilon$}}] at (axis cs:{182.531},{-22.620}) {}; % Crv,  3.01 
\node[pin={[pin distance=-0.4\onedegree,Bayer]90:{$\boldsymbol\eta$}}] at (axis cs:{188.018},{-16.196}) {}; % Crv,  4.31 
\node[pin={[pin distance=-0.4\onedegree,Bayer]00:{$\boldsymbol\gamma$}}] at (axis cs:{183.952},{-17.542}) {}; % Crv,  2.59 
\node[pin={[pin distance=-0.6\onedegree,Bayer]00:{$\boldsymbol\zeta$}}] at (axis cs:{185.140},{-22.216}) {}; % Crv,  5.21 
\node[pin={[pin distance=-0.6\onedegree,Flaamsted]00:{ 3}}] at (axis cs:{182.766},{-23.602}) {}; % Crv,  5.46 
\node[pin={[pin distance=-0.6\onedegree,Flaamsted]00:{ 6}}] at (axis cs:{185.840},{-24.841}) {}; % Crv,  5.66
 
\node[pin={[pin distance=-0.6\onedegree,Bayer]00:{$\boldsymbol\alpha^{1,2}$}}] at (axis cs:{194.002},{38.315}) {}; % CVn,  5.48 
%\node[pin={[pin distance=-0.6\onedegree,Bayer]00:{$\boldsymbol\alpha^2$}}] at (axis cs:{194.007},{38.318}) {}; % CVn,  2.89 
\node[pin={[pin distance=-0.6\onedegree,Flaamsted]00:{10}}] at (axis cs:{191.248},{39.279}) {}; % CVn,  5.96 
\node[pin={[pin distance=-0.6\onedegree,Flaamsted]00:{14}}] at (axis cs:{196.435},{35.799}) {}; % CVn,  5.20 
\node[pin={[pin distance=-0.6\onedegree,Flaamsted]00:{17}}] at (axis cs:{197.513},{38.499}) {}; % CVn,  5.93 
\node[pin={[pin distance=-0.6\onedegree,Flaamsted]00:{ 6}}] at (axis cs:{186.462},{39.019}) {}; % CVn,  5.02
 
\node[pin={[pin distance=-0.4\onedegree,Bayer]00:{$\boldsymbol\alpha$}}] at (axis cs:{113.649},{31.888}) {}; % Gem,  1.58 
\node[pin={[pin distance=-0.4\onedegree,Bayer]00:{$\boldsymbol\beta$}}] at (axis cs:{116.329},{28.026}) {}; % Gem,  1.22 
\node[pin={[pin distance=-0.4\onedegree,Bayer]00:{$\boldsymbol\gamma$}}] at (axis cs:{99.428},{16.399}) {}; % Gem,  2.02 
\node[pin={[pin distance=-0.6\onedegree,Bayer]00:{$\boldsymbol\chi$}}] at (axis cs:{120.880},{27.794}) {}; % Gem,  4.94 
\node[pin={[pin distance=-0.6\onedegree,Bayer]00:{$\boldsymbol\delta$}}] at (axis cs:{110.031},{21.982}) {}; % Gem,  3.54 
\node[pin={[pin distance=-0.6\onedegree,Bayer]00:{$\boldsymbol\epsilon$}}] at (axis cs:{100.983},{25.131}) {}; % Gem,  3.01 
\node[pin={[pin distance=-0.6\onedegree,Bayer]00:{$\boldsymbol\eta$}}] at (axis cs:{93.719},{22.507}) {}; % Gem,  3.30 
\node[pin={[pin distance=-0.6\onedegree,Bayer]90:{$\boldsymbol\iota$}}] at (axis cs:{111.432},{27.798}) {}; % Gem,  3.79 
\node[pin={[pin distance=-0.6\onedegree,Bayer]00:{$\boldsymbol\kappa$}}] at (axis cs:{116.112},{24.398}) {}; % Gem,  3.57 
\node[pin={[pin distance=-0.6\onedegree,Bayer]00:{$\boldsymbol\lambda$}}] at (axis cs:{109.523},{16.540}) {}; % Gem,  3.58 
\node[pin={[pin distance=-0.6\onedegree,Bayer]00:{$\boldsymbol\mu$}}] at (axis cs:{95.740},{22.513}) {}; % Gem,  2.90 
\node[pin={[pin distance=-0.6\onedegree,Bayer]00:{$\boldsymbol\nu$}}] at (axis cs:{97.241},{20.212}) {}; % Gem,  4.14 
\node[pin={[pin distance=-0.6\onedegree,Bayer]00:{$\boldsymbol\omega$}}] at (axis cs:{105.603},{24.215}) {}; % Gem,  5.18 
\node[pin={[pin distance=-0.6\onedegree,Bayer]00:{$\boldsymbol\omicron$}}] at (axis cs:{114.791},{34.584}) {}; % Gem,  4.89 
\node[pin={[pin distance=-0.6\onedegree,Bayer]00:{$\boldsymbol\pi$}}] at (axis cs:{116.876},{33.415}) {}; % Gem,  5.14 
\node[pin={[pin distance=-0.6\onedegree,Bayer]00:{$\boldsymbol\rho$}}] at (axis cs:{112.278},{31.785}) {}; % Gem,  4.18 
\node[pin={[pin distance=-0.6\onedegree,Bayer]00:{$\boldsymbol\sigma$}}] at (axis cs:{115.828},{28.884}) {}; % Gem,  4.25 
\node[pin={[pin distance=-0.6\onedegree,Bayer]00:{$\boldsymbol\tau$}}] at (axis cs:{107.785},{30.245}) {}; % Gem,  4.39 
\node[pin={[pin distance=-0.6\onedegree,Bayer]00:{$\boldsymbol\upsilon$}}] at (axis cs:{113.981},{26.896}) {}; % Gem,  4.07 
\node[pin={[pin distance=-0.6\onedegree,Bayer]00:{$\boldsymbol\varphi$}}] at (axis cs:{118.374},{26.766}) {}; % Gem,  4.98 
\node[pin={[pin distance=-0.6\onedegree,Bayer]00:{$\boldsymbol\vartheta$}}] at (axis cs:{103.197},{33.961}) {}; % Gem,  3.61 
\node[pin={[pin distance=-0.6\onedegree,Bayer]-90:{$\boldsymbol\xi$}}] at (axis cs:{101.322},{12.895}) {}; % Gem,  3.34 
\node[pin={[pin distance=-0.6\onedegree,Bayer]00:{$\boldsymbol\zeta$}}] at (axis cs:{106.027},{20.570}) {}; % Gem,  4.01 
\node[pin={[pin distance=-0.6\onedegree,Flaamsted]00:{ 1}}] at (axis cs:{91.030},{23.263}) {}; % Gem,  4.18 
\node[pin={[pin distance=-0.6\onedegree,Flaamsted]00:{26}}] at (axis cs:{100.601},{17.645}) {}; % Gem,  5.29 
\node[pin={[pin distance=-0.6\onedegree,Flaamsted]00:{28}}] at (axis cs:{101.189},{28.971}) {}; % Gem,  5.42 
\node[pin={[pin distance=-0.6\onedegree,Flaamsted]00:{30}}] at (axis cs:{100.997},{13.228}) {}; % Gem,  4.49 
\node[pin={[pin distance=-0.6\onedegree,Flaamsted]00:{33}}] at (axis cs:{102.458},{16.203}) {}; % Gem,  5.87 
\node[pin={[pin distance=-0.6\onedegree,Flaamsted]00:{35}}] at (axis cs:{102.606},{13.413}) {}; % Gem,  5.68 
\node[pin={[pin distance=-0.6\onedegree,Flaamsted]00:{36}}] at (axis cs:{102.888},{21.761}) {}; % Gem,  5.27 
\node[pin={[pin distance=-0.6\onedegree,Flaamsted]00:{37}}] at (axis cs:{103.828},{25.376}) {}; % Gem,  5.75 
\node[pin={[pin distance=-0.6\onedegree,Flaamsted]00:{38}}] at (axis cs:{103.661},{13.178}) {}; % Gem,  4.70 
\node[pin={[pin distance=-0.6\onedegree,Flaamsted]00:{ 3}}] at (axis cs:{92.433},{23.113}) {}; % Gem,  5.76 
\node[pin={[pin distance=-0.6\onedegree,Flaamsted]00:{41}}] at (axis cs:{105.066},{16.079}) {}; % Gem,  5.72 
\node[pin={[pin distance=-0.6\onedegree,Flaamsted]00:{45}}] at (axis cs:{107.092},{15.930}) {}; % Gem,  5.49 
\node[pin={[pin distance=-0.6\onedegree,Flaamsted]00:{47}}] at (axis cs:{107.846},{26.856}) {}; % Gem,  5.76 
\node[pin={[pin distance=-0.6\onedegree,Flaamsted]00:{48}}] at (axis cs:{108.110},{24.128}) {}; % Gem,  5.85 
\node[pin={[pin distance=-0.6\onedegree,Flaamsted]00:{51}}] at (axis cs:{108.343},{16.159}) {}; % Gem,  5.06 
\node[pin={[pin distance=-0.6\onedegree,Flaamsted]00:{52}}] at (axis cs:{108.675},{24.885}) {}; % Gem,  5.84 
\node[pin={[pin distance=-0.6\onedegree,Flaamsted]00:{53}}] at (axis cs:{108.988},{27.897}) {}; % Gem,  5.75 
\node[pin={[pin distance=-0.6\onedegree,Flaamsted]00:{56}}] at (axis cs:{110.487},{20.443}) {}; % Gem,  5.09 
\node[pin={[pin distance=-0.6\onedegree,Flaamsted]00:{57}}] at (axis cs:{110.869},{25.051}) {}; % Gem,  5.03 
\node[pin={[pin distance=-0.6\onedegree,Flaamsted]00:{59}}] at (axis cs:{111.139},{27.638}) {}; % Gem,  5.78 
\node[pin={[pin distance=-0.6\onedegree,Flaamsted]180:{ 5}}] at (axis cs:{92.885},{24.420}) {}; % Gem,  5.83 
\node[pin={[pin distance=-0.6\onedegree,Flaamsted]00:{61}}] at (axis cs:{111.735},{20.257}) {}; % Gem,  5.94 
\node[pin={[pin distance=-0.6\onedegree,Flaamsted]00:{63}}] at (axis cs:{111.935},{21.445}) {}; % Gem,  5.24 
\node[pin={[pin distance=-0.6\onedegree,Flaamsted]90:{64}}] at (axis cs:{112.335},{28.118}) {}; % Gem,  5.07 
\node[pin={[pin distance=-0.6\onedegree,Flaamsted]180:{65}}] at (axis cs:{112.453},{27.916}) {}; % Gem,  5.01 
\node[pin={[pin distance=-0.6\onedegree,Flaamsted]00:{68}}] at (axis cs:{113.402},{15.826}) {}; % Gem,  5.27 
\node[pin={[pin distance=-0.6\onedegree,Flaamsted]00:{70}}] at (axis cs:{114.637},{35.048}) {}; % Gem,  5.58 
\node[pin={[pin distance=-0.6\onedegree,Flaamsted]00:{74}}] at (axis cs:{114.869},{17.674}) {}; % Gem,  5.04 
\node[pin={[pin distance=-0.6\onedegree,Flaamsted]00:{76}}] at (axis cs:{116.029},{25.784}) {}; % Gem,  5.29 
\node[pin={[pin distance=-0.6\onedegree,Flaamsted]00:{81}}] at (axis cs:{116.531},{18.510}) {}; % Gem,  4.87 
\node[pin={[pin distance=-0.6\onedegree,Flaamsted]00:{85}}] at (axis cs:{118.916},{19.884}) {}; % Gem,  5.38 

\node[pin={[pin distance=-0.4\onedegree,Bayer]00:{$\boldsymbol\alpha$}}] at (axis cs:{141.897},{-8.658}) {}; % Hya,  1.99 
\node[pin={[pin distance=-0.6\onedegree,Bayer]00:{$\boldsymbol\beta$}}] at (axis cs:{178.227},{-33.908}) {}; % Hya,  4.29 
\node[pin={[pin distance=-0.6\onedegree,Bayer]00:{$\boldsymbol\chi^{1,2}$}}] at (axis cs:{166.333},{-27.293}) {}; % Hya,  4.93 
%\node[pin={[pin distance=-0.6\onedegree,Bayer]00:{$\boldsymbol\chi^2$}}] at (axis cs:{166.490},{-27.288}) {}; % Hya,  5.71 
\node[pin={[pin distance=-0.6\onedegree,Bayer]00:{$\boldsymbol\delta$}}] at (axis cs:{129.414},{5.704}) {}; % Hya,  4.14 
\node[pin={[pin distance=-0.6\onedegree,Bayer]00:{$\boldsymbol\epsilon$}}] at (axis cs:{131.694},{6.419}) {}; % Hya,  3.40 
\node[pin={[pin distance=-0.6\onedegree,Bayer]00:{$\boldsymbol\eta$}}] at (axis cs:{130.806},{3.398}) {}; % Hya,  4.29 
\node[pin={[pin distance=-0.6\onedegree,Bayer]00:{$\boldsymbol\gamma$}}] at (axis cs:{199.730},{-23.171}) {}; % Hya,  3.00 
\node[pin={[pin distance=-0.6\onedegree,Bayer]00:{$\boldsymbol\iota$}}] at (axis cs:{144.964},{-1.143}) {}; % Hya,  3.90 
\node[pin={[pin distance=-0.6\onedegree,Bayer]00:{$\boldsymbol\kappa$}}] at (axis cs:{145.077},{-14.332}) {}; % Hya,  5.06 
\node[pin={[pin distance=-0.6\onedegree,Bayer]00:{$\boldsymbol\lambda$}}] at (axis cs:{152.647},{-12.354}) {}; % Hya,  3.61 
\node[pin={[pin distance=-0.6\onedegree,Bayer]00:{$\boldsymbol\mu$}}] at (axis cs:{156.523},{-16.836}) {}; % Hya,  3.82 
\node[pin={[pin distance=-0.6\onedegree,Bayer]00:{$\boldsymbol\nu$}}] at (axis cs:{162.406},{-16.193}) {}; % Hya,  3.11 
\node[pin={[pin distance=-0.6\onedegree,Bayer]00:{$\boldsymbol\omega$}}] at (axis cs:{136.493},{5.092}) {}; % Hya,  4.99 
\node[pin={[pin distance=-0.6\onedegree,Bayer]00:{$\boldsymbol\omicron$}}] at (axis cs:{175.053},{-34.744}) {}; % Hya,  4.70 
\node[pin={[pin distance=-0.6\onedegree,Bayer]00:{$\boldsymbol\psi$}}] at (axis cs:{197.264},{-23.118}) {}; % Hya,  4.94 
\node[pin={[pin distance=-0.6\onedegree,Bayer]-90:{$\boldsymbol\rho$}}] at (axis cs:{132.108},{5.838}) {}; % Hya,  4.35 
\node[pin={[pin distance=-0.6\onedegree,Bayer]00:{$\boldsymbol\sigma$}}] at (axis cs:{129.689},{3.341}) {}; % Hya,  4.45 
\node[pin={[pin distance=-0.6\onedegree,Bayer]00:{$\boldsymbol\tau^1$}}] at (axis cs:{142.287},{-2.769}) {}; % Hya,  4.92 
\node[pin={[pin distance=-0.6\onedegree,Bayer]00:{$\boldsymbol\tau^2$}}] at (axis cs:{142.996},{-1.185}) {}; % Hya,  4.55 
\node[pin={[pin distance=-0.6\onedegree,Bayer]00:{$\boldsymbol\upsilon^1$}}] at (axis cs:{147.870},{-14.846}) {}; % Hya,  4.11 
\node[pin={[pin distance=-0.6\onedegree,Bayer]00:{$\boldsymbol\upsilon^2$}}] at (axis cs:{151.281},{-13.064}) {}; % Hya,  4.60 
\node[pin={[pin distance=-0.6\onedegree,Bayer]00:{$\boldsymbol\varphi^3$}}] at (axis cs:{159.646},{-16.876}) {}; % Hya,  4.91 
\node[pin={[pin distance=-0.6\onedegree,Bayer]00:{$\boldsymbol\vartheta$}}] at (axis cs:{138.591},{2.314}) {}; % Hya,  3.89 
\node[pin={[pin distance=-0.6\onedegree,Bayer]00:{$\boldsymbol\xi$}}] at (axis cs:{173.250},{-31.858}) {}; % Hya,  3.54 
\node[pin={[pin distance=-0.6\onedegree,Bayer]00:{$\boldsymbol\zeta$}}] at (axis cs:{133.848},{5.945}) {}; % Hya,  3.12 
\node[pin={[pin distance=-0.6\onedegree,Flaamsted]00:{12}}] at (axis cs:{131.594},{-13.547}) {}; % Hya,  4.32 
\node[pin={[pin distance=-0.6\onedegree,Flaamsted]00:{14}}] at (axis cs:{132.341},{-3.443}) {}; % Hya,  5.30 
\node[pin={[pin distance=-0.6\onedegree,Flaamsted]00:{15}}] at (axis cs:{132.893},{-7.177}) {}; % Hya,  5.57 
\node[pin={[pin distance=-0.6\onedegree,Flaamsted]00:{19}}] at (axis cs:{137.176},{-8.589}) {}; % Hya,  5.62 
\node[pin={[pin distance=-0.6\onedegree,Flaamsted]00:{ 1}}] at (axis cs:{126.146},{-3.751}) {}; % Hya,  5.62 
\node[pin={[pin distance=-0.6\onedegree,Flaamsted]-90:{20}}] at (axis cs:{137.398},{-8.788}) {}; % Hya,  5.46 
\node[pin={[pin distance=-0.6\onedegree,Flaamsted]00:{23}}] at (axis cs:{139.174},{-6.353}) {}; % Hya,  5.23 
\node[pin={[pin distance=-0.6\onedegree,Flaamsted]00:{24}}] at (axis cs:{139.172},{-8.745}) {}; % Hya,  5.48 
\node[pin={[pin distance=-0.6\onedegree,Flaamsted]00:{26}}] at (axis cs:{139.943},{-11.975}) {}; % Hya,  4.78 
\node[pin={[pin distance=-0.6\onedegree,Flaamsted]00:{27}}] at (axis cs:{140.121},{-9.556}) {}; % Hya,  4.81 
\node[pin={[pin distance=-0.6\onedegree,Flaamsted]00:{28}}] at (axis cs:{141.350},{-5.117}) {}; % Hya,  5.58 
\node[pin={[pin distance=-0.6\onedegree,Flaamsted]-90:{ 2}}] at (axis cs:{126.613},{-3.987}) {}; % Hya,  5.59 
\node[pin={[pin distance=-0.6\onedegree,Flaamsted]00:{33}}] at (axis cs:{143.636},{-5.915}) {}; % Hya,  5.56 
\node[pin={[pin distance=-0.6\onedegree,Flaamsted]00:{ 3}}] at (axis cs:{128.867},{-7.982}) {}; % Hya,  5.73 
\node[pin={[pin distance=-0.6\onedegree,Flaamsted]00:{44}}] at (axis cs:{158.504},{-23.745}) {}; % Hya,  5.07 
\node[pin={[pin distance=-0.6\onedegree,Flaamsted]00:{ 6}}] at (axis cs:{130.006},{-12.475}) {}; % Hya,  4.97 
\node[pin={[pin distance=-0.6\onedegree,Flaamsted]00:{ 9}}] at (axis cs:{130.431},{-15.943}) {}; % Hya,  4.87 

\node[pin={[pin distance=-0.4\onedegree,Bayer]45:{$\boldsymbol\alpha$}}] at (axis cs:{152.093},{11.967}) {}; % Leo,  1.41 
\node[pin={[pin distance=-0.4\onedegree,Bayer]00:{$\boldsymbol\beta$}}] at (axis cs:{177.265},{14.572}) {}; % Leo,  2.13 
\node[pin={[pin distance=-0.4\onedegree,Bayer]00:{$\boldsymbol\gamma^{1,2}$}}] at (axis cs:{154.993},{19.841}) {}; % Leo,  2.23 
\node[pin={[pin distance=-0.6\onedegree,Bayer]00:{$\boldsymbol\chi$}}] at (axis cs:{166.254},{7.336}) {}; % Leo,  4.62 
\node[pin={[pin distance=-0.8\onedegree,Bayer]00:{$\boldsymbol\delta$}}] at (axis cs:{168.527},{20.524}) {}; % Leo,  2.56 
\node[pin={[pin distance=-0.6\onedegree,Bayer]00:{$\boldsymbol\epsilon$}}] at (axis cs:{146.463},{23.774}) {}; % Leo,  2.97 
\node[pin={[pin distance=-0.6\onedegree,Bayer]00:{$\boldsymbol\eta$}}] at (axis cs:{151.833},{16.762}) {}; % Leo,  3.52 
%\node[pin={[pin distance=-0.6\onedegree,Bayer]00:{$\boldsymbol\gamma^2$}}] at (axis cs:{154.994},{19.841}) {}; % Leo,  3.47 
\node[pin={[pin distance=-0.6\onedegree,Bayer]00:{$\boldsymbol\iota$}}] at (axis cs:{170.981},{10.529}) {}; % Leo,  3.96 
\node[pin={[pin distance=-0.6\onedegree,Bayer]180:{$\boldsymbol\kappa$}}] at (axis cs:{141.164},{26.182}) {}; % Leo,  4.47 
\node[pin={[pin distance=-0.6\onedegree,Bayer]00:{$\boldsymbol\lambda$}}] at (axis cs:{142.930},{22.968}) {}; % Leo,  4.30 
\node[pin={[pin distance=-0.6\onedegree,Bayer]00:{$\boldsymbol\mu$}}] at (axis cs:{148.191},{26.007}) {}; % Leo,  3.88 
\node[pin={[pin distance=-0.6\onedegree,Bayer]90:{$\boldsymbol\nu$}}] at (axis cs:{149.556},{12.445}) {}; % Leo,  5.26 
\node[pin={[pin distance=-0.6\onedegree,Bayer]00:{$\boldsymbol\omega$}}] at (axis cs:{142.114},{9.057}) {}; % Leo,  5.42 
\node[pin={[pin distance=-0.6\onedegree,Bayer]00:{$\boldsymbol\omicron$}}] at (axis cs:{145.288},{9.892}) {}; % Leo,  3.53 
\node[pin={[pin distance=-0.6\onedegree,Bayer]00:{$\boldsymbol\pi$}}] at (axis cs:{150.053},{8.044}) {}; % Leo,  4.67 
\node[pin={[pin distance=-0.6\onedegree,Bayer]90:{$\boldsymbol\psi$}}] at (axis cs:{145.933},{14.022}) {}; % Leo,  5.37 
\node[pin={[pin distance=-0.6\onedegree,Bayer]90:{$\boldsymbol\rho$}}] at (axis cs:{158.203},{9.306}) {}; % Leo,  3.85 
\node[pin={[pin distance=-0.6\onedegree,Bayer]00:{$\boldsymbol\sigma$}}] at (axis cs:{170.284},{6.029}) {}; % Leo,  4.05 
\node[pin={[pin distance=-0.6\onedegree,Bayer]00:{$\boldsymbol\tau$}}] at (axis cs:{171.984},{2.856}) {}; % Leo,  4.94 
\node[pin={[pin distance=-0.6\onedegree,Bayer]00:{$\boldsymbol\upsilon$}}] at (axis cs:{174.237},{-0.824}) {}; % Leo,  4.30 
\node[pin={[pin distance=-0.6\onedegree,Bayer]00:{$\boldsymbol\varphi$}}] at (axis cs:{169.165},{-3.652}) {}; % Leo,  4.47 
\node[pin={[pin distance=-0.6\onedegree,Bayer]00:{$\boldsymbol\vartheta$}}] at (axis cs:{168.560},{15.429}) {}; % Leo,  3.33 
\node[pin={[pin distance=-0.6\onedegree,Bayer]00:{$\boldsymbol\xi$}}] at (axis cs:{142.986},{11.300}) {}; % Leo,  4.97 
\node[pin={[pin distance=-0.6\onedegree,Bayer]00:{$\boldsymbol\zeta$}}] at (axis cs:{154.173},{23.417}) {}; % Leo,  3.44 
\node[pin={[pin distance=-0.6\onedegree,Flaamsted]00:{10}}] at (axis cs:{144.303},{6.836}) {}; % Leo,  5.01 
\node[pin={[pin distance=-0.6\onedegree,Flaamsted]00:{15}}] at (axis cs:{145.889},{29.974}) {}; % Leo,  5.64 
\node[pin={[pin distance=-0.6\onedegree,Flaamsted]00:{18}}] at (axis cs:{146.597},{11.810}) {}; % Leo,  5.66 
\node[pin={[pin distance=-0.6\onedegree,Flaamsted]00:{22}}] at (axis cs:{147.971},{24.395}) {}; % Leo,  5.29 
\node[pin={[pin distance=-0.6\onedegree,Flaamsted]-90:{31}}] at (axis cs:{151.976},{9.997}) {}; % Leo,  4.37 
%\node[pin={[pin distance=-0.6\onedegree,Flaamsted]00:{35}}] at (axis cs:{154.135},{23.503}) {}; % Leo,  5.97 
\node[pin={[pin distance=-0.6\onedegree,Flaamsted]00:{37}}] at (axis cs:{154.170},{13.728}) {}; % Leo,  5.42 
%\node[pin={[pin distance=-0.6\onedegree,Flaamsted]00:{39}}] at (axis cs:{154.311},{23.106}) {}; % Leo,  5.82 
\node[pin={[pin distance=-0.6\onedegree,Flaamsted]00:{ 3}}] at (axis cs:{142.122},{8.188}) {}; % Leo,  5.71 
%\node[pin={[pin distance=-0.6\onedegree,Flaamsted]00:{40}}] at (axis cs:{154.934},{19.471}) {}; % Leo,  4.78 
\node[pin={[pin distance=-0.6\onedegree,Flaamsted]00:{44}}] at (axis cs:{156.313},{8.785}) {}; % Leo,  5.62 
\node[pin={[pin distance=-0.6\onedegree,Flaamsted]00:{46}}] at (axis cs:{158.049},{14.137}) {}; % Leo,  5.44 
\node[pin={[pin distance=-0.6\onedegree,Flaamsted]00:{48}}] at (axis cs:{158.700},{6.954}) {}; % Leo,  5.08 
\node[pin={[pin distance=-0.6\onedegree,Flaamsted]00:{49}}] at (axis cs:{158.759},{8.650}) {}; % Leo,  5.67 
\node[pin={[pin distance=-0.6\onedegree,Flaamsted]00:{51}}] at (axis cs:{161.602},{18.891}) {}; % Leo,  5.50 
\node[pin={[pin distance=-0.6\onedegree,Flaamsted]00:{52}}] at (axis cs:{161.605},{14.194}) {}; % Leo,  5.48 
\node[pin={[pin distance=-0.6\onedegree,Flaamsted]00:{53}}] at (axis cs:{162.314},{10.545}) {}; % Leo,  5.32 
\node[pin={[pin distance=-0.6\onedegree,Flaamsted]-90:{54}}] at (axis cs:{163.903},{24.749}) {}; % Leo,  4.47 
\node[pin={[pin distance=-0.6\onedegree,Flaamsted]00:{55}}] at (axis cs:{163.927},{0.737}) {}; % Leo,  5.92 
\node[pin={[pin distance=-0.6\onedegree,Flaamsted]00:{56}}] at (axis cs:{164.006},{6.185}) {}; % Leo,  5.98 
\node[pin={[pin distance=-0.6\onedegree,Flaamsted]00:{58}}] at (axis cs:{165.140},{3.617}) {}; % Leo,  4.85 
\node[pin={[pin distance=-0.6\onedegree,Flaamsted]00:{59}}] at (axis cs:{165.187},{6.101}) {}; % Leo,  5.00 
\node[pin={[pin distance=-0.6\onedegree,Flaamsted]00:{60}}] at (axis cs:{165.582},{20.180}) {}; % Leo,  4.41 
\node[pin={[pin distance=-0.6\onedegree,Flaamsted]00:{61}}] at (axis cs:{165.457},{-2.485}) {}; % Leo,  4.72 
\node[pin={[pin distance=-0.6\onedegree,Flaamsted]-90:{62}}] at (axis cs:{165.902},{-0.001}) {}; % Leo,  5.94 
\node[pin={[pin distance=-0.6\onedegree,Flaamsted]00:{65}}] at (axis cs:{166.726},{1.955}) {}; % Leo,  5.53 
\node[pin={[pin distance=-0.6\onedegree,Flaamsted]00:{67}}] at (axis cs:{167.205},{24.658}) {}; % Leo,  5.69 
\node[pin={[pin distance=-0.6\onedegree,Flaamsted]-90:{69}}] at (axis cs:{168.440},{-0.069}) {}; % Leo,  5.40 
\node[pin={[pin distance=-0.6\onedegree,Flaamsted]00:{ 6}}] at (axis cs:{142.990},{9.716}) {}; % Leo,  5.07 
\node[pin={[pin distance=-0.6\onedegree,Flaamsted]00:{72}}] at (axis cs:{168.801},{23.095}) {}; % Leo,  4.57 
\node[pin={[pin distance=-0.6\onedegree,Flaamsted]00:{73}}] at (axis cs:{168.966},{13.307}) {}; % Leo,  5.31 
\node[pin={[pin distance=-0.6\onedegree,Flaamsted]00:{75}}] at (axis cs:{169.323},{2.010}) {}; % Leo,  5.17 
\node[pin={[pin distance=-0.6\onedegree,Flaamsted]00:{76}}] at (axis cs:{169.729},{1.650}) {}; % Leo,  5.90 
\node[pin={[pin distance=-0.6\onedegree,Flaamsted]90:{79}}] at (axis cs:{171.010},{1.408}) {}; % Leo,  5.39 
\node[pin={[pin distance=-0.6\onedegree,Flaamsted]00:{81}}] at (axis cs:{171.402},{16.456}) {}; % Leo,  5.58 
\node[pin={[pin distance=-0.6\onedegree,Flaamsted]00:{85}}] at (axis cs:{172.424},{15.413}) {}; % Leo,  5.73 
\node[pin={[pin distance=-0.6\onedegree,Flaamsted]00:{86}}] at (axis cs:{172.621},{18.410}) {}; % Leo,  5.55 
\node[pin={[pin distance=-0.6\onedegree,Flaamsted]00:{87}}] at (axis cs:{172.579},{-3.003}) {}; % Leo,  4.76 
\node[pin={[pin distance=-0.6\onedegree,Flaamsted]90:{89}}] at (axis cs:{173.591},{3.060}) {}; % Leo,  5.76 
\node[pin={[pin distance=-0.6\onedegree,Flaamsted]00:{ 8}}] at (axis cs:{144.261},{16.438}) {}; % Leo,  5.73 
\node[pin={[pin distance=-0.6\onedegree,Flaamsted]00:{90}}] at (axis cs:{173.677},{16.797}) {}; % Leo,  5.95 
\node[pin={[pin distance=-0.6\onedegree,Flaamsted]00:{92}}] at (axis cs:{175.196},{21.353}) {}; % Leo,  5.26 
\node[pin={[pin distance=-0.6\onedegree,Flaamsted]00:{93}}] at (axis cs:{176.996},{20.219}) {}; % Leo,  4.53 
\node[pin={[pin distance=-0.6\onedegree,Flaamsted]00:{95}}] at (axis cs:{178.919},{15.647}) {}; % Leo,  5.53 
\node[pin={[pin distance=-0.6\onedegree,Bayer]00:{$\boldsymbol\vartheta$}}] at (axis cs:{91.539},{-14.935}) {}; % Lep,  4.67 
\node[pin={[pin distance=-0.6\onedegree,Flaamsted]00:{17}}] at (axis cs:{91.246},{-16.484}) {}; % Lep,  4.97 
\node[pin={[pin distance=-0.6\onedegree,Flaamsted]180:{19}}] at (axis cs:{91.923},{-19.166}) {}; % Lep,  5.28 
\node[pin={[pin distance=-0.6\onedegree,Bayer]00:{$\boldsymbol\beta$}}] at (axis cs:{156.971},{36.707}) {}; % LMi,  4.21 
\node[pin={[pin distance=-0.6\onedegree,Flaamsted]00:{10}}] at (axis cs:{143.556},{36.397}) {}; % LMi,  4.55 
\node[pin={[pin distance=-0.6\onedegree,Flaamsted]00:{11}}] at (axis cs:{143.915},{35.810}) {}; % LMi,  5.39 
\node[pin={[pin distance=-0.6\onedegree,Flaamsted]180:{20}}] at (axis cs:{150.253},{31.924}) {}; % LMi,  5.38 
\node[pin={[pin distance=-0.6\onedegree,Flaamsted]00:{21}}] at (axis cs:{151.857},{35.245}) {}; % LMi,  4.49 
\node[pin={[pin distance=-0.6\onedegree,Flaamsted]00:{23}}] at (axis cs:{154.060},{29.310}) {}; % LMi,  5.49 
\node[pin={[pin distance=-0.6\onedegree,Flaamsted]00:{27}}] at (axis cs:{155.776},{33.908}) {}; % LMi,  5.89 
\node[pin={[pin distance=-0.6\onedegree,Flaamsted]-90:{28}}] at (axis cs:{156.036},{33.719}) {}; % LMi,  5.52 
\node[pin={[pin distance=-0.6\onedegree,Flaamsted]180:{30}}] at (axis cs:{156.478},{33.796}) {}; % LMi,  4.73 
\node[pin={[pin distance=-0.6\onedegree,Flaamsted]00:{32}}] at (axis cs:{157.527},{38.925}) {}; % LMi,  5.79 
\node[pin={[pin distance=-0.6\onedegree,Flaamsted]00:{33}}] at (axis cs:{157.964},{32.379}) {}; % LMi,  5.91 
\node[pin={[pin distance=-0.6\onedegree,Flaamsted]00:{34}}] at (axis cs:{158.379},{34.989}) {}; % LMi,  5.58 
\node[pin={[pin distance=-0.6\onedegree,Flaamsted]00:{37}}] at (axis cs:{159.680},{31.976}) {}; % LMi,  4.68 
\node[pin={[pin distance=-0.6\onedegree,Flaamsted]00:{38}}] at (axis cs:{159.782},{37.910}) {}; % LMi,  5.84 
\node[pin={[pin distance=-0.6\onedegree,Flaamsted]00:{40}}] at (axis cs:{160.758},{26.325}) {}; % LMi,  5.52 
\node[pin={[pin distance=-0.6\onedegree,Flaamsted]00:{41}}] at (axis cs:{160.854},{23.188}) {}; % LMi,  5.08 
\node[pin={[pin distance=-0.6\onedegree,Flaamsted]00:{42}}] at (axis cs:{161.466},{30.682}) {}; % LMi,  5.26 
\node[pin={[pin distance=-0.6\onedegree,Flaamsted]00:{46}}] at (axis cs:{163.328},{34.215}) {}; % LMi,  3.79 
\node[pin={[pin distance=-0.6\onedegree,Flaamsted]00:{ 7}}] at (axis cs:{142.680},{33.656}) {}; % LMi,  5.86 
\node[pin={[pin distance=-0.6\onedegree,Flaamsted]00:{ 8}}] at (axis cs:{142.885},{35.103}) {}; % LMi,  5.38 

\node[pin={[pin distance=-0.4\onedegree,Bayer]00:{$\boldsymbol\alpha$}}] at (axis cs:{140.264},{34.392}) {}; % Lyn,  3.13 
\node[pin={[pin distance=-0.6\onedegree,Flaamsted]90:{33}}] at (axis cs:{128.683},{36.420}) {}; % Lyn,  5.76 
\node[pin={[pin distance=-0.5\onedegree,Flaamsted]00:{38}}] at (axis cs:{139.711},{36.803}) {}; % Lyn,  3.82 
\node[pin={[pin distance=-0.6\onedegree,Flaamsted]00:{43}}] at (axis cs:{145.501},{39.758}) {}; % Lyn,  5.61 

\node[pin={[pin distance=-0.6\onedegree,Bayer]00:{$\boldsymbol\alpha$}}] at (axis cs:{115.312},{-9.551}) {}; % Mon,  3.94 
\node[pin={[pin distance=-0.6\onedegree,Bayer]00:{$\boldsymbol\beta$}}] at (axis cs:{97.204},{-7.033}) {}; % Mon,  4.64 
\node[pin={[pin distance=-0.6\onedegree,Bayer]00:{$\boldsymbol\delta$}}] at (axis cs:{107.966},{-0.493}) {}; % Mon,  4.15 
\node[pin={[pin distance=-0.6\onedegree,Bayer]00:{$\boldsymbol\epsilon$}}] at (axis cs:{95.942},{4.593}) {}; % Mon,  4.41 
\node[pin={[pin distance=-0.6\onedegree,Bayer]00:{$\boldsymbol\gamma$}}] at (axis cs:{93.714},{-6.275}) {}; % Mon,  3.98 
\node[pin={[pin distance=-0.6\onedegree,Bayer]00:{$\boldsymbol\zeta$}}] at (axis cs:{122.149},{-2.984}) {}; % Mon,  4.36 
\node[pin={[pin distance=-0.6\onedegree,Flaamsted]00:{10}}] at (axis cs:{96.990},{-4.762}) {}; % Mon,  5.05 
\node[pin={[pin distance=-0.6\onedegree,Flaamsted]00:{12}}] at (axis cs:{98.080},{4.856}) {}; % Mon,  5.85 
\node[pin={[pin distance=-0.6\onedegree,Flaamsted]-90:{13}}] at (axis cs:{98.226},{7.333}) {}; % Mon,  4.51 
\node[pin={[pin distance=-0.6\onedegree,Flaamsted]00:{16}}] at (axis cs:{101.635},{8.587}) {}; % Mon,  5.91 
\node[pin={[pin distance=-0.6\onedegree,Flaamsted]00:{17}}] at (axis cs:{101.833},{8.037}) {}; % Mon,  4.77 
\node[pin={[pin distance=-0.6\onedegree,Flaamsted]00:{18}}] at (axis cs:{101.965},{2.412}) {}; % Mon,  4.47 
\node[pin={[pin distance=-0.6\onedegree,Flaamsted]00:{19}}] at (axis cs:{105.728},{-4.239}) {}; % Mon,  4.98 
\node[pin={[pin distance=-0.6\onedegree,Flaamsted]00:{20}}] at (axis cs:{107.557},{-4.237}) {}; % Mon,  4.92 
%\node[pin={[pin distance=-0.6\onedegree,Flaamsted]00:{21}}] at (axis cs:{107.848},{-0.302}) {}; % Mon,  5.43 
\node[pin={[pin distance=-0.6\onedegree,Flaamsted]00:{25}}] at (axis cs:{114.320},{-4.111}) {}; % Mon,  5.14 
\node[pin={[pin distance=-0.6\onedegree,Flaamsted]00:{27}}] at (axis cs:{119.934},{-3.679}) {}; % Mon,  4.94 
\node[pin={[pin distance=-0.6\onedegree,Flaamsted]00:{28}}] at (axis cs:{120.306},{-1.392}) {}; % Mon,  4.68 
\node[pin={[pin distance=-0.6\onedegree,Flaamsted]00:{ 3}}] at (axis cs:{90.460},{-10.598}) {}; % Mon,  5.00 
\node[pin={[pin distance=-0.6\onedegree,Flaamsted]00:{ 7}}] at (axis cs:{94.928},{-7.823}) {}; % Mon,  5.26 

\node[pin={[pin distance=-0.6\onedegree,Bayer]00:{$\boldsymbol\chi^2$}}] at (axis cs:{90.980},{20.138}) {}; % Ori,  4.64 
\node[pin={[pin distance=-0.6\onedegree,Bayer]180:{$\boldsymbol\mu$}}] at (axis cs:{90.596},{9.647}) {}; % Ori,  4.13 
\node[pin={[pin distance=-0.6\onedegree,Bayer]00:{$\boldsymbol\nu$}}] at (axis cs:{91.893},{14.768}) {}; % Ori,  4.42 
\node[pin={[pin distance=-0.6\onedegree,Bayer]00:{$\boldsymbol\xi$}}] at (axis cs:{92.985},{14.209}) {}; % Ori,  4.46 
\node[pin={[pin distance=-0.6\onedegree,Flaamsted]00:{63}}] at (axis cs:{91.242},{5.420}) {}; % Ori,  5.66 
\node[pin={[pin distance=-0.6\onedegree,Flaamsted]-90:{64}}] at (axis cs:{90.864},{19.690}) {}; % Ori,  5.15 
\node[pin={[pin distance=-0.6\onedegree,Flaamsted]00:{66}}] at (axis cs:{91.243},{4.159}) {}; % Ori,  5.63 
\node[pin={[pin distance=-0.6\onedegree,Flaamsted]00:{68}}] at (axis cs:{93.006},{19.790}) {}; % Ori,  5.76 
\node[pin={[pin distance=-0.6\onedegree,Flaamsted]00:{69}}] at (axis cs:{93.014},{16.130}) {}; % Ori,  4.96 
\node[pin={[pin distance=-0.6\onedegree,Flaamsted]00:{71}}] at (axis cs:{93.712},{19.156}) {}; % Ori,  5.21 
\node[pin={[pin distance=-0.6\onedegree,Flaamsted]180:{72}}] at (axis cs:{93.855},{16.143}) {}; % Ori,  5.34 
\node[pin={[pin distance=-0.6\onedegree,Flaamsted]-90:{73}}] at (axis cs:{93.937},{12.551}) {}; % Ori,  5.48 
\node[pin={[pin distance=-0.6\onedegree,Flaamsted]180:{74}}] at (axis cs:{94.111},{12.272}) {}; % Ori,  5.04 
\node[pin={[pin distance=-0.6\onedegree,Flaamsted]00:{75}}] at (axis cs:{94.278},{9.942}) {}; % Ori,  5.40 

\node[pin={[pin distance=-0.6\onedegree,Bayer]00:{$\boldsymbol\omicron$}}] at (axis cs:{117.022},{-25.937}) {}; % Pup,  4.50 
\node[pin={[pin distance=-0.4\onedegree,Bayer]00:{$\boldsymbol\pi$}}] at (axis cs:{109.286},{-37.097}) {}; % Pup,  2.71 
\node[pin={[pin distance=-0.6\onedegree,Bayer]00:{$\boldsymbol\rho$}}] at (axis cs:{121.886},{-24.304}) {}; % Pup,  2.83 
\node[pin={[pin distance=-0.6\onedegree,Bayer]00:{$\boldsymbol\xi$}}] at (axis cs:{117.324},{-24.860}) {}; % Pup,  3.33 
\node[pin={[pin distance=-0.6\onedegree,Flaamsted]00:{10}}] at (axis cs:{118.079},{-14.846}) {}; % Pup,  5.69 
\node[pin={[pin distance=-0.6\onedegree,Flaamsted]00:{11}}] at (axis cs:{119.215},{-22.880}) {}; % Pup,  4.20 
\node[pin={[pin distance=-0.6\onedegree,Flaamsted]180:{12}}] at (axis cs:{119.774},{-23.310}) {}; % Pup,  5.09 
\node[pin={[pin distance=-0.6\onedegree,Flaamsted]00:{16}}] at (axis cs:{122.257},{-19.245}) {}; % Pup,  4.40 
\node[pin={[pin distance=-0.6\onedegree,Flaamsted]00:{18}}] at (axis cs:{122.666},{-13.799}) {}; % Pup,  5.53 
\node[pin={[pin distance=-0.6\onedegree,Flaamsted]00:{19}}] at (axis cs:{122.818},{-12.927}) {}; % Pup,  4.72 
\node[pin={[pin distance=-0.6\onedegree,Flaamsted]00:{ 1}}] at (axis cs:{115.885},{-28.411}) {}; % Pup,  4.61 
\node[pin={[pin distance=-0.6\onedegree,Flaamsted]00:{20}}] at (axis cs:{123.333},{-15.788}) {}; % Pup,  4.99 
\node[pin={[pin distance=-0.6\onedegree,Flaamsted]00:{ 3}}] at (axis cs:{115.952},{-28.955}) {}; % Pup,  3.98 
\node[pin={[pin distance=-0.6\onedegree,Flaamsted]00:{ 4}}] at (axis cs:{116.487},{-14.564}) {}; % Pup,  5.04 
\node[pin={[pin distance=-0.6\onedegree,Flaamsted]00:{ 5}}] at (axis cs:{116.986},{-12.193}) {}; % Pup,  5.62 
\node[pin={[pin distance=-0.6\onedegree,Flaamsted]00:{ 6}}] at (axis cs:{117.422},{-17.228}) {}; % Pup,  5.17 
\node[pin={[pin distance=-0.6\onedegree,Flaamsted]00:{ 9}}] at (axis cs:{117.943},{-13.898}) {}; % Pup,  5.17 
\node[pin={[pin distance=-0.6\onedegree,Bayer]00:{$\boldsymbol\alpha$}}] at (axis cs:{130.898},{-33.186}) {}; % Pyx,  3.69 
\node[pin={[pin distance=-0.6\onedegree,Bayer]00:{$\boldsymbol\beta$}}] at (axis cs:{130.026},{-35.308}) {}; % Pyx,  3.97 
\node[pin={[pin distance=-0.6\onedegree,Bayer]00:{$\boldsymbol\delta$}}] at (axis cs:{133.882},{-27.682}) {}; % Pyx,  4.88 
\node[pin={[pin distance=-0.6\onedegree,Bayer]00:{$\boldsymbol\epsilon$}}] at (axis cs:{137.485},{-30.365}) {}; % Pyx,  5.60 
\node[pin={[pin distance=-0.6\onedegree,Bayer]00:{$\boldsymbol\eta$}}] at (axis cs:{129.467},{-26.255}) {}; % Pyx,  5.25 
\node[pin={[pin distance=-0.6\onedegree,Bayer]00:{$\boldsymbol\gamma$}}] at (axis cs:{132.633},{-27.710}) {}; % Pyx,  4.02 
\node[pin={[pin distance=-0.6\onedegree,Bayer]00:{$\boldsymbol\kappa$}}] at (axis cs:{137.012},{-25.858}) {}; % Pyx,  4.61 
\node[pin={[pin distance=-0.6\onedegree,Bayer]00:{$\boldsymbol\lambda$}}] at (axis cs:{140.801},{-28.834}) {}; % Pyx,  4.72 
\node[pin={[pin distance=-0.6\onedegree,Bayer]00:{$\boldsymbol\vartheta$}}] at (axis cs:{140.373},{-25.965}) {}; % Pyx,  4.71 
\node[pin={[pin distance=-0.6\onedegree,Bayer]180:{$\boldsymbol\zeta$}}] at (axis cs:{129.927},{-29.561}) {}; % Pyx,  4.88 
\node[pin={[pin distance=-0.6\onedegree,Bayer]00:{$\boldsymbol\alpha$}}] at (axis cs:{151.985},{-0.372}) {}; % Sex,  4.48 
\node[pin={[pin distance=-0.6\onedegree,Bayer]00:{$\boldsymbol\beta$}}] at (axis cs:{157.573},{-0.637}) {}; % Sex,  5.07 
\node[pin={[pin distance=-0.6\onedegree,Bayer]00:{$\boldsymbol\delta$}}] at (axis cs:{157.370},{-2.739}) {}; % Sex,  5.186
\node[pin={[pin distance=-0.6\onedegree,Bayer]00:{$\boldsymbol\epsilon$}}] at (axis cs:{154.408},{-8.069}) {}; % Sex,  5.24 
\node[pin={[pin distance=-0.6\onedegree,Bayer]00:{$\boldsymbol\gamma$}}] at (axis cs:{148.127},{-8.105}) {}; % Sex,  5.08 
\node[pin={[pin distance=-0.6\onedegree,Flaamsted]00:{17}}] at (axis cs:{152.531},{-8.408}) {}; % Sex,  5.90 
\node[pin={[pin distance=-0.6\onedegree,Flaamsted]180:{18}}] at (axis cs:{152.733},{-8.418}) {}; % Sex,  5.62 
\node[pin={[pin distance=-0.6\onedegree,Flaamsted]00:{19}}] at (axis cs:{153.202},{4.614}) {}; % Sex,  5.78 
\node[pin={[pin distance=-0.6\onedegree,Flaamsted]00:{25}}] at (axis cs:{155.860},{-4.074}) {}; % Sex,  5.93 
\node[pin={[pin distance=-0.6\onedegree,Flaamsted]00:{41}}] at (axis cs:{162.575},{-8.898}) {}; % Sex,  5.80 
\node[pin={[pin distance=-0.6\onedegree,Bayer]00:{$\boldsymbol\nu$}}] at (axis cs:{169.620},{33.094}) {}; % UMa,  3.49 
\node[pin={[pin distance=-0.6\onedegree,Bayer]00:{$\boldsymbol\xi$}}] at (axis cs:{169.546},{31.529}) {}; % UMa,  3.79 
\node[pin={[pin distance=-0.6\onedegree,Flaamsted]180:{46}}] at (axis cs:{163.935},{33.507}) {}; % UMa,  5.03 
\node[pin={[pin distance=-0.6\onedegree,Flaamsted]00:{49}}] at (axis cs:{165.210},{39.212}) {}; % UMa,  5.07 
\node[pin={[pin distance=-0.6\onedegree,Flaamsted]00:{55}}] at (axis cs:{169.783},{38.186}) {}; % UMa,  4.76 
\node[pin={[pin distance=-0.6\onedegree,Flaamsted]00:{57}}] at (axis cs:{172.267},{39.337}) {}; % UMa,  5.36 
\node[pin={[pin distance=-0.6\onedegree,Flaamsted]00:{61}}] at (axis cs:{175.263},{34.202}) {}; % UMa,  5.31 
\node[pin={[pin distance=-0.6\onedegree,Flaamsted]00:{62}}] at (axis cs:{175.393},{31.746}) {}; % UMa,  5.75 
\node[pin={[pin distance=-0.6\onedegree,Bayer]00:{$\boldsymbol\beta$}}] at (axis cs:{177.674},{1.764}) {}; % Vir,  3.60 
\node[pin={[pin distance=-0.6\onedegree,Bayer]00:{$\boldsymbol\chi$}}] at (axis cs:{189.812},{-7.995}) {}; % Vir,  4.65 
\node[pin={[pin distance=-0.4\onedegree,Bayer]00:{$\boldsymbol\delta$}}] at (axis cs:{193.901},{3.397}) {}; % Vir,  3.42 
\node[pin={[pin distance=-0.4\onedegree,Bayer]180:{$\boldsymbol\epsilon$}}] at (axis cs:{195.544},{10.959}) {}; % Vir,  2.84 
\node[pin={[pin distance=-0.4\onedegree,Bayer]00:{$\boldsymbol\eta$}}] at (axis cs:{184.976},{-0.667}) {}; % Vir,  3.89 
\node[pin={[pin distance=-0.4\onedegree,Bayer]00:{$\boldsymbol\gamma$}}] at (axis cs:{190.415},{-1.449}) {}; % Vir,  3.48 
\node[pin={[pin distance=-0.6\onedegree,Bayer]00:{$\boldsymbol\nu$}}] at (axis cs:{176.465},{6.529}) {}; % Vir,  4.03 
\node[pin={[pin distance=-0.6\onedegree,Bayer]180:{$\boldsymbol\omega$}}] at (axis cs:{174.615},{8.134}) {}; % Vir,  5.32 
\node[pin={[pin distance=-0.6\onedegree,Bayer]00:{$\boldsymbol\omicron$}}] at (axis cs:{181.302},{8.733}) {}; % Vir,  4.12 
\node[pin={[pin distance=-0.6\onedegree,Bayer]00:{$\boldsymbol\pi$}}] at (axis cs:{180.218},{6.614}) {}; % Vir,  4.66 
\node[pin={[pin distance=-0.6\onedegree,Bayer]00:{$\boldsymbol\psi$}}] at (axis cs:{193.588},{-9.539}) {}; % Vir,  4.79 
\node[pin={[pin distance=-0.6\onedegree,Bayer]180:{$\boldsymbol\rho$}}] at (axis cs:{190.471},{10.236}) {}; % Vir,  4.87 
\node[pin={[pin distance=-0.6\onedegree,Bayer]00:{$\boldsymbol\sigma$}}] at (axis cs:{199.401},{5.470}) {}; % Vir,  4.79 
\node[pin={[pin distance=-0.6\onedegree,Bayer]00:{$\boldsymbol\vartheta$}}] at (axis cs:{197.487},{-5.539}) {}; % Vir,  4.39 
\node[pin={[pin distance=-0.6\onedegree,Bayer]90:{$\boldsymbol\xi$}}] at (axis cs:{176.321},{8.258}) {}; % Vir,  4.84 
\node[pin={[pin distance=-0.6\onedegree,Flaamsted]00:{10}}] at (axis cs:{182.422},{1.898}) {}; % Vir,  5.96 
\node[pin={[pin distance=-0.6\onedegree,Flaamsted]00:{11}}] at (axis cs:{182.514},{5.807}) {}; % Vir,  5.71 
\node[pin={[pin distance=-0.6\onedegree,Flaamsted]00:{12}}] at (axis cs:{183.358},{10.262}) {}; % Vir,  5.85 
%\node[pin={[pin distance=-0.6\onedegree,Flaamsted]00:{13}}] at (axis cs:{184.668},{-0.787}) {}; % Vir,  5.91 
\node[pin={[pin distance=-0.6\onedegree,Flaamsted]00:{16}}] at (axis cs:{185.087},{3.312}) {}; % Vir,  4.97 
\node[pin={[pin distance=-0.6\onedegree,Flaamsted]00:{21}}] at (axis cs:{188.445},{-9.452}) {}; % Vir,  5.50 
\node[pin={[pin distance=-0.6\onedegree,Flaamsted]00:{25}}] at (axis cs:{189.197},{-5.832}) {}; % Vir,  5.88 
\node[pin={[pin distance=-0.6\onedegree,Flaamsted]-90:{31}}] at (axis cs:{190.488},{6.807}) {}; % Vir,  5.58 
\node[pin={[pin distance=-0.6\onedegree,Flaamsted]00:{32}}] at (axis cs:{191.404},{7.673}) {}; % Vir,  5.21 
\node[pin={[pin distance=-0.6\onedegree,Flaamsted]00:{33}}] at (axis cs:{191.594},{9.540}) {}; % Vir,  5.66 
\node[pin={[pin distance=-0.6\onedegree,Flaamsted]00:{44}}] at (axis cs:{194.915},{-3.812}) {}; % Vir,  5.80 
\node[pin={[pin distance=-0.6\onedegree,Flaamsted]00:{46}}] at (axis cs:{195.150},{-3.368}) {}; % Vir,  5.99 
\node[pin={[pin distance=-0.6\onedegree,Flaamsted]00:{49}}] at (axis cs:{196.974},{-10.740}) {}; % Vir,  5.16 
\node[pin={[pin distance=-0.6\onedegree,Flaamsted]-90:{ 4}}] at (axis cs:{176.979},{8.246}) {}; % Vir,  5.32 
\node[pin={[pin distance=-0.6\onedegree,Flaamsted]00:{50}}] at (axis cs:{197.439},{-10.329}) {}; % Vir,  5.96 
\node[pin={[pin distance=-0.6\onedegree,Flaamsted]00:{53}}] at (axis cs:{198.015},{-16.198}) {}; % Vir,  5.04 
\node[pin={[pin distance=-0.6\onedegree,Flaamsted]00:{55}}] at (axis cs:{198.545},{-19.931}) {}; % Vir,  5.32 
\node[pin={[pin distance=-0.6\onedegree,Flaamsted]-90:{57}}] at (axis cs:{198.995},{-19.943}) {}; % Vir,  5.22 
\node[pin={[pin distance=-0.6\onedegree,Flaamsted]00:{59}}] at (axis cs:{199.194},{9.424}) {}; % Vir,  5.20 
\node[pin={[pin distance=-0.6\onedegree,Flaamsted]00:{61}}] at (axis cs:{199.601},{-18.311}) {}; % Vir,  4.74 
\node[pin={[pin distance=-0.6\onedegree,Flaamsted]00:{ 6}}] at (axis cs:{178.763},{8.444}) {}; % Vir,  5.58 
\node[pin={[pin distance=-0.6\onedegree,Flaamsted]00:{ 7}}] at (axis cs:{179.987},{3.655}) {}; % Vir,  5.36 
\end{axis}

%

%
% Comets, own objects etc.
\begin{axis}[name=transients,axis lines=none]
% Format for label:
%\node[transient,pin={[pin distance=-0.8\onedegree,transient-label]90:{LABEL}}] at (axis cs:{ *15 + /4 + /240 },{  + /60 + /3600 })  {+};
%
% Note for Southern positions, if in sexagesimal, decl. signs need to be inverted: 
%   \(axis cs:{ *15 + /4 + /240 },{   - /60 - /3600 })  {+};
%
% Note "08" is interpreted as Octal 8, and triggers a problem in computation

%%%%%%%%%%%%%%%%%%%%%%%%%%%%%%%%%%%%%%%%%%%%%%%%%%%%%%%%%%%%%%%%%%%%%%%%%%%%%%%%%%%%%%%%
% Comet C/2023 A3 (Tsuchinshan-ATLAS) in 2024, Apr-Dec\
% Coords taken from https://in-the-sky.org/ephemeris.php?objtxt=CK23A030
%%%%%%%%%%%%%%%%%%%%%%%%%%%%%%%%%%%%%%%%%%%%%%%%%%%%%%%%%%%%%%%%%%%%%%%%%%%%%%%%%%%%%%%%
%
% \node[transient,pin={[pin distance=-0.8\onedegree,transient-label]90:{Apr\,01}}] at (axis cs:{14 *15 + 32/4 +  30/240 },{ -04  - 59/60 -  6/3600 })  {+};
% \node[transient,pin={[pin distance=-0.8\onedegree,transient-label]90:{Apr\,08}}] at (axis cs:{14 *15 + 19/4 +  22/240 },{ -04  -  8/60 - 35/3600 })  {+};
% \node[transient,pin={[pin distance=-0.8\onedegree,transient-label]90:{Apr\,15}}] at (axis cs:{14 *15 +  4/4 +  04/240 },{ -03  - 11/60 - 56/3600 })  {+};
% \node[transient,pin={[pin distance=-0.8\onedegree,transient-label]90:{Apr\,22}}] at (axis cs:{13 *15 + 46/4 +  53/240 },{ -02  - 10/60 - 50/3600 })  {+};
% \node[transient,pin={[pin distance=-0.8\onedegree,transient-label]90:{May\,01}}] at (axis cs:{13 *15 + 22/4 +  47/240 },{ -00  - 50/60 -  5/3600 })  {+};
% \node[transient,pin={[pin distance=-0.8\onedegree,transient-label]90:{May\,08}}] at (axis cs:{13 *15 +  3/4 +  21/240 },{ +00  + 10/60 +  8/3600 })  {+};
% \node[transient,pin={[pin distance=-0.8\onedegree,transient-label]90:{May\,15}}] at (axis cs:{12 *15 + 44/4 +   8/240 },{ +01  +  4/60 + 17/3600 })  {+};
% \node[transient,pin={[pin distance=-0.8\onedegree,transient-label]90:{May\,22}}] at (axis cs:{12 *15 + 25/4 +  54/240 },{ +01  + 49/60 + 32/3600 })  {+};
% \node[transient,pin={[pin distance=-0.8\onedegree,transient-label]90:{Jun\,01}}] at (axis cs:{12 *15 +  2/4 +  36/240 },{ +02  + 35/60 + 35/3600 })  {+};
% \node[transient,pin={[pin distance=-0.8\onedegree,transient-label]90:{Jun\,08}}] at (axis cs:{11 *15 + 48/4 +  37/240 },{ +02  + 54/60 + 16/3600 })  {+};
% \node[transient,pin={[pin distance=-0.8\onedegree,transient-label]90:{Jun\,15}}] at (axis cs:{11 *15 + 36/4 +  38/240 },{ +03  +  2/60 + 21/3600 })  {+};
% \node[transient,pin={[pin distance=-0.8\onedegree,transient-label]90:{Jun\,22}}] at (axis cs:{11 *15 + 26/4 +  34/240 },{ +03  +  0/60 + 47/3600 })  {+};
% \node[transient,pin={[pin distance=-0.8\onedegree,transient-label]90:{Jul\,01}}] at (axis cs:{11 *15 + 16/4 +  10/240 },{ +02  + 46/60 + 21/3600 })  {+};
% \node[transient,pin={[pin distance=-0.8\onedegree,transient-label]90:{Jul\,08}}] at (axis cs:{11 *15 +  9/4 +  44/240 },{ +02  + 26/60 + 41/3600 })  {+};
% \node[transient,pin={[pin distance=-0.8\onedegree,transient-label]90:{Jul\,15}}] at (axis cs:{11 *15 +  4/4 +  32/240 },{ +02  +  0/60 + 32/3600 })  {+};
% \node[transient,pin={[pin distance=-0.8\onedegree,transient-label]90:{Jul\,22}}] at (axis cs:{11 *15 +  0/4 +  19/240 },{ +01  + 28/60 + 34/3600 })  {+};
% \node[transient,pin={[pin distance=-0.8\onedegree,transient-label]90:{Aug\,01}}] at (axis cs:{10 *15 + 55/4 +  30/240 },{ +00  + 33/60 + 43/3600 })  {+};
% \node[transient,pin={[pin distance=-0.8\onedegree,transient-label]90:{Aug\,08}}] at (axis cs:{10 *15 + 52/4 +  42/240 },{ -00  - 10/60 - 40/3600 })  {+};
% \node[transient,pin={[pin distance=-0.8\onedegree,transient-label]90:{Aug\,15}}] at (axis cs:{10 *15 + 50/4 +  05/240 },{ -00  - 59/60 - 45/3600 })  {+};
% \node[transient,pin={[pin distance=-0.8\onedegree,transient-label]90:{Aug\,22}}] at (axis cs:{10 *15 + 47/4 +  26/240 },{ -01  - 53/60 - 19/3600 })  {+};
% \node[transient,pin={[pin distance=-0.8\onedegree,transient-label]90:{Sep\,01}}] at (axis cs:{10 *15 + 43/4 +  06/240 },{ -03  - 16/60 - 41/3600 })  {+};
% \node[transient,pin={[pin distance=-0.8\onedegree,transient-label]90:{Sep\,08}}] at (axis cs:{10 *15 + 39/4 +  30/240 },{ -04  - 17/60 - 50/3600 })  {+};
% \node[transient,pin={[pin distance=-0.8\onedegree,transient-label]90:{Sep\,15}}] at (axis cs:{10 *15 + 35/4 +  57/240 },{ -05  - 16/60 -  5/3600 })  {+};
% \node[transient,pin={[pin distance=-0.8\onedegree,transient-label]90:{Sep\,22}}] at (axis cs:{10 *15 + 35/4 +  49/240 },{ -05  - 58/60 -  7/3600 })  {+};
% \node[transient,pin={[pin distance=-0.8\onedegree,transient-label]90:{Oct\,01}}] at (axis cs:{11 *15 +  4/4 +  03/240 },{ -05  - 41/60 - 36/3600 })  {+};
% \node[transient,pin={[pin distance=-0.8\onedegree,transient-label]90:{Oct\,08}}] at (axis cs:{12 *15 + 30/4 +  27/240 },{ -03  - 36/60 - 16/3600 })  {+};
% \node[transient,pin={[pin distance=-0.8\onedegree,transient-label]90:{Oct\,15}}] at (axis cs:{14 *15 + 59/4 +  40/240 },{ +00  + 23/60 + 14/3600 })  {+};
% \node[transient,pin={[pin distance=-0.8\onedegree,transient-label]90:{Oct\,22}}] at (axis cs:{16 *15 + 49/4 +  25/240 },{ +02  + 50/60 + 11/3600 })  {+};
% \node[transient,pin={[pin distance=-0.8\onedegree,transient-label]90:{Nov\,01}}] at (axis cs:{17 *15 + 59/4 +  40/240 },{ +03  + 45/60 + 12/3600 })  {+};
% \node[transient,pin={[pin distance=-0.8\onedegree,transient-label]90:{Nov\,08}}] at (axis cs:{18 *15 + 24/4 +  28/240 },{ +03  + 55/60 + 51/3600 })  {+};
% \node[transient,pin={[pin distance=-0.8\onedegree,transient-label]90:{Nov\,15}}] at (axis cs:{18 *15 + 41/4 +  24/240 },{ +04  +  3/60 + 27/3600 })  {+};
% \node[transient,pin={[pin distance=-0.8\onedegree,transient-label]90:{Nov\,22}}] at (axis cs:{18 *15 + 54/4 +  18/240 },{ +04  + 12/60 + 40/3600 })  {+};
% \node[transient,pin={[pin distance=-0.8\onedegree,transient-label]90:{Dec\,01}}] at (axis cs:{19 *15 +  7/4 +  39/240 },{ +04  + 29/60 + 30/3600 })  {+};
% \node[transient,pin={[pin distance=-0.8\onedegree,transient-label]90:{Dec\,08}}] at (axis cs:{19 *15 + 16/4 +  32/240 },{ +04  + 47/60 + 12/3600 })  {+};
% \node[transient,pin={[pin distance=-0.8\onedegree,transient-label]90:{Dec\,15}}] at (axis cs:{19 *15 + 24/4 +  32/240 },{ +05  +  9/60 +  7/3600 })  {+};
% \node[transient,pin={[pin distance=-0.8\onedegree,transient-label]90:{Dec\,22}}] at (axis cs:{19 *15 + 31/4 +  54/240 },{ +05  + 35/60 + 17/3600 })  {+};
% 
% 
% \draw[transient]
% (axis cs:{14 *15 + 32/4 + 30/240 },{ -04  - 59/60 -  6/3600 }) --
% (axis cs:{14 *15 + 19/4 + 22/240 },{ -04  -  8/60 - 35/3600 }) --
% (axis cs:{14 *15 +  4/4 + 04/240 },{ -03  - 11/60 - 56/3600 }) --
% (axis cs:{13 *15 + 46/4 + 53/240 },{ -02  - 10/60 - 50/3600 }) --
% (axis cs:{13 *15 + 22/4 + 47/240 },{ -00  - 50/60 -  5/3600 }) --
% (axis cs:{13 *15 +  3/4 + 21/240 },{ +00  + 10/60 +  8/3600 }) --
% (axis cs:{12 *15 + 44/4 +  8/240 },{ +01  +  4/60 + 17/3600 }) --
% (axis cs:{12 *15 + 25/4 + 54/240 },{ +01  + 49/60 + 32/3600 }) --
% (axis cs:{12 *15 +  2/4 + 36/240 },{ +02  + 35/60 + 35/3600 }) --
% (axis cs:{11 *15 + 48/4 + 37/240 },{ +02  + 54/60 + 16/3600 }) --
% (axis cs:{11 *15 + 36/4 + 38/240 },{ +03  +  2/60 + 21/3600 }) --
% (axis cs:{11 *15 + 26/4 + 34/240 },{ +03  +  0/60 + 47/3600 }) --
% (axis cs:{11 *15 + 16/4 + 10/240 },{ +02  + 46/60 + 21/3600 }) --
% (axis cs:{11 *15 +  9/4 + 44/240 },{ +02  + 26/60 + 41/3600 }) --
% (axis cs:{11 *15 +  4/4 + 32/240 },{ +02  +  0/60 + 32/3600 }) --
% (axis cs:{11 *15 +  0/4 + 19/240 },{ +01  + 28/60 + 34/3600 }) --
% (axis cs:{10 *15 + 55/4 + 30/240 },{ +00  + 33/60 + 43/3600 }) --
% (axis cs:{10 *15 + 52/4 + 42/240 },{ -00  - 10/60 - 40/3600 }) --
% (axis cs:{10 *15 + 50/4 + 05/240 },{ -00  - 59/60 - 45/3600 }) --
% (axis cs:{10 *15 + 47/4 + 26/240 },{ -01  - 53/60 - 19/3600 }) --
% (axis cs:{10 *15 + 43/4 + 06/240 },{ -03  - 16/60 - 41/3600 }) --
% (axis cs:{10 *15 + 39/4 + 30/240 },{ -04  - 17/60 - 50/3600 }) --
% (axis cs:{10 *15 + 35/4 + 57/240 },{ -05  - 16/60 -  5/3600 }) --
% (axis cs:{10 *15 + 35/4 + 49/240 },{ -05  - 58/60 -  7/3600 }) --
% (axis cs:{11 *15 +  4/4 + 03/240 },{ -05  - 41/60 - 36/3600 }) --
% (axis cs:{12 *15 + 30/4 + 27/240 },{ -03  - 36/60 - 16/3600 }) --
% (axis cs:{14 *15 + 59/4 + 40/240 },{ +00  + 23/60 + 14/3600 }) --
% (axis cs:{16 *15 + 49/4 + 25/240 },{ +02  + 50/60 + 11/3600 }) --
% (axis cs:{17 *15 + 59/4 + 40/240 },{ +03  + 45/60 + 12/3600 }) --
% (axis cs:{18 *15 + 24/4 + 28/240 },{ +03  + 55/60 + 51/3600 }) --
% (axis cs:{18 *15 + 41/4 + 24/240 },{ +04  +  3/60 + 27/3600 }) --
% (axis cs:{18 *15 + 54/4 + 18/240 },{ +04  + 12/60 + 40/3600 }) --
% (axis cs:{19 *15 +  7/4 + 39/240 },{ +04  + 29/60 + 30/3600 }) --
% (axis cs:{19 *15 + 16/4 + 32/240 },{ +04  + 47/60 + 12/3600 }) --
% (axis cs:{19 *15 + 24/4 + 32/240 },{ +05  +  9/60 +  7/3600 }) --
% (axis cs:{19 *15 + 31/4 + 54/240 },{ +05  + 35/60 + 17/3600 }) ;
% 
% 

\end{axis}

\end{tikzpicture}


\end{document}



