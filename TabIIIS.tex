\documentclass[10pt,landscape]{article}
% rm TabulaIII_S.pdf ; pdflatex -shell-escape TabIIIS.tex 


\usepackage[margin=1cm,a3paper]{geometry}

\def\pgfsysdriver{pgfsys-pdftex.def}
\usepackage{pgfplots}
\usepgfplotslibrary{external} 
\usetikzlibrary{pgfplots.external}
\usetikzlibrary{calc}
\usetikzlibrary{shapes}
\usetikzlibrary{patterns}
\usetikzlibrary{plotmarks}
\usepgflibrary{arrows}
\usepgfplotslibrary{polar}
\tikzexternalize

\usepackage{times,latexsym,amssymb}
\usepackage{eulervm} % math font
\pagestyle{empty}

\usepackage{amsmath} 
\usepackage{nicefrac} 
%
% This is used because from 1.11 onwards the use of \pgfdeclareplotmark will
% introduce a shift of the stellar markers.
%
\pgfplotsset{compat=1.10}
%
% Use to increase spacing for constellation labels
\usepackage[letterspace=400]{microtype}
%
\usepackage{pagecolor}
\usepackage{natbib}
%


%\pagecolor{black}
\pagecolor{white}

%%%%%%%%%%%%%%%%%%%%%%%%%%%%%%%%%%%%%%%%%%%%%%%%%%%%%%%%%%%%%%%%%%%%%%
\begin{document}




%%%%%%%%%%%%%%%%%%%%%%%%%%%%%%%%%%%%%%%%%%%%%%%%%%%%%%%%%%%%%%%%%%%%%%%%%%%%%%%%%%%%%%%%%%
%%%%%%%%%%%%%%%%%%%%%%%%%%%%%%%%%%%%%%%%%%%%%%%%%%%%%%%%%%%%%%%%%%%%%%%%%%%%%%%%%%%%%%%%%%
%   Magnitude markers
%%%%%%%%%%%%%%%%%%%%%%%%%%%%%%%%%%%%%%%%%%%%%%%%%%%%%%%%%%%%%%%%%%%%%%%%%%%%%%%%%%%%%%%%%%
%%%%%%%%%%%%%%%%%%%%%%%%%%%%%%%%%%%%%%%%%%%%%%%%%%%%%%%%%%%%%%%%%%%%%%%%%%%%%%%%%%%%%%%%%%

\pgfdeclareplotmark{m1b}{%
\node[scale=\mOnescale*\AtoBscale*\symscale] at (0,0) {\tikz {%
\draw[fill=black,line width=.3pt,even odd rule,rotate=90] 
(0.*25.71428:10pt)  -- (0.5*25.71428:5pt)   -- 
(1.*25.71428:10pt)  -- (1.5*25.71428:5pt)   -- 
(2.*25.71428:10pt)  -- (2.5*25.71428:5pt)   -- 
(3.*25.71428:10pt)  -- (3.5*25.71428:5pt)   -- 
(4.*25.71428:10pt)  -- (4.5*25.71428:5pt)   -- 
(5.*25.71428:10pt)  -- (5.5*25.71428:5pt)   -- 
(6.*25.71428:10pt)  -- (6.5*25.71428:5pt)   -- 
(7.*25.71428:10pt)  -- (7.5*25.71428:5pt)   -- 
(8.*25.71428:10pt)  -- (8.5*25.71428:5pt)   -- 
(9.*25.71428:10pt)  -- (9.5*25.71428:5pt)   -- 
(10.*25.71428:10pt) -- (10.5*25.71428:5pt)  -- 
(11.*25.71428:10pt) -- (11.5*25.71428:5pt)  -- 
(12.*25.71428:10pt) -- (12.5*25.71428:5pt)  -- 
(13.*25.71428:10pt) -- (13.5*25.71428:5pt)  --  cycle;
}}}


\pgfdeclareplotmark{m1bv}{%
\node[scale=\mOnescale*\AtoBscale*\symscale] at (0,0) {\tikz {%
\draw[fill=black,line width=.3pt,even odd rule,rotate=90] 
(0.*25.71428:9pt)  -- (0.5*25.71428:4.5pt)   -- 
(1.*25.71428:9pt)  -- (1.5*25.71428:4.5pt)   -- 
(2.*25.71428:9pt)  -- (2.5*25.71428:4.5pt)   -- 
(3.*25.71428:9pt)  -- (3.5*25.71428:4.5pt)   -- 
(4.*25.71428:9pt)  -- (4.5*25.71428:4.5pt)   -- 
(5.*25.71428:9pt)  -- (5.5*25.71428:4.5pt)   -- 
(6.*25.71428:9pt)  -- (6.5*25.71428:4.5pt)   -- 
(7.*25.71428:9pt)  -- (7.5*25.71428:4.5pt)   -- 
(8.*25.71428:9pt)  -- (8.5*25.71428:4.5pt)   -- 
(9.*25.71428:9pt)  -- (9.5*25.71428:4.5pt)   -- 
(10.*25.71428:9pt) -- (10.5*25.71428:4.5pt)  -- 
(11.*25.71428:9pt) -- (11.5*25.71428:4.5pt)  -- 
(12.*25.71428:9pt) -- (12.5*25.71428:4.5pt)  -- 
(13.*25.71428:9pt) -- (13.5*25.71428:4.5pt)  --cycle;
\draw (0,0) circle (9pt);
%\draw[line width=.2pt,even odd rule,rotate=90] 
%(0.*25.71428:11pt)  -- (0.5*25.71428:6.5pt)   -- 
%(1.*25.71428:11pt)  -- (1.5*25.71428:6.5pt)   -- 
%(2.*25.71428:11pt)  -- (2.5*25.71428:6.5pt)   -- 
%(3.*25.71428:11pt)  -- (3.5*25.71428:6.5pt)   -- 
%(4.*25.71428:11pt)  -- (4.5*25.71428:6.5pt)   -- 
%(5.*25.71428:11pt)  -- (5.5*25.71428:6.5pt)   -- 
%(6.*25.71428:11pt)  -- (6.5*25.71428:6.5pt)   -- 
%(7.*25.71428:11pt)  -- (7.5*25.71428:6.5pt)   -- 
%(8.*25.71428:11pt)  -- (8.5*25.71428:6.5pt)   -- 
%(9.*25.71428:11pt)  -- (9.5*25.71428:6.5pt)   -- 
%(10.*25.71428:11pt) -- (10.5*25.71428:6.5pt)  -- 
%(11.*25.71428:11pt) -- (11.5*25.71428:6.5pt)  -- 
%(12.*25.71428:11pt) -- (12.5*25.71428:6.5pt)  -- 
%(13.*25.71428:11pt) -- (13.5*25.71428:6.5pt)  --  cycle;
}}}



\pgfdeclareplotmark{m1bb}{%
\node[scale=\mOnescale*\AtoBscale*\symscale] at (0,0) {\tikz {%
\draw[fill=black,line width=.3pt,even odd rule,rotate=90] 
(0.*25.71428:10pt)  -- (0.5*25.71428:5pt)   -- 
(1.*25.71428:10pt)  -- (1.5*25.71428:5pt)   -- 
(2.*25.71428:10pt)  -- (2.5*25.71428:5pt)   -- 
(3.*25.71428:10pt)  -- (3.5*25.71428:5pt)   -- 
(4.*25.71428:10pt)  -- (4.5*25.71428:5pt)   -- 
(5.*25.71428:10pt)  -- (5.5*25.71428:5pt)   -- 
(6.*25.71428:10pt)  -- (6.5*25.71428:5pt)   -- 
(7.*25.71428:10pt)  -- (7.5*25.71428:5pt)   -- 
(8.*25.71428:10pt)  -- (8.5*25.71428:5pt)   -- 
(9.*25.71428:10pt)  -- (9.5*25.71428:5pt)   -- 
(10.*25.71428:10pt) -- (10.5*25.71428:5pt)  -- 
(11.*25.71428:10pt) -- (11.5*25.71428:5pt)  -- 
(12.*25.71428:10pt) -- (12.5*25.71428:5pt)  -- 
(13.*25.71428:10pt) -- (13.5*25.71428:5pt)  --   cycle;
\draw[line width=.4pt,even odd rule,rotate=90] 
(0.5*25.71428:9pt)  -- (0.5*25.71428:5pt)   
(1.5*25.71428:9pt)  -- (1.5*25.71428:5pt)   
(2.5*25.71428:9pt)  -- (2.5*25.71428:5pt)   
(3.5*25.71428:9pt)  -- (3.5*25.71428:5pt)   
(4.5*25.71428:9pt)  -- (4.5*25.71428:5pt)   
(5.5*25.71428:9pt)  -- (5.5*25.71428:5pt)   
(6.5*25.71428:9pt)  -- (6.5*25.71428:5pt)   
(7.5*25.71428:9pt)  -- (7.5*25.71428:5pt)   
(8.5*25.71428:9pt)  -- (8.5*25.71428:5pt)   
(9.5*25.71428:9pt)  -- (9.5*25.71428:5pt)   
(10.5*25.71428:9pt) -- (10.5*25.71428:5pt)  
(11.5*25.71428:9pt) -- (11.5*25.71428:5pt)  
(12.5*25.71428:9pt) -- (12.5*25.71428:5pt)  
(13.5*25.71428:9pt) -- (13.5*25.71428:5pt)  ;
 }}}

\pgfdeclareplotmark{m1bvb}{%
\node[scale=\mOnescale*\AtoBscale*\symscale] at (0,0) {\tikz {%
\draw[fill=black,line width=.3pt,even odd rule,rotate=90] 
(0.*25.71428:9pt)  -- (0.5*25.71428:4.5pt)   -- 
(1.*25.71428:9pt)  -- (1.5*25.71428:4.5pt)   -- 
(2.*25.71428:9pt)  -- (2.5*25.71428:4.5pt)   -- 
(3.*25.71428:9pt)  -- (3.5*25.71428:4.5pt)   -- 
(4.*25.71428:9pt)  -- (4.5*25.71428:4.5pt)   -- 
(5.*25.71428:9pt)  -- (5.5*25.71428:4.5pt)   -- 
(6.*25.71428:9pt)  -- (6.5*25.71428:4.5pt)   -- 
(7.*25.71428:9pt)  -- (7.5*25.71428:4.5pt)   -- 
(8.*25.71428:9pt)  -- (8.5*25.71428:4.5pt)   -- 
(9.*25.71428:9pt)  -- (9.5*25.71428:4.5pt)   -- 
(10.*25.71428:9pt) -- (10.5*25.71428:4.5pt)  -- 
(11.*25.71428:9pt) -- (11.5*25.71428:4.5pt)  -- 
(12.*25.71428:9pt) -- (12.5*25.71428:4.5pt)  -- 
(13.*25.71428:9pt) -- (13.5*25.71428:4.5pt)  --  cycle;
\draw (0,0) circle (9pt);
%\draw[line width=.2pt,even odd rule,rotate=90] 
%(0.*25.71428:11pt)  -- (0.5*25.71428:6.5pt)   -- 
%(1.*25.71428:11pt)  -- (1.5*25.71428:6.5pt)   -- 
%(2.*25.71428:11pt)  -- (2.5*25.71428:6.5pt)   -- 
%(3.*25.71428:11pt)  -- (3.5*25.71428:6.5pt)   -- 
%(4.*25.71428:11pt)  -- (4.5*25.71428:6.5pt)   -- 
%(5.*25.71428:11pt)  -- (5.5*25.71428:6.5pt)   -- 
%(6.*25.71428:11pt)  -- (6.5*25.71428:6.5pt)   -- 
%(7.*25.71428:11pt)  -- (7.5*25.71428:6.5pt)   -- 
%(8.*25.71428:11pt)  -- (8.5*25.71428:6.5pt)   -- 
%(9.*25.71428:11pt)  -- (9.5*25.71428:6.5pt)   -- 
%(10.*25.71428:11pt) -- (10.5*25.71428:6.5pt)  -- 
%(11.*25.71428:11pt) -- (11.5*25.71428:6.5pt)  -- 
%(12.*25.71428:11pt) -- (12.5*25.71428:6.5pt)  -- 
%(13.*25.71428:11pt) -- (13.5*25.71428:6.5pt)  --  cycle;
\draw[line width=.4pt,even odd rule,rotate=90] 
(0.5*25.71428:10pt)  -- (0.5*25.71428:6.5pt)   
(1.5*25.71428:10pt)  -- (1.5*25.71428:6.5pt)   
(2.5*25.71428:10pt)  -- (2.5*25.71428:6.5pt)   
(3.5*25.71428:10pt)  -- (3.5*25.71428:6.5pt)   
(4.5*25.71428:10pt)  -- (4.5*25.71428:6.5pt)   
(5.5*25.71428:10pt)  -- (5.5*25.71428:6.5pt)   
(6.5*25.71428:10pt)  -- (6.5*25.71428:6.5pt)   
(7.5*25.71428:10pt)  -- (7.5*25.71428:6.5pt)   
(8.5*25.71428:10pt)  -- (8.5*25.71428:6.5pt)   
(9.5*25.71428:10pt)  -- (9.5*25.71428:6.5pt)   
(10.5*25.71428:10pt) -- (10.5*25.71428:6.5pt)  
(11.5*25.71428:10pt) -- (11.5*25.71428:6.5pt)  
(12.5*25.71428:10pt) -- (12.5*25.71428:6.5pt)  
(13.5*25.71428:10pt) -- (13.5*25.71428:6.5pt)  ;
 }}}



\pgfdeclareplotmark{m1c}{%
\node[scale=\mOnescale*\AtoCscale*\symscale] at (0,0) {\tikz {%
\draw[fill=black,line width=.3pt,even odd rule,rotate=90] 
(0.*25.71428:10pt)  -- (0.5*25.71428:5pt)   -- 
(1.*25.71428:10pt)  -- (1.5*25.71428:5pt)   -- 
(2.*25.71428:10pt)  -- (2.5*25.71428:5pt)   -- 
(3.*25.71428:10pt)  -- (3.5*25.71428:5pt)   -- 
(4.*25.71428:10pt)  -- (4.5*25.71428:5pt)   -- 
(5.*25.71428:10pt)  -- (5.5*25.71428:5pt)   -- 
(6.*25.71428:10pt)  -- (6.5*25.71428:5pt)   -- 
(7.*25.71428:10pt)  -- (7.5*25.71428:5pt)   -- 
(8.*25.71428:10pt)  -- (8.5*25.71428:5pt)   -- 
(9.*25.71428:10pt)  -- (9.5*25.71428:5pt)   -- 
(10.*25.71428:10pt) -- (10.5*25.71428:5pt)  -- 
(11.*25.71428:10pt) -- (11.5*25.71428:5pt)  -- 
(12.*25.71428:10pt) -- (12.5*25.71428:5pt)  -- 
(13.*25.71428:10pt) -- (13.5*25.71428:5pt)  --  cycle
(0,0) circle (2pt);
 }}}


\pgfdeclareplotmark{m1cv}{%
\node[scale=\mOnescale*\AtoCscale*\symscale] at (0,0) {\tikz {%
\draw[fill=black,line width=.3pt,even odd rule,rotate=90] 
(0.*25.71428:9pt)  -- (0.5*25.71428:4.5pt)   -- 
(1.*25.71428:9pt)  -- (1.5*25.71428:4.5pt)   -- 
(2.*25.71428:9pt)  -- (2.5*25.71428:4.5pt)   -- 
(3.*25.71428:9pt)  -- (3.5*25.71428:4.5pt)   -- 
(4.*25.71428:9pt)  -- (4.5*25.71428:4.5pt)   -- 
(5.*25.71428:9pt)  -- (5.5*25.71428:4.5pt)   -- 
(6.*25.71428:9pt)  -- (6.5*25.71428:4.5pt)   -- 
(7.*25.71428:9pt)  -- (7.5*25.71428:4.5pt)   -- 
(8.*25.71428:9pt)  -- (8.5*25.71428:4.5pt)   -- 
(9.*25.71428:9pt)  -- (9.5*25.71428:4.5pt)   -- 
(10.*25.71428:9pt) -- (10.5*25.71428:4.5pt)  -- 
(11.*25.71428:9pt) -- (11.5*25.71428:4.5pt)  -- 
(12.*25.71428:9pt) -- (12.5*25.71428:4.5pt)  -- 
(13.*25.71428:9pt) -- (13.5*25.71428:4.5pt)  -- cycle
(0,0) circle (2pt);
\draw (0,0) circle (9pt);
%\draw[line width=.2pt,even odd rule,rotate=90] 
%(0.*25.71428:11pt)  -- (0.5*25.71428:6.5pt)   -- 
%(1.*25.71428:11pt)  -- (1.5*25.71428:6.5pt)   -- 
%(2.*25.71428:11pt)  -- (2.5*25.71428:6.5pt)   -- 
%(3.*25.71428:11pt)  -- (3.5*25.71428:6.5pt)   -- 
%(4.*25.71428:11pt)  -- (4.5*25.71428:6.5pt)   -- 
%(5.*25.71428:11pt)  -- (5.5*25.71428:6.5pt)   -- 
%(6.*25.71428:11pt)  -- (6.5*25.71428:6.5pt)   -- 
%(7.*25.71428:11pt)  -- (7.5*25.71428:6.5pt)   -- 
%(8.*25.71428:11pt)  -- (8.5*25.71428:6.5pt)   -- 
%(9.*25.71428:11pt)  -- (9.5*25.71428:6.5pt)   -- 
%(10.*25.71428:11pt) -- (10.5*25.71428:6.5pt)  -- 
%(11.*25.71428:11pt) -- (11.5*25.71428:6.5pt)  -- 
%(12.*25.71428:11pt) -- (12.5*25.71428:6.5pt)  -- 
%(13.*25.71428:11pt) -- (13.5*25.71428:6.5pt)  --  cycle;
 }}}



\pgfdeclareplotmark{m1cb}{%
\node[scale=\mOnescale*\AtoCscale*\symscale] at (0,0) {\tikz {%
\draw[fill=black,line width=.3pt,even odd rule,rotate=90] 
(0.*25.71428:10pt)  -- (0.5*25.71428:5pt)   -- 
(1.*25.71428:10pt)  -- (1.5*25.71428:5pt)   -- 
(2.*25.71428:10pt)  -- (2.5*25.71428:5pt)   -- 
(3.*25.71428:10pt)  -- (3.5*25.71428:5pt)   -- 
(4.*25.71428:10pt)  -- (4.5*25.71428:5pt)   -- 
(5.*25.71428:10pt)  -- (5.5*25.71428:5pt)   -- 
(6.*25.71428:10pt)  -- (6.5*25.71428:5pt)   -- 
(7.*25.71428:10pt)  -- (7.5*25.71428:5pt)   -- 
(8.*25.71428:10pt)  -- (8.5*25.71428:5pt)   -- 
(9.*25.71428:10pt)  -- (9.5*25.71428:5pt)   -- 
(10.*25.71428:10pt) -- (10.5*25.71428:5pt)  -- 
(11.*25.71428:10pt) -- (11.5*25.71428:5pt)  -- 
(12.*25.71428:10pt) -- (12.5*25.71428:5pt)  -- 
(13.*25.71428:10pt) -- (13.5*25.71428:5pt)  --  cycle
(0,0) circle (2pt);
\draw[line width=.4pt,even odd rule,rotate=90] 
(0.5*25.71428:9pt)  -- (0.5*25.71428:5pt)   
(1.5*25.71428:9pt)  -- (1.5*25.71428:5pt)   
(2.5*25.71428:9pt)  -- (2.5*25.71428:5pt)   
(3.5*25.71428:9pt)  -- (3.5*25.71428:5pt)   
(4.5*25.71428:9pt)  -- (4.5*25.71428:5pt)   
(5.5*25.71428:9pt)  -- (5.5*25.71428:5pt)   
(6.5*25.71428:9pt)  -- (6.5*25.71428:5pt)   
(7.5*25.71428:9pt)  -- (7.5*25.71428:5pt)   
(8.5*25.71428:9pt)  -- (8.5*25.71428:5pt)   
(9.5*25.71428:9pt)  -- (9.5*25.71428:5pt)   
(10.5*25.71428:9pt) -- (10.5*25.71428:5pt)  
(11.5*25.71428:9pt) -- (11.5*25.71428:5pt)  
(12.5*25.71428:9pt) -- (12.5*25.71428:5pt)  
(13.5*25.71428:9pt) -- (13.5*25.71428:5pt)  ;
 }}}

\pgfdeclareplotmark{m1cvb}{%
\node[scale=\mOnescale*\AtoCscale*\symscale] at (0,0) {\tikz {%
\draw[fill=black,line width=.3pt,even odd rule,rotate=90] 
(0.*25.71428:9pt)  -- (0.5*25.71428:4.5pt)   -- 
(1.*25.71428:9pt)  -- (1.5*25.71428:4.5pt)   -- 
(2.*25.71428:9pt)  -- (2.5*25.71428:4.5pt)   -- 
(3.*25.71428:9pt)  -- (3.5*25.71428:4.5pt)   -- 
(4.*25.71428:9pt)  -- (4.5*25.71428:4.5pt)   -- 
(5.*25.71428:9pt)  -- (5.5*25.71428:4.5pt)   -- 
(6.*25.71428:9pt)  -- (6.5*25.71428:4.5pt)   -- 
(7.*25.71428:9pt)  -- (7.5*25.71428:4.5pt)   -- 
(8.*25.71428:9pt)  -- (8.5*25.71428:4.5pt)   -- 
(9.*25.71428:9pt)  -- (9.5*25.71428:4.5pt)   -- 
(10.*25.71428:9pt) -- (10.5*25.71428:4.5pt)  -- 
(11.*25.71428:9pt) -- (11.5*25.71428:4.5pt)  -- 
(12.*25.71428:9pt) -- (12.5*25.71428:4.5pt)  -- 
(13.*25.71428:9pt) -- (13.5*25.71428:4.5pt)  --  cycle
(0,0) circle (2pt);
\draw (0,0) circle (9pt);
%\draw[line width=.2pt,even odd rule,rotate=90] 
%(0.*25.71428:11pt)  -- (0.5*25.71428:6.5pt)   -- 
%(1.*25.71428:11pt)  -- (1.5*25.71428:6.5pt)   -- 
%(2.*25.71428:11pt)  -- (2.5*25.71428:6.5pt)   -- 
%(3.*25.71428:11pt)  -- (3.5*25.71428:6.5pt)   -- 
%(4.*25.71428:11pt)  -- (4.5*25.71428:6.5pt)   -- 
%(5.*25.71428:11pt)  -- (5.5*25.71428:6.5pt)   -- 
%(6.*25.71428:11pt)  -- (6.5*25.71428:6.5pt)   -- 
%(7.*25.71428:11pt)  -- (7.5*25.71428:6.5pt)   -- 
%(8.*25.71428:11pt)  -- (8.5*25.71428:6.5pt)   -- 
%(9.*25.71428:11pt)  -- (9.5*25.71428:6.5pt)   -- 
%(10.*25.71428:11pt) -- (10.5*25.71428:6.5pt)  -- 
%(11.*25.71428:11pt) -- (11.5*25.71428:6.5pt)  -- 
%(12.*25.71428:11pt) -- (12.5*25.71428:6.5pt)  -- 
%(13.*25.71428:11pt) -- (13.5*25.71428:6.5pt)  --  cycle;
\draw[line width=.4pt,even odd rule,rotate=90] 
(0.5*25.71428:10pt)  -- (0.5*25.71428:6.5pt)   
(1.5*25.71428:10pt)  -- (1.5*25.71428:6.5pt)   
(2.5*25.71428:10pt)  -- (2.5*25.71428:6.5pt)   
(3.5*25.71428:10pt)  -- (3.5*25.71428:6.5pt)   
(4.5*25.71428:10pt)  -- (4.5*25.71428:6.5pt)   
(5.5*25.71428:10pt)  -- (5.5*25.71428:6.5pt)   
(6.5*25.71428:10pt)  -- (6.5*25.71428:6.5pt)   
(7.5*25.71428:10pt)  -- (7.5*25.71428:6.5pt)   
(8.5*25.71428:10pt)  -- (8.5*25.71428:6.5pt)   
(9.5*25.71428:10pt)  -- (9.5*25.71428:6.5pt)   
(10.5*25.71428:10pt) -- (10.5*25.71428:6.5pt)  
(11.5*25.71428:10pt) -- (11.5*25.71428:6.5pt)  
(12.5*25.71428:10pt) -- (12.5*25.71428:6.5pt)  
(13.5*25.71428:10pt) -- (13.5*25.71428:6.5pt)  ;
 }}}



\pgfdeclareplotmark{m2a}{%
\node[scale=\mTwoscale*\symscale] at (0,0) {\tikz {%
\draw[fill=black,line width=.3pt,even odd rule,rotate=90] 
(0.*30.0:10pt)  -- (0.5*30.0:5pt)   -- 
(1.*30.0:10pt)  -- (1.5*30.0:5pt)   -- 
(2.*30.0:10pt)  -- (2.5*30.0:5pt)   -- 
(3.*30.0:10pt)  -- (3.5*30.0:5pt)   -- 
(4.*30.0:10pt)  -- (4.5*30.0:5pt)   -- 
(5.*30.0:10pt)  -- (5.5*30.0:5pt)   -- 
(6.*30.0:10pt)  -- (6.5*30.0:5pt)   -- 
(7.*30.0:10pt)  -- (7.5*30.0:5pt)   -- 
(8.*30.0:10pt)  -- (8.5*30.0:5pt)   -- 
(9.*30.0:10pt)  -- (9.5*30.0:5pt)   -- 
(10.*30.0:10pt) -- (10.5*30.0:5pt)  -- 
(11.*30.0:10pt) -- (11.5*30.0:5pt)  --  cycle;
 }}}


\pgfdeclareplotmark{m2av}{%
\node[scale=\mTwoscale*\symscale] at (0,0) {\tikz {%
\draw[fill=black,line width=.3pt,even odd rule,rotate=90] 
(0.*30.0:9pt)  -- (0.5*30.0:4.5pt)   -- 
(1.*30.0:9pt)  -- (1.5*30.0:4.5pt)   -- 
(2.*30.0:9pt)  -- (2.5*30.0:4.5pt)   -- 
(3.*30.0:9pt)  -- (3.5*30.0:4.5pt)   -- 
(4.*30.0:9pt)  -- (4.5*30.0:4.5pt)   -- 
(5.*30.0:9pt)  -- (5.5*30.0:4.5pt)   -- 
(6.*30.0:9pt)  -- (6.5*30.0:4.5pt)   -- 
(7.*30.0:9pt)  -- (7.5*30.0:4.5pt)   -- 
(8.*30.0:9pt)  -- (8.5*30.0:4.5pt)   -- 
(9.*30.0:9pt)  -- (9.5*30.0:4.5pt)   -- 
(10.*30.0:9pt) -- (10.5*30.0:4.5pt)  -- 
(11.*30.0:9pt) -- (11.5*30.0:4.5pt)  -- cycle;
\draw (0,0) circle (9pt);
%\draw[line width=.2pt,even odd rule,rotate=90] 
%(0.*30.0:11pt)  -- (0.5*30.0:6.5pt)   -- 
%(1.*30.0:11pt)  -- (1.5*30.0:6.5pt)   -- 
%(2.*30.0:11pt)  -- (2.5*30.0:6.5pt)   -- 
%(3.*30.0:11pt)  -- (3.5*30.0:6.5pt)   -- 
%(4.*30.0:11pt)  -- (4.5*30.0:6.5pt)   -- 
%(5.*30.0:11pt)  -- (5.5*30.0:6.5pt)   -- 
%(6.*30.0:11pt)  -- (6.5*30.0:6.5pt)   -- 
%(7.*30.0:11pt)  -- (7.5*30.0:6.5pt)   -- 
%(8.*30.0:11pt)  -- (8.5*30.0:6.5pt)   -- 
%(9.*30.0:11pt)  -- (9.5*30.0:6.5pt)   -- 
%(10.*30.0:11pt) -- (10.5*30.0:6.5pt)  -- 
%(11.*30.0:11pt) -- (11.5*30.0:6.5pt)  --  cycle;
 }}}



\pgfdeclareplotmark{m2ab}{%
\node[scale=\mTwoscale*\symscale] at (0,0) {\tikz {%
\draw[fill=black,line width=.3pt,even odd rule,rotate=90] 
(0.*30.0:10pt)  -- (0.5*30.0:5pt)   -- 
(1.*30.0:10pt)  -- (1.5*30.0:5pt)   -- 
(2.*30.0:10pt)  -- (2.5*30.0:5pt)   -- 
(3.*30.0:10pt)  -- (3.5*30.0:5pt)   -- 
(4.*30.0:10pt)  -- (4.5*30.0:5pt)   -- 
(5.*30.0:10pt)  -- (5.5*30.0:5pt)   -- 
(6.*30.0:10pt)  -- (6.5*30.0:5pt)   -- 
(7.*30.0:10pt)  -- (7.5*30.0:5pt)   -- 
(8.*30.0:10pt)  -- (8.5*30.0:5pt)   -- 
(9.*30.0:10pt)  -- (9.5*30.0:5pt)   -- 
(10.*30.0:10pt) -- (10.5*30.0:5pt)  -- 
(11.*30.0:10pt) -- (11.5*30.0:5pt)  --  cycle;
\draw[line width=.4pt,even odd rule,rotate=90] 
(0.5*30.0:9pt)  -- (0.5*30.0:5pt)   
(1.5*30.0:9pt)  -- (1.5*30.0:5pt)   
(2.5*30.0:9pt)  -- (2.5*30.0:5pt)   
(3.5*30.0:9pt)  -- (3.5*30.0:5pt)   
(4.5*30.0:9pt)  -- (4.5*30.0:5pt)   
(5.5*30.0:9pt)  -- (5.5*30.0:5pt)   
(6.5*30.0:9pt)  -- (6.5*30.0:5pt)   
(7.5*30.0:9pt)  -- (7.5*30.0:5pt)   
(8.5*30.0:9pt)  -- (8.5*30.0:5pt)   
(9.5*30.0:9pt)  -- (9.5*30.0:5pt)   
(10.5*30.0:9pt) -- (10.5*30.0:5pt)  
(11.5*30.0:9pt) -- (11.5*30.0:5pt) ;
 }}}

\pgfdeclareplotmark{m2avb}{%
\node[scale=\mTwoscale*\symscale] at (0,0) {\tikz {%
\draw[fill=black,line width=.3pt,even odd rule,rotate=90] 
(0.*30.0:9pt)  -- (0.5*30.0:4.5pt)   -- 
(1.*30.0:9pt)  -- (1.5*30.0:4.5pt)   -- 
(2.*30.0:9pt)  -- (2.5*30.0:4.5pt)   -- 
(3.*30.0:9pt)  -- (3.5*30.0:4.5pt)   -- 
(4.*30.0:9pt)  -- (4.5*30.0:4.5pt)   -- 
(5.*30.0:9pt)  -- (5.5*30.0:4.5pt)   -- 
(6.*30.0:9pt)  -- (6.5*30.0:4.5pt)   -- 
(7.*30.0:9pt)  -- (7.5*30.0:4.5pt)   -- 
(8.*30.0:9pt)  -- (8.5*30.0:4.5pt)   -- 
(9.*30.0:9pt)  -- (9.5*30.0:4.5pt)   -- 
(10.*30.0:9pt) -- (10.5*30.0:4.5pt)  -- 
(11.*30.0:9pt) -- (11.5*30.0:4.5pt)  --  cycle;
\draw (0,0) circle (9pt);
%\draw[line width=.2pt,even odd rule,rotate=90] 
%(0.*30.0:11pt)  -- (0.5*30.0:6.5pt)   -- 
%(1.*30.0:11pt)  -- (1.5*30.0:6.5pt)   -- 
%(2.*30.0:11pt)  -- (2.5*30.0:6.5pt)   -- 
%(3.*30.0:11pt)  -- (3.5*30.0:6.5pt)   -- 
%(4.*30.0:11pt)  -- (4.5*30.0:6.5pt)   -- 
%(5.*30.0:11pt)  -- (5.5*30.0:6.5pt)   -- 
%(6.*30.0:11pt)  -- (6.5*30.0:6.5pt)   -- 
%(7.*30.0:11pt)  -- (7.5*30.0:6.5pt)   -- 
%(8.*30.0:11pt)  -- (8.5*30.0:6.5pt)   -- 
%(9.*30.0:11pt)  -- (9.5*30.0:6.5pt)   -- 
%(10.*30.0:11pt) -- (10.5*30.0:6.5pt)  -- 
%(11.*30.0:11pt) -- (11.5*30.0:6.5pt)  -- cycle;
\draw[line width=.4pt,even odd rule,rotate=90] 
(0.5*30.0:10pt)  -- (0.5*30.0:6.5pt)   
(1.5*30.0:10pt)  -- (1.5*30.0:6.5pt)   
(2.5*30.0:10pt)  -- (2.5*30.0:6.5pt)   
(3.5*30.0:10pt)  -- (3.5*30.0:6.5pt)   
(4.5*30.0:10pt)  -- (4.5*30.0:6.5pt)   
(5.5*30.0:10pt)  -- (5.5*30.0:6.5pt)   
(6.5*30.0:10pt)  -- (6.5*30.0:6.5pt)   
(7.5*30.0:10pt)  -- (7.5*30.0:6.5pt)   
(8.5*30.0:10pt)  -- (8.5*30.0:6.5pt)   
(9.5*30.0:10pt)  -- (9.5*30.0:6.5pt)   
(10.5*30.0:10pt) -- (10.5*30.0:6.5pt)  
(11.5*30.0:10pt) -- (11.5*30.0:6.5pt)  ;
 }}}



\pgfdeclareplotmark{m2b}{%
\node[scale=\mTwoscale*\AtoBscale*\symscale] at (0,0) {\tikz {%
\draw[fill=black,line width=.3pt,even odd rule,rotate=90] 
(0.*30.0:10pt)  -- (0.5*30.0:5pt)   -- 
(1.*30.0:10pt)  -- (1.5*30.0:5pt)   -- 
(2.*30.0:10pt)  -- (2.5*30.0:5pt)   -- 
(3.*30.0:10pt)  -- (3.5*30.0:5pt)   -- 
(4.*30.0:10pt)  -- (4.5*30.0:5pt)   -- 
(5.*30.0:10pt)  -- (5.5*30.0:5pt)   -- 
(6.*30.0:10pt)  -- (6.5*30.0:5pt)   -- 
(7.*30.0:10pt)  -- (7.5*30.0:5pt)   -- 
(8.*30.0:10pt)  -- (8.5*30.0:5pt)   -- 
(9.*30.0:10pt)  -- (9.5*30.0:5pt)   -- 
(10.*30.0:10pt) -- (10.5*30.0:5pt)  -- 
(11.*30.0:10pt) -- (11.5*30.0:5pt)  -- cycle
(0,0) circle (2pt);
 }}}


\pgfdeclareplotmark{m2bv}{%
\node[scale=\mTwoscale*\AtoBscale*\symscale] at (0,0) {\tikz {%
\draw[fill=black,line width=.3pt,even odd rule,rotate=90] 
(0.*30.0:9pt)  -- (0.5*30.0:4.5pt)   -- 
(1.*30.0:9pt)  -- (1.5*30.0:4.5pt)   -- 
(2.*30.0:9pt)  -- (2.5*30.0:4.5pt)   -- 
(3.*30.0:9pt)  -- (3.5*30.0:4.5pt)   -- 
(4.*30.0:9pt)  -- (4.5*30.0:4.5pt)   -- 
(5.*30.0:9pt)  -- (5.5*30.0:4.5pt)   -- 
(6.*30.0:9pt)  -- (6.5*30.0:4.5pt)   -- 
(7.*30.0:9pt)  -- (7.5*30.0:4.5pt)   -- 
(8.*30.0:9pt)  -- (8.5*30.0:4.5pt)   -- 
(9.*30.0:9pt)  -- (9.5*30.0:4.5pt)   -- 
(10.*30.0:9pt) -- (10.5*30.0:4.5pt)  -- 
(11.*30.0:9pt) -- (11.5*30.0:4.5pt)  --  cycle
(0,0) circle (2pt);
\draw (0,0) circle (9pt);
%\draw[line width=.2pt,even odd rule,rotate=90] 
%(0.*30.0:11pt)  -- (0.5*30.0:6.5pt)   -- 
%(1.*30.0:11pt)  -- (1.5*30.0:6.5pt)   -- 
%(2.*30.0:11pt)  -- (2.5*30.0:6.5pt)   -- 
%(3.*30.0:11pt)  -- (3.5*30.0:6.5pt)   -- 
%(4.*30.0:11pt)  -- (4.5*30.0:6.5pt)   -- 
%(5.*30.0:11pt)  -- (5.5*30.0:6.5pt)   -- 
%(6.*30.0:11pt)  -- (6.5*30.0:6.5pt)   -- 
%(7.*30.0:11pt)  -- (7.5*30.0:6.5pt)   -- 
%(8.*30.0:11pt)  -- (8.5*30.0:6.5pt)   -- 
%(9.*30.0:11pt)  -- (9.5*30.0:6.5pt)   -- 
%(10.*30.0:11pt) -- (10.5*30.0:6.5pt)  -- 
%(11.*30.0:11pt) -- (11.5*30.0:6.5pt)  --  cycle;
 }}}



\pgfdeclareplotmark{m2bb}{%
\node[scale=\mTwoscale*\AtoBscale*\symscale] at (0,0) {\tikz {%
\draw[fill=black,line width=.3pt,even odd rule,rotate=90] 
(0.*30.0:10pt)  -- (0.5*30.0:5pt)   -- 
(1.*30.0:10pt)  -- (1.5*30.0:5pt)   -- 
(2.*30.0:10pt)  -- (2.5*30.0:5pt)   -- 
(3.*30.0:10pt)  -- (3.5*30.0:5pt)   -- 
(4.*30.0:10pt)  -- (4.5*30.0:5pt)   -- 
(5.*30.0:10pt)  -- (5.5*30.0:5pt)   -- 
(6.*30.0:10pt)  -- (6.5*30.0:5pt)   -- 
(7.*30.0:10pt)  -- (7.5*30.0:5pt)   -- 
(8.*30.0:10pt)  -- (8.5*30.0:5pt)   -- 
(9.*30.0:10pt)  -- (9.5*30.0:5pt)   -- 
(10.*30.0:10pt) -- (10.5*30.0:5pt)  -- 
(11.*30.0:10pt) -- (11.5*30.0:5pt)  --  cycle
(0,0) circle (2pt);
\draw[line width=.4pt,even odd rule,rotate=90] 
(0.5*30.0:9pt)  -- (0.5*30.0:5pt)   
(1.5*30.0:9pt)  -- (1.5*30.0:5pt)   
(2.5*30.0:9pt)  -- (2.5*30.0:5pt)   
(3.5*30.0:9pt)  -- (3.5*30.0:5pt)   
(4.5*30.0:9pt)  -- (4.5*30.0:5pt)   
(5.5*30.0:9pt)  -- (5.5*30.0:5pt)   
(6.5*30.0:9pt)  -- (6.5*30.0:5pt)   
(7.5*30.0:9pt)  -- (7.5*30.0:5pt)   
(8.5*30.0:9pt)  -- (8.5*30.0:5pt)   
(9.5*30.0:9pt)  -- (9.5*30.0:5pt)   
(10.5*30.0:9pt) -- (10.5*30.0:5pt)  
(11.5*30.0:9pt) -- (11.5*30.0:5pt)  ;
 }}}

\pgfdeclareplotmark{m2bvb}{%
\node[scale=\mTwoscale*\AtoBscale*\symscale] at (0,0) {\tikz {%
\draw[fill=black,line width=.3pt,even odd rule,rotate=90] 
(0.*30.0:9pt)  -- (0.5*30.0:4.5pt)   -- 
(1.*30.0:9pt)  -- (1.5*30.0:4.5pt)   -- 
(2.*30.0:9pt)  -- (2.5*30.0:4.5pt)   -- 
(3.*30.0:9pt)  -- (3.5*30.0:4.5pt)   -- 
(4.*30.0:9pt)  -- (4.5*30.0:4.5pt)   -- 
(5.*30.0:9pt)  -- (5.5*30.0:4.5pt)   -- 
(6.*30.0:9pt)  -- (6.5*30.0:4.5pt)   -- 
(7.*30.0:9pt)  -- (7.5*30.0:4.5pt)   -- 
(8.*30.0:9pt)  -- (8.5*30.0:4.5pt)   -- 
(9.*30.0:9pt)  -- (9.5*30.0:4.5pt)   -- 
(10.*30.0:9pt) -- (10.5*30.0:4.5pt)  -- 
(11.*30.0:9pt) -- (11.5*30.0:4.5pt)  --  cycle
(0,0) circle (2pt);
\draw (0,0) circle (9pt);
%\draw[line width=.2pt,even odd rule,rotate=90] 
%(0.*30.0:11pt)  -- (0.5*30.0:6.5pt)   -- 
%(1.*30.0:11pt)  -- (1.5*30.0:6.5pt)   -- 
%(2.*30.0:11pt)  -- (2.5*30.0:6.5pt)   -- 
%(3.*30.0:11pt)  -- (3.5*30.0:6.5pt)   -- 
%(4.*30.0:11pt)  -- (4.5*30.0:6.5pt)   -- 
%(5.*30.0:11pt)  -- (5.5*30.0:6.5pt)   -- 
%(6.*30.0:11pt)  -- (6.5*30.0:6.5pt)   -- 
%(7.*30.0:11pt)  -- (7.5*30.0:6.5pt)   -- 
%(8.*30.0:11pt)  -- (8.5*30.0:6.5pt)   -- 
%(9.*30.0:11pt)  -- (9.5*30.0:6.5pt)   -- 
%(10.*30.0:11pt) -- (10.5*30.0:6.5pt)  -- 
%(11.*30.0:11pt) -- (11.5*30.0:6.5pt)  --  cycle;
\draw[line width=.4pt,even odd rule,rotate=90] 
(0.5*30.0:10pt)  -- (0.5*30.0:6.5pt)   
(1.5*30.0:10pt)  -- (1.5*30.0:6.5pt)   
(2.5*30.0:10pt)  -- (2.5*30.0:6.5pt)   
(3.5*30.0:10pt)  -- (3.5*30.0:6.5pt)   
(4.5*30.0:10pt)  -- (4.5*30.0:6.5pt)   
(5.5*30.0:10pt)  -- (5.5*30.0:6.5pt)   
(6.5*30.0:10pt)  -- (6.5*30.0:6.5pt)   
(7.5*30.0:10pt)  -- (7.5*30.0:6.5pt)   
(8.5*30.0:10pt)  -- (8.5*30.0:6.5pt)   
(9.5*30.0:10pt)  -- (9.5*30.0:6.5pt)   
(10.5*30.0:10pt) -- (10.5*30.0:6.5pt)  
(11.5*30.0:10pt) -- (11.5*30.0:6.5pt)  ;
 }}}



\pgfdeclareplotmark{m2c}{%
\node[scale=\mTwoscale*\AtoCscale*\symscale] at (0,0) {\tikz {%
\draw[fill=black,line width=.3pt,even odd rule,rotate=90] 
(0.*30.0:10pt)  -- (0.5*30.0:5pt)   -- 
(1.*30.0:10pt)  -- (1.5*30.0:5pt)   -- 
(2.*30.0:10pt)  -- (2.5*30.0:5pt)   -- 
(3.*30.0:10pt)  -- (3.5*30.0:5pt)   -- 
(4.*30.0:10pt)  -- (4.5*30.0:5pt)   -- 
(5.*30.0:10pt)  -- (5.5*30.0:5pt)   -- 
(6.*30.0:10pt)  -- (6.5*30.0:5pt)   -- 
(7.*30.0:10pt)  -- (7.5*30.0:5pt)   -- 
(8.*30.0:10pt)  -- (8.5*30.0:5pt)   -- 
(9.*30.0:10pt)  -- (9.5*30.0:5pt)   -- 
(10.*30.0:10pt) -- (10.5*30.0:5pt)  -- 
(11.*30.0:10pt) -- (11.5*30.0:5pt)  --cycle
(0,0) circle (3pt);
 }}}


\pgfdeclareplotmark{m2cv}{%
\node[scale=\mTwoscale*\AtoCscale*\symscale] at (0,0) {\tikz {%
\draw[fill=black,line width=.3pt,even odd rule,rotate=90] 
(0.*30.0:9pt)  -- (0.5*30.0:4.5pt)   -- 
(1.*30.0:9pt)  -- (1.5*30.0:4.5pt)   -- 
(2.*30.0:9pt)  -- (2.5*30.0:4.5pt)   -- 
(3.*30.0:9pt)  -- (3.5*30.0:4.5pt)   -- 
(4.*30.0:9pt)  -- (4.5*30.0:4.5pt)   -- 
(5.*30.0:9pt)  -- (5.5*30.0:4.5pt)   -- 
(6.*30.0:9pt)  -- (6.5*30.0:4.5pt)   -- 
(7.*30.0:9pt)  -- (7.5*30.0:4.5pt)   -- 
(8.*30.0:9pt)  -- (8.5*30.0:4.5pt)   -- 
(9.*30.0:9pt)  -- (9.5*30.0:4.5pt)   -- 
(10.*30.0:9pt) -- (10.5*30.0:4.5pt)  -- 
(11.*30.0:9pt) -- (11.5*30.0:4.5pt)  --  cycle
(0,0) circle (3pt);
\draw (0,0) circle (9pt);
%\draw[line width=.2pt,even odd rule,rotate=90] 
%(0.*30.0:11pt)  -- (0.5*30.0:6.5pt)   -- 
%(1.*30.0:11pt)  -- (1.5*30.0:6.5pt)   -- 
%(2.*30.0:11pt)  -- (2.5*30.0:6.5pt)   -- 
%(3.*30.0:11pt)  -- (3.5*30.0:6.5pt)   -- 
%(4.*30.0:11pt)  -- (4.5*30.0:6.5pt)   -- 
%(5.*30.0:11pt)  -- (5.5*30.0:6.5pt)   -- 
%(6.*30.0:11pt)  -- (6.5*30.0:6.5pt)   -- 
%(7.*30.0:11pt)  -- (7.5*30.0:6.5pt)   -- 
%(8.*30.0:11pt)  -- (8.5*30.0:6.5pt)   -- 
%(9.*30.0:11pt)  -- (9.5*30.0:6.5pt)   -- 
%(10.*30.0:11pt) -- (10.5*30.0:6.5pt)  -- 
%(11.*30.0:11pt) -- (11.5*30.0:6.5pt)  --  cycle;
 }}}



\pgfdeclareplotmark{m2cb}{%
\node[scale=\mTwoscale*\AtoCscale*\symscale] at (0,0) {\tikz {%
\draw[fill=black,line width=.3pt,even odd rule,rotate=90] 
(0.*30.0:10pt)  -- (0.5*30.0:5pt)   -- 
(1.*30.0:10pt)  -- (1.5*30.0:5pt)   -- 
(2.*30.0:10pt)  -- (2.5*30.0:5pt)   -- 
(3.*30.0:10pt)  -- (3.5*30.0:5pt)   -- 
(4.*30.0:10pt)  -- (4.5*30.0:5pt)   -- 
(5.*30.0:10pt)  -- (5.5*30.0:5pt)   -- 
(6.*30.0:10pt)  -- (6.5*30.0:5pt)   -- 
(7.*30.0:10pt)  -- (7.5*30.0:5pt)   -- 
(8.*30.0:10pt)  -- (8.5*30.0:5pt)   -- 
(9.*30.0:10pt)  -- (9.5*30.0:5pt)   -- 
(10.*30.0:10pt) -- (10.5*30.0:5pt)  -- 
(11.*30.0:10pt) -- (11.5*30.0:5pt)  --  cycle
(0,0) circle (3pt);
\draw[line width=.4pt,even odd rule,rotate=90] 
(0.5*30.0:9pt)  -- (0.5*30.0:5pt)   
(1.5*30.0:9pt)  -- (1.5*30.0:5pt)   
(2.5*30.0:9pt)  -- (2.5*30.0:5pt)   
(3.5*30.0:9pt)  -- (3.5*30.0:5pt)   
(4.5*30.0:9pt)  -- (4.5*30.0:5pt)   
(5.5*30.0:9pt)  -- (5.5*30.0:5pt)   
(6.5*30.0:9pt)  -- (6.5*30.0:5pt)   
(7.5*30.0:9pt)  -- (7.5*30.0:5pt)   
(8.5*30.0:9pt)  -- (8.5*30.0:5pt)   
(9.5*30.0:9pt)  -- (9.5*30.0:5pt)   
(10.5*30.0:9pt) -- (10.5*30.0:5pt)  
(11.5*30.0:9pt) -- (11.5*30.0:5pt)  ;
 }}}

\pgfdeclareplotmark{m2cvb}{%
\node[scale=\mTwoscale*\AtoCscale*\symscale] at (0,0) {\tikz {%
\draw[fill=black,line width=.3pt,even odd rule,rotate=90] 
(0.*30.0:9pt)  -- (0.5*30.0:4.5pt)   -- 
(1.*30.0:9pt)  -- (1.5*30.0:4.5pt)   -- 
(2.*30.0:9pt)  -- (2.5*30.0:4.5pt)   -- 
(3.*30.0:9pt)  -- (3.5*30.0:4.5pt)   -- 
(4.*30.0:9pt)  -- (4.5*30.0:4.5pt)   -- 
(5.*30.0:9pt)  -- (5.5*30.0:4.5pt)   -- 
(6.*30.0:9pt)  -- (6.5*30.0:4.5pt)   -- 
(7.*30.0:9pt)  -- (7.5*30.0:4.5pt)   -- 
(8.*30.0:9pt)  -- (8.5*30.0:4.5pt)   -- 
(9.*30.0:9pt)  -- (9.5*30.0:4.5pt)   -- 
(10.*30.0:9pt) -- (10.5*30.0:4.5pt)  -- 
(11.*30.0:9pt) -- (11.5*30.0:4.5pt)  --  cycle
(0,0) circle (3pt);
\draw (0,0) circle (9pt);
%\draw[line width=.2pt,even odd rule,rotate=90] 
%(0.*30.0:11pt)  -- (0.5*30.0:6.5pt)   -- 
%(1.*30.0:11pt)  -- (1.5*30.0:6.5pt)   -- 
%(2.*30.0:11pt)  -- (2.5*30.0:6.5pt)   -- 
%(3.*30.0:11pt)  -- (3.5*30.0:6.5pt)   -- 
%(4.*30.0:11pt)  -- (4.5*30.0:6.5pt)   -- 
%(5.*30.0:11pt)  -- (5.5*30.0:6.5pt)   -- 
%(6.*30.0:11pt)  -- (6.5*30.0:6.5pt)   -- 
%(7.*30.0:11pt)  -- (7.5*30.0:6.5pt)   -- 
%(8.*30.0:11pt)  -- (8.5*30.0:6.5pt)   -- 
%(9.*30.0:11pt)  -- (9.5*30.0:6.5pt)   -- 
%(10.*30.0:11pt) -- (10.5*30.0:6.5pt)  -- 
%(11.*30.0:11pt) -- (11.5*30.0:6.5pt)  --  cycle;
\draw[line width=.4pt,even odd rule,rotate=90] 
(0.5*30.0:10pt)  -- (0.5*30.0:6.5pt)   
(1.5*30.0:10pt)  -- (1.5*30.0:6.5pt)   
(2.5*30.0:10pt)  -- (2.5*30.0:6.5pt)   
(3.5*30.0:10pt)  -- (3.5*30.0:6.5pt)   
(4.5*30.0:10pt)  -- (4.5*30.0:6.5pt)   
(5.5*30.0:10pt)  -- (5.5*30.0:6.5pt)   
(6.5*30.0:10pt)  -- (6.5*30.0:6.5pt)   
(7.5*30.0:10pt)  -- (7.5*30.0:6.5pt)   
(8.5*30.0:10pt)  -- (8.5*30.0:6.5pt)   
(9.5*30.0:10pt)  -- (9.5*30.0:6.5pt)   
(10.5*30.0:10pt) -- (10.5*30.0:6.5pt)  
(11.5*30.0:10pt) -- (11.5*30.0:6.5pt)  ;
 }}}



\pgfdeclareplotmark{m3a}{%
\node[scale=\mThreescale*\symscale] at (0,0) {\tikz {%
\draw[fill=black,line width=.3pt,even odd rule,rotate=90] 
(0.*36.0:10pt)  -- (0.5*36.0:5pt)   -- 
(1.*36.0:10pt)  -- (1.5*36.0:5pt)   -- 
(2.*36.0:10pt)  -- (2.5*36.0:5pt)   -- 
(3.*36.0:10pt)  -- (3.5*36.0:5pt)   -- 
(4.*36.0:10pt)  -- (4.5*36.0:5pt)   -- 
(5.*36.0:10pt)  -- (5.5*36.0:5pt)   -- 
(6.*36.0:10pt)  -- (6.5*36.0:5pt)   -- 
(7.*36.0:10pt)  -- (7.5*36.0:5pt)   -- 
(8.*36.0:10pt)  -- (8.5*36.0:5pt)   -- 
(9.*36.0:10pt)  -- (9.5*36.0:5pt)   --  cycle;
 }}}


\pgfdeclareplotmark{m3av}{%
\node[scale=\mThreescale*\symscale] at (0,0) {\tikz {%
\draw[fill=black,line width=.3pt,even odd rule,rotate=90] 
(0.*36.0:9pt)  -- (0.5*36.0:4.5pt)   -- 
(1.*36.0:9pt)  -- (1.5*36.0:4.5pt)   -- 
(2.*36.0:9pt)  -- (2.5*36.0:4.5pt)   -- 
(3.*36.0:9pt)  -- (3.5*36.0:4.5pt)   -- 
(4.*36.0:9pt)  -- (4.5*36.0:4.5pt)   -- 
(5.*36.0:9pt)  -- (5.5*36.0:4.5pt)   -- 
(6.*36.0:9pt)  -- (6.5*36.0:4.5pt)   -- 
(7.*36.0:9pt)  -- (7.5*36.0:4.5pt)   -- 
(8.*36.0:9pt)  -- (8.5*36.0:4.5pt)   -- 
(9.*36.0:9pt)  -- (9.5*36.0:4.5pt)   -- cycle;
\draw (0,0) circle (9pt);
%\draw[line width=.2pt,even odd rule,rotate=90] 
%(0.*36.0:11pt)  -- (0.5*36.0:6.5pt)   -- 
%(1.*36.0:11pt)  -- (1.5*36.0:6.5pt)   -- 
%(2.*36.0:11pt)  -- (2.5*36.0:6.5pt)   -- 
%(3.*36.0:11pt)  -- (3.5*36.0:6.5pt)   -- 
%(4.*36.0:11pt)  -- (4.5*36.0:6.5pt)   -- 
%(5.*36.0:11pt)  -- (5.5*36.0:6.5pt)   -- 
%(6.*36.0:11pt)  -- (6.5*36.0:6.5pt)   -- 
%(7.*36.0:11pt)  -- (7.5*36.0:6.5pt)   -- 
%(8.*36.0:11pt)  -- (8.5*36.0:6.5pt)   -- 
%(9.*36.0:11pt)  -- (9.5*36.0:6.5pt)   --  cycle;
 }}}



\pgfdeclareplotmark{m3ab}{%
\node[scale=\mThreescale*\symscale] at (0,0) {\tikz {%
\draw[fill=black,line width=.3pt,even odd rule,rotate=90] 
(0.*36.0:10pt)  -- (0.5*36.0:5pt)   -- 
(1.*36.0:10pt)  -- (1.5*36.0:5pt)   -- 
(2.*36.0:10pt)  -- (2.5*36.0:5pt)   -- 
(3.*36.0:10pt)  -- (3.5*36.0:5pt)   -- 
(4.*36.0:10pt)  -- (4.5*36.0:5pt)   -- 
(5.*36.0:10pt)  -- (5.5*36.0:5pt)   -- 
(6.*36.0:10pt)  -- (6.5*36.0:5pt)   -- 
(7.*36.0:10pt)  -- (7.5*36.0:5pt)   -- 
(8.*36.0:10pt)  -- (8.5*36.0:5pt)   -- 
(9.*36.0:10pt)  -- (9.5*36.0:5pt)   --  cycle;
\draw[line width=.4pt,even odd rule,rotate=90] 
(0.5*36.0:9pt)  -- (0.5*36.0:5pt)   
(1.5*36.0:9pt)  -- (1.5*36.0:5pt)   
(2.5*36.0:9pt)  -- (2.5*36.0:5pt)   
(3.5*36.0:9pt)  -- (3.5*36.0:5pt)   
(4.5*36.0:9pt)  -- (4.5*36.0:5pt)   
(5.5*36.0:9pt)  -- (5.5*36.0:5pt)   
(6.5*36.0:9pt)  -- (6.5*36.0:5pt)   
(7.5*36.0:9pt)  -- (7.5*36.0:5pt)   
(8.5*36.0:9pt)  -- (8.5*36.0:5pt)   
(9.5*36.0:9pt)  -- (9.5*36.0:5pt)  ;
 }}}

\pgfdeclareplotmark{m3avb}{%
\node[scale=\mThreescale*\symscale] at (0,0) {\tikz {%
\draw[fill=black,line width=.3pt,even odd rule,rotate=90] 
(0.*36.0:9pt)  -- (0.5*36.0:4.5pt)   -- 
(1.*36.0:9pt)  -- (1.5*36.0:4.5pt)   -- 
(2.*36.0:9pt)  -- (2.5*36.0:4.5pt)   -- 
(3.*36.0:9pt)  -- (3.5*36.0:4.5pt)   -- 
(4.*36.0:9pt)  -- (4.5*36.0:4.5pt)   -- 
(5.*36.0:9pt)  -- (5.5*36.0:4.5pt)   -- 
(6.*36.0:9pt)  -- (6.5*36.0:4.5pt)   -- 
(7.*36.0:9pt)  -- (7.5*36.0:4.5pt)   -- 
(8.*36.0:9pt)  -- (8.5*36.0:4.5pt)   -- 
(9.*36.0:9pt)  -- (9.5*36.0:4.5pt)   --  cycle;
\draw (0,0) circle (9pt);
%\draw[line width=.2pt,even odd rule,rotate=90] 
%(0.*36.0:11pt)  -- (0.5*36.0:6.5pt)   -- 
%(1.*36.0:11pt)  -- (1.5*36.0:6.5pt)   -- 
%(2.*36.0:11pt)  -- (2.5*36.0:6.5pt)   -- 
%(3.*36.0:11pt)  -- (3.5*36.0:6.5pt)   -- 
%(4.*36.0:11pt)  -- (4.5*36.0:6.5pt)   -- 
%(5.*36.0:11pt)  -- (5.5*36.0:6.5pt)   -- 
%(6.*36.0:11pt)  -- (6.5*36.0:6.5pt)   -- 
%(7.*36.0:11pt)  -- (7.5*36.0:6.5pt)   -- 
%(8.*36.0:11pt)  -- (8.5*36.0:6.5pt)   -- 
%(9.*36.0:11pt)  -- (9.5*36.0:6.5pt)   --  cycle;
\draw[line width=.4pt,even odd rule,rotate=90] 
(0.5*36.0:10pt)  -- (0.5*36.0:6.5pt)   
(1.5*36.0:10pt)  -- (1.5*36.0:6.5pt)   
(2.5*36.0:10pt)  -- (2.5*36.0:6.5pt)   
(3.5*36.0:10pt)  -- (3.5*36.0:6.5pt)   
(4.5*36.0:10pt)  -- (4.5*36.0:6.5pt)   
(5.5*36.0:10pt)  -- (5.5*36.0:6.5pt)   
(6.5*36.0:10pt)  -- (6.5*36.0:6.5pt)   
(7.5*36.0:10pt)  -- (7.5*36.0:6.5pt)   
(8.5*36.0:10pt)  -- (8.5*36.0:6.5pt)   
(9.5*36.0:10pt)  -- (9.5*36.0:6.5pt)   ;
 }}}



\pgfdeclareplotmark{m3b}{%
\node[scale=\mThreescale*\AtoBscale*\symscale] at (0,0) {\tikz {%
\draw[fill=black,line width=.3pt,even odd rule,rotate=90] 
(0.*36.0:10pt)  -- (0.5*36.0:5pt)   -- 
(1.*36.0:10pt)  -- (1.5*36.0:5pt)   -- 
(2.*36.0:10pt)  -- (2.5*36.0:5pt)   -- 
(3.*36.0:10pt)  -- (3.5*36.0:5pt)   -- 
(4.*36.0:10pt)  -- (4.5*36.0:5pt)   -- 
(5.*36.0:10pt)  -- (5.5*36.0:5pt)   -- 
(6.*36.0:10pt)  -- (6.5*36.0:5pt)   -- 
(7.*36.0:10pt)  -- (7.5*36.0:5pt)   -- 
(8.*36.0:10pt)  -- (8.5*36.0:5pt)   -- 
(9.*36.0:10pt)  -- (9.5*36.0:5pt)   -- cycle
(0,0) circle (2pt);
 }}}


\pgfdeclareplotmark{m3bv}{%
\node[scale=\mThreescale*\AtoBscale*\symscale] at (0,0) {\tikz {%
\draw[fill=black,line width=.3pt,even odd rule,rotate=90] 
(0.*36.0:9pt)  -- (0.5*36.0:4.5pt)   -- 
(1.*36.0:9pt)  -- (1.5*36.0:4.5pt)   -- 
(2.*36.0:9pt)  -- (2.5*36.0:4.5pt)   -- 
(3.*36.0:9pt)  -- (3.5*36.0:4.5pt)   -- 
(4.*36.0:9pt)  -- (4.5*36.0:4.5pt)   -- 
(5.*36.0:9pt)  -- (5.5*36.0:4.5pt)   -- 
(6.*36.0:9pt)  -- (6.5*36.0:4.5pt)   -- 
(7.*36.0:9pt)  -- (7.5*36.0:4.5pt)   -- 
(8.*36.0:9pt)  -- (8.5*36.0:4.5pt)   -- 
(9.*36.0:9pt)  -- (9.5*36.0:4.5pt)   -- cycle
(0,0) circle (2pt);
\draw (0,0) circle (9pt);
%\draw[line width=.2pt,even odd rule,rotate=90] 
%(0.*36.0:11pt)  -- (0.5*36.0:6.5pt)   -- 
%(1.*36.0:11pt)  -- (1.5*36.0:6.5pt)   -- 
%(2.*36.0:11pt)  -- (2.5*36.0:6.5pt)   -- 
%(3.*36.0:11pt)  -- (3.5*36.0:6.5pt)   -- 
%(4.*36.0:11pt)  -- (4.5*36.0:6.5pt)   -- 
%(5.*36.0:11pt)  -- (5.5*36.0:6.5pt)   -- 
%(6.*36.0:11pt)  -- (6.5*36.0:6.5pt)   -- 
%(7.*36.0:11pt)  -- (7.5*36.0:6.5pt)   -- 
%(8.*36.0:11pt)  -- (8.5*36.0:6.5pt)   -- 
%(9.*36.0:11pt)  -- (9.5*36.0:6.5pt)   -- cycle;
 }}}



\pgfdeclareplotmark{m3bb}{%
\node[scale=\mThreescale*\AtoBscale*\symscale] at (0,0) {\tikz {%
\draw[fill=black,line width=.3pt,even odd rule,rotate=90] 
(0.*36.0:10pt)  -- (0.5*36.0:5pt)   -- 
(1.*36.0:10pt)  -- (1.5*36.0:5pt)   -- 
(2.*36.0:10pt)  -- (2.5*36.0:5pt)   -- 
(3.*36.0:10pt)  -- (3.5*36.0:5pt)   -- 
(4.*36.0:10pt)  -- (4.5*36.0:5pt)   -- 
(5.*36.0:10pt)  -- (5.5*36.0:5pt)   -- 
(6.*36.0:10pt)  -- (6.5*36.0:5pt)   -- 
(7.*36.0:10pt)  -- (7.5*36.0:5pt)   -- 
(8.*36.0:10pt)  -- (8.5*36.0:5pt)   -- 
(9.*36.0:10pt)  -- (9.5*36.0:5pt)   --  cycle
(0,0) circle (2pt);
\draw[line width=.4pt,even odd rule,rotate=90] 
(0.5*36.0:9pt)  -- (0.5*36.0:5pt)   
(1.5*36.0:9pt)  -- (1.5*36.0:5pt)   
(2.5*36.0:9pt)  -- (2.5*36.0:5pt)   
(3.5*36.0:9pt)  -- (3.5*36.0:5pt)   
(4.5*36.0:9pt)  -- (4.5*36.0:5pt)   
(5.5*36.0:9pt)  -- (5.5*36.0:5pt)   
(6.5*36.0:9pt)  -- (6.5*36.0:5pt)   
(7.5*36.0:9pt)  -- (7.5*36.0:5pt)   
(8.5*36.0:9pt)  -- (8.5*36.0:5pt)   
(9.5*36.0:9pt)  -- (9.5*36.0:5pt)  ;
 }}}

\pgfdeclareplotmark{m3bvb}{%
\node[scale=\mThreescale*\AtoBscale*\symscale] at (0,0) {\tikz {%
\draw[fill=black,line width=.3pt,even odd rule,rotate=90] 
(0.*36.0:9pt)  -- (0.5*36.0:4.5pt)   -- 
(1.*36.0:9pt)  -- (1.5*36.0:4.5pt)   -- 
(2.*36.0:9pt)  -- (2.5*36.0:4.5pt)   -- 
(3.*36.0:9pt)  -- (3.5*36.0:4.5pt)   -- 
(4.*36.0:9pt)  -- (4.5*36.0:4.5pt)   -- 
(5.*36.0:9pt)  -- (5.5*36.0:4.5pt)   -- 
(6.*36.0:9pt)  -- (6.5*36.0:4.5pt)   -- 
(7.*36.0:9pt)  -- (7.5*36.0:4.5pt)   -- 
(8.*36.0:9pt)  -- (8.5*36.0:4.5pt)   -- 
(9.*36.0:9pt)  -- (9.5*36.0:4.5pt)   --  cycle
(0,0) circle (2pt);
\draw (0,0) circle (9pt);
%\draw[line width=.2pt,even odd rule,rotate=90] 
%(0.*36.0:11pt)  -- (0.5*36.0:6.5pt)   -- 
%(1.*36.0:11pt)  -- (1.5*36.0:6.5pt)   -- 
%(2.*36.0:11pt)  -- (2.5*36.0:6.5pt)   -- 
%(3.*36.0:11pt)  -- (3.5*36.0:6.5pt)   -- 
%(4.*36.0:11pt)  -- (4.5*36.0:6.5pt)   -- 
%(5.*36.0:11pt)  -- (5.5*36.0:6.5pt)   -- 
%(6.*36.0:11pt)  -- (6.5*36.0:6.5pt)   -- 
%(7.*36.0:11pt)  -- (7.5*36.0:6.5pt)   -- 
%(8.*36.0:11pt)  -- (8.5*36.0:6.5pt)   -- 
%(9.*36.0:11pt)  -- (9.5*36.0:6.5pt)   -- cycle;
\draw[line width=.4pt,even odd rule,rotate=90] 
(0.5*36.0:10pt)  -- (0.5*36.0:6.5pt)   
(1.5*36.0:10pt)  -- (1.5*36.0:6.5pt)   
(2.5*36.0:10pt)  -- (2.5*36.0:6.5pt)   
(3.5*36.0:10pt)  -- (3.5*36.0:6.5pt)   
(4.5*36.0:10pt)  -- (4.5*36.0:6.5pt)   
(5.5*36.0:10pt)  -- (5.5*36.0:6.5pt)   
(6.5*36.0:10pt)  -- (6.5*36.0:6.5pt)   
(7.5*36.0:10pt)  -- (7.5*36.0:6.5pt)   
(8.5*36.0:10pt)  -- (8.5*36.0:6.5pt)   
(9.5*36.0:10pt)  -- (9.5*36.0:6.5pt)  ;
 }}}



\pgfdeclareplotmark{m3c}{%
\node[scale=\mThreescale*\AtoCscale*\symscale] at (0,0) {\tikz {%
\draw[fill=black,line width=.3pt,even odd rule,rotate=90] 
(0.*36.0:10pt)  -- (0.5*36.0:5pt)   -- 
(1.*36.0:10pt)  -- (1.5*36.0:5pt)   -- 
(2.*36.0:10pt)  -- (2.5*36.0:5pt)   -- 
(3.*36.0:10pt)  -- (3.5*36.0:5pt)   -- 
(4.*36.0:10pt)  -- (4.5*36.0:5pt)   -- 
(5.*36.0:10pt)  -- (5.5*36.0:5pt)   -- 
(6.*36.0:10pt)  -- (6.5*36.0:5pt)   -- 
(7.*36.0:10pt)  -- (7.5*36.0:5pt)   -- 
(8.*36.0:10pt)  -- (8.5*36.0:5pt)   -- 
(9.*36.0:10pt)  -- (9.5*36.0:5pt)   -- cycle
(0,0) circle (3pt);
 }}}


\pgfdeclareplotmark{m3cv}{%
\node[scale=\mThreescale*\AtoCscale*\symscale] at (0,0) {\tikz {%
\draw[fill=black,line width=.3pt,even odd rule,rotate=90] 
(0.*36.0:9pt)  -- (0.5*36.0:4.5pt)   -- 
(1.*36.0:9pt)  -- (1.5*36.0:4.5pt)   -- 
(2.*36.0:9pt)  -- (2.5*36.0:4.5pt)   -- 
(3.*36.0:9pt)  -- (3.5*36.0:4.5pt)   -- 
(4.*36.0:9pt)  -- (4.5*36.0:4.5pt)   -- 
(5.*36.0:9pt)  -- (5.5*36.0:4.5pt)   -- 
(6.*36.0:9pt)  -- (6.5*36.0:4.5pt)   -- 
(7.*36.0:9pt)  -- (7.5*36.0:4.5pt)   -- 
(8.*36.0:9pt)  -- (8.5*36.0:4.5pt)   -- 
(9.*36.0:9pt)  -- (9.5*36.0:4.5pt)   --  cycle
(0,0) circle (3pt);
\draw (0,0) circle (9pt);
%\draw[line width=.2pt,even odd rule,rotate=90] 
%(0.*36.0:11pt)  -- (0.5*36.0:6.5pt)   -- 
%(1.*36.0:11pt)  -- (1.5*36.0:6.5pt)   -- 
%(2.*36.0:11pt)  -- (2.5*36.0:6.5pt)   -- 
%(3.*36.0:11pt)  -- (3.5*36.0:6.5pt)   -- 
%(4.*36.0:11pt)  -- (4.5*36.0:6.5pt)   -- 
%(5.*36.0:11pt)  -- (5.5*36.0:6.5pt)   -- 
%(6.*36.0:11pt)  -- (6.5*36.0:6.5pt)   -- 
%(7.*36.0:11pt)  -- (7.5*36.0:6.5pt)   -- 
%(8.*36.0:11pt)  -- (8.5*36.0:6.5pt)   -- 
%(9.*36.0:11pt)  -- (9.5*36.0:6.5pt)   -- cycle;
 }}}



\pgfdeclareplotmark{m3cb}{%
\node[scale=\mThreescale*\AtoCscale*\symscale] at (0,0) {\tikz {%
\draw[fill=black,line width=.3pt,even odd rule,rotate=90] 
(0.*36.0:10pt)  -- (0.5*36.0:5pt)   -- 
(1.*36.0:10pt)  -- (1.5*36.0:5pt)   -- 
(2.*36.0:10pt)  -- (2.5*36.0:5pt)   -- 
(3.*36.0:10pt)  -- (3.5*36.0:5pt)   -- 
(4.*36.0:10pt)  -- (4.5*36.0:5pt)   -- 
(5.*36.0:10pt)  -- (5.5*36.0:5pt)   -- 
(6.*36.0:10pt)  -- (6.5*36.0:5pt)   -- 
(7.*36.0:10pt)  -- (7.5*36.0:5pt)   -- 
(8.*36.0:10pt)  -- (8.5*36.0:5pt)   -- 
(9.*36.0:10pt)  -- (9.5*36.0:5pt)   -- cycle
(0,0) circle (3pt);
\draw[line width=.4pt,even odd rule,rotate=90] 
(0.5*36.0:9pt)  -- (0.5*36.0:5pt)   
(1.5*36.0:9pt)  -- (1.5*36.0:5pt)   
(2.5*36.0:9pt)  -- (2.5*36.0:5pt)   
(3.5*36.0:9pt)  -- (3.5*36.0:5pt)   
(4.5*36.0:9pt)  -- (4.5*36.0:5pt)   
(5.5*36.0:9pt)  -- (5.5*36.0:5pt)   
(6.5*36.0:9pt)  -- (6.5*36.0:5pt)   
(7.5*36.0:9pt)  -- (7.5*36.0:5pt)   
(8.5*36.0:9pt)  -- (8.5*36.0:5pt)   
(9.5*36.0:9pt)  -- (9.5*36.0:5pt)   ;
 }}}

\pgfdeclareplotmark{m3cvb}{%
\node[scale=\mThreescale*\AtoCscale*\symscale] at (0,0) {\tikz {%
\draw[fill=black,line width=.3pt,even odd rule,rotate=90] 
(0.*36.0:9pt)  -- (0.5*36.0:4.5pt)   -- 
(1.*36.0:9pt)  -- (1.5*36.0:4.5pt)   -- 
(2.*36.0:9pt)  -- (2.5*36.0:4.5pt)   -- 
(3.*36.0:9pt)  -- (3.5*36.0:4.5pt)   -- 
(4.*36.0:9pt)  -- (4.5*36.0:4.5pt)   -- 
(5.*36.0:9pt)  -- (5.5*36.0:4.5pt)   -- 
(6.*36.0:9pt)  -- (6.5*36.0:4.5pt)   -- 
(7.*36.0:9pt)  -- (7.5*36.0:4.5pt)   -- 
(8.*36.0:9pt)  -- (8.5*36.0:4.5pt)   -- 
(9.*36.0:9pt)  -- (9.5*36.0:4.5pt)   -- cycle
(0,0) circle (3pt);
\draw (0,0) circle (9pt);
%\draw[line width=.2pt,even odd rule,rotate=90] 
%(0.*36.0:11pt)  -- (0.5*36.0:6.5pt)   -- 
%(1.*36.0:11pt)  -- (1.5*36.0:6.5pt)   -- 
%(2.*36.0:11pt)  -- (2.5*36.0:6.5pt)   -- 
%(3.*36.0:11pt)  -- (3.5*36.0:6.5pt)   -- 
%(4.*36.0:11pt)  -- (4.5*36.0:6.5pt)   -- 
%(5.*36.0:11pt)  -- (5.5*36.0:6.5pt)   -- 
%(6.*36.0:11pt)  -- (6.5*36.0:6.5pt)   -- 
%(7.*36.0:11pt)  -- (7.5*36.0:6.5pt)   -- 
%(8.*36.0:11pt)  -- (8.5*36.0:6.5pt)   -- 
%(9.*36.0:11pt)  -- (9.5*36.0:6.5pt)   --  cycle;
\draw[line width=.4pt,even odd rule,rotate=90] 
(0.5*36.0:10pt)  -- (0.5*36.0:6.5pt)   
(1.5*36.0:10pt)  -- (1.5*36.0:6.5pt)   
(2.5*36.0:10pt)  -- (2.5*36.0:6.5pt)   
(3.5*36.0:10pt)  -- (3.5*36.0:6.5pt)   
(4.5*36.0:10pt)  -- (4.5*36.0:6.5pt)   
(5.5*36.0:10pt)  -- (5.5*36.0:6.5pt)   
(6.5*36.0:10pt)  -- (6.5*36.0:6.5pt)   
(7.5*36.0:10pt)  -- (7.5*36.0:6.5pt)   
(8.5*36.0:10pt)  -- (8.5*36.0:6.5pt)   
(9.5*36.0:10pt)  -- (9.5*36.0:6.5pt)   ;
 }}}



\pgfdeclareplotmark{m4a}{%
\node[scale=\mFourscale*\symscale] at (0,0) {\tikz {%
\draw[fill=black,line width=.3pt,even odd rule,rotate=90] 
(0.*45.0:10pt)  -- (0.5*45.0:5pt)   -- 
(1.*45.0:10pt)  -- (1.5*45.0:5pt)   -- 
(2.*45.0:10pt)  -- (2.5*45.0:5pt)   -- 
(3.*45.0:10pt)  -- (3.5*45.0:5pt)   -- 
(4.*45.0:10pt)  -- (4.5*45.0:5pt)   -- 
(5.*45.0:10pt)  -- (5.5*45.0:5pt)   -- 
(6.*45.0:10pt)  -- (6.5*45.0:5pt)   -- 
(7.*45.0:10pt)  -- (7.5*45.0:5pt)   --  cycle;
 }}}


\pgfdeclareplotmark{m4av}{%
\node[scale=\mFourscale*\symscale] at (0,0) {\tikz {%
\draw[fill=black,line width=.3pt,even odd rule,rotate=90] 
(0.*45.0:9pt)  -- (0.5*45.0:4.5pt)   -- 
(1.*45.0:9pt)  -- (1.5*45.0:4.5pt)   -- 
(2.*45.0:9pt)  -- (2.5*45.0:4.5pt)   -- 
(3.*45.0:9pt)  -- (3.5*45.0:4.5pt)   -- 
(4.*45.0:9pt)  -- (4.5*45.0:4.5pt)   -- 
(5.*45.0:9pt)  -- (5.5*45.0:4.5pt)   -- 
(6.*45.0:9pt)  -- (6.5*45.0:4.5pt)   -- 
(7.*45.0:9pt)  -- (7.5*45.0:4.5pt)   -- cycle;
\draw (0,0) circle (9pt);
%\draw[line width=.3pt,even odd rule,rotate=90] 
%(0.*45.0:11pt)  -- (0.5*45.0:6.5pt)   -- 
%(1.*45.0:11pt)  -- (1.5*45.0:6.5pt)   -- 
%(2.*45.0:11pt)  -- (2.5*45.0:6.5pt)   -- 
%(3.*45.0:11pt)  -- (3.5*45.0:6.5pt)   -- 
%(4.*45.0:11pt)  -- (4.5*45.0:6.5pt)   -- 
%(5.*45.0:11pt)  -- (5.5*45.0:6.5pt)   -- 
%(6.*45.0:11pt)  -- (6.5*45.0:6.5pt)   -- 
%(7.*45.0:11pt)  -- (7.5*45.0:6.5pt)   --  cycle;
 }}}



\pgfdeclareplotmark{m4ab}{%
\node[scale=\mFourscale*\symscale] at (0,0) {\tikz {%
\draw[fill=black,line width=.3pt,even odd rule,rotate=90] 
(0.*45.0:10pt)  -- (0.5*45.0:5pt)   -- 
(1.*45.0:10pt)  -- (1.5*45.0:5pt)   -- 
(2.*45.0:10pt)  -- (2.5*45.0:5pt)   -- 
(3.*45.0:10pt)  -- (3.5*45.0:5pt)   -- 
(4.*45.0:10pt)  -- (4.5*45.0:5pt)   -- 
(5.*45.0:10pt)  -- (5.5*45.0:5pt)   -- 
(6.*45.0:10pt)  -- (6.5*45.0:5pt)   -- 
(7.*45.0:10pt)  -- (7.5*45.0:5pt)   --  cycle;
\draw[line width=.4pt,even odd rule,rotate=90] 
(0.5*45.0:9pt)  -- (0.5*45.0:5pt)   
(1.5*45.0:9pt)  -- (1.5*45.0:5pt)   
(2.5*45.0:9pt)  -- (2.5*45.0:5pt)   
(3.5*45.0:9pt)  -- (3.5*45.0:5pt)   
(4.5*45.0:9pt)  -- (4.5*45.0:5pt)   
(5.5*45.0:9pt)  -- (5.5*45.0:5pt)   
(6.5*45.0:9pt)  -- (6.5*45.0:5pt)   
(7.5*45.0:9pt)  -- (7.5*45.0:5pt)  ;
 }}}

\pgfdeclareplotmark{m4avb}{%
\node[scale=\mFourscale*\symscale] at (0,0) {\tikz {%
\draw[fill=black,line width=.3pt,even odd rule,rotate=90] 
(0.*45.0:9pt)  -- (0.5*45.0:4.5pt)   -- 
(1.*45.0:9pt)  -- (1.5*45.0:4.5pt)   -- 
(2.*45.0:9pt)  -- (2.5*45.0:4.5pt)   -- 
(3.*45.0:9pt)  -- (3.5*45.0:4.5pt)   -- 
(4.*45.0:9pt)  -- (4.5*45.0:4.5pt)   -- 
(5.*45.0:9pt)  -- (5.5*45.0:4.5pt)   -- 
(6.*45.0:9pt)  -- (6.5*45.0:4.5pt)   -- 
(7.*45.0:9pt)  -- (7.5*45.0:4.5pt)   --  cycle;
\draw (0,0) circle (9pt);
%\draw[line width=.3pt,even odd rule,rotate=90] 
%(0.*45.0:11pt)  -- (0.5*45.0:6.5pt)   -- 
%(1.*45.0:11pt)  -- (1.5*45.0:6.5pt)   -- 
%(2.*45.0:11pt)  -- (2.5*45.0:6.5pt)   -- 
%(3.*45.0:11pt)  -- (3.5*45.0:6.5pt)   -- 
%(4.*45.0:11pt)  -- (4.5*45.0:6.5pt)   -- 
%(5.*45.0:11pt)  -- (5.5*45.0:6.5pt)   -- 
%(6.*45.0:11pt)  -- (6.5*45.0:6.5pt)   -- 
%(7.*45.0:11pt)  -- (7.5*45.0:6.5pt)   --  cycle;
\draw[line width=.4pt,even odd rule,rotate=90] 
(0.5*45.0:10pt)  -- (0.5*45.0:6.5pt)   
(1.5*45.0:10pt)  -- (1.5*45.0:6.5pt)   
(2.5*45.0:10pt)  -- (2.5*45.0:6.5pt)   
(3.5*45.0:10pt)  -- (3.5*45.0:6.5pt)   
(4.5*45.0:10pt)  -- (4.5*45.0:6.5pt)   
(5.5*45.0:10pt)  -- (5.5*45.0:6.5pt)   
(6.5*45.0:10pt)  -- (6.5*45.0:6.5pt)   
(7.5*45.0:10pt)  -- (7.5*45.0:6.5pt)   ;
 }}}



\pgfdeclareplotmark{m4b}{%
\node[scale=\mFourscale*\AtoBscale*\symscale] at (0,0) {\tikz {%
\draw[fill=black,line width=.3pt,even odd rule,rotate=90] 
(0.*45.0:10pt)  -- (0.5*45.0:5pt)   -- 
(1.*45.0:10pt)  -- (1.5*45.0:5pt)   -- 
(2.*45.0:10pt)  -- (2.5*45.0:5pt)   -- 
(3.*45.0:10pt)  -- (3.5*45.0:5pt)   -- 
(4.*45.0:10pt)  -- (4.5*45.0:5pt)   -- 
(5.*45.0:10pt)  -- (5.5*45.0:5pt)   -- 
(6.*45.0:10pt)  -- (6.5*45.0:5pt)   -- 
(7.*45.0:10pt)  -- (7.5*45.0:5pt)   -- cycle
(0,0) circle (2pt);
 }}}


\pgfdeclareplotmark{m4bv}{%
\node[scale=\mFourscale*\AtoBscale*\symscale] at (0,0) {\tikz {%
\draw[fill=black,line width=.3pt,even odd rule,rotate=90] 
(0.*45.0:9pt)  -- (0.5*45.0:4.5pt)   -- 
(1.*45.0:9pt)  -- (1.5*45.0:4.5pt)   -- 
(2.*45.0:9pt)  -- (2.5*45.0:4.5pt)   -- 
(3.*45.0:9pt)  -- (3.5*45.0:4.5pt)   -- 
(4.*45.0:9pt)  -- (4.5*45.0:4.5pt)   -- 
(5.*45.0:9pt)  -- (5.5*45.0:4.5pt)   -- 
(6.*45.0:9pt)  -- (6.5*45.0:4.5pt)   -- 
(7.*45.0:9pt)  -- (7.5*45.0:4.5pt)   -- cycle
(0,0) circle (2pt);
\draw (0,0) circle (9pt);
%\draw[line width=.3pt,even odd rule,rotate=90] 
%(0.*45.0:11pt)  -- (0.5*45.0:6.5pt)   -- 
%(1.*45.0:11pt)  -- (1.5*45.0:6.5pt)   -- 
%(2.*45.0:11pt)  -- (2.5*45.0:6.5pt)   -- 
%(3.*45.0:11pt)  -- (3.5*45.0:6.5pt)   -- 
%(4.*45.0:11pt)  -- (4.5*45.0:6.5pt)   -- 
%(5.*45.0:11pt)  -- (5.5*45.0:6.5pt)   -- 
%(6.*45.0:11pt)  -- (6.5*45.0:6.5pt)   -- 
%(7.*45.0:11pt)  -- (7.5*45.0:6.5pt)   --cycle;
 }}}



\pgfdeclareplotmark{m4bb}{%
\node[scale=\mFourscale*\AtoBscale*\symscale] at (0,0) {\tikz {%
\draw[fill=black,line width=.3pt,even odd rule,rotate=90] 
(0.*45.0:10pt)  -- (0.5*45.0:5pt)   -- 
(1.*45.0:10pt)  -- (1.5*45.0:5pt)   -- 
(2.*45.0:10pt)  -- (2.5*45.0:5pt)   -- 
(3.*45.0:10pt)  -- (3.5*45.0:5pt)   -- 
(4.*45.0:10pt)  -- (4.5*45.0:5pt)   -- 
(5.*45.0:10pt)  -- (5.5*45.0:5pt)   -- 
(6.*45.0:10pt)  -- (6.5*45.0:5pt)   -- 
(7.*45.0:10pt)  -- (7.5*45.0:5pt)   --  cycle
(0,0) circle (2pt);
\draw[line width=.4pt,even odd rule,rotate=90] 
(0.5*45.0:9pt)  -- (0.5*45.0:5pt)   
(1.5*45.0:9pt)  -- (1.5*45.0:5pt)   
(2.5*45.0:9pt)  -- (2.5*45.0:5pt)   
(3.5*45.0:9pt)  -- (3.5*45.0:5pt)   
(4.5*45.0:9pt)  -- (4.5*45.0:5pt)   
(5.5*45.0:9pt)  -- (5.5*45.0:5pt)   
(6.5*45.0:9pt)  -- (6.5*45.0:5pt)   
(7.5*45.0:9pt)  -- (7.5*45.0:5pt)  ;
 }}}

\pgfdeclareplotmark{m4bvb}{%
\node[scale=\mFourscale*\AtoBscale*\symscale] at (0,0) {\tikz {%
\draw[fill=black,line width=.3pt,even odd rule,rotate=90] 
(0.*45.0:9pt)  -- (0.5*45.0:4.5pt)   -- 
(1.*45.0:9pt)  -- (1.5*45.0:4.5pt)   -- 
(2.*45.0:9pt)  -- (2.5*45.0:4.5pt)   -- 
(3.*45.0:9pt)  -- (3.5*45.0:4.5pt)   -- 
(4.*45.0:9pt)  -- (4.5*45.0:4.5pt)   -- 
(5.*45.0:9pt)  -- (5.5*45.0:4.5pt)   -- 
(6.*45.0:9pt)  -- (6.5*45.0:4.5pt)   -- 
(7.*45.0:9pt)  -- (7.5*45.0:4.5pt)   --  cycle
(0,0) circle (2pt);
\draw (0,0) circle (9pt);
%\draw[line width=.3pt,even odd rule,rotate=90] 
%(0.*45.0:11pt)  -- (0.5*45.0:6.5pt)   -- 
%(1.*45.0:11pt)  -- (1.5*45.0:6.5pt)   -- 
%(2.*45.0:11pt)  -- (2.5*45.0:6.5pt)   -- 
%(3.*45.0:11pt)  -- (3.5*45.0:6.5pt)   -- 
%(4.*45.0:11pt)  -- (4.5*45.0:6.5pt)   -- 
%(5.*45.0:11pt)  -- (5.5*45.0:6.5pt)   -- 
%(6.*45.0:11pt)  -- (6.5*45.0:6.5pt)   -- 
%(7.*45.0:11pt)  -- (7.5*45.0:6.5pt)   -- cycle;
\draw[line width=.4pt,even odd rule,rotate=90] 
(0.5*45.0:10pt)  -- (0.5*45.0:6.5pt)   
(1.5*45.0:10pt)  -- (1.5*45.0:6.5pt)   
(2.5*45.0:10pt)  -- (2.5*45.0:6.5pt)   
(3.5*45.0:10pt)  -- (3.5*45.0:6.5pt)   
(4.5*45.0:10pt)  -- (4.5*45.0:6.5pt)   
(5.5*45.0:10pt)  -- (5.5*45.0:6.5pt)   
(6.5*45.0:10pt)  -- (6.5*45.0:6.5pt)   
(7.5*45.0:10pt)  -- (7.5*45.0:6.5pt)  ;
 }}}



\pgfdeclareplotmark{m4c}{%
\node[scale=\mFourscale*\AtoCscale*\symscale] at (0,0) {\tikz {%
\draw[fill=black,line width=.3pt,even odd rule,rotate=90] 
(0.*45.0:10pt)  -- (0.5*45.0:5pt)   -- 
(1.*45.0:10pt)  -- (1.5*45.0:5pt)   -- 
(2.*45.0:10pt)  -- (2.5*45.0:5pt)   -- 
(3.*45.0:10pt)  -- (3.5*45.0:5pt)   -- 
(4.*45.0:10pt)  -- (4.5*45.0:5pt)   -- 
(5.*45.0:10pt)  -- (5.5*45.0:5pt)   -- 
(6.*45.0:10pt)  -- (6.5*45.0:5pt)   -- 
(7.*45.0:10pt)  -- (7.5*45.0:5pt)   -- cycle
(0,0) circle (3pt);
 }}}


\pgfdeclareplotmark{m4cv}{%
\node[scale=\mFourscale*\AtoCscale*\symscale] at (0,0) {\tikz {%
\draw[fill=black,line width=.3pt,even odd rule,rotate=90] 
(0.*45.0:9pt)  -- (0.5*45.0:4.5pt)   -- 
(1.*45.0:9pt)  -- (1.5*45.0:4.5pt)   -- 
(2.*45.0:9pt)  -- (2.5*45.0:4.5pt)   -- 
(3.*45.0:9pt)  -- (3.5*45.0:4.5pt)   -- 
(4.*45.0:9pt)  -- (4.5*45.0:4.5pt)   -- 
(5.*45.0:9pt)  -- (5.5*45.0:4.5pt)   -- 
(6.*45.0:9pt)  -- (6.5*45.0:4.5pt)   -- 
(7.*45.0:9pt)  -- (7.5*45.0:4.5pt)   --  cycle
(0,0) circle (3pt);
\draw (0,0) circle (9pt);
%\draw[line width=.3pt,even odd rule,rotate=90] 
%(0.*45.0:11pt)  -- (0.5*45.0:6.5pt)   -- 
%(1.*45.0:11pt)  -- (1.5*45.0:6.5pt)   -- 
%(2.*45.0:11pt)  -- (2.5*45.0:6.5pt)   -- 
%(3.*45.0:11pt)  -- (3.5*45.0:6.5pt)   -- 
%(4.*45.0:11pt)  -- (4.5*45.0:6.5pt)   -- 
%(5.*45.0:11pt)  -- (5.5*45.0:6.5pt)   -- 
%(6.*45.0:11pt)  -- (6.5*45.0:6.5pt)   -- 
%(7.*45.0:11pt)  -- (7.5*45.0:6.5pt)   -- cycle;
 }}}



\pgfdeclareplotmark{m4cb}{%
\node[scale=\mFourscale*\AtoCscale*\symscale] at (0,0) {\tikz {%
\draw[fill=black,line width=.3pt,even odd rule,rotate=90] 
(0.*45.0:10pt)  -- (0.5*45.0:5pt)   -- 
(1.*45.0:10pt)  -- (1.5*45.0:5pt)   -- 
(2.*45.0:10pt)  -- (2.5*45.0:5pt)   -- 
(3.*45.0:10pt)  -- (3.5*45.0:5pt)   -- 
(4.*45.0:10pt)  -- (4.5*45.0:5pt)   -- 
(5.*45.0:10pt)  -- (5.5*45.0:5pt)   -- 
(6.*45.0:10pt)  -- (6.5*45.0:5pt)   -- 
(7.*45.0:10pt)  -- (7.5*45.0:5pt)   -- cycle
(0,0) circle (3pt);
\draw[line width=.4pt,even odd rule,rotate=90] 
(0.5*45.0:9pt)  -- (0.5*45.0:5pt)   
(1.5*45.0:9pt)  -- (1.5*45.0:5pt)   
(2.5*45.0:9pt)  -- (2.5*45.0:5pt)   
(3.5*45.0:9pt)  -- (3.5*45.0:5pt)   
(4.5*45.0:9pt)  -- (4.5*45.0:5pt)   
(5.5*45.0:9pt)  -- (5.5*45.0:5pt)   
(6.5*45.0:9pt)  -- (6.5*45.0:5pt)   
(7.5*45.0:9pt)  -- (7.5*45.0:5pt)  ;
 }}}

\pgfdeclareplotmark{m4cvb}{%
\node[scale=\mFourscale*\AtoCscale*\symscale] at (0,0) {\tikz {%
\draw[fill=black,line width=.3pt,even odd rule,rotate=90] 
(0.*45.0:9pt)  -- (0.5*45.0:4.5pt)   -- 
(1.*45.0:9pt)  -- (1.5*45.0:4.5pt)   -- 
(2.*45.0:9pt)  -- (2.5*45.0:4.5pt)   -- 
(3.*45.0:9pt)  -- (3.5*45.0:4.5pt)   -- 
(4.*45.0:9pt)  -- (4.5*45.0:4.5pt)   -- 
(5.*45.0:9pt)  -- (5.5*45.0:4.5pt)   -- 
(6.*45.0:9pt)  -- (6.5*45.0:4.5pt)   -- 
(7.*45.0:9pt)  -- (7.5*45.0:4.5pt)   --  cycle
(0,0) circle (3pt);
\draw (0,0) circle (9pt);
%\draw[line width=.3pt,even odd rule,rotate=90] 
%(0.*45.0:11pt)  -- (0.5*45.0:6.5pt)   -- 
%(1.*45.0:11pt)  -- (1.5*45.0:6.5pt)   -- 
%(2.*45.0:11pt)  -- (2.5*45.0:6.5pt)   -- 
%(3.*45.0:11pt)  -- (3.5*45.0:6.5pt)   -- 
%(4.*45.0:11pt)  -- (4.5*45.0:6.5pt)   -- 
%(5.*45.0:11pt)  -- (5.5*45.0:6.5pt)   -- 
%(6.*45.0:11pt)  -- (6.5*45.0:6.5pt)   -- 
%(7.*45.0:11pt)  -- (7.5*45.0:6.5pt)   --  cycle;
\draw[line width=.4pt,even odd rule,rotate=90] 
(0.5*45.0:10pt)  -- (0.5*45.0:6.5pt)   
(1.5*45.0:10pt)  -- (1.5*45.0:6.5pt)   
(2.5*45.0:10pt)  -- (2.5*45.0:6.5pt)   
(3.5*45.0:10pt)  -- (3.5*45.0:6.5pt)   
(4.5*45.0:10pt)  -- (4.5*45.0:6.5pt)   
(5.5*45.0:10pt)  -- (5.5*45.0:6.5pt)   
(6.5*45.0:10pt)  -- (6.5*45.0:6.5pt)   
(7.5*45.0:10pt)  -- (7.5*45.0:6.5pt)   ;
 }}}



\pgfdeclareplotmark{m5a}{%
\node[scale=\mFivescale*\symscale] at (0,0) {\tikz {%
\draw[fill=black,line width=.3pt,even odd rule,rotate=90] 
(0.*60.0:10pt)  -- (0.5*60.0:1.14*5.0pt)   -- 
(1.*60.0:10pt)  -- (1.5*60.0:1.14*5.0pt)   -- 
(2.*60.0:10pt)  -- (2.5*60.0:1.14*5.0pt)   -- 
(3.*60.0:10pt)  -- (3.5*60.0:1.14*5.0pt)   -- 
(4.*60.0:10pt)  -- (4.5*60.0:1.14*5.0pt)   -- 
(5.*60.0:10pt)  -- (5.5*60.0:1.14*5.0pt)   --  cycle;
 }}}


\pgfdeclareplotmark{m5av}{%
\node[scale=\mFivescale*\symscale] at (0,0) {\tikz {%
\draw[fill=black,line width=.3pt,even odd rule,rotate=90] 
(0.*60.0:9pt)  -- (0.5*60.0:1.14*4.5pt)   -- 
(1.*60.0:9pt)  -- (1.5*60.0:1.14*4.5pt)   -- 
(2.*60.0:9pt)  -- (2.5*60.0:1.14*4.5pt)   -- 
(3.*60.0:9pt)  -- (3.5*60.0:1.14*4.5pt)   -- 
(4.*60.0:9pt)  -- (4.5*60.0:1.14*4.5pt)   -- 
(5.*60.0:9pt)  -- (5.5*60.0:1.14*4.5pt)   -- cycle;
\draw (0,0) circle (9pt);
%\draw[line width=.35pt,even odd rule,rotate=90] 
%(0.*60.0:11pt)  -- (0.5*60.0:1.14*1.00*6.0pt)   -- 
%(1.*60.0:11pt)  -- (1.5*60.0:1.14*1.00*6.0pt)   -- 
%(2.*60.0:11pt)  -- (2.5*60.0:1.14*1.00*6.0pt)   -- 
%(3.*60.0:11pt)  -- (3.5*60.0:1.14*1.00*6.0pt)   -- 
%(4.*60.0:11pt)  -- (4.5*60.0:1.14*1.00*6.0pt)   -- 
%(5.*60.0:11pt)  -- (5.5*60.0:1.14*1.00*6.0pt)   --  cycle;
 }}}



\pgfdeclareplotmark{m5ab}{%
\node[scale=\mFivescale*\symscale] at (0,0) {\tikz {%
\draw[fill=black,line width=.3pt,even odd rule,rotate=90] 
(0.*60.0:10pt)  -- (0.5*60.0:1.14*5.0pt)   -- 
(1.*60.0:10pt)  -- (1.5*60.0:1.14*5.0pt)   -- 
(2.*60.0:10pt)  -- (2.5*60.0:1.14*5.0pt)   -- 
(3.*60.0:10pt)  -- (3.5*60.0:1.14*5.0pt)   -- 
(4.*60.0:10pt)  -- (4.5*60.0:1.14*5.0pt)   -- 
(5.*60.0:10pt)  -- (5.5*60.0:1.14*5.0pt)   --  cycle;
\draw[line width=0.6pt,even odd rule,rotate=90] 
(0.5*60.0:9pt)  -- (0.5*60.0:1.14*5.0pt)   
(1.5*60.0:9pt)  -- (1.5*60.0:1.14*5.0pt)   
(2.5*60.0:9pt)  -- (2.5*60.0:1.14*5.0pt)   
(3.5*60.0:9pt)  -- (3.5*60.0:1.14*5.0pt)   
(4.5*60.0:9pt)  -- (4.5*60.0:1.14*5.0pt)   
(5.5*60.0:9pt)  -- (5.5*60.0:1.14*5.0pt)  ;
 }}}

\pgfdeclareplotmark{m5avb}{%
\node[scale=\mFivescale*\symscale] at (0,0) {\tikz {%
\draw[fill=black,line width=.3pt,even odd rule,rotate=90] 
(0.*60.0:9pt)  -- (0.5*60.0:1.14*4.5pt)   -- 
(1.*60.0:9pt)  -- (1.5*60.0:1.14*4.5pt)   -- 
(2.*60.0:9pt)  -- (2.5*60.0:1.14*4.5pt)   -- 
(3.*60.0:9pt)  -- (3.5*60.0:1.14*4.5pt)   -- 
(4.*60.0:9pt)  -- (4.5*60.0:1.14*4.5pt)   -- 
(5.*60.0:9pt)  -- (5.5*60.0:1.14*4.5pt)   --  cycle;
\draw (0,0) circle (9pt);
%\draw[line width=.35pt,even odd rule,rotate=90] 
%(0.*60.0:11pt)  -- (0.5*60.0:1.14*1.00*6.0pt)   -- 
%(1.*60.0:11pt)  -- (1.5*60.0:1.14*1.00*6.0pt)   -- 
%(2.*60.0:11pt)  -- (2.5*60.0:1.14*1.00*6.0pt)   -- 
%(3.*60.0:11pt)  -- (3.5*60.0:1.14*1.00*6.0pt)   -- 
%(4.*60.0:11pt)  -- (4.5*60.0:1.14*1.00*6.0pt)   -- 
%(5.*60.0:11pt)  -- (5.5*60.0:1.14*1.00*6.0pt)   --  cycle;
\draw[line width=0.6pt,even odd rule,rotate=90] 
(0.5*60.0:10pt)  -- (0.5*60.0:1.14*1.00*6.0pt)   
(1.5*60.0:10pt)  -- (1.5*60.0:1.14*1.00*6.0pt)   
(2.5*60.0:10pt)  -- (2.5*60.0:1.14*1.00*6.0pt)   
(3.5*60.0:10pt)  -- (3.5*60.0:1.14*1.00*6.0pt)   
(4.5*60.0:10pt)  -- (4.5*60.0:1.14*1.00*6.0pt)   
(5.5*60.0:10pt)  -- (5.5*60.0:1.14*1.00*6.0pt)   ;
 }}}



\pgfdeclareplotmark{m5b}{%
\node[scale=\mFivescale*\AtoBscale*\symscale] at (0,0) {\tikz {%
\draw[fill=black,line width=.3pt,even odd rule,rotate=90] 
(0.*60.0:10pt)  -- (0.5*60.0:1.14*5.0pt)   -- 
(1.*60.0:10pt)  -- (1.5*60.0:1.14*5.0pt)   -- 
(2.*60.0:10pt)  -- (2.5*60.0:1.14*5.0pt)   -- 
(3.*60.0:10pt)  -- (3.5*60.0:1.14*5.0pt)   -- 
(4.*60.0:10pt)  -- (4.5*60.0:1.14*5.0pt)   -- 
(5.*60.0:10pt)  -- (5.5*60.0:1.14*5.0pt)   -- cycle
(0,0) circle (2pt);
 }}}


\pgfdeclareplotmark{m5bv}{%
\node[scale=\mFivescale*\AtoBscale*\symscale] at (0,0) {\tikz {%
\draw[fill=black,line width=.3pt,even odd rule,rotate=90] 
(0.*60.0:9pt)  -- (0.5*60.0:1.14*4.5pt)   -- 
(1.*60.0:9pt)  -- (1.5*60.0:1.14*4.5pt)   -- 
(2.*60.0:9pt)  -- (2.5*60.0:1.14*4.5pt)   -- 
(3.*60.0:9pt)  -- (3.5*60.0:1.14*4.5pt)   -- 
(4.*60.0:9pt)  -- (4.5*60.0:1.14*4.5pt)   -- 
(5.*60.0:9pt)  -- (5.5*60.0:1.14*4.5pt)   -- cycle
(0,0) circle (2pt);
\draw (0,0) circle (9pt);
%\draw[line width=.35pt,even odd rule,rotate=90] 
%(0.*60.0:11pt)  -- (0.5*60.0:1.14*1.00*6.0pt)   -- 
%(1.*60.0:11pt)  -- (1.5*60.0:1.14*1.00*6.0pt)   -- 
%(2.*60.0:11pt)  -- (2.5*60.0:1.14*1.00*6.0pt)   -- 
%(3.*60.0:11pt)  -- (3.5*60.0:1.14*1.00*6.0pt)   -- 
%(4.*60.0:11pt)  -- (4.5*60.0:1.14*1.00*6.0pt)   -- 
%(5.*60.0:11pt)  -- (5.5*60.0:1.14*1.00*6.0pt)   --  cycle;
 }}}



\pgfdeclareplotmark{m5bb}{%
\node[scale=\mFivescale*\AtoBscale*\symscale] at (0,0) {\tikz {%
\draw[fill=black,line width=.3pt,even odd rule,rotate=90] 
(0.*60.0:10pt)  -- (0.5*60.0:1.14*5.0pt)   -- 
(1.*60.0:10pt)  -- (1.5*60.0:1.14*5.0pt)   -- 
(2.*60.0:10pt)  -- (2.5*60.0:1.14*5.0pt)   -- 
(3.*60.0:10pt)  -- (3.5*60.0:1.14*5.0pt)   -- 
(4.*60.0:10pt)  -- (4.5*60.0:1.14*5.0pt)   -- 
(5.*60.0:10pt)  -- (5.5*60.0:1.14*5.0pt)   --  cycle
(0,0) circle (2pt);
\draw[line width=0.6pt,even odd rule,rotate=90] 
(0.5*60.0:9pt)  -- (0.5*60.0:1.14*5.0pt)   
(1.5*60.0:9pt)  -- (1.5*60.0:1.14*5.0pt)   
(2.5*60.0:9pt)  -- (2.5*60.0:1.14*5.0pt)   
(3.5*60.0:9pt)  -- (3.5*60.0:1.14*5.0pt)   
(4.5*60.0:9pt)  -- (4.5*60.0:1.14*5.0pt)   
(5.5*60.0:9pt)  -- (5.5*60.0:1.14*5.0pt)  ;
 }}}

\pgfdeclareplotmark{m5bvb}{%
\node[scale=\mFivescale*\AtoBscale*\symscale] at (0,0) {\tikz {%
\draw[fill=black,line width=.3pt,even odd rule,rotate=90] 
(0.*60.0:9pt)  -- (0.5*60.0:1.14*4.5pt)   -- 
(1.*60.0:9pt)  -- (1.5*60.0:1.14*4.5pt)   -- 
(2.*60.0:9pt)  -- (2.5*60.0:1.14*4.5pt)   -- 
(3.*60.0:9pt)  -- (3.5*60.0:1.14*4.5pt)   -- 
(4.*60.0:9pt)  -- (4.5*60.0:1.14*4.5pt)   -- 
(5.*60.0:9pt)  -- (5.5*60.0:1.14*4.5pt)   --  cycle
(0,0) circle (2pt);
\draw (0,0) circle (9pt);
%\draw[line width=.35pt,even odd rule,rotate=90] 
%(0.*60.0:11pt)  -- (0.5*60.0:1.14*1.00*6.0pt)   -- 
%(1.*60.0:11pt)  -- (1.5*60.0:1.14*1.00*6.0pt)   -- 
%(2.*60.0:11pt)  -- (2.5*60.0:1.14*1.00*6.0pt)   -- 
%(3.*60.0:11pt)  -- (3.5*60.0:1.14*1.00*6.0pt)   -- 
%(4.*60.0:11pt)  -- (4.5*60.0:1.14*1.00*6.0pt)   -- 
%(5.*60.0:11pt)  -- (5.5*60.0:1.14*1.00*6.0pt)   -- cycle;
\draw[line width=0.6pt,even odd rule,rotate=90] 
(0.5*60.0:10pt)  -- (0.5*60.0:1.14*1.00*6.0pt)   
(1.5*60.0:10pt)  -- (1.5*60.0:1.14*1.00*6.0pt)   
(2.5*60.0:10pt)  -- (2.5*60.0:1.14*1.00*6.0pt)   
(3.5*60.0:10pt)  -- (3.5*60.0:1.14*1.00*6.0pt)   
(4.5*60.0:10pt)  -- (4.5*60.0:1.14*1.00*6.0pt)   
(5.5*60.0:10pt)  -- (5.5*60.0:1.14*1.00*6.0pt)  ;
 }}}



\pgfdeclareplotmark{m5c}{%
\node[scale=\mFivescale*\AtoCscale*\symscale] at (0,0) {\tikz {%
\draw[fill=black,line width=.3pt,even odd rule,rotate=90] 
(0.*60.0:10pt)  -- (0.5*60.0:1.14*5.0pt)   -- 
(1.*60.0:10pt)  -- (1.5*60.0:1.14*5.0pt)   -- 
(2.*60.0:10pt)  -- (2.5*60.0:1.14*5.0pt)   -- 
(3.*60.0:10pt)  -- (3.5*60.0:1.14*5.0pt)   -- 
(4.*60.0:10pt)  -- (4.5*60.0:1.14*5.0pt)   -- 
(5.*60.0:10pt)  -- (5.5*60.0:1.14*5.0pt)   -- cycle
(0,0) circle (3.25pt);
 }}}


\pgfdeclareplotmark{m5cv}{%
\node[scale=\mFivescale*\AtoCscale*\symscale] at (0,0) {\tikz {%
\draw[fill=black,line width=.3pt,even odd rule,rotate=90] 
(0.*60.0:9pt)  -- (0.5*60.0:1.14*4.5pt)   -- 
(1.*60.0:9pt)  -- (1.5*60.0:1.14*4.5pt)   -- 
(2.*60.0:9pt)  -- (2.5*60.0:1.14*4.5pt)   -- 
(3.*60.0:9pt)  -- (3.5*60.0:1.14*4.5pt)   -- 
(4.*60.0:9pt)  -- (4.5*60.0:1.14*4.5pt)   -- 
(5.*60.0:9pt)  -- (5.5*60.0:1.14*4.5pt)   --  cycle
(0,0) circle (3.25pt);
\draw (0,0) circle (9pt);
%\draw[line width=.35pt,even odd rule,rotate=90] 
%(0.*60.0:11pt)  -- (0.5*60.0:1.14*1.00*6.0pt)   -- 
%(1.*60.0:11pt)  -- (1.5*60.0:1.14*1.00*6.0pt)   -- 
%(2.*60.0:11pt)  -- (2.5*60.0:1.14*1.00*6.0pt)   -- 
%(3.*60.0:11pt)  -- (3.5*60.0:1.14*1.00*6.0pt)   -- 
%(4.*60.0:11pt)  -- (4.5*60.0:1.14*1.00*6.0pt)   -- 
%(5.*60.0:11pt)  -- (5.5*60.0:1.14*1.00*6.0pt)   -- cycle;
 }}}



\pgfdeclareplotmark{m5cb}{%
\node[scale=\mFivescale*\AtoCscale*\symscale] at (0,0) {\tikz {%
\draw[fill=black,line width=.3pt,even odd rule,rotate=90] 
(0.*60.0:10pt)  -- (0.5*60.0:1.14*5.0pt)   -- 
(1.*60.0:10pt)  -- (1.5*60.0:1.14*5.0pt)   -- 
(2.*60.0:10pt)  -- (2.5*60.0:1.14*5.0pt)   -- 
(3.*60.0:10pt)  -- (3.5*60.0:1.14*5.0pt)   -- 
(4.*60.0:10pt)  -- (4.5*60.0:1.14*5.0pt)   -- 
(5.*60.0:10pt)  -- (5.5*60.0:1.14*5.0pt)   -- cycle
(0,0) circle (3.25pt);
\draw[line width=0.6pt,even odd rule,rotate=90] 
(0.5*60.0:9pt)  -- (0.5*60.0:1.14*5.0pt)   
(1.5*60.0:9pt)  -- (1.5*60.0:1.14*5.0pt)   
(2.5*60.0:9pt)  -- (2.5*60.0:1.14*5.0pt)   
(3.5*60.0:9pt)  -- (3.5*60.0:1.14*5.0pt)   
(4.5*60.0:9pt)  -- (4.5*60.0:1.14*5.0pt)   
(5.5*60.0:9pt)  -- (5.5*60.0:1.14*5.0pt)  ;
 }}}

\pgfdeclareplotmark{m5cvb}{%
\node[scale=\mFivescale*\AtoCscale*\symscale] at (0,0) {\tikz {%
\draw[fill=black,line width=.3pt,even odd rule,rotate=90] 
(0.*60.0:9pt)  -- (0.5*60.0:1.14*4.5pt)   -- 
(1.*60.0:9pt)  -- (1.5*60.0:1.14*4.5pt)   -- 
(2.*60.0:9pt)  -- (2.5*60.0:1.14*4.5pt)   -- 
(3.*60.0:9pt)  -- (3.5*60.0:1.14*4.5pt)   -- 
(4.*60.0:9pt)  -- (4.5*60.0:1.14*4.5pt)   -- 
(5.*60.0:9pt)  -- (5.5*60.0:1.14*4.5pt)   --  cycle
(0,0) circle (3.25pt);
\draw (0,0) circle (9pt);
%\draw[line width=.35pt,even odd rule,rotate=90] 
%(0.*60.0:11pt)  -- (0.5*60.0:1.14*1.00*6.0pt)   -- 
%(1.*60.0:11pt)  -- (1.5*60.0:1.14*1.00*6.0pt)   -- 
%(2.*60.0:11pt)  -- (2.5*60.0:1.14*1.00*6.0pt)   -- 
%(3.*60.0:11pt)  -- (3.5*60.0:1.14*1.00*6.0pt)   -- 
%(4.*60.0:11pt)  -- (4.5*60.0:1.14*1.00*6.0pt)   -- 
%(5.*60.0:11pt)  -- (5.5*60.0:1.14*1.00*6.0pt)   -- cycle;
\draw[line width=0.6pt,even odd rule,rotate=90] 
(0.5*60.0:10pt)  -- (0.5*60.0:1.14*1.00*6.0pt)   
(1.5*60.0:10pt)  -- (1.5*60.0:1.14*1.00*6.0pt)   
(2.5*60.0:10pt)  -- (2.5*60.0:1.14*1.00*6.0pt)   
(3.5*60.0:10pt)  -- (3.5*60.0:1.14*1.00*6.0pt)   
(4.5*60.0:10pt)  -- (4.5*60.0:1.14*1.00*6.0pt)   
(5.5*60.0:10pt)  -- (5.5*60.0:1.14*1.00*6.0pt)   ;
 }}}

\pgfdeclareplotmark{m6a}{%
\node[scale=\mSixscale*\symscale] at (0,0) {\tikz {%
\draw[fill,line width=0.2pt,even odd rule,rotate=90] %
(0:10pt) -- (36:3.8pt)    -- 
(72:10pt) -- (108:3.8pt)  -- 
(144:10pt) -- (180:3.8pt) -- 
(216:10pt) -- (252:3.8pt) --
(288:10pt) -- (324:3.8pt) -- cycle
(0pt,0pt) circle (2.5pt);
% The following is needed to center the darn thing.
\draw[opacity=0] (0pt,0pt) circle (13pt) ;
 }}}


\pgfdeclareplotmark{m6av}{%
\node[scale=\mSixscale*\symscale] at (0,0) {\tikz {%
\draw[fill,line width=0.2pt,even odd rule,rotate=90] %
(0:0.9*10pt) -- (36:0.9*3.8pt)    -- 
(72:0.9*10pt) -- (108:0.9*3.8pt)  -- 
(144:0.9*10pt) -- (180:0.9*3.8pt) -- 
(216:0.9*10pt) -- (252:0.9*3.8pt) --
(288:0.9*10pt) -- (324:0.9*3.8pt) -- cycle
(0pt,0pt) circle (2.5pt);
\draw (0,0) circle (10pt);
%\draw[line width=0.6pt,even odd rule,rotate=90] %
%(0:0.9*13pt) --  (36:0.9*6.0pt)    -- 
%(72:0.9*13pt) -- (108:0.9*6.0pt)  -- 
%(144:0.9*13pt) -- (180:0.9*6.0pt) -- 
%(216:0.9*13pt) -- (252:0.9*6.0pt) --
%(288:0.9*13pt) -- (324:0.9*6.0pt) -- cycle ;
% The following is needed to center the darn thing.
\draw[opacity=0] (0pt,0pt) circle (13pt) ;
 }}}

\pgfdeclareplotmark{m6ab}{%
\node[scale=\mSixscale*\symscale] at (0,0) {\tikz {%
\draw[fill,line width=0.2pt,even odd rule,rotate=90] %
(0:10pt) -- (36:3.8pt)    -- 
(72:10pt) -- (108:3.8pt)  -- 
(144:10pt) -- (180:3.8pt) -- 
(216:10pt) -- (252:3.8pt) --
(288:10pt) -- (324:3.8pt) -- cycle
(0pt,0pt) circle (2.5pt);
\draw[line width=0.9pt,even odd rule,rotate=90] %
(36:9pt) -- (36:3.8pt)    
(108:9pt) -- (108:3.8pt)  
(180:9pt) -- (180:3.8pt) 
(252:9pt) -- (252:3.8pt) 
(324:9pt) -- (324:3.8pt) 
(0pt,0pt) circle (2.5pt);
% The following is needed to center the darn thing.
\draw[opacity=0] (0pt,0pt) circle (13pt) ;
 }}}

\pgfdeclareplotmark{m6avb}{%
\node[scale=\mSixscale*\symscale] at (0,0) {\tikz {%
\draw[fill,line width=0.2pt,even odd rule,rotate=90] %
(0:10pt) -- (36:3.8pt)    -- 
(72:10pt) -- (108:3.8pt)  -- 
(144:10pt) -- (180:3.8pt) -- 
(216:10pt) -- (252:3.8pt) --
(288:10pt) -- (324:3.8pt) -- cycle
(0pt,0pt) circle (2.5pt);
\draw (0,0) circle (10pt);
%\draw[line width=0.6pt,even odd rule,rotate=90] %
%(0:0.9*13pt) --  (36:0.9*6.0pt)    -- 
%(72:0.9*13pt) -- (108:0.9*6.0pt)  -- 
%(144:0.9*13pt) -- (180:0.9*6.0pt) -- 
%(216:0.9*13pt) -- (252:0.9*6.0pt) --
%(288:0.9*13pt) -- (324:0.9*6.0pt) -- cycle ;
\draw[line width=0.9pt,even odd rule,rotate=90] %
(36:12pt) -- (36:6.0pt)    
(108:12pt) -- (108:6.0pt)  
(180:12pt) -- (180:6.0pt) 
(252:12pt) -- (252:6.0pt) 
(324:12pt) -- (324:6.0pt) 
(0pt,0pt) circle (2.5pt);
% The following is needed to center the darn thing.
\draw[opacity=0] (0pt,0pt) circle (13pt) ;
 }}}



\pgfdeclareplotmark{m6b}{%
\node[scale=\mSixscale*\AtoBscale*\symscale] at (0,0) {\tikz {%
\draw[fill,line width=0.2pt,even odd rule,rotate=45] %
(0:10pt)   --  (45:5pt) -- 
(90:10pt)  -- (135:5pt) -- 
(180:10pt) -- (225:5pt) --
(270:10pt) -- (315:5pt) -- cycle 
(0pt,0pt) circle (3pt) ;
 }}}


\pgfdeclareplotmark{m6bv}{%
\node[scale=\mSixscale*\AtoBscale*\symscale] at (0,0) {\tikz {%
\draw[fill,line width=0.2pt,even odd rule,rotate=45] %
(0:10pt)   --  (45:5pt) -- 
(90:10pt)  -- (135:5pt) -- 
(180:10pt) -- (225:5pt) --
(270:10pt) -- (315:5pt) -- cycle 
(0pt,0pt) circle (3pt) ;
\draw (0,0) circle (10pt);
%\draw[line width=0.6pt,even odd rule,rotate=45] %
%(0:13pt)   --  (45:7pt) -- 
%(90:13pt)  -- (135:7pt) -- 
%(180:13pt) -- (225:7pt) --
%(270:13pt) -- (315:7pt) -- cycle ;
 }}}

\pgfdeclareplotmark{m6bb}{%
\node[scale=\mSixscale*\AtoBscale*\symscale] at (0,0) {\tikz {%
\draw[fill,line width=0.2pt,even odd rule,rotate=45] %
(0:10pt)   --  (45:5pt) -- 
(90:10pt)  -- (135:5pt) -- 
(180:10pt) -- (225:5pt) --
(270:10pt) -- (315:5pt) -- cycle 
(0pt,0pt) circle (3pt) ;
\draw[line width=1.2pt,even odd rule,rotate=45] %
(45:9pt)   --  (45:5pt) 
(135:9pt)  -- (135:5pt) 
(225:9pt) -- (225:5pt) 
(315:9pt) -- (315:5pt) ;
 }}}

\pgfdeclareplotmark{m6bvb}{%
\node[scale=\mSixscale*\AtoBscale*\symscale] at (0,0) {\tikz {%
\draw[fill,line width=0.2pt,even odd rule,rotate=45] %
(0:10pt)   --  (45:5pt) -- 
(90:10pt)  -- (135:5pt) -- 
(180:10pt) -- (225:5pt) --
(270:10pt) -- (315:5pt) -- cycle 
(0pt,0pt) circle (3pt) ;
\draw (0,0) circle (10pt);
%\draw[line width=0.6pt,even odd rule,rotate=45] %
%(0:13pt)   --  (45:7pt) -- 
%(90:13pt)  -- (135:7pt) -- 
%(180:13pt) -- (225:7pt) --
%(270:13pt) -- (315:7pt) -- cycle ;
\draw[line width=1.2pt,even odd rule,rotate=45] %
(45:11pt)   --  (45:7pt) 
(135:11pt)  -- (135:7pt) 
(225:11pt)  -- (225:7pt) 
(315:11pt)  -- (315:7pt) ;
 }}}


\pgfdeclareplotmark{m6c}{%
\node[scale=\mSixscale*\AtoCscale*\symscale] at (0,0) {\tikz {%
\draw[fill,line width=0.2pt,even odd rule,rotate=90] %
(0pt,0pt) circle (5pt) ;
 }}}


\pgfdeclareplotmark{m6cv}{%
\node[scale=\mSixscale*\AtoCscale*\symscale] at (0,0) {\tikz {%
\draw[fill,line width=0.2pt,even odd rule,rotate=90] %
(0pt,0pt) circle (5pt) ;
\draw[line width=0.6pt,even odd rule,rotate=90] %
(0pt,0pt) circle (7pt) ;
 }}}

\pgfdeclareplotmark{m6cb}{%
\node[scale=\mSixscale*\AtoCscale*\symscale] at (0,0) {\tikz {%
\draw[fill,line width=0.2pt,even odd rule,rotate=90] %
(0pt,0pt) circle (5pt) ;
\draw[line width=2pt,even odd rule] %
(-8pt,0pt) -- (8pt,0pt) ;
 }}}

\pgfdeclareplotmark{m6cvb}{%
\node[scale=\mSixscale*\AtoCscale*\symscale] at (0,0) {\tikz {%
\draw[fill,line width=0.2pt,even odd rule,rotate=90] %
(0pt,0pt) circle (5pt) ;
\draw[line width=0.6pt,even odd rule,rotate=90] %
(0pt,0pt) circle (7pt) ;
\draw[line width=2pt,even odd rule] %
(-9pt,0pt) -- (-7pt,0pt) 
(9pt,0pt) -- (7pt,0pt) ;
 }}}




%%%%%%%%%%%%%%%%%%%%%%%%%%%%%%%%%%%%%%%%%%%%%%%%%%%%%%%%%%%%%%%%%%%%%%%%%%%%%%%%%%%%%%%%%%
%%%%%%%%%%%%%%%%%%%%%%%%%%%%%%%%%%%%%%%%%%%%%%%%%%%%%%%%%%%%%%%%%%%%%%%%%%%%%%%%%%%%%%%%%%
%   Nebulae markers
%%%%%%%%%%%%%%%%%%%%%%%%%%%%%%%%%%%%%%%%%%%%%%%%%%%%%%%%%%%%%%%%%%%%%%%%%%%%%%%%%%%%%%%%%%
%%%%%%%%%%%%%%%%%%%%%%%%%%%%%%%%%%%%%%%%%%%%%%%%%%%%%%%%%%%%%%%%%%%%%%%%%%%%%%%%%%%%%%%%%%


\pgfdeclarepatternformonly{dense crosshatch dots}{\pgfqpoint{-1pt}{-1pt}}{\pgfqpoint{.6pt}{.6pt}}{\pgfqpoint{.8pt}{.8pt}}%
{
    \pgfpathcircle{\pgfqpoint{0pt}{0pt}}{.3pt}
    \pgfpathcircle{\pgfqpoint{1pt}{1pt}}{.3pt}
    \pgfusepath{fill}
}

% Open Cluster
%\pgfdeclareplotmark{OC}{%
%\node at (0,0) {\tikz { 
%\draw[scale=3,draw opacity=0,line width=.2pt,pattern=dense crosshatch dots] %
%(0:1pt) -- (30:0.75pt)  -- (60:1pt) -- (90:0.75pt)  -- (120:1pt) --  (150:0.75pt)  -- (180:1pt) --
%(210:0.75pt) -- (240:1pt)  -- (270:0.75pt) -- (300:1pt) -- (330:0.75pt)  -- (0:1pt) ;
%};};   
%}

\pgfdeclareplotmark{OC}{%
\node at (0,0) {\tikz {%
%\clip (0:1pt) -- (30:0.75pt)  -- (60:1pt) -- (90:0.75pt)  -- (120:1pt) --  (150:0.75pt)  -- (180:1pt) --
%(210:0.75pt) -- (240:1pt)  -- (270:0.75pt) -- (300:1pt) -- (330:0.75pt)  -- (0:1pt) ;
\filldraw[scale=3,draw opacity=1,line width=.2pt]
(0,0) circle (.1pt)
(0:.4pt) circle (.1pt) (60:.4pt) circle (.1pt) (120:.4pt) circle (.1pt)
(180:.4pt) circle (.1pt) (240:.4pt) circle (.1pt) (300:.4pt) circle (.1pt)
%
(30:.75pt) circle (.1pt) (90:.75pt) circle (.1pt) (150:.75pt) circle (.1pt)
(210:.75pt) circle (.1pt) (270:.75pt) circle (.1pt) (330:.75pt) circle (.1pt)
;
}}}

% Globular Cluster
\pgfdeclareplotmark{GC}{%
\node at (0,0) {\tikz { 
\filldraw[scale=3,draw opacity=1,line width=.2pt]
(0,0) circle (.1pt)
(0:.4pt) circle (.1pt) (60:.4pt) circle (.1pt) (120:.4pt) circle (.1pt)
(180:.4pt) circle (.1pt) (240:.4pt) circle (.1pt) (300:.4pt) circle (.1pt)
%
(30:.75pt) circle (.1pt) (90:.75pt) circle (.1pt) (150:.75pt) circle (.1pt)
(210:.75pt) circle (.1pt) (270:.75pt) circle (.1pt) (330:.75pt) circle (.1pt)
;
\draw[scale=3,line width=.3pt] %
circle (1pt) ;
}}}

% Emission Nebula
\pgfdeclareplotmark{EN}{%
\node[opacity=0] at (0,0) {\tikz { 
\clip (0,0) circle (2.5pt);
\draw[opacity=1,line width=.3pt] %
(-4pt,-1pt) -- (4pt,7pt) 
(-4pt,-2.5pt) -- (4pt,5.5pt)  
(-4pt,-4pt) -- (4pt,4pt) 
(-4pt,-5.5pt) -- (4pt,2.5pt) 
(-4pt,-7pt) -- (4pt,1pt) 
;
}}}

% Nebuluous Cluster
\pgfdeclareplotmark{CN}{%
\node[opacity=0] at (0,0) {\tikz { 
\clip (0,0) circle (2.5pt);
\draw[opacity=1,line width=.3pt] %
(-4pt,-1pt) -- (4pt,7pt) 
(-4pt,-2.5pt) -- (4pt,5.5pt)  
(-4pt,-4pt) -- (4pt,4pt) 
(-4pt,-5.5pt) -- (4pt,2.5pt) 
(-4pt,-7pt) -- (4pt,1pt) ;
}}  ;
\node at (0,0) {\tikz { 
\filldraw[scale=3,draw opacity=1,line width=.2pt]
(0,0) circle (.1pt)
(0:.4pt) circle (.1pt) (60:.4pt) circle (.1pt) (120:.4pt) circle (.1pt)
(180:.4pt) circle (.1pt) (240:.4pt) circle (.1pt) (300:.4pt) circle (.1pt)
%
(30:.75pt) circle (.1pt) (90:.75pt) circle (.1pt) (150:.75pt) circle (.1pt)
(210:.75pt) circle (.1pt) (270:.75pt) circle (.1pt) (330:.75pt) circle (.1pt)
;
}}}


% Planetary Nebula
\pgfdeclareplotmark{PN}{%
\node at (0,0) {\tikz { 
\draw[line width=.3pt] %
circle (2.7pt);
\clip (0,0) circle (2.7pt);
\draw[opacity=1,line width=.3pt] %
(-4pt,-1pt) -- (4pt,7pt) 
(-4pt,-2.5pt) -- (4pt,5.5pt)  
(-4pt,-4pt) -- (4pt,4pt) 
(-4pt,-5.5pt) -- (4pt,2.5pt) 
(-4pt,-7pt) -- (4pt,1pt) 
;
\draw[line width=.3pt,fill=white] %
circle (1.2pt) ;
}}}

% Galaxies
\pgfdeclareplotmark{GAL}{%
\node [rotate=-15] at (0,0) {\tikz {
\draw (0,0) ellipse (4pt and 1.5pt);
\clip (0,0) ellipse (4pt and 1.5pt);
\draw[opacity=1,line width=.3pt] %
(-4pt,-1pt) -- (4pt,7pt) 
(-4pt,-2.5pt) -- (4pt,5.5pt)  
(-4pt,-4pt) -- (4pt,4pt) 
(-4pt,-5.5pt) -- (4pt,2.5pt) 
(-4pt,-7pt) -- (4pt,1pt) 
;
}}}  ;


 


% 
% Colors
\definecolor{Burgundy}{rgb}{.75,.25,.25}
\definecolor{IndianRed4}{HTML}{8B3A3A}
\definecolor{IndianRed3}{HTML}{CD5555}
\definecolor{IndianRed2}{HTML}{EE6363}
\definecolor{IndianRed1}{HTML}{FF6A6A}
\definecolor{GoldenRod1}{HTML}{FFC125}
\definecolor{Orchid}{rgb}{0.85, 0.44, 0.84}
\definecolor{Mauve}{HTML}{B784A7}
\colorlet{cConstellation}{IndianRed4}
%\colorlet{cConstellation}{burgundy}
\definecolor{ForestGreen}{rgb}{0.13, 0.55, 0.13}

\colorlet{cFrame}{black}
\colorlet{cStars}{black}
\colorlet{cAxes}{black}


\colorlet{cProper}{IndianRed2}
\colorlet{cBayer}{IndianRed3}
\colorlet{cFlamsteed}{IndianRed4}

\colorlet{cTransient}{ForestGreen}

\colorlet{cNGC}{Orchid}
\colorlet{cMessier}{Orchid}

%\colorlet{cNGC}{Mauve}
%\colorlet{cMessier}{Mauve}

%\colorlet{cNGC}{IndianRed4}
%\colorlet{cMessier}{IndianRed4}

%\colorlet{cNGC}{green!80!black}
%\colorlet{cMessier}{blue}

\colorlet{cEcliptic}{GoldenRod1!80!black}
\colorlet{cEquator}{gray}

\colorlet{cGrid}{black}

\definecolor{DeepSkyBlue}{HTML}{00BFFF}
% Milky Way  color
\colorlet{cMW}{DeepSkyBlue}
\colorlet{cMW0}{white}
\colorlet{cMW1}{DeepSkyBlue!10}
\colorlet{cMW2}{DeepSkyBlue!14}
\colorlet{cMW3}{DeepSkyBlue!18}
\colorlet{cMW4}{DeepSkyBlue!22}
\colorlet{cMW5}{DeepSkyBlue!26}

%\colorlet{cMW1}{DeepSkyBlue!15}
%\colorlet{cMW2}{DeepSkyBlue!30}
%\colorlet{cMW3}{DeepSkyBlue!45}
%\colorlet{cMW4}{DeepSkyBlue!60}
%\colorlet{cMW5}{DeepSkyBlue!75}

%\colorlet{cMW6}{DeepSkyBlue!40}
%\colorlet{cMW7}{DeepSkyBlue!50}
%\colorlet{cMW8}{DeepSkyBlue!60}
\colorlet{cMW9}{DeepSkyBlue!70}

\def\symscale {1.5}

\def\AtoBscale {0.9}
\def\AtoCscale {0.8}

\def\mOnescale   {0.48}
\def\mTwoscale   {0.38}
\def\mThreescale {0.3}
\def\mFourscale  {0.23}
\def\mFivescale  {0.15}
\def\mSixscale   {0.1}


% Styles:
\pgfplotsset{MW0/.style={mark=,color=cMW0,fill=cMW0,opacity=1}}
\pgfplotsset{MW1/.style={mark=,color=cMW1,fill=cMW1,opacity=1}}
\pgfplotsset{MW2/.style={mark=,color=cMW2,fill=cMW2,opacity=1}}
\pgfplotsset{MW3/.style={mark=,color=cMW3,fill=cMW3,opacity=1}}
\pgfplotsset{MW4/.style={mark=,color=cMW4,fill=cMW4,opacity=1}}
\pgfplotsset{MW5/.style={mark=,color=cMW5,fill=cMW5,opacity=1}}

%\pgfplotsset{MW6/.style={mark=,color=cMW6,opacity=1}}
%\pgfplotsset{MW7/.style={mark=,color=cMW7,opacity=1}}
%\pgfplotsset{MW8/.style={mark=,color=cMW8,opacity=1}}
%\pgfplotsset{MW9/.style={mark=,color=cMW9,opacity=1}}
\pgfplotsset{MWX/.style={mark=,opacity=1}}
\pgfplotsset{MWR/.style={mark=,color=red,opacity=1}}


% Coordinate grid overlay
\pgfplotsset{coordinategrid/.style={grid=major,major grid style={thin,color=cGrid}}}
\tikzset{coordinategrid/.style={thin,color=cGrid}}
% Ecliptics
\tikzset{ecliptics-full/.style={color=cEcliptic,line width=.4pt,mark=,fill=cEcliptic,opacity=1}}
\tikzset{ecliptics-empty/.style={color=cEcliptic,line width=.4pt,mark=,fill=,fill opacity=0}}
\tikzset{ecliptics-label/.style={color=black,font=\scriptsize}}
% Galactic Equator
\tikzset{MWE-full/.style={color=cMW,line width=.4pt,mark=,fill=cMW,opacity=1}}
\tikzset{MWE-empty/.style={color=cMW,line width=.4pt,mark=,fill=,fill opacity=0}}
\tikzset{MWE-label/.style={color=black,font=\scriptsize}}
% Equator
\tikzset{Equator-full/.style={color=cEquator,line width=.4pt,mark=,fill=cEquator,opacity=1}}
\tikzset{Equator-empty/.style={color=cEquator,line width=.4pt,mark=,fill=,fill opacity=0}}
\tikzset{Equator-label/.style={color=black,font=\scriptsize}}
% Transients
\tikzset{transient-label/.style={color=cTransient,font=\bfseries\scriptsize}}
\tikzset{transient/.style={color=cTransient,line width=1pt,mark=,opacity=1}}

% To enable consistent input scripting:
\newcommand\omicron{o}

% Scale of panels:
\newlength{\tendegree}
\setlength{\tendegree}{3cm}
\newlength{\onedegree}
\setlength{\onedegree}{0.1\tendegree}

% General style definitions
\pgfkeys{/pgf/number format/.cd,fixed,precision=4}


% For RA tickmark labeling
\newcommand{\fh}{$^{\rm h}$}
\newcommand{\fm}{$^{\rm m}$}



% Style for constellation boundaries and names:
\pgfplotsset{constellation/.style={thin,mark=,color=cConstellation,dashed}}
\tikzset{constellation-label/.style={color=cConstellation!80,font=\bfseries}}
\tikzset{constellation-boundary/.style={thin,mark=,color=cConstellation,dashed}}


\tikzset{designation-label/.style={color=cProper,font=\bfseries\scriptsize}}
 
% Style for Star markers and names
\tikzset{stars/.style={color=black}}

\tikzset{Proper/.style={color=cProper,font=\bfseries\footnotesize},pin edge={draw opacity=0}}
\tikzset{Bayer/.style={color=cBayer,font=\bfseries\scriptsize},pin edge={draw opacity=0}}
\tikzset{Flaamsted/.style={color=cFlamsteed,font=\bfseries\tiny},pin edge={draw opacity=0}}
% Intersting objects label
\tikzset{interest/.style={color=cProper}}
\tikzset{interest-label/.style={color=cProper,font=\bfseries\tiny}}

% Style for Nebulae markers and names
\tikzset{NGC/.style={color=cNGC,pattern color=cNGC}}
\tikzset{NGC-label/.style={color=cNGC,font=\bfseries\tiny},pin edge={draw opacity=0}}

\tikzset{Caldwell/.style={color=cMessier,pattern color=cMessier}}
\tikzset{Caldwell-label/.style={color=cMessier,font=\bfseries\scriptsize},pin edge={draw opacity=0}}

\tikzset{Messier/.style={color=cMessier,pattern color=cMessier}}
\tikzset{Messier-label/.style={color=cMessier,font=\bfseries\scriptsize},pin edge={draw opacity=0}}
% An extra one for the Messier galaxies in the Virgo-group 
\tikzset{Messier-label-crowded/.style={pin edge={draw opacity=1,color=cMessier,thin},color=cMessier,font=\bfseries\footnotesize}}



\pgfplotsset{tick label style={font=\normalsize}, label style={font=\large}, legend style={font=\large}
  ,every axis/.append style={scale only axis},scaled ticks=false
  % A3paper has size 297 × 420mm
  ,width=11\tendegree,height=8\tendegree
  ,/pgf/number format/set thousands separator={\,}
%
  ,x dir=reverse  ,ytick={-90,-80,...,90},minor y tick num=0
  ,yticklabels={$-90^\circ$,$-80^\circ$,$-70^\circ$,$-60^\circ$,$-50^\circ$,$-40^\circ$,$-30^\circ$,$-20^\circ$,$-10^\circ$,$0^\circ$,
    $+10^\circ$,$+20^\circ$,$+30^\circ$,$+40^\circ$,$+50^\circ$,$+60^\circ$,$+70^\circ$,$+80^\circ$,$+90^\circ$}
%  
}

% RA tickmark labels in hours
\pgfplotsset{RA_in_hours/.style={,xtick={0,10,...,390},minor x tick num=0
      ,xticklabels={0\fh{\small 00}\fm,0\fh{\small 40}\fm,1\fh{\small 20}\fm,2\fh{\small
      00}\fm,2\fh{\small 40}\fm,3\fh{\small 20}\fm,4\fh{\small 00}\fm,4\fh{\small
      40}\fm,5\fh{\small 20}\fm,6\fh{\small 00}\fm,6\fh{\small 40}\fm,7\fh{\small
      20}\fm,8\fh{\small 00}\fm,8\fh{\small 40}\fm,9\fh{\small 20}\fm,10\fh{\small
      00}\fm,10\fh{\small 40}\fm,11\fh{\small 20}\fm,12\fh{\small 00}\fm,12\fh{\small
      40}\fm,13\fh{\small 20}\fm,14\fh{\small 00}\fm,14\fh{\small 40}\fm,15\fh{\small
      20}\fm,16\fh{\small 00}\fm,16\fh{\small 40}\fm,17\fh{\small 20}\fm,18\fh{\small
      00}\fm,18\fh{\small 40}\fm,19\fh{\small 20}\fm,20\fh{\small 00}\fm,20\fh{\small
      40}\fm,21\fh{\small 20}\fm,22\fh{\small 00}\fm,22\fh{\small 40}\fm,23\fh{\small
      20}\fm,0\fh{\small 00}\fm,0\fh{\small 40}\fm,1\fh{\small 20}\fm}}}

% RA tickmark labels in degree
\pgfplotsset{RA_in_deg/.style={,xtick={0,10,...,390},minor x tick num=0%,x dir=reverse,
      ,tick label style={font=\footnotesize},xticklabels={$0^\circ$,$10^\circ$,$20^\circ$,$30^\circ$, 
      $40^\circ$,$50^\circ$,$60^\circ$,$70^\circ$,$80^\circ$,$90^\circ$,$100^\circ$,$110^\circ$,$120^\circ$,
      $130^\circ$,$140^\circ$,$150^\circ$,$160^\circ$,$170^\circ$,$180^\circ$,$190^\circ$,$200^\circ$,
      $210^\circ$,$220^\circ$,$230^\circ$,$240^\circ$,$250^\circ$,$260^\circ$,$270^\circ$,$280^\circ$,
      $290^\circ$,$300^\circ$,$310^\circ$,$320^\circ$,$330^\circ$,$340^\circ$,$350^\circ$,
      $0^\circ$,$10^\circ$,$20^\circ$}  
}} 


\center

\tikzsetnextfilename{TabulaIII_S}
 
%
% Coordinate limit of tab to plot
\pgfplotsset{xmin=0,xmax=110,ymin=-40,ymax=40,y dir=reverse,x dir=normal}
\begin{tikzpicture}
%\
% Empty plot to establish anchor points
\begin{axis}[name=base,axis lines=none]
\end{axis}

% The Milky Way first, as it should not obstruct coordinate systems 


\begin{axis}[name=milkyWay,axis lines=none]

\addplot[MW1] table[x index=0,y index=1] {./MW/III_MWbase.dat}  -- cycle ;

\addplot[MW2] table[x index=0,y index=1] {./MW/III_ol2_100.dat}  -- cycle ;
\addplot[MW2] table[x index=0,y index=1] {./MW/III_ol2_16.dat}  -- cycle ;
\addplot[MW2] table[x index=0,y index=1] {./MW/III_ol2_19.dat}  -- cycle ;
\addplot[MW2] table[x index=0,y index=1] {./MW/III_ol2_21.dat}  -- cycle ;
\addplot[MW2] table[x index=0,y index=1] {./MW/III_ol2_30.dat}  -- cycle ;
\addplot[MW2] table[x index=0,y index=1] {./MW/III_ol2_32.dat}  -- cycle ;
\addplot[MW2] table[x index=0,y index=1] {./MW/III_ol2_40.dat}  -- cycle ;
\addplot[MW2] table[x index=0,y index=1] {./MW/III_ol2_50.dat}  -- cycle ;
\addplot[MW2] table[x index=0,y index=1] {./MW/III_ol2_55.dat}  -- cycle ;
\addplot[MW2] table[x index=0,y index=1] {./MW/III_ol2_59.dat}  -- cycle ;
\addplot[MW2] table[x index=0,y index=1] {./MW/III_ol2_66.dat}  -- cycle ;
\addplot[MW2] table[x index=0,y index=1] {./MW/III_ol2_69.dat}  -- cycle ;
\addplot[MW2] table[x index=0,y index=1] {./MW/III_ol2_6.dat}  -- cycle ;
\addplot[MW2] table[x index=0,y index=1] {./MW/III_ol2_76.dat}  -- cycle ;
\addplot[MW2] table[x index=0,y index=1] {./MW/III_ol2_82.dat}  -- cycle ;
\addplot[MW2] table[x index=0,y index=1] {./MW/III_ol2_93.dat}  -- cycle ;
\addplot[MW2] table[x index=0,y index=1] {./MW/III_ol2_98.dat}  -- cycle ;

\end{axis}


% The Galactic coordinate system
%
% Some are commented out because they are surrounded by already plotted borders
% The x-coordinate is in fractional hours, so must be times 15
%

\begin{axis}[name=constellations,at=(base.center),anchor=center,axis lines=none]

\draw[MWE-full] (axis cs:2.66308E+02,-2.88819E+01) -- (axis cs:2.66899E+02,-2.80275E+01) -- (axis cs:2.67093E+02,-2.81309E+01) -- (axis cs:2.66503E+02,-2.89861E+01) -- cycle ; 
\draw[MWE-full] (axis cs:2.67481E+02,-2.71706E+01) -- (axis cs:2.68054E+02,-2.63113E+01) -- (axis cs:2.68246E+02,-2.64131E+01) -- (axis cs:2.67674E+02,-2.72732E+01) -- cycle ; 
\draw[MWE-full] (axis cs:2.68618E+02,-2.54498E+01) -- (axis cs:2.69174E+02,-2.45861E+01) -- (axis cs:2.69365E+02,-2.46865E+01) -- (axis cs:2.68809E+02,-2.55508E+01) -- cycle ; 
\draw[MWE-full] (axis cs:2.69723E+02,-2.37204E+01) -- (axis cs:2.70264E+02,-2.28528E+01) -- (axis cs:2.70453E+02,-2.29518E+01) -- (axis cs:2.69912E+02,-2.38201E+01) -- cycle ; 
\draw[MWE-full] (axis cs:2.70799E+02,-2.19834E+01) -- (axis cs:2.71327E+02,-2.11122E+01) -- (axis cs:2.71514E+02,-2.12100E+01) -- (axis cs:2.70986E+02,-2.20817E+01) -- cycle ; 
\draw[MWE-full] (axis cs:2.71848E+02,-2.02394E+01) -- (axis cs:2.72364E+02,-1.93650E+01) -- (axis cs:2.72550E+02,-1.94617E+01) -- (axis cs:2.72035E+02,-2.03366E+01) -- cycle ; 
\draw[MWE-full] (axis cs:2.72874E+02,-1.84892E+01) -- (axis cs:2.73379E+02,-1.76121E+01) -- (axis cs:2.73564E+02,-1.77078E+01) -- (axis cs:2.73059E+02,-1.85854E+01) -- cycle ; 
\draw[MWE-full] (axis cs:2.73879E+02,-1.67336E+01) -- (axis cs:2.74375E+02,-1.58540E+01) -- (axis cs:2.74558E+02,-1.59488E+01) -- (axis cs:2.74063E+02,-1.68289E+01) -- cycle ; 
\draw[MWE-full] (axis cs:2.74866E+02,-1.49732E+01) -- (axis cs:2.75353E+02,-1.40913E+01) -- (axis cs:2.75535E+02,-1.41854E+01) -- (axis cs:2.75049E+02,-1.50676E+01) -- cycle ; 
\draw[MWE-full] (axis cs:2.75837E+02,-1.32085E+01) -- (axis cs:2.76316E+02,-1.23248E+01) -- (axis cs:2.76498E+02,-1.24181E+01) -- (axis cs:2.76018E+02,-1.33022E+01) -- cycle ; 
\draw[MWE-full] (axis cs:2.76793E+02,-1.14402E+01) -- (axis cs:2.77267E+02,-1.05548E+01) -- (axis cs:2.77447E+02,-1.06476E+01) -- (axis cs:2.76974E+02,-1.15333E+01) -- cycle ; 
\draw[MWE-full] (axis cs:2.77738E+02,-9.66873E+00) -- (axis cs:2.78206E+02,-8.78200E+00) -- (axis cs:2.78385E+02,-8.87430E+00) -- (axis cs:2.77917E+02,-9.76126E+00) -- cycle ; 
\draw[MWE-full] (axis cs:2.78672E+02,-7.89468E+00) -- (axis cs:2.79136E+02,-7.00682E+00) -- (axis cs:2.79315E+02,-7.09872E+00) -- (axis cs:2.78851E+02,-7.98676E+00) -- cycle ; 
\draw[MWE-full] (axis cs:2.79598E+02,-6.11851E+00) -- (axis cs:2.80059E+02,-5.22978E+00) -- (axis cs:2.80237E+02,-5.32137E+00) -- (axis cs:2.79777E+02,-6.21024E+00) -- cycle ; 
\draw[MWE-full] (axis cs:2.80518E+02,-4.34070E+00) -- (axis cs:2.80977E+02,-3.45134E+00) -- (axis cs:2.81155E+02,-3.54271E+00) -- (axis cs:2.80697E+02,-4.43217E+00) -- cycle ; 
\draw[MWE-full] (axis cs:2.81434E+02,-2.56174E+00) -- (axis cs:2.81891E+02,-1.67196E+00) -- (axis cs:2.82069E+02,-1.76320E+00) -- (axis cs:2.81612E+02,-2.65303E+00) -- cycle ; 
\draw[MWE-full] (axis cs:2.82347E+02,-7.82062E-01) -- (axis cs:2.82917E+02,1.07899E-01) -- (axis cs:2.82981E+02,1.66967E-02) -- (axis cs:2.82525E+02,-8.73274E-01) -- cycle ; 
\draw[MWE-full] (axis cs:2.83259E+02,9.97867E-01) -- (axis cs:2.83715E+02,1.88779E+00) -- (axis cs:2.83893E+02,1.79654E+00) -- (axis cs:2.83437E+02,9.06652E-01) -- cycle ; 
\draw[MWE-full] (axis cs:2.84172E+02,2.77760E+00) -- (axis cs:2.84629E+02,3.66725E+00) -- (axis cs:2.84808E+02,3.57586E+00) -- (axis cs:2.84350E+02,2.68629E+00) -- cycle ; 
\draw[MWE-full] (axis cs:2.85088E+02,4.55668E+00) -- (axis cs:2.85547E+02,5.44583E+00) -- (axis cs:2.85725E+02,5.35422E+00) -- (axis cs:2.85266E+02,4.46519E+00) -- cycle ; 
\draw[MWE-full] (axis cs:2.86008E+02,6.33465E+00) -- (axis cs:2.86470E+02,7.22308E+00) -- (axis cs:2.86649E+02,7.13115E+00) -- (axis cs:2.86186E+02,6.24290E+00) -- cycle ; 
\draw[MWE-full] (axis cs:2.86934E+02,8.11105E+00) -- (axis cs:2.87400E+02,8.99851E+00) -- (axis cs:2.87580E+02,8.90618E+00) -- (axis cs:2.87113E+02,8.01893E+00) -- cycle ; 
\draw[MWE-full] (axis cs:2.87868E+02,9.88539E+00) -- (axis cs:2.88339E+02,1.07716E+01) -- (axis cs:2.88519E+02,1.06788E+01) -- (axis cs:2.88048E+02,9.79283E+00) -- cycle ; 
\draw[MWE-full] (axis cs:2.88813E+02,1.16572E+01) -- (axis cs:2.89289E+02,1.25420E+01) -- (axis cs:2.89470E+02,1.24486E+01) -- (axis cs:2.88993E+02,1.15641E+01) -- cycle ; 
\draw[MWE-full] (axis cs:2.89769E+02,1.34259E+01) -- (axis cs:2.90253E+02,1.43090E+01) -- (axis cs:2.90435E+02,1.42149E+01) -- (axis cs:2.89951E+02,1.33322E+01) -- cycle ; 
\draw[MWE-full] (axis cs:2.90740E+02,1.51910E+01) -- (axis cs:2.91231E+02,1.60721E+01) -- (axis cs:2.91414E+02,1.59772E+01) -- (axis cs:2.90922E+02,1.50966E+01) -- cycle ; 
\draw[MWE-full] (axis cs:2.91727E+02,1.69520E+01) -- (axis cs:2.92227E+02,1.78307E+01) -- (axis cs:2.92411E+02,1.77350E+01) -- (axis cs:2.91910E+02,1.68567E+01) -- cycle ; 
\draw[MWE-full] (axis cs:2.92732E+02,1.87082E+01) -- (axis cs:2.93242E+02,1.95843E+01) -- (axis cs:2.93428E+02,1.94875E+01) -- (axis cs:2.92917E+02,1.86119E+01) -- cycle ; 
\draw[MWE-full] (axis cs:2.93758E+02,2.04590E+01) -- (axis cs:2.94280E+02,2.13321E+01) -- (axis cs:2.94467E+02,2.12342E+01) -- (axis cs:2.93944E+02,2.03617E+01) -- cycle ; 
\draw[MWE-full] (axis cs:2.94808E+02,2.22037E+01) -- (axis cs:2.95342E+02,2.30735E+01) -- (axis cs:2.95531E+02,2.29744E+01) -- (axis cs:2.94996E+02,2.21052E+01) -- cycle ; 
\draw[MWE-full] (axis cs:2.95884E+02,2.39415E+01) -- (axis cs:2.96433E+02,2.48076E+01) -- (axis cs:2.96623E+02,2.47072E+01) -- (axis cs:2.96073E+02,2.38417E+01) -- cycle ; 
\draw[MWE-full] (axis cs:2.96989E+02,2.56717E+01) -- (axis cs:2.97554E+02,2.65337E+01) -- (axis cs:2.97746E+02,2.64318E+01) -- (axis cs:2.97180E+02,2.55705E+01) -- cycle ; 
\draw[MWE-full] (axis cs:2.98127E+02,2.73934E+01) -- (axis cs:2.98709E+02,2.82508E+01) -- (axis cs:2.98903E+02,2.81473E+01) -- (axis cs:2.98320E+02,2.72907E+01) -- cycle ; 
\draw[MWE-full] (axis cs:2.99300E+02,2.91056E+01) -- (axis cs:2.99901E+02,2.99579E+01) -- (axis cs:3.00098E+02,2.98527E+01) -- (axis cs:2.99495E+02,2.90013E+01) -- cycle ; 
\draw[MWE-full] (axis cs:3.00513E+02,3.08074E+01) -- (axis cs:3.01135E+02,3.16540E+01) -- (axis cs:3.01334E+02,3.15469E+01) -- (axis cs:3.00710E+02,3.07013E+01) -- cycle ; 
\draw[MWE-full] (axis cs:3.01769E+02,3.24975E+01) -- (axis cs:3.02415E+02,3.33378E+01) -- (axis cs:3.02616E+02,3.32287E+01) -- (axis cs:3.01969E+02,3.23894E+01) -- cycle ; 
\draw[MWE-full] (axis cs:3.03073E+02,3.41747E+01) -- (axis cs:3.03745E+02,3.50081E+01) -- (axis cs:3.03948E+02,3.48968E+01) -- (axis cs:3.03275E+02,3.40646E+01) -- cycle ; 
\draw[MWE-full] (axis cs:3.04430E+02,3.58377E+01) -- (axis cs:3.05130E+02,3.66633E+01) -- (axis cs:3.05335E+02,3.65497E+01) -- (axis cs:3.04634E+02,3.57252E+01) -- cycle ; 
\draw[MWE-full] (axis cs:3.05844E+02,3.74848E+01) -- (axis cs:3.06575E+02,3.83019E+01) -- (axis cs:3.06782E+02,3.81857E+01) -- (axis cs:3.06051E+02,3.73699E+01) -- cycle ; 
\draw[MWE-full] (axis cs:3.07322E+02,3.91143E+01) -- (axis cs:3.08087E+02,3.99220E+01) -- (axis cs:3.08296E+02,3.98031E+01) -- (axis cs:3.07531E+02,3.89969E+01) -- cycle ; 
\draw[MWE-full] (axis cs:3.08869E+02,4.07244E+01) -- (axis cs:3.09671E+02,4.15215E+01) -- (axis cs:3.09883E+02,4.13998E+01) -- (axis cs:3.09080E+02,4.06042E+01) -- cycle ; 
\draw[MWE-full] (axis cs:3.10492E+02,4.23130E+01) -- (axis cs:3.11335E+02,4.30984E+01) -- (axis cs:3.11548E+02,4.29736E+01) -- (axis cs:3.10705E+02,4.21897E+01) -- cycle ; 
\draw[MWE-full] (axis cs:3.12199E+02,4.38775E+01) -- (axis cs:3.13086E+02,4.46499E+01) -- (axis cs:3.13301E+02,4.45218E+01) -- (axis cs:3.12414E+02,4.37511E+01) -- cycle ; 
\draw[MWE-full] (axis cs:3.13996E+02,4.54153E+01) -- (axis cs:3.14932E+02,4.61733E+01) -- (axis cs:3.15149E+02,4.60418E+01) -- (axis cs:3.14213E+02,4.52855E+01) -- cycle ; 
\draw[MWE-full] (axis cs:3.15893E+02,4.69235E+01) -- (axis cs:3.16882E+02,4.76654E+01) -- (axis cs:3.17100E+02,4.75301E+01) -- (axis cs:3.16111E+02,4.67901E+01) -- cycle ; 
\draw[MWE-full] (axis cs:3.17899E+02,4.83986E+01) -- (axis cs:3.18945E+02,4.91226E+01) -- (axis cs:3.19165E+02,4.89834E+01) -- (axis cs:3.18118E+02,4.82614E+01) -- cycle ; 
\draw[MWE-full] (axis cs:3.20023E+02,4.98368E+01) -- (axis cs:3.21132E+02,5.05409E+01) -- (axis cs:3.21351E+02,5.03976E+01) -- (axis cs:3.20242E+02,4.96956E+01) -- cycle ; 
\draw[MWE-full] (axis cs:3.22274E+02,5.12341E+01) -- (axis cs:3.23452E+02,5.19158E+01) -- (axis cs:3.23670E+02,5.17682E+01) -- (axis cs:3.22493E+02,5.10886E+01) -- cycle ; 
\draw[MWE-full] (axis cs:3.24665E+02,5.25855E+01) -- (axis cs:3.25915E+02,5.32424E+01) -- (axis cs:3.26132E+02,5.30902E+01) -- (axis cs:3.24882E+02,5.24356E+01) -- cycle ; 
\draw[MWE-full] (axis cs:3.27204E+02,5.38858E+01) -- (axis cs:3.28533E+02,5.45151E+01) -- (axis cs:3.28746E+02,5.43582E+01) -- (axis cs:3.27419E+02,5.37314E+01) -- cycle ; 
\draw[MWE-full] (axis cs:3.29903E+02,5.51293E+01) -- (axis cs:3.31315E+02,5.57277E+01) -- (axis cs:3.31523E+02,5.55661E+01) -- (axis cs:3.30114E+02,5.49701E+01) -- cycle ; 
\draw[MWE-full] (axis cs:3.32770E+02,5.63095E+01) -- (axis cs:3.34269E+02,5.68737E+01) -- (axis cs:3.34472E+02,5.67072E+01) -- (axis cs:3.32976E+02,5.61454E+01) -- cycle ; 
\draw[MWE-full] (axis cs:3.35814E+02,5.74194E+01) -- (axis cs:3.37405E+02,5.79457E+01) -- (axis cs:3.37598E+02,5.77743E+01) -- (axis cs:3.36012E+02,5.72505E+01) -- cycle ; 
\draw[MWE-full] (axis cs:3.39042E+02,5.84516E+01) -- (axis cs:3.40725E+02,5.89360E+01) -- (axis cs:3.40908E+02,5.87597E+01) -- (axis cs:3.39230E+02,5.82777E+01) -- cycle ; 
\draw[MWE-full] (axis cs:3.42456E+02,5.93980E+01) -- (axis cs:3.44233E+02,5.98365E+01) -- (axis cs:3.44402E+02,5.96555E+01) -- (axis cs:3.42632E+02,5.92193E+01) -- cycle ; 
\draw[MWE-full] (axis cs:3.46056E+02,6.02505E+01) -- (axis cs:3.47924E+02,6.06388E+01) -- (axis cs:3.48077E+02,6.04533E+01) -- (axis cs:3.46217E+02,6.00672E+01) -- cycle ; 
\draw[MWE-full] (axis cs:3.49837E+02,6.10005E+01) -- (axis cs:3.51791E+02,6.13346E+01) -- (axis cs:3.51924E+02,6.11451E+01) -- (axis cs:3.49979E+02,6.08130E+01) -- cycle ; 
\draw[MWE-full] (axis cs:3.53786E+02,6.16401E+01) -- (axis cs:3.55819E+02,6.19160E+01) -- (axis cs:3.55929E+02,6.17229E+01) -- (axis cs:3.53908E+02,6.14487E+01) -- cycle ; 
\draw[MWE-full] (axis cs:3.57886E+02,6.21614E+01) -- (axis cs:3.59986E+02,6.23756E+01) -- (axis cs:7.05566E-02,6.21796E+01) -- (axis cs:3.57984E+02,6.19668E+01) -- cycle ; 
\draw[MWE-full] (axis cs:2.11273E+00,6.25578E+01) -- (axis cs:4.26373E+00,6.27073E+01) -- (axis cs:4.32138E+00,6.25091E+01) -- (axis cs:2.18430E+00,6.23606E+01) -- cycle ; 
\draw[MWE-full] (axis cs:6.43417E+00,6.28237E+01) -- (axis cs:8.61932E+00,6.29064E+01) -- (axis cs:8.64801E+00,6.27068E+01) -- (axis cs:6.47748E+00,6.26247E+01) -- cycle ; 
\draw[MWE-full] (axis cs:1.08143E+01,6.29552E+01) -- (axis cs:1.30140E+01,6.29699E+01) -- (axis cs:1.30130E+01,6.27699E+01) -- (axis cs:1.08282E+01,6.27553E+01) -- cycle ; 
\draw[MWE-full] (axis cs:1.52134E+01,6.29504E+01) -- (axis cs:1.74073E+01,6.28969E+01) -- (axis cs:1.73765E+01,6.26974E+01) -- (axis cs:1.51974E+01,6.27506E+01) -- cycle ; 
\draw[MWE-full] (axis cs:1.95907E+01,6.28094E+01) -- (axis cs:2.17586E+01,6.26884E+01) -- (axis cs:2.16990E+01,6.24903E+01) -- (axis cs:1.95453E+01,6.26105E+01) -- cycle ; 
\draw[MWE-full] (axis cs:2.39066E+01,6.25342E+01) -- (axis cs:2.60301E+01,6.23475E+01) -- (axis cs:2.59432E+01,6.21517E+01) -- (axis cs:2.38331E+01,6.23372E+01) -- cycle ; 
\draw[MWE-full] (axis cs:2.81251E+01,6.21289E+01) -- (axis cs:3.01880E+01,6.18792E+01) -- (axis cs:3.00762E+01,6.16863E+01) -- (axis cs:2.80255E+01,6.19345E+01) -- cycle ; 
\draw[MWE-full] (axis cs:3.22156E+01,6.15991E+01) -- (axis cs:3.42049E+01,6.12895E+01) -- (axis cs:3.40709E+01,6.11003E+01) -- (axis cs:3.20923E+01,6.14079E+01) -- cycle ; 
\draw[MWE-full] (axis cs:3.61537E+01,6.09515E+01) -- (axis cs:3.80599E+01,6.05860E+01) -- (axis cs:3.79064E+01,6.04008E+01) -- (axis cs:3.60096E+01,6.07642E+01) -- cycle ; 
\draw[MWE-full] (axis cs:3.99219E+01,6.01940E+01) -- (axis cs:4.17386E+01,5.97766E+01) -- (axis cs:4.15687E+01,5.95959E+01) -- (axis cs:3.97599E+01,6.00110E+01) -- cycle ; 
\draw[MWE-full] (axis cs:4.35091E+01,5.93347E+01) -- (axis cs:4.52331E+01,5.88695E+01) -- (axis cs:4.50497E+01,5.86936E+01) -- (axis cs:4.33322E+01,5.91564E+01) -- cycle ; 
\draw[MWE-full] (axis cs:4.69102E+01,5.83820E+01) -- (axis cs:4.85407E+01,5.78732E+01) -- (axis cs:4.83464E+01,5.77022E+01) -- (axis cs:4.67210E+01,5.82085E+01) -- cycle ; 
\draw[MWE-full] (axis cs:5.01248E+01,5.73442E+01) -- (axis cs:5.16632E+01,5.67958E+01) -- (axis cs:5.14604E+01,5.66296E+01) -- (axis cs:4.99260E+01,5.71755E+01) -- cycle ; 
\draw[MWE-full] (axis cs:5.31565E+01,5.62291E+01) -- (axis cs:5.46057E+01,5.56449E+01) -- (axis cs:5.43965E+01,5.54837E+01) -- (axis cs:5.29503E+01,5.60654E+01) -- cycle ; 
\draw[MWE-full] (axis cs:5.60117E+01,5.50442E+01) -- (axis cs:5.73757E+01,5.44279E+01) -- (axis cs:5.71620E+01,5.42714E+01) -- (axis cs:5.58001E+01,5.48854E+01) -- cycle ; 
\draw[MWE-full] (axis cs:5.86988E+01,5.37966E+01) -- (axis cs:5.99822E+01,5.31512E+01) -- (axis cs:5.97655E+01,5.29994E+01) -- (axis cs:5.84834E+01,5.36425E+01) -- cycle ; 
\draw[MWE-full] (axis cs:6.12273E+01,5.24925E+01) -- (axis cs:6.24353E+01,5.18211E+01) -- (axis cs:6.22168E+01,5.16738E+01) -- (axis cs:6.10096E+01,5.23429E+01) -- cycle ; 
\draw[MWE-full] (axis cs:6.36076E+01,5.11377E+01) -- (axis cs:6.47454E+01,5.04429E+01) -- (axis cs:6.45262E+01,5.02999E+01) -- (axis cs:6.33886E+01,5.09926E+01) -- cycle ; 
\draw[MWE-full] (axis cs:6.58501E+01,4.97374E+01) -- (axis cs:6.69230E+01,4.90218E+01) -- (axis cs:6.67038E+01,4.88829E+01) -- (axis cs:6.56308E+01,4.95965E+01) -- cycle ; 
\draw[MWE-full] (axis cs:6.79652E+01,4.82965E+01) -- (axis cs:6.89781E+01,4.75620E+01) -- (axis cs:6.87597E+01,4.74270E+01) -- (axis cs:6.77464E+01,4.81596E+01) -- cycle ; 
\draw[MWE-full] (axis cs:6.99629E+01,4.68189E+01) -- (axis cs:7.09207E+01,4.60676E+01) -- (axis cs:7.07035E+01,4.59363E+01) -- (axis cs:6.97450E+01,4.66858E+01) -- cycle ; 
\draw[MWE-full] (axis cs:7.18527E+01,4.53086E+01) -- (axis cs:7.27599E+01,4.45422E+01) -- (axis cs:7.25443E+01,4.44144E+01) -- (axis cs:7.16362E+01,4.51791E+01) -- cycle ; 
\draw[MWE-full] (axis cs:7.36436E+01,4.37688E+01) -- (axis cs:7.45045E+01,4.29888E+01) -- (axis cs:7.42908E+01,4.28642E+01) -- (axis cs:7.34289E+01,4.36426E+01) -- cycle ; 
\draw[MWE-full] (axis cs:7.53439E+01,4.22025E+01) -- (axis cs:7.61626E+01,4.14103E+01) -- (axis cs:7.59510E+01,4.12888E+01) -- (axis cs:7.51312E+01,4.20795E+01) -- cycle ; 
\draw[MWE-full] (axis cs:7.69615E+01,4.06124E+01) -- (axis cs:7.77415E+01,3.98092E+01) -- (axis cs:7.75321E+01,3.96906E+01) -- (axis cs:7.67510E+01,4.04924E+01) -- cycle ; 
\draw[MWE-full] (axis cs:7.85035E+01,3.90009E+01) -- (axis cs:7.92483E+01,3.81877E+01) -- (axis cs:7.90411E+01,3.80718E+01) -- (axis cs:7.82952E+01,3.88836E+01) -- cycle ; 
\draw[MWE-full] (axis cs:7.99766E+01,3.73700E+01) -- (axis cs:8.06892E+01,3.65480E+01) -- (axis cs:8.04843E+01,3.64345E+01) -- (axis cs:7.97705E+01,3.72554E+01) -- cycle ; 
\draw[MWE-full] (axis cs:8.13867E+01,3.57218E+01) -- (axis cs:8.20700E+01,3.48916E+01) -- (axis cs:8.18674E+01,3.47805E+01) -- (axis cs:8.11830E+01,3.56095E+01) -- cycle ; 
\draw[MWE-full] (axis cs:8.27396E+01,3.40578E+01) -- (axis cs:8.33961E+01,3.32204E+01) -- (axis cs:8.31957E+01,3.31114E+01) -- (axis cs:8.25381E+01,3.39478E+01) -- cycle ; 
\draw[MWE-full] (axis cs:8.40402E+01,3.23796E+01) -- (axis cs:8.46724E+01,3.15356E+01) -- (axis cs:8.44742E+01,3.14287E+01) -- (axis cs:8.38409E+01,3.22717E+01) -- cycle ; 
\draw[MWE-full] (axis cs:8.52933E+01,3.06886E+01) -- (axis cs:8.59034E+01,2.98387E+01) -- (axis cs:8.57074E+01,2.97337E+01) -- (axis cs:8.50962E+01,3.05826E+01) -- cycle ; 
\draw[MWE-full] (axis cs:8.65033E+01,2.89861E+01) -- (axis cs:8.70933E+01,2.81309E+01) -- (axis cs:8.68993E+01,2.80275E+01) -- (axis cs:8.63082E+01,2.88819E+01) -- cycle ; 
\draw[MWE-full] (axis cs:8.76740E+01,2.72732E+01) -- (axis cs:8.82459E+01,2.64131E+01) -- (axis cs:8.80537E+01,2.63113E+01) -- (axis cs:8.74810E+01,2.71706E+01) -- cycle ; 
\draw[MWE-full] (axis cs:8.88093E+01,2.55508E+01) -- (axis cs:8.93646E+01,2.46865E+01) -- (axis cs:8.91743E+01,2.45861E+01) -- (axis cs:8.86180E+01,2.54498E+01) -- cycle ; 
\draw[MWE-full] (axis cs:8.99124E+01,2.38201E+01) -- (axis cs:9.04529E+01,2.29518E+01) -- (axis cs:9.02642E+01,2.28528E+01) -- (axis cs:8.97229E+01,2.37204E+01) -- cycle ; 
\draw[MWE-full] (axis cs:9.09865E+01,2.20817E+01) -- (axis cs:9.15135E+01,2.12100E+01) -- (axis cs:9.13265E+01,2.11122E+01) -- (axis cs:9.07986E+01,2.19834E+01) -- cycle ; 
\draw[MWE-full] (axis cs:9.20345E+01,2.03366E+01) -- (axis cs:9.25496E+01,1.94617E+01) -- (axis cs:9.23640E+01,1.93650E+01) -- (axis cs:9.18481E+01,2.02394E+01) -- cycle ; 
\draw[MWE-full] (axis cs:9.30592E+01,1.85854E+01) -- (axis cs:9.35636E+01,1.77078E+01) -- (axis cs:9.33793E+01,1.76121E+01) -- (axis cs:9.28742E+01,1.84892E+01) -- cycle ; 
\draw[MWE-full] (axis cs:9.40631E+01,1.68289E+01) -- (axis cs:9.45580E+01,1.59488E+01) -- (axis cs:9.43750E+01,1.58540E+01) -- (axis cs:9.38795E+01,1.67336E+01) -- cycle ; 
\draw[MWE-full] (axis cs:9.50487E+01,1.50676E+01) -- (axis cs:9.55353E+01,1.41854E+01) -- (axis cs:9.53533E+01,1.40913E+01) -- (axis cs:9.48661E+01,1.49732E+01) -- cycle ; 
\draw[MWE-full] (axis cs:9.60182E+01,1.33022E+01) -- (axis cs:9.64975E+01,1.24181E+01) -- (axis cs:9.63165E+01,1.23248E+01) -- (axis cs:9.58366E+01,1.32085E+01) -- cycle ; 
\draw[MWE-full] (axis cs:9.69737E+01,1.15333E+01) -- (axis cs:9.74469E+01,1.06476E+01) -- (axis cs:9.72666E+01,1.05548E+01) -- (axis cs:9.67931E+01,1.14402E+01) -- cycle ; 
\draw[MWE-full] (axis cs:9.79174E+01,9.76126E+00) -- (axis cs:9.83854E+01,8.87430E+00) -- (axis cs:9.82058E+01,8.78200E+00) -- (axis cs:9.77375E+01,9.66874E+00) -- cycle ; 
\draw[MWE-full] (axis cs:9.88512E+01,7.98676E+00) -- (axis cs:9.93149E+01,7.09872E+00) -- (axis cs:9.91360E+01,7.00683E+00) -- (axis cs:9.86719E+01,7.89468E+00) -- cycle ; 
\draw[MWE-full] (axis cs:9.97769E+01,6.21024E+00) -- (axis cs:1.00237E+02,5.32137E+00) -- (axis cs:1.00059E+02,5.22978E+00) -- (axis cs:9.95982E+01,6.11851E+00) -- cycle ; 
\draw[MWE-full] (axis cs:1.00697E+02,4.43217E+00) -- (axis cs:1.01155E+02,3.54271E+00) -- (axis cs:1.00977E+02,3.45134E+00) -- (axis cs:1.00518E+02,4.34070E+00) -- cycle ; 
\draw[MWE-full] (axis cs:1.01612E+02,2.65303E+00) -- (axis cs:1.02069E+02,1.76320E+00) -- (axis cs:1.01891E+02,1.67196E+00) -- (axis cs:1.01434E+02,2.56174E+00) -- cycle ; 
\draw[MWE-full] (axis cs:1.02525E+02,8.73274E-01) -- (axis cs:1.02981E+02,-1.66966E-02) -- (axis cs:1.02917E+02,-1.07899E-01) -- (axis cs:1.02347E+02,7.82062E-01) -- cycle ; 
\draw[MWE-full] (axis cs:1.03437E+02,-9.06652E-01) -- (axis cs:1.03893E+02,-1.79654E+00) -- (axis cs:1.03715E+02,-1.88779E+00) -- (axis cs:1.03259E+02,-9.97867E-01) -- cycle ; 
\draw[MWE-full] (axis cs:1.04350E+02,-2.68629E+00) -- (axis cs:1.04808E+02,-3.57586E+00) -- (axis cs:1.04629E+02,-3.66725E+00) -- (axis cs:1.04172E+02,-2.77760E+00) -- cycle ; 
\draw[MWE-full] (axis cs:1.05266E+02,-4.46519E+00) -- (axis cs:1.05726E+02,-5.35422E+00) -- (axis cs:1.05547E+02,-5.44583E+00) -- (axis cs:1.05088E+02,-4.55668E+00) -- cycle ; 
\draw[MWE-full] (axis cs:1.06186E+02,-6.24290E+00) -- (axis cs:1.06649E+02,-7.13115E+00) -- (axis cs:1.06470E+02,-7.22308E+00) -- (axis cs:1.06008E+02,-6.33465E+00) -- cycle ; 
\draw[MWE-full] (axis cs:1.07113E+02,-8.01893E+00) -- (axis cs:1.07580E+02,-8.90618E+00) -- (axis cs:1.07400E+02,-8.99851E+00) -- (axis cs:1.06934E+02,-8.11105E+00) -- cycle ; 
\draw[MWE-full] (axis cs:1.08048E+02,-9.79283E+00) -- (axis cs:1.08520E+02,-1.06788E+01) -- (axis cs:1.08339E+02,-1.07716E+01) -- (axis cs:1.07868E+02,-9.88539E+00) -- cycle ; 
\draw[MWE-full] (axis cs:1.08993E+02,-1.15641E+01) -- (axis cs:1.09471E+02,-1.24486E+01) -- (axis cs:1.09289E+02,-1.25420E+01) -- (axis cs:1.08813E+02,-1.16572E+01) -- cycle ; 
\draw[MWE-full] (axis cs:1.09951E+02,-1.33322E+01) -- (axis cs:1.10435E+02,-1.42149E+01) -- (axis cs:1.10253E+02,-1.43090E+01) -- (axis cs:1.09769E+02,-1.34259E+01) -- cycle ; 
\draw[MWE-full] (axis cs:1.10922E+02,-1.50966E+01) -- (axis cs:1.11414E+02,-1.59772E+01) -- (axis cs:1.11231E+02,-1.60721E+01) -- (axis cs:1.10740E+02,-1.51910E+01) -- cycle ; 
\draw[MWE-full] (axis cs:1.11910E+02,-1.68567E+01) -- (axis cs:1.12411E+02,-1.77350E+01) -- (axis cs:1.12227E+02,-1.78307E+01) -- (axis cs:1.11727E+02,-1.69520E+01) -- cycle ; 
\draw[MWE-full] (axis cs:1.12917E+02,-1.86119E+01) -- (axis cs:1.13428E+02,-1.94875E+01) -- (axis cs:1.13242E+02,-1.95843E+01) -- (axis cs:1.12732E+02,-1.87082E+01) -- cycle ; 
\draw[MWE-full] (axis cs:1.13945E+02,-2.03617E+01) -- (axis cs:1.14467E+02,-2.12342E+01) -- (axis cs:1.14280E+02,-2.13321E+01) -- (axis cs:1.13758E+02,-2.04590E+01) -- cycle ; 
\draw[MWE-full] (axis cs:1.14996E+02,-2.21052E+01) -- (axis cs:1.15531E+02,-2.29744E+01) -- (axis cs:1.15342E+02,-2.30735E+01) -- (axis cs:1.14808E+02,-2.22037E+01) -- cycle ; 
\draw[MWE-full] (axis cs:1.16073E+02,-2.38417E+01) -- (axis cs:1.16623E+02,-2.47072E+01) -- (axis cs:1.16433E+02,-2.48076E+01) -- (axis cs:1.15884E+02,-2.39415E+01) -- cycle ; 
\draw[MWE-full] (axis cs:1.17180E+02,-2.55705E+01) -- (axis cs:1.17746E+02,-2.64318E+01) -- (axis cs:1.17554E+02,-2.65337E+01) -- (axis cs:1.16989E+02,-2.56717E+01) -- cycle ; 
\draw[MWE-full] (axis cs:1.18320E+02,-2.72907E+01) -- (axis cs:1.18903E+02,-2.81473E+01) -- (axis cs:1.18709E+02,-2.82508E+01) -- (axis cs:1.18127E+02,-2.73934E+01) -- cycle ; 
\draw[MWE-full] (axis cs:1.19495E+02,-2.90013E+01) -- (axis cs:1.20098E+02,-2.98527E+01) -- (axis cs:1.19901E+02,-2.99579E+01) -- (axis cs:1.19300E+02,-2.91056E+01) -- cycle ; 
\draw[MWE-full] (axis cs:1.20710E+02,-3.07013E+01) -- (axis cs:1.21334E+02,-3.15469E+01) -- (axis cs:1.21135E+02,-3.16540E+01) -- (axis cs:1.20513E+02,-3.08074E+01) -- cycle ; 
\draw[MWE-full] (axis cs:1.21969E+02,-3.23894E+01) -- (axis cs:1.22616E+02,-3.32287E+01) -- (axis cs:1.22415E+02,-3.33378E+01) -- (axis cs:1.21769E+02,-3.24975E+01) -- cycle ; 
\draw[MWE-full] (axis cs:1.23275E+02,-3.40646E+01) -- (axis cs:1.23948E+02,-3.48968E+01) -- (axis cs:1.23745E+02,-3.50081E+01) -- (axis cs:1.23073E+02,-3.41747E+01) -- cycle ; 
\draw[MWE-full] (axis cs:1.24634E+02,-3.57252E+01) -- (axis cs:1.25335E+02,-3.65497E+01) -- (axis cs:1.25130E+02,-3.66633E+01) -- (axis cs:1.24430E+02,-3.58377E+01) -- cycle ; 
\draw[MWE-full] (axis cs:1.26051E+02,-3.73699E+01) -- (axis cs:1.26782E+02,-3.81857E+01) -- (axis cs:1.26575E+02,-3.83019E+01) -- (axis cs:1.25844E+02,-3.74848E+01) -- cycle ; 
\draw[MWE-full] (axis cs:1.27531E+02,-3.89969E+01) -- (axis cs:1.28296E+02,-3.98031E+01) -- (axis cs:1.28087E+02,-3.99220E+01) -- (axis cs:1.27322E+02,-3.91143E+01) -- cycle ; 
\draw[MWE-full] (axis cs:1.29080E+02,-4.06042E+01) -- (axis cs:1.29883E+02,-4.13998E+01) -- (axis cs:1.29671E+02,-4.15215E+01) -- (axis cs:1.28869E+02,-4.07244E+01) -- cycle ; 
\draw[MWE-full] (axis cs:1.30705E+02,-4.21897E+01) -- (axis cs:1.31549E+02,-4.29736E+01) -- (axis cs:1.31335E+02,-4.30984E+01) -- (axis cs:1.30492E+02,-4.23130E+01) -- cycle ; 
\draw[MWE-full] (axis cs:1.32414E+02,-4.37511E+01) -- (axis cs:1.33301E+02,-4.45218E+01) -- (axis cs:1.33086E+02,-4.46499E+01) -- (axis cs:1.32199E+02,-4.38775E+01) -- cycle ; 
\draw[MWE-full] (axis cs:1.34213E+02,-4.52855E+01) -- (axis cs:1.35149E+02,-4.60418E+01) -- (axis cs:1.34932E+02,-4.61733E+01) -- (axis cs:1.33996E+02,-4.54153E+01) -- cycle ; 
\draw[MWE-full] (axis cs:1.36111E+02,-4.67901E+01) -- (axis cs:1.37100E+02,-4.75301E+01) -- (axis cs:1.36882E+02,-4.76654E+01) -- (axis cs:1.35893E+02,-4.69235E+01) -- cycle ; 
\draw[MWE-full] (axis cs:1.38118E+02,-4.82614E+01) -- (axis cs:1.39165E+02,-4.89834E+01) -- (axis cs:1.38945E+02,-4.91226E+01) -- (axis cs:1.37899E+02,-4.83986E+01) -- cycle ; 
\draw[MWE-full] (axis cs:1.40242E+02,-4.96956E+01) -- (axis cs:1.41351E+02,-5.03976E+01) -- (axis cs:1.41132E+02,-5.05409E+01) -- (axis cs:1.40023E+02,-4.98368E+01) -- cycle ; 
\draw[MWE-full] (axis cs:1.42493E+02,-5.10886E+01) -- (axis cs:1.43670E+02,-5.17682E+01) -- (axis cs:1.43452E+02,-5.19158E+01) -- (axis cs:1.42274E+02,-5.12341E+01) -- cycle ; 
\draw[MWE-full] (axis cs:1.44882E+02,-5.24356E+01) -- (axis cs:1.46132E+02,-5.30902E+01) -- (axis cs:1.45915E+02,-5.32424E+01) -- (axis cs:1.44665E+02,-5.25855E+01) -- cycle ; 
\draw[MWE-full] (axis cs:1.47419E+02,-5.37314E+01) -- (axis cs:1.48746E+02,-5.43582E+01) -- (axis cs:1.48533E+02,-5.45151E+01) -- (axis cs:1.47204E+02,-5.38858E+01) -- cycle ; 
\draw[MWE-full] (axis cs:1.50114E+02,-5.49701E+01) -- (axis cs:1.51523E+02,-5.55661E+01) -- (axis cs:1.51315E+02,-5.57277E+01) -- (axis cs:1.49903E+02,-5.51293E+01) -- cycle ; 
\draw[MWE-full] (axis cs:1.52976E+02,-5.61454E+01) -- (axis cs:1.54472E+02,-5.67072E+01) -- (axis cs:1.54269E+02,-5.68737E+01) -- (axis cs:1.52770E+02,-5.63095E+01) -- cycle ; 
\draw[MWE-full] (axis cs:1.56012E+02,-5.72505E+01) -- (axis cs:1.57598E+02,-5.77743E+01) -- (axis cs:1.57405E+02,-5.79457E+01) -- (axis cs:1.55814E+02,-5.74194E+01) -- cycle ; 
\draw[MWE-full] (axis cs:1.59230E+02,-5.82777E+01) -- (axis cs:1.60908E+02,-5.87597E+01) -- (axis cs:1.60725E+02,-5.89360E+01) -- (axis cs:1.59042E+02,-5.84516E+01) -- cycle ; 
\draw[MWE-full] (axis cs:1.62632E+02,-5.92193E+01) -- (axis cs:1.64402E+02,-5.96555E+01) -- (axis cs:1.64233E+02,-5.98365E+01) -- (axis cs:1.62456E+02,-5.93980E+01) -- cycle ; 
\draw[MWE-full] (axis cs:1.66217E+02,-6.00672E+01) -- (axis cs:1.68077E+02,-6.04533E+01) -- (axis cs:1.67924E+02,-6.06388E+01) -- (axis cs:1.66056E+02,-6.02505E+01) -- cycle ; 
\draw[MWE-full] (axis cs:1.69979E+02,-6.08130E+01) -- (axis cs:1.71924E+02,-6.11451E+01) -- (axis cs:1.71791E+02,-6.13346E+01) -- (axis cs:1.69837E+02,-6.10005E+01) -- cycle ; 
\draw[MWE-full] (axis cs:1.73908E+02,-6.14487E+01) -- (axis cs:1.75929E+02,-6.17229E+01) -- (axis cs:1.75819E+02,-6.19160E+01) -- (axis cs:1.73786E+02,-6.16401E+01) -- cycle ; 
\draw[MWE-full] (axis cs:1.77984E+02,-6.19668E+01) -- (axis cs:1.80071E+02,-6.21796E+01) -- (axis cs:1.79986E+02,-6.23756E+01) -- (axis cs:1.77886E+02,-6.21614E+01) -- cycle ; 
\draw[MWE-full] (axis cs:1.82184E+02,-6.23606E+01) -- (axis cs:1.84321E+02,-6.25091E+01) -- (axis cs:1.84264E+02,-6.27073E+01) -- (axis cs:1.82113E+02,-6.25578E+01) -- cycle ; 
\draw[MWE-full] (axis cs:1.86478E+02,-6.26247E+01) -- (axis cs:1.88648E+02,-6.27068E+01) -- (axis cs:1.88619E+02,-6.29064E+01) -- (axis cs:1.86434E+02,-6.28237E+01) -- cycle ; 
\draw[MWE-full] (axis cs:1.90828E+02,-6.27553E+01) -- (axis cs:1.93013E+02,-6.27699E+01) -- (axis cs:1.93014E+02,-6.29699E+01) -- (axis cs:1.90814E+02,-6.29552E+01) -- cycle ; 
\draw[MWE-full] (axis cs:1.95197E+02,-6.27506E+01) -- (axis cs:1.97377E+02,-6.26974E+01) -- (axis cs:1.97407E+02,-6.28969E+01) -- (axis cs:1.95213E+02,-6.29504E+01) -- cycle ; 
\draw[MWE-full] (axis cs:1.99545E+02,-6.26105E+01) -- (axis cs:2.01699E+02,-6.24903E+01) -- (axis cs:2.01759E+02,-6.26884E+01) -- (axis cs:1.99591E+02,-6.28094E+01) -- cycle ; 
\draw[MWE-full] (axis cs:2.03833E+02,-6.23372E+01) -- (axis cs:2.05943E+02,-6.21517E+01) -- (axis cs:2.06030E+02,-6.23475E+01) -- (axis cs:2.03907E+02,-6.25342E+01) -- cycle ; 
\draw[MWE-full] (axis cs:2.08026E+02,-6.19345E+01) -- (axis cs:2.10076E+02,-6.16863E+01) -- (axis cs:2.10188E+02,-6.18792E+01) -- (axis cs:2.08125E+02,-6.21289E+01) -- cycle ; 
\draw[MWE-full] (axis cs:2.12092E+02,-6.14079E+01) -- (axis cs:2.14071E+02,-6.11003E+01) -- (axis cs:2.14205E+02,-6.12895E+01) -- (axis cs:2.12216E+02,-6.15991E+01) -- cycle ; 
\draw[MWE-full] (axis cs:2.16010E+02,-6.07642E+01) -- (axis cs:2.17906E+02,-6.04008E+01) -- (axis cs:2.18060E+02,-6.05860E+01) -- (axis cs:2.16154E+02,-6.09515E+01) -- cycle ; 
\draw[MWE-full] (axis cs:2.19760E+02,-6.00110E+01) -- (axis cs:2.21569E+02,-5.95959E+01) -- (axis cs:2.21739E+02,-5.97766E+01) -- (axis cs:2.19922E+02,-6.01940E+01) -- cycle ; 
\draw[MWE-full] (axis cs:2.23332E+02,-5.91564E+01) -- (axis cs:2.25050E+02,-5.86936E+01) -- (axis cs:2.25233E+02,-5.88695E+01) -- (axis cs:2.23509E+02,-5.93347E+01) -- cycle ; 
\draw[MWE-full] (axis cs:2.26721E+02,-5.82085E+01) -- (axis cs:2.28346E+02,-5.77022E+01) -- (axis cs:2.28541E+02,-5.78732E+01) -- (axis cs:2.26910E+02,-5.83820E+01) -- cycle ; 
\draw[MWE-full] (axis cs:2.29926E+02,-5.71755E+01) -- (axis cs:2.31460E+02,-5.66296E+01) -- (axis cs:2.31663E+02,-5.67958E+01) -- (axis cs:2.30125E+02,-5.73442E+01) -- cycle ; 
\draw[MWE-full] (axis cs:2.32950E+02,-5.60654E+01) -- (axis cs:2.34397E+02,-5.54837E+01) -- (axis cs:2.34606E+02,-5.56449E+01) -- (axis cs:2.33157E+02,-5.62291E+01) -- cycle ; 
\draw[MWE-full] (axis cs:2.35800E+02,-5.48854E+01) -- (axis cs:2.37162E+02,-5.42714E+01) -- (axis cs:2.37376E+02,-5.44279E+01) -- (axis cs:2.36012E+02,-5.50442E+01) -- cycle ; 
\draw[MWE-full] (axis cs:2.38483E+02,-5.36425E+01) -- (axis cs:2.39766E+02,-5.29994E+01) -- (axis cs:2.39982E+02,-5.31512E+01) -- (axis cs:2.38699E+02,-5.37966E+01) -- cycle ; 
\draw[MWE-full] (axis cs:2.41010E+02,-5.23429E+01) -- (axis cs:2.42217E+02,-5.16738E+01) -- (axis cs:2.42435E+02,-5.18211E+01) -- (axis cs:2.41227E+02,-5.24925E+01) -- cycle ; 
\draw[MWE-full] (axis cs:2.43389E+02,-5.09926E+01) -- (axis cs:2.44526E+02,-5.02999E+01) -- (axis cs:2.44745E+02,-5.04429E+01) -- (axis cs:2.43608E+02,-5.11377E+01) -- cycle ; 
\draw[MWE-full] (axis cs:2.45631E+02,-4.95965E+01) -- (axis cs:2.46704E+02,-4.88829E+01) -- (axis cs:2.46923E+02,-4.90218E+01) -- (axis cs:2.45850E+02,-4.97374E+01) -- cycle ; 
\draw[MWE-full] (axis cs:2.47746E+02,-4.81596E+01) -- (axis cs:2.48760E+02,-4.74270E+01) -- (axis cs:2.48978E+02,-4.75620E+01) -- (axis cs:2.47965E+02,-4.82965E+01) -- cycle ; 
\draw[MWE-full] (axis cs:2.49745E+02,-4.66858E+01) -- (axis cs:2.50704E+02,-4.59363E+01) -- (axis cs:2.50921E+02,-4.60676E+01) -- (axis cs:2.49963E+02,-4.68189E+01) -- cycle ; 
\draw[MWE-full] (axis cs:2.51636E+02,-4.51791E+01) -- (axis cs:2.52544E+02,-4.44144E+01) -- (axis cs:2.52760E+02,-4.45422E+01) -- (axis cs:2.51853E+02,-4.53086E+01) -- cycle ; 
\draw[MWE-full] (axis cs:2.53429E+02,-4.36426E+01) -- (axis cs:2.54291E+02,-4.28642E+01) -- (axis cs:2.54505E+02,-4.29888E+01) -- (axis cs:2.53644E+02,-4.37688E+01) -- cycle ; 
\draw[MWE-full] (axis cs:2.55131E+02,-4.20795E+01) -- (axis cs:2.55951E+02,-4.12888E+01) -- (axis cs:2.56163E+02,-4.14103E+01) -- (axis cs:2.55344E+02,-4.22025E+01) -- cycle ; 
\draw[MWE-full] (axis cs:2.56751E+02,-4.04924E+01) -- (axis cs:2.57532E+02,-3.96906E+01) -- (axis cs:2.57742E+02,-3.98092E+01) -- (axis cs:2.56962E+02,-4.06124E+01) -- cycle ; 
\draw[MWE-full] (axis cs:2.58295E+02,-3.88836E+01) -- (axis cs:2.59041E+02,-3.80718E+01) -- (axis cs:2.59248E+02,-3.81877E+01) -- (axis cs:2.58504E+02,-3.90009E+01) -- cycle ; 
\draw[MWE-full] (axis cs:2.59771E+02,-3.72554E+01) -- (axis cs:2.60484E+02,-3.64345E+01) -- (axis cs:2.60689E+02,-3.65480E+01) -- (axis cs:2.59977E+02,-3.73700E+01) -- cycle ; 
\draw[MWE-full] (axis cs:2.61183E+02,-3.56095E+01) -- (axis cs:2.61867E+02,-3.47805E+01) -- (axis cs:2.62070E+02,-3.48916E+01) -- (axis cs:2.61387E+02,-3.57218E+01) -- cycle ; 
\draw[MWE-full] (axis cs:2.62538E+02,-3.39478E+01) -- (axis cs:2.63196E+02,-3.31114E+01) -- (axis cs:2.63396E+02,-3.32204E+01) -- (axis cs:2.62740E+02,-3.40578E+01) -- cycle ; 
\draw[MWE-full] (axis cs:2.63841E+02,-3.22717E+01) -- (axis cs:2.64474E+02,-3.14287E+01) -- (axis cs:2.64672E+02,-3.15356E+01) -- (axis cs:2.64040E+02,-3.23796E+01) -- cycle ; 
\draw[MWE-full] (axis cs:2.65096E+02,-3.05826E+01) -- (axis cs:2.65707E+02,-2.97337E+01) -- (axis cs:2.65903E+02,-2.98387E+01) -- (axis cs:2.65293E+02,-3.06886E+01) -- cycle ; 
\draw[MWE-full] (axis cs:2.66308E+02,-2.88819E+01) -- (axis cs:2.66899E+02,-2.80275E+01) -- (axis cs:2.67093E+02,-2.81309E+01) -- (axis cs:2.66503E+02,-2.89861E+01) -- cycle ; 
\draw[MWE-empty] (axis cs:2.66899E+02,-2.80275E+01) -- (axis cs:2.67481E+02,-2.71706E+01) -- (axis cs:2.67674E+02,-2.72732E+01) -- (axis cs:2.67093E+02,-2.81309E+01) -- cycle ; 
\draw[MWE-empty] (axis cs:2.68054E+02,-2.63113E+01) -- (axis cs:2.68618E+02,-2.54498E+01) -- (axis cs:2.68809E+02,-2.55508E+01) -- (axis cs:2.68246E+02,-2.64131E+01) -- cycle ; 
\draw[MWE-empty] (axis cs:2.69174E+02,-2.45861E+01) -- (axis cs:2.69723E+02,-2.37204E+01) -- (axis cs:2.69912E+02,-2.38201E+01) -- (axis cs:2.69365E+02,-2.46865E+01) -- cycle ; 
\draw[MWE-empty] (axis cs:2.70264E+02,-2.28528E+01) -- (axis cs:2.70799E+02,-2.19834E+01) -- (axis cs:2.70986E+02,-2.20817E+01) -- (axis cs:2.70453E+02,-2.29518E+01) -- cycle ; 
\draw[MWE-empty] (axis cs:2.71327E+02,-2.11122E+01) -- (axis cs:2.71848E+02,-2.02394E+01) -- (axis cs:2.72035E+02,-2.03366E+01) -- (axis cs:2.71514E+02,-2.12100E+01) -- cycle ; 
\draw[MWE-empty] (axis cs:2.72364E+02,-1.93650E+01) -- (axis cs:2.72874E+02,-1.84892E+01) -- (axis cs:2.73059E+02,-1.85854E+01) -- (axis cs:2.72550E+02,-1.94617E+01) -- cycle ; 
\draw[MWE-empty] (axis cs:2.73379E+02,-1.76121E+01) -- (axis cs:2.73879E+02,-1.67336E+01) -- (axis cs:2.74063E+02,-1.68289E+01) -- (axis cs:2.73564E+02,-1.77078E+01) -- cycle ; 
\draw[MWE-empty] (axis cs:2.74375E+02,-1.58540E+01) -- (axis cs:2.74866E+02,-1.49732E+01) -- (axis cs:2.75049E+02,-1.50676E+01) -- (axis cs:2.74558E+02,-1.59488E+01) -- cycle ; 
\draw[MWE-empty] (axis cs:2.75353E+02,-1.40913E+01) -- (axis cs:2.75837E+02,-1.32085E+01) -- (axis cs:2.76018E+02,-1.33022E+01) -- (axis cs:2.75535E+02,-1.41854E+01) -- cycle ; 
\draw[MWE-empty] (axis cs:2.76316E+02,-1.23248E+01) -- (axis cs:2.76793E+02,-1.14402E+01) -- (axis cs:2.76974E+02,-1.15333E+01) -- (axis cs:2.76498E+02,-1.24181E+01) -- cycle ; 
\draw[MWE-empty] (axis cs:2.77267E+02,-1.05548E+01) -- (axis cs:2.77738E+02,-9.66873E+00) -- (axis cs:2.77917E+02,-9.76126E+00) -- (axis cs:2.77447E+02,-1.06476E+01) -- cycle ; 
\draw[MWE-empty] (axis cs:2.78206E+02,-8.78200E+00) -- (axis cs:2.78672E+02,-7.89468E+00) -- (axis cs:2.78851E+02,-7.98676E+00) -- (axis cs:2.78385E+02,-8.87430E+00) -- cycle ; 
\draw[MWE-empty] (axis cs:2.79136E+02,-7.00682E+00) -- (axis cs:2.79598E+02,-6.11851E+00) -- (axis cs:2.79777E+02,-6.21024E+00) -- (axis cs:2.79315E+02,-7.09872E+00) -- cycle ; 
\draw[MWE-empty] (axis cs:2.80059E+02,-5.22978E+00) -- (axis cs:2.80518E+02,-4.34070E+00) -- (axis cs:2.80697E+02,-4.43217E+00) -- (axis cs:2.80237E+02,-5.32137E+00) -- cycle ; 
\draw[MWE-empty] (axis cs:2.80977E+02,-3.45134E+00) -- (axis cs:2.81434E+02,-2.56174E+00) -- (axis cs:2.81612E+02,-2.65303E+00) -- (axis cs:2.81155E+02,-3.54271E+00) -- cycle ; 
\draw[MWE-empty] (axis cs:2.81891E+02,-1.67196E+00) -- (axis cs:2.82347E+02,-7.82062E-01) -- (axis cs:2.82525E+02,-8.73274E-01) -- (axis cs:2.82069E+02,-1.76320E+00) -- cycle ; 
\draw[MWE-empty] (axis cs:2.82917E+02,1.07899E-01) -- (axis cs:2.83259E+02,9.97867E-01) -- (axis cs:2.83437E+02,9.06652E-01) -- (axis cs:2.82981E+02,1.66967E-02) -- cycle ; 
\draw[MWE-empty] (axis cs:2.83715E+02,1.88779E+00) -- (axis cs:2.84172E+02,2.77760E+00) -- (axis cs:2.84350E+02,2.68629E+00) -- (axis cs:2.83893E+02,1.79654E+00) -- cycle ; 
\draw[MWE-empty] (axis cs:2.84629E+02,3.66725E+00) -- (axis cs:2.85088E+02,4.55668E+00) -- (axis cs:2.85266E+02,4.46519E+00) -- (axis cs:2.84808E+02,3.57586E+00) -- cycle ; 
\draw[MWE-empty] (axis cs:2.85547E+02,5.44583E+00) -- (axis cs:2.86008E+02,6.33465E+00) -- (axis cs:2.86186E+02,6.24290E+00) -- (axis cs:2.85725E+02,5.35422E+00) -- cycle ; 
\draw[MWE-empty] (axis cs:2.86470E+02,7.22308E+00) -- (axis cs:2.86934E+02,8.11105E+00) -- (axis cs:2.87113E+02,8.01893E+00) -- (axis cs:2.86649E+02,7.13115E+00) -- cycle ; 
\draw[MWE-empty] (axis cs:2.87400E+02,8.99851E+00) -- (axis cs:2.87868E+02,9.88539E+00) -- (axis cs:2.88048E+02,9.79283E+00) -- (axis cs:2.87580E+02,8.90618E+00) -- cycle ; 
\draw[MWE-empty] (axis cs:2.88339E+02,1.07716E+01) -- (axis cs:2.88813E+02,1.16572E+01) -- (axis cs:2.88993E+02,1.15641E+01) -- (axis cs:2.88519E+02,1.06788E+01) -- cycle ; 
\draw[MWE-empty] (axis cs:2.89289E+02,1.25420E+01) -- (axis cs:2.89769E+02,1.34259E+01) -- (axis cs:2.89951E+02,1.33322E+01) -- (axis cs:2.89470E+02,1.24486E+01) -- cycle ; 
\draw[MWE-empty] (axis cs:2.90253E+02,1.43090E+01) -- (axis cs:2.90740E+02,1.51910E+01) -- (axis cs:2.90922E+02,1.50966E+01) -- (axis cs:2.90435E+02,1.42149E+01) -- cycle ; 
\draw[MWE-empty] (axis cs:2.91231E+02,1.60721E+01) -- (axis cs:2.91727E+02,1.69520E+01) -- (axis cs:2.91910E+02,1.68567E+01) -- (axis cs:2.91414E+02,1.59772E+01) -- cycle ; 
\draw[MWE-empty] (axis cs:2.92227E+02,1.78307E+01) -- (axis cs:2.92732E+02,1.87082E+01) -- (axis cs:2.92917E+02,1.86119E+01) -- (axis cs:2.92411E+02,1.77350E+01) -- cycle ; 
\draw[MWE-empty] (axis cs:2.93242E+02,1.95843E+01) -- (axis cs:2.93758E+02,2.04590E+01) -- (axis cs:2.93944E+02,2.03617E+01) -- (axis cs:2.93428E+02,1.94875E+01) -- cycle ; 
\draw[MWE-empty] (axis cs:2.94280E+02,2.13321E+01) -- (axis cs:2.94808E+02,2.22037E+01) -- (axis cs:2.94996E+02,2.21052E+01) -- (axis cs:2.94467E+02,2.12342E+01) -- cycle ; 
\draw[MWE-empty] (axis cs:2.95342E+02,2.30735E+01) -- (axis cs:2.95884E+02,2.39415E+01) -- (axis cs:2.96073E+02,2.38417E+01) -- (axis cs:2.95531E+02,2.29744E+01) -- cycle ; 
\draw[MWE-empty] (axis cs:2.96433E+02,2.48076E+01) -- (axis cs:2.96989E+02,2.56717E+01) -- (axis cs:2.97180E+02,2.55705E+01) -- (axis cs:2.96623E+02,2.47072E+01) -- cycle ; 
\draw[MWE-empty] (axis cs:2.97554E+02,2.65337E+01) -- (axis cs:2.98127E+02,2.73934E+01) -- (axis cs:2.98320E+02,2.72907E+01) -- (axis cs:2.97746E+02,2.64318E+01) -- cycle ; 
\draw[MWE-empty] (axis cs:2.98709E+02,2.82508E+01) -- (axis cs:2.99300E+02,2.91056E+01) -- (axis cs:2.99495E+02,2.90013E+01) -- (axis cs:2.98903E+02,2.81473E+01) -- cycle ; 
\draw[MWE-empty] (axis cs:2.99901E+02,2.99579E+01) -- (axis cs:3.00513E+02,3.08074E+01) -- (axis cs:3.00710E+02,3.07013E+01) -- (axis cs:3.00098E+02,2.98527E+01) -- cycle ; 
\draw[MWE-empty] (axis cs:3.01135E+02,3.16540E+01) -- (axis cs:3.01769E+02,3.24975E+01) -- (axis cs:3.01969E+02,3.23894E+01) -- (axis cs:3.01334E+02,3.15469E+01) -- cycle ; 
\draw[MWE-empty] (axis cs:3.02415E+02,3.33378E+01) -- (axis cs:3.03073E+02,3.41747E+01) -- (axis cs:3.03275E+02,3.40646E+01) -- (axis cs:3.02616E+02,3.32287E+01) -- cycle ; 
\draw[MWE-empty] (axis cs:3.03745E+02,3.50081E+01) -- (axis cs:3.04430E+02,3.58377E+01) -- (axis cs:3.04634E+02,3.57252E+01) -- (axis cs:3.03948E+02,3.48968E+01) -- cycle ; 
\draw[MWE-empty] (axis cs:3.05130E+02,3.66633E+01) -- (axis cs:3.05844E+02,3.74848E+01) -- (axis cs:3.06051E+02,3.73699E+01) -- (axis cs:3.05335E+02,3.65497E+01) -- cycle ; 
\draw[MWE-empty] (axis cs:3.06575E+02,3.83019E+01) -- (axis cs:3.07322E+02,3.91143E+01) -- (axis cs:3.07531E+02,3.89969E+01) -- (axis cs:3.06782E+02,3.81857E+01) -- cycle ; 
\draw[MWE-empty] (axis cs:3.08087E+02,3.99220E+01) -- (axis cs:3.08869E+02,4.07244E+01) -- (axis cs:3.09080E+02,4.06042E+01) -- (axis cs:3.08296E+02,3.98031E+01) -- cycle ; 
\draw[MWE-empty] (axis cs:3.09671E+02,4.15215E+01) -- (axis cs:3.10492E+02,4.23130E+01) -- (axis cs:3.10705E+02,4.21897E+01) -- (axis cs:3.09883E+02,4.13998E+01) -- cycle ; 
\draw[MWE-empty] (axis cs:3.11335E+02,4.30984E+01) -- (axis cs:3.12199E+02,4.38775E+01) -- (axis cs:3.12414E+02,4.37511E+01) -- (axis cs:3.11548E+02,4.29736E+01) -- cycle ; 
\draw[MWE-empty] (axis cs:3.13086E+02,4.46499E+01) -- (axis cs:3.13996E+02,4.54153E+01) -- (axis cs:3.14213E+02,4.52855E+01) -- (axis cs:3.13301E+02,4.45218E+01) -- cycle ; 
\draw[MWE-empty] (axis cs:3.14932E+02,4.61733E+01) -- (axis cs:3.15893E+02,4.69235E+01) -- (axis cs:3.16111E+02,4.67901E+01) -- (axis cs:3.15149E+02,4.60418E+01) -- cycle ; 
\draw[MWE-empty] (axis cs:3.16882E+02,4.76654E+01) -- (axis cs:3.17899E+02,4.83986E+01) -- (axis cs:3.18118E+02,4.82614E+01) -- (axis cs:3.17100E+02,4.75301E+01) -- cycle ; 
\draw[MWE-empty] (axis cs:3.18945E+02,4.91226E+01) -- (axis cs:3.20023E+02,4.98368E+01) -- (axis cs:3.20242E+02,4.96956E+01) -- (axis cs:3.19165E+02,4.89834E+01) -- cycle ; 
\draw[MWE-empty] (axis cs:3.21132E+02,5.05409E+01) -- (axis cs:3.22274E+02,5.12341E+01) -- (axis cs:3.22493E+02,5.10886E+01) -- (axis cs:3.21351E+02,5.03976E+01) -- cycle ; 
\draw[MWE-empty] (axis cs:3.23452E+02,5.19158E+01) -- (axis cs:3.24665E+02,5.25855E+01) -- (axis cs:3.24882E+02,5.24356E+01) -- (axis cs:3.23670E+02,5.17682E+01) -- cycle ; 
\draw[MWE-empty] (axis cs:3.25915E+02,5.32424E+01) -- (axis cs:3.27204E+02,5.38858E+01) -- (axis cs:3.27419E+02,5.37314E+01) -- (axis cs:3.26132E+02,5.30902E+01) -- cycle ; 
\draw[MWE-empty] (axis cs:3.28533E+02,5.45151E+01) -- (axis cs:3.29903E+02,5.51293E+01) -- (axis cs:3.30114E+02,5.49701E+01) -- (axis cs:3.28746E+02,5.43582E+01) -- cycle ; 
\draw[MWE-empty] (axis cs:3.31315E+02,5.57277E+01) -- (axis cs:3.32770E+02,5.63095E+01) -- (axis cs:3.32976E+02,5.61454E+01) -- (axis cs:3.31523E+02,5.55661E+01) -- cycle ; 
\draw[MWE-empty] (axis cs:3.34269E+02,5.68737E+01) -- (axis cs:3.35814E+02,5.74194E+01) -- (axis cs:3.36012E+02,5.72505E+01) -- (axis cs:3.34472E+02,5.67072E+01) -- cycle ; 
\draw[MWE-empty] (axis cs:3.37405E+02,5.79457E+01) -- (axis cs:3.39042E+02,5.84516E+01) -- (axis cs:3.39230E+02,5.82777E+01) -- (axis cs:3.37598E+02,5.77743E+01) -- cycle ; 
\draw[MWE-empty] (axis cs:3.40725E+02,5.89360E+01) -- (axis cs:3.42456E+02,5.93980E+01) -- (axis cs:3.42632E+02,5.92193E+01) -- (axis cs:3.40908E+02,5.87597E+01) -- cycle ; 
\draw[MWE-empty] (axis cs:3.44233E+02,5.98365E+01) -- (axis cs:3.46056E+02,6.02505E+01) -- (axis cs:3.46217E+02,6.00672E+01) -- (axis cs:3.44402E+02,5.96555E+01) -- cycle ; 
\draw[MWE-empty] (axis cs:3.47924E+02,6.06388E+01) -- (axis cs:3.49837E+02,6.10005E+01) -- (axis cs:3.49979E+02,6.08130E+01) -- (axis cs:3.48077E+02,6.04533E+01) -- cycle ; 
\draw[MWE-empty] (axis cs:3.51791E+02,6.13346E+01) -- (axis cs:3.53786E+02,6.16401E+01) -- (axis cs:3.53908E+02,6.14487E+01) -- (axis cs:3.51924E+02,6.11451E+01) -- cycle ; 
\draw[MWE-empty] (axis cs:3.55819E+02,6.19160E+01) -- (axis cs:3.57886E+02,6.21614E+01) -- (axis cs:3.57984E+02,6.19668E+01) -- (axis cs:3.55929E+02,6.17229E+01) -- cycle ; 
\draw[MWE-empty] (axis cs:3.59986E+02,6.23756E+01) -- (axis cs:2.11273E+00,6.25578E+01) -- (axis cs:2.18430E+00,6.23606E+01) -- (axis cs:7.05566E-02,6.21796E+01) -- cycle ; 
\draw[MWE-empty] (axis cs:4.26373E+00,6.27073E+01) -- (axis cs:6.43417E+00,6.28237E+01) -- (axis cs:6.47748E+00,6.26247E+01) -- (axis cs:4.32138E+00,6.25091E+01) -- cycle ; 
\draw[MWE-empty] (axis cs:8.61932E+00,6.29064E+01) -- (axis cs:1.08143E+01,6.29552E+01) -- (axis cs:1.08282E+01,6.27553E+01) -- (axis cs:8.64801E+00,6.27068E+01) -- cycle ; 
\draw[MWE-empty] (axis cs:1.30140E+01,6.29699E+01) -- (axis cs:1.52134E+01,6.29504E+01) -- (axis cs:1.51974E+01,6.27506E+01) -- (axis cs:1.30130E+01,6.27699E+01) -- cycle ; 
\draw[MWE-empty] (axis cs:1.74073E+01,6.28969E+01) -- (axis cs:1.95907E+01,6.28094E+01) -- (axis cs:1.95453E+01,6.26105E+01) -- (axis cs:1.73765E+01,6.26974E+01) -- cycle ; 
\draw[MWE-empty] (axis cs:2.17586E+01,6.26884E+01) -- (axis cs:2.39066E+01,6.25342E+01) -- (axis cs:2.38331E+01,6.23372E+01) -- (axis cs:2.16990E+01,6.24903E+01) -- cycle ; 
\draw[MWE-empty] (axis cs:2.60301E+01,6.23475E+01) -- (axis cs:2.81251E+01,6.21289E+01) -- (axis cs:2.80255E+01,6.19345E+01) -- (axis cs:2.59432E+01,6.21517E+01) -- cycle ; 
\draw[MWE-empty] (axis cs:3.01880E+01,6.18792E+01) -- (axis cs:3.22156E+01,6.15991E+01) -- (axis cs:3.20923E+01,6.14079E+01) -- (axis cs:3.00762E+01,6.16863E+01) -- cycle ; 
\draw[MWE-empty] (axis cs:3.42049E+01,6.12895E+01) -- (axis cs:3.61537E+01,6.09515E+01) -- (axis cs:3.60096E+01,6.07642E+01) -- (axis cs:3.40709E+01,6.11003E+01) -- cycle ; 
\draw[MWE-empty] (axis cs:3.80599E+01,6.05860E+01) -- (axis cs:3.99219E+01,6.01940E+01) -- (axis cs:3.97599E+01,6.00110E+01) -- (axis cs:3.79064E+01,6.04008E+01) -- cycle ; 
\draw[MWE-empty] (axis cs:4.17386E+01,5.97766E+01) -- (axis cs:4.35091E+01,5.93347E+01) -- (axis cs:4.33322E+01,5.91564E+01) -- (axis cs:4.15687E+01,5.95959E+01) -- cycle ; 
\draw[MWE-empty] (axis cs:4.52331E+01,5.88695E+01) -- (axis cs:4.69102E+01,5.83820E+01) -- (axis cs:4.67210E+01,5.82085E+01) -- (axis cs:4.50497E+01,5.86936E+01) -- cycle ; 
\draw[MWE-empty] (axis cs:4.85407E+01,5.78732E+01) -- (axis cs:5.01248E+01,5.73442E+01) -- (axis cs:4.99260E+01,5.71755E+01) -- (axis cs:4.83464E+01,5.77022E+01) -- cycle ; 
\draw[MWE-empty] (axis cs:5.16632E+01,5.67958E+01) -- (axis cs:5.31565E+01,5.62291E+01) -- (axis cs:5.29503E+01,5.60654E+01) -- (axis cs:5.14604E+01,5.66296E+01) -- cycle ; 
\draw[MWE-empty] (axis cs:5.46057E+01,5.56449E+01) -- (axis cs:5.60117E+01,5.50442E+01) -- (axis cs:5.58001E+01,5.48854E+01) -- (axis cs:5.43965E+01,5.54837E+01) -- cycle ; 
\draw[MWE-empty] (axis cs:5.73757E+01,5.44279E+01) -- (axis cs:5.86988E+01,5.37966E+01) -- (axis cs:5.84834E+01,5.36425E+01) -- (axis cs:5.71620E+01,5.42714E+01) -- cycle ; 
\draw[MWE-empty] (axis cs:5.99822E+01,5.31512E+01) -- (axis cs:6.12273E+01,5.24925E+01) -- (axis cs:6.10096E+01,5.23429E+01) -- (axis cs:5.97655E+01,5.29994E+01) -- cycle ; 
\draw[MWE-empty] (axis cs:6.24353E+01,5.18211E+01) -- (axis cs:6.36076E+01,5.11377E+01) -- (axis cs:6.33886E+01,5.09926E+01) -- (axis cs:6.22168E+01,5.16738E+01) -- cycle ; 
\draw[MWE-empty] (axis cs:6.47454E+01,5.04429E+01) -- (axis cs:6.58501E+01,4.97374E+01) -- (axis cs:6.56308E+01,4.95965E+01) -- (axis cs:6.45262E+01,5.02999E+01) -- cycle ; 
\draw[MWE-empty] (axis cs:6.69230E+01,4.90218E+01) -- (axis cs:6.79652E+01,4.82965E+01) -- (axis cs:6.77464E+01,4.81596E+01) -- (axis cs:6.67038E+01,4.88829E+01) -- cycle ; 
\draw[MWE-empty] (axis cs:6.89781E+01,4.75620E+01) -- (axis cs:6.99629E+01,4.68189E+01) -- (axis cs:6.97450E+01,4.66858E+01) -- (axis cs:6.87597E+01,4.74270E+01) -- cycle ; 
\draw[MWE-empty] (axis cs:7.09207E+01,4.60676E+01) -- (axis cs:7.18527E+01,4.53086E+01) -- (axis cs:7.16362E+01,4.51791E+01) -- (axis cs:7.07035E+01,4.59363E+01) -- cycle ; 
\draw[MWE-empty] (axis cs:7.27599E+01,4.45422E+01) -- (axis cs:7.36436E+01,4.37688E+01) -- (axis cs:7.34289E+01,4.36426E+01) -- (axis cs:7.25443E+01,4.44144E+01) -- cycle ; 
\draw[MWE-empty] (axis cs:7.45045E+01,4.29888E+01) -- (axis cs:7.53439E+01,4.22025E+01) -- (axis cs:7.51312E+01,4.20795E+01) -- (axis cs:7.42908E+01,4.28642E+01) -- cycle ; 
\draw[MWE-empty] (axis cs:7.61626E+01,4.14103E+01) -- (axis cs:7.69615E+01,4.06124E+01) -- (axis cs:7.67510E+01,4.04924E+01) -- (axis cs:7.59510E+01,4.12888E+01) -- cycle ; 
\draw[MWE-empty] (axis cs:7.77415E+01,3.98092E+01) -- (axis cs:7.85035E+01,3.90009E+01) -- (axis cs:7.82952E+01,3.88836E+01) -- (axis cs:7.75321E+01,3.96906E+01) -- cycle ; 
\draw[MWE-empty] (axis cs:7.92483E+01,3.81877E+01) -- (axis cs:7.99766E+01,3.73700E+01) -- (axis cs:7.97705E+01,3.72554E+01) -- (axis cs:7.90411E+01,3.80718E+01) -- cycle ; 
\draw[MWE-empty] (axis cs:8.06892E+01,3.65480E+01) -- (axis cs:8.13867E+01,3.57218E+01) -- (axis cs:8.11830E+01,3.56095E+01) -- (axis cs:8.04843E+01,3.64345E+01) -- cycle ; 
\draw[MWE-empty] (axis cs:8.20700E+01,3.48916E+01) -- (axis cs:8.27396E+01,3.40578E+01) -- (axis cs:8.25381E+01,3.39478E+01) -- (axis cs:8.18674E+01,3.47805E+01) -- cycle ; 
\draw[MWE-empty] (axis cs:8.33961E+01,3.32204E+01) -- (axis cs:8.40402E+01,3.23796E+01) -- (axis cs:8.38409E+01,3.22717E+01) -- (axis cs:8.31957E+01,3.31114E+01) -- cycle ; 
\draw[MWE-empty] (axis cs:8.46724E+01,3.15356E+01) -- (axis cs:8.52933E+01,3.06886E+01) -- (axis cs:8.50962E+01,3.05826E+01) -- (axis cs:8.44742E+01,3.14287E+01) -- cycle ; 
\draw[MWE-empty] (axis cs:8.59034E+01,2.98387E+01) -- (axis cs:8.65033E+01,2.89861E+01) -- (axis cs:8.63082E+01,2.88819E+01) -- (axis cs:8.57074E+01,2.97337E+01) -- cycle ; 
\draw[MWE-empty] (axis cs:8.70933E+01,2.81309E+01) -- (axis cs:8.76740E+01,2.72732E+01) -- (axis cs:8.74810E+01,2.71706E+01) -- (axis cs:8.68993E+01,2.80275E+01) -- cycle ; 
\draw[MWE-empty] (axis cs:8.82459E+01,2.64131E+01) -- (axis cs:8.88093E+01,2.55508E+01) -- (axis cs:8.86180E+01,2.54498E+01) -- (axis cs:8.80537E+01,2.63113E+01) -- cycle ; 
\draw[MWE-empty] (axis cs:8.93646E+01,2.46865E+01) -- (axis cs:8.99124E+01,2.38201E+01) -- (axis cs:8.97229E+01,2.37204E+01) -- (axis cs:8.91743E+01,2.45861E+01) -- cycle ; 
\draw[MWE-empty] (axis cs:9.04529E+01,2.29518E+01) -- (axis cs:9.09865E+01,2.20817E+01) -- (axis cs:9.07986E+01,2.19834E+01) -- (axis cs:9.02642E+01,2.28528E+01) -- cycle ; 
\draw[MWE-empty] (axis cs:9.15135E+01,2.12100E+01) -- (axis cs:9.20345E+01,2.03366E+01) -- (axis cs:9.18481E+01,2.02394E+01) -- (axis cs:9.13265E+01,2.11122E+01) -- cycle ; 
\draw[MWE-empty] (axis cs:9.25496E+01,1.94617E+01) -- (axis cs:9.30592E+01,1.85854E+01) -- (axis cs:9.28742E+01,1.84892E+01) -- (axis cs:9.23640E+01,1.93650E+01) -- cycle ; 
\draw[MWE-empty] (axis cs:9.35636E+01,1.77078E+01) -- (axis cs:9.40631E+01,1.68289E+01) -- (axis cs:9.38795E+01,1.67336E+01) -- (axis cs:9.33793E+01,1.76121E+01) -- cycle ; 
\draw[MWE-empty] (axis cs:9.45580E+01,1.59488E+01) -- (axis cs:9.50487E+01,1.50676E+01) -- (axis cs:9.48661E+01,1.49732E+01) -- (axis cs:9.43750E+01,1.58540E+01) -- cycle ; 
\draw[MWE-empty] (axis cs:9.55353E+01,1.41854E+01) -- (axis cs:9.60182E+01,1.33022E+01) -- (axis cs:9.58366E+01,1.32085E+01) -- (axis cs:9.53533E+01,1.40913E+01) -- cycle ; 
\draw[MWE-empty] (axis cs:9.64975E+01,1.24181E+01) -- (axis cs:9.69737E+01,1.15333E+01) -- (axis cs:9.67931E+01,1.14402E+01) -- (axis cs:9.63165E+01,1.23248E+01) -- cycle ; 
\draw[MWE-empty] (axis cs:9.74469E+01,1.06476E+01) -- (axis cs:9.79174E+01,9.76126E+00) -- (axis cs:9.77375E+01,9.66874E+00) -- (axis cs:9.72666E+01,1.05548E+01) -- cycle ; 
\draw[MWE-empty] (axis cs:9.83854E+01,8.87430E+00) -- (axis cs:9.88512E+01,7.98676E+00) -- (axis cs:9.86719E+01,7.89468E+00) -- (axis cs:9.82058E+01,8.78200E+00) -- cycle ; 
\draw[MWE-empty] (axis cs:9.93149E+01,7.09872E+00) -- (axis cs:9.97769E+01,6.21024E+00) -- (axis cs:9.95982E+01,6.11851E+00) -- (axis cs:9.91360E+01,7.00683E+00) -- cycle ; 
\draw[MWE-empty] (axis cs:1.00237E+02,5.32137E+00) -- (axis cs:1.00697E+02,4.43217E+00) -- (axis cs:1.00518E+02,4.34070E+00) -- (axis cs:1.00059E+02,5.22978E+00) -- cycle ; 
\draw[MWE-empty] (axis cs:1.01155E+02,3.54271E+00) -- (axis cs:1.01612E+02,2.65303E+00) -- (axis cs:1.01434E+02,2.56174E+00) -- (axis cs:1.00977E+02,3.45134E+00) -- cycle ; 
\draw[MWE-empty] (axis cs:1.02069E+02,1.76320E+00) -- (axis cs:1.02525E+02,8.73274E-01) -- (axis cs:1.02347E+02,7.82062E-01) -- (axis cs:1.01891E+02,1.67196E+00) -- cycle ; 
\draw[MWE-empty] (axis cs:1.02981E+02,-1.66966E-02) -- (axis cs:1.03437E+02,-9.06652E-01) -- (axis cs:1.03259E+02,-9.97867E-01) -- (axis cs:1.02917E+02,-1.07899E-01) -- cycle ; 
\draw[MWE-empty] (axis cs:1.03893E+02,-1.79654E+00) -- (axis cs:1.04350E+02,-2.68629E+00) -- (axis cs:1.04172E+02,-2.77760E+00) -- (axis cs:1.03715E+02,-1.88779E+00) -- cycle ; 
\draw[MWE-empty] (axis cs:1.04808E+02,-3.57586E+00) -- (axis cs:1.05266E+02,-4.46519E+00) -- (axis cs:1.05088E+02,-4.55668E+00) -- (axis cs:1.04629E+02,-3.66725E+00) -- cycle ; 
\draw[MWE-empty] (axis cs:1.05726E+02,-5.35422E+00) -- (axis cs:1.06186E+02,-6.24290E+00) -- (axis cs:1.06008E+02,-6.33465E+00) -- (axis cs:1.05547E+02,-5.44583E+00) -- cycle ; 
\draw[MWE-empty] (axis cs:1.06649E+02,-7.13115E+00) -- (axis cs:1.07113E+02,-8.01893E+00) -- (axis cs:1.06934E+02,-8.11105E+00) -- (axis cs:1.06470E+02,-7.22308E+00) -- cycle ; 
\draw[MWE-empty] (axis cs:1.07580E+02,-8.90618E+00) -- (axis cs:1.08048E+02,-9.79283E+00) -- (axis cs:1.07868E+02,-9.88539E+00) -- (axis cs:1.07400E+02,-8.99851E+00) -- cycle ; 
\draw[MWE-empty] (axis cs:1.08520E+02,-1.06788E+01) -- (axis cs:1.08993E+02,-1.15641E+01) -- (axis cs:1.08813E+02,-1.16572E+01) -- (axis cs:1.08339E+02,-1.07716E+01) -- cycle ; 
\draw[MWE-empty] (axis cs:1.09471E+02,-1.24486E+01) -- (axis cs:1.09951E+02,-1.33322E+01) -- (axis cs:1.09769E+02,-1.34259E+01) -- (axis cs:1.09289E+02,-1.25420E+01) -- cycle ; 
\draw[MWE-empty] (axis cs:1.10435E+02,-1.42149E+01) -- (axis cs:1.10922E+02,-1.50966E+01) -- (axis cs:1.10740E+02,-1.51910E+01) -- (axis cs:1.10253E+02,-1.43090E+01) -- cycle ; 
\draw[MWE-empty] (axis cs:1.11414E+02,-1.59772E+01) -- (axis cs:1.11910E+02,-1.68567E+01) -- (axis cs:1.11727E+02,-1.69520E+01) -- (axis cs:1.11231E+02,-1.60721E+01) -- cycle ; 
\draw[MWE-empty] (axis cs:1.12411E+02,-1.77350E+01) -- (axis cs:1.12917E+02,-1.86119E+01) -- (axis cs:1.12732E+02,-1.87082E+01) -- (axis cs:1.12227E+02,-1.78307E+01) -- cycle ; 
\draw[MWE-empty] (axis cs:1.13428E+02,-1.94875E+01) -- (axis cs:1.13945E+02,-2.03617E+01) -- (axis cs:1.13758E+02,-2.04590E+01) -- (axis cs:1.13242E+02,-1.95843E+01) -- cycle ; 
\draw[MWE-empty] (axis cs:1.14467E+02,-2.12342E+01) -- (axis cs:1.14996E+02,-2.21052E+01) -- (axis cs:1.14808E+02,-2.22037E+01) -- (axis cs:1.14280E+02,-2.13321E+01) -- cycle ; 
\draw[MWE-empty] (axis cs:1.15531E+02,-2.29744E+01) -- (axis cs:1.16073E+02,-2.38417E+01) -- (axis cs:1.15884E+02,-2.39415E+01) -- (axis cs:1.15342E+02,-2.30735E+01) -- cycle ; 
\draw[MWE-empty] (axis cs:1.16623E+02,-2.47072E+01) -- (axis cs:1.17180E+02,-2.55705E+01) -- (axis cs:1.16989E+02,-2.56717E+01) -- (axis cs:1.16433E+02,-2.48076E+01) -- cycle ; 
\draw[MWE-empty] (axis cs:1.17746E+02,-2.64318E+01) -- (axis cs:1.18320E+02,-2.72907E+01) -- (axis cs:1.18127E+02,-2.73934E+01) -- (axis cs:1.17554E+02,-2.65337E+01) -- cycle ; 
\draw[MWE-empty] (axis cs:1.18903E+02,-2.81473E+01) -- (axis cs:1.19495E+02,-2.90013E+01) -- (axis cs:1.19300E+02,-2.91056E+01) -- (axis cs:1.18709E+02,-2.82508E+01) -- cycle ; 
\draw[MWE-empty] (axis cs:1.20098E+02,-2.98527E+01) -- (axis cs:1.20710E+02,-3.07013E+01) -- (axis cs:1.20513E+02,-3.08074E+01) -- (axis cs:1.19901E+02,-2.99579E+01) -- cycle ; 
\draw[MWE-empty] (axis cs:1.21334E+02,-3.15469E+01) -- (axis cs:1.21969E+02,-3.23894E+01) -- (axis cs:1.21769E+02,-3.24975E+01) -- (axis cs:1.21135E+02,-3.16540E+01) -- cycle ; 
\draw[MWE-empty] (axis cs:1.22616E+02,-3.32287E+01) -- (axis cs:1.23275E+02,-3.40646E+01) -- (axis cs:1.23073E+02,-3.41747E+01) -- (axis cs:1.22415E+02,-3.33378E+01) -- cycle ; 
\draw[MWE-empty] (axis cs:1.23948E+02,-3.48968E+01) -- (axis cs:1.24634E+02,-3.57252E+01) -- (axis cs:1.24430E+02,-3.58377E+01) -- (axis cs:1.23745E+02,-3.50081E+01) -- cycle ; 
\draw[MWE-empty] (axis cs:1.25335E+02,-3.65497E+01) -- (axis cs:1.26051E+02,-3.73699E+01) -- (axis cs:1.25844E+02,-3.74848E+01) -- (axis cs:1.25130E+02,-3.66633E+01) -- cycle ; 
\draw[MWE-empty] (axis cs:1.26782E+02,-3.81857E+01) -- (axis cs:1.27531E+02,-3.89969E+01) -- (axis cs:1.27322E+02,-3.91143E+01) -- (axis cs:1.26575E+02,-3.83019E+01) -- cycle ; 
\draw[MWE-empty] (axis cs:1.28296E+02,-3.98031E+01) -- (axis cs:1.29080E+02,-4.06042E+01) -- (axis cs:1.28869E+02,-4.07244E+01) -- (axis cs:1.28087E+02,-3.99220E+01) -- cycle ; 
\draw[MWE-empty] (axis cs:1.29883E+02,-4.13998E+01) -- (axis cs:1.30705E+02,-4.21897E+01) -- (axis cs:1.30492E+02,-4.23130E+01) -- (axis cs:1.29671E+02,-4.15215E+01) -- cycle ; 
\draw[MWE-empty] (axis cs:1.31549E+02,-4.29736E+01) -- (axis cs:1.32414E+02,-4.37511E+01) -- (axis cs:1.32199E+02,-4.38775E+01) -- (axis cs:1.31335E+02,-4.30984E+01) -- cycle ; 
\draw[MWE-empty] (axis cs:1.33301E+02,-4.45218E+01) -- (axis cs:1.34213E+02,-4.52855E+01) -- (axis cs:1.33996E+02,-4.54153E+01) -- (axis cs:1.33086E+02,-4.46499E+01) -- cycle ; 
\draw[MWE-empty] (axis cs:1.35149E+02,-4.60418E+01) -- (axis cs:1.36111E+02,-4.67901E+01) -- (axis cs:1.35893E+02,-4.69235E+01) -- (axis cs:1.34932E+02,-4.61733E+01) -- cycle ; 
\draw[MWE-empty] (axis cs:1.37100E+02,-4.75301E+01) -- (axis cs:1.38118E+02,-4.82614E+01) -- (axis cs:1.37899E+02,-4.83986E+01) -- (axis cs:1.36882E+02,-4.76654E+01) -- cycle ; 
\draw[MWE-empty] (axis cs:1.39165E+02,-4.89834E+01) -- (axis cs:1.40242E+02,-4.96956E+01) -- (axis cs:1.40023E+02,-4.98368E+01) -- (axis cs:1.38945E+02,-4.91226E+01) -- cycle ; 
\draw[MWE-empty] (axis cs:1.41351E+02,-5.03976E+01) -- (axis cs:1.42493E+02,-5.10886E+01) -- (axis cs:1.42274E+02,-5.12341E+01) -- (axis cs:1.41132E+02,-5.05409E+01) -- cycle ; 
\draw[MWE-empty] (axis cs:1.43670E+02,-5.17682E+01) -- (axis cs:1.44882E+02,-5.24356E+01) -- (axis cs:1.44665E+02,-5.25855E+01) -- (axis cs:1.43452E+02,-5.19158E+01) -- cycle ; 
\draw[MWE-empty] (axis cs:1.46132E+02,-5.30902E+01) -- (axis cs:1.47419E+02,-5.37314E+01) -- (axis cs:1.47204E+02,-5.38858E+01) -- (axis cs:1.45915E+02,-5.32424E+01) -- cycle ; 
\draw[MWE-empty] (axis cs:1.48746E+02,-5.43582E+01) -- (axis cs:1.50114E+02,-5.49701E+01) -- (axis cs:1.49903E+02,-5.51293E+01) -- (axis cs:1.48533E+02,-5.45151E+01) -- cycle ; 
\draw[MWE-empty] (axis cs:1.51523E+02,-5.55661E+01) -- (axis cs:1.52976E+02,-5.61454E+01) -- (axis cs:1.52770E+02,-5.63095E+01) -- (axis cs:1.51315E+02,-5.57277E+01) -- cycle ; 
\draw[MWE-empty] (axis cs:1.54472E+02,-5.67072E+01) -- (axis cs:1.56012E+02,-5.72505E+01) -- (axis cs:1.55814E+02,-5.74194E+01) -- (axis cs:1.54269E+02,-5.68737E+01) -- cycle ; 
\draw[MWE-empty] (axis cs:1.57598E+02,-5.77743E+01) -- (axis cs:1.59230E+02,-5.82777E+01) -- (axis cs:1.59042E+02,-5.84516E+01) -- (axis cs:1.57405E+02,-5.79457E+01) -- cycle ; 
\draw[MWE-empty] (axis cs:1.60908E+02,-5.87597E+01) -- (axis cs:1.62632E+02,-5.92193E+01) -- (axis cs:1.62456E+02,-5.93980E+01) -- (axis cs:1.60725E+02,-5.89360E+01) -- cycle ; 
\draw[MWE-empty] (axis cs:1.64402E+02,-5.96555E+01) -- (axis cs:1.66217E+02,-6.00672E+01) -- (axis cs:1.66056E+02,-6.02505E+01) -- (axis cs:1.64233E+02,-5.98365E+01) -- cycle ; 
\draw[MWE-empty] (axis cs:1.68077E+02,-6.04533E+01) -- (axis cs:1.69979E+02,-6.08130E+01) -- (axis cs:1.69837E+02,-6.10005E+01) -- (axis cs:1.67924E+02,-6.06388E+01) -- cycle ; 
\draw[MWE-empty] (axis cs:1.71924E+02,-6.11451E+01) -- (axis cs:1.73908E+02,-6.14487E+01) -- (axis cs:1.73786E+02,-6.16401E+01) -- (axis cs:1.71791E+02,-6.13346E+01) -- cycle ; 
\draw[MWE-empty] (axis cs:1.75929E+02,-6.17229E+01) -- (axis cs:1.77984E+02,-6.19668E+01) -- (axis cs:1.77886E+02,-6.21614E+01) -- (axis cs:1.75819E+02,-6.19160E+01) -- cycle ; 
\draw[MWE-empty] (axis cs:1.80071E+02,-6.21796E+01) -- (axis cs:1.82184E+02,-6.23606E+01) -- (axis cs:1.82113E+02,-6.25578E+01) -- (axis cs:1.79986E+02,-6.23756E+01) -- cycle ; 
\draw[MWE-empty] (axis cs:1.84321E+02,-6.25091E+01) -- (axis cs:1.86478E+02,-6.26247E+01) -- (axis cs:1.86434E+02,-6.28237E+01) -- (axis cs:1.84264E+02,-6.27073E+01) -- cycle ; 
\draw[MWE-empty] (axis cs:1.88648E+02,-6.27068E+01) -- (axis cs:1.90828E+02,-6.27553E+01) -- (axis cs:1.90814E+02,-6.29552E+01) -- (axis cs:1.88619E+02,-6.29064E+01) -- cycle ; 
\draw[MWE-empty] (axis cs:1.93013E+02,-6.27699E+01) -- (axis cs:1.95197E+02,-6.27506E+01) -- (axis cs:1.95213E+02,-6.29504E+01) -- (axis cs:1.93014E+02,-6.29699E+01) -- cycle ; 
\draw[MWE-empty] (axis cs:1.97377E+02,-6.26974E+01) -- (axis cs:1.99545E+02,-6.26105E+01) -- (axis cs:1.99591E+02,-6.28094E+01) -- (axis cs:1.97407E+02,-6.28969E+01) -- cycle ; 
\draw[MWE-empty] (axis cs:2.01699E+02,-6.24903E+01) -- (axis cs:2.03833E+02,-6.23372E+01) -- (axis cs:2.03907E+02,-6.25342E+01) -- (axis cs:2.01759E+02,-6.26884E+01) -- cycle ; 
\draw[MWE-empty] (axis cs:2.05943E+02,-6.21517E+01) -- (axis cs:2.08026E+02,-6.19345E+01) -- (axis cs:2.08125E+02,-6.21289E+01) -- (axis cs:2.06030E+02,-6.23475E+01) -- cycle ; 
\draw[MWE-empty] (axis cs:2.10076E+02,-6.16863E+01) -- (axis cs:2.12092E+02,-6.14079E+01) -- (axis cs:2.12216E+02,-6.15991E+01) -- (axis cs:2.10188E+02,-6.18792E+01) -- cycle ; 
\draw[MWE-empty] (axis cs:2.14071E+02,-6.11003E+01) -- (axis cs:2.16010E+02,-6.07642E+01) -- (axis cs:2.16154E+02,-6.09515E+01) -- (axis cs:2.14205E+02,-6.12895E+01) -- cycle ; 
\draw[MWE-empty] (axis cs:2.17906E+02,-6.04008E+01) -- (axis cs:2.19760E+02,-6.00110E+01) -- (axis cs:2.19922E+02,-6.01940E+01) -- (axis cs:2.18060E+02,-6.05860E+01) -- cycle ; 
\draw[MWE-empty] (axis cs:2.21569E+02,-5.95959E+01) -- (axis cs:2.23332E+02,-5.91564E+01) -- (axis cs:2.23509E+02,-5.93347E+01) -- (axis cs:2.21739E+02,-5.97766E+01) -- cycle ; 
\draw[MWE-empty] (axis cs:2.25050E+02,-5.86936E+01) -- (axis cs:2.26721E+02,-5.82085E+01) -- (axis cs:2.26910E+02,-5.83820E+01) -- (axis cs:2.25233E+02,-5.88695E+01) -- cycle ; 
\draw[MWE-empty] (axis cs:2.28346E+02,-5.77022E+01) -- (axis cs:2.29926E+02,-5.71755E+01) -- (axis cs:2.30125E+02,-5.73442E+01) -- (axis cs:2.28541E+02,-5.78732E+01) -- cycle ; 
\draw[MWE-empty] (axis cs:2.31460E+02,-5.66296E+01) -- (axis cs:2.32950E+02,-5.60654E+01) -- (axis cs:2.33157E+02,-5.62291E+01) -- (axis cs:2.31663E+02,-5.67958E+01) -- cycle ; 
\draw[MWE-empty] (axis cs:2.34397E+02,-5.54837E+01) -- (axis cs:2.35800E+02,-5.48854E+01) -- (axis cs:2.36012E+02,-5.50442E+01) -- (axis cs:2.34606E+02,-5.56449E+01) -- cycle ; 
\draw[MWE-empty] (axis cs:2.37162E+02,-5.42714E+01) -- (axis cs:2.38483E+02,-5.36425E+01) -- (axis cs:2.38699E+02,-5.37966E+01) -- (axis cs:2.37376E+02,-5.44279E+01) -- cycle ; 
\draw[MWE-empty] (axis cs:2.39766E+02,-5.29994E+01) -- (axis cs:2.41010E+02,-5.23429E+01) -- (axis cs:2.41227E+02,-5.24925E+01) -- (axis cs:2.39982E+02,-5.31512E+01) -- cycle ; 
\draw[MWE-empty] (axis cs:2.42217E+02,-5.16738E+01) -- (axis cs:2.43389E+02,-5.09926E+01) -- (axis cs:2.43608E+02,-5.11377E+01) -- (axis cs:2.42435E+02,-5.18211E+01) -- cycle ; 
\draw[MWE-empty] (axis cs:2.44526E+02,-5.02999E+01) -- (axis cs:2.45631E+02,-4.95965E+01) -- (axis cs:2.45850E+02,-4.97374E+01) -- (axis cs:2.44745E+02,-5.04429E+01) -- cycle ; 
\draw[MWE-empty] (axis cs:2.46704E+02,-4.88829E+01) -- (axis cs:2.47746E+02,-4.81596E+01) -- (axis cs:2.47965E+02,-4.82965E+01) -- (axis cs:2.46923E+02,-4.90218E+01) -- cycle ; 
\draw[MWE-empty] (axis cs:2.48760E+02,-4.74270E+01) -- (axis cs:2.49745E+02,-4.66858E+01) -- (axis cs:2.49963E+02,-4.68189E+01) -- (axis cs:2.48978E+02,-4.75620E+01) -- cycle ; 
\draw[MWE-empty] (axis cs:2.50704E+02,-4.59363E+01) -- (axis cs:2.51636E+02,-4.51791E+01) -- (axis cs:2.51853E+02,-4.53086E+01) -- (axis cs:2.50921E+02,-4.60676E+01) -- cycle ; 
\draw[MWE-empty] (axis cs:2.52544E+02,-4.44144E+01) -- (axis cs:2.53429E+02,-4.36426E+01) -- (axis cs:2.53644E+02,-4.37688E+01) -- (axis cs:2.52760E+02,-4.45422E+01) -- cycle ; 
\draw[MWE-empty] (axis cs:2.54291E+02,-4.28642E+01) -- (axis cs:2.55131E+02,-4.20795E+01) -- (axis cs:2.55344E+02,-4.22025E+01) -- (axis cs:2.54505E+02,-4.29888E+01) -- cycle ; 
\draw[MWE-empty] (axis cs:2.55951E+02,-4.12888E+01) -- (axis cs:2.56751E+02,-4.04924E+01) -- (axis cs:2.56962E+02,-4.06124E+01) -- (axis cs:2.56163E+02,-4.14103E+01) -- cycle ; 
\draw[MWE-empty] (axis cs:2.57532E+02,-3.96906E+01) -- (axis cs:2.58295E+02,-3.88836E+01) -- (axis cs:2.58504E+02,-3.90009E+01) -- (axis cs:2.57742E+02,-3.98092E+01) -- cycle ; 
\draw[MWE-empty] (axis cs:2.59041E+02,-3.80718E+01) -- (axis cs:2.59771E+02,-3.72554E+01) -- (axis cs:2.59977E+02,-3.73700E+01) -- (axis cs:2.59248E+02,-3.81877E+01) -- cycle ; 
\draw[MWE-empty] (axis cs:2.60484E+02,-3.64345E+01) -- (axis cs:2.61183E+02,-3.56095E+01) -- (axis cs:2.61387E+02,-3.57218E+01) -- (axis cs:2.60689E+02,-3.65480E+01) -- cycle ; 
\draw[MWE-empty] (axis cs:2.61867E+02,-3.47805E+01) -- (axis cs:2.62538E+02,-3.39478E+01) -- (axis cs:2.62740E+02,-3.40578E+01) -- (axis cs:2.62070E+02,-3.48916E+01) -- cycle ; 
\draw[MWE-empty] (axis cs:2.63196E+02,-3.31114E+01) -- (axis cs:2.63841E+02,-3.22717E+01) -- (axis cs:2.64040E+02,-3.23796E+01) -- (axis cs:2.63396E+02,-3.32204E+01) -- cycle ; 
\draw[MWE-empty] (axis cs:2.64474E+02,-3.14287E+01) -- (axis cs:2.65096E+02,-3.05826E+01) -- (axis cs:2.65293E+02,-3.06886E+01) -- (axis cs:2.64672E+02,-3.15356E+01) -- cycle ; 
\draw[MWE-empty] (axis cs:2.65707E+02,-2.97337E+01) -- (axis cs:2.66308E+02,-2.88819E+01) -- (axis cs:2.66503E+02,-2.89861E+01) -- (axis cs:2.65903E+02,-2.98387E+01) -- cycle ; 

 
\draw [MWE-empty] (axis cs:2.66016E+02,-2.87251E+01) -- (axis cs:2.66797E+02,-2.91419E+01)   ;
\node[pin={[pin distance=-0.4\onedegree,MWE-label]0:{0$^\circ$}}] at (axis cs:2.66016E+02,-2.87251E+01) {} ;
\draw [MWE-empty] (axis cs:2.71569E+02,-2.00932E+01) -- (axis cs:2.72314E+02,-2.04821E+01)   ;
\node[pin={[pin distance=-0.4\onedegree,MWE-label]00:{10$^\circ$}}] at (axis cs:2.71569E+02,-2.00932E+01) {} ;
\draw [MWE-empty] (axis cs:2.76522E+02,-1.13004E+01) -- (axis cs:2.77245E+02,-1.16726E+01)   ;
\node[pin={[pin distance=-0.4\onedegree,MWE-label]00:{20$^\circ$}}] at (axis cs:2.76522E+02,-1.13004E+01) {} ;
\draw [MWE-empty] (axis cs:2.81167E+02,-2.42475E+00) -- (axis cs:2.81879E+02,-2.78993E+00)   ;
\node[pin={[pin distance=-0.4\onedegree,MWE-label]00:{30$^\circ$}}] at (axis cs:2.81167E+02,-2.42475E+00) {} ;
\draw [MWE-empty] (axis cs:2.85739E+02,6.47217E+00) -- (axis cs:2.86454E+02,6.10515E+00)   ;
\node[pin={[pin distance=-0.4\onedegree,MWE-label]00:{40$^\circ$}}] at (axis cs:2.85739E+02,6.47217E+00) {} ;
\draw [MWE-empty] (axis cs:2.90465E+02,1.53325E+01) -- (axis cs:2.91196E+02,1.49546E+01)   ;
\node[pin={[pin distance=-0.4\onedegree,MWE-label]00:{50$^\circ$}}] at (axis cs:2.90465E+02,1.53325E+01) {} ;
\draw [MWE-empty] (axis cs:2.95599E+02,2.40907E+01) -- (axis cs:2.96357E+02,2.36917E+01)   ;
\node[pin={[pin distance=-0.4\onedegree,MWE-label]00:{60$^\circ$}}] at (axis cs:2.95599E+02,2.40907E+01) {} ;
\draw [MWE-empty] (axis cs:3.01469E+02,3.26590E+01) -- (axis cs:3.02267E+02,3.22267E+01)   ;
\node[pin={[pin distance=-0.4\onedegree,MWE-label]00:{70$^\circ$}}] at (axis cs:3.01469E+02,3.26590E+01) {} ;
\draw [MWE-empty] (axis cs:3.08552E+02,4.09041E+01) -- (axis cs:3.09394E+02,4.04231E+01)   ;
\node[pin={[pin distance=-0.4\onedegree,MWE-label]00:{80$^\circ$}}] at (axis cs:3.08552E+02,4.09041E+01) {} ;
\draw [MWE-empty] (axis cs:3.17568E+02,4.86036E+01) -- (axis cs:3.18444E+02,4.80548E+01)   ;
\node[pin={[pin distance=-0.4\onedegree,MWE-label]90:{90$^\circ$}}] at (axis cs:3.17568E+02,4.86036E+01) {} ;
\draw [MWE-empty] (axis cs:3.29582E+02,5.53674E+01) -- (axis cs:3.30428E+02,5.47306E+01)   ;
\node[pin={[pin distance=-0.4\onedegree,MWE-label]090:{100$^\circ$}}] at (axis cs:3.29582E+02,5.53674E+01) {} ;
\draw [MWE-empty] (axis cs:3.45811E+02,6.05250E+01) -- (axis cs:3.46455E+02,5.97919E+01)   ;
\node[pin={[pin distance=-0.4\onedegree,MWE-label]090:{110$^\circ$}}] at (axis cs:3.45811E+02,6.05250E+01) {} ;
\draw [MWE-empty] (axis cs:6.36807E+00,6.31222E+01) -- (axis cs:6.54138E+00,6.23261E+01)   ;
\node[pin={[pin distance=-0.4\onedegree,MWE-label]090:{120$^\circ$}}] at (axis cs:6.36807E+00,6.31222E+01) {} ;
\draw [MWE-empty] (axis cs:2.82769E+01,6.24205E+01) -- (axis cs:2.78784E+01,6.16426E+01)   ;
\node[pin={[pin distance=-0.4\onedegree,MWE-label]090:{130$^\circ$}}] at (axis cs:2.82769E+01,6.24205E+01) {} ;
\draw [MWE-empty] (axis cs:4.71975E+01,5.86418E+01) -- (axis cs:4.64407E+01,5.79477E+01)   ;
\node[pin={[pin distance=-0.4\onedegree,MWE-label]090:{140$^\circ$}}] at (axis cs:4.71975E+01,5.86418E+01) {} ;
\draw [MWE-empty] (axis cs:6.15567E+01,5.27160E+01) -- (axis cs:6.06857E+01,5.21179E+01)   ;
\node[pin={[pin distance=-0.4\onedegree,MWE-label]090:{150$^\circ$}}] at (axis cs:6.15567E+01,5.27160E+01) {} ;
\draw [MWE-empty] (axis cs:7.21792E+01,4.55021E+01) -- (axis cs:7.13134E+01,4.49840E+01)   ;
\node[pin={[pin distance=-0.4\onedegree,MWE-label]090:{160$^\circ$}}] at (axis cs:7.21792E+01,4.55021E+01) {} ;
\draw [MWE-empty] (axis cs:8.02868E+01,3.75414E+01) -- (axis cs:7.94627E+01,3.70827E+01)   ;
\node[pin={[pin distance=-0.4\onedegree,MWE-label]090:{170$^\circ$}}] at (axis cs:8.02868E+01,3.75414E+01) {} ;
\draw [MWE-empty] (axis cs:8.67966E+01,2.91419E+01) -- (axis cs:8.60164E+01,2.87251E+01)   ;
\node[pin={[pin distance=-0.4\onedegree,MWE-label]090:{180$^\circ$}}] at (axis cs:8.67966E+01,2.91419E+01) {} ;
\draw [MWE-empty] (axis cs:9.23144E+01,2.04821E+01) -- (axis cs:9.15691E+01,2.00932E+01)   ;
\node[pin={[pin distance=-0.4\onedegree,MWE-label]180:{190$^\circ$}}] at (axis cs:9.23144E+01,2.04821E+01) {} ;
\draw [MWE-empty] (axis cs:9.72449E+01,1.16726E+01) -- (axis cs:9.65223E+01,1.13004E+01)   ;
\node[pin={[pin distance=-0.4\onedegree,MWE-label]180:{200$^\circ$}}] at (axis cs:9.72449E+01,1.16726E+01) {} ;
\draw [MWE-empty] (axis cs:1.01879E+02,2.78993E+00) -- (axis cs:1.01167E+02,2.42475E+00)   ;
\node[pin={[pin distance=-0.4\onedegree,MWE-label]180:{210$^\circ$}}] at (axis cs:1.01879E+02,2.78993E+00) {} ;
\draw [MWE-empty] (axis cs:1.06454E+02,-6.10515E+00) -- (axis cs:1.05739E+02,-6.47217E+00)   ;
\node[pin={[pin distance=-0.4\onedegree,MWE-label]180:{220$^\circ$}}] at (axis cs:1.06454E+02,-6.10515E+00) {} ;
\draw [MWE-empty] (axis cs:1.11196E+02,-1.49546E+01) -- (axis cs:1.10466E+02,-1.53325E+01)   ;
\node[pin={[pin distance=-0.4\onedegree,MWE-label]180:{230$^\circ$}}] at (axis cs:1.11196E+02,-1.49546E+01) {} ;
\draw [MWE-empty] (axis cs:1.16357E+02,-2.36917E+01) -- (axis cs:1.15599E+02,-2.40907E+01)   ;
\node[pin={[pin distance=-0.4\onedegree,MWE-label]180:{240$^\circ$}}] at (axis cs:1.16357E+02,-2.36917E+01) {} ;
\draw [MWE-empty] (axis cs:1.22267E+02,-3.22267E+01) -- (axis cs:1.21469E+02,-3.26590E+01)   ;
\node[pin={[pin distance=-0.4\onedegree,MWE-label]180:{250$^\circ$}}] at (axis cs:1.22267E+02,-3.22267E+01) {} ;
\draw [MWE-empty] (axis cs:1.29395E+02,-4.04231E+01) -- (axis cs:1.28552E+02,-4.09041E+01)   ;
\node[pin={[pin distance=-0.4\onedegree,MWE-label]180:{260$^\circ$}}] at (axis cs:1.29395E+02,-4.04231E+01) {} ;
\draw [MWE-empty] (axis cs:1.38444E+02,-4.80548E+01) -- (axis cs:1.37568E+02,-4.86036E+01)   ;
\node[pin={[pin distance=-0.4\onedegree,MWE-label]090:{270$^\circ$}}] at (axis cs:1.38444E+02,-4.80548E+01) {} ;
\draw [MWE-empty] (axis cs:1.50428E+02,-5.47306E+01) -- (axis cs:1.49582E+02,-5.53674E+01)   ;
\node[pin={[pin distance=-0.4\onedegree,MWE-label]090:{280$^\circ$}}] at (axis cs:1.50428E+02,-5.47306E+01) {} ;
\draw [MWE-empty] (axis cs:1.66455E+02,-5.97919E+01) -- (axis cs:1.65811E+02,-6.05250E+01)   ;
\node[pin={[pin distance=-0.4\onedegree,MWE-label]090:{290$^\circ$}}] at (axis cs:1.66455E+02,-5.97919E+01) {} ;
\draw [MWE-empty] (axis cs:1.86541E+02,-6.23261E+01) -- (axis cs:1.86368E+02,-6.31222E+01)   ;
\node[pin={[pin distance=-0.4\onedegree,MWE-label]090:{300$^\circ$}}] at (axis cs:1.86541E+02,-6.23261E+01) {} ;
\draw [MWE-empty] (axis cs:2.07878E+02,-6.16426E+01) -- (axis cs:2.08277E+02,-6.24205E+01)   ;
\node[pin={[pin distance=-0.4\onedegree,MWE-label]090:{310$^\circ$}}] at (axis cs:2.07878E+02,-6.16426E+01) {} ;
\draw [MWE-empty] (axis cs:2.26441E+02,-5.79477E+01) -- (axis cs:2.27198E+02,-5.86418E+01)   ;
\node[pin={[pin distance=-0.4\onedegree,MWE-label]090:{320$^\circ$}}] at (axis cs:2.26441E+02,-5.79477E+01) {} ;
\draw [MWE-empty] (axis cs:2.40686E+02,-5.21179E+01) -- (axis cs:2.41557E+02,-5.27160E+01)   ;
\node[pin={[pin distance=-0.4\onedegree,MWE-label]090:{330$^\circ$}}] at (axis cs:2.40686E+02,-5.21179E+01) {} ;
\draw [MWE-empty] (axis cs:2.51313E+02,-4.49840E+01) -- (axis cs:2.52179E+02,-4.55021E+01)   ;
\node[pin={[pin distance=-0.4\onedegree,MWE-label]090:{340$^\circ$}}] at (axis cs:2.51313E+02,-4.49840E+01) {} ;
\draw [MWE-empty] (axis cs:2.59463E+02,-3.70827E+01) -- (axis cs:2.60287E+02,-3.75414E+01)   ;
\node[pin={[pin distance=-0.4\onedegree,MWE-label]090:{350$^\circ$}}] at (axis cs:2.59463E+02,-3.70827E+01) {} ;


\end{axis}

% Ecliptic coordinate system


\begin{axis}[name=ecliptics,axis lines=none,at=(base.center),anchor=center]

% This simply plots the line
%\addplot[ecliptics,domain=0:390,samples=390] {23.433*sin(x)};

\draw [ecliptics-full] (axis cs:3.59960E+02,9.17485E-02) -- (axis cs:360.0398,-9.17483E-02)  -- (axis cs:3.59122E+02,-4.89506E-01) -- (axis cs:3.59043E+02,-3.06005E-01) -- cycle  ;

\draw [ecliptics-full] (axis cs:{8.77725E-01},{4.89506E-01}) -- (axis cs:{9.57272E-01},{3.06005E-01})  -- (axis cs:{1.87485E+00},{7.03652E-01}) -- (axis cs:{1.79532E+00},{8.87166E-01}) -- cycle  ;
\draw [ecliptics-full] (axis cs:2.71311E+00,1.28463E+00) -- (axis cs:2.79259E+00,1.10109E+00)  -- (axis cs:3.71059E+00,1.49822E+00) -- (axis cs:3.63117E+00,1.68179E+00) -- cycle  ;
\draw [ecliptics-full] (axis cs:4.54959E+00,2.07854E+00) -- (axis cs:4.62893E+00,1.89493E+00)  -- (axis cs:5.54771E+00,2.29113E+00) -- (axis cs:5.46845E+00,2.47479E+00) -- cycle  ;
\draw [ecliptics-full] (axis cs:6.38786E+00,2.87043E+00) -- (axis cs:6.46701E+00,2.68672E+00)  -- (axis cs:7.38691E+00,3.08158E+00) -- (axis cs:7.30789E+00,3.26536E+00) -- cycle  ;
\draw [ecliptics-full] (axis cs:8.22863E+00,3.65948E+00) -- (axis cs:8.30751E+00,3.47563E+00)  -- (axis cs:9.22889E+00,3.86875E+00) -- (axis cs:9.15017E+00,4.05268E+00) -- cycle  ;
\draw [ecliptics-full] (axis cs:1.00726E+01,4.44487E+00) -- (axis cs:1.01511E+01,4.26084E+00)  -- (axis cs:1.10743E+01,4.65180E+00) -- (axis cs:1.09960E+01,4.83593E+00) -- cycle  ;
\draw [ecliptics-full] (axis cs:1.19204E+01,5.22577E+00) -- (axis cs:1.19986E+01,5.04154E+00)  -- (axis cs:1.29239E+01,5.42993E+00) -- (axis cs:1.28460E+01,5.61429E+00) -- cycle  ;
\draw [ecliptics-full] (axis cs:1.37728E+01,6.00137E+00) -- (axis cs:1.38505E+01,5.81689E+00)  -- (axis cs:1.47783E+01,6.20231E+00) -- (axis cs:1.47009E+01,6.38692E+00) -- cycle  ;
\draw [ecliptics-full] (axis cs:1.56304E+01,6.77084E+00) -- (axis cs:1.57075E+01,6.58609E+00)  -- (axis cs:1.66382E+01,6.96811E+00) -- (axis cs:1.65614E+01,7.15301E+00) -- cycle  ;
\draw [ecliptics-full] (axis cs:1.74939E+01,7.53334E+00) -- (axis cs:1.75704E+01,7.34829E+00)  -- (axis cs:1.85041E+01,7.72651E+00) -- (axis cs:1.84280E+01,7.91173E+00) -- cycle  ;
\draw [ecliptics-full] (axis cs:1.93638E+01,8.28806E+00) -- (axis cs:1.94396E+01,8.10267E+00)  -- (axis cs:2.03768E+01,8.47667E+00) -- (axis cs:2.03014E+01,8.66224E+00) -- cycle  ;
\draw [ecliptics-full] (axis cs:2.12408E+01,9.03415E+00) -- (axis cs:2.13159E+01,8.84840E+00)  -- (axis cs:2.22568E+01,9.21776E+00) -- (axis cs:2.21822E+01,9.40370E+00) -- cycle  ;
\draw [ecliptics-full] (axis cs:2.31256E+01,9.77078E+00) -- (axis cs:2.31998E+01,9.58464E+00)  -- (axis cs:2.41448E+01,9.94894E+00) -- (axis cs:2.40710E+01,1.01353E+01) -- cycle  ;
\draw [ecliptics-full] (axis cs:2.50186E+01,1.04971E+01) -- (axis cs:2.50918E+01,1.03105E+01)  -- (axis cs:2.60411E+01,1.06694E+01) -- (axis cs:2.59683E+01,1.08562E+01) -- cycle  ;
\draw [ecliptics-full] (axis cs:2.69204E+01,1.12123E+01) -- (axis cs:2.69926E+01,1.10253E+01)  -- (axis cs:2.79465E+01,1.13782E+01) -- (axis cs:2.78747E+01,1.15655E+01) -- cycle  ;
\draw [ecliptics-full] (axis cs:2.88315E+01,1.19155E+01) -- (axis cs:2.89026E+01,1.17280E+01)  -- (axis cs:2.98613E+01,1.20746E+01) -- (axis cs:2.97907E+01,1.22623E+01) -- cycle  ;
\draw [ecliptics-full] (axis cs:3.07524E+01,1.26059E+01) -- (axis cs:3.08224E+01,1.24179E+01)  -- (axis cs:3.17860E+01,1.27578E+01) -- (axis cs:3.17166E+01,1.29460E+01) -- cycle  ;
\draw [ecliptics-full] (axis cs:3.26835E+01,1.32826E+01) -- (axis cs:3.27522E+01,1.30941E+01)  -- (axis cs:3.37211E+01,1.34268E+01) -- (axis cs:3.36530E+01,1.36155E+01) -- cycle  ;
\draw [ecliptics-full] (axis cs:3.46253E+01,1.39448E+01) -- (axis cs:3.46927E+01,1.37558E+01)  -- (axis cs:3.56669E+01,1.40809E+01) -- (axis cs:3.56002E+01,1.42701E+01) -- cycle  ;
\draw [ecliptics-full] (axis cs:3.65780E+01,1.45916E+01) -- (axis cs:3.66440E+01,1.44021E+01)  -- (axis cs:3.76238E+01,1.47192E+01) -- (axis cs:3.75586E+01,1.49090E+01) -- cycle  ;
\draw [ecliptics-full] (axis cs:3.85421E+01,1.52222E+01) -- (axis cs:3.86065E+01,1.50321E+01)  -- (axis cs:3.95921E+01,1.53408E+01) -- (axis cs:3.95285E+01,1.55312E+01) -- cycle  ;
\draw [ecliptics-full] (axis cs:4.05178E+01,1.58358E+01) -- (axis cs:4.05806E+01,1.56451E+01)  -- (axis cs:4.15720E+01,1.59450E+01) -- (axis cs:4.15101E+01,1.61359E+01) -- cycle  ;
\draw [ecliptics-full] (axis cs:4.25053E+01,1.64315E+01) -- (axis cs:4.25664E+01,1.62403E+01)  -- (axis cs:4.35638E+01,1.65309E+01) -- (axis cs:4.35036E+01,1.67224E+01) -- cycle  ;
\draw [ecliptics-full] (axis cs:4.45048E+01,1.70085E+01) -- (axis cs:4.45641E+01,1.68167E+01)  -- (axis cs:4.55675E+01,1.70976E+01) -- (axis cs:4.55091E+01,1.72897E+01) -- cycle  ;
\draw [ecliptics-full] (axis cs:4.65165E+01,1.75660E+01) -- (axis cs:4.65738E+01,1.73736E+01)  -- (axis cs:4.75832E+01,1.76445E+01) -- (axis cs:4.75268E+01,1.78371E+01) -- cycle  ;
\draw [ecliptics-full] (axis cs:4.85402E+01,1.81031E+01) -- (axis cs:4.85956E+01,1.79102E+01)  -- (axis cs:4.96110E+01,1.81706E+01) -- (axis cs:4.95567E+01,1.83638E+01) -- cycle  ;
\draw [ecliptics-full] (axis cs:5.05762E+01,1.86192E+01) -- (axis cs:5.06294E+01,1.84256E+01)  -- (axis cs:5.16509E+01,1.86752E+01) -- (axis cs:5.15987E+01,1.88690E+01) -- cycle  ;
\draw [ecliptics-full] (axis cs:5.26242E+01,1.91133E+01) -- (axis cs:5.26753E+01,1.89192E+01)  -- (axis cs:5.37027E+01,1.91575E+01) -- (axis cs:5.36527E+01,1.93519E+01) -- cycle  ;
\draw [ecliptics-full] (axis cs:5.46842E+01,1.95848E+01) -- (axis cs:5.47330E+01,1.93901E+01)  -- (axis cs:5.57662E+01,1.96168E+01) -- (axis cs:5.57187E+01,1.98118E+01) -- cycle  ;
\draw [ecliptics-full] (axis cs:5.67560E+01,2.00328E+01) -- (axis cs:5.68024E+01,1.98376E+01)  -- (axis cs:5.78414E+01,2.00524E+01) -- (axis cs:5.77962E+01,2.02479E+01) -- cycle  ;
\draw [ecliptics-full] (axis cs:5.88393E+01,2.04568E+01) -- (axis cs:5.88831E+01,2.02610E+01)  -- (axis cs:5.99277E+01,2.04635E+01) -- (axis cs:5.98851E+01,2.06595E+01) -- cycle  ;
\draw [ecliptics-full] (axis cs:6.09337E+01,2.08559E+01) -- (axis cs:6.09750E+01,2.06597E+01)  -- (axis cs:6.20250E+01,2.08495E+01) -- (axis cs:6.19850E+01,2.10459E+01) -- cycle  ;
\draw [ecliptics-full] (axis cs:6.30389E+01,2.12295E+01) -- (axis cs:6.30775E+01,2.10328E+01)  -- (axis cs:6.41327E+01,2.12096E+01) -- (axis cs:6.40954E+01,2.14066E+01) -- cycle  ;
\draw [ecliptics-full] (axis cs:6.51544E+01,2.15771E+01) -- (axis cs:6.51903E+01,2.13799E+01)  -- (axis cs:6.62503E+01,2.15434E+01) -- (axis cs:6.62158E+01,2.17408E+01) -- cycle  ;
\draw [ecliptics-full] (axis cs:6.72796E+01,2.18979E+01) -- (axis cs:6.73127E+01,2.17002E+01)  -- (axis cs:6.83773E+01,2.18502E+01) -- (axis cs:6.83457E+01,2.20481E+01) -- cycle  ;
\draw [ecliptics-full] (axis cs:6.94140E+01,2.21914E+01) -- (axis cs:6.94442E+01,2.19934E+01)  -- (axis cs:7.05131E+01,2.21296E+01) -- (axis cs:7.04845E+01,2.23278E+01) -- cycle  ;
\draw [ecliptics-full] (axis cs:7.15569E+01,2.24572E+01) -- (axis cs:7.15841E+01,2.22587E+01)  -- (axis cs:7.26570E+01,2.23809E+01) -- (axis cs:7.26313E+01,2.25795E+01) -- cycle  ;
\draw [ecliptics-full] (axis cs:7.37076E+01,2.26946E+01) -- (axis cs:7.37317E+01,2.24959E+01)  -- (axis cs:7.48082E+01,2.26037E+01) -- (axis cs:7.47856E+01,2.28026E+01) -- cycle  ;
\draw [ecliptics-full] (axis cs:7.58653E+01,2.29034E+01) -- (axis cs:7.58863E+01,2.27043E+01)  -- (axis cs:7.69660E+01,2.27977E+01) -- (axis cs:7.69465E+01,2.29969E+01) -- cycle  ;
\draw [ecliptics-full] (axis cs:7.80291E+01,2.30830E+01) -- (axis cs:7.80470E+01,2.28837E+01)  -- (axis cs:7.91294E+01,2.29624E+01) -- (axis cs:7.91131E+01,2.31619E+01) -- cycle  ;
\draw [ecliptics-full] (axis cs:8.01982E+01,2.32333E+01) -- (axis cs:8.02130E+01,2.30338E+01)  -- (axis cs:8.12976E+01,2.30977E+01) -- (axis cs:8.12845E+01,2.32973E+01) -- cycle  ;
\draw [ecliptics-full] (axis cs:8.23718E+01,2.33538E+01) -- (axis cs:8.23833E+01,2.31541E+01)  -- (axis cs:8.34698E+01,2.32031E+01) -- (axis cs:8.34599E+01,2.34029E+01) -- cycle  ;
\draw [ecliptics-full] (axis cs:8.45488E+01,2.34444E+01) -- (axis cs:8.45570E+01,2.32446E+01)  -- (axis cs:8.56449E+01,2.32785E+01) -- (axis cs:8.56383E+01,2.34785E+01) -- cycle  ;
\draw [ecliptics-full] (axis cs:8.67283E+01,2.35049E+01) -- (axis cs:8.67332E+01,2.33050E+01)  -- (axis cs:8.78219E+01,2.33239E+01) -- (axis cs:8.78186E+01,2.35239E+01) -- cycle  ;
\draw [ecliptics-full] (axis cs:8.89093E+01,2.35352E+01) -- (axis cs:8.89109E+01,2.33352E+01)  -- (axis cs:9.00000E+01,2.33390E+01) -- (axis cs:9.00000E+01,2.35390E+01) -- cycle  ;
\draw [ecliptics-full] (axis cs:9.10907E+01,2.35352E+01) -- (axis cs:9.10891E+01,2.33352E+01)  -- (axis cs:9.21781E+01,2.33239E+01) -- (axis cs:9.21814E+01,2.35239E+01) -- cycle  ;
\draw [ecliptics-full] (axis cs:9.32717E+01,2.35049E+01) -- (axis cs:9.32668E+01,2.33050E+01)  -- (axis cs:9.43551E+01,2.32785E+01) -- (axis cs:9.43617E+01,2.34785E+01) -- cycle  ;
\draw [ecliptics-full] (axis cs:9.54512E+01,2.34444E+01) -- (axis cs:9.54430E+01,2.32446E+01)  -- (axis cs:9.65302E+01,2.32031E+01) -- (axis cs:9.65401E+01,2.34029E+01) -- cycle  ;
\draw [ecliptics-full] (axis cs:9.76282E+01,2.33538E+01) -- (axis cs:9.76167E+01,2.31541E+01)  -- (axis cs:9.87024E+01,2.30977E+01) -- (axis cs:9.87155E+01,2.32973E+01) -- cycle  ;
\draw [ecliptics-full] (axis cs:9.98018E+01,2.32333E+01) -- (axis cs:9.97870E+01,2.30338E+01)  -- (axis cs:1.00871E+02,2.29624E+01) -- (axis cs:1.00887E+02,2.31619E+01) -- cycle  ;
\draw [ecliptics-full] (axis cs:1.01971E+02,2.30830E+01) -- (axis cs:1.01953E+02,2.28837E+01)  -- (axis cs:1.03034E+02,2.27977E+01) -- (axis cs:1.03054E+02,2.29969E+01) -- cycle  ;
\draw [ecliptics-full] (axis cs:1.04135E+02,2.29034E+01) -- (axis cs:1.04114E+02,2.27043E+01)  -- (axis cs:1.05192E+02,2.26037E+01) -- (axis cs:1.05214E+02,2.28026E+01) -- cycle  ;
\draw [ecliptics-full] (axis cs:1.06292E+02,2.26946E+01) -- (axis cs:1.06268E+02,2.24959E+01)  -- (axis cs:1.07343E+02,2.23809E+01) -- (axis cs:1.07369E+02,2.25795E+01) -- cycle  ;
\draw [ecliptics-full] (axis cs:1.08443E+02,2.24572E+01) -- (axis cs:1.08416E+02,2.22587E+01)  -- (axis cs:1.09487E+02,2.21296E+01) -- (axis cs:1.09516E+02,2.23278E+01) -- cycle  ;
\draw [ecliptics-full] (axis cs:1.10586E+02,2.21914E+01) -- (axis cs:1.10556E+02,2.19934E+01)  -- (axis cs:1.11623E+02,2.18502E+01) -- (axis cs:1.11654E+02,2.20481E+01) -- cycle  ;
\draw [ecliptics-full] (axis cs:1.12720E+02,2.18979E+01) -- (axis cs:1.12687E+02,2.17002E+01)  -- (axis cs:1.13750E+02,2.15434E+01) -- (axis cs:1.13784E+02,2.17408E+01) -- cycle  ;
\draw [ecliptics-full] (axis cs:1.14846E+02,2.15771E+01) -- (axis cs:1.14810E+02,2.13799E+01)  -- (axis cs:1.15867E+02,2.12096E+01) -- (axis cs:1.15905E+02,2.14066E+01) -- cycle  ;
\draw [ecliptics-full] (axis cs:1.16961E+02,2.12295E+01) -- (axis cs:1.16922E+02,2.10328E+01)  -- (axis cs:1.17975E+02,2.08495E+01) -- (axis cs:1.18015E+02,2.10459E+01) -- cycle  ;
\draw [ecliptics-full] (axis cs:1.19066E+02,2.08559E+01) -- (axis cs:1.19025E+02,2.06597E+01)  -- (axis cs:1.20072E+02,2.04635E+01) -- (axis cs:1.20115E+02,2.06595E+01) -- cycle  ;
\draw [ecliptics-full] (axis cs:1.21161E+02,2.04568E+01) -- (axis cs:1.21117E+02,2.02610E+01)  -- (axis cs:1.22159E+02,2.00524E+01) -- (axis cs:1.22204E+02,2.02479E+01) -- cycle  ;
\draw [ecliptics-full] (axis cs:1.23244E+02,2.00328E+01) -- (axis cs:1.23198E+02,1.98376E+01)  -- (axis cs:1.24234E+02,1.96168E+01) -- (axis cs:1.24281E+02,1.98118E+01) -- cycle  ;
\draw [ecliptics-full] (axis cs:1.25316E+02,1.95848E+01) -- (axis cs:1.25267E+02,1.93901E+01)  -- (axis cs:1.26297E+02,1.91575E+01) -- (axis cs:1.26347E+02,1.93519E+01) -- cycle  ;
\draw [ecliptics-full] (axis cs:1.27376E+02,1.91133E+01) -- (axis cs:1.27325E+02,1.89192E+01)  -- (axis cs:1.28349E+02,1.86752E+01) -- (axis cs:1.28401E+02,1.88690E+01) -- cycle  ;
\draw [ecliptics-full] (axis cs:1.29424E+02,1.86192E+01) -- (axis cs:1.29371E+02,1.84256E+01)  -- (axis cs:1.30389E+02,1.81706E+01) -- (axis cs:1.30443E+02,1.83638E+01) -- cycle  ;
\draw [ecliptics-full] (axis cs:1.31460E+02,1.81031E+01) -- (axis cs:1.31404E+02,1.79102E+01)  -- (axis cs:1.32417E+02,1.76445E+01) -- (axis cs:1.32473E+02,1.78371E+01) -- cycle  ;
\draw [ecliptics-full] (axis cs:1.33484E+02,1.75660E+01) -- (axis cs:1.33426E+02,1.73736E+01)  -- (axis cs:1.34433E+02,1.70976E+01) -- (axis cs:1.34491E+02,1.72897E+01) -- cycle  ;
\draw [ecliptics-full] (axis cs:1.35495E+02,1.70085E+01) -- (axis cs:1.35436E+02,1.68167E+01)  -- (axis cs:1.36436E+02,1.65309E+01) -- (axis cs:1.36496E+02,1.67224E+01) -- cycle  ;
\draw [ecliptics-full] (axis cs:1.37495E+02,1.64315E+01) -- (axis cs:1.37434E+02,1.62403E+01)  -- (axis cs:1.38428E+02,1.59450E+01) -- (axis cs:1.38490E+02,1.61359E+01) -- cycle  ;
\draw [ecliptics-full] (axis cs:1.39482E+02,1.58358E+01) -- (axis cs:1.39419E+02,1.56451E+01)  -- (axis cs:1.40408E+02,1.53408E+01) -- (axis cs:1.40471E+02,1.55312E+01) -- cycle  ;
\draw [ecliptics-full] (axis cs:1.41458E+02,1.52222E+01) -- (axis cs:1.41393E+02,1.50321E+01)  -- (axis cs:1.42376E+02,1.47192E+01) -- (axis cs:1.42441E+02,1.49090E+01) -- cycle  ;
\draw [ecliptics-full] (axis cs:1.43422E+02,1.45916E+01) -- (axis cs:1.43356E+02,1.44021E+01)  -- (axis cs:1.44333E+02,1.40809E+01) -- (axis cs:1.44400E+02,1.42701E+01) -- cycle  ;
\draw [ecliptics-full] (axis cs:1.45375E+02,1.39448E+01) -- (axis cs:1.45307E+02,1.37558E+01)  -- (axis cs:1.46279E+02,1.34268E+01) -- (axis cs:1.46347E+02,1.36155E+01) -- cycle  ;
\draw [ecliptics-full] (axis cs:1.47317E+02,1.32826E+01) -- (axis cs:1.47248E+02,1.30941E+01)  -- (axis cs:1.48214E+02,1.27578E+01) -- (axis cs:1.48283E+02,1.29460E+01) -- cycle  ;
\draw [ecliptics-full] (axis cs:1.49248E+02,1.26059E+01) -- (axis cs:1.49178E+02,1.24179E+01)  -- (axis cs:1.50139E+02,1.20746E+01) -- (axis cs:1.50209E+02,1.22623E+01) -- cycle  ;
\draw [ecliptics-full] (axis cs:1.51169E+02,1.19155E+01) -- (axis cs:1.51097E+02,1.17280E+01)  -- (axis cs:1.52054E+02,1.13782E+01) -- (axis cs:1.52125E+02,1.15655E+01) -- cycle  ;
\draw [ecliptics-full] (axis cs:1.53080E+02,1.12123E+01) -- (axis cs:1.53007E+02,1.10253E+01)  -- (axis cs:1.53959E+02,1.06694E+01) -- (axis cs:1.54032E+02,1.08562E+01) -- cycle  ;
\draw [ecliptics-full] (axis cs:1.54981E+02,1.04971E+01) -- (axis cs:1.54908E+02,1.03105E+01)  -- (axis cs:1.55855E+02,9.94894E+00) -- (axis cs:1.55929E+02,1.01353E+01) -- cycle  ;
\draw [ecliptics-full] (axis cs:1.56874E+02,9.77078E+00) -- (axis cs:1.56800E+02,9.58464E+00)  -- (axis cs:1.57743E+02,9.21776E+00) -- (axis cs:1.57818E+02,9.40370E+00) -- cycle  ;
\draw [ecliptics-full] (axis cs:1.58759E+02,9.03415E+00) -- (axis cs:1.58684E+02,8.84840E+00)  -- (axis cs:1.59623E+02,8.47667E+00) -- (axis cs:1.59699E+02,8.66224E+00) -- cycle  ;
\draw [ecliptics-full] (axis cs:1.60636E+02,8.28806E+00) -- (axis cs:1.60560E+02,8.10267E+00)  -- (axis cs:1.61496E+02,7.72651E+00) -- (axis cs:1.61572E+02,7.91173E+00) -- cycle  ;
\draw [ecliptics-full] (axis cs:1.62506E+02,7.53334E+00) -- (axis cs:1.62430E+02,7.34829E+00)  -- (axis cs:1.63362E+02,6.96811E+00) -- (axis cs:1.63439E+02,7.15301E+00) -- cycle  ;
\draw [ecliptics-full] (axis cs:1.64370E+02,6.77084E+00) -- (axis cs:1.64292E+02,6.58609E+00)  -- (axis cs:1.65222E+02,6.20231E+00) -- (axis cs:1.65299E+02,6.38692E+00) -- cycle  ;
\draw [ecliptics-full] (axis cs:1.66227E+02,6.00137E+00) -- (axis cs:1.66150E+02,5.81689E+00)  -- (axis cs:1.67076E+02,5.42993E+00) -- (axis cs:1.67154E+02,5.61429E+00) -- cycle  ;
\draw [ecliptics-full] (axis cs:1.68080E+02,5.22577E+00) -- (axis cs:1.68001E+02,5.04154E+00)  -- (axis cs:1.68926E+02,4.65180E+00) -- (axis cs:1.69004E+02,4.83593E+00) -- cycle  ;
\draw [ecliptics-full] (axis cs:1.69927E+02,4.44487E+00) -- (axis cs:1.69849E+02,4.26084E+00)  -- (axis cs:1.70771E+02,3.86875E+00) -- (axis cs:1.70850E+02,4.05268E+00) -- cycle  ;
\draw [ecliptics-full] (axis cs:1.71771E+02,3.65948E+00) -- (axis cs:1.71692E+02,3.47563E+00)  -- (axis cs:1.72613E+02,3.08158E+00) -- (axis cs:1.72692E+02,3.26536E+00) -- cycle  ;
\draw [ecliptics-full] (axis cs:1.73612E+02,2.87043E+00) -- (axis cs:1.73533E+02,2.68672E+00)  -- (axis cs:1.74452E+02,2.29113E+00) -- (axis cs:1.74532E+02,2.47479E+00) -- cycle  ;
\draw [ecliptics-full] (axis cs:1.75450E+02,2.07854E+00) -- (axis cs:1.75371E+02,1.89493E+00)  -- (axis cs:1.76289E+02,1.49822E+00) -- (axis cs:1.76369E+02,1.68179E+00) -- cycle  ;
\draw [ecliptics-full] (axis cs:1.77287E+02,1.28463E+00) -- (axis cs:1.77207E+02,1.10109E+00)  -- (axis cs:1.78125E+02,7.03652E-01) -- (axis cs:1.78205E+02,8.87166E-01) -- cycle  ;
\draw [ecliptics-full] (axis cs:1.79122E+02,4.89506E-01) -- (axis cs:1.79043E+02,3.06005E-01)  -- (axis cs:1.79960E+02,-9.17483E-02) -- (axis cs:1.80040E+02,9.17485E-02) -- cycle ;




\draw [ecliptics-full] (axis cs:3.58125E+02,-7.03651E-01) -- (axis cs:3.58205E+02,-8.87166E-01)  -- (axis cs:3.57287E+02,-1.28463E+00) -- (axis cs:3.57207E+02,-1.10109E+00) -- cycle  ;
\draw [ecliptics-full] (axis cs:3.56289E+02,-1.49822E+00) -- (axis cs:3.56369E+02,-1.68179E+00)  -- (axis cs:3.55450E+02,-2.07854E+00) -- (axis cs:3.55371E+02,-1.89493E+00) -- cycle  ;
\draw [ecliptics-full] (axis cs:3.54452E+02,-2.29113E+00) -- (axis cs:3.54532E+02,-2.47479E+00)  -- (axis cs:3.53612E+02,-2.87043E+00) -- (axis cs:3.53533E+02,-2.68672E+00) -- cycle  ;
\draw [ecliptics-full] (axis cs:3.52613E+02,-3.08158E+00) -- (axis cs:3.52692E+02,-3.26536E+00)  -- (axis cs:3.51771E+02,-3.65948E+00) -- (axis cs:3.51693E+02,-3.47563E+00) -- cycle  ;
\draw [ecliptics-full] (axis cs:3.50771E+02,-3.86875E+00) -- (axis cs:3.50850E+02,-4.05268E+00)  -- (axis cs:3.49927E+02,-4.44487E+00) -- (axis cs:3.49849E+02,-4.26084E+00) -- cycle  ;
\draw [ecliptics-full] (axis cs:3.48926E+02,-4.65180E+00) -- (axis cs:3.49004E+02,-4.83593E+00)  -- (axis cs:3.48080E+02,-5.22577E+00) -- (axis cs:3.48001E+02,-5.04154E+00) -- cycle  ;
\draw [ecliptics-full] (axis cs:3.47076E+02,-5.42993E+00) -- (axis cs:3.47154E+02,-5.61429E+00)  -- (axis cs:3.46227E+02,-6.00137E+00) -- (axis cs:3.46150E+02,-5.81689E+00) -- cycle  ;
\draw [ecliptics-full] (axis cs:3.45222E+02,-6.20231E+00) -- (axis cs:3.45299E+02,-6.38692E+00)  -- (axis cs:3.44370E+02,-6.77084E+00) -- (axis cs:3.44292E+02,-6.58609E+00) -- cycle  ;
\draw [ecliptics-full] (axis cs:3.43362E+02,-6.96811E+00) -- (axis cs:3.43439E+02,-7.15301E+00)  -- (axis cs:3.42506E+02,-7.53334E+00) -- (axis cs:3.42430E+02,-7.34829E+00) -- cycle  ;
\draw [ecliptics-full] (axis cs:3.41496E+02,-7.72651E+00) -- (axis cs:3.41572E+02,-7.91173E+00)  -- (axis cs:3.40636E+02,-8.28806E+00) -- (axis cs:3.40560E+02,-8.10267E+00) -- cycle  ;
\draw [ecliptics-full] (axis cs:3.39623E+02,-8.47667E+00) -- (axis cs:3.39699E+02,-8.66224E+00)  -- (axis cs:3.38759E+02,-9.03415E+00) -- (axis cs:3.38684E+02,-8.84840E+00) -- cycle  ;
\draw [ecliptics-full] (axis cs:3.37743E+02,-9.21776E+00) -- (axis cs:3.37818E+02,-9.40370E+00)  -- (axis cs:3.36874E+02,-9.77078E+00) -- (axis cs:3.36800E+02,-9.58464E+00) -- cycle  ;
\draw [ecliptics-full] (axis cs:3.35855E+02,-9.94894E+00) -- (axis cs:3.35929E+02,-1.01353E+01)  -- (axis cs:3.34981E+02,-1.04971E+01) -- (axis cs:3.34908E+02,-1.03105E+01) -- cycle  ;
\draw [ecliptics-full] (axis cs:3.33959E+02,-1.06694E+01) -- (axis cs:3.34032E+02,-1.08562E+01)  -- (axis cs:3.33080E+02,-1.12123E+01) -- (axis cs:3.33007E+02,-1.10253E+01) -- cycle  ;
\draw [ecliptics-full] (axis cs:3.32054E+02,-1.13782E+01) -- (axis cs:3.32125E+02,-1.15655E+01)  -- (axis cs:3.31169E+02,-1.19155E+01) -- (axis cs:3.31097E+02,-1.17280E+01) -- cycle  ;
\draw [ecliptics-full] (axis cs:3.30139E+02,-1.20746E+01) -- (axis cs:3.30209E+02,-1.22623E+01)  -- (axis cs:3.29248E+02,-1.26059E+01) -- (axis cs:3.29178E+02,-1.24179E+01) -- cycle  ;
\draw [ecliptics-full] (axis cs:3.28214E+02,-1.27578E+01) -- (axis cs:3.28283E+02,-1.29460E+01)  -- (axis cs:3.27316E+02,-1.32826E+01) -- (axis cs:3.27248E+02,-1.30941E+01) -- cycle  ;
\draw [ecliptics-full] (axis cs:3.26279E+02,-1.34268E+01) -- (axis cs:3.26347E+02,-1.36155E+01)  -- (axis cs:3.25375E+02,-1.39448E+01) -- (axis cs:3.25307E+02,-1.37558E+01) -- cycle  ;
\draw [ecliptics-full] (axis cs:3.24333E+02,-1.40809E+01) -- (axis cs:3.24400E+02,-1.42701E+01)  -- (axis cs:3.23422E+02,-1.45916E+01) -- (axis cs:3.23356E+02,-1.44021E+01) -- cycle  ;
\draw [ecliptics-full] (axis cs:3.22376E+02,-1.47192E+01) -- (axis cs:3.22441E+02,-1.49090E+01)  -- (axis cs:3.21458E+02,-1.52222E+01) -- (axis cs:3.21393E+02,-1.50321E+01) -- cycle  ;
\draw [ecliptics-full] (axis cs:3.20408E+02,-1.53408E+01) -- (axis cs:3.20471E+02,-1.55312E+01)  -- (axis cs:3.19482E+02,-1.58358E+01) -- (axis cs:3.19419E+02,-1.56451E+01) -- cycle  ;
\draw [ecliptics-full] (axis cs:3.18428E+02,-1.59450E+01) -- (axis cs:3.18490E+02,-1.61359E+01)  -- (axis cs:3.17495E+02,-1.64315E+01) -- (axis cs:3.17434E+02,-1.62403E+01) -- cycle  ;
\draw [ecliptics-full] (axis cs:3.16436E+02,-1.65309E+01) -- (axis cs:3.16496E+02,-1.67224E+01)  -- (axis cs:3.15495E+02,-1.70085E+01) -- (axis cs:3.15436E+02,-1.68167E+01) -- cycle  ;
\draw [ecliptics-full] (axis cs:3.14433E+02,-1.70976E+01) -- (axis cs:3.14491E+02,-1.72897E+01)  -- (axis cs:3.13484E+02,-1.75660E+01) -- (axis cs:3.13426E+02,-1.73736E+01) -- cycle  ;
\draw [ecliptics-full] (axis cs:3.12417E+02,-1.76445E+01) -- (axis cs:3.12473E+02,-1.78371E+01)  -- (axis cs:3.11460E+02,-1.81031E+01) -- (axis cs:3.11404E+02,-1.79102E+01) -- cycle  ;
\draw [ecliptics-full] (axis cs:3.10389E+02,-1.81706E+01) -- (axis cs:3.10443E+02,-1.83638E+01)  -- (axis cs:3.09424E+02,-1.86192E+01) -- (axis cs:3.09371E+02,-1.84256E+01) -- cycle  ;
\draw [ecliptics-full] (axis cs:3.08349E+02,-1.86752E+01) -- (axis cs:3.08401E+02,-1.88690E+01)  -- (axis cs:3.07376E+02,-1.91133E+01) -- (axis cs:3.07325E+02,-1.89192E+01) -- cycle  ;
\draw [ecliptics-full] (axis cs:3.06297E+02,-1.91575E+01) -- (axis cs:3.06347E+02,-1.93519E+01)  -- (axis cs:3.05316E+02,-1.95848E+01) -- (axis cs:3.05267E+02,-1.93901E+01) -- cycle  ;
\draw [ecliptics-full] (axis cs:3.04234E+02,-1.96168E+01) -- (axis cs:3.04281E+02,-1.98118E+01)  -- (axis cs:3.03244E+02,-2.00328E+01) -- (axis cs:3.03198E+02,-1.98376E+01) -- cycle  ;
\draw [ecliptics-full] (axis cs:3.02159E+02,-2.00524E+01) -- (axis cs:3.02204E+02,-2.02479E+01)  -- (axis cs:3.01161E+02,-2.04568E+01) -- (axis cs:3.01117E+02,-2.02610E+01) -- cycle  ;
\draw [ecliptics-full] (axis cs:3.00072E+02,-2.04635E+01) -- (axis cs:3.00115E+02,-2.06595E+01)  -- (axis cs:2.99066E+02,-2.08559E+01) -- (axis cs:2.99025E+02,-2.06597E+01) -- cycle  ;
\draw [ecliptics-full] (axis cs:2.97975E+02,-2.08495E+01) -- (axis cs:2.98015E+02,-2.10459E+01)  -- (axis cs:2.96961E+02,-2.12295E+01) -- (axis cs:2.96922E+02,-2.10328E+01) -- cycle  ;
\draw [ecliptics-full] (axis cs:2.95867E+02,-2.12096E+01) -- (axis cs:2.95905E+02,-2.14066E+01)  -- (axis cs:2.94846E+02,-2.15771E+01) -- (axis cs:2.94810E+02,-2.13799E+01) -- cycle  ;
\draw [ecliptics-full] (axis cs:2.93750E+02,-2.15434E+01) -- (axis cs:2.93784E+02,-2.17408E+01)  -- (axis cs:2.92720E+02,-2.18979E+01) -- (axis cs:2.92687E+02,-2.17002E+01) -- cycle  ;
\draw [ecliptics-full] (axis cs:2.91623E+02,-2.18502E+01) -- (axis cs:2.91654E+02,-2.20481E+01)  -- (axis cs:2.90586E+02,-2.21914E+01) -- (axis cs:2.90556E+02,-2.19934E+01) -- cycle  ;
\draw [ecliptics-full] (axis cs:2.89487E+02,-2.21296E+01) -- (axis cs:2.89516E+02,-2.23278E+01)  -- (axis cs:2.88443E+02,-2.24572E+01) -- (axis cs:2.88416E+02,-2.22587E+01) -- cycle  ;
\draw [ecliptics-full] (axis cs:2.87343E+02,-2.23809E+01) -- (axis cs:2.87369E+02,-2.25795E+01)  -- (axis cs:2.86292E+02,-2.26946E+01) -- (axis cs:2.86268E+02,-2.24959E+01) -- cycle  ;
\draw [ecliptics-full] (axis cs:2.85192E+02,-2.26037E+01) -- (axis cs:2.85214E+02,-2.28026E+01)  -- (axis cs:2.84135E+02,-2.29034E+01) -- (axis cs:2.84114E+02,-2.27043E+01) -- cycle  ;
\draw [ecliptics-full] (axis cs:2.83034E+02,-2.27977E+01) -- (axis cs:2.83054E+02,-2.29969E+01)  -- (axis cs:2.81971E+02,-2.30830E+01) -- (axis cs:2.81953E+02,-2.28837E+01) -- cycle  ;
\draw [ecliptics-full] (axis cs:2.80871E+02,-2.29624E+01) -- (axis cs:2.80887E+02,-2.31619E+01)  -- (axis cs:2.79802E+02,-2.32333E+01) -- (axis cs:2.79787E+02,-2.30338E+01) -- cycle  ;
\draw [ecliptics-full] (axis cs:2.78702E+02,-2.30977E+01) -- (axis cs:2.78715E+02,-2.32973E+01)  -- (axis cs:2.77628E+02,-2.33538E+01) -- (axis cs:2.77617E+02,-2.31541E+01) -- cycle  ;
\draw [ecliptics-full] (axis cs:2.76530E+02,-2.32031E+01) -- (axis cs:2.76540E+02,-2.34029E+01)  -- (axis cs:2.75451E+02,-2.34444E+01) -- (axis cs:2.75443E+02,-2.32446E+01) -- cycle  ;
\draw [ecliptics-full] (axis cs:2.74355E+02,-2.32785E+01) -- (axis cs:2.74362E+02,-2.34785E+01)  -- (axis cs:2.73272E+02,-2.35049E+01) -- (axis cs:2.73267E+02,-2.33050E+01) -- cycle  ;
\draw [ecliptics-full] (axis cs:2.72178E+02,-2.33239E+01) -- (axis cs:2.72181E+02,-2.35239E+01)  -- (axis cs:2.71091E+02,-2.35352E+01) -- (axis cs:2.71089E+02,-2.33352E+01) -- cycle  ;
\draw [ecliptics-full] (axis cs:2.70000E+02,-2.33390E+01) -- (axis cs:2.70000E+02,-2.35390E+01)  -- (axis cs:2.68909E+02,-2.35352E+01) -- (axis cs:2.68911E+02,-2.33352E+01) -- cycle  ;
\draw [ecliptics-full] (axis cs:2.67822E+02,-2.33239E+01) -- (axis cs:2.67819E+02,-2.35239E+01)  -- (axis cs:2.66728E+02,-2.35049E+01) -- (axis cs:2.66733E+02,-2.33050E+01) -- cycle  ;
\draw [ecliptics-full] (axis cs:2.65645E+02,-2.32785E+01) -- (axis cs:2.65638E+02,-2.34785E+01)  -- (axis cs:2.64549E+02,-2.34444E+01) -- (axis cs:2.64557E+02,-2.32446E+01) -- cycle  ;
\draw [ecliptics-full] (axis cs:2.63470E+02,-2.32031E+01) -- (axis cs:2.63460E+02,-2.34029E+01)  -- (axis cs:2.62372E+02,-2.33538E+01) -- (axis cs:2.62383E+02,-2.31541E+01) -- cycle  ;
\draw [ecliptics-full] (axis cs:2.61298E+02,-2.30977E+01) -- (axis cs:2.61285E+02,-2.32973E+01)  -- (axis cs:2.60198E+02,-2.32333E+01) -- (axis cs:2.60213E+02,-2.30338E+01) -- cycle  ;
\draw [ecliptics-full] (axis cs:2.59129E+02,-2.29624E+01) -- (axis cs:2.59113E+02,-2.31619E+01)  -- (axis cs:2.58029E+02,-2.30830E+01) -- (axis cs:2.58047E+02,-2.28837E+01) -- cycle  ;
\draw [ecliptics-full] (axis cs:2.56966E+02,-2.27977E+01) -- (axis cs:2.56946E+02,-2.29969E+01)  -- (axis cs:2.55865E+02,-2.29034E+01) -- (axis cs:2.55886E+02,-2.27043E+01) -- cycle  ;
\draw [ecliptics-full] (axis cs:2.54808E+02,-2.26037E+01) -- (axis cs:2.54786E+02,-2.28026E+01)  -- (axis cs:2.53708E+02,-2.26946E+01) -- (axis cs:2.53732E+02,-2.24959E+01) -- cycle  ;
\draw [ecliptics-full] (axis cs:2.52657E+02,-2.23809E+01) -- (axis cs:2.52631E+02,-2.25795E+01)  -- (axis cs:2.51557E+02,-2.24572E+01) -- (axis cs:2.51584E+02,-2.22587E+01) -- cycle  ;
\draw [ecliptics-full] (axis cs:2.50513E+02,-2.21296E+01) -- (axis cs:2.50484E+02,-2.23278E+01)  -- (axis cs:2.49414E+02,-2.21914E+01) -- (axis cs:2.49444E+02,-2.19934E+01) -- cycle  ;
\draw [ecliptics-full] (axis cs:2.48377E+02,-2.18502E+01) -- (axis cs:2.48346E+02,-2.20481E+01)  -- (axis cs:2.47280E+02,-2.18979E+01) -- (axis cs:2.47313E+02,-2.17002E+01) -- cycle  ;
\draw [ecliptics-full] (axis cs:2.46250E+02,-2.15434E+01) -- (axis cs:2.46216E+02,-2.17408E+01)  -- (axis cs:2.45154E+02,-2.15771E+01) -- (axis cs:2.45190E+02,-2.13799E+01) -- cycle  ;
\draw [ecliptics-full] (axis cs:2.44133E+02,-2.12096E+01) -- (axis cs:2.44095E+02,-2.14066E+01)  -- (axis cs:2.43039E+02,-2.12295E+01) -- (axis cs:2.43078E+02,-2.10328E+01) -- cycle  ;
\draw [ecliptics-full] (axis cs:2.42025E+02,-2.08495E+01) -- (axis cs:2.41985E+02,-2.10459E+01)  -- (axis cs:2.40934E+02,-2.08559E+01) -- (axis cs:2.40975E+02,-2.06597E+01) -- cycle  ;
\draw [ecliptics-full] (axis cs:2.39928E+02,-2.04635E+01) -- (axis cs:2.39885E+02,-2.06595E+01)  -- (axis cs:2.38839E+02,-2.04568E+01) -- (axis cs:2.38883E+02,-2.02610E+01) -- cycle  ;
\draw [ecliptics-full] (axis cs:2.37841E+02,-2.00524E+01) -- (axis cs:2.37796E+02,-2.02479E+01)  -- (axis cs:2.36756E+02,-2.00328E+01) -- (axis cs:2.36802E+02,-1.98376E+01) -- cycle  ;
\draw [ecliptics-full] (axis cs:2.35766E+02,-1.96168E+01) -- (axis cs:2.35719E+02,-1.98118E+01)  -- (axis cs:2.34684E+02,-1.95848E+01) -- (axis cs:2.34733E+02,-1.93901E+01) -- cycle  ;
\draw [ecliptics-full] (axis cs:2.33703E+02,-1.91575E+01) -- (axis cs:2.33653E+02,-1.93519E+01)  -- (axis cs:2.32624E+02,-1.91133E+01) -- (axis cs:2.32675E+02,-1.89192E+01) -- cycle  ;
\draw [ecliptics-full] (axis cs:2.31651E+02,-1.86752E+01) -- (axis cs:2.31599E+02,-1.88690E+01)  -- (axis cs:2.30576E+02,-1.86192E+01) -- (axis cs:2.30629E+02,-1.84256E+01) -- cycle  ;
\draw [ecliptics-full] (axis cs:2.29611E+02,-1.81706E+01) -- (axis cs:2.29557E+02,-1.83638E+01)  -- (axis cs:2.28540E+02,-1.81031E+01) -- (axis cs:2.28596E+02,-1.79102E+01) -- cycle  ;
\draw [ecliptics-full] (axis cs:2.27583E+02,-1.76445E+01) -- (axis cs:2.27527E+02,-1.78371E+01)  -- (axis cs:2.26516E+02,-1.75660E+01) -- (axis cs:2.26574E+02,-1.73736E+01) -- cycle  ;
\draw [ecliptics-full] (axis cs:2.25567E+02,-1.70976E+01) -- (axis cs:2.25509E+02,-1.72897E+01)  -- (axis cs:2.24505E+02,-1.70085E+01) -- (axis cs:2.24564E+02,-1.68167E+01) -- cycle  ;
\draw [ecliptics-full] (axis cs:2.23564E+02,-1.65309E+01) -- (axis cs:2.23504E+02,-1.67224E+01)  -- (axis cs:2.22505E+02,-1.64315E+01) -- (axis cs:2.22566E+02,-1.62403E+01) -- cycle  ;
\draw [ecliptics-full] (axis cs:2.21572E+02,-1.59450E+01) -- (axis cs:2.21510E+02,-1.61359E+01)  -- (axis cs:2.20518E+02,-1.58358E+01) -- (axis cs:2.20581E+02,-1.56451E+01) -- cycle  ;
\draw [ecliptics-full] (axis cs:2.19592E+02,-1.53408E+01) -- (axis cs:2.19529E+02,-1.55312E+01)  -- (axis cs:2.18542E+02,-1.52222E+01) -- (axis cs:2.18607E+02,-1.50321E+01) -- cycle  ;
\draw [ecliptics-full] (axis cs:2.17624E+02,-1.47192E+01) -- (axis cs:2.17559E+02,-1.49090E+01)  -- (axis cs:2.16578E+02,-1.45916E+01) -- (axis cs:2.16644E+02,-1.44021E+01) -- cycle  ;
\draw [ecliptics-full] (axis cs:2.15667E+02,-1.40809E+01) -- (axis cs:2.15600E+02,-1.42701E+01)  -- (axis cs:2.14625E+02,-1.39448E+01) -- (axis cs:2.14693E+02,-1.37558E+01) -- cycle  ;
\draw [ecliptics-full] (axis cs:2.13721E+02,-1.34268E+01) -- (axis cs:2.13653E+02,-1.36155E+01)  -- (axis cs:2.12683E+02,-1.32826E+01) -- (axis cs:2.12752E+02,-1.30941E+01) -- cycle  ;
\draw [ecliptics-full] (axis cs:2.11786E+02,-1.27578E+01) -- (axis cs:2.11717E+02,-1.29460E+01)  -- (axis cs:2.10752E+02,-1.26059E+01) -- (axis cs:2.10822E+02,-1.24179E+01) -- cycle  ;
\draw [ecliptics-full] (axis cs:2.09861E+02,-1.20746E+01) -- (axis cs:2.09791E+02,-1.22623E+01)  -- (axis cs:2.08831E+02,-1.19155E+01) -- (axis cs:2.08903E+02,-1.17280E+01) -- cycle  ;
\draw [ecliptics-full] (axis cs:2.07946E+02,-1.13782E+01) -- (axis cs:2.07875E+02,-1.15655E+01)  -- (axis cs:2.06920E+02,-1.12123E+01) -- (axis cs:2.06993E+02,-1.10253E+01) -- cycle  ;
\draw [ecliptics-full] (axis cs:2.06041E+02,-1.06694E+01) -- (axis cs:2.05968E+02,-1.08562E+01)  -- (axis cs:2.05019E+02,-1.04971E+01) -- (axis cs:2.05092E+02,-1.03105E+01) -- cycle  ;
\draw [ecliptics-full] (axis cs:2.04145E+02,-9.94894E+00) -- (axis cs:2.04071E+02,-1.01353E+01)  -- (axis cs:2.03126E+02,-9.77078E+00) -- (axis cs:2.03200E+02,-9.58464E+00) -- cycle  ;
\draw [ecliptics-full] (axis cs:2.02257E+02,-9.21776E+00) -- (axis cs:2.02182E+02,-9.40370E+00)  -- (axis cs:2.01241E+02,-9.03415E+00) -- (axis cs:2.01316E+02,-8.84840E+00) -- cycle  ;
\draw [ecliptics-full] (axis cs:2.00377E+02,-8.47667E+00) -- (axis cs:2.00301E+02,-8.66224E+00)  -- (axis cs:1.99364E+02,-8.28806E+00) -- (axis cs:1.99440E+02,-8.10267E+00) -- cycle  ;
\draw [ecliptics-full] (axis cs:1.98504E+02,-7.72651E+00) -- (axis cs:1.98428E+02,-7.91173E+00)  -- (axis cs:1.97494E+02,-7.53334E+00) -- (axis cs:1.97570E+02,-7.34829E+00) -- cycle  ;
\draw [ecliptics-full] (axis cs:1.96638E+02,-6.96811E+00) -- (axis cs:1.96561E+02,-7.15301E+00)  -- (axis cs:1.95630E+02,-6.77084E+00) -- (axis cs:1.95708E+02,-6.58609E+00) -- cycle  ;
\draw [ecliptics-full] (axis cs:1.94778E+02,-6.20231E+00) -- (axis cs:1.94701E+02,-6.38692E+00)  -- (axis cs:1.93773E+02,-6.00137E+00) -- (axis cs:1.93850E+02,-5.81689E+00) -- cycle  ;
\draw [ecliptics-full] (axis cs:1.92924E+02,-5.42993E+00) -- (axis cs:1.92846E+02,-5.61429E+00)  -- (axis cs:1.91920E+02,-5.22577E+00) -- (axis cs:1.91999E+02,-5.04154E+00) -- cycle  ;
\draw [ecliptics-full] (axis cs:1.91074E+02,-4.65180E+00) -- (axis cs:1.90996E+02,-4.83593E+00)  -- (axis cs:1.90073E+02,-4.44487E+00) -- (axis cs:1.90151E+02,-4.26084E+00) -- cycle  ;
\draw [ecliptics-full] (axis cs:1.89229E+02,-3.86875E+00) -- (axis cs:1.89150E+02,-4.05268E+00)  -- (axis cs:1.88229E+02,-3.65948E+00) -- (axis cs:1.88308E+02,-3.47563E+00) -- cycle  ;
\draw [ecliptics-full] (axis cs:1.87387E+02,-3.08158E+00) -- (axis cs:1.87308E+02,-3.26536E+00)  -- (axis cs:1.86388E+02,-2.87043E+00) -- (axis cs:1.86467E+02,-2.68672E+00) -- cycle  ;
\draw [ecliptics-full] (axis cs:1.85548E+02,-2.29113E+00) -- (axis cs:1.85468E+02,-2.47479E+00)  -- (axis cs:1.84550E+02,-2.07854E+00) -- (axis cs:1.84629E+02,-1.89493E+00) -- cycle  ;
\draw [ecliptics-full] (axis cs:1.83711E+02,-1.49822E+00) -- (axis cs:1.83631E+02,-1.68179E+00)  -- (axis cs:1.82713E+02,-1.28463E+00) -- (axis cs:1.82793E+02,-1.10109E+00) -- cycle  ;
\draw [ecliptics-full] (axis cs:1.81875E+02,-7.03652E-01) -- (axis cs:1.81795E+02,-8.87166E-01)  -- (axis cs:1.80878E+02,-4.89506E-01) -- (axis cs:1.80957E+02,-3.06005E-01) -- cycle  ;

\draw [ecliptics-empty] (axis cs:3.59043E+02,-3.06005E-01) -- (axis cs:3.59122E+02,-4.89506E-01)  -- (axis cs:3.58205E+02,-8.87166E-01) -- (axis cs:3.58125E+02,-7.03651E-01) -- cycle  ;
\draw [ecliptics-empty] (axis cs:3.57207E+02,-1.10109E+00) -- (axis cs:3.57287E+02,-1.28463E+00)  -- (axis cs:3.56369E+02,-1.68179E+00) -- (axis cs:3.56289E+02,-1.49822E+00) -- cycle  ;
\draw [ecliptics-empty] (axis cs:3.55371E+02,-1.89493E+00) -- (axis cs:3.55450E+02,-2.07854E+00)  -- (axis cs:3.54532E+02,-2.47479E+00) -- (axis cs:3.54452E+02,-2.29113E+00) -- cycle  ;
\draw [ecliptics-empty] (axis cs:3.53533E+02,-2.68672E+00) -- (axis cs:3.53612E+02,-2.87043E+00)  -- (axis cs:3.52692E+02,-3.26536E+00) -- (axis cs:3.52613E+02,-3.08158E+00) -- cycle  ;
\draw [ecliptics-empty] (axis cs:3.51693E+02,-3.47563E+00) -- (axis cs:3.51771E+02,-3.65948E+00)  -- (axis cs:3.50850E+02,-4.05268E+00) -- (axis cs:3.50771E+02,-3.86875E+00) -- cycle  ;
\draw [ecliptics-empty] (axis cs:3.49849E+02,-4.26084E+00) -- (axis cs:3.49927E+02,-4.44487E+00)  -- (axis cs:3.49004E+02,-4.83593E+00) -- (axis cs:3.48926E+02,-4.65180E+00) -- cycle  ;
\draw [ecliptics-empty] (axis cs:3.48001E+02,-5.04154E+00) -- (axis cs:3.48080E+02,-5.22577E+00)  -- (axis cs:3.47154E+02,-5.61429E+00) -- (axis cs:3.47076E+02,-5.42993E+00) -- cycle  ;
\draw [ecliptics-empty] (axis cs:3.46150E+02,-5.81689E+00) -- (axis cs:3.46227E+02,-6.00137E+00)  -- (axis cs:3.45299E+02,-6.38692E+00) -- (axis cs:3.45222E+02,-6.20231E+00) -- cycle  ;
\draw [ecliptics-empty] (axis cs:3.44292E+02,-6.58609E+00) -- (axis cs:3.44370E+02,-6.77084E+00)  -- (axis cs:3.43439E+02,-7.15301E+00) -- (axis cs:3.43362E+02,-6.96811E+00) -- cycle  ;
\draw [ecliptics-empty] (axis cs:3.42430E+02,-7.34829E+00) -- (axis cs:3.42506E+02,-7.53334E+00)  -- (axis cs:3.41572E+02,-7.91173E+00) -- (axis cs:3.41496E+02,-7.72651E+00) -- cycle  ;
\draw [ecliptics-empty] (axis cs:3.40560E+02,-8.10267E+00) -- (axis cs:3.40636E+02,-8.28806E+00)  -- (axis cs:3.39699E+02,-8.66224E+00) -- (axis cs:3.39623E+02,-8.47667E+00) -- cycle  ;
\draw [ecliptics-empty] (axis cs:3.38684E+02,-8.84840E+00) -- (axis cs:3.38759E+02,-9.03415E+00)  -- (axis cs:3.37818E+02,-9.40370E+00) -- (axis cs:3.37743E+02,-9.21776E+00) -- cycle  ;
\draw [ecliptics-empty] (axis cs:3.36800E+02,-9.58464E+00) -- (axis cs:3.36874E+02,-9.77078E+00)  -- (axis cs:3.35929E+02,-1.01353E+01) -- (axis cs:3.35855E+02,-9.94894E+00) -- cycle  ;
\draw [ecliptics-empty] (axis cs:3.34908E+02,-1.03105E+01) -- (axis cs:3.34981E+02,-1.04971E+01)  -- (axis cs:3.34032E+02,-1.08562E+01) -- (axis cs:3.33959E+02,-1.06694E+01) -- cycle  ;
\draw [ecliptics-empty] (axis cs:3.33007E+02,-1.10253E+01) -- (axis cs:3.33080E+02,-1.12123E+01)  -- (axis cs:3.32125E+02,-1.15655E+01) -- (axis cs:3.32054E+02,-1.13782E+01) -- cycle  ;
\draw [ecliptics-empty] (axis cs:3.31097E+02,-1.17280E+01) -- (axis cs:3.31169E+02,-1.19155E+01)  -- (axis cs:3.30209E+02,-1.22623E+01) -- (axis cs:3.30139E+02,-1.20746E+01) -- cycle  ;
\draw [ecliptics-empty] (axis cs:3.29178E+02,-1.24179E+01) -- (axis cs:3.29248E+02,-1.26059E+01)  -- (axis cs:3.28283E+02,-1.29460E+01) -- (axis cs:3.28214E+02,-1.27578E+01) -- cycle  ;
\draw [ecliptics-empty] (axis cs:3.27248E+02,-1.30941E+01) -- (axis cs:3.27316E+02,-1.32826E+01)  -- (axis cs:3.26347E+02,-1.36155E+01) -- (axis cs:3.26279E+02,-1.34268E+01) -- cycle  ;
\draw [ecliptics-empty] (axis cs:3.25307E+02,-1.37558E+01) -- (axis cs:3.25375E+02,-1.39448E+01)  -- (axis cs:3.24400E+02,-1.42701E+01) -- (axis cs:3.24333E+02,-1.40809E+01) -- cycle  ;
\draw [ecliptics-empty] (axis cs:3.23356E+02,-1.44021E+01) -- (axis cs:3.23422E+02,-1.45916E+01)  -- (axis cs:3.22441E+02,-1.49090E+01) -- (axis cs:3.22376E+02,-1.47192E+01) -- cycle  ;
\draw [ecliptics-empty] (axis cs:3.21393E+02,-1.50321E+01) -- (axis cs:3.21458E+02,-1.52222E+01)  -- (axis cs:3.20471E+02,-1.55312E+01) -- (axis cs:3.20408E+02,-1.53408E+01) -- cycle  ;
\draw [ecliptics-empty] (axis cs:3.19419E+02,-1.56451E+01) -- (axis cs:3.19482E+02,-1.58358E+01)  -- (axis cs:3.18490E+02,-1.61359E+01) -- (axis cs:3.18428E+02,-1.59450E+01) -- cycle  ;
\draw [ecliptics-empty] (axis cs:3.17434E+02,-1.62403E+01) -- (axis cs:3.17495E+02,-1.64315E+01)  -- (axis cs:3.16496E+02,-1.67224E+01) -- (axis cs:3.16436E+02,-1.65309E+01) -- cycle  ;
\draw [ecliptics-empty] (axis cs:3.15436E+02,-1.68167E+01) -- (axis cs:3.15495E+02,-1.70085E+01)  -- (axis cs:3.14491E+02,-1.72897E+01) -- (axis cs:3.14433E+02,-1.70976E+01) -- cycle  ;
\draw [ecliptics-empty] (axis cs:3.13426E+02,-1.73736E+01) -- (axis cs:3.13484E+02,-1.75660E+01)  -- (axis cs:3.12473E+02,-1.78371E+01) -- (axis cs:3.12417E+02,-1.76445E+01) -- cycle  ;
\draw [ecliptics-empty] (axis cs:3.11404E+02,-1.79102E+01) -- (axis cs:3.11460E+02,-1.81031E+01)  -- (axis cs:3.10443E+02,-1.83638E+01) -- (axis cs:3.10389E+02,-1.81706E+01) -- cycle  ;
\draw [ecliptics-empty] (axis cs:3.09371E+02,-1.84256E+01) -- (axis cs:3.09424E+02,-1.86192E+01)  -- (axis cs:3.08401E+02,-1.88690E+01) -- (axis cs:3.08349E+02,-1.86752E+01) -- cycle  ;
\draw [ecliptics-empty] (axis cs:3.07325E+02,-1.89192E+01) -- (axis cs:3.07376E+02,-1.91133E+01)  -- (axis cs:3.06347E+02,-1.93519E+01) -- (axis cs:3.06297E+02,-1.91575E+01) -- cycle  ;
\draw [ecliptics-empty] (axis cs:3.05267E+02,-1.93901E+01) -- (axis cs:3.05316E+02,-1.95848E+01)  -- (axis cs:3.04281E+02,-1.98118E+01) -- (axis cs:3.04234E+02,-1.96168E+01) -- cycle  ;
\draw [ecliptics-empty] (axis cs:3.03198E+02,-1.98376E+01) -- (axis cs:3.03244E+02,-2.00328E+01)  -- (axis cs:3.02204E+02,-2.02479E+01) -- (axis cs:3.02159E+02,-2.00524E+01) -- cycle  ;
\draw [ecliptics-empty] (axis cs:3.01117E+02,-2.02610E+01) -- (axis cs:3.01161E+02,-2.04568E+01)  -- (axis cs:3.00115E+02,-2.06595E+01) -- (axis cs:3.00072E+02,-2.04635E+01) -- cycle  ;
\draw [ecliptics-empty] (axis cs:2.99025E+02,-2.06597E+01) -- (axis cs:2.99066E+02,-2.08559E+01)  -- (axis cs:2.98015E+02,-2.10459E+01) -- (axis cs:2.97975E+02,-2.08495E+01) -- cycle  ;
\draw [ecliptics-empty] (axis cs:2.96922E+02,-2.10328E+01) -- (axis cs:2.96961E+02,-2.12295E+01)  -- (axis cs:2.95905E+02,-2.14066E+01) -- (axis cs:2.95867E+02,-2.12096E+01) -- cycle  ;
\draw [ecliptics-empty] (axis cs:2.94810E+02,-2.13799E+01) -- (axis cs:2.94846E+02,-2.15771E+01)  -- (axis cs:2.93784E+02,-2.17408E+01) -- (axis cs:2.93750E+02,-2.15434E+01) -- cycle  ;
\draw [ecliptics-empty] (axis cs:2.92687E+02,-2.17002E+01) -- (axis cs:2.92720E+02,-2.18979E+01)  -- (axis cs:2.91654E+02,-2.20481E+01) -- (axis cs:2.91623E+02,-2.18502E+01) -- cycle  ;
\draw [ecliptics-empty] (axis cs:2.90556E+02,-2.19934E+01) -- (axis cs:2.90586E+02,-2.21914E+01)  -- (axis cs:2.89516E+02,-2.23278E+01) -- (axis cs:2.89487E+02,-2.21296E+01) -- cycle  ;
\draw [ecliptics-empty] (axis cs:2.88416E+02,-2.22587E+01) -- (axis cs:2.88443E+02,-2.24572E+01)  -- (axis cs:2.87369E+02,-2.25795E+01) -- (axis cs:2.87343E+02,-2.23809E+01) -- cycle  ;
\draw [ecliptics-empty] (axis cs:2.86268E+02,-2.24959E+01) -- (axis cs:2.86292E+02,-2.26946E+01)  -- (axis cs:2.85214E+02,-2.28026E+01) -- (axis cs:2.85192E+02,-2.26037E+01) -- cycle  ;
\draw [ecliptics-empty] (axis cs:2.84114E+02,-2.27043E+01) -- (axis cs:2.84135E+02,-2.29034E+01)  -- (axis cs:2.83054E+02,-2.29969E+01) -- (axis cs:2.83034E+02,-2.27977E+01) -- cycle  ;
\draw [ecliptics-empty] (axis cs:2.81953E+02,-2.28837E+01) -- (axis cs:2.81971E+02,-2.30830E+01)  -- (axis cs:2.80887E+02,-2.31619E+01) -- (axis cs:2.80871E+02,-2.29624E+01) -- cycle  ;
\draw [ecliptics-empty] (axis cs:2.79787E+02,-2.30338E+01) -- (axis cs:2.79802E+02,-2.32333E+01)  -- (axis cs:2.78715E+02,-2.32973E+01) -- (axis cs:2.78702E+02,-2.30977E+01) -- cycle  ;
\draw [ecliptics-empty] (axis cs:2.77617E+02,-2.31541E+01) -- (axis cs:2.77628E+02,-2.33538E+01)  -- (axis cs:2.76540E+02,-2.34029E+01) -- (axis cs:2.76530E+02,-2.32031E+01) -- cycle  ;
\draw [ecliptics-empty] (axis cs:2.75443E+02,-2.32446E+01) -- (axis cs:2.75451E+02,-2.34444E+01)  -- (axis cs:2.74362E+02,-2.34785E+01) -- (axis cs:2.74355E+02,-2.32785E+01) -- cycle  ;
\draw [ecliptics-empty] (axis cs:2.73267E+02,-2.33050E+01) -- (axis cs:2.73272E+02,-2.35049E+01)  -- (axis cs:2.72181E+02,-2.35239E+01) -- (axis cs:2.72178E+02,-2.33239E+01) -- cycle  ;
\draw [ecliptics-empty] (axis cs:2.71089E+02,-2.33352E+01) -- (axis cs:2.71091E+02,-2.35352E+01)  -- (axis cs:2.70000E+02,-2.35390E+01) -- (axis cs:2.70000E+02,-2.33390E+01) -- cycle  ;
\draw [ecliptics-empty] (axis cs:2.68911E+02,-2.33352E+01) -- (axis cs:2.68909E+02,-2.35352E+01)  -- (axis cs:2.67819E+02,-2.35239E+01) -- (axis cs:2.67822E+02,-2.33239E+01) -- cycle  ;
\draw [ecliptics-empty] (axis cs:2.66733E+02,-2.33050E+01) -- (axis cs:2.66728E+02,-2.35049E+01)  -- (axis cs:2.65638E+02,-2.34785E+01) -- (axis cs:2.65645E+02,-2.32785E+01) -- cycle  ;
\draw [ecliptics-empty] (axis cs:2.64557E+02,-2.32446E+01) -- (axis cs:2.64549E+02,-2.34444E+01)  -- (axis cs:2.63460E+02,-2.34029E+01) -- (axis cs:2.63470E+02,-2.32031E+01) -- cycle  ;
\draw [ecliptics-empty] (axis cs:2.62383E+02,-2.31541E+01) -- (axis cs:2.62372E+02,-2.33538E+01)  -- (axis cs:2.61285E+02,-2.32973E+01) -- (axis cs:2.61298E+02,-2.30977E+01) -- cycle  ;
\draw [ecliptics-empty] (axis cs:2.60213E+02,-2.30338E+01) -- (axis cs:2.60198E+02,-2.32333E+01)  -- (axis cs:2.59113E+02,-2.31619E+01) -- (axis cs:2.59129E+02,-2.29624E+01) -- cycle  ;
\draw [ecliptics-empty] (axis cs:2.58047E+02,-2.28837E+01) -- (axis cs:2.58029E+02,-2.30830E+01)  -- (axis cs:2.56946E+02,-2.29969E+01) -- (axis cs:2.56966E+02,-2.27977E+01) -- cycle  ;
\draw [ecliptics-empty] (axis cs:2.55886E+02,-2.27043E+01) -- (axis cs:2.55865E+02,-2.29034E+01)  -- (axis cs:2.54786E+02,-2.28026E+01) -- (axis cs:2.54808E+02,-2.26037E+01) -- cycle  ;
\draw [ecliptics-empty] (axis cs:2.53732E+02,-2.24959E+01) -- (axis cs:2.53708E+02,-2.26946E+01)  -- (axis cs:2.52631E+02,-2.25795E+01) -- (axis cs:2.52657E+02,-2.23809E+01) -- cycle  ;
\draw [ecliptics-empty] (axis cs:2.51584E+02,-2.22587E+01) -- (axis cs:2.51557E+02,-2.24572E+01)  -- (axis cs:2.50484E+02,-2.23278E+01) -- (axis cs:2.50513E+02,-2.21296E+01) -- cycle  ;
\draw [ecliptics-empty] (axis cs:2.49444E+02,-2.19934E+01) -- (axis cs:2.49414E+02,-2.21914E+01)  -- (axis cs:2.48346E+02,-2.20481E+01) -- (axis cs:2.48377E+02,-2.18502E+01) -- cycle  ;
\draw [ecliptics-empty] (axis cs:2.47313E+02,-2.17002E+01) -- (axis cs:2.47280E+02,-2.18979E+01)  -- (axis cs:2.46216E+02,-2.17408E+01) -- (axis cs:2.46250E+02,-2.15434E+01) -- cycle  ;
\draw [ecliptics-empty] (axis cs:2.45190E+02,-2.13799E+01) -- (axis cs:2.45154E+02,-2.15771E+01)  -- (axis cs:2.44095E+02,-2.14066E+01) -- (axis cs:2.44133E+02,-2.12096E+01) -- cycle  ;
\draw [ecliptics-empty] (axis cs:2.43078E+02,-2.10328E+01) -- (axis cs:2.43039E+02,-2.12295E+01)  -- (axis cs:2.41985E+02,-2.10459E+01) -- (axis cs:2.42025E+02,-2.08495E+01) -- cycle  ;
\draw [ecliptics-empty] (axis cs:2.40975E+02,-2.06597E+01) -- (axis cs:2.40934E+02,-2.08559E+01)  -- (axis cs:2.39885E+02,-2.06595E+01) -- (axis cs:2.39928E+02,-2.04635E+01) -- cycle  ;
\draw [ecliptics-empty] (axis cs:2.38883E+02,-2.02610E+01) -- (axis cs:2.38839E+02,-2.04568E+01)  -- (axis cs:2.37796E+02,-2.02479E+01) -- (axis cs:2.37841E+02,-2.00524E+01) -- cycle  ;
\draw [ecliptics-empty] (axis cs:2.36802E+02,-1.98376E+01) -- (axis cs:2.36756E+02,-2.00328E+01)  -- (axis cs:2.35719E+02,-1.98118E+01) -- (axis cs:2.35766E+02,-1.96168E+01) -- cycle  ;
\draw [ecliptics-empty] (axis cs:2.34733E+02,-1.93901E+01) -- (axis cs:2.34684E+02,-1.95848E+01)  -- (axis cs:2.33653E+02,-1.93519E+01) -- (axis cs:2.33703E+02,-1.91575E+01) -- cycle  ;
\draw [ecliptics-empty] (axis cs:2.32675E+02,-1.89192E+01) -- (axis cs:2.32624E+02,-1.91133E+01)  -- (axis cs:2.31599E+02,-1.88690E+01) -- (axis cs:2.31651E+02,-1.86752E+01) -- cycle  ;
\draw [ecliptics-empty] (axis cs:2.30629E+02,-1.84256E+01) -- (axis cs:2.30576E+02,-1.86192E+01)  -- (axis cs:2.29557E+02,-1.83638E+01) -- (axis cs:2.29611E+02,-1.81706E+01) -- cycle  ;
\draw [ecliptics-empty] (axis cs:2.28596E+02,-1.79102E+01) -- (axis cs:2.28540E+02,-1.81031E+01)  -- (axis cs:2.27527E+02,-1.78371E+01) -- (axis cs:2.27583E+02,-1.76445E+01) -- cycle  ;
\draw [ecliptics-empty] (axis cs:2.26574E+02,-1.73736E+01) -- (axis cs:2.26516E+02,-1.75660E+01)  -- (axis cs:2.25509E+02,-1.72897E+01) -- (axis cs:2.25567E+02,-1.70976E+01) -- cycle  ;
\draw [ecliptics-empty] (axis cs:2.24564E+02,-1.68167E+01) -- (axis cs:2.24505E+02,-1.70085E+01)  -- (axis cs:2.23504E+02,-1.67224E+01) -- (axis cs:2.23564E+02,-1.65309E+01) -- cycle  ;
\draw [ecliptics-empty] (axis cs:2.22566E+02,-1.62403E+01) -- (axis cs:2.22505E+02,-1.64315E+01)  -- (axis cs:2.21510E+02,-1.61359E+01) -- (axis cs:2.21572E+02,-1.59450E+01) -- cycle  ;
\draw [ecliptics-empty] (axis cs:2.20581E+02,-1.56451E+01) -- (axis cs:2.20518E+02,-1.58358E+01)  -- (axis cs:2.19529E+02,-1.55312E+01) -- (axis cs:2.19592E+02,-1.53408E+01) -- cycle  ;
\draw [ecliptics-empty] (axis cs:2.18607E+02,-1.50321E+01) -- (axis cs:2.18542E+02,-1.52222E+01)  -- (axis cs:2.17559E+02,-1.49090E+01) -- (axis cs:2.17624E+02,-1.47192E+01) -- cycle  ;
\draw [ecliptics-empty] (axis cs:2.16644E+02,-1.44021E+01) -- (axis cs:2.16578E+02,-1.45916E+01)  -- (axis cs:2.15600E+02,-1.42701E+01) -- (axis cs:2.15667E+02,-1.40809E+01) -- cycle  ;
\draw [ecliptics-empty] (axis cs:2.14693E+02,-1.37558E+01) -- (axis cs:2.14625E+02,-1.39448E+01)  -- (axis cs:2.13653E+02,-1.36155E+01) -- (axis cs:2.13721E+02,-1.34268E+01) -- cycle  ;
\draw [ecliptics-empty] (axis cs:2.12752E+02,-1.30941E+01) -- (axis cs:2.12683E+02,-1.32826E+01)  -- (axis cs:2.11717E+02,-1.29460E+01) -- (axis cs:2.11786E+02,-1.27578E+01) -- cycle  ;
\draw [ecliptics-empty] (axis cs:2.10822E+02,-1.24179E+01) -- (axis cs:2.10752E+02,-1.26059E+01)  -- (axis cs:2.09791E+02,-1.22623E+01) -- (axis cs:2.09861E+02,-1.20746E+01) -- cycle  ;
\draw [ecliptics-empty] (axis cs:2.08903E+02,-1.17280E+01) -- (axis cs:2.08831E+02,-1.19155E+01)  -- (axis cs:2.07875E+02,-1.15655E+01) -- (axis cs:2.07946E+02,-1.13782E+01) -- cycle  ;
\draw [ecliptics-empty] (axis cs:2.06993E+02,-1.10253E+01) -- (axis cs:2.06920E+02,-1.12123E+01)  -- (axis cs:2.05968E+02,-1.08562E+01) -- (axis cs:2.06041E+02,-1.06694E+01) -- cycle  ;
\draw [ecliptics-empty] (axis cs:2.05092E+02,-1.03105E+01) -- (axis cs:2.05019E+02,-1.04971E+01)  -- (axis cs:2.04071E+02,-1.01353E+01) -- (axis cs:2.04145E+02,-9.94894E+00) -- cycle  ;
\draw [ecliptics-empty] (axis cs:2.03200E+02,-9.58464E+00) -- (axis cs:2.03126E+02,-9.77078E+00)  -- (axis cs:2.02182E+02,-9.40370E+00) -- (axis cs:2.02257E+02,-9.21776E+00) -- cycle  ;
\draw [ecliptics-empty] (axis cs:2.01316E+02,-8.84840E+00) -- (axis cs:2.01241E+02,-9.03415E+00)  -- (axis cs:2.00301E+02,-8.66224E+00) -- (axis cs:2.00377E+02,-8.47667E+00) -- cycle  ;
\draw [ecliptics-empty] (axis cs:1.99440E+02,-8.10267E+00) -- (axis cs:1.99364E+02,-8.28806E+00)  -- (axis cs:1.98428E+02,-7.91173E+00) -- (axis cs:1.98504E+02,-7.72651E+00) -- cycle  ;
\draw [ecliptics-empty] (axis cs:1.97570E+02,-7.34829E+00) -- (axis cs:1.97494E+02,-7.53334E+00)  -- (axis cs:1.96561E+02,-7.15301E+00) -- (axis cs:1.96638E+02,-6.96811E+00) -- cycle  ;
\draw [ecliptics-empty] (axis cs:1.95708E+02,-6.58609E+00) -- (axis cs:1.95630E+02,-6.77084E+00)  -- (axis cs:1.94701E+02,-6.38692E+00) -- (axis cs:1.94778E+02,-6.20231E+00) -- cycle  ;
\draw [ecliptics-empty] (axis cs:1.93850E+02,-5.81689E+00) -- (axis cs:1.93773E+02,-6.00137E+00)  -- (axis cs:1.92846E+02,-5.61429E+00) -- (axis cs:1.92924E+02,-5.42993E+00) -- cycle  ;
\draw [ecliptics-empty] (axis cs:1.91999E+02,-5.04154E+00) -- (axis cs:1.91920E+02,-5.22577E+00)  -- (axis cs:1.90996E+02,-4.83593E+00) -- (axis cs:1.91074E+02,-4.65180E+00) -- cycle  ;
\draw [ecliptics-empty] (axis cs:1.90151E+02,-4.26084E+00) -- (axis cs:1.90073E+02,-4.44487E+00)  -- (axis cs:1.89150E+02,-4.05268E+00) -- (axis cs:1.89229E+02,-3.86875E+00) -- cycle  ;
\draw [ecliptics-empty] (axis cs:1.88308E+02,-3.47563E+00) -- (axis cs:1.88229E+02,-3.65948E+00)  -- (axis cs:1.87308E+02,-3.26536E+00) -- (axis cs:1.87387E+02,-3.08158E+00) -- cycle  ;
\draw [ecliptics-empty] (axis cs:1.86467E+02,-2.68672E+00) -- (axis cs:1.86388E+02,-2.87043E+00)  -- (axis cs:1.85468E+02,-2.47479E+00) -- (axis cs:1.85548E+02,-2.29113E+00) -- cycle  ;
\draw [ecliptics-empty] (axis cs:1.84629E+02,-1.89493E+00) -- (axis cs:1.84550E+02,-2.07854E+00)  -- (axis cs:1.83631E+02,-1.68179E+00) -- (axis cs:1.83711E+02,-1.49822E+00) -- cycle  ;
\draw [ecliptics-empty] (axis cs:1.82793E+02,-1.10109E+00) -- (axis cs:1.82713E+02,-1.28463E+00)  -- (axis cs:1.81795E+02,-8.87166E-01) -- (axis cs:1.81875E+02,-7.03652E-01) -- cycle  ;
\draw [ecliptics-empty] (axis cs:1.80957E+02,-3.06005E-01) -- (axis cs:1.80878E+02,-4.89506E-01)  -- (axis cs:1.79960E+02,-9.17483E-02) -- (axis cs:1.80040E+02,9.17485E-02) -- cycle  ;



\draw [ecliptics-empty] (axis cs:-0.4000854E-01,9.17484E-02) -- (axis cs:0.0398,-9.17484E-02)  -- (axis cs:9.57272E-01,3.06005E-01) -- (axis cs:8.77725E-01,4.89506E-01) -- cycle  ;
\draw [ecliptics-empty] (axis cs:1.79532E+00,8.87166E-01) -- (axis cs:1.87485E+00,7.03652E-01)  -- (axis cs:2.79259E+00,1.10109E+00) -- (axis cs:2.71311E+00,1.28463E+00) -- cycle  ;
\draw [ecliptics-empty] (axis cs:3.63117E+00,1.68179E+00) -- (axis cs:3.71059E+00,1.49822E+00)  -- (axis cs:4.62893E+00,1.89493E+00) -- (axis cs:4.54959E+00,2.07854E+00) -- cycle  ;
\draw [ecliptics-empty] (axis cs:5.46845E+00,2.47479E+00) -- (axis cs:5.54771E+00,2.29113E+00)  -- (axis cs:6.46701E+00,2.68672E+00) -- (axis cs:6.38786E+00,2.87043E+00) -- cycle  ;
\draw [ecliptics-empty] (axis cs:7.30789E+00,3.26536E+00) -- (axis cs:7.38691E+00,3.08158E+00)  -- (axis cs:8.30751E+00,3.47563E+00) -- (axis cs:8.22863E+00,3.65948E+00) -- cycle  ;
\draw [ecliptics-empty] (axis cs:9.15017E+00,4.05268E+00) -- (axis cs:9.22889E+00,3.86875E+00)  -- (axis cs:1.01511E+01,4.26084E+00) -- (axis cs:1.00726E+01,4.44487E+00) -- cycle  ;
\draw [ecliptics-empty] (axis cs:1.09960E+01,4.83593E+00) -- (axis cs:1.10743E+01,4.65180E+00)  -- (axis cs:1.19986E+01,5.04154E+00) -- (axis cs:1.19204E+01,5.22577E+00) -- cycle  ;
\draw [ecliptics-empty] (axis cs:1.28460E+01,5.61429E+00) -- (axis cs:1.29239E+01,5.42993E+00)  -- (axis cs:1.38505E+01,5.81689E+00) -- (axis cs:1.37728E+01,6.00137E+00) -- cycle  ;
\draw [ecliptics-empty] (axis cs:1.47009E+01,6.38692E+00) -- (axis cs:1.47783E+01,6.20231E+00)  -- (axis cs:1.57075E+01,6.58609E+00) -- (axis cs:1.56304E+01,6.77084E+00) -- cycle  ;
\draw [ecliptics-empty] (axis cs:1.65614E+01,7.15301E+00) -- (axis cs:1.66382E+01,6.96811E+00)  -- (axis cs:1.75704E+01,7.34829E+00) -- (axis cs:1.74939E+01,7.53334E+00) -- cycle  ;
\draw [ecliptics-empty] (axis cs:1.84280E+01,7.91173E+00) -- (axis cs:1.85041E+01,7.72651E+00)  -- (axis cs:1.94396E+01,8.10267E+00) -- (axis cs:1.93638E+01,8.28806E+00) -- cycle  ;
\draw [ecliptics-empty] (axis cs:2.03014E+01,8.66224E+00) -- (axis cs:2.03768E+01,8.47667E+00)  -- (axis cs:2.13159E+01,8.84840E+00) -- (axis cs:2.12408E+01,9.03415E+00) -- cycle  ;
\draw [ecliptics-empty] (axis cs:2.21822E+01,9.40370E+00) -- (axis cs:2.22568E+01,9.21776E+00)  -- (axis cs:2.31998E+01,9.58464E+00) -- (axis cs:2.31256E+01,9.77078E+00) -- cycle  ;
\draw [ecliptics-empty] (axis cs:2.40710E+01,1.01353E+01) -- (axis cs:2.41448E+01,9.94894E+00)  -- (axis cs:2.50918E+01,1.03105E+01) -- (axis cs:2.50186E+01,1.04971E+01) -- cycle  ;
\draw [ecliptics-empty] (axis cs:2.59683E+01,1.08562E+01) -- (axis cs:2.60411E+01,1.06694E+01)  -- (axis cs:2.69926E+01,1.10253E+01) -- (axis cs:2.69204E+01,1.12123E+01) -- cycle  ;
\draw [ecliptics-empty] (axis cs:2.78747E+01,1.15655E+01) -- (axis cs:2.79465E+01,1.13782E+01)  -- (axis cs:2.89026E+01,1.17280E+01) -- (axis cs:2.88315E+01,1.19155E+01) -- cycle  ;
\draw [ecliptics-empty] (axis cs:2.97907E+01,1.22623E+01) -- (axis cs:2.98613E+01,1.20746E+01)  -- (axis cs:3.08224E+01,1.24179E+01) -- (axis cs:3.07524E+01,1.26059E+01) -- cycle  ;
\draw [ecliptics-empty] (axis cs:3.17166E+01,1.29460E+01) -- (axis cs:3.17860E+01,1.27578E+01)  -- (axis cs:3.27522E+01,1.30941E+01) -- (axis cs:3.26835E+01,1.32826E+01) -- cycle  ;
\draw [ecliptics-empty] (axis cs:3.36530E+01,1.36155E+01) -- (axis cs:3.37211E+01,1.34268E+01)  -- (axis cs:3.46927E+01,1.37558E+01) -- (axis cs:3.46253E+01,1.39448E+01) -- cycle  ;
\draw [ecliptics-empty] (axis cs:3.56002E+01,1.42701E+01) -- (axis cs:3.56669E+01,1.40809E+01)  -- (axis cs:3.66440E+01,1.44021E+01) -- (axis cs:3.65780E+01,1.45916E+01) -- cycle  ;
\draw [ecliptics-empty] (axis cs:3.75586E+01,1.49090E+01) -- (axis cs:3.76238E+01,1.47192E+01)  -- (axis cs:3.86065E+01,1.50321E+01) -- (axis cs:3.85421E+01,1.52222E+01) -- cycle  ;
\draw [ecliptics-empty] (axis cs:3.95285E+01,1.55312E+01) -- (axis cs:3.95921E+01,1.53408E+01)  -- (axis cs:4.05806E+01,1.56451E+01) -- (axis cs:4.05178E+01,1.58358E+01) -- cycle  ;
\draw [ecliptics-empty] (axis cs:4.15101E+01,1.61359E+01) -- (axis cs:4.15720E+01,1.59450E+01)  -- (axis cs:4.25664E+01,1.62403E+01) -- (axis cs:4.25053E+01,1.64315E+01) -- cycle  ;
\draw [ecliptics-empty] (axis cs:4.35036E+01,1.67224E+01) -- (axis cs:4.35638E+01,1.65309E+01)  -- (axis cs:4.45641E+01,1.68167E+01) -- (axis cs:4.45048E+01,1.70085E+01) -- cycle  ;
\draw [ecliptics-empty] (axis cs:4.55091E+01,1.72897E+01) -- (axis cs:4.55675E+01,1.70976E+01)  -- (axis cs:4.65738E+01,1.73736E+01) -- (axis cs:4.65165E+01,1.75660E+01) -- cycle  ;
\draw [ecliptics-empty] (axis cs:4.75268E+01,1.78371E+01) -- (axis cs:4.75832E+01,1.76445E+01)  -- (axis cs:4.85956E+01,1.79102E+01) -- (axis cs:4.85402E+01,1.81031E+01) -- cycle  ;
\draw [ecliptics-empty] (axis cs:4.95567E+01,1.83638E+01) -- (axis cs:4.96110E+01,1.81706E+01)  -- (axis cs:5.06294E+01,1.84256E+01) -- (axis cs:5.05762E+01,1.86192E+01) -- cycle  ;
\draw [ecliptics-empty] (axis cs:5.15987E+01,1.88690E+01) -- (axis cs:5.16509E+01,1.86752E+01)  -- (axis cs:5.26753E+01,1.89192E+01) -- (axis cs:5.26242E+01,1.91133E+01) -- cycle  ;
\draw [ecliptics-empty] (axis cs:5.36527E+01,1.93519E+01) -- (axis cs:5.37027E+01,1.91575E+01)  -- (axis cs:5.47330E+01,1.93901E+01) -- (axis cs:5.46842E+01,1.95848E+01) -- cycle  ;
\draw [ecliptics-empty] (axis cs:5.57187E+01,1.98118E+01) -- (axis cs:5.57662E+01,1.96168E+01)  -- (axis cs:5.68024E+01,1.98376E+01) -- (axis cs:5.67560E+01,2.00328E+01) -- cycle  ;
\draw [ecliptics-empty] (axis cs:5.77962E+01,2.02479E+01) -- (axis cs:5.78414E+01,2.00524E+01)  -- (axis cs:5.88831E+01,2.02610E+01) -- (axis cs:5.88393E+01,2.04568E+01) -- cycle  ;
\draw [ecliptics-empty] (axis cs:5.98851E+01,2.06595E+01) -- (axis cs:5.99277E+01,2.04635E+01)  -- (axis cs:6.09750E+01,2.06597E+01) -- (axis cs:6.09337E+01,2.08559E+01) -- cycle  ;
\draw [ecliptics-empty] (axis cs:6.19850E+01,2.10459E+01) -- (axis cs:6.20250E+01,2.08495E+01)  -- (axis cs:6.30775E+01,2.10328E+01) -- (axis cs:6.30389E+01,2.12295E+01) -- cycle  ;
\draw [ecliptics-empty] (axis cs:6.40954E+01,2.14066E+01) -- (axis cs:6.41327E+01,2.12096E+01)  -- (axis cs:6.51903E+01,2.13799E+01) -- (axis cs:6.51544E+01,2.15771E+01) -- cycle  ;
\draw [ecliptics-empty] (axis cs:6.62158E+01,2.17408E+01) -- (axis cs:6.62503E+01,2.15434E+01)  -- (axis cs:6.73127E+01,2.17002E+01) -- (axis cs:6.72796E+01,2.18979E+01) -- cycle  ;
\draw [ecliptics-empty] (axis cs:6.83457E+01,2.20481E+01) -- (axis cs:6.83773E+01,2.18502E+01)  -- (axis cs:6.94442E+01,2.19934E+01) -- (axis cs:6.94140E+01,2.21914E+01) -- cycle  ;
\draw [ecliptics-empty] (axis cs:7.04845E+01,2.23278E+01) -- (axis cs:7.05131E+01,2.21296E+01)  -- (axis cs:7.15841E+01,2.22587E+01) -- (axis cs:7.15569E+01,2.24572E+01) -- cycle  ;
\draw [ecliptics-empty] (axis cs:7.26313E+01,2.25795E+01) -- (axis cs:7.26570E+01,2.23809E+01)  -- (axis cs:7.37317E+01,2.24959E+01) -- (axis cs:7.37076E+01,2.26946E+01) -- cycle  ;
\draw [ecliptics-empty] (axis cs:7.47856E+01,2.28026E+01) -- (axis cs:7.48082E+01,2.26037E+01)  -- (axis cs:7.58863E+01,2.27043E+01) -- (axis cs:7.58653E+01,2.29034E+01) -- cycle  ;
\draw [ecliptics-empty] (axis cs:7.69465E+01,2.29969E+01) -- (axis cs:7.69660E+01,2.27977E+01)  -- (axis cs:7.80470E+01,2.28837E+01) -- (axis cs:7.80291E+01,2.30830E+01) -- cycle  ;
\draw [ecliptics-empty] (axis cs:7.91131E+01,2.31619E+01) -- (axis cs:7.91294E+01,2.29624E+01)  -- (axis cs:8.02130E+01,2.30338E+01) -- (axis cs:8.01982E+01,2.32333E+01) -- cycle  ;
\draw [ecliptics-empty] (axis cs:8.12845E+01,2.32973E+01) -- (axis cs:8.12976E+01,2.30977E+01)  -- (axis cs:8.23833E+01,2.31541E+01) -- (axis cs:8.23718E+01,2.33538E+01) -- cycle  ;
\draw [ecliptics-empty] (axis cs:8.34599E+01,2.34029E+01) -- (axis cs:8.34698E+01,2.32031E+01)  -- (axis cs:8.45570E+01,2.32446E+01) -- (axis cs:8.45488E+01,2.34444E+01) -- cycle  ;
\draw [ecliptics-empty] (axis cs:8.56383E+01,2.34785E+01) -- (axis cs:8.56449E+01,2.32785E+01)  -- (axis cs:8.67332E+01,2.33050E+01) -- (axis cs:8.67283E+01,2.35049E+01) -- cycle  ;
\draw [ecliptics-empty] (axis cs:8.78186E+01,2.35239E+01) -- (axis cs:8.78219E+01,2.33239E+01)  -- (axis cs:8.89109E+01,2.33352E+01) -- (axis cs:8.89093E+01,2.35352E+01) -- cycle  ;
\draw [ecliptics-empty] (axis cs:9.00000E+01,2.35390E+01) -- (axis cs:9.00000E+01,2.33390E+01)  -- (axis cs:9.10891E+01,2.33352E+01) -- (axis cs:9.10907E+01,2.35352E+01) -- cycle  ;
\draw [ecliptics-empty] (axis cs:9.21814E+01,2.35239E+01) -- (axis cs:9.21781E+01,2.33239E+01)  -- (axis cs:9.32668E+01,2.33050E+01) -- (axis cs:9.32717E+01,2.35049E+01) -- cycle  ;
\draw [ecliptics-empty] (axis cs:9.43617E+01,2.34785E+01) -- (axis cs:9.43551E+01,2.32785E+01)  -- (axis cs:9.54430E+01,2.32446E+01) -- (axis cs:9.54512E+01,2.34444E+01) -- cycle  ;
\draw [ecliptics-empty] (axis cs:9.65401E+01,2.34029E+01) -- (axis cs:9.65302E+01,2.32031E+01)  -- (axis cs:9.76167E+01,2.31541E+01) -- (axis cs:9.76282E+01,2.33538E+01) -- cycle  ;
\draw [ecliptics-empty] (axis cs:9.87155E+01,2.32973E+01) -- (axis cs:9.87024E+01,2.30977E+01)  -- (axis cs:9.97870E+01,2.30338E+01) -- (axis cs:9.98018E+01,2.32333E+01) -- cycle  ;
\draw [ecliptics-empty] (axis cs:1.00887E+02,2.31619E+01) -- (axis cs:1.00871E+02,2.29624E+01)  -- (axis cs:1.01953E+02,2.28837E+01) -- (axis cs:1.01971E+02,2.30830E+01) -- cycle  ;
\draw [ecliptics-empty] (axis cs:1.03054E+02,2.29969E+01) -- (axis cs:1.03034E+02,2.27977E+01)  -- (axis cs:1.04114E+02,2.27043E+01) -- (axis cs:1.04135E+02,2.29034E+01) -- cycle  ;
\draw [ecliptics-empty] (axis cs:1.05214E+02,2.28026E+01) -- (axis cs:1.05192E+02,2.26037E+01)  -- (axis cs:1.06268E+02,2.24959E+01) -- (axis cs:1.06292E+02,2.26946E+01) -- cycle  ;
\draw [ecliptics-empty] (axis cs:1.07369E+02,2.25795E+01) -- (axis cs:1.07343E+02,2.23809E+01)  -- (axis cs:1.08416E+02,2.22587E+01) -- (axis cs:1.08443E+02,2.24572E+01) -- cycle  ;
\draw [ecliptics-empty] (axis cs:1.09516E+02,2.23278E+01) -- (axis cs:1.09487E+02,2.21296E+01)  -- (axis cs:1.10556E+02,2.19934E+01) -- (axis cs:1.10586E+02,2.21914E+01) -- cycle  ;
\draw [ecliptics-empty] (axis cs:1.11654E+02,2.20481E+01) -- (axis cs:1.11623E+02,2.18502E+01)  -- (axis cs:1.12687E+02,2.17002E+01) -- (axis cs:1.12720E+02,2.18979E+01) -- cycle  ;
\draw [ecliptics-empty] (axis cs:1.13784E+02,2.17408E+01) -- (axis cs:1.13750E+02,2.15434E+01)  -- (axis cs:1.14810E+02,2.13799E+01) -- (axis cs:1.14846E+02,2.15771E+01) -- cycle  ;
\draw [ecliptics-empty] (axis cs:1.15905E+02,2.14066E+01) -- (axis cs:1.15867E+02,2.12096E+01)  -- (axis cs:1.16922E+02,2.10328E+01) -- (axis cs:1.16961E+02,2.12295E+01) -- cycle  ;
\draw [ecliptics-empty] (axis cs:1.18015E+02,2.10459E+01) -- (axis cs:1.17975E+02,2.08495E+01)  -- (axis cs:1.19025E+02,2.06597E+01) -- (axis cs:1.19066E+02,2.08559E+01) -- cycle  ;
\draw [ecliptics-empty] (axis cs:1.20115E+02,2.06595E+01) -- (axis cs:1.20072E+02,2.04635E+01)  -- (axis cs:1.21117E+02,2.02610E+01) -- (axis cs:1.21161E+02,2.04568E+01) -- cycle  ;
\draw [ecliptics-empty] (axis cs:1.22204E+02,2.02479E+01) -- (axis cs:1.22159E+02,2.00524E+01)  -- (axis cs:1.23198E+02,1.98376E+01) -- (axis cs:1.23244E+02,2.00328E+01) -- cycle  ;
\draw [ecliptics-empty] (axis cs:1.24281E+02,1.98118E+01) -- (axis cs:1.24234E+02,1.96168E+01)  -- (axis cs:1.25267E+02,1.93901E+01) -- (axis cs:1.25316E+02,1.95848E+01) -- cycle  ;
\draw [ecliptics-empty] (axis cs:1.26347E+02,1.93519E+01) -- (axis cs:1.26297E+02,1.91575E+01)  -- (axis cs:1.27325E+02,1.89192E+01) -- (axis cs:1.27376E+02,1.91133E+01) -- cycle  ;
\draw [ecliptics-empty] (axis cs:1.28401E+02,1.88690E+01) -- (axis cs:1.28349E+02,1.86752E+01)  -- (axis cs:1.29371E+02,1.84256E+01) -- (axis cs:1.29424E+02,1.86192E+01) -- cycle  ;
\draw [ecliptics-empty] (axis cs:1.30443E+02,1.83638E+01) -- (axis cs:1.30389E+02,1.81706E+01)  -- (axis cs:1.31404E+02,1.79102E+01) -- (axis cs:1.31460E+02,1.81031E+01) -- cycle  ;
\draw [ecliptics-empty] (axis cs:1.32473E+02,1.78371E+01) -- (axis cs:1.32417E+02,1.76445E+01)  -- (axis cs:1.33426E+02,1.73736E+01) -- (axis cs:1.33484E+02,1.75660E+01) -- cycle  ;
\draw [ecliptics-empty] (axis cs:1.34491E+02,1.72897E+01) -- (axis cs:1.34433E+02,1.70976E+01)  -- (axis cs:1.35436E+02,1.68167E+01) -- (axis cs:1.35495E+02,1.70085E+01) -- cycle  ;
\draw [ecliptics-empty] (axis cs:1.36496E+02,1.67224E+01) -- (axis cs:1.36436E+02,1.65309E+01)  -- (axis cs:1.37434E+02,1.62403E+01) -- (axis cs:1.37495E+02,1.64315E+01) -- cycle  ;
\draw [ecliptics-empty] (axis cs:1.38490E+02,1.61359E+01) -- (axis cs:1.38428E+02,1.59450E+01)  -- (axis cs:1.39419E+02,1.56451E+01) -- (axis cs:1.39482E+02,1.58358E+01) -- cycle  ;
\draw [ecliptics-empty] (axis cs:1.40471E+02,1.55312E+01) -- (axis cs:1.40408E+02,1.53408E+01)  -- (axis cs:1.41393E+02,1.50321E+01) -- (axis cs:1.41458E+02,1.52222E+01) -- cycle  ;
\draw [ecliptics-empty] (axis cs:1.42441E+02,1.49090E+01) -- (axis cs:1.42376E+02,1.47192E+01)  -- (axis cs:1.43356E+02,1.44021E+01) -- (axis cs:1.43422E+02,1.45916E+01) -- cycle  ;
\draw [ecliptics-empty] (axis cs:1.44400E+02,1.42701E+01) -- (axis cs:1.44333E+02,1.40809E+01)  -- (axis cs:1.45307E+02,1.37558E+01) -- (axis cs:1.45375E+02,1.39448E+01) -- cycle  ;
\draw [ecliptics-empty] (axis cs:1.46347E+02,1.36155E+01) -- (axis cs:1.46279E+02,1.34268E+01)  -- (axis cs:1.47248E+02,1.30941E+01) -- (axis cs:1.47317E+02,1.32826E+01) -- cycle  ;
\draw [ecliptics-empty] (axis cs:1.48283E+02,1.29460E+01) -- (axis cs:1.48214E+02,1.27578E+01)  -- (axis cs:1.49178E+02,1.24179E+01) -- (axis cs:1.49248E+02,1.26059E+01) -- cycle  ;
\draw [ecliptics-empty] (axis cs:1.50209E+02,1.22623E+01) -- (axis cs:1.50139E+02,1.20746E+01)  -- (axis cs:1.51097E+02,1.17280E+01) -- (axis cs:1.51169E+02,1.19155E+01) -- cycle  ;
\draw [ecliptics-empty] (axis cs:1.52125E+02,1.15655E+01) -- (axis cs:1.52054E+02,1.13782E+01)  -- (axis cs:1.53007E+02,1.10253E+01) -- (axis cs:1.53080E+02,1.12123E+01) -- cycle  ;
\draw [ecliptics-empty] (axis cs:1.54032E+02,1.08562E+01) -- (axis cs:1.53959E+02,1.06694E+01)  -- (axis cs:1.54908E+02,1.03105E+01) -- (axis cs:1.54981E+02,1.04971E+01) -- cycle  ;
\draw [ecliptics-empty] (axis cs:1.55929E+02,1.01353E+01) -- (axis cs:1.55855E+02,9.94894E+00)  -- (axis cs:1.56800E+02,9.58464E+00) -- (axis cs:1.56874E+02,9.77078E+00) -- cycle  ;
\draw [ecliptics-empty] (axis cs:1.57818E+02,9.40370E+00) -- (axis cs:1.57743E+02,9.21776E+00)  -- (axis cs:1.58684E+02,8.84840E+00) -- (axis cs:1.58759E+02,9.03415E+00) -- cycle  ;
\draw [ecliptics-empty] (axis cs:1.59699E+02,8.66224E+00) -- (axis cs:1.59623E+02,8.47667E+00)  -- (axis cs:1.60560E+02,8.10267E+00) -- (axis cs:1.60636E+02,8.28806E+00) -- cycle  ;
\draw [ecliptics-empty] (axis cs:1.61572E+02,7.91173E+00) -- (axis cs:1.61496E+02,7.72651E+00)  -- (axis cs:1.62430E+02,7.34829E+00) -- (axis cs:1.62506E+02,7.53334E+00) -- cycle  ;
\draw [ecliptics-empty] (axis cs:1.63439E+02,7.15301E+00) -- (axis cs:1.63362E+02,6.96811E+00)  -- (axis cs:1.64292E+02,6.58609E+00) -- (axis cs:1.64370E+02,6.77084E+00) -- cycle  ;
\draw [ecliptics-empty] (axis cs:1.65299E+02,6.38692E+00) -- (axis cs:1.65222E+02,6.20231E+00)  -- (axis cs:1.66150E+02,5.81689E+00) -- (axis cs:1.66227E+02,6.00137E+00) -- cycle  ;
\draw [ecliptics-empty] (axis cs:1.67154E+02,5.61429E+00) -- (axis cs:1.67076E+02,5.42993E+00)  -- (axis cs:1.68001E+02,5.04154E+00) -- (axis cs:1.68080E+02,5.22577E+00) -- cycle  ;
\draw [ecliptics-empty] (axis cs:1.69004E+02,4.83593E+00) -- (axis cs:1.68926E+02,4.65180E+00)  -- (axis cs:1.69849E+02,4.26084E+00) -- (axis cs:1.69927E+02,4.44487E+00) -- cycle  ;
\draw [ecliptics-empty] (axis cs:1.70850E+02,4.05268E+00) -- (axis cs:1.70771E+02,3.86875E+00)  -- (axis cs:1.71692E+02,3.47563E+00) -- (axis cs:1.71771E+02,3.65948E+00) -- cycle  ;
\draw [ecliptics-empty] (axis cs:1.72692E+02,3.26536E+00) -- (axis cs:1.72613E+02,3.08158E+00)  -- (axis cs:1.73533E+02,2.68672E+00) -- (axis cs:1.73612E+02,2.87043E+00) -- cycle  ;
\draw [ecliptics-empty] (axis cs:1.74532E+02,2.47479E+00) -- (axis cs:1.74452E+02,2.29113E+00)  -- (axis cs:1.75371E+02,1.89493E+00) -- (axis cs:1.75450E+02,2.07854E+00) -- cycle  ;
\draw [ecliptics-empty] (axis cs:1.76369E+02,1.68179E+00) -- (axis cs:1.76289E+02,1.49822E+00)  -- (axis cs:1.77207E+02,1.10109E+00) -- (axis cs:1.77287E+02,1.28463E+00) -- cycle  ;
\draw [ecliptics-empty] (axis cs:1.78205E+02,8.87166E-01) -- (axis cs:1.78125E+02,7.03652E-01)  -- (axis cs:1.79043E+02,3.06005E-01) -- (axis cs:1.79122E+02,4.89506E-01) -- cycle  ;




\draw [ecliptics-empty] (axis cs:-0.4000854E-01+360,9.17484E-02) -- (axis cs:0.0398+360,-9.17484E-02)  -- (axis cs:9.57272E-01+360,3.06005E-01) -- (axis cs:8.77725E-01+360,4.89506E-01) -- cycle  ;
\draw [ecliptics-empty] (axis cs:1.79532E+00+360,8.87166E-01) -- (axis cs:1.87485E+00+360,7.03652E-01)  -- (axis cs:2.79259E+00+360,1.10109E+00) -- (axis cs:2.71311E+00+360,1.28463E+00) -- cycle  ;
\draw [ecliptics-empty] (axis cs:3.63117E+00+360,1.68179E+00) -- (axis cs:3.71059E+00+360,1.49822E+00)  -- (axis cs:4.62893E+00+360,1.89493E+00) -- (axis cs:4.54959E+00+360,2.07854E+00) -- cycle  ;
\draw [ecliptics-empty] (axis cs:5.46845E+00+360,2.47479E+00) -- (axis cs:5.54771E+00+360,2.29113E+00)  -- (axis cs:6.46701E+00+360,2.68672E+00) -- (axis cs:6.38786E+00+360,2.87043E+00) -- cycle  ;
\draw [ecliptics-empty] (axis cs:7.30789E+00+360,3.26536E+00) -- (axis cs:7.38691E+00+360,3.08158E+00)  -- (axis cs:8.30751E+00+360,3.47563E+00) -- (axis cs:8.22863E+00+360,3.65948E+00) -- cycle  ;
\draw [ecliptics-empty] (axis cs:9.15017E+00+360,4.05268E+00) -- (axis cs:9.22889E+00+360,3.86875E+00)  -- (axis cs:1.01511E+01+360,4.26084E+00) -- (axis cs:1.00726E+01+360,4.44487E+00) -- cycle  ;
\draw [ecliptics-empty] (axis cs:1.09960E+01+360,4.83593E+00) -- (axis cs:1.10743E+01+360,4.65180E+00)  -- (axis cs:1.19986E+01+360,5.04154E+00) -- (axis cs:1.19204E+01+360,5.22577E+00) -- cycle  ;
\draw [ecliptics-empty] (axis cs:1.28460E+01+360,5.61429E+00) -- (axis cs:1.29239E+01+360,5.42993E+00)  -- (axis cs:1.38505E+01+360,5.81689E+00) -- (axis cs:1.37728E+01+360,6.00137E+00) -- cycle  ;
\draw [ecliptics-empty] (axis cs:1.47009E+01+360,6.38692E+00) -- (axis cs:1.47783E+01+360,6.20231E+00)  -- (axis cs:1.57075E+01+360,6.58609E+00) -- (axis cs:1.56304E+01+360,6.77084E+00) -- cycle  ;
\draw [ecliptics-empty] (axis cs:1.65614E+01+360,7.15301E+00) -- (axis cs:1.66382E+01+360,6.96811E+00)  -- (axis cs:1.75704E+01+360,7.34829E+00) -- (axis cs:1.74939E+01+360,7.53334E+00) -- cycle  ;
\draw [ecliptics-empty] (axis cs:1.84280E+01+360,7.91173E+00) -- (axis cs:1.85041E+01+360,7.72651E+00)  -- (axis cs:1.94396E+01+360,8.10267E+00) -- (axis cs:1.93638E+01+360,8.28806E+00) -- cycle  ;
\draw [ecliptics-empty] (axis cs:2.03014E+01+360,8.66224E+00) -- (axis cs:2.03768E+01+360,8.47667E+00)  -- (axis cs:2.13159E+01+360,8.84840E+00) -- (axis cs:2.12408E+01+360,9.03415E+00) -- cycle  ;


\draw [ecliptics-full] (axis cs:8.77725E-01+360,{4.89506E-01}) -- (axis cs:9.57272E-01+360,{3.06005E-01})  -- (axis cs:1.87485E+00+360,{7.03652E-01}) -- (axis cs:1.79532E+00+360,{8.87166E-01}) -- cycle  ;
\draw [ecliptics-full] (axis cs:2.71311E+00+360,1.28463E+00) -- (axis cs:2.79259E+00+360,1.10109E+00)  -- (axis cs:3.71059E+00+360,1.49822E+00) -- (axis cs:3.63117E+00+360,1.68179E+00) -- cycle  ;
\draw [ecliptics-full] (axis cs:4.54959E+00+360,2.07854E+00) -- (axis cs:4.62893E+00+360,1.89493E+00)  -- (axis cs:5.54771E+00+360,2.29113E+00) -- (axis cs:5.46845E+00+360,2.47479E+00) -- cycle  ;
\draw [ecliptics-full] (axis cs:6.38786E+00+360,2.87043E+00) -- (axis cs:6.46701E+00+360,2.68672E+00)  -- (axis cs:7.38691E+00+360,3.08158E+00) -- (axis cs:7.30789E+00+360,3.26536E+00) -- cycle  ;
\draw [ecliptics-full] (axis cs:8.22863E+00+360,3.65948E+00) -- (axis cs:8.30751E+00+360,3.47563E+00)  -- (axis cs:9.22889E+00+360,3.86875E+00) -- (axis cs:9.15017E+00+360,4.05268E+00) -- cycle  ;
\draw [ecliptics-full] (axis cs:1.00726E+01+360,4.44487E+00) -- (axis cs:1.01511E+01+360,4.26084E+00)  -- (axis cs:1.10743E+01+360,4.65180E+00) -- (axis cs:1.09960E+01+360,4.83593E+00) -- cycle  ;
\draw [ecliptics-full] (axis cs:1.19204E+01+360,5.22577E+00) -- (axis cs:1.19986E+01+360,5.04154E+00)  -- (axis cs:1.29239E+01+360,5.42993E+00) -- (axis cs:1.28460E+01+360,5.61429E+00) -- cycle  ;
\draw [ecliptics-full] (axis cs:1.37728E+01+360,6.00137E+00) -- (axis cs:1.38505E+01+360,5.81689E+00)  -- (axis cs:1.47783E+01+360,6.20231E+00) -- (axis cs:1.47009E+01+360,6.38692E+00) -- cycle  ;
\draw [ecliptics-full] (axis cs:1.56304E+01+360,6.77084E+00) -- (axis cs:1.57075E+01+360,6.58609E+00)  -- (axis cs:1.66382E+01+360,6.96811E+00) -- (axis cs:1.65614E+01+360,7.15301E+00) -- cycle  ;
\draw [ecliptics-full] (axis cs:1.74939E+01+360,7.53334E+00) -- (axis cs:1.75704E+01+360,7.34829E+00)  -- (axis cs:1.85041E+01+360,7.72651E+00) -- (axis cs:1.84280E+01+360,7.91173E+00) -- cycle  ;
\draw [ecliptics-full] (axis cs:1.93638E+01+360,8.28806E+00) -- (axis cs:1.94396E+01+360,8.10267E+00)  -- (axis cs:2.03768E+01+360,8.47667E+00) -- (axis cs:2.03014E+01+360,8.66224E+00) -- cycle  ;


\draw [ecliptics-empty] (axis cs:-1.59111E-01,3.66993E-01) -- (axis cs:1.59111E-01,-3.66993E-01)   ;
\node[pin={[pin distance=-0.4\onedegree,ecliptics-label]+90:{0$^\circ$}}] at (axis cs:1.59111E-01,3.66993E-01) {} ;
\draw [ecliptics-empty] (axis cs:9.03202E+00,4.32857E+00) -- (axis cs:9.34691E+00,3.59283E+00)   ;
\node[pin={[pin distance=-0.4\onedegree,ecliptics-label]-90:{10$^\circ$}}] at (axis cs:9.03202E+00,4.32857E+00) {} ;
\draw [ecliptics-empty] (axis cs:1.83136E+01,8.18953E+00) -- (axis cs:1.86183E+01,7.44866E+00)   ;
\node[pin={[pin distance=-0.4\onedegree,ecliptics-label]-90:{20$^\circ$}}] at (axis cs:1.83136E+01,8.18953E+00) {} ;
\draw [ecliptics-empty] (axis cs:2.77669E+01,1.18463E+01) -- (axis cs:2.80539E+01,1.10973E+01)   ;
\node[pin={[pin distance=-0.4\onedegree,ecliptics-label]-90:{30$^\circ$}}] at (axis cs:2.77669E+01,1.18463E+01) {} ;
\draw [ecliptics-empty] (axis cs:3.74606E+01,1.51936E+01) -- (axis cs:3.77214E+01,1.44344E+01)   ;
\node[pin={[pin distance=-0.4\onedegree,ecliptics-label]-90:{40$^\circ$}}] at (axis cs:3.74606E+01,1.51936E+01) {} ;
\draw [ecliptics-empty] (axis cs:4.74420E+01,1.81261E+01) -- (axis cs:4.76675E+01,1.73555E+01)   ;
\node[pin={[pin distance=-0.4\onedegree,ecliptics-label]-90:{50$^\circ$}}] at (axis cs:4.74420E+01,1.81261E+01) {} ;
\draw [ecliptics-empty] (axis cs:5.77283E+01,2.05410E+01) -- (axis cs:5.79088E+01,1.97592E+01)   ;
\node[pin={[pin distance=-0.4\onedegree,ecliptics-label]-90:{60$^\circ$}}] at (axis cs:5.77283E+01,2.05410E+01) {} ;
\draw [ecliptics-empty] (axis cs:6.82981E+01,2.23448E+01) -- (axis cs:6.84246E+01,2.15535E+01)   ;
\node[pin={[pin distance=-0.4\onedegree,ecliptics-label]-90:{70$^\circ$}}] at (axis cs:6.82981E+01,2.23448E+01) {} ;
\draw [ecliptics-empty] (axis cs:7.90885E+01,2.34610E+01) -- (axis cs:7.91538E+01,2.26633E+01)   ;
\node[pin={[pin distance=-0.4\onedegree,ecliptics-label]-90:{80$^\circ$}}] at (axis cs:7.90885E+01,2.34610E+01) {} ;
\draw [ecliptics-empty] (axis cs:9.00000E+01,2.38390E+01) -- (axis cs:9.00000E+01,2.30390E+01)   ;
\node[pin={[pin distance=-0.4\onedegree,ecliptics-label]-90:{90$^\circ$}}] at (axis cs:9.00000E+01,2.38390E+01) {} ;
\draw [ecliptics-empty] (axis cs:1.00911E+02,2.34610E+01) -- (axis cs:1.00846E+02,2.26633E+01)   ;
\node[pin={[pin distance=-0.4\onedegree,ecliptics-label]-90:{100$^\circ$}}] at (axis cs:1.00911E+02,2.34610E+01) {} ;
\draw [ecliptics-empty] (axis cs:1.11702E+02,2.23448E+01) -- (axis cs:1.11575E+02,2.15535E+01)   ;
\node[pin={[pin distance=-0.4\onedegree,ecliptics-label]-90:{110$^\circ$}}] at (axis cs:1.11702E+02,2.23448E+01) {} ;
\draw [ecliptics-empty] (axis cs:1.22272E+02,2.05410E+01) -- (axis cs:1.22091E+02,1.97592E+01)   ;
\node[pin={[pin distance=-0.4\onedegree,ecliptics-label]-90:{120$^\circ$}}] at (axis cs:1.22272E+02,2.05410E+01) {} ;
\draw [ecliptics-empty] (axis cs:1.32558E+02,1.81261E+01) -- (axis cs:1.32332E+02,1.73555E+01)   ;
\node[pin={[pin distance=-0.4\onedegree,ecliptics-label]-90:{130$^\circ$}}] at (axis cs:1.32558E+02,1.81261E+01) {} ;
\draw [ecliptics-empty] (axis cs:1.42539E+02,1.51936E+01) -- (axis cs:1.42279E+02,1.44344E+01)   ;
\node[pin={[pin distance=-0.4\onedegree,ecliptics-label]-90:{140$^\circ$}}] at (axis cs:1.42539E+02,1.51936E+01) {} ;
\draw [ecliptics-empty] (axis cs:1.52233E+02,1.18463E+01) -- (axis cs:1.51946E+02,1.10973E+01)   ;
\node[pin={[pin distance=0\onedegree,ecliptics-label]-45:{150$^\circ$}}] at (axis cs:1.52233E+02,1.18463E+01) {} ;
\draw [ecliptics-empty] (axis cs:1.61686E+02,8.18953E+00) -- (axis cs:1.61382E+02,7.44866E+00)   ;
\node[pin={[pin distance=-0.4\onedegree,ecliptics-label]-90:{160$^\circ$}}] at (axis cs:1.61686E+02,8.18953E+00) {} ;
\draw [ecliptics-empty] (axis cs:1.70968E+02,4.32857E+00) -- (axis cs:1.70653E+02,3.59283E+00)   ;
\node[pin={[pin distance=-0.4\onedegree,ecliptics-label]-90:{170$^\circ$}}] at (axis cs:1.70968E+02,4.32857E+00) {} ;
\draw [ecliptics-empty] (axis cs:360-1.79841E+02,3.66993E-01) -- (axis cs:1.79841E+02,-3.66993E-01)   ;
%\node[pin={[pin distance=0.4\onedegree,ecliptics-label]+90:{180$^\circ$}}] at (axis cs:1.79841E+02,3.66993E-01) {} ;

\draw [ecliptics-empty] (axis cs:3.50653E+02,-3.59283E+00) -- (axis cs:3.50968E+02,-4.32857E+00)   ;
\node[pin={[pin distance=-0.4\onedegree,ecliptics-label]-90:{350$^\circ$}}] at (axis cs:3.50653E+02,-3.59283E+00) {} ;
\draw [ecliptics-empty] (axis cs:3.41382E+02,-7.44866E+00) -- (axis cs:3.41686E+02,-8.18953E+00)   ;
\node[pin={[pin distance=-0.4\onedegree,ecliptics-label]-90:{340$^\circ$}}] at (axis cs:3.41382E+02,-7.44866E+00) {} ;
\draw [ecliptics-empty] (axis cs:3.31946E+02,-1.10973E+01) -- (axis cs:3.32233E+02,-1.18463E+01)   ;
\node[pin={[pin distance=-0.4\onedegree,ecliptics-label]-90:{330$^\circ$}}] at (axis cs:3.31946E+02,-1.10973E+01) {} ;
\draw [ecliptics-empty] (axis cs:3.22279E+02,-1.44344E+01) -- (axis cs:3.22539E+02,-1.51936E+01)   ;
\node[pin={[pin distance=-0.4\onedegree,ecliptics-label]-90:{320$^\circ$}}] at (axis cs:3.22279E+02,-1.44344E+01) {} ;
\draw [ecliptics-empty] (axis cs:3.12332E+02,-1.73555E+01) -- (axis cs:3.12558E+02,-1.81261E+01)   ;
\node[pin={[pin distance=-0.4\onedegree,ecliptics-label]-90:{310$^\circ$}}] at (axis cs:3.12332E+02,-1.73555E+01) {} ;
\draw [ecliptics-empty] (axis cs:3.02091E+02,-1.97592E+01) -- (axis cs:3.02272E+02,-2.05410E+01)   ;
\node[pin={[pin distance=-0.4\onedegree,ecliptics-label]-90:{300$^\circ$}}] at (axis cs:3.02091E+02,-1.97592E+01) {} ;
\draw [ecliptics-empty] (axis cs:2.91575E+02,-2.15535E+01) -- (axis cs:2.91702E+02,-2.23448E+01)   ;
\node[pin={[pin distance=-0.4\onedegree,ecliptics-label]-90:{290$^\circ$}}] at (axis cs:2.91575E+02,-2.15535E+01) {} ;
\draw [ecliptics-empty] (axis cs:2.80846E+02,-2.26633E+01) -- (axis cs:2.80911E+02,-2.34610E+01)   ;
\node[pin={[pin distance=-0.4\onedegree,ecliptics-label]-90:{280$^\circ$}}] at (axis cs:2.80846E+02,-2.26633E+01) {} ;
\draw [ecliptics-empty] (axis cs:2.70000E+02,-2.30390E+01) -- (axis cs:2.70000E+02,-2.38390E+01)   ;
\node[pin={[pin distance=-0.4\onedegree,ecliptics-label]-90:{270$^\circ$}}] at (axis cs:2.70000E+02,-2.30390E+01) {} ;
\draw [ecliptics-empty] (axis cs:2.59154E+02,-2.26633E+01) -- (axis cs:2.59089E+02,-2.34610E+01)   ;
\node[pin={[pin distance=-0.4\onedegree,ecliptics-label]-90:{260$^\circ$}}] at (axis cs:2.59154E+02,-2.26633E+01) {} ;
\draw [ecliptics-empty] (axis cs:2.48425E+02,-2.15535E+01) -- (axis cs:2.48298E+02,-2.23448E+01)   ;
\node[pin={[pin distance=-0.4\onedegree,ecliptics-label]-90:{250$^\circ$}}] at (axis cs:2.48425E+02,-2.15535E+01) {} ;
\draw [ecliptics-empty] (axis cs:2.37909E+02,-1.97592E+01) -- (axis cs:2.37728E+02,-2.05410E+01)   ;
\node[pin={[pin distance=-0.4\onedegree,ecliptics-label]-90:{240$^\circ$}}] at (axis cs:2.37909E+02,-1.97592E+01) {} ;
\draw [ecliptics-empty] (axis cs:2.27668E+02,-1.73555E+01) -- (axis cs:2.27442E+02,-1.81261E+01)   ;
\node[pin={[pin distance=-0.4\onedegree,ecliptics-label]-90:{230$^\circ$}}] at (axis cs:2.27668E+02,-1.73555E+01) {} ;
\draw [ecliptics-empty] (axis cs:2.17721E+02,-1.44344E+01) -- (axis cs:2.17461E+02,-1.51936E+01)   ;
\node[pin={[pin distance=-0.4\onedegree,ecliptics-label]-90:{220$^\circ$}}] at (axis cs:2.17721E+02,-1.44344E+01) {} ;
\draw [ecliptics-empty] (axis cs:2.08054E+02,-1.10973E+01) -- (axis cs:2.07767E+02,-1.18463E+01)   ;
\node[pin={[pin distance=-0.4\onedegree,ecliptics-label]-90:{210$^\circ$}}] at (axis cs:2.08054E+02,-1.10973E+01) {} ;
\draw [ecliptics-empty] (axis cs:1.98618E+02,-7.44866E+00) -- (axis cs:1.98314E+02,-8.18953E+00)   ;
\node[pin={[pin distance=-0.4\onedegree,ecliptics-label]-90:{200$^\circ$}}] at (axis cs:1.98618E+02,-7.44866E+00) {} ;
\draw [ecliptics-empty] (axis cs:1.89347E+02,-3.59283E+00) -- (axis cs:1.89032E+02,-4.32857E+00)   ;
\node[pin={[pin distance=-0.4\onedegree,ecliptics-label]-90:{190$^\circ$}}] at (axis cs:1.89347E+02,-3.59283E+00) {} ;

%\draw [ecliptics-empty] (axis cs:-1.59111E-01+360,3.66993E-01) -- (axis cs:1.59111E-01+360,-3.66993E-01)   ;
%\node[pin={[pin distance=-0.4\onedegree,ecliptics-label]+90:{0$^\circ$}}] at (axis cs:1.59111E-01+360,3.66993E-01) {} ;
\draw [ecliptics-empty] (axis cs:9.03202E+00+360,4.32857E+00) -- (axis cs:9.34691E+00+360,3.59283E+00)   ;
\node[pin={[pin distance=-0.4\onedegree,ecliptics-label]-90:{10$^\circ$}}] at (axis cs:9.03202E+00+360,4.32857E+00) {} ;
\draw [ecliptics-empty] (axis cs:1.83136E+01+360,8.18953E+00) -- (axis cs:1.86183E+01+360,7.44866E+00)   ;
\node[pin={[pin distance=-0.4\onedegree,ecliptics-label]-90:{20$^\circ$}}] at (axis cs:1.83136E+01+360,8.18953E+00) {} ;



\end{axis}

% Equatorial coordinate system
%
% Some are commented out because they are surrounded by already plotted borders
% The x-coordinate is in fractional hours, so must be times 15
%

\begin{axis}[name=constellations,at=(base.center),anchor=center,axis lines=none]


\draw[Equator-full] (axis cs:0.00000E+00,1.00000E-01) -- (axis cs:1.00000E+00,1.00000E-01) -- (axis cs:1.00000E+00,-1.00000E-01) -- (axis cs:0.00000E+00,-1.00000E-01) -- cycle ; 
\draw[Equator-full] (axis cs:2.00000E+00,1.00000E-01) -- (axis cs:3.00000E+00,1.00000E-01) -- (axis cs:3.00000E+00,-1.00000E-01) -- (axis cs:2.00000E+00,-1.00000E-01) -- cycle ; 
\draw[Equator-full] (axis cs:4.00000E+00,1.00000E-01) -- (axis cs:5.00000E+00,1.00000E-01) -- (axis cs:5.00000E+00,-1.00000E-01) -- (axis cs:4.00000E+00,-1.00000E-01) -- cycle ; 
\draw[Equator-full] (axis cs:6.00000E+00,1.00000E-01) -- (axis cs:7.00000E+00,1.00000E-01) -- (axis cs:7.00000E+00,-1.00000E-01) -- (axis cs:6.00000E+00,-1.00000E-01) -- cycle ; 
\draw[Equator-full] (axis cs:8.00000E+00,1.00000E-01) -- (axis cs:9.00000E+00,1.00000E-01) -- (axis cs:9.00000E+00,-1.00000E-01) -- (axis cs:8.00000E+00,-1.00000E-01) -- cycle ; 
\draw[Equator-full] (axis cs:1.00000E+01,1.00000E-01) -- (axis cs:1.10000E+01,1.00000E-01) -- (axis cs:1.10000E+01,-1.00000E-01) -- (axis cs:1.00000E+01,-1.00000E-01) -- cycle ; 
\draw[Equator-full] (axis cs:1.20000E+01,1.00000E-01) -- (axis cs:1.30000E+01,1.00000E-01) -- (axis cs:1.30000E+01,-1.00000E-01) -- (axis cs:1.20000E+01,-1.00000E-01) -- cycle ; 
\draw[Equator-full] (axis cs:1.40000E+01,1.00000E-01) -- (axis cs:1.50000E+01,1.00000E-01) -- (axis cs:1.50000E+01,-1.00000E-01) -- (axis cs:1.40000E+01,-1.00000E-01) -- cycle ; 
\draw[Equator-full] (axis cs:1.60000E+01,1.00000E-01) -- (axis cs:1.70000E+01,1.00000E-01) -- (axis cs:1.70000E+01,-1.00000E-01) -- (axis cs:1.60000E+01,-1.00000E-01) -- cycle ; 
\draw[Equator-full] (axis cs:1.80000E+01,1.00000E-01) -- (axis cs:1.90000E+01,1.00000E-01) -- (axis cs:1.90000E+01,-1.00000E-01) -- (axis cs:1.80000E+01,-1.00000E-01) -- cycle ; 
\draw[Equator-full] (axis cs:2.00000E+01,1.00000E-01) -- (axis cs:2.10000E+01,1.00000E-01) -- (axis cs:2.10000E+01,-1.00000E-01) -- (axis cs:2.00000E+01,-1.00000E-01) -- cycle ; 
\draw[Equator-full] (axis cs:2.20000E+01,1.00000E-01) -- (axis cs:2.30000E+01,1.00000E-01) -- (axis cs:2.30000E+01,-1.00000E-01) -- (axis cs:2.20000E+01,-1.00000E-01) -- cycle ; 
\draw[Equator-full] (axis cs:2.40000E+01,1.00000E-01) -- (axis cs:2.50000E+01,1.00000E-01) -- (axis cs:2.50000E+01,-1.00000E-01) -- (axis cs:2.40000E+01,-1.00000E-01) -- cycle ; 
\draw[Equator-full] (axis cs:2.60000E+01,1.00000E-01) -- (axis cs:2.70000E+01,1.00000E-01) -- (axis cs:2.70000E+01,-1.00000E-01) -- (axis cs:2.60000E+01,-1.00000E-01) -- cycle ; 
\draw[Equator-full] (axis cs:2.80000E+01,1.00000E-01) -- (axis cs:2.90000E+01,1.00000E-01) -- (axis cs:2.90000E+01,-1.00000E-01) -- (axis cs:2.80000E+01,-1.00000E-01) -- cycle ; 
\draw[Equator-full] (axis cs:3.00000E+01,1.00000E-01) -- (axis cs:3.10000E+01,1.00000E-01) -- (axis cs:3.10000E+01,-1.00000E-01) -- (axis cs:3.00000E+01,-1.00000E-01) -- cycle ; 
\draw[Equator-full] (axis cs:3.20000E+01,1.00000E-01) -- (axis cs:3.30000E+01,1.00000E-01) -- (axis cs:3.30000E+01,-1.00000E-01) -- (axis cs:3.20000E+01,-1.00000E-01) -- cycle ; 
\draw[Equator-full] (axis cs:3.40000E+01,1.00000E-01) -- (axis cs:3.50000E+01,1.00000E-01) -- (axis cs:3.50000E+01,-1.00000E-01) -- (axis cs:3.40000E+01,-1.00000E-01) -- cycle ; 
\draw[Equator-full] (axis cs:3.60000E+01,1.00000E-01) -- (axis cs:3.70000E+01,1.00000E-01) -- (axis cs:3.70000E+01,-1.00000E-01) -- (axis cs:3.60000E+01,-1.00000E-01) -- cycle ; 
\draw[Equator-full] (axis cs:3.80000E+01,1.00000E-01) -- (axis cs:3.90000E+01,1.00000E-01) -- (axis cs:3.90000E+01,-1.00000E-01) -- (axis cs:3.80000E+01,-1.00000E-01) -- cycle ; 
\draw[Equator-full] (axis cs:4.00000E+01,1.00000E-01) -- (axis cs:4.10000E+01,1.00000E-01) -- (axis cs:4.10000E+01,-1.00000E-01) -- (axis cs:4.00000E+01,-1.00000E-01) -- cycle ; 
\draw[Equator-full] (axis cs:4.20000E+01,1.00000E-01) -- (axis cs:4.30000E+01,1.00000E-01) -- (axis cs:4.30000E+01,-1.00000E-01) -- (axis cs:4.20000E+01,-1.00000E-01) -- cycle ; 
\draw[Equator-full] (axis cs:4.40000E+01,1.00000E-01) -- (axis cs:4.50000E+01,1.00000E-01) -- (axis cs:4.50000E+01,-1.00000E-01) -- (axis cs:4.40000E+01,-1.00000E-01) -- cycle ; 
\draw[Equator-full] (axis cs:4.60000E+01,1.00000E-01) -- (axis cs:4.70000E+01,1.00000E-01) -- (axis cs:4.70000E+01,-1.00000E-01) -- (axis cs:4.60000E+01,-1.00000E-01) -- cycle ; 
\draw[Equator-full] (axis cs:4.80000E+01,1.00000E-01) -- (axis cs:4.90000E+01,1.00000E-01) -- (axis cs:4.90000E+01,-1.00000E-01) -- (axis cs:4.80000E+01,-1.00000E-01) -- cycle ; 
\draw[Equator-full] (axis cs:5.00000E+01,1.00000E-01) -- (axis cs:5.10000E+01,1.00000E-01) -- (axis cs:5.10000E+01,-1.00000E-01) -- (axis cs:5.00000E+01,-1.00000E-01) -- cycle ; 
\draw[Equator-full] (axis cs:5.20000E+01,1.00000E-01) -- (axis cs:5.30000E+01,1.00000E-01) -- (axis cs:5.30000E+01,-1.00000E-01) -- (axis cs:5.20000E+01,-1.00000E-01) -- cycle ; 
\draw[Equator-full] (axis cs:5.40000E+01,1.00000E-01) -- (axis cs:5.50000E+01,1.00000E-01) -- (axis cs:5.50000E+01,-1.00000E-01) -- (axis cs:5.40000E+01,-1.00000E-01) -- cycle ; 
\draw[Equator-full] (axis cs:5.60000E+01,1.00000E-01) -- (axis cs:5.70000E+01,1.00000E-01) -- (axis cs:5.70000E+01,-1.00000E-01) -- (axis cs:5.60000E+01,-1.00000E-01) -- cycle ; 
\draw[Equator-full] (axis cs:5.80000E+01,1.00000E-01) -- (axis cs:5.90000E+01,1.00000E-01) -- (axis cs:5.90000E+01,-1.00000E-01) -- (axis cs:5.80000E+01,-1.00000E-01) -- cycle ; 
\draw[Equator-full] (axis cs:6.00000E+01,1.00000E-01) -- (axis cs:6.10000E+01,1.00000E-01) -- (axis cs:6.10000E+01,-1.00000E-01) -- (axis cs:6.00000E+01,-1.00000E-01) -- cycle ; 
\draw[Equator-full] (axis cs:6.20000E+01,1.00000E-01) -- (axis cs:6.30000E+01,1.00000E-01) -- (axis cs:6.30000E+01,-1.00000E-01) -- (axis cs:6.20000E+01,-1.00000E-01) -- cycle ; 
\draw[Equator-full] (axis cs:6.40000E+01,1.00000E-01) -- (axis cs:6.50000E+01,1.00000E-01) -- (axis cs:6.50000E+01,-1.00000E-01) -- (axis cs:6.40000E+01,-1.00000E-01) -- cycle ; 
\draw[Equator-full] (axis cs:6.60000E+01,1.00000E-01) -- (axis cs:6.70000E+01,1.00000E-01) -- (axis cs:6.70000E+01,-1.00000E-01) -- (axis cs:6.60000E+01,-1.00000E-01) -- cycle ; 
\draw[Equator-full] (axis cs:6.80000E+01,1.00000E-01) -- (axis cs:6.90000E+01,1.00000E-01) -- (axis cs:6.90000E+01,-1.00000E-01) -- (axis cs:6.80000E+01,-1.00000E-01) -- cycle ; 
\draw[Equator-full] (axis cs:7.00000E+01,1.00000E-01) -- (axis cs:7.10000E+01,1.00000E-01) -- (axis cs:7.10000E+01,-1.00000E-01) -- (axis cs:7.00000E+01,-1.00000E-01) -- cycle ; 
\draw[Equator-full] (axis cs:7.20000E+01,1.00000E-01) -- (axis cs:7.30000E+01,1.00000E-01) -- (axis cs:7.30000E+01,-1.00000E-01) -- (axis cs:7.20000E+01,-1.00000E-01) -- cycle ; 
\draw[Equator-full] (axis cs:7.40000E+01,1.00000E-01) -- (axis cs:7.50000E+01,1.00000E-01) -- (axis cs:7.50000E+01,-1.00000E-01) -- (axis cs:7.40000E+01,-1.00000E-01) -- cycle ; 
\draw[Equator-full] (axis cs:7.60000E+01,1.00000E-01) -- (axis cs:7.70000E+01,1.00000E-01) -- (axis cs:7.70000E+01,-1.00000E-01) -- (axis cs:7.60000E+01,-1.00000E-01) -- cycle ; 
\draw[Equator-full] (axis cs:7.80000E+01,1.00000E-01) -- (axis cs:7.90000E+01,1.00000E-01) -- (axis cs:7.90000E+01,-1.00000E-01) -- (axis cs:7.80000E+01,-1.00000E-01) -- cycle ; 
\draw[Equator-full] (axis cs:8.00000E+01,1.00000E-01) -- (axis cs:8.10000E+01,1.00000E-01) -- (axis cs:8.10000E+01,-1.00000E-01) -- (axis cs:8.00000E+01,-1.00000E-01) -- cycle ; 
\draw[Equator-full] (axis cs:8.20000E+01,1.00000E-01) -- (axis cs:8.30000E+01,1.00000E-01) -- (axis cs:8.30000E+01,-1.00000E-01) -- (axis cs:8.20000E+01,-1.00000E-01) -- cycle ; 
\draw[Equator-full] (axis cs:8.40000E+01,1.00000E-01) -- (axis cs:8.50000E+01,1.00000E-01) -- (axis cs:8.50000E+01,-1.00000E-01) -- (axis cs:8.40000E+01,-1.00000E-01) -- cycle ; 
\draw[Equator-full] (axis cs:8.60000E+01,1.00000E-01) -- (axis cs:8.70000E+01,1.00000E-01) -- (axis cs:8.70000E+01,-1.00000E-01) -- (axis cs:8.60000E+01,-1.00000E-01) -- cycle ; 
\draw[Equator-full] (axis cs:8.80000E+01,1.00000E-01) -- (axis cs:8.90000E+01,1.00000E-01) -- (axis cs:8.90000E+01,-1.00000E-01) -- (axis cs:8.80000E+01,-1.00000E-01) -- cycle ; 
\draw[Equator-full] (axis cs:9.00000E+01,1.00000E-01) -- (axis cs:9.10000E+01,1.00000E-01) -- (axis cs:9.10000E+01,-1.00000E-01) -- (axis cs:9.00000E+01,-1.00000E-01) -- cycle ; 
\draw[Equator-full] (axis cs:9.20000E+01,1.00000E-01) -- (axis cs:9.30000E+01,1.00000E-01) -- (axis cs:9.30000E+01,-1.00000E-01) -- (axis cs:9.20000E+01,-1.00000E-01) -- cycle ; 
\draw[Equator-full] (axis cs:9.40000E+01,1.00000E-01) -- (axis cs:9.50000E+01,1.00000E-01) -- (axis cs:9.50000E+01,-1.00000E-01) -- (axis cs:9.40000E+01,-1.00000E-01) -- cycle ; 
\draw[Equator-full] (axis cs:9.60000E+01,1.00000E-01) -- (axis cs:9.70000E+01,1.00000E-01) -- (axis cs:9.70000E+01,-1.00000E-01) -- (axis cs:9.60000E+01,-1.00000E-01) -- cycle ; 
\draw[Equator-full] (axis cs:9.80000E+01,1.00000E-01) -- (axis cs:9.90000E+01,1.00000E-01) -- (axis cs:9.90000E+01,-1.00000E-01) -- (axis cs:9.80000E+01,-1.00000E-01) -- cycle ; 
\draw[Equator-full] (axis cs:1.00000E+02,1.00000E-01) -- (axis cs:1.01000E+02,1.00000E-01) -- (axis cs:1.01000E+02,-1.00000E-01) -- (axis cs:1.00000E+02,-1.00000E-01) -- cycle ; 
\draw[Equator-full] (axis cs:1.02000E+02,1.00000E-01) -- (axis cs:1.03000E+02,1.00000E-01) -- (axis cs:1.03000E+02,-1.00000E-01) -- (axis cs:1.02000E+02,-1.00000E-01) -- cycle ; 
\draw[Equator-full] (axis cs:1.04000E+02,1.00000E-01) -- (axis cs:1.05000E+02,1.00000E-01) -- (axis cs:1.05000E+02,-1.00000E-01) -- (axis cs:1.04000E+02,-1.00000E-01) -- cycle ; 
\draw[Equator-full] (axis cs:1.06000E+02,1.00000E-01) -- (axis cs:1.07000E+02,1.00000E-01) -- (axis cs:1.07000E+02,-1.00000E-01) -- (axis cs:1.06000E+02,-1.00000E-01) -- cycle ; 
\draw[Equator-full] (axis cs:1.08000E+02,1.00000E-01) -- (axis cs:1.09000E+02,1.00000E-01) -- (axis cs:1.09000E+02,-1.00000E-01) -- (axis cs:1.08000E+02,-1.00000E-01) -- cycle ; 
\draw[Equator-full] (axis cs:1.10000E+02,1.00000E-01) -- (axis cs:1.11000E+02,1.00000E-01) -- (axis cs:1.11000E+02,-1.00000E-01) -- (axis cs:1.10000E+02,-1.00000E-01) -- cycle ; 
\draw[Equator-full] (axis cs:1.12000E+02,1.00000E-01) -- (axis cs:1.13000E+02,1.00000E-01) -- (axis cs:1.13000E+02,-1.00000E-01) -- (axis cs:1.12000E+02,-1.00000E-01) -- cycle ; 
\draw[Equator-full] (axis cs:1.14000E+02,1.00000E-01) -- (axis cs:1.15000E+02,1.00000E-01) -- (axis cs:1.15000E+02,-1.00000E-01) -- (axis cs:1.14000E+02,-1.00000E-01) -- cycle ; 
\draw[Equator-full] (axis cs:1.16000E+02,1.00000E-01) -- (axis cs:1.17000E+02,1.00000E-01) -- (axis cs:1.17000E+02,-1.00000E-01) -- (axis cs:1.16000E+02,-1.00000E-01) -- cycle ; 
\draw[Equator-full] (axis cs:1.18000E+02,1.00000E-01) -- (axis cs:1.19000E+02,1.00000E-01) -- (axis cs:1.19000E+02,-1.00000E-01) -- (axis cs:1.18000E+02,-1.00000E-01) -- cycle ; 
\draw[Equator-full] (axis cs:1.20000E+02,1.00000E-01) -- (axis cs:1.21000E+02,1.00000E-01) -- (axis cs:1.21000E+02,-1.00000E-01) -- (axis cs:1.20000E+02,-1.00000E-01) -- cycle ; 
\draw[Equator-full] (axis cs:1.22000E+02,1.00000E-01) -- (axis cs:1.23000E+02,1.00000E-01) -- (axis cs:1.23000E+02,-1.00000E-01) -- (axis cs:1.22000E+02,-1.00000E-01) -- cycle ; 
\draw[Equator-full] (axis cs:1.24000E+02,1.00000E-01) -- (axis cs:1.25000E+02,1.00000E-01) -- (axis cs:1.25000E+02,-1.00000E-01) -- (axis cs:1.24000E+02,-1.00000E-01) -- cycle ; 
\draw[Equator-full] (axis cs:1.26000E+02,1.00000E-01) -- (axis cs:1.27000E+02,1.00000E-01) -- (axis cs:1.27000E+02,-1.00000E-01) -- (axis cs:1.26000E+02,-1.00000E-01) -- cycle ; 
\draw[Equator-full] (axis cs:1.28000E+02,1.00000E-01) -- (axis cs:1.29000E+02,1.00000E-01) -- (axis cs:1.29000E+02,-1.00000E-01) -- (axis cs:1.28000E+02,-1.00000E-01) -- cycle ; 
\draw[Equator-full] (axis cs:1.30000E+02,1.00000E-01) -- (axis cs:1.31000E+02,1.00000E-01) -- (axis cs:1.31000E+02,-1.00000E-01) -- (axis cs:1.30000E+02,-1.00000E-01) -- cycle ; 
\draw[Equator-full] (axis cs:1.32000E+02,1.00000E-01) -- (axis cs:1.33000E+02,1.00000E-01) -- (axis cs:1.33000E+02,-1.00000E-01) -- (axis cs:1.32000E+02,-1.00000E-01) -- cycle ; 
\draw[Equator-full] (axis cs:1.34000E+02,1.00000E-01) -- (axis cs:1.35000E+02,1.00000E-01) -- (axis cs:1.35000E+02,-1.00000E-01) -- (axis cs:1.34000E+02,-1.00000E-01) -- cycle ; 
\draw[Equator-full] (axis cs:1.36000E+02,1.00000E-01) -- (axis cs:1.37000E+02,1.00000E-01) -- (axis cs:1.37000E+02,-1.00000E-01) -- (axis cs:1.36000E+02,-1.00000E-01) -- cycle ; 
\draw[Equator-full] (axis cs:1.38000E+02,1.00000E-01) -- (axis cs:1.39000E+02,1.00000E-01) -- (axis cs:1.39000E+02,-1.00000E-01) -- (axis cs:1.38000E+02,-1.00000E-01) -- cycle ; 
\draw[Equator-full] (axis cs:1.40000E+02,1.00000E-01) -- (axis cs:1.41000E+02,1.00000E-01) -- (axis cs:1.41000E+02,-1.00000E-01) -- (axis cs:1.40000E+02,-1.00000E-01) -- cycle ; 
\draw[Equator-full] (axis cs:1.42000E+02,1.00000E-01) -- (axis cs:1.43000E+02,1.00000E-01) -- (axis cs:1.43000E+02,-1.00000E-01) -- (axis cs:1.42000E+02,-1.00000E-01) -- cycle ; 
\draw[Equator-full] (axis cs:1.44000E+02,1.00000E-01) -- (axis cs:1.45000E+02,1.00000E-01) -- (axis cs:1.45000E+02,-1.00000E-01) -- (axis cs:1.44000E+02,-1.00000E-01) -- cycle ; 
\draw[Equator-full] (axis cs:1.46000E+02,1.00000E-01) -- (axis cs:1.47000E+02,1.00000E-01) -- (axis cs:1.47000E+02,-1.00000E-01) -- (axis cs:1.46000E+02,-1.00000E-01) -- cycle ; 
\draw[Equator-full] (axis cs:1.48000E+02,1.00000E-01) -- (axis cs:1.49000E+02,1.00000E-01) -- (axis cs:1.49000E+02,-1.00000E-01) -- (axis cs:1.48000E+02,-1.00000E-01) -- cycle ; 
\draw[Equator-full] (axis cs:1.50000E+02,1.00000E-01) -- (axis cs:1.51000E+02,1.00000E-01) -- (axis cs:1.51000E+02,-1.00000E-01) -- (axis cs:1.50000E+02,-1.00000E-01) -- cycle ; 
\draw[Equator-full] (axis cs:1.52000E+02,1.00000E-01) -- (axis cs:1.53000E+02,1.00000E-01) -- (axis cs:1.53000E+02,-1.00000E-01) -- (axis cs:1.52000E+02,-1.00000E-01) -- cycle ; 
\draw[Equator-full] (axis cs:1.54000E+02,1.00000E-01) -- (axis cs:1.55000E+02,1.00000E-01) -- (axis cs:1.55000E+02,-1.00000E-01) -- (axis cs:1.54000E+02,-1.00000E-01) -- cycle ; 
\draw[Equator-full] (axis cs:1.56000E+02,1.00000E-01) -- (axis cs:1.57000E+02,1.00000E-01) -- (axis cs:1.57000E+02,-1.00000E-01) -- (axis cs:1.56000E+02,-1.00000E-01) -- cycle ; 
\draw[Equator-full] (axis cs:1.58000E+02,1.00000E-01) -- (axis cs:1.59000E+02,1.00000E-01) -- (axis cs:1.59000E+02,-1.00000E-01) -- (axis cs:1.58000E+02,-1.00000E-01) -- cycle ; 
\draw[Equator-full] (axis cs:1.60000E+02,1.00000E-01) -- (axis cs:1.61000E+02,1.00000E-01) -- (axis cs:1.61000E+02,-1.00000E-01) -- (axis cs:1.60000E+02,-1.00000E-01) -- cycle ; 
\draw[Equator-full] (axis cs:1.62000E+02,1.00000E-01) -- (axis cs:1.63000E+02,1.00000E-01) -- (axis cs:1.63000E+02,-1.00000E-01) -- (axis cs:1.62000E+02,-1.00000E-01) -- cycle ; 
\draw[Equator-full] (axis cs:1.64000E+02,1.00000E-01) -- (axis cs:1.65000E+02,1.00000E-01) -- (axis cs:1.65000E+02,-1.00000E-01) -- (axis cs:1.64000E+02,-1.00000E-01) -- cycle ; 
\draw[Equator-full] (axis cs:1.66000E+02,1.00000E-01) -- (axis cs:1.67000E+02,1.00000E-01) -- (axis cs:1.67000E+02,-1.00000E-01) -- (axis cs:1.66000E+02,-1.00000E-01) -- cycle ; 
\draw[Equator-full] (axis cs:1.68000E+02,1.00000E-01) -- (axis cs:1.69000E+02,1.00000E-01) -- (axis cs:1.69000E+02,-1.00000E-01) -- (axis cs:1.68000E+02,-1.00000E-01) -- cycle ; 
\draw[Equator-full] (axis cs:1.70000E+02,1.00000E-01) -- (axis cs:1.71000E+02,1.00000E-01) -- (axis cs:1.71000E+02,-1.00000E-01) -- (axis cs:1.70000E+02,-1.00000E-01) -- cycle ; 
\draw[Equator-full] (axis cs:1.72000E+02,1.00000E-01) -- (axis cs:1.73000E+02,1.00000E-01) -- (axis cs:1.73000E+02,-1.00000E-01) -- (axis cs:1.72000E+02,-1.00000E-01) -- cycle ; 
\draw[Equator-full] (axis cs:1.74000E+02,1.00000E-01) -- (axis cs:1.75000E+02,1.00000E-01) -- (axis cs:1.75000E+02,-1.00000E-01) -- (axis cs:1.74000E+02,-1.00000E-01) -- cycle ; 
\draw[Equator-full] (axis cs:1.76000E+02,1.00000E-01) -- (axis cs:1.77000E+02,1.00000E-01) -- (axis cs:1.77000E+02,-1.00000E-01) -- (axis cs:1.76000E+02,-1.00000E-01) -- cycle ; 
\draw[Equator-full] (axis cs:1.78000E+02,1.00000E-01) -- (axis cs:1.79000E+02,1.00000E-01) -- (axis cs:1.79000E+02,-1.00000E-01) -- (axis cs:1.78000E+02,-1.00000E-01) -- cycle ; 
\draw[Equator-full] (axis cs:1.80000E+02,1.00000E-01) -- (axis cs:1.81000E+02,1.00000E-01) -- (axis cs:1.81000E+02,-1.00000E-01) -- (axis cs:1.80000E+02,-1.00000E-01) -- cycle ; 
\draw[Equator-full] (axis cs:1.82000E+02,1.00000E-01) -- (axis cs:1.83000E+02,1.00000E-01) -- (axis cs:1.83000E+02,-1.00000E-01) -- (axis cs:1.82000E+02,-1.00000E-01) -- cycle ; 
\draw[Equator-full] (axis cs:1.84000E+02,1.00000E-01) -- (axis cs:1.85000E+02,1.00000E-01) -- (axis cs:1.85000E+02,-1.00000E-01) -- (axis cs:1.84000E+02,-1.00000E-01) -- cycle ; 
\draw[Equator-full] (axis cs:1.86000E+02,1.00000E-01) -- (axis cs:1.87000E+02,1.00000E-01) -- (axis cs:1.87000E+02,-1.00000E-01) -- (axis cs:1.86000E+02,-1.00000E-01) -- cycle ; 
\draw[Equator-full] (axis cs:1.88000E+02,1.00000E-01) -- (axis cs:1.89000E+02,1.00000E-01) -- (axis cs:1.89000E+02,-1.00000E-01) -- (axis cs:1.88000E+02,-1.00000E-01) -- cycle ; 
\draw[Equator-full] (axis cs:1.90000E+02,1.00000E-01) -- (axis cs:1.91000E+02,1.00000E-01) -- (axis cs:1.91000E+02,-1.00000E-01) -- (axis cs:1.90000E+02,-1.00000E-01) -- cycle ; 
\draw[Equator-full] (axis cs:1.92000E+02,1.00000E-01) -- (axis cs:1.93000E+02,1.00000E-01) -- (axis cs:1.93000E+02,-1.00000E-01) -- (axis cs:1.92000E+02,-1.00000E-01) -- cycle ; 
\draw[Equator-full] (axis cs:1.94000E+02,1.00000E-01) -- (axis cs:1.95000E+02,1.00000E-01) -- (axis cs:1.95000E+02,-1.00000E-01) -- (axis cs:1.94000E+02,-1.00000E-01) -- cycle ; 
\draw[Equator-full] (axis cs:1.96000E+02,1.00000E-01) -- (axis cs:1.97000E+02,1.00000E-01) -- (axis cs:1.97000E+02,-1.00000E-01) -- (axis cs:1.96000E+02,-1.00000E-01) -- cycle ; 
\draw[Equator-full] (axis cs:1.98000E+02,1.00000E-01) -- (axis cs:1.99000E+02,1.00000E-01) -- (axis cs:1.99000E+02,-1.00000E-01) -- (axis cs:1.98000E+02,-1.00000E-01) -- cycle ; 
\draw[Equator-full] (axis cs:2.00000E+02,1.00000E-01) -- (axis cs:2.01000E+02,1.00000E-01) -- (axis cs:2.01000E+02,-1.00000E-01) -- (axis cs:2.00000E+02,-1.00000E-01) -- cycle ; 
\draw[Equator-full] (axis cs:2.02000E+02,1.00000E-01) -- (axis cs:2.03000E+02,1.00000E-01) -- (axis cs:2.03000E+02,-1.00000E-01) -- (axis cs:2.02000E+02,-1.00000E-01) -- cycle ; 
\draw[Equator-full] (axis cs:2.04000E+02,1.00000E-01) -- (axis cs:2.05000E+02,1.00000E-01) -- (axis cs:2.05000E+02,-1.00000E-01) -- (axis cs:2.04000E+02,-1.00000E-01) -- cycle ; 
\draw[Equator-full] (axis cs:2.06000E+02,1.00000E-01) -- (axis cs:2.07000E+02,1.00000E-01) -- (axis cs:2.07000E+02,-1.00000E-01) -- (axis cs:2.06000E+02,-1.00000E-01) -- cycle ; 
\draw[Equator-full] (axis cs:2.08000E+02,1.00000E-01) -- (axis cs:2.09000E+02,1.00000E-01) -- (axis cs:2.09000E+02,-1.00000E-01) -- (axis cs:2.08000E+02,-1.00000E-01) -- cycle ; 
\draw[Equator-full] (axis cs:2.10000E+02,1.00000E-01) -- (axis cs:2.11000E+02,1.00000E-01) -- (axis cs:2.11000E+02,-1.00000E-01) -- (axis cs:2.10000E+02,-1.00000E-01) -- cycle ; 
\draw[Equator-full] (axis cs:2.12000E+02,1.00000E-01) -- (axis cs:2.13000E+02,1.00000E-01) -- (axis cs:2.13000E+02,-1.00000E-01) -- (axis cs:2.12000E+02,-1.00000E-01) -- cycle ; 
\draw[Equator-full] (axis cs:2.14000E+02,1.00000E-01) -- (axis cs:2.15000E+02,1.00000E-01) -- (axis cs:2.15000E+02,-1.00000E-01) -- (axis cs:2.14000E+02,-1.00000E-01) -- cycle ; 
\draw[Equator-full] (axis cs:2.16000E+02,1.00000E-01) -- (axis cs:2.17000E+02,1.00000E-01) -- (axis cs:2.17000E+02,-1.00000E-01) -- (axis cs:2.16000E+02,-1.00000E-01) -- cycle ; 
\draw[Equator-full] (axis cs:2.18000E+02,1.00000E-01) -- (axis cs:2.19000E+02,1.00000E-01) -- (axis cs:2.19000E+02,-1.00000E-01) -- (axis cs:2.18000E+02,-1.00000E-01) -- cycle ; 
\draw[Equator-full] (axis cs:2.20000E+02,1.00000E-01) -- (axis cs:2.21000E+02,1.00000E-01) -- (axis cs:2.21000E+02,-1.00000E-01) -- (axis cs:2.20000E+02,-1.00000E-01) -- cycle ; 
\draw[Equator-full] (axis cs:2.22000E+02,1.00000E-01) -- (axis cs:2.23000E+02,1.00000E-01) -- (axis cs:2.23000E+02,-1.00000E-01) -- (axis cs:2.22000E+02,-1.00000E-01) -- cycle ; 
\draw[Equator-full] (axis cs:2.24000E+02,1.00000E-01) -- (axis cs:2.25000E+02,1.00000E-01) -- (axis cs:2.25000E+02,-1.00000E-01) -- (axis cs:2.24000E+02,-1.00000E-01) -- cycle ; 
\draw[Equator-full] (axis cs:2.26000E+02,1.00000E-01) -- (axis cs:2.27000E+02,1.00000E-01) -- (axis cs:2.27000E+02,-1.00000E-01) -- (axis cs:2.26000E+02,-1.00000E-01) -- cycle ; 
\draw[Equator-full] (axis cs:2.28000E+02,1.00000E-01) -- (axis cs:2.29000E+02,1.00000E-01) -- (axis cs:2.29000E+02,-1.00000E-01) -- (axis cs:2.28000E+02,-1.00000E-01) -- cycle ; 
\draw[Equator-full] (axis cs:2.30000E+02,1.00000E-01) -- (axis cs:2.31000E+02,1.00000E-01) -- (axis cs:2.31000E+02,-1.00000E-01) -- (axis cs:2.30000E+02,-1.00000E-01) -- cycle ; 
\draw[Equator-full] (axis cs:2.32000E+02,1.00000E-01) -- (axis cs:2.33000E+02,1.00000E-01) -- (axis cs:2.33000E+02,-1.00000E-01) -- (axis cs:2.32000E+02,-1.00000E-01) -- cycle ; 
\draw[Equator-full] (axis cs:2.34000E+02,1.00000E-01) -- (axis cs:2.35000E+02,1.00000E-01) -- (axis cs:2.35000E+02,-1.00000E-01) -- (axis cs:2.34000E+02,-1.00000E-01) -- cycle ; 
\draw[Equator-full] (axis cs:2.36000E+02,1.00000E-01) -- (axis cs:2.37000E+02,1.00000E-01) -- (axis cs:2.37000E+02,-1.00000E-01) -- (axis cs:2.36000E+02,-1.00000E-01) -- cycle ; 
\draw[Equator-full] (axis cs:2.38000E+02,1.00000E-01) -- (axis cs:2.39000E+02,1.00000E-01) -- (axis cs:2.39000E+02,-1.00000E-01) -- (axis cs:2.38000E+02,-1.00000E-01) -- cycle ; 
\draw[Equator-full] (axis cs:2.40000E+02,1.00000E-01) -- (axis cs:2.41000E+02,1.00000E-01) -- (axis cs:2.41000E+02,-1.00000E-01) -- (axis cs:2.40000E+02,-1.00000E-01) -- cycle ; 
\draw[Equator-full] (axis cs:2.42000E+02,1.00000E-01) -- (axis cs:2.43000E+02,1.00000E-01) -- (axis cs:2.43000E+02,-1.00000E-01) -- (axis cs:2.42000E+02,-1.00000E-01) -- cycle ; 
\draw[Equator-full] (axis cs:2.44000E+02,1.00000E-01) -- (axis cs:2.45000E+02,1.00000E-01) -- (axis cs:2.45000E+02,-1.00000E-01) -- (axis cs:2.44000E+02,-1.00000E-01) -- cycle ; 
\draw[Equator-full] (axis cs:2.46000E+02,1.00000E-01) -- (axis cs:2.47000E+02,1.00000E-01) -- (axis cs:2.47000E+02,-1.00000E-01) -- (axis cs:2.46000E+02,-1.00000E-01) -- cycle ; 
\draw[Equator-full] (axis cs:2.48000E+02,1.00000E-01) -- (axis cs:2.49000E+02,1.00000E-01) -- (axis cs:2.49000E+02,-1.00000E-01) -- (axis cs:2.48000E+02,-1.00000E-01) -- cycle ; 
\draw[Equator-full] (axis cs:2.50000E+02,1.00000E-01) -- (axis cs:2.51000E+02,1.00000E-01) -- (axis cs:2.51000E+02,-1.00000E-01) -- (axis cs:2.50000E+02,-1.00000E-01) -- cycle ; 
\draw[Equator-full] (axis cs:2.52000E+02,1.00000E-01) -- (axis cs:2.53000E+02,1.00000E-01) -- (axis cs:2.53000E+02,-1.00000E-01) -- (axis cs:2.52000E+02,-1.00000E-01) -- cycle ; 
\draw[Equator-full] (axis cs:2.54000E+02,1.00000E-01) -- (axis cs:2.55000E+02,1.00000E-01) -- (axis cs:2.55000E+02,-1.00000E-01) -- (axis cs:2.54000E+02,-1.00000E-01) -- cycle ; 
\draw[Equator-full] (axis cs:2.56000E+02,1.00000E-01) -- (axis cs:2.57000E+02,1.00000E-01) -- (axis cs:2.57000E+02,-1.00000E-01) -- (axis cs:2.56000E+02,-1.00000E-01) -- cycle ; 
\draw[Equator-full] (axis cs:2.58000E+02,1.00000E-01) -- (axis cs:2.59000E+02,1.00000E-01) -- (axis cs:2.59000E+02,-1.00000E-01) -- (axis cs:2.58000E+02,-1.00000E-01) -- cycle ; 
\draw[Equator-full] (axis cs:2.60000E+02,1.00000E-01) -- (axis cs:2.61000E+02,1.00000E-01) -- (axis cs:2.61000E+02,-1.00000E-01) -- (axis cs:2.60000E+02,-1.00000E-01) -- cycle ; 
\draw[Equator-full] (axis cs:2.62000E+02,1.00000E-01) -- (axis cs:2.63000E+02,1.00000E-01) -- (axis cs:2.63000E+02,-1.00000E-01) -- (axis cs:2.62000E+02,-1.00000E-01) -- cycle ; 
\draw[Equator-full] (axis cs:2.64000E+02,1.00000E-01) -- (axis cs:2.65000E+02,1.00000E-01) -- (axis cs:2.65000E+02,-1.00000E-01) -- (axis cs:2.64000E+02,-1.00000E-01) -- cycle ; 
\draw[Equator-full] (axis cs:2.66000E+02,1.00000E-01) -- (axis cs:2.67000E+02,1.00000E-01) -- (axis cs:2.67000E+02,-1.00000E-01) -- (axis cs:2.66000E+02,-1.00000E-01) -- cycle ; 
\draw[Equator-full] (axis cs:2.68000E+02,1.00000E-01) -- (axis cs:2.69000E+02,1.00000E-01) -- (axis cs:2.69000E+02,-1.00000E-01) -- (axis cs:2.68000E+02,-1.00000E-01) -- cycle ; 
\draw[Equator-full] (axis cs:2.70000E+02,1.00000E-01) -- (axis cs:2.71000E+02,1.00000E-01) -- (axis cs:2.71000E+02,-1.00000E-01) -- (axis cs:2.70000E+02,-1.00000E-01) -- cycle ; 
\draw[Equator-full] (axis cs:2.72000E+02,1.00000E-01) -- (axis cs:2.73000E+02,1.00000E-01) -- (axis cs:2.73000E+02,-1.00000E-01) -- (axis cs:2.72000E+02,-1.00000E-01) -- cycle ; 
\draw[Equator-full] (axis cs:2.74000E+02,1.00000E-01) -- (axis cs:2.75000E+02,1.00000E-01) -- (axis cs:2.75000E+02,-1.00000E-01) -- (axis cs:2.74000E+02,-1.00000E-01) -- cycle ; 
\draw[Equator-full] (axis cs:2.76000E+02,1.00000E-01) -- (axis cs:2.77000E+02,1.00000E-01) -- (axis cs:2.77000E+02,-1.00000E-01) -- (axis cs:2.76000E+02,-1.00000E-01) -- cycle ; 
\draw[Equator-full] (axis cs:2.78000E+02,1.00000E-01) -- (axis cs:2.79000E+02,1.00000E-01) -- (axis cs:2.79000E+02,-1.00000E-01) -- (axis cs:2.78000E+02,-1.00000E-01) -- cycle ; 
\draw[Equator-full] (axis cs:2.80000E+02,1.00000E-01) -- (axis cs:2.81000E+02,1.00000E-01) -- (axis cs:2.81000E+02,-1.00000E-01) -- (axis cs:2.80000E+02,-1.00000E-01) -- cycle ; 
\draw[Equator-full] (axis cs:2.82000E+02,1.00000E-01) -- (axis cs:2.83000E+02,1.00000E-01) -- (axis cs:2.83000E+02,-1.00000E-01) -- (axis cs:2.82000E+02,-1.00000E-01) -- cycle ; 
\draw[Equator-full] (axis cs:2.84000E+02,1.00000E-01) -- (axis cs:2.85000E+02,1.00000E-01) -- (axis cs:2.85000E+02,-1.00000E-01) -- (axis cs:2.84000E+02,-1.00000E-01) -- cycle ; 
\draw[Equator-full] (axis cs:2.86000E+02,1.00000E-01) -- (axis cs:2.87000E+02,1.00000E-01) -- (axis cs:2.87000E+02,-1.00000E-01) -- (axis cs:2.86000E+02,-1.00000E-01) -- cycle ; 
\draw[Equator-full] (axis cs:2.88000E+02,1.00000E-01) -- (axis cs:2.89000E+02,1.00000E-01) -- (axis cs:2.89000E+02,-1.00000E-01) -- (axis cs:2.88000E+02,-1.00000E-01) -- cycle ; 
\draw[Equator-full] (axis cs:2.90000E+02,1.00000E-01) -- (axis cs:2.91000E+02,1.00000E-01) -- (axis cs:2.91000E+02,-1.00000E-01) -- (axis cs:2.90000E+02,-1.00000E-01) -- cycle ; 
\draw[Equator-full] (axis cs:2.92000E+02,1.00000E-01) -- (axis cs:2.93000E+02,1.00000E-01) -- (axis cs:2.93000E+02,-1.00000E-01) -- (axis cs:2.92000E+02,-1.00000E-01) -- cycle ; 
\draw[Equator-full] (axis cs:2.94000E+02,1.00000E-01) -- (axis cs:2.95000E+02,1.00000E-01) -- (axis cs:2.95000E+02,-1.00000E-01) -- (axis cs:2.94000E+02,-1.00000E-01) -- cycle ; 
\draw[Equator-full] (axis cs:2.96000E+02,1.00000E-01) -- (axis cs:2.97000E+02,1.00000E-01) -- (axis cs:2.97000E+02,-1.00000E-01) -- (axis cs:2.96000E+02,-1.00000E-01) -- cycle ; 
\draw[Equator-full] (axis cs:2.98000E+02,1.00000E-01) -- (axis cs:2.99000E+02,1.00000E-01) -- (axis cs:2.99000E+02,-1.00000E-01) -- (axis cs:2.98000E+02,-1.00000E-01) -- cycle ; 
\draw[Equator-full] (axis cs:3.00000E+02,1.00000E-01) -- (axis cs:3.01000E+02,1.00000E-01) -- (axis cs:3.01000E+02,-1.00000E-01) -- (axis cs:3.00000E+02,-1.00000E-01) -- cycle ; 
\draw[Equator-full] (axis cs:3.02000E+02,1.00000E-01) -- (axis cs:3.03000E+02,1.00000E-01) -- (axis cs:3.03000E+02,-1.00000E-01) -- (axis cs:3.02000E+02,-1.00000E-01) -- cycle ; 
\draw[Equator-full] (axis cs:3.04000E+02,1.00000E-01) -- (axis cs:3.05000E+02,1.00000E-01) -- (axis cs:3.05000E+02,-1.00000E-01) -- (axis cs:3.04000E+02,-1.00000E-01) -- cycle ; 
\draw[Equator-full] (axis cs:3.06000E+02,1.00000E-01) -- (axis cs:3.07000E+02,1.00000E-01) -- (axis cs:3.07000E+02,-1.00000E-01) -- (axis cs:3.06000E+02,-1.00000E-01) -- cycle ; 
\draw[Equator-full] (axis cs:3.08000E+02,1.00000E-01) -- (axis cs:3.09000E+02,1.00000E-01) -- (axis cs:3.09000E+02,-1.00000E-01) -- (axis cs:3.08000E+02,-1.00000E-01) -- cycle ; 
\draw[Equator-full] (axis cs:3.10000E+02,1.00000E-01) -- (axis cs:3.11000E+02,1.00000E-01) -- (axis cs:3.11000E+02,-1.00000E-01) -- (axis cs:3.10000E+02,-1.00000E-01) -- cycle ; 
\draw[Equator-full] (axis cs:3.12000E+02,1.00000E-01) -- (axis cs:3.13000E+02,1.00000E-01) -- (axis cs:3.13000E+02,-1.00000E-01) -- (axis cs:3.12000E+02,-1.00000E-01) -- cycle ; 
\draw[Equator-full] (axis cs:3.14000E+02,1.00000E-01) -- (axis cs:3.15000E+02,1.00000E-01) -- (axis cs:3.15000E+02,-1.00000E-01) -- (axis cs:3.14000E+02,-1.00000E-01) -- cycle ; 
\draw[Equator-full] (axis cs:3.16000E+02,1.00000E-01) -- (axis cs:3.17000E+02,1.00000E-01) -- (axis cs:3.17000E+02,-1.00000E-01) -- (axis cs:3.16000E+02,-1.00000E-01) -- cycle ; 
\draw[Equator-full] (axis cs:3.18000E+02,1.00000E-01) -- (axis cs:3.19000E+02,1.00000E-01) -- (axis cs:3.19000E+02,-1.00000E-01) -- (axis cs:3.18000E+02,-1.00000E-01) -- cycle ; 
\draw[Equator-full] (axis cs:3.20000E+02,1.00000E-01) -- (axis cs:3.21000E+02,1.00000E-01) -- (axis cs:3.21000E+02,-1.00000E-01) -- (axis cs:3.20000E+02,-1.00000E-01) -- cycle ; 
\draw[Equator-full] (axis cs:3.22000E+02,1.00000E-01) -- (axis cs:3.23000E+02,1.00000E-01) -- (axis cs:3.23000E+02,-1.00000E-01) -- (axis cs:3.22000E+02,-1.00000E-01) -- cycle ; 
\draw[Equator-full] (axis cs:3.24000E+02,1.00000E-01) -- (axis cs:3.25000E+02,1.00000E-01) -- (axis cs:3.25000E+02,-1.00000E-01) -- (axis cs:3.24000E+02,-1.00000E-01) -- cycle ; 
\draw[Equator-full] (axis cs:3.26000E+02,1.00000E-01) -- (axis cs:3.27000E+02,1.00000E-01) -- (axis cs:3.27000E+02,-1.00000E-01) -- (axis cs:3.26000E+02,-1.00000E-01) -- cycle ; 
\draw[Equator-full] (axis cs:3.28000E+02,1.00000E-01) -- (axis cs:3.29000E+02,1.00000E-01) -- (axis cs:3.29000E+02,-1.00000E-01) -- (axis cs:3.28000E+02,-1.00000E-01) -- cycle ; 
\draw[Equator-full] (axis cs:3.30000E+02,1.00000E-01) -- (axis cs:3.31000E+02,1.00000E-01) -- (axis cs:3.31000E+02,-1.00000E-01) -- (axis cs:3.30000E+02,-1.00000E-01) -- cycle ; 
\draw[Equator-full] (axis cs:3.32000E+02,1.00000E-01) -- (axis cs:3.33000E+02,1.00000E-01) -- (axis cs:3.33000E+02,-1.00000E-01) -- (axis cs:3.32000E+02,-1.00000E-01) -- cycle ; 
\draw[Equator-full] (axis cs:3.34000E+02,1.00000E-01) -- (axis cs:3.35000E+02,1.00000E-01) -- (axis cs:3.35000E+02,-1.00000E-01) -- (axis cs:3.34000E+02,-1.00000E-01) -- cycle ; 
\draw[Equator-full] (axis cs:3.36000E+02,1.00000E-01) -- (axis cs:3.37000E+02,1.00000E-01) -- (axis cs:3.37000E+02,-1.00000E-01) -- (axis cs:3.36000E+02,-1.00000E-01) -- cycle ; 
\draw[Equator-full] (axis cs:3.38000E+02,1.00000E-01) -- (axis cs:3.39000E+02,1.00000E-01) -- (axis cs:3.39000E+02,-1.00000E-01) -- (axis cs:3.38000E+02,-1.00000E-01) -- cycle ; 
\draw[Equator-full] (axis cs:3.40000E+02,1.00000E-01) -- (axis cs:3.41000E+02,1.00000E-01) -- (axis cs:3.41000E+02,-1.00000E-01) -- (axis cs:3.40000E+02,-1.00000E-01) -- cycle ; 
\draw[Equator-full] (axis cs:3.42000E+02,1.00000E-01) -- (axis cs:3.43000E+02,1.00000E-01) -- (axis cs:3.43000E+02,-1.00000E-01) -- (axis cs:3.42000E+02,-1.00000E-01) -- cycle ; 
\draw[Equator-full] (axis cs:3.44000E+02,1.00000E-01) -- (axis cs:3.45000E+02,1.00000E-01) -- (axis cs:3.45000E+02,-1.00000E-01) -- (axis cs:3.44000E+02,-1.00000E-01) -- cycle ; 
\draw[Equator-full] (axis cs:3.46000E+02,1.00000E-01) -- (axis cs:3.47000E+02,1.00000E-01) -- (axis cs:3.47000E+02,-1.00000E-01) -- (axis cs:3.46000E+02,-1.00000E-01) -- cycle ; 
\draw[Equator-full] (axis cs:3.48000E+02,1.00000E-01) -- (axis cs:3.49000E+02,1.00000E-01) -- (axis cs:3.49000E+02,-1.00000E-01) -- (axis cs:3.48000E+02,-1.00000E-01) -- cycle ; 
\draw[Equator-full] (axis cs:3.50000E+02,1.00000E-01) -- (axis cs:3.51000E+02,1.00000E-01) -- (axis cs:3.51000E+02,-1.00000E-01) -- (axis cs:3.50000E+02,-1.00000E-01) -- cycle ; 
\draw[Equator-full] (axis cs:3.52000E+02,1.00000E-01) -- (axis cs:3.53000E+02,1.00000E-01) -- (axis cs:3.53000E+02,-1.00000E-01) -- (axis cs:3.52000E+02,-1.00000E-01) -- cycle ; 
\draw[Equator-full] (axis cs:3.54000E+02,1.00000E-01) -- (axis cs:3.55000E+02,1.00000E-01) -- (axis cs:3.55000E+02,-1.00000E-01) -- (axis cs:3.54000E+02,-1.00000E-01) -- cycle ; 
\draw[Equator-full] (axis cs:3.56000E+02,1.00000E-01) -- (axis cs:3.57000E+02,1.00000E-01) -- (axis cs:3.57000E+02,-1.00000E-01) -- (axis cs:3.56000E+02,-1.00000E-01) -- cycle ; 
\draw[Equator-full] (axis cs:3.58000E+02,1.00000E-01) -- (axis cs:3.59000E+02,1.00000E-01) -- (axis cs:3.59000E+02,-1.00000E-01) -- (axis cs:3.58000E+02,-1.00000E-01) -- cycle ; 
\draw[Equator-empty] (axis cs:1.00000E+00,1.00000E-01) -- (axis cs:2.00000E+00,1.00000E-01) -- (axis cs:2.00000E+00,-1.00000E-01) -- (axis cs:1.00000E+00,-1.00000E-01) -- cycle ; 
\draw[Equator-empty] (axis cs:3.00000E+00,1.00000E-01) -- (axis cs:4.00000E+00,1.00000E-01) -- (axis cs:4.00000E+00,-1.00000E-01) -- (axis cs:3.00000E+00,-1.00000E-01) -- cycle ; 
\draw[Equator-empty] (axis cs:5.00000E+00,1.00000E-01) -- (axis cs:6.00000E+00,1.00000E-01) -- (axis cs:6.00000E+00,-1.00000E-01) -- (axis cs:5.00000E+00,-1.00000E-01) -- cycle ; 
\draw[Equator-empty] (axis cs:7.00000E+00,1.00000E-01) -- (axis cs:8.00000E+00,1.00000E-01) -- (axis cs:8.00000E+00,-1.00000E-01) -- (axis cs:7.00000E+00,-1.00000E-01) -- cycle ; 
\draw[Equator-empty] (axis cs:9.00000E+00,1.00000E-01) -- (axis cs:1.00000E+01,1.00000E-01) -- (axis cs:1.00000E+01,-1.00000E-01) -- (axis cs:9.00000E+00,-1.00000E-01) -- cycle ; 
\draw[Equator-empty] (axis cs:1.10000E+01,1.00000E-01) -- (axis cs:1.20000E+01,1.00000E-01) -- (axis cs:1.20000E+01,-1.00000E-01) -- (axis cs:1.10000E+01,-1.00000E-01) -- cycle ; 
\draw[Equator-empty] (axis cs:1.30000E+01,1.00000E-01) -- (axis cs:1.40000E+01,1.00000E-01) -- (axis cs:1.40000E+01,-1.00000E-01) -- (axis cs:1.30000E+01,-1.00000E-01) -- cycle ; 
\draw[Equator-empty] (axis cs:1.50000E+01,1.00000E-01) -- (axis cs:1.60000E+01,1.00000E-01) -- (axis cs:1.60000E+01,-1.00000E-01) -- (axis cs:1.50000E+01,-1.00000E-01) -- cycle ; 
\draw[Equator-empty] (axis cs:1.70000E+01,1.00000E-01) -- (axis cs:1.80000E+01,1.00000E-01) -- (axis cs:1.80000E+01,-1.00000E-01) -- (axis cs:1.70000E+01,-1.00000E-01) -- cycle ; 
\draw[Equator-empty] (axis cs:1.90000E+01,1.00000E-01) -- (axis cs:2.00000E+01,1.00000E-01) -- (axis cs:2.00000E+01,-1.00000E-01) -- (axis cs:1.90000E+01,-1.00000E-01) -- cycle ; 
\draw[Equator-empty] (axis cs:2.10000E+01,1.00000E-01) -- (axis cs:2.20000E+01,1.00000E-01) -- (axis cs:2.20000E+01,-1.00000E-01) -- (axis cs:2.10000E+01,-1.00000E-01) -- cycle ; 
\draw[Equator-empty] (axis cs:2.30000E+01,1.00000E-01) -- (axis cs:2.40000E+01,1.00000E-01) -- (axis cs:2.40000E+01,-1.00000E-01) -- (axis cs:2.30000E+01,-1.00000E-01) -- cycle ; 
\draw[Equator-empty] (axis cs:2.50000E+01,1.00000E-01) -- (axis cs:2.60000E+01,1.00000E-01) -- (axis cs:2.60000E+01,-1.00000E-01) -- (axis cs:2.50000E+01,-1.00000E-01) -- cycle ; 
\draw[Equator-empty] (axis cs:2.70000E+01,1.00000E-01) -- (axis cs:2.80000E+01,1.00000E-01) -- (axis cs:2.80000E+01,-1.00000E-01) -- (axis cs:2.70000E+01,-1.00000E-01) -- cycle ; 
\draw[Equator-empty] (axis cs:2.90000E+01,1.00000E-01) -- (axis cs:3.00000E+01,1.00000E-01) -- (axis cs:3.00000E+01,-1.00000E-01) -- (axis cs:2.90000E+01,-1.00000E-01) -- cycle ; 
\draw[Equator-empty] (axis cs:3.10000E+01,1.00000E-01) -- (axis cs:3.20000E+01,1.00000E-01) -- (axis cs:3.20000E+01,-1.00000E-01) -- (axis cs:3.10000E+01,-1.00000E-01) -- cycle ; 
\draw[Equator-empty] (axis cs:3.30000E+01,1.00000E-01) -- (axis cs:3.40000E+01,1.00000E-01) -- (axis cs:3.40000E+01,-1.00000E-01) -- (axis cs:3.30000E+01,-1.00000E-01) -- cycle ; 
\draw[Equator-empty] (axis cs:3.50000E+01,1.00000E-01) -- (axis cs:3.60000E+01,1.00000E-01) -- (axis cs:3.60000E+01,-1.00000E-01) -- (axis cs:3.50000E+01,-1.00000E-01) -- cycle ; 
\draw[Equator-empty] (axis cs:3.70000E+01,1.00000E-01) -- (axis cs:3.80000E+01,1.00000E-01) -- (axis cs:3.80000E+01,-1.00000E-01) -- (axis cs:3.70000E+01,-1.00000E-01) -- cycle ; 
\draw[Equator-empty] (axis cs:3.90000E+01,1.00000E-01) -- (axis cs:4.00000E+01,1.00000E-01) -- (axis cs:4.00000E+01,-1.00000E-01) -- (axis cs:3.90000E+01,-1.00000E-01) -- cycle ; 
\draw[Equator-empty] (axis cs:4.10000E+01,1.00000E-01) -- (axis cs:4.20000E+01,1.00000E-01) -- (axis cs:4.20000E+01,-1.00000E-01) -- (axis cs:4.10000E+01,-1.00000E-01) -- cycle ; 
\draw[Equator-empty] (axis cs:4.30000E+01,1.00000E-01) -- (axis cs:4.40000E+01,1.00000E-01) -- (axis cs:4.40000E+01,-1.00000E-01) -- (axis cs:4.30000E+01,-1.00000E-01) -- cycle ; 
\draw[Equator-empty] (axis cs:4.50000E+01,1.00000E-01) -- (axis cs:4.60000E+01,1.00000E-01) -- (axis cs:4.60000E+01,-1.00000E-01) -- (axis cs:4.50000E+01,-1.00000E-01) -- cycle ; 
\draw[Equator-empty] (axis cs:4.70000E+01,1.00000E-01) -- (axis cs:4.80000E+01,1.00000E-01) -- (axis cs:4.80000E+01,-1.00000E-01) -- (axis cs:4.70000E+01,-1.00000E-01) -- cycle ; 
\draw[Equator-empty] (axis cs:4.90000E+01,1.00000E-01) -- (axis cs:5.00000E+01,1.00000E-01) -- (axis cs:5.00000E+01,-1.00000E-01) -- (axis cs:4.90000E+01,-1.00000E-01) -- cycle ; 
\draw[Equator-empty] (axis cs:5.10000E+01,1.00000E-01) -- (axis cs:5.20000E+01,1.00000E-01) -- (axis cs:5.20000E+01,-1.00000E-01) -- (axis cs:5.10000E+01,-1.00000E-01) -- cycle ; 
\draw[Equator-empty] (axis cs:5.30000E+01,1.00000E-01) -- (axis cs:5.40000E+01,1.00000E-01) -- (axis cs:5.40000E+01,-1.00000E-01) -- (axis cs:5.30000E+01,-1.00000E-01) -- cycle ; 
\draw[Equator-empty] (axis cs:5.50000E+01,1.00000E-01) -- (axis cs:5.60000E+01,1.00000E-01) -- (axis cs:5.60000E+01,-1.00000E-01) -- (axis cs:5.50000E+01,-1.00000E-01) -- cycle ; 
\draw[Equator-empty] (axis cs:5.70000E+01,1.00000E-01) -- (axis cs:5.80000E+01,1.00000E-01) -- (axis cs:5.80000E+01,-1.00000E-01) -- (axis cs:5.70000E+01,-1.00000E-01) -- cycle ; 
\draw[Equator-empty] (axis cs:5.90000E+01,1.00000E-01) -- (axis cs:6.00000E+01,1.00000E-01) -- (axis cs:6.00000E+01,-1.00000E-01) -- (axis cs:5.90000E+01,-1.00000E-01) -- cycle ; 
\draw[Equator-empty] (axis cs:6.10000E+01,1.00000E-01) -- (axis cs:6.20000E+01,1.00000E-01) -- (axis cs:6.20000E+01,-1.00000E-01) -- (axis cs:6.10000E+01,-1.00000E-01) -- cycle ; 
\draw[Equator-empty] (axis cs:6.30000E+01,1.00000E-01) -- (axis cs:6.40000E+01,1.00000E-01) -- (axis cs:6.40000E+01,-1.00000E-01) -- (axis cs:6.30000E+01,-1.00000E-01) -- cycle ; 
\draw[Equator-empty] (axis cs:6.50000E+01,1.00000E-01) -- (axis cs:6.60000E+01,1.00000E-01) -- (axis cs:6.60000E+01,-1.00000E-01) -- (axis cs:6.50000E+01,-1.00000E-01) -- cycle ; 
\draw[Equator-empty] (axis cs:6.70000E+01,1.00000E-01) -- (axis cs:6.80000E+01,1.00000E-01) -- (axis cs:6.80000E+01,-1.00000E-01) -- (axis cs:6.70000E+01,-1.00000E-01) -- cycle ; 
\draw[Equator-empty] (axis cs:6.90000E+01,1.00000E-01) -- (axis cs:7.00000E+01,1.00000E-01) -- (axis cs:7.00000E+01,-1.00000E-01) -- (axis cs:6.90000E+01,-1.00000E-01) -- cycle ; 
\draw[Equator-empty] (axis cs:7.10000E+01,1.00000E-01) -- (axis cs:7.20000E+01,1.00000E-01) -- (axis cs:7.20000E+01,-1.00000E-01) -- (axis cs:7.10000E+01,-1.00000E-01) -- cycle ; 
\draw[Equator-empty] (axis cs:7.30000E+01,1.00000E-01) -- (axis cs:7.40000E+01,1.00000E-01) -- (axis cs:7.40000E+01,-1.00000E-01) -- (axis cs:7.30000E+01,-1.00000E-01) -- cycle ; 
\draw[Equator-empty] (axis cs:7.50000E+01,1.00000E-01) -- (axis cs:7.60000E+01,1.00000E-01) -- (axis cs:7.60000E+01,-1.00000E-01) -- (axis cs:7.50000E+01,-1.00000E-01) -- cycle ; 
\draw[Equator-empty] (axis cs:7.70000E+01,1.00000E-01) -- (axis cs:7.80000E+01,1.00000E-01) -- (axis cs:7.80000E+01,-1.00000E-01) -- (axis cs:7.70000E+01,-1.00000E-01) -- cycle ; 
\draw[Equator-empty] (axis cs:7.90000E+01,1.00000E-01) -- (axis cs:8.00000E+01,1.00000E-01) -- (axis cs:8.00000E+01,-1.00000E-01) -- (axis cs:7.90000E+01,-1.00000E-01) -- cycle ; 
\draw[Equator-empty] (axis cs:8.10000E+01,1.00000E-01) -- (axis cs:8.20000E+01,1.00000E-01) -- (axis cs:8.20000E+01,-1.00000E-01) -- (axis cs:8.10000E+01,-1.00000E-01) -- cycle ; 
\draw[Equator-empty] (axis cs:8.30000E+01,1.00000E-01) -- (axis cs:8.40000E+01,1.00000E-01) -- (axis cs:8.40000E+01,-1.00000E-01) -- (axis cs:8.30000E+01,-1.00000E-01) -- cycle ; 
\draw[Equator-empty] (axis cs:8.50000E+01,1.00000E-01) -- (axis cs:8.60000E+01,1.00000E-01) -- (axis cs:8.60000E+01,-1.00000E-01) -- (axis cs:8.50000E+01,-1.00000E-01) -- cycle ; 
\draw[Equator-empty] (axis cs:8.70000E+01,1.00000E-01) -- (axis cs:8.80000E+01,1.00000E-01) -- (axis cs:8.80000E+01,-1.00000E-01) -- (axis cs:8.70000E+01,-1.00000E-01) -- cycle ; 
\draw[Equator-empty] (axis cs:8.90000E+01,1.00000E-01) -- (axis cs:9.00000E+01,1.00000E-01) -- (axis cs:9.00000E+01,-1.00000E-01) -- (axis cs:8.90000E+01,-1.00000E-01) -- cycle ; 
\draw[Equator-empty] (axis cs:9.10000E+01,1.00000E-01) -- (axis cs:9.20000E+01,1.00000E-01) -- (axis cs:9.20000E+01,-1.00000E-01) -- (axis cs:9.10000E+01,-1.00000E-01) -- cycle ; 
\draw[Equator-empty] (axis cs:9.30000E+01,1.00000E-01) -- (axis cs:9.40000E+01,1.00000E-01) -- (axis cs:9.40000E+01,-1.00000E-01) -- (axis cs:9.30000E+01,-1.00000E-01) -- cycle ; 
\draw[Equator-empty] (axis cs:9.50000E+01,1.00000E-01) -- (axis cs:9.60000E+01,1.00000E-01) -- (axis cs:9.60000E+01,-1.00000E-01) -- (axis cs:9.50000E+01,-1.00000E-01) -- cycle ; 
\draw[Equator-empty] (axis cs:9.70000E+01,1.00000E-01) -- (axis cs:9.80000E+01,1.00000E-01) -- (axis cs:9.80000E+01,-1.00000E-01) -- (axis cs:9.70000E+01,-1.00000E-01) -- cycle ; 
\draw[Equator-empty] (axis cs:9.90000E+01,1.00000E-01) -- (axis cs:1.00000E+02,1.00000E-01) -- (axis cs:1.00000E+02,-1.00000E-01) -- (axis cs:9.90000E+01,-1.00000E-01) -- cycle ; 
\draw[Equator-empty] (axis cs:1.01000E+02,1.00000E-01) -- (axis cs:1.02000E+02,1.00000E-01) -- (axis cs:1.02000E+02,-1.00000E-01) -- (axis cs:1.01000E+02,-1.00000E-01) -- cycle ; 
\draw[Equator-empty] (axis cs:1.03000E+02,1.00000E-01) -- (axis cs:1.04000E+02,1.00000E-01) -- (axis cs:1.04000E+02,-1.00000E-01) -- (axis cs:1.03000E+02,-1.00000E-01) -- cycle ; 
\draw[Equator-empty] (axis cs:1.05000E+02,1.00000E-01) -- (axis cs:1.06000E+02,1.00000E-01) -- (axis cs:1.06000E+02,-1.00000E-01) -- (axis cs:1.05000E+02,-1.00000E-01) -- cycle ; 
\draw[Equator-empty] (axis cs:1.07000E+02,1.00000E-01) -- (axis cs:1.08000E+02,1.00000E-01) -- (axis cs:1.08000E+02,-1.00000E-01) -- (axis cs:1.07000E+02,-1.00000E-01) -- cycle ; 
\draw[Equator-empty] (axis cs:1.09000E+02,1.00000E-01) -- (axis cs:1.10000E+02,1.00000E-01) -- (axis cs:1.10000E+02,-1.00000E-01) -- (axis cs:1.09000E+02,-1.00000E-01) -- cycle ; 
\draw[Equator-empty] (axis cs:1.11000E+02,1.00000E-01) -- (axis cs:1.12000E+02,1.00000E-01) -- (axis cs:1.12000E+02,-1.00000E-01) -- (axis cs:1.11000E+02,-1.00000E-01) -- cycle ; 
\draw[Equator-empty] (axis cs:1.13000E+02,1.00000E-01) -- (axis cs:1.14000E+02,1.00000E-01) -- (axis cs:1.14000E+02,-1.00000E-01) -- (axis cs:1.13000E+02,-1.00000E-01) -- cycle ; 
\draw[Equator-empty] (axis cs:1.15000E+02,1.00000E-01) -- (axis cs:1.16000E+02,1.00000E-01) -- (axis cs:1.16000E+02,-1.00000E-01) -- (axis cs:1.15000E+02,-1.00000E-01) -- cycle ; 
\draw[Equator-empty] (axis cs:1.17000E+02,1.00000E-01) -- (axis cs:1.18000E+02,1.00000E-01) -- (axis cs:1.18000E+02,-1.00000E-01) -- (axis cs:1.17000E+02,-1.00000E-01) -- cycle ; 
\draw[Equator-empty] (axis cs:1.19000E+02,1.00000E-01) -- (axis cs:1.20000E+02,1.00000E-01) -- (axis cs:1.20000E+02,-1.00000E-01) -- (axis cs:1.19000E+02,-1.00000E-01) -- cycle ; 
\draw[Equator-empty] (axis cs:1.21000E+02,1.00000E-01) -- (axis cs:1.22000E+02,1.00000E-01) -- (axis cs:1.22000E+02,-1.00000E-01) -- (axis cs:1.21000E+02,-1.00000E-01) -- cycle ; 
\draw[Equator-empty] (axis cs:1.23000E+02,1.00000E-01) -- (axis cs:1.24000E+02,1.00000E-01) -- (axis cs:1.24000E+02,-1.00000E-01) -- (axis cs:1.23000E+02,-1.00000E-01) -- cycle ; 
\draw[Equator-empty] (axis cs:1.25000E+02,1.00000E-01) -- (axis cs:1.26000E+02,1.00000E-01) -- (axis cs:1.26000E+02,-1.00000E-01) -- (axis cs:1.25000E+02,-1.00000E-01) -- cycle ; 
\draw[Equator-empty] (axis cs:1.27000E+02,1.00000E-01) -- (axis cs:1.28000E+02,1.00000E-01) -- (axis cs:1.28000E+02,-1.00000E-01) -- (axis cs:1.27000E+02,-1.00000E-01) -- cycle ; 
\draw[Equator-empty] (axis cs:1.29000E+02,1.00000E-01) -- (axis cs:1.30000E+02,1.00000E-01) -- (axis cs:1.30000E+02,-1.00000E-01) -- (axis cs:1.29000E+02,-1.00000E-01) -- cycle ; 
\draw[Equator-empty] (axis cs:1.31000E+02,1.00000E-01) -- (axis cs:1.32000E+02,1.00000E-01) -- (axis cs:1.32000E+02,-1.00000E-01) -- (axis cs:1.31000E+02,-1.00000E-01) -- cycle ; 
\draw[Equator-empty] (axis cs:1.33000E+02,1.00000E-01) -- (axis cs:1.34000E+02,1.00000E-01) -- (axis cs:1.34000E+02,-1.00000E-01) -- (axis cs:1.33000E+02,-1.00000E-01) -- cycle ; 
\draw[Equator-empty] (axis cs:1.35000E+02,1.00000E-01) -- (axis cs:1.36000E+02,1.00000E-01) -- (axis cs:1.36000E+02,-1.00000E-01) -- (axis cs:1.35000E+02,-1.00000E-01) -- cycle ; 
\draw[Equator-empty] (axis cs:1.37000E+02,1.00000E-01) -- (axis cs:1.38000E+02,1.00000E-01) -- (axis cs:1.38000E+02,-1.00000E-01) -- (axis cs:1.37000E+02,-1.00000E-01) -- cycle ; 
\draw[Equator-empty] (axis cs:1.39000E+02,1.00000E-01) -- (axis cs:1.40000E+02,1.00000E-01) -- (axis cs:1.40000E+02,-1.00000E-01) -- (axis cs:1.39000E+02,-1.00000E-01) -- cycle ; 
\draw[Equator-empty] (axis cs:1.41000E+02,1.00000E-01) -- (axis cs:1.42000E+02,1.00000E-01) -- (axis cs:1.42000E+02,-1.00000E-01) -- (axis cs:1.41000E+02,-1.00000E-01) -- cycle ; 
\draw[Equator-empty] (axis cs:1.43000E+02,1.00000E-01) -- (axis cs:1.44000E+02,1.00000E-01) -- (axis cs:1.44000E+02,-1.00000E-01) -- (axis cs:1.43000E+02,-1.00000E-01) -- cycle ; 
\draw[Equator-empty] (axis cs:1.45000E+02,1.00000E-01) -- (axis cs:1.46000E+02,1.00000E-01) -- (axis cs:1.46000E+02,-1.00000E-01) -- (axis cs:1.45000E+02,-1.00000E-01) -- cycle ; 
\draw[Equator-empty] (axis cs:1.47000E+02,1.00000E-01) -- (axis cs:1.48000E+02,1.00000E-01) -- (axis cs:1.48000E+02,-1.00000E-01) -- (axis cs:1.47000E+02,-1.00000E-01) -- cycle ; 
\draw[Equator-empty] (axis cs:1.49000E+02,1.00000E-01) -- (axis cs:1.50000E+02,1.00000E-01) -- (axis cs:1.50000E+02,-1.00000E-01) -- (axis cs:1.49000E+02,-1.00000E-01) -- cycle ; 
\draw[Equator-empty] (axis cs:1.51000E+02,1.00000E-01) -- (axis cs:1.52000E+02,1.00000E-01) -- (axis cs:1.52000E+02,-1.00000E-01) -- (axis cs:1.51000E+02,-1.00000E-01) -- cycle ; 
\draw[Equator-empty] (axis cs:1.53000E+02,1.00000E-01) -- (axis cs:1.54000E+02,1.00000E-01) -- (axis cs:1.54000E+02,-1.00000E-01) -- (axis cs:1.53000E+02,-1.00000E-01) -- cycle ; 
\draw[Equator-empty] (axis cs:1.55000E+02,1.00000E-01) -- (axis cs:1.56000E+02,1.00000E-01) -- (axis cs:1.56000E+02,-1.00000E-01) -- (axis cs:1.55000E+02,-1.00000E-01) -- cycle ; 
\draw[Equator-empty] (axis cs:1.57000E+02,1.00000E-01) -- (axis cs:1.58000E+02,1.00000E-01) -- (axis cs:1.58000E+02,-1.00000E-01) -- (axis cs:1.57000E+02,-1.00000E-01) -- cycle ; 
\draw[Equator-empty] (axis cs:1.59000E+02,1.00000E-01) -- (axis cs:1.60000E+02,1.00000E-01) -- (axis cs:1.60000E+02,-1.00000E-01) -- (axis cs:1.59000E+02,-1.00000E-01) -- cycle ; 
\draw[Equator-empty] (axis cs:1.61000E+02,1.00000E-01) -- (axis cs:1.62000E+02,1.00000E-01) -- (axis cs:1.62000E+02,-1.00000E-01) -- (axis cs:1.61000E+02,-1.00000E-01) -- cycle ; 
\draw[Equator-empty] (axis cs:1.63000E+02,1.00000E-01) -- (axis cs:1.64000E+02,1.00000E-01) -- (axis cs:1.64000E+02,-1.00000E-01) -- (axis cs:1.63000E+02,-1.00000E-01) -- cycle ; 
\draw[Equator-empty] (axis cs:1.65000E+02,1.00000E-01) -- (axis cs:1.66000E+02,1.00000E-01) -- (axis cs:1.66000E+02,-1.00000E-01) -- (axis cs:1.65000E+02,-1.00000E-01) -- cycle ; 
\draw[Equator-empty] (axis cs:1.67000E+02,1.00000E-01) -- (axis cs:1.68000E+02,1.00000E-01) -- (axis cs:1.68000E+02,-1.00000E-01) -- (axis cs:1.67000E+02,-1.00000E-01) -- cycle ; 
\draw[Equator-empty] (axis cs:1.69000E+02,1.00000E-01) -- (axis cs:1.70000E+02,1.00000E-01) -- (axis cs:1.70000E+02,-1.00000E-01) -- (axis cs:1.69000E+02,-1.00000E-01) -- cycle ; 
\draw[Equator-empty] (axis cs:1.71000E+02,1.00000E-01) -- (axis cs:1.72000E+02,1.00000E-01) -- (axis cs:1.72000E+02,-1.00000E-01) -- (axis cs:1.71000E+02,-1.00000E-01) -- cycle ; 
\draw[Equator-empty] (axis cs:1.73000E+02,1.00000E-01) -- (axis cs:1.74000E+02,1.00000E-01) -- (axis cs:1.74000E+02,-1.00000E-01) -- (axis cs:1.73000E+02,-1.00000E-01) -- cycle ; 
\draw[Equator-empty] (axis cs:1.75000E+02,1.00000E-01) -- (axis cs:1.76000E+02,1.00000E-01) -- (axis cs:1.76000E+02,-1.00000E-01) -- (axis cs:1.75000E+02,-1.00000E-01) -- cycle ; 
\draw[Equator-empty] (axis cs:1.77000E+02,1.00000E-01) -- (axis cs:1.78000E+02,1.00000E-01) -- (axis cs:1.78000E+02,-1.00000E-01) -- (axis cs:1.77000E+02,-1.00000E-01) -- cycle ; 
\draw[Equator-empty] (axis cs:1.79000E+02,1.00000E-01) -- (axis cs:1.80000E+02,1.00000E-01) -- (axis cs:1.80000E+02,-1.00000E-01) -- (axis cs:1.79000E+02,-1.00000E-01) -- cycle ; 
\draw[Equator-empty] (axis cs:1.81000E+02,1.00000E-01) -- (axis cs:1.82000E+02,1.00000E-01) -- (axis cs:1.82000E+02,-1.00000E-01) -- (axis cs:1.81000E+02,-1.00000E-01) -- cycle ; 
\draw[Equator-empty] (axis cs:1.83000E+02,1.00000E-01) -- (axis cs:1.84000E+02,1.00000E-01) -- (axis cs:1.84000E+02,-1.00000E-01) -- (axis cs:1.83000E+02,-1.00000E-01) -- cycle ; 
\draw[Equator-empty] (axis cs:1.85000E+02,1.00000E-01) -- (axis cs:1.86000E+02,1.00000E-01) -- (axis cs:1.86000E+02,-1.00000E-01) -- (axis cs:1.85000E+02,-1.00000E-01) -- cycle ; 
\draw[Equator-empty] (axis cs:1.87000E+02,1.00000E-01) -- (axis cs:1.88000E+02,1.00000E-01) -- (axis cs:1.88000E+02,-1.00000E-01) -- (axis cs:1.87000E+02,-1.00000E-01) -- cycle ; 
\draw[Equator-empty] (axis cs:1.89000E+02,1.00000E-01) -- (axis cs:1.90000E+02,1.00000E-01) -- (axis cs:1.90000E+02,-1.00000E-01) -- (axis cs:1.89000E+02,-1.00000E-01) -- cycle ; 
\draw[Equator-empty] (axis cs:1.91000E+02,1.00000E-01) -- (axis cs:1.92000E+02,1.00000E-01) -- (axis cs:1.92000E+02,-1.00000E-01) -- (axis cs:1.91000E+02,-1.00000E-01) -- cycle ; 
\draw[Equator-empty] (axis cs:1.93000E+02,1.00000E-01) -- (axis cs:1.94000E+02,1.00000E-01) -- (axis cs:1.94000E+02,-1.00000E-01) -- (axis cs:1.93000E+02,-1.00000E-01) -- cycle ; 
\draw[Equator-empty] (axis cs:1.95000E+02,1.00000E-01) -- (axis cs:1.96000E+02,1.00000E-01) -- (axis cs:1.96000E+02,-1.00000E-01) -- (axis cs:1.95000E+02,-1.00000E-01) -- cycle ; 
\draw[Equator-empty] (axis cs:1.97000E+02,1.00000E-01) -- (axis cs:1.98000E+02,1.00000E-01) -- (axis cs:1.98000E+02,-1.00000E-01) -- (axis cs:1.97000E+02,-1.00000E-01) -- cycle ; 
\draw[Equator-empty] (axis cs:1.99000E+02,1.00000E-01) -- (axis cs:2.00000E+02,1.00000E-01) -- (axis cs:2.00000E+02,-1.00000E-01) -- (axis cs:1.99000E+02,-1.00000E-01) -- cycle ; 
\draw[Equator-empty] (axis cs:2.01000E+02,1.00000E-01) -- (axis cs:2.02000E+02,1.00000E-01) -- (axis cs:2.02000E+02,-1.00000E-01) -- (axis cs:2.01000E+02,-1.00000E-01) -- cycle ; 
\draw[Equator-empty] (axis cs:2.03000E+02,1.00000E-01) -- (axis cs:2.04000E+02,1.00000E-01) -- (axis cs:2.04000E+02,-1.00000E-01) -- (axis cs:2.03000E+02,-1.00000E-01) -- cycle ; 
\draw[Equator-empty] (axis cs:2.05000E+02,1.00000E-01) -- (axis cs:2.06000E+02,1.00000E-01) -- (axis cs:2.06000E+02,-1.00000E-01) -- (axis cs:2.05000E+02,-1.00000E-01) -- cycle ; 
\draw[Equator-empty] (axis cs:2.07000E+02,1.00000E-01) -- (axis cs:2.08000E+02,1.00000E-01) -- (axis cs:2.08000E+02,-1.00000E-01) -- (axis cs:2.07000E+02,-1.00000E-01) -- cycle ; 
\draw[Equator-empty] (axis cs:2.09000E+02,1.00000E-01) -- (axis cs:2.10000E+02,1.00000E-01) -- (axis cs:2.10000E+02,-1.00000E-01) -- (axis cs:2.09000E+02,-1.00000E-01) -- cycle ; 
\draw[Equator-empty] (axis cs:2.11000E+02,1.00000E-01) -- (axis cs:2.12000E+02,1.00000E-01) -- (axis cs:2.12000E+02,-1.00000E-01) -- (axis cs:2.11000E+02,-1.00000E-01) -- cycle ; 
\draw[Equator-empty] (axis cs:2.13000E+02,1.00000E-01) -- (axis cs:2.14000E+02,1.00000E-01) -- (axis cs:2.14000E+02,-1.00000E-01) -- (axis cs:2.13000E+02,-1.00000E-01) -- cycle ; 
\draw[Equator-empty] (axis cs:2.15000E+02,1.00000E-01) -- (axis cs:2.16000E+02,1.00000E-01) -- (axis cs:2.16000E+02,-1.00000E-01) -- (axis cs:2.15000E+02,-1.00000E-01) -- cycle ; 
\draw[Equator-empty] (axis cs:2.17000E+02,1.00000E-01) -- (axis cs:2.18000E+02,1.00000E-01) -- (axis cs:2.18000E+02,-1.00000E-01) -- (axis cs:2.17000E+02,-1.00000E-01) -- cycle ; 
\draw[Equator-empty] (axis cs:2.19000E+02,1.00000E-01) -- (axis cs:2.20000E+02,1.00000E-01) -- (axis cs:2.20000E+02,-1.00000E-01) -- (axis cs:2.19000E+02,-1.00000E-01) -- cycle ; 
\draw[Equator-empty] (axis cs:2.21000E+02,1.00000E-01) -- (axis cs:2.22000E+02,1.00000E-01) -- (axis cs:2.22000E+02,-1.00000E-01) -- (axis cs:2.21000E+02,-1.00000E-01) -- cycle ; 
\draw[Equator-empty] (axis cs:2.23000E+02,1.00000E-01) -- (axis cs:2.24000E+02,1.00000E-01) -- (axis cs:2.24000E+02,-1.00000E-01) -- (axis cs:2.23000E+02,-1.00000E-01) -- cycle ; 
\draw[Equator-empty] (axis cs:2.25000E+02,1.00000E-01) -- (axis cs:2.26000E+02,1.00000E-01) -- (axis cs:2.26000E+02,-1.00000E-01) -- (axis cs:2.25000E+02,-1.00000E-01) -- cycle ; 
\draw[Equator-empty] (axis cs:2.27000E+02,1.00000E-01) -- (axis cs:2.28000E+02,1.00000E-01) -- (axis cs:2.28000E+02,-1.00000E-01) -- (axis cs:2.27000E+02,-1.00000E-01) -- cycle ; 
\draw[Equator-empty] (axis cs:2.29000E+02,1.00000E-01) -- (axis cs:2.30000E+02,1.00000E-01) -- (axis cs:2.30000E+02,-1.00000E-01) -- (axis cs:2.29000E+02,-1.00000E-01) -- cycle ; 
\draw[Equator-empty] (axis cs:2.31000E+02,1.00000E-01) -- (axis cs:2.32000E+02,1.00000E-01) -- (axis cs:2.32000E+02,-1.00000E-01) -- (axis cs:2.31000E+02,-1.00000E-01) -- cycle ; 
\draw[Equator-empty] (axis cs:2.33000E+02,1.00000E-01) -- (axis cs:2.34000E+02,1.00000E-01) -- (axis cs:2.34000E+02,-1.00000E-01) -- (axis cs:2.33000E+02,-1.00000E-01) -- cycle ; 
\draw[Equator-empty] (axis cs:2.35000E+02,1.00000E-01) -- (axis cs:2.36000E+02,1.00000E-01) -- (axis cs:2.36000E+02,-1.00000E-01) -- (axis cs:2.35000E+02,-1.00000E-01) -- cycle ; 
\draw[Equator-empty] (axis cs:2.37000E+02,1.00000E-01) -- (axis cs:2.38000E+02,1.00000E-01) -- (axis cs:2.38000E+02,-1.00000E-01) -- (axis cs:2.37000E+02,-1.00000E-01) -- cycle ; 
\draw[Equator-empty] (axis cs:2.39000E+02,1.00000E-01) -- (axis cs:2.40000E+02,1.00000E-01) -- (axis cs:2.40000E+02,-1.00000E-01) -- (axis cs:2.39000E+02,-1.00000E-01) -- cycle ; 
\draw[Equator-empty] (axis cs:2.41000E+02,1.00000E-01) -- (axis cs:2.42000E+02,1.00000E-01) -- (axis cs:2.42000E+02,-1.00000E-01) -- (axis cs:2.41000E+02,-1.00000E-01) -- cycle ; 
\draw[Equator-empty] (axis cs:2.43000E+02,1.00000E-01) -- (axis cs:2.44000E+02,1.00000E-01) -- (axis cs:2.44000E+02,-1.00000E-01) -- (axis cs:2.43000E+02,-1.00000E-01) -- cycle ; 
\draw[Equator-empty] (axis cs:2.45000E+02,1.00000E-01) -- (axis cs:2.46000E+02,1.00000E-01) -- (axis cs:2.46000E+02,-1.00000E-01) -- (axis cs:2.45000E+02,-1.00000E-01) -- cycle ; 
\draw[Equator-empty] (axis cs:2.47000E+02,1.00000E-01) -- (axis cs:2.48000E+02,1.00000E-01) -- (axis cs:2.48000E+02,-1.00000E-01) -- (axis cs:2.47000E+02,-1.00000E-01) -- cycle ; 
\draw[Equator-empty] (axis cs:2.49000E+02,1.00000E-01) -- (axis cs:2.50000E+02,1.00000E-01) -- (axis cs:2.50000E+02,-1.00000E-01) -- (axis cs:2.49000E+02,-1.00000E-01) -- cycle ; 
\draw[Equator-empty] (axis cs:2.51000E+02,1.00000E-01) -- (axis cs:2.52000E+02,1.00000E-01) -- (axis cs:2.52000E+02,-1.00000E-01) -- (axis cs:2.51000E+02,-1.00000E-01) -- cycle ; 
\draw[Equator-empty] (axis cs:2.53000E+02,1.00000E-01) -- (axis cs:2.54000E+02,1.00000E-01) -- (axis cs:2.54000E+02,-1.00000E-01) -- (axis cs:2.53000E+02,-1.00000E-01) -- cycle ; 
\draw[Equator-empty] (axis cs:2.55000E+02,1.00000E-01) -- (axis cs:2.56000E+02,1.00000E-01) -- (axis cs:2.56000E+02,-1.00000E-01) -- (axis cs:2.55000E+02,-1.00000E-01) -- cycle ; 
\draw[Equator-empty] (axis cs:2.57000E+02,1.00000E-01) -- (axis cs:2.58000E+02,1.00000E-01) -- (axis cs:2.58000E+02,-1.00000E-01) -- (axis cs:2.57000E+02,-1.00000E-01) -- cycle ; 
\draw[Equator-empty] (axis cs:2.59000E+02,1.00000E-01) -- (axis cs:2.60000E+02,1.00000E-01) -- (axis cs:2.60000E+02,-1.00000E-01) -- (axis cs:2.59000E+02,-1.00000E-01) -- cycle ; 
\draw[Equator-empty] (axis cs:2.61000E+02,1.00000E-01) -- (axis cs:2.62000E+02,1.00000E-01) -- (axis cs:2.62000E+02,-1.00000E-01) -- (axis cs:2.61000E+02,-1.00000E-01) -- cycle ; 
\draw[Equator-empty] (axis cs:2.63000E+02,1.00000E-01) -- (axis cs:2.64000E+02,1.00000E-01) -- (axis cs:2.64000E+02,-1.00000E-01) -- (axis cs:2.63000E+02,-1.00000E-01) -- cycle ; 
\draw[Equator-empty] (axis cs:2.65000E+02,1.00000E-01) -- (axis cs:2.66000E+02,1.00000E-01) -- (axis cs:2.66000E+02,-1.00000E-01) -- (axis cs:2.65000E+02,-1.00000E-01) -- cycle ; 
\draw[Equator-empty] (axis cs:2.67000E+02,1.00000E-01) -- (axis cs:2.68000E+02,1.00000E-01) -- (axis cs:2.68000E+02,-1.00000E-01) -- (axis cs:2.67000E+02,-1.00000E-01) -- cycle ; 
\draw[Equator-empty] (axis cs:2.69000E+02,1.00000E-01) -- (axis cs:2.70000E+02,1.00000E-01) -- (axis cs:2.70000E+02,-1.00000E-01) -- (axis cs:2.69000E+02,-1.00000E-01) -- cycle ; 
\draw[Equator-empty] (axis cs:2.71000E+02,1.00000E-01) -- (axis cs:2.72000E+02,1.00000E-01) -- (axis cs:2.72000E+02,-1.00000E-01) -- (axis cs:2.71000E+02,-1.00000E-01) -- cycle ; 
\draw[Equator-empty] (axis cs:2.73000E+02,1.00000E-01) -- (axis cs:2.74000E+02,1.00000E-01) -- (axis cs:2.74000E+02,-1.00000E-01) -- (axis cs:2.73000E+02,-1.00000E-01) -- cycle ; 
\draw[Equator-empty] (axis cs:2.75000E+02,1.00000E-01) -- (axis cs:2.76000E+02,1.00000E-01) -- (axis cs:2.76000E+02,-1.00000E-01) -- (axis cs:2.75000E+02,-1.00000E-01) -- cycle ; 
\draw[Equator-empty] (axis cs:2.77000E+02,1.00000E-01) -- (axis cs:2.78000E+02,1.00000E-01) -- (axis cs:2.78000E+02,-1.00000E-01) -- (axis cs:2.77000E+02,-1.00000E-01) -- cycle ; 
\draw[Equator-empty] (axis cs:2.79000E+02,1.00000E-01) -- (axis cs:2.80000E+02,1.00000E-01) -- (axis cs:2.80000E+02,-1.00000E-01) -- (axis cs:2.79000E+02,-1.00000E-01) -- cycle ; 
\draw[Equator-empty] (axis cs:2.81000E+02,1.00000E-01) -- (axis cs:2.82000E+02,1.00000E-01) -- (axis cs:2.82000E+02,-1.00000E-01) -- (axis cs:2.81000E+02,-1.00000E-01) -- cycle ; 
\draw[Equator-empty] (axis cs:2.83000E+02,1.00000E-01) -- (axis cs:2.84000E+02,1.00000E-01) -- (axis cs:2.84000E+02,-1.00000E-01) -- (axis cs:2.83000E+02,-1.00000E-01) -- cycle ; 
\draw[Equator-empty] (axis cs:2.85000E+02,1.00000E-01) -- (axis cs:2.86000E+02,1.00000E-01) -- (axis cs:2.86000E+02,-1.00000E-01) -- (axis cs:2.85000E+02,-1.00000E-01) -- cycle ; 
\draw[Equator-empty] (axis cs:2.87000E+02,1.00000E-01) -- (axis cs:2.88000E+02,1.00000E-01) -- (axis cs:2.88000E+02,-1.00000E-01) -- (axis cs:2.87000E+02,-1.00000E-01) -- cycle ; 
\draw[Equator-empty] (axis cs:2.89000E+02,1.00000E-01) -- (axis cs:2.90000E+02,1.00000E-01) -- (axis cs:2.90000E+02,-1.00000E-01) -- (axis cs:2.89000E+02,-1.00000E-01) -- cycle ; 
\draw[Equator-empty] (axis cs:2.91000E+02,1.00000E-01) -- (axis cs:2.92000E+02,1.00000E-01) -- (axis cs:2.92000E+02,-1.00000E-01) -- (axis cs:2.91000E+02,-1.00000E-01) -- cycle ; 
\draw[Equator-empty] (axis cs:2.93000E+02,1.00000E-01) -- (axis cs:2.94000E+02,1.00000E-01) -- (axis cs:2.94000E+02,-1.00000E-01) -- (axis cs:2.93000E+02,-1.00000E-01) -- cycle ; 
\draw[Equator-empty] (axis cs:2.95000E+02,1.00000E-01) -- (axis cs:2.96000E+02,1.00000E-01) -- (axis cs:2.96000E+02,-1.00000E-01) -- (axis cs:2.95000E+02,-1.00000E-01) -- cycle ; 
\draw[Equator-empty] (axis cs:2.97000E+02,1.00000E-01) -- (axis cs:2.98000E+02,1.00000E-01) -- (axis cs:2.98000E+02,-1.00000E-01) -- (axis cs:2.97000E+02,-1.00000E-01) -- cycle ; 
\draw[Equator-empty] (axis cs:2.99000E+02,1.00000E-01) -- (axis cs:3.00000E+02,1.00000E-01) -- (axis cs:3.00000E+02,-1.00000E-01) -- (axis cs:2.99000E+02,-1.00000E-01) -- cycle ; 
\draw[Equator-empty] (axis cs:3.01000E+02,1.00000E-01) -- (axis cs:3.02000E+02,1.00000E-01) -- (axis cs:3.02000E+02,-1.00000E-01) -- (axis cs:3.01000E+02,-1.00000E-01) -- cycle ; 
\draw[Equator-empty] (axis cs:3.03000E+02,1.00000E-01) -- (axis cs:3.04000E+02,1.00000E-01) -- (axis cs:3.04000E+02,-1.00000E-01) -- (axis cs:3.03000E+02,-1.00000E-01) -- cycle ; 
\draw[Equator-empty] (axis cs:3.05000E+02,1.00000E-01) -- (axis cs:3.06000E+02,1.00000E-01) -- (axis cs:3.06000E+02,-1.00000E-01) -- (axis cs:3.05000E+02,-1.00000E-01) -- cycle ; 
\draw[Equator-empty] (axis cs:3.07000E+02,1.00000E-01) -- (axis cs:3.08000E+02,1.00000E-01) -- (axis cs:3.08000E+02,-1.00000E-01) -- (axis cs:3.07000E+02,-1.00000E-01) -- cycle ; 
\draw[Equator-empty] (axis cs:3.09000E+02,1.00000E-01) -- (axis cs:3.10000E+02,1.00000E-01) -- (axis cs:3.10000E+02,-1.00000E-01) -- (axis cs:3.09000E+02,-1.00000E-01) -- cycle ; 
\draw[Equator-empty] (axis cs:3.11000E+02,1.00000E-01) -- (axis cs:3.12000E+02,1.00000E-01) -- (axis cs:3.12000E+02,-1.00000E-01) -- (axis cs:3.11000E+02,-1.00000E-01) -- cycle ; 
\draw[Equator-empty] (axis cs:3.13000E+02,1.00000E-01) -- (axis cs:3.14000E+02,1.00000E-01) -- (axis cs:3.14000E+02,-1.00000E-01) -- (axis cs:3.13000E+02,-1.00000E-01) -- cycle ; 
\draw[Equator-empty] (axis cs:3.15000E+02,1.00000E-01) -- (axis cs:3.16000E+02,1.00000E-01) -- (axis cs:3.16000E+02,-1.00000E-01) -- (axis cs:3.15000E+02,-1.00000E-01) -- cycle ; 
\draw[Equator-empty] (axis cs:3.17000E+02,1.00000E-01) -- (axis cs:3.18000E+02,1.00000E-01) -- (axis cs:3.18000E+02,-1.00000E-01) -- (axis cs:3.17000E+02,-1.00000E-01) -- cycle ; 
\draw[Equator-empty] (axis cs:3.19000E+02,1.00000E-01) -- (axis cs:3.20000E+02,1.00000E-01) -- (axis cs:3.20000E+02,-1.00000E-01) -- (axis cs:3.19000E+02,-1.00000E-01) -- cycle ; 
\draw[Equator-empty] (axis cs:3.21000E+02,1.00000E-01) -- (axis cs:3.22000E+02,1.00000E-01) -- (axis cs:3.22000E+02,-1.00000E-01) -- (axis cs:3.21000E+02,-1.00000E-01) -- cycle ; 
\draw[Equator-empty] (axis cs:3.23000E+02,1.00000E-01) -- (axis cs:3.24000E+02,1.00000E-01) -- (axis cs:3.24000E+02,-1.00000E-01) -- (axis cs:3.23000E+02,-1.00000E-01) -- cycle ; 
\draw[Equator-empty] (axis cs:3.25000E+02,1.00000E-01) -- (axis cs:3.26000E+02,1.00000E-01) -- (axis cs:3.26000E+02,-1.00000E-01) -- (axis cs:3.25000E+02,-1.00000E-01) -- cycle ; 
\draw[Equator-empty] (axis cs:3.27000E+02,1.00000E-01) -- (axis cs:3.28000E+02,1.00000E-01) -- (axis cs:3.28000E+02,-1.00000E-01) -- (axis cs:3.27000E+02,-1.00000E-01) -- cycle ; 
\draw[Equator-empty] (axis cs:3.29000E+02,1.00000E-01) -- (axis cs:3.30000E+02,1.00000E-01) -- (axis cs:3.30000E+02,-1.00000E-01) -- (axis cs:3.29000E+02,-1.00000E-01) -- cycle ; 
\draw[Equator-empty] (axis cs:3.31000E+02,1.00000E-01) -- (axis cs:3.32000E+02,1.00000E-01) -- (axis cs:3.32000E+02,-1.00000E-01) -- (axis cs:3.31000E+02,-1.00000E-01) -- cycle ; 
\draw[Equator-empty] (axis cs:3.33000E+02,1.00000E-01) -- (axis cs:3.34000E+02,1.00000E-01) -- (axis cs:3.34000E+02,-1.00000E-01) -- (axis cs:3.33000E+02,-1.00000E-01) -- cycle ; 
\draw[Equator-empty] (axis cs:3.35000E+02,1.00000E-01) -- (axis cs:3.36000E+02,1.00000E-01) -- (axis cs:3.36000E+02,-1.00000E-01) -- (axis cs:3.35000E+02,-1.00000E-01) -- cycle ; 
\draw[Equator-empty] (axis cs:3.37000E+02,1.00000E-01) -- (axis cs:3.38000E+02,1.00000E-01) -- (axis cs:3.38000E+02,-1.00000E-01) -- (axis cs:3.37000E+02,-1.00000E-01) -- cycle ; 
\draw[Equator-empty] (axis cs:3.39000E+02,1.00000E-01) -- (axis cs:3.40000E+02,1.00000E-01) -- (axis cs:3.40000E+02,-1.00000E-01) -- (axis cs:3.39000E+02,-1.00000E-01) -- cycle ; 
\draw[Equator-empty] (axis cs:3.41000E+02,1.00000E-01) -- (axis cs:3.42000E+02,1.00000E-01) -- (axis cs:3.42000E+02,-1.00000E-01) -- (axis cs:3.41000E+02,-1.00000E-01) -- cycle ; 
\draw[Equator-empty] (axis cs:3.43000E+02,1.00000E-01) -- (axis cs:3.44000E+02,1.00000E-01) -- (axis cs:3.44000E+02,-1.00000E-01) -- (axis cs:3.43000E+02,-1.00000E-01) -- cycle ; 
\draw[Equator-empty] (axis cs:3.45000E+02,1.00000E-01) -- (axis cs:3.46000E+02,1.00000E-01) -- (axis cs:3.46000E+02,-1.00000E-01) -- (axis cs:3.45000E+02,-1.00000E-01) -- cycle ; 
\draw[Equator-empty] (axis cs:3.47000E+02,1.00000E-01) -- (axis cs:3.48000E+02,1.00000E-01) -- (axis cs:3.48000E+02,-1.00000E-01) -- (axis cs:3.47000E+02,-1.00000E-01) -- cycle ; 
\draw[Equator-empty] (axis cs:3.49000E+02,1.00000E-01) -- (axis cs:3.50000E+02,1.00000E-01) -- (axis cs:3.50000E+02,-1.00000E-01) -- (axis cs:3.49000E+02,-1.00000E-01) -- cycle ; 
\draw[Equator-empty] (axis cs:3.51000E+02,1.00000E-01) -- (axis cs:3.52000E+02,1.00000E-01) -- (axis cs:3.52000E+02,-1.00000E-01) -- (axis cs:3.51000E+02,-1.00000E-01) -- cycle ; 
\draw[Equator-empty] (axis cs:3.53000E+02,1.00000E-01) -- (axis cs:3.54000E+02,1.00000E-01) -- (axis cs:3.54000E+02,-1.00000E-01) -- (axis cs:3.53000E+02,-1.00000E-01) -- cycle ; 
\draw[Equator-empty] (axis cs:3.55000E+02,1.00000E-01) -- (axis cs:3.56000E+02,1.00000E-01) -- (axis cs:3.56000E+02,-1.00000E-01) -- (axis cs:3.55000E+02,-1.00000E-01) -- cycle ; 
\draw[Equator-empty] (axis cs:3.57000E+02,1.00000E-01) -- (axis cs:3.58000E+02,1.00000E-01) -- (axis cs:3.58000E+02,-1.00000E-01) -- (axis cs:3.57000E+02,-1.00000E-01) -- cycle ; 
\draw[Equator-empty] (axis cs:3.59000E+02,1.00000E-01) -- (axis cs:3.60000E+02,1.00000E-01) -- (axis cs:3.60000E+02,-1.00000E-01) -- (axis cs:3.59000E+02,-1.00000E-01) -- cycle ; 

\draw[Equator-full] (axis cs:3.60000E+02,1.00000E-01) -- (axis cs:3.61000E+02,1.00000E-01) -- (axis cs:3.61000E+02,-1.00000E-01) -- (axis cs:3.60000E+02,-1.00000E-01) -- cycle ; 
\draw[Equator-full] (axis cs:3.62000E+02,1.00000E-01) -- (axis cs:3.63000E+02,1.00000E-01) -- (axis cs:3.63000E+02,-1.00000E-01) -- (axis cs:3.62000E+02,-1.00000E-01) -- cycle ; 
\draw[Equator-full] (axis cs:3.64000E+02,1.00000E-01) -- (axis cs:3.65000E+02,1.00000E-01) -- (axis cs:3.65000E+02,-1.00000E-01) -- (axis cs:3.64000E+02,-1.00000E-01) -- cycle ; 
\draw[Equator-full] (axis cs:3.66000E+02,1.00000E-01) -- (axis cs:3.67000E+02,1.00000E-01) -- (axis cs:3.67000E+02,-1.00000E-01) -- (axis cs:3.66000E+02,-1.00000E-01) -- cycle ; 
\draw[Equator-full] (axis cs:3.68000E+02,1.00000E-01) -- (axis cs:3.69000E+02,1.00000E-01) -- (axis cs:3.69000E+02,-1.00000E-01) -- (axis cs:3.68000E+02,-1.00000E-01) -- cycle ; 
\draw[Equator-full] (axis cs:3.70000E+02,1.00000E-01) -- (axis cs:3.71000E+02,1.00000E-01) -- (axis cs:3.71000E+02,-1.00000E-01) -- (axis cs:3.70000E+02,-1.00000E-01) -- cycle ; 
\draw[Equator-full] (axis cs:3.72000E+02,1.00000E-01) -- (axis cs:3.73000E+02,1.00000E-01) -- (axis cs:3.73000E+02,-1.00000E-01) -- (axis cs:3.72000E+02,-1.00000E-01) -- cycle ; 
\draw[Equator-full] (axis cs:3.74000E+02,1.00000E-01) -- (axis cs:3.75000E+02,1.00000E-01) -- (axis cs:3.75000E+02,-1.00000E-01) -- (axis cs:3.74000E+02,-1.00000E-01) -- cycle ; 
\draw[Equator-full] (axis cs:3.76000E+02,1.00000E-01) -- (axis cs:3.77000E+02,1.00000E-01) -- (axis cs:3.77000E+02,-1.00000E-01) -- (axis cs:3.76000E+02,-1.00000E-01) -- cycle ; 
\draw[Equator-full] (axis cs:3.78000E+02,1.00000E-01) -- (axis cs:3.79000E+02,1.00000E-01) -- (axis cs:3.79000E+02,-1.00000E-01) -- (axis cs:3.78000E+02,-1.00000E-01) -- cycle ; 

\draw[Equator-empty] (axis cs:3.61000E+02,1.00000E-01) -- (axis cs:3.62000E+02,1.00000E-01) -- (axis cs:3.62000E+02,-1.00000E-01) -- (axis cs:3.61000E+02,-1.00000E-01) -- cycle ; 
\draw[Equator-empty] (axis cs:3.63000E+02,1.00000E-01) -- (axis cs:3.64000E+02,1.00000E-01) -- (axis cs:3.64000E+02,-1.00000E-01) -- (axis cs:3.63000E+02,-1.00000E-01) -- cycle ; 
\draw[Equator-empty] (axis cs:3.65000E+02,1.00000E-01) -- (axis cs:3.66000E+02,1.00000E-01) -- (axis cs:3.66000E+02,-1.00000E-01) -- (axis cs:3.65000E+02,-1.00000E-01) -- cycle ; 
\draw[Equator-empty] (axis cs:3.67000E+02,1.00000E-01) -- (axis cs:3.68000E+02,1.00000E-01) -- (axis cs:3.68000E+02,-1.00000E-01) -- (axis cs:3.67000E+02,-1.00000E-01) -- cycle ; 
\draw[Equator-empty] (axis cs:3.69000E+02,1.00000E-01) -- (axis cs:3.70000E+02,1.00000E-01) -- (axis cs:3.70000E+02,-1.00000E-01) -- (axis cs:3.69000E+02,-1.00000E-01) -- cycle ; 
\draw[Equator-empty] (axis cs:3.71000E+02,1.00000E-01) -- (axis cs:3.72000E+02,1.00000E-01) -- (axis cs:3.72000E+02,-1.00000E-01) -- (axis cs:3.71000E+02,-1.00000E-01) -- cycle ; 
\draw[Equator-empty] (axis cs:3.73000E+02,1.00000E-01) -- (axis cs:3.74000E+02,1.00000E-01) -- (axis cs:3.74000E+02,-1.00000E-01) -- (axis cs:3.73000E+02,-1.00000E-01) -- cycle ; 
\draw[Equator-empty] (axis cs:3.75000E+02,1.00000E-01) -- (axis cs:3.76000E+02,1.00000E-01) -- (axis cs:3.76000E+02,-1.00000E-01) -- (axis cs:3.75000E+02,-1.00000E-01) -- cycle ; 
\draw[Equator-empty] (axis cs:3.77000E+02,1.00000E-01) -- (axis cs:3.78000E+02,1.00000E-01) -- (axis cs:3.78000E+02,-1.00000E-01) -- (axis cs:3.77000E+02,-1.00000E-01) -- cycle ; 
\draw[Equator-empty] (axis cs:3.79000E+02,1.00000E-01) -- (axis cs:3.80000E+02,1.00000E-01) -- (axis cs:3.80000E+02,-1.00000E-01) -- (axis cs:3.79000E+02,-1.00000E-01) -- cycle ; 


\draw [Equator-empty] (axis cs:0.00000E+00,4.00000E-01) -- (axis cs:0.00000E+00,-4.00000E-01)   ;
\node[pin={[pin distance=-0.4\onedegree,Equator-label]90:{0$^\circ$}}] at (axis cs:0.00000E+00,4.00000E-01) {} ;
\draw [Equator-empty] (axis cs:1.00000E+01,4.00000E-01) -- (axis cs:1.00000E+01,-4.00000E-01)   ;
\node[pin={[pin distance=-0.4\onedegree,Equator-label]90:{10$^\circ$}}] at (axis cs:1.00000E+01,4.00000E-01) {} ;
\draw [Equator-empty] (axis cs:2.00000E+01,4.00000E-01) -- (axis cs:2.00000E+01,-4.00000E-01)   ;
\node[pin={[pin distance=-0.4\onedegree,Equator-label]90:{20$^\circ$}}] at (axis cs:2.00000E+01,4.00000E-01) {} ;
\draw [Equator-empty] (axis cs:3.00000E+01,4.00000E-01) -- (axis cs:3.00000E+01,-4.00000E-01)   ;
\node[pin={[pin distance=-0.4\onedegree,Equator-label]90:{30$^\circ$}}] at (axis cs:3.00000E+01,4.00000E-01) {} ;
\draw [Equator-empty] (axis cs:4.00000E+01,4.00000E-01) -- (axis cs:4.00000E+01,-4.00000E-01)   ;
\node[pin={[pin distance=-0.4\onedegree,Equator-label]90:{40$^\circ$}}] at (axis cs:4.00000E+01,4.00000E-01) {} ;
\draw [Equator-empty] (axis cs:5.00000E+01,4.00000E-01) -- (axis cs:5.00000E+01,-4.00000E-01)   ;
\node[pin={[pin distance=-0.4\onedegree,Equator-label]90:{50$^\circ$}}] at (axis cs:5.00000E+01,4.00000E-01) {} ;
\draw [Equator-empty] (axis cs:6.00000E+01,4.00000E-01) -- (axis cs:6.00000E+01,-4.00000E-01)   ;
\node[pin={[pin distance=-0.4\onedegree,Equator-label]90:{60$^\circ$}}] at (axis cs:6.00000E+01,4.00000E-01) {} ;
\draw [Equator-empty] (axis cs:7.00000E+01,4.00000E-01) -- (axis cs:7.00000E+01,-4.00000E-01)   ;
\node[pin={[pin distance=-0.4\onedegree,Equator-label]90:{70$^\circ$}}] at (axis cs:7.00000E+01,4.00000E-01) {} ;
\draw [Equator-empty] (axis cs:8.00000E+01,4.00000E-01) -- (axis cs:8.00000E+01,-4.00000E-01)   ;
\node[pin={[pin distance=-0.4\onedegree,Equator-label]90:{80$^\circ$}}] at (axis cs:8.00000E+01,4.00000E-01) {} ;
\draw [Equator-empty] (axis cs:9.00000E+01,4.00000E-01) -- (axis cs:9.00000E+01,-4.00000E-01)   ;
\node[pin={[pin distance=-0.4\onedegree,Equator-label]90:{90$^\circ$}}] at (axis cs:9.00000E+01,4.00000E-01) {} ;
\draw [Equator-empty] (axis cs:1.00000E+02,4.00000E-01) -- (axis cs:1.00000E+02,-4.00000E-01)   ;
\node[pin={[pin distance=-0.4\onedegree,Equator-label]090:{100$^\circ$}}] at (axis cs:1.00000E+02,4.00000E-01) {} ;
\draw [Equator-empty] (axis cs:1.10000E+02,4.00000E-01) -- (axis cs:1.10000E+02,-4.00000E-01)   ;
\node[pin={[pin distance=-0.4\onedegree,Equator-label]090:{110$^\circ$}}] at (axis cs:1.10000E+02,4.00000E-01) {} ;
\draw [Equator-empty] (axis cs:1.20000E+02,4.00000E-01) -- (axis cs:1.20000E+02,-4.00000E-01)   ;
\node[pin={[pin distance=-0.4\onedegree,Equator-label]090:{120$^\circ$}}] at (axis cs:1.20000E+02,4.00000E-01) {} ;
\draw [Equator-empty] (axis cs:1.30000E+02,4.00000E-01) -- (axis cs:1.30000E+02,-4.00000E-01)   ;
\node[pin={[pin distance=-0.4\onedegree,Equator-label]090:{130$^\circ$}}] at (axis cs:1.30000E+02,4.00000E-01) {} ;
\draw [Equator-empty] (axis cs:1.40000E+02,4.00000E-01) -- (axis cs:1.40000E+02,-4.00000E-01)   ;
\node[pin={[pin distance=-0.4\onedegree,Equator-label]090:{140$^\circ$}}] at (axis cs:1.40000E+02,4.00000E-01) {} ;
\draw [Equator-empty] (axis cs:1.50000E+02,4.00000E-01) -- (axis cs:1.50000E+02,-4.00000E-01)   ;
\node[pin={[pin distance=-0.4\onedegree,Equator-label]090:{150$^\circ$}}] at (axis cs:1.50000E+02,4.00000E-01) {} ;
\draw [Equator-empty] (axis cs:1.60000E+02,4.00000E-01) -- (axis cs:1.60000E+02,-4.00000E-01)   ;
\node[pin={[pin distance=-0.4\onedegree,Equator-label]090:{160$^\circ$}}] at (axis cs:1.60000E+02,4.00000E-01) {} ;
\draw [Equator-empty] (axis cs:1.70000E+02,4.00000E-01) -- (axis cs:1.70000E+02,-4.00000E-01)   ;
\node[pin={[pin distance=-0.4\onedegree,Equator-label]090:{170$^\circ$}}] at (axis cs:1.70000E+02,4.00000E-01) {} ;
\draw [Equator-empty] (axis cs:1.80000E+02,4.00000E-01) -- (axis cs:1.80000E+02,-4.00000E-01)   ;
\node[pin={[pin distance=-0.4\onedegree,Equator-label]090:{180$^\circ$}}] at (axis cs:1.80000E+02,4.00000E-01) {} ;
\draw [Equator-empty] (axis cs:1.90000E+02,4.00000E-01) -- (axis cs:1.90000E+02,-4.00000E-01)   ;
\node[pin={[pin distance=-0.4\onedegree,Equator-label]090:{190$^\circ$}}] at (axis cs:1.90000E+02,4.00000E-01) {} ;
\draw [Equator-empty] (axis cs:2.00000E+02,4.00000E-01) -- (axis cs:2.00000E+02,-4.00000E-01)   ;
\node[pin={[pin distance=-0.4\onedegree,Equator-label]090:{200$^\circ$}}] at (axis cs:2.00000E+02,4.00000E-01) {} ;
\draw [Equator-empty] (axis cs:2.10000E+02,4.00000E-01) -- (axis cs:2.10000E+02,-4.00000E-01)   ;
\node[pin={[pin distance=-0.4\onedegree,Equator-label]090:{210$^\circ$}}] at (axis cs:2.10000E+02,4.00000E-01) {} ;
\draw [Equator-empty] (axis cs:2.20000E+02,4.00000E-01) -- (axis cs:2.20000E+02,-4.00000E-01)   ;
\node[pin={[pin distance=-0.4\onedegree,Equator-label]090:{220$^\circ$}}] at (axis cs:2.20000E+02,4.00000E-01) {} ;
\draw [Equator-empty] (axis cs:2.30000E+02,4.00000E-01) -- (axis cs:2.30000E+02,-4.00000E-01)   ;
\node[pin={[pin distance=-0.4\onedegree,Equator-label]090:{230$^\circ$}}] at (axis cs:2.30000E+02,4.00000E-01) {} ;
\draw [Equator-empty] (axis cs:2.40000E+02,4.00000E-01) -- (axis cs:2.40000E+02,-4.00000E-01)   ;
\node[pin={[pin distance=-0.4\onedegree,Equator-label]090:{240$^\circ$}}] at (axis cs:2.40000E+02,4.00000E-01) {} ;
\draw [Equator-empty] (axis cs:2.50000E+02,4.00000E-01) -- (axis cs:2.50000E+02,-4.00000E-01)   ;
\node[pin={[pin distance=-0.4\onedegree,Equator-label]090:{250$^\circ$}}] at (axis cs:2.50000E+02,4.00000E-01) {} ;
\draw [Equator-empty] (axis cs:2.60000E+02,4.00000E-01) -- (axis cs:2.60000E+02,-4.00000E-01)   ;
\node[pin={[pin distance=-0.4\onedegree,Equator-label]090:{260$^\circ$}}] at (axis cs:2.60000E+02,4.00000E-01) {} ;
\draw [Equator-empty] (axis cs:2.70000E+02,4.00000E-01) -- (axis cs:2.70000E+02,-4.00000E-01)   ;
\node[pin={[pin distance=-0.4\onedegree,Equator-label]090:{270$^\circ$}}] at (axis cs:2.70000E+02,4.00000E-01) {} ;
\draw [Equator-empty] (axis cs:2.80000E+02,4.00000E-01) -- (axis cs:2.80000E+02,-4.00000E-01)   ;
\node[pin={[pin distance=-0.4\onedegree,Equator-label]090:{280$^\circ$}}] at (axis cs:2.80000E+02,4.00000E-01) {} ;
\draw [Equator-empty] (axis cs:2.90000E+02,4.00000E-01) -- (axis cs:2.90000E+02,-4.00000E-01)   ;
\node[pin={[pin distance=-0.4\onedegree,Equator-label]090:{290$^\circ$}}] at (axis cs:2.90000E+02,4.00000E-01) {} ;
\draw [Equator-empty] (axis cs:3.00000E+02,4.00000E-01) -- (axis cs:3.00000E+02,-4.00000E-01)   ;
\node[pin={[pin distance=-0.4\onedegree,Equator-label]090:{300$^\circ$}}] at (axis cs:3.00000E+02,4.00000E-01) {} ;
\draw [Equator-empty] (axis cs:3.10000E+02,4.00000E-01) -- (axis cs:3.10000E+02,-4.00000E-01)   ;
\node[pin={[pin distance=-0.4\onedegree,Equator-label]090:{310$^\circ$}}] at (axis cs:3.10000E+02,4.00000E-01) {} ;
\draw [Equator-empty] (axis cs:3.20000E+02,4.00000E-01) -- (axis cs:3.20000E+02,-4.00000E-01)   ;
\node[pin={[pin distance=-0.4\onedegree,Equator-label]090:{320$^\circ$}}] at (axis cs:3.20000E+02,4.00000E-01) {} ;
\draw [Equator-empty] (axis cs:3.30000E+02,4.00000E-01) -- (axis cs:3.30000E+02,-4.00000E-01)   ;
\node[pin={[pin distance=-0.4\onedegree,Equator-label]090:{330$^\circ$}}] at (axis cs:3.30000E+02,4.00000E-01) {} ;
\draw [Equator-empty] (axis cs:3.40000E+02,4.00000E-01) -- (axis cs:3.40000E+02,-4.00000E-01)   ;
\node[pin={[pin distance=-0.4\onedegree,Equator-label]090:{340$^\circ$}}] at (axis cs:3.40000E+02,4.00000E-01) {} ;
\draw [Equator-empty] (axis cs:3.50000E+02,4.00000E-01) -- (axis cs:3.50000E+02,-4.00000E-01)   ;
\node[pin={[pin distance=-0.4\onedegree,Equator-label]090:{350$^\circ$}}] at (axis cs:3.50000E+02,4.00000E-01) {} ;

\draw [Equator-empty] (axis cs:3.60000E+02,4.00000E-01) -- (axis cs:3.60000E+02,-4.00000E-01)   ;
\node[pin={[pin distance=-0.4\onedegree,Equator-label]090:{0$^\circ$}}] at (axis cs:3.60000E+02,4.00000E-01) {} ;
\draw [Equator-empty] (axis cs:3.70000E+02,4.00000E-01) -- (axis cs:3.70000E+02,-4.00000E-01)   ;
\node[pin={[pin distance=-0.4\onedegree,Equator-label]090:{10$^\circ$}}] at (axis cs:3.70000E+02,4.00000E-01) {} ;




\end{axis}


% Coordinate axes 


% Left and bottom axis with gridlines
  \begin{axis}[at=(base.center),anchor=center,RA_in_hours,color=cAxes] \end{axis}
% Right and top axis, labelling in RA hours (offset)
  \begin{axis}[at=(base.center),anchor=center,yticklabel pos=right,xticklabel pos=right,RA_in_hours
  ,xticklabel style={yshift=2.5ex, anchor=south},color=cAxes]\end{axis}
% Top axis labelling in RA degree (small offset)
  \begin{axis}[at=(base.center),anchor=center,axis y line=none,xticklabel pos=top,RA_in_deg
     ,xticklabel style={font=\footnotesize},xticklabel style={yshift=0.5ex, anchor=south} ,color=cAxes]
  \end{axis}

% Coordinate grid

\begin{axis}[at=(base.center),anchor=center,coordinategrid,RA_in_deg,
    separate axis lines,
    y axis line style= { draw opacity=0 },
    x axis line style= { draw opacity=0 },
    ,xticklabels={},yticklabels={}
] \end{axis}


% Ornamental frame


  \begin{axis}[name=frame,at={($(base.center)+(0cm,+.7\onedegree)$)},anchor=center,width=11.75\tendegree,height=8.53\tendegree,ticks=none
  ,yticklabels={},xticklabels={},axis line style={line width=1pt},color=cFrame] \end{axis}

  \begin{axis}[name=frame2,at={($(frame.center)+(0cm,+0cm)$)},anchor=center,width=11.85\tendegree,height=8.63\tendegree,ticks=none
  ,yticklabels={},xticklabels={},axis line style={line width=3pt},color=cFrame] \end{axis}
  

% Symbol legend


 
 \begin{axis}[name=legend,at=(frame2.south),anchor=north,axis y line=none,axis x
   line=none,xmin=-55,xmax=55,height=0.5\tendegree   ,ymin=-1.5,ymax=1.5,  ,y dir=normal,x dir=reverse, color=cAxes]

\node at (axis cs:54,+1.0) {\small\it 1};                  \node at (axis cs:54,0.3)  {\tikz{\pgfuseplotmark{m1b};}};           
\node at (axis cs:52,+1.0) {\small\it 1{\nicefrac{1}{3}}}; \node at (axis cs:52,0.3)  {\tikz{\pgfuseplotmark{m1c};}};           
\node at (axis cs:53,-0.5)  {\small First};                  

\node at (axis cs:48,+1.0) {\small\it 1{\nicefrac{2}{3}}}; \node at (axis cs:48,0.3)  {\tikz{\pgfuseplotmark{m2a};}};           
\node at (axis cs:46,+1.0) {\small\it 2};                  \node at (axis cs:46,0.3)  {\tikz{\pgfuseplotmark{m2b};}};           
\node at (axis cs:44,+1.0) {\small\it 2{\nicefrac{1}{3}}}; \node at (axis cs:44,0.3)  {\tikz{\pgfuseplotmark{m2c};}};           
\node at (axis cs:46,-0.5)  {\small Second};

\node at (axis cs:40,+1.0) {\small\it 2{\nicefrac{2}{3}}}; \node at (axis cs:40,0.3)  {\tikz{\pgfuseplotmark{m3a};}};           
\node at (axis cs:38,+1.0) {\small\it 3};                  \node at (axis cs:38,0.3)  {\tikz{\pgfuseplotmark{m3b};}};           
\node at (axis cs:36,+1.0) {\small\it 3{\nicefrac{1}{3}}}; \node at (axis cs:36,0.3)  {\tikz{\pgfuseplotmark{m3c};}};           
\node at (axis cs:38,-0.5)  {\small Third};

\node at (axis cs:32,+1.0) {\small\it 3{\nicefrac{2}{3}}}; \node at (axis cs:32,0.3)  {\tikz{\pgfuseplotmark{m4a};}};           
\node at (axis cs:30,+1.0) {\small\it 4};                  \node at (axis cs:30,0.3)  {\tikz{\pgfuseplotmark{m4b};}};           
\node at (axis cs:28,+1.0) {\small\it 4{\nicefrac{1}{3}}}; \node at (axis cs:28,0.3)  {\tikz{\pgfuseplotmark{m4c};}};           
\node at (axis cs:30,-0.5)  {\small Fourth};                                                                      
                                                                                                                  
\node at (axis cs:24,+1.0) {\small\it 4{\nicefrac{2}{3}}}; \node at (axis cs:24,0.3)  {\tikz{\pgfuseplotmark{m5a};}};           
\node at (axis cs:22,+1.0) {\small\it 5};                  \node at (axis cs:22,0.3)  {\tikz{\pgfuseplotmark{m5b};}};           
\node at (axis cs:20,+1.0) {\small\it 5{\nicefrac{1}{3}}}; \node at (axis cs:20,0.3)  {\tikz{\pgfuseplotmark{m5c};}};           
\node at (axis cs:22,-0.5)  {\small Fifth};

\node at (axis cs:16,+1.0) {\small\it 5{\nicefrac{2}{3}}}; \node at (axis cs:16,0.3)  {\tikz{\pgfuseplotmark{m6a};}};           
\node at (axis cs:14,+1.0) {\small\it 6};                  \node at (axis cs:14,0.3)  {\tikz{\pgfuseplotmark{m6b};}};           
\node at (axis cs:12,+1.0) {\small\it 6{\nicefrac{1}{3}}}; \node at (axis cs:12,0.3)  {\tikz{\pgfuseplotmark{m6c};}};           
\node at (axis cs:14,-0.5)  {\small Sixth};                                                                       
                                                                                                                  
                                                                                                                 
\node at  (axis cs:6,+1.0)   {\small\it Variable};  \node at  (axis cs:6,+0.3)       {\tikz{\pgfuseplotmark{m2av};}};           
\node at  (axis cs:2,+.992)    {\small\it Binary};   \node at  (axis cs:2,+0.3)      {\tikz{\pgfuseplotmark{m2ab};}};           
\node at  (axis cs:-2,+1.0)   {\small\it Var. \& Bin.};  \node at  (axis cs:-2,+0.3) {\tikz{\pgfuseplotmark{m2avb};}};           


\node at  (axis cs:-17,+.992)   {\small\it Planetary};  \node at  (axis cs:-18,+0.3)  {\tikz{\pgfuseplotmark{PN};}};           
\node at  (axis cs:-22,+1.0)   {\small\it Emission};   \node at  (axis cs:-22,+0.3)   {\tikz{\pgfuseplotmark{EN};}};           
\node at  (axis cs:-27,+1.0)   {\small\it w/ Cluster};  \node at  (axis cs:-26,+0.3)  {\tikz{\pgfuseplotmark{CN};}};           
\node at (axis cs:-22,-0.5)  {\small Nebulae};


\node at  (axis cs:-32,+.994)   {\small\it Open};      \node at  (axis cs:-32,+0.3)   {\tikz{\pgfuseplotmark{OC};}};           
\node at  (axis cs:-37,+1.0)   {\small\it Globular};   \node at  (axis cs:-37,+0.3)   {\tikz{\pgfuseplotmark{GC};}};           
\node at (axis cs:-34.5,-0.5)  {\small Cluster};

\node at  (axis cs:-42,+.994)   {\small\it Galaxy};      \node at  (axis cs:-42,+0.3) {\tikz{\pgfuseplotmark{GAL};}};         

%
\end{axis}
         



% Constellations
%
% Some are commented out because they are surrounded by already plotted borders
% The x-coordinate is in fractional hours, so must be times 15
%

\begin{axis}[name=constellations,at=(base.center),anchor=center,axis lines=none]

\draw[constellation-boundary]   ++ (0,0);  \draw[constellation-boundary]  (axis cs:39.6793,    +37.293149) --  (axis cs:39.6823,    +37.543138) --  (axis cs:39.6945,    +38.543092) --  (axis cs:39.707,    +39.543044) --  (axis cs:39.7199,    +40.542995) --  (axis cs:39.7333,    +41.542945) --  (axis cs:39.747,    +42.542893) --  (axis cs:39.7612,    +43.542839) --  (axis cs:39.7758,    +44.542784) ;
  \draw[constellation-boundary]  (axis cs:31.8711,    +37.347078) --  (axis cs:32.8789,    +37.340717) --  (axis cs:33.8866,    +37.334175) --  (axis cs:34.8943,    +37.327454) --  (axis cs:35.9018,    +37.320556) --  (axis cs:36.9093,    +37.313483) --  (axis cs:37.9166,    +37.306237) --  (axis cs:38.9239,    +37.298822) --  (axis cs:39.931,    +37.291238) --  (axis cs:40.4346,    +37.287384) ;
  \draw[constellation-boundary]  (axis cs:31.8542,    +35.597131) --  (axis cs:31.8638,    +36.597101) --  (axis cs:31.8711,    +37.347078) ;
  \draw[constellation-boundary]  (axis cs:22.9109,    +35.645330) --  (axis cs:23.7928,    +35.641253) --  (axis cs:24.8008,    +35.636410) --  (axis cs:25.8086,    +35.631371) --  (axis cs:26.8164,    +35.626139) --  (axis cs:27.8241,    +35.620714) --  (axis cs:28.8317,    +35.615099) --  (axis cs:29.8393,    +35.609296) --  (axis cs:30.8468,    +35.603306) --  (axis cs:31.8542,    +35.597131) ;
  \draw[constellation-boundary]  (axis cs:12.443,    +33.681888) --  (axis cs:12.695,    +33.681269) --  (axis cs:13.7028,    +33.678660) --  (axis cs:14.7106,    +33.675843) --  (axis cs:15.7184,    +33.672819) --  (axis cs:16.7261,    +33.669589) --  (axis cs:17.7338,    +33.666153) --  (axis cs:18.7414,    +33.662512) --  (axis cs:19.749,    +33.658669) --  (axis cs:20.7566,    +33.654623) --  (axis cs:21.7641,    +33.650376) --  (axis cs:22.7715,    +33.645930) --  (axis cs:22.8975,    +33.645360) ;
  \draw[constellation-boundary]  (axis cs:12.4134,    +24.431924) --  (axis cs:12.4142,    +24.681923) --  (axis cs:12.4171,    +25.681919) --  (axis cs:12.4201,    +26.681916) --  (axis cs:12.4232,    +27.681912) --  (axis cs:12.4264,    +28.681908) --  (axis cs:12.4295,    +29.681904) --  (axis cs:12.4328,    +30.681900) --  (axis cs:12.4361,    +31.681896) --  (axis cs:12.4395,    +32.681892) --  (axis cs:12.443,    +33.681888) ;
  \draw[constellation-boundary]  (axis cs:12.4134,    +24.431924) --  (axis cs:12.6648,    +24.431306) --  (axis cs:13.6701,    +24.428704) --  (axis cs:14.424,    +24.426616) ;
  \draw[constellation-boundary]  (axis cs:14.4148,    +21.676629) --  (axis cs:14.4181,    +22.676625) --  (axis cs:14.4215,    +23.676620) --  (axis cs:14.424,    +24.426616) ;
  \draw[constellation-boundary]  (axis cs:3.73985,    +21.695185) --  (axis cs:4.61901,    +21.694558) --  (axis cs:5.62376,    +21.693643) --  (axis cs:6.62849,    +21.692515) --  (axis cs:7.63322,    +21.691176) --  (axis cs:8.63794,    +21.689625) --  (axis cs:9.64265,    +21.687863) --  (axis cs:10.6473,    +21.685890) --  (axis cs:11.652,    +21.683708) --  (axis cs:12.6567,    +21.681316) --  (axis cs:13.6613,    +21.678716) --  (axis cs:14.4148,    +21.676629) ;
  \draw[constellation-boundary]  (axis cs:2.60995,    +22.695751) --  (axis cs:3.61495,    +22.695261) --  (axis cs:3.74058,    +22.695184) ;
  \draw[constellation-boundary]  (axis cs:2.60995,    +22.695751) --  (axis cs:2.6104,    +23.695751) --  (axis cs:2.61086,    +24.695751) --  (axis cs:2.61133,    +25.695750) --  (axis cs:2.6118,    +26.695750) --  (axis cs:2.61228,    +27.695750) --  (axis cs:2.61277,    +28.695750) ;
  \draw[constellation-boundary]  (axis cs:1.6062,    +28.696028) --  (axis cs:2.61277,    +28.695750) ;
  \draw[constellation-boundary]  (axis cs:1.6062,    +28.696028) --  (axis cs:1.60642,    +29.696028) --  (axis cs:1.60665,    +30.696028) --  (axis cs:1.60688,    +31.696028) --  (axis cs:1.60696,    +32.029361) ;
  \draw[constellation-boundary]  (axis cs:-2.171,    +32.028500) --  (axis cs:-1.416,    +32.028913) --  (axis cs:-0.408,    +32.029276) --  (axis cs:0.599433,    +32.029425) --  (axis cs:1.60696,    +32.029361) ;
  \draw[constellation-boundary]  (axis cs:-2.171,    +32.028500) --  (axis cs:-2.172,    +32.695167) --  (axis cs:-2.172,    +32.778500) ;
  \draw[constellation-boundary]  (axis cs:-4.439,    +32.776543) --  (axis cs:-3.432,    +32.777546) --  (axis cs:-2.424,    +32.778336) --  (axis cs:-2.172,    +32.778500) ;
  \draw[constellation-boundary]  (axis cs:-4.395,    -03.306766) --  (axis cs:-3.396,    -03.305772) --  (axis cs:-2.397,    -03.304988) --  (axis cs:-1.397,    -03.304417) --  (axis cs:-0.898,    -03.304210) ;
  \draw[constellation-boundary]  (axis cs:-0.897,    -06.304210) --  (axis cs:-0.397,    -06.304056) --  (axis cs:0.60122,    -06.303908) --  (axis cs:1.5998,    -06.303972) --  (axis cs:2.59839,    -06.304247) --  (axis cs:3.59697,    -06.304734) --  (axis cs:4.59556,    -06.305432) --  (axis cs:5.59415,    -06.306342) --  (axis cs:6.59274,    -06.307463) ;
  \draw[constellation-boundary]  (axis cs:-0.889,    -24.804209) --  (axis cs:-0.89,    -24.304209) --  (axis cs:-0.89,    -23.304209) --  (axis cs:-0.89,    -22.304209) --  (axis cs:-0.891,    -21.304209) --  (axis cs:-0.891,    -20.304209) --  (axis cs:-0.892,    -19.304209) --  (axis cs:-0.892,    -18.304209) --  (axis cs:-0.893,    -17.304209) --  (axis cs:-0.893,    -16.304209) --  (axis cs:-0.893,    -15.304209) --  (axis cs:-0.894,    -14.304209) --  (axis cs:-0.894,    -13.304209) --  (axis cs:-0.894,    -12.304209) --  (axis cs:-0.895,    -11.304209) --  (axis cs:-0.895,    -10.304210) --  (axis cs:-0.896,    -09.304210) --  (axis cs:-0.896,    -08.304210) --  (axis cs:-0.896,    -07.304210) --  (axis cs:-0.897,    -06.304210) --  (axis cs:-0.897,    -05.304210) --  (axis cs:-0.897,    -04.304210) --  (axis cs:-0.898,    -03.304210) ;
  \draw[constellation-boundary]  (axis cs:26.6557,    +10.543230) --  (axis cs:27.6576,    +10.537837) --  (axis cs:28.6596,    +10.532254) --  (axis cs:29.6615,    +10.526483) --  (axis cs:30.6634,    +10.520526) --  (axis cs:31.6652,    +10.514385) --  (axis cs:32.6671,    +10.508062) --  (axis cs:33.6689,    +10.501558) --  (axis cs:34.6707,    +10.494876) --  (axis cs:35.6725,    +10.488017) --  (axis cs:36.6743,    +10.480985) --  (axis cs:37.676,    +10.473780) --  (axis cs:38.6777,    +10.466405) --  (axis cs:39.6794,    +10.458862) --  (axis cs:40.681,    +10.451155) --  (axis cs:41.6827,    +10.443284) --  (axis cs:42.6843,    +10.435253) --  (axis cs:43.6859,    +10.427063) --  (axis cs:44.6874,    +10.418718) --  (axis cs:45.6889,    +10.410220) --  (axis cs:46.6904,    +10.401571) --  (axis cs:47.6919,    +10.392775) --  (axis cs:48.6933,    +10.383833) --  (axis cs:49.6947,    +10.374750) --  (axis cs:50.6961,    +10.365526) --  (axis cs:50.9464,    +10.363199) ;
  \draw[constellation-boundary]  (axis cs:26.6557,    +10.543230) --  (axis cs:26.6561,    +10.626562) --  (axis cs:26.6616,    +11.626548) --  (axis cs:26.6671,    +12.626533) --  (axis cs:26.6727,    +13.626518) --  (axis cs:26.6783,    +14.626504) --  (axis cs:26.684,    +15.626489) --  (axis cs:26.6897,    +16.626473) --  (axis cs:26.6955,    +17.626458) --  (axis cs:26.7014,    +18.626443) --  (axis cs:26.7073,    +19.626427) --  (axis cs:26.7133,    +20.626411) --  (axis cs:26.7194,    +21.626395) --  (axis cs:26.7256,    +22.626379) --  (axis cs:26.7318,    +23.626362) --  (axis cs:26.7382,    +24.626345) --  (axis cs:26.7446,    +25.626328) --  (axis cs:26.7512,    +26.626311) --  (axis cs:26.7578,    +27.626293) --  (axis cs:26.7646,    +28.626275) ;
  \draw[constellation-boundary]  (axis cs:26.7446,    +25.626328) --  (axis cs:27.7497,    +25.620918) --  (axis cs:28.7548,    +25.615317) --  (axis cs:29.7599,    +25.609528) --  (axis cs:30.5136,    +25.605064) ;
  \draw[constellation-boundary]  (axis cs:30.5136,    +25.605064) --  (axis cs:30.5211,    +26.605042) --  (axis cs:30.5286,    +27.605019) --  (axis cs:30.5305,    +27.855014) ;
  \draw[constellation-boundary]  (axis cs:30.5305,    +27.855014) --  (axis cs:30.7819,    +27.853502) --  (axis cs:31.7874,    +27.847339) --  (axis cs:32.7928,    +27.840993) --  (axis cs:33.7981,    +27.834466) --  (axis cs:34.8034,    +27.827761) --  (axis cs:35.8086,    +27.820879) --  (axis cs:36.8137,    +27.813822) --  (axis cs:37.8188,    +27.806593) --  (axis cs:38.0701,    +27.804759) ;
  \draw[constellation-boundary]  (axis cs:38.0701,    +27.804759) --  (axis cs:38.0771,    +28.554734) --  (axis cs:38.0867,    +29.554699) --  (axis cs:38.0965,    +30.554663) --  (axis cs:38.1031,    +31.221305) ;
  \draw[constellation-boundary]  (axis cs:38.1031,    +31.221305) --  (axis cs:38.8574,    +31.215736) --  (axis cs:39.8631,    +31.208163) --  (axis cs:40.8687,    +31.200425) --  (axis cs:41.8742,    +31.192524) --  (axis cs:42.8797,    +31.184461) --  (axis cs:43.885,    +31.176241) --  (axis cs:44.8902,    +31.167865) --  (axis cs:45.8954,    +31.159336) --  (axis cs:46.9005,    +31.150657) --  (axis cs:47.9054,    +31.141829) --  (axis cs:48.9103,    +31.132857) --  (axis cs:49.9151,    +31.123742) --  (axis cs:50.9198,    +31.114488) --  (axis cs:51.9243,    +31.105098) --  (axis cs:52.9288,    +31.095573) --  (axis cs:53.9332,    +31.085918) --  (axis cs:54.9375,    +31.076136) --  (axis cs:55.9416,    +31.066228) --  (axis cs:56.9457,    +31.056199) --  (axis cs:57.9496,    +31.046051) --  (axis cs:58.9535,    +31.035788) --  (axis cs:59.9572,    +31.025412) --  (axis cs:60.9608,    +31.014928) --  (axis cs:61.9643,    +31.004337) --  (axis cs:62.9677,    +30.993644) --  (axis cs:63.971,    +30.982851) --  (axis cs:64.9741,    +30.971963) --  (axis cs:65.9772,    +30.960982) --  (axis cs:66.9801,    +30.949911) --  (axis cs:67.9829,    +30.938755) --  (axis cs:68.9856,    +30.927516) --  (axis cs:69.4869,    +30.921867) ;
  \draw[constellation-boundary]  (axis cs:42.6283,    +31.186492) --  (axis cs:42.632,    +31.519810) --  (axis cs:42.6432,    +32.519765) --  (axis cs:42.6547,    +33.519719) --  (axis cs:42.6664,    +34.519672) ;
  \draw[constellation-boundary]  (axis cs:52.2906,    +19.434330) --  (axis cs:52.3013,    +20.434280) --  (axis cs:52.3122,    +21.434228) --  (axis cs:52.3232,    +22.434176) --  (axis cs:52.3344,    +23.434123) --  (axis cs:52.3457,    +24.434069) --  (axis cs:52.3572,    +25.434014) --  (axis cs:52.369,    +26.433959) --  (axis cs:52.3809,    +27.433902) --  (axis cs:52.393,    +28.433845) --  (axis cs:52.4054,    +29.433786) --  (axis cs:52.418,    +30.433726) --  (axis cs:52.4266,    +31.100352) ;
  \draw[constellation-boundary]  (axis cs:51.0373,    +19.446109) --  (axis cs:51.7892,    +19.439067) --  (axis cs:52.2906,    +19.434330) ;
  \draw[constellation-boundary]  (axis cs:69.5738,    +36.254710) --  (axis cs:70.0754,    +36.249038) --  (axis cs:71.0784,    +36.237638) --  (axis cs:72.0812,    +36.226166) --  (axis cs:72.4573,    +36.221847) ;
  \draw[constellation-boundary]  (axis cs:72.4573,    +36.221847) --  (axis cs:72.4752,    +37.221744) --  (axis cs:72.4935,    +38.221638) --  (axis cs:72.5124,    +39.221530) --  (axis cs:72.5319,    +40.221418) --  (axis cs:72.5519,    +41.221303) --  (axis cs:72.5725,    +42.221184) --  (axis cs:72.5939,    +43.221062) --  (axis cs:72.6159,    +44.220935) ;
  \draw[constellation-boundary]  (axis cs:104.265,    +44.341840) --  (axis cs:104.277,    +44.841771) ;
  \draw[constellation-boundary]  (axis cs:104.265,    +44.341840) --  (axis cs:105.262,    +44.330051) --  (axis cs:106.259,    +44.318314) --  (axis cs:107.256,    +44.306633) --  (axis cs:108.252,    +44.295012) --  (axis cs:109.249,    +44.283455) --  (axis cs:110.245,    +44.271964) --  (axis cs:111.241,    +44.260544) --  (axis cs:112.236,    +44.249198) --  (axis cs:112.734,    +44.243553) ;
  \draw[constellation-boundary]  (axis cs:112.561,    +35.244534) --  (axis cs:112.569,    +35.744486) --  (axis cs:112.587,    +36.744388) --  (axis cs:112.604,    +37.744286) --  (axis cs:112.623,    +38.744183) --  (axis cs:112.642,    +39.744076) --  (axis cs:112.661,    +40.743966) --  (axis cs:112.681,    +41.743853) --  (axis cs:112.702,    +42.743736) --  (axis cs:112.723,    +43.743615) --  (axis cs:112.734,    +44.243553) ;
  \draw[constellation-boundary]  (axis cs:100.09,    +35.390565) --  (axis cs:101.089,    +35.378593) --  (axis cs:102.087,    +35.366659) --  (axis cs:103.085,    +35.354766) --  (axis cs:104.083,    +35.342919) --  (axis cs:105.081,    +35.331120) --  (axis cs:106.079,    +35.319373) --  (axis cs:107.076,    +35.307682) --  (axis cs:108.074,    +35.296050) --  (axis cs:109.071,    +35.284481) --  (axis cs:110.068,    +35.272978) --  (axis cs:111.065,    +35.261545) --  (axis cs:112.062,    +35.250186) --  (axis cs:113.059,    +35.238903) --  (axis cs:114.056,    +35.227700) ;
  \draw[constellation-boundary]  (axis cs:99.9657,    +27.891313) --  (axis cs:99.9812,    +28.891220) --  (axis cs:99.997,    +29.891125) --  (axis cs:100.013,    +30.891028) --  (axis cs:100.03,    +31.890929) --  (axis cs:100.046,    +32.890828) --  (axis cs:100.064,    +33.890724) --  (axis cs:100.081,    +34.890618) --  (axis cs:100.09,    +35.390565) ;
  \draw[constellation-boundary]  (axis cs:90.2211,    +28.009289) --  (axis cs:90.9711,    +28.000178) --  (axis cs:91.9709,    +27.988030) --  (axis cs:92.9707,    +27.975886) --  (axis cs:93.9703,    +27.963751) --  (axis cs:94.9698,    +27.951627) --  (axis cs:95.9692,    +27.939519) --  (axis cs:96.9685,    +27.927429) --  (axis cs:97.9677,    +27.915363) --  (axis cs:98.9667,    +27.903323) --  (axis cs:99.9657,    +27.891313) ;
  \draw[constellation-boundary]  (axis cs:73.2125,    +28.712437) --  (axis cs:73.9639,    +28.703734) --  (axis cs:74.9657,    +28.692076) --  (axis cs:75.9674,    +28.680359) --  (axis cs:76.9689,    +28.668589) --  (axis cs:77.9704,    +28.656768) --  (axis cs:78.9717,    +28.644899) --  (axis cs:79.973,    +28.632987) --  (axis cs:80.9741,    +28.621035) --  (axis cs:81.9751,    +28.609047) --  (axis cs:82.976,    +28.597026) --  (axis cs:83.9767,    +28.584977) --  (axis cs:84.9774,    +28.572902) --  (axis cs:85.9779,    +28.560806) --  (axis cs:86.9783,    +28.548691) --  (axis cs:87.9787,    +28.536563) --  (axis cs:88.9788,    +28.524424) --  (axis cs:89.9789,    +28.512279) --  (axis cs:90.2289,    +28.509242) ;
  \draw[constellation-boundary]  (axis cs:73.2125,    +28.712437) --  (axis cs:73.22,    +29.212394) --  (axis cs:73.2353,    +30.212305) ;
  \draw[constellation-boundary]  (axis cs:69.4768,    +30.255257) --  (axis cs:69.492,    +31.255171) --  (axis cs:69.5077,    +32.255083) --  (axis cs:69.5236,    +33.254993) --  (axis cs:69.54,    +34.254901) --  (axis cs:69.5567,    +35.254806) --  (axis cs:69.5738,    +36.254710) ;
  \draw[constellation-boundary]  (axis cs:69.4768,    +30.255257) --  (axis cs:69.978,    +30.249589) --  (axis cs:70.9804,    +30.238197) --  (axis cs:71.9827,    +30.226732) --  (axis cs:72.9848,    +30.215199) --  (axis cs:73.2353,    +30.212305) ;
  \draw[constellation-boundary]  (axis cs:64.976,    -44.700182) --  (axis cs:64.9974,    -43.700299) --  (axis cs:65.0181,    -42.700413) --  (axis cs:65.0381,    -41.700523) --  (axis cs:65.0575,    -40.700630) --  (axis cs:65.0764,    -39.700734) --  (axis cs:65.0947,    -38.700834) --  (axis cs:65.1125,    -37.700932) --  (axis cs:65.1298,    -36.701027) ;
  \draw[constellation-boundary]  (axis cs:73.4377,    -44.796112) --  (axis cs:73.4604,    -43.796244) --  (axis cs:73.4824,    -42.796372) ;
  \draw[constellation-boundary]  (axis cs:73.4824,    -42.796372) --  (axis cs:73.9807,    -42.802179) --  (axis cs:74.9775,    -42.813838) --  (axis cs:75.9744,    -42.825554) --  (axis cs:76.9716,    -42.837325) --  (axis cs:77.9689,    -42.849147) --  (axis cs:78.9664,    -42.861015) --  (axis cs:79.9641,    -42.872927) --  (axis cs:80.962,    -42.884878) --  (axis cs:81.9601,    -42.896866) --  (axis cs:82.9584,    -42.908886) --  (axis cs:83.9569,    -42.920935) --  (axis cs:84.9556,    -42.933010) --  (axis cs:85.9544,    -42.945105) --  (axis cs:86.9535,    -42.957219) --  (axis cs:87.9528,    -42.969347) --  (axis cs:88.9522,    -42.981486) --  (axis cs:89.9519,    -42.993631) --  (axis cs:90.9518,    -43.005780) --  (axis cs:91.9518,    -43.017927) --  (axis cs:92.9521,    -43.030071) --  (axis cs:93.9525,    -43.042207) --  (axis cs:94.9532,    -43.054331) --  (axis cs:95.954,    -43.066440) --  (axis cs:96.9551,    -43.078529) --  (axis cs:97.9563,    -43.090596) --  (axis cs:98.9577,    -43.102636) --  (axis cs:99.709,    -43.111647) ;
  \draw[constellation-boundary]  (axis cs:75.9744,    -42.825554) --  (axis cs:75.996,    -41.825681) --  (axis cs:76.0169,    -40.825804) --  (axis cs:76.0371,    -39.825924) --  (axis cs:76.0568,    -38.826040) --  (axis cs:76.076,    -37.826152) --  (axis cs:76.0946,    -36.826262) --  (axis cs:76.1127,    -35.826369) --  (axis cs:76.1304,    -34.826473) --  (axis cs:76.1477,    -33.826575) --  (axis cs:76.1646,    -32.826674) --  (axis cs:76.1811,    -31.826771) --  (axis cs:76.1972,    -30.826866) --  (axis cs:76.213,    -29.826960) --  (axis cs:76.2285,    -28.827051) --  (axis cs:76.2437,    -27.827140) --  (axis cs:76.2549,    -27.077206) ;
  \draw[constellation-boundary]  (axis cs:73.7593,    -27.047983) --  (axis cs:73.763,    -26.798004) --  (axis cs:73.7775,    -25.798089) --  (axis cs:73.7917,    -24.798171) --  (axis cs:73.8057,    -23.798253) --  (axis cs:73.8195,    -22.798333) --  (axis cs:73.8331,    -21.798412) --  (axis cs:73.8465,    -20.798490) --  (axis cs:73.8597,    -19.798567) --  (axis cs:73.8728,    -18.798643) --  (axis cs:73.8857,    -17.798718) --  (axis cs:73.8985,    -16.798793) --  (axis cs:73.9111,    -15.798866) --  (axis cs:73.9236,    -14.798939) --  (axis cs:73.9298,    -14.298975) ;
  \draw[constellation-boundary]  (axis cs:71.7633,    -27.024881) --  (axis cs:72.2623,    -27.030632) --  (axis cs:73.2603,    -27.042183) --  (axis cs:74.2584,    -27.053798) --  (axis cs:75.2566,    -27.065474) --  (axis cs:76.2549,    -27.077206) --  (axis cs:77.2533,    -27.088992) --  (axis cs:78.2519,    -27.100827) --  (axis cs:79.2505,    -27.112708) --  (axis cs:80.2492,    -27.124631) --  (axis cs:81.2481,    -27.136594) --  (axis cs:82.247,    -27.148591) --  (axis cs:83.2461,    -27.160620) --  (axis cs:84.2452,    -27.172677) --  (axis cs:85.2445,    -27.184757) --  (axis cs:86.2439,    -27.196859) --  (axis cs:87.2434,    -27.208977) --  (axis cs:88.2429,    -27.221109) --  (axis cs:89.2426,    -27.233249) --  (axis cs:90.2425,    -27.245396) --  (axis cs:91.2424,    -27.257545) --  (axis cs:92.2424,    -27.269692) --  (axis cs:92.9925,    -27.278799) ;
  \draw[constellation-boundary]  (axis cs:71.7224,    -29.774645) --  (axis cs:71.7375,    -28.774732) --  (axis cs:71.7524,    -27.774818) --  (axis cs:71.7633,    -27.024881) ;
  \draw[constellation-boundary]  (axis cs:69.9767,    -29.754662) --  (axis cs:70.2261,    -29.757503) --  (axis cs:71.2236,    -29.768913) --  (axis cs:71.7224,    -29.774645) ;
  \draw[constellation-boundary]  (axis cs:69.8624,    -36.754011) --  (axis cs:69.8799,    -35.754111) --  (axis cs:69.897,    -34.754208) --  (axis cs:69.9137,    -33.754303) --  (axis cs:69.9299,    -32.754396) --  (axis cs:69.9459,    -31.754486) --  (axis cs:69.9614,    -30.754575) --  (axis cs:69.9767,    -29.754662) ;
  \draw[constellation-boundary]  (axis cs:65.1298,    -36.701027) --  (axis cs:66.1259,    -36.712022) --  (axis cs:67.1221,    -36.723105) --  (axis cs:68.1184,    -36.734273) --  (axis cs:69.1149,    -36.745523) --  (axis cs:69.8624,    -36.754011) ;
  \draw[constellation-boundary]  (axis cs:111.677,    -33.250465) --  (axis cs:111.693,    -32.250556) --  (axis cs:111.709,    -31.250644) --  (axis cs:111.724,    -30.250731) --  (axis cs:111.739,    -29.250816) --  (axis cs:111.754,    -28.250899) --  (axis cs:111.768,    -27.250981) --  (axis cs:111.783,    -26.251062) --  (axis cs:111.796,    -25.251141) --  (axis cs:111.81,    -24.251218) --  (axis cs:111.824,    -23.251295) --  (axis cs:111.837,    -22.251370) --  (axis cs:111.85,    -21.251444) --  (axis cs:111.863,    -20.251518) --  (axis cs:111.876,    -19.251590) --  (axis cs:111.888,    -18.251661) --  (axis cs:111.901,    -17.251732) --  (axis cs:111.913,    -16.251802) --  (axis cs:111.926,    -15.251871) --  (axis cs:111.938,    -14.251940) --  (axis cs:111.95,    -13.252007) --  (axis cs:111.962,    -12.252075) --  (axis cs:111.973,    -11.252142) ;
  \draw[constellation-boundary]  (axis cs:92.899,    -33.028232) --  (axis cs:92.9161,    -32.028335) --  (axis cs:92.9328,    -31.028437) --  (axis cs:92.9491,    -30.028536) --  (axis cs:92.9652,    -29.028633) --  (axis cs:92.9809,    -28.028729) --  (axis cs:92.9963,    -27.028822) --  (axis cs:93.0115,    -26.028914) --  (axis cs:93.0264,    -25.029005) --  (axis cs:93.0411,    -24.029094) --  (axis cs:93.0555,    -23.029181) --  (axis cs:93.0697,    -22.029268) --  (axis cs:93.0838,    -21.029353) --  (axis cs:93.0976,    -20.029437) --  (axis cs:93.1113,    -19.029520) --  (axis cs:93.1248,    -18.029602) --  (axis cs:93.1381,    -17.029683) --  (axis cs:93.1513,    -16.029763) --  (axis cs:93.1644,    -15.029842) --  (axis cs:93.1773,    -14.029921) --  (axis cs:93.1902,    -13.029999) --  (axis cs:93.2029,    -12.030076) --  (axis cs:93.2156,    -11.030153) ;
  \draw[constellation-boundary]  (axis cs:92.899,    -33.028232) --  (axis cs:93.1491,    -33.031266) --  (axis cs:94.1494,    -33.043400) --  (axis cs:95.1498,    -33.055521) --  (axis cs:96.1504,    -33.067627) --  (axis cs:97.1512,    -33.079712) --  (axis cs:98.152,    -33.091774) --  (axis cs:99.153,    -33.103809) --  (axis cs:100.154,    -33.115813) --  (axis cs:101.155,    -33.127782) --  (axis cs:102.157,    -33.139714) --  (axis cs:103.158,    -33.151603) --  (axis cs:104.16,    -33.163447) --  (axis cs:105.162,    -33.175242) --  (axis cs:106.164,    -33.186984) --  (axis cs:107.166,    -33.198670) --  (axis cs:108.168,    -33.210297) --  (axis cs:109.171,    -33.221859) --  (axis cs:110.173,    -33.233355) --  (axis cs:111.176,    -33.244780) --  (axis cs:111.677,    -33.250465) ;
  \draw[constellation-boundary]  (axis cs:109.6,    -00.224326) --  (axis cs:110.6,    -00.235792) --  (axis cs:111.6,    -00.247187) --  (axis cs:112.599,    -00.258506) --  (axis cs:113.599,    -00.269746) --  (axis cs:114.599,    -00.280904) ;
  \draw[constellation-boundary]  (axis cs:109.6,    -00.224326) --  (axis cs:109.611,    +00.775608) --  (axis cs:109.617,    +01.275574) ;
  \draw[constellation-boundary]  (axis cs:106.867,    +01.307445) --  (axis cs:107.617,    +01.298706) --  (axis cs:108.617,    +01.287108) --  (axis cs:109.617,    +01.275574) ;
  \draw[constellation-boundary]  (axis cs:106.867,    +01.307445) --  (axis cs:106.873,    +01.807411) --  (axis cs:106.885,    +02.807342) --  (axis cs:106.897,    +03.807274) --  (axis cs:106.908,    +04.807205) --  (axis cs:106.914,    +05.307171) ;
  \draw[constellation-boundary]  (axis cs:106.664,    +05.310091) --  (axis cs:106.914,    +05.307171) ;
  \draw[constellation-boundary]  (axis cs:106.664,    +05.310091) --  (axis cs:106.67,    +05.810056) --  (axis cs:106.682,    +06.809987) --  (axis cs:106.694,    +07.809918) --  (axis cs:106.706,    +08.809848) --  (axis cs:106.718,    +09.809778) --  (axis cs:106.73,    +10.809707) --  (axis cs:106.742,    +11.809636) --  (axis cs:106.748,    +12.309600) ;
  \draw[constellation-boundary]  (axis cs:106.748,    +12.309600) --  (axis cs:107.747,    +12.297949) --  (axis cs:108.747,    +12.286359) --  (axis cs:109.746,    +12.274834) --  (axis cs:110.745,    +12.263379) --  (axis cs:111.744,    +12.251995) --  (axis cs:112.743,    +12.240687) --  (axis cs:113.742,    +12.229458) --  (axis cs:114.241,    +12.223875) ;
  \draw[constellation-boundary]  (axis cs:114.241,    +12.223875) --  (axis cs:114.247,    +12.723842) --  (axis cs:114.253,    +13.223809) ;
  \draw[constellation-boundary]  (axis cs:114.253,    +13.223809) --  (axis cs:114.752,    +13.218247) ;
  \draw[constellation-boundary]  (axis cs:-4.37,    -24.806752) --  (axis cs:-3.375,    -24.805762) --  (axis cs:-2.381,    -24.804983) --  (axis cs:-1.387,    -24.804414) --  (axis cs:-0.392,    -24.804056) --  (axis cs:0.602072,    -24.803908) --  (axis cs:1.59638,    -24.803972) --  (axis cs:2.5907,    -24.804246) --  (axis cs:3.58501,    -24.804730) --  (axis cs:4.57933,    -24.805426) --  (axis cs:5.57366,    -24.806332) --  (axis cs:6.56799,    -24.807448) --  (axis cs:7.56234,    -24.808773) --  (axis cs:8.55669,    -24.810308) --  (axis cs:9.55106,    -24.812052) --  (axis cs:10.5454,    -24.814004) --  (axis cs:11.5398,    -24.816165) --  (axis cs:12.5343,    -24.818532) --  (axis cs:13.5287,    -24.821106) --  (axis cs:14.5232,    -24.823885) --  (axis cs:15.5176,    -24.826869) --  (axis cs:16.5121,    -24.830057) --  (axis cs:17.5067,    -24.833449) --  (axis cs:18.5012,    -24.837042) --  (axis cs:19.4958,    -24.840836) --  (axis cs:20.4904,    -24.844830) --  (axis cs:21.4851,    -24.849022) --  (axis cs:22.4797,    -24.853412) --  (axis cs:23.4745,    -24.857997) --  (axis cs:24.4692,    -24.862777) --  (axis cs:25.464,    -24.867751) --  (axis cs:26.4588,    -24.872916) ;
  \draw[constellation-boundary]  (axis cs:90.9287,    -44.005639) --  (axis cs:90.9518,    -43.005780) ;
  \draw[constellation-boundary]  (axis cs:6.60382,    +02.692530) --  (axis cs:7.60431,    +02.691197) --  (axis cs:8.60481,    +02.689652) --  (axis cs:9.6053,    +02.687897) --  (axis cs:10.6058,    +02.685933) --  (axis cs:11.6063,    +02.683760) --  (axis cs:12.6068,    +02.681378) --  (axis cs:13.6072,    +02.678789) --  (axis cs:14.6077,    +02.675992) --  (axis cs:15.6082,    +02.672990) --  (axis cs:16.6087,    +02.669783) --  (axis cs:17.6091,    +02.666372) --  (axis cs:18.6096,    +02.662757) --  (axis cs:19.6101,    +02.658941) --  (axis cs:20.6105,    +02.654923) --  (axis cs:21.611,    +02.650706) --  (axis cs:22.6114,    +02.646291) --  (axis cs:23.6119,    +02.641679) --  (axis cs:24.6123,    +02.636872) --  (axis cs:25.6127,    +02.631870) --  (axis cs:26.6132,    +02.626676) --  (axis cs:27.6136,    +02.621291) --  (axis cs:28.614,    +02.615716) --  (axis cs:29.6144,    +02.609954) --  (axis cs:30.6148,    +02.604006) --  (axis cs:31.6152,    +02.597874) ;
  \draw[constellation-boundary]  (axis cs:31.6152,    +02.597874) --  (axis cs:31.6215,    +03.597855) --  (axis cs:31.6277,    +04.597835) --  (axis cs:31.634,    +05.597816) --  (axis cs:31.6403,    +06.597796) --  (axis cs:31.6466,    +07.597777) --  (axis cs:31.653,    +08.597757) --  (axis cs:31.6593,    +09.597737) --  (axis cs:31.6652,    +10.514385) ;
  \draw[constellation-boundary]  (axis cs:50.8367,    -01.302957) --  (axis cs:50.8437,    -00.552989) --  (axis cs:50.853,    +00.446967) --  (axis cs:50.8623,    +01.446924) --  (axis cs:50.8717,    +02.446880) --  (axis cs:50.881,    +03.446837) --  (axis cs:50.8904,    +04.446793) --  (axis cs:50.8997,    +05.446750) --  (axis cs:50.9092,    +06.446706) --  (axis cs:50.9186,    +07.446662) --  (axis cs:50.9281,    +08.446618) --  (axis cs:50.9376,    +09.446573) --  (axis cs:50.9472,    +10.446528) --  (axis cs:50.9569,    +11.446483) --  (axis cs:50.9666,    +12.446438) --  (axis cs:50.9764,    +13.446392) --  (axis cs:50.9863,    +14.446346) --  (axis cs:50.9963,    +15.446300) --  (axis cs:51.0064,    +16.446253) --  (axis cs:51.0166,    +17.446206) --  (axis cs:51.0269,    +18.446158) --  (axis cs:51.0373,    +19.446109) ;
  \draw[constellation-boundary]  (axis cs:41.3392,    -01.221032) --  (axis cs:41.5891,    -01.223011) --  (axis cs:42.5889,    -01.231028) --  (axis cs:43.5886,    -01.239202) --  (axis cs:44.5883,    -01.247532) --  (axis cs:45.5881,    -01.256015) --  (axis cs:46.5878,    -01.264648) --  (axis cs:47.5875,    -01.273430) --  (axis cs:48.5873,    -01.282356) --  (axis cs:49.587,    -01.291425) --  (axis cs:50.5868,    -01.300633) --  (axis cs:51.5865,    -01.309979) --  (axis cs:52.5863,    -01.319458) --  (axis cs:53.586,    -01.329068) --  (axis cs:54.5858,    -01.338807) --  (axis cs:55.3356,    -01.346193) ;
  \draw[constellation-boundary]  (axis cs:41.1487,    -23.853614) --  (axis cs:41.1524,    -23.470295) --  (axis cs:41.1617,    -22.470331) --  (axis cs:41.1709,    -21.470368) --  (axis cs:41.1799,    -20.470403) --  (axis cs:41.1889,    -19.470439) --  (axis cs:41.1977,    -18.470474) --  (axis cs:41.2064,    -17.470508) --  (axis cs:41.215,    -16.470542) --  (axis cs:41.2236,    -15.470576) --  (axis cs:41.2321,    -14.470609) --  (axis cs:41.2404,    -13.470642) --  (axis cs:41.2488,    -12.470675) --  (axis cs:41.257,    -11.470708) --  (axis cs:41.2652,    -10.470740) --  (axis cs:41.2734,    -09.470772) --  (axis cs:41.2814,    -08.470804) --  (axis cs:41.2895,    -07.470836) --  (axis cs:41.2975,    -06.470868) --  (axis cs:41.3055,    -05.470899) --  (axis cs:41.3135,    -04.470931) --  (axis cs:41.3214,    -03.470962) --  (axis cs:41.3293,    -02.470993) --  (axis cs:41.3372,    -01.471025) --  (axis cs:41.3392,    -01.221032) ;
  \draw[constellation-boundary]  (axis cs:26.4659,    -23.756268) --  (axis cs:27.4611,    -23.761625) --  (axis cs:28.4562,    -23.767170) --  (axis cs:29.4514,    -23.772902) --  (axis cs:30.4467,    -23.778820) --  (axis cs:31.442,    -23.784920) --  (axis cs:32.4373,    -23.791203) --  (axis cs:33.4327,    -23.797665) --  (axis cs:34.4282,    -23.804304) --  (axis cs:35.4236,    -23.811120) --  (axis cs:36.4192,    -23.818109) --  (axis cs:37.4148,    -23.825270) --  (axis cs:38.4104,    -23.832600) --  (axis cs:39.4061,    -23.840097) --  (axis cs:40.4019,    -23.847760) --  (axis cs:41.3977,    -23.855585) --  (axis cs:42.3936,    -23.863570) --  (axis cs:43.3895,    -23.871714) --  (axis cs:44.3855,    -23.880013) --  (axis cs:45.3816,    -23.888465) --  (axis cs:46.3777,    -23.897067) --  (axis cs:47.3739,    -23.905817) --  (axis cs:48.3702,    -23.914712) --  (axis cs:49.3665,    -23.923750) --  (axis cs:50.3629,    -23.932928) --  (axis cs:51.3593,    -23.942242) --  (axis cs:52.3559,    -23.951691) --  (axis cs:53.3525,    -23.961272) --  (axis cs:54.3492,    -23.970980) --  (axis cs:55.3459,    -23.980815) --  (axis cs:56.3427,    -23.990772) --  (axis cs:57.3396,    -24.000848) --  (axis cs:57.5889,    -24.003386) ;
  \draw[constellation-boundary]  (axis cs:6.59274,    -06.307463) --  (axis cs:6.59398,    -05.307464) --  (axis cs:6.59521,    -04.307464) --  (axis cs:6.59645,    -03.307465) --  (axis cs:6.59768,    -02.307466) --  (axis cs:6.59891,    -01.307467) --  (axis cs:6.60013,    -00.307467) --  (axis cs:6.60136,    +00.692532) --  (axis cs:6.60259,    +01.692531) --  (axis cs:6.60382,    +02.692530) ;
  \draw[constellation-boundary]  (axis cs:99.709,    -43.111647) --  (axis cs:99.7311,    -42.111780) --  (axis cs:99.7525,    -41.111908) --  (axis cs:99.7733,    -40.112033) --  (axis cs:99.7935,    -39.112154) --  (axis cs:99.8131,    -38.112271) --  (axis cs:99.8322,    -37.112386) --  (axis cs:99.8508,    -36.112497) --  (axis cs:99.8689,    -35.112606) --  (axis cs:99.8866,    -34.112711) --  (axis cs:99.9039,    -33.112815) ;
  \draw[constellation-boundary]  (axis cs:55.3529,    +00.403721) --  (axis cs:55.6029,    +00.401243) --  (axis cs:56.6029,    +00.391254) --  (axis cs:57.6029,    +00.381147) --  (axis cs:58.6028,    +00.370923) --  (axis cs:59.6028,    +00.360587) --  (axis cs:60.6028,    +00.350141) --  (axis cs:61.6027,    +00.339588) --  (axis cs:62.6027,    +00.328931) --  (axis cs:63.6027,    +00.318175) --  (axis cs:64.6026,    +00.307321) --  (axis cs:65.6026,    +00.296374) --  (axis cs:66.6025,    +00.285337) --  (axis cs:67.6025,    +00.274212) --  (axis cs:68.6024,    +00.263005) --  (axis cs:69.6024,    +00.251717) --  (axis cs:70.6023,    +00.240352) --  (axis cs:71.6022,    +00.228914) ;
  \draw[constellation-boundary]  (axis cs:70.8523,    +00.237499) --  (axis cs:70.8637,    +01.237434) --  (axis cs:70.8751,    +02.237369) --  (axis cs:70.8866,    +03.237304) --  (axis cs:70.8981,    +04.237238) --  (axis cs:70.9096,    +05.237172) --  (axis cs:70.9211,    +06.237107) --  (axis cs:70.9327,    +07.237040) --  (axis cs:70.9443,    +08.236974) --  (axis cs:70.956,    +09.236907) --  (axis cs:70.9678,    +10.236840) --  (axis cs:70.9796,    +11.236773) --  (axis cs:70.9915,    +12.236705) --  (axis cs:71.0035,    +13.236636) --  (axis cs:71.0156,    +14.236567) --  (axis cs:71.0278,    +15.236497) --  (axis cs:71.034,    +15.736462) ;
  \draw[constellation-boundary]  (axis cs:71.5563,    -03.770822) --  (axis cs:71.5678,    -02.770888) --  (axis cs:71.5793,    -01.770954) --  (axis cs:71.5908,    -00.771020) --  (axis cs:71.6022,    +00.228914) ;
  \draw[constellation-boundary]  (axis cs:71.5563,    -03.770822) --  (axis cs:72.556,    -03.782326) --  (axis cs:73.5556,    -03.793897) --  (axis cs:74.5553,    -03.805531) --  (axis cs:75.555,    -03.817224) --  (axis cs:76.5548,    -03.828972) --  (axis cs:77.5545,    -03.840773) --  (axis cs:77.8044,    -03.843731) ;
  \draw[constellation-boundary]  (axis cs:77.7201,    -10.843231) --  (axis cs:77.7323,    -09.843304) --  (axis cs:77.7445,    -08.843376) --  (axis cs:77.7566,    -07.843447) --  (axis cs:77.7686,    -06.843519) --  (axis cs:77.7806,    -05.843590) --  (axis cs:77.7925,    -04.843660) --  (axis cs:77.8044,    -03.843731) ;
  \draw[constellation-boundary]  (axis cs:75.2217,    -10.813806) --  (axis cs:75.4715,    -10.816734) --  (axis cs:76.4708,    -10.828478) --  (axis cs:77.4702,    -10.840274) --  (axis cs:78.4696,    -10.852120) --  (axis cs:79.4691,    -10.864010) --  (axis cs:80.4686,    -10.875943) --  (axis cs:81.4681,    -10.887913) --  (axis cs:82.4677,    -10.899918) --  (axis cs:83.4673,    -10.911953) --  (axis cs:84.4669,    -10.924015) --  (axis cs:85.4666,    -10.936101) --  (axis cs:86.4663,    -10.948207) --  (axis cs:87.4661,    -10.960328) --  (axis cs:88.4659,    -10.972462) --  (axis cs:89.4658,    -10.984604) --  (axis cs:90.4656,    -10.996752) --  (axis cs:91.4656,    -11.008901) --  (axis cs:92.4656,    -11.021047) --  (axis cs:93.4656,    -11.033187) --  (axis cs:94.4656,    -11.045317) --  (axis cs:95.4657,    -11.057434) --  (axis cs:96.4659,    -11.069533) --  (axis cs:97.466,    -11.081611) --  (axis cs:98.4663,    -11.093665) --  (axis cs:99.4665,    -11.105691) --  (axis cs:100.467,    -11.117684) --  (axis cs:101.467,    -11.129642) --  (axis cs:102.468,    -11.141561) --  (axis cs:103.468,    -11.153437) --  (axis cs:104.468,    -11.165266) --  (axis cs:105.469,    -11.177045) --  (axis cs:106.47,    -11.188771) --  (axis cs:107.47,    -11.200439) --  (axis cs:108.471,    -11.212046) --  (axis cs:109.471,    -11.223589) --  (axis cs:110.472,    -11.235064) --  (axis cs:111.473,    -11.246468) --  (axis cs:112.474,    -11.257796) --  (axis cs:113.475,    -11.269047) --  (axis cs:114.475,    -11.280215) ;
  \draw[constellation-boundary]  (axis cs:75.1787,    -14.313554) --  (axis cs:75.1849,    -13.813591) --  (axis cs:75.1972,    -12.813663) --  (axis cs:75.2095,    -11.813735) --  (axis cs:75.2217,    -10.813806) ;
  \draw[constellation-boundary]  (axis cs:73.9298,    -14.298975) --  (axis cs:74.4293,    -14.304796) --  (axis cs:75.1787,    -14.313554) ;
  \draw[constellation-boundary]  (axis cs:59.1058,    -39.636831) --  (axis cs:60.1005,    -39.647223) --  (axis cs:61.0954,    -39.657722) --  (axis cs:62.0904,    -39.668326) --  (axis cs:63.0856,    -39.679032) --  (axis cs:64.0809,    -39.689835) --  (axis cs:65.0764,    -39.700734) ;
  \draw[constellation-boundary]  (axis cs:59.0313,    -43.636443) --  (axis cs:59.0508,    -42.636545) --  (axis cs:59.0697,    -41.636643) --  (axis cs:59.088,    -40.636738) --  (axis cs:59.1058,    -39.636831) ;
  \draw[constellation-boundary]  (axis cs:52.3267,    -43.569408) --  (axis cs:53.0713,    -43.576578) --  (axis cs:54.0642,    -43.586250) --  (axis cs:55.0573,    -43.596049) --  (axis cs:56.0506,    -43.605971) --  (axis cs:57.044,    -43.616012) --  (axis cs:58.0376,    -43.626171) --  (axis cs:59.0313,    -43.636443) ;
  \draw[constellation-boundary]  (axis cs:52.3082,    -44.569319) --  (axis cs:52.3267,    -43.569408) ;
  \draw[constellation-boundary]  (axis cs:36.1997,    -44.433996) --  (axis cs:36.2134,    -43.434044) --  (axis cs:36.2267,    -42.434092) --  (axis cs:36.2395,    -41.434137) --  (axis cs:36.252,    -40.434181) --  (axis cs:36.2641,    -39.434224) ;
  \draw[constellation-boundary]  (axis cs:46.1872,    -39.512904) --  (axis cs:46.1933,    -39.096264) ;
  \draw[constellation-boundary]  (axis cs:46.1933,    -39.096264) --  (axis cs:47.1864,    -39.104987) --  (axis cs:48.1795,    -39.113854) --  (axis cs:49.1728,    -39.122865) --  (axis cs:50.1663,    -39.132016) --  (axis cs:51.1598,    -39.141303) --  (axis cs:52.1535,    -39.150725) --  (axis cs:53.1473,    -39.160279) --  (axis cs:53.6443,    -39.165104) ;
  \draw[constellation-boundary]  (axis cs:53.6443,    -39.165104) --  (axis cs:53.6536,    -38.581817) --  (axis cs:53.6694,    -37.581893) --  (axis cs:53.6847,    -36.581968) --  (axis cs:53.6996,    -35.582040) ;
  \draw[constellation-boundary]  (axis cs:53.6996,    -35.582040) --  (axis cs:54.1969,    -35.586901) --  (axis cs:55.1917,    -35.596716) --  (axis cs:56.1867,    -35.606654) --  (axis cs:57.1817,    -35.616712) --  (axis cs:57.4305,    -35.619245) ;
  \draw[constellation-boundary]  (axis cs:57.4305,    -35.619245) --  (axis cs:57.4457,    -34.619323) --  (axis cs:57.4606,    -33.619399) --  (axis cs:57.4751,    -32.619473) --  (axis cs:57.4894,    -31.619545) --  (axis cs:57.5033,    -30.619616) --  (axis cs:57.5169,    -29.619686) --  (axis cs:57.5303,    -28.619754) --  (axis cs:57.5434,    -27.619820) --  (axis cs:57.5562,    -26.619886) --  (axis cs:57.5689,    -25.619951) --  (axis cs:57.5813,    -24.620014) --  (axis cs:57.5889,    -24.003386) ;
  \draw[constellation-boundary]  (axis cs:55.3356,    -01.346193) --  (axis cs:55.343,    -00.596230) --  (axis cs:55.3529,    +00.403721) ;
  \draw[constellation-boundary]  (axis cs:26.3507,    -39.372630) --  (axis cs:26.3594,    -38.372653) --  (axis cs:26.3679,    -37.372676) --  (axis cs:26.3761,    -36.372697) --  (axis cs:26.3842,    -35.372719) --  (axis cs:26.392,    -34.372740) --  (axis cs:26.3997,    -33.372760) --  (axis cs:26.4072,    -32.372780) --  (axis cs:26.4145,    -31.372799) --  (axis cs:26.4217,    -30.372818) --  (axis cs:26.4288,    -29.372837) --  (axis cs:26.4357,    -28.372855) --  (axis cs:26.4424,    -27.372873) --  (axis cs:26.4491,    -26.372890) --  (axis cs:26.4556,    -25.372908) --  (axis cs:26.462,    -24.372925) --  (axis cs:26.4659,    -23.756268) ;
  \draw[constellation-boundary]  (axis cs:96.3728,    +11.933297) --  (axis cs:96.7477,    +11.928764) --  (axis cs:97.7473,    +11.916692) --  (axis cs:98.7469,    +11.904646) --  (axis cs:99.7465,    +11.892629) --  (axis cs:100.746,    +11.880645) --  (axis cs:101.745,    +11.868697) --  (axis cs:102.745,    +11.856790) --  (axis cs:103.744,    +11.844927) --  (axis cs:104.744,    +11.833111) --  (axis cs:105.743,    +11.821346) ;
  \draw[constellation-boundary]  (axis cs:95.0695,    +17.449508) --  (axis cs:95.8192,    +17.440427) --  (axis cs:96.4439,    +17.432866) ;
  \draw[constellation-boundary]  (axis cs:95.0695,    +17.449508) --  (axis cs:95.0762,    +17.949468) --  (axis cs:95.0896,    +18.949386) --  (axis cs:95.1033,    +19.949304) --  (axis cs:95.1171,    +20.949220) --  (axis cs:95.1241,    +21.449178) ;
  \draw[constellation-boundary]  (axis cs:90.1252,    +21.509872) --  (axis cs:90.1322,    +22.009829) --  (axis cs:90.1465,    +23.009743) --  (axis cs:90.1609,    +24.009655) --  (axis cs:90.1756,    +25.009566) --  (axis cs:90.1905,    +26.009475) --  (axis cs:90.2057,    +27.009383) --  (axis cs:90.2211,    +28.009289) --  (axis cs:90.2289,    +28.509242) ;
  \draw[constellation-boundary]  (axis cs:90.1252,    +21.509872) --  (axis cs:90.8751,    +21.500761) --  (axis cs:91.875,    +21.488612) --  (axis cs:92.8748,    +21.476468) --  (axis cs:93.8745,    +21.464332) --  (axis cs:94.8742,    +21.452207) --  (axis cs:95.1241,    +21.449178) ;
  \draw[constellation-boundary]  (axis cs:105.718,    +09.821489) --  (axis cs:106.718,    +09.809778) ;
  \draw[constellation-boundary]  (axis cs:105.718,    +09.821489) --  (axis cs:105.731,    +10.821418) --  (axis cs:105.743,    +11.821346) ;
  \draw[constellation-boundary]  (axis cs:-4.347,    -39.306740) --  (axis cs:-3.357,    -39.305754) --  (axis cs:-2.367,    -39.304978) --  (axis cs:-1.377,    -39.304412) --  (axis cs:-0.387,    -39.304055) --  (axis cs:0.602937,    -39.303908) --  (axis cs:1.59292,    -39.303971) --  (axis cs:2.5829,    -39.304244) --  (axis cs:3.57289,    -39.304727) --  (axis cs:4.56289,    -39.305419) --  (axis cs:5.55289,    -39.306321) --  (axis cs:6.54292,    -39.307432) --  (axis cs:7.53296,    -39.308752) --  (axis cs:8.52301,    -39.310280) --  (axis cs:9.51309,    -39.312017) --  (axis cs:10.5032,    -39.313961) --  (axis cs:11.4933,    -39.316112) --  (axis cs:12.4835,    -39.318469) --  (axis cs:13.4737,    -39.321032) --  (axis cs:14.4639,    -39.323799) --  (axis cs:15.4542,    -39.326771) --  (axis cs:16.4445,    -39.329946) --  (axis cs:17.4349,    -39.333323) --  (axis cs:18.4253,    -39.336901) --  (axis cs:19.4158,    -39.340679) --  (axis cs:20.4063,    -39.344656) --  (axis cs:21.3969,    -39.348832) --  (axis cs:22.3875,    -39.353203) --  (axis cs:23.3782,    -39.357771) --  (axis cs:24.3689,    -39.362532) --  (axis cs:25.3598,    -39.367485) --  (axis cs:26.3507,    -39.372630) --  (axis cs:27.3416,    -39.377964) --  (axis cs:28.3327,    -39.383487) --  (axis cs:29.3238,    -39.389195) --  (axis cs:30.315,    -39.395089) --  (axis cs:31.3063,    -39.401165) --  (axis cs:32.2977,    -39.407422) --  (axis cs:33.2891,    -39.413859) --  (axis cs:34.2807,    -39.420473) --  (axis cs:35.2723,    -39.427262) --  (axis cs:36.2641,    -39.434224) --  (axis cs:37.2559,    -39.441358) --  (axis cs:38.2478,    -39.448661) --  (axis cs:39.2399,    -39.456131) --  (axis cs:40.232,    -39.463766) --  (axis cs:41.2243,    -39.471563) --  (axis cs:42.2166,    -39.479521) --  (axis cs:43.2091,    -39.487636) --  (axis cs:44.2017,    -39.495907) --  (axis cs:45.1944,    -39.504330) --  (axis cs:46.1872,    -39.512904) ;
  \draw[constellation-boundary]  (axis cs:88.9658,    -10.978532) --  (axis cs:88.9784,    -09.978609) --  (axis cs:88.9909,    -08.978684) --  (axis cs:89.0033,    -07.978760) --  (axis cs:89.0156,    -06.978835) --  (axis cs:89.0279,    -05.978910) --  (axis cs:89.0402,    -04.978984) --  (axis cs:89.0524,    -03.979058) ;
  \draw[constellation-boundary]  (axis cs:95.348,    +09.945550) --  (axis cs:95.7229,    +09.941010) --  (axis cs:96.3477,    +09.933448) ;
  \draw[constellation-boundary]  (axis cs:96.3477,    +09.933448) --  (axis cs:96.3602,    +10.933373) --  (axis cs:96.3728,    +11.933297) --  (axis cs:96.3855,    +12.933220) --  (axis cs:96.3983,    +13.933143) --  (axis cs:96.4112,    +14.933065) --  (axis cs:96.4242,    +15.932986) --  (axis cs:96.4373,    +16.932906) --  (axis cs:96.4439,    +17.432866) ;
  \draw[constellation-boundary]  (axis cs:89.0524,    -03.979058) --  (axis cs:89.5524,    -03.985130) --  (axis cs:90.5523,    -03.997278) --  (axis cs:91.5522,    -04.009427) --  (axis cs:92.5522,    -04.021573) --  (axis cs:93.5521,    -04.033712) --  (axis cs:94.5521,    -04.045841) --  (axis cs:95.1771,    -04.053415) ;
  \draw[constellation-boundary]  (axis cs:95.1771,    -04.053415) --  (axis cs:95.1892,    -03.053489) --  (axis cs:95.2014,    -02.053562) --  (axis cs:95.2135,    -01.053636) --  (axis cs:95.2256,    -00.053709) --  (axis cs:95.2377,    +00.946218) --  (axis cs:95.2498,    +01.946144) --  (axis cs:95.262,    +02.946071) --  (axis cs:95.2741,    +03.945997) --  (axis cs:95.2863,    +04.945923) --  (axis cs:95.2985,    +05.945849) --  (axis cs:95.3108,    +06.945775) --  (axis cs:95.3231,    +07.945700) --  (axis cs:95.3355,    +08.945625) --  (axis cs:95.348,    +09.945550) ;
  \draw[constellation-boundary]  (axis cs:71.034,    +15.736462) --  (axis cs:71.7848,    +15.727867) --  (axis cs:72.7858,    +15.716347) --  (axis cs:73.7867,    +15.704762) --  (axis cs:74.7876,    +15.693114) --  (axis cs:75.7885,    +15.681408) --  (axis cs:76.2889,    +15.675534) ;
  \draw[constellation-boundary]  (axis cs:76.2889,    +15.675534) --  (axis cs:76.2952,    +16.175497) ;
  \draw[constellation-boundary]  (axis cs:76.2952,    +16.175497) --  (axis cs:76.7956,    +16.169610) --  (axis cs:77.7964,    +16.157797) --  (axis cs:78.797,    +16.145936) --  (axis cs:79.7977,    +16.134031) --  (axis cs:80.7982,    +16.122086) --  (axis cs:81.7987,    +16.110104) ;
  \draw[constellation-boundary]  (axis cs:81.7923,    +15.610143) --  (axis cs:81.7987,    +16.110104) ;
  \draw[constellation-boundary]  (axis cs:81.7923,    +15.610143) --  (axis cs:82.7927,    +15.598128) --  (axis cs:83.7931,    +15.586084) --  (axis cs:84.7934,    +15.574013) --  (axis cs:85.7936,    +15.561920) ;
  \draw[constellation-boundary]  (axis cs:85.755,    +12.562154) --  (axis cs:85.7614,    +13.062115) --  (axis cs:85.7742,    +14.062038) --  (axis cs:85.7871,    +15.061960) --  (axis cs:85.7936,    +15.561920) ;
  \draw[constellation-boundary]  (axis cs:85.755,    +12.562154) --  (axis cs:86.7552,    +12.550043) --  (axis cs:87.7553,    +12.537918) --  (axis cs:88.2553,    +12.531851) ;
  \draw[constellation-boundary]  (axis cs:88.2553,    +12.531851) --  (axis cs:88.2617,    +13.031812) --  (axis cs:88.2745,    +14.031734) --  (axis cs:88.2875,    +15.031656) --  (axis cs:88.3005,    +16.031576) --  (axis cs:88.3137,    +17.031496) --  (axis cs:88.3271,    +18.031415) ;
  \draw[constellation-boundary]  (axis cs:87.327,    +18.043547) --  (axis cs:87.827,    +18.037483) --  (axis cs:88.3271,    +18.031415) ;
  \draw[constellation-boundary]  (axis cs:87.327,    +18.043547) --  (axis cs:87.3405,    +19.043466) --  (axis cs:87.3541,    +20.043383) --  (axis cs:87.3679,    +21.043299) --  (axis cs:87.3819,    +22.043214) --  (axis cs:87.3938,    +22.876476) ;
  \draw[constellation-boundary]  (axis cs:87.3938,    +22.876476) --  (axis cs:87.8939,    +22.870411) --  (axis cs:88.894,    +22.858273) --  (axis cs:89.8941,    +22.846128) --  (axis cs:90.1441,    +22.843091) ;
  \draw[constellation-boundary]  (axis cs:3.73405,    +13.195186) --  (axis cs:3.73438,    +13.695186) --  (axis cs:3.73504,    +14.695186) --  (axis cs:3.7357,    +15.695186) --  (axis cs:3.73638,    +16.695186) --  (axis cs:3.73705,    +17.695186) --  (axis cs:3.73774,    +18.695185) --  (axis cs:3.73844,    +19.695185) --  (axis cs:3.73914,    +20.695185) --  (axis cs:3.73985,    +21.695185) --  (axis cs:3.74058,    +22.695184) ;
  \draw[constellation-boundary]  (axis cs:1.60316,    +13.196028) --  (axis cs:2.60593,    +13.195751) --  (axis cs:3.6087,    +13.195263) --  (axis cs:3.73405,    +13.195186) ;
  \draw[constellation-boundary]  (axis cs:1.60271,    +10.696028) --  (axis cs:1.60289,    +11.696028) --  (axis cs:1.60307,    +12.696028) --  (axis cs:1.60316,    +13.196028) ;
  \draw[constellation-boundary]  (axis cs:-0.903,    +10.695789) --  (axis cs:-0.402,    +10.695943) --  (axis cs:0.600492,    +10.696092) --  (axis cs:1.60271,    +10.696028) ;
  \draw[constellation-boundary]  (axis cs:-0.902,    +08.195789) --  (axis cs:-0.902,    +08.695789) --  (axis cs:-0.902,    +09.695789) --  (axis cs:-0.903,    +10.695789) ;
  \draw[constellation-boundary]  (axis cs:-4.408,    +08.193227) --  (axis cs:-3.406,    +08.194224) --  (axis cs:-2.404,    +08.195009) --  (axis cs:-1.403,    +08.195582) --  (axis cs:-0.902,    +08.195789) ;
  \draw[constellation-boundary]  (axis cs:40.4023,    +34.537508) --  (axis cs:40.9055,    +34.533615) --  (axis cs:41.9118,    +34.525708) --  (axis cs:42.6664,    +34.519672) ;
  \draw[constellation-boundary]  (axis cs:40.4023,    +34.537508) --  (axis cs:40.4138,    +35.537464) --  (axis cs:40.4256,    +36.537418) --  (axis cs:40.4346,    +37.287384) ;
  \draw[constellation-boundary]  (axis cs:22.8665,    +28.645431) --  (axis cs:22.8724,    +29.645417) --  (axis cs:22.8785,    +30.645404) --  (axis cs:22.8847,    +31.645390) --  (axis cs:22.891,    +32.645375) --  (axis cs:22.8975,    +33.645360) --  (axis cs:22.9041,    +34.645345) --  (axis cs:22.9109,    +35.645330) ;
  \draw[constellation-boundary]  (axis cs:22.8665,    +28.645431) --  (axis cs:23.7467,    +28.641362) --  (axis cs:24.7528,    +28.636527) --  (axis cs:25.7587,    +28.631498) --  (axis cs:26.7646,    +28.626275) ;
  



% Labels

% Phoenix too small to label

\node[constellation-label] at (axis cs:{40,23.25})  {\Huge \textls{Aries}};  
\node[constellation-label] at (axis cs:{20,-18})  {\Huge \textls{Cetus}};
\node[constellation-label] at (axis cs:{15,13.3})  {\Huge \textls{Pisces}};
\node[constellation-label] at (axis cs:{80,5.1})   {\Huge \textls{Orion}};
\node[constellation-label] at (axis cs:{58,13.8})  {\Huge \textls{Taurus}};
\node[constellation-label] at (axis cs:{60,-16}) {\Huge \textls{Eridanus}};
\node[constellation-label] at (axis cs:{83,-19}) {\huge \textls{Lepus}};
\node[constellation-label] at (axis cs:{40,-33}) {\LARGE \textls{Fornax}};
\node[constellation-label] at (axis cs:{12,-35}) {\LARGE \textls{Sculptor}};
\node[constellation-label] at (axis cs:{90,35})  {\Large \textls{Auriga}};
\node[constellation-label] at (axis cs:{101.5,26}) {\Large \textls{Gemini}};
\node[constellation-label] at (axis cs:{54,36})  {\Large \textls{Perseus}};
\node[constellation-label] at (axis cs:{102,-6}) {\Large \textls{Monoceros}};
\node[constellation-label] at (axis cs:{85,-36.5}) {\Large \textls{Columba}};
\node[constellation-label] at (axis cs:{10,37})  {\Large \textls{Andromeda}};  
\node[constellation-label] at (axis cs:{103,-19}) {\Large \textls{Major}};
  \node[constellation-label] at (axis cs:{103,-22.5}) {\Large \textls{Canis}};
\node[constellation-label] at (axis cs:{33,32.5})   {\large \textls{Triangulum}};     
\node[constellation-label] at (axis cs:{105,-37}) {\large \textls{Puppis}};
\node[constellation-label,rotate=-70] at (axis cs:{73,-35}) {\large \textls{Caelum}};
\node[constellation-label,rotate=-90] at (axis cs:{1.5,21})  {Pegasus};
\node[constellation-label,rotate=90] at (axis cs:{108.5,7})  {Canis Minor};



\end{axis}


% Nebulae and nebulae names
\begin{axis}[name=NGC,axis lines=none]


\node[NGC,pin={[pin distance=-1.2\onedegree,NGC-label]90:{7814}}] at (axis cs:0.825,16.150) {\tikz\pgfuseplotmark{GAL};}; % 10.5
\node[NGC,pin={[pin distance=-1.2\onedegree,NGC-label]90:{45}}] at (axis cs:3.525,-23.183) {\tikz\pgfuseplotmark{GAL};}; % 10.4
\node[NGC,pin={[pin distance=-1.2\onedegree,NGC-label]90:{55}}] at (axis cs:3.725,-39.183) {\tikz\pgfuseplotmark{GAL};}; %  8. 
\node[NGC,pin={[pin distance=-1.2\onedegree,NGC-label]90:{134}}] at (axis cs:7.600,-33.250) {\tikz\pgfuseplotmark{GAL};}; % 10.1
\node[NGC,pin={[pin distance=-1.2\onedegree,NGC-label]90:{157}}] at (axis cs:8.700,-8.400) {\tikz\pgfuseplotmark{GAL};}; % 10.4
\node[NGC,pin={[pin distance=-1.2\onedegree,NGC-label]90:{210}}] at (axis cs:10.150,-13.867) {\tikz\pgfuseplotmark{GAL};}; % 10.9
\node[NGC,pin={[pin distance=-1.2\onedegree,NGC-label]90:{247}}] at (axis cs:11.775,-20.767) {\tikz\pgfuseplotmark{GAL};}; %  8.9
\node[NGC,pin={[pin distance=-1.2\onedegree,NGC-label]-90:{253}}] at (axis cs:11.900,-25.283) {\tikz\pgfuseplotmark{GAL};}; %  7.1
\node[NGC,pin={[pin distance=-1.2\onedegree,NGC-label]90:{300}}] at (axis cs:13.725,-37.683) {\tikz\pgfuseplotmark{GAL};}; %  9. 
\node[NGC,pin={[pin distance=-1.2\onedegree,NGC-label]180:{I1613}}] at (axis cs:16.200,2.117) {\tikz\pgfuseplotmark{GAL};}; %  9.3
\node[NGC,pin={[pin distance=-1.2\onedegree,NGC-label]-90:{404}}] at (axis cs:17.350,35.717) {\tikz\pgfuseplotmark{GAL};}; % 10.1
\node[NGC,pin={[pin distance=-1.2\onedegree,NGC-label]90:{488}}] at (axis cs:20.450,5.250) {\tikz\pgfuseplotmark{GAL};}; % 10.3
\node[NGC,pin={[pin distance=-1.2\onedegree,NGC-label]90:{524}}] at (axis cs:21.200,9.533) {\tikz\pgfuseplotmark{GAL};}; % 10.6
\node[NGC,pin={[pin distance=-1.2\onedegree,NGC-label]90:{578}}] at (axis cs:22.625,-22.667) {\tikz\pgfuseplotmark{GAL};}; % 10.9
\node[NGC,pin={[pin distance=-1.2\onedegree,NGC-label]-90:{584}}] at (axis cs:22.825,-6.867) {\tikz\pgfuseplotmark{GAL};}; % 10.4
\node[NGC,pin={[pin distance=-1.2\onedegree,NGC-label]0:{596}}] at (axis cs:23.225,-7.033) {\tikz\pgfuseplotmark{GAL};}; % 10.9
\node[NGC,pin={[pin distance=-1.2\onedegree,NGC-label]-90:{613}}] at (axis cs:23.575,-29.417) {\tikz\pgfuseplotmark{GAL};}; % 10.0
\node[NGC,pin={[pin distance=-1.2\onedegree,NGC-label]90:{660}}] at (axis cs:25.750,13.633) {\tikz\pgfuseplotmark{GAL};}; % 10.8
\node[NGC,pin={[pin distance=-1.2\onedegree,NGC-label]90:{672}}] at (axis cs:26.975,27.433) {\tikz\pgfuseplotmark{GAL};}; % 10.8
\node[NGC,pin={[pin distance=-1.2\onedegree,NGC-label]90:{720}}] at (axis cs:28.250,-13.733) {\tikz\pgfuseplotmark{GAL};}; % 10.2
\node[NGC,pin={[pin distance=-1.2\onedegree,NGC-label]90:{772}}] at (axis cs:29.825,19.017) {\tikz\pgfuseplotmark{GAL};}; % 10.3
\node[NGC,pin={[pin distance=-1.2\onedegree,NGC-label]90:{821}}] at (axis cs:32.100,11) {\tikz\pgfuseplotmark{GAL};}; % 10.8
\node[NGC,pin={[pin distance=-1.2\onedegree,NGC-label]90:{908}}] at (axis cs:35.775,-21.233) {\tikz\pgfuseplotmark{GAL};}; % 10.2
\node[NGC,pin={[pin distance=-1.2\onedegree,NGC-label]-90:{925}}] at (axis cs:36.825,33.583) {\tikz\pgfuseplotmark{GAL};}; % 10.0
\node[NGC,pin={[pin distance=-1.2\onedegree,NGC-label]90:{936}}] at (axis cs:36.900,-1.150) {\tikz\pgfuseplotmark{GAL};}; % 10.1
\node[NGC,pin={[pin distance=-1.2\onedegree,NGC-label]90:{1023}}] at (axis cs:40.100,39.067) {\tikz\pgfuseplotmark{GAL};}; %  9.5
\node[NGC,pin={[pin distance=-1.2\onedegree,NGC-label]90:{1042}}] at (axis cs:40.100,-8.433) {\tikz\pgfuseplotmark{GAL};}; % 10.9
\node[NGC,pin={[pin distance=-1.2\onedegree,NGC-label]-90:{1052}}] at (axis cs:40.275,-8.250) {\tikz\pgfuseplotmark{GAL};}; % 10.6
\node[NGC,pin={[pin distance=-1.2\onedegree,NGC-label]-90:{1055}}] at (axis cs:40.450,0.433) {\tikz\pgfuseplotmark{GAL};}; % 10.6
\node[NGC,pin={[pin distance=-1.2\onedegree,NGC-label]90:{1084}}] at (axis cs:41.500,-7.583) {\tikz\pgfuseplotmark{GAL};}; % 10.6
\node[NGC,pin={[pin distance=-1.2\onedegree,NGC-label]90:{1097}}] at (axis cs:41.575,-30.283) {\tikz\pgfuseplotmark{GAL};}; %  9.3
\node[NGC,pin={[pin distance=-1.2\onedegree,NGC-label]90:{1201}}] at (axis cs:46.025,-26.067) {\tikz\pgfuseplotmark{GAL};}; % 10.6
\node[NGC,pin={[pin distance=-1.2\onedegree,NGC-label]90:{1232}}] at (axis cs:47.450,-20.583) {\tikz\pgfuseplotmark{GAL};}; %  9.9
\node[NGC,pin={[pin distance=-1.2\onedegree,NGC-label]90:{1300}}] at (axis cs:49.925,-19.417) {\tikz\pgfuseplotmark{GAL};}; % 10.4
\node[NGC,pin={[pin distance=-1.2\onedegree,NGC-label]90:{1316}}] at (axis cs:50.675,-37.200) {\tikz\pgfuseplotmark{GAL};}; %  8.9
\node[NGC,pin={[pin distance=-1.2\onedegree,NGC-label]180:{1326}}] at (axis cs:50.975,-36.467) {\tikz\pgfuseplotmark{GAL};}; % 10.5
\node[NGC,pin={[pin distance=-1.2\onedegree,NGC-label]90:{1332}}] at (axis cs:51.575,-21.333) {\tikz\pgfuseplotmark{GAL};}; % 10.3
\node[NGC,pin={[pin distance=-1.2\onedegree,NGC-label]90:{1344}}] at (axis cs:52.075,-31.067) {\tikz\pgfuseplotmark{GAL};}; % 10.3
\node[NGC,pin={[pin distance=-1.2\onedegree,NGC-label]90:{1350}}] at (axis cs:52.775,-33.633) {\tikz\pgfuseplotmark{GAL};}; % 10.5
\node[NGC,pin={[pin distance=-1.2\onedegree,NGC-label]90:{1365}}] at (axis cs:53.400,-36.133) {\tikz\pgfuseplotmark{GAL};}; %  9.5
\node[NGC,pin={[pin distance=-1.2\onedegree,NGC-label]-90:{1399}}] at (axis cs:54.625,-35.450) {\tikz\pgfuseplotmark{GAL};}; %  9.9
\node[NGC,pin={[pin distance=-1.2\onedegree,NGC-label]90:{1398}}] at (axis cs:54.725,-26.333) {\tikz\pgfuseplotmark{GAL};}; %  9.7
\node[NGC,pin={[pin distance=-1.2\onedegree,NGC-label]90:{1404}}] at (axis cs:54.725,-35.583) {\tikz\pgfuseplotmark{GAL};}; % 10.3
\node[NGC,pin={[pin distance=-1.2\onedegree,NGC-label]90:{1407}}] at (axis cs:55.050,-18.583) {\tikz\pgfuseplotmark{GAL};}; %  9.8
\node[NGC,pin={[pin distance=-1.2\onedegree,NGC-label]90:{1637}}] at (axis cs:70.375,-2.850) {\tikz\pgfuseplotmark{GAL};}; % 10.9
\node[NGC,pin={[pin distance=-1.2\onedegree,NGC-label]180:{1792}}] at (axis cs:76.300,-37.983) {\tikz\pgfuseplotmark{GAL};}; % 10.2
\node[NGC,pin={[pin distance=-1.2\onedegree,NGC-label]-90:{1808}}] at (axis cs:76.925,-37.517) {\tikz\pgfuseplotmark{GAL};}; %  9.9
\node[NGC,pin={[pin distance=-1.2\onedegree,NGC-label]90:{1964}}] at (axis cs:83.350,-21.950) {\tikz\pgfuseplotmark{GAL};}; % 10.8
\node[NGC,pin={[pin distance=-1.2\onedegree,NGC-label]90:{2207}}] at (axis cs:94.100,-21.367) {\tikz\pgfuseplotmark{GAL};}; % 10.7
\node[NGC,pin={[pin distance=-1.2\onedegree,NGC-label]90:{2217}}] at (axis cs:95.425,-27.233) {\tikz\pgfuseplotmark{GAL};}; % 10.4

\node[NGC,pin={[pin distance=-1.2\onedegree,NGC-label]90:{246}}] at (axis cs:11.750,-11.883) {\tikz\pgfuseplotmark{PN};}; %  8. 
\node[NGC,pin={[pin distance=-1.2\onedegree,NGC-label]90:{1514}}] at (axis cs:62.300,30.783) {\tikz\pgfuseplotmark{PN};}; % 10. 
\node[NGC,pin={[pin distance=-1.2\onedegree,NGC-label]90:{1535}}] at (axis cs:63.550,-12.733) {\tikz\pgfuseplotmark{PN};}; % 10. 
\node[NGC,pin={[pin distance=-1.2\onedegree,NGC-label]90:{I348}}] at (axis cs:56.125,32.283) {\tikz\pgfuseplotmark{CN};}; %  7.3
\node[NGC,pin={[pin distance=-1.2\onedegree,NGC-label]90:{2264}}] at (axis cs:100.275,9.883) {\tikz\pgfuseplotmark{CN};}; %  3.9
%\node[NGC,pin={[pin distance=-1.2\onedegree,NGC-label]90:{2362}}] at (axis cs:109.700,-24.950) {\tikz\pgfuseplotmark{CN};}; %  4.1
\node[NGC,pin={[pin distance=-1.2\onedegree,NGC-label]-90:{752}}] at (axis cs:29.450,37.683) {\tikz\pgfuseplotmark{OC};}; %  5.7
\node[NGC,pin={[pin distance=-1.2\onedegree,NGC-label]90:{1342}}] at (axis cs:52.900,37.333) {\tikz\pgfuseplotmark{OC};}; %  6.7
\node[NGC,pin={[pin distance=-1.2\onedegree,NGC-label]90:{1647}}] at (axis cs:71.500,19.067) {\tikz\pgfuseplotmark{OC};}; %  6.4
\node[NGC,pin={[pin distance=-1.2\onedegree,NGC-label]90:{1662}}] at (axis cs:72.125,10.933) {\tikz\pgfuseplotmark{OC};}; %  6.4
\node[NGC,pin={[pin distance=-1.2\onedegree,NGC-label]90:{1746}}] at (axis cs:75.900,23.817) {\tikz\pgfuseplotmark{OC};}; %  6. 
\node[NGC,pin={[pin distance=-1.2\onedegree,NGC-label]90:{1778}}] at (axis cs:77.025,37.050) {\tikz\pgfuseplotmark{OC};}; %  7.7
\node[NGC,pin={[pin distance=-1.2\onedegree,NGC-label]0:{1807}}] at (axis cs:77.675,16.533) {\tikz\pgfuseplotmark{OC};}; %  7.0
\node[NGC,pin={[pin distance=-1.2\onedegree,NGC-label]180:{1817}}] at (axis cs:78.025,16.700) {\tikz\pgfuseplotmark{OC};}; %  7.7
\node[NGC,pin={[pin distance=-1.2\onedegree,NGC-label]90:{1857}}] at (axis cs:80.050,39.350) {\tikz\pgfuseplotmark{OC};}; %  7.0
\node[NGC,pin={[pin distance=-1.2\onedegree,NGC-label]-90:{1893}}] at (axis cs:80.675,33.400) {\tikz\pgfuseplotmark{OC};}; %  7.5
\node[NGC,pin={[pin distance=-1.2\onedegree,NGC-label]0:{1907}}] at (axis cs:82,35.317) {\tikz\pgfuseplotmark{OC};}; %  8.2
\node[NGC,pin={[pin distance=-1.2\onedegree,NGC-label]-90:{1981}}] at (axis cs:83.800,-4.433) {\tikz\pgfuseplotmark{OC};}; %  4.6
\node[NGC,pin={[pin distance=-1.2\onedegree,NGC-label]-90:{2112}}] at (axis cs:88.475,0.400) {\tikz\pgfuseplotmark{OC};}; %  9. 
\node[NGC,pin={[pin distance=-1.2\onedegree,NGC-label]90:{2129}}] at (axis cs:90.250,23.300) {\tikz\pgfuseplotmark{OC};}; %  6.7
\node[NGC,pin={[pin distance=-1.2\onedegree,NGC-label]-90:{2141}}] at (axis cs:90.775,10.433) {\tikz\pgfuseplotmark{OC};}; %  9.4
%\node[NGC,pin={[pin distance=-1.2\onedegree,NGC-label]0:{I2157}}] at (axis cs:91.250,24) {\tikz\pgfuseplotmark{OC};}; %  8.4
\node[NGC,pin={[pin distance=-1.2\onedegree,NGC-label]180:{2158}}] at (axis cs:91.875,24.100) {\tikz\pgfuseplotmark{OC};}; %  8.6
\node[NGC,pin={[pin distance=-1.2\onedegree,NGC-label]180:{2169}}] at (axis cs:92.100,13.950) {\tikz\pgfuseplotmark{OC};}; %  5.9
\node[NGC,pin={[pin distance=-1.2\onedegree,NGC-label]-90:{2175}}] at (axis cs:92.450,20.317) {\tikz\pgfuseplotmark{OC};}; %  6.8
\node[NGC,pin={[pin distance=-1.2\onedegree,NGC-label]90:{2186}}] at (axis cs:93.050,5.450) {\tikz\pgfuseplotmark{OC};}; %  8.7
\node[NGC,pin={[pin distance=-1.2\onedegree,NGC-label]180:{2194}}] at (axis cs:93.450,12.800) {\tikz\pgfuseplotmark{OC};}; %  8.5
\node[NGC,pin={[pin distance=-1.2\onedegree,NGC-label]-90:{2204}}] at (axis cs:93.925,-18.650) {\tikz\pgfuseplotmark{OC};}; %  8.6
\node[NGC,pin={[pin distance=-1.2\onedegree,NGC-label]-90:{2215}}] at (axis cs:95.250,-7.283) {\tikz\pgfuseplotmark{OC};}; %  8.4
\node[NGC,pin={[pin distance=-1.2\onedegree,NGC-label]90:{2232}}] at (axis cs:96.650,-4.750) {\tikz\pgfuseplotmark{OC};}; %  3.9
\node[NGC,pin={[pin distance=-1.2\onedegree,NGC-label]-90:{2236}}] at (axis cs:97.425,6.833) {\tikz\pgfuseplotmark{OC};}; %  8.5
\node[NGC,pin={[pin distance=-1.2\onedegree,NGC-label]90:{2243}}] at (axis cs:97.450,-31.283) {\tikz\pgfuseplotmark{OC};}; %  9.4
\node[NGC,pin={[pin distance=-1.2\onedegree,NGC-label]90:{2244}}] at (axis cs:98.100,4.867) {\tikz\pgfuseplotmark{OC};}; %  4.8
\node[NGC,pin={[pin distance=-1.2\onedegree,NGC-label]-90:{2250}}] at (axis cs:98.200,-5.033) {\tikz\pgfuseplotmark{OC};}; %  9. 
\node[NGC,pin={[pin distance=-1.2\onedegree,NGC-label]180:{2251}}] at (axis cs:98.675,8.367) {\tikz\pgfuseplotmark{OC};}; %  7.3
\node[NGC,pin={[pin distance=-1.2\onedegree,NGC-label]-90:{2252}}] at (axis cs:98.750,5.383) {\tikz\pgfuseplotmark{OC};}; %  8. 
\node[NGC,pin={[pin distance=-1.2\onedegree,NGC-label]0:{2254}}] at (axis cs:99,7.667) {\tikz\pgfuseplotmark{OC};}; %  9.7
\node[NGC,pin={[pin distance=-1.2\onedegree,NGC-label]-90:{2266}}] at (axis cs:100.800,26.967) {\tikz\pgfuseplotmark{OC};}; % 10. 
\node[NGC,pin={[pin distance=-1.2\onedegree,NGC-label]180:{2269}}] at (axis cs:100.975,4.567) {\tikz\pgfuseplotmark{OC};}; % 10.0
\node[NGC,pin={[pin distance=-1.2\onedegree,NGC-label]90:{2286}}] at (axis cs:101.900,-3.167) {\tikz\pgfuseplotmark{OC};}; %  7.5
\node[NGC,pin={[pin distance=-1.2\onedegree,NGC-label]180:{2301}}] at (axis cs:102.950,0.467) {\tikz\pgfuseplotmark{OC};}; %  6.0
\node[NGC,pin={[pin distance=-1.2\onedegree,NGC-label]-90:{2302}}] at (axis cs:102.975,-7.067) {\tikz\pgfuseplotmark{OC};}; %  8.9
\node[NGC,pin={[pin distance=-1.2\onedegree,NGC-label]90:{2304}}] at (axis cs:103.750,18.017) {\tikz\pgfuseplotmark{OC};}; % 10. 
\node[NGC,pin={[pin distance=-1.2\onedegree,NGC-label]180:{2311}}] at (axis cs:104.450,-4.583) {\tikz\pgfuseplotmark{OC};}; % 10. 
\node[NGC,pin={[pin distance=-1.2\onedegree,NGC-label]90:{2324}}] at (axis cs:106.050,1.050) {\tikz\pgfuseplotmark{OC};}; %  8.4
\node[NGC,pin={[pin distance=-1.2\onedegree,NGC-label]180:{2335}}] at (axis cs:106.650,-10.083) {\tikz\pgfuseplotmark{OC};}; %  7.2
\node[NGC,pin={[pin distance=-1.2\onedegree,NGC-label]90:{2331}}] at (axis cs:106.800,27.350) {\tikz\pgfuseplotmark{OC};}; %  9. 
\node[NGC,pin={[pin distance=-1.2\onedegree,NGC-label]180:{2343}}] at (axis cs:107.075,-10.650) {\tikz\pgfuseplotmark{OC};}; %  6.7
\node[NGC,pin={[pin distance=-1.2\onedegree,NGC-label]0:{2345}}] at (axis cs:107.075,-13.167) {\tikz\pgfuseplotmark{OC};}; %  7.7
\node[NGC,pin={[pin distance=-1.2\onedegree,NGC-label]-90:{2354}}] at (axis cs:108.575,-25.733) {\tikz\pgfuseplotmark{OC};}; %  6.5
\node[NGC,pin={[pin distance=-0.8\onedegree,NGC-label]90:{2353}}] at (axis cs:108.650,-10.300) {\tikz\pgfuseplotmark{OC};}; %  7.1
\node[NGC,pin={[pin distance=-1.2\onedegree,NGC-label]90:{2355}}] at (axis cs:109.225,13.783) {\tikz\pgfuseplotmark{OC};}; % 10. 
\node[NGC,pin={[pin distance=-1.2\onedegree,NGC-label]90:{2360}}] at (axis cs:109.450,-15.617) {\tikz\pgfuseplotmark{OC};}; %  7.2
\node[NGC,pin={[pin distance=-1.2\onedegree,NGC-label]90:{288}}] at (axis cs:13.200,-26.583) {\tikz\pgfuseplotmark{GC};}; %  8.1
\node[NGC,pin={[pin distance=-1.2\onedegree,NGC-label]90:{2298}}] at (axis cs:102.250,-36) {\tikz\pgfuseplotmark{GC};}; %  9.4

\node[Messier,pin={[pin distance=-0.8\onedegree,Messier-label]-90:{M38}}] at (axis cs:82.175,35.833) {\tikz\pgfuseplotmark{OC};}; %  6.4
\node[Messier,pin={[pin distance=-0.8\onedegree,Messier-label]0:{M36}}] at (axis cs:84.025,34.133) {\tikz\pgfuseplotmark{OC};}; %  6.0
\node[Messier,pin={[pin distance=-1.2\onedegree,Messier-label]-90:{M37}}] at (axis cs:88.100,32.550) {\tikz\pgfuseplotmark{OC};}; %  5.6
\node[Messier,pin={[pin distance=-0.8\onedegree,Messier-label]90:{M33}}] at (axis cs:23.475,30.650) {\tikz\pgfuseplotmark{GAL};}; %  5.7
\node[Messier,pin={[pin distance=-0.8\onedegree,Messier-label]-90:{M35}}] at (axis cs:92.225,24.333) {\tikz\pgfuseplotmark{OC};}; %  5.1
\node[Messier,pin={[pin distance=-1.2\onedegree,Messier-label]-90:{M1}}] at (axis cs:83.625,22.017) {\tikz\pgfuseplotmark{EN};}; %  8.4
\node[Messier,pin={[pin distance=-0.8\onedegree,Messier-label]-90:{M74}}] at (axis cs:24.175,15.783) {\tikz\pgfuseplotmark{GAL};}; %  9.2
\node[Messier,pin={[pin distance=-1.2\onedegree,Messier-label]-45:{M77}}] at (axis cs:40.675,-0.017) {\tikz\pgfuseplotmark{GAL};}; %  8.8
\node[Messier,pin={[pin distance=-1.2\onedegree,Messier-label]270:{M78}}] at (axis cs:86.675,0.050) {\tikz\pgfuseplotmark{EN};}; %  8. 
\node[Messier,pin={[pin distance=-0.8\onedegree,Messier-label]180:{M42/3}}] at (axis cs:83.850,-5.450) {\tikz\pgfuseplotmark{EN};}; %  4. 
%\node[Messier,pin={[pin distance=-0.8\onedegree,Messier-label]90:{M43}}] at (axis cs:83.900,-5.267) {\tikz\pgfuseplotmark{EN};}; %  9. 
\node[Messier,pin={[pin distance=-0.8\onedegree,Messier-label]90:{M50}}] at (axis cs:105.800,-8.333) {\tikz\pgfuseplotmark{OC};}; %  5.9
\node[Messier,pin={[pin distance=-1.2\onedegree,Messier-label]90:{M41}}] at (axis cs:101.750,-20.733) {\tikz\pgfuseplotmark{OC};}; %  4.5
\node[Messier,pin={[pin distance=-0.8\onedegree,Messier-label]90:{M79}}] at (axis cs:81.125,-24.550) {\tikz\pgfuseplotmark{GC};}; %  8.0

\node[Messier,pin={[pin distance=0\onedegree,Messier-label]135:{M45}}] at (axis cs:57,24) {\tikz\draw[dashed]  circle (1.5\onedegree) ;}; %  8.0
\node[Messier,pin={[pin distance=\onedegree,Messier-label]-90:{Pleiades}}] at (axis cs:55,24.5) {}; %  8.0


\node[Messier] at (axis cs:66.5,17) {\tikz\draw[dashed]  circle (3\onedegree) ;}; %  8.0
\node[Messier-label] at (axis cs:61.5,16.5) {\normalsize Hyades}; %  8.0

\node[interest] at (axis cs:53.57,-35.666) {\tikz\draw[dashed]  circle (3.5\onedegree) ;}; 
\node[interest-label] at (axis cs:50,-32) {Fornax cluster}; 
\node[interest,pin={[pin distance=-1\onedegree,interest-label]0:{Fornax Dwarf}}] at (axis cs:40,-34.25) {\tikz\pgfuseplotmark{GAL};}; %  8.0
\node[interest,pin={[pin distance=-1\onedegree,interest-label]0:{Sculptor Dwarf}}] at (axis cs:15,-33.7) {\tikz\pgfuseplotmark{GAL};}; %  8.0
\end{axis}

% Stars and star names
\begin{axis}[name=stars,axis lines=none]

\node[stars] at (axis cs:{0.100},{26.918}) {\tikz\pgfuseplotmark{m6c};};
\node[stars] at (axis cs:{0.456},{-3.027}) {\tikz\pgfuseplotmark{m5b};};
\node[stars] at (axis cs:{0.490},{-6.014}) {\tikz\pgfuseplotmark{m4cv};};
\node[stars] at (axis cs:{0.542},{27.082}) {\tikz\pgfuseplotmark{m6av};};
\node[stars] at (axis cs:{0.583},{-29.720}) {\tikz\pgfuseplotmark{m5b};};
\node[stars] at (axis cs:{0.601},{8.957}) {\tikz\pgfuseplotmark{m6c};};
\node[stars] at (axis cs:{0.624},{8.485}) {\tikz\pgfuseplotmark{m6a};};
\node[stars] at (axis cs:{0.740},{-20.046}) {\tikz\pgfuseplotmark{m6c};};
\node[stars] at (axis cs:{0.783},{-24.145}) {\tikz\pgfuseplotmark{m6c};};
\node[stars] at (axis cs:{0.935},{-17.336}) {\tikz\pgfuseplotmark{m5a};};
\node[stars] at (axis cs:{1.082},{-16.529}) {\tikz\pgfuseplotmark{m6a};};
\node[stars] at (axis cs:{1.085},{-29.269}) {\tikz\pgfuseplotmark{m6c};};
\node[stars] at (axis cs:{1.125},{-10.509}) {\tikz\pgfuseplotmark{m5bv};};
\node[stars] at (axis cs:{1.224},{34.660}) {\tikz\pgfuseplotmark{m6b};};
\node[stars] at (axis cs:{1.233},{26.649}) {\tikz\pgfuseplotmark{m6c};};
\node[stars] at (axis cs:{1.255},{27.675}) {\tikz\pgfuseplotmark{m6c};};
\node[stars] at (axis cs:{1.266},{-0.503}) {\tikz\pgfuseplotmark{m6c};};
\node[stars] at (axis cs:{1.334},{-5.707}) {\tikz\pgfuseplotmark{m5av};};
\node[stars] at (axis cs:{1.425},{13.396}) {\tikz\pgfuseplotmark{m6a};};
\node[stars] at (axis cs:{1.653},{29.021}) {\tikz\pgfuseplotmark{m6bvb};};
\node[stars] at (axis cs:{1.709},{-23.107}) {\tikz\pgfuseplotmark{m6c};};
\node[stars] at (axis cs:{1.826},{-17.387}) {\tikz\pgfuseplotmark{m6c};};
\node[stars] at (axis cs:{1.934},{-2.548}) {\tikz\pgfuseplotmark{m6cv};};
\node[stars] at (axis cs:{1.945},{-22.508}) {\tikz\pgfuseplotmark{m6b};};
\node[stars] at (axis cs:{2.015},{-33.529}) {\tikz\pgfuseplotmark{m6a};};
\node[stars] at (axis cs:{2.050},{-2.447}) {\tikz\pgfuseplotmark{m6cv};};
\node[stars] at (axis cs:{2.073},{-8.824}) {\tikz\pgfuseplotmark{m6b};};
\node[stars] at (axis cs:{2.097},{29.090}) {\tikz\pgfuseplotmark{m2bv};};
\node[stars] at (axis cs:{2.140},{-17.578}) {\tikz\pgfuseplotmark{m6bv};};
\node[stars] at (axis cs:{2.171},{36.627}) {\tikz\pgfuseplotmark{m6c};};
\node[stars] at (axis cs:{2.217},{25.463}) {\tikz\pgfuseplotmark{m6c};};
\node[stars] at (axis cs:{2.251},{28.247}) {\tikz\pgfuseplotmark{m6c};};
\node[stars] at (axis cs:{2.260},{18.212}) {\tikz\pgfuseplotmark{m6a};};
\node[stars] at (axis cs:{2.338},{-27.988}) {\tikz\pgfuseplotmark{m5cb};};
\node[stars] at (axis cs:{2.509},{11.146}) {\tikz\pgfuseplotmark{m6avb};};
\node[stars] at (axis cs:{2.579},{-5.248}) {\tikz\pgfuseplotmark{m6a};};
\node[stars] at (axis cs:{2.678},{-12.580}) {\tikz\pgfuseplotmark{m6b};};
\node[stars] at (axis cs:{2.816},{-15.468}) {\tikz\pgfuseplotmark{m5b};};
\node[stars] at (axis cs:{2.893},{-27.799}) {\tikz\pgfuseplotmark{m5c};};
\node[stars] at (axis cs:{2.933},{-35.133}) {\tikz\pgfuseplotmark{m5c};};
\node[stars] at (axis cs:{3.042},{-17.938}) {\tikz\pgfuseplotmark{m5c};};
\node[stars] at (axis cs:{3.309},{15.184}) {\tikz\pgfuseplotmark{m3avb};};
\node[stars] at (axis cs:{3.350},{26.987}) {\tikz\pgfuseplotmark{m6c};};
\node[stars] at (axis cs:{3.426},{-26.022}) {\tikz\pgfuseplotmark{m6bv};};
\node[stars] at (axis cs:{3.435},{-26.285}) {\tikz\pgfuseplotmark{m6b};};
\node[stars] at (axis cs:{3.510},{33.206}) {\tikz\pgfuseplotmark{m6c};};
\node[stars] at (axis cs:{3.615},{-7.780}) {\tikz\pgfuseplotmark{m5cv};};
\node[stars] at (axis cs:{3.651},{20.207}) {\tikz\pgfuseplotmark{m5av};};
\node[stars] at (axis cs:{3.660},{-18.933}) {\tikz\pgfuseplotmark{m4cv};};
\node[stars] at (axis cs:{3.727},{-9.570}) {\tikz\pgfuseplotmark{m6a};};
\node[stars] at (axis cs:{3.733},{22.284}) {\tikz\pgfuseplotmark{m6cv};};
\node[stars] at (axis cs:{3.743},{-34.904}) {\tikz\pgfuseplotmark{m6c};};
\node[stars] at (axis cs:{3.745},{8.821}) {\tikz\pgfuseplotmark{m6bvb};};
\node[stars] at (axis cs:{3.779},{31.536}) {\tikz\pgfuseplotmark{m6c};};
\node[stars] at (axis cs:{3.794},{27.283}) {\tikz\pgfuseplotmark{m6c};};
\node[stars] at (axis cs:{4.037},{-31.446}) {\tikz\pgfuseplotmark{m6a};};
\node[stars] at (axis cs:{4.142},{8.240}) {\tikz\pgfuseplotmark{m6b};};
\node[stars] at (axis cs:{4.273},{38.681}) {\tikz\pgfuseplotmark{m5av};};
\node[stars] at (axis cs:{4.386},{-19.051}) {\tikz\pgfuseplotmark{m6c};};
\node[stars] at (axis cs:{4.449},{1.689}) {\tikz\pgfuseplotmark{m6c};};
\node[stars] at (axis cs:{4.572},{11.206}) {\tikz\pgfuseplotmark{m6b};};
\node[stars] at (axis cs:{4.582},{36.785}) {\tikz\pgfuseplotmark{m5av};};
\node[stars] at (axis cs:{4.659},{31.517}) {\tikz\pgfuseplotmark{m6b};};
\node[stars] at (axis cs:{4.674},{-8.053}) {\tikz\pgfuseplotmark{m6c};};
\node[stars] at (axis cs:{4.857},{-8.824}) {\tikz\pgfuseplotmark{m4av};};
\node[stars] at (axis cs:{5.102},{30.936}) {\tikz\pgfuseplotmark{m6b};};
\node[stars] at (axis cs:{5.149},{8.190}) {\tikz\pgfuseplotmark{m5c};};
\node[stars] at (axis cs:{5.190},{32.911}) {\tikz\pgfuseplotmark{m6a};};
\node[stars] at (axis cs:{5.280},{37.969}) {\tikz\pgfuseplotmark{m5c};};
\node[stars] at (axis cs:{5.380},{-28.981}) {\tikz\pgfuseplotmark{m5c};};
\node[stars] at (axis cs:{5.443},{-20.058}) {\tikz\pgfuseplotmark{m6av};};
\node[stars] at (axis cs:{5.606},{13.482}) {\tikz\pgfuseplotmark{m6c};};
\node[stars] at (axis cs:{5.716},{-12.209}) {\tikz\pgfuseplotmark{m6cv};};
\node[stars] at (axis cs:{5.768},{-15.942}) {\tikz\pgfuseplotmark{m6c};};
\node[stars] at (axis cs:{6.124},{-2.219}) {\tikz\pgfuseplotmark{m6b};};
\node[stars] at (axis cs:{6.159},{14.315}) {\tikz\pgfuseplotmark{m6c};};
\node[stars] at (axis cs:{6.351},{1.940}) {\tikz\pgfuseplotmark{m6av};};
\node[stars] at (axis cs:{6.591},{-18.693}) {\tikz\pgfuseplotmark{m6cv};};
\node[stars] at (axis cs:{6.656},{-0.049}) {\tikz\pgfuseplotmark{m6c};};
\node[stars] at (axis cs:{6.811},{-25.547}) {\tikz\pgfuseplotmark{m6b};};
\node[stars] at (axis cs:{6.879},{16.025}) {\tikz\pgfuseplotmark{m6cb};};
\node[stars] at (axis cs:{6.982},{-33.007}) {\tikz\pgfuseplotmark{m5bv};};
\node[stars] at (axis cs:{6.994},{19.514}) {\tikz\pgfuseplotmark{m6c};};
\node[stars] at (axis cs:{7.012},{17.893}) {\tikz\pgfuseplotmark{m5bv};};
\node[stars] at (axis cs:{7.053},{16.445}) {\tikz\pgfuseplotmark{m6b};};
\node[stars] at (axis cs:{7.084},{10.190}) {\tikz\pgfuseplotmark{m6b};};
\node[stars] at (axis cs:{7.088},{-20.335}) {\tikz\pgfuseplotmark{m6c};};
\node[stars] at (axis cs:{7.111},{-39.915}) {\tikz\pgfuseplotmark{m5c};};
\node[stars] at (axis cs:{7.236},{36.899}) {\tikz\pgfuseplotmark{m6c};};
\node[stars] at (axis cs:{7.466},{-14.864}) {\tikz\pgfuseplotmark{m6c};};
\node[stars] at (axis cs:{7.510},{-3.957}) {\tikz\pgfuseplotmark{m6a};};
\node[stars] at (axis cs:{7.531},{29.752}) {\tikz\pgfuseplotmark{m5cv};};
\node[stars] at (axis cs:{7.594},{-23.788}) {\tikz\pgfuseplotmark{m5c};};
\node[stars] at (axis cs:{7.857},{33.581}) {\tikz\pgfuseplotmark{m6bb};};
\node[stars] at (axis cs:{8.099},{6.955}) {\tikz\pgfuseplotmark{m6avb};};
\node[stars] at (axis cs:{8.148},{20.294}) {\tikz\pgfuseplotmark{m5c};};
\node[stars] at (axis cs:{8.205},{28.280}) {\tikz\pgfuseplotmark{m6c};};
\node[stars] at (axis cs:{8.421},{-29.558}) {\tikz\pgfuseplotmark{m6a};};
\node[stars] at (axis cs:{8.731},{13.371}) {\tikz\pgfuseplotmark{m6c};};
\node[stars] at (axis cs:{8.812},{-3.593}) {\tikz\pgfuseplotmark{m5cv};};
\node[stars] at (axis cs:{8.887},{-0.506}) {\tikz\pgfuseplotmark{m6b};};
\node[stars] at (axis cs:{8.978},{13.207}) {\tikz\pgfuseplotmark{m6c};};
\node[stars] at (axis cs:{9.013},{-14.973}) {\tikz\pgfuseplotmark{m6c};};
\node[stars] at (axis cs:{9.029},{-22.842}) {\tikz\pgfuseplotmark{m6bv};};
\node[stars] at (axis cs:{9.197},{15.231}) {\tikz\pgfuseplotmark{m6bv};};
\node[stars] at (axis cs:{9.220},{33.719}) {\tikz\pgfuseplotmark{m4cvb};};
\node[stars] at (axis cs:{9.280},{24.014}) {\tikz\pgfuseplotmark{m6c};};
\node[stars] at (axis cs:{9.336},{-24.767}) {\tikz\pgfuseplotmark{m6a};};
\node[stars] at (axis cs:{9.338},{35.399}) {\tikz\pgfuseplotmark{m5c};};
\node[stars] at (axis cs:{9.377},{3.135}) {\tikz\pgfuseplotmark{m6c};};
\node[stars] at (axis cs:{9.639},{29.312}) {\tikz\pgfuseplotmark{m4c};};
\node[stars] at (axis cs:{9.703},{-25.108}) {\tikz\pgfuseplotmark{m6cb};};
\node[stars] at (axis cs:{9.832},{30.861}) {\tikz\pgfuseplotmark{m3c};};
\node[stars] at (axis cs:{9.841},{21.250}) {\tikz\pgfuseplotmark{m6bv};};
\node[stars] at (axis cs:{9.982},{21.438}) {\tikz\pgfuseplotmark{m5cb};};
\node[stars] at (axis cs:{10.119},{-16.517}) {\tikz\pgfuseplotmark{m6cb};};
\node[stars] at (axis cs:{10.137},{-23.805}) {\tikz\pgfuseplotmark{m6c};};
\node[stars] at (axis cs:{10.177},{-4.352}) {\tikz\pgfuseplotmark{m6bb};};
\node[stars] at (axis cs:{10.280},{39.459}) {\tikz\pgfuseplotmark{m5c};};
\node[stars] at (axis cs:{10.400},{24.629}) {\tikz\pgfuseplotmark{m6bv};};
\node[stars] at (axis cs:{10.679},{-38.463}) {\tikz\pgfuseplotmark{m6b};};
\node[stars] at (axis cs:{10.897},{-17.987}) {\tikz\pgfuseplotmark{m2bv};};
\node[stars] at (axis cs:{10.959},{-12.012}) {\tikz\pgfuseplotmark{m6b};};
\node[stars] at (axis cs:{11.048},{-10.609}) {\tikz\pgfuseplotmark{m5av};};
\node[stars] at (axis cs:{11.050},{-38.422}) {\tikz\pgfuseplotmark{m6b};};
\node[stars] at (axis cs:{11.185},{-22.006}) {\tikz\pgfuseplotmark{m5c};};
\node[stars] at (axis cs:{11.351},{-4.629}) {\tikz\pgfuseplotmark{m6c};};
\node[stars] at (axis cs:{11.370},{-12.881}) {\tikz\pgfuseplotmark{m6c};};
\node[stars] at (axis cs:{11.549},{-22.522}) {\tikz\pgfuseplotmark{m5c};};
\node[stars] at (axis cs:{11.637},{15.475}) {\tikz\pgfuseplotmark{m5cv};};
\node[stars] at (axis cs:{11.756},{11.974}) {\tikz\pgfuseplotmark{m6a};};
\node[stars] at (axis cs:{11.807},{19.579}) {\tikz\pgfuseplotmark{m6bv};};
\node[stars] at (axis cs:{11.835},{24.267}) {\tikz\pgfuseplotmark{m4bv};};
\node[stars] at (axis cs:{11.848},{6.741}) {\tikz\pgfuseplotmark{m6b};};
\node[stars] at (axis cs:{11.930},{-18.061}) {\tikz\pgfuseplotmark{m6a};};
\node[stars] at (axis cs:{12.004},{-21.722}) {\tikz\pgfuseplotmark{m6a};};
\node[stars] at (axis cs:{12.073},{7.300}) {\tikz\pgfuseplotmark{m6b};};
\node[stars] at (axis cs:{12.096},{5.280}) {\tikz\pgfuseplotmark{m6ab};};
\node[stars] at (axis cs:{12.171},{7.585}) {\tikz\pgfuseplotmark{m4c};};
\node[stars] at (axis cs:{12.245},{16.941}) {\tikz\pgfuseplotmark{m5b};};
\node[stars] at (axis cs:{12.308},{-24.137}) {\tikz\pgfuseplotmark{m6b};};
\node[stars] at (axis cs:{12.357},{-13.561}) {\tikz\pgfuseplotmark{m6a};};
\node[stars] at (axis cs:{12.389},{-23.362}) {\tikz\pgfuseplotmark{m6c};};
\node[stars] at (axis cs:{12.472},{27.710}) {\tikz\pgfuseplotmark{m6ab};};
\node[stars] at (axis cs:{12.532},{-10.644}) {\tikz\pgfuseplotmark{m5cv};};
\node[stars] at (axis cs:{12.826},{3.385}) {\tikz\pgfuseplotmark{m6c};};
\node[stars] at (axis cs:{13.169},{-24.006}) {\tikz\pgfuseplotmark{m5c};};
\node[stars] at (axis cs:{13.252},{-1.144}) {\tikz\pgfuseplotmark{m5a};};
\node[stars] at (axis cs:{13.368},{37.418}) {\tikz\pgfuseplotmark{m6b};};
\node[stars] at (axis cs:{13.573},{-8.741}) {\tikz\pgfuseplotmark{m6c};};
\node[stars] at (axis cs:{13.647},{19.188}) {\tikz\pgfuseplotmark{m6a};};
\node[stars] at (axis cs:{13.742},{23.628}) {\tikz\pgfuseplotmark{m5cvb};};
\node[stars] at (axis cs:{13.811},{24.557}) {\tikz\pgfuseplotmark{m6cv};};
\node[stars] at (axis cs:{13.927},{-7.347}) {\tikz\pgfuseplotmark{m6b};};
\node[stars] at (axis cs:{13.981},{-27.776}) {\tikz\pgfuseplotmark{m6b};};
\node[stars] at (axis cs:{13.994},{27.209}) {\tikz\pgfuseplotmark{m6b};};
\node[stars] at (axis cs:{14.006},{-11.266}) {\tikz\pgfuseplotmark{m5c};};
\node[stars] at (axis cs:{14.038},{13.952}) {\tikz\pgfuseplotmark{m6c};};
\node[stars] at (axis cs:{14.188},{38.499}) {\tikz\pgfuseplotmark{m4b};};
\node[stars] at (axis cs:{14.302},{23.418}) {\tikz\pgfuseplotmark{m4c};};
\node[stars] at (axis cs:{14.459},{28.992}) {\tikz\pgfuseplotmark{m5c};};
\node[stars] at (axis cs:{14.477},{13.696}) {\tikz\pgfuseplotmark{m6c};};
\node[stars] at (axis cs:{14.559},{33.951}) {\tikz\pgfuseplotmark{m6b};};
\node[stars] at (axis cs:{14.579},{21.404}) {\tikz\pgfuseplotmark{m6c};};
\node[stars] at (axis cs:{14.652},{-29.357}) {\tikz\pgfuseplotmark{m4cv};};
\node[stars] at (axis cs:{14.683},{-11.380}) {\tikz\pgfuseplotmark{m6a};};
\node[stars] at (axis cs:{14.957},{6.483}) {\tikz\pgfuseplotmark{m6cv};};
\node[stars] at (axis cs:{15.326},{-38.916}) {\tikz\pgfuseplotmark{m6a};};
\node[stars] at (axis cs:{15.411},{-16.265}) {\tikz\pgfuseplotmark{m6c};};
\node[stars] at (axis cs:{15.610},{-31.552}) {\tikz\pgfuseplotmark{m6av};};
\node[stars] at (axis cs:{15.705},{31.804}) {\tikz\pgfuseplotmark{m5c};};
\node[stars] at (axis cs:{15.736},{7.890}) {\tikz\pgfuseplotmark{m4c};};
\node[stars] at (axis cs:{15.761},{-4.837}) {\tikz\pgfuseplotmark{m5c};};
\node[stars] at (axis cs:{15.824},{-29.526}) {\tikz\pgfuseplotmark{m6c};};
\node[stars] at (axis cs:{15.954},{1.367}) {\tikz\pgfuseplotmark{m6bb};};
\node[stars] at (axis cs:{16.115},{29.659}) {\tikz\pgfuseplotmark{m6c};};
\node[stars] at (axis cs:{16.136},{-33.533}) {\tikz\pgfuseplotmark{m6cb};};
\node[stars] at (axis cs:{16.219},{5.656}) {\tikz\pgfuseplotmark{m6b};};
\node[stars] at (axis cs:{16.272},{14.946}) {\tikz\pgfuseplotmark{m6a};};
\node[stars] at (axis cs:{16.404},{-9.979}) {\tikz\pgfuseplotmark{m6b};};
\node[stars] at (axis cs:{16.421},{21.473}) {\tikz\pgfuseplotmark{m5cvb};};
\node[stars] at (axis cs:{16.455},{4.908}) {\tikz\pgfuseplotmark{m6cb};};
\node[stars] at (axis cs:{16.521},{-9.839}) {\tikz\pgfuseplotmark{m6a};};
\node[stars] at (axis cs:{16.532},{-23.992}) {\tikz\pgfuseplotmark{m6b};};
\node[stars] at (axis cs:{16.547},{32.181}) {\tikz\pgfuseplotmark{m6c};};
\node[stars] at (axis cs:{16.640},{12.956}) {\tikz\pgfuseplotmark{m6c};};
\node[stars] at (axis cs:{16.804},{-23.996}) {\tikz\pgfuseplotmark{m6c};};
\node[stars] at (axis cs:{16.943},{-9.786}) {\tikz\pgfuseplotmark{m6a};};
\node[stars] at (axis cs:{16.988},{20.739}) {\tikz\pgfuseplotmark{m6a};};
\node[stars] at (axis cs:{16.999},{1.993}) {\tikz\pgfuseplotmark{m6c};};
\node[stars] at (axis cs:{17.006},{32.012}) {\tikz\pgfuseplotmark{m6c};};
\node[stars] at (axis cs:{17.092},{5.650}) {\tikz\pgfuseplotmark{m6av};};
\node[stars] at (axis cs:{17.147},{-10.182}) {\tikz\pgfuseplotmark{m3c};};
\node[stars] at (axis cs:{17.433},{35.620}) {\tikz\pgfuseplotmark{m2bvb};};
\node[stars] at (axis cs:{17.455},{19.658}) {\tikz\pgfuseplotmark{m6a};};
\node[stars] at (axis cs:{17.548},{15.674}) {\tikz\pgfuseplotmark{m6b};};
\node[stars] at (axis cs:{17.550},{-8.906}) {\tikz\pgfuseplotmark{m6c};};
\node[stars] at (axis cs:{17.581},{25.458}) {\tikz\pgfuseplotmark{m6a};};
\node[stars] at (axis cs:{17.640},{2.445}) {\tikz\pgfuseplotmark{m6bv};};
\node[stars] at (axis cs:{17.778},{31.425}) {\tikz\pgfuseplotmark{m5c};};
\node[stars] at (axis cs:{17.793},{37.724}) {\tikz\pgfuseplotmark{m6a};};
\node[stars] at (axis cs:{17.863},{21.035}) {\tikz\pgfuseplotmark{m5a};};
\node[stars] at (axis cs:{17.915},{30.090}) {\tikz\pgfuseplotmark{m5a};};
\node[stars] at (axis cs:{17.931},{-2.251}) {\tikz\pgfuseplotmark{m6b};};
\node[stars] at (axis cs:{18.189},{-37.856}) {\tikz\pgfuseplotmark{m6bv};};
\node[stars] at (axis cs:{18.248},{30.064}) {\tikz\pgfuseplotmark{m6c};};
\node[stars] at (axis cs:{18.433},{7.575}) {\tikz\pgfuseplotmark{m5cb};};
\node[stars] at (axis cs:{18.437},{24.584}) {\tikz\pgfuseplotmark{m5a};};
\node[stars] at (axis cs:{18.520},{28.529}) {\tikz\pgfuseplotmark{m6cv};};
\node[stars] at (axis cs:{18.532},{16.133}) {\tikz\pgfuseplotmark{m6bv};};
\node[stars] at (axis cs:{18.600},{-7.923}) {\tikz\pgfuseplotmark{m5cvb};};
\node[stars] at (axis cs:{18.677},{6.995}) {\tikz\pgfuseplotmark{m6b};};
\node[stars] at (axis cs:{18.705},{-0.974}) {\tikz\pgfuseplotmark{m6a};};
\node[stars] at (axis cs:{19.079},{33.114}) {\tikz\pgfuseplotmark{m6b};};
\node[stars] at (axis cs:{19.151},{-2.500}) {\tikz\pgfuseplotmark{m5cv};};
\node[stars] at (axis cs:{19.350},{31.744}) {\tikz\pgfuseplotmark{m6c};};
\node[stars] at (axis cs:{19.450},{3.614}) {\tikz\pgfuseplotmark{m5c};};
\node[stars] at (axis cs:{19.696},{37.386}) {\tikz\pgfuseplotmark{m6cb};};
\node[stars] at (axis cs:{19.867},{27.264}) {\tikz\pgfuseplotmark{m5a};};
\node[stars] at (axis cs:{19.951},{-0.509}) {\tikz\pgfuseplotmark{m6cv};};
\node[stars] at (axis cs:{20.116},{-11.239}) {\tikz\pgfuseplotmark{m6c};};
\node[stars] at (axis cs:{20.144},{-3.247}) {\tikz\pgfuseplotmark{m6c};};
\node[stars] at (axis cs:{20.281},{28.738}) {\tikz\pgfuseplotmark{m5c};};
\node[stars] at (axis cs:{20.627},{-19.081}) {\tikz\pgfuseplotmark{m6cb};};
\node[stars] at (axis cs:{20.645},{-0.449}) {\tikz\pgfuseplotmark{m6cv};};
\node[stars] at (axis cs:{20.654},{1.726}) {\tikz\pgfuseplotmark{m6c};};
\node[stars] at (axis cs:{20.854},{20.469}) {\tikz\pgfuseplotmark{m6b};};
\node[stars] at (axis cs:{20.879},{-30.945}) {\tikz\pgfuseplotmark{m6a};};
\node[stars] at (axis cs:{20.906},{34.246}) {\tikz\pgfuseplotmark{m6c};};
\node[stars] at (axis cs:{20.919},{37.715}) {\tikz\pgfuseplotmark{m6a};};
\node[stars] at (axis cs:{21.006},{-8.183}) {\tikz\pgfuseplotmark{m4a};};
\node[stars] at (axis cs:{21.011},{-8.007}) {\tikz\pgfuseplotmark{m6cv};};
\node[stars] at (axis cs:{21.085},{-6.914}) {\tikz\pgfuseplotmark{m6b};};
\node[stars] at (axis cs:{21.166},{-15.660}) {\tikz\pgfuseplotmark{m6c};};
\node[stars] at (axis cs:{21.203},{-2.848}) {\tikz\pgfuseplotmark{m6c};};
\node[stars] at (axis cs:{21.399},{23.511}) {\tikz\pgfuseplotmark{m6c};};
\node[stars] at (axis cs:{21.405},{-14.599}) {\tikz\pgfuseplotmark{m5b};};
\node[stars] at (axis cs:{21.468},{-3.927}) {\tikz\pgfuseplotmark{m6cv};};
\node[stars] at (axis cs:{21.537},{34.579}) {\tikz\pgfuseplotmark{m6c};};
\node[stars] at (axis cs:{21.564},{19.172}) {\tikz\pgfuseplotmark{m5cv};};
\node[stars] at (axis cs:{21.598},{20.071}) {\tikz\pgfuseplotmark{m6c};};
\node[stars] at (axis cs:{21.614},{-0.399}) {\tikz\pgfuseplotmark{m6cv};};
\node[stars] at (axis cs:{21.674},{19.240}) {\tikz\pgfuseplotmark{m5c};};
\node[stars] at (axis cs:{21.715},{-13.056}) {\tikz\pgfuseplotmark{m6av};};
\node[stars] at (axis cs:{21.776},{34.377}) {\tikz\pgfuseplotmark{m6c};};
\node[stars] at (axis cs:{21.847},{-22.338}) {\tikz\pgfuseplotmark{m6c};};
\node[stars] at (axis cs:{21.944},{-10.902}) {\tikz\pgfuseplotmark{m6cb};};
\node[stars] at (axis cs:{22.095},{7.961}) {\tikz\pgfuseplotmark{m6cvb};};
\node[stars] at (axis cs:{22.401},{-21.629}) {\tikz\pgfuseplotmark{m5b};};
\node[stars] at (axis cs:{22.428},{-25.618}) {\tikz\pgfuseplotmark{m6c};};
\node[stars] at (axis cs:{22.470},{18.356}) {\tikz\pgfuseplotmark{m6bv};};
\node[stars] at (axis cs:{22.546},{6.144}) {\tikz\pgfuseplotmark{m5ab};};
\node[stars] at (axis cs:{22.595},{-26.208}) {\tikz\pgfuseplotmark{m6b};};
\node[stars] at (axis cs:{22.871},{15.346}) {\tikz\pgfuseplotmark{m4av};};
\node[stars] at (axis cs:{22.930},{-30.283}) {\tikz\pgfuseplotmark{m6a};};
\node[stars] at (axis cs:{23.032},{34.800}) {\tikz\pgfuseplotmark{m6c};};
\node[stars] at (axis cs:{23.234},{-36.865}) {\tikz\pgfuseplotmark{m5c};};
\node[stars] at (axis cs:{23.326},{8.209}) {\tikz\pgfuseplotmark{m6c};};
\node[stars] at (axis cs:{23.428},{-7.025}) {\tikz\pgfuseplotmark{m6a};};
\node[stars] at (axis cs:{23.569},{37.237}) {\tikz\pgfuseplotmark{m6bv};};
\node[stars] at (axis cs:{23.657},{-15.676}) {\tikz\pgfuseplotmark{m6a};};
\node[stars] at (axis cs:{23.704},{18.460}) {\tikz\pgfuseplotmark{m6b};};
\node[stars] at (axis cs:{23.712},{-31.892}) {\tikz\pgfuseplotmark{m6bv};};
\node[stars] at (axis cs:{23.714},{-3.524}) {\tikz\pgfuseplotmark{m6c};};
\node[stars] at (axis cs:{23.715},{-23.699}) {\tikz\pgfuseplotmark{m6c};};
\node[stars] at (axis cs:{23.944},{14.661}) {\tikz\pgfuseplotmark{m6c};};
\node[stars] at (axis cs:{23.961},{-39.947}) {\tikz\pgfuseplotmark{m6c};};
\node[stars] at (axis cs:{23.978},{17.434}) {\tikz\pgfuseplotmark{m6b};};
\node[stars] at (axis cs:{23.996},{-15.400}) {\tikz\pgfuseplotmark{m5c};};
\node[stars] at (axis cs:{24.035},{-29.907}) {\tikz\pgfuseplotmark{m6a};};
\node[stars] at (axis cs:{24.181},{7.831}) {\tikz\pgfuseplotmark{m6cv};};
\node[stars] at (axis cs:{24.275},{12.141}) {\tikz\pgfuseplotmark{m6a};};
\node[stars] at (axis cs:{24.407},{-9.404}) {\tikz\pgfuseplotmark{m6c};};
\node[stars] at (axis cs:{24.467},{-3.441}) {\tikz\pgfuseplotmark{m6c};};
\node[stars] at (axis cs:{24.615},{-36.528}) {\tikz\pgfuseplotmark{m6b};};
\node[stars] at (axis cs:{24.716},{-21.275}) {\tikz\pgfuseplotmark{m6a};};
\node[stars] at (axis cs:{24.920},{16.406}) {\tikz\pgfuseplotmark{m6b};};
\node[stars] at (axis cs:{25.146},{8.761}) {\tikz\pgfuseplotmark{m6c};};
\node[stars] at (axis cs:{25.327},{25.746}) {\tikz\pgfuseplotmark{m6c};};
\node[stars] at (axis cs:{25.358},{5.488}) {\tikz\pgfuseplotmark{m4c};};
\node[stars] at (axis cs:{25.364},{-38.133}) {\tikz\pgfuseplotmark{m6c};};
\node[stars] at (axis cs:{25.413},{30.047}) {\tikz\pgfuseplotmark{m6b};};
\node[stars] at (axis cs:{25.437},{-11.325}) {\tikz\pgfuseplotmark{m6a};};
\node[stars] at (axis cs:{25.512},{-36.832}) {\tikz\pgfuseplotmark{m6a};};
\node[stars] at (axis cs:{25.515},{35.246}) {\tikz\pgfuseplotmark{m6a};};
\node[stars] at (axis cs:{25.536},{-32.327}) {\tikz\pgfuseplotmark{m5c};};
\node[stars] at (axis cs:{25.624},{20.268}) {\tikz\pgfuseplotmark{m5cv};};
\node[stars] at (axis cs:{25.681},{-3.690}) {\tikz\pgfuseplotmark{m5b};};
\node[stars] at (axis cs:{25.959},{32.192}) {\tikz\pgfuseplotmark{m6c};};
\node[stars] at (axis cs:{25.978},{-4.765}) {\tikz\pgfuseplotmark{m6c};};
\node[stars] at (axis cs:{26.017},{-15.937}) {\tikz\pgfuseplotmark{m3c};};
\node[stars] at (axis cs:{26.182},{-6.766}) {\tikz\pgfuseplotmark{m6cb};};
\node[stars] at (axis cs:{26.233},{20.083}) {\tikz\pgfuseplotmark{m6c};};
\node[stars] at (axis cs:{26.348},{9.158}) {\tikz\pgfuseplotmark{m4cv};};
\node[stars] at (axis cs:{26.411},{-25.053}) {\tikz\pgfuseplotmark{m5cb};};
\node[stars] at (axis cs:{26.497},{-5.733}) {\tikz\pgfuseplotmark{m5c};};
\node[stars] at (axis cs:{26.504},{-27.349}) {\tikz\pgfuseplotmark{m6c};};
\node[stars] at (axis cs:{26.949},{-37.159}) {\tikz\pgfuseplotmark{m6c};};
\node[stars] at (axis cs:{27.046},{16.955}) {\tikz\pgfuseplotmark{m6a};};
\node[stars] at (axis cs:{27.108},{3.685}) {\tikz\pgfuseplotmark{m6b};};
\node[stars] at (axis cs:{27.162},{37.953}) {\tikz\pgfuseplotmark{m6b};};
\node[stars] at (axis cs:{27.173},{32.690}) {\tikz\pgfuseplotmark{m6a};};
\node[stars] at (axis cs:{27.331},{-31.072}) {\tikz\pgfuseplotmark{m6cv};};
\node[stars] at (axis cs:{27.396},{-10.686}) {\tikz\pgfuseplotmark{m5ab};};
\node[stars] at (axis cs:{27.454},{-38.403}) {\tikz\pgfuseplotmark{m6c};};
\node[stars] at (axis cs:{27.536},{22.275}) {\tikz\pgfuseplotmark{m6ab};};
\node[stars] at (axis cs:{27.717},{11.043}) {\tikz\pgfuseplotmark{m6bv};};
\node[stars] at (axis cs:{27.860},{-39.836}) {\tikz\pgfuseplotmark{m6c};};
\node[stars] at (axis cs:{27.865},{-10.335}) {\tikz\pgfuseplotmark{m4av};};
\node[stars] at (axis cs:{27.908},{-6.874}) {\tikz\pgfuseplotmark{m6c};};
\node[stars] at (axis cs:{27.942},{-15.647}) {\tikz\pgfuseplotmark{m6c};};
\node[stars] at (axis cs:{28.217},{-16.929}) {\tikz\pgfuseplotmark{m6av};};
\node[stars] at (axis cs:{28.270},{29.579}) {\tikz\pgfuseplotmark{m3cvb};};
\node[stars] at (axis cs:{28.347},{-38.594}) {\tikz\pgfuseplotmark{m6b};};
\node[stars] at (axis cs:{28.383},{19.294}) {\tikz\pgfuseplotmark{m4bvb};};
\node[stars] at (axis cs:{28.389},{3.187}) {\tikz\pgfuseplotmark{m5av};};
\node[stars] at (axis cs:{28.660},{20.808}) {\tikz\pgfuseplotmark{m3av};};
\node[stars] at (axis cs:{28.740},{37.128}) {\tikz\pgfuseplotmark{m6c};};
\node[stars] at (axis cs:{28.963},{23.577}) {\tikz\pgfuseplotmark{m6av};};
\node[stars] at (axis cs:{28.974},{1.850}) {\tikz\pgfuseplotmark{m6b};};
\node[stars] at (axis cs:{29.039},{37.252}) {\tikz\pgfuseplotmark{m6ab};};
\node[stars] at (axis cs:{29.167},{-22.527}) {\tikz\pgfuseplotmark{m5b};};
\node[stars] at (axis cs:{29.290},{-10.242}) {\tikz\pgfuseplotmark{m6c};};
\node[stars] at (axis cs:{29.338},{17.818}) {\tikz\pgfuseplotmark{m5b};};
\node[stars] at (axis cs:{29.432},{27.804}) {\tikz\pgfuseplotmark{m6a};};
\node[stars] at (axis cs:{29.482},{23.596}) {\tikz\pgfuseplotmark{m5ab};};
\node[stars] at (axis cs:{29.611},{-33.067}) {\tikz\pgfuseplotmark{m6c};};
\node[stars] at (axis cs:{29.692},{-11.301}) {\tikz\pgfuseplotmark{m6c};};
\node[stars] at (axis cs:{29.858},{12.295}) {\tikz\pgfuseplotmark{m6b};};
\node[stars] at (axis cs:{29.942},{-20.824}) {\tikz\pgfuseplotmark{m5c};};
\node[stars] at (axis cs:{30.001},{-21.078}) {\tikz\pgfuseplotmark{m4b};};
\node[stars] at (axis cs:{30.038},{3.097}) {\tikz\pgfuseplotmark{m6b};};
\node[stars] at (axis cs:{30.112},{-8.524}) {\tikz\pgfuseplotmark{m6avb};};
\node[stars] at (axis cs:{30.311},{-30.002}) {\tikz\pgfuseplotmark{m5c};};
\node[stars] at (axis cs:{30.512},{2.764}) {\tikz\pgfuseplotmark{m4avb};};
\node[stars] at (axis cs:{30.618},{-29.665}) {\tikz\pgfuseplotmark{m6c};};
\node[stars] at (axis cs:{30.646},{13.477}) {\tikz\pgfuseplotmark{m6bv};};
\node[stars] at (axis cs:{30.716},{-23.887}) {\tikz\pgfuseplotmark{m6c};};
\node[stars] at (axis cs:{30.741},{33.284}) {\tikz\pgfuseplotmark{m6a};};
\node[stars] at (axis cs:{30.744},{-15.306}) {\tikz\pgfuseplotmark{m6a};};
\node[stars] at (axis cs:{30.799},{0.128}) {\tikz\pgfuseplotmark{m5cv};};
\node[stars] at (axis cs:{30.914},{25.935}) {\tikz\pgfuseplotmark{m6ab};};
\node[stars] at (axis cs:{30.919},{-4.103}) {\tikz\pgfuseplotmark{m6a};};
\node[stars] at (axis cs:{30.928},{18.253}) {\tikz\pgfuseplotmark{m6c};};
\node[stars] at (axis cs:{30.951},{-0.340}) {\tikz\pgfuseplotmark{m6bv};};
\node[stars] at (axis cs:{31.052},{-11.860}) {\tikz\pgfuseplotmark{m6cv};};
\node[stars] at (axis cs:{31.123},{-29.297}) {\tikz\pgfuseplotmark{m5av};};
\node[stars] at (axis cs:{31.213},{7.736}) {\tikz\pgfuseplotmark{m6c};};
\node[stars] at (axis cs:{31.551},{8.248}) {\tikz\pgfuseplotmark{m6cv};};
\node[stars] at (axis cs:{31.622},{0.035}) {\tikz\pgfuseplotmark{m6c};};
\node[stars] at (axis cs:{31.641},{22.648}) {\tikz\pgfuseplotmark{m5b};};
\node[stars] at (axis cs:{31.705},{25.704}) {\tikz\pgfuseplotmark{m6b};};
\node[stars] at (axis cs:{31.717},{-19.139}) {\tikz\pgfuseplotmark{m6c};};
\node[stars] at (axis cs:{31.793},{23.462}) {\tikz\pgfuseplotmark{m2bv};};
\node[stars] at (axis cs:{32.122},{37.859}) {\tikz\pgfuseplotmark{m5a};};
\node[stars] at (axis cs:{32.190},{-17.779}) {\tikz\pgfuseplotmark{m6bv};};
\node[stars] at (axis cs:{32.346},{17.224}) {\tikz\pgfuseplotmark{m6c};};
\node[stars] at (axis cs:{32.356},{25.940}) {\tikz\pgfuseplotmark{m5bb};};
\node[stars] at (axis cs:{32.386},{34.987}) {\tikz\pgfuseplotmark{m3b};};
\node[stars] at (axis cs:{32.395},{-24.346}) {\tikz\pgfuseplotmark{m6c};};
\node[stars] at (axis cs:{32.657},{19.500}) {\tikz\pgfuseplotmark{m6av};};
\node[stars] at (axis cs:{32.720},{39.039}) {\tikz\pgfuseplotmark{m6bb};};
\node[stars] at (axis cs:{32.800},{25.937}) {\tikz\pgfuseplotmark{m6b};};
\node[stars] at (axis cs:{32.838},{8.570}) {\tikz\pgfuseplotmark{m6a};};
\node[stars] at (axis cs:{32.843},{-10.052}) {\tikz\pgfuseplotmark{m6b};};
\node[stars] at (axis cs:{32.854},{31.526}) {\tikz\pgfuseplotmark{m6c};};
\node[stars] at (axis cs:{32.899},{-1.825}) {\tikz\pgfuseplotmark{m6b};};
\node[stars] at (axis cs:{33.066},{2.745}) {\tikz\pgfuseplotmark{m6c};};
\node[stars] at (axis cs:{33.093},{30.303}) {\tikz\pgfuseplotmark{m5cvb};};
\node[stars] at (axis cs:{33.156},{24.168}) {\tikz\pgfuseplotmark{m6b};};
\node[stars] at (axis cs:{33.198},{-2.393}) {\tikz\pgfuseplotmark{m6ab};};
\node[stars] at (axis cs:{33.200},{21.211}) {\tikz\pgfuseplotmark{m5c};};
\node[stars] at (axis cs:{33.227},{-30.724}) {\tikz\pgfuseplotmark{m5c};};
\node[stars] at (axis cs:{33.250},{8.846}) {\tikz\pgfuseplotmark{m4cv};};
\node[stars] at (axis cs:{33.254},{-21.000}) {\tikz\pgfuseplotmark{m6b};};
\node[stars] at (axis cs:{33.264},{15.280}) {\tikz\pgfuseplotmark{m6av};};
\node[stars] at (axis cs:{33.473},{-9.064}) {\tikz\pgfuseplotmark{m6c};};
\node[stars] at (axis cs:{33.658},{28.691}) {\tikz\pgfuseplotmark{m6c};};
\node[stars] at (axis cs:{33.928},{25.043}) {\tikz\pgfuseplotmark{m6a};};
\node[stars] at (axis cs:{33.942},{25.783}) {\tikz\pgfuseplotmark{m6a};};
\node[stars] at (axis cs:{33.985},{33.359}) {\tikz\pgfuseplotmark{m5c};};
\node[stars] at (axis cs:{34.246},{-6.422}) {\tikz\pgfuseplotmark{m5c};};
\node[stars] at (axis cs:{34.263},{34.224}) {\tikz\pgfuseplotmark{m5b};};
\node[stars] at (axis cs:{34.329},{33.847}) {\tikz\pgfuseplotmark{m4b};};
\node[stars] at (axis cs:{34.506},{1.758}) {\tikz\pgfuseplotmark{m6a};};
\node[stars] at (axis cs:{34.531},{19.901}) {\tikz\pgfuseplotmark{m6a};};
\node[stars] at (axis cs:{34.737},{28.643}) {\tikz\pgfuseplotmark{m5c};};
\node[stars] at (axis cs:{34.741},{23.168}) {\tikz\pgfuseplotmark{m6c};};
\node[stars] at (axis cs:{34.744},{-25.945}) {\tikz\pgfuseplotmark{m6c};};
\node[stars] at (axis cs:{34.837},{-2.977}) {\tikz\pgfuseplotmark{m5bvb};};
\node[stars] at (axis cs:{35.486},{0.395}) {\tikz\pgfuseplotmark{m5cv};};
\node[stars] at (axis cs:{35.506},{-10.777}) {\tikz\pgfuseplotmark{m5c};};
\node[stars] at (axis cs:{35.521},{-17.662}) {\tikz\pgfuseplotmark{m6b};};
\node[stars] at (axis cs:{35.552},{-0.885}) {\tikz\pgfuseplotmark{m5cv};};
\node[stars] at (axis cs:{35.636},{-23.816}) {\tikz\pgfuseplotmark{m5c};};
\node[stars] at (axis cs:{35.740},{-18.355}) {\tikz\pgfuseplotmark{m6c};};
\node[stars] at (axis cs:{36.084},{-25.847}) {\tikz\pgfuseplotmark{m6c};};
\node[stars] at (axis cs:{36.204},{10.610}) {\tikz\pgfuseplotmark{m5c};};
\node[stars] at (axis cs:{36.243},{-2.780}) {\tikz\pgfuseplotmark{m6c};};
\node[stars] at (axis cs:{36.488},{-12.290}) {\tikz\pgfuseplotmark{m5b};};
\node[stars] at (axis cs:{36.501},{-15.341}) {\tikz\pgfuseplotmark{m6avb};};
\node[stars] at (axis cs:{36.647},{-20.042}) {\tikz\pgfuseplotmark{m6b};};
\node[stars] at (axis cs:{36.780},{27.012}) {\tikz\pgfuseplotmark{m6b};};
\node[stars] at (axis cs:{36.847},{10.198}) {\tikz\pgfuseplotmark{m6c};};
\node[stars] at (axis cs:{36.866},{31.801}) {\tikz\pgfuseplotmark{m6a};};
\node[stars] at (axis cs:{37.000},{1.961}) {\tikz\pgfuseplotmark{m6c};};
\node[stars] at (axis cs:{37.007},{-33.811}) {\tikz\pgfuseplotmark{m5b};};
\node[stars] at (axis cs:{37.040},{8.460}) {\tikz\pgfuseplotmark{m4c};};
\node[stars] at (axis cs:{37.042},{29.669}) {\tikz\pgfuseplotmark{m5c};};
\node[stars] at (axis cs:{37.148},{-31.102}) {\tikz\pgfuseplotmark{m6b};};
\node[stars] at (axis cs:{37.202},{29.932}) {\tikz\pgfuseplotmark{m6b};};
\node[stars] at (axis cs:{37.307},{23.469}) {\tikz\pgfuseplotmark{m6c};};
\node[stars] at (axis cs:{37.397},{9.565}) {\tikz\pgfuseplotmark{m6bb};};
\node[stars] at (axis cs:{37.569},{33.834}) {\tikz\pgfuseplotmark{m6c};};
\node[stars] at (axis cs:{37.635},{25.235}) {\tikz\pgfuseplotmark{m6b};};
\node[stars] at (axis cs:{37.637},{-22.545}) {\tikz\pgfuseplotmark{m6c};};
\node[stars] at (axis cs:{37.660},{19.855}) {\tikz\pgfuseplotmark{m6cv};};
\node[stars] at (axis cs:{37.688},{0.256}) {\tikz\pgfuseplotmark{m6bv};};
\node[stars] at (axis cs:{37.727},{17.704}) {\tikz\pgfuseplotmark{m6c};};
\node[stars] at (axis cs:{37.875},{2.267}) {\tikz\pgfuseplotmark{m5c};};
\node[stars] at (axis cs:{38.022},{-15.244}) {\tikz\pgfuseplotmark{m5a};};
\node[stars] at (axis cs:{38.026},{36.147}) {\tikz\pgfuseplotmark{m5c};};
\node[stars] at (axis cs:{38.039},{-1.035}) {\tikz\pgfuseplotmark{m5c};};
\node[stars] at (axis cs:{38.062},{-36.427}) {\tikz\pgfuseplotmark{m6c};};
\node[stars] at (axis cs:{38.219},{34.542}) {\tikz\pgfuseplotmark{m6a};};
\node[stars] at (axis cs:{38.226},{15.035}) {\tikz\pgfuseplotmark{m6b};};
\node[stars] at (axis cs:{38.279},{-34.650}) {\tikz\pgfuseplotmark{m6b};};
\node[stars] at (axis cs:{38.418},{-20.002}) {\tikz\pgfuseplotmark{m6c};};
\node[stars] at (axis cs:{38.461},{-28.232}) {\tikz\pgfuseplotmark{m5bb};};
\node[stars] at (axis cs:{38.678},{-7.859}) {\tikz\pgfuseplotmark{m6a};};
\node[stars] at (axis cs:{38.767},{7.471}) {\tikz\pgfuseplotmark{m6c};};
\node[stars] at (axis cs:{38.866},{39.664}) {\tikz\pgfuseplotmark{m6c};};
\node[stars] at (axis cs:{38.911},{37.312}) {\tikz\pgfuseplotmark{m6a};};
\node[stars] at (axis cs:{38.945},{34.687}) {\tikz\pgfuseplotmark{m5cvb};};
\node[stars] at (axis cs:{38.969},{5.593}) {\tikz\pgfuseplotmark{m5b};};
\node[stars] at (axis cs:{39.000},{-7.831}) {\tikz\pgfuseplotmark{m6a};};
\node[stars] at (axis cs:{39.020},{6.887}) {\tikz\pgfuseplotmark{m6a};};
\node[stars] at (axis cs:{39.039},{-30.045}) {\tikz\pgfuseplotmark{m6a};};
\node[stars] at (axis cs:{39.146},{7.730}) {\tikz\pgfuseplotmark{m6av};};
\node[stars] at (axis cs:{39.158},{12.447}) {\tikz\pgfuseplotmark{m6a};};
\node[stars] at (axis cs:{39.179},{31.608}) {\tikz\pgfuseplotmark{m6b};};
\node[stars] at (axis cs:{39.238},{38.734}) {\tikz\pgfuseplotmark{m6b};};
\node[stars] at (axis cs:{39.244},{-34.578}) {\tikz\pgfuseplotmark{m6a};};
\node[stars] at (axis cs:{39.252},{24.647}) {\tikz\pgfuseplotmark{m6cb};};
\node[stars] at (axis cs:{39.277},{32.891}) {\tikz\pgfuseplotmark{m6c};};
\node[stars] at (axis cs:{39.424},{-3.396}) {\tikz\pgfuseplotmark{m6a};};
\node[stars] at (axis cs:{39.503},{7.695}) {\tikz\pgfuseplotmark{m6c};};
\node[stars] at (axis cs:{39.574},{37.727}) {\tikz\pgfuseplotmark{m6c};};
\node[stars] at (axis cs:{39.578},{-30.194}) {\tikz\pgfuseplotmark{m6a};};
\node[stars] at (axis cs:{39.603},{-37.990}) {\tikz\pgfuseplotmark{m6c};};
\node[stars] at (axis cs:{39.616},{38.089}) {\tikz\pgfuseplotmark{m6c};};
\node[stars] at (axis cs:{39.653},{3.443}) {\tikz\pgfuseplotmark{m6c};};
\node[stars] at (axis cs:{39.704},{21.961}) {\tikz\pgfuseplotmark{m5c};};
\node[stars] at (axis cs:{39.871},{0.328}) {\tikz\pgfuseplotmark{m4bv};};
\node[stars] at (axis cs:{39.891},{-11.872}) {\tikz\pgfuseplotmark{m5a};};
\node[stars] at (axis cs:{40.052},{-9.453}) {\tikz\pgfuseplotmark{m6a};};
\node[stars] at (axis cs:{40.065},{6.112}) {\tikz\pgfuseplotmark{m6c};};
\node[stars] at (axis cs:{40.167},{-39.855}) {\tikz\pgfuseplotmark{m4b};};
\node[stars] at (axis cs:{40.171},{27.061}) {\tikz\pgfuseplotmark{m5c};};
\node[stars] at (axis cs:{40.308},{-0.695}) {\tikz\pgfuseplotmark{m6avb};};
\node[stars] at (axis cs:{40.392},{-14.549}) {\tikz\pgfuseplotmark{m6b};};
\node[stars] at (axis cs:{40.451},{-3.214}) {\tikz\pgfuseplotmark{m6b};};
\node[stars] at (axis cs:{40.528},{-38.384}) {\tikz\pgfuseplotmark{m6b};};
\node[stars] at (axis cs:{40.591},{20.011}) {\tikz\pgfuseplotmark{m6a};};
\node[stars] at (axis cs:{40.621},{10.742}) {\tikz\pgfuseplotmark{m6cv};};
\node[stars] at (axis cs:{40.825},{3.236}) {\tikz\pgfuseplotmark{m3cb};};
\node[stars] at (axis cs:{40.863},{27.707}) {\tikz\pgfuseplotmark{m5a};};
\node[stars] at (axis cs:{40.870},{-2.531}) {\tikz\pgfuseplotmark{m6c};};
\node[stars] at (axis cs:{40.964},{25.638}) {\tikz\pgfuseplotmark{m6cb};};
\node[stars] at (axis cs:{41.031},{-13.859}) {\tikz\pgfuseplotmark{m4c};};
\node[stars] at (axis cs:{41.080},{17.764}) {\tikz\pgfuseplotmark{m6c};};
\node[stars] at (axis cs:{41.086},{-32.525}) {\tikz\pgfuseplotmark{m6c};};
\node[stars] at (axis cs:{41.137},{15.312}) {\tikz\pgfuseplotmark{m6a};};
\node[stars] at (axis cs:{41.236},{10.114}) {\tikz\pgfuseplotmark{m4cv};};
\node[stars] at (axis cs:{41.240},{12.446}) {\tikz\pgfuseplotmark{m5cv};};
\node[stars] at (axis cs:{41.276},{-18.572}) {\tikz\pgfuseplotmark{m4c};};
\node[stars] at (axis cs:{41.337},{4.711}) {\tikz\pgfuseplotmark{m6b};};
\node[stars] at (axis cs:{41.688},{-21.639}) {\tikz\pgfuseplotmark{m6c};};
\node[stars] at (axis cs:{41.743},{35.984}) {\tikz\pgfuseplotmark{m6c};};
\node[stars] at (axis cs:{41.765},{35.555}) {\tikz\pgfuseplotmark{m6c};};
\node[stars] at (axis cs:{41.796},{-22.486}) {\tikz\pgfuseplotmark{m6c};};
\node[stars] at (axis cs:{41.977},{29.247}) {\tikz\pgfuseplotmark{m5a};};
\node[stars] at (axis cs:{42.134},{18.284}) {\tikz\pgfuseplotmark{m6a};};
\node[stars] at (axis cs:{42.191},{25.188}) {\tikz\pgfuseplotmark{m6bv};};
\node[stars] at (axis cs:{42.273},{-32.406}) {\tikz\pgfuseplotmark{m4c};};
\node[stars] at (axis cs:{42.323},{17.464}) {\tikz\pgfuseplotmark{m5cb};};
\node[stars] at (axis cs:{42.362},{37.326}) {\tikz\pgfuseplotmark{m6c};};
\node[stars] at (axis cs:{42.462},{-24.560}) {\tikz\pgfuseplotmark{m6c};};
\node[stars] at (axis cs:{42.476},{-27.942}) {\tikz\pgfuseplotmark{m5c};};
\node[stars] at (axis cs:{42.496},{27.260}) {\tikz\pgfuseplotmark{m4avb};};
\node[stars] at (axis cs:{42.562},{-35.843}) {\tikz\pgfuseplotmark{m6bb};};
\node[stars] at (axis cs:{42.646},{38.319}) {\tikz\pgfuseplotmark{m4cv};};
\node[stars] at (axis cs:{42.647},{36.948}) {\tikz\pgfuseplotmark{m6c};};
\node[stars] at (axis cs:{42.668},{-35.676}) {\tikz\pgfuseplotmark{m5c};};
\node[stars] at (axis cs:{42.699},{-39.932}) {\tikz\pgfuseplotmark{m6c};};
\node[stars] at (axis cs:{42.760},{-21.004}) {\tikz\pgfuseplotmark{m5a};};
\node[stars] at (axis cs:{42.873},{15.082}) {\tikz\pgfuseplotmark{m6a};};
\node[stars] at (axis cs:{42.878},{35.060}) {\tikz\pgfuseplotmark{m5av};};
\node[stars] at (axis cs:{42.984},{-30.814}) {\tikz\pgfuseplotmark{m6c};};
\node[stars] at (axis cs:{43.134},{-12.770}) {\tikz\pgfuseplotmark{m6bv};};
\node[stars] at (axis cs:{43.211},{-9.441}) {\tikz\pgfuseplotmark{m6c};};
\node[stars] at (axis cs:{43.299},{16.483}) {\tikz\pgfuseplotmark{m6c};};
\node[stars] at (axis cs:{43.393},{-38.437}) {\tikz\pgfuseplotmark{m6b};};
\node[stars] at (axis cs:{43.397},{-22.376}) {\tikz\pgfuseplotmark{m6b};};
\node[stars] at (axis cs:{43.428},{38.337}) {\tikz\pgfuseplotmark{m5cv};};
\node[stars] at (axis cs:{43.952},{18.331}) {\tikz\pgfuseplotmark{m6av};};
\node[stars] at (axis cs:{44.057},{8.381}) {\tikz\pgfuseplotmark{m6b};};
\node[stars] at (axis cs:{44.107},{-8.898}) {\tikz\pgfuseplotmark{m4bv};};
\node[stars] at (axis cs:{44.109},{18.023}) {\tikz\pgfuseplotmark{m6av};};
\node[stars] at (axis cs:{44.156},{-3.712}) {\tikz\pgfuseplotmark{m5c};};
\node[stars] at (axis cs:{44.269},{4.501}) {\tikz\pgfuseplotmark{m6cv};};
\node[stars] at (axis cs:{44.304},{-29.855}) {\tikz\pgfuseplotmark{m6c};};
\node[stars] at (axis cs:{44.322},{31.934}) {\tikz\pgfuseplotmark{m5bv};};
\node[stars] at (axis cs:{44.349},{-23.862}) {\tikz\pgfuseplotmark{m5c};};
\node[stars] at (axis cs:{44.386},{-38.191}) {\tikz\pgfuseplotmark{m6c};};
\node[stars] at (axis cs:{44.510},{38.615}) {\tikz\pgfuseplotmark{m6b};};
\node[stars] at (axis cs:{44.522},{20.669}) {\tikz\pgfuseplotmark{m6av};};
\node[stars] at (axis cs:{44.524},{-23.606}) {\tikz\pgfuseplotmark{m6a};};
\node[stars] at (axis cs:{44.675},{-2.783}) {\tikz\pgfuseplotmark{m5c};};
\node[stars] at (axis cs:{44.690},{39.663}) {\tikz\pgfuseplotmark{m5a};};
\node[stars] at (axis cs:{44.697},{-9.776}) {\tikz\pgfuseplotmark{m6c};};
\node[stars] at (axis cs:{44.765},{35.183}) {\tikz\pgfuseplotmark{m5b};};
\node[stars] at (axis cs:{44.777},{-28.907}) {\tikz\pgfuseplotmark{m6c};};
\node[stars] at (axis cs:{44.803},{21.340}) {\tikz\pgfuseplotmark{m5ab};};
\node[stars] at (axis cs:{44.901},{-25.274}) {\tikz\pgfuseplotmark{m6a};};
\node[stars] at (axis cs:{44.910},{-32.507}) {\tikz\pgfuseplotmark{m6c};};
\node[stars] at (axis cs:{44.921},{-2.465}) {\tikz\pgfuseplotmark{m6a};};
\node[stars] at (axis cs:{44.929},{8.907}) {\tikz\pgfuseplotmark{m5a};};
\node[stars] at (axis cs:{45.050},{38.131}) {\tikz\pgfuseplotmark{m6b};};
\node[stars] at (axis cs:{45.184},{10.870}) {\tikz\pgfuseplotmark{m6b};};
\node[stars] at (axis cs:{45.212},{-2.878}) {\tikz\pgfuseplotmark{m6bv};};
\node[stars] at (axis cs:{45.292},{-7.663}) {\tikz\pgfuseplotmark{m6a};};
\node[stars] at (axis cs:{45.407},{-28.091}) {\tikz\pgfuseplotmark{m6b};};
\node[stars] at (axis cs:{45.468},{5.336}) {\tikz\pgfuseplotmark{m6c};};
\node[stars] at (axis cs:{45.476},{26.462}) {\tikz\pgfuseplotmark{m6bv};};
\node[stars] at (axis cs:{45.484},{-9.961}) {\tikz\pgfuseplotmark{m6a};};
\node[stars] at (axis cs:{45.539},{-6.494}) {\tikz\pgfuseplotmark{m6c};};
\node[stars] at (axis cs:{45.570},{4.090}) {\tikz\pgfuseplotmark{m3av};};
\node[stars] at (axis cs:{45.594},{4.353}) {\tikz\pgfuseplotmark{m6a};};
\node[stars] at (axis cs:{45.598},{-23.624}) {\tikz\pgfuseplotmark{m4b};};
\node[stars] at (axis cs:{45.676},{-7.685}) {\tikz\pgfuseplotmark{m5cb};};
\node[stars] at (axis cs:{45.876},{28.270}) {\tikz\pgfuseplotmark{m6c};};
\node[stars] at (axis cs:{46.069},{-7.601}) {\tikz\pgfuseplotmark{m5c};};
\node[stars] at (axis cs:{46.159},{1.863}) {\tikz\pgfuseplotmark{m6b};};
\node[stars] at (axis cs:{46.294},{38.840}) {\tikz\pgfuseplotmark{m3cv};};
\node[stars] at (axis cs:{46.297},{-8.274}) {\tikz\pgfuseplotmark{m6c};};
\node[stars] at (axis cs:{46.361},{25.255}) {\tikz\pgfuseplotmark{m5c};};
\node[stars] at (axis cs:{46.599},{13.187}) {\tikz\pgfuseplotmark{m6a};};
\node[stars] at (axis cs:{46.640},{-6.088}) {\tikz\pgfuseplotmark{m5cv};};
\node[stars] at (axis cs:{46.857},{17.880}) {\tikz\pgfuseplotmark{m6cv};};
\node[stars] at (axis cs:{46.962},{-27.831}) {\tikz\pgfuseplotmark{m6c};};
\node[stars] at (axis cs:{47.088},{18.795}) {\tikz\pgfuseplotmark{m6cv};};
\node[stars] at (axis cs:{47.162},{8.471}) {\tikz\pgfuseplotmark{m6c};};
\node[stars] at (axis cs:{47.334},{20.761}) {\tikz\pgfuseplotmark{m6c};};
\node[stars] at (axis cs:{47.403},{29.077}) {\tikz\pgfuseplotmark{m6a};};
\node[stars] at (axis cs:{47.537},{27.820}) {\tikz\pgfuseplotmark{m6c};};
\node[stars] at (axis cs:{47.613},{26.896}) {\tikz\pgfuseplotmark{m6b};};
\node[stars] at (axis cs:{47.647},{-23.738}) {\tikz\pgfuseplotmark{m6c};};
\node[stars] at (axis cs:{47.662},{11.873}) {\tikz\pgfuseplotmark{m6b};};
\node[stars] at (axis cs:{47.772},{-13.264}) {\tikz\pgfuseplotmark{m6c};};
\node[stars] at (axis cs:{47.820},{-16.025}) {\tikz\pgfuseplotmark{m6c};};
\node[stars] at (axis cs:{47.822},{39.611}) {\tikz\pgfuseplotmark{m5a};};
\node[stars] at (axis cs:{47.828},{-3.811}) {\tikz\pgfuseplotmark{m6b};};
\node[stars] at (axis cs:{47.841},{13.048}) {\tikz\pgfuseplotmark{m6b};};
\node[stars] at (axis cs:{47.907},{19.727}) {\tikz\pgfuseplotmark{m4cv};};
\node[stars] at (axis cs:{48.019},{-28.988}) {\tikz\pgfuseplotmark{m4ab};};
\node[stars] at (axis cs:{48.059},{27.257}) {\tikz\pgfuseplotmark{m6av};};
\node[stars] at (axis cs:{48.110},{6.661}) {\tikz\pgfuseplotmark{m6av};};
\node[stars] at (axis cs:{48.193},{-1.196}) {\tikz\pgfuseplotmark{m5bb};};
\node[stars] at (axis cs:{48.256},{-35.944}) {\tikz\pgfuseplotmark{m6cv};};
\node[stars] at (axis cs:{48.408},{-29.804}) {\tikz\pgfuseplotmark{m6c};};
\node[stars] at (axis cs:{48.725},{21.044}) {\tikz\pgfuseplotmark{m5b};};
\node[stars] at (axis cs:{48.751},{-26.100}) {\tikz\pgfuseplotmark{m6c};};
\node[stars] at (axis cs:{48.835},{30.557}) {\tikz\pgfuseplotmark{m6a};};
\node[stars] at (axis cs:{48.946},{32.857}) {\tikz\pgfuseplotmark{m6c};};
\node[stars] at (axis cs:{48.958},{-8.820}) {\tikz\pgfuseplotmark{m5av};};
\node[stars] at (axis cs:{49.004},{-5.919}) {\tikz\pgfuseplotmark{m6c};};
\node[stars] at (axis cs:{49.008},{34.688}) {\tikz\pgfuseplotmark{m6cv};};
\node[stars] at (axis cs:{49.091},{-5.730}) {\tikz\pgfuseplotmark{m6cv};};
\node[stars] at (axis cs:{49.147},{32.184}) {\tikz\pgfuseplotmark{m6bv};};
\node[stars] at (axis cs:{49.149},{-9.155}) {\tikz\pgfuseplotmark{m6c};};
\node[stars] at (axis cs:{49.417},{31.127}) {\tikz\pgfuseplotmark{m6c};};
\node[stars] at (axis cs:{49.441},{39.283}) {\tikz\pgfuseplotmark{m6b};};
\node[stars] at (axis cs:{49.511},{-28.797}) {\tikz\pgfuseplotmark{m6b};};
\node[stars] at (axis cs:{49.592},{-22.511}) {\tikz\pgfuseplotmark{m5b};};
\node[stars] at (axis cs:{49.593},{-0.930}) {\tikz\pgfuseplotmark{m5cb};};
\node[stars] at (axis cs:{49.671},{-18.560}) {\tikz\pgfuseplotmark{m6ab};};
\node[stars] at (axis cs:{49.683},{34.223}) {\tikz\pgfuseplotmark{m5a};};
\node[stars] at (axis cs:{49.840},{3.370}) {\tikz\pgfuseplotmark{m5avb};};
\node[stars] at (axis cs:{49.879},{-21.758}) {\tikz\pgfuseplotmark{m4avb};};
\node[stars] at (axis cs:{49.895},{-24.123}) {\tikz\pgfuseplotmark{m6av};};
\node[stars] at (axis cs:{49.982},{27.071}) {\tikz\pgfuseplotmark{m6b};};
\node[stars] at (axis cs:{50.085},{29.048}) {\tikz\pgfuseplotmark{m4c};};
\node[stars] at (axis cs:{50.107},{25.663}) {\tikz\pgfuseplotmark{m6c};};
\node[stars] at (axis cs:{50.188},{-26.606}) {\tikz\pgfuseplotmark{m6c};};
\node[stars] at (axis cs:{50.278},{3.675}) {\tikz\pgfuseplotmark{m6a};};
\node[stars] at (axis cs:{50.307},{21.147}) {\tikz\pgfuseplotmark{m5cv};};
\node[stars] at (axis cs:{50.350},{-23.635}) {\tikz\pgfuseplotmark{m5c};};
\node[stars] at (axis cs:{50.550},{27.607}) {\tikz\pgfuseplotmark{m6a};};
\node[stars] at (axis cs:{50.568},{-25.588}) {\tikz\pgfuseplotmark{m6c};};
\node[stars] at (axis cs:{50.689},{20.742}) {\tikz\pgfuseplotmark{m5b};};
\node[stars] at (axis cs:{50.824},{-7.794}) {\tikz\pgfuseplotmark{m6c};};
\node[stars] at (axis cs:{50.901},{0.910}) {\tikz\pgfuseplotmark{m6c};};
\node[stars] at (axis cs:{50.912},{4.882}) {\tikz\pgfuseplotmark{m6c};};
\node[stars] at (axis cs:{51.042},{12.629}) {\tikz\pgfuseplotmark{m6b};};
\node[stars] at (axis cs:{51.077},{24.724}) {\tikz\pgfuseplotmark{m5c};};
\node[stars] at (axis cs:{51.109},{20.803}) {\tikz\pgfuseplotmark{m6b};};
\node[stars] at (axis cs:{51.124},{33.536}) {\tikz\pgfuseplotmark{m6avb};};
\node[stars] at (axis cs:{51.203},{9.029}) {\tikz\pgfuseplotmark{m4av};};
\node[stars] at (axis cs:{51.483},{-35.921}) {\tikz\pgfuseplotmark{m6c};};
\node[stars] at (axis cs:{51.594},{-27.317}) {\tikz\pgfuseplotmark{m6b};};
\node[stars] at (axis cs:{51.647},{28.715}) {\tikz\pgfuseplotmark{m6cv};};
\node[stars] at (axis cs:{51.792},{9.732}) {\tikz\pgfuseplotmark{m4av};};
\node[stars] at (axis cs:{51.828},{12.735}) {\tikz\pgfuseplotmark{m6c};};
\node[stars] at (axis cs:{51.889},{-35.681}) {\tikz\pgfuseplotmark{m6a};};
\node[stars] at (axis cs:{52.004},{-11.287}) {\tikz\pgfuseplotmark{m6a};};
\node[stars] at (axis cs:{52.086},{33.807}) {\tikz\pgfuseplotmark{m6a};};
\node[stars] at (axis cs:{52.111},{22.804}) {\tikz\pgfuseplotmark{m6b};};
\node[stars] at (axis cs:{52.266},{3.248}) {\tikz\pgfuseplotmark{m6c};};
\node[stars] at (axis cs:{52.400},{-12.675}) {\tikz\pgfuseplotmark{m6a};};
\node[stars] at (axis cs:{52.413},{-6.804}) {\tikz\pgfuseplotmark{m6bv};};
\node[stars] at (axis cs:{52.602},{11.336}) {\tikz\pgfuseplotmark{m5b};};
\node[stars] at (axis cs:{52.654},{-5.075}) {\tikz\pgfuseplotmark{m5a};};
\node[stars] at (axis cs:{52.689},{6.189}) {\tikz\pgfuseplotmark{m6b};};
\node[stars] at (axis cs:{52.718},{12.937}) {\tikz\pgfuseplotmark{m4b};};
\node[stars] at (axis cs:{52.974},{-25.614}) {\tikz\pgfuseplotmark{m6c};};
\node[stars] at (axis cs:{53.150},{9.373}) {\tikz\pgfuseplotmark{m6a};};
\node[stars] at (axis cs:{53.167},{35.462}) {\tikz\pgfuseplotmark{m6b};};
\node[stars] at (axis cs:{53.233},{-9.458}) {\tikz\pgfuseplotmark{m4av};};
\node[stars] at (axis cs:{53.396},{39.899}) {\tikz\pgfuseplotmark{m6av};};
\node[stars] at (axis cs:{53.447},{-21.633}) {\tikz\pgfuseplotmark{m4c};};
\node[stars] at (axis cs:{53.487},{-31.080}) {\tikz\pgfuseplotmark{m6c};};
\node[stars] at (axis cs:{53.535},{17.833}) {\tikz\pgfuseplotmark{m6c};};
\node[stars] at (axis cs:{53.611},{24.464}) {\tikz\pgfuseplotmark{m6bvb};};
\node[stars] at (axis cs:{53.640},{-31.875}) {\tikz\pgfuseplotmark{m6cb};};
\node[stars] at (axis cs:{53.656},{-9.868}) {\tikz\pgfuseplotmark{m6c};};
\node[stars] at (axis cs:{53.705},{6.418}) {\tikz\pgfuseplotmark{m6c};};
\node[stars] at (axis cs:{53.718},{-25.583}) {\tikz\pgfuseplotmark{m6c};};
\node[stars] at (axis cs:{53.990},{-11.194}) {\tikz\pgfuseplotmark{m6a};};
\node[stars] at (axis cs:{54.073},{-17.467}) {\tikz\pgfuseplotmark{m5cv};};
\node[stars] at (axis cs:{54.197},{0.588}) {\tikz\pgfuseplotmark{m6bvb};};
\node[stars] at (axis cs:{54.218},{0.402}) {\tikz\pgfuseplotmark{m4c};};
\node[stars] at (axis cs:{54.449},{15.430}) {\tikz\pgfuseplotmark{m6c};};
\node[stars] at (axis cs:{54.622},{-7.392}) {\tikz\pgfuseplotmark{m6a};};
\node[stars] at (axis cs:{54.699},{-27.943}) {\tikz\pgfuseplotmark{m6b};};
\node[stars] at (axis cs:{54.750},{20.916}) {\tikz\pgfuseplotmark{m6c};};
\node[stars] at (axis cs:{54.755},{-5.626}) {\tikz\pgfuseplotmark{m6b};};
\node[stars] at (axis cs:{54.856},{-10.437}) {\tikz\pgfuseplotmark{m6c};};
\node[stars] at (axis cs:{54.857},{16.537}) {\tikz\pgfuseplotmark{m6c};};
\node[stars] at (axis cs:{54.910},{-3.393}) {\tikz\pgfuseplotmark{m6c};};
\node[stars] at (axis cs:{54.963},{3.057}) {\tikz\pgfuseplotmark{m6a};};
\node[stars] at (axis cs:{54.998},{-1.121}) {\tikz\pgfuseplotmark{m6b};};
\node[stars] at (axis cs:{55.048},{-15.226}) {\tikz\pgfuseplotmark{m6c};};
\node[stars] at (axis cs:{55.160},{-5.211}) {\tikz\pgfuseplotmark{m6av};};
\node[stars] at (axis cs:{55.193},{25.329}) {\tikz\pgfuseplotmark{m6b};};
\node[stars] at (axis cs:{55.283},{37.580}) {\tikz\pgfuseplotmark{m6av};};
\node[stars] at (axis cs:{55.307},{-11.802}) {\tikz\pgfuseplotmark{m6c};};
\node[stars] at (axis cs:{55.562},{-31.938}) {\tikz\pgfuseplotmark{m5b};};
\node[stars] at (axis cs:{55.579},{19.700}) {\tikz\pgfuseplotmark{m6a};};
\node[stars] at (axis cs:{55.594},{33.965}) {\tikz\pgfuseplotmark{m5b};};
\node[stars] at (axis cs:{55.709},{-37.313}) {\tikz\pgfuseplotmark{m5a};};
\node[stars] at (axis cs:{55.812},{-9.763}) {\tikz\pgfuseplotmark{m4av};};
\node[stars] at (axis cs:{55.891},{-10.486}) {\tikz\pgfuseplotmark{m6a};};
\node[stars] at (axis cs:{55.947},{19.665}) {\tikz\pgfuseplotmark{m6c};};
\node[stars] at (axis cs:{56.080},{32.288}) {\tikz\pgfuseplotmark{m4avb};};
\node[stars] at (axis cs:{56.118},{20.929}) {\tikz\pgfuseplotmark{m6b};};
\node[stars] at (axis cs:{56.127},{-1.163}) {\tikz\pgfuseplotmark{m5c};};
\node[stars] at (axis cs:{56.131},{36.460}) {\tikz\pgfuseplotmark{m6a};};
\node[stars] at (axis cs:{56.201},{24.289}) {\tikz\pgfuseplotmark{m5c};};
\node[stars] at (axis cs:{56.219},{24.113}) {\tikz\pgfuseplotmark{m4a};};
\node[stars] at (axis cs:{56.235},{-0.297}) {\tikz\pgfuseplotmark{m6a};};
\node[stars] at (axis cs:{56.291},{24.839}) {\tikz\pgfuseplotmark{m6a};};
\node[stars] at (axis cs:{56.302},{24.467}) {\tikz\pgfuseplotmark{m4c};};
\node[stars] at (axis cs:{56.405},{38.676}) {\tikz\pgfuseplotmark{m6c};};
\node[stars] at (axis cs:{56.419},{6.050}) {\tikz\pgfuseplotmark{m5c};};
\node[stars] at (axis cs:{56.457},{24.368}) {\tikz\pgfuseplotmark{m4b};};
\node[stars] at (axis cs:{56.477},{24.554}) {\tikz\pgfuseplotmark{m6a};};
\node[stars] at (axis cs:{56.512},{24.528}) {\tikz\pgfuseplotmark{m6c};};
\node[stars] at (axis cs:{56.536},{-12.102}) {\tikz\pgfuseplotmark{m4cv};};
\node[stars] at (axis cs:{56.539},{6.803}) {\tikz\pgfuseplotmark{m6b};};
\node[stars] at (axis cs:{56.582},{23.948}) {\tikz\pgfuseplotmark{m4cv};};
\node[stars] at (axis cs:{56.614},{-29.338}) {\tikz\pgfuseplotmark{m6b};};
\node[stars] at (axis cs:{56.712},{-23.249}) {\tikz\pgfuseplotmark{m4c};};
\node[stars] at (axis cs:{56.871},{24.105}) {\tikz\pgfuseplotmark{m3bvb};};
\node[stars] at (axis cs:{56.915},{-23.875}) {\tikz\pgfuseplotmark{m5cv};};
\node[stars] at (axis cs:{56.954},{32.195}) {\tikz\pgfuseplotmark{m6c};};
\node[stars] at (axis cs:{56.957},{-36.106}) {\tikz\pgfuseplotmark{m6c};};
\node[stars] at (axis cs:{56.984},{-30.168}) {\tikz\pgfuseplotmark{m6av};};
\node[stars] at (axis cs:{57.027},{24.988}) {\tikz\pgfuseplotmark{m6c};};
\node[stars] at (axis cs:{57.068},{11.143}) {\tikz\pgfuseplotmark{m5b};};
\node[stars] at (axis cs:{57.087},{23.421}) {\tikz\pgfuseplotmark{m5c};};
\node[stars] at (axis cs:{57.149},{-20.903}) {\tikz\pgfuseplotmark{m6a};};
\node[stars] at (axis cs:{57.149},{-37.620}) {\tikz\pgfuseplotmark{m5ab};};
\node[stars] at (axis cs:{57.162},{0.228}) {\tikz\pgfuseplotmark{m6b};};
\node[stars] at (axis cs:{57.237},{23.857}) {\tikz\pgfuseplotmark{m6c};};
\node[stars] at (axis cs:{57.291},{24.053}) {\tikz\pgfuseplotmark{m4avb};};
\node[stars] at (axis cs:{57.297},{24.137}) {\tikz\pgfuseplotmark{m5bv};};
\node[stars] at (axis cs:{57.364},{-36.200}) {\tikz\pgfuseplotmark{m4c};};
\node[stars] at (axis cs:{57.386},{33.091}) {\tikz\pgfuseplotmark{m5cv};};
\node[stars] at (axis cs:{57.431},{23.712}) {\tikz\pgfuseplotmark{m6cv};};
\node[stars] at (axis cs:{57.479},{22.244}) {\tikz\pgfuseplotmark{m6bv};};
\node[stars] at (axis cs:{57.567},{-1.522}) {\tikz\pgfuseplotmark{m6cb};};
\node[stars] at (axis cs:{57.579},{25.579}) {\tikz\pgfuseplotmark{m5c};};
\node[stars] at (axis cs:{57.816},{13.046}) {\tikz\pgfuseplotmark{m6cv};};
\node[stars] at (axis cs:{57.974},{34.359}) {\tikz\pgfuseplotmark{m6a};};
\node[stars] at (axis cs:{58.001},{6.535}) {\tikz\pgfuseplotmark{m6a};};
\node[stars] at (axis cs:{58.019},{31.169}) {\tikz\pgfuseplotmark{m6c};};
\node[stars] at (axis cs:{58.174},{-5.361}) {\tikz\pgfuseplotmark{m5c};};
\node[stars] at (axis cs:{58.292},{17.327}) {\tikz\pgfuseplotmark{m6bv};};
\node[stars] at (axis cs:{58.304},{-18.434}) {\tikz\pgfuseplotmark{m6c};};
\node[stars] at (axis cs:{58.394},{25.683}) {\tikz\pgfuseplotmark{m6cv};};
\node[stars] at (axis cs:{58.412},{-34.732}) {\tikz\pgfuseplotmark{m5b};};
\node[stars] at (axis cs:{58.428},{-24.612}) {\tikz\pgfuseplotmark{m5av};};
\node[stars] at (axis cs:{58.533},{31.884}) {\tikz\pgfuseplotmark{m3bvb};};
\node[stars] at (axis cs:{58.573},{-2.955}) {\tikz\pgfuseplotmark{m5ab};};
\node[stars] at (axis cs:{58.817},{-12.099}) {\tikz\pgfuseplotmark{m6bv};};
\node[stars] at (axis cs:{59.115},{-13.597}) {\tikz\pgfuseplotmark{m6cv};};
\node[stars] at (axis cs:{59.120},{35.081}) {\tikz\pgfuseplotmark{m5cv};};
\node[stars] at (axis cs:{59.158},{-9.751}) {\tikz\pgfuseplotmark{m6cv};};
\node[stars] at (axis cs:{59.217},{22.478}) {\tikz\pgfuseplotmark{m6a};};
\node[stars] at (axis cs:{59.257},{6.040}) {\tikz\pgfuseplotmark{m6b};};
\node[stars] at (axis cs:{59.266},{23.175}) {\tikz\pgfuseplotmark{m6bv};};
\node[stars] at (axis cs:{59.360},{24.462}) {\tikz\pgfuseplotmark{m6c};};
\node[stars] at (axis cs:{59.507},{-13.508}) {\tikz\pgfuseplotmark{m3bv};};
\node[stars] at (axis cs:{59.622},{38.840}) {\tikz\pgfuseplotmark{m6c};};
\node[stars] at (axis cs:{59.718},{-5.470}) {\tikz\pgfuseplotmark{m6a};};
\node[stars] at (axis cs:{59.741},{35.791}) {\tikz\pgfuseplotmark{m4bv};};
\node[stars] at (axis cs:{59.876},{-12.574}) {\tikz\pgfuseplotmark{m6av};};
\node[stars] at (axis cs:{59.919},{10.330}) {\tikz\pgfuseplotmark{m6c};};
\node[stars] at (axis cs:{59.981},{-24.016}) {\tikz\pgfuseplotmark{m5av};};
\node[stars] at (axis cs:{60.154},{17.296}) {\tikz\pgfuseplotmark{m6c};};
\node[stars] at (axis cs:{60.169},{-30.491}) {\tikz\pgfuseplotmark{m6b};};
\node[stars] at (axis cs:{60.170},{12.490}) {\tikz\pgfuseplotmark{m3cv};};
\node[stars] at (axis cs:{60.203},{18.194}) {\tikz\pgfuseplotmark{m6b};};
\node[stars] at (axis cs:{60.312},{36.989}) {\tikz\pgfuseplotmark{m6c};};
\node[stars] at (axis cs:{60.384},{-1.549}) {\tikz\pgfuseplotmark{m5cv};};
\node[stars] at (axis cs:{60.442},{9.998}) {\tikz\pgfuseplotmark{m6a};};
\node[stars] at (axis cs:{60.653},{-0.269}) {\tikz\pgfuseplotmark{m5c};};
\node[stars] at (axis cs:{60.789},{5.989}) {\tikz\pgfuseplotmark{m4b};};
\node[stars] at (axis cs:{60.853},{-20.144}) {\tikz\pgfuseplotmark{m6c};};
\node[stars] at (axis cs:{60.936},{5.436}) {\tikz\pgfuseplotmark{m5cv};};
\node[stars] at (axis cs:{60.986},{8.197}) {\tikz\pgfuseplotmark{m5c};};
\node[stars] at (axis cs:{61.036},{-16.589}) {\tikz\pgfuseplotmark{m6c};};
\node[stars] at (axis cs:{61.041},{2.827}) {\tikz\pgfuseplotmark{m5c};};
\node[stars] at (axis cs:{61.090},{24.106}) {\tikz\pgfuseplotmark{m6a};};
\node[stars] at (axis cs:{61.095},{-12.792}) {\tikz\pgfuseplotmark{m6av};};
\node[stars] at (axis cs:{61.170},{-20.382}) {\tikz\pgfuseplotmark{m6b};};
\node[stars] at (axis cs:{61.174},{22.082}) {\tikz\pgfuseplotmark{m4c};};
\node[stars] at (axis cs:{61.334},{22.009}) {\tikz\pgfuseplotmark{m6bb};};
\node[stars] at (axis cs:{61.406},{-27.652}) {\tikz\pgfuseplotmark{m6a};};
\node[stars] at (axis cs:{61.445},{-20.512}) {\tikz\pgfuseplotmark{m6c};};
\node[stars] at (axis cs:{61.485},{-8.856}) {\tikz\pgfuseplotmark{m6c};};
\node[stars] at (axis cs:{61.652},{27.600}) {\tikz\pgfuseplotmark{m5cv};};
\node[stars] at (axis cs:{61.752},{29.001}) {\tikz\pgfuseplotmark{m5c};};
\node[stars] at (axis cs:{61.925},{15.163}) {\tikz\pgfuseplotmark{m6bvb};};
\node[stars] at (axis cs:{61.998},{17.340}) {\tikz\pgfuseplotmark{m6bb};};
\node[stars] at (axis cs:{62.064},{37.727}) {\tikz\pgfuseplotmark{m6b};};
\node[stars] at (axis cs:{62.153},{38.040}) {\tikz\pgfuseplotmark{m6av};};
\node[stars] at (axis cs:{62.257},{13.398}) {\tikz\pgfuseplotmark{m6b};};
\node[stars] at (axis cs:{62.292},{19.609}) {\tikz\pgfuseplotmark{m6a};};
\node[stars] at (axis cs:{62.324},{-16.386}) {\tikz\pgfuseplotmark{m5cv};};
\node[stars] at (axis cs:{62.429},{3.323}) {\tikz\pgfuseplotmark{m6c};};
\node[stars] at (axis cs:{62.594},{-6.924}) {\tikz\pgfuseplotmark{m5c};};
\node[stars] at (axis cs:{62.691},{-35.274}) {\tikz\pgfuseplotmark{m6c};};
\node[stars] at (axis cs:{62.699},{-8.820}) {\tikz\pgfuseplotmark{m6a};};
\node[stars] at (axis cs:{62.708},{26.481}) {\tikz\pgfuseplotmark{m5cv};};
\node[stars] at (axis cs:{62.746},{33.587}) {\tikz\pgfuseplotmark{m6a};};
\node[stars] at (axis cs:{62.835},{5.523}) {\tikz\pgfuseplotmark{m6a};};
\node[stars] at (axis cs:{62.901},{-20.356}) {\tikz\pgfuseplotmark{m6av};};
\node[stars] at (axis cs:{62.966},{-6.838}) {\tikz\pgfuseplotmark{m4bv};};
\node[stars] at (axis cs:{62.984},{-8.837}) {\tikz\pgfuseplotmark{m6cb};};
\node[stars] at (axis cs:{63.001},{-17.275}) {\tikz\pgfuseplotmark{m6c};};
\node[stars] at (axis cs:{63.131},{17.277}) {\tikz\pgfuseplotmark{m6b};};
\node[stars] at (axis cs:{63.214},{22.413}) {\tikz\pgfuseplotmark{m6cv};};
\node[stars] at (axis cs:{63.388},{7.716}) {\tikz\pgfuseplotmark{m5c};};
\node[stars] at (axis cs:{63.394},{10.212}) {\tikz\pgfuseplotmark{m6c};};
\node[stars] at (axis cs:{63.409},{-1.149}) {\tikz\pgfuseplotmark{m6cv};};
\node[stars] at (axis cs:{63.458},{12.754}) {\tikz\pgfuseplotmark{m6c};};
\node[stars] at (axis cs:{63.485},{9.264}) {\tikz\pgfuseplotmark{m5bb};};
\node[stars] at (axis cs:{63.499},{37.967}) {\tikz\pgfuseplotmark{m6c};};
\node[stars] at (axis cs:{63.599},{-10.256}) {\tikz\pgfuseplotmark{m5bb};};
\node[stars] at (axis cs:{63.651},{10.011}) {\tikz\pgfuseplotmark{m5c};};
\node[stars] at (axis cs:{63.717},{37.543}) {\tikz\pgfuseplotmark{m6c};};
\node[stars] at (axis cs:{63.818},{-7.653}) {\tikz\pgfuseplotmark{m4cb};};
\node[stars] at (axis cs:{63.870},{6.187}) {\tikz\pgfuseplotmark{m6cvb};};
\node[stars] at (axis cs:{63.884},{8.892}) {\tikz\pgfuseplotmark{m4c};};
\node[stars] at (axis cs:{63.943},{15.401}) {\tikz\pgfuseplotmark{m6cv};};
\node[stars] at (axis cs:{64.315},{20.578}) {\tikz\pgfuseplotmark{m5b};};
\node[stars] at (axis cs:{64.330},{-6.472}) {\tikz\pgfuseplotmark{m6b};};
\node[stars] at (axis cs:{64.474},{-33.798}) {\tikz\pgfuseplotmark{m4a};};
\node[stars] at (axis cs:{64.567},{-20.715}) {\tikz\pgfuseplotmark{m6bv};};
\node[stars] at (axis cs:{64.597},{21.579}) {\tikz\pgfuseplotmark{m6a};};
\node[stars] at (axis cs:{64.656},{-22.970}) {\tikz\pgfuseplotmark{m6b};};
\node[stars] at (axis cs:{64.859},{21.142}) {\tikz\pgfuseplotmark{m5cv};};
\node[stars] at (axis cs:{64.903},{21.773}) {\tikz\pgfuseplotmark{m5cv};};
\node[stars] at (axis cs:{64.906},{10.121}) {\tikz\pgfuseplotmark{m6c};};
\node[stars] at (axis cs:{64.948},{15.627}) {\tikz\pgfuseplotmark{m4av};};
\node[stars] at (axis cs:{64.990},{14.035}) {\tikz\pgfuseplotmark{m6av};};
\node[stars] at (axis cs:{65.041},{31.953}) {\tikz\pgfuseplotmark{m6c};};
\node[stars] at (axis cs:{65.088},{27.351}) {\tikz\pgfuseplotmark{m5bb};};
\node[stars] at (axis cs:{65.103},{34.567}) {\tikz\pgfuseplotmark{m5b};};
\node[stars] at (axis cs:{65.105},{18.742}) {\tikz\pgfuseplotmark{m6b};};
\node[stars] at (axis cs:{65.151},{15.095}) {\tikz\pgfuseplotmark{m5cv};};
\node[stars] at (axis cs:{65.161},{-6.246}) {\tikz\pgfuseplotmark{m6cv};};
\node[stars] at (axis cs:{65.163},{-20.640}) {\tikz\pgfuseplotmark{m5c};};
\node[stars] at (axis cs:{65.172},{6.131}) {\tikz\pgfuseplotmark{m6a};};
\node[stars] at (axis cs:{65.178},{-7.592}) {\tikz\pgfuseplotmark{m6av};};
\node[stars] at (axis cs:{65.220},{13.864}) {\tikz\pgfuseplotmark{m6c};};
\node[stars] at (axis cs:{65.363},{-0.098}) {\tikz\pgfuseplotmark{m6a};};
\node[stars] at (axis cs:{65.380},{-25.728}) {\tikz\pgfuseplotmark{m6bvb};};
\node[stars] at (axis cs:{65.406},{-6.284}) {\tikz\pgfuseplotmark{m6c};};
\node[stars] at (axis cs:{65.515},{14.077}) {\tikz\pgfuseplotmark{m6av};};
\node[stars] at (axis cs:{65.595},{20.821}) {\tikz\pgfuseplotmark{m6bvb};};
\node[stars] at (axis cs:{65.646},{25.629}) {\tikz\pgfuseplotmark{m5cb};};
\node[stars] at (axis cs:{65.734},{17.542}) {\tikz\pgfuseplotmark{m4av};};
\node[stars] at (axis cs:{65.774},{-24.892}) {\tikz\pgfuseplotmark{m6av};};
\node[stars] at (axis cs:{65.782},{-35.545}) {\tikz\pgfuseplotmark{m6c};};
\node[stars] at (axis cs:{65.854},{16.777}) {\tikz\pgfuseplotmark{m6av};};
\node[stars] at (axis cs:{65.885},{20.982}) {\tikz\pgfuseplotmark{m6bv};};
\node[stars] at (axis cs:{65.920},{-3.745}) {\tikz\pgfuseplotmark{m5cv};};
\node[stars] at (axis cs:{65.966},{9.461}) {\tikz\pgfuseplotmark{m5b};};
\node[stars] at (axis cs:{65.999},{24.301}) {\tikz\pgfuseplotmark{m6cb};};
\node[stars] at (axis cs:{66.009},{-34.017}) {\tikz\pgfuseplotmark{m4b};};
\node[stars] at (axis cs:{66.024},{17.444}) {\tikz\pgfuseplotmark{m5a};};
\node[stars] at (axis cs:{66.060},{12.158}) {\tikz\pgfuseplotmark{m6c};};
\node[stars] at (axis cs:{66.121},{34.131}) {\tikz\pgfuseplotmark{m6a};};
\node[stars] at (axis cs:{66.156},{33.960}) {\tikz\pgfuseplotmark{m6ab};};
\node[stars] at (axis cs:{66.238},{19.042}) {\tikz\pgfuseplotmark{m6b};};
\node[stars] at (axis cs:{66.342},{22.294}) {\tikz\pgfuseplotmark{m4cv};};
\node[stars] at (axis cs:{66.354},{22.200}) {\tikz\pgfuseplotmark{m5cv};};
\node[stars] at (axis cs:{66.372},{17.928}) {\tikz\pgfuseplotmark{m4cvb};};
\node[stars] at (axis cs:{66.405},{15.941}) {\tikz\pgfuseplotmark{m6cv};};
\node[stars] at (axis cs:{66.506},{4.373}) {\tikz\pgfuseplotmark{m6c};};
\node[stars] at (axis cs:{66.526},{31.439}) {\tikz\pgfuseplotmark{m5c};};
\node[stars] at (axis cs:{66.577},{22.813}) {\tikz\pgfuseplotmark{m4cv};};
\node[stars] at (axis cs:{66.586},{15.618}) {\tikz\pgfuseplotmark{m4cv};};
\node[stars] at (axis cs:{66.588},{8.590}) {\tikz\pgfuseplotmark{m6bv};};
\node[stars] at (axis cs:{66.652},{14.714}) {\tikz\pgfuseplotmark{m5a};};
\node[stars] at (axis cs:{66.737},{-24.081}) {\tikz\pgfuseplotmark{m6b};};
\node[stars] at (axis cs:{66.753},{2.079}) {\tikz\pgfuseplotmark{m6c};};
\node[stars] at (axis cs:{66.823},{22.996}) {\tikz\pgfuseplotmark{m6a};};
\node[stars] at (axis cs:{66.870},{11.212}) {\tikz\pgfuseplotmark{m6b};};
\node[stars] at (axis cs:{67.003},{21.620}) {\tikz\pgfuseplotmark{m6a};};
\node[stars] at (axis cs:{67.015},{1.858}) {\tikz\pgfuseplotmark{m6c};};
\node[stars] at (axis cs:{67.098},{14.741}) {\tikz\pgfuseplotmark{m6b};};
\node[stars] at (axis cs:{67.110},{16.360}) {\tikz\pgfuseplotmark{m5b};};
\node[stars] at (axis cs:{67.134},{1.381}) {\tikz\pgfuseplotmark{m6a};};
\node[stars] at (axis cs:{67.154},{19.180}) {\tikz\pgfuseplotmark{m4a};};
\node[stars] at (axis cs:{67.163},{-19.459}) {\tikz\pgfuseplotmark{m6b};};
\node[stars] at (axis cs:{67.166},{15.871}) {\tikz\pgfuseplotmark{m3cvb};};
\node[stars] at (axis cs:{67.209},{13.047}) {\tikz\pgfuseplotmark{m5bv};};
\node[stars] at (axis cs:{67.217},{30.361}) {\tikz\pgfuseplotmark{m6cb};};
\node[stars] at (axis cs:{67.279},{-13.048}) {\tikz\pgfuseplotmark{m6av};};
\node[stars] at (axis cs:{67.510},{10.262}) {\tikz\pgfuseplotmark{m6c};};
\node[stars] at (axis cs:{67.536},{15.638}) {\tikz\pgfuseplotmark{m6ab};};
\node[stars] at (axis cs:{67.542},{-13.592}) {\tikz\pgfuseplotmark{m6c};};
\node[stars] at (axis cs:{67.640},{16.194}) {\tikz\pgfuseplotmark{m5avb};};
\node[stars] at (axis cs:{67.656},{13.724}) {\tikz\pgfuseplotmark{m5cb};};
\node[stars] at (axis cs:{67.660},{32.458}) {\tikz\pgfuseplotmark{m6c};};
\node[stars] at (axis cs:{67.662},{15.692}) {\tikz\pgfuseplotmark{m5cvb};};
\node[stars] at (axis cs:{67.668},{-35.653}) {\tikz\pgfuseplotmark{m6b};};
\node[stars] at (axis cs:{67.780},{15.105}) {\tikz\pgfuseplotmark{m6cv};};
\node[stars] at (axis cs:{67.858},{-13.644}) {\tikz\pgfuseplotmark{m6cb};};
\node[stars] at (axis cs:{67.966},{15.852}) {\tikz\pgfuseplotmark{m6b};};
\node[stars] at (axis cs:{67.969},{-0.044}) {\tikz\pgfuseplotmark{m5b};};
\node[stars] at (axis cs:{67.987},{36.743}) {\tikz\pgfuseplotmark{m6cv};};
\node[stars] at (axis cs:{68.020},{5.410}) {\tikz\pgfuseplotmark{m6c};};
\node[stars] at (axis cs:{68.156},{-3.209}) {\tikz\pgfuseplotmark{m6av};};
\node[stars] at (axis cs:{68.342},{-10.785}) {\tikz\pgfuseplotmark{m6b};};
\node[stars] at (axis cs:{68.377},{-29.766}) {\tikz\pgfuseplotmark{m5a};};
\node[stars] at (axis cs:{68.388},{18.017}) {\tikz\pgfuseplotmark{m6cb};};
\node[stars] at (axis cs:{68.451},{9.413}) {\tikz\pgfuseplotmark{m6b};};
\node[stars] at (axis cs:{68.462},{14.844}) {\tikz\pgfuseplotmark{m5av};};
\node[stars] at (axis cs:{68.478},{-6.739}) {\tikz\pgfuseplotmark{m6avb};};
\node[stars] at (axis cs:{68.534},{5.569}) {\tikz\pgfuseplotmark{m6a};};
\node[stars] at (axis cs:{68.548},{-8.231}) {\tikz\pgfuseplotmark{m5cv};};
\node[stars] at (axis cs:{68.549},{-8.970}) {\tikz\pgfuseplotmark{m5c};};
\node[stars] at (axis cs:{68.559},{-6.838}) {\tikz\pgfuseplotmark{m6b};};
\node[stars] at (axis cs:{68.658},{28.961}) {\tikz\pgfuseplotmark{m6b};};
\node[stars] at (axis cs:{68.659},{-24.040}) {\tikz\pgfuseplotmark{m6c};};
\node[stars] at (axis cs:{68.753},{-19.921}) {\tikz\pgfuseplotmark{m6b};};
\node[stars] at (axis cs:{68.888},{-30.562}) {\tikz\pgfuseplotmark{m4a};};
\node[stars] at (axis cs:{68.914},{10.161}) {\tikz\pgfuseplotmark{m4cvb};};
\node[stars] at (axis cs:{68.927},{19.882}) {\tikz\pgfuseplotmark{m6c};};
\node[stars] at (axis cs:{68.980},{16.509}) {\tikz\pgfuseplotmark{m1bvb};};
\node[stars] at (axis cs:{69.007},{-3.611}) {\tikz\pgfuseplotmark{m6cv};};
\node[stars] at (axis cs:{69.080},{-3.352}) {\tikz\pgfuseplotmark{m4bv};};
\node[stars] at (axis cs:{69.121},{23.341}) {\tikz\pgfuseplotmark{m6bv};};
\node[stars] at (axis cs:{69.212},{-30.717}) {\tikz\pgfuseplotmark{m6c};};
\node[stars] at (axis cs:{69.307},{0.998}) {\tikz\pgfuseplotmark{m5cv};};
\node[stars] at (axis cs:{69.312},{18.543}) {\tikz\pgfuseplotmark{m6cv};};
\node[stars] at (axis cs:{69.401},{-2.473}) {\tikz\pgfuseplotmark{m5c};};
\node[stars] at (axis cs:{69.539},{16.033}) {\tikz\pgfuseplotmark{m6a};};
\node[stars] at (axis cs:{69.539},{12.511}) {\tikz\pgfuseplotmark{m4c};};
\node[stars] at (axis cs:{69.545},{-14.304}) {\tikz\pgfuseplotmark{m4bb};};
\node[stars] at (axis cs:{69.566},{20.685}) {\tikz\pgfuseplotmark{m6av};};
\node[stars] at (axis cs:{69.723},{-12.123}) {\tikz\pgfuseplotmark{m5b};};
\node[stars] at (axis cs:{69.776},{7.871}) {\tikz\pgfuseplotmark{m5cb};};
\node[stars] at (axis cs:{69.819},{15.918}) {\tikz\pgfuseplotmark{m5ab};};
\node[stars] at (axis cs:{69.832},{-14.359}) {\tikz\pgfuseplotmark{m5c};};
\node[stars] at (axis cs:{69.846},{25.218}) {\tikz\pgfuseplotmark{m6c};};
\node[stars] at (axis cs:{69.947},{-1.052}) {\tikz\pgfuseplotmark{m6b};};
\node[stars] at (axis cs:{70.014},{12.197}) {\tikz\pgfuseplotmark{m5cv};};
\node[stars] at (axis cs:{70.028},{-24.482}) {\tikz\pgfuseplotmark{m6a};};
\node[stars] at (axis cs:{70.110},{-19.671}) {\tikz\pgfuseplotmark{m4cv};};
\node[stars] at (axis cs:{70.332},{28.615}) {\tikz\pgfuseplotmark{m6a};};
\node[stars] at (axis cs:{70.459},{38.280}) {\tikz\pgfuseplotmark{m6b};};
\node[stars] at (axis cs:{70.515},{-37.144}) {\tikz\pgfuseplotmark{m5b};};
\node[stars] at (axis cs:{70.561},{22.957}) {\tikz\pgfuseplotmark{m4cb};};
\node[stars] at (axis cs:{70.789},{-30.765}) {\tikz\pgfuseplotmark{m6a};};
\node[stars] at (axis cs:{70.807},{24.089}) {\tikz\pgfuseplotmark{m6c};};
\node[stars] at (axis cs:{71.022},{-8.504}) {\tikz\pgfuseplotmark{m6av};};
\node[stars] at (axis cs:{71.033},{-18.666}) {\tikz\pgfuseplotmark{m6a};};
\node[stars] at (axis cs:{71.108},{11.146}) {\tikz\pgfuseplotmark{m5cb};};
\node[stars] at (axis cs:{71.267},{-21.283}) {\tikz\pgfuseplotmark{m6a};};
\node[stars] at (axis cs:{71.376},{-3.255}) {\tikz\pgfuseplotmark{m4b};};
\node[stars] at (axis cs:{71.427},{23.628}) {\tikz\pgfuseplotmark{m6c};};
\node[stars] at (axis cs:{71.481},{-39.357}) {\tikz\pgfuseplotmark{m6b};};
\node[stars] at (axis cs:{71.507},{11.705}) {\tikz\pgfuseplotmark{m5c};};
\node[stars] at (axis cs:{71.570},{18.735}) {\tikz\pgfuseplotmark{m6b};};
\node[stars] at (axis cs:{71.601},{-2.954}) {\tikz\pgfuseplotmark{m6c};};
\node[stars] at (axis cs:{71.607},{-28.087}) {\tikz\pgfuseplotmark{m6cv};};
\node[stars] at (axis cs:{71.901},{-16.934}) {\tikz\pgfuseplotmark{m5cv};};
\node[stars] at (axis cs:{71.957},{-30.020}) {\tikz\pgfuseplotmark{m6c};};
\node[stars] at (axis cs:{72.136},{-16.329}) {\tikz\pgfuseplotmark{m6av};};
\node[stars] at (axis cs:{72.152},{-5.674}) {\tikz\pgfuseplotmark{m6a};};
\node[stars] at (axis cs:{72.186},{3.588}) {\tikz\pgfuseplotmark{m6b};};
\node[stars] at (axis cs:{72.304},{31.437}) {\tikz\pgfuseplotmark{m6a};};
\node[stars] at (axis cs:{72.329},{32.588}) {\tikz\pgfuseplotmark{m6a};};
\node[stars] at (axis cs:{72.426},{-13.770}) {\tikz\pgfuseplotmark{m6c};};
\node[stars] at (axis cs:{72.434},{15.904}) {\tikz\pgfuseplotmark{m6bb};};
\node[stars] at (axis cs:{72.460},{6.961}) {\tikz\pgfuseplotmark{m3cv};};
\node[stars] at (axis cs:{72.478},{37.488}) {\tikz\pgfuseplotmark{m5b};};
\node[stars] at (axis cs:{72.548},{-16.217}) {\tikz\pgfuseplotmark{m5b};};
\node[stars] at (axis cs:{72.653},{8.900}) {\tikz\pgfuseplotmark{m4c};};
\node[stars] at (axis cs:{72.802},{5.605}) {\tikz\pgfuseplotmark{m4av};};
\node[stars] at (axis cs:{72.844},{18.840}) {\tikz\pgfuseplotmark{m5bv};};
\node[stars] at (axis cs:{72.868},{-34.906}) {\tikz\pgfuseplotmark{m6a};};
\node[stars] at (axis cs:{72.931},{9.975}) {\tikz\pgfuseplotmark{m6b};};
\node[stars] at (axis cs:{72.958},{13.655}) {\tikz\pgfuseplotmark{m6c};};
\node[stars] at (axis cs:{73.133},{14.251}) {\tikz\pgfuseplotmark{m5av};};
\node[stars] at (axis cs:{73.158},{36.703}) {\tikz\pgfuseplotmark{m5a};};
\node[stars] at (axis cs:{73.196},{27.897}) {\tikz\pgfuseplotmark{m6b};};
\node[stars] at (axis cs:{73.224},{-5.453}) {\tikz\pgfuseplotmark{m4c};};
\node[stars] at (axis cs:{73.345},{2.508}) {\tikz\pgfuseplotmark{m5cv};};
\node[stars] at (axis cs:{73.563},{2.441}) {\tikz\pgfuseplotmark{m4av};};
\node[stars] at (axis cs:{73.695},{11.426}) {\tikz\pgfuseplotmark{m5c};};
\node[stars] at (axis cs:{73.699},{7.779}) {\tikz\pgfuseplotmark{m5c};};
\node[stars] at (axis cs:{73.711},{0.467}) {\tikz\pgfuseplotmark{m6b};};
\node[stars] at (axis cs:{73.724},{10.151}) {\tikz\pgfuseplotmark{m5a};};
\node[stars] at (axis cs:{73.728},{-39.628}) {\tikz\pgfuseplotmark{m6b};};
\node[stars] at (axis cs:{73.743},{19.485}) {\tikz\pgfuseplotmark{m6c};};
\node[stars] at (axis cs:{73.778},{-16.741}) {\tikz\pgfuseplotmark{m6a};};
\node[stars] at (axis cs:{73.828},{-16.418}) {\tikz\pgfuseplotmark{m6a};};
\node[stars] at (axis cs:{73.959},{15.040}) {\tikz\pgfuseplotmark{m6a};};
\node[stars] at (axis cs:{74.037},{7.905}) {\tikz\pgfuseplotmark{m6c};};
\node[stars] at (axis cs:{74.065},{24.592}) {\tikz\pgfuseplotmark{m6c};};
\node[stars] at (axis cs:{74.084},{36.169}) {\tikz\pgfuseplotmark{m6b};};
\node[stars] at (axis cs:{74.093},{13.514}) {\tikz\pgfuseplotmark{m4b};};
\node[stars] at (axis cs:{74.101},{-5.171}) {\tikz\pgfuseplotmark{m5cb};};
\node[stars] at (axis cs:{74.248},{33.166}) {\tikz\pgfuseplotmark{m3av};};
\node[stars] at (axis cs:{74.322},{-1.067}) {\tikz\pgfuseplotmark{m6c};};
\node[stars] at (axis cs:{74.343},{17.154}) {\tikz\pgfuseplotmark{m5c};};
\node[stars] at (axis cs:{74.436},{-14.232}) {\tikz\pgfuseplotmark{m6c};};
\node[stars] at (axis cs:{74.453},{23.948}) {\tikz\pgfuseplotmark{m6a};};
\node[stars] at (axis cs:{74.539},{25.050}) {\tikz\pgfuseplotmark{m6a};};
\node[stars] at (axis cs:{74.545},{-2.213}) {\tikz\pgfuseplotmark{m6c};};
\node[stars] at (axis cs:{74.637},{1.714}) {\tikz\pgfuseplotmark{m4cv};};
\node[stars] at (axis cs:{74.748},{14.543}) {\tikz\pgfuseplotmark{m6bvb};};
\node[stars] at (axis cs:{74.756},{-16.376}) {\tikz\pgfuseplotmark{m6av};};
\node[stars] at (axis cs:{74.814},{37.890}) {\tikz\pgfuseplotmark{m5bb};};
\node[stars] at (axis cs:{74.960},{-10.263}) {\tikz\pgfuseplotmark{m5c};};
\node[stars] at (axis cs:{74.982},{-12.537}) {\tikz\pgfuseplotmark{m5av};};
\node[stars] at (axis cs:{75.076},{39.395}) {\tikz\pgfuseplotmark{m6bb};};
\node[stars] at (axis cs:{75.097},{39.655}) {\tikz\pgfuseplotmark{m6c};};
\node[stars] at (axis cs:{75.166},{-2.066}) {\tikz\pgfuseplotmark{m6c};};
\node[stars] at (axis cs:{75.204},{-5.754}) {\tikz\pgfuseplotmark{m6cv};};
\node[stars] at (axis cs:{75.357},{-20.052}) {\tikz\pgfuseplotmark{m5b};};
\node[stars] at (axis cs:{75.360},{-7.174}) {\tikz\pgfuseplotmark{m5av};};
\node[stars] at (axis cs:{75.394},{-39.718}) {\tikz\pgfuseplotmark{m6b};};
\node[stars] at (axis cs:{75.460},{0.722}) {\tikz\pgfuseplotmark{m6b};};
\node[stars] at (axis cs:{75.541},{-26.275}) {\tikz\pgfuseplotmark{m5b};};
\node[stars] at (axis cs:{75.595},{-31.771}) {\tikz\pgfuseplotmark{m6b};};
\node[stars] at (axis cs:{75.687},{-22.795}) {\tikz\pgfuseplotmark{m6a};};
\node[stars] at (axis cs:{75.689},{-4.210}) {\tikz\pgfuseplotmark{m6a};};
\node[stars] at (axis cs:{75.774},{21.590}) {\tikz\pgfuseplotmark{m5a};};
\node[stars] at (axis cs:{75.967},{-14.370}) {\tikz\pgfuseplotmark{m6c};};
\node[stars] at (axis cs:{75.972},{-24.388}) {\tikz\pgfuseplotmark{m6a};};
\node[stars] at (axis cs:{76.061},{30.494}) {\tikz\pgfuseplotmark{m6c};};
\node[stars] at (axis cs:{76.090},{21.278}) {\tikz\pgfuseplotmark{m6c};};
\node[stars] at (axis cs:{76.102},{-35.483}) {\tikz\pgfuseplotmark{m5ab};};
\node[stars] at (axis cs:{76.109},{-35.705}) {\tikz\pgfuseplotmark{m6cv};};
\node[stars] at (axis cs:{76.142},{15.404}) {\tikz\pgfuseplotmark{m5av};};
\node[stars] at (axis cs:{76.227},{-3.040}) {\tikz\pgfuseplotmark{m6b};};
\node[stars] at (axis cs:{76.317},{-26.152}) {\tikz\pgfuseplotmark{m6a};};
\node[stars] at (axis cs:{76.349},{1.177}) {\tikz\pgfuseplotmark{m6bv};};
\node[stars] at (axis cs:{76.365},{-22.371}) {\tikz\pgfuseplotmark{m3cv};};
\node[stars] at (axis cs:{76.653},{-13.123}) {\tikz\pgfuseplotmark{m6b};};
\node[stars] at (axis cs:{76.690},{-4.655}) {\tikz\pgfuseplotmark{m5bv};};
\node[stars] at (axis cs:{76.791},{-19.392}) {\tikz\pgfuseplotmark{m6c};};
\node[stars] at (axis cs:{76.854},{-12.491}) {\tikz\pgfuseplotmark{m6bv};};
\node[stars] at (axis cs:{76.863},{18.645}) {\tikz\pgfuseplotmark{m5b};};
\node[stars] at (axis cs:{76.910},{9.472}) {\tikz\pgfuseplotmark{m6cb};};
\node[stars] at (axis cs:{76.952},{20.418}) {\tikz\pgfuseplotmark{m5c};};
\node[stars] at (axis cs:{76.958},{-12.593}) {\tikz\pgfuseplotmark{m6c};};
\node[stars] at (axis cs:{76.962},{-5.086}) {\tikz\pgfuseplotmark{m3av};};
\node[stars] at (axis cs:{76.970},{8.498}) {\tikz\pgfuseplotmark{m5c};};
\node[stars] at (axis cs:{76.981},{21.705}) {\tikz\pgfuseplotmark{m6avb};};
\node[stars] at (axis cs:{77.028},{24.265}) {\tikz\pgfuseplotmark{m6a};};
\node[stars] at (axis cs:{77.055},{-15.095}) {\tikz\pgfuseplotmark{m6c};};
\node[stars] at (axis cs:{77.084},{-8.665}) {\tikz\pgfuseplotmark{m6ab};};
\node[stars] at (axis cs:{77.182},{-4.456}) {\tikz\pgfuseplotmark{m5b};};
\node[stars] at (axis cs:{77.287},{-8.754}) {\tikz\pgfuseplotmark{m4cv};};
\node[stars] at (axis cs:{77.332},{9.829}) {\tikz\pgfuseplotmark{m5cv};};
\node[stars] at (axis cs:{77.425},{15.597}) {\tikz\pgfuseplotmark{m5a};};
\node[stars] at (axis cs:{77.438},{28.030}) {\tikz\pgfuseplotmark{m6bvb};};
\node[stars] at (axis cs:{77.514},{-0.565}) {\tikz\pgfuseplotmark{m6b};};
\node[stars] at (axis cs:{77.578},{37.302}) {\tikz\pgfuseplotmark{m6cb};};
\node[stars] at (axis cs:{77.686},{-25.910}) {\tikz\pgfuseplotmark{m6c};};
\node[stars] at (axis cs:{77.742},{-2.254}) {\tikz\pgfuseplotmark{m6c};};
\node[stars] at (axis cs:{77.830},{-2.491}) {\tikz\pgfuseplotmark{m6b};};
\node[stars] at (axis cs:{77.845},{-11.849}) {\tikz\pgfuseplotmark{m6av};};
\node[stars] at (axis cs:{77.910},{29.904}) {\tikz\pgfuseplotmark{m6c};};
\node[stars] at (axis cs:{77.923},{16.045}) {\tikz\pgfuseplotmark{m5c};};
\node[stars] at (axis cs:{77.939},{1.037}) {\tikz\pgfuseplotmark{m6bb};};
\node[stars] at (axis cs:{78.075},{-11.869}) {\tikz\pgfuseplotmark{m4c};};
\node[stars] at (axis cs:{78.201},{-6.057}) {\tikz\pgfuseplotmark{m6b};};
\node[stars] at (axis cs:{78.233},{-16.205}) {\tikz\pgfuseplotmark{m3cv};};
\node[stars] at (axis cs:{78.308},{-12.941}) {\tikz\pgfuseplotmark{m4cvb};};
\node[stars] at (axis cs:{78.323},{37.337}) {\tikz\pgfuseplotmark{m6cb};};
\node[stars] at (axis cs:{78.323},{2.861}) {\tikz\pgfuseplotmark{m4cvb};};
\node[stars] at (axis cs:{78.357},{38.484}) {\tikz\pgfuseplotmark{m5a};};
\node[stars] at (axis cs:{78.381},{1.968}) {\tikz\pgfuseplotmark{m6b};};
\node[stars] at (axis cs:{78.389},{-8.148}) {\tikz\pgfuseplotmark{m6c};};
\node[stars] at (axis cs:{78.447},{0.560}) {\tikz\pgfuseplotmark{m6cv};};
\node[stars] at (axis cs:{78.500},{-14.607}) {\tikz\pgfuseplotmark{m6c};};
\node[stars] at (axis cs:{78.620},{-35.977}) {\tikz\pgfuseplotmark{m6a};};
\node[stars] at (axis cs:{78.634},{-8.202}) {\tikz\pgfuseplotmark{m1bv};};
\node[stars] at (axis cs:{78.684},{5.156}) {\tikz\pgfuseplotmark{m5cv};};
\node[stars] at (axis cs:{78.827},{-1.409}) {\tikz\pgfuseplotmark{m6c};};
\node[stars] at (axis cs:{78.852},{-26.943}) {\tikz\pgfuseplotmark{m5b};};
\node[stars] at (axis cs:{78.852},{32.688}) {\tikz\pgfuseplotmark{m5bvb};};
\node[stars] at (axis cs:{78.865},{22.285}) {\tikz\pgfuseplotmark{m6c};};
\node[stars] at (axis cs:{79.017},{11.341}) {\tikz\pgfuseplotmark{m6a};};
\node[stars] at (axis cs:{79.044},{-11.353}) {\tikz\pgfuseplotmark{m6c};};
\node[stars] at (axis cs:{79.076},{34.312}) {\tikz\pgfuseplotmark{m6bvb};};
\node[stars] at (axis cs:{79.171},{1.947}) {\tikz\pgfuseplotmark{m6c};};
\node[stars] at (axis cs:{79.371},{-34.895}) {\tikz\pgfuseplotmark{m5a};};
\node[stars] at (axis cs:{79.402},{-6.844}) {\tikz\pgfuseplotmark{m4ab};};
\node[stars] at (axis cs:{79.418},{-13.520}) {\tikz\pgfuseplotmark{m5cv};};
\node[stars] at (axis cs:{79.544},{33.372}) {\tikz\pgfuseplotmark{m5av};};
\node[stars] at (axis cs:{79.579},{33.767}) {\tikz\pgfuseplotmark{m6cv};};
\node[stars] at (axis cs:{79.710},{-18.130}) {\tikz\pgfuseplotmark{m6bb};};
\node[stars] at (axis cs:{79.750},{33.748}) {\tikz\pgfuseplotmark{m5cv};};
\node[stars] at (axis cs:{79.797},{2.596}) {\tikz\pgfuseplotmark{m5c};};
\node[stars] at (axis cs:{79.811},{20.135}) {\tikz\pgfuseplotmark{m6bb};};
\node[stars] at (axis cs:{79.819},{22.096}) {\tikz\pgfuseplotmark{m5b};};
\node[stars] at (axis cs:{79.823},{-18.520}) {\tikz\pgfuseplotmark{m6cvb};};
\node[stars] at (axis cs:{79.849},{-27.369}) {\tikz\pgfuseplotmark{m6b};};
\node[stars] at (axis cs:{79.849},{33.985}) {\tikz\pgfuseplotmark{m6c};};
\node[stars] at (axis cs:{79.894},{-13.177}) {\tikz\pgfuseplotmark{m4c};};
\node[stars] at (axis cs:{79.897},{-1.412}) {\tikz\pgfuseplotmark{m6c};};
\node[stars] at (axis cs:{79.996},{-12.315}) {\tikz\pgfuseplotmark{m5c};};
\node[stars] at (axis cs:{80.004},{33.958}) {\tikz\pgfuseplotmark{m5b};};
\node[stars] at (axis cs:{80.086},{-34.699}) {\tikz\pgfuseplotmark{m6c};};
\node[stars] at (axis cs:{80.110},{-5.368}) {\tikz\pgfuseplotmark{m6cb};};
\node[stars] at (axis cs:{80.112},{-21.240}) {\tikz\pgfuseplotmark{m5avb};};
\node[stars] at (axis cs:{80.182},{2.545}) {\tikz\pgfuseplotmark{m6c};};
\node[stars] at (axis cs:{80.236},{19.814}) {\tikz\pgfuseplotmark{m6c};};
\node[stars] at (axis cs:{80.247},{27.957}) {\tikz\pgfuseplotmark{m6c};};
\node[stars] at (axis cs:{80.303},{29.570}) {\tikz\pgfuseplotmark{m6a};};
\node[stars] at (axis cs:{80.320},{-34.345}) {\tikz\pgfuseplotmark{m6bv};};
\node[stars] at (axis cs:{80.383},{-0.416}) {\tikz\pgfuseplotmark{m6a};};
\node[stars] at (axis cs:{80.432},{8.428}) {\tikz\pgfuseplotmark{m6a};};
\node[stars] at (axis cs:{80.441},{-0.382}) {\tikz\pgfuseplotmark{m5av};};
\node[stars] at (axis cs:{80.443},{-24.773}) {\tikz\pgfuseplotmark{m5bb};};
\node[stars] at (axis cs:{80.480},{-17.604}) {\tikz\pgfuseplotmark{m6c};};
\node[stars] at (axis cs:{80.708},{3.544}) {\tikz\pgfuseplotmark{m5bb};};
\node[stars] at (axis cs:{80.800},{-26.705}) {\tikz\pgfuseplotmark{m6c};};
\node[stars] at (axis cs:{80.827},{-8.415}) {\tikz\pgfuseplotmark{m6cb};};
\node[stars] at (axis cs:{80.845},{28.937}) {\tikz\pgfuseplotmark{m6c};};
\node[stars] at (axis cs:{80.850},{-39.678}) {\tikz\pgfuseplotmark{m6av};};
\node[stars] at (axis cs:{80.876},{-13.927}) {\tikz\pgfuseplotmark{m5c};};
\node[stars] at (axis cs:{80.879},{5.323}) {\tikz\pgfuseplotmark{m6c};};
\node[stars] at (axis cs:{80.907},{16.699}) {\tikz\pgfuseplotmark{m6b};};
\node[stars] at (axis cs:{80.926},{-0.160}) {\tikz\pgfuseplotmark{m6av};};
\node[stars] at (axis cs:{80.964},{-0.866}) {\tikz\pgfuseplotmark{m6bb};};
\node[stars] at (axis cs:{80.987},{-7.808}) {\tikz\pgfuseplotmark{m4c};};
\node[stars] at (axis cs:{81.106},{17.384}) {\tikz\pgfuseplotmark{m5bvb};};
\node[stars] at (axis cs:{81.119},{-16.976}) {\tikz\pgfuseplotmark{m6a};};
\node[stars] at (axis cs:{81.119},{-2.397}) {\tikz\pgfuseplotmark{m3cvb};};
\node[stars] at (axis cs:{81.120},{-0.891}) {\tikz\pgfuseplotmark{m5b};};
\node[stars] at (axis cs:{81.150},{2.353}) {\tikz\pgfuseplotmark{m6c};};
\node[stars] at (axis cs:{81.160},{31.224}) {\tikz\pgfuseplotmark{m6c};};
\node[stars] at (axis cs:{81.160},{31.143}) {\tikz\pgfuseplotmark{m6b};};
\node[stars] at (axis cs:{81.163},{37.385}) {\tikz\pgfuseplotmark{m5bb};};
\node[stars] at (axis cs:{81.187},{36.200}) {\tikz\pgfuseplotmark{m6c};};
\node[stars] at (axis cs:{81.187},{1.846}) {\tikz\pgfuseplotmark{m5bv};};
\node[stars] at (axis cs:{81.257},{-10.329}) {\tikz\pgfuseplotmark{m6a};};
\node[stars] at (axis cs:{81.283},{6.349}) {\tikz\pgfuseplotmark{m2av};};
\node[stars] at (axis cs:{81.402},{-1.491}) {\tikz\pgfuseplotmark{m6c};};
\node[stars] at (axis cs:{81.446},{0.520}) {\tikz\pgfuseplotmark{m6cv};};
\node[stars] at (axis cs:{81.499},{-19.695}) {\tikz\pgfuseplotmark{m6ab};};
\node[stars] at (axis cs:{81.510},{-5.518}) {\tikz\pgfuseplotmark{m6c};};
\node[stars] at (axis cs:{81.524},{16.700}) {\tikz\pgfuseplotmark{m6cv};};
\node[stars] at (axis cs:{81.573},{28.607}) {\tikz\pgfuseplotmark{m2ab};};
\node[stars] at (axis cs:{81.662},{6.869}) {\tikz\pgfuseplotmark{m6c};};
\node[stars] at (axis cs:{81.703},{34.392}) {\tikz\pgfuseplotmark{m6b};};
\node[stars] at (axis cs:{81.709},{3.095}) {\tikz\pgfuseplotmark{m5av};};
\node[stars] at (axis cs:{81.714},{33.262}) {\tikz\pgfuseplotmark{m6c};};
\node[stars] at (axis cs:{81.726},{35.457}) {\tikz\pgfuseplotmark{m6c};};
\node[stars] at (axis cs:{81.770},{-11.901}) {\tikz\pgfuseplotmark{m6cv};};
\node[stars] at (axis cs:{81.784},{30.208}) {\tikz\pgfuseplotmark{m6a};};
\node[stars] at (axis cs:{81.792},{17.962}) {\tikz\pgfuseplotmark{m5cb};};
\node[stars] at (axis cs:{81.808},{15.258}) {\tikz\pgfuseplotmark{m6c};};
\node[stars] at (axis cs:{81.814},{2.341}) {\tikz\pgfuseplotmark{m6c};};
\node[stars] at (axis cs:{81.902},{-21.376}) {\tikz\pgfuseplotmark{m6b};};
\node[stars] at (axis cs:{81.909},{21.937}) {\tikz\pgfuseplotmark{m5b};};
\node[stars] at (axis cs:{81.912},{34.476}) {\tikz\pgfuseplotmark{m5bb};};
\node[stars] at (axis cs:{81.940},{15.874}) {\tikz\pgfuseplotmark{m6a};};
\node[stars] at (axis cs:{82.004},{33.763}) {\tikz\pgfuseplotmark{m6cb};};
\node[stars] at (axis cs:{82.006},{1.298}) {\tikz\pgfuseplotmark{m6c};};
\node[stars] at (axis cs:{82.007},{17.239}) {\tikz\pgfuseplotmark{m6a};};
\node[stars] at (axis cs:{82.061},{-20.759}) {\tikz\pgfuseplotmark{m3a};};
\node[stars] at (axis cs:{82.064},{-37.231}) {\tikz\pgfuseplotmark{m6a};};
\node[stars] at (axis cs:{82.145},{13.679}) {\tikz\pgfuseplotmark{m6c};};
\node[stars] at (axis cs:{82.237},{-3.307}) {\tikz\pgfuseplotmark{m6c};};
\node[stars] at (axis cs:{82.319},{25.150}) {\tikz\pgfuseplotmark{m6ab};};
\node[stars] at (axis cs:{82.349},{-3.446}) {\tikz\pgfuseplotmark{m6av};};
\node[stars] at (axis cs:{82.419},{29.186}) {\tikz\pgfuseplotmark{m6c};};
\node[stars] at (axis cs:{82.433},{-1.092}) {\tikz\pgfuseplotmark{m5a};};
\node[stars] at (axis cs:{82.478},{1.789}) {\tikz\pgfuseplotmark{m6a};};
\node[stars] at (axis cs:{82.583},{4.205}) {\tikz\pgfuseplotmark{m6cv};};
\node[stars] at (axis cs:{82.586},{-7.435}) {\tikz\pgfuseplotmark{m6c};};
\node[stars] at (axis cs:{82.609},{15.360}) {\tikz\pgfuseplotmark{m6b};};
\node[stars] at (axis cs:{82.681},{22.462}) {\tikz\pgfuseplotmark{m6c};};
\node[stars] at (axis cs:{82.688},{39.826}) {\tikz\pgfuseplotmark{m6cb};};
\node[stars] at (axis cs:{82.696},{5.948}) {\tikz\pgfuseplotmark{m4cb};};
\node[stars] at (axis cs:{82.782},{-20.863}) {\tikz\pgfuseplotmark{m6a};};
\node[stars] at (axis cs:{82.803},{-35.470}) {\tikz\pgfuseplotmark{m4b};};
\node[stars] at (axis cs:{82.811},{3.292}) {\tikz\pgfuseplotmark{m5cb};};
\node[stars] at (axis cs:{82.837},{-6.708}) {\tikz\pgfuseplotmark{m6c};};
\node[stars] at (axis cs:{82.983},{-7.302}) {\tikz\pgfuseplotmark{m5av};};
\node[stars] at (axis cs:{83.002},{-0.299}) {\tikz\pgfuseplotmark{m2cv};};
\node[stars] at (axis cs:{83.053},{18.594}) {\tikz\pgfuseplotmark{m4cv};};
\node[stars] at (axis cs:{83.059},{17.058}) {\tikz\pgfuseplotmark{m6bvb};};
\node[stars] at (axis cs:{83.158},{0.012}) {\tikz\pgfuseplotmark{m6c};};
\node[stars] at (axis cs:{83.172},{-1.592}) {\tikz\pgfuseplotmark{m5c};};
\node[stars] at (axis cs:{83.182},{32.192}) {\tikz\pgfuseplotmark{m5a};};
\node[stars] at (axis cs:{83.183},{-17.822}) {\tikz\pgfuseplotmark{m3a};};
\node[stars] at (axis cs:{83.214},{-38.513}) {\tikz\pgfuseplotmark{m5c};};
\node[stars] at (axis cs:{83.281},{-35.139}) {\tikz\pgfuseplotmark{m6a};};
\node[stars] at (axis cs:{83.364},{32.801}) {\tikz\pgfuseplotmark{m6c};};
\node[stars] at (axis cs:{83.381},{-1.156}) {\tikz\pgfuseplotmark{m5cv};};
\node[stars] at (axis cs:{83.382},{18.540}) {\tikz\pgfuseplotmark{m6av};};
\node[stars] at (axis cs:{83.409},{34.725}) {\tikz\pgfuseplotmark{m6c};};
\node[stars] at (axis cs:{83.412},{20.474}) {\tikz\pgfuseplotmark{m6c};};
\node[stars] at (axis cs:{83.476},{14.305}) {\tikz\pgfuseplotmark{m6a};};
\node[stars] at (axis cs:{83.510},{-7.024}) {\tikz\pgfuseplotmark{m6c};};
\node[stars] at (axis cs:{83.516},{-1.036}) {\tikz\pgfuseplotmark{m6c};};
\node[stars] at (axis cs:{83.517},{-1.470}) {\tikz\pgfuseplotmark{m6b};};
\node[stars] at (axis cs:{83.570},{3.767}) {\tikz\pgfuseplotmark{m5c};};
\node[stars] at (axis cs:{83.622},{-0.012}) {\tikz\pgfuseplotmark{m6c};};
\node[stars] at (axis cs:{83.705},{9.489}) {\tikz\pgfuseplotmark{m4c};};
\node[stars] at (axis cs:{83.761},{-6.002}) {\tikz\pgfuseplotmark{m5ab};};
\node[stars] at (axis cs:{83.784},{9.934}) {\tikz\pgfuseplotmark{m3cb};};
\node[stars] at (axis cs:{83.805},{10.240}) {\tikz\pgfuseplotmark{m6a};};
\node[stars] at (axis cs:{83.814},{-33.080}) {\tikz\pgfuseplotmark{m6a};};
\node[stars] at (axis cs:{83.843},{-4.424}) {\tikz\pgfuseplotmark{m6c};};
\node[stars] at (axis cs:{83.845},{-5.416}) {\tikz\pgfuseplotmark{m5bb};};
\node[stars] at (axis cs:{83.847},{-4.838}) {\tikz\pgfuseplotmark{m5ab};};
\node[stars] at (axis cs:{83.858},{-5.910}) {\tikz\pgfuseplotmark{m3ab};};
\node[stars] at (axis cs:{83.863},{24.039}) {\tikz\pgfuseplotmark{m5c};};
\node[stars] at (axis cs:{83.879},{-4.364}) {\tikz\pgfuseplotmark{m6cb};};
\node[stars] at (axis cs:{83.900},{-3.253}) {\tikz\pgfuseplotmark{m6cv};};
\node[stars] at (axis cs:{83.915},{-4.856}) {\tikz\pgfuseplotmark{m5c};};
\node[stars] at (axis cs:{83.981},{27.662}) {\tikz\pgfuseplotmark{m6c};};
\node[stars] at (axis cs:{84.043},{-28.708}) {\tikz\pgfuseplotmark{m6c};};
\node[stars] at (axis cs:{84.053},{-1.202}) {\tikz\pgfuseplotmark{m2av};};
\node[stars] at (axis cs:{84.149},{-6.065}) {\tikz\pgfuseplotmark{m6ab};};
\node[stars] at (axis cs:{84.227},{9.291}) {\tikz\pgfuseplotmark{m4b};};
\node[stars] at (axis cs:{84.266},{17.040}) {\tikz\pgfuseplotmark{m6a};};
\node[stars] at (axis cs:{84.268},{11.035}) {\tikz\pgfuseplotmark{m6b};};
\node[stars] at (axis cs:{84.287},{-11.775}) {\tikz\pgfuseplotmark{m6b};};
\node[stars] at (axis cs:{84.287},{26.924}) {\tikz\pgfuseplotmark{m6bb};};
\node[stars] at (axis cs:{84.319},{-27.871}) {\tikz\pgfuseplotmark{m6c};};
\node[stars] at (axis cs:{84.330},{8.952}) {\tikz\pgfuseplotmark{m6b};};
\node[stars] at (axis cs:{84.364},{-5.938}) {\tikz\pgfuseplotmark{m6bv};};
\node[stars] at (axis cs:{84.411},{21.142}) {\tikz\pgfuseplotmark{m3bv};};
\node[stars] at (axis cs:{84.436},{-28.690}) {\tikz\pgfuseplotmark{m5c};};
\node[stars] at (axis cs:{84.441},{33.559}) {\tikz\pgfuseplotmark{m6c};};
\node[stars] at (axis cs:{84.472},{-4.814}) {\tikz\pgfuseplotmark{m6cv};};
\node[stars] at (axis cs:{84.505},{7.541}) {\tikz\pgfuseplotmark{m6b};};
\node[stars] at (axis cs:{84.658},{-6.574}) {\tikz\pgfuseplotmark{m6bv};};
\node[stars] at (axis cs:{84.659},{30.492}) {\tikz\pgfuseplotmark{m5cvb};};
\node[stars] at (axis cs:{84.687},{-2.600}) {\tikz\pgfuseplotmark{m4avb};};
\node[stars] at (axis cs:{84.721},{-7.213}) {\tikz\pgfuseplotmark{m5a};};
\node[stars] at (axis cs:{84.739},{26.618}) {\tikz\pgfuseplotmark{m6c};};
\node[stars] at (axis cs:{84.796},{4.121}) {\tikz\pgfuseplotmark{m5av};};
\node[stars] at (axis cs:{84.818},{-17.849}) {\tikz\pgfuseplotmark{m6cb};};
\node[stars] at (axis cs:{84.826},{29.215}) {\tikz\pgfuseplotmark{m6bv};};
\node[stars] at (axis cs:{84.863},{21.763}) {\tikz\pgfuseplotmark{m6c};};
\node[stars] at (axis cs:{84.879},{-9.707}) {\tikz\pgfuseplotmark{m6c};};
\node[stars] at (axis cs:{84.880},{-3.564}) {\tikz\pgfuseplotmark{m6b};};
\node[stars] at (axis cs:{84.912},{-34.074}) {\tikz\pgfuseplotmark{m3av};};
\node[stars] at (axis cs:{84.934},{25.897}) {\tikz\pgfuseplotmark{m5c};};
\node[stars] at (axis cs:{84.958},{-32.629}) {\tikz\pgfuseplotmark{m5c};};
\node[stars] at (axis cs:{85.150},{31.358}) {\tikz\pgfuseplotmark{m6bv};};
\node[stars] at (axis cs:{85.155},{-2.825}) {\tikz\pgfuseplotmark{m6c};};
\node[stars] at (axis cs:{85.175},{31.921}) {\tikz\pgfuseplotmark{m6cv};};
\node[stars] at (axis cs:{85.190},{-1.942}) {\tikz\pgfuseplotmark{m2avb};};
\node[stars] at (axis cs:{85.192},{-10.409}) {\tikz\pgfuseplotmark{m6cv};};
\node[stars] at (axis cs:{85.211},{-1.129}) {\tikz\pgfuseplotmark{m5bv};};
\node[stars] at (axis cs:{85.273},{0.338}) {\tikz\pgfuseplotmark{m6b};};
\node[stars] at (axis cs:{85.324},{16.534}) {\tikz\pgfuseplotmark{m5av};};
\node[stars] at (axis cs:{85.337},{29.487}) {\tikz\pgfuseplotmark{m6cb};};
\node[stars] at (axis cs:{85.362},{-33.401}) {\tikz\pgfuseplotmark{m6c};};
\node[stars] at (axis cs:{85.418},{-2.896}) {\tikz\pgfuseplotmark{m6cv};};
\node[stars] at (axis cs:{85.423},{-16.726}) {\tikz\pgfuseplotmark{m6c};};
\node[stars] at (axis cs:{85.517},{22.660}) {\tikz\pgfuseplotmark{m6c};};
\node[stars] at (axis cs:{85.548},{-30.535}) {\tikz\pgfuseplotmark{m6c};};
\node[stars] at (axis cs:{85.558},{-22.374}) {\tikz\pgfuseplotmark{m6b};};
\node[stars] at (axis cs:{85.560},{-17.530}) {\tikz\pgfuseplotmark{m6c};};
\node[stars] at (axis cs:{85.563},{-34.668}) {\tikz\pgfuseplotmark{m5cv};};
\node[stars] at (axis cs:{85.619},{1.474}) {\tikz\pgfuseplotmark{m5b};};
\node[stars] at (axis cs:{85.725},{-6.796}) {\tikz\pgfuseplotmark{m6b};};
\node[stars] at (axis cs:{85.789},{-1.613}) {\tikz\pgfuseplotmark{m6cv};};
\node[stars] at (axis cs:{85.831},{23.204}) {\tikz\pgfuseplotmark{m6cv};};
\node[stars] at (axis cs:{85.840},{-18.557}) {\tikz\pgfuseplotmark{m6a};};
\node[stars] at (axis cs:{85.876},{-39.407}) {\tikz\pgfuseplotmark{m6c};};
\node[stars] at (axis cs:{86.116},{-22.448}) {\tikz\pgfuseplotmark{m4ab};};
\node[stars] at (axis cs:{86.118},{-20.126}) {\tikz\pgfuseplotmark{m6c};};
\node[stars] at (axis cs:{86.255},{12.888}) {\tikz\pgfuseplotmark{m6c};};
\node[stars] at (axis cs:{86.257},{4.008}) {\tikz\pgfuseplotmark{m6bb};};
\node[stars] at (axis cs:{86.500},{-32.306}) {\tikz\pgfuseplotmark{m5c};};
\node[stars] at (axis cs:{86.512},{-4.268}) {\tikz\pgfuseplotmark{m6cb};};
\node[stars] at (axis cs:{86.645},{1.168}) {\tikz\pgfuseplotmark{m6b};};
\node[stars] at (axis cs:{86.690},{15.822}) {\tikz\pgfuseplotmark{m6b};};
\node[stars] at (axis cs:{86.717},{9.522}) {\tikz\pgfuseplotmark{m6a};};
\node[stars] at (axis cs:{86.739},{-14.822}) {\tikz\pgfuseplotmark{m4a};};
\node[stars] at (axis cs:{86.769},{-28.639}) {\tikz\pgfuseplotmark{m6c};};
\node[stars] at (axis cs:{86.783},{-16.238}) {\tikz\pgfuseplotmark{m6c};};
\node[stars] at (axis cs:{86.805},{14.488}) {\tikz\pgfuseplotmark{m6av};};
\node[stars] at (axis cs:{86.828},{-35.674}) {\tikz\pgfuseplotmark{m6c};};
\node[stars] at (axis cs:{86.859},{17.729}) {\tikz\pgfuseplotmark{m5c};};
\node[stars] at (axis cs:{86.862},{-10.533}) {\tikz\pgfuseplotmark{m6bv};};
\node[stars] at (axis cs:{86.894},{25.569}) {\tikz\pgfuseplotmark{m6c};};
\node[stars] at (axis cs:{86.929},{13.899}) {\tikz\pgfuseplotmark{m5c};};
\node[stars] at (axis cs:{86.939},{-9.670}) {\tikz\pgfuseplotmark{m2bv};};
\node[stars] at (axis cs:{87.001},{6.454}) {\tikz\pgfuseplotmark{m5c};};
\node[stars] at (axis cs:{87.093},{20.869}) {\tikz\pgfuseplotmark{m6bb};};
\node[stars] at (axis cs:{87.146},{-4.094}) {\tikz\pgfuseplotmark{m6c};};
\node[stars] at (axis cs:{87.254},{24.568}) {\tikz\pgfuseplotmark{m5b};};
\node[stars] at (axis cs:{87.293},{39.181}) {\tikz\pgfuseplotmark{m5a};};
\node[stars] at (axis cs:{87.387},{12.651}) {\tikz\pgfuseplotmark{m5b};};
\node[stars] at (axis cs:{87.402},{-14.484}) {\tikz\pgfuseplotmark{m5cb};};
\node[stars] at (axis cs:{87.473},{-22.972}) {\tikz\pgfuseplotmark{m6b};};
\node[stars] at (axis cs:{87.511},{9.871}) {\tikz\pgfuseplotmark{m6a};};
\node[stars] at (axis cs:{87.554},{4.423}) {\tikz\pgfuseplotmark{m6b};};
\node[stars] at (axis cs:{87.620},{14.306}) {\tikz\pgfuseplotmark{m6c};};
\node[stars] at (axis cs:{87.625},{2.025}) {\tikz\pgfuseplotmark{m6b};};
\node[stars] at (axis cs:{87.740},{-35.768}) {\tikz\pgfuseplotmark{m3b};};
\node[stars] at (axis cs:{87.742},{27.968}) {\tikz\pgfuseplotmark{m6a};};
\node[stars] at (axis cs:{87.760},{37.305}) {\tikz\pgfuseplotmark{m5av};};
\node[stars] at (axis cs:{87.830},{-20.879}) {\tikz\pgfuseplotmark{m4a};};
\node[stars] at (axis cs:{87.842},{-7.518}) {\tikz\pgfuseplotmark{m5cv};};
\node[stars] at (axis cs:{87.857},{32.125}) {\tikz\pgfuseplotmark{m6cv};};
\node[stars] at (axis cs:{87.869},{-22.927}) {\tikz\pgfuseplotmark{m6c};};
\node[stars] at (axis cs:{87.872},{39.148}) {\tikz\pgfuseplotmark{m4b};};
\node[stars] at (axis cs:{87.998},{-29.449}) {\tikz\pgfuseplotmark{m6c};};
\node[stars] at (axis cs:{88.032},{-9.042}) {\tikz\pgfuseplotmark{m6b};};
\node[stars] at (axis cs:{88.093},{14.172}) {\tikz\pgfuseplotmark{m6av};};
\node[stars] at (axis cs:{88.098},{19.868}) {\tikz\pgfuseplotmark{m6bv};};
\node[stars] at (axis cs:{88.110},{1.855}) {\tikz\pgfuseplotmark{m5av};};
\node[stars] at (axis cs:{88.138},{-37.631}) {\tikz\pgfuseplotmark{m6a};};
\node[stars] at (axis cs:{88.165},{39.575}) {\tikz\pgfuseplotmark{m6c};};
\node[stars] at (axis cs:{88.167},{33.917}) {\tikz\pgfuseplotmark{m6bv};};
\node[stars] at (axis cs:{88.279},{-33.801}) {\tikz\pgfuseplotmark{m5bv};};
\node[stars] at (axis cs:{88.332},{27.612}) {\tikz\pgfuseplotmark{m5av};};
\node[stars] at (axis cs:{88.556},{10.586}) {\tikz\pgfuseplotmark{m6b};};
\node[stars] at (axis cs:{88.558},{-29.147}) {\tikz\pgfuseplotmark{m6c};};
\node[stars] at (axis cs:{88.566},{3.225}) {\tikz\pgfuseplotmark{m6c};};
\node[stars] at (axis cs:{88.596},{20.276}) {\tikz\pgfuseplotmark{m4cv};};
\node[stars] at (axis cs:{88.682},{-11.774}) {\tikz\pgfuseplotmark{m6a};};
\node[stars] at (axis cs:{88.683},{0.969}) {\tikz\pgfuseplotmark{m6b};};
\node[stars] at (axis cs:{88.719},{-39.958}) {\tikz\pgfuseplotmark{m6a};};
\node[stars] at (axis cs:{88.736},{19.749}) {\tikz\pgfuseplotmark{m6bv};};
\node[stars] at (axis cs:{88.746},{31.701}) {\tikz\pgfuseplotmark{m6b};};
\node[stars] at (axis cs:{88.793},{7.407}) {\tikz\pgfuseplotmark{m1bv};};
\node[stars] at (axis cs:{88.875},{-37.121}) {\tikz\pgfuseplotmark{m5b};};
\node[stars] at (axis cs:{88.876},{-4.616}) {\tikz\pgfuseplotmark{m6b};};
\node[stars] at (axis cs:{88.897},{-4.789}) {\tikz\pgfuseplotmark{m6c};};
\node[stars] at (axis cs:{88.955},{20.175}) {\tikz\pgfuseplotmark{m6bv};};
\node[stars] at (axis cs:{89.059},{-22.840}) {\tikz\pgfuseplotmark{m6b};};
\node[stars] at (axis cs:{89.087},{-31.382}) {\tikz\pgfuseplotmark{m6a};};
\node[stars] at (axis cs:{89.101},{-14.168}) {\tikz\pgfuseplotmark{m4a};};
\node[stars] at (axis cs:{89.117},{9.509}) {\tikz\pgfuseplotmark{m6b};};
\node[stars] at (axis cs:{89.141},{28.942}) {\tikz\pgfuseplotmark{m6c};};
\node[stars] at (axis cs:{89.143},{-23.215}) {\tikz\pgfuseplotmark{m6c};};
\node[stars] at (axis cs:{89.204},{-31.976}) {\tikz\pgfuseplotmark{m6c};};
\node[stars] at (axis cs:{89.206},{11.521}) {\tikz\pgfuseplotmark{m6b};};
\node[stars] at (axis cs:{89.234},{24.249}) {\tikz\pgfuseplotmark{m6b};};
\node[stars] at (axis cs:{89.384},{-35.283}) {\tikz\pgfuseplotmark{m4cv};};
\node[stars] at (axis cs:{89.477},{1.224}) {\tikz\pgfuseplotmark{m6c};};
\node[stars] at (axis cs:{89.499},{25.954}) {\tikz\pgfuseplotmark{m5a};};
\node[stars] at (axis cs:{89.549},{-0.994}) {\tikz\pgfuseplotmark{m6cv};};
\node[stars] at (axis cs:{89.602},{1.837}) {\tikz\pgfuseplotmark{m6bvb};};
\node[stars] at (axis cs:{89.707},{0.553}) {\tikz\pgfuseplotmark{m5cv};};
\node[stars] at (axis cs:{89.722},{12.808}) {\tikz\pgfuseplotmark{m6a};};
\node[stars] at (axis cs:{89.754},{-9.382}) {\tikz\pgfuseplotmark{m6cv};};
\node[stars] at (axis cs:{89.768},{-9.558}) {\tikz\pgfuseplotmark{m5b};};
\node[stars] at (axis cs:{89.930},{37.212}) {\tikz\pgfuseplotmark{m3av};};
\node[stars] at (axis cs:{90.014},{-3.074}) {\tikz\pgfuseplotmark{m5av};};
\node[stars] at (axis cs:{90.074},{-12.900}) {\tikz\pgfuseplotmark{m6cv};};
\node[stars] at (axis cs:{90.252},{27.572}) {\tikz\pgfuseplotmark{m6b};};
\node[stars] at (axis cs:{90.292},{31.034}) {\tikz\pgfuseplotmark{m6c};};
\node[stars] at (axis cs:{90.305},{-25.418}) {\tikz\pgfuseplotmark{m6b};};
\node[stars] at (axis cs:{90.318},{-33.912}) {\tikz\pgfuseplotmark{m6a};};
\node[stars] at (axis cs:{90.423},{22.401}) {\tikz\pgfuseplotmark{m6c};};
\node[stars] at (axis cs:{90.460},{-10.598}) {\tikz\pgfuseplotmark{m5bb};};
\node[stars] at (axis cs:{90.596},{9.647}) {\tikz\pgfuseplotmark{m4bv};};
\node[stars] at (axis cs:{90.641},{-14.497}) {\tikz\pgfuseplotmark{m6c};};
\node[stars] at (axis cs:{90.730},{32.636}) {\tikz\pgfuseplotmark{m6c};};
\node[stars] at (axis cs:{90.815},{-26.285}) {\tikz\pgfuseplotmark{m5b};};
\node[stars] at (axis cs:{90.853},{11.681}) {\tikz\pgfuseplotmark{m6b};};
\node[stars] at (axis cs:{90.864},{19.690}) {\tikz\pgfuseplotmark{m5cv};};
\node[stars] at (axis cs:{90.980},{20.138}) {\tikz\pgfuseplotmark{m5av};};
\node[stars] at (axis cs:{91.030},{23.263}) {\tikz\pgfuseplotmark{m4c};};
\node[stars] at (axis cs:{91.056},{-6.709}) {\tikz\pgfuseplotmark{m5cv};};
\node[stars] at (axis cs:{91.084},{-32.172}) {\tikz\pgfuseplotmark{m6av};};
\node[stars] at (axis cs:{91.242},{5.420}) {\tikz\pgfuseplotmark{m6a};};
\node[stars] at (axis cs:{91.243},{4.159}) {\tikz\pgfuseplotmark{m6a};};
\node[stars] at (axis cs:{91.246},{-16.484}) {\tikz\pgfuseplotmark{m5bv};};
\node[stars] at (axis cs:{91.261},{37.964}) {\tikz\pgfuseplotmark{m6c};};
\node[stars] at (axis cs:{91.363},{-10.242}) {\tikz\pgfuseplotmark{m6bv};};
\node[stars] at (axis cs:{91.363},{-35.513}) {\tikz\pgfuseplotmark{m6a};};
\node[stars] at (axis cs:{91.391},{33.599}) {\tikz\pgfuseplotmark{m6c};};
\node[stars] at (axis cs:{91.523},{-29.758}) {\tikz\pgfuseplotmark{m6a};};
\node[stars] at (axis cs:{91.536},{35.388}) {\tikz\pgfuseplotmark{m6bb};};
\node[stars] at (axis cs:{91.539},{-14.935}) {\tikz\pgfuseplotmark{m5a};};
\node[stars] at (axis cs:{91.594},{29.512}) {\tikz\pgfuseplotmark{m6bv};};
\node[stars] at (axis cs:{91.634},{-23.111}) {\tikz\pgfuseplotmark{m5c};};
\node[stars] at (axis cs:{91.646},{38.482}) {\tikz\pgfuseplotmark{m5cv};};
\node[stars] at (axis cs:{91.661},{-4.194}) {\tikz\pgfuseplotmark{m5c};};
\node[stars] at (axis cs:{91.740},{-21.812}) {\tikz\pgfuseplotmark{m6av};};
\node[stars] at (axis cs:{91.765},{-34.312}) {\tikz\pgfuseplotmark{m6av};};
\node[stars] at (axis cs:{91.882},{-37.253}) {\tikz\pgfuseplotmark{m5b};};
\node[stars] at (axis cs:{91.893},{14.768}) {\tikz\pgfuseplotmark{m4c};};
\node[stars] at (axis cs:{91.923},{-19.166}) {\tikz\pgfuseplotmark{m5cv};};
\node[stars] at (axis cs:{92.199},{-6.821}) {\tikz\pgfuseplotmark{m6c};};
\node[stars] at (axis cs:{92.241},{-22.427}) {\tikz\pgfuseplotmark{m5c};};
\node[stars] at (axis cs:{92.241},{2.499}) {\tikz\pgfuseplotmark{m6ab};};
\node[stars] at (axis cs:{92.334},{-18.126}) {\tikz\pgfuseplotmark{m6c};};
\node[stars] at (axis cs:{92.385},{22.190}) {\tikz\pgfuseplotmark{m6bv};};
\node[stars] at (axis cs:{92.394},{-14.585}) {\tikz\pgfuseplotmark{m6a};};
\node[stars] at (axis cs:{92.401},{-5.711}) {\tikz\pgfuseplotmark{m6c};};
\node[stars] at (axis cs:{92.433},{23.113}) {\tikz\pgfuseplotmark{m6av};};
\node[stars] at (axis cs:{92.446},{-26.701}) {\tikz\pgfuseplotmark{m6c};};
\node[stars] at (axis cs:{92.450},{-22.774}) {\tikz\pgfuseplotmark{m6a};};
\node[stars] at (axis cs:{92.645},{-27.154}) {\tikz\pgfuseplotmark{m6a};};
\node[stars] at (axis cs:{92.755},{-6.754}) {\tikz\pgfuseplotmark{m6cv};};
\node[stars] at (axis cs:{92.757},{18.130}) {\tikz\pgfuseplotmark{m6cv};};
\node[stars] at (axis cs:{92.807},{-26.482}) {\tikz\pgfuseplotmark{m6b};};
\node[stars] at (axis cs:{92.866},{13.638}) {\tikz\pgfuseplotmark{m6b};};
\node[stars] at (axis cs:{92.885},{24.420}) {\tikz\pgfuseplotmark{m6a};};
\node[stars] at (axis cs:{92.932},{-4.665}) {\tikz\pgfuseplotmark{m6cv};};
\node[stars] at (axis cs:{92.966},{-6.550}) {\tikz\pgfuseplotmark{m5b};};
\node[stars] at (axis cs:{92.985},{14.209}) {\tikz\pgfuseplotmark{m4cb};};
\node[stars] at (axis cs:{93.006},{19.790}) {\tikz\pgfuseplotmark{m6ab};};
\node[stars] at (axis cs:{93.014},{16.130}) {\tikz\pgfuseplotmark{m5b};};
\node[stars] at (axis cs:{93.080},{22.908}) {\tikz\pgfuseplotmark{m6cv};};
\node[stars] at (axis cs:{93.084},{32.693}) {\tikz\pgfuseplotmark{m6a};};
\node[stars] at (axis cs:{93.248},{6.016}) {\tikz\pgfuseplotmark{m6cv};};
\node[stars] at (axis cs:{93.302},{10.627}) {\tikz\pgfuseplotmark{m6cv};};
\node[stars] at (axis cs:{93.439},{-23.862}) {\tikz\pgfuseplotmark{m6c};};
\node[stars] at (axis cs:{93.476},{-3.741}) {\tikz\pgfuseplotmark{m6a};};
\node[stars] at (axis cs:{93.619},{17.906}) {\tikz\pgfuseplotmark{m6b};};
\node[stars] at (axis cs:{93.653},{-4.568}) {\tikz\pgfuseplotmark{m6a};};
\node[stars] at (axis cs:{93.712},{19.156}) {\tikz\pgfuseplotmark{m5cb};};
\node[stars] at (axis cs:{93.714},{-6.275}) {\tikz\pgfuseplotmark{m4b};};
\node[stars] at (axis cs:{93.719},{22.507}) {\tikz\pgfuseplotmark{m3cvb};};
\node[stars] at (axis cs:{93.785},{-20.272}) {\tikz\pgfuseplotmark{m6b};};
\node[stars] at (axis cs:{93.785},{13.851}) {\tikz\pgfuseplotmark{m6bv};};
\node[stars] at (axis cs:{93.824},{-18.477}) {\tikz\pgfuseplotmark{m6b};};
\node[stars] at (axis cs:{93.845},{29.498}) {\tikz\pgfuseplotmark{m4c};};
\node[stars] at (axis cs:{93.855},{16.143}) {\tikz\pgfuseplotmark{m5c};};
\node[stars] at (axis cs:{93.859},{-9.036}) {\tikz\pgfuseplotmark{m6b};};
\node[stars] at (axis cs:{93.874},{-4.914}) {\tikz\pgfuseplotmark{m6bb};};
\node[stars] at (axis cs:{93.893},{-0.512}) {\tikz\pgfuseplotmark{m6a};};
\node[stars] at (axis cs:{93.917},{6.066}) {\tikz\pgfuseplotmark{m6b};};
\node[stars] at (axis cs:{93.937},{-13.718}) {\tikz\pgfuseplotmark{m5b};};
\node[stars] at (axis cs:{93.937},{12.551}) {\tikz\pgfuseplotmark{m5c};};
\node[stars] at (axis cs:{93.975},{1.169}) {\tikz\pgfuseplotmark{m6c};};
\node[stars] at (axis cs:{94.032},{-16.618}) {\tikz\pgfuseplotmark{m6bv};};
\node[stars] at (axis cs:{94.079},{23.970}) {\tikz\pgfuseplotmark{m6bv};};
\node[stars] at (axis cs:{94.099},{17.181}) {\tikz\pgfuseplotmark{m6c};};
\node[stars] at (axis cs:{94.111},{12.272}) {\tikz\pgfuseplotmark{m5bb};};
\node[stars] at (axis cs:{94.138},{-35.140}) {\tikz\pgfuseplotmark{m4cv};};
\node[stars] at (axis cs:{94.148},{-39.264}) {\tikz\pgfuseplotmark{m6b};};
\node[stars] at (axis cs:{94.245},{23.741}) {\tikz\pgfuseplotmark{m6cv};};
\node[stars] at (axis cs:{94.255},{-37.737}) {\tikz\pgfuseplotmark{m6a};};
\node[stars] at (axis cs:{94.265},{-22.715}) {\tikz\pgfuseplotmark{m6b};};
\node[stars] at (axis cs:{94.278},{9.942}) {\tikz\pgfuseplotmark{m5cb};};
\node[stars] at (axis cs:{94.290},{-37.253}) {\tikz\pgfuseplotmark{m6bv};};
\node[stars] at (axis cs:{94.317},{5.100}) {\tikz\pgfuseplotmark{m6a};};
\node[stars] at (axis cs:{94.424},{-16.816}) {\tikz\pgfuseplotmark{m5c};};
\node[stars] at (axis cs:{94.523},{14.383}) {\tikz\pgfuseplotmark{m6b};};
\node[stars] at (axis cs:{94.557},{-19.967}) {\tikz\pgfuseplotmark{m6a};};
\node[stars] at (axis cs:{94.668},{9.047}) {\tikz\pgfuseplotmark{m6c};};
\node[stars] at (axis cs:{94.703},{-15.025}) {\tikz\pgfuseplotmark{m6bv};};
\node[stars] at (axis cs:{94.711},{-9.390}) {\tikz\pgfuseplotmark{m5c};};
\node[stars] at (axis cs:{94.746},{-20.925}) {\tikz\pgfuseplotmark{m6av};};
\node[stars] at (axis cs:{94.758},{17.325}) {\tikz\pgfuseplotmark{m6cv};};
\node[stars] at (axis cs:{94.783},{-8.586}) {\tikz\pgfuseplotmark{m6c};};
\node[stars] at (axis cs:{94.909},{-22.103}) {\tikz\pgfuseplotmark{m6c};};
\node[stars] at (axis cs:{94.921},{-34.396}) {\tikz\pgfuseplotmark{m6a};};
\node[stars] at (axis cs:{94.928},{-7.823}) {\tikz\pgfuseplotmark{m5c};};
\node[stars] at (axis cs:{94.998},{-2.944}) {\tikz\pgfuseplotmark{m5b};};
\node[stars] at (axis cs:{95.018},{14.651}) {\tikz\pgfuseplotmark{m6av};};
\node[stars] at (axis cs:{95.078},{-30.063}) {\tikz\pgfuseplotmark{m3bv};};
\node[stars] at (axis cs:{95.146},{-23.638}) {\tikz\pgfuseplotmark{m6c};};
\node[stars] at (axis cs:{95.151},{-34.144}) {\tikz\pgfuseplotmark{m6av};};
\node[stars] at (axis cs:{95.273},{-11.817}) {\tikz\pgfuseplotmark{m6c};};
\node[stars] at (axis cs:{95.300},{29.541}) {\tikz\pgfuseplotmark{m6c};};
\node[stars] at (axis cs:{95.353},{-11.773}) {\tikz\pgfuseplotmark{m6av};};
\node[stars] at (axis cs:{95.357},{2.269}) {\tikz\pgfuseplotmark{m6cv};};
\node[stars] at (axis cs:{95.358},{17.763}) {\tikz\pgfuseplotmark{m6c};};
\node[stars] at (axis cs:{95.528},{-33.436}) {\tikz\pgfuseplotmark{m4a};};
\node[stars] at (axis cs:{95.652},{12.570}) {\tikz\pgfuseplotmark{m6b};};
\node[stars] at (axis cs:{95.675},{-17.956}) {\tikz\pgfuseplotmark{m2bv};};
\node[stars] at (axis cs:{95.740},{22.513}) {\tikz\pgfuseplotmark{m3bv};};
\node[stars] at (axis cs:{95.810},{-31.790}) {\tikz\pgfuseplotmark{m6c};};
\node[stars] at (axis cs:{95.827},{3.764}) {\tikz\pgfuseplotmark{m6c};};
\node[stars] at (axis cs:{95.900},{-9.875}) {\tikz\pgfuseplotmark{m6cv};};
\node[stars] at (axis cs:{95.942},{4.593}) {\tikz\pgfuseplotmark{m4cb};};
\node[stars] at (axis cs:{95.942},{-15.071}) {\tikz\pgfuseplotmark{m6c};};
\node[stars] at (axis cs:{95.983},{-25.577}) {\tikz\pgfuseplotmark{m6a};};
\node[stars] at (axis cs:{96.004},{-36.708}) {\tikz\pgfuseplotmark{m6ab};};
\node[stars] at (axis cs:{96.010},{8.885}) {\tikz\pgfuseplotmark{m6c};};
\node[stars] at (axis cs:{96.043},{-11.530}) {\tikz\pgfuseplotmark{m5c};};
\node[stars] at (axis cs:{96.086},{-12.962}) {\tikz\pgfuseplotmark{m6bv};};
\node[stars] at (axis cs:{96.182},{25.048}) {\tikz\pgfuseplotmark{m6c};};
\node[stars] at (axis cs:{96.183},{-28.780}) {\tikz\pgfuseplotmark{m6c};};
\node[stars] at (axis cs:{96.220},{16.057}) {\tikz\pgfuseplotmark{m6c};};
\node[stars] at (axis cs:{96.304},{7.086}) {\tikz\pgfuseplotmark{m6cv};};
\node[stars] at (axis cs:{96.319},{-0.946}) {\tikz\pgfuseplotmark{m6b};};
\node[stars] at (axis cs:{96.367},{14.722}) {\tikz\pgfuseplotmark{m6cv};};
\node[stars] at (axis cs:{96.375},{-35.064}) {\tikz\pgfuseplotmark{m6cb};};
\node[stars] at (axis cs:{96.387},{23.327}) {\tikz\pgfuseplotmark{m6b};};
\node[stars] at (axis cs:{96.446},{-3.889}) {\tikz\pgfuseplotmark{m6c};};
\node[stars] at (axis cs:{96.468},{11.126}) {\tikz\pgfuseplotmark{m6c};};
\node[stars] at (axis cs:{96.495},{-7.895}) {\tikz\pgfuseplotmark{m6c};};
\node[stars] at (axis cs:{96.540},{-3.515}) {\tikz\pgfuseplotmark{m6c};};
\node[stars] at (axis cs:{96.644},{-4.597}) {\tikz\pgfuseplotmark{m6c};};
\node[stars] at (axis cs:{96.665},{-1.507}) {\tikz\pgfuseplotmark{m6b};};
\node[stars] at (axis cs:{96.678},{-14.603}) {\tikz\pgfuseplotmark{m6c};};
\node[stars] at (axis cs:{96.687},{-7.512}) {\tikz\pgfuseplotmark{m6cb};};
\node[stars] at (axis cs:{96.782},{-37.895}) {\tikz\pgfuseplotmark{m6c};};
\node[stars] at (axis cs:{96.797},{-25.856}) {\tikz\pgfuseplotmark{m6b};};
\node[stars] at (axis cs:{96.807},{0.299}) {\tikz\pgfuseplotmark{m5c};};
\node[stars] at (axis cs:{96.815},{-0.276}) {\tikz\pgfuseplotmark{m6a};};
\node[stars] at (axis cs:{96.835},{2.908}) {\tikz\pgfuseplotmark{m6a};};
\node[stars] at (axis cs:{96.898},{32.563}) {\tikz\pgfuseplotmark{m6c};};
\node[stars] at (axis cs:{96.986},{20.496}) {\tikz\pgfuseplotmark{m6c};};
\node[stars] at (axis cs:{96.990},{-4.762}) {\tikz\pgfuseplotmark{m5b};};
\node[stars] at (axis cs:{97.043},{-32.580}) {\tikz\pgfuseplotmark{m4c};};
\node[stars] at (axis cs:{97.070},{1.912}) {\tikz\pgfuseplotmark{m6c};};
\node[stars] at (axis cs:{97.078},{10.304}) {\tikz\pgfuseplotmark{m6c};};
\node[stars] at (axis cs:{97.117},{16.238}) {\tikz\pgfuseplotmark{m6c};};
\node[stars] at (axis cs:{97.142},{30.493}) {\tikz\pgfuseplotmark{m6av};};
\node[stars] at (axis cs:{97.156},{-17.466}) {\tikz\pgfuseplotmark{m6a};};
\node[stars] at (axis cs:{97.164},{-32.371}) {\tikz\pgfuseplotmark{m6avb};};
\node[stars] at (axis cs:{97.204},{-7.033}) {\tikz\pgfuseplotmark{m5avb};};
\node[stars] at (axis cs:{97.241},{20.212}) {\tikz\pgfuseplotmark{m4cb};};
\node[stars] at (axis cs:{97.312},{2.646}) {\tikz\pgfuseplotmark{m6c};};
\node[stars] at (axis cs:{97.547},{-10.081}) {\tikz\pgfuseplotmark{m6b};};
\node[stars] at (axis cs:{97.549},{-19.215}) {\tikz\pgfuseplotmark{m6c};};
\node[stars] at (axis cs:{97.645},{-13.148}) {\tikz\pgfuseplotmark{m6cv};};
\node[stars] at (axis cs:{97.693},{-27.770}) {\tikz\pgfuseplotmark{m6b};};
\node[stars] at (axis cs:{97.790},{11.251}) {\tikz\pgfuseplotmark{m6cb};};
\node[stars] at (axis cs:{97.792},{16.938}) {\tikz\pgfuseplotmark{m6cb};};
\node[stars] at (axis cs:{97.805},{-35.259}) {\tikz\pgfuseplotmark{m6a};};
\node[stars] at (axis cs:{97.846},{-12.392}) {\tikz\pgfuseplotmark{m5c};};
\node[stars] at (axis cs:{97.850},{-32.869}) {\tikz\pgfuseplotmark{m6cv};};
\node[stars] at (axis cs:{97.896},{-36.940}) {\tikz\pgfuseplotmark{m6cv};};
\node[stars] at (axis cs:{97.906},{15.903}) {\tikz\pgfuseplotmark{m6c};};
\node[stars] at (axis cs:{97.951},{11.544}) {\tikz\pgfuseplotmark{m5c};};
\node[stars] at (axis cs:{97.959},{-8.158}) {\tikz\pgfuseplotmark{m5c};};
\node[stars] at (axis cs:{97.964},{-23.418}) {\tikz\pgfuseplotmark{m4cv};};
\node[stars] at (axis cs:{98.077},{17.784}) {\tikz\pgfuseplotmark{m6cb};};
\node[stars] at (axis cs:{98.080},{4.856}) {\tikz\pgfuseplotmark{m6av};};
\node[stars] at (axis cs:{98.089},{-37.696}) {\tikz\pgfuseplotmark{m5c};};
\node[stars] at (axis cs:{98.096},{-5.869}) {\tikz\pgfuseplotmark{m6a};};
\node[stars] at (axis cs:{98.097},{11.673}) {\tikz\pgfuseplotmark{m6bb};};
\node[stars] at (axis cs:{98.113},{32.455}) {\tikz\pgfuseplotmark{m6av};};
\node[stars] at (axis cs:{98.162},{-32.030}) {\tikz\pgfuseplotmark{m6ab};};
\node[stars] at (axis cs:{98.195},{-11.166}) {\tikz\pgfuseplotmark{m6cv};};
\node[stars] at (axis cs:{98.226},{7.333}) {\tikz\pgfuseplotmark{m5av};};
\node[stars] at (axis cs:{98.293},{-38.625}) {\tikz\pgfuseplotmark{m6c};};
\node[stars] at (axis cs:{98.361},{-20.924}) {\tikz\pgfuseplotmark{m6c};};
\node[stars] at (axis cs:{98.401},{14.155}) {\tikz\pgfuseplotmark{m6a};};
\node[stars] at (axis cs:{98.408},{-1.220}) {\tikz\pgfuseplotmark{m5b};};
\node[stars] at (axis cs:{98.456},{-36.232}) {\tikz\pgfuseplotmark{m5c};};
\node[stars] at (axis cs:{98.647},{-32.716}) {\tikz\pgfuseplotmark{m6a};};
\node[stars] at (axis cs:{98.693},{7.572}) {\tikz\pgfuseplotmark{m6cv};};
\node[stars] at (axis cs:{98.764},{-22.965}) {\tikz\pgfuseplotmark{m5a};};
\node[stars] at (axis cs:{98.800},{28.022}) {\tikz\pgfuseplotmark{m5cv};};
\node[stars] at (axis cs:{98.816},{0.890}) {\tikz\pgfuseplotmark{m6a};};
\node[stars] at (axis cs:{98.823},{9.988}) {\tikz\pgfuseplotmark{m6b};};
\node[stars] at (axis cs:{98.851},{-36.780}) {\tikz\pgfuseplotmark{m6ab};};
\node[stars] at (axis cs:{98.878},{-22.109}) {\tikz\pgfuseplotmark{m6c};};
\node[stars] at (axis cs:{98.975},{-36.089}) {\tikz\pgfuseplotmark{m6cb};};
\node[stars] at (axis cs:{99.095},{-18.660}) {\tikz\pgfuseplotmark{m6ab};};
\node[stars] at (axis cs:{99.137},{38.445}) {\tikz\pgfuseplotmark{m5cv};};
\node[stars] at (axis cs:{99.147},{-5.211}) {\tikz\pgfuseplotmark{m6a};};
\node[stars] at (axis cs:{99.171},{-19.256}) {\tikz\pgfuseplotmark{m4bv};};
\node[stars] at (axis cs:{99.171},{-22.614}) {\tikz\pgfuseplotmark{m6cvb};};
\node[stars] at (axis cs:{99.194},{-13.321}) {\tikz\pgfuseplotmark{m6bv};};
\node[stars] at (axis cs:{99.258},{-38.146}) {\tikz\pgfuseplotmark{m6b};};
\node[stars] at (axis cs:{99.308},{-36.991}) {\tikz\pgfuseplotmark{m6a};};
\node[stars] at (axis cs:{99.350},{6.135}) {\tikz\pgfuseplotmark{m6bv};};
\node[stars] at (axis cs:{99.364},{24.591}) {\tikz\pgfuseplotmark{m6c};};
\node[stars] at (axis cs:{99.404},{10.853}) {\tikz\pgfuseplotmark{m6c};};
\node[stars] at (axis cs:{99.418},{2.704}) {\tikz\pgfuseplotmark{m6c};};
\node[stars] at (axis cs:{99.420},{-12.985}) {\tikz\pgfuseplotmark{m6b};};
\node[stars] at (axis cs:{99.428},{16.399}) {\tikz\pgfuseplotmark{m2b};};
\node[stars] at (axis cs:{99.448},{-32.340}) {\tikz\pgfuseplotmark{m5c};};
\node[stars] at (axis cs:{99.470},{4.957}) {\tikz\pgfuseplotmark{m6cv};};
\node[stars] at (axis cs:{99.473},{-18.237}) {\tikz\pgfuseplotmark{m4c};};
\node[stars] at (axis cs:{99.585},{-2.544}) {\tikz\pgfuseplotmark{m6b};};
\node[stars] at (axis cs:{99.595},{-23.580}) {\tikz\pgfuseplotmark{m6c};};
\node[stars] at (axis cs:{99.596},{28.984}) {\tikz\pgfuseplotmark{m6a};};
\node[stars] at (axis cs:{99.648},{-16.873}) {\tikz\pgfuseplotmark{m6b};};
\node[stars] at (axis cs:{99.659},{1.613}) {\tikz\pgfuseplotmark{m6cv};};
\node[stars] at (axis cs:{99.665},{39.391}) {\tikz\pgfuseplotmark{m6a};};
\node[stars] at (axis cs:{99.705},{39.903}) {\tikz\pgfuseplotmark{m5c};};
\node[stars] at (axis cs:{99.772},{22.031}) {\tikz\pgfuseplotmark{m6b};};
\node[stars] at (axis cs:{99.820},{-14.146}) {\tikz\pgfuseplotmark{m5a};};
\node[stars] at (axis cs:{99.881},{24.600}) {\tikz\pgfuseplotmark{m6c};};
\node[stars] at (axis cs:{99.888},{28.263}) {\tikz\pgfuseplotmark{m6bv};};
\node[stars] at (axis cs:{99.901},{-23.695}) {\tikz\pgfuseplotmark{m6b};};
\node[stars] at (axis cs:{99.928},{-30.470}) {\tikz\pgfuseplotmark{m6a};};
\node[stars] at (axis cs:{99.949},{12.983}) {\tikz\pgfuseplotmark{m6b};};
\node[stars] at (axis cs:{100.244},{9.896}) {\tikz\pgfuseplotmark{m5avb};};
\node[stars] at (axis cs:{100.273},{0.495}) {\tikz\pgfuseplotmark{m6a};};
\node[stars] at (axis cs:{100.322},{11.003}) {\tikz\pgfuseplotmark{m6cv};};
\node[stars] at (axis cs:{100.337},{28.196}) {\tikz\pgfuseplotmark{m6c};};
\node[stars] at (axis cs:{100.341},{16.397}) {\tikz\pgfuseplotmark{m6c};};
\node[stars] at (axis cs:{100.407},{35.932}) {\tikz\pgfuseplotmark{m6c};};
\node[stars] at (axis cs:{100.485},{-9.168}) {\tikz\pgfuseplotmark{m5c};};
\node[stars] at (axis cs:{100.497},{6.345}) {\tikz\pgfuseplotmark{m6c};};
\node[stars] at (axis cs:{100.601},{17.645}) {\tikz\pgfuseplotmark{m5c};};
\node[stars] at (axis cs:{100.692},{-22.448}) {\tikz\pgfuseplotmark{m6cvb};};
\node[stars] at (axis cs:{100.778},{3.034}) {\tikz\pgfuseplotmark{m6c};};
\node[stars] at (axis cs:{100.807},{37.147}) {\tikz\pgfuseplotmark{m6c};};
\node[stars] at (axis cs:{100.847},{-39.193}) {\tikz\pgfuseplotmark{m6c};};
\node[stars] at (axis cs:{100.911},{3.932}) {\tikz\pgfuseplotmark{m6bv};};
\node[stars] at (axis cs:{100.983},{25.131}) {\tikz\pgfuseplotmark{m3bv};};
\node[stars] at (axis cs:{100.997},{13.228}) {\tikz\pgfuseplotmark{m4c};};
\node[stars] at (axis cs:{101.052},{36.109}) {\tikz\pgfuseplotmark{m6c};};
\node[stars] at (axis cs:{101.119},{-31.070}) {\tikz\pgfuseplotmark{m5cv};};
\node[stars] at (axis cs:{101.189},{28.971}) {\tikz\pgfuseplotmark{m5c};};
\node[stars] at (axis cs:{101.216},{-27.342}) {\tikz\pgfuseplotmark{m6c};};
\node[stars] at (axis cs:{101.287},{-16.716}) {\tikz\pgfuseplotmark{m1b};};
\node[stars] at (axis cs:{101.322},{12.895}) {\tikz\pgfuseplotmark{m3cv};};
\node[stars] at (axis cs:{101.346},{-31.793}) {\tikz\pgfuseplotmark{m6b};};
\node[stars] at (axis cs:{101.347},{-23.462}) {\tikz\pgfuseplotmark{m6b};};
\node[stars] at (axis cs:{101.380},{-30.949}) {\tikz\pgfuseplotmark{m6avb};};
\node[stars] at (axis cs:{101.476},{12.693}) {\tikz\pgfuseplotmark{m6c};};
\node[stars] at (axis cs:{101.497},{-14.796}) {\tikz\pgfuseplotmark{m5c};};
\node[stars] at (axis cs:{101.551},{-37.775}) {\tikz\pgfuseplotmark{m6cv};};
\node[stars] at (axis cs:{101.635},{8.587}) {\tikz\pgfuseplotmark{m6b};};
\node[stars] at (axis cs:{101.663},{-10.107}) {\tikz\pgfuseplotmark{m6a};};
\node[stars] at (axis cs:{101.713},{-14.426}) {\tikz\pgfuseplotmark{m5c};};
\node[stars] at (axis cs:{101.756},{-21.015}) {\tikz\pgfuseplotmark{m6bv};};
\node[stars] at (axis cs:{101.833},{8.037}) {\tikz\pgfuseplotmark{m5a};};
\node[stars] at (axis cs:{101.839},{-37.929}) {\tikz\pgfuseplotmark{m5c};};
\node[stars] at (axis cs:{101.848},{18.193}) {\tikz\pgfuseplotmark{m6c};};
\node[stars] at (axis cs:{101.905},{-8.998}) {\tikz\pgfuseplotmark{m5bv};};
\node[stars] at (axis cs:{101.965},{2.412}) {\tikz\pgfuseplotmark{m4c};};
\node[stars] at (axis cs:{102.079},{-1.319}) {\tikz\pgfuseplotmark{m6a};};
\node[stars] at (axis cs:{102.241},{-15.145}) {\tikz\pgfuseplotmark{m5cv};};
\node[stars] at (axis cs:{102.265},{1.002}) {\tikz\pgfuseplotmark{m6cv};};
\node[stars] at (axis cs:{102.318},{-2.272}) {\tikz\pgfuseplotmark{m6a};};
\node[stars] at (axis cs:{102.422},{32.607}) {\tikz\pgfuseplotmark{m6av};};
\node[stars] at (axis cs:{102.458},{16.203}) {\tikz\pgfuseplotmark{m6bv};};
\node[stars] at (axis cs:{102.460},{-32.508}) {\tikz\pgfuseplotmark{m4av};};
\node[stars] at (axis cs:{102.591},{-17.085}) {\tikz\pgfuseplotmark{m6a};};
\node[stars] at (axis cs:{102.597},{-31.706}) {\tikz\pgfuseplotmark{m6avb};};
\node[stars] at (axis cs:{102.606},{13.413}) {\tikz\pgfuseplotmark{m6a};};
\node[stars] at (axis cs:{102.654},{-25.778}) {\tikz\pgfuseplotmark{m6c};};
\node[stars] at (axis cs:{102.676},{-8.041}) {\tikz\pgfuseplotmark{m6cv};};
\node[stars] at (axis cs:{102.708},{-0.541}) {\tikz\pgfuseplotmark{m6a};};
\node[stars] at (axis cs:{102.718},{-34.367}) {\tikz\pgfuseplotmark{m5b};};
\node[stars] at (axis cs:{102.888},{21.761}) {\tikz\pgfuseplotmark{m5c};};
\node[stars] at (axis cs:{102.914},{3.042}) {\tikz\pgfuseplotmark{m6c};};
\node[stars] at (axis cs:{102.927},{-36.230}) {\tikz\pgfuseplotmark{m6b};};
\node[stars] at (axis cs:{103.000},{23.602}) {\tikz\pgfuseplotmark{m6a};};
\node[stars] at (axis cs:{103.095},{-5.316}) {\tikz\pgfuseplotmark{m6c};};
\node[stars] at (axis cs:{103.197},{33.961}) {\tikz\pgfuseplotmark{m4a};};
\node[stars] at (axis cs:{103.206},{8.380}) {\tikz\pgfuseplotmark{m6a};};
\node[stars] at (axis cs:{103.251},{-26.958}) {\tikz\pgfuseplotmark{m6cv};};
\node[stars] at (axis cs:{103.256},{38.869}) {\tikz\pgfuseplotmark{m6bvb};};
\node[stars] at (axis cs:{103.265},{35.788}) {\tikz\pgfuseplotmark{m6b};};
\node[stars] at (axis cs:{103.306},{38.438}) {\tikz\pgfuseplotmark{m6c};};
\node[stars] at (axis cs:{103.328},{-19.032}) {\tikz\pgfuseplotmark{m6a};};
\node[stars] at (axis cs:{103.341},{-18.933}) {\tikz\pgfuseplotmark{m6b};};
\node[stars] at (axis cs:{103.343},{10.996}) {\tikz\pgfuseplotmark{m6c};};
\node[stars] at (axis cs:{103.387},{-20.224}) {\tikz\pgfuseplotmark{m5av};};
\node[stars] at (axis cs:{103.391},{-28.540}) {\tikz\pgfuseplotmark{m6b};};
\node[stars] at (axis cs:{103.481},{-24.539}) {\tikz\pgfuseplotmark{m6c};};
\node[stars] at (axis cs:{103.488},{38.505}) {\tikz\pgfuseplotmark{m6c};};
\node[stars] at (axis cs:{103.533},{-24.184}) {\tikz\pgfuseplotmark{m4av};};
\node[stars] at (axis cs:{103.536},{-5.852}) {\tikz\pgfuseplotmark{m6cb};};
\node[stars] at (axis cs:{103.547},{-12.039}) {\tikz\pgfuseplotmark{m4b};};
\node[stars] at (axis cs:{103.603},{-1.127}) {\tikz\pgfuseplotmark{m5c};};
\node[stars] at (axis cs:{103.661},{13.178}) {\tikz\pgfuseplotmark{m5ab};};
\node[stars] at (axis cs:{103.675},{-1.756}) {\tikz\pgfuseplotmark{m6cv};};
\node[stars] at (axis cs:{103.745},{-2.804}) {\tikz\pgfuseplotmark{m6b};};
\node[stars] at (axis cs:{103.761},{-20.405}) {\tikz\pgfuseplotmark{m6ab};};
\node[stars] at (axis cs:{103.828},{25.376}) {\tikz\pgfuseplotmark{m6av};};
\node[stars] at (axis cs:{103.894},{8.324}) {\tikz\pgfuseplotmark{m6c};};
\node[stars] at (axis cs:{103.906},{-20.136}) {\tikz\pgfuseplotmark{m5av};};
\node[stars] at (axis cs:{103.946},{-22.941}) {\tikz\pgfuseplotmark{m5cv};};
\node[stars] at (axis cs:{103.978},{-31.790}) {\tikz\pgfuseplotmark{m6cv};};
\node[stars] at (axis cs:{104.028},{-14.043}) {\tikz\pgfuseplotmark{m5bb};};
\node[stars] at (axis cs:{104.034},{-17.054}) {\tikz\pgfuseplotmark{m4cv};};
\node[stars] at (axis cs:{104.108},{9.956}) {\tikz\pgfuseplotmark{m6b};};
\node[stars] at (axis cs:{104.190},{-35.341}) {\tikz\pgfuseplotmark{m6c};};
\node[stars] at (axis cs:{104.250},{-8.179}) {\tikz\pgfuseplotmark{m6c};};
\node[stars] at (axis cs:{104.252},{33.681}) {\tikz\pgfuseplotmark{m6b};};
\node[stars] at (axis cs:{104.323},{-35.507}) {\tikz\pgfuseplotmark{m6cv};};
\node[stars] at (axis cs:{104.357},{11.907}) {\tikz\pgfuseplotmark{m6c};};
\node[stars] at (axis cs:{104.391},{-24.631}) {\tikz\pgfuseplotmark{m5c};};
\node[stars] at (axis cs:{104.428},{-27.537}) {\tikz\pgfuseplotmark{m6cv};};
\node[stars] at (axis cs:{104.531},{-27.164}) {\tikz\pgfuseplotmark{m6cv};};
\node[stars] at (axis cs:{104.605},{-34.112}) {\tikz\pgfuseplotmark{m5b};};
\node[stars] at (axis cs:{104.650},{-25.414}) {\tikz\pgfuseplotmark{m6a};};
\node[stars] at (axis cs:{104.656},{-28.972}) {\tikz\pgfuseplotmark{m2a};};
\node[stars] at (axis cs:{104.662},{7.622}) {\tikz\pgfuseplotmark{m6c};};
\node[stars] at (axis cs:{104.682},{-30.998}) {\tikz\pgfuseplotmark{m6cb};};
\node[stars] at (axis cs:{104.698},{26.081}) {\tikz\pgfuseplotmark{m6cv};};
\node[stars] at (axis cs:{104.738},{3.602}) {\tikz\pgfuseplotmark{m6b};};
\node[stars] at (axis cs:{104.762},{38.052}) {\tikz\pgfuseplotmark{m6b};};
\node[stars] at (axis cs:{104.834},{7.317}) {\tikz\pgfuseplotmark{m6c};};
\node[stars] at (axis cs:{104.866},{25.914}) {\tikz\pgfuseplotmark{m6c};};
\node[stars] at (axis cs:{104.914},{-21.603}) {\tikz\pgfuseplotmark{m6c};};
\node[stars] at (axis cs:{105.035},{-20.158}) {\tikz\pgfuseplotmark{m6c};};
\node[stars] at (axis cs:{105.066},{16.079}) {\tikz\pgfuseplotmark{m6av};};
\node[stars] at (axis cs:{105.075},{-5.367}) {\tikz\pgfuseplotmark{m6c};};
\node[stars] at (axis cs:{105.099},{-8.407}) {\tikz\pgfuseplotmark{m6b};};
\node[stars] at (axis cs:{105.177},{-28.489}) {\tikz\pgfuseplotmark{m6c};};
\node[stars] at (axis cs:{105.207},{-33.465}) {\tikz\pgfuseplotmark{m6c};};
\node[stars] at (axis cs:{105.275},{-25.215}) {\tikz\pgfuseplotmark{m6av};};
\node[stars] at (axis cs:{105.430},{-27.935}) {\tikz\pgfuseplotmark{m3cv};};
\node[stars] at (axis cs:{105.470},{-1.346}) {\tikz\pgfuseplotmark{m6cb};};
\node[stars] at (axis cs:{105.485},{-5.722}) {\tikz\pgfuseplotmark{m5cv};};
\node[stars] at (axis cs:{105.573},{15.336}) {\tikz\pgfuseplotmark{m6a};};
\node[stars] at (axis cs:{105.603},{24.215}) {\tikz\pgfuseplotmark{m5cv};};
\node[stars] at (axis cs:{105.606},{17.755}) {\tikz\pgfuseplotmark{m6bv};};
\node[stars] at (axis cs:{105.639},{16.674}) {\tikz\pgfuseplotmark{m6av};};
\node[stars] at (axis cs:{105.728},{-4.239}) {\tikz\pgfuseplotmark{m5bv};};
\node[stars] at (axis cs:{105.756},{-23.833}) {\tikz\pgfuseplotmark{m3bv};};
\node[stars] at (axis cs:{105.825},{9.138}) {\tikz\pgfuseplotmark{m6b};};
\node[stars] at (axis cs:{105.877},{29.337}) {\tikz\pgfuseplotmark{m6b};};
\node[stars] at (axis cs:{105.909},{10.952}) {\tikz\pgfuseplotmark{m5b};};
\node[stars] at (axis cs:{105.940},{-15.633}) {\tikz\pgfuseplotmark{m4b};};
\node[stars] at (axis cs:{105.965},{12.594}) {\tikz\pgfuseplotmark{m6b};};
\node[stars] at (axis cs:{105.989},{-10.124}) {\tikz\pgfuseplotmark{m6c};};
\node[stars] at (axis cs:{106.022},{-5.324}) {\tikz\pgfuseplotmark{m6a};};
\node[stars] at (axis cs:{106.027},{20.570}) {\tikz\pgfuseplotmark{m4bvb};};
\node[stars] at (axis cs:{106.196},{-22.031}) {\tikz\pgfuseplotmark{m6b};};
\node[stars] at (axis cs:{106.327},{22.637}) {\tikz\pgfuseplotmark{m6b};};
\node[stars] at (axis cs:{106.383},{-34.778}) {\tikz\pgfuseplotmark{m6cb};};
\node[stars] at (axis cs:{106.413},{9.186}) {\tikz\pgfuseplotmark{m6a};};
\node[stars] at (axis cs:{106.502},{-30.656}) {\tikz\pgfuseplotmark{m6cv};};
\node[stars] at (axis cs:{106.509},{-38.383}) {\tikz\pgfuseplotmark{m6b};};
\node[stars] at (axis cs:{106.548},{34.474}) {\tikz\pgfuseplotmark{m6a};};
\node[stars] at (axis cs:{106.650},{-12.394}) {\tikz\pgfuseplotmark{m6c};};
\node[stars] at (axis cs:{106.670},{-11.294}) {\tikz\pgfuseplotmark{m5cvb};};
\node[stars] at (axis cs:{106.719},{-24.960}) {\tikz\pgfuseplotmark{m6b};};
\node[stars] at (axis cs:{106.777},{4.910}) {\tikz\pgfuseplotmark{m6b};};
\node[stars] at (axis cs:{106.843},{34.009}) {\tikz\pgfuseplotmark{m6b};};
\node[stars] at (axis cs:{106.844},{-23.841}) {\tikz\pgfuseplotmark{m6av};};
\node[stars] at (axis cs:{106.854},{28.177}) {\tikz\pgfuseplotmark{m6c};};
\node[stars] at (axis cs:{106.956},{7.471}) {\tikz\pgfuseplotmark{m6a};};
\node[stars] at (axis cs:{107.055},{33.832}) {\tikz\pgfuseplotmark{m6c};};
\node[stars] at (axis cs:{107.092},{15.930}) {\tikz\pgfuseplotmark{m5c};};
\node[stars] at (axis cs:{107.098},{-26.393}) {\tikz\pgfuseplotmark{m2av};};
\node[stars] at (axis cs:{107.151},{37.445}) {\tikz\pgfuseplotmark{m6c};};
\node[stars] at (axis cs:{107.213},{-39.656}) {\tikz\pgfuseplotmark{m5av};};
\node[stars] at (axis cs:{107.334},{-10.346}) {\tikz\pgfuseplotmark{m6c};};
\node[stars] at (axis cs:{107.389},{-16.234}) {\tikz\pgfuseplotmark{m6bv};};
\node[stars] at (axis cs:{107.429},{-25.231}) {\tikz\pgfuseplotmark{m6av};};
\node[stars] at (axis cs:{107.528},{21.247}) {\tikz\pgfuseplotmark{m6c};};
\node[stars] at (axis cs:{107.539},{-18.686}) {\tikz\pgfuseplotmark{m6c};};
\node[stars] at (axis cs:{107.557},{-4.237}) {\tikz\pgfuseplotmark{m5bb};};
\node[stars] at (axis cs:{107.581},{-27.491}) {\tikz\pgfuseplotmark{m5c};};
\node[stars] at (axis cs:{107.763},{-3.898}) {\tikz\pgfuseplotmark{m6c};};
\node[stars] at (axis cs:{107.785},{30.245}) {\tikz\pgfuseplotmark{m4cv};};
\node[stars] at (axis cs:{107.846},{26.856}) {\tikz\pgfuseplotmark{m6av};};
\node[stars] at (axis cs:{107.848},{-0.302}) {\tikz\pgfuseplotmark{m5cv};};
\node[stars] at (axis cs:{107.914},{39.320}) {\tikz\pgfuseplotmark{m5b};};
\node[stars] at (axis cs:{107.923},{-20.883}) {\tikz\pgfuseplotmark{m6a};};
\node[stars] at (axis cs:{107.964},{5.655}) {\tikz\pgfuseplotmark{m6b};};
\node[stars] at (axis cs:{107.966},{-0.493}) {\tikz\pgfuseplotmark{m4c};};
\node[stars] at (axis cs:{108.017},{-30.821}) {\tikz\pgfuseplotmark{m6b};};
\node[stars] at (axis cs:{108.031},{5.474}) {\tikz\pgfuseplotmark{m6c};};
\node[stars] at (axis cs:{108.051},{-25.942}) {\tikz\pgfuseplotmark{m6bv};};
\node[stars] at (axis cs:{108.108},{-36.544}) {\tikz\pgfuseplotmark{m6bvb};};
\node[stars] at (axis cs:{108.110},{24.128}) {\tikz\pgfuseplotmark{m6a};};
\node[stars] at (axis cs:{108.280},{-11.251}) {\tikz\pgfuseplotmark{m6a};};
\node[stars] at (axis cs:{108.343},{16.159}) {\tikz\pgfuseplotmark{m5bv};};
\node[stars] at (axis cs:{108.350},{-22.674}) {\tikz\pgfuseplotmark{m6bv};};
\node[stars] at (axis cs:{108.402},{-27.356}) {\tikz\pgfuseplotmark{m6bv};};
\node[stars] at (axis cs:{108.452},{-22.906}) {\tikz\pgfuseplotmark{m6cb};};
\node[stars] at (axis cs:{108.489},{-30.340}) {\tikz\pgfuseplotmark{m6c};};
\node[stars] at (axis cs:{108.545},{-3.902}) {\tikz\pgfuseplotmark{m6b};};
\node[stars] at (axis cs:{108.563},{-26.352}) {\tikz\pgfuseplotmark{m4cv};};
\node[stars] at (axis cs:{108.565},{-9.947}) {\tikz\pgfuseplotmark{m6b};};
\node[stars] at (axis cs:{108.584},{3.111}) {\tikz\pgfuseplotmark{m5c};};
\node[stars] at (axis cs:{108.618},{-10.316}) {\tikz\pgfuseplotmark{m6b};};
\node[stars] at (axis cs:{108.636},{12.116}) {\tikz\pgfuseplotmark{m6a};};
\node[stars] at (axis cs:{108.675},{24.885}) {\tikz\pgfuseplotmark{m6av};};
\node[stars] at (axis cs:{108.703},{-26.773}) {\tikz\pgfuseplotmark{m4bv};};
\node[stars] at (axis cs:{108.713},{-27.038}) {\tikz\pgfuseplotmark{m6a};};
\node[stars] at (axis cs:{108.831},{-0.161}) {\tikz\pgfuseplotmark{m6c};};
\node[stars] at (axis cs:{108.838},{-30.686}) {\tikz\pgfuseplotmark{m5cv};};
\node[stars] at (axis cs:{108.914},{7.978}) {\tikz\pgfuseplotmark{m6av};};
\node[stars] at (axis cs:{108.930},{-10.584}) {\tikz\pgfuseplotmark{m6b};};
\node[stars] at (axis cs:{108.947},{-23.740}) {\tikz\pgfuseplotmark{m6c};};
\node[stars] at (axis cs:{108.988},{27.897}) {\tikz\pgfuseplotmark{m6av};};
\node[stars] at (axis cs:{109.061},{-15.586}) {\tikz\pgfuseplotmark{m5cv};};
\node[stars] at (axis cs:{109.133},{-38.319}) {\tikz\pgfuseplotmark{m6a};};
\node[stars] at (axis cs:{109.146},{-27.881}) {\tikz\pgfuseplotmark{m5av};};
\node[stars] at (axis cs:{109.153},{-23.315}) {\tikz\pgfuseplotmark{m5avb};};
\node[stars] at (axis cs:{109.206},{-36.592}) {\tikz\pgfuseplotmark{m5bv};};
\node[stars] at (axis cs:{109.238},{-30.897}) {\tikz\pgfuseplotmark{m6cb};};
\node[stars] at (axis cs:{109.264},{26.689}) {\tikz\pgfuseplotmark{m6c};};
\node[stars] at (axis cs:{109.286},{-37.097}) {\tikz\pgfuseplotmark{m3av};};
\node[stars] at (axis cs:{109.382},{-6.680}) {\tikz\pgfuseplotmark{m6c};};
\node[stars] at (axis cs:{109.428},{38.876}) {\tikz\pgfuseplotmark{m6c};};
\node[stars] at (axis cs:{109.450},{-26.797}) {\tikz\pgfuseplotmark{m6c};};
\node[stars] at (axis cs:{109.517},{30.956}) {\tikz\pgfuseplotmark{m6c};};
\node[stars] at (axis cs:{109.523},{16.540}) {\tikz\pgfuseplotmark{m4av};};
\node[stars] at (axis cs:{109.577},{-36.734}) {\tikz\pgfuseplotmark{m5avb};};
\node[stars] at (axis cs:{109.640},{-39.210}) {\tikz\pgfuseplotmark{m5c};};
\node[stars] at (axis cs:{109.668},{-24.559}) {\tikz\pgfuseplotmark{m5bv};};
\node[stars] at (axis cs:{109.677},{-24.954}) {\tikz\pgfuseplotmark{m4cvb};};
\node[stars] at (axis cs:{109.714},{-26.586}) {\tikz\pgfuseplotmark{m5cv};};
\node[stars] at (axis cs:{109.758},{-19.280}) {\tikz\pgfuseplotmark{m6b};};
\node[stars] at (axis cs:{109.807},{-33.727}) {\tikz\pgfuseplotmark{m6cv};};
\node[stars] at (axis cs:{109.843},{2.741}) {\tikz\pgfuseplotmark{m6b};};
\node[stars] at (axis cs:{109.867},{-16.395}) {\tikz\pgfuseplotmark{m6av};};
\node[stars] at (axis cs:{109.949},{7.143}) {\tikz\pgfuseplotmark{m6b};};
\end{axis}




\begin{axis}[name=base,axis lines=none]
 
\node[designation-label,anchor=north west]  at (axis cs:{04*15+35/4},{+16+30/ 60  })  {Aldebaran};    

\node[designation-label,anchor=north west]  at (axis cs:{05*15+36/4},{-01-12/ 60  })  {Alnilam};   
\node[designation-label,anchor=south west]  at (axis cs:{05*15+40/4},{-01-56/ 60  })  {Alnitak};   
\node[designation-label,anchor=north east]  at (axis cs:{05*15+55/4},{+07+24/ 60  })  {Betelgeuse};  

\node[designation-label,anchor=south west]  at (axis cs:{05*15+14/4},{- 8-12/ 60  })  {Rigel};   
\node[designation-label,anchor=north west]  at (axis cs:{05*15+25/4},{+06+20/ 60  })  {Bellatrix};   
\node[designation-label,anchor=south west]  at (axis cs:{05*15+26/4},{+28+36/ 60  })  {Alnath};   
\node[designation-label,anchor=north west]  at (axis cs:{06*15+22/4},{-17-57/ 60  })  {Mirzam};   
\node[designation-label,anchor=south west]  at (axis cs:{06*15+37/4},{+16+23/ 60  })  {Alhena};   
\node[designation-label,anchor=south west]  at (axis cs:{06*15+45/4},{-16-42/ 60  })  {Sirius};   
\node[designation-label,anchor=south west]  at (axis cs:{06*15+58/4},{-28-58/ 60  })  {Adara};   
\node[designation-label,anchor=east]  at (axis cs:{07*15+ 8/4},{-26-23/ 60  })  {Wezen}; 



\node[interest,pin={[pin distance=-0.4\onedegree,interest-label]-90:{V838}}] at (axis cs:7*15+4.08/4,-5+50.7/60) {\pgfuseplotmark{+}} ; % V838 Mon
\node[interest,pin={[pin distance=-0.4\onedegree,interest-label]-90:{v.\,Maanen's star}}] at (axis cs:12.29,5.39) {\pgfuseplotmark{+}} ;
\node[interest,pin={[pin distance=-0.4\onedegree,interest-label]0:{T}}] at (axis cs:65.498,19.535) {\pgfuseplotmark{+}} ; % T Tau

\node[stars,pin={[pin distance=-0.6\onedegree,Flaamsted]180:{AE}}] at (axis cs:79.06,34.31) {} ; % AE Aur

\node[stars,pin={[pin distance=-0.6\onedegree,Flaamsted]0:{UU}}] at (axis cs:099.1368212654100,+38.4455052837800) {} ; % UU Aur
\node[stars,pin={[pin distance=-0.6\onedegree,Flaamsted]180:{WW}}] at (axis cs:098.1132698606000,+32.4548978969700) {} ; % WW Aur

\node[stars,pin={[pin distance=-0.6\onedegree,Flaamsted]0:{R}}] at (axis cs:6.008,38.577)  {\tikz\pgfuseplotmark{m6av};} ; % R And
\node[stars,pin={[pin distance=-0.6\onedegree,Flaamsted]0:{R}}] at (axis cs:39.26,34.264) {\tikz\pgfuseplotmark{m6cv};} ; % R Tri
\node[stars,pin={[pin distance=-0.6\onedegree,Flaamsted]-45:{X}}] at (axis cs:058.8461572733700,+31.0458443989300)  {\tikz\pgfuseplotmark{m6cv};} ; % X Per
\node[stars,pin={[pin distance=-0.6\onedegree,Flaamsted]270:{T}}] at (axis cs:005.4428049075,-20.0580235628) {} ; % T Cet

\node[stars,pin={[pin distance=-0.6\onedegree,Flaamsted]-90:{R}}] at (axis cs:73.8277409664600,-16.4177706596900) {} ; % R Eri
\node[stars,pin={[pin distance=-0.6\onedegree,Flaamsted]0:{S}}] at (axis cs:091.4397903266700,-24.1955597292800)  {\tikz\pgfuseplotmark{m6cv};} ; % S Lep
\node[stars,pin={[pin distance=-0.6\onedegree,Flaamsted]0:{R}}] at (axis cs:74.9014528838000,-14.8062506886500)  {\tikz\pgfuseplotmark{m6cv};} ; % R Lep
\node[stars,pin={[pin distance=-0.6\onedegree,Flaamsted]0:{T}}] at (axis cs:096.3041653170500,+07.0857107174700) {} ; % T Mon


\node[pin={[pin distance=-0.2\onedegree,Bayer]-90:{$\alpha$}}] at (axis cs:{2.097},{29.090}) {}; % And,  2.06 
\node[pin={[pin distance=-0.2\onedegree,Bayer]00:{$\beta$}}] at (axis cs:{17.433},{35.620}) {}; % And,  2.08 
\node[pin={[pin distance=-0.4\onedegree,Bayer]00:{$\delta$}}] at (axis cs:{9.832},{30.861}) {}; % And,  3.27 
\node[pin={[pin distance=-0.4\onedegree,Bayer]00:{$\epsilon$}}] at (axis cs:{9.639},{29.312}) {}; % And,  4.35 
\node[pin={[pin distance=-0.4\onedegree,Bayer]135:{$\eta$}}] at (axis cs:{14.302},{23.418}) {}; % And,  4.40 
\node[pin={[pin distance=-0.4\onedegree,Bayer]00:{$\mu$}}] at (axis cs:{14.188},{38.499}) {}; % And,  3.87 
\node[pin={[pin distance=-0.4\onedegree,Bayer]00:{$\pi$}}] at (axis cs:{9.220},{33.719}) {}; % And,  4.35 
\node[pin={[pin distance=-0.4\onedegree,Bayer]00:{$\rho$}}] at (axis cs:{5.280},{37.969}) {}; % And,  5.16 
\node[pin={[pin distance=-0.4\onedegree,Bayer]270:{$\sigma$}}] at (axis cs:{4.582},{36.785}) {}; % And,  4.51 
\node[pin={[pin distance=-0.4\onedegree,Bayer]00:{$\vartheta$}}] at (axis cs:{4.273},{38.681}) {}; % And,  4.62 
\node[pin={[pin distance=-0.4\onedegree,Bayer]180:{$\zeta$}}] at (axis cs:{11.835},{24.267}) {}; % And,  4.09 
\node[pin={[pin distance=-0.4\onedegree,Flaamsted]00:{$ 28$}}] at (axis cs:{7.531},{29.752}) {}; % And,  5.22 
\node[pin={[pin distance=-0.4\onedegree,Flaamsted]00:{$ 32$}}] at (axis cs:{10.280},{39.459}) {}; % And,  5.31 
\node[pin={[pin distance=-0.4\onedegree,Flaamsted]180:{$ 36$}}] at (axis cs:{13.742},{23.628}) {}; % And,  5.50 
\node[pin={[pin distance=-0.4\onedegree,Flaamsted]00:{$ 45$}}] at (axis cs:{17.793},{37.724}) {}; % And,  5.79 
\node[pin={[pin distance=-0.4\onedegree,Flaamsted]00:{$ 47$}}] at (axis cs:{20.919},{37.715}) {}; % And,  5.59 
\node[pin={[pin distance=-0.4\onedegree,Flaamsted]90:{$ 56$}}] at (axis cs:{29.039},{37.252}) {}; % And,  5.69 
\node[pin={[pin distance=-0.4\onedegree,Flaamsted]00:{$ 58$}}] at (axis cs:{32.122},{37.859}) {}; % And,  4.79 
\node[pin={[pin distance=-0.2\onedegree,Bayer]00:{$\alpha$}}] at (axis cs:{31.793},{23.462}) {}; % Ari,  2.02 
\node[pin={[pin distance=-0.4\onedegree,Bayer]00:{$\beta$}}] at (axis cs:{28.660},{20.808}) {}; % Ari,  2.66 
\node[pin={[pin distance=-0.4\onedegree,Bayer]00:{$\delta$}}] at (axis cs:{47.907},{19.727}) {}; % Ari,  4.35 
\node[pin={[pin distance=-0.4\onedegree,Bayer]00:{$\epsilon$}}] at (axis cs:{44.803},{21.340}) {}; % Ari,  4.63 
\node[pin={[pin distance=-0.4\onedegree,Bayer]00:{$\epsilon$}}] at (axis cs:{44.803},{21.340}) {}; % Ari,  4.63 
\node[pin={[pin distance=-0.4\onedegree,Bayer]00:{$\eta$}}] at (axis cs:{33.200},{21.211}) {}; % Ari,  5.23 
\node[pin={[pin distance=-0.4\onedegree,Bayer]00:{$\gamma$}}] at (axis cs:{28.383},{19.296}) {}; % Ari,  4.64 
\node[pin={[pin distance=-0.4\onedegree,Bayer]00:{$\iota$}}] at (axis cs:{29.338},{17.818}) {}; % Ari,  5.10 
\node[pin={[pin distance=-0.4\onedegree,Bayer]00:{$\kappa$}}] at (axis cs:{31.641},{22.648}) {}; % Ari,  5.03 
\node[pin={[pin distance=-0.4\onedegree,Bayer]90:{$\lambda$}}] at (axis cs:{29.482},{23.596}) {}; % Ari,  4.77 
\node[pin={[pin distance=-0.4\onedegree,Bayer]00:{$\mu$}}] at (axis cs:{40.591},{20.011}) {}; % Ari,  5.74 
\node[pin={[pin distance=-0.4\onedegree,Bayer]00:{$\nu$}}] at (axis cs:{39.704},{21.961}) {}; % Ari,  5.45 
\node[pin={[pin distance=-0.4\onedegree,Bayer]00:{$\omicron$}}] at (axis cs:{41.137},{15.312}) {}; % Ari,  5.78 
\node[pin={[pin distance=-0.4\onedegree,Bayer]180:{$\pi$}}] at (axis cs:{42.323},{17.464}) {}; % Ari,  5.32 
\node[pin={[pin distance=-0.4\onedegree,Bayer]-90:{$\rho^2$}}] at (axis cs:{43.952},{18.331}) {}; % Ari,  5.82 
\node[pin={[pin distance=-0.4\onedegree,Bayer]0:{$\rho^3$}}] at (axis cs:{44.109},{18.023}) {}; % Ari,  5.59 
\node[pin={[pin distance=-0.4\onedegree,Bayer]00:{$\sigma$}}] at (axis cs:{42.873},{15.082}) {}; % Ari,  5.52 
\node[pin={[pin distance=-0.4\onedegree,Bayer]-90:{$\tau^1$}}] at (axis cs:{50.307},{21.147}) {}; % Ari,  5.31 
\node[pin={[pin distance=-0.4\onedegree,Bayer]90:{$\tau^2$}}] at (axis cs:{50.689},{20.742}) {}; % Ari,  5.08 
\node[pin={[pin distance=-0.4\onedegree,Bayer]00:{$\vartheta$}}] at (axis cs:{34.531},{19.901}) {}; % Ari,  5.58 
\node[pin={[pin distance=-0.4\onedegree,Bayer]90:{$\xi$}}] at (axis cs:{36.204},{10.610}) {}; % Ari,  5.48 
\node[pin={[pin distance=-0.4\onedegree,Bayer]180:{$\zeta$}}] at (axis cs:{48.725},{21.044}) {}; % Ari,  4.88 
\node[pin={[pin distance=-0.4\onedegree,Flaamsted]90:{$ 10$}}] at (axis cs:{30.914},{25.935}) {}; % Ari,  5.69 
\node[pin={[pin distance=-0.4\onedegree,Flaamsted]90:{$ 14$}}] at (axis cs:{32.356},{25.940}) {}; % Ari,  4.98 
\node[pin={[pin distance=-0.4\onedegree,Flaamsted]00:{$ 15$}}] at (axis cs:{32.657},{19.500}) {}; % Ari,  5.71 
\node[pin={[pin distance=-0.4\onedegree,Flaamsted]00:{$ 19$}}] at (axis cs:{33.264},{15.280}) {}; % Ari,  5.71 
\node[pin={[pin distance=-0.4\onedegree,Flaamsted]00:{$  1$}}] at (axis cs:{27.536},{22.275}) {}; % Ari,  5.83 
\node[pin={[pin distance=-0.4\onedegree,Flaamsted]00:{$ 20$}}] at (axis cs:{33.942},{25.783}) {}; % Ari,  5.79 
\node[pin={[pin distance=-0.4\onedegree,Flaamsted]00:{$ 21$}}] at (axis cs:{33.928},{25.043}) {}; % Ari,  5.58 
\node[pin={[pin distance=-0.4\onedegree,Flaamsted]00:{$ 31$}}] at (axis cs:{39.158},{12.447}) {}; % Ari,  5.65 
\node[pin={[pin distance=-0.4\onedegree,Flaamsted]00:{$ 33$}}] at (axis cs:{40.171},{27.061}) {}; % Ari,  5.30 
\node[pin={[pin distance=-0.4\onedegree,Flaamsted]00:{$ 35$}}] at (axis cs:{40.863},{27.707}) {}; % Ari,  4.65 
\node[pin={[pin distance=-0.4\onedegree,Flaamsted]00:{$ 38$}}] at (axis cs:{41.240},{12.446}) {}; % Ari,  5.19 
\node[pin={[pin distance=-0.4\onedegree,Flaamsted]00:{$ 39$}}] at (axis cs:{41.977},{29.247}) {}; % Ari,  4.52 
\node[pin={[pin distance=-0.4\onedegree,Flaamsted]180:{$ 40$}}] at (axis cs:{42.134},{18.284}) {}; % Ari,  5.83 
\node[pin={[pin distance=-0.4\onedegree,Flaamsted]00:{$ 41$}}] at (axis cs:{42.496},{27.260}) {}; % Ari,  3.62 
\node[pin={[pin distance=-0.4\onedegree,Flaamsted]00:{$ 47$}}] at (axis cs:{44.522},{20.669}) {}; % Ari,  5.80 
\node[pin={[pin distance=-0.4\onedegree,Flaamsted]00:{$ 49$}}] at (axis cs:{45.476},{26.462}) {}; % Ari,  5.92 
\node[pin={[pin distance=-0.4\onedegree,Flaamsted]180:{$  4$}}] at (axis cs:{27.046},{16.955}) {}; % Ari,  5.86 
\node[pin={[pin distance=-0.4\onedegree,Flaamsted]00:{$ 52$}}] at (axis cs:{46.361},{25.255}) {}; % Ari,  5.46 
\node[pin={[pin distance=-0.4\onedegree,Flaamsted]00:{$ 55$}}] at (axis cs:{47.403},{29.077}) {}; % Ari,  5.75 
\node[pin={[pin distance=-0.4\onedegree,Flaamsted]00:{$ 56$}}] at (axis cs:{48.059},{27.257}) {}; % Ari,  5.78 
\node[pin={[pin distance=-0.4\onedegree,Flaamsted]00:{$ 59$}}] at (axis cs:{49.982},{27.071}) {}; % Ari,  5.92 
\node[pin={[pin distance=-0.4\onedegree,Flaamsted]00:{$ 62$}}] at (axis cs:{50.550},{27.607}) {}; % Ari,  5.55 
\node[pin={[pin distance=-0.4\onedegree,Flaamsted]00:{$ 64$}}] at (axis cs:{51.077},{24.724}) {}; % Ari,  5.50 
\node[pin={[pin distance=-0.4\onedegree,Flaamsted]00:{$  7$}}] at (axis cs:{28.963},{23.577}) {}; % Ari,  5.75 
\node[pin={[pin distance=-0.4\onedegree,Bayer]00:{$\chi$}}] at (axis cs:{83.182},{32.192}) {}; % Aur,  4.74 
\node[pin={[pin distance=-0.4\onedegree,Bayer]00:{$\iota$}}] at (axis cs:{74.248},{33.166}) {}; % Aur,  2.68 
\node[pin={[pin distance=-0.4\onedegree,Bayer]00:{$\kappa$}}] at (axis cs:{93.845},{29.498}) {}; % Aur,  4.33 
\node[pin={[pin distance=-0.4\onedegree,Bayer]180:{$\mu$}}] at (axis cs:{78.357},{38.484}) {}; % Aur,  4.83 
\node[pin={[pin distance=-0.4\onedegree,Bayer]90:{$\nu$}}] at (axis cs:{87.872},{39.148}) {}; % Aur,  3.97 
\node[pin={[pin distance=-0.4\onedegree,Bayer]00:{$\omega$}}] at (axis cs:{74.814},{37.890}) {}; % Aur,  4.99 
\node[pin={[pin distance=-0.4\onedegree,Bayer]-135:{$\psi^3$}}] at (axis cs:{99.705},{39.903}) {}; % Aur,  5.35 
\node[pin={[pin distance=-0.4\onedegree,Bayer]00:{$\sigma$}}] at (axis cs:{81.163},{37.385}) {}; % Aur,  4.99 
\node[pin={[pin distance=-0.4\onedegree,Bayer]180:{$\tau$}}] at (axis cs:{87.293},{39.181}) {}; % Aur,  4.52 
\node[pin={[pin distance=-0.4\onedegree,Bayer]00:{$\upsilon$}}] at (axis cs:{87.760},{37.305}) {}; % Aur,  4.73 
\node[pin={[pin distance=-0.4\onedegree,Bayer]180:{$\varphi$}}] at (axis cs:{81.912},{34.476}) {}; % Aur,  5.06 
\node[pin={[pin distance=-0.4\onedegree,Bayer]00:{$\vartheta$}}] at (axis cs:{89.930},{37.212}) {}; % Aur,  2.65 
\node[pin={[pin distance=-0.4\onedegree,Flaamsted]180:{$ 14$}}] at (axis cs:{78.852},{32.688}) {}; % Aur,  4.99 
\node[pin={[pin distance=-0.4\onedegree,Flaamsted]90:{$ 16$}}] at (axis cs:{79.544},{33.372}) {}; % Aur,  4.54 
\node[pin={[pin distance=-0.4\onedegree,Flaamsted]-90:{$ 19$}}] at (axis cs:{80.004},{33.958}) {}; % Aur,  5.04 
\node[pin={[pin distance=-0.4\onedegree,Flaamsted]00:{$ 26$}}] at (axis cs:{84.659},{30.492}) {}; % Aur,  5.41 
\node[pin={[pin distance=-0.4\onedegree,Flaamsted]00:{$  2$}}] at (axis cs:{73.158},{36.703}) {}; % Aur,  4.78 
\node[pin={[pin distance=-0.4\onedegree,Flaamsted]00:{$ 40$}}] at (axis cs:{91.646},{38.482}) {}; % Aur,  5.35 
\node[pin={[pin distance=-0.4\onedegree,Flaamsted]00:{$ 48$}}] at (axis cs:{97.142},{30.493}) {}; % Aur,  5.75 
\node[pin={[pin distance=-0.4\onedegree,Flaamsted]-90:{$ 49$}}] at (axis cs:{98.800},{28.022}) {}; % Aur,  5.27 
\node[pin={[pin distance=-0.4\onedegree,Flaamsted]00:{$ 51$}}] at (axis cs:{99.665},{39.391}) {}; % Aur,  5.70 
\node[pin={[pin distance=-0.4\onedegree,Flaamsted]-90:{$ 53$}}] at (axis cs:{99.596},{28.984}) {}; % Aur,  5.75 
\node[pin={[pin distance=-0.4\onedegree,Flaamsted]00:{$  5$}}] at (axis cs:{75.076},{39.395}) {}; % Aur,  5.98 
\node[pin={[pin distance=-0.4\onedegree,Flaamsted]00:{$ 63$}}] at (axis cs:{107.914},{39.320}) {}; % Aur,  4.89 
\node[pin={[pin distance=-0.4\onedegree,Bayer]00:{$\beta$}}] at (axis cs:{70.515},{-37.144}) {}; % Cae,  5.04 
\node[pin={[pin distance=-0.4\onedegree,Bayer]00:{$\gamma^1$}}] at (axis cs:{76.102},{-35.483}) {}; % Cae,  4.57 


\node[pin={[pin distance=-0.2\onedegree,Bayer]90:{$\beta$}}] at (axis cs:{10.897},{-17.987}) {}; % Cet,  2.05 
\node[pin={[pin distance=-0.4\onedegree,Bayer]00:{$\alpha$}}] at (axis cs:{45.570},{4.090}) {}; % Cet,  2.55 
\node[pin={[pin distance=-0.4\onedegree,Bayer]180:{$\chi$}}] at (axis cs:{27.396},{-10.686}) {}; % Cet,  4.66 
\node[pin={[pin distance=-0.4\onedegree,Bayer]180:{$\delta$}}] at (axis cs:{39.871},{0.328}) {}; % Cet,  4.07 
\node[pin={[pin distance=-0.4\onedegree,Bayer]00:{$\epsilon$}}] at (axis cs:{39.891},{-11.872}) {}; % Cet,  4.83 
\node[pin={[pin distance=-0.4\onedegree,Bayer]90:{$\eta$}}] at (axis cs:{17.147},{-10.182}) {}; % Cet,  3.46 
\node[pin={[pin distance=-0.4\onedegree,Bayer]00:{$\gamma$}}] at (axis cs:{40.825},{3.236}) {}; % Cet,  3.47 
\node[pin={[pin distance=-0.4\onedegree,Bayer]00:{$\iota$}}] at (axis cs:{4.857},{-8.824}) {}; % Cet,  3.55 
\node[pin={[pin distance=-0.4\onedegree,Bayer]90:{$\kappa^1$}}] at (axis cs:{49.840},{3.370}) {}; % Cet,  4.85 
\node[pin={[pin distance=-0.4\onedegree,Bayer]-90:{$\kappa^2$}}] at (axis cs:{50.278},{3.675}) {}; % Cet,  5.68 
\node[pin={[pin distance=-0.4\onedegree,Bayer]00:{$\lambda$}}] at (axis cs:{44.929},{8.907}) {}; % Cet,  4.71 
\node[pin={[pin distance=-0.4\onedegree,Bayer]00:{$\mu$}}] at (axis cs:{41.236},{10.114}) {}; % Cet,  4.27 
\node[pin={[pin distance=-0.4\onedegree,Bayer]00:{$\nu$}}] at (axis cs:{38.969},{5.593}) {}; % Cet,  4.88 
\node[pin={[pin distance=-0.4\onedegree,Bayer]00:{$\omicron$}}] at (axis cs:{34.837},{-2.977}) {}; % Cet,  4.95 
\node[pin={[pin distance=-0.4\onedegree,Proper]90:{Mira}}] at (axis cs:{34.837},{-2.977}) {}; % Cet,  4.95 
\node[pin={[pin distance=-0.4\onedegree,Bayer]00:{$\pi$}}] at (axis cs:{41.031},{-13.859}) {}; % Cet,  4.25 
\node[pin={[pin distance=-0.4\onedegree,Bayer]00:{$\rho$}}] at (axis cs:{36.488},{-12.290}) {}; % Cet,  4.88 
\node[pin={[pin distance=-0.4\onedegree,Bayer]00:{$\sigma$}}] at (axis cs:{38.022},{-15.244}) {}; % Cet,  4.74 
\node[pin={[pin distance=-0.4\onedegree,Bayer]00:{$\tau$}}] at (axis cs:{26.017},{-15.937}) {}; % Cet,  3.49 
\node[pin={[pin distance=-0.4\onedegree,Bayer]00:{$\upsilon$}}] at (axis cs:{30.001},{-21.078}) {}; % Cet,  4.00 
\node[pin={[pin distance=-0.4\onedegree,Bayer]180:{$\varphi^1$}}] at (axis cs:{11.048},{-10.609}) {}; % Cet,  4.77 
\node[pin={[pin distance=-0.4\onedegree,Bayer]-90:{$\varphi^2$}}] at (axis cs:{12.532},{-10.644}) {}; % Cet,  5.17 
\node[pin={[pin distance=-0.4\onedegree,Bayer]-90:{$\varphi^3$}}] at (axis cs:{14.006},{-11.266}) {}; % Cet,  5.34 
\node[pin={[pin distance=-0.4\onedegree,Bayer]90:{$\varphi^4$}}] at (axis cs:{14.683},{-11.380}) {}; % Cet,  5.62 
\node[pin={[pin distance=-0.4\onedegree,Bayer]00:{$\vartheta$}}] at (axis cs:{21.006},{-8.183}) {}; % Cet,  3.61 
\node[pin={[pin distance=-0.4\onedegree,Bayer]0:{$\xi^1$}}] at (axis cs:{33.250},{8.846}) {}; % Cet,  4.37 
\node[pin={[pin distance=-0.4\onedegree,Bayer]00:{$\xi^2$}}] at (axis cs:{37.040},{8.460}) {}; % Cet,  4.28 
\node[pin={[pin distance=-0.4\onedegree,Bayer]-90:{$\zeta$}}] at (axis cs:{27.865},{-10.335}) {}; % Cet,  3.73 
\node[pin={[pin distance=-0.4\onedegree,Flaamsted]-90:{$ 12$}}] at (axis cs:{7.510},{-3.957}) {}; % Cet,  5.73 
\node[pin={[pin distance=-0.4\onedegree,Flaamsted]00:{$ 13$}}] at (axis cs:{8.812},{-3.593}) {}; % Cet,  5.20 
\node[pin={[pin distance=-0.4\onedegree,Flaamsted]180:{$ 14$}}] at (axis cs:{8.887},{-0.506}) {}; % Cet,  5.94 
\node[pin={[pin distance=-0.4\onedegree,Flaamsted]00:{$ 20$}}] at (axis cs:{13.252},{-1.144}) {}; % Cet,  4.76 
\node[pin={[pin distance=-0.4\onedegree,Flaamsted]00:{$ 25$}}] at (axis cs:{15.761},{-4.837}) {}; % Cet,  5.40 
\node[pin={[pin distance=-0.4\onedegree,Flaamsted]180:{$ 28$}}] at (axis cs:{16.521},{-9.839}) {}; % Cet,  5.58 
\node[pin={[pin distance=-0.4\onedegree,Flaamsted]00:{$  2$}}] at (axis cs:{0.935},{-17.336}) {}; % Cet,  4.55 
\node[pin={[pin distance=-0.4\onedegree,Flaamsted]00:{$ 30$}}] at (axis cs:{16.943},{-9.786}) {}; % Cet,  5.71 
\node[pin={[pin distance=-0.4\onedegree,Flaamsted]00:{$ 33$}}] at (axis cs:{17.640},{2.445}) {}; % Cet,  5.95 
\node[pin={[pin distance=-0.4\onedegree,Flaamsted]00:{$ 34$}}] at (axis cs:{17.931},{-2.251}) {}; % Cet,  5.93 
\node[pin={[pin distance=-0.4\onedegree,Flaamsted]90:{$ 37$}}] at (axis cs:{18.600},{-7.923}) {}; % Cet,  5.15 
\node[pin={[pin distance=-0.4\onedegree,Flaamsted]00:{$ 38$}}] at (axis cs:{18.705},{-0.974}) {}; % Cet,  5.70 
\node[pin={[pin distance=-0.4\onedegree,Flaamsted]00:{$ 39$}}] at (axis cs:{19.151},{-2.500}) {}; % Cet,  5.43 
\node[pin={[pin distance=-0.4\onedegree,Flaamsted]00:{$  3$}}] at (axis cs:{1.125},{-10.509}) {}; % Cet,  4.94 
\node[pin={[pin distance=-0.4\onedegree,Flaamsted]00:{$ 46$}}] at (axis cs:{21.405},{-14.599}) {}; % Cet,  4.91 
\node[pin={[pin distance=-0.4\onedegree,Flaamsted]00:{$ 47$}}] at (axis cs:{21.715},{-13.056}) {}; % Cet,  5.51 
\node[pin={[pin distance=-0.4\onedegree,Flaamsted]00:{$ 48$}}] at (axis cs:{22.401},{-21.629}) {}; % Cet,  5.11 
\node[pin={[pin distance=-0.4\onedegree,Flaamsted]180:{$ 49$}}] at (axis cs:{23.657},{-15.676}) {}; % Cet,  5.62 
\node[pin={[pin distance=-0.4\onedegree,Flaamsted]-90:{$ 50$}}] at (axis cs:{23.996},{-15.400}) {}; % Cet,  5.41 
\node[pin={[pin distance=-0.4\onedegree,Flaamsted]00:{$ 56$}}] at (axis cs:{29.167},{-22.527}) {}; % Cet,  4.90 
\node[pin={[pin distance=-0.4\onedegree,Flaamsted]-90:{$ 57$}}] at (axis cs:{29.942},{-20.824}) {}; % Cet,  5.44 
\node[pin={[pin distance=-0.4\onedegree,Flaamsted]-90:{$ 60$}}] at (axis cs:{30.799},{0.128}) {}; % Cet,  5.43 
\node[pin={[pin distance=-0.4\onedegree,Flaamsted]00:{$ 61$}}] at (axis cs:{30.951},{-0.340}) {}; % Cet,  5.95 
\node[pin={[pin distance=-0.4\onedegree,Flaamsted]00:{$ 63$}}] at (axis cs:{32.899},{-1.825}) {}; % Cet,  5.94 
\node[pin={[pin distance=-0.4\onedegree,Flaamsted]90:{$ 64$}}] at (axis cs:{32.838},{8.570}) {}; % Cet,  5.64 
\node[pin={[pin distance=-0.4\onedegree,Flaamsted]00:{$ 66$}}] at (axis cs:{33.198},{-2.393}) {}; % Cet,  5.65 
\node[pin={[pin distance=-0.4\onedegree,Flaamsted]00:{$ 67$}}] at (axis cs:{34.246},{-6.422}) {}; % Cet,  5.50 
\node[pin={[pin distance=-0.4\onedegree,Flaamsted]00:{$ 69$}}] at (axis cs:{35.486},{0.395}) {}; % Cet,  5.30 
\node[pin={[pin distance=-0.4\onedegree,Flaamsted]00:{$  6$}}] at (axis cs:{2.816},{-15.468}) {}; % Cet,  4.90 
\node[pin={[pin distance=-0.4\onedegree,Flaamsted]00:{$ 70$}}] at (axis cs:{35.552},{-0.885}) {}; % Cet,  5.43 
\node[pin={[pin distance=-0.4\onedegree,Flaamsted]0:{$ 75$}}] at (axis cs:{38.039},{-1.035}) {}; % Cet,  5.36 
\node[pin={[pin distance=-0.4\onedegree,Flaamsted]180:{$ 77$}}] at (axis cs:{38.678},{-7.859}) {}; % Cet,  5.73 
\node[pin={[pin distance=-0.4\onedegree,Flaamsted]180:{$  7$}}] at (axis cs:{3.660},{-18.933}) {}; % Cet,  4.46 
\node[pin={[pin distance=-0.4\onedegree,Flaamsted]-90:{$ 80$}}] at (axis cs:{39.000},{-7.831}) {}; % Cet,  5.53 
\node[pin={[pin distance=-0.4\onedegree,Flaamsted]180:{$ 81$}}] at (axis cs:{39.424},{-3.396}) {}; % Cet,  5.65 
\node[pin={[pin distance=-0.4\onedegree,Flaamsted]00:{$ 84$}}] at (axis cs:{40.308},{-0.695}) {}; % Cet,  5.76 
%\node[pin={[pin distance=-0.4\onedegree,Flaamsted]90:{$ 93$}}] at (axis cs:{45.594},{4.353}) {}; % Cet,  5.62 
\node[pin={[pin distance=-0.4\onedegree,Flaamsted]90:{$ 94$}}] at (axis cs:{48.193},{-1.196}) {}; % Cet,  5.07 
\node[pin={[pin distance=-0.4\onedegree,Flaamsted]00:{$ 95$}}] at (axis cs:{49.593},{-0.930}) {}; % Cet,  5.42 

\node[pin={[pin distance=-0.2\onedegree,Bayer]00:{$\alpha$}}] at (axis cs:{101.287},{-16.716}) {}; % CMa, -1.44 
\node[pin={[pin distance=-0.2\onedegree,Bayer]00:{$\beta$}}] at (axis cs:{95.675},{-17.956}) {}; % CMa,  1.96 
\node[pin={[pin distance=-0.2\onedegree,Bayer]-90:{$\delta$}}] at (axis cs:{107.098},{-26.393}) {}; % CMa,  1.84 
\node[pin={[pin distance=-0.2\onedegree,Bayer]00:{$\epsilon$}}] at (axis cs:{104.656},{-28.972}) {}; % CMa,  1.53 
\node[pin={[pin distance=-0.4\onedegree,Bayer]00:{$\gamma$}}] at (axis cs:{105.940},{-15.633}) {}; % CMa,  4.11 
\node[pin={[pin distance=-0.4\onedegree,Bayer]00:{$\iota$}}] at (axis cs:{104.034},{-17.054}) {}; % CMa,  4.39 
\node[pin={[pin distance=-0.4\onedegree,Bayer]00:{$\kappa$}}] at (axis cs:{102.460},{-32.508}) {}; % CMa,  3.53 
\node[pin={[pin distance=-0.4\onedegree,Bayer]00:{$\lambda$}}] at (axis cs:{97.043},{-32.580}) {}; % CMa,  4.47 
\node[pin={[pin distance=-0.4\onedegree,Bayer]00:{$\mu$}}] at (axis cs:{104.028},{-14.043}) {}; % CMa,  4.97 
\node[pin={[pin distance=-0.4\onedegree,Bayer]180:{$\nu^1$}}] at (axis cs:{99.095},{-18.660}) {}; % CMa,  5.70 
\node[pin={[pin distance=-0.4\onedegree,Bayer]180:{$\nu^2$}}] at (axis cs:{99.171},{-19.256}) {}; % CMa,  3.95 
\node[pin={[pin distance=-0.8\onedegree,Bayer]-90:{$\nu^3$}}] at (axis cs:{99.473},{-18.237}) {}; % CMa,  4.43 
\node[pin={[pin distance=-0.4\onedegree,Bayer]180:{$\omega$}}] at (axis cs:{108.703},{-26.773}) {}; % CMa,  4.04 
\node[pin={[pin distance=-0.4\onedegree,Bayer]90:{$\omicron^1$}}] at (axis cs:{103.533},{-24.184}) {}; % CMa,  3.84 
\node[pin={[pin distance=-0.4\onedegree,Bayer]-90:{$\omicron^2$}}] at (axis cs:{105.756},{-23.833}) {}; % CMa,  3.04 
\node[pin={[pin distance=-0.4\onedegree,Bayer]00:{$\pi$}}] at (axis cs:{103.906},{-20.136}) {}; % CMa,  4.68 
\node[pin={[pin distance=-0.4\onedegree,Bayer]00:{$\sigma$}}] at (axis cs:{105.430},{-27.935}) {}; % CMa,  3.47 
\node[pin={[pin distance=-0.4\onedegree,Bayer]90:{$\tau$}}] at (axis cs:{109.677},{-24.954}) {}; % CMa,  4.40 
\node[pin={[pin distance=-0.4\onedegree,Bayer]00:{$\vartheta$}}] at (axis cs:{103.547},{-12.039}) {}; % CMa,  4.07 
\node[pin={[pin distance=-0.4\onedegree,Bayer]180:{$\xi^1$}}] at (axis cs:{97.964},{-23.418}) {}; % CMa,  4.33 
\node[pin={[pin distance=-0.8\onedegree,Bayer]200:{$\xi^2$}}] at (axis cs:{98.764},{-22.965}) {}; % CMa,  4.53 
\node[pin={[pin distance=-0.4\onedegree,Bayer]00:{$\zeta$}}] at (axis cs:{95.078},{-30.063}) {}; % CMa,  3.02 
\node[pin={[pin distance=-0.4\onedegree,Flaamsted]00:{$ 10$}}] at (axis cs:{101.119},{-31.070}) {}; % CMa,  5.23 
\node[pin={[pin distance=-0.4\onedegree,Flaamsted]00:{$ 11$}}] at (axis cs:{101.713},{-14.426}) {}; % CMa,  5.28 
\node[pin={[pin distance=-0.4\onedegree,Flaamsted]180:{$ 15$}}] at (axis cs:{103.387},{-20.224}) {}; % CMa,  4.82 
\node[pin={[pin distance=-0.4\onedegree,Flaamsted]90:{$ 17$}}] at (axis cs:{103.761},{-20.405}) {}; % CMa,  5.80 
%\node[pin={[pin distance=-0.4\onedegree,Flaamsted]00:{$ 26$}}] at (axis cs:{108.051},{-25.942}) {}; % CMa,  5.91 
\node[pin={[pin distance=-0.8\onedegree,Flaamsted]285:{$ 27$}}] at (axis cs:{108.563},{-26.352}) {}; % CMa,  4.45 
%\node[pin={[pin distance=-0.8\onedegree,Flaamsted]-90:{$ 29$}}] at (axis cs:{109.668},{-24.559}) {}; % CMa,  4.92 

\node[pin={[pin distance=-0.4\onedegree,Bayer]00:{$\alpha$}}] at (axis cs:{84.912},{-34.074}) {}; % Col,  2.66 
\node[pin={[pin distance=-0.4\onedegree,Bayer]-90:{$\beta$}}] at (axis cs:{87.740},{-35.768}) {}; % Col,  3.11 
\node[pin={[pin distance=-0.4\onedegree,Bayer]00:{$\delta$}}] at (axis cs:{95.528},{-33.436}) {}; % Col,  3.85 
\node[pin={[pin distance=-0.4\onedegree,Bayer]90:{$\epsilon$}}] at (axis cs:{82.803},{-35.470}) {}; % Col,  3.87 
\node[pin={[pin distance=-0.4\onedegree,Bayer]00:{$\gamma$}}] at (axis cs:{89.384},{-35.283}) {}; % Col,  4.36 
\node[pin={[pin distance=-0.4\onedegree,Bayer]00:{$\kappa$}}] at (axis cs:{94.138},{-35.140}) {}; % Col,  4.37 
\node[pin={[pin distance=-0.4\onedegree,Bayer]00:{$\lambda$}}] at (axis cs:{88.279},{-33.801}) {}; % Col,  4.87 
\node[pin={[pin distance=-0.4\onedegree,Bayer]00:{$\mu$}}] at (axis cs:{86.500},{-32.306}) {}; % Col,  5.16 
\node[pin={[pin distance=-0.4\onedegree,Bayer]00:{$\nu^2$}}] at (axis cs:{84.436},{-28.690}) {}; % Col,  5.28 
\node[pin={[pin distance=-0.4\onedegree,Bayer]00:{$\omicron$}}] at (axis cs:{79.371},{-34.895}) {}; % Col,  4.82 
\node[pin={[pin distance=-0.4\onedegree,Bayer]00:{$\sigma$}}] at (axis cs:{89.087},{-31.382}) {}; % Col,  5.52 
\node[pin={[pin distance=-0.4\onedegree,Bayer]00:{$\vartheta$}}] at (axis cs:{91.882},{-37.253}) {}; % Col,  5.00 
\node[pin={[pin distance=-0.4\onedegree,Bayer]00:{$\xi$}}] at (axis cs:{88.875},{-37.121}) {}; % Col,  4.96 

\node[pin={[pin distance=-0.4\onedegree,Bayer]90:{$\beta$}}] at (axis cs:{76.962},{-5.086}) {}; % Eri,  2.79 
\node[pin={[pin distance=-0.4\onedegree,Bayer]00:{$\delta$}}] at (axis cs:{55.812},{-9.763}) {}; % Eri,  3.53 
\node[pin={[pin distance=-0.4\onedegree,Bayer]00:{$\epsilon$}}] at (axis cs:{53.233},{-9.458}) {}; % Eri,  3.73 
\node[pin={[pin distance=-0.4\onedegree,Bayer]00:{$\eta$}}] at (axis cs:{44.107},{-8.898}) {}; % Eri,  3.90 
\node[pin={[pin distance=-0.4\onedegree,Bayer]00:{$\gamma$}}] at (axis cs:{59.507},{-13.508}) {}; % Eri,  2.96 
%\node[pin={[pin distance=-0.4\onedegree,Bayer]00:{$\iota$}}] at (axis cs:{40.167},{-39.855}) {}; % Eri,  4.12 
\node[pin={[pin distance=-0.4\onedegree,Bayer]180:{$\lambda$}}] at (axis cs:{77.287},{-8.754}) {}; % Eri,  4.26 
\node[pin={[pin distance=-0.4\onedegree,Bayer]90:{$\mu$}}] at (axis cs:{71.376},{-3.255}) {}; % Eri,  4.02 
\node[pin={[pin distance=-0.4\onedegree,Bayer]-90:{$\nu$}}] at (axis cs:{69.080},{-3.352}) {}; % Eri,  3.93 
\node[pin={[pin distance=-0.4\onedegree,Bayer]-90:{$\omega$}}] at (axis cs:{73.224},{-5.453}) {}; % Eri,  4.38 
\node[pin={[pin distance=-0.4\onedegree,Bayer]-90:{$\omicron^1$}}] at (axis cs:{62.966},{-6.838}) {}; % Eri,  4.04 
\node[pin={[pin distance=-0.4\onedegree,Bayer]90:{$\omicron^2$}}] at (axis cs:{63.818},{-7.653}) {}; % Eri,  4.42 
\node[pin={[pin distance=-0.4\onedegree,Bayer]00:{$\pi$}}] at (axis cs:{56.536},{-12.102}) {}; % Eri,  4.43 
\node[pin={[pin distance=-0.4\onedegree,Bayer]180:{$\rho^1$}}] at (axis cs:{45.292},{-7.663}) {}; % Eri,  5.75 
\node[pin={[pin distance=-0.4\onedegree,Bayer]90:{$\rho^2$}}] at (axis cs:{45.676},{-7.685}) {}; % Eri,  5.33 
\node[pin={[pin distance=-0.4\onedegree,Bayer]-90:{$\rho^3$}}] at (axis cs:{46.069},{-7.601}) {}; % Eri,  5.26 
\node[pin={[pin distance=-0.4\onedegree,Bayer]00:{$\psi$}}] at (axis cs:{75.360},{-7.174}) {}; % Eri,  4.79 
\node[pin={[pin distance=-0.4\onedegree,Bayer]180:{$\tau^1$}}] at (axis cs:{41.276},{-18.572}) {}; % Eri,  4.47 
\node[pin={[pin distance=-0.4\onedegree,Bayer]00:{$\tau^2$}}] at (axis cs:{42.760},{-21.004}) {}; % Eri,  4.77 
\node[pin={[pin distance=-0.4\onedegree,Bayer]-90:{$\tau^3$}}] at (axis cs:{45.598},{-23.624}) {}; % Eri,  4.09 
\node[pin={[pin distance=-0.4\onedegree,Bayer]-90:{$\tau^4$}}] at (axis cs:{49.879},{-21.758}) {}; % Eri,  3.74 
\node[pin={[pin distance=-0.4\onedegree,Bayer]00:{$\tau^5$}}] at (axis cs:{53.447},{-21.633}) {}; % Eri,  4.27 
\node[pin={[pin distance=-0.4\onedegree,Bayer]00:{$\tau^6$}}] at (axis cs:{56.712},{-23.249}) {}; % Eri,  4.22 
\node[pin={[pin distance=-0.4\onedegree,Bayer]180:{$\tau^7$}}] at (axis cs:{56.915},{-23.875}) {}; % Eri,  5.24 
\node[pin={[pin distance=-0.4\onedegree,Bayer]-90:{$\tau^8$}}] at (axis cs:{58.428},{-24.612}) {}; % Eri,  4.64 
\node[pin={[pin distance=-0.4\onedegree,Bayer]00:{$\tau^9$}}] at (axis cs:{59.981},{-24.016}) {}; % Eri,  4.63 
\node[pin={[pin distance=-0.4\onedegree,Bayer]180:{$\upsilon^1$}}] at (axis cs:{68.377},{-29.766}) {}; % Eri,  4.51 
\node[pin={[pin distance=-0.4\onedegree,Bayer]90:{$\upsilon^2$}}] at (axis cs:{68.888},{-30.562}) {}; % Eri,  3.81 
\node[pin={[pin distance=-0.4\onedegree,Bayer]180:{$\upsilon^4$}}] at (axis cs:{64.474},{-33.798}) {}; % Eri,  3.56 
\node[pin={[pin distance=-0.4\onedegree,Bayer]00:{$\xi$}}] at (axis cs:{65.920},{-3.745}) {}; % Eri,  5.17 
\node[pin={[pin distance=-0.4\onedegree,Bayer]00:{$\zeta$}}] at (axis cs:{48.958},{-8.820}) {}; % Eri,  4.81 
\node[pin={[pin distance=-0.4\onedegree,Flaamsted]00:{$ 15$}}] at (axis cs:{49.592},{-22.511}) {}; % Eri,  4.87 
\node[pin={[pin distance=-0.4\onedegree,Flaamsted]00:{$ 17$}}] at (axis cs:{52.654},{-5.075}) {}; % Eri,  4.74 
\node[pin={[pin distance=-0.4\onedegree,Flaamsted]00:{$ 20$}}] at (axis cs:{54.073},{-17.467}) {}; % Eri,  5.25 
\node[pin={[pin distance=-0.4\onedegree,Flaamsted]00:{$ 21$}}] at (axis cs:{54.755},{-5.626}) {}; % Eri,  5.97 
\node[pin={[pin distance=-0.4\onedegree,Flaamsted]00:{$ 22$}}] at (axis cs:{55.160},{-5.211}) {}; % Eri,  5.53 
\node[pin={[pin distance=-0.4\onedegree,Flaamsted]90:{$ 24$}}] at (axis cs:{56.127},{-1.163}) {}; % Eri,  5.24 
\node[pin={[pin distance=-0.4\onedegree,Flaamsted]00:{$ 25$}}] at (axis cs:{56.235},{-0.297}) {}; % Eri,  5.56 
\node[pin={[pin distance=-0.4\onedegree,Flaamsted]00:{$ 30$}}] at (axis cs:{58.174},{-5.361}) {}; % Eri,  5.49 
\node[pin={[pin distance=-0.4\onedegree,Flaamsted]00:{$ 32$}}] at (axis cs:{58.573},{-2.955}) {}; % Eri,  4.72 
\node[pin={[pin distance=-0.4\onedegree,Flaamsted]00:{$ 35$}}] at (axis cs:{60.384},{-1.549}) {}; % Eri,  5.28 
\node[pin={[pin distance=-0.4\onedegree,Flaamsted]180:{$ 37$}}] at (axis cs:{62.594},{-6.924}) {}; % Eri,  5.44 
\node[pin={[pin distance=-0.4\onedegree,Flaamsted]00:{$ 39$}}] at (axis cs:{63.599},{-10.256}) {}; % Eri,  4.90 
\node[pin={[pin distance=-0.4\onedegree,Flaamsted]00:{$ 43$}}] at (axis cs:{66.009},{-34.017}) {}; % Eri,  3.96 
\node[pin={[pin distance=-0.4\onedegree,Flaamsted]90:{$ 45$}}] at (axis cs:{67.969},{-0.044}) {}; % Eri,  4.90 
\node[pin={[pin distance=-0.4\onedegree,Flaamsted]00:{$ 46$}}] at (axis cs:{68.478},{-6.739}) {}; % Eri,  5.73 
\node[pin={[pin distance=-0.4\onedegree,Flaamsted]00:{$ 47$}}] at (axis cs:{68.548},{-8.231}) {}; % Eri,  5.20 
\node[pin={[pin distance=-0.4\onedegree,Flaamsted]180:{$  4$}}] at (axis cs:{44.349},{-23.862}) {}; % Eri,  5.44 
\node[pin={[pin distance=-0.4\onedegree,Flaamsted]00:{$ 51$}}] at (axis cs:{69.401},{-2.473}) {}; % Eri,  5.22 
\node[pin={[pin distance=-0.4\onedegree,Flaamsted]00:{$ 53$}}] at (axis cs:{69.545},{-14.304}) {}; % Eri,  3.87 
\node[pin={[pin distance=-0.4\onedegree,Flaamsted]00:{$ 54$}}] at (axis cs:{70.110},{-19.671}) {}; % Eri,  4.34 
\node[pin={[pin distance=-0.4\onedegree,Flaamsted]00:{$ 56$}}] at (axis cs:{71.022},{-8.504}) {}; % Eri,  5.78 
\node[pin={[pin distance=-0.4\onedegree,Flaamsted]00:{$ 58$}}] at (axis cs:{71.901},{-16.934}) {}; % Eri,  5.49 
\node[pin={[pin distance=-0.4\onedegree,Flaamsted]180:{$ 59$}}] at (axis cs:{72.136},{-16.329}) {}; % Eri,  5.76 
\node[pin={[pin distance=-0.4\onedegree,Flaamsted]00:{$  5$}}] at (axis cs:{44.921},{-2.465}) {}; % Eri,  5.55 
\node[pin={[pin distance=-0.4\onedegree,Flaamsted]-90:{$ 60$}}] at (axis cs:{72.548},{-16.217}) {}; % Eri,  5.03 
\node[pin={[pin distance=-0.4\onedegree,Flaamsted]90:{$ 62$}}] at (axis cs:{74.101},{-5.171}) {}; % Eri,  5.50 
\node[pin={[pin distance=-0.4\onedegree,Flaamsted]00:{$ 63$}}] at (axis cs:{74.960},{-10.263}) {}; % Eri,  5.39 
\node[pin={[pin distance=-0.4\onedegree,Flaamsted]180:{$ 64$}}] at (axis cs:{74.982},{-12.537}) {}; % Eri,  4.78 
\node[pin={[pin distance=-0.4\onedegree,Flaamsted]180:{$ 66$}}] at (axis cs:{76.690},{-4.655}) {}; % Eri,  5.10 
\node[pin={[pin distance=-0.4\onedegree,Flaamsted]-90:{$ 68$}}] at (axis cs:{77.182},{-4.456}) {}; % Eri,  5.12 
\node[pin={[pin distance=-0.4\onedegree,Flaamsted]-90:{$  6$}}] at (axis cs:{44.524},{-23.606}) {}; % Eri,  5.81

\node[pin={[pin distance=-0.4\onedegree,Bayer]00:{$\alpha$}}] at (axis cs:{48.019},{-28.988}) {}; % For,  3.80 
\node[pin={[pin distance=-0.4\onedegree,Bayer]00:{$\beta$}}] at (axis cs:{42.273},{-32.406}) {}; % For,  4.46 
\node[pin={[pin distance=-0.4\onedegree,Bayer]00:{$\chi^2$}}] at (axis cs:{51.889},{-35.681}) {}; % For,  5.70 
\node[pin={[pin distance=-0.4\onedegree,Bayer]00:{$\delta$}}] at (axis cs:{55.562},{-31.938}) {}; % For,  4.98 
\node[pin={[pin distance=-0.4\onedegree,Bayer]00:{$\epsilon$}}] at (axis cs:{45.407},{-28.091}) {}; % For,  5.88 
\node[pin={[pin distance=-0.4\onedegree,Bayer]180:{$\eta^2$}}] at (axis cs:{42.562},{-35.843}) {}; % For,  5.93 
\node[pin={[pin distance=-0.4\onedegree,Bayer]0:{$\eta^3$}}] at (axis cs:{42.668},{-35.676}) {}; % For,  5.48 
\node[pin={[pin distance=-0.4\onedegree,Bayer]00:{$\gamma^2$}}] at (axis cs:{42.476},{-27.942}) {}; % For,  5.39 
\node[pin={[pin distance=-0.4\onedegree,Bayer]90:{$\iota^1$}}] at (axis cs:{39.039},{-30.045}) {}; % For,  5.73 
\node[pin={[pin distance=-0.4\onedegree,Bayer]180:{$\iota^2$}}] at (axis cs:{39.578},{-30.194}) {}; % For,  5.84 
\node[pin={[pin distance=-0.4\onedegree,Bayer]-90:{$\kappa$}}] at (axis cs:{35.636},{-23.816}) {}; % For,  5.20 
\node[pin={[pin distance=-0.4\onedegree,Bayer]90:{$\lambda^1$}}] at (axis cs:{38.279},{-34.650}) {}; % For,  5.90 
\node[pin={[pin distance=-0.4\onedegree,Bayer]90:{$\lambda^2$}}] at (axis cs:{39.244},{-34.578}) {}; % For,  5.78 
\node[pin={[pin distance=-0.4\onedegree,Bayer]00:{$\mu$}}] at (axis cs:{33.227},{-30.724}) {}; % For,  5.27 
\node[pin={[pin distance=-0.4\onedegree,Bayer]00:{$\nu$}}] at (axis cs:{31.123},{-29.297}) {}; % For,  4.70 
\node[pin={[pin distance=-0.4\onedegree,Bayer]00:{$\omega$}}] at (axis cs:{38.461},{-28.232}) {}; % For,  4.97 
\node[pin={[pin distance=-0.4\onedegree,Bayer]00:{$\pi$}}] at (axis cs:{30.311},{-30.002}) {}; % For,  5.36 
\node[pin={[pin distance=-0.4\onedegree,Bayer]00:{$\rho$}}] at (axis cs:{56.984},{-30.168}) {}; % For,  5.53 
\node[pin={[pin distance=-0.4\onedegree,Bayer]00:{$\psi$}}] at (axis cs:{43.393},{-38.437}) {}; % For,  5.93 
\node[pin={[pin distance=-0.4\onedegree,Bayer]00:{$\sigma$}}] at (axis cs:{56.614},{-29.338}) {}; % For,  5.92 
\node[pin={[pin distance=-0.4\onedegree,Bayer]00:{$\varphi$}}] at (axis cs:{37.007},{-33.811}) {}; % For,  5.13 
\node[pin={[pin distance=-0.4\onedegree,Bayer]00:{$\zeta$}}] at (axis cs:{44.901},{-25.274}) {}; % For,  5.70 
\node[pin={[pin distance=-0.2\onedegree,Bayer]00:{$\gamma$}}] at (axis cs:{99.428},{16.399}) {}; % Gem,  2.02 
\node[pin={[pin distance=-0.4\onedegree,Bayer]0:{$\epsilon$}}] at (axis cs:{100.983},{25.131}) {}; % Gem,  3.01 
\node[pin={[pin distance=-0.4\onedegree,Bayer]00:{$\eta$}}] at (axis cs:{93.719},{22.507}) {}; % Gem,  3.30 
\node[pin={[pin distance=-0.4\onedegree,Bayer]00:{$\lambda$}}] at (axis cs:{109.523},{16.540}) {}; % Gem,  3.58 
\node[pin={[pin distance=-0.4\onedegree,Bayer]00:{$\mu$}}] at (axis cs:{95.740},{22.513}) {}; % Gem,  2.90 
\node[pin={[pin distance=-0.4\onedegree,Bayer]00:{$\nu$}}] at (axis cs:{97.241},{20.212}) {}; % Gem,  4.14 
\node[pin={[pin distance=-0.4\onedegree,Bayer]-90:{$\omega$}}] at (axis cs:{105.603},{24.215}) {}; % Gem,  5.18 
\node[pin={[pin distance=-0.4\onedegree,Bayer]00:{$\tau$}}] at (axis cs:{107.785},{30.245}) {}; % Gem,  4.39 
\node[pin={[pin distance=-0.4\onedegree,Bayer]00:{$\vartheta$}}] at (axis cs:{103.197},{33.961}) {}; % Gem,  3.61 
\node[pin={[pin distance=-0.4\onedegree,Bayer]00:{$\xi$}}] at (axis cs:{101.322},{12.895}) {}; % Gem,  3.34 
\node[pin={[pin distance=-0.4\onedegree,Bayer]00:{$\zeta$}}] at (axis cs:{106.027},{20.570}) {}; % Gem,  4.01 
\node[pin={[pin distance=-0.4\onedegree,Flaamsted]-90:{$  1$}}] at (axis cs:{91.030},{23.263}) {}; % Gem,  4.18 
\node[pin={[pin distance=-0.4\onedegree,Flaamsted]00:{$ 26$}}] at (axis cs:{100.601},{17.645}) {}; % Gem,  5.29 
\node[pin={[pin distance=-0.4\onedegree,Flaamsted]00:{$ 28$}}] at (axis cs:{101.189},{28.971}) {}; % Gem,  5.42 
\node[pin={[pin distance=-0.4\onedegree,Flaamsted]-90:{$ 30$}}] at (axis cs:{100.997},{13.228}) {}; % Gem,  4.49 
\node[pin={[pin distance=-0.4\onedegree,Flaamsted]00:{$ 33$}}] at (axis cs:{102.458},{16.203}) {}; % Gem,  5.87 
\node[pin={[pin distance=-0.4\onedegree,Flaamsted]-90:{$ 35$}}] at (axis cs:{102.606},{13.413}) {}; % Gem,  5.68 
\node[pin={[pin distance=-0.4\onedegree,Flaamsted]00:{$ 36$}}] at (axis cs:{102.888},{21.761}) {}; % Gem,  5.27 
\node[pin={[pin distance=-0.4\onedegree,Flaamsted]90:{$ 37$}}] at (axis cs:{103.828},{25.376}) {}; % Gem,  5.75 
\node[pin={[pin distance=-0.4\onedegree,Flaamsted]00:{$ 38$}}] at (axis cs:{103.661},{13.178}) {}; % Gem,  4.70 
\node[pin={[pin distance=-0.4\onedegree,Flaamsted]00:{$  3$}}] at (axis cs:{92.433},{23.113}) {}; % Gem,  5.76 
\node[pin={[pin distance=-0.4\onedegree,Flaamsted]00:{$ 41$}}] at (axis cs:{105.066},{16.079}) {}; % Gem,  5.72 
\node[pin={[pin distance=-0.4\onedegree,Flaamsted]00:{$ 45$}}] at (axis cs:{107.092},{15.930}) {}; % Gem,  5.49 
\node[pin={[pin distance=-0.4\onedegree,Flaamsted]00:{$ 47$}}] at (axis cs:{107.846},{26.856}) {}; % Gem,  5.76 
\node[pin={[pin distance=-0.4\onedegree,Flaamsted]00:{$ 48$}}] at (axis cs:{108.110},{24.128}) {}; % Gem,  5.85 
\node[pin={[pin distance=-0.4\onedegree,Flaamsted]00:{$ 51$}}] at (axis cs:{108.343},{16.159}) {}; % Gem,  5.06 
\node[pin={[pin distance=-0.4\onedegree,Flaamsted]00:{$ 52$}}] at (axis cs:{108.675},{24.885}) {}; % Gem,  5.84 
\node[pin={[pin distance=-0.4\onedegree,Flaamsted]00:{$ 53$}}] at (axis cs:{108.988},{27.897}) {}; % Gem,  5.75 
\node[pin={[pin distance=-0.4\onedegree,Flaamsted]180:{$  5$}}] at (axis cs:{92.885},{24.420}) {}; % Gem,  5.83 
\node[pin={[pin distance=-0.4\onedegree,Bayer]00:{$\alpha$}}] at (axis cs:{83.183},{-17.822}) {}; % Lep,  2.59 
\node[pin={[pin distance=-0.4\onedegree,Bayer]-90:{$\beta$}}] at (axis cs:{82.061},{-20.759}) {}; % Lep,  2.84 
\node[pin={[pin distance=-0.4\onedegree,Bayer]00:{$\delta$}}] at (axis cs:{87.830},{-20.879}) {}; % Lep,  3.78 
\node[pin={[pin distance=-0.4\onedegree,Bayer]00:{$\epsilon$}}] at (axis cs:{76.365},{-22.371}) {}; % Lep,  3.18 
\node[pin={[pin distance=-0.4\onedegree,Bayer]00:{$\eta$}}] at (axis cs:{89.101},{-14.168}) {}; % Lep,  3.72 
\node[pin={[pin distance=-0.4\onedegree,Bayer]0:{$\gamma$}}] at (axis cs:{86.116},{-22.448}) {}; % Lep,  3.60 
\node[pin={[pin distance=-0.4\onedegree,Bayer]00:{$\iota$}}] at (axis cs:{78.075},{-11.869}) {}; % Lep,  4.47 
\node[pin={[pin distance=-0.4\onedegree,Bayer]00:{$\kappa$}}] at (axis cs:{78.308},{-12.941}) {}; % Lep,  4.43 
\node[pin={[pin distance=-0.4\onedegree,Bayer]0:{$\lambda$}}] at (axis cs:{79.894},{-13.177}) {}; % Lep,  4.28 
\node[pin={[pin distance=-0.4\onedegree,Bayer]00:{$\mu$}}] at (axis cs:{78.233},{-16.205}) {}; % Lep,  3.29 
\node[pin={[pin distance=-0.4\onedegree,Bayer]00:{$\nu$}}] at (axis cs:{79.996},{-12.315}) {}; % Lep,  5.29 
\node[pin={[pin distance=-0.4\onedegree,Bayer]00:{$\vartheta$}}] at (axis cs:{91.539},{-14.935}) {}; % Lep,  4.67 
\node[pin={[pin distance=-0.4\onedegree,Bayer]00:{$\zeta$}}] at (axis cs:{86.739},{-14.822}) {}; % Lep,  3.55 
\node[pin={[pin distance=-0.4\onedegree,Flaamsted]0:{$ 10$}}] at (axis cs:{82.782},{-20.863}) {}; % Lep,  5.54 
\node[pin={[pin distance=-0.4\onedegree,Flaamsted]180:{$ 12$}}] at (axis cs:{85.558},{-22.374}) {}; % Lep,  5.88 
\node[pin={[pin distance=-0.4\onedegree,Flaamsted]00:{$ 17$}}] at (axis cs:{91.246},{-16.484}) {}; % Lep,  4.97 
\node[pin={[pin distance=-0.4\onedegree,Flaamsted]00:{$ 19$}}] at (axis cs:{91.923},{-19.166}) {}; % Lep,  5.28 
\node[pin={[pin distance=-0.4\onedegree,Flaamsted]180:{$  1$}}] at (axis cs:{75.687},{-22.795}) {}; % Lep,  5.74 
\node[pin={[pin distance=-0.4\onedegree,Flaamsted]00:{$  8$}}] at (axis cs:{80.876},{-13.927}) {}; % Lep,  5.24 
\node[pin={[pin distance=-0.4\onedegree,Bayer]00:{$\beta$}}] at (axis cs:{97.204},{-7.033}) {}; % Mon,  4.64 
\node[pin={[pin distance=-0.4\onedegree,Bayer]00:{$\delta$}}] at (axis cs:{107.966},{-0.493}) {}; % Mon,  4.15 
\node[pin={[pin distance=-0.4\onedegree,Bayer]-90:{$\epsilon$}}] at (axis cs:{95.942},{4.593}) {}; % Mon,  4.41 
\node[pin={[pin distance=-0.4\onedegree,Bayer]-90:{$\gamma$}}] at (axis cs:{93.714},{-6.275}) {}; % Mon,  3.98 
\node[pin={[pin distance=-0.4\onedegree,Flaamsted]-90:{$ 10$}}] at (axis cs:{96.990},{-4.762}) {}; % Mon,  5.05 
\node[pin={[pin distance=-0.4\onedegree,Flaamsted]00:{$ 12$}}] at (axis cs:{98.080},{4.856}) {}; % Mon,  5.85 
\node[pin={[pin distance=-0.4\onedegree,Flaamsted]00:{$ 13$}}] at (axis cs:{98.226},{7.333}) {}; % Mon,  4.51 
\node[pin={[pin distance=-0.4\onedegree,Flaamsted]00:{$ 15$}}] at (axis cs:{100.244},{9.896}) {}; % Mon,  4.67 
\node[pin={[pin distance=-0.4\onedegree,Flaamsted]00:{$ 16$}}] at (axis cs:{101.635},{8.587}) {}; % Mon,  5.91 
\node[pin={[pin distance=-0.4\onedegree,Flaamsted]00:{$ 17$}}] at (axis cs:{101.833},{8.037}) {}; % Mon,  4.77 
\node[pin={[pin distance=-0.4\onedegree,Flaamsted]00:{$ 18$}}] at (axis cs:{101.965},{2.412}) {}; % Mon,  4.47 
\node[pin={[pin distance=-0.4\onedegree,Flaamsted]90:{$ 19$}}] at (axis cs:{105.728},{-4.239}) {}; % Mon,  4.98 
\node[pin={[pin distance=-0.4\onedegree,Flaamsted]00:{$ 20$}}] at (axis cs:{107.557},{-4.237}) {}; % Mon,  4.92 
\node[pin={[pin distance=-0.4\onedegree,Flaamsted]90:{$ 21$}}] at (axis cs:{107.848},{-0.302}) {}; % Mon,  5.43 
\node[pin={[pin distance=-0.4\onedegree,Flaamsted]00:{$  2$}}] at (axis cs:{89.768},{-9.558}) {}; % Mon,  5.04 
\node[pin={[pin distance=-0.4\onedegree,Flaamsted]00:{$  3$}}] at (axis cs:{90.460},{-10.598}) {}; % Mon,  5.00 
\node[pin={[pin distance=-0.4\onedegree,Flaamsted]00:{$  7$}}] at (axis cs:{94.928},{-7.823}) {}; % Mon,  5.26 

\node[pin={[pin distance=-0.2\onedegree,Bayer]00:{$\alpha$}}] at (axis cs:{88.793},{7.407}) {}; % Ori,  0.57 
\node[pin={[pin distance=-0.2\onedegree,Bayer]00:{$\beta$}}] at (axis cs:{78.634},{-8.202}) {}; % Ori,  0.28 
\node[pin={[pin distance=-0.2\onedegree,Bayer]-90:{$\delta$}}] at (axis cs:{83.002},{-0.299}) {}; % Ori,  2.23 
\node[pin={[pin distance=-0.4\onedegree,Bayer]90:{$\epsilon$}}] at (axis cs:{84.053},{-1.202}) {}; % Ori,  1.72 
\node[pin={[pin distance=-0.2\onedegree,Bayer]00:{$\gamma$}}] at (axis cs:{81.283},{6.349}) {}; % Ori,  1.66 
\node[pin={[pin distance=-0.2\onedegree,Bayer]00:{$\kappa$}}] at (axis cs:{86.939},{-9.670}) {}; % Ori,  2.06 
\node[pin={[pin distance=-0.2\onedegree,Bayer]00:{$\zeta$}}] at (axis cs:{85.190},{-1.942}) {}; % Ori,  1.74 
\node[pin={[pin distance=-0.4\onedegree,Bayer]-90:{$\chi^1$}}] at (axis cs:{88.596},{20.276}) {}; % Ori,  4.40 
\node[pin={[pin distance=-0.4\onedegree,Bayer]180:{$\chi^2$}}] at (axis cs:{90.980},{20.138}) {}; % Ori,  4.64 
\node[pin={[pin distance=-0.4\onedegree,Bayer]00:{$\eta$}}] at (axis cs:{81.119},{-2.397}) {}; % Ori,  3.39 
\node[pin={[pin distance=-0.2\onedegree,Bayer]90:{$\iota$}}] at (axis cs:{83.858},{-5.910}) {}; % Ori,  2.78 
\node[pin={[pin distance=-0.4\onedegree,Bayer]00:{$\lambda$}}] at (axis cs:{83.784},{9.934}) {}; % Ori,  3.39 
%\node[pin={[pin distance=-0.4\onedegree,Bayer]00:{$\lambda$}}] at (axis cs:{83.785},{9.935}) {}; % Ori,  5.61 
\node[pin={[pin distance=-0.4\onedegree,Bayer]00:{$\mu$}}] at (axis cs:{90.596},{9.647}) {}; % Ori,  4.13 
\node[pin={[pin distance=-0.4\onedegree,Bayer]-90:{$\nu$}}] at (axis cs:{91.893},{14.768}) {}; % Ori,  4.42 
\node[pin={[pin distance=-0.4\onedegree,Bayer]00:{$\omega$}}] at (axis cs:{84.796},{4.121}) {}; % Ori,  4.54 
\node[pin={[pin distance=-0.4\onedegree,Bayer]-90:{$\omicron^1$}}] at (axis cs:{73.133},{14.251}) {}; % Ori,  4.74 
\node[pin={[pin distance=-0.4\onedegree,Bayer]00:{$\omicron^2$}}] at (axis cs:{74.093},{13.514}) {}; % Ori,  4.08 
\node[pin={[pin distance=-0.4\onedegree,Bayer]0:{$\pi^1$}}] at (axis cs:{73.724},{10.151}) {}; % Ori,  4.66 
\node[pin={[pin distance=-0.4\onedegree,Bayer]00:{$\pi^2$}}] at (axis cs:{72.653},{8.900}) {}; % Ori,  4.36 
\node[pin={[pin distance=-0.4\onedegree,Bayer]00:{$\pi^3$}}] at (axis cs:{72.460},{6.961}) {}; % Ori,  3.19 
\node[pin={[pin distance=-0.4\onedegree,Bayer]00:{$\pi^4$}}] at (axis cs:{72.802},{5.605}) {}; % Ori,  3.68 
\node[pin={[pin distance=-0.4\onedegree,Bayer]00:{$\pi^5$}}] at (axis cs:{73.563},{2.441}) {}; % Ori,  3.71 
\node[pin={[pin distance=-0.4\onedegree,Bayer]00:{$\pi^6$}}] at (axis cs:{74.637},{1.714}) {}; % Ori,  4.46 
\node[pin={[pin distance=-0.4\onedegree,Bayer]00:{$\rho$}}] at (axis cs:{78.323},{2.861}) {}; % Ori,  4.48 
\node[pin={[pin distance=-0.4\onedegree,Bayer]90:{$\psi^1$}}] at (axis cs:{81.187},{1.846}) {}; % Ori,  4.89 
\node[pin={[pin distance=-0.4\onedegree,Bayer]-90:{$\psi^2$}}] at (axis cs:{81.709},{3.095}) {}; % Ori,  4.61 
\node[pin={[pin distance=-0.4\onedegree,Bayer]90:{$\sigma$}}] at (axis cs:{84.687},{-2.600}) {}; % Ori,  3.80 
\node[pin={[pin distance=-0.4\onedegree,Bayer]00:{$\tau$}}] at (axis cs:{79.402},{-6.844}) {}; % Ori,  3.60 
\node[pin={[pin distance=-0.4\onedegree,Bayer]-90:{$\upsilon$}}] at (axis cs:{82.983},{-7.302}) {}; % Ori,  4.61 
\node[pin={[pin distance=-0.4\onedegree,Bayer]90:{$\varphi^1$}}] at (axis cs:{83.705},{9.489}) {}; % Ori,  4.41 
\node[pin={[pin distance=-0.4\onedegree,Bayer]0:{$\varphi^2$}}] at (axis cs:{84.227},{9.291}) {}; % Ori,  4.09 
\node[pin={[pin distance=-0.4\onedegree,Bayer]00:{$\vartheta$}}] at (axis cs:{83.819},{-5.390}) {}; % Ori,  5.13 
\node[pin={[pin distance=-0.4\onedegree,Bayer]-90:{$\xi$}}] at (axis cs:{92.985},{14.209}) {}; % Ori,  4.46 
\node[pin={[pin distance=-0.4\onedegree,Flaamsted]00:{$ 11$}}] at (axis cs:{76.142},{15.404}) {}; % Ori,  4.67 
\node[pin={[pin distance=-0.4\onedegree,Flaamsted]00:{$ 14$}}] at (axis cs:{76.970},{8.498}) {}; % Ori,  5.35 
\node[pin={[pin distance=-0.4\onedegree,Flaamsted]00:{$ 15$}}] at (axis cs:{77.425},{15.597}) {}; % Ori,  4.82 
\node[pin={[pin distance=-0.4\onedegree,Flaamsted]00:{$ 16$}}] at (axis cs:{77.332},{9.829}) {}; % Ori,  5.43 
\node[pin={[pin distance=-0.4\onedegree,Flaamsted]00:{$ 18$}}] at (axis cs:{79.017},{11.341}) {}; % Ori,  5.53 
\node[pin={[pin distance=-0.4\onedegree,Flaamsted]90:{$ 21$}}] at (axis cs:{79.797},{2.596}) {}; % Ori,  5.34 
\node[pin={[pin distance=-0.4\onedegree,Flaamsted]90:{$ 22$}}] at (axis cs:{80.441},{-0.382}) {}; % Ori,  4.72 
\node[pin={[pin distance=-0.4\onedegree,Flaamsted]90:{$ 23$}}] at (axis cs:{80.708},{3.544}) {}; % Ori,  4.98 
\node[pin={[pin distance=-0.4\onedegree,Flaamsted]90:{$ 27$}}] at (axis cs:{81.120},{-0.891}) {}; % Ori,  5.07 
\node[pin={[pin distance=-0.4\onedegree,Flaamsted]00:{$ 29$}}] at (axis cs:{80.987},{-7.808}) {}; % Ori,  4.14 
\node[pin={[pin distance=-0.4\onedegree,Flaamsted]90:{$ 31$}}] at (axis cs:{82.433},{-1.092}) {}; % Ori,  4.70 
\node[pin={[pin distance=-0.4\onedegree,Flaamsted]00:{$ 32$}}] at (axis cs:{82.696},{5.948}) {}; % Ori,  4.23 
\node[pin={[pin distance=-0.4\onedegree,Flaamsted]00:{$ 33$}}] at (axis cs:{82.811},{3.292}) {}; % Ori,  5.46 
\node[pin={[pin distance=-0.4\onedegree,Flaamsted]00:{$ 35$}}] at (axis cs:{83.476},{14.305}) {}; % Ori,  5.60 
\node[pin={[pin distance=-0.4\onedegree,Flaamsted]00:{$ 38$}}] at (axis cs:{83.570},{3.767}) {}; % Ori,  5.33 
\node[pin={[pin distance=-0.4\onedegree,Flaamsted]180:{$ 42$}}] at (axis cs:{83.847},{-4.838}) {}; % Ori,  4.60 
\node[pin={[pin distance=-1.0\onedegree,Flaamsted]-45:{$ 45$}}] at (axis cs:{83.915},{-4.856}) {}; % Ori,  5.24 
\node[pin={[pin distance=-0.4\onedegree,Flaamsted]00:{$ 49$}}] at (axis cs:{84.721},{-7.213}) {}; % Ori,  4.79 
\node[pin={[pin distance=-0.4\onedegree,Flaamsted]00:{$ 51$}}] at (axis cs:{85.619},{1.474}) {}; % Ori,  4.90 
\node[pin={[pin distance=-0.4\onedegree,Flaamsted]00:{$ 52$}}] at (axis cs:{87.001},{6.454}) {}; % Ori,  5.30 
\node[pin={[pin distance=-0.4\onedegree,Flaamsted]00:{$ 55$}}] at (axis cs:{87.842},{-7.518}) {}; % Ori,  5.35 
\node[pin={[pin distance=-0.4\onedegree,Flaamsted]-90:{$ 56$}}] at (axis cs:{88.110},{1.855}) {}; % Ori,  4.75 
\node[pin={[pin distance=-0.4\onedegree,Flaamsted]90:{$ 57$}}] at (axis cs:{88.736},{19.749}) {}; % Ori,  5.91 
\node[pin={[pin distance=-0.4\onedegree,Flaamsted]00:{$ 59$}}] at (axis cs:{89.602},{1.837}) {}; % Ori,  5.90 
\node[pin={[pin distance=-0.4\onedegree,Flaamsted]90:{$  5$}}] at (axis cs:{73.345},{2.508}) {}; % Ori,  5.33 
\node[pin={[pin distance=-0.4\onedegree,Flaamsted]00:{$ 60$}}] at (axis cs:{89.707},{0.553}) {}; % Ori,  5.22 
\node[pin={[pin distance=-0.4\onedegree,Flaamsted]00:{$ 63$}}] at (axis cs:{91.242},{5.420}) {}; % Ori,  5.66 
\node[pin={[pin distance=-0.4\onedegree,Flaamsted]90:{$ 64$}}] at (axis cs:{90.864},{19.690}) {}; % Ori,  5.15 
\node[pin={[pin distance=-0.4\onedegree,Flaamsted]00:{$ 66$}}] at (axis cs:{91.243},{4.159}) {}; % Ori,  5.63 
\node[pin={[pin distance=-0.4\onedegree,Flaamsted]00:{$ 68$}}] at (axis cs:{93.006},{19.790}) {}; % Ori,  5.76 
\node[pin={[pin distance=-0.4\onedegree,Flaamsted]180:{$ 69$}}] at (axis cs:{93.014},{16.130}) {}; % Ori,  4.96 
\node[pin={[pin distance=-0.4\onedegree,Flaamsted]00:{$  6$}}] at (axis cs:{73.695},{11.426}) {}; % Ori,  5.19 
\node[pin={[pin distance=-0.4\onedegree,Flaamsted]00:{$ 71$}}] at (axis cs:{93.712},{19.156}) {}; % Ori,  5.21 
\node[pin={[pin distance=-0.4\onedegree,Flaamsted]-90:{$ 72$}}] at (axis cs:{93.855},{16.143}) {}; % Ori,  5.34 
\node[pin={[pin distance=-0.4\onedegree,Flaamsted]0:{$ 73$}}] at (axis cs:{93.937},{12.551}) {}; % Ori,  5.48 
\node[pin={[pin distance=-0.4\onedegree,Flaamsted]90:{$ 74$}}] at (axis cs:{94.111},{12.272}) {}; % Ori,  5.04 
\node[pin={[pin distance=-0.4\onedegree,Flaamsted]00:{$ 75$}}] at (axis cs:{94.278},{9.942}) {}; % Ori,  5.40 

\node[pin={[pin distance=-0.4\onedegree,Bayer]180:{$\chi$}}] at (axis cs:{3.651},{20.207}) {}; % Peg,  4.80 
\node[pin={[pin distance=-0.4\onedegree,Bayer]180:{$\gamma$}}] at (axis cs:{3.309},{15.184}) {}; % Peg,  2.83 
\node[pin={[pin distance=-0.4\onedegree,Flaamsted]00:{$ 85$}}] at (axis cs:{0.542},{27.082}) {}; % Peg,  5.80 
\node[pin={[pin distance=-0.4\onedegree,Flaamsted]00:{$ 86$}}] at (axis cs:{1.425},{13.396}) {}; % Peg,  5.55 
\node[pin={[pin distance=-0.4\onedegree,Flaamsted]00:{$ 87$}}] at (axis cs:{2.260},{18.212}) {}; % Peg,  5.56 
\node[pin={[pin distance=-0.4\onedegree,Bayer]00:{$\omega$}}] at (axis cs:{47.822},{39.611}) {}; % Per,  4.61 
\node[pin={[pin distance=-0.4\onedegree,Bayer]00:{$\omicron$}}] at (axis cs:{56.080},{32.288}) {}; % Per,  3.86 
\node[pin={[pin distance=-0.4\onedegree,Bayer]00:{$\pi$}}] at (axis cs:{44.690},{39.663}) {}; % Per,  4.69 
\node[pin={[pin distance=-0.4\onedegree,Bayer]00:{$\rho$}}] at (axis cs:{46.294},{38.840}) {}; % Per,  3.41 
\node[pin={[pin distance=-0.4\onedegree,Bayer]00:{$\xi$}}] at (axis cs:{59.741},{35.791}) {}; % Per,  4.05 
\node[pin={[pin distance=-0.4\onedegree,Bayer]-90:{$\zeta$}}] at (axis cs:{58.533},{31.884}) {}; % Per,  2.88 
\node[pin={[pin distance=-0.4\onedegree,Flaamsted]180:{$ 16$}}] at (axis cs:{42.646},{38.319}) {}; % Per,  4.22 
\node[pin={[pin distance=-0.4\onedegree,Flaamsted]00:{$ 17$}}] at (axis cs:{42.878},{35.060}) {}; % Per,  4.55 
\node[pin={[pin distance=-0.4\onedegree,Flaamsted]-90:{$ 20$}}] at (axis cs:{43.428},{38.337}) {}; % Per,  5.35 
\node[pin={[pin distance=-0.4\onedegree,Flaamsted]00:{$ 21$}}] at (axis cs:{44.322},{31.934}) {}; % Per,  5.10 
\node[pin={[pin distance=-0.4\onedegree,Flaamsted]00:{$ 24$}}] at (axis cs:{44.765},{35.183}) {}; % Per,  4.94 
\node[pin={[pin distance=-0.4\onedegree,Flaamsted]00:{$ 40$}}] at (axis cs:{55.594},{33.965}) {}; % Per,  4.98 
\node[pin={[pin distance=-0.4\onedegree,Flaamsted]180:{$ 42$}}] at (axis cs:{57.386},{33.091}) {}; % Per,  5.15 
\node[pin={[pin distance=-0.4\onedegree,Flaamsted]00:{$ 50$}}] at (axis cs:{62.153},{38.040}) {}; % Per,  5.52 
\node[pin={[pin distance=-0.4\onedegree,Flaamsted]00:{$ 54$}}] at (axis cs:{65.103},{34.567}) {}; % Per,  4.93 
\node[pin={[pin distance=-0.4\onedegree,Flaamsted]90:{$ 55$}}] at (axis cs:{66.121},{34.131}) {}; % Per,  5.73 
\node[pin={[pin distance=-0.4\onedegree,Flaamsted]00:{$ 56$}}] at (axis cs:{66.156},{33.960}) {}; % Per,  5.79 

\node[pin={[pin distance=-0.4\onedegree,Bayer]00:{$\alpha$}}] at (axis cs:{30.511},{2.764}) {}; % Psc,  3.82 
\node[pin={[pin distance=-0.4\onedegree,Bayer]00:{$\alpha$}}] at (axis cs:{30.512},{2.764}) {}; % Psc,  3.82 
\node[pin={[pin distance=-0.4\onedegree,Bayer]0:{$\chi$}}] at (axis cs:{17.863},{21.035}) {}; % Psc,  4.66 
\node[pin={[pin distance=-0.4\onedegree,Bayer]-90:{$\delta$}}] at (axis cs:{12.171},{7.585}) {}; % Psc,  4.42 
\node[pin={[pin distance=-0.4\onedegree,Bayer]00:{$\epsilon$}}] at (axis cs:{15.736},{7.890}) {}; % Psc,  4.28 
\node[pin={[pin distance=-0.4\onedegree,Bayer]00:{$\eta$}}] at (axis cs:{22.871},{15.346}) {}; % Psc,  3.63 
\node[pin={[pin distance=-0.4\onedegree,Bayer]00:{$\mu$}}] at (axis cs:{22.546},{6.144}) {}; % Psc,  4.84 
\node[pin={[pin distance=-0.4\onedegree,Bayer]00:{$\nu$}}] at (axis cs:{25.358},{5.488}) {}; % Psc,  4.44 
\node[pin={[pin distance=-0.4\onedegree,Bayer]00:{$\omicron$}}] at (axis cs:{26.348},{9.158}) {}; % Psc,  4.27 
\node[pin={[pin distance=-0.4\onedegree,Bayer]00:{$\pi$}}] at (axis cs:{24.275},{12.141}) {}; % Psc,  5.55 
\node[pin={[pin distance=-0.4\onedegree,Bayer]180:{$\rho$}}] at (axis cs:{21.564},{19.172}) {}; % Psc,  5.35 
\node[pin={[pin distance=-0.4\onedegree,Bayer]-90:{$\psi^1$}}] at (axis cs:{16.424},{21.465}) {}; % Psc,  5.54 
\node[pin={[pin distance=-0.4\onedegree,Bayer]180:{$\psi^2$}}] at (axis cs:{16.988},{20.739}) {}; % Psc,  5.57 
\node[pin={[pin distance=-0.4\onedegree,Bayer]90:{$\psi^3$}}] at (axis cs:{17.455},{19.658}) {}; % Psc,  5.56 
\node[pin={[pin distance=-0.4\onedegree,Bayer]00:{$\sigma$}}] at (axis cs:{15.705},{31.804}) {}; % Psc,  5.50 
\node[pin={[pin distance=-0.4\onedegree,Bayer]00:{$\tau$}}] at (axis cs:{17.915},{30.090}) {}; % Psc,  4.52 
\node[pin={[pin distance=-0.4\onedegree,Bayer]00:{$\upsilon$}}] at (axis cs:{19.867},{27.264}) {}; % Psc,  4.75 
\node[pin={[pin distance=-0.4\onedegree,Bayer]00:{$\varphi$}}] at (axis cs:{18.437},{24.584}) {}; % Psc,  4.67 
\node[pin={[pin distance=-0.4\onedegree,Bayer]0:{$\xi$}}] at (axis cs:{28.389},{3.187}) {}; % Psc,  4.62 
\node[pin={[pin distance=-0.4\onedegree,Bayer]180:{$\zeta$}}] at (axis cs:{18.433},{7.575}) {}; % Psc,  5.20 
\node[pin={[pin distance=-0.4\onedegree,Flaamsted]00:{$105$}}] at (axis cs:{24.920},{16.406}) {}; % Psc,  5.99 
\node[pin={[pin distance=-0.4\onedegree,Flaamsted]180:{$107$}}] at (axis cs:{25.624},{20.268}) {}; % Psc,  5.24 
\node[pin={[pin distance=-0.4\onedegree,Flaamsted]-90:{$112$}}] at (axis cs:{30.038},{3.097}) {}; % Psc,  5.88 
\node[pin={[pin distance=-0.4\onedegree,Flaamsted]90:{$ 29$}}] at (axis cs:{0.456},{-3.027}) {}; % Psc,  5.13 
\node[pin={[pin distance=-0.4\onedegree,Flaamsted]-90:{$ 30$}}] at (axis cs:{0.490},{-6.014}) {}; % Psc,  4.40 
\node[pin={[pin distance=-0.4\onedegree,Flaamsted]90:{$ 32$}}] at (axis cs:{0.624},{8.485}) {}; % Psc,  5.70 
\node[pin={[pin distance=-0.4\onedegree,Flaamsted]-90:{$ 33$}}] at (axis cs:{1.334},{-5.707}) {}; % Psc,  4.62 
\node[pin={[pin distance=-0.4\onedegree,Flaamsted]180:{$ 34$}}] at (axis cs:{2.509},{11.146}) {}; % Psc,  5.54 
\node[pin={[pin distance=-0.4\onedegree,Flaamsted]0:{$ 41$}}] at (axis cs:{5.149},{8.190}) {}; % Psc,  5.38 
\node[pin={[pin distance=-0.4\onedegree,Flaamsted]180:{$ 44$}}] at (axis cs:{6.351},{1.940}) {}; % Psc,  5.77 
\node[pin={[pin distance=-0.4\onedegree,Flaamsted]00:{$ 47$}}] at (axis cs:{7.012},{17.893}) {}; % Psc,  5.06 
\node[pin={[pin distance=-0.4\onedegree,Flaamsted]00:{$ 51$}}] at (axis cs:{8.099},{6.955}) {}; % Psc,  5.67 
\node[pin={[pin distance=-0.4\onedegree,Flaamsted]00:{$ 52$}}] at (axis cs:{8.148},{20.294}) {}; % Psc,  5.37 
\node[pin={[pin distance=-0.4\onedegree,Flaamsted]00:{$ 53$}}] at (axis cs:{9.197},{15.231}) {}; % Psc,  5.88 
\node[pin={[pin distance=-0.4\onedegree,Flaamsted]0:{$ 54$}}] at (axis cs:{9.841},{21.250}) {}; % Psc,  5.88 
\node[pin={[pin distance=-0.4\onedegree,Flaamsted]180:{$ 55$}}] at (axis cs:{9.982},{21.438}) {}; % Psc,  5.43 
\node[pin={[pin distance=-0.4\onedegree,Flaamsted]00:{$ 57$}}] at (axis cs:{11.637},{15.475}) {}; % Psc,  5.41 
\node[pin={[pin distance=-0.4\onedegree,Flaamsted]00:{$ 58$}}] at (axis cs:{11.756},{11.974}) {}; % Psc,  5.51 
\node[pin={[pin distance=-0.4\onedegree,Flaamsted]180:{$ 60$}}] at (axis cs:{11.848},{6.741}) {}; % Psc,  5.98 
\node[pin={[pin distance=-0.4\onedegree,Flaamsted]00:{$ 62$}}] at (axis cs:{12.073},{7.300}) {}; % Psc,  5.92 
\node[pin={[pin distance=-0.4\onedegree,Flaamsted]00:{$ 64$}}] at (axis cs:{12.245},{16.941}) {}; % Psc,  5.07 
\node[pin={[pin distance=-0.4\onedegree,Flaamsted]00:{$ 65$}}] at (axis cs:{12.470},{27.711}) {}; % Psc,  5.55 
\node[pin={[pin distance=-0.4\onedegree,Flaamsted]00:{$ 65$}}] at (axis cs:{12.472},{27.710}) {}; % Psc,  5.55 
\node[pin={[pin distance=-0.4\onedegree,Flaamsted]00:{$ 66$}}] at (axis cs:{13.647},{19.188}) {}; % Psc,  5.80 
\node[pin={[pin distance=-0.4\onedegree,Flaamsted]00:{$ 68$}}] at (axis cs:{14.459},{28.992}) {}; % Psc,  5.43 
\node[pin={[pin distance=-0.4\onedegree,Flaamsted]00:{$ 72$}}] at (axis cs:{16.272},{14.946}) {}; % Psc,  5.64 
\node[pin={[pin distance=-0.4\onedegree,Flaamsted]180:{$ 80$}}] at (axis cs:{17.092},{5.650}) {}; % Psc,  5.51 
\node[pin={[pin distance=-0.4\onedegree,Flaamsted]00:{$ 82$}}] at (axis cs:{17.778},{31.425}) {}; % Psc,  5.16 
\node[pin={[pin distance=-0.4\onedegree,Flaamsted]00:{$ 87$}}] at (axis cs:{18.532},{16.133}) {}; % Psc,  5.97 
\node[pin={[pin distance=-0.4\onedegree,Flaamsted]00:{$ 89$}}] at (axis cs:{19.450},{3.614}) {}; % Psc,  5.15 
\node[pin={[pin distance=-0.4\onedegree,Flaamsted]00:{$ 91$}}] at (axis cs:{20.281},{28.738}) {}; % Psc,  5.22 
\node[pin={[pin distance=-0.4\onedegree,Flaamsted]0:{$ 94$}}] at (axis cs:{21.674},{19.240}) {}; % Psc,  5.50 

\node[pin={[pin distance=-0.4\onedegree,Bayer]90:{$\pi$}}] at (axis cs:{109.286},{-37.097}) {}; % Pup,  2.71 
\node[pin={[pin distance=-0.4\onedegree,Bayer]00:{$\alpha$}}] at (axis cs:{14.652},{-29.357}) {}; % Scl,  4.31 
\node[pin={[pin distance=-0.4\onedegree,Bayer]00:{$\epsilon$}}] at (axis cs:{26.411},{-25.053}) {}; % Scl,  5.33 
\node[pin={[pin distance=-0.4\onedegree,Bayer]-90:{$\eta$}}] at (axis cs:{6.982},{-33.007}) {}; % Scl,  4.87 
\node[pin={[pin distance=-0.4\onedegree,Bayer]00:{$\iota$}}] at (axis cs:{5.380},{-28.981}) {}; % Scl,  5.17 
\node[pin={[pin distance=-0.4\onedegree,Bayer]180:{$\kappa^1$}}] at (axis cs:{2.338},{-27.988}) {}; % Scl,  5.44 
\node[pin={[pin distance=-0.4\onedegree,Bayer]-90:{$\kappa^2$}}] at (axis cs:{2.893},{-27.799}) {}; % Scl,  5.41 
\node[pin={[pin distance=-0.4\onedegree,Bayer]180:{$\lambda^2$}}] at (axis cs:{11.050},{-38.422}) {}; % Scl,  5.90 
\node[pin={[pin distance=-0.4\onedegree,Bayer]00:{$\pi$}}] at (axis cs:{25.536},{-32.327}) {}; % Scl,  5.26 
\node[pin={[pin distance=-0.4\onedegree,Bayer]00:{$\sigma$}}] at (axis cs:{15.610},{-31.552}) {}; % Scl,  5.51 
\node[pin={[pin distance=-0.4\onedegree,Bayer]00:{$\tau$}}] at (axis cs:{24.035},{-29.907}) {}; % Scl,  5.70 
\node[pin={[pin distance=-0.4\onedegree,Bayer]00:{$\vartheta$}}] at (axis cs:{2.933},{-35.133}) {}; % Scl,  5.24 
\node[pin={[pin distance=-0.4\onedegree,Bayer]00:{$\xi$}}] at (axis cs:{15.326},{-38.916}) {}; % Scl,  5.59 
\node[pin={[pin distance=-0.4\onedegree,Bayer]180:{$\zeta$}}] at (axis cs:{0.583},{-29.720}) {}; % Scl,  5.04 

\node[pin={[pin distance=-0.4\onedegree,Bayer]0:{$\alpha$}}] at (axis cs:{68.980},{16.509}) {}; % Tau,  0.99 
\node[pin={[pin distance=-0.2\onedegree,Bayer]90:{$\beta$}}] at (axis cs:{81.573},{28.607}) {}; % Tau,  1.68 
\node[pin={[pin distance=-0.4\onedegree,Bayer]00:{$\chi$}}] at (axis cs:{65.646},{25.629}) {}; % Tau,  5.38 
\node[pin={[pin distance=-0.4\onedegree,Bayer]180:{$\delta^{1}$}}] at (axis cs:{65.734},{17.542}) {}; % Tau,  3.76 
\node[pin={[pin distance=-0.8\onedegree,Bayer]45:{$\delta^2$}}] at (axis cs:{66.024},{17.444}) {}; % Tau,  4.80 
\node[pin={[pin distance=-0.4\onedegree,Bayer]0:{$\delta^3$}}] at (axis cs:{66.372},{17.928}) {}; % Tau,  4.31 
\node[pin={[pin distance=-0.4\onedegree,Bayer]00:{$\epsilon$}}] at (axis cs:{67.154},{19.180}) {}; % Tau,  3.54 
\node[pin={[pin distance=-0.4\onedegree,Bayer]-90:{$\eta$}}] at (axis cs:{56.871},{24.105}) {}; % Tau,  2.87 
\node[pin={[pin distance=-0.4\onedegree,Bayer]180:{$\gamma$}}] at (axis cs:{64.948},{15.627}) {}; % Tau,  3.65 
\node[pin={[pin distance=-0.4\onedegree,Bayer]00:{$\iota$}}] at (axis cs:{75.774},{21.590}) {}; % Tau,  4.62 
\node[pin={[pin distance=-0.4\onedegree,Bayer]00:{$\kappa$}}] at (axis cs:{66.342},{22.294}) {}; % Tau,  4.21 
%\node[pin={[pin distance=-0.4\onedegree,Bayer]00:{$\kappa^2$}}] at (axis cs:{66.354},{22.200}) {}; % Tau,  5.27 
\node[pin={[pin distance=-0.4\onedegree,Bayer]00:{$\lambda$}}] at (axis cs:{60.170},{12.490}) {}; % Tau,  3.42 
\node[pin={[pin distance=-0.4\onedegree,Bayer]90:{$\mu$}}] at (axis cs:{63.884},{8.892}) {}; % Tau,  4.29 
\node[pin={[pin distance=-0.4\onedegree,Bayer]00:{$\nu$}}] at (axis cs:{60.789},{5.989}) {}; % Tau,  3.90 
\node[pin={[pin distance=-0.4\onedegree,Bayer]00:{$\omega^1$}}] at (axis cs:{62.292},{19.609}) {}; % Tau,  5.51 
\node[pin={[pin distance=-0.4\onedegree,Bayer]180:{$\omega^2$}}] at (axis cs:{64.315},{20.578}) {}; % Tau,  4.92 
\node[pin={[pin distance=-0.4\onedegree,Bayer]0:{$\omicron$}}] at (axis cs:{51.203},{9.029}) {}; % Tau,  3.61 
\node[pin={[pin distance=-0.4\onedegree,Bayer]90:{$\pi$}}] at (axis cs:{66.652},{14.714}) {}; % Tau,  4.69 
\node[pin={[pin distance=-0.4\onedegree,Bayer]0:{$\rho$}}] at (axis cs:{68.462},{14.844}) {}; % Tau,  4.65 
\node[pin={[pin distance=-0.4\onedegree,Bayer]180:{$\psi$}}] at (axis cs:{61.752},{29.001}) {}; % Tau,  5.22 
\node[pin={[pin distance=-0.4\onedegree,Bayer]90:{$\sigma$}}] at (axis cs:{69.788},{15.800}) {}; % Tau,  5.09 
%\node[pin={[pin distance=-0.4\onedegree,Bayer]180:{$\sigma^2$}}] at (axis cs:{69.819},{15.918}) {}; % Tau,  4.67 
\node[pin={[pin distance=-0.4\onedegree,Bayer]00:{$\tau$}}] at (axis cs:{70.561},{22.957}) {}; % Tau,  4.27 
\node[pin={[pin distance=-0.4\onedegree,Bayer]00:{$\upsilon$}}] at (axis cs:{66.577},{22.813}) {}; % Tau,  4.29 
\node[pin={[pin distance=-0.4\onedegree,Bayer]00:{$\varphi$}}] at (axis cs:{65.088},{27.351}) {}; % Tau,  4.95 
\node[pin={[pin distance=-0.8\onedegree,Bayer]90:{$\vartheta$}}] at (axis cs:{67.0},{15.45}) {}; % Tau,  3.41 
\node[pin={[pin distance=-0.4\onedegree,Bayer]-90:{$\xi$}}] at (axis cs:{51.792},{9.732}) {}; % Tau,  3.73 
\node[pin={[pin distance=-0.4\onedegree,Bayer]00:{$\zeta$}}] at (axis cs:{84.411},{21.142}) {}; % Tau,  3.00 
\node[pin={[pin distance=-0.4\onedegree,Flaamsted]180:{$103$}}] at (axis cs:{77.028},{24.265}) {}; % Tau,  5.51 
\node[pin={[pin distance=-0.4\onedegree,Flaamsted]00:{$104$}}] at (axis cs:{76.863},{18.645}) {}; % Tau,  4.92 
%\node[pin={[pin distance=-0.4\onedegree,Flaamsted]180:{$105$}}] at (axis cs:{76.981},{21.705}) {}; % Tau,  5.84 
\node[pin={[pin distance=-0.4\onedegree,Flaamsted]00:{$106$}}] at (axis cs:{76.952},{20.418}) {}; % Tau,  5.29 
\node[pin={[pin distance=-0.4\onedegree,Flaamsted]-90:{$109$}}] at (axis cs:{79.819},{22.096}) {}; % Tau,  4.95 
\node[pin={[pin distance=-0.4\onedegree,Flaamsted]00:{$ 10$}}] at (axis cs:{54.218},{0.402}) {}; % Tau,  4.30 
\node[pin={[pin distance=-0.4\onedegree,Flaamsted]180:{$111$}}] at (axis cs:{81.106},{17.384}) {}; % Tau,  5.00 
\node[pin={[pin distance=-0.4\onedegree,Flaamsted]00:{$114$}}] at (axis cs:{81.909},{21.937}) {}; % Tau,  4.88 
\node[pin={[pin distance=-0.4\onedegree,Flaamsted]00:{$115$}}] at (axis cs:{81.792},{17.962}) {}; % Tau,  5.41 
\node[pin={[pin distance=-0.4\onedegree,Flaamsted]0:{$116$}}] at (axis cs:{81.940},{15.874}) {}; % Tau,  5.52 
\node[pin={[pin distance=-0.4\onedegree,Flaamsted]90:{$117$}}] at (axis cs:{82.007},{17.239}) {}; % Tau,  5.77 
\node[pin={[pin distance=-0.4\onedegree,Flaamsted]00:{$118$}}] at (axis cs:{82.319},{25.150}) {}; % Tau,  5.85 
\node[pin={[pin distance=-0.4\onedegree,Flaamsted]180:{$119$}}] at (axis cs:{83.053},{18.594}) {}; % Tau,  4.36 
\node[pin={[pin distance=-0.4\onedegree,Flaamsted]0:{$120$}}] at (axis cs:{83.382},{18.540}) {}; % Tau,  5.67 
\node[pin={[pin distance=-0.4\onedegree,Flaamsted]00:{$121$}}] at (axis cs:{83.863},{24.039}) {}; % Tau,  5.38 
\node[pin={[pin distance=-0.4\onedegree,Flaamsted]90:{$122$}}] at (axis cs:{84.266},{17.040}) {}; % Tau,  5.53 
\node[pin={[pin distance=-0.4\onedegree,Flaamsted]00:{$125$}}] at (axis cs:{84.934},{25.897}) {}; % Tau,  5.17 
\node[pin={[pin distance=-0.4\onedegree,Flaamsted]00:{$126$}}] at (axis cs:{85.324},{16.534}) {}; % Tau,  4.84 
\node[pin={[pin distance=-0.4\onedegree,Flaamsted]00:{$ 12$}}] at (axis cs:{54.963},{3.057}) {}; % Tau,  5.55 
\node[pin={[pin distance=-0.4\onedegree,Flaamsted]00:{$130$}}] at (axis cs:{86.859},{17.729}) {}; % Tau,  5.48 
\node[pin={[pin distance=-0.4\onedegree,Flaamsted]-90:{$131$}}] at (axis cs:{86.805},{14.488}) {}; % Tau,  5.73 
\node[pin={[pin distance=-0.4\onedegree,Flaamsted]00:{$132$}}] at (axis cs:{87.254},{24.568}) {}; % Tau,  4.89 
\node[pin={[pin distance=-0.4\onedegree,Flaamsted]180:{$133$}}] at (axis cs:{86.929},{13.899}) {}; % Tau,  5.28 
\node[pin={[pin distance=-0.4\onedegree,Flaamsted]-90:{$134$}}] at (axis cs:{87.387},{12.651}) {}; % Tau,  4.88 
\node[pin={[pin distance=-0.4\onedegree,Flaamsted]180:{$136$}}] at (axis cs:{88.332},{27.612}) {}; % Tau,  4.56 
\node[pin={[pin distance=-0.4\onedegree,Flaamsted]-90:{$137$}}] at (axis cs:{88.093},{14.172}) {}; % Tau,  5.60 
\node[pin={[pin distance=-0.4\onedegree,Flaamsted]90:{$139$}}] at (axis cs:{89.499},{25.954}) {}; % Tau,  4.83 
\node[pin={[pin distance=-0.4\onedegree,Flaamsted]00:{$ 13$}}] at (axis cs:{55.579},{19.700}) {}; % Tau,  5.69 
\node[pin={[pin distance=-0.4\onedegree,Flaamsted]180:{$ 16$}}] at (axis cs:{56.201},{24.289}) {}; % Tau,  5.46 
\node[pin={[pin distance=-0.8\onedegree,Flaamsted]135:{$ 17$}}] at (axis cs:{56.219},{24.113}) {}; % Tau,  3.71 
\node[pin={[pin distance=-0.4\onedegree,Flaamsted]-90:{$ 18$}}] at (axis cs:{56.291},{24.839}) {}; % Tau,  5.66 
\node[pin={[pin distance=-0.8\onedegree,Flaamsted]-135:{$ 19$}}] at (axis cs:{56.302},{24.467}) {}; % Tau,  4.30 
%\node[pin={[pin distance=-0.4\onedegree,Flaamsted]00:{$ 20$}}] at (axis cs:{56.457},{24.368}) {}; % Tau,  3.88 
%\node[pin={[pin distance=-0.4\onedegree,Flaamsted]00:{$ 21$}}] at (axis cs:{56.477},{24.554}) {}; % Tau,  5.77 
\node[pin={[pin distance=-0.4\onedegree,Flaamsted]90:{$ 23$}}] at (axis cs:{56.582},{23.948}) {}; % Tau,  4.17 
\node[pin={[pin distance=-0.4\onedegree,Flaamsted]0:{$ 27$}}] at (axis cs:{57.291},{24.053}) {}; % Tau,  3.63 
%\node[pin={[pin distance=-0.8\onedegree,Flaamsted]-135:{$ 28$}}] at (axis cs:{57.297},{24.137}) {}; % Tau,  5.06 
\node[pin={[pin distance=-0.4\onedegree,Flaamsted]00:{$ 29$}}] at (axis cs:{56.419},{6.050}) {}; % Tau,  5.34 
\node[pin={[pin distance=-0.4\onedegree,Flaamsted]00:{$ 30$}}] at (axis cs:{57.068},{11.143}) {}; % Tau,  5.08 
\node[pin={[pin distance=-0.4\onedegree,Flaamsted]00:{$ 31$}}] at (axis cs:{58.001},{6.535}) {}; % Tau,  5.69 
\node[pin={[pin distance=-0.4\onedegree,Flaamsted]00:{$ 32$}}] at (axis cs:{59.217},{22.478}) {}; % Tau,  5.62 
\node[pin={[pin distance=-0.4\onedegree,Flaamsted]00:{$ 36$}}] at (axis cs:{61.090},{24.106}) {}; % Tau,  5.51 
\node[pin={[pin distance=-0.4\onedegree,Flaamsted]00:{$ 37$}}] at (axis cs:{61.174},{22.082}) {}; % Tau,  4.35 
\node[pin={[pin distance=-0.4\onedegree,Flaamsted]180:{$ 39$}}] at (axis cs:{61.334},{22.009}) {}; % Tau,  5.89 
\node[pin={[pin distance=-0.4\onedegree,Flaamsted]00:{$ 40$}}] at (axis cs:{60.936},{5.436}) {}; % Tau,  5.32 
\node[pin={[pin distance=-0.4\onedegree,Flaamsted]00:{$ 41$}}] at (axis cs:{61.652},{27.600}) {}; % Tau,  5.18 
\node[pin={[pin distance=-0.4\onedegree,Flaamsted]00:{$ 44$}}] at (axis cs:{62.708},{26.481}) {}; % Tau,  5.40 
\node[pin={[pin distance=-0.4\onedegree,Flaamsted]00:{$ 45$}}] at (axis cs:{62.835},{5.523}) {}; % Tau,  5.71 
\node[pin={[pin distance=-0.4\onedegree,Flaamsted]00:{$ 46$}}] at (axis cs:{63.388},{7.716}) {}; % Tau,  5.29 
\node[pin={[pin distance=-0.4\onedegree,Flaamsted]180:{$ 47$}}] at (axis cs:{63.485},{9.264}) {}; % Tau,  4.89 
\node[pin={[pin distance=-0.4\onedegree,Flaamsted]00:{$  4$}}] at (axis cs:{52.602},{11.336}) {}; % Tau,  5.13 
\node[pin={[pin distance=-0.4\onedegree,Flaamsted]180:{$ 51$}}] at (axis cs:{64.597},{21.579}) {}; % Tau,  5.64 
\node[pin={[pin distance=-0.4\onedegree,Flaamsted]90:{$ 53$}}] at (axis cs:{64.859},{21.142}) {}; % Tau,  5.50 
\node[pin={[pin distance=-0.4\onedegree,Flaamsted]180:{$ 56$}}] at (axis cs:{64.903},{21.773}) {}; % Tau,  5.36 
%\node[pin={[pin distance=-0.4\onedegree,Flaamsted]90:{$ 57$}}] at (axis cs:{64.990},{14.035}) {}; % Tau,  5.58 
\node[pin={[pin distance=-0.4\onedegree,Flaamsted]180:{$ 58$}}] at (axis cs:{65.151},{15.095}) {}; % Tau,  5.26 
\node[pin={[pin distance=-0.4\onedegree,Flaamsted]-90:{$  5$}}] at (axis cs:{52.718},{12.937}) {}; % Tau,  4.13 
%\node[pin={[pin distance=-0.4\onedegree,Flaamsted]45:{$ 60$}}] at (axis cs:{65.515},{14.077}) {}; % Tau,  5.72 
\node[pin={[pin distance=-0.4\onedegree,Flaamsted]180:{$ 63$}}] at (axis cs:{65.854},{16.777}) {}; % Tau,  5.63 
\node[pin={[pin distance=-0.4\onedegree,Flaamsted]00:{$ 66$}}] at (axis cs:{65.966},{9.461}) {}; % Tau,  5.11 
\node[pin={[pin distance=-0.4\onedegree,Flaamsted]00:{$  6$}}] at (axis cs:{53.150},{9.373}) {}; % Tau,  5.76 
\node[pin={[pin distance=-0.4\onedegree,Flaamsted]0:{$ 71$}}] at (axis cs:{66.586},{15.618}) {}; % Tau,  4.49 
%\node[pin={[pin distance=-0.4\onedegree,Flaamsted]90:{$ 72$}}] at (axis cs:{66.823},{22.996}) {}; % Tau,  5.52 
\node[pin={[pin distance=-0.4\onedegree,Flaamsted]180:{$ 75$}}] at (axis cs:{67.110},{16.360}) {}; % Tau,  4.97 
%\node[pin={[pin distance=-0.4\onedegree,Flaamsted]90:{$ 76$}}] at (axis cs:{67.098},{14.741}) {}; % Tau,  5.91 
\node[pin={[pin distance=-0.4\onedegree,Flaamsted]00:{$ 79$}}] at (axis cs:{67.209},{13.047}) {}; % Tau,  5.02 
\node[pin={[pin distance=-0.4\onedegree,Flaamsted]00:{$  7$}}] at (axis cs:{53.611},{24.464}) {}; % Tau,  5.99 
%\node[pin={[pin distance=-0.4\onedegree,Flaamsted]90:{$ 80$}}] at (axis cs:{67.536},{15.638}) {}; % Tau,  5.64 
\node[pin={[pin distance=-0.4\onedegree,Flaamsted]0:{$ 81$}}] at (axis cs:{67.662},{15.692}) {}; % Tau,  5.46 
\node[pin={[pin distance=-0.4\onedegree,Flaamsted]00:{$ 83$}}] at (axis cs:{67.656},{13.724}) {}; % Tau,  5.40 
\node[pin={[pin distance=-0.4\onedegree,Flaamsted]00:{$ 88$}}] at (axis cs:{68.914},{10.161}) {}; % Tau,  4.25 
%\node[pin={[pin distance=-0.8\onedegree,Flaamsted]-45:{$ 89$}}] at (axis cs:{69.539},{16.033}) {}; % Tau,  5.78 
\node[pin={[pin distance=-0.4\onedegree,Flaamsted]180:{$ 90$}}] at (axis cs:{69.539},{12.511}) {}; % Tau,  4.27 
\node[pin={[pin distance=-0.4\onedegree,Flaamsted]90:{$ 93$}}] at (axis cs:{70.014},{12.197}) {}; % Tau,  5.46 
\node[pin={[pin distance=-0.4\onedegree,Flaamsted]00:{$ 97$}}] at (axis cs:{72.844},{18.840}) {}; % Tau,  5.09 
\node[pin={[pin distance=-0.4\onedegree,Flaamsted]00:{$ 98$}}] at (axis cs:{74.539},{25.050}) {}; % Tau,  5.79 
\node[pin={[pin distance=-0.4\onedegree,Flaamsted]00:{$ 99$}}] at (axis cs:{74.453},{23.948}) {}; % Tau,  5.81 
\node[pin={[pin distance=-0.4\onedegree,Bayer]00:{$\alpha$}}] at (axis cs:{28.270},{29.579}) {}; % Tri,  3.42 
\node[pin={[pin distance=-0.4\onedegree,Bayer]00:{$\beta$}}] at (axis cs:{32.386},{34.987}) {}; % Tri,  3.02 
\node[pin={[pin distance=-0.4\onedegree,Bayer]-90:{$\delta$}}] at (axis cs:{34.263},{34.224}) {}; % Tri,  4.87 
\node[pin={[pin distance=-0.4\onedegree,Bayer]90:{$\epsilon$}}] at (axis cs:{30.741},{33.284}) {}; % Tri,  5.52 
\node[pin={[pin distance=-0.4\onedegree,Bayer]00:{$\gamma$}}] at (axis cs:{34.329},{33.847}) {}; % Tri,  4.01 
\node[pin={[pin distance=-0.4\onedegree,Flaamsted]00:{$ 10$}}] at (axis cs:{34.737},{28.643}) {}; % Tri,  5.30 
\node[pin={[pin distance=-0.4\onedegree,Flaamsted]00:{$ 11$}}] at (axis cs:{36.866},{31.801}) {}; % Tri,  5.56 
\node[pin={[pin distance=-0.4\onedegree,Flaamsted]00:{$ 12$}}] at (axis cs:{37.042},{29.669}) {}; % Tri,  5.29 
\node[pin={[pin distance=-0.4\onedegree,Flaamsted]-90:{$ 13$}}] at (axis cs:{37.202},{29.932}) {}; % Tri,  5.90 
\node[pin={[pin distance=-0.4\onedegree,Flaamsted]00:{$ 14$}}] at (axis cs:{38.026},{36.147}) {}; % Tri,  5.15 
\node[pin={[pin distance=-0.4\onedegree,Flaamsted]-90:{$ 15$}}] at (axis cs:{38.945},{34.687}) {}; % Tri,  5.39 
\node[pin={[pin distance=-0.4\onedegree,Flaamsted]00:{$  6$}}] at (axis cs:{33.093},{30.303}) {}; % Tri,  5.15 
\node[pin={[pin distance=-0.4\onedegree,Flaamsted]00:{$  7$}}] at (axis cs:{33.985},{33.359}) {}; % Tri,  5.25 
\end{axis}


%
% Comets, own objects etc.
\begin{axis}[name=transients,axis lines=none]
% Format for label:
%\node[transient,pin={[pin distance=-0.8\onedegree,transient-label]90:{LABEL}}] at (axis cs:{ *15 + /4 + /240 },{  + /60 + /3600 })  {+};
%
% Note for Southern positions, if in sexagesimal, decl. signs need to be inverted: 
%   \(axis cs:{ *15 + /4 + /240 },{   - /60 - /3600 })  {+};
%
% Note "08" is interpreted as Octal 8, and triggers a problem in computation

%%%%%%%%%%%%%%%%%%%%%%%%%%%%%%%%%%%%%%%%%%%%%%%%%%%%%%%%%%%%%%%%%%%%%%%%%%%%%%%%%%%%%%%%
% Comet C/2023 A3 (Tsuchinshan-ATLAS) in 2024, Apr-Dec\
% Coords taken from https://in-the-sky.org/ephemeris.php?objtxt=CK23A030
%%%%%%%%%%%%%%%%%%%%%%%%%%%%%%%%%%%%%%%%%%%%%%%%%%%%%%%%%%%%%%%%%%%%%%%%%%%%%%%%%%%%%%%%
%
% \node[transient,pin={[pin distance=-0.8\onedegree,transient-label]90:{Apr\,01}}] at (axis cs:{14 *15 + 32/4 +  30/240 },{ -04  - 59/60 -  6/3600 })  {+};
% \node[transient,pin={[pin distance=-0.8\onedegree,transient-label]90:{Apr\,08}}] at (axis cs:{14 *15 + 19/4 +  22/240 },{ -04  -  8/60 - 35/3600 })  {+};
% \node[transient,pin={[pin distance=-0.8\onedegree,transient-label]90:{Apr\,15}}] at (axis cs:{14 *15 +  4/4 +  04/240 },{ -03  - 11/60 - 56/3600 })  {+};
% \node[transient,pin={[pin distance=-0.8\onedegree,transient-label]90:{Apr\,22}}] at (axis cs:{13 *15 + 46/4 +  53/240 },{ -02  - 10/60 - 50/3600 })  {+};
% \node[transient,pin={[pin distance=-0.8\onedegree,transient-label]90:{May\,01}}] at (axis cs:{13 *15 + 22/4 +  47/240 },{ -00  - 50/60 -  5/3600 })  {+};
% \node[transient,pin={[pin distance=-0.8\onedegree,transient-label]90:{May\,08}}] at (axis cs:{13 *15 +  3/4 +  21/240 },{ +00  + 10/60 +  8/3600 })  {+};
% \node[transient,pin={[pin distance=-0.8\onedegree,transient-label]90:{May\,15}}] at (axis cs:{12 *15 + 44/4 +   8/240 },{ +01  +  4/60 + 17/3600 })  {+};
% \node[transient,pin={[pin distance=-0.8\onedegree,transient-label]90:{May\,22}}] at (axis cs:{12 *15 + 25/4 +  54/240 },{ +01  + 49/60 + 32/3600 })  {+};
% \node[transient,pin={[pin distance=-0.8\onedegree,transient-label]90:{Jun\,01}}] at (axis cs:{12 *15 +  2/4 +  36/240 },{ +02  + 35/60 + 35/3600 })  {+};
% \node[transient,pin={[pin distance=-0.8\onedegree,transient-label]90:{Jun\,08}}] at (axis cs:{11 *15 + 48/4 +  37/240 },{ +02  + 54/60 + 16/3600 })  {+};
% \node[transient,pin={[pin distance=-0.8\onedegree,transient-label]90:{Jun\,15}}] at (axis cs:{11 *15 + 36/4 +  38/240 },{ +03  +  2/60 + 21/3600 })  {+};
% \node[transient,pin={[pin distance=-0.8\onedegree,transient-label]90:{Jun\,22}}] at (axis cs:{11 *15 + 26/4 +  34/240 },{ +03  +  0/60 + 47/3600 })  {+};
% \node[transient,pin={[pin distance=-0.8\onedegree,transient-label]90:{Jul\,01}}] at (axis cs:{11 *15 + 16/4 +  10/240 },{ +02  + 46/60 + 21/3600 })  {+};
% \node[transient,pin={[pin distance=-0.8\onedegree,transient-label]90:{Jul\,08}}] at (axis cs:{11 *15 +  9/4 +  44/240 },{ +02  + 26/60 + 41/3600 })  {+};
% \node[transient,pin={[pin distance=-0.8\onedegree,transient-label]90:{Jul\,15}}] at (axis cs:{11 *15 +  4/4 +  32/240 },{ +02  +  0/60 + 32/3600 })  {+};
% \node[transient,pin={[pin distance=-0.8\onedegree,transient-label]90:{Jul\,22}}] at (axis cs:{11 *15 +  0/4 +  19/240 },{ +01  + 28/60 + 34/3600 })  {+};
% \node[transient,pin={[pin distance=-0.8\onedegree,transient-label]90:{Aug\,01}}] at (axis cs:{10 *15 + 55/4 +  30/240 },{ +00  + 33/60 + 43/3600 })  {+};
% \node[transient,pin={[pin distance=-0.8\onedegree,transient-label]90:{Aug\,08}}] at (axis cs:{10 *15 + 52/4 +  42/240 },{ -00  - 10/60 - 40/3600 })  {+};
% \node[transient,pin={[pin distance=-0.8\onedegree,transient-label]90:{Aug\,15}}] at (axis cs:{10 *15 + 50/4 +  05/240 },{ -00  - 59/60 - 45/3600 })  {+};
% \node[transient,pin={[pin distance=-0.8\onedegree,transient-label]90:{Aug\,22}}] at (axis cs:{10 *15 + 47/4 +  26/240 },{ -01  - 53/60 - 19/3600 })  {+};
% \node[transient,pin={[pin distance=-0.8\onedegree,transient-label]90:{Sep\,01}}] at (axis cs:{10 *15 + 43/4 +  06/240 },{ -03  - 16/60 - 41/3600 })  {+};
% \node[transient,pin={[pin distance=-0.8\onedegree,transient-label]90:{Sep\,08}}] at (axis cs:{10 *15 + 39/4 +  30/240 },{ -04  - 17/60 - 50/3600 })  {+};
% \node[transient,pin={[pin distance=-0.8\onedegree,transient-label]90:{Sep\,15}}] at (axis cs:{10 *15 + 35/4 +  57/240 },{ -05  - 16/60 -  5/3600 })  {+};
% \node[transient,pin={[pin distance=-0.8\onedegree,transient-label]90:{Sep\,22}}] at (axis cs:{10 *15 + 35/4 +  49/240 },{ -05  - 58/60 -  7/3600 })  {+};
% \node[transient,pin={[pin distance=-0.8\onedegree,transient-label]90:{Oct\,01}}] at (axis cs:{11 *15 +  4/4 +  03/240 },{ -05  - 41/60 - 36/3600 })  {+};
% \node[transient,pin={[pin distance=-0.8\onedegree,transient-label]90:{Oct\,08}}] at (axis cs:{12 *15 + 30/4 +  27/240 },{ -03  - 36/60 - 16/3600 })  {+};
% \node[transient,pin={[pin distance=-0.8\onedegree,transient-label]90:{Oct\,15}}] at (axis cs:{14 *15 + 59/4 +  40/240 },{ +00  + 23/60 + 14/3600 })  {+};
% \node[transient,pin={[pin distance=-0.8\onedegree,transient-label]90:{Oct\,22}}] at (axis cs:{16 *15 + 49/4 +  25/240 },{ +02  + 50/60 + 11/3600 })  {+};
% \node[transient,pin={[pin distance=-0.8\onedegree,transient-label]90:{Nov\,01}}] at (axis cs:{17 *15 + 59/4 +  40/240 },{ +03  + 45/60 + 12/3600 })  {+};
% \node[transient,pin={[pin distance=-0.8\onedegree,transient-label]90:{Nov\,08}}] at (axis cs:{18 *15 + 24/4 +  28/240 },{ +03  + 55/60 + 51/3600 })  {+};
% \node[transient,pin={[pin distance=-0.8\onedegree,transient-label]90:{Nov\,15}}] at (axis cs:{18 *15 + 41/4 +  24/240 },{ +04  +  3/60 + 27/3600 })  {+};
% \node[transient,pin={[pin distance=-0.8\onedegree,transient-label]90:{Nov\,22}}] at (axis cs:{18 *15 + 54/4 +  18/240 },{ +04  + 12/60 + 40/3600 })  {+};
% \node[transient,pin={[pin distance=-0.8\onedegree,transient-label]90:{Dec\,01}}] at (axis cs:{19 *15 +  7/4 +  39/240 },{ +04  + 29/60 + 30/3600 })  {+};
% \node[transient,pin={[pin distance=-0.8\onedegree,transient-label]90:{Dec\,08}}] at (axis cs:{19 *15 + 16/4 +  32/240 },{ +04  + 47/60 + 12/3600 })  {+};
% \node[transient,pin={[pin distance=-0.8\onedegree,transient-label]90:{Dec\,15}}] at (axis cs:{19 *15 + 24/4 +  32/240 },{ +05  +  9/60 +  7/3600 })  {+};
% \node[transient,pin={[pin distance=-0.8\onedegree,transient-label]90:{Dec\,22}}] at (axis cs:{19 *15 + 31/4 +  54/240 },{ +05  + 35/60 + 17/3600 })  {+};
% 
% 
% \draw[transient]
% (axis cs:{14 *15 + 32/4 + 30/240 },{ -04  - 59/60 -  6/3600 }) --
% (axis cs:{14 *15 + 19/4 + 22/240 },{ -04  -  8/60 - 35/3600 }) --
% (axis cs:{14 *15 +  4/4 + 04/240 },{ -03  - 11/60 - 56/3600 }) --
% (axis cs:{13 *15 + 46/4 + 53/240 },{ -02  - 10/60 - 50/3600 }) --
% (axis cs:{13 *15 + 22/4 + 47/240 },{ -00  - 50/60 -  5/3600 }) --
% (axis cs:{13 *15 +  3/4 + 21/240 },{ +00  + 10/60 +  8/3600 }) --
% (axis cs:{12 *15 + 44/4 +  8/240 },{ +01  +  4/60 + 17/3600 }) --
% (axis cs:{12 *15 + 25/4 + 54/240 },{ +01  + 49/60 + 32/3600 }) --
% (axis cs:{12 *15 +  2/4 + 36/240 },{ +02  + 35/60 + 35/3600 }) --
% (axis cs:{11 *15 + 48/4 + 37/240 },{ +02  + 54/60 + 16/3600 }) --
% (axis cs:{11 *15 + 36/4 + 38/240 },{ +03  +  2/60 + 21/3600 }) --
% (axis cs:{11 *15 + 26/4 + 34/240 },{ +03  +  0/60 + 47/3600 }) --
% (axis cs:{11 *15 + 16/4 + 10/240 },{ +02  + 46/60 + 21/3600 }) --
% (axis cs:{11 *15 +  9/4 + 44/240 },{ +02  + 26/60 + 41/3600 }) --
% (axis cs:{11 *15 +  4/4 + 32/240 },{ +02  +  0/60 + 32/3600 }) --
% (axis cs:{11 *15 +  0/4 + 19/240 },{ +01  + 28/60 + 34/3600 }) --
% (axis cs:{10 *15 + 55/4 + 30/240 },{ +00  + 33/60 + 43/3600 }) --
% (axis cs:{10 *15 + 52/4 + 42/240 },{ -00  - 10/60 - 40/3600 }) --
% (axis cs:{10 *15 + 50/4 + 05/240 },{ -00  - 59/60 - 45/3600 }) --
% (axis cs:{10 *15 + 47/4 + 26/240 },{ -01  - 53/60 - 19/3600 }) --
% (axis cs:{10 *15 + 43/4 + 06/240 },{ -03  - 16/60 - 41/3600 }) --
% (axis cs:{10 *15 + 39/4 + 30/240 },{ -04  - 17/60 - 50/3600 }) --
% (axis cs:{10 *15 + 35/4 + 57/240 },{ -05  - 16/60 -  5/3600 }) --
% (axis cs:{10 *15 + 35/4 + 49/240 },{ -05  - 58/60 -  7/3600 }) --
% (axis cs:{11 *15 +  4/4 + 03/240 },{ -05  - 41/60 - 36/3600 }) --
% (axis cs:{12 *15 + 30/4 + 27/240 },{ -03  - 36/60 - 16/3600 }) --
% (axis cs:{14 *15 + 59/4 + 40/240 },{ +00  + 23/60 + 14/3600 }) --
% (axis cs:{16 *15 + 49/4 + 25/240 },{ +02  + 50/60 + 11/3600 }) --
% (axis cs:{17 *15 + 59/4 + 40/240 },{ +03  + 45/60 + 12/3600 }) --
% (axis cs:{18 *15 + 24/4 + 28/240 },{ +03  + 55/60 + 51/3600 }) --
% (axis cs:{18 *15 + 41/4 + 24/240 },{ +04  +  3/60 + 27/3600 }) --
% (axis cs:{18 *15 + 54/4 + 18/240 },{ +04  + 12/60 + 40/3600 }) --
% (axis cs:{19 *15 +  7/4 + 39/240 },{ +04  + 29/60 + 30/3600 }) --
% (axis cs:{19 *15 + 16/4 + 32/240 },{ +04  + 47/60 + 12/3600 }) --
% (axis cs:{19 *15 + 24/4 + 32/240 },{ +05  +  9/60 +  7/3600 }) --
% (axis cs:{19 *15 + 31/4 + 54/240 },{ +05  + 35/60 + 17/3600 }) ;
% 
% 

\end{axis}

%

\end{tikzpicture}

\end{document}
