
\documentclass[10pt,landscape]{article}


\usepackage[margin=1cm,a3paper]{geometry}

\def\pgfsysdriver{pgfsys-pdftex.def}
\usepackage{pgfplots}
\usepgfplotslibrary{external} 
\usetikzlibrary{pgfplots.external}
\usetikzlibrary{calc}
\usetikzlibrary{shapes}
\usetikzlibrary{patterns}
\usetikzlibrary{plotmarks}
\usepgflibrary{arrows}
\usepgfplotslibrary{polar}
\tikzexternalize

\usepackage{times,latexsym,amssymb}
\usepackage{eulervm} % math font
\pagestyle{empty}

\usepackage{amsmath} 
\usepackage{nicefrac} 
%
% This is used because from 1.11 onwards the use of \pgfdeclareplotmark will
% introduce a shift of the stellar markers.
%
\pgfplotsset{compat=1.10}
%
% Use to increase spacing for constellation labels
\usepackage[letterspace=400]{microtype}
%
\usepackage{pagecolor}
\usepackage{natbib}
%


%\pagecolor{black}
\pagecolor{white}

%%%%%%%%%%%%%%%%%%%%%%%%%%%%%%%%%%%%%%%%%%%%%%%%%%%%%%%%%%%%%%%%%%%%%%
\begin{document}




%%%%%%%%%%%%%%%%%%%%%%%%%%%%%%%%%%%%%%%%%%%%%%%%%%%%%%%%%%%%%%%%%%%%%%%%%%%%%%%%%%%%%%%%%%
%%%%%%%%%%%%%%%%%%%%%%%%%%%%%%%%%%%%%%%%%%%%%%%%%%%%%%%%%%%%%%%%%%%%%%%%%%%%%%%%%%%%%%%%%%
%   Magnitude markers
%%%%%%%%%%%%%%%%%%%%%%%%%%%%%%%%%%%%%%%%%%%%%%%%%%%%%%%%%%%%%%%%%%%%%%%%%%%%%%%%%%%%%%%%%%
%%%%%%%%%%%%%%%%%%%%%%%%%%%%%%%%%%%%%%%%%%%%%%%%%%%%%%%%%%%%%%%%%%%%%%%%%%%%%%%%%%%%%%%%%%

\pgfdeclareplotmark{m1b}{%
\node[scale=\mOnescale*\AtoBscale*\symscale] at (0,0) {\tikz {%
\draw[fill=black,line width=.3pt,even odd rule,rotate=90] 
(0.*25.71428:10pt)  -- (0.5*25.71428:5pt)   -- 
(1.*25.71428:10pt)  -- (1.5*25.71428:5pt)   -- 
(2.*25.71428:10pt)  -- (2.5*25.71428:5pt)   -- 
(3.*25.71428:10pt)  -- (3.5*25.71428:5pt)   -- 
(4.*25.71428:10pt)  -- (4.5*25.71428:5pt)   -- 
(5.*25.71428:10pt)  -- (5.5*25.71428:5pt)   -- 
(6.*25.71428:10pt)  -- (6.5*25.71428:5pt)   -- 
(7.*25.71428:10pt)  -- (7.5*25.71428:5pt)   -- 
(8.*25.71428:10pt)  -- (8.5*25.71428:5pt)   -- 
(9.*25.71428:10pt)  -- (9.5*25.71428:5pt)   -- 
(10.*25.71428:10pt) -- (10.5*25.71428:5pt)  -- 
(11.*25.71428:10pt) -- (11.5*25.71428:5pt)  -- 
(12.*25.71428:10pt) -- (12.5*25.71428:5pt)  -- 
(13.*25.71428:10pt) -- (13.5*25.71428:5pt)  --  cycle;
}}}


\pgfdeclareplotmark{m1bv}{%
\node[scale=\mOnescale*\AtoBscale*\symscale] at (0,0) {\tikz {%
\draw[fill=black,line width=.3pt,even odd rule,rotate=90] 
(0.*25.71428:9pt)  -- (0.5*25.71428:4.5pt)   -- 
(1.*25.71428:9pt)  -- (1.5*25.71428:4.5pt)   -- 
(2.*25.71428:9pt)  -- (2.5*25.71428:4.5pt)   -- 
(3.*25.71428:9pt)  -- (3.5*25.71428:4.5pt)   -- 
(4.*25.71428:9pt)  -- (4.5*25.71428:4.5pt)   -- 
(5.*25.71428:9pt)  -- (5.5*25.71428:4.5pt)   -- 
(6.*25.71428:9pt)  -- (6.5*25.71428:4.5pt)   -- 
(7.*25.71428:9pt)  -- (7.5*25.71428:4.5pt)   -- 
(8.*25.71428:9pt)  -- (8.5*25.71428:4.5pt)   -- 
(9.*25.71428:9pt)  -- (9.5*25.71428:4.5pt)   -- 
(10.*25.71428:9pt) -- (10.5*25.71428:4.5pt)  -- 
(11.*25.71428:9pt) -- (11.5*25.71428:4.5pt)  -- 
(12.*25.71428:9pt) -- (12.5*25.71428:4.5pt)  -- 
(13.*25.71428:9pt) -- (13.5*25.71428:4.5pt)  --cycle;
\draw (0,0) circle (9pt);
%\draw[line width=.2pt,even odd rule,rotate=90] 
%(0.*25.71428:11pt)  -- (0.5*25.71428:6.5pt)   -- 
%(1.*25.71428:11pt)  -- (1.5*25.71428:6.5pt)   -- 
%(2.*25.71428:11pt)  -- (2.5*25.71428:6.5pt)   -- 
%(3.*25.71428:11pt)  -- (3.5*25.71428:6.5pt)   -- 
%(4.*25.71428:11pt)  -- (4.5*25.71428:6.5pt)   -- 
%(5.*25.71428:11pt)  -- (5.5*25.71428:6.5pt)   -- 
%(6.*25.71428:11pt)  -- (6.5*25.71428:6.5pt)   -- 
%(7.*25.71428:11pt)  -- (7.5*25.71428:6.5pt)   -- 
%(8.*25.71428:11pt)  -- (8.5*25.71428:6.5pt)   -- 
%(9.*25.71428:11pt)  -- (9.5*25.71428:6.5pt)   -- 
%(10.*25.71428:11pt) -- (10.5*25.71428:6.5pt)  -- 
%(11.*25.71428:11pt) -- (11.5*25.71428:6.5pt)  -- 
%(12.*25.71428:11pt) -- (12.5*25.71428:6.5pt)  -- 
%(13.*25.71428:11pt) -- (13.5*25.71428:6.5pt)  --  cycle;
}}}



\pgfdeclareplotmark{m1bb}{%
\node[scale=\mOnescale*\AtoBscale*\symscale] at (0,0) {\tikz {%
\draw[fill=black,line width=.3pt,even odd rule,rotate=90] 
(0.*25.71428:10pt)  -- (0.5*25.71428:5pt)   -- 
(1.*25.71428:10pt)  -- (1.5*25.71428:5pt)   -- 
(2.*25.71428:10pt)  -- (2.5*25.71428:5pt)   -- 
(3.*25.71428:10pt)  -- (3.5*25.71428:5pt)   -- 
(4.*25.71428:10pt)  -- (4.5*25.71428:5pt)   -- 
(5.*25.71428:10pt)  -- (5.5*25.71428:5pt)   -- 
(6.*25.71428:10pt)  -- (6.5*25.71428:5pt)   -- 
(7.*25.71428:10pt)  -- (7.5*25.71428:5pt)   -- 
(8.*25.71428:10pt)  -- (8.5*25.71428:5pt)   -- 
(9.*25.71428:10pt)  -- (9.5*25.71428:5pt)   -- 
(10.*25.71428:10pt) -- (10.5*25.71428:5pt)  -- 
(11.*25.71428:10pt) -- (11.5*25.71428:5pt)  -- 
(12.*25.71428:10pt) -- (12.5*25.71428:5pt)  -- 
(13.*25.71428:10pt) -- (13.5*25.71428:5pt)  --   cycle;
\draw[line width=.4pt,even odd rule,rotate=90] 
(0.5*25.71428:9pt)  -- (0.5*25.71428:5pt)   
(1.5*25.71428:9pt)  -- (1.5*25.71428:5pt)   
(2.5*25.71428:9pt)  -- (2.5*25.71428:5pt)   
(3.5*25.71428:9pt)  -- (3.5*25.71428:5pt)   
(4.5*25.71428:9pt)  -- (4.5*25.71428:5pt)   
(5.5*25.71428:9pt)  -- (5.5*25.71428:5pt)   
(6.5*25.71428:9pt)  -- (6.5*25.71428:5pt)   
(7.5*25.71428:9pt)  -- (7.5*25.71428:5pt)   
(8.5*25.71428:9pt)  -- (8.5*25.71428:5pt)   
(9.5*25.71428:9pt)  -- (9.5*25.71428:5pt)   
(10.5*25.71428:9pt) -- (10.5*25.71428:5pt)  
(11.5*25.71428:9pt) -- (11.5*25.71428:5pt)  
(12.5*25.71428:9pt) -- (12.5*25.71428:5pt)  
(13.5*25.71428:9pt) -- (13.5*25.71428:5pt)  ;
 }}}

\pgfdeclareplotmark{m1bvb}{%
\node[scale=\mOnescale*\AtoBscale*\symscale] at (0,0) {\tikz {%
\draw[fill=black,line width=.3pt,even odd rule,rotate=90] 
(0.*25.71428:9pt)  -- (0.5*25.71428:4.5pt)   -- 
(1.*25.71428:9pt)  -- (1.5*25.71428:4.5pt)   -- 
(2.*25.71428:9pt)  -- (2.5*25.71428:4.5pt)   -- 
(3.*25.71428:9pt)  -- (3.5*25.71428:4.5pt)   -- 
(4.*25.71428:9pt)  -- (4.5*25.71428:4.5pt)   -- 
(5.*25.71428:9pt)  -- (5.5*25.71428:4.5pt)   -- 
(6.*25.71428:9pt)  -- (6.5*25.71428:4.5pt)   -- 
(7.*25.71428:9pt)  -- (7.5*25.71428:4.5pt)   -- 
(8.*25.71428:9pt)  -- (8.5*25.71428:4.5pt)   -- 
(9.*25.71428:9pt)  -- (9.5*25.71428:4.5pt)   -- 
(10.*25.71428:9pt) -- (10.5*25.71428:4.5pt)  -- 
(11.*25.71428:9pt) -- (11.5*25.71428:4.5pt)  -- 
(12.*25.71428:9pt) -- (12.5*25.71428:4.5pt)  -- 
(13.*25.71428:9pt) -- (13.5*25.71428:4.5pt)  --  cycle;
\draw (0,0) circle (9pt);
%\draw[line width=.2pt,even odd rule,rotate=90] 
%(0.*25.71428:11pt)  -- (0.5*25.71428:6.5pt)   -- 
%(1.*25.71428:11pt)  -- (1.5*25.71428:6.5pt)   -- 
%(2.*25.71428:11pt)  -- (2.5*25.71428:6.5pt)   -- 
%(3.*25.71428:11pt)  -- (3.5*25.71428:6.5pt)   -- 
%(4.*25.71428:11pt)  -- (4.5*25.71428:6.5pt)   -- 
%(5.*25.71428:11pt)  -- (5.5*25.71428:6.5pt)   -- 
%(6.*25.71428:11pt)  -- (6.5*25.71428:6.5pt)   -- 
%(7.*25.71428:11pt)  -- (7.5*25.71428:6.5pt)   -- 
%(8.*25.71428:11pt)  -- (8.5*25.71428:6.5pt)   -- 
%(9.*25.71428:11pt)  -- (9.5*25.71428:6.5pt)   -- 
%(10.*25.71428:11pt) -- (10.5*25.71428:6.5pt)  -- 
%(11.*25.71428:11pt) -- (11.5*25.71428:6.5pt)  -- 
%(12.*25.71428:11pt) -- (12.5*25.71428:6.5pt)  -- 
%(13.*25.71428:11pt) -- (13.5*25.71428:6.5pt)  --  cycle;
\draw[line width=.4pt,even odd rule,rotate=90] 
(0.5*25.71428:10pt)  -- (0.5*25.71428:6.5pt)   
(1.5*25.71428:10pt)  -- (1.5*25.71428:6.5pt)   
(2.5*25.71428:10pt)  -- (2.5*25.71428:6.5pt)   
(3.5*25.71428:10pt)  -- (3.5*25.71428:6.5pt)   
(4.5*25.71428:10pt)  -- (4.5*25.71428:6.5pt)   
(5.5*25.71428:10pt)  -- (5.5*25.71428:6.5pt)   
(6.5*25.71428:10pt)  -- (6.5*25.71428:6.5pt)   
(7.5*25.71428:10pt)  -- (7.5*25.71428:6.5pt)   
(8.5*25.71428:10pt)  -- (8.5*25.71428:6.5pt)   
(9.5*25.71428:10pt)  -- (9.5*25.71428:6.5pt)   
(10.5*25.71428:10pt) -- (10.5*25.71428:6.5pt)  
(11.5*25.71428:10pt) -- (11.5*25.71428:6.5pt)  
(12.5*25.71428:10pt) -- (12.5*25.71428:6.5pt)  
(13.5*25.71428:10pt) -- (13.5*25.71428:6.5pt)  ;
 }}}



\pgfdeclareplotmark{m1c}{%
\node[scale=\mOnescale*\AtoCscale*\symscale] at (0,0) {\tikz {%
\draw[fill=black,line width=.3pt,even odd rule,rotate=90] 
(0.*25.71428:10pt)  -- (0.5*25.71428:5pt)   -- 
(1.*25.71428:10pt)  -- (1.5*25.71428:5pt)   -- 
(2.*25.71428:10pt)  -- (2.5*25.71428:5pt)   -- 
(3.*25.71428:10pt)  -- (3.5*25.71428:5pt)   -- 
(4.*25.71428:10pt)  -- (4.5*25.71428:5pt)   -- 
(5.*25.71428:10pt)  -- (5.5*25.71428:5pt)   -- 
(6.*25.71428:10pt)  -- (6.5*25.71428:5pt)   -- 
(7.*25.71428:10pt)  -- (7.5*25.71428:5pt)   -- 
(8.*25.71428:10pt)  -- (8.5*25.71428:5pt)   -- 
(9.*25.71428:10pt)  -- (9.5*25.71428:5pt)   -- 
(10.*25.71428:10pt) -- (10.5*25.71428:5pt)  -- 
(11.*25.71428:10pt) -- (11.5*25.71428:5pt)  -- 
(12.*25.71428:10pt) -- (12.5*25.71428:5pt)  -- 
(13.*25.71428:10pt) -- (13.5*25.71428:5pt)  --  cycle
(0,0) circle (2pt);
 }}}


\pgfdeclareplotmark{m1cv}{%
\node[scale=\mOnescale*\AtoCscale*\symscale] at (0,0) {\tikz {%
\draw[fill=black,line width=.3pt,even odd rule,rotate=90] 
(0.*25.71428:9pt)  -- (0.5*25.71428:4.5pt)   -- 
(1.*25.71428:9pt)  -- (1.5*25.71428:4.5pt)   -- 
(2.*25.71428:9pt)  -- (2.5*25.71428:4.5pt)   -- 
(3.*25.71428:9pt)  -- (3.5*25.71428:4.5pt)   -- 
(4.*25.71428:9pt)  -- (4.5*25.71428:4.5pt)   -- 
(5.*25.71428:9pt)  -- (5.5*25.71428:4.5pt)   -- 
(6.*25.71428:9pt)  -- (6.5*25.71428:4.5pt)   -- 
(7.*25.71428:9pt)  -- (7.5*25.71428:4.5pt)   -- 
(8.*25.71428:9pt)  -- (8.5*25.71428:4.5pt)   -- 
(9.*25.71428:9pt)  -- (9.5*25.71428:4.5pt)   -- 
(10.*25.71428:9pt) -- (10.5*25.71428:4.5pt)  -- 
(11.*25.71428:9pt) -- (11.5*25.71428:4.5pt)  -- 
(12.*25.71428:9pt) -- (12.5*25.71428:4.5pt)  -- 
(13.*25.71428:9pt) -- (13.5*25.71428:4.5pt)  -- cycle
(0,0) circle (2pt);
\draw (0,0) circle (9pt);
%\draw[line width=.2pt,even odd rule,rotate=90] 
%(0.*25.71428:11pt)  -- (0.5*25.71428:6.5pt)   -- 
%(1.*25.71428:11pt)  -- (1.5*25.71428:6.5pt)   -- 
%(2.*25.71428:11pt)  -- (2.5*25.71428:6.5pt)   -- 
%(3.*25.71428:11pt)  -- (3.5*25.71428:6.5pt)   -- 
%(4.*25.71428:11pt)  -- (4.5*25.71428:6.5pt)   -- 
%(5.*25.71428:11pt)  -- (5.5*25.71428:6.5pt)   -- 
%(6.*25.71428:11pt)  -- (6.5*25.71428:6.5pt)   -- 
%(7.*25.71428:11pt)  -- (7.5*25.71428:6.5pt)   -- 
%(8.*25.71428:11pt)  -- (8.5*25.71428:6.5pt)   -- 
%(9.*25.71428:11pt)  -- (9.5*25.71428:6.5pt)   -- 
%(10.*25.71428:11pt) -- (10.5*25.71428:6.5pt)  -- 
%(11.*25.71428:11pt) -- (11.5*25.71428:6.5pt)  -- 
%(12.*25.71428:11pt) -- (12.5*25.71428:6.5pt)  -- 
%(13.*25.71428:11pt) -- (13.5*25.71428:6.5pt)  --  cycle;
 }}}



\pgfdeclareplotmark{m1cb}{%
\node[scale=\mOnescale*\AtoCscale*\symscale] at (0,0) {\tikz {%
\draw[fill=black,line width=.3pt,even odd rule,rotate=90] 
(0.*25.71428:10pt)  -- (0.5*25.71428:5pt)   -- 
(1.*25.71428:10pt)  -- (1.5*25.71428:5pt)   -- 
(2.*25.71428:10pt)  -- (2.5*25.71428:5pt)   -- 
(3.*25.71428:10pt)  -- (3.5*25.71428:5pt)   -- 
(4.*25.71428:10pt)  -- (4.5*25.71428:5pt)   -- 
(5.*25.71428:10pt)  -- (5.5*25.71428:5pt)   -- 
(6.*25.71428:10pt)  -- (6.5*25.71428:5pt)   -- 
(7.*25.71428:10pt)  -- (7.5*25.71428:5pt)   -- 
(8.*25.71428:10pt)  -- (8.5*25.71428:5pt)   -- 
(9.*25.71428:10pt)  -- (9.5*25.71428:5pt)   -- 
(10.*25.71428:10pt) -- (10.5*25.71428:5pt)  -- 
(11.*25.71428:10pt) -- (11.5*25.71428:5pt)  -- 
(12.*25.71428:10pt) -- (12.5*25.71428:5pt)  -- 
(13.*25.71428:10pt) -- (13.5*25.71428:5pt)  --  cycle
(0,0) circle (2pt);
\draw[line width=.4pt,even odd rule,rotate=90] 
(0.5*25.71428:9pt)  -- (0.5*25.71428:5pt)   
(1.5*25.71428:9pt)  -- (1.5*25.71428:5pt)   
(2.5*25.71428:9pt)  -- (2.5*25.71428:5pt)   
(3.5*25.71428:9pt)  -- (3.5*25.71428:5pt)   
(4.5*25.71428:9pt)  -- (4.5*25.71428:5pt)   
(5.5*25.71428:9pt)  -- (5.5*25.71428:5pt)   
(6.5*25.71428:9pt)  -- (6.5*25.71428:5pt)   
(7.5*25.71428:9pt)  -- (7.5*25.71428:5pt)   
(8.5*25.71428:9pt)  -- (8.5*25.71428:5pt)   
(9.5*25.71428:9pt)  -- (9.5*25.71428:5pt)   
(10.5*25.71428:9pt) -- (10.5*25.71428:5pt)  
(11.5*25.71428:9pt) -- (11.5*25.71428:5pt)  
(12.5*25.71428:9pt) -- (12.5*25.71428:5pt)  
(13.5*25.71428:9pt) -- (13.5*25.71428:5pt)  ;
 }}}

\pgfdeclareplotmark{m1cvb}{%
\node[scale=\mOnescale*\AtoCscale*\symscale] at (0,0) {\tikz {%
\draw[fill=black,line width=.3pt,even odd rule,rotate=90] 
(0.*25.71428:9pt)  -- (0.5*25.71428:4.5pt)   -- 
(1.*25.71428:9pt)  -- (1.5*25.71428:4.5pt)   -- 
(2.*25.71428:9pt)  -- (2.5*25.71428:4.5pt)   -- 
(3.*25.71428:9pt)  -- (3.5*25.71428:4.5pt)   -- 
(4.*25.71428:9pt)  -- (4.5*25.71428:4.5pt)   -- 
(5.*25.71428:9pt)  -- (5.5*25.71428:4.5pt)   -- 
(6.*25.71428:9pt)  -- (6.5*25.71428:4.5pt)   -- 
(7.*25.71428:9pt)  -- (7.5*25.71428:4.5pt)   -- 
(8.*25.71428:9pt)  -- (8.5*25.71428:4.5pt)   -- 
(9.*25.71428:9pt)  -- (9.5*25.71428:4.5pt)   -- 
(10.*25.71428:9pt) -- (10.5*25.71428:4.5pt)  -- 
(11.*25.71428:9pt) -- (11.5*25.71428:4.5pt)  -- 
(12.*25.71428:9pt) -- (12.5*25.71428:4.5pt)  -- 
(13.*25.71428:9pt) -- (13.5*25.71428:4.5pt)  --  cycle
(0,0) circle (2pt);
\draw (0,0) circle (9pt);
%\draw[line width=.2pt,even odd rule,rotate=90] 
%(0.*25.71428:11pt)  -- (0.5*25.71428:6.5pt)   -- 
%(1.*25.71428:11pt)  -- (1.5*25.71428:6.5pt)   -- 
%(2.*25.71428:11pt)  -- (2.5*25.71428:6.5pt)   -- 
%(3.*25.71428:11pt)  -- (3.5*25.71428:6.5pt)   -- 
%(4.*25.71428:11pt)  -- (4.5*25.71428:6.5pt)   -- 
%(5.*25.71428:11pt)  -- (5.5*25.71428:6.5pt)   -- 
%(6.*25.71428:11pt)  -- (6.5*25.71428:6.5pt)   -- 
%(7.*25.71428:11pt)  -- (7.5*25.71428:6.5pt)   -- 
%(8.*25.71428:11pt)  -- (8.5*25.71428:6.5pt)   -- 
%(9.*25.71428:11pt)  -- (9.5*25.71428:6.5pt)   -- 
%(10.*25.71428:11pt) -- (10.5*25.71428:6.5pt)  -- 
%(11.*25.71428:11pt) -- (11.5*25.71428:6.5pt)  -- 
%(12.*25.71428:11pt) -- (12.5*25.71428:6.5pt)  -- 
%(13.*25.71428:11pt) -- (13.5*25.71428:6.5pt)  --  cycle;
\draw[line width=.4pt,even odd rule,rotate=90] 
(0.5*25.71428:10pt)  -- (0.5*25.71428:6.5pt)   
(1.5*25.71428:10pt)  -- (1.5*25.71428:6.5pt)   
(2.5*25.71428:10pt)  -- (2.5*25.71428:6.5pt)   
(3.5*25.71428:10pt)  -- (3.5*25.71428:6.5pt)   
(4.5*25.71428:10pt)  -- (4.5*25.71428:6.5pt)   
(5.5*25.71428:10pt)  -- (5.5*25.71428:6.5pt)   
(6.5*25.71428:10pt)  -- (6.5*25.71428:6.5pt)   
(7.5*25.71428:10pt)  -- (7.5*25.71428:6.5pt)   
(8.5*25.71428:10pt)  -- (8.5*25.71428:6.5pt)   
(9.5*25.71428:10pt)  -- (9.5*25.71428:6.5pt)   
(10.5*25.71428:10pt) -- (10.5*25.71428:6.5pt)  
(11.5*25.71428:10pt) -- (11.5*25.71428:6.5pt)  
(12.5*25.71428:10pt) -- (12.5*25.71428:6.5pt)  
(13.5*25.71428:10pt) -- (13.5*25.71428:6.5pt)  ;
 }}}



\pgfdeclareplotmark{m2a}{%
\node[scale=\mTwoscale*\symscale] at (0,0) {\tikz {%
\draw[fill=black,line width=.3pt,even odd rule,rotate=90] 
(0.*30.0:10pt)  -- (0.5*30.0:5pt)   -- 
(1.*30.0:10pt)  -- (1.5*30.0:5pt)   -- 
(2.*30.0:10pt)  -- (2.5*30.0:5pt)   -- 
(3.*30.0:10pt)  -- (3.5*30.0:5pt)   -- 
(4.*30.0:10pt)  -- (4.5*30.0:5pt)   -- 
(5.*30.0:10pt)  -- (5.5*30.0:5pt)   -- 
(6.*30.0:10pt)  -- (6.5*30.0:5pt)   -- 
(7.*30.0:10pt)  -- (7.5*30.0:5pt)   -- 
(8.*30.0:10pt)  -- (8.5*30.0:5pt)   -- 
(9.*30.0:10pt)  -- (9.5*30.0:5pt)   -- 
(10.*30.0:10pt) -- (10.5*30.0:5pt)  -- 
(11.*30.0:10pt) -- (11.5*30.0:5pt)  --  cycle;
 }}}


\pgfdeclareplotmark{m2av}{%
\node[scale=\mTwoscale*\symscale] at (0,0) {\tikz {%
\draw[fill=black,line width=.3pt,even odd rule,rotate=90] 
(0.*30.0:9pt)  -- (0.5*30.0:4.5pt)   -- 
(1.*30.0:9pt)  -- (1.5*30.0:4.5pt)   -- 
(2.*30.0:9pt)  -- (2.5*30.0:4.5pt)   -- 
(3.*30.0:9pt)  -- (3.5*30.0:4.5pt)   -- 
(4.*30.0:9pt)  -- (4.5*30.0:4.5pt)   -- 
(5.*30.0:9pt)  -- (5.5*30.0:4.5pt)   -- 
(6.*30.0:9pt)  -- (6.5*30.0:4.5pt)   -- 
(7.*30.0:9pt)  -- (7.5*30.0:4.5pt)   -- 
(8.*30.0:9pt)  -- (8.5*30.0:4.5pt)   -- 
(9.*30.0:9pt)  -- (9.5*30.0:4.5pt)   -- 
(10.*30.0:9pt) -- (10.5*30.0:4.5pt)  -- 
(11.*30.0:9pt) -- (11.5*30.0:4.5pt)  -- cycle;
\draw (0,0) circle (9pt);
%\draw[line width=.2pt,even odd rule,rotate=90] 
%(0.*30.0:11pt)  -- (0.5*30.0:6.5pt)   -- 
%(1.*30.0:11pt)  -- (1.5*30.0:6.5pt)   -- 
%(2.*30.0:11pt)  -- (2.5*30.0:6.5pt)   -- 
%(3.*30.0:11pt)  -- (3.5*30.0:6.5pt)   -- 
%(4.*30.0:11pt)  -- (4.5*30.0:6.5pt)   -- 
%(5.*30.0:11pt)  -- (5.5*30.0:6.5pt)   -- 
%(6.*30.0:11pt)  -- (6.5*30.0:6.5pt)   -- 
%(7.*30.0:11pt)  -- (7.5*30.0:6.5pt)   -- 
%(8.*30.0:11pt)  -- (8.5*30.0:6.5pt)   -- 
%(9.*30.0:11pt)  -- (9.5*30.0:6.5pt)   -- 
%(10.*30.0:11pt) -- (10.5*30.0:6.5pt)  -- 
%(11.*30.0:11pt) -- (11.5*30.0:6.5pt)  --  cycle;
 }}}



\pgfdeclareplotmark{m2ab}{%
\node[scale=\mTwoscale*\symscale] at (0,0) {\tikz {%
\draw[fill=black,line width=.3pt,even odd rule,rotate=90] 
(0.*30.0:10pt)  -- (0.5*30.0:5pt)   -- 
(1.*30.0:10pt)  -- (1.5*30.0:5pt)   -- 
(2.*30.0:10pt)  -- (2.5*30.0:5pt)   -- 
(3.*30.0:10pt)  -- (3.5*30.0:5pt)   -- 
(4.*30.0:10pt)  -- (4.5*30.0:5pt)   -- 
(5.*30.0:10pt)  -- (5.5*30.0:5pt)   -- 
(6.*30.0:10pt)  -- (6.5*30.0:5pt)   -- 
(7.*30.0:10pt)  -- (7.5*30.0:5pt)   -- 
(8.*30.0:10pt)  -- (8.5*30.0:5pt)   -- 
(9.*30.0:10pt)  -- (9.5*30.0:5pt)   -- 
(10.*30.0:10pt) -- (10.5*30.0:5pt)  -- 
(11.*30.0:10pt) -- (11.5*30.0:5pt)  --  cycle;
\draw[line width=.4pt,even odd rule,rotate=90] 
(0.5*30.0:9pt)  -- (0.5*30.0:5pt)   
(1.5*30.0:9pt)  -- (1.5*30.0:5pt)   
(2.5*30.0:9pt)  -- (2.5*30.0:5pt)   
(3.5*30.0:9pt)  -- (3.5*30.0:5pt)   
(4.5*30.0:9pt)  -- (4.5*30.0:5pt)   
(5.5*30.0:9pt)  -- (5.5*30.0:5pt)   
(6.5*30.0:9pt)  -- (6.5*30.0:5pt)   
(7.5*30.0:9pt)  -- (7.5*30.0:5pt)   
(8.5*30.0:9pt)  -- (8.5*30.0:5pt)   
(9.5*30.0:9pt)  -- (9.5*30.0:5pt)   
(10.5*30.0:9pt) -- (10.5*30.0:5pt)  
(11.5*30.0:9pt) -- (11.5*30.0:5pt) ;
 }}}

\pgfdeclareplotmark{m2avb}{%
\node[scale=\mTwoscale*\symscale] at (0,0) {\tikz {%
\draw[fill=black,line width=.3pt,even odd rule,rotate=90] 
(0.*30.0:9pt)  -- (0.5*30.0:4.5pt)   -- 
(1.*30.0:9pt)  -- (1.5*30.0:4.5pt)   -- 
(2.*30.0:9pt)  -- (2.5*30.0:4.5pt)   -- 
(3.*30.0:9pt)  -- (3.5*30.0:4.5pt)   -- 
(4.*30.0:9pt)  -- (4.5*30.0:4.5pt)   -- 
(5.*30.0:9pt)  -- (5.5*30.0:4.5pt)   -- 
(6.*30.0:9pt)  -- (6.5*30.0:4.5pt)   -- 
(7.*30.0:9pt)  -- (7.5*30.0:4.5pt)   -- 
(8.*30.0:9pt)  -- (8.5*30.0:4.5pt)   -- 
(9.*30.0:9pt)  -- (9.5*30.0:4.5pt)   -- 
(10.*30.0:9pt) -- (10.5*30.0:4.5pt)  -- 
(11.*30.0:9pt) -- (11.5*30.0:4.5pt)  --  cycle;
\draw (0,0) circle (9pt);
%\draw[line width=.2pt,even odd rule,rotate=90] 
%(0.*30.0:11pt)  -- (0.5*30.0:6.5pt)   -- 
%(1.*30.0:11pt)  -- (1.5*30.0:6.5pt)   -- 
%(2.*30.0:11pt)  -- (2.5*30.0:6.5pt)   -- 
%(3.*30.0:11pt)  -- (3.5*30.0:6.5pt)   -- 
%(4.*30.0:11pt)  -- (4.5*30.0:6.5pt)   -- 
%(5.*30.0:11pt)  -- (5.5*30.0:6.5pt)   -- 
%(6.*30.0:11pt)  -- (6.5*30.0:6.5pt)   -- 
%(7.*30.0:11pt)  -- (7.5*30.0:6.5pt)   -- 
%(8.*30.0:11pt)  -- (8.5*30.0:6.5pt)   -- 
%(9.*30.0:11pt)  -- (9.5*30.0:6.5pt)   -- 
%(10.*30.0:11pt) -- (10.5*30.0:6.5pt)  -- 
%(11.*30.0:11pt) -- (11.5*30.0:6.5pt)  -- cycle;
\draw[line width=.4pt,even odd rule,rotate=90] 
(0.5*30.0:10pt)  -- (0.5*30.0:6.5pt)   
(1.5*30.0:10pt)  -- (1.5*30.0:6.5pt)   
(2.5*30.0:10pt)  -- (2.5*30.0:6.5pt)   
(3.5*30.0:10pt)  -- (3.5*30.0:6.5pt)   
(4.5*30.0:10pt)  -- (4.5*30.0:6.5pt)   
(5.5*30.0:10pt)  -- (5.5*30.0:6.5pt)   
(6.5*30.0:10pt)  -- (6.5*30.0:6.5pt)   
(7.5*30.0:10pt)  -- (7.5*30.0:6.5pt)   
(8.5*30.0:10pt)  -- (8.5*30.0:6.5pt)   
(9.5*30.0:10pt)  -- (9.5*30.0:6.5pt)   
(10.5*30.0:10pt) -- (10.5*30.0:6.5pt)  
(11.5*30.0:10pt) -- (11.5*30.0:6.5pt)  ;
 }}}



\pgfdeclareplotmark{m2b}{%
\node[scale=\mTwoscale*\AtoBscale*\symscale] at (0,0) {\tikz {%
\draw[fill=black,line width=.3pt,even odd rule,rotate=90] 
(0.*30.0:10pt)  -- (0.5*30.0:5pt)   -- 
(1.*30.0:10pt)  -- (1.5*30.0:5pt)   -- 
(2.*30.0:10pt)  -- (2.5*30.0:5pt)   -- 
(3.*30.0:10pt)  -- (3.5*30.0:5pt)   -- 
(4.*30.0:10pt)  -- (4.5*30.0:5pt)   -- 
(5.*30.0:10pt)  -- (5.5*30.0:5pt)   -- 
(6.*30.0:10pt)  -- (6.5*30.0:5pt)   -- 
(7.*30.0:10pt)  -- (7.5*30.0:5pt)   -- 
(8.*30.0:10pt)  -- (8.5*30.0:5pt)   -- 
(9.*30.0:10pt)  -- (9.5*30.0:5pt)   -- 
(10.*30.0:10pt) -- (10.5*30.0:5pt)  -- 
(11.*30.0:10pt) -- (11.5*30.0:5pt)  -- cycle
(0,0) circle (2pt);
 }}}


\pgfdeclareplotmark{m2bv}{%
\node[scale=\mTwoscale*\AtoBscale*\symscale] at (0,0) {\tikz {%
\draw[fill=black,line width=.3pt,even odd rule,rotate=90] 
(0.*30.0:9pt)  -- (0.5*30.0:4.5pt)   -- 
(1.*30.0:9pt)  -- (1.5*30.0:4.5pt)   -- 
(2.*30.0:9pt)  -- (2.5*30.0:4.5pt)   -- 
(3.*30.0:9pt)  -- (3.5*30.0:4.5pt)   -- 
(4.*30.0:9pt)  -- (4.5*30.0:4.5pt)   -- 
(5.*30.0:9pt)  -- (5.5*30.0:4.5pt)   -- 
(6.*30.0:9pt)  -- (6.5*30.0:4.5pt)   -- 
(7.*30.0:9pt)  -- (7.5*30.0:4.5pt)   -- 
(8.*30.0:9pt)  -- (8.5*30.0:4.5pt)   -- 
(9.*30.0:9pt)  -- (9.5*30.0:4.5pt)   -- 
(10.*30.0:9pt) -- (10.5*30.0:4.5pt)  -- 
(11.*30.0:9pt) -- (11.5*30.0:4.5pt)  --  cycle
(0,0) circle (2pt);
\draw (0,0) circle (9pt);
%\draw[line width=.2pt,even odd rule,rotate=90] 
%(0.*30.0:11pt)  -- (0.5*30.0:6.5pt)   -- 
%(1.*30.0:11pt)  -- (1.5*30.0:6.5pt)   -- 
%(2.*30.0:11pt)  -- (2.5*30.0:6.5pt)   -- 
%(3.*30.0:11pt)  -- (3.5*30.0:6.5pt)   -- 
%(4.*30.0:11pt)  -- (4.5*30.0:6.5pt)   -- 
%(5.*30.0:11pt)  -- (5.5*30.0:6.5pt)   -- 
%(6.*30.0:11pt)  -- (6.5*30.0:6.5pt)   -- 
%(7.*30.0:11pt)  -- (7.5*30.0:6.5pt)   -- 
%(8.*30.0:11pt)  -- (8.5*30.0:6.5pt)   -- 
%(9.*30.0:11pt)  -- (9.5*30.0:6.5pt)   -- 
%(10.*30.0:11pt) -- (10.5*30.0:6.5pt)  -- 
%(11.*30.0:11pt) -- (11.5*30.0:6.5pt)  --  cycle;
 }}}



\pgfdeclareplotmark{m2bb}{%
\node[scale=\mTwoscale*\AtoBscale*\symscale] at (0,0) {\tikz {%
\draw[fill=black,line width=.3pt,even odd rule,rotate=90] 
(0.*30.0:10pt)  -- (0.5*30.0:5pt)   -- 
(1.*30.0:10pt)  -- (1.5*30.0:5pt)   -- 
(2.*30.0:10pt)  -- (2.5*30.0:5pt)   -- 
(3.*30.0:10pt)  -- (3.5*30.0:5pt)   -- 
(4.*30.0:10pt)  -- (4.5*30.0:5pt)   -- 
(5.*30.0:10pt)  -- (5.5*30.0:5pt)   -- 
(6.*30.0:10pt)  -- (6.5*30.0:5pt)   -- 
(7.*30.0:10pt)  -- (7.5*30.0:5pt)   -- 
(8.*30.0:10pt)  -- (8.5*30.0:5pt)   -- 
(9.*30.0:10pt)  -- (9.5*30.0:5pt)   -- 
(10.*30.0:10pt) -- (10.5*30.0:5pt)  -- 
(11.*30.0:10pt) -- (11.5*30.0:5pt)  --  cycle
(0,0) circle (2pt);
\draw[line width=.4pt,even odd rule,rotate=90] 
(0.5*30.0:9pt)  -- (0.5*30.0:5pt)   
(1.5*30.0:9pt)  -- (1.5*30.0:5pt)   
(2.5*30.0:9pt)  -- (2.5*30.0:5pt)   
(3.5*30.0:9pt)  -- (3.5*30.0:5pt)   
(4.5*30.0:9pt)  -- (4.5*30.0:5pt)   
(5.5*30.0:9pt)  -- (5.5*30.0:5pt)   
(6.5*30.0:9pt)  -- (6.5*30.0:5pt)   
(7.5*30.0:9pt)  -- (7.5*30.0:5pt)   
(8.5*30.0:9pt)  -- (8.5*30.0:5pt)   
(9.5*30.0:9pt)  -- (9.5*30.0:5pt)   
(10.5*30.0:9pt) -- (10.5*30.0:5pt)  
(11.5*30.0:9pt) -- (11.5*30.0:5pt)  ;
 }}}

\pgfdeclareplotmark{m2bvb}{%
\node[scale=\mTwoscale*\AtoBscale*\symscale] at (0,0) {\tikz {%
\draw[fill=black,line width=.3pt,even odd rule,rotate=90] 
(0.*30.0:9pt)  -- (0.5*30.0:4.5pt)   -- 
(1.*30.0:9pt)  -- (1.5*30.0:4.5pt)   -- 
(2.*30.0:9pt)  -- (2.5*30.0:4.5pt)   -- 
(3.*30.0:9pt)  -- (3.5*30.0:4.5pt)   -- 
(4.*30.0:9pt)  -- (4.5*30.0:4.5pt)   -- 
(5.*30.0:9pt)  -- (5.5*30.0:4.5pt)   -- 
(6.*30.0:9pt)  -- (6.5*30.0:4.5pt)   -- 
(7.*30.0:9pt)  -- (7.5*30.0:4.5pt)   -- 
(8.*30.0:9pt)  -- (8.5*30.0:4.5pt)   -- 
(9.*30.0:9pt)  -- (9.5*30.0:4.5pt)   -- 
(10.*30.0:9pt) -- (10.5*30.0:4.5pt)  -- 
(11.*30.0:9pt) -- (11.5*30.0:4.5pt)  --  cycle
(0,0) circle (2pt);
\draw (0,0) circle (9pt);
%\draw[line width=.2pt,even odd rule,rotate=90] 
%(0.*30.0:11pt)  -- (0.5*30.0:6.5pt)   -- 
%(1.*30.0:11pt)  -- (1.5*30.0:6.5pt)   -- 
%(2.*30.0:11pt)  -- (2.5*30.0:6.5pt)   -- 
%(3.*30.0:11pt)  -- (3.5*30.0:6.5pt)   -- 
%(4.*30.0:11pt)  -- (4.5*30.0:6.5pt)   -- 
%(5.*30.0:11pt)  -- (5.5*30.0:6.5pt)   -- 
%(6.*30.0:11pt)  -- (6.5*30.0:6.5pt)   -- 
%(7.*30.0:11pt)  -- (7.5*30.0:6.5pt)   -- 
%(8.*30.0:11pt)  -- (8.5*30.0:6.5pt)   -- 
%(9.*30.0:11pt)  -- (9.5*30.0:6.5pt)   -- 
%(10.*30.0:11pt) -- (10.5*30.0:6.5pt)  -- 
%(11.*30.0:11pt) -- (11.5*30.0:6.5pt)  --  cycle;
\draw[line width=.4pt,even odd rule,rotate=90] 
(0.5*30.0:10pt)  -- (0.5*30.0:6.5pt)   
(1.5*30.0:10pt)  -- (1.5*30.0:6.5pt)   
(2.5*30.0:10pt)  -- (2.5*30.0:6.5pt)   
(3.5*30.0:10pt)  -- (3.5*30.0:6.5pt)   
(4.5*30.0:10pt)  -- (4.5*30.0:6.5pt)   
(5.5*30.0:10pt)  -- (5.5*30.0:6.5pt)   
(6.5*30.0:10pt)  -- (6.5*30.0:6.5pt)   
(7.5*30.0:10pt)  -- (7.5*30.0:6.5pt)   
(8.5*30.0:10pt)  -- (8.5*30.0:6.5pt)   
(9.5*30.0:10pt)  -- (9.5*30.0:6.5pt)   
(10.5*30.0:10pt) -- (10.5*30.0:6.5pt)  
(11.5*30.0:10pt) -- (11.5*30.0:6.5pt)  ;
 }}}



\pgfdeclareplotmark{m2c}{%
\node[scale=\mTwoscale*\AtoCscale*\symscale] at (0,0) {\tikz {%
\draw[fill=black,line width=.3pt,even odd rule,rotate=90] 
(0.*30.0:10pt)  -- (0.5*30.0:5pt)   -- 
(1.*30.0:10pt)  -- (1.5*30.0:5pt)   -- 
(2.*30.0:10pt)  -- (2.5*30.0:5pt)   -- 
(3.*30.0:10pt)  -- (3.5*30.0:5pt)   -- 
(4.*30.0:10pt)  -- (4.5*30.0:5pt)   -- 
(5.*30.0:10pt)  -- (5.5*30.0:5pt)   -- 
(6.*30.0:10pt)  -- (6.5*30.0:5pt)   -- 
(7.*30.0:10pt)  -- (7.5*30.0:5pt)   -- 
(8.*30.0:10pt)  -- (8.5*30.0:5pt)   -- 
(9.*30.0:10pt)  -- (9.5*30.0:5pt)   -- 
(10.*30.0:10pt) -- (10.5*30.0:5pt)  -- 
(11.*30.0:10pt) -- (11.5*30.0:5pt)  --cycle
(0,0) circle (3pt);
 }}}


\pgfdeclareplotmark{m2cv}{%
\node[scale=\mTwoscale*\AtoCscale*\symscale] at (0,0) {\tikz {%
\draw[fill=black,line width=.3pt,even odd rule,rotate=90] 
(0.*30.0:9pt)  -- (0.5*30.0:4.5pt)   -- 
(1.*30.0:9pt)  -- (1.5*30.0:4.5pt)   -- 
(2.*30.0:9pt)  -- (2.5*30.0:4.5pt)   -- 
(3.*30.0:9pt)  -- (3.5*30.0:4.5pt)   -- 
(4.*30.0:9pt)  -- (4.5*30.0:4.5pt)   -- 
(5.*30.0:9pt)  -- (5.5*30.0:4.5pt)   -- 
(6.*30.0:9pt)  -- (6.5*30.0:4.5pt)   -- 
(7.*30.0:9pt)  -- (7.5*30.0:4.5pt)   -- 
(8.*30.0:9pt)  -- (8.5*30.0:4.5pt)   -- 
(9.*30.0:9pt)  -- (9.5*30.0:4.5pt)   -- 
(10.*30.0:9pt) -- (10.5*30.0:4.5pt)  -- 
(11.*30.0:9pt) -- (11.5*30.0:4.5pt)  --  cycle
(0,0) circle (3pt);
\draw (0,0) circle (9pt);
%\draw[line width=.2pt,even odd rule,rotate=90] 
%(0.*30.0:11pt)  -- (0.5*30.0:6.5pt)   -- 
%(1.*30.0:11pt)  -- (1.5*30.0:6.5pt)   -- 
%(2.*30.0:11pt)  -- (2.5*30.0:6.5pt)   -- 
%(3.*30.0:11pt)  -- (3.5*30.0:6.5pt)   -- 
%(4.*30.0:11pt)  -- (4.5*30.0:6.5pt)   -- 
%(5.*30.0:11pt)  -- (5.5*30.0:6.5pt)   -- 
%(6.*30.0:11pt)  -- (6.5*30.0:6.5pt)   -- 
%(7.*30.0:11pt)  -- (7.5*30.0:6.5pt)   -- 
%(8.*30.0:11pt)  -- (8.5*30.0:6.5pt)   -- 
%(9.*30.0:11pt)  -- (9.5*30.0:6.5pt)   -- 
%(10.*30.0:11pt) -- (10.5*30.0:6.5pt)  -- 
%(11.*30.0:11pt) -- (11.5*30.0:6.5pt)  --  cycle;
 }}}



\pgfdeclareplotmark{m2cb}{%
\node[scale=\mTwoscale*\AtoCscale*\symscale] at (0,0) {\tikz {%
\draw[fill=black,line width=.3pt,even odd rule,rotate=90] 
(0.*30.0:10pt)  -- (0.5*30.0:5pt)   -- 
(1.*30.0:10pt)  -- (1.5*30.0:5pt)   -- 
(2.*30.0:10pt)  -- (2.5*30.0:5pt)   -- 
(3.*30.0:10pt)  -- (3.5*30.0:5pt)   -- 
(4.*30.0:10pt)  -- (4.5*30.0:5pt)   -- 
(5.*30.0:10pt)  -- (5.5*30.0:5pt)   -- 
(6.*30.0:10pt)  -- (6.5*30.0:5pt)   -- 
(7.*30.0:10pt)  -- (7.5*30.0:5pt)   -- 
(8.*30.0:10pt)  -- (8.5*30.0:5pt)   -- 
(9.*30.0:10pt)  -- (9.5*30.0:5pt)   -- 
(10.*30.0:10pt) -- (10.5*30.0:5pt)  -- 
(11.*30.0:10pt) -- (11.5*30.0:5pt)  --  cycle
(0,0) circle (3pt);
\draw[line width=.4pt,even odd rule,rotate=90] 
(0.5*30.0:9pt)  -- (0.5*30.0:5pt)   
(1.5*30.0:9pt)  -- (1.5*30.0:5pt)   
(2.5*30.0:9pt)  -- (2.5*30.0:5pt)   
(3.5*30.0:9pt)  -- (3.5*30.0:5pt)   
(4.5*30.0:9pt)  -- (4.5*30.0:5pt)   
(5.5*30.0:9pt)  -- (5.5*30.0:5pt)   
(6.5*30.0:9pt)  -- (6.5*30.0:5pt)   
(7.5*30.0:9pt)  -- (7.5*30.0:5pt)   
(8.5*30.0:9pt)  -- (8.5*30.0:5pt)   
(9.5*30.0:9pt)  -- (9.5*30.0:5pt)   
(10.5*30.0:9pt) -- (10.5*30.0:5pt)  
(11.5*30.0:9pt) -- (11.5*30.0:5pt)  ;
 }}}

\pgfdeclareplotmark{m2cvb}{%
\node[scale=\mTwoscale*\AtoCscale*\symscale] at (0,0) {\tikz {%
\draw[fill=black,line width=.3pt,even odd rule,rotate=90] 
(0.*30.0:9pt)  -- (0.5*30.0:4.5pt)   -- 
(1.*30.0:9pt)  -- (1.5*30.0:4.5pt)   -- 
(2.*30.0:9pt)  -- (2.5*30.0:4.5pt)   -- 
(3.*30.0:9pt)  -- (3.5*30.0:4.5pt)   -- 
(4.*30.0:9pt)  -- (4.5*30.0:4.5pt)   -- 
(5.*30.0:9pt)  -- (5.5*30.0:4.5pt)   -- 
(6.*30.0:9pt)  -- (6.5*30.0:4.5pt)   -- 
(7.*30.0:9pt)  -- (7.5*30.0:4.5pt)   -- 
(8.*30.0:9pt)  -- (8.5*30.0:4.5pt)   -- 
(9.*30.0:9pt)  -- (9.5*30.0:4.5pt)   -- 
(10.*30.0:9pt) -- (10.5*30.0:4.5pt)  -- 
(11.*30.0:9pt) -- (11.5*30.0:4.5pt)  --  cycle
(0,0) circle (3pt);
\draw (0,0) circle (9pt);
%\draw[line width=.2pt,even odd rule,rotate=90] 
%(0.*30.0:11pt)  -- (0.5*30.0:6.5pt)   -- 
%(1.*30.0:11pt)  -- (1.5*30.0:6.5pt)   -- 
%(2.*30.0:11pt)  -- (2.5*30.0:6.5pt)   -- 
%(3.*30.0:11pt)  -- (3.5*30.0:6.5pt)   -- 
%(4.*30.0:11pt)  -- (4.5*30.0:6.5pt)   -- 
%(5.*30.0:11pt)  -- (5.5*30.0:6.5pt)   -- 
%(6.*30.0:11pt)  -- (6.5*30.0:6.5pt)   -- 
%(7.*30.0:11pt)  -- (7.5*30.0:6.5pt)   -- 
%(8.*30.0:11pt)  -- (8.5*30.0:6.5pt)   -- 
%(9.*30.0:11pt)  -- (9.5*30.0:6.5pt)   -- 
%(10.*30.0:11pt) -- (10.5*30.0:6.5pt)  -- 
%(11.*30.0:11pt) -- (11.5*30.0:6.5pt)  --  cycle;
\draw[line width=.4pt,even odd rule,rotate=90] 
(0.5*30.0:10pt)  -- (0.5*30.0:6.5pt)   
(1.5*30.0:10pt)  -- (1.5*30.0:6.5pt)   
(2.5*30.0:10pt)  -- (2.5*30.0:6.5pt)   
(3.5*30.0:10pt)  -- (3.5*30.0:6.5pt)   
(4.5*30.0:10pt)  -- (4.5*30.0:6.5pt)   
(5.5*30.0:10pt)  -- (5.5*30.0:6.5pt)   
(6.5*30.0:10pt)  -- (6.5*30.0:6.5pt)   
(7.5*30.0:10pt)  -- (7.5*30.0:6.5pt)   
(8.5*30.0:10pt)  -- (8.5*30.0:6.5pt)   
(9.5*30.0:10pt)  -- (9.5*30.0:6.5pt)   
(10.5*30.0:10pt) -- (10.5*30.0:6.5pt)  
(11.5*30.0:10pt) -- (11.5*30.0:6.5pt)  ;
 }}}



\pgfdeclareplotmark{m3a}{%
\node[scale=\mThreescale*\symscale] at (0,0) {\tikz {%
\draw[fill=black,line width=.3pt,even odd rule,rotate=90] 
(0.*36.0:10pt)  -- (0.5*36.0:5pt)   -- 
(1.*36.0:10pt)  -- (1.5*36.0:5pt)   -- 
(2.*36.0:10pt)  -- (2.5*36.0:5pt)   -- 
(3.*36.0:10pt)  -- (3.5*36.0:5pt)   -- 
(4.*36.0:10pt)  -- (4.5*36.0:5pt)   -- 
(5.*36.0:10pt)  -- (5.5*36.0:5pt)   -- 
(6.*36.0:10pt)  -- (6.5*36.0:5pt)   -- 
(7.*36.0:10pt)  -- (7.5*36.0:5pt)   -- 
(8.*36.0:10pt)  -- (8.5*36.0:5pt)   -- 
(9.*36.0:10pt)  -- (9.5*36.0:5pt)   --  cycle;
 }}}


\pgfdeclareplotmark{m3av}{%
\node[scale=\mThreescale*\symscale] at (0,0) {\tikz {%
\draw[fill=black,line width=.3pt,even odd rule,rotate=90] 
(0.*36.0:9pt)  -- (0.5*36.0:4.5pt)   -- 
(1.*36.0:9pt)  -- (1.5*36.0:4.5pt)   -- 
(2.*36.0:9pt)  -- (2.5*36.0:4.5pt)   -- 
(3.*36.0:9pt)  -- (3.5*36.0:4.5pt)   -- 
(4.*36.0:9pt)  -- (4.5*36.0:4.5pt)   -- 
(5.*36.0:9pt)  -- (5.5*36.0:4.5pt)   -- 
(6.*36.0:9pt)  -- (6.5*36.0:4.5pt)   -- 
(7.*36.0:9pt)  -- (7.5*36.0:4.5pt)   -- 
(8.*36.0:9pt)  -- (8.5*36.0:4.5pt)   -- 
(9.*36.0:9pt)  -- (9.5*36.0:4.5pt)   -- cycle;
\draw (0,0) circle (9pt);
%\draw[line width=.2pt,even odd rule,rotate=90] 
%(0.*36.0:11pt)  -- (0.5*36.0:6.5pt)   -- 
%(1.*36.0:11pt)  -- (1.5*36.0:6.5pt)   -- 
%(2.*36.0:11pt)  -- (2.5*36.0:6.5pt)   -- 
%(3.*36.0:11pt)  -- (3.5*36.0:6.5pt)   -- 
%(4.*36.0:11pt)  -- (4.5*36.0:6.5pt)   -- 
%(5.*36.0:11pt)  -- (5.5*36.0:6.5pt)   -- 
%(6.*36.0:11pt)  -- (6.5*36.0:6.5pt)   -- 
%(7.*36.0:11pt)  -- (7.5*36.0:6.5pt)   -- 
%(8.*36.0:11pt)  -- (8.5*36.0:6.5pt)   -- 
%(9.*36.0:11pt)  -- (9.5*36.0:6.5pt)   --  cycle;
 }}}



\pgfdeclareplotmark{m3ab}{%
\node[scale=\mThreescale*\symscale] at (0,0) {\tikz {%
\draw[fill=black,line width=.3pt,even odd rule,rotate=90] 
(0.*36.0:10pt)  -- (0.5*36.0:5pt)   -- 
(1.*36.0:10pt)  -- (1.5*36.0:5pt)   -- 
(2.*36.0:10pt)  -- (2.5*36.0:5pt)   -- 
(3.*36.0:10pt)  -- (3.5*36.0:5pt)   -- 
(4.*36.0:10pt)  -- (4.5*36.0:5pt)   -- 
(5.*36.0:10pt)  -- (5.5*36.0:5pt)   -- 
(6.*36.0:10pt)  -- (6.5*36.0:5pt)   -- 
(7.*36.0:10pt)  -- (7.5*36.0:5pt)   -- 
(8.*36.0:10pt)  -- (8.5*36.0:5pt)   -- 
(9.*36.0:10pt)  -- (9.5*36.0:5pt)   --  cycle;
\draw[line width=.4pt,even odd rule,rotate=90] 
(0.5*36.0:9pt)  -- (0.5*36.0:5pt)   
(1.5*36.0:9pt)  -- (1.5*36.0:5pt)   
(2.5*36.0:9pt)  -- (2.5*36.0:5pt)   
(3.5*36.0:9pt)  -- (3.5*36.0:5pt)   
(4.5*36.0:9pt)  -- (4.5*36.0:5pt)   
(5.5*36.0:9pt)  -- (5.5*36.0:5pt)   
(6.5*36.0:9pt)  -- (6.5*36.0:5pt)   
(7.5*36.0:9pt)  -- (7.5*36.0:5pt)   
(8.5*36.0:9pt)  -- (8.5*36.0:5pt)   
(9.5*36.0:9pt)  -- (9.5*36.0:5pt)  ;
 }}}

\pgfdeclareplotmark{m3avb}{%
\node[scale=\mThreescale*\symscale] at (0,0) {\tikz {%
\draw[fill=black,line width=.3pt,even odd rule,rotate=90] 
(0.*36.0:9pt)  -- (0.5*36.0:4.5pt)   -- 
(1.*36.0:9pt)  -- (1.5*36.0:4.5pt)   -- 
(2.*36.0:9pt)  -- (2.5*36.0:4.5pt)   -- 
(3.*36.0:9pt)  -- (3.5*36.0:4.5pt)   -- 
(4.*36.0:9pt)  -- (4.5*36.0:4.5pt)   -- 
(5.*36.0:9pt)  -- (5.5*36.0:4.5pt)   -- 
(6.*36.0:9pt)  -- (6.5*36.0:4.5pt)   -- 
(7.*36.0:9pt)  -- (7.5*36.0:4.5pt)   -- 
(8.*36.0:9pt)  -- (8.5*36.0:4.5pt)   -- 
(9.*36.0:9pt)  -- (9.5*36.0:4.5pt)   --  cycle;
\draw (0,0) circle (9pt);
%\draw[line width=.2pt,even odd rule,rotate=90] 
%(0.*36.0:11pt)  -- (0.5*36.0:6.5pt)   -- 
%(1.*36.0:11pt)  -- (1.5*36.0:6.5pt)   -- 
%(2.*36.0:11pt)  -- (2.5*36.0:6.5pt)   -- 
%(3.*36.0:11pt)  -- (3.5*36.0:6.5pt)   -- 
%(4.*36.0:11pt)  -- (4.5*36.0:6.5pt)   -- 
%(5.*36.0:11pt)  -- (5.5*36.0:6.5pt)   -- 
%(6.*36.0:11pt)  -- (6.5*36.0:6.5pt)   -- 
%(7.*36.0:11pt)  -- (7.5*36.0:6.5pt)   -- 
%(8.*36.0:11pt)  -- (8.5*36.0:6.5pt)   -- 
%(9.*36.0:11pt)  -- (9.5*36.0:6.5pt)   --  cycle;
\draw[line width=.4pt,even odd rule,rotate=90] 
(0.5*36.0:10pt)  -- (0.5*36.0:6.5pt)   
(1.5*36.0:10pt)  -- (1.5*36.0:6.5pt)   
(2.5*36.0:10pt)  -- (2.5*36.0:6.5pt)   
(3.5*36.0:10pt)  -- (3.5*36.0:6.5pt)   
(4.5*36.0:10pt)  -- (4.5*36.0:6.5pt)   
(5.5*36.0:10pt)  -- (5.5*36.0:6.5pt)   
(6.5*36.0:10pt)  -- (6.5*36.0:6.5pt)   
(7.5*36.0:10pt)  -- (7.5*36.0:6.5pt)   
(8.5*36.0:10pt)  -- (8.5*36.0:6.5pt)   
(9.5*36.0:10pt)  -- (9.5*36.0:6.5pt)   ;
 }}}



\pgfdeclareplotmark{m3b}{%
\node[scale=\mThreescale*\AtoBscale*\symscale] at (0,0) {\tikz {%
\draw[fill=black,line width=.3pt,even odd rule,rotate=90] 
(0.*36.0:10pt)  -- (0.5*36.0:5pt)   -- 
(1.*36.0:10pt)  -- (1.5*36.0:5pt)   -- 
(2.*36.0:10pt)  -- (2.5*36.0:5pt)   -- 
(3.*36.0:10pt)  -- (3.5*36.0:5pt)   -- 
(4.*36.0:10pt)  -- (4.5*36.0:5pt)   -- 
(5.*36.0:10pt)  -- (5.5*36.0:5pt)   -- 
(6.*36.0:10pt)  -- (6.5*36.0:5pt)   -- 
(7.*36.0:10pt)  -- (7.5*36.0:5pt)   -- 
(8.*36.0:10pt)  -- (8.5*36.0:5pt)   -- 
(9.*36.0:10pt)  -- (9.5*36.0:5pt)   -- cycle
(0,0) circle (2pt);
 }}}


\pgfdeclareplotmark{m3bv}{%
\node[scale=\mThreescale*\AtoBscale*\symscale] at (0,0) {\tikz {%
\draw[fill=black,line width=.3pt,even odd rule,rotate=90] 
(0.*36.0:9pt)  -- (0.5*36.0:4.5pt)   -- 
(1.*36.0:9pt)  -- (1.5*36.0:4.5pt)   -- 
(2.*36.0:9pt)  -- (2.5*36.0:4.5pt)   -- 
(3.*36.0:9pt)  -- (3.5*36.0:4.5pt)   -- 
(4.*36.0:9pt)  -- (4.5*36.0:4.5pt)   -- 
(5.*36.0:9pt)  -- (5.5*36.0:4.5pt)   -- 
(6.*36.0:9pt)  -- (6.5*36.0:4.5pt)   -- 
(7.*36.0:9pt)  -- (7.5*36.0:4.5pt)   -- 
(8.*36.0:9pt)  -- (8.5*36.0:4.5pt)   -- 
(9.*36.0:9pt)  -- (9.5*36.0:4.5pt)   -- cycle
(0,0) circle (2pt);
\draw (0,0) circle (9pt);
%\draw[line width=.2pt,even odd rule,rotate=90] 
%(0.*36.0:11pt)  -- (0.5*36.0:6.5pt)   -- 
%(1.*36.0:11pt)  -- (1.5*36.0:6.5pt)   -- 
%(2.*36.0:11pt)  -- (2.5*36.0:6.5pt)   -- 
%(3.*36.0:11pt)  -- (3.5*36.0:6.5pt)   -- 
%(4.*36.0:11pt)  -- (4.5*36.0:6.5pt)   -- 
%(5.*36.0:11pt)  -- (5.5*36.0:6.5pt)   -- 
%(6.*36.0:11pt)  -- (6.5*36.0:6.5pt)   -- 
%(7.*36.0:11pt)  -- (7.5*36.0:6.5pt)   -- 
%(8.*36.0:11pt)  -- (8.5*36.0:6.5pt)   -- 
%(9.*36.0:11pt)  -- (9.5*36.0:6.5pt)   -- cycle;
 }}}



\pgfdeclareplotmark{m3bb}{%
\node[scale=\mThreescale*\AtoBscale*\symscale] at (0,0) {\tikz {%
\draw[fill=black,line width=.3pt,even odd rule,rotate=90] 
(0.*36.0:10pt)  -- (0.5*36.0:5pt)   -- 
(1.*36.0:10pt)  -- (1.5*36.0:5pt)   -- 
(2.*36.0:10pt)  -- (2.5*36.0:5pt)   -- 
(3.*36.0:10pt)  -- (3.5*36.0:5pt)   -- 
(4.*36.0:10pt)  -- (4.5*36.0:5pt)   -- 
(5.*36.0:10pt)  -- (5.5*36.0:5pt)   -- 
(6.*36.0:10pt)  -- (6.5*36.0:5pt)   -- 
(7.*36.0:10pt)  -- (7.5*36.0:5pt)   -- 
(8.*36.0:10pt)  -- (8.5*36.0:5pt)   -- 
(9.*36.0:10pt)  -- (9.5*36.0:5pt)   --  cycle
(0,0) circle (2pt);
\draw[line width=.4pt,even odd rule,rotate=90] 
(0.5*36.0:9pt)  -- (0.5*36.0:5pt)   
(1.5*36.0:9pt)  -- (1.5*36.0:5pt)   
(2.5*36.0:9pt)  -- (2.5*36.0:5pt)   
(3.5*36.0:9pt)  -- (3.5*36.0:5pt)   
(4.5*36.0:9pt)  -- (4.5*36.0:5pt)   
(5.5*36.0:9pt)  -- (5.5*36.0:5pt)   
(6.5*36.0:9pt)  -- (6.5*36.0:5pt)   
(7.5*36.0:9pt)  -- (7.5*36.0:5pt)   
(8.5*36.0:9pt)  -- (8.5*36.0:5pt)   
(9.5*36.0:9pt)  -- (9.5*36.0:5pt)  ;
 }}}

\pgfdeclareplotmark{m3bvb}{%
\node[scale=\mThreescale*\AtoBscale*\symscale] at (0,0) {\tikz {%
\draw[fill=black,line width=.3pt,even odd rule,rotate=90] 
(0.*36.0:9pt)  -- (0.5*36.0:4.5pt)   -- 
(1.*36.0:9pt)  -- (1.5*36.0:4.5pt)   -- 
(2.*36.0:9pt)  -- (2.5*36.0:4.5pt)   -- 
(3.*36.0:9pt)  -- (3.5*36.0:4.5pt)   -- 
(4.*36.0:9pt)  -- (4.5*36.0:4.5pt)   -- 
(5.*36.0:9pt)  -- (5.5*36.0:4.5pt)   -- 
(6.*36.0:9pt)  -- (6.5*36.0:4.5pt)   -- 
(7.*36.0:9pt)  -- (7.5*36.0:4.5pt)   -- 
(8.*36.0:9pt)  -- (8.5*36.0:4.5pt)   -- 
(9.*36.0:9pt)  -- (9.5*36.0:4.5pt)   --  cycle
(0,0) circle (2pt);
\draw (0,0) circle (9pt);
%\draw[line width=.2pt,even odd rule,rotate=90] 
%(0.*36.0:11pt)  -- (0.5*36.0:6.5pt)   -- 
%(1.*36.0:11pt)  -- (1.5*36.0:6.5pt)   -- 
%(2.*36.0:11pt)  -- (2.5*36.0:6.5pt)   -- 
%(3.*36.0:11pt)  -- (3.5*36.0:6.5pt)   -- 
%(4.*36.0:11pt)  -- (4.5*36.0:6.5pt)   -- 
%(5.*36.0:11pt)  -- (5.5*36.0:6.5pt)   -- 
%(6.*36.0:11pt)  -- (6.5*36.0:6.5pt)   -- 
%(7.*36.0:11pt)  -- (7.5*36.0:6.5pt)   -- 
%(8.*36.0:11pt)  -- (8.5*36.0:6.5pt)   -- 
%(9.*36.0:11pt)  -- (9.5*36.0:6.5pt)   -- cycle;
\draw[line width=.4pt,even odd rule,rotate=90] 
(0.5*36.0:10pt)  -- (0.5*36.0:6.5pt)   
(1.5*36.0:10pt)  -- (1.5*36.0:6.5pt)   
(2.5*36.0:10pt)  -- (2.5*36.0:6.5pt)   
(3.5*36.0:10pt)  -- (3.5*36.0:6.5pt)   
(4.5*36.0:10pt)  -- (4.5*36.0:6.5pt)   
(5.5*36.0:10pt)  -- (5.5*36.0:6.5pt)   
(6.5*36.0:10pt)  -- (6.5*36.0:6.5pt)   
(7.5*36.0:10pt)  -- (7.5*36.0:6.5pt)   
(8.5*36.0:10pt)  -- (8.5*36.0:6.5pt)   
(9.5*36.0:10pt)  -- (9.5*36.0:6.5pt)  ;
 }}}



\pgfdeclareplotmark{m3c}{%
\node[scale=\mThreescale*\AtoCscale*\symscale] at (0,0) {\tikz {%
\draw[fill=black,line width=.3pt,even odd rule,rotate=90] 
(0.*36.0:10pt)  -- (0.5*36.0:5pt)   -- 
(1.*36.0:10pt)  -- (1.5*36.0:5pt)   -- 
(2.*36.0:10pt)  -- (2.5*36.0:5pt)   -- 
(3.*36.0:10pt)  -- (3.5*36.0:5pt)   -- 
(4.*36.0:10pt)  -- (4.5*36.0:5pt)   -- 
(5.*36.0:10pt)  -- (5.5*36.0:5pt)   -- 
(6.*36.0:10pt)  -- (6.5*36.0:5pt)   -- 
(7.*36.0:10pt)  -- (7.5*36.0:5pt)   -- 
(8.*36.0:10pt)  -- (8.5*36.0:5pt)   -- 
(9.*36.0:10pt)  -- (9.5*36.0:5pt)   -- cycle
(0,0) circle (3pt);
 }}}


\pgfdeclareplotmark{m3cv}{%
\node[scale=\mThreescale*\AtoCscale*\symscale] at (0,0) {\tikz {%
\draw[fill=black,line width=.3pt,even odd rule,rotate=90] 
(0.*36.0:9pt)  -- (0.5*36.0:4.5pt)   -- 
(1.*36.0:9pt)  -- (1.5*36.0:4.5pt)   -- 
(2.*36.0:9pt)  -- (2.5*36.0:4.5pt)   -- 
(3.*36.0:9pt)  -- (3.5*36.0:4.5pt)   -- 
(4.*36.0:9pt)  -- (4.5*36.0:4.5pt)   -- 
(5.*36.0:9pt)  -- (5.5*36.0:4.5pt)   -- 
(6.*36.0:9pt)  -- (6.5*36.0:4.5pt)   -- 
(7.*36.0:9pt)  -- (7.5*36.0:4.5pt)   -- 
(8.*36.0:9pt)  -- (8.5*36.0:4.5pt)   -- 
(9.*36.0:9pt)  -- (9.5*36.0:4.5pt)   --  cycle
(0,0) circle (3pt);
\draw (0,0) circle (9pt);
%\draw[line width=.2pt,even odd rule,rotate=90] 
%(0.*36.0:11pt)  -- (0.5*36.0:6.5pt)   -- 
%(1.*36.0:11pt)  -- (1.5*36.0:6.5pt)   -- 
%(2.*36.0:11pt)  -- (2.5*36.0:6.5pt)   -- 
%(3.*36.0:11pt)  -- (3.5*36.0:6.5pt)   -- 
%(4.*36.0:11pt)  -- (4.5*36.0:6.5pt)   -- 
%(5.*36.0:11pt)  -- (5.5*36.0:6.5pt)   -- 
%(6.*36.0:11pt)  -- (6.5*36.0:6.5pt)   -- 
%(7.*36.0:11pt)  -- (7.5*36.0:6.5pt)   -- 
%(8.*36.0:11pt)  -- (8.5*36.0:6.5pt)   -- 
%(9.*36.0:11pt)  -- (9.5*36.0:6.5pt)   -- cycle;
 }}}



\pgfdeclareplotmark{m3cb}{%
\node[scale=\mThreescale*\AtoCscale*\symscale] at (0,0) {\tikz {%
\draw[fill=black,line width=.3pt,even odd rule,rotate=90] 
(0.*36.0:10pt)  -- (0.5*36.0:5pt)   -- 
(1.*36.0:10pt)  -- (1.5*36.0:5pt)   -- 
(2.*36.0:10pt)  -- (2.5*36.0:5pt)   -- 
(3.*36.0:10pt)  -- (3.5*36.0:5pt)   -- 
(4.*36.0:10pt)  -- (4.5*36.0:5pt)   -- 
(5.*36.0:10pt)  -- (5.5*36.0:5pt)   -- 
(6.*36.0:10pt)  -- (6.5*36.0:5pt)   -- 
(7.*36.0:10pt)  -- (7.5*36.0:5pt)   -- 
(8.*36.0:10pt)  -- (8.5*36.0:5pt)   -- 
(9.*36.0:10pt)  -- (9.5*36.0:5pt)   -- cycle
(0,0) circle (3pt);
\draw[line width=.4pt,even odd rule,rotate=90] 
(0.5*36.0:9pt)  -- (0.5*36.0:5pt)   
(1.5*36.0:9pt)  -- (1.5*36.0:5pt)   
(2.5*36.0:9pt)  -- (2.5*36.0:5pt)   
(3.5*36.0:9pt)  -- (3.5*36.0:5pt)   
(4.5*36.0:9pt)  -- (4.5*36.0:5pt)   
(5.5*36.0:9pt)  -- (5.5*36.0:5pt)   
(6.5*36.0:9pt)  -- (6.5*36.0:5pt)   
(7.5*36.0:9pt)  -- (7.5*36.0:5pt)   
(8.5*36.0:9pt)  -- (8.5*36.0:5pt)   
(9.5*36.0:9pt)  -- (9.5*36.0:5pt)   ;
 }}}

\pgfdeclareplotmark{m3cvb}{%
\node[scale=\mThreescale*\AtoCscale*\symscale] at (0,0) {\tikz {%
\draw[fill=black,line width=.3pt,even odd rule,rotate=90] 
(0.*36.0:9pt)  -- (0.5*36.0:4.5pt)   -- 
(1.*36.0:9pt)  -- (1.5*36.0:4.5pt)   -- 
(2.*36.0:9pt)  -- (2.5*36.0:4.5pt)   -- 
(3.*36.0:9pt)  -- (3.5*36.0:4.5pt)   -- 
(4.*36.0:9pt)  -- (4.5*36.0:4.5pt)   -- 
(5.*36.0:9pt)  -- (5.5*36.0:4.5pt)   -- 
(6.*36.0:9pt)  -- (6.5*36.0:4.5pt)   -- 
(7.*36.0:9pt)  -- (7.5*36.0:4.5pt)   -- 
(8.*36.0:9pt)  -- (8.5*36.0:4.5pt)   -- 
(9.*36.0:9pt)  -- (9.5*36.0:4.5pt)   -- cycle
(0,0) circle (3pt);
\draw (0,0) circle (9pt);
%\draw[line width=.2pt,even odd rule,rotate=90] 
%(0.*36.0:11pt)  -- (0.5*36.0:6.5pt)   -- 
%(1.*36.0:11pt)  -- (1.5*36.0:6.5pt)   -- 
%(2.*36.0:11pt)  -- (2.5*36.0:6.5pt)   -- 
%(3.*36.0:11pt)  -- (3.5*36.0:6.5pt)   -- 
%(4.*36.0:11pt)  -- (4.5*36.0:6.5pt)   -- 
%(5.*36.0:11pt)  -- (5.5*36.0:6.5pt)   -- 
%(6.*36.0:11pt)  -- (6.5*36.0:6.5pt)   -- 
%(7.*36.0:11pt)  -- (7.5*36.0:6.5pt)   -- 
%(8.*36.0:11pt)  -- (8.5*36.0:6.5pt)   -- 
%(9.*36.0:11pt)  -- (9.5*36.0:6.5pt)   --  cycle;
\draw[line width=.4pt,even odd rule,rotate=90] 
(0.5*36.0:10pt)  -- (0.5*36.0:6.5pt)   
(1.5*36.0:10pt)  -- (1.5*36.0:6.5pt)   
(2.5*36.0:10pt)  -- (2.5*36.0:6.5pt)   
(3.5*36.0:10pt)  -- (3.5*36.0:6.5pt)   
(4.5*36.0:10pt)  -- (4.5*36.0:6.5pt)   
(5.5*36.0:10pt)  -- (5.5*36.0:6.5pt)   
(6.5*36.0:10pt)  -- (6.5*36.0:6.5pt)   
(7.5*36.0:10pt)  -- (7.5*36.0:6.5pt)   
(8.5*36.0:10pt)  -- (8.5*36.0:6.5pt)   
(9.5*36.0:10pt)  -- (9.5*36.0:6.5pt)   ;
 }}}



\pgfdeclareplotmark{m4a}{%
\node[scale=\mFourscale*\symscale] at (0,0) {\tikz {%
\draw[fill=black,line width=.3pt,even odd rule,rotate=90] 
(0.*45.0:10pt)  -- (0.5*45.0:5pt)   -- 
(1.*45.0:10pt)  -- (1.5*45.0:5pt)   -- 
(2.*45.0:10pt)  -- (2.5*45.0:5pt)   -- 
(3.*45.0:10pt)  -- (3.5*45.0:5pt)   -- 
(4.*45.0:10pt)  -- (4.5*45.0:5pt)   -- 
(5.*45.0:10pt)  -- (5.5*45.0:5pt)   -- 
(6.*45.0:10pt)  -- (6.5*45.0:5pt)   -- 
(7.*45.0:10pt)  -- (7.5*45.0:5pt)   --  cycle;
 }}}


\pgfdeclareplotmark{m4av}{%
\node[scale=\mFourscale*\symscale] at (0,0) {\tikz {%
\draw[fill=black,line width=.3pt,even odd rule,rotate=90] 
(0.*45.0:9pt)  -- (0.5*45.0:4.5pt)   -- 
(1.*45.0:9pt)  -- (1.5*45.0:4.5pt)   -- 
(2.*45.0:9pt)  -- (2.5*45.0:4.5pt)   -- 
(3.*45.0:9pt)  -- (3.5*45.0:4.5pt)   -- 
(4.*45.0:9pt)  -- (4.5*45.0:4.5pt)   -- 
(5.*45.0:9pt)  -- (5.5*45.0:4.5pt)   -- 
(6.*45.0:9pt)  -- (6.5*45.0:4.5pt)   -- 
(7.*45.0:9pt)  -- (7.5*45.0:4.5pt)   -- cycle;
\draw (0,0) circle (9pt);
%\draw[line width=.3pt,even odd rule,rotate=90] 
%(0.*45.0:11pt)  -- (0.5*45.0:6.5pt)   -- 
%(1.*45.0:11pt)  -- (1.5*45.0:6.5pt)   -- 
%(2.*45.0:11pt)  -- (2.5*45.0:6.5pt)   -- 
%(3.*45.0:11pt)  -- (3.5*45.0:6.5pt)   -- 
%(4.*45.0:11pt)  -- (4.5*45.0:6.5pt)   -- 
%(5.*45.0:11pt)  -- (5.5*45.0:6.5pt)   -- 
%(6.*45.0:11pt)  -- (6.5*45.0:6.5pt)   -- 
%(7.*45.0:11pt)  -- (7.5*45.0:6.5pt)   --  cycle;
 }}}



\pgfdeclareplotmark{m4ab}{%
\node[scale=\mFourscale*\symscale] at (0,0) {\tikz {%
\draw[fill=black,line width=.3pt,even odd rule,rotate=90] 
(0.*45.0:10pt)  -- (0.5*45.0:5pt)   -- 
(1.*45.0:10pt)  -- (1.5*45.0:5pt)   -- 
(2.*45.0:10pt)  -- (2.5*45.0:5pt)   -- 
(3.*45.0:10pt)  -- (3.5*45.0:5pt)   -- 
(4.*45.0:10pt)  -- (4.5*45.0:5pt)   -- 
(5.*45.0:10pt)  -- (5.5*45.0:5pt)   -- 
(6.*45.0:10pt)  -- (6.5*45.0:5pt)   -- 
(7.*45.0:10pt)  -- (7.5*45.0:5pt)   --  cycle;
\draw[line width=.4pt,even odd rule,rotate=90] 
(0.5*45.0:9pt)  -- (0.5*45.0:5pt)   
(1.5*45.0:9pt)  -- (1.5*45.0:5pt)   
(2.5*45.0:9pt)  -- (2.5*45.0:5pt)   
(3.5*45.0:9pt)  -- (3.5*45.0:5pt)   
(4.5*45.0:9pt)  -- (4.5*45.0:5pt)   
(5.5*45.0:9pt)  -- (5.5*45.0:5pt)   
(6.5*45.0:9pt)  -- (6.5*45.0:5pt)   
(7.5*45.0:9pt)  -- (7.5*45.0:5pt)  ;
 }}}

\pgfdeclareplotmark{m4avb}{%
\node[scale=\mFourscale*\symscale] at (0,0) {\tikz {%
\draw[fill=black,line width=.3pt,even odd rule,rotate=90] 
(0.*45.0:9pt)  -- (0.5*45.0:4.5pt)   -- 
(1.*45.0:9pt)  -- (1.5*45.0:4.5pt)   -- 
(2.*45.0:9pt)  -- (2.5*45.0:4.5pt)   -- 
(3.*45.0:9pt)  -- (3.5*45.0:4.5pt)   -- 
(4.*45.0:9pt)  -- (4.5*45.0:4.5pt)   -- 
(5.*45.0:9pt)  -- (5.5*45.0:4.5pt)   -- 
(6.*45.0:9pt)  -- (6.5*45.0:4.5pt)   -- 
(7.*45.0:9pt)  -- (7.5*45.0:4.5pt)   --  cycle;
\draw (0,0) circle (9pt);
%\draw[line width=.3pt,even odd rule,rotate=90] 
%(0.*45.0:11pt)  -- (0.5*45.0:6.5pt)   -- 
%(1.*45.0:11pt)  -- (1.5*45.0:6.5pt)   -- 
%(2.*45.0:11pt)  -- (2.5*45.0:6.5pt)   -- 
%(3.*45.0:11pt)  -- (3.5*45.0:6.5pt)   -- 
%(4.*45.0:11pt)  -- (4.5*45.0:6.5pt)   -- 
%(5.*45.0:11pt)  -- (5.5*45.0:6.5pt)   -- 
%(6.*45.0:11pt)  -- (6.5*45.0:6.5pt)   -- 
%(7.*45.0:11pt)  -- (7.5*45.0:6.5pt)   --  cycle;
\draw[line width=.4pt,even odd rule,rotate=90] 
(0.5*45.0:10pt)  -- (0.5*45.0:6.5pt)   
(1.5*45.0:10pt)  -- (1.5*45.0:6.5pt)   
(2.5*45.0:10pt)  -- (2.5*45.0:6.5pt)   
(3.5*45.0:10pt)  -- (3.5*45.0:6.5pt)   
(4.5*45.0:10pt)  -- (4.5*45.0:6.5pt)   
(5.5*45.0:10pt)  -- (5.5*45.0:6.5pt)   
(6.5*45.0:10pt)  -- (6.5*45.0:6.5pt)   
(7.5*45.0:10pt)  -- (7.5*45.0:6.5pt)   ;
 }}}



\pgfdeclareplotmark{m4b}{%
\node[scale=\mFourscale*\AtoBscale*\symscale] at (0,0) {\tikz {%
\draw[fill=black,line width=.3pt,even odd rule,rotate=90] 
(0.*45.0:10pt)  -- (0.5*45.0:5pt)   -- 
(1.*45.0:10pt)  -- (1.5*45.0:5pt)   -- 
(2.*45.0:10pt)  -- (2.5*45.0:5pt)   -- 
(3.*45.0:10pt)  -- (3.5*45.0:5pt)   -- 
(4.*45.0:10pt)  -- (4.5*45.0:5pt)   -- 
(5.*45.0:10pt)  -- (5.5*45.0:5pt)   -- 
(6.*45.0:10pt)  -- (6.5*45.0:5pt)   -- 
(7.*45.0:10pt)  -- (7.5*45.0:5pt)   -- cycle
(0,0) circle (2pt);
 }}}


\pgfdeclareplotmark{m4bv}{%
\node[scale=\mFourscale*\AtoBscale*\symscale] at (0,0) {\tikz {%
\draw[fill=black,line width=.3pt,even odd rule,rotate=90] 
(0.*45.0:9pt)  -- (0.5*45.0:4.5pt)   -- 
(1.*45.0:9pt)  -- (1.5*45.0:4.5pt)   -- 
(2.*45.0:9pt)  -- (2.5*45.0:4.5pt)   -- 
(3.*45.0:9pt)  -- (3.5*45.0:4.5pt)   -- 
(4.*45.0:9pt)  -- (4.5*45.0:4.5pt)   -- 
(5.*45.0:9pt)  -- (5.5*45.0:4.5pt)   -- 
(6.*45.0:9pt)  -- (6.5*45.0:4.5pt)   -- 
(7.*45.0:9pt)  -- (7.5*45.0:4.5pt)   -- cycle
(0,0) circle (2pt);
\draw (0,0) circle (9pt);
%\draw[line width=.3pt,even odd rule,rotate=90] 
%(0.*45.0:11pt)  -- (0.5*45.0:6.5pt)   -- 
%(1.*45.0:11pt)  -- (1.5*45.0:6.5pt)   -- 
%(2.*45.0:11pt)  -- (2.5*45.0:6.5pt)   -- 
%(3.*45.0:11pt)  -- (3.5*45.0:6.5pt)   -- 
%(4.*45.0:11pt)  -- (4.5*45.0:6.5pt)   -- 
%(5.*45.0:11pt)  -- (5.5*45.0:6.5pt)   -- 
%(6.*45.0:11pt)  -- (6.5*45.0:6.5pt)   -- 
%(7.*45.0:11pt)  -- (7.5*45.0:6.5pt)   --cycle;
 }}}



\pgfdeclareplotmark{m4bb}{%
\node[scale=\mFourscale*\AtoBscale*\symscale] at (0,0) {\tikz {%
\draw[fill=black,line width=.3pt,even odd rule,rotate=90] 
(0.*45.0:10pt)  -- (0.5*45.0:5pt)   -- 
(1.*45.0:10pt)  -- (1.5*45.0:5pt)   -- 
(2.*45.0:10pt)  -- (2.5*45.0:5pt)   -- 
(3.*45.0:10pt)  -- (3.5*45.0:5pt)   -- 
(4.*45.0:10pt)  -- (4.5*45.0:5pt)   -- 
(5.*45.0:10pt)  -- (5.5*45.0:5pt)   -- 
(6.*45.0:10pt)  -- (6.5*45.0:5pt)   -- 
(7.*45.0:10pt)  -- (7.5*45.0:5pt)   --  cycle
(0,0) circle (2pt);
\draw[line width=.4pt,even odd rule,rotate=90] 
(0.5*45.0:9pt)  -- (0.5*45.0:5pt)   
(1.5*45.0:9pt)  -- (1.5*45.0:5pt)   
(2.5*45.0:9pt)  -- (2.5*45.0:5pt)   
(3.5*45.0:9pt)  -- (3.5*45.0:5pt)   
(4.5*45.0:9pt)  -- (4.5*45.0:5pt)   
(5.5*45.0:9pt)  -- (5.5*45.0:5pt)   
(6.5*45.0:9pt)  -- (6.5*45.0:5pt)   
(7.5*45.0:9pt)  -- (7.5*45.0:5pt)  ;
 }}}

\pgfdeclareplotmark{m4bvb}{%
\node[scale=\mFourscale*\AtoBscale*\symscale] at (0,0) {\tikz {%
\draw[fill=black,line width=.3pt,even odd rule,rotate=90] 
(0.*45.0:9pt)  -- (0.5*45.0:4.5pt)   -- 
(1.*45.0:9pt)  -- (1.5*45.0:4.5pt)   -- 
(2.*45.0:9pt)  -- (2.5*45.0:4.5pt)   -- 
(3.*45.0:9pt)  -- (3.5*45.0:4.5pt)   -- 
(4.*45.0:9pt)  -- (4.5*45.0:4.5pt)   -- 
(5.*45.0:9pt)  -- (5.5*45.0:4.5pt)   -- 
(6.*45.0:9pt)  -- (6.5*45.0:4.5pt)   -- 
(7.*45.0:9pt)  -- (7.5*45.0:4.5pt)   --  cycle
(0,0) circle (2pt);
\draw (0,0) circle (9pt);
%\draw[line width=.3pt,even odd rule,rotate=90] 
%(0.*45.0:11pt)  -- (0.5*45.0:6.5pt)   -- 
%(1.*45.0:11pt)  -- (1.5*45.0:6.5pt)   -- 
%(2.*45.0:11pt)  -- (2.5*45.0:6.5pt)   -- 
%(3.*45.0:11pt)  -- (3.5*45.0:6.5pt)   -- 
%(4.*45.0:11pt)  -- (4.5*45.0:6.5pt)   -- 
%(5.*45.0:11pt)  -- (5.5*45.0:6.5pt)   -- 
%(6.*45.0:11pt)  -- (6.5*45.0:6.5pt)   -- 
%(7.*45.0:11pt)  -- (7.5*45.0:6.5pt)   -- cycle;
\draw[line width=.4pt,even odd rule,rotate=90] 
(0.5*45.0:10pt)  -- (0.5*45.0:6.5pt)   
(1.5*45.0:10pt)  -- (1.5*45.0:6.5pt)   
(2.5*45.0:10pt)  -- (2.5*45.0:6.5pt)   
(3.5*45.0:10pt)  -- (3.5*45.0:6.5pt)   
(4.5*45.0:10pt)  -- (4.5*45.0:6.5pt)   
(5.5*45.0:10pt)  -- (5.5*45.0:6.5pt)   
(6.5*45.0:10pt)  -- (6.5*45.0:6.5pt)   
(7.5*45.0:10pt)  -- (7.5*45.0:6.5pt)  ;
 }}}



\pgfdeclareplotmark{m4c}{%
\node[scale=\mFourscale*\AtoCscale*\symscale] at (0,0) {\tikz {%
\draw[fill=black,line width=.3pt,even odd rule,rotate=90] 
(0.*45.0:10pt)  -- (0.5*45.0:5pt)   -- 
(1.*45.0:10pt)  -- (1.5*45.0:5pt)   -- 
(2.*45.0:10pt)  -- (2.5*45.0:5pt)   -- 
(3.*45.0:10pt)  -- (3.5*45.0:5pt)   -- 
(4.*45.0:10pt)  -- (4.5*45.0:5pt)   -- 
(5.*45.0:10pt)  -- (5.5*45.0:5pt)   -- 
(6.*45.0:10pt)  -- (6.5*45.0:5pt)   -- 
(7.*45.0:10pt)  -- (7.5*45.0:5pt)   -- cycle
(0,0) circle (3pt);
 }}}


\pgfdeclareplotmark{m4cv}{%
\node[scale=\mFourscale*\AtoCscale*\symscale] at (0,0) {\tikz {%
\draw[fill=black,line width=.3pt,even odd rule,rotate=90] 
(0.*45.0:9pt)  -- (0.5*45.0:4.5pt)   -- 
(1.*45.0:9pt)  -- (1.5*45.0:4.5pt)   -- 
(2.*45.0:9pt)  -- (2.5*45.0:4.5pt)   -- 
(3.*45.0:9pt)  -- (3.5*45.0:4.5pt)   -- 
(4.*45.0:9pt)  -- (4.5*45.0:4.5pt)   -- 
(5.*45.0:9pt)  -- (5.5*45.0:4.5pt)   -- 
(6.*45.0:9pt)  -- (6.5*45.0:4.5pt)   -- 
(7.*45.0:9pt)  -- (7.5*45.0:4.5pt)   --  cycle
(0,0) circle (3pt);
\draw (0,0) circle (9pt);
%\draw[line width=.3pt,even odd rule,rotate=90] 
%(0.*45.0:11pt)  -- (0.5*45.0:6.5pt)   -- 
%(1.*45.0:11pt)  -- (1.5*45.0:6.5pt)   -- 
%(2.*45.0:11pt)  -- (2.5*45.0:6.5pt)   -- 
%(3.*45.0:11pt)  -- (3.5*45.0:6.5pt)   -- 
%(4.*45.0:11pt)  -- (4.5*45.0:6.5pt)   -- 
%(5.*45.0:11pt)  -- (5.5*45.0:6.5pt)   -- 
%(6.*45.0:11pt)  -- (6.5*45.0:6.5pt)   -- 
%(7.*45.0:11pt)  -- (7.5*45.0:6.5pt)   -- cycle;
 }}}



\pgfdeclareplotmark{m4cb}{%
\node[scale=\mFourscale*\AtoCscale*\symscale] at (0,0) {\tikz {%
\draw[fill=black,line width=.3pt,even odd rule,rotate=90] 
(0.*45.0:10pt)  -- (0.5*45.0:5pt)   -- 
(1.*45.0:10pt)  -- (1.5*45.0:5pt)   -- 
(2.*45.0:10pt)  -- (2.5*45.0:5pt)   -- 
(3.*45.0:10pt)  -- (3.5*45.0:5pt)   -- 
(4.*45.0:10pt)  -- (4.5*45.0:5pt)   -- 
(5.*45.0:10pt)  -- (5.5*45.0:5pt)   -- 
(6.*45.0:10pt)  -- (6.5*45.0:5pt)   -- 
(7.*45.0:10pt)  -- (7.5*45.0:5pt)   -- cycle
(0,0) circle (3pt);
\draw[line width=.4pt,even odd rule,rotate=90] 
(0.5*45.0:9pt)  -- (0.5*45.0:5pt)   
(1.5*45.0:9pt)  -- (1.5*45.0:5pt)   
(2.5*45.0:9pt)  -- (2.5*45.0:5pt)   
(3.5*45.0:9pt)  -- (3.5*45.0:5pt)   
(4.5*45.0:9pt)  -- (4.5*45.0:5pt)   
(5.5*45.0:9pt)  -- (5.5*45.0:5pt)   
(6.5*45.0:9pt)  -- (6.5*45.0:5pt)   
(7.5*45.0:9pt)  -- (7.5*45.0:5pt)  ;
 }}}

\pgfdeclareplotmark{m4cvb}{%
\node[scale=\mFourscale*\AtoCscale*\symscale] at (0,0) {\tikz {%
\draw[fill=black,line width=.3pt,even odd rule,rotate=90] 
(0.*45.0:9pt)  -- (0.5*45.0:4.5pt)   -- 
(1.*45.0:9pt)  -- (1.5*45.0:4.5pt)   -- 
(2.*45.0:9pt)  -- (2.5*45.0:4.5pt)   -- 
(3.*45.0:9pt)  -- (3.5*45.0:4.5pt)   -- 
(4.*45.0:9pt)  -- (4.5*45.0:4.5pt)   -- 
(5.*45.0:9pt)  -- (5.5*45.0:4.5pt)   -- 
(6.*45.0:9pt)  -- (6.5*45.0:4.5pt)   -- 
(7.*45.0:9pt)  -- (7.5*45.0:4.5pt)   --  cycle
(0,0) circle (3pt);
\draw (0,0) circle (9pt);
%\draw[line width=.3pt,even odd rule,rotate=90] 
%(0.*45.0:11pt)  -- (0.5*45.0:6.5pt)   -- 
%(1.*45.0:11pt)  -- (1.5*45.0:6.5pt)   -- 
%(2.*45.0:11pt)  -- (2.5*45.0:6.5pt)   -- 
%(3.*45.0:11pt)  -- (3.5*45.0:6.5pt)   -- 
%(4.*45.0:11pt)  -- (4.5*45.0:6.5pt)   -- 
%(5.*45.0:11pt)  -- (5.5*45.0:6.5pt)   -- 
%(6.*45.0:11pt)  -- (6.5*45.0:6.5pt)   -- 
%(7.*45.0:11pt)  -- (7.5*45.0:6.5pt)   --  cycle;
\draw[line width=.4pt,even odd rule,rotate=90] 
(0.5*45.0:10pt)  -- (0.5*45.0:6.5pt)   
(1.5*45.0:10pt)  -- (1.5*45.0:6.5pt)   
(2.5*45.0:10pt)  -- (2.5*45.0:6.5pt)   
(3.5*45.0:10pt)  -- (3.5*45.0:6.5pt)   
(4.5*45.0:10pt)  -- (4.5*45.0:6.5pt)   
(5.5*45.0:10pt)  -- (5.5*45.0:6.5pt)   
(6.5*45.0:10pt)  -- (6.5*45.0:6.5pt)   
(7.5*45.0:10pt)  -- (7.5*45.0:6.5pt)   ;
 }}}



\pgfdeclareplotmark{m5a}{%
\node[scale=\mFivescale*\symscale] at (0,0) {\tikz {%
\draw[fill=black,line width=.3pt,even odd rule,rotate=90] 
(0.*60.0:10pt)  -- (0.5*60.0:1.14*5.0pt)   -- 
(1.*60.0:10pt)  -- (1.5*60.0:1.14*5.0pt)   -- 
(2.*60.0:10pt)  -- (2.5*60.0:1.14*5.0pt)   -- 
(3.*60.0:10pt)  -- (3.5*60.0:1.14*5.0pt)   -- 
(4.*60.0:10pt)  -- (4.5*60.0:1.14*5.0pt)   -- 
(5.*60.0:10pt)  -- (5.5*60.0:1.14*5.0pt)   --  cycle;
 }}}


\pgfdeclareplotmark{m5av}{%
\node[scale=\mFivescale*\symscale] at (0,0) {\tikz {%
\draw[fill=black,line width=.3pt,even odd rule,rotate=90] 
(0.*60.0:9pt)  -- (0.5*60.0:1.14*4.5pt)   -- 
(1.*60.0:9pt)  -- (1.5*60.0:1.14*4.5pt)   -- 
(2.*60.0:9pt)  -- (2.5*60.0:1.14*4.5pt)   -- 
(3.*60.0:9pt)  -- (3.5*60.0:1.14*4.5pt)   -- 
(4.*60.0:9pt)  -- (4.5*60.0:1.14*4.5pt)   -- 
(5.*60.0:9pt)  -- (5.5*60.0:1.14*4.5pt)   -- cycle;
\draw (0,0) circle (9pt);
%\draw[line width=.35pt,even odd rule,rotate=90] 
%(0.*60.0:11pt)  -- (0.5*60.0:1.14*1.00*6.0pt)   -- 
%(1.*60.0:11pt)  -- (1.5*60.0:1.14*1.00*6.0pt)   -- 
%(2.*60.0:11pt)  -- (2.5*60.0:1.14*1.00*6.0pt)   -- 
%(3.*60.0:11pt)  -- (3.5*60.0:1.14*1.00*6.0pt)   -- 
%(4.*60.0:11pt)  -- (4.5*60.0:1.14*1.00*6.0pt)   -- 
%(5.*60.0:11pt)  -- (5.5*60.0:1.14*1.00*6.0pt)   --  cycle;
 }}}



\pgfdeclareplotmark{m5ab}{%
\node[scale=\mFivescale*\symscale] at (0,0) {\tikz {%
\draw[fill=black,line width=.3pt,even odd rule,rotate=90] 
(0.*60.0:10pt)  -- (0.5*60.0:1.14*5.0pt)   -- 
(1.*60.0:10pt)  -- (1.5*60.0:1.14*5.0pt)   -- 
(2.*60.0:10pt)  -- (2.5*60.0:1.14*5.0pt)   -- 
(3.*60.0:10pt)  -- (3.5*60.0:1.14*5.0pt)   -- 
(4.*60.0:10pt)  -- (4.5*60.0:1.14*5.0pt)   -- 
(5.*60.0:10pt)  -- (5.5*60.0:1.14*5.0pt)   --  cycle;
\draw[line width=0.6pt,even odd rule,rotate=90] 
(0.5*60.0:9pt)  -- (0.5*60.0:1.14*5.0pt)   
(1.5*60.0:9pt)  -- (1.5*60.0:1.14*5.0pt)   
(2.5*60.0:9pt)  -- (2.5*60.0:1.14*5.0pt)   
(3.5*60.0:9pt)  -- (3.5*60.0:1.14*5.0pt)   
(4.5*60.0:9pt)  -- (4.5*60.0:1.14*5.0pt)   
(5.5*60.0:9pt)  -- (5.5*60.0:1.14*5.0pt)  ;
 }}}

\pgfdeclareplotmark{m5avb}{%
\node[scale=\mFivescale*\symscale] at (0,0) {\tikz {%
\draw[fill=black,line width=.3pt,even odd rule,rotate=90] 
(0.*60.0:9pt)  -- (0.5*60.0:1.14*4.5pt)   -- 
(1.*60.0:9pt)  -- (1.5*60.0:1.14*4.5pt)   -- 
(2.*60.0:9pt)  -- (2.5*60.0:1.14*4.5pt)   -- 
(3.*60.0:9pt)  -- (3.5*60.0:1.14*4.5pt)   -- 
(4.*60.0:9pt)  -- (4.5*60.0:1.14*4.5pt)   -- 
(5.*60.0:9pt)  -- (5.5*60.0:1.14*4.5pt)   --  cycle;
\draw (0,0) circle (9pt);
%\draw[line width=.35pt,even odd rule,rotate=90] 
%(0.*60.0:11pt)  -- (0.5*60.0:1.14*1.00*6.0pt)   -- 
%(1.*60.0:11pt)  -- (1.5*60.0:1.14*1.00*6.0pt)   -- 
%(2.*60.0:11pt)  -- (2.5*60.0:1.14*1.00*6.0pt)   -- 
%(3.*60.0:11pt)  -- (3.5*60.0:1.14*1.00*6.0pt)   -- 
%(4.*60.0:11pt)  -- (4.5*60.0:1.14*1.00*6.0pt)   -- 
%(5.*60.0:11pt)  -- (5.5*60.0:1.14*1.00*6.0pt)   --  cycle;
\draw[line width=0.6pt,even odd rule,rotate=90] 
(0.5*60.0:10pt)  -- (0.5*60.0:1.14*1.00*6.0pt)   
(1.5*60.0:10pt)  -- (1.5*60.0:1.14*1.00*6.0pt)   
(2.5*60.0:10pt)  -- (2.5*60.0:1.14*1.00*6.0pt)   
(3.5*60.0:10pt)  -- (3.5*60.0:1.14*1.00*6.0pt)   
(4.5*60.0:10pt)  -- (4.5*60.0:1.14*1.00*6.0pt)   
(5.5*60.0:10pt)  -- (5.5*60.0:1.14*1.00*6.0pt)   ;
 }}}



\pgfdeclareplotmark{m5b}{%
\node[scale=\mFivescale*\AtoBscale*\symscale] at (0,0) {\tikz {%
\draw[fill=black,line width=.3pt,even odd rule,rotate=90] 
(0.*60.0:10pt)  -- (0.5*60.0:1.14*5.0pt)   -- 
(1.*60.0:10pt)  -- (1.5*60.0:1.14*5.0pt)   -- 
(2.*60.0:10pt)  -- (2.5*60.0:1.14*5.0pt)   -- 
(3.*60.0:10pt)  -- (3.5*60.0:1.14*5.0pt)   -- 
(4.*60.0:10pt)  -- (4.5*60.0:1.14*5.0pt)   -- 
(5.*60.0:10pt)  -- (5.5*60.0:1.14*5.0pt)   -- cycle
(0,0) circle (2pt);
 }}}


\pgfdeclareplotmark{m5bv}{%
\node[scale=\mFivescale*\AtoBscale*\symscale] at (0,0) {\tikz {%
\draw[fill=black,line width=.3pt,even odd rule,rotate=90] 
(0.*60.0:9pt)  -- (0.5*60.0:1.14*4.5pt)   -- 
(1.*60.0:9pt)  -- (1.5*60.0:1.14*4.5pt)   -- 
(2.*60.0:9pt)  -- (2.5*60.0:1.14*4.5pt)   -- 
(3.*60.0:9pt)  -- (3.5*60.0:1.14*4.5pt)   -- 
(4.*60.0:9pt)  -- (4.5*60.0:1.14*4.5pt)   -- 
(5.*60.0:9pt)  -- (5.5*60.0:1.14*4.5pt)   -- cycle
(0,0) circle (2pt);
\draw (0,0) circle (9pt);
%\draw[line width=.35pt,even odd rule,rotate=90] 
%(0.*60.0:11pt)  -- (0.5*60.0:1.14*1.00*6.0pt)   -- 
%(1.*60.0:11pt)  -- (1.5*60.0:1.14*1.00*6.0pt)   -- 
%(2.*60.0:11pt)  -- (2.5*60.0:1.14*1.00*6.0pt)   -- 
%(3.*60.0:11pt)  -- (3.5*60.0:1.14*1.00*6.0pt)   -- 
%(4.*60.0:11pt)  -- (4.5*60.0:1.14*1.00*6.0pt)   -- 
%(5.*60.0:11pt)  -- (5.5*60.0:1.14*1.00*6.0pt)   --  cycle;
 }}}



\pgfdeclareplotmark{m5bb}{%
\node[scale=\mFivescale*\AtoBscale*\symscale] at (0,0) {\tikz {%
\draw[fill=black,line width=.3pt,even odd rule,rotate=90] 
(0.*60.0:10pt)  -- (0.5*60.0:1.14*5.0pt)   -- 
(1.*60.0:10pt)  -- (1.5*60.0:1.14*5.0pt)   -- 
(2.*60.0:10pt)  -- (2.5*60.0:1.14*5.0pt)   -- 
(3.*60.0:10pt)  -- (3.5*60.0:1.14*5.0pt)   -- 
(4.*60.0:10pt)  -- (4.5*60.0:1.14*5.0pt)   -- 
(5.*60.0:10pt)  -- (5.5*60.0:1.14*5.0pt)   --  cycle
(0,0) circle (2pt);
\draw[line width=0.6pt,even odd rule,rotate=90] 
(0.5*60.0:9pt)  -- (0.5*60.0:1.14*5.0pt)   
(1.5*60.0:9pt)  -- (1.5*60.0:1.14*5.0pt)   
(2.5*60.0:9pt)  -- (2.5*60.0:1.14*5.0pt)   
(3.5*60.0:9pt)  -- (3.5*60.0:1.14*5.0pt)   
(4.5*60.0:9pt)  -- (4.5*60.0:1.14*5.0pt)   
(5.5*60.0:9pt)  -- (5.5*60.0:1.14*5.0pt)  ;
 }}}

\pgfdeclareplotmark{m5bvb}{%
\node[scale=\mFivescale*\AtoBscale*\symscale] at (0,0) {\tikz {%
\draw[fill=black,line width=.3pt,even odd rule,rotate=90] 
(0.*60.0:9pt)  -- (0.5*60.0:1.14*4.5pt)   -- 
(1.*60.0:9pt)  -- (1.5*60.0:1.14*4.5pt)   -- 
(2.*60.0:9pt)  -- (2.5*60.0:1.14*4.5pt)   -- 
(3.*60.0:9pt)  -- (3.5*60.0:1.14*4.5pt)   -- 
(4.*60.0:9pt)  -- (4.5*60.0:1.14*4.5pt)   -- 
(5.*60.0:9pt)  -- (5.5*60.0:1.14*4.5pt)   --  cycle
(0,0) circle (2pt);
\draw (0,0) circle (9pt);
%\draw[line width=.35pt,even odd rule,rotate=90] 
%(0.*60.0:11pt)  -- (0.5*60.0:1.14*1.00*6.0pt)   -- 
%(1.*60.0:11pt)  -- (1.5*60.0:1.14*1.00*6.0pt)   -- 
%(2.*60.0:11pt)  -- (2.5*60.0:1.14*1.00*6.0pt)   -- 
%(3.*60.0:11pt)  -- (3.5*60.0:1.14*1.00*6.0pt)   -- 
%(4.*60.0:11pt)  -- (4.5*60.0:1.14*1.00*6.0pt)   -- 
%(5.*60.0:11pt)  -- (5.5*60.0:1.14*1.00*6.0pt)   -- cycle;
\draw[line width=0.6pt,even odd rule,rotate=90] 
(0.5*60.0:10pt)  -- (0.5*60.0:1.14*1.00*6.0pt)   
(1.5*60.0:10pt)  -- (1.5*60.0:1.14*1.00*6.0pt)   
(2.5*60.0:10pt)  -- (2.5*60.0:1.14*1.00*6.0pt)   
(3.5*60.0:10pt)  -- (3.5*60.0:1.14*1.00*6.0pt)   
(4.5*60.0:10pt)  -- (4.5*60.0:1.14*1.00*6.0pt)   
(5.5*60.0:10pt)  -- (5.5*60.0:1.14*1.00*6.0pt)  ;
 }}}



\pgfdeclareplotmark{m5c}{%
\node[scale=\mFivescale*\AtoCscale*\symscale] at (0,0) {\tikz {%
\draw[fill=black,line width=.3pt,even odd rule,rotate=90] 
(0.*60.0:10pt)  -- (0.5*60.0:1.14*5.0pt)   -- 
(1.*60.0:10pt)  -- (1.5*60.0:1.14*5.0pt)   -- 
(2.*60.0:10pt)  -- (2.5*60.0:1.14*5.0pt)   -- 
(3.*60.0:10pt)  -- (3.5*60.0:1.14*5.0pt)   -- 
(4.*60.0:10pt)  -- (4.5*60.0:1.14*5.0pt)   -- 
(5.*60.0:10pt)  -- (5.5*60.0:1.14*5.0pt)   -- cycle
(0,0) circle (3.25pt);
 }}}


\pgfdeclareplotmark{m5cv}{%
\node[scale=\mFivescale*\AtoCscale*\symscale] at (0,0) {\tikz {%
\draw[fill=black,line width=.3pt,even odd rule,rotate=90] 
(0.*60.0:9pt)  -- (0.5*60.0:1.14*4.5pt)   -- 
(1.*60.0:9pt)  -- (1.5*60.0:1.14*4.5pt)   -- 
(2.*60.0:9pt)  -- (2.5*60.0:1.14*4.5pt)   -- 
(3.*60.0:9pt)  -- (3.5*60.0:1.14*4.5pt)   -- 
(4.*60.0:9pt)  -- (4.5*60.0:1.14*4.5pt)   -- 
(5.*60.0:9pt)  -- (5.5*60.0:1.14*4.5pt)   --  cycle
(0,0) circle (3.25pt);
\draw (0,0) circle (9pt);
%\draw[line width=.35pt,even odd rule,rotate=90] 
%(0.*60.0:11pt)  -- (0.5*60.0:1.14*1.00*6.0pt)   -- 
%(1.*60.0:11pt)  -- (1.5*60.0:1.14*1.00*6.0pt)   -- 
%(2.*60.0:11pt)  -- (2.5*60.0:1.14*1.00*6.0pt)   -- 
%(3.*60.0:11pt)  -- (3.5*60.0:1.14*1.00*6.0pt)   -- 
%(4.*60.0:11pt)  -- (4.5*60.0:1.14*1.00*6.0pt)   -- 
%(5.*60.0:11pt)  -- (5.5*60.0:1.14*1.00*6.0pt)   -- cycle;
 }}}



\pgfdeclareplotmark{m5cb}{%
\node[scale=\mFivescale*\AtoCscale*\symscale] at (0,0) {\tikz {%
\draw[fill=black,line width=.3pt,even odd rule,rotate=90] 
(0.*60.0:10pt)  -- (0.5*60.0:1.14*5.0pt)   -- 
(1.*60.0:10pt)  -- (1.5*60.0:1.14*5.0pt)   -- 
(2.*60.0:10pt)  -- (2.5*60.0:1.14*5.0pt)   -- 
(3.*60.0:10pt)  -- (3.5*60.0:1.14*5.0pt)   -- 
(4.*60.0:10pt)  -- (4.5*60.0:1.14*5.0pt)   -- 
(5.*60.0:10pt)  -- (5.5*60.0:1.14*5.0pt)   -- cycle
(0,0) circle (3.25pt);
\draw[line width=0.6pt,even odd rule,rotate=90] 
(0.5*60.0:9pt)  -- (0.5*60.0:1.14*5.0pt)   
(1.5*60.0:9pt)  -- (1.5*60.0:1.14*5.0pt)   
(2.5*60.0:9pt)  -- (2.5*60.0:1.14*5.0pt)   
(3.5*60.0:9pt)  -- (3.5*60.0:1.14*5.0pt)   
(4.5*60.0:9pt)  -- (4.5*60.0:1.14*5.0pt)   
(5.5*60.0:9pt)  -- (5.5*60.0:1.14*5.0pt)  ;
 }}}

\pgfdeclareplotmark{m5cvb}{%
\node[scale=\mFivescale*\AtoCscale*\symscale] at (0,0) {\tikz {%
\draw[fill=black,line width=.3pt,even odd rule,rotate=90] 
(0.*60.0:9pt)  -- (0.5*60.0:1.14*4.5pt)   -- 
(1.*60.0:9pt)  -- (1.5*60.0:1.14*4.5pt)   -- 
(2.*60.0:9pt)  -- (2.5*60.0:1.14*4.5pt)   -- 
(3.*60.0:9pt)  -- (3.5*60.0:1.14*4.5pt)   -- 
(4.*60.0:9pt)  -- (4.5*60.0:1.14*4.5pt)   -- 
(5.*60.0:9pt)  -- (5.5*60.0:1.14*4.5pt)   --  cycle
(0,0) circle (3.25pt);
\draw (0,0) circle (9pt);
%\draw[line width=.35pt,even odd rule,rotate=90] 
%(0.*60.0:11pt)  -- (0.5*60.0:1.14*1.00*6.0pt)   -- 
%(1.*60.0:11pt)  -- (1.5*60.0:1.14*1.00*6.0pt)   -- 
%(2.*60.0:11pt)  -- (2.5*60.0:1.14*1.00*6.0pt)   -- 
%(3.*60.0:11pt)  -- (3.5*60.0:1.14*1.00*6.0pt)   -- 
%(4.*60.0:11pt)  -- (4.5*60.0:1.14*1.00*6.0pt)   -- 
%(5.*60.0:11pt)  -- (5.5*60.0:1.14*1.00*6.0pt)   -- cycle;
\draw[line width=0.6pt,even odd rule,rotate=90] 
(0.5*60.0:10pt)  -- (0.5*60.0:1.14*1.00*6.0pt)   
(1.5*60.0:10pt)  -- (1.5*60.0:1.14*1.00*6.0pt)   
(2.5*60.0:10pt)  -- (2.5*60.0:1.14*1.00*6.0pt)   
(3.5*60.0:10pt)  -- (3.5*60.0:1.14*1.00*6.0pt)   
(4.5*60.0:10pt)  -- (4.5*60.0:1.14*1.00*6.0pt)   
(5.5*60.0:10pt)  -- (5.5*60.0:1.14*1.00*6.0pt)   ;
 }}}

\pgfdeclareplotmark{m6a}{%
\node[scale=\mSixscale*\symscale] at (0,0) {\tikz {%
\draw[fill,line width=0.2pt,even odd rule,rotate=90] %
(0:10pt) -- (36:3.8pt)    -- 
(72:10pt) -- (108:3.8pt)  -- 
(144:10pt) -- (180:3.8pt) -- 
(216:10pt) -- (252:3.8pt) --
(288:10pt) -- (324:3.8pt) -- cycle
(0pt,0pt) circle (2.5pt);
% The following is needed to center the darn thing.
\draw[opacity=0] (0pt,0pt) circle (13pt) ;
 }}}


\pgfdeclareplotmark{m6av}{%
\node[scale=\mSixscale*\symscale] at (0,0) {\tikz {%
\draw[fill,line width=0.2pt,even odd rule,rotate=90] %
(0:0.9*10pt) -- (36:0.9*3.8pt)    -- 
(72:0.9*10pt) -- (108:0.9*3.8pt)  -- 
(144:0.9*10pt) -- (180:0.9*3.8pt) -- 
(216:0.9*10pt) -- (252:0.9*3.8pt) --
(288:0.9*10pt) -- (324:0.9*3.8pt) -- cycle
(0pt,0pt) circle (2.5pt);
\draw (0,0) circle (10pt);
%\draw[line width=0.6pt,even odd rule,rotate=90] %
%(0:0.9*13pt) --  (36:0.9*6.0pt)    -- 
%(72:0.9*13pt) -- (108:0.9*6.0pt)  -- 
%(144:0.9*13pt) -- (180:0.9*6.0pt) -- 
%(216:0.9*13pt) -- (252:0.9*6.0pt) --
%(288:0.9*13pt) -- (324:0.9*6.0pt) -- cycle ;
% The following is needed to center the darn thing.
\draw[opacity=0] (0pt,0pt) circle (13pt) ;
 }}}

\pgfdeclareplotmark{m6ab}{%
\node[scale=\mSixscale*\symscale] at (0,0) {\tikz {%
\draw[fill,line width=0.2pt,even odd rule,rotate=90] %
(0:10pt) -- (36:3.8pt)    -- 
(72:10pt) -- (108:3.8pt)  -- 
(144:10pt) -- (180:3.8pt) -- 
(216:10pt) -- (252:3.8pt) --
(288:10pt) -- (324:3.8pt) -- cycle
(0pt,0pt) circle (2.5pt);
\draw[line width=0.9pt,even odd rule,rotate=90] %
(36:9pt) -- (36:3.8pt)    
(108:9pt) -- (108:3.8pt)  
(180:9pt) -- (180:3.8pt) 
(252:9pt) -- (252:3.8pt) 
(324:9pt) -- (324:3.8pt) 
(0pt,0pt) circle (2.5pt);
% The following is needed to center the darn thing.
\draw[opacity=0] (0pt,0pt) circle (13pt) ;
 }}}

\pgfdeclareplotmark{m6avb}{%
\node[scale=\mSixscale*\symscale] at (0,0) {\tikz {%
\draw[fill,line width=0.2pt,even odd rule,rotate=90] %
(0:10pt) -- (36:3.8pt)    -- 
(72:10pt) -- (108:3.8pt)  -- 
(144:10pt) -- (180:3.8pt) -- 
(216:10pt) -- (252:3.8pt) --
(288:10pt) -- (324:3.8pt) -- cycle
(0pt,0pt) circle (2.5pt);
\draw (0,0) circle (10pt);
%\draw[line width=0.6pt,even odd rule,rotate=90] %
%(0:0.9*13pt) --  (36:0.9*6.0pt)    -- 
%(72:0.9*13pt) -- (108:0.9*6.0pt)  -- 
%(144:0.9*13pt) -- (180:0.9*6.0pt) -- 
%(216:0.9*13pt) -- (252:0.9*6.0pt) --
%(288:0.9*13pt) -- (324:0.9*6.0pt) -- cycle ;
\draw[line width=0.9pt,even odd rule,rotate=90] %
(36:12pt) -- (36:6.0pt)    
(108:12pt) -- (108:6.0pt)  
(180:12pt) -- (180:6.0pt) 
(252:12pt) -- (252:6.0pt) 
(324:12pt) -- (324:6.0pt) 
(0pt,0pt) circle (2.5pt);
% The following is needed to center the darn thing.
\draw[opacity=0] (0pt,0pt) circle (13pt) ;
 }}}



\pgfdeclareplotmark{m6b}{%
\node[scale=\mSixscale*\AtoBscale*\symscale] at (0,0) {\tikz {%
\draw[fill,line width=0.2pt,even odd rule,rotate=45] %
(0:10pt)   --  (45:5pt) -- 
(90:10pt)  -- (135:5pt) -- 
(180:10pt) -- (225:5pt) --
(270:10pt) -- (315:5pt) -- cycle 
(0pt,0pt) circle (3pt) ;
 }}}


\pgfdeclareplotmark{m6bv}{%
\node[scale=\mSixscale*\AtoBscale*\symscale] at (0,0) {\tikz {%
\draw[fill,line width=0.2pt,even odd rule,rotate=45] %
(0:10pt)   --  (45:5pt) -- 
(90:10pt)  -- (135:5pt) -- 
(180:10pt) -- (225:5pt) --
(270:10pt) -- (315:5pt) -- cycle 
(0pt,0pt) circle (3pt) ;
\draw (0,0) circle (10pt);
%\draw[line width=0.6pt,even odd rule,rotate=45] %
%(0:13pt)   --  (45:7pt) -- 
%(90:13pt)  -- (135:7pt) -- 
%(180:13pt) -- (225:7pt) --
%(270:13pt) -- (315:7pt) -- cycle ;
 }}}

\pgfdeclareplotmark{m6bb}{%
\node[scale=\mSixscale*\AtoBscale*\symscale] at (0,0) {\tikz {%
\draw[fill,line width=0.2pt,even odd rule,rotate=45] %
(0:10pt)   --  (45:5pt) -- 
(90:10pt)  -- (135:5pt) -- 
(180:10pt) -- (225:5pt) --
(270:10pt) -- (315:5pt) -- cycle 
(0pt,0pt) circle (3pt) ;
\draw[line width=1.2pt,even odd rule,rotate=45] %
(45:9pt)   --  (45:5pt) 
(135:9pt)  -- (135:5pt) 
(225:9pt) -- (225:5pt) 
(315:9pt) -- (315:5pt) ;
 }}}

\pgfdeclareplotmark{m6bvb}{%
\node[scale=\mSixscale*\AtoBscale*\symscale] at (0,0) {\tikz {%
\draw[fill,line width=0.2pt,even odd rule,rotate=45] %
(0:10pt)   --  (45:5pt) -- 
(90:10pt)  -- (135:5pt) -- 
(180:10pt) -- (225:5pt) --
(270:10pt) -- (315:5pt) -- cycle 
(0pt,0pt) circle (3pt) ;
\draw (0,0) circle (10pt);
%\draw[line width=0.6pt,even odd rule,rotate=45] %
%(0:13pt)   --  (45:7pt) -- 
%(90:13pt)  -- (135:7pt) -- 
%(180:13pt) -- (225:7pt) --
%(270:13pt) -- (315:7pt) -- cycle ;
\draw[line width=1.2pt,even odd rule,rotate=45] %
(45:11pt)   --  (45:7pt) 
(135:11pt)  -- (135:7pt) 
(225:11pt)  -- (225:7pt) 
(315:11pt)  -- (315:7pt) ;
 }}}


\pgfdeclareplotmark{m6c}{%
\node[scale=\mSixscale*\AtoCscale*\symscale] at (0,0) {\tikz {%
\draw[fill,line width=0.2pt,even odd rule,rotate=90] %
(0pt,0pt) circle (5pt) ;
 }}}


\pgfdeclareplotmark{m6cv}{%
\node[scale=\mSixscale*\AtoCscale*\symscale] at (0,0) {\tikz {%
\draw[fill,line width=0.2pt,even odd rule,rotate=90] %
(0pt,0pt) circle (5pt) ;
\draw[line width=0.6pt,even odd rule,rotate=90] %
(0pt,0pt) circle (7pt) ;
 }}}

\pgfdeclareplotmark{m6cb}{%
\node[scale=\mSixscale*\AtoCscale*\symscale] at (0,0) {\tikz {%
\draw[fill,line width=0.2pt,even odd rule,rotate=90] %
(0pt,0pt) circle (5pt) ;
\draw[line width=2pt,even odd rule] %
(-8pt,0pt) -- (8pt,0pt) ;
 }}}

\pgfdeclareplotmark{m6cvb}{%
\node[scale=\mSixscale*\AtoCscale*\symscale] at (0,0) {\tikz {%
\draw[fill,line width=0.2pt,even odd rule,rotate=90] %
(0pt,0pt) circle (5pt) ;
\draw[line width=0.6pt,even odd rule,rotate=90] %
(0pt,0pt) circle (7pt) ;
\draw[line width=2pt,even odd rule] %
(-9pt,0pt) -- (-7pt,0pt) 
(9pt,0pt) -- (7pt,0pt) ;
 }}}




%%%%%%%%%%%%%%%%%%%%%%%%%%%%%%%%%%%%%%%%%%%%%%%%%%%%%%%%%%%%%%%%%%%%%%%%%%%%%%%%%%%%%%%%%%
%%%%%%%%%%%%%%%%%%%%%%%%%%%%%%%%%%%%%%%%%%%%%%%%%%%%%%%%%%%%%%%%%%%%%%%%%%%%%%%%%%%%%%%%%%
%   Nebulae markers
%%%%%%%%%%%%%%%%%%%%%%%%%%%%%%%%%%%%%%%%%%%%%%%%%%%%%%%%%%%%%%%%%%%%%%%%%%%%%%%%%%%%%%%%%%
%%%%%%%%%%%%%%%%%%%%%%%%%%%%%%%%%%%%%%%%%%%%%%%%%%%%%%%%%%%%%%%%%%%%%%%%%%%%%%%%%%%%%%%%%%


\pgfdeclarepatternformonly{dense crosshatch dots}{\pgfqpoint{-1pt}{-1pt}}{\pgfqpoint{.6pt}{.6pt}}{\pgfqpoint{.8pt}{.8pt}}%
{
    \pgfpathcircle{\pgfqpoint{0pt}{0pt}}{.3pt}
    \pgfpathcircle{\pgfqpoint{1pt}{1pt}}{.3pt}
    \pgfusepath{fill}
}

% Open Cluster
%\pgfdeclareplotmark{OC}{%
%\node at (0,0) {\tikz { 
%\draw[scale=3,draw opacity=0,line width=.2pt,pattern=dense crosshatch dots] %
%(0:1pt) -- (30:0.75pt)  -- (60:1pt) -- (90:0.75pt)  -- (120:1pt) --  (150:0.75pt)  -- (180:1pt) --
%(210:0.75pt) -- (240:1pt)  -- (270:0.75pt) -- (300:1pt) -- (330:0.75pt)  -- (0:1pt) ;
%};};   
%}

\pgfdeclareplotmark{OC}{%
\node at (0,0) {\tikz {%
%\clip (0:1pt) -- (30:0.75pt)  -- (60:1pt) -- (90:0.75pt)  -- (120:1pt) --  (150:0.75pt)  -- (180:1pt) --
%(210:0.75pt) -- (240:1pt)  -- (270:0.75pt) -- (300:1pt) -- (330:0.75pt)  -- (0:1pt) ;
\filldraw[scale=3,draw opacity=1,line width=.2pt]
(0,0) circle (.1pt)
(0:.4pt) circle (.1pt) (60:.4pt) circle (.1pt) (120:.4pt) circle (.1pt)
(180:.4pt) circle (.1pt) (240:.4pt) circle (.1pt) (300:.4pt) circle (.1pt)
%
(30:.75pt) circle (.1pt) (90:.75pt) circle (.1pt) (150:.75pt) circle (.1pt)
(210:.75pt) circle (.1pt) (270:.75pt) circle (.1pt) (330:.75pt) circle (.1pt)
;
}}}

% Globular Cluster
\pgfdeclareplotmark{GC}{%
\node at (0,0) {\tikz { 
\filldraw[scale=3,draw opacity=1,line width=.2pt]
(0,0) circle (.1pt)
(0:.4pt) circle (.1pt) (60:.4pt) circle (.1pt) (120:.4pt) circle (.1pt)
(180:.4pt) circle (.1pt) (240:.4pt) circle (.1pt) (300:.4pt) circle (.1pt)
%
(30:.75pt) circle (.1pt) (90:.75pt) circle (.1pt) (150:.75pt) circle (.1pt)
(210:.75pt) circle (.1pt) (270:.75pt) circle (.1pt) (330:.75pt) circle (.1pt)
;
\draw[scale=3,line width=.3pt] %
circle (1pt) ;
}}}

% Emission Nebula
\pgfdeclareplotmark{EN}{%
\node[opacity=0] at (0,0) {\tikz { 
\clip (0,0) circle (2.5pt);
\draw[opacity=1,line width=.3pt] %
(-4pt,-1pt) -- (4pt,7pt) 
(-4pt,-2.5pt) -- (4pt,5.5pt)  
(-4pt,-4pt) -- (4pt,4pt) 
(-4pt,-5.5pt) -- (4pt,2.5pt) 
(-4pt,-7pt) -- (4pt,1pt) 
;
}}}

% Nebuluous Cluster
\pgfdeclareplotmark{CN}{%
\node[opacity=0] at (0,0) {\tikz { 
\clip (0,0) circle (2.5pt);
\draw[opacity=1,line width=.3pt] %
(-4pt,-1pt) -- (4pt,7pt) 
(-4pt,-2.5pt) -- (4pt,5.5pt)  
(-4pt,-4pt) -- (4pt,4pt) 
(-4pt,-5.5pt) -- (4pt,2.5pt) 
(-4pt,-7pt) -- (4pt,1pt) ;
}}  ;
\node at (0,0) {\tikz { 
\filldraw[scale=3,draw opacity=1,line width=.2pt]
(0,0) circle (.1pt)
(0:.4pt) circle (.1pt) (60:.4pt) circle (.1pt) (120:.4pt) circle (.1pt)
(180:.4pt) circle (.1pt) (240:.4pt) circle (.1pt) (300:.4pt) circle (.1pt)
%
(30:.75pt) circle (.1pt) (90:.75pt) circle (.1pt) (150:.75pt) circle (.1pt)
(210:.75pt) circle (.1pt) (270:.75pt) circle (.1pt) (330:.75pt) circle (.1pt)
;
}}}


% Planetary Nebula
\pgfdeclareplotmark{PN}{%
\node at (0,0) {\tikz { 
\draw[line width=.3pt] %
circle (2.7pt);
\clip (0,0) circle (2.7pt);
\draw[opacity=1,line width=.3pt] %
(-4pt,-1pt) -- (4pt,7pt) 
(-4pt,-2.5pt) -- (4pt,5.5pt)  
(-4pt,-4pt) -- (4pt,4pt) 
(-4pt,-5.5pt) -- (4pt,2.5pt) 
(-4pt,-7pt) -- (4pt,1pt) 
;
\draw[line width=.3pt,fill=white] %
circle (1.2pt) ;
}}}

% Galaxies
\pgfdeclareplotmark{GAL}{%
\node [rotate=-15] at (0,0) {\tikz {
\draw (0,0) ellipse (4pt and 1.5pt);
\clip (0,0) ellipse (4pt and 1.5pt);
\draw[opacity=1,line width=.3pt] %
(-4pt,-1pt) -- (4pt,7pt) 
(-4pt,-2.5pt) -- (4pt,5.5pt)  
(-4pt,-4pt) -- (4pt,4pt) 
(-4pt,-5.5pt) -- (4pt,2.5pt) 
(-4pt,-7pt) -- (4pt,1pt) 
;
}}}  ;


 


% 
% Colors
\definecolor{Burgundy}{rgb}{.75,.25,.25}
\definecolor{IndianRed4}{HTML}{8B3A3A}
\definecolor{IndianRed3}{HTML}{CD5555}
\definecolor{IndianRed2}{HTML}{EE6363}
\definecolor{IndianRed1}{HTML}{FF6A6A}
\definecolor{GoldenRod1}{HTML}{FFC125}
\definecolor{Orchid}{rgb}{0.85, 0.44, 0.84}
\definecolor{Mauve}{HTML}{B784A7}
\colorlet{cConstellation}{IndianRed4}
%\colorlet{cConstellation}{burgundy}
\definecolor{ForestGreen}{rgb}{0.13, 0.55, 0.13}

\colorlet{cFrame}{black}
\colorlet{cStars}{black}
\colorlet{cAxes}{black}


\colorlet{cProper}{IndianRed2}
\colorlet{cBayer}{IndianRed3}
\colorlet{cFlamsteed}{IndianRed4}

\colorlet{cTransient}{ForestGreen}

\colorlet{cNGC}{Orchid}
\colorlet{cMessier}{Orchid}

%\colorlet{cNGC}{Mauve}
%\colorlet{cMessier}{Mauve}

%\colorlet{cNGC}{IndianRed4}
%\colorlet{cMessier}{IndianRed4}

%\colorlet{cNGC}{green!80!black}
%\colorlet{cMessier}{blue}

\colorlet{cEcliptic}{GoldenRod1!80!black}
\colorlet{cEquator}{gray}

\colorlet{cGrid}{black}

\definecolor{DeepSkyBlue}{HTML}{00BFFF}
% Milky Way  color
\colorlet{cMW}{DeepSkyBlue}
\colorlet{cMW0}{white}
\colorlet{cMW1}{DeepSkyBlue!10}
\colorlet{cMW2}{DeepSkyBlue!14}
\colorlet{cMW3}{DeepSkyBlue!18}
\colorlet{cMW4}{DeepSkyBlue!22}
\colorlet{cMW5}{DeepSkyBlue!26}

%\colorlet{cMW1}{DeepSkyBlue!15}
%\colorlet{cMW2}{DeepSkyBlue!30}
%\colorlet{cMW3}{DeepSkyBlue!45}
%\colorlet{cMW4}{DeepSkyBlue!60}
%\colorlet{cMW5}{DeepSkyBlue!75}

%\colorlet{cMW6}{DeepSkyBlue!40}
%\colorlet{cMW7}{DeepSkyBlue!50}
%\colorlet{cMW8}{DeepSkyBlue!60}
\colorlet{cMW9}{DeepSkyBlue!70}

\def\symscale {1.5}

\def\AtoBscale {0.9}
\def\AtoCscale {0.8}

\def\mOnescale   {0.48}
\def\mTwoscale   {0.38}
\def\mThreescale {0.3}
\def\mFourscale  {0.23}
\def\mFivescale  {0.15}
\def\mSixscale   {0.1}


% Styles:
\pgfplotsset{MW0/.style={mark=,color=cMW0,fill=cMW0,opacity=1}}
\pgfplotsset{MW1/.style={mark=,color=cMW1,fill=cMW1,opacity=1}}
\pgfplotsset{MW2/.style={mark=,color=cMW2,fill=cMW2,opacity=1}}
\pgfplotsset{MW3/.style={mark=,color=cMW3,fill=cMW3,opacity=1}}
\pgfplotsset{MW4/.style={mark=,color=cMW4,fill=cMW4,opacity=1}}
\pgfplotsset{MW5/.style={mark=,color=cMW5,fill=cMW5,opacity=1}}

%\pgfplotsset{MW6/.style={mark=,color=cMW6,opacity=1}}
%\pgfplotsset{MW7/.style={mark=,color=cMW7,opacity=1}}
%\pgfplotsset{MW8/.style={mark=,color=cMW8,opacity=1}}
%\pgfplotsset{MW9/.style={mark=,color=cMW9,opacity=1}}
\pgfplotsset{MWX/.style={mark=,opacity=1}}
\pgfplotsset{MWR/.style={mark=,color=red,opacity=1}}


% Coordinate grid overlay
\pgfplotsset{coordinategrid/.style={grid=major,major grid style={thin,color=cGrid}}}
\tikzset{coordinategrid/.style={thin,color=cGrid}}
% Ecliptics
\tikzset{ecliptics-full/.style={color=cEcliptic,line width=.4pt,mark=,fill=cEcliptic,opacity=1}}
\tikzset{ecliptics-empty/.style={color=cEcliptic,line width=.4pt,mark=,fill=,fill opacity=0}}
\tikzset{ecliptics-label/.style={color=black,font=\scriptsize}}
% Galactic Equator
\tikzset{MWE-full/.style={color=cMW,line width=.4pt,mark=,fill=cMW,opacity=1}}
\tikzset{MWE-empty/.style={color=cMW,line width=.4pt,mark=,fill=,fill opacity=0}}
\tikzset{MWE-label/.style={color=black,font=\scriptsize}}
% Equator
\tikzset{Equator-full/.style={color=cEquator,line width=.4pt,mark=,fill=cEquator,opacity=1}}
\tikzset{Equator-empty/.style={color=cEquator,line width=.4pt,mark=,fill=,fill opacity=0}}
\tikzset{Equator-label/.style={color=black,font=\scriptsize}}
% Transients
\tikzset{transient-label/.style={color=cTransient,font=\bfseries\scriptsize}}
\tikzset{transient/.style={color=cTransient,line width=1pt,mark=,opacity=1}}

% To enable consistent input scripting:
\newcommand\omicron{o}

% Scale of panels:
\newlength{\tendegree}
\setlength{\tendegree}{3cm}
\newlength{\onedegree}
\setlength{\onedegree}{0.1\tendegree}

% General style definitions
\pgfkeys{/pgf/number format/.cd,fixed,precision=4}


% For RA tickmark labeling
\newcommand{\fh}{$^{\rm h}$}
\newcommand{\fm}{$^{\rm m}$}



% Style for constellation boundaries and names:
\pgfplotsset{constellation/.style={thin,mark=,color=cConstellation,dashed}}
\tikzset{constellation-label/.style={color=cConstellation!80,font=\bfseries}}
\tikzset{constellation-boundary/.style={thin,mark=,color=cConstellation,dashed}}


\tikzset{designation-label/.style={color=cProper,font=\bfseries\scriptsize}}
 
% Style for Star markers and names
\tikzset{stars/.style={color=black}}

\tikzset{Proper/.style={color=cProper,font=\bfseries\footnotesize},pin edge={draw opacity=0}}
\tikzset{Bayer/.style={color=cBayer,font=\bfseries\scriptsize},pin edge={draw opacity=0}}
\tikzset{Flaamsted/.style={color=cFlamsteed,font=\bfseries\tiny},pin edge={draw opacity=0}}
% Intersting objects label
\tikzset{interest/.style={color=cProper}}
\tikzset{interest-label/.style={color=cProper,font=\bfseries\tiny}}

% Style for Nebulae markers and names
\tikzset{NGC/.style={color=cNGC,pattern color=cNGC}}
\tikzset{NGC-label/.style={color=cNGC,font=\bfseries\tiny},pin edge={draw opacity=0}}

\tikzset{Caldwell/.style={color=cMessier,pattern color=cMessier}}
\tikzset{Caldwell-label/.style={color=cMessier,font=\bfseries\scriptsize},pin edge={draw opacity=0}}

\tikzset{Messier/.style={color=cMessier,pattern color=cMessier}}
\tikzset{Messier-label/.style={color=cMessier,font=\bfseries\scriptsize},pin edge={draw opacity=0}}
% An extra one for the Messier galaxies in the Virgo-group 
\tikzset{Messier-label-crowded/.style={pin edge={draw opacity=1,color=cMessier,thin},color=cMessier,font=\bfseries\footnotesize}}



\pgfplotsset{tick label style={font=\normalsize}, label style={font=\large}, legend style={font=\large}
  ,every axis/.append style={scale only axis},scaled ticks=false
  % A3paper has size 297 × 420mm
  ,width=11\tendegree,height=8\tendegree
  ,/pgf/number format/set thousands separator={\,}
%
  ,x dir=reverse  ,ytick={-90,-80,...,90},minor y tick num=0
  ,yticklabels={$-90^\circ$,$-80^\circ$,$-70^\circ$,$-60^\circ$,$-50^\circ$,$-40^\circ$,$-30^\circ$,$-20^\circ$,$-10^\circ$,$0^\circ$,
    $+10^\circ$,$+20^\circ$,$+30^\circ$,$+40^\circ$,$+50^\circ$,$+60^\circ$,$+70^\circ$,$+80^\circ$,$+90^\circ$}
%  
}

% RA tickmark labels in hours
\pgfplotsset{RA_in_hours/.style={,xtick={0,10,...,390},minor x tick num=0
      ,xticklabels={0\fh{\small 00}\fm,0\fh{\small 40}\fm,1\fh{\small 20}\fm,2\fh{\small
      00}\fm,2\fh{\small 40}\fm,3\fh{\small 20}\fm,4\fh{\small 00}\fm,4\fh{\small
      40}\fm,5\fh{\small 20}\fm,6\fh{\small 00}\fm,6\fh{\small 40}\fm,7\fh{\small
      20}\fm,8\fh{\small 00}\fm,8\fh{\small 40}\fm,9\fh{\small 20}\fm,10\fh{\small
      00}\fm,10\fh{\small 40}\fm,11\fh{\small 20}\fm,12\fh{\small 00}\fm,12\fh{\small
      40}\fm,13\fh{\small 20}\fm,14\fh{\small 00}\fm,14\fh{\small 40}\fm,15\fh{\small
      20}\fm,16\fh{\small 00}\fm,16\fh{\small 40}\fm,17\fh{\small 20}\fm,18\fh{\small
      00}\fm,18\fh{\small 40}\fm,19\fh{\small 20}\fm,20\fh{\small 00}\fm,20\fh{\small
      40}\fm,21\fh{\small 20}\fm,22\fh{\small 00}\fm,22\fh{\small 40}\fm,23\fh{\small
      20}\fm,0\fh{\small 00}\fm,0\fh{\small 40}\fm,1\fh{\small 20}\fm}}}

% RA tickmark labels in degree
\pgfplotsset{RA_in_deg/.style={,xtick={0,10,...,390},minor x tick num=0%,x dir=reverse,
      ,tick label style={font=\footnotesize},xticklabels={$0^\circ$,$10^\circ$,$20^\circ$,$30^\circ$, 
      $40^\circ$,$50^\circ$,$60^\circ$,$70^\circ$,$80^\circ$,$90^\circ$,$100^\circ$,$110^\circ$,$120^\circ$,
      $130^\circ$,$140^\circ$,$150^\circ$,$160^\circ$,$170^\circ$,$180^\circ$,$190^\circ$,$200^\circ$,
      $210^\circ$,$220^\circ$,$230^\circ$,$240^\circ$,$250^\circ$,$260^\circ$,$270^\circ$,$280^\circ$,
      $290^\circ$,$300^\circ$,$310^\circ$,$320^\circ$,$330^\circ$,$340^\circ$,$350^\circ$,
      $0^\circ$,$10^\circ$,$20^\circ$}  
}} 


\center

\tikzsetnextfilename{TabulaV_S}
%
% Coordinate limit of tab to plot
\pgfplotsset{xmin=180,xmax=290,ymin=-40,ymax=40,y dir=reverse,x dir=normal}

\begin{tikzpicture}
%
% Empty plot to establish anchor points
\begin{axis}[name=base,axis lines=none]\end{axis}

% The Milky Way first, as it should not obstruct coordinate grid 


\begin{axis}[name=milkyWay,axis lines=none]
%
% Broad band
\addplot[MW1] table[x index=0,y index=1] {./MW/V_MWbase.dat}  -- cycle ;
%
% Light areas in braod band
\addplot[MW0] table[x index=0,y index=1] {./MW/V_ol1_10.dat}  -- cycle ;
\addplot[MW0] table[x index=0,y index=1] {./MW/V_ol1_3.dat}  -- cycle ;
\addplot[MW0] table[x index=0,y index=1] {./MW/V_ol1_5.dat}  -- cycle ;
\addplot[MW0] table[x index=0,y index=1] {./MW/V_ol1_6.dat}  -- cycle ;
\addplot[MW0] table[x index=0,y index=1] {./MW/V_ol1_7.dat}  -- cycle ;
\addplot[MW0] table[x index=0,y index=1] {./MW/V_ol1_8.dat}  -- cycle ;
\addplot[MW0] table[x index=0,y index=1] {./MW/V_ol1_9.dat}  -- cycle ;
%
% Darker level I
\addplot[MW2] table[x index=0,y index=1] {./MW/V_ol2_105.dat}  -- cycle ;
\addplot[MW2] table[x index=0,y index=1] {./MW/V_ol2_10.dat}  -- cycle ;
\addplot[MW2] table[x index=0,y index=1] {./MW/V_ol2_110.dat}  -- cycle ;
\addplot[MW2] table[x index=0,y index=1] {./MW/V_ol2_111.dat}  -- cycle ;
\addplot[MW2] table[x index=0,y index=1] {./MW/V_ol2_113.dat}  -- cycle ;
\addplot[MW2] table[x index=0,y index=1] {./MW/V_ol2_13.dat}  -- cycle ;
\addplot[MW2] table[x index=0,y index=1] {./MW/V_ol2_14.dat}  -- cycle ;
\addplot[MW2] table[x index=0,y index=1] {./MW/V_ol2_18.dat}  -- cycle ;
\addplot[MW2] table[x index=0,y index=1] {./MW/V_ol2_1.dat}  -- cycle ;
\addplot[MW2] table[x index=0,y index=1] {./MW/V_ol2_20.dat}  -- cycle ;
\addplot[MW2] table[x index=0,y index=1] {./MW/V_ol2_22.dat}  -- cycle ;
\addplot[MW2] table[x index=0,y index=1] {./MW/V_ol2_23.dat}  -- cycle ;
\addplot[MW2] table[x index=0,y index=1] {./MW/V_ol2_24.dat}  -- cycle ;
\addplot[MW2] table[x index=0,y index=1] {./MW/V_ol2_25.dat}  -- cycle ;
\addplot[MW2] table[x index=0,y index=1] {./MW/V_ol2_29.dat}  -- cycle ;
\addplot[MW2] table[x index=0,y index=1] {./MW/V_ol2_36.dat}  -- cycle ;
\addplot[MW2] table[x index=0,y index=1] {./MW/V_ol2_39.dat}  -- cycle ;
\addplot[MW2] table[x index=0,y index=1] {./MW/V_ol2_3.dat}  -- cycle ;
\addplot[MW2] table[x index=0,y index=1] {./MW/V_ol2_41.dat}  -- cycle ;
\addplot[MW2] table[x index=0,y index=1] {./MW/V_ol2_42.dat}  -- cycle ;
\addplot[MW2] table[x index=0,y index=1] {./MW/V_ol2_43.dat}  -- cycle ;
\addplot[MW2] table[x index=0,y index=1] {./MW/V_ol2_46.dat}  -- cycle ;
\addplot[MW2] table[x index=0,y index=1] {./MW/V_ol2_47.dat}  -- cycle ;
\addplot[MW2] table[x index=0,y index=1] {./MW/V_ol2_4.dat}  -- cycle ;
\addplot[MW2] table[x index=0,y index=1] {./MW/V_ol2_53.dat}  -- cycle ;
\addplot[MW2] table[x index=0,y index=1] {./MW/V_ol2_56.dat}  -- cycle ;
\addplot[MW2] table[x index=0,y index=1] {./MW/V_ol2_57.dat}  -- cycle ;
\addplot[MW2] table[x index=0,y index=1] {./MW/V_ol2_58.dat}  -- cycle ;
\addplot[MW2] table[x index=0,y index=1] {./MW/V_ol2_5.dat}  -- cycle ;
\addplot[MW2] table[x index=0,y index=1] {./MW/V_ol2_60.dat}  -- cycle ;
\addplot[MW2] table[x index=0,y index=1] {./MW/V_ol2_63.dat}  -- cycle ;
\addplot[MW2] table[x index=0,y index=1] {./MW/V_ol2_64.dat}  -- cycle ;
\addplot[MW2] table[x index=0,y index=1] {./MW/V_ol2_65.dat}  -- cycle ;
\addplot[MW2] table[x index=0,y index=1] {./MW/V_ol2_70.dat}  -- cycle ;
\addplot[MW2] table[x index=0,y index=1] {./MW/V_ol2_72.dat}  -- cycle ;
\addplot[MW2] table[x index=0,y index=1] {./MW/V_ol2_77.dat}  -- cycle ;
\addplot[MW2] table[x index=0,y index=1] {./MW/V_ol2_79.dat}  -- cycle ;
\addplot[MW2] table[x index=0,y index=1] {./MW/V_ol2_7.dat}  -- cycle ;
\addplot[MW2] table[x index=0,y index=1] {./MW/V_ol2_81.dat}  -- cycle ;
\addplot[MW2] table[x index=0,y index=1] {./MW/V_ol2_85.dat}  -- cycle ;
\addplot[MW2] table[x index=0,y index=1] {./MW/V_ol2_88.dat}  -- cycle ;
\addplot[MW2] table[x index=0,y index=1] {./MW/V_ol2_89.dat}  -- cycle ;
\addplot[MW2] table[x index=0,y index=1] {./MW/V_ol2_8.dat}  -- cycle ;
\addplot[MW2] table[x index=0,y index=1] {./MW/V_ol2_99.dat}  -- cycle ;
\addplot[MW2] table[x index=0,y index=1] {./MW/V_ol2_9.dat}  -- cycle ;
%
% Light areas in Darker level I
\addplot[MW1] table[x index=0,y index=1] {./MW/V_ol2_91.dat}  -- cycle ;
\addplot[MW1] table[x index=0,y index=1] {./MW/V_ol2_80.dat}  -- cycle ;
\addplot[MW1] table[x index=0,y index=1] {./MW/V_ol2_73.dat}  -- cycle ;
\addplot[MW1] table[x index=0,y index=1] {./MW/V_ol2_75.dat}  -- cycle ;
\addplot[MW1] table[x index=0,y index=1] {./MW/V_ol2_51.dat}  -- cycle ;
\addplot[MW1] table[x index=0,y index=1] {./MW/V_ol2_68.dat}  -- cycle ;
%
% Darker level II
\addplot[MW3] table[x index=0,y index=1] {./MW/V_ol3_11.dat}  -- cycle ;
\addplot[MW3] table[x index=0,y index=1] {./MW/V_ol3_12.dat}  -- cycle ;
\addplot[MW3] table[x index=0,y index=1] {./MW/V_ol3_13.dat}  -- cycle ;
\addplot[MW3] table[x index=0,y index=1] {./MW/V_ol3_16.dat}  -- cycle ;
\addplot[MW3] table[x index=0,y index=1] {./MW/V_ol3_17.dat}  -- cycle ;
\addplot[MW3] table[x index=0,y index=1] {./MW/V_ol3_19.dat}  -- cycle ;
\addplot[MW3] table[x index=0,y index=1] {./MW/V_ol3_1.dat}  -- cycle ;
\addplot[MW3] table[x index=0,y index=1] {./MW/V_ol3_20.dat}  -- cycle ;
\addplot[MW3] table[x index=0,y index=1] {./MW/V_ol3_21.dat}  -- cycle ;
\addplot[MW3] table[x index=0,y index=1] {./MW/V_ol3_22.dat}  -- cycle ;
\addplot[MW3] table[x index=0,y index=1] {./MW/V_ol3_23.dat}  -- cycle ;
\addplot[MW3] table[x index=0,y index=1] {./MW/V_ol3_28.dat}  -- cycle ;
\addplot[MW3] table[x index=0,y index=1] {./MW/V_ol3_2.dat}  -- cycle ;
\addplot[MW3] table[x index=0,y index=1] {./MW/V_ol3_31.dat}  -- cycle ;
\addplot[MW3] table[x index=0,y index=1] {./MW/V_ol3_32.dat}  -- cycle ;
\addplot[MW3] table[x index=0,y index=1] {./MW/V_ol3_34.dat}  -- cycle ;
\addplot[MW3] table[x index=0,y index=1] {./MW/V_ol3_37.dat}  -- cycle ;
\addplot[MW3] table[x index=0,y index=1] {./MW/V_ol3_39.dat}  -- cycle ;
\addplot[MW3] table[x index=0,y index=1] {./MW/V_ol3_40.dat}  -- cycle ;
\addplot[MW3] table[x index=0,y index=1] {./MW/V_ol3_42.dat}  -- cycle ;
\addplot[MW3] table[x index=0,y index=1] {./MW/V_ol3_45.dat}  -- cycle ;
\addplot[MW3] table[x index=0,y index=1] {./MW/V_ol3_46.dat}  -- cycle ;
\addplot[MW3] table[x index=0,y index=1] {./MW/V_ol3_5.dat}  -- cycle ;
\addplot[MW3] table[x index=0,y index=1] {./MW/V_ol3_6.dat}  -- cycle ;
\addplot[MW3] table[x index=0,y index=1] {./MW/V_ol3_7.dat}  -- cycle ;
\addplot[MW3] table[x index=0,y index=1] {./MW/V_ol3_8.dat}  -- cycle ;
%
% Light areas in Darker level II
\addplot[MW2] table[x index=0,y index=1] {./MW/V_ol3_43.dat}  -- cycle ;
\addplot[MW2] table[x index=0,y index=1] {./MW/V_ol3_41.dat}  -- cycle ;
\addplot[MW2] table[x index=0,y index=1] {./MW/V_ol3_33.dat}  -- cycle ;
\addplot[MW2] table[x index=0,y index=1] {./MW/V_ol3_26.dat}  -- cycle ;
\addplot[MW2] table[x index=0,y index=1] {./MW/V_ol3_27.dat}  -- cycle ;
%
% Darker level III
\addplot[MW4] table[x index=0,y index=1] {./MW/V_ol4_10.dat}  -- cycle ;
\addplot[MW4] table[x index=0,y index=1] {./MW/V_ol4_11.dat}  -- cycle ;
\addplot[MW4] table[x index=0,y index=1] {./MW/V_ol4_13.dat}  -- cycle ;
\addplot[MW4] table[x index=0,y index=1] {./MW/V_ol4_17.dat}  -- cycle ;
\addplot[MW4] table[x index=0,y index=1] {./MW/V_ol4_1.dat}  -- cycle ;
\addplot[MW4] table[x index=0,y index=1] {./MW/V_ol4_21.dat}  -- cycle ;
\addplot[MW4] table[x index=0,y index=1] {./MW/V_ol4_22.dat}  -- cycle ;
\addplot[MW4] table[x index=0,y index=1] {./MW/V_ol4_23.dat}  -- cycle ;
\addplot[MW4] table[x index=0,y index=1] {./MW/V_ol4_24.dat}  -- cycle ;
\addplot[MW4] table[x index=0,y index=1] {./MW/V_ol4_25.dat}  -- cycle ;
\addplot[MW4] table[x index=0,y index=1] {./MW/V_ol4_26.dat}  -- cycle ;
\addplot[MW4] table[x index=0,y index=1] {./MW/V_ol4_27.dat}  -- cycle ;
\addplot[MW4] table[x index=0,y index=1] {./MW/V_ol4_2.dat}  -- cycle ;
\addplot[MW4] table[x index=0,y index=1] {./MW/V_ol4_4.dat}  -- cycle ;
\addplot[MW4] table[x index=0,y index=1] {./MW/V_ol4_5.dat}  -- cycle ;
\addplot[MW4] table[x index=0,y index=1] {./MW/V_ol4_7.dat}  -- cycle ;
\addplot[MW4] table[x index=0,y index=1] {./MW/V_ol4_8.dat}  -- cycle ;
%
% Darker level IV
\addplot[MW5] table[x index=0,y index=1] {./MW/V_ol5_1.dat}  -- cycle ;
\addplot[MW5] table[x index=0,y index=1] {./MW/V_ol5_2.dat}  -- cycle ;
\addplot[MW5] table[x index=0,y index=1] {./MW/V_ol5_3.dat}  -- cycle ;
\addplot[MW5] table[x index=0,y index=1] {./MW/V_ol5_6.dat}  -- cycle ;


\end{axis}




% The Galactic coordinate system
%
% Some are commented out because they are surrounded by already plotted borders
% The x-coordinate is in fractional hours, so must be times 15
%

\begin{axis}[name=constellations,at=(base.center),anchor=center,axis lines=none]

\draw[MWE-full] (axis cs:2.66308E+02,-2.88819E+01) -- (axis cs:2.66899E+02,-2.80275E+01) -- (axis cs:2.67093E+02,-2.81309E+01) -- (axis cs:2.66503E+02,-2.89861E+01) -- cycle ; 
\draw[MWE-full] (axis cs:2.67481E+02,-2.71706E+01) -- (axis cs:2.68054E+02,-2.63113E+01) -- (axis cs:2.68246E+02,-2.64131E+01) -- (axis cs:2.67674E+02,-2.72732E+01) -- cycle ; 
\draw[MWE-full] (axis cs:2.68618E+02,-2.54498E+01) -- (axis cs:2.69174E+02,-2.45861E+01) -- (axis cs:2.69365E+02,-2.46865E+01) -- (axis cs:2.68809E+02,-2.55508E+01) -- cycle ; 
\draw[MWE-full] (axis cs:2.69723E+02,-2.37204E+01) -- (axis cs:2.70264E+02,-2.28528E+01) -- (axis cs:2.70453E+02,-2.29518E+01) -- (axis cs:2.69912E+02,-2.38201E+01) -- cycle ; 
\draw[MWE-full] (axis cs:2.70799E+02,-2.19834E+01) -- (axis cs:2.71327E+02,-2.11122E+01) -- (axis cs:2.71514E+02,-2.12100E+01) -- (axis cs:2.70986E+02,-2.20817E+01) -- cycle ; 
\draw[MWE-full] (axis cs:2.71848E+02,-2.02394E+01) -- (axis cs:2.72364E+02,-1.93650E+01) -- (axis cs:2.72550E+02,-1.94617E+01) -- (axis cs:2.72035E+02,-2.03366E+01) -- cycle ; 
\draw[MWE-full] (axis cs:2.72874E+02,-1.84892E+01) -- (axis cs:2.73379E+02,-1.76121E+01) -- (axis cs:2.73564E+02,-1.77078E+01) -- (axis cs:2.73059E+02,-1.85854E+01) -- cycle ; 
\draw[MWE-full] (axis cs:2.73879E+02,-1.67336E+01) -- (axis cs:2.74375E+02,-1.58540E+01) -- (axis cs:2.74558E+02,-1.59488E+01) -- (axis cs:2.74063E+02,-1.68289E+01) -- cycle ; 
\draw[MWE-full] (axis cs:2.74866E+02,-1.49732E+01) -- (axis cs:2.75353E+02,-1.40913E+01) -- (axis cs:2.75535E+02,-1.41854E+01) -- (axis cs:2.75049E+02,-1.50676E+01) -- cycle ; 
\draw[MWE-full] (axis cs:2.75837E+02,-1.32085E+01) -- (axis cs:2.76316E+02,-1.23248E+01) -- (axis cs:2.76498E+02,-1.24181E+01) -- (axis cs:2.76018E+02,-1.33022E+01) -- cycle ; 
\draw[MWE-full] (axis cs:2.76793E+02,-1.14402E+01) -- (axis cs:2.77267E+02,-1.05548E+01) -- (axis cs:2.77447E+02,-1.06476E+01) -- (axis cs:2.76974E+02,-1.15333E+01) -- cycle ; 
\draw[MWE-full] (axis cs:2.77738E+02,-9.66873E+00) -- (axis cs:2.78206E+02,-8.78200E+00) -- (axis cs:2.78385E+02,-8.87430E+00) -- (axis cs:2.77917E+02,-9.76126E+00) -- cycle ; 
\draw[MWE-full] (axis cs:2.78672E+02,-7.89468E+00) -- (axis cs:2.79136E+02,-7.00682E+00) -- (axis cs:2.79315E+02,-7.09872E+00) -- (axis cs:2.78851E+02,-7.98676E+00) -- cycle ; 
\draw[MWE-full] (axis cs:2.79598E+02,-6.11851E+00) -- (axis cs:2.80059E+02,-5.22978E+00) -- (axis cs:2.80237E+02,-5.32137E+00) -- (axis cs:2.79777E+02,-6.21024E+00) -- cycle ; 
\draw[MWE-full] (axis cs:2.80518E+02,-4.34070E+00) -- (axis cs:2.80977E+02,-3.45134E+00) -- (axis cs:2.81155E+02,-3.54271E+00) -- (axis cs:2.80697E+02,-4.43217E+00) -- cycle ; 
\draw[MWE-full] (axis cs:2.81434E+02,-2.56174E+00) -- (axis cs:2.81891E+02,-1.67196E+00) -- (axis cs:2.82069E+02,-1.76320E+00) -- (axis cs:2.81612E+02,-2.65303E+00) -- cycle ; 
\draw[MWE-full] (axis cs:2.82347E+02,-7.82062E-01) -- (axis cs:2.82917E+02,1.07899E-01) -- (axis cs:2.82981E+02,1.66967E-02) -- (axis cs:2.82525E+02,-8.73274E-01) -- cycle ; 
\draw[MWE-full] (axis cs:2.83259E+02,9.97867E-01) -- (axis cs:2.83715E+02,1.88779E+00) -- (axis cs:2.83893E+02,1.79654E+00) -- (axis cs:2.83437E+02,9.06652E-01) -- cycle ; 
\draw[MWE-full] (axis cs:2.84172E+02,2.77760E+00) -- (axis cs:2.84629E+02,3.66725E+00) -- (axis cs:2.84808E+02,3.57586E+00) -- (axis cs:2.84350E+02,2.68629E+00) -- cycle ; 
\draw[MWE-full] (axis cs:2.85088E+02,4.55668E+00) -- (axis cs:2.85547E+02,5.44583E+00) -- (axis cs:2.85725E+02,5.35422E+00) -- (axis cs:2.85266E+02,4.46519E+00) -- cycle ; 
\draw[MWE-full] (axis cs:2.86008E+02,6.33465E+00) -- (axis cs:2.86470E+02,7.22308E+00) -- (axis cs:2.86649E+02,7.13115E+00) -- (axis cs:2.86186E+02,6.24290E+00) -- cycle ; 
\draw[MWE-full] (axis cs:2.86934E+02,8.11105E+00) -- (axis cs:2.87400E+02,8.99851E+00) -- (axis cs:2.87580E+02,8.90618E+00) -- (axis cs:2.87113E+02,8.01893E+00) -- cycle ; 
\draw[MWE-full] (axis cs:2.87868E+02,9.88539E+00) -- (axis cs:2.88339E+02,1.07716E+01) -- (axis cs:2.88519E+02,1.06788E+01) -- (axis cs:2.88048E+02,9.79283E+00) -- cycle ; 
\draw[MWE-full] (axis cs:2.88813E+02,1.16572E+01) -- (axis cs:2.89289E+02,1.25420E+01) -- (axis cs:2.89470E+02,1.24486E+01) -- (axis cs:2.88993E+02,1.15641E+01) -- cycle ; 
\draw[MWE-full] (axis cs:2.89769E+02,1.34259E+01) -- (axis cs:2.90253E+02,1.43090E+01) -- (axis cs:2.90435E+02,1.42149E+01) -- (axis cs:2.89951E+02,1.33322E+01) -- cycle ; 
\draw[MWE-full] (axis cs:2.90740E+02,1.51910E+01) -- (axis cs:2.91231E+02,1.60721E+01) -- (axis cs:2.91414E+02,1.59772E+01) -- (axis cs:2.90922E+02,1.50966E+01) -- cycle ; 
\draw[MWE-full] (axis cs:2.91727E+02,1.69520E+01) -- (axis cs:2.92227E+02,1.78307E+01) -- (axis cs:2.92411E+02,1.77350E+01) -- (axis cs:2.91910E+02,1.68567E+01) -- cycle ; 
\draw[MWE-full] (axis cs:2.92732E+02,1.87082E+01) -- (axis cs:2.93242E+02,1.95843E+01) -- (axis cs:2.93428E+02,1.94875E+01) -- (axis cs:2.92917E+02,1.86119E+01) -- cycle ; 
\draw[MWE-full] (axis cs:2.93758E+02,2.04590E+01) -- (axis cs:2.94280E+02,2.13321E+01) -- (axis cs:2.94467E+02,2.12342E+01) -- (axis cs:2.93944E+02,2.03617E+01) -- cycle ; 
\draw[MWE-full] (axis cs:2.94808E+02,2.22037E+01) -- (axis cs:2.95342E+02,2.30735E+01) -- (axis cs:2.95531E+02,2.29744E+01) -- (axis cs:2.94996E+02,2.21052E+01) -- cycle ; 
\draw[MWE-full] (axis cs:2.95884E+02,2.39415E+01) -- (axis cs:2.96433E+02,2.48076E+01) -- (axis cs:2.96623E+02,2.47072E+01) -- (axis cs:2.96073E+02,2.38417E+01) -- cycle ; 
\draw[MWE-full] (axis cs:2.96989E+02,2.56717E+01) -- (axis cs:2.97554E+02,2.65337E+01) -- (axis cs:2.97746E+02,2.64318E+01) -- (axis cs:2.97180E+02,2.55705E+01) -- cycle ; 
\draw[MWE-full] (axis cs:2.98127E+02,2.73934E+01) -- (axis cs:2.98709E+02,2.82508E+01) -- (axis cs:2.98903E+02,2.81473E+01) -- (axis cs:2.98320E+02,2.72907E+01) -- cycle ; 
\draw[MWE-full] (axis cs:2.99300E+02,2.91056E+01) -- (axis cs:2.99901E+02,2.99579E+01) -- (axis cs:3.00098E+02,2.98527E+01) -- (axis cs:2.99495E+02,2.90013E+01) -- cycle ; 
\draw[MWE-full] (axis cs:3.00513E+02,3.08074E+01) -- (axis cs:3.01135E+02,3.16540E+01) -- (axis cs:3.01334E+02,3.15469E+01) -- (axis cs:3.00710E+02,3.07013E+01) -- cycle ; 
\draw[MWE-full] (axis cs:3.01769E+02,3.24975E+01) -- (axis cs:3.02415E+02,3.33378E+01) -- (axis cs:3.02616E+02,3.32287E+01) -- (axis cs:3.01969E+02,3.23894E+01) -- cycle ; 
\draw[MWE-full] (axis cs:3.03073E+02,3.41747E+01) -- (axis cs:3.03745E+02,3.50081E+01) -- (axis cs:3.03948E+02,3.48968E+01) -- (axis cs:3.03275E+02,3.40646E+01) -- cycle ; 
\draw[MWE-full] (axis cs:3.04430E+02,3.58377E+01) -- (axis cs:3.05130E+02,3.66633E+01) -- (axis cs:3.05335E+02,3.65497E+01) -- (axis cs:3.04634E+02,3.57252E+01) -- cycle ; 
\draw[MWE-full] (axis cs:3.05844E+02,3.74848E+01) -- (axis cs:3.06575E+02,3.83019E+01) -- (axis cs:3.06782E+02,3.81857E+01) -- (axis cs:3.06051E+02,3.73699E+01) -- cycle ; 
\draw[MWE-full] (axis cs:3.07322E+02,3.91143E+01) -- (axis cs:3.08087E+02,3.99220E+01) -- (axis cs:3.08296E+02,3.98031E+01) -- (axis cs:3.07531E+02,3.89969E+01) -- cycle ; 
\draw[MWE-full] (axis cs:3.08869E+02,4.07244E+01) -- (axis cs:3.09671E+02,4.15215E+01) -- (axis cs:3.09883E+02,4.13998E+01) -- (axis cs:3.09080E+02,4.06042E+01) -- cycle ; 
\draw[MWE-full] (axis cs:3.10492E+02,4.23130E+01) -- (axis cs:3.11335E+02,4.30984E+01) -- (axis cs:3.11548E+02,4.29736E+01) -- (axis cs:3.10705E+02,4.21897E+01) -- cycle ; 
\draw[MWE-full] (axis cs:3.12199E+02,4.38775E+01) -- (axis cs:3.13086E+02,4.46499E+01) -- (axis cs:3.13301E+02,4.45218E+01) -- (axis cs:3.12414E+02,4.37511E+01) -- cycle ; 
\draw[MWE-full] (axis cs:3.13996E+02,4.54153E+01) -- (axis cs:3.14932E+02,4.61733E+01) -- (axis cs:3.15149E+02,4.60418E+01) -- (axis cs:3.14213E+02,4.52855E+01) -- cycle ; 
\draw[MWE-full] (axis cs:3.15893E+02,4.69235E+01) -- (axis cs:3.16882E+02,4.76654E+01) -- (axis cs:3.17100E+02,4.75301E+01) -- (axis cs:3.16111E+02,4.67901E+01) -- cycle ; 
\draw[MWE-full] (axis cs:3.17899E+02,4.83986E+01) -- (axis cs:3.18945E+02,4.91226E+01) -- (axis cs:3.19165E+02,4.89834E+01) -- (axis cs:3.18118E+02,4.82614E+01) -- cycle ; 
\draw[MWE-full] (axis cs:3.20023E+02,4.98368E+01) -- (axis cs:3.21132E+02,5.05409E+01) -- (axis cs:3.21351E+02,5.03976E+01) -- (axis cs:3.20242E+02,4.96956E+01) -- cycle ; 
\draw[MWE-full] (axis cs:3.22274E+02,5.12341E+01) -- (axis cs:3.23452E+02,5.19158E+01) -- (axis cs:3.23670E+02,5.17682E+01) -- (axis cs:3.22493E+02,5.10886E+01) -- cycle ; 
\draw[MWE-full] (axis cs:3.24665E+02,5.25855E+01) -- (axis cs:3.25915E+02,5.32424E+01) -- (axis cs:3.26132E+02,5.30902E+01) -- (axis cs:3.24882E+02,5.24356E+01) -- cycle ; 
\draw[MWE-full] (axis cs:3.27204E+02,5.38858E+01) -- (axis cs:3.28533E+02,5.45151E+01) -- (axis cs:3.28746E+02,5.43582E+01) -- (axis cs:3.27419E+02,5.37314E+01) -- cycle ; 
\draw[MWE-full] (axis cs:3.29903E+02,5.51293E+01) -- (axis cs:3.31315E+02,5.57277E+01) -- (axis cs:3.31523E+02,5.55661E+01) -- (axis cs:3.30114E+02,5.49701E+01) -- cycle ; 
\draw[MWE-full] (axis cs:3.32770E+02,5.63095E+01) -- (axis cs:3.34269E+02,5.68737E+01) -- (axis cs:3.34472E+02,5.67072E+01) -- (axis cs:3.32976E+02,5.61454E+01) -- cycle ; 
\draw[MWE-full] (axis cs:3.35814E+02,5.74194E+01) -- (axis cs:3.37405E+02,5.79457E+01) -- (axis cs:3.37598E+02,5.77743E+01) -- (axis cs:3.36012E+02,5.72505E+01) -- cycle ; 
\draw[MWE-full] (axis cs:3.39042E+02,5.84516E+01) -- (axis cs:3.40725E+02,5.89360E+01) -- (axis cs:3.40908E+02,5.87597E+01) -- (axis cs:3.39230E+02,5.82777E+01) -- cycle ; 
\draw[MWE-full] (axis cs:3.42456E+02,5.93980E+01) -- (axis cs:3.44233E+02,5.98365E+01) -- (axis cs:3.44402E+02,5.96555E+01) -- (axis cs:3.42632E+02,5.92193E+01) -- cycle ; 
\draw[MWE-full] (axis cs:3.46056E+02,6.02505E+01) -- (axis cs:3.47924E+02,6.06388E+01) -- (axis cs:3.48077E+02,6.04533E+01) -- (axis cs:3.46217E+02,6.00672E+01) -- cycle ; 
\draw[MWE-full] (axis cs:3.49837E+02,6.10005E+01) -- (axis cs:3.51791E+02,6.13346E+01) -- (axis cs:3.51924E+02,6.11451E+01) -- (axis cs:3.49979E+02,6.08130E+01) -- cycle ; 
\draw[MWE-full] (axis cs:3.53786E+02,6.16401E+01) -- (axis cs:3.55819E+02,6.19160E+01) -- (axis cs:3.55929E+02,6.17229E+01) -- (axis cs:3.53908E+02,6.14487E+01) -- cycle ; 
\draw[MWE-full] (axis cs:3.57886E+02,6.21614E+01) -- (axis cs:3.59986E+02,6.23756E+01) -- (axis cs:7.05566E-02,6.21796E+01) -- (axis cs:3.57984E+02,6.19668E+01) -- cycle ; 
\draw[MWE-full] (axis cs:2.11273E+00,6.25578E+01) -- (axis cs:4.26373E+00,6.27073E+01) -- (axis cs:4.32138E+00,6.25091E+01) -- (axis cs:2.18430E+00,6.23606E+01) -- cycle ; 
\draw[MWE-full] (axis cs:6.43417E+00,6.28237E+01) -- (axis cs:8.61932E+00,6.29064E+01) -- (axis cs:8.64801E+00,6.27068E+01) -- (axis cs:6.47748E+00,6.26247E+01) -- cycle ; 
\draw[MWE-full] (axis cs:1.08143E+01,6.29552E+01) -- (axis cs:1.30140E+01,6.29699E+01) -- (axis cs:1.30130E+01,6.27699E+01) -- (axis cs:1.08282E+01,6.27553E+01) -- cycle ; 
\draw[MWE-full] (axis cs:1.52134E+01,6.29504E+01) -- (axis cs:1.74073E+01,6.28969E+01) -- (axis cs:1.73765E+01,6.26974E+01) -- (axis cs:1.51974E+01,6.27506E+01) -- cycle ; 
\draw[MWE-full] (axis cs:1.95907E+01,6.28094E+01) -- (axis cs:2.17586E+01,6.26884E+01) -- (axis cs:2.16990E+01,6.24903E+01) -- (axis cs:1.95453E+01,6.26105E+01) -- cycle ; 
\draw[MWE-full] (axis cs:2.39066E+01,6.25342E+01) -- (axis cs:2.60301E+01,6.23475E+01) -- (axis cs:2.59432E+01,6.21517E+01) -- (axis cs:2.38331E+01,6.23372E+01) -- cycle ; 
\draw[MWE-full] (axis cs:2.81251E+01,6.21289E+01) -- (axis cs:3.01880E+01,6.18792E+01) -- (axis cs:3.00762E+01,6.16863E+01) -- (axis cs:2.80255E+01,6.19345E+01) -- cycle ; 
\draw[MWE-full] (axis cs:3.22156E+01,6.15991E+01) -- (axis cs:3.42049E+01,6.12895E+01) -- (axis cs:3.40709E+01,6.11003E+01) -- (axis cs:3.20923E+01,6.14079E+01) -- cycle ; 
\draw[MWE-full] (axis cs:3.61537E+01,6.09515E+01) -- (axis cs:3.80599E+01,6.05860E+01) -- (axis cs:3.79064E+01,6.04008E+01) -- (axis cs:3.60096E+01,6.07642E+01) -- cycle ; 
\draw[MWE-full] (axis cs:3.99219E+01,6.01940E+01) -- (axis cs:4.17386E+01,5.97766E+01) -- (axis cs:4.15687E+01,5.95959E+01) -- (axis cs:3.97599E+01,6.00110E+01) -- cycle ; 
\draw[MWE-full] (axis cs:4.35091E+01,5.93347E+01) -- (axis cs:4.52331E+01,5.88695E+01) -- (axis cs:4.50497E+01,5.86936E+01) -- (axis cs:4.33322E+01,5.91564E+01) -- cycle ; 
\draw[MWE-full] (axis cs:4.69102E+01,5.83820E+01) -- (axis cs:4.85407E+01,5.78732E+01) -- (axis cs:4.83464E+01,5.77022E+01) -- (axis cs:4.67210E+01,5.82085E+01) -- cycle ; 
\draw[MWE-full] (axis cs:5.01248E+01,5.73442E+01) -- (axis cs:5.16632E+01,5.67958E+01) -- (axis cs:5.14604E+01,5.66296E+01) -- (axis cs:4.99260E+01,5.71755E+01) -- cycle ; 
\draw[MWE-full] (axis cs:5.31565E+01,5.62291E+01) -- (axis cs:5.46057E+01,5.56449E+01) -- (axis cs:5.43965E+01,5.54837E+01) -- (axis cs:5.29503E+01,5.60654E+01) -- cycle ; 
\draw[MWE-full] (axis cs:5.60117E+01,5.50442E+01) -- (axis cs:5.73757E+01,5.44279E+01) -- (axis cs:5.71620E+01,5.42714E+01) -- (axis cs:5.58001E+01,5.48854E+01) -- cycle ; 
\draw[MWE-full] (axis cs:5.86988E+01,5.37966E+01) -- (axis cs:5.99822E+01,5.31512E+01) -- (axis cs:5.97655E+01,5.29994E+01) -- (axis cs:5.84834E+01,5.36425E+01) -- cycle ; 
\draw[MWE-full] (axis cs:6.12273E+01,5.24925E+01) -- (axis cs:6.24353E+01,5.18211E+01) -- (axis cs:6.22168E+01,5.16738E+01) -- (axis cs:6.10096E+01,5.23429E+01) -- cycle ; 
\draw[MWE-full] (axis cs:6.36076E+01,5.11377E+01) -- (axis cs:6.47454E+01,5.04429E+01) -- (axis cs:6.45262E+01,5.02999E+01) -- (axis cs:6.33886E+01,5.09926E+01) -- cycle ; 
\draw[MWE-full] (axis cs:6.58501E+01,4.97374E+01) -- (axis cs:6.69230E+01,4.90218E+01) -- (axis cs:6.67038E+01,4.88829E+01) -- (axis cs:6.56308E+01,4.95965E+01) -- cycle ; 
\draw[MWE-full] (axis cs:6.79652E+01,4.82965E+01) -- (axis cs:6.89781E+01,4.75620E+01) -- (axis cs:6.87597E+01,4.74270E+01) -- (axis cs:6.77464E+01,4.81596E+01) -- cycle ; 
\draw[MWE-full] (axis cs:6.99629E+01,4.68189E+01) -- (axis cs:7.09207E+01,4.60676E+01) -- (axis cs:7.07035E+01,4.59363E+01) -- (axis cs:6.97450E+01,4.66858E+01) -- cycle ; 
\draw[MWE-full] (axis cs:7.18527E+01,4.53086E+01) -- (axis cs:7.27599E+01,4.45422E+01) -- (axis cs:7.25443E+01,4.44144E+01) -- (axis cs:7.16362E+01,4.51791E+01) -- cycle ; 
\draw[MWE-full] (axis cs:7.36436E+01,4.37688E+01) -- (axis cs:7.45045E+01,4.29888E+01) -- (axis cs:7.42908E+01,4.28642E+01) -- (axis cs:7.34289E+01,4.36426E+01) -- cycle ; 
\draw[MWE-full] (axis cs:7.53439E+01,4.22025E+01) -- (axis cs:7.61626E+01,4.14103E+01) -- (axis cs:7.59510E+01,4.12888E+01) -- (axis cs:7.51312E+01,4.20795E+01) -- cycle ; 
\draw[MWE-full] (axis cs:7.69615E+01,4.06124E+01) -- (axis cs:7.77415E+01,3.98092E+01) -- (axis cs:7.75321E+01,3.96906E+01) -- (axis cs:7.67510E+01,4.04924E+01) -- cycle ; 
\draw[MWE-full] (axis cs:7.85035E+01,3.90009E+01) -- (axis cs:7.92483E+01,3.81877E+01) -- (axis cs:7.90411E+01,3.80718E+01) -- (axis cs:7.82952E+01,3.88836E+01) -- cycle ; 
\draw[MWE-full] (axis cs:7.99766E+01,3.73700E+01) -- (axis cs:8.06892E+01,3.65480E+01) -- (axis cs:8.04843E+01,3.64345E+01) -- (axis cs:7.97705E+01,3.72554E+01) -- cycle ; 
\draw[MWE-full] (axis cs:8.13867E+01,3.57218E+01) -- (axis cs:8.20700E+01,3.48916E+01) -- (axis cs:8.18674E+01,3.47805E+01) -- (axis cs:8.11830E+01,3.56095E+01) -- cycle ; 
\draw[MWE-full] (axis cs:8.27396E+01,3.40578E+01) -- (axis cs:8.33961E+01,3.32204E+01) -- (axis cs:8.31957E+01,3.31114E+01) -- (axis cs:8.25381E+01,3.39478E+01) -- cycle ; 
\draw[MWE-full] (axis cs:8.40402E+01,3.23796E+01) -- (axis cs:8.46724E+01,3.15356E+01) -- (axis cs:8.44742E+01,3.14287E+01) -- (axis cs:8.38409E+01,3.22717E+01) -- cycle ; 
\draw[MWE-full] (axis cs:8.52933E+01,3.06886E+01) -- (axis cs:8.59034E+01,2.98387E+01) -- (axis cs:8.57074E+01,2.97337E+01) -- (axis cs:8.50962E+01,3.05826E+01) -- cycle ; 
\draw[MWE-full] (axis cs:8.65033E+01,2.89861E+01) -- (axis cs:8.70933E+01,2.81309E+01) -- (axis cs:8.68993E+01,2.80275E+01) -- (axis cs:8.63082E+01,2.88819E+01) -- cycle ; 
\draw[MWE-full] (axis cs:8.76740E+01,2.72732E+01) -- (axis cs:8.82459E+01,2.64131E+01) -- (axis cs:8.80537E+01,2.63113E+01) -- (axis cs:8.74810E+01,2.71706E+01) -- cycle ; 
\draw[MWE-full] (axis cs:8.88093E+01,2.55508E+01) -- (axis cs:8.93646E+01,2.46865E+01) -- (axis cs:8.91743E+01,2.45861E+01) -- (axis cs:8.86180E+01,2.54498E+01) -- cycle ; 
\draw[MWE-full] (axis cs:8.99124E+01,2.38201E+01) -- (axis cs:9.04529E+01,2.29518E+01) -- (axis cs:9.02642E+01,2.28528E+01) -- (axis cs:8.97229E+01,2.37204E+01) -- cycle ; 
\draw[MWE-full] (axis cs:9.09865E+01,2.20817E+01) -- (axis cs:9.15135E+01,2.12100E+01) -- (axis cs:9.13265E+01,2.11122E+01) -- (axis cs:9.07986E+01,2.19834E+01) -- cycle ; 
\draw[MWE-full] (axis cs:9.20345E+01,2.03366E+01) -- (axis cs:9.25496E+01,1.94617E+01) -- (axis cs:9.23640E+01,1.93650E+01) -- (axis cs:9.18481E+01,2.02394E+01) -- cycle ; 
\draw[MWE-full] (axis cs:9.30592E+01,1.85854E+01) -- (axis cs:9.35636E+01,1.77078E+01) -- (axis cs:9.33793E+01,1.76121E+01) -- (axis cs:9.28742E+01,1.84892E+01) -- cycle ; 
\draw[MWE-full] (axis cs:9.40631E+01,1.68289E+01) -- (axis cs:9.45580E+01,1.59488E+01) -- (axis cs:9.43750E+01,1.58540E+01) -- (axis cs:9.38795E+01,1.67336E+01) -- cycle ; 
\draw[MWE-full] (axis cs:9.50487E+01,1.50676E+01) -- (axis cs:9.55353E+01,1.41854E+01) -- (axis cs:9.53533E+01,1.40913E+01) -- (axis cs:9.48661E+01,1.49732E+01) -- cycle ; 
\draw[MWE-full] (axis cs:9.60182E+01,1.33022E+01) -- (axis cs:9.64975E+01,1.24181E+01) -- (axis cs:9.63165E+01,1.23248E+01) -- (axis cs:9.58366E+01,1.32085E+01) -- cycle ; 
\draw[MWE-full] (axis cs:9.69737E+01,1.15333E+01) -- (axis cs:9.74469E+01,1.06476E+01) -- (axis cs:9.72666E+01,1.05548E+01) -- (axis cs:9.67931E+01,1.14402E+01) -- cycle ; 
\draw[MWE-full] (axis cs:9.79174E+01,9.76126E+00) -- (axis cs:9.83854E+01,8.87430E+00) -- (axis cs:9.82058E+01,8.78200E+00) -- (axis cs:9.77375E+01,9.66874E+00) -- cycle ; 
\draw[MWE-full] (axis cs:9.88512E+01,7.98676E+00) -- (axis cs:9.93149E+01,7.09872E+00) -- (axis cs:9.91360E+01,7.00683E+00) -- (axis cs:9.86719E+01,7.89468E+00) -- cycle ; 
\draw[MWE-full] (axis cs:9.97769E+01,6.21024E+00) -- (axis cs:1.00237E+02,5.32137E+00) -- (axis cs:1.00059E+02,5.22978E+00) -- (axis cs:9.95982E+01,6.11851E+00) -- cycle ; 
\draw[MWE-full] (axis cs:1.00697E+02,4.43217E+00) -- (axis cs:1.01155E+02,3.54271E+00) -- (axis cs:1.00977E+02,3.45134E+00) -- (axis cs:1.00518E+02,4.34070E+00) -- cycle ; 
\draw[MWE-full] (axis cs:1.01612E+02,2.65303E+00) -- (axis cs:1.02069E+02,1.76320E+00) -- (axis cs:1.01891E+02,1.67196E+00) -- (axis cs:1.01434E+02,2.56174E+00) -- cycle ; 
\draw[MWE-full] (axis cs:1.02525E+02,8.73274E-01) -- (axis cs:1.02981E+02,-1.66966E-02) -- (axis cs:1.02917E+02,-1.07899E-01) -- (axis cs:1.02347E+02,7.82062E-01) -- cycle ; 
\draw[MWE-full] (axis cs:1.03437E+02,-9.06652E-01) -- (axis cs:1.03893E+02,-1.79654E+00) -- (axis cs:1.03715E+02,-1.88779E+00) -- (axis cs:1.03259E+02,-9.97867E-01) -- cycle ; 
\draw[MWE-full] (axis cs:1.04350E+02,-2.68629E+00) -- (axis cs:1.04808E+02,-3.57586E+00) -- (axis cs:1.04629E+02,-3.66725E+00) -- (axis cs:1.04172E+02,-2.77760E+00) -- cycle ; 
\draw[MWE-full] (axis cs:1.05266E+02,-4.46519E+00) -- (axis cs:1.05726E+02,-5.35422E+00) -- (axis cs:1.05547E+02,-5.44583E+00) -- (axis cs:1.05088E+02,-4.55668E+00) -- cycle ; 
\draw[MWE-full] (axis cs:1.06186E+02,-6.24290E+00) -- (axis cs:1.06649E+02,-7.13115E+00) -- (axis cs:1.06470E+02,-7.22308E+00) -- (axis cs:1.06008E+02,-6.33465E+00) -- cycle ; 
\draw[MWE-full] (axis cs:1.07113E+02,-8.01893E+00) -- (axis cs:1.07580E+02,-8.90618E+00) -- (axis cs:1.07400E+02,-8.99851E+00) -- (axis cs:1.06934E+02,-8.11105E+00) -- cycle ; 
\draw[MWE-full] (axis cs:1.08048E+02,-9.79283E+00) -- (axis cs:1.08520E+02,-1.06788E+01) -- (axis cs:1.08339E+02,-1.07716E+01) -- (axis cs:1.07868E+02,-9.88539E+00) -- cycle ; 
\draw[MWE-full] (axis cs:1.08993E+02,-1.15641E+01) -- (axis cs:1.09471E+02,-1.24486E+01) -- (axis cs:1.09289E+02,-1.25420E+01) -- (axis cs:1.08813E+02,-1.16572E+01) -- cycle ; 
\draw[MWE-full] (axis cs:1.09951E+02,-1.33322E+01) -- (axis cs:1.10435E+02,-1.42149E+01) -- (axis cs:1.10253E+02,-1.43090E+01) -- (axis cs:1.09769E+02,-1.34259E+01) -- cycle ; 
\draw[MWE-full] (axis cs:1.10922E+02,-1.50966E+01) -- (axis cs:1.11414E+02,-1.59772E+01) -- (axis cs:1.11231E+02,-1.60721E+01) -- (axis cs:1.10740E+02,-1.51910E+01) -- cycle ; 
\draw[MWE-full] (axis cs:1.11910E+02,-1.68567E+01) -- (axis cs:1.12411E+02,-1.77350E+01) -- (axis cs:1.12227E+02,-1.78307E+01) -- (axis cs:1.11727E+02,-1.69520E+01) -- cycle ; 
\draw[MWE-full] (axis cs:1.12917E+02,-1.86119E+01) -- (axis cs:1.13428E+02,-1.94875E+01) -- (axis cs:1.13242E+02,-1.95843E+01) -- (axis cs:1.12732E+02,-1.87082E+01) -- cycle ; 
\draw[MWE-full] (axis cs:1.13945E+02,-2.03617E+01) -- (axis cs:1.14467E+02,-2.12342E+01) -- (axis cs:1.14280E+02,-2.13321E+01) -- (axis cs:1.13758E+02,-2.04590E+01) -- cycle ; 
\draw[MWE-full] (axis cs:1.14996E+02,-2.21052E+01) -- (axis cs:1.15531E+02,-2.29744E+01) -- (axis cs:1.15342E+02,-2.30735E+01) -- (axis cs:1.14808E+02,-2.22037E+01) -- cycle ; 
\draw[MWE-full] (axis cs:1.16073E+02,-2.38417E+01) -- (axis cs:1.16623E+02,-2.47072E+01) -- (axis cs:1.16433E+02,-2.48076E+01) -- (axis cs:1.15884E+02,-2.39415E+01) -- cycle ; 
\draw[MWE-full] (axis cs:1.17180E+02,-2.55705E+01) -- (axis cs:1.17746E+02,-2.64318E+01) -- (axis cs:1.17554E+02,-2.65337E+01) -- (axis cs:1.16989E+02,-2.56717E+01) -- cycle ; 
\draw[MWE-full] (axis cs:1.18320E+02,-2.72907E+01) -- (axis cs:1.18903E+02,-2.81473E+01) -- (axis cs:1.18709E+02,-2.82508E+01) -- (axis cs:1.18127E+02,-2.73934E+01) -- cycle ; 
\draw[MWE-full] (axis cs:1.19495E+02,-2.90013E+01) -- (axis cs:1.20098E+02,-2.98527E+01) -- (axis cs:1.19901E+02,-2.99579E+01) -- (axis cs:1.19300E+02,-2.91056E+01) -- cycle ; 
\draw[MWE-full] (axis cs:1.20710E+02,-3.07013E+01) -- (axis cs:1.21334E+02,-3.15469E+01) -- (axis cs:1.21135E+02,-3.16540E+01) -- (axis cs:1.20513E+02,-3.08074E+01) -- cycle ; 
\draw[MWE-full] (axis cs:1.21969E+02,-3.23894E+01) -- (axis cs:1.22616E+02,-3.32287E+01) -- (axis cs:1.22415E+02,-3.33378E+01) -- (axis cs:1.21769E+02,-3.24975E+01) -- cycle ; 
\draw[MWE-full] (axis cs:1.23275E+02,-3.40646E+01) -- (axis cs:1.23948E+02,-3.48968E+01) -- (axis cs:1.23745E+02,-3.50081E+01) -- (axis cs:1.23073E+02,-3.41747E+01) -- cycle ; 
\draw[MWE-full] (axis cs:1.24634E+02,-3.57252E+01) -- (axis cs:1.25335E+02,-3.65497E+01) -- (axis cs:1.25130E+02,-3.66633E+01) -- (axis cs:1.24430E+02,-3.58377E+01) -- cycle ; 
\draw[MWE-full] (axis cs:1.26051E+02,-3.73699E+01) -- (axis cs:1.26782E+02,-3.81857E+01) -- (axis cs:1.26575E+02,-3.83019E+01) -- (axis cs:1.25844E+02,-3.74848E+01) -- cycle ; 
\draw[MWE-full] (axis cs:1.27531E+02,-3.89969E+01) -- (axis cs:1.28296E+02,-3.98031E+01) -- (axis cs:1.28087E+02,-3.99220E+01) -- (axis cs:1.27322E+02,-3.91143E+01) -- cycle ; 
\draw[MWE-full] (axis cs:1.29080E+02,-4.06042E+01) -- (axis cs:1.29883E+02,-4.13998E+01) -- (axis cs:1.29671E+02,-4.15215E+01) -- (axis cs:1.28869E+02,-4.07244E+01) -- cycle ; 
\draw[MWE-full] (axis cs:1.30705E+02,-4.21897E+01) -- (axis cs:1.31549E+02,-4.29736E+01) -- (axis cs:1.31335E+02,-4.30984E+01) -- (axis cs:1.30492E+02,-4.23130E+01) -- cycle ; 
\draw[MWE-full] (axis cs:1.32414E+02,-4.37511E+01) -- (axis cs:1.33301E+02,-4.45218E+01) -- (axis cs:1.33086E+02,-4.46499E+01) -- (axis cs:1.32199E+02,-4.38775E+01) -- cycle ; 
\draw[MWE-full] (axis cs:1.34213E+02,-4.52855E+01) -- (axis cs:1.35149E+02,-4.60418E+01) -- (axis cs:1.34932E+02,-4.61733E+01) -- (axis cs:1.33996E+02,-4.54153E+01) -- cycle ; 
\draw[MWE-full] (axis cs:1.36111E+02,-4.67901E+01) -- (axis cs:1.37100E+02,-4.75301E+01) -- (axis cs:1.36882E+02,-4.76654E+01) -- (axis cs:1.35893E+02,-4.69235E+01) -- cycle ; 
\draw[MWE-full] (axis cs:1.38118E+02,-4.82614E+01) -- (axis cs:1.39165E+02,-4.89834E+01) -- (axis cs:1.38945E+02,-4.91226E+01) -- (axis cs:1.37899E+02,-4.83986E+01) -- cycle ; 
\draw[MWE-full] (axis cs:1.40242E+02,-4.96956E+01) -- (axis cs:1.41351E+02,-5.03976E+01) -- (axis cs:1.41132E+02,-5.05409E+01) -- (axis cs:1.40023E+02,-4.98368E+01) -- cycle ; 
\draw[MWE-full] (axis cs:1.42493E+02,-5.10886E+01) -- (axis cs:1.43670E+02,-5.17682E+01) -- (axis cs:1.43452E+02,-5.19158E+01) -- (axis cs:1.42274E+02,-5.12341E+01) -- cycle ; 
\draw[MWE-full] (axis cs:1.44882E+02,-5.24356E+01) -- (axis cs:1.46132E+02,-5.30902E+01) -- (axis cs:1.45915E+02,-5.32424E+01) -- (axis cs:1.44665E+02,-5.25855E+01) -- cycle ; 
\draw[MWE-full] (axis cs:1.47419E+02,-5.37314E+01) -- (axis cs:1.48746E+02,-5.43582E+01) -- (axis cs:1.48533E+02,-5.45151E+01) -- (axis cs:1.47204E+02,-5.38858E+01) -- cycle ; 
\draw[MWE-full] (axis cs:1.50114E+02,-5.49701E+01) -- (axis cs:1.51523E+02,-5.55661E+01) -- (axis cs:1.51315E+02,-5.57277E+01) -- (axis cs:1.49903E+02,-5.51293E+01) -- cycle ; 
\draw[MWE-full] (axis cs:1.52976E+02,-5.61454E+01) -- (axis cs:1.54472E+02,-5.67072E+01) -- (axis cs:1.54269E+02,-5.68737E+01) -- (axis cs:1.52770E+02,-5.63095E+01) -- cycle ; 
\draw[MWE-full] (axis cs:1.56012E+02,-5.72505E+01) -- (axis cs:1.57598E+02,-5.77743E+01) -- (axis cs:1.57405E+02,-5.79457E+01) -- (axis cs:1.55814E+02,-5.74194E+01) -- cycle ; 
\draw[MWE-full] (axis cs:1.59230E+02,-5.82777E+01) -- (axis cs:1.60908E+02,-5.87597E+01) -- (axis cs:1.60725E+02,-5.89360E+01) -- (axis cs:1.59042E+02,-5.84516E+01) -- cycle ; 
\draw[MWE-full] (axis cs:1.62632E+02,-5.92193E+01) -- (axis cs:1.64402E+02,-5.96555E+01) -- (axis cs:1.64233E+02,-5.98365E+01) -- (axis cs:1.62456E+02,-5.93980E+01) -- cycle ; 
\draw[MWE-full] (axis cs:1.66217E+02,-6.00672E+01) -- (axis cs:1.68077E+02,-6.04533E+01) -- (axis cs:1.67924E+02,-6.06388E+01) -- (axis cs:1.66056E+02,-6.02505E+01) -- cycle ; 
\draw[MWE-full] (axis cs:1.69979E+02,-6.08130E+01) -- (axis cs:1.71924E+02,-6.11451E+01) -- (axis cs:1.71791E+02,-6.13346E+01) -- (axis cs:1.69837E+02,-6.10005E+01) -- cycle ; 
\draw[MWE-full] (axis cs:1.73908E+02,-6.14487E+01) -- (axis cs:1.75929E+02,-6.17229E+01) -- (axis cs:1.75819E+02,-6.19160E+01) -- (axis cs:1.73786E+02,-6.16401E+01) -- cycle ; 
\draw[MWE-full] (axis cs:1.77984E+02,-6.19668E+01) -- (axis cs:1.80071E+02,-6.21796E+01) -- (axis cs:1.79986E+02,-6.23756E+01) -- (axis cs:1.77886E+02,-6.21614E+01) -- cycle ; 
\draw[MWE-full] (axis cs:1.82184E+02,-6.23606E+01) -- (axis cs:1.84321E+02,-6.25091E+01) -- (axis cs:1.84264E+02,-6.27073E+01) -- (axis cs:1.82113E+02,-6.25578E+01) -- cycle ; 
\draw[MWE-full] (axis cs:1.86478E+02,-6.26247E+01) -- (axis cs:1.88648E+02,-6.27068E+01) -- (axis cs:1.88619E+02,-6.29064E+01) -- (axis cs:1.86434E+02,-6.28237E+01) -- cycle ; 
\draw[MWE-full] (axis cs:1.90828E+02,-6.27553E+01) -- (axis cs:1.93013E+02,-6.27699E+01) -- (axis cs:1.93014E+02,-6.29699E+01) -- (axis cs:1.90814E+02,-6.29552E+01) -- cycle ; 
\draw[MWE-full] (axis cs:1.95197E+02,-6.27506E+01) -- (axis cs:1.97377E+02,-6.26974E+01) -- (axis cs:1.97407E+02,-6.28969E+01) -- (axis cs:1.95213E+02,-6.29504E+01) -- cycle ; 
\draw[MWE-full] (axis cs:1.99545E+02,-6.26105E+01) -- (axis cs:2.01699E+02,-6.24903E+01) -- (axis cs:2.01759E+02,-6.26884E+01) -- (axis cs:1.99591E+02,-6.28094E+01) -- cycle ; 
\draw[MWE-full] (axis cs:2.03833E+02,-6.23372E+01) -- (axis cs:2.05943E+02,-6.21517E+01) -- (axis cs:2.06030E+02,-6.23475E+01) -- (axis cs:2.03907E+02,-6.25342E+01) -- cycle ; 
\draw[MWE-full] (axis cs:2.08026E+02,-6.19345E+01) -- (axis cs:2.10076E+02,-6.16863E+01) -- (axis cs:2.10188E+02,-6.18792E+01) -- (axis cs:2.08125E+02,-6.21289E+01) -- cycle ; 
\draw[MWE-full] (axis cs:2.12092E+02,-6.14079E+01) -- (axis cs:2.14071E+02,-6.11003E+01) -- (axis cs:2.14205E+02,-6.12895E+01) -- (axis cs:2.12216E+02,-6.15991E+01) -- cycle ; 
\draw[MWE-full] (axis cs:2.16010E+02,-6.07642E+01) -- (axis cs:2.17906E+02,-6.04008E+01) -- (axis cs:2.18060E+02,-6.05860E+01) -- (axis cs:2.16154E+02,-6.09515E+01) -- cycle ; 
\draw[MWE-full] (axis cs:2.19760E+02,-6.00110E+01) -- (axis cs:2.21569E+02,-5.95959E+01) -- (axis cs:2.21739E+02,-5.97766E+01) -- (axis cs:2.19922E+02,-6.01940E+01) -- cycle ; 
\draw[MWE-full] (axis cs:2.23332E+02,-5.91564E+01) -- (axis cs:2.25050E+02,-5.86936E+01) -- (axis cs:2.25233E+02,-5.88695E+01) -- (axis cs:2.23509E+02,-5.93347E+01) -- cycle ; 
\draw[MWE-full] (axis cs:2.26721E+02,-5.82085E+01) -- (axis cs:2.28346E+02,-5.77022E+01) -- (axis cs:2.28541E+02,-5.78732E+01) -- (axis cs:2.26910E+02,-5.83820E+01) -- cycle ; 
\draw[MWE-full] (axis cs:2.29926E+02,-5.71755E+01) -- (axis cs:2.31460E+02,-5.66296E+01) -- (axis cs:2.31663E+02,-5.67958E+01) -- (axis cs:2.30125E+02,-5.73442E+01) -- cycle ; 
\draw[MWE-full] (axis cs:2.32950E+02,-5.60654E+01) -- (axis cs:2.34397E+02,-5.54837E+01) -- (axis cs:2.34606E+02,-5.56449E+01) -- (axis cs:2.33157E+02,-5.62291E+01) -- cycle ; 
\draw[MWE-full] (axis cs:2.35800E+02,-5.48854E+01) -- (axis cs:2.37162E+02,-5.42714E+01) -- (axis cs:2.37376E+02,-5.44279E+01) -- (axis cs:2.36012E+02,-5.50442E+01) -- cycle ; 
\draw[MWE-full] (axis cs:2.38483E+02,-5.36425E+01) -- (axis cs:2.39766E+02,-5.29994E+01) -- (axis cs:2.39982E+02,-5.31512E+01) -- (axis cs:2.38699E+02,-5.37966E+01) -- cycle ; 
\draw[MWE-full] (axis cs:2.41010E+02,-5.23429E+01) -- (axis cs:2.42217E+02,-5.16738E+01) -- (axis cs:2.42435E+02,-5.18211E+01) -- (axis cs:2.41227E+02,-5.24925E+01) -- cycle ; 
\draw[MWE-full] (axis cs:2.43389E+02,-5.09926E+01) -- (axis cs:2.44526E+02,-5.02999E+01) -- (axis cs:2.44745E+02,-5.04429E+01) -- (axis cs:2.43608E+02,-5.11377E+01) -- cycle ; 
\draw[MWE-full] (axis cs:2.45631E+02,-4.95965E+01) -- (axis cs:2.46704E+02,-4.88829E+01) -- (axis cs:2.46923E+02,-4.90218E+01) -- (axis cs:2.45850E+02,-4.97374E+01) -- cycle ; 
\draw[MWE-full] (axis cs:2.47746E+02,-4.81596E+01) -- (axis cs:2.48760E+02,-4.74270E+01) -- (axis cs:2.48978E+02,-4.75620E+01) -- (axis cs:2.47965E+02,-4.82965E+01) -- cycle ; 
\draw[MWE-full] (axis cs:2.49745E+02,-4.66858E+01) -- (axis cs:2.50704E+02,-4.59363E+01) -- (axis cs:2.50921E+02,-4.60676E+01) -- (axis cs:2.49963E+02,-4.68189E+01) -- cycle ; 
\draw[MWE-full] (axis cs:2.51636E+02,-4.51791E+01) -- (axis cs:2.52544E+02,-4.44144E+01) -- (axis cs:2.52760E+02,-4.45422E+01) -- (axis cs:2.51853E+02,-4.53086E+01) -- cycle ; 
\draw[MWE-full] (axis cs:2.53429E+02,-4.36426E+01) -- (axis cs:2.54291E+02,-4.28642E+01) -- (axis cs:2.54505E+02,-4.29888E+01) -- (axis cs:2.53644E+02,-4.37688E+01) -- cycle ; 
\draw[MWE-full] (axis cs:2.55131E+02,-4.20795E+01) -- (axis cs:2.55951E+02,-4.12888E+01) -- (axis cs:2.56163E+02,-4.14103E+01) -- (axis cs:2.55344E+02,-4.22025E+01) -- cycle ; 
\draw[MWE-full] (axis cs:2.56751E+02,-4.04924E+01) -- (axis cs:2.57532E+02,-3.96906E+01) -- (axis cs:2.57742E+02,-3.98092E+01) -- (axis cs:2.56962E+02,-4.06124E+01) -- cycle ; 
\draw[MWE-full] (axis cs:2.58295E+02,-3.88836E+01) -- (axis cs:2.59041E+02,-3.80718E+01) -- (axis cs:2.59248E+02,-3.81877E+01) -- (axis cs:2.58504E+02,-3.90009E+01) -- cycle ; 
\draw[MWE-full] (axis cs:2.59771E+02,-3.72554E+01) -- (axis cs:2.60484E+02,-3.64345E+01) -- (axis cs:2.60689E+02,-3.65480E+01) -- (axis cs:2.59977E+02,-3.73700E+01) -- cycle ; 
\draw[MWE-full] (axis cs:2.61183E+02,-3.56095E+01) -- (axis cs:2.61867E+02,-3.47805E+01) -- (axis cs:2.62070E+02,-3.48916E+01) -- (axis cs:2.61387E+02,-3.57218E+01) -- cycle ; 
\draw[MWE-full] (axis cs:2.62538E+02,-3.39478E+01) -- (axis cs:2.63196E+02,-3.31114E+01) -- (axis cs:2.63396E+02,-3.32204E+01) -- (axis cs:2.62740E+02,-3.40578E+01) -- cycle ; 
\draw[MWE-full] (axis cs:2.63841E+02,-3.22717E+01) -- (axis cs:2.64474E+02,-3.14287E+01) -- (axis cs:2.64672E+02,-3.15356E+01) -- (axis cs:2.64040E+02,-3.23796E+01) -- cycle ; 
\draw[MWE-full] (axis cs:2.65096E+02,-3.05826E+01) -- (axis cs:2.65707E+02,-2.97337E+01) -- (axis cs:2.65903E+02,-2.98387E+01) -- (axis cs:2.65293E+02,-3.06886E+01) -- cycle ; 
\draw[MWE-full] (axis cs:2.66308E+02,-2.88819E+01) -- (axis cs:2.66899E+02,-2.80275E+01) -- (axis cs:2.67093E+02,-2.81309E+01) -- (axis cs:2.66503E+02,-2.89861E+01) -- cycle ; 
\draw[MWE-empty] (axis cs:2.66899E+02,-2.80275E+01) -- (axis cs:2.67481E+02,-2.71706E+01) -- (axis cs:2.67674E+02,-2.72732E+01) -- (axis cs:2.67093E+02,-2.81309E+01) -- cycle ; 
\draw[MWE-empty] (axis cs:2.68054E+02,-2.63113E+01) -- (axis cs:2.68618E+02,-2.54498E+01) -- (axis cs:2.68809E+02,-2.55508E+01) -- (axis cs:2.68246E+02,-2.64131E+01) -- cycle ; 
\draw[MWE-empty] (axis cs:2.69174E+02,-2.45861E+01) -- (axis cs:2.69723E+02,-2.37204E+01) -- (axis cs:2.69912E+02,-2.38201E+01) -- (axis cs:2.69365E+02,-2.46865E+01) -- cycle ; 
\draw[MWE-empty] (axis cs:2.70264E+02,-2.28528E+01) -- (axis cs:2.70799E+02,-2.19834E+01) -- (axis cs:2.70986E+02,-2.20817E+01) -- (axis cs:2.70453E+02,-2.29518E+01) -- cycle ; 
\draw[MWE-empty] (axis cs:2.71327E+02,-2.11122E+01) -- (axis cs:2.71848E+02,-2.02394E+01) -- (axis cs:2.72035E+02,-2.03366E+01) -- (axis cs:2.71514E+02,-2.12100E+01) -- cycle ; 
\draw[MWE-empty] (axis cs:2.72364E+02,-1.93650E+01) -- (axis cs:2.72874E+02,-1.84892E+01) -- (axis cs:2.73059E+02,-1.85854E+01) -- (axis cs:2.72550E+02,-1.94617E+01) -- cycle ; 
\draw[MWE-empty] (axis cs:2.73379E+02,-1.76121E+01) -- (axis cs:2.73879E+02,-1.67336E+01) -- (axis cs:2.74063E+02,-1.68289E+01) -- (axis cs:2.73564E+02,-1.77078E+01) -- cycle ; 
\draw[MWE-empty] (axis cs:2.74375E+02,-1.58540E+01) -- (axis cs:2.74866E+02,-1.49732E+01) -- (axis cs:2.75049E+02,-1.50676E+01) -- (axis cs:2.74558E+02,-1.59488E+01) -- cycle ; 
\draw[MWE-empty] (axis cs:2.75353E+02,-1.40913E+01) -- (axis cs:2.75837E+02,-1.32085E+01) -- (axis cs:2.76018E+02,-1.33022E+01) -- (axis cs:2.75535E+02,-1.41854E+01) -- cycle ; 
\draw[MWE-empty] (axis cs:2.76316E+02,-1.23248E+01) -- (axis cs:2.76793E+02,-1.14402E+01) -- (axis cs:2.76974E+02,-1.15333E+01) -- (axis cs:2.76498E+02,-1.24181E+01) -- cycle ; 
\draw[MWE-empty] (axis cs:2.77267E+02,-1.05548E+01) -- (axis cs:2.77738E+02,-9.66873E+00) -- (axis cs:2.77917E+02,-9.76126E+00) -- (axis cs:2.77447E+02,-1.06476E+01) -- cycle ; 
\draw[MWE-empty] (axis cs:2.78206E+02,-8.78200E+00) -- (axis cs:2.78672E+02,-7.89468E+00) -- (axis cs:2.78851E+02,-7.98676E+00) -- (axis cs:2.78385E+02,-8.87430E+00) -- cycle ; 
\draw[MWE-empty] (axis cs:2.79136E+02,-7.00682E+00) -- (axis cs:2.79598E+02,-6.11851E+00) -- (axis cs:2.79777E+02,-6.21024E+00) -- (axis cs:2.79315E+02,-7.09872E+00) -- cycle ; 
\draw[MWE-empty] (axis cs:2.80059E+02,-5.22978E+00) -- (axis cs:2.80518E+02,-4.34070E+00) -- (axis cs:2.80697E+02,-4.43217E+00) -- (axis cs:2.80237E+02,-5.32137E+00) -- cycle ; 
\draw[MWE-empty] (axis cs:2.80977E+02,-3.45134E+00) -- (axis cs:2.81434E+02,-2.56174E+00) -- (axis cs:2.81612E+02,-2.65303E+00) -- (axis cs:2.81155E+02,-3.54271E+00) -- cycle ; 
\draw[MWE-empty] (axis cs:2.81891E+02,-1.67196E+00) -- (axis cs:2.82347E+02,-7.82062E-01) -- (axis cs:2.82525E+02,-8.73274E-01) -- (axis cs:2.82069E+02,-1.76320E+00) -- cycle ; 
\draw[MWE-empty] (axis cs:2.82917E+02,1.07899E-01) -- (axis cs:2.83259E+02,9.97867E-01) -- (axis cs:2.83437E+02,9.06652E-01) -- (axis cs:2.82981E+02,1.66967E-02) -- cycle ; 
\draw[MWE-empty] (axis cs:2.83715E+02,1.88779E+00) -- (axis cs:2.84172E+02,2.77760E+00) -- (axis cs:2.84350E+02,2.68629E+00) -- (axis cs:2.83893E+02,1.79654E+00) -- cycle ; 
\draw[MWE-empty] (axis cs:2.84629E+02,3.66725E+00) -- (axis cs:2.85088E+02,4.55668E+00) -- (axis cs:2.85266E+02,4.46519E+00) -- (axis cs:2.84808E+02,3.57586E+00) -- cycle ; 
\draw[MWE-empty] (axis cs:2.85547E+02,5.44583E+00) -- (axis cs:2.86008E+02,6.33465E+00) -- (axis cs:2.86186E+02,6.24290E+00) -- (axis cs:2.85725E+02,5.35422E+00) -- cycle ; 
\draw[MWE-empty] (axis cs:2.86470E+02,7.22308E+00) -- (axis cs:2.86934E+02,8.11105E+00) -- (axis cs:2.87113E+02,8.01893E+00) -- (axis cs:2.86649E+02,7.13115E+00) -- cycle ; 
\draw[MWE-empty] (axis cs:2.87400E+02,8.99851E+00) -- (axis cs:2.87868E+02,9.88539E+00) -- (axis cs:2.88048E+02,9.79283E+00) -- (axis cs:2.87580E+02,8.90618E+00) -- cycle ; 
\draw[MWE-empty] (axis cs:2.88339E+02,1.07716E+01) -- (axis cs:2.88813E+02,1.16572E+01) -- (axis cs:2.88993E+02,1.15641E+01) -- (axis cs:2.88519E+02,1.06788E+01) -- cycle ; 
\draw[MWE-empty] (axis cs:2.89289E+02,1.25420E+01) -- (axis cs:2.89769E+02,1.34259E+01) -- (axis cs:2.89951E+02,1.33322E+01) -- (axis cs:2.89470E+02,1.24486E+01) -- cycle ; 
\draw[MWE-empty] (axis cs:2.90253E+02,1.43090E+01) -- (axis cs:2.90740E+02,1.51910E+01) -- (axis cs:2.90922E+02,1.50966E+01) -- (axis cs:2.90435E+02,1.42149E+01) -- cycle ; 
\draw[MWE-empty] (axis cs:2.91231E+02,1.60721E+01) -- (axis cs:2.91727E+02,1.69520E+01) -- (axis cs:2.91910E+02,1.68567E+01) -- (axis cs:2.91414E+02,1.59772E+01) -- cycle ; 
\draw[MWE-empty] (axis cs:2.92227E+02,1.78307E+01) -- (axis cs:2.92732E+02,1.87082E+01) -- (axis cs:2.92917E+02,1.86119E+01) -- (axis cs:2.92411E+02,1.77350E+01) -- cycle ; 
\draw[MWE-empty] (axis cs:2.93242E+02,1.95843E+01) -- (axis cs:2.93758E+02,2.04590E+01) -- (axis cs:2.93944E+02,2.03617E+01) -- (axis cs:2.93428E+02,1.94875E+01) -- cycle ; 
\draw[MWE-empty] (axis cs:2.94280E+02,2.13321E+01) -- (axis cs:2.94808E+02,2.22037E+01) -- (axis cs:2.94996E+02,2.21052E+01) -- (axis cs:2.94467E+02,2.12342E+01) -- cycle ; 
\draw[MWE-empty] (axis cs:2.95342E+02,2.30735E+01) -- (axis cs:2.95884E+02,2.39415E+01) -- (axis cs:2.96073E+02,2.38417E+01) -- (axis cs:2.95531E+02,2.29744E+01) -- cycle ; 
\draw[MWE-empty] (axis cs:2.96433E+02,2.48076E+01) -- (axis cs:2.96989E+02,2.56717E+01) -- (axis cs:2.97180E+02,2.55705E+01) -- (axis cs:2.96623E+02,2.47072E+01) -- cycle ; 
\draw[MWE-empty] (axis cs:2.97554E+02,2.65337E+01) -- (axis cs:2.98127E+02,2.73934E+01) -- (axis cs:2.98320E+02,2.72907E+01) -- (axis cs:2.97746E+02,2.64318E+01) -- cycle ; 
\draw[MWE-empty] (axis cs:2.98709E+02,2.82508E+01) -- (axis cs:2.99300E+02,2.91056E+01) -- (axis cs:2.99495E+02,2.90013E+01) -- (axis cs:2.98903E+02,2.81473E+01) -- cycle ; 
\draw[MWE-empty] (axis cs:2.99901E+02,2.99579E+01) -- (axis cs:3.00513E+02,3.08074E+01) -- (axis cs:3.00710E+02,3.07013E+01) -- (axis cs:3.00098E+02,2.98527E+01) -- cycle ; 
\draw[MWE-empty] (axis cs:3.01135E+02,3.16540E+01) -- (axis cs:3.01769E+02,3.24975E+01) -- (axis cs:3.01969E+02,3.23894E+01) -- (axis cs:3.01334E+02,3.15469E+01) -- cycle ; 
\draw[MWE-empty] (axis cs:3.02415E+02,3.33378E+01) -- (axis cs:3.03073E+02,3.41747E+01) -- (axis cs:3.03275E+02,3.40646E+01) -- (axis cs:3.02616E+02,3.32287E+01) -- cycle ; 
\draw[MWE-empty] (axis cs:3.03745E+02,3.50081E+01) -- (axis cs:3.04430E+02,3.58377E+01) -- (axis cs:3.04634E+02,3.57252E+01) -- (axis cs:3.03948E+02,3.48968E+01) -- cycle ; 
\draw[MWE-empty] (axis cs:3.05130E+02,3.66633E+01) -- (axis cs:3.05844E+02,3.74848E+01) -- (axis cs:3.06051E+02,3.73699E+01) -- (axis cs:3.05335E+02,3.65497E+01) -- cycle ; 
\draw[MWE-empty] (axis cs:3.06575E+02,3.83019E+01) -- (axis cs:3.07322E+02,3.91143E+01) -- (axis cs:3.07531E+02,3.89969E+01) -- (axis cs:3.06782E+02,3.81857E+01) -- cycle ; 
\draw[MWE-empty] (axis cs:3.08087E+02,3.99220E+01) -- (axis cs:3.08869E+02,4.07244E+01) -- (axis cs:3.09080E+02,4.06042E+01) -- (axis cs:3.08296E+02,3.98031E+01) -- cycle ; 
\draw[MWE-empty] (axis cs:3.09671E+02,4.15215E+01) -- (axis cs:3.10492E+02,4.23130E+01) -- (axis cs:3.10705E+02,4.21897E+01) -- (axis cs:3.09883E+02,4.13998E+01) -- cycle ; 
\draw[MWE-empty] (axis cs:3.11335E+02,4.30984E+01) -- (axis cs:3.12199E+02,4.38775E+01) -- (axis cs:3.12414E+02,4.37511E+01) -- (axis cs:3.11548E+02,4.29736E+01) -- cycle ; 
\draw[MWE-empty] (axis cs:3.13086E+02,4.46499E+01) -- (axis cs:3.13996E+02,4.54153E+01) -- (axis cs:3.14213E+02,4.52855E+01) -- (axis cs:3.13301E+02,4.45218E+01) -- cycle ; 
\draw[MWE-empty] (axis cs:3.14932E+02,4.61733E+01) -- (axis cs:3.15893E+02,4.69235E+01) -- (axis cs:3.16111E+02,4.67901E+01) -- (axis cs:3.15149E+02,4.60418E+01) -- cycle ; 
\draw[MWE-empty] (axis cs:3.16882E+02,4.76654E+01) -- (axis cs:3.17899E+02,4.83986E+01) -- (axis cs:3.18118E+02,4.82614E+01) -- (axis cs:3.17100E+02,4.75301E+01) -- cycle ; 
\draw[MWE-empty] (axis cs:3.18945E+02,4.91226E+01) -- (axis cs:3.20023E+02,4.98368E+01) -- (axis cs:3.20242E+02,4.96956E+01) -- (axis cs:3.19165E+02,4.89834E+01) -- cycle ; 
\draw[MWE-empty] (axis cs:3.21132E+02,5.05409E+01) -- (axis cs:3.22274E+02,5.12341E+01) -- (axis cs:3.22493E+02,5.10886E+01) -- (axis cs:3.21351E+02,5.03976E+01) -- cycle ; 
\draw[MWE-empty] (axis cs:3.23452E+02,5.19158E+01) -- (axis cs:3.24665E+02,5.25855E+01) -- (axis cs:3.24882E+02,5.24356E+01) -- (axis cs:3.23670E+02,5.17682E+01) -- cycle ; 
\draw[MWE-empty] (axis cs:3.25915E+02,5.32424E+01) -- (axis cs:3.27204E+02,5.38858E+01) -- (axis cs:3.27419E+02,5.37314E+01) -- (axis cs:3.26132E+02,5.30902E+01) -- cycle ; 
\draw[MWE-empty] (axis cs:3.28533E+02,5.45151E+01) -- (axis cs:3.29903E+02,5.51293E+01) -- (axis cs:3.30114E+02,5.49701E+01) -- (axis cs:3.28746E+02,5.43582E+01) -- cycle ; 
\draw[MWE-empty] (axis cs:3.31315E+02,5.57277E+01) -- (axis cs:3.32770E+02,5.63095E+01) -- (axis cs:3.32976E+02,5.61454E+01) -- (axis cs:3.31523E+02,5.55661E+01) -- cycle ; 
\draw[MWE-empty] (axis cs:3.34269E+02,5.68737E+01) -- (axis cs:3.35814E+02,5.74194E+01) -- (axis cs:3.36012E+02,5.72505E+01) -- (axis cs:3.34472E+02,5.67072E+01) -- cycle ; 
\draw[MWE-empty] (axis cs:3.37405E+02,5.79457E+01) -- (axis cs:3.39042E+02,5.84516E+01) -- (axis cs:3.39230E+02,5.82777E+01) -- (axis cs:3.37598E+02,5.77743E+01) -- cycle ; 
\draw[MWE-empty] (axis cs:3.40725E+02,5.89360E+01) -- (axis cs:3.42456E+02,5.93980E+01) -- (axis cs:3.42632E+02,5.92193E+01) -- (axis cs:3.40908E+02,5.87597E+01) -- cycle ; 
\draw[MWE-empty] (axis cs:3.44233E+02,5.98365E+01) -- (axis cs:3.46056E+02,6.02505E+01) -- (axis cs:3.46217E+02,6.00672E+01) -- (axis cs:3.44402E+02,5.96555E+01) -- cycle ; 
\draw[MWE-empty] (axis cs:3.47924E+02,6.06388E+01) -- (axis cs:3.49837E+02,6.10005E+01) -- (axis cs:3.49979E+02,6.08130E+01) -- (axis cs:3.48077E+02,6.04533E+01) -- cycle ; 
\draw[MWE-empty] (axis cs:3.51791E+02,6.13346E+01) -- (axis cs:3.53786E+02,6.16401E+01) -- (axis cs:3.53908E+02,6.14487E+01) -- (axis cs:3.51924E+02,6.11451E+01) -- cycle ; 
\draw[MWE-empty] (axis cs:3.55819E+02,6.19160E+01) -- (axis cs:3.57886E+02,6.21614E+01) -- (axis cs:3.57984E+02,6.19668E+01) -- (axis cs:3.55929E+02,6.17229E+01) -- cycle ; 
\draw[MWE-empty] (axis cs:3.59986E+02,6.23756E+01) -- (axis cs:2.11273E+00,6.25578E+01) -- (axis cs:2.18430E+00,6.23606E+01) -- (axis cs:7.05566E-02,6.21796E+01) -- cycle ; 
\draw[MWE-empty] (axis cs:4.26373E+00,6.27073E+01) -- (axis cs:6.43417E+00,6.28237E+01) -- (axis cs:6.47748E+00,6.26247E+01) -- (axis cs:4.32138E+00,6.25091E+01) -- cycle ; 
\draw[MWE-empty] (axis cs:8.61932E+00,6.29064E+01) -- (axis cs:1.08143E+01,6.29552E+01) -- (axis cs:1.08282E+01,6.27553E+01) -- (axis cs:8.64801E+00,6.27068E+01) -- cycle ; 
\draw[MWE-empty] (axis cs:1.30140E+01,6.29699E+01) -- (axis cs:1.52134E+01,6.29504E+01) -- (axis cs:1.51974E+01,6.27506E+01) -- (axis cs:1.30130E+01,6.27699E+01) -- cycle ; 
\draw[MWE-empty] (axis cs:1.74073E+01,6.28969E+01) -- (axis cs:1.95907E+01,6.28094E+01) -- (axis cs:1.95453E+01,6.26105E+01) -- (axis cs:1.73765E+01,6.26974E+01) -- cycle ; 
\draw[MWE-empty] (axis cs:2.17586E+01,6.26884E+01) -- (axis cs:2.39066E+01,6.25342E+01) -- (axis cs:2.38331E+01,6.23372E+01) -- (axis cs:2.16990E+01,6.24903E+01) -- cycle ; 
\draw[MWE-empty] (axis cs:2.60301E+01,6.23475E+01) -- (axis cs:2.81251E+01,6.21289E+01) -- (axis cs:2.80255E+01,6.19345E+01) -- (axis cs:2.59432E+01,6.21517E+01) -- cycle ; 
\draw[MWE-empty] (axis cs:3.01880E+01,6.18792E+01) -- (axis cs:3.22156E+01,6.15991E+01) -- (axis cs:3.20923E+01,6.14079E+01) -- (axis cs:3.00762E+01,6.16863E+01) -- cycle ; 
\draw[MWE-empty] (axis cs:3.42049E+01,6.12895E+01) -- (axis cs:3.61537E+01,6.09515E+01) -- (axis cs:3.60096E+01,6.07642E+01) -- (axis cs:3.40709E+01,6.11003E+01) -- cycle ; 
\draw[MWE-empty] (axis cs:3.80599E+01,6.05860E+01) -- (axis cs:3.99219E+01,6.01940E+01) -- (axis cs:3.97599E+01,6.00110E+01) -- (axis cs:3.79064E+01,6.04008E+01) -- cycle ; 
\draw[MWE-empty] (axis cs:4.17386E+01,5.97766E+01) -- (axis cs:4.35091E+01,5.93347E+01) -- (axis cs:4.33322E+01,5.91564E+01) -- (axis cs:4.15687E+01,5.95959E+01) -- cycle ; 
\draw[MWE-empty] (axis cs:4.52331E+01,5.88695E+01) -- (axis cs:4.69102E+01,5.83820E+01) -- (axis cs:4.67210E+01,5.82085E+01) -- (axis cs:4.50497E+01,5.86936E+01) -- cycle ; 
\draw[MWE-empty] (axis cs:4.85407E+01,5.78732E+01) -- (axis cs:5.01248E+01,5.73442E+01) -- (axis cs:4.99260E+01,5.71755E+01) -- (axis cs:4.83464E+01,5.77022E+01) -- cycle ; 
\draw[MWE-empty] (axis cs:5.16632E+01,5.67958E+01) -- (axis cs:5.31565E+01,5.62291E+01) -- (axis cs:5.29503E+01,5.60654E+01) -- (axis cs:5.14604E+01,5.66296E+01) -- cycle ; 
\draw[MWE-empty] (axis cs:5.46057E+01,5.56449E+01) -- (axis cs:5.60117E+01,5.50442E+01) -- (axis cs:5.58001E+01,5.48854E+01) -- (axis cs:5.43965E+01,5.54837E+01) -- cycle ; 
\draw[MWE-empty] (axis cs:5.73757E+01,5.44279E+01) -- (axis cs:5.86988E+01,5.37966E+01) -- (axis cs:5.84834E+01,5.36425E+01) -- (axis cs:5.71620E+01,5.42714E+01) -- cycle ; 
\draw[MWE-empty] (axis cs:5.99822E+01,5.31512E+01) -- (axis cs:6.12273E+01,5.24925E+01) -- (axis cs:6.10096E+01,5.23429E+01) -- (axis cs:5.97655E+01,5.29994E+01) -- cycle ; 
\draw[MWE-empty] (axis cs:6.24353E+01,5.18211E+01) -- (axis cs:6.36076E+01,5.11377E+01) -- (axis cs:6.33886E+01,5.09926E+01) -- (axis cs:6.22168E+01,5.16738E+01) -- cycle ; 
\draw[MWE-empty] (axis cs:6.47454E+01,5.04429E+01) -- (axis cs:6.58501E+01,4.97374E+01) -- (axis cs:6.56308E+01,4.95965E+01) -- (axis cs:6.45262E+01,5.02999E+01) -- cycle ; 
\draw[MWE-empty] (axis cs:6.69230E+01,4.90218E+01) -- (axis cs:6.79652E+01,4.82965E+01) -- (axis cs:6.77464E+01,4.81596E+01) -- (axis cs:6.67038E+01,4.88829E+01) -- cycle ; 
\draw[MWE-empty] (axis cs:6.89781E+01,4.75620E+01) -- (axis cs:6.99629E+01,4.68189E+01) -- (axis cs:6.97450E+01,4.66858E+01) -- (axis cs:6.87597E+01,4.74270E+01) -- cycle ; 
\draw[MWE-empty] (axis cs:7.09207E+01,4.60676E+01) -- (axis cs:7.18527E+01,4.53086E+01) -- (axis cs:7.16362E+01,4.51791E+01) -- (axis cs:7.07035E+01,4.59363E+01) -- cycle ; 
\draw[MWE-empty] (axis cs:7.27599E+01,4.45422E+01) -- (axis cs:7.36436E+01,4.37688E+01) -- (axis cs:7.34289E+01,4.36426E+01) -- (axis cs:7.25443E+01,4.44144E+01) -- cycle ; 
\draw[MWE-empty] (axis cs:7.45045E+01,4.29888E+01) -- (axis cs:7.53439E+01,4.22025E+01) -- (axis cs:7.51312E+01,4.20795E+01) -- (axis cs:7.42908E+01,4.28642E+01) -- cycle ; 
\draw[MWE-empty] (axis cs:7.61626E+01,4.14103E+01) -- (axis cs:7.69615E+01,4.06124E+01) -- (axis cs:7.67510E+01,4.04924E+01) -- (axis cs:7.59510E+01,4.12888E+01) -- cycle ; 
\draw[MWE-empty] (axis cs:7.77415E+01,3.98092E+01) -- (axis cs:7.85035E+01,3.90009E+01) -- (axis cs:7.82952E+01,3.88836E+01) -- (axis cs:7.75321E+01,3.96906E+01) -- cycle ; 
\draw[MWE-empty] (axis cs:7.92483E+01,3.81877E+01) -- (axis cs:7.99766E+01,3.73700E+01) -- (axis cs:7.97705E+01,3.72554E+01) -- (axis cs:7.90411E+01,3.80718E+01) -- cycle ; 
\draw[MWE-empty] (axis cs:8.06892E+01,3.65480E+01) -- (axis cs:8.13867E+01,3.57218E+01) -- (axis cs:8.11830E+01,3.56095E+01) -- (axis cs:8.04843E+01,3.64345E+01) -- cycle ; 
\draw[MWE-empty] (axis cs:8.20700E+01,3.48916E+01) -- (axis cs:8.27396E+01,3.40578E+01) -- (axis cs:8.25381E+01,3.39478E+01) -- (axis cs:8.18674E+01,3.47805E+01) -- cycle ; 
\draw[MWE-empty] (axis cs:8.33961E+01,3.32204E+01) -- (axis cs:8.40402E+01,3.23796E+01) -- (axis cs:8.38409E+01,3.22717E+01) -- (axis cs:8.31957E+01,3.31114E+01) -- cycle ; 
\draw[MWE-empty] (axis cs:8.46724E+01,3.15356E+01) -- (axis cs:8.52933E+01,3.06886E+01) -- (axis cs:8.50962E+01,3.05826E+01) -- (axis cs:8.44742E+01,3.14287E+01) -- cycle ; 
\draw[MWE-empty] (axis cs:8.59034E+01,2.98387E+01) -- (axis cs:8.65033E+01,2.89861E+01) -- (axis cs:8.63082E+01,2.88819E+01) -- (axis cs:8.57074E+01,2.97337E+01) -- cycle ; 
\draw[MWE-empty] (axis cs:8.70933E+01,2.81309E+01) -- (axis cs:8.76740E+01,2.72732E+01) -- (axis cs:8.74810E+01,2.71706E+01) -- (axis cs:8.68993E+01,2.80275E+01) -- cycle ; 
\draw[MWE-empty] (axis cs:8.82459E+01,2.64131E+01) -- (axis cs:8.88093E+01,2.55508E+01) -- (axis cs:8.86180E+01,2.54498E+01) -- (axis cs:8.80537E+01,2.63113E+01) -- cycle ; 
\draw[MWE-empty] (axis cs:8.93646E+01,2.46865E+01) -- (axis cs:8.99124E+01,2.38201E+01) -- (axis cs:8.97229E+01,2.37204E+01) -- (axis cs:8.91743E+01,2.45861E+01) -- cycle ; 
\draw[MWE-empty] (axis cs:9.04529E+01,2.29518E+01) -- (axis cs:9.09865E+01,2.20817E+01) -- (axis cs:9.07986E+01,2.19834E+01) -- (axis cs:9.02642E+01,2.28528E+01) -- cycle ; 
\draw[MWE-empty] (axis cs:9.15135E+01,2.12100E+01) -- (axis cs:9.20345E+01,2.03366E+01) -- (axis cs:9.18481E+01,2.02394E+01) -- (axis cs:9.13265E+01,2.11122E+01) -- cycle ; 
\draw[MWE-empty] (axis cs:9.25496E+01,1.94617E+01) -- (axis cs:9.30592E+01,1.85854E+01) -- (axis cs:9.28742E+01,1.84892E+01) -- (axis cs:9.23640E+01,1.93650E+01) -- cycle ; 
\draw[MWE-empty] (axis cs:9.35636E+01,1.77078E+01) -- (axis cs:9.40631E+01,1.68289E+01) -- (axis cs:9.38795E+01,1.67336E+01) -- (axis cs:9.33793E+01,1.76121E+01) -- cycle ; 
\draw[MWE-empty] (axis cs:9.45580E+01,1.59488E+01) -- (axis cs:9.50487E+01,1.50676E+01) -- (axis cs:9.48661E+01,1.49732E+01) -- (axis cs:9.43750E+01,1.58540E+01) -- cycle ; 
\draw[MWE-empty] (axis cs:9.55353E+01,1.41854E+01) -- (axis cs:9.60182E+01,1.33022E+01) -- (axis cs:9.58366E+01,1.32085E+01) -- (axis cs:9.53533E+01,1.40913E+01) -- cycle ; 
\draw[MWE-empty] (axis cs:9.64975E+01,1.24181E+01) -- (axis cs:9.69737E+01,1.15333E+01) -- (axis cs:9.67931E+01,1.14402E+01) -- (axis cs:9.63165E+01,1.23248E+01) -- cycle ; 
\draw[MWE-empty] (axis cs:9.74469E+01,1.06476E+01) -- (axis cs:9.79174E+01,9.76126E+00) -- (axis cs:9.77375E+01,9.66874E+00) -- (axis cs:9.72666E+01,1.05548E+01) -- cycle ; 
\draw[MWE-empty] (axis cs:9.83854E+01,8.87430E+00) -- (axis cs:9.88512E+01,7.98676E+00) -- (axis cs:9.86719E+01,7.89468E+00) -- (axis cs:9.82058E+01,8.78200E+00) -- cycle ; 
\draw[MWE-empty] (axis cs:9.93149E+01,7.09872E+00) -- (axis cs:9.97769E+01,6.21024E+00) -- (axis cs:9.95982E+01,6.11851E+00) -- (axis cs:9.91360E+01,7.00683E+00) -- cycle ; 
\draw[MWE-empty] (axis cs:1.00237E+02,5.32137E+00) -- (axis cs:1.00697E+02,4.43217E+00) -- (axis cs:1.00518E+02,4.34070E+00) -- (axis cs:1.00059E+02,5.22978E+00) -- cycle ; 
\draw[MWE-empty] (axis cs:1.01155E+02,3.54271E+00) -- (axis cs:1.01612E+02,2.65303E+00) -- (axis cs:1.01434E+02,2.56174E+00) -- (axis cs:1.00977E+02,3.45134E+00) -- cycle ; 
\draw[MWE-empty] (axis cs:1.02069E+02,1.76320E+00) -- (axis cs:1.02525E+02,8.73274E-01) -- (axis cs:1.02347E+02,7.82062E-01) -- (axis cs:1.01891E+02,1.67196E+00) -- cycle ; 
\draw[MWE-empty] (axis cs:1.02981E+02,-1.66966E-02) -- (axis cs:1.03437E+02,-9.06652E-01) -- (axis cs:1.03259E+02,-9.97867E-01) -- (axis cs:1.02917E+02,-1.07899E-01) -- cycle ; 
\draw[MWE-empty] (axis cs:1.03893E+02,-1.79654E+00) -- (axis cs:1.04350E+02,-2.68629E+00) -- (axis cs:1.04172E+02,-2.77760E+00) -- (axis cs:1.03715E+02,-1.88779E+00) -- cycle ; 
\draw[MWE-empty] (axis cs:1.04808E+02,-3.57586E+00) -- (axis cs:1.05266E+02,-4.46519E+00) -- (axis cs:1.05088E+02,-4.55668E+00) -- (axis cs:1.04629E+02,-3.66725E+00) -- cycle ; 
\draw[MWE-empty] (axis cs:1.05726E+02,-5.35422E+00) -- (axis cs:1.06186E+02,-6.24290E+00) -- (axis cs:1.06008E+02,-6.33465E+00) -- (axis cs:1.05547E+02,-5.44583E+00) -- cycle ; 
\draw[MWE-empty] (axis cs:1.06649E+02,-7.13115E+00) -- (axis cs:1.07113E+02,-8.01893E+00) -- (axis cs:1.06934E+02,-8.11105E+00) -- (axis cs:1.06470E+02,-7.22308E+00) -- cycle ; 
\draw[MWE-empty] (axis cs:1.07580E+02,-8.90618E+00) -- (axis cs:1.08048E+02,-9.79283E+00) -- (axis cs:1.07868E+02,-9.88539E+00) -- (axis cs:1.07400E+02,-8.99851E+00) -- cycle ; 
\draw[MWE-empty] (axis cs:1.08520E+02,-1.06788E+01) -- (axis cs:1.08993E+02,-1.15641E+01) -- (axis cs:1.08813E+02,-1.16572E+01) -- (axis cs:1.08339E+02,-1.07716E+01) -- cycle ; 
\draw[MWE-empty] (axis cs:1.09471E+02,-1.24486E+01) -- (axis cs:1.09951E+02,-1.33322E+01) -- (axis cs:1.09769E+02,-1.34259E+01) -- (axis cs:1.09289E+02,-1.25420E+01) -- cycle ; 
\draw[MWE-empty] (axis cs:1.10435E+02,-1.42149E+01) -- (axis cs:1.10922E+02,-1.50966E+01) -- (axis cs:1.10740E+02,-1.51910E+01) -- (axis cs:1.10253E+02,-1.43090E+01) -- cycle ; 
\draw[MWE-empty] (axis cs:1.11414E+02,-1.59772E+01) -- (axis cs:1.11910E+02,-1.68567E+01) -- (axis cs:1.11727E+02,-1.69520E+01) -- (axis cs:1.11231E+02,-1.60721E+01) -- cycle ; 
\draw[MWE-empty] (axis cs:1.12411E+02,-1.77350E+01) -- (axis cs:1.12917E+02,-1.86119E+01) -- (axis cs:1.12732E+02,-1.87082E+01) -- (axis cs:1.12227E+02,-1.78307E+01) -- cycle ; 
\draw[MWE-empty] (axis cs:1.13428E+02,-1.94875E+01) -- (axis cs:1.13945E+02,-2.03617E+01) -- (axis cs:1.13758E+02,-2.04590E+01) -- (axis cs:1.13242E+02,-1.95843E+01) -- cycle ; 
\draw[MWE-empty] (axis cs:1.14467E+02,-2.12342E+01) -- (axis cs:1.14996E+02,-2.21052E+01) -- (axis cs:1.14808E+02,-2.22037E+01) -- (axis cs:1.14280E+02,-2.13321E+01) -- cycle ; 
\draw[MWE-empty] (axis cs:1.15531E+02,-2.29744E+01) -- (axis cs:1.16073E+02,-2.38417E+01) -- (axis cs:1.15884E+02,-2.39415E+01) -- (axis cs:1.15342E+02,-2.30735E+01) -- cycle ; 
\draw[MWE-empty] (axis cs:1.16623E+02,-2.47072E+01) -- (axis cs:1.17180E+02,-2.55705E+01) -- (axis cs:1.16989E+02,-2.56717E+01) -- (axis cs:1.16433E+02,-2.48076E+01) -- cycle ; 
\draw[MWE-empty] (axis cs:1.17746E+02,-2.64318E+01) -- (axis cs:1.18320E+02,-2.72907E+01) -- (axis cs:1.18127E+02,-2.73934E+01) -- (axis cs:1.17554E+02,-2.65337E+01) -- cycle ; 
\draw[MWE-empty] (axis cs:1.18903E+02,-2.81473E+01) -- (axis cs:1.19495E+02,-2.90013E+01) -- (axis cs:1.19300E+02,-2.91056E+01) -- (axis cs:1.18709E+02,-2.82508E+01) -- cycle ; 
\draw[MWE-empty] (axis cs:1.20098E+02,-2.98527E+01) -- (axis cs:1.20710E+02,-3.07013E+01) -- (axis cs:1.20513E+02,-3.08074E+01) -- (axis cs:1.19901E+02,-2.99579E+01) -- cycle ; 
\draw[MWE-empty] (axis cs:1.21334E+02,-3.15469E+01) -- (axis cs:1.21969E+02,-3.23894E+01) -- (axis cs:1.21769E+02,-3.24975E+01) -- (axis cs:1.21135E+02,-3.16540E+01) -- cycle ; 
\draw[MWE-empty] (axis cs:1.22616E+02,-3.32287E+01) -- (axis cs:1.23275E+02,-3.40646E+01) -- (axis cs:1.23073E+02,-3.41747E+01) -- (axis cs:1.22415E+02,-3.33378E+01) -- cycle ; 
\draw[MWE-empty] (axis cs:1.23948E+02,-3.48968E+01) -- (axis cs:1.24634E+02,-3.57252E+01) -- (axis cs:1.24430E+02,-3.58377E+01) -- (axis cs:1.23745E+02,-3.50081E+01) -- cycle ; 
\draw[MWE-empty] (axis cs:1.25335E+02,-3.65497E+01) -- (axis cs:1.26051E+02,-3.73699E+01) -- (axis cs:1.25844E+02,-3.74848E+01) -- (axis cs:1.25130E+02,-3.66633E+01) -- cycle ; 
\draw[MWE-empty] (axis cs:1.26782E+02,-3.81857E+01) -- (axis cs:1.27531E+02,-3.89969E+01) -- (axis cs:1.27322E+02,-3.91143E+01) -- (axis cs:1.26575E+02,-3.83019E+01) -- cycle ; 
\draw[MWE-empty] (axis cs:1.28296E+02,-3.98031E+01) -- (axis cs:1.29080E+02,-4.06042E+01) -- (axis cs:1.28869E+02,-4.07244E+01) -- (axis cs:1.28087E+02,-3.99220E+01) -- cycle ; 
\draw[MWE-empty] (axis cs:1.29883E+02,-4.13998E+01) -- (axis cs:1.30705E+02,-4.21897E+01) -- (axis cs:1.30492E+02,-4.23130E+01) -- (axis cs:1.29671E+02,-4.15215E+01) -- cycle ; 
\draw[MWE-empty] (axis cs:1.31549E+02,-4.29736E+01) -- (axis cs:1.32414E+02,-4.37511E+01) -- (axis cs:1.32199E+02,-4.38775E+01) -- (axis cs:1.31335E+02,-4.30984E+01) -- cycle ; 
\draw[MWE-empty] (axis cs:1.33301E+02,-4.45218E+01) -- (axis cs:1.34213E+02,-4.52855E+01) -- (axis cs:1.33996E+02,-4.54153E+01) -- (axis cs:1.33086E+02,-4.46499E+01) -- cycle ; 
\draw[MWE-empty] (axis cs:1.35149E+02,-4.60418E+01) -- (axis cs:1.36111E+02,-4.67901E+01) -- (axis cs:1.35893E+02,-4.69235E+01) -- (axis cs:1.34932E+02,-4.61733E+01) -- cycle ; 
\draw[MWE-empty] (axis cs:1.37100E+02,-4.75301E+01) -- (axis cs:1.38118E+02,-4.82614E+01) -- (axis cs:1.37899E+02,-4.83986E+01) -- (axis cs:1.36882E+02,-4.76654E+01) -- cycle ; 
\draw[MWE-empty] (axis cs:1.39165E+02,-4.89834E+01) -- (axis cs:1.40242E+02,-4.96956E+01) -- (axis cs:1.40023E+02,-4.98368E+01) -- (axis cs:1.38945E+02,-4.91226E+01) -- cycle ; 
\draw[MWE-empty] (axis cs:1.41351E+02,-5.03976E+01) -- (axis cs:1.42493E+02,-5.10886E+01) -- (axis cs:1.42274E+02,-5.12341E+01) -- (axis cs:1.41132E+02,-5.05409E+01) -- cycle ; 
\draw[MWE-empty] (axis cs:1.43670E+02,-5.17682E+01) -- (axis cs:1.44882E+02,-5.24356E+01) -- (axis cs:1.44665E+02,-5.25855E+01) -- (axis cs:1.43452E+02,-5.19158E+01) -- cycle ; 
\draw[MWE-empty] (axis cs:1.46132E+02,-5.30902E+01) -- (axis cs:1.47419E+02,-5.37314E+01) -- (axis cs:1.47204E+02,-5.38858E+01) -- (axis cs:1.45915E+02,-5.32424E+01) -- cycle ; 
\draw[MWE-empty] (axis cs:1.48746E+02,-5.43582E+01) -- (axis cs:1.50114E+02,-5.49701E+01) -- (axis cs:1.49903E+02,-5.51293E+01) -- (axis cs:1.48533E+02,-5.45151E+01) -- cycle ; 
\draw[MWE-empty] (axis cs:1.51523E+02,-5.55661E+01) -- (axis cs:1.52976E+02,-5.61454E+01) -- (axis cs:1.52770E+02,-5.63095E+01) -- (axis cs:1.51315E+02,-5.57277E+01) -- cycle ; 
\draw[MWE-empty] (axis cs:1.54472E+02,-5.67072E+01) -- (axis cs:1.56012E+02,-5.72505E+01) -- (axis cs:1.55814E+02,-5.74194E+01) -- (axis cs:1.54269E+02,-5.68737E+01) -- cycle ; 
\draw[MWE-empty] (axis cs:1.57598E+02,-5.77743E+01) -- (axis cs:1.59230E+02,-5.82777E+01) -- (axis cs:1.59042E+02,-5.84516E+01) -- (axis cs:1.57405E+02,-5.79457E+01) -- cycle ; 
\draw[MWE-empty] (axis cs:1.60908E+02,-5.87597E+01) -- (axis cs:1.62632E+02,-5.92193E+01) -- (axis cs:1.62456E+02,-5.93980E+01) -- (axis cs:1.60725E+02,-5.89360E+01) -- cycle ; 
\draw[MWE-empty] (axis cs:1.64402E+02,-5.96555E+01) -- (axis cs:1.66217E+02,-6.00672E+01) -- (axis cs:1.66056E+02,-6.02505E+01) -- (axis cs:1.64233E+02,-5.98365E+01) -- cycle ; 
\draw[MWE-empty] (axis cs:1.68077E+02,-6.04533E+01) -- (axis cs:1.69979E+02,-6.08130E+01) -- (axis cs:1.69837E+02,-6.10005E+01) -- (axis cs:1.67924E+02,-6.06388E+01) -- cycle ; 
\draw[MWE-empty] (axis cs:1.71924E+02,-6.11451E+01) -- (axis cs:1.73908E+02,-6.14487E+01) -- (axis cs:1.73786E+02,-6.16401E+01) -- (axis cs:1.71791E+02,-6.13346E+01) -- cycle ; 
\draw[MWE-empty] (axis cs:1.75929E+02,-6.17229E+01) -- (axis cs:1.77984E+02,-6.19668E+01) -- (axis cs:1.77886E+02,-6.21614E+01) -- (axis cs:1.75819E+02,-6.19160E+01) -- cycle ; 
\draw[MWE-empty] (axis cs:1.80071E+02,-6.21796E+01) -- (axis cs:1.82184E+02,-6.23606E+01) -- (axis cs:1.82113E+02,-6.25578E+01) -- (axis cs:1.79986E+02,-6.23756E+01) -- cycle ; 
\draw[MWE-empty] (axis cs:1.84321E+02,-6.25091E+01) -- (axis cs:1.86478E+02,-6.26247E+01) -- (axis cs:1.86434E+02,-6.28237E+01) -- (axis cs:1.84264E+02,-6.27073E+01) -- cycle ; 
\draw[MWE-empty] (axis cs:1.88648E+02,-6.27068E+01) -- (axis cs:1.90828E+02,-6.27553E+01) -- (axis cs:1.90814E+02,-6.29552E+01) -- (axis cs:1.88619E+02,-6.29064E+01) -- cycle ; 
\draw[MWE-empty] (axis cs:1.93013E+02,-6.27699E+01) -- (axis cs:1.95197E+02,-6.27506E+01) -- (axis cs:1.95213E+02,-6.29504E+01) -- (axis cs:1.93014E+02,-6.29699E+01) -- cycle ; 
\draw[MWE-empty] (axis cs:1.97377E+02,-6.26974E+01) -- (axis cs:1.99545E+02,-6.26105E+01) -- (axis cs:1.99591E+02,-6.28094E+01) -- (axis cs:1.97407E+02,-6.28969E+01) -- cycle ; 
\draw[MWE-empty] (axis cs:2.01699E+02,-6.24903E+01) -- (axis cs:2.03833E+02,-6.23372E+01) -- (axis cs:2.03907E+02,-6.25342E+01) -- (axis cs:2.01759E+02,-6.26884E+01) -- cycle ; 
\draw[MWE-empty] (axis cs:2.05943E+02,-6.21517E+01) -- (axis cs:2.08026E+02,-6.19345E+01) -- (axis cs:2.08125E+02,-6.21289E+01) -- (axis cs:2.06030E+02,-6.23475E+01) -- cycle ; 
\draw[MWE-empty] (axis cs:2.10076E+02,-6.16863E+01) -- (axis cs:2.12092E+02,-6.14079E+01) -- (axis cs:2.12216E+02,-6.15991E+01) -- (axis cs:2.10188E+02,-6.18792E+01) -- cycle ; 
\draw[MWE-empty] (axis cs:2.14071E+02,-6.11003E+01) -- (axis cs:2.16010E+02,-6.07642E+01) -- (axis cs:2.16154E+02,-6.09515E+01) -- (axis cs:2.14205E+02,-6.12895E+01) -- cycle ; 
\draw[MWE-empty] (axis cs:2.17906E+02,-6.04008E+01) -- (axis cs:2.19760E+02,-6.00110E+01) -- (axis cs:2.19922E+02,-6.01940E+01) -- (axis cs:2.18060E+02,-6.05860E+01) -- cycle ; 
\draw[MWE-empty] (axis cs:2.21569E+02,-5.95959E+01) -- (axis cs:2.23332E+02,-5.91564E+01) -- (axis cs:2.23509E+02,-5.93347E+01) -- (axis cs:2.21739E+02,-5.97766E+01) -- cycle ; 
\draw[MWE-empty] (axis cs:2.25050E+02,-5.86936E+01) -- (axis cs:2.26721E+02,-5.82085E+01) -- (axis cs:2.26910E+02,-5.83820E+01) -- (axis cs:2.25233E+02,-5.88695E+01) -- cycle ; 
\draw[MWE-empty] (axis cs:2.28346E+02,-5.77022E+01) -- (axis cs:2.29926E+02,-5.71755E+01) -- (axis cs:2.30125E+02,-5.73442E+01) -- (axis cs:2.28541E+02,-5.78732E+01) -- cycle ; 
\draw[MWE-empty] (axis cs:2.31460E+02,-5.66296E+01) -- (axis cs:2.32950E+02,-5.60654E+01) -- (axis cs:2.33157E+02,-5.62291E+01) -- (axis cs:2.31663E+02,-5.67958E+01) -- cycle ; 
\draw[MWE-empty] (axis cs:2.34397E+02,-5.54837E+01) -- (axis cs:2.35800E+02,-5.48854E+01) -- (axis cs:2.36012E+02,-5.50442E+01) -- (axis cs:2.34606E+02,-5.56449E+01) -- cycle ; 
\draw[MWE-empty] (axis cs:2.37162E+02,-5.42714E+01) -- (axis cs:2.38483E+02,-5.36425E+01) -- (axis cs:2.38699E+02,-5.37966E+01) -- (axis cs:2.37376E+02,-5.44279E+01) -- cycle ; 
\draw[MWE-empty] (axis cs:2.39766E+02,-5.29994E+01) -- (axis cs:2.41010E+02,-5.23429E+01) -- (axis cs:2.41227E+02,-5.24925E+01) -- (axis cs:2.39982E+02,-5.31512E+01) -- cycle ; 
\draw[MWE-empty] (axis cs:2.42217E+02,-5.16738E+01) -- (axis cs:2.43389E+02,-5.09926E+01) -- (axis cs:2.43608E+02,-5.11377E+01) -- (axis cs:2.42435E+02,-5.18211E+01) -- cycle ; 
\draw[MWE-empty] (axis cs:2.44526E+02,-5.02999E+01) -- (axis cs:2.45631E+02,-4.95965E+01) -- (axis cs:2.45850E+02,-4.97374E+01) -- (axis cs:2.44745E+02,-5.04429E+01) -- cycle ; 
\draw[MWE-empty] (axis cs:2.46704E+02,-4.88829E+01) -- (axis cs:2.47746E+02,-4.81596E+01) -- (axis cs:2.47965E+02,-4.82965E+01) -- (axis cs:2.46923E+02,-4.90218E+01) -- cycle ; 
\draw[MWE-empty] (axis cs:2.48760E+02,-4.74270E+01) -- (axis cs:2.49745E+02,-4.66858E+01) -- (axis cs:2.49963E+02,-4.68189E+01) -- (axis cs:2.48978E+02,-4.75620E+01) -- cycle ; 
\draw[MWE-empty] (axis cs:2.50704E+02,-4.59363E+01) -- (axis cs:2.51636E+02,-4.51791E+01) -- (axis cs:2.51853E+02,-4.53086E+01) -- (axis cs:2.50921E+02,-4.60676E+01) -- cycle ; 
\draw[MWE-empty] (axis cs:2.52544E+02,-4.44144E+01) -- (axis cs:2.53429E+02,-4.36426E+01) -- (axis cs:2.53644E+02,-4.37688E+01) -- (axis cs:2.52760E+02,-4.45422E+01) -- cycle ; 
\draw[MWE-empty] (axis cs:2.54291E+02,-4.28642E+01) -- (axis cs:2.55131E+02,-4.20795E+01) -- (axis cs:2.55344E+02,-4.22025E+01) -- (axis cs:2.54505E+02,-4.29888E+01) -- cycle ; 
\draw[MWE-empty] (axis cs:2.55951E+02,-4.12888E+01) -- (axis cs:2.56751E+02,-4.04924E+01) -- (axis cs:2.56962E+02,-4.06124E+01) -- (axis cs:2.56163E+02,-4.14103E+01) -- cycle ; 
\draw[MWE-empty] (axis cs:2.57532E+02,-3.96906E+01) -- (axis cs:2.58295E+02,-3.88836E+01) -- (axis cs:2.58504E+02,-3.90009E+01) -- (axis cs:2.57742E+02,-3.98092E+01) -- cycle ; 
\draw[MWE-empty] (axis cs:2.59041E+02,-3.80718E+01) -- (axis cs:2.59771E+02,-3.72554E+01) -- (axis cs:2.59977E+02,-3.73700E+01) -- (axis cs:2.59248E+02,-3.81877E+01) -- cycle ; 
\draw[MWE-empty] (axis cs:2.60484E+02,-3.64345E+01) -- (axis cs:2.61183E+02,-3.56095E+01) -- (axis cs:2.61387E+02,-3.57218E+01) -- (axis cs:2.60689E+02,-3.65480E+01) -- cycle ; 
\draw[MWE-empty] (axis cs:2.61867E+02,-3.47805E+01) -- (axis cs:2.62538E+02,-3.39478E+01) -- (axis cs:2.62740E+02,-3.40578E+01) -- (axis cs:2.62070E+02,-3.48916E+01) -- cycle ; 
\draw[MWE-empty] (axis cs:2.63196E+02,-3.31114E+01) -- (axis cs:2.63841E+02,-3.22717E+01) -- (axis cs:2.64040E+02,-3.23796E+01) -- (axis cs:2.63396E+02,-3.32204E+01) -- cycle ; 
\draw[MWE-empty] (axis cs:2.64474E+02,-3.14287E+01) -- (axis cs:2.65096E+02,-3.05826E+01) -- (axis cs:2.65293E+02,-3.06886E+01) -- (axis cs:2.64672E+02,-3.15356E+01) -- cycle ; 
\draw[MWE-empty] (axis cs:2.65707E+02,-2.97337E+01) -- (axis cs:2.66308E+02,-2.88819E+01) -- (axis cs:2.66503E+02,-2.89861E+01) -- (axis cs:2.65903E+02,-2.98387E+01) -- cycle ; 

 
\draw [MWE-empty] (axis cs:2.66016E+02,-2.87251E+01) -- (axis cs:2.66797E+02,-2.91419E+01)   ;
\node[pin={[pin distance=-0.4\onedegree,MWE-label]0:{0$^\circ$}}] at (axis cs:2.66016E+02,-2.87251E+01) {} ;
\draw [MWE-empty] (axis cs:2.71569E+02,-2.00932E+01) -- (axis cs:2.72314E+02,-2.04821E+01)   ;
\node[pin={[pin distance=-0.4\onedegree,MWE-label]00:{10$^\circ$}}] at (axis cs:2.71569E+02,-2.00932E+01) {} ;
\draw [MWE-empty] (axis cs:2.76522E+02,-1.13004E+01) -- (axis cs:2.77245E+02,-1.16726E+01)   ;
\node[pin={[pin distance=-0.4\onedegree,MWE-label]00:{20$^\circ$}}] at (axis cs:2.76522E+02,-1.13004E+01) {} ;
\draw [MWE-empty] (axis cs:2.81167E+02,-2.42475E+00) -- (axis cs:2.81879E+02,-2.78993E+00)   ;
\node[pin={[pin distance=-0.4\onedegree,MWE-label]00:{30$^\circ$}}] at (axis cs:2.81167E+02,-2.42475E+00) {} ;
\draw [MWE-empty] (axis cs:2.85739E+02,6.47217E+00) -- (axis cs:2.86454E+02,6.10515E+00)   ;
\node[pin={[pin distance=-0.4\onedegree,MWE-label]00:{40$^\circ$}}] at (axis cs:2.85739E+02,6.47217E+00) {} ;
\draw [MWE-empty] (axis cs:2.90465E+02,1.53325E+01) -- (axis cs:2.91196E+02,1.49546E+01)   ;
\node[pin={[pin distance=-0.4\onedegree,MWE-label]00:{50$^\circ$}}] at (axis cs:2.90465E+02,1.53325E+01) {} ;
\draw [MWE-empty] (axis cs:2.95599E+02,2.40907E+01) -- (axis cs:2.96357E+02,2.36917E+01)   ;
\node[pin={[pin distance=-0.4\onedegree,MWE-label]00:{60$^\circ$}}] at (axis cs:2.95599E+02,2.40907E+01) {} ;
\draw [MWE-empty] (axis cs:3.01469E+02,3.26590E+01) -- (axis cs:3.02267E+02,3.22267E+01)   ;
\node[pin={[pin distance=-0.4\onedegree,MWE-label]00:{70$^\circ$}}] at (axis cs:3.01469E+02,3.26590E+01) {} ;
\draw [MWE-empty] (axis cs:3.08552E+02,4.09041E+01) -- (axis cs:3.09394E+02,4.04231E+01)   ;
\node[pin={[pin distance=-0.4\onedegree,MWE-label]00:{80$^\circ$}}] at (axis cs:3.08552E+02,4.09041E+01) {} ;
\draw [MWE-empty] (axis cs:3.17568E+02,4.86036E+01) -- (axis cs:3.18444E+02,4.80548E+01)   ;
\node[pin={[pin distance=-0.4\onedegree,MWE-label]90:{90$^\circ$}}] at (axis cs:3.17568E+02,4.86036E+01) {} ;
\draw [MWE-empty] (axis cs:3.29582E+02,5.53674E+01) -- (axis cs:3.30428E+02,5.47306E+01)   ;
\node[pin={[pin distance=-0.4\onedegree,MWE-label]090:{100$^\circ$}}] at (axis cs:3.29582E+02,5.53674E+01) {} ;
\draw [MWE-empty] (axis cs:3.45811E+02,6.05250E+01) -- (axis cs:3.46455E+02,5.97919E+01)   ;
\node[pin={[pin distance=-0.4\onedegree,MWE-label]090:{110$^\circ$}}] at (axis cs:3.45811E+02,6.05250E+01) {} ;
\draw [MWE-empty] (axis cs:6.36807E+00,6.31222E+01) -- (axis cs:6.54138E+00,6.23261E+01)   ;
\node[pin={[pin distance=-0.4\onedegree,MWE-label]090:{120$^\circ$}}] at (axis cs:6.36807E+00,6.31222E+01) {} ;
\draw [MWE-empty] (axis cs:2.82769E+01,6.24205E+01) -- (axis cs:2.78784E+01,6.16426E+01)   ;
\node[pin={[pin distance=-0.4\onedegree,MWE-label]090:{130$^\circ$}}] at (axis cs:2.82769E+01,6.24205E+01) {} ;
\draw [MWE-empty] (axis cs:4.71975E+01,5.86418E+01) -- (axis cs:4.64407E+01,5.79477E+01)   ;
\node[pin={[pin distance=-0.4\onedegree,MWE-label]090:{140$^\circ$}}] at (axis cs:4.71975E+01,5.86418E+01) {} ;
\draw [MWE-empty] (axis cs:6.15567E+01,5.27160E+01) -- (axis cs:6.06857E+01,5.21179E+01)   ;
\node[pin={[pin distance=-0.4\onedegree,MWE-label]090:{150$^\circ$}}] at (axis cs:6.15567E+01,5.27160E+01) {} ;
\draw [MWE-empty] (axis cs:7.21792E+01,4.55021E+01) -- (axis cs:7.13134E+01,4.49840E+01)   ;
\node[pin={[pin distance=-0.4\onedegree,MWE-label]090:{160$^\circ$}}] at (axis cs:7.21792E+01,4.55021E+01) {} ;
\draw [MWE-empty] (axis cs:8.02868E+01,3.75414E+01) -- (axis cs:7.94627E+01,3.70827E+01)   ;
\node[pin={[pin distance=-0.4\onedegree,MWE-label]090:{170$^\circ$}}] at (axis cs:8.02868E+01,3.75414E+01) {} ;
\draw [MWE-empty] (axis cs:8.67966E+01,2.91419E+01) -- (axis cs:8.60164E+01,2.87251E+01)   ;
\node[pin={[pin distance=-0.4\onedegree,MWE-label]090:{180$^\circ$}}] at (axis cs:8.67966E+01,2.91419E+01) {} ;
\draw [MWE-empty] (axis cs:9.23144E+01,2.04821E+01) -- (axis cs:9.15691E+01,2.00932E+01)   ;
\node[pin={[pin distance=-0.4\onedegree,MWE-label]180:{190$^\circ$}}] at (axis cs:9.23144E+01,2.04821E+01) {} ;
\draw [MWE-empty] (axis cs:9.72449E+01,1.16726E+01) -- (axis cs:9.65223E+01,1.13004E+01)   ;
\node[pin={[pin distance=-0.4\onedegree,MWE-label]180:{200$^\circ$}}] at (axis cs:9.72449E+01,1.16726E+01) {} ;
\draw [MWE-empty] (axis cs:1.01879E+02,2.78993E+00) -- (axis cs:1.01167E+02,2.42475E+00)   ;
\node[pin={[pin distance=-0.4\onedegree,MWE-label]180:{210$^\circ$}}] at (axis cs:1.01879E+02,2.78993E+00) {} ;
\draw [MWE-empty] (axis cs:1.06454E+02,-6.10515E+00) -- (axis cs:1.05739E+02,-6.47217E+00)   ;
\node[pin={[pin distance=-0.4\onedegree,MWE-label]180:{220$^\circ$}}] at (axis cs:1.06454E+02,-6.10515E+00) {} ;
\draw [MWE-empty] (axis cs:1.11196E+02,-1.49546E+01) -- (axis cs:1.10466E+02,-1.53325E+01)   ;
\node[pin={[pin distance=-0.4\onedegree,MWE-label]180:{230$^\circ$}}] at (axis cs:1.11196E+02,-1.49546E+01) {} ;
\draw [MWE-empty] (axis cs:1.16357E+02,-2.36917E+01) -- (axis cs:1.15599E+02,-2.40907E+01)   ;
\node[pin={[pin distance=-0.4\onedegree,MWE-label]180:{240$^\circ$}}] at (axis cs:1.16357E+02,-2.36917E+01) {} ;
\draw [MWE-empty] (axis cs:1.22267E+02,-3.22267E+01) -- (axis cs:1.21469E+02,-3.26590E+01)   ;
\node[pin={[pin distance=-0.4\onedegree,MWE-label]180:{250$^\circ$}}] at (axis cs:1.22267E+02,-3.22267E+01) {} ;
\draw [MWE-empty] (axis cs:1.29395E+02,-4.04231E+01) -- (axis cs:1.28552E+02,-4.09041E+01)   ;
\node[pin={[pin distance=-0.4\onedegree,MWE-label]180:{260$^\circ$}}] at (axis cs:1.29395E+02,-4.04231E+01) {} ;
\draw [MWE-empty] (axis cs:1.38444E+02,-4.80548E+01) -- (axis cs:1.37568E+02,-4.86036E+01)   ;
\node[pin={[pin distance=-0.4\onedegree,MWE-label]090:{270$^\circ$}}] at (axis cs:1.38444E+02,-4.80548E+01) {} ;
\draw [MWE-empty] (axis cs:1.50428E+02,-5.47306E+01) -- (axis cs:1.49582E+02,-5.53674E+01)   ;
\node[pin={[pin distance=-0.4\onedegree,MWE-label]090:{280$^\circ$}}] at (axis cs:1.50428E+02,-5.47306E+01) {} ;
\draw [MWE-empty] (axis cs:1.66455E+02,-5.97919E+01) -- (axis cs:1.65811E+02,-6.05250E+01)   ;
\node[pin={[pin distance=-0.4\onedegree,MWE-label]090:{290$^\circ$}}] at (axis cs:1.66455E+02,-5.97919E+01) {} ;
\draw [MWE-empty] (axis cs:1.86541E+02,-6.23261E+01) -- (axis cs:1.86368E+02,-6.31222E+01)   ;
\node[pin={[pin distance=-0.4\onedegree,MWE-label]090:{300$^\circ$}}] at (axis cs:1.86541E+02,-6.23261E+01) {} ;
\draw [MWE-empty] (axis cs:2.07878E+02,-6.16426E+01) -- (axis cs:2.08277E+02,-6.24205E+01)   ;
\node[pin={[pin distance=-0.4\onedegree,MWE-label]090:{310$^\circ$}}] at (axis cs:2.07878E+02,-6.16426E+01) {} ;
\draw [MWE-empty] (axis cs:2.26441E+02,-5.79477E+01) -- (axis cs:2.27198E+02,-5.86418E+01)   ;
\node[pin={[pin distance=-0.4\onedegree,MWE-label]090:{320$^\circ$}}] at (axis cs:2.26441E+02,-5.79477E+01) {} ;
\draw [MWE-empty] (axis cs:2.40686E+02,-5.21179E+01) -- (axis cs:2.41557E+02,-5.27160E+01)   ;
\node[pin={[pin distance=-0.4\onedegree,MWE-label]090:{330$^\circ$}}] at (axis cs:2.40686E+02,-5.21179E+01) {} ;
\draw [MWE-empty] (axis cs:2.51313E+02,-4.49840E+01) -- (axis cs:2.52179E+02,-4.55021E+01)   ;
\node[pin={[pin distance=-0.4\onedegree,MWE-label]090:{340$^\circ$}}] at (axis cs:2.51313E+02,-4.49840E+01) {} ;
\draw [MWE-empty] (axis cs:2.59463E+02,-3.70827E+01) -- (axis cs:2.60287E+02,-3.75414E+01)   ;
\node[pin={[pin distance=-0.4\onedegree,MWE-label]090:{350$^\circ$}}] at (axis cs:2.59463E+02,-3.70827E+01) {} ;


\end{axis}

% Ecliptic coordinate system


\begin{axis}[name=ecliptics,axis lines=none,at=(base.center),anchor=center]

% This simply plots the line
%\addplot[ecliptics,domain=0:390,samples=390] {23.433*sin(x)};

\draw [ecliptics-full] (axis cs:3.59960E+02,9.17485E-02) -- (axis cs:360.0398,-9.17483E-02)  -- (axis cs:3.59122E+02,-4.89506E-01) -- (axis cs:3.59043E+02,-3.06005E-01) -- cycle  ;

\draw [ecliptics-full] (axis cs:{8.77725E-01},{4.89506E-01}) -- (axis cs:{9.57272E-01},{3.06005E-01})  -- (axis cs:{1.87485E+00},{7.03652E-01}) -- (axis cs:{1.79532E+00},{8.87166E-01}) -- cycle  ;
\draw [ecliptics-full] (axis cs:2.71311E+00,1.28463E+00) -- (axis cs:2.79259E+00,1.10109E+00)  -- (axis cs:3.71059E+00,1.49822E+00) -- (axis cs:3.63117E+00,1.68179E+00) -- cycle  ;
\draw [ecliptics-full] (axis cs:4.54959E+00,2.07854E+00) -- (axis cs:4.62893E+00,1.89493E+00)  -- (axis cs:5.54771E+00,2.29113E+00) -- (axis cs:5.46845E+00,2.47479E+00) -- cycle  ;
\draw [ecliptics-full] (axis cs:6.38786E+00,2.87043E+00) -- (axis cs:6.46701E+00,2.68672E+00)  -- (axis cs:7.38691E+00,3.08158E+00) -- (axis cs:7.30789E+00,3.26536E+00) -- cycle  ;
\draw [ecliptics-full] (axis cs:8.22863E+00,3.65948E+00) -- (axis cs:8.30751E+00,3.47563E+00)  -- (axis cs:9.22889E+00,3.86875E+00) -- (axis cs:9.15017E+00,4.05268E+00) -- cycle  ;
\draw [ecliptics-full] (axis cs:1.00726E+01,4.44487E+00) -- (axis cs:1.01511E+01,4.26084E+00)  -- (axis cs:1.10743E+01,4.65180E+00) -- (axis cs:1.09960E+01,4.83593E+00) -- cycle  ;
\draw [ecliptics-full] (axis cs:1.19204E+01,5.22577E+00) -- (axis cs:1.19986E+01,5.04154E+00)  -- (axis cs:1.29239E+01,5.42993E+00) -- (axis cs:1.28460E+01,5.61429E+00) -- cycle  ;
\draw [ecliptics-full] (axis cs:1.37728E+01,6.00137E+00) -- (axis cs:1.38505E+01,5.81689E+00)  -- (axis cs:1.47783E+01,6.20231E+00) -- (axis cs:1.47009E+01,6.38692E+00) -- cycle  ;
\draw [ecliptics-full] (axis cs:1.56304E+01,6.77084E+00) -- (axis cs:1.57075E+01,6.58609E+00)  -- (axis cs:1.66382E+01,6.96811E+00) -- (axis cs:1.65614E+01,7.15301E+00) -- cycle  ;
\draw [ecliptics-full] (axis cs:1.74939E+01,7.53334E+00) -- (axis cs:1.75704E+01,7.34829E+00)  -- (axis cs:1.85041E+01,7.72651E+00) -- (axis cs:1.84280E+01,7.91173E+00) -- cycle  ;
\draw [ecliptics-full] (axis cs:1.93638E+01,8.28806E+00) -- (axis cs:1.94396E+01,8.10267E+00)  -- (axis cs:2.03768E+01,8.47667E+00) -- (axis cs:2.03014E+01,8.66224E+00) -- cycle  ;
\draw [ecliptics-full] (axis cs:2.12408E+01,9.03415E+00) -- (axis cs:2.13159E+01,8.84840E+00)  -- (axis cs:2.22568E+01,9.21776E+00) -- (axis cs:2.21822E+01,9.40370E+00) -- cycle  ;
\draw [ecliptics-full] (axis cs:2.31256E+01,9.77078E+00) -- (axis cs:2.31998E+01,9.58464E+00)  -- (axis cs:2.41448E+01,9.94894E+00) -- (axis cs:2.40710E+01,1.01353E+01) -- cycle  ;
\draw [ecliptics-full] (axis cs:2.50186E+01,1.04971E+01) -- (axis cs:2.50918E+01,1.03105E+01)  -- (axis cs:2.60411E+01,1.06694E+01) -- (axis cs:2.59683E+01,1.08562E+01) -- cycle  ;
\draw [ecliptics-full] (axis cs:2.69204E+01,1.12123E+01) -- (axis cs:2.69926E+01,1.10253E+01)  -- (axis cs:2.79465E+01,1.13782E+01) -- (axis cs:2.78747E+01,1.15655E+01) -- cycle  ;
\draw [ecliptics-full] (axis cs:2.88315E+01,1.19155E+01) -- (axis cs:2.89026E+01,1.17280E+01)  -- (axis cs:2.98613E+01,1.20746E+01) -- (axis cs:2.97907E+01,1.22623E+01) -- cycle  ;
\draw [ecliptics-full] (axis cs:3.07524E+01,1.26059E+01) -- (axis cs:3.08224E+01,1.24179E+01)  -- (axis cs:3.17860E+01,1.27578E+01) -- (axis cs:3.17166E+01,1.29460E+01) -- cycle  ;
\draw [ecliptics-full] (axis cs:3.26835E+01,1.32826E+01) -- (axis cs:3.27522E+01,1.30941E+01)  -- (axis cs:3.37211E+01,1.34268E+01) -- (axis cs:3.36530E+01,1.36155E+01) -- cycle  ;
\draw [ecliptics-full] (axis cs:3.46253E+01,1.39448E+01) -- (axis cs:3.46927E+01,1.37558E+01)  -- (axis cs:3.56669E+01,1.40809E+01) -- (axis cs:3.56002E+01,1.42701E+01) -- cycle  ;
\draw [ecliptics-full] (axis cs:3.65780E+01,1.45916E+01) -- (axis cs:3.66440E+01,1.44021E+01)  -- (axis cs:3.76238E+01,1.47192E+01) -- (axis cs:3.75586E+01,1.49090E+01) -- cycle  ;
\draw [ecliptics-full] (axis cs:3.85421E+01,1.52222E+01) -- (axis cs:3.86065E+01,1.50321E+01)  -- (axis cs:3.95921E+01,1.53408E+01) -- (axis cs:3.95285E+01,1.55312E+01) -- cycle  ;
\draw [ecliptics-full] (axis cs:4.05178E+01,1.58358E+01) -- (axis cs:4.05806E+01,1.56451E+01)  -- (axis cs:4.15720E+01,1.59450E+01) -- (axis cs:4.15101E+01,1.61359E+01) -- cycle  ;
\draw [ecliptics-full] (axis cs:4.25053E+01,1.64315E+01) -- (axis cs:4.25664E+01,1.62403E+01)  -- (axis cs:4.35638E+01,1.65309E+01) -- (axis cs:4.35036E+01,1.67224E+01) -- cycle  ;
\draw [ecliptics-full] (axis cs:4.45048E+01,1.70085E+01) -- (axis cs:4.45641E+01,1.68167E+01)  -- (axis cs:4.55675E+01,1.70976E+01) -- (axis cs:4.55091E+01,1.72897E+01) -- cycle  ;
\draw [ecliptics-full] (axis cs:4.65165E+01,1.75660E+01) -- (axis cs:4.65738E+01,1.73736E+01)  -- (axis cs:4.75832E+01,1.76445E+01) -- (axis cs:4.75268E+01,1.78371E+01) -- cycle  ;
\draw [ecliptics-full] (axis cs:4.85402E+01,1.81031E+01) -- (axis cs:4.85956E+01,1.79102E+01)  -- (axis cs:4.96110E+01,1.81706E+01) -- (axis cs:4.95567E+01,1.83638E+01) -- cycle  ;
\draw [ecliptics-full] (axis cs:5.05762E+01,1.86192E+01) -- (axis cs:5.06294E+01,1.84256E+01)  -- (axis cs:5.16509E+01,1.86752E+01) -- (axis cs:5.15987E+01,1.88690E+01) -- cycle  ;
\draw [ecliptics-full] (axis cs:5.26242E+01,1.91133E+01) -- (axis cs:5.26753E+01,1.89192E+01)  -- (axis cs:5.37027E+01,1.91575E+01) -- (axis cs:5.36527E+01,1.93519E+01) -- cycle  ;
\draw [ecliptics-full] (axis cs:5.46842E+01,1.95848E+01) -- (axis cs:5.47330E+01,1.93901E+01)  -- (axis cs:5.57662E+01,1.96168E+01) -- (axis cs:5.57187E+01,1.98118E+01) -- cycle  ;
\draw [ecliptics-full] (axis cs:5.67560E+01,2.00328E+01) -- (axis cs:5.68024E+01,1.98376E+01)  -- (axis cs:5.78414E+01,2.00524E+01) -- (axis cs:5.77962E+01,2.02479E+01) -- cycle  ;
\draw [ecliptics-full] (axis cs:5.88393E+01,2.04568E+01) -- (axis cs:5.88831E+01,2.02610E+01)  -- (axis cs:5.99277E+01,2.04635E+01) -- (axis cs:5.98851E+01,2.06595E+01) -- cycle  ;
\draw [ecliptics-full] (axis cs:6.09337E+01,2.08559E+01) -- (axis cs:6.09750E+01,2.06597E+01)  -- (axis cs:6.20250E+01,2.08495E+01) -- (axis cs:6.19850E+01,2.10459E+01) -- cycle  ;
\draw [ecliptics-full] (axis cs:6.30389E+01,2.12295E+01) -- (axis cs:6.30775E+01,2.10328E+01)  -- (axis cs:6.41327E+01,2.12096E+01) -- (axis cs:6.40954E+01,2.14066E+01) -- cycle  ;
\draw [ecliptics-full] (axis cs:6.51544E+01,2.15771E+01) -- (axis cs:6.51903E+01,2.13799E+01)  -- (axis cs:6.62503E+01,2.15434E+01) -- (axis cs:6.62158E+01,2.17408E+01) -- cycle  ;
\draw [ecliptics-full] (axis cs:6.72796E+01,2.18979E+01) -- (axis cs:6.73127E+01,2.17002E+01)  -- (axis cs:6.83773E+01,2.18502E+01) -- (axis cs:6.83457E+01,2.20481E+01) -- cycle  ;
\draw [ecliptics-full] (axis cs:6.94140E+01,2.21914E+01) -- (axis cs:6.94442E+01,2.19934E+01)  -- (axis cs:7.05131E+01,2.21296E+01) -- (axis cs:7.04845E+01,2.23278E+01) -- cycle  ;
\draw [ecliptics-full] (axis cs:7.15569E+01,2.24572E+01) -- (axis cs:7.15841E+01,2.22587E+01)  -- (axis cs:7.26570E+01,2.23809E+01) -- (axis cs:7.26313E+01,2.25795E+01) -- cycle  ;
\draw [ecliptics-full] (axis cs:7.37076E+01,2.26946E+01) -- (axis cs:7.37317E+01,2.24959E+01)  -- (axis cs:7.48082E+01,2.26037E+01) -- (axis cs:7.47856E+01,2.28026E+01) -- cycle  ;
\draw [ecliptics-full] (axis cs:7.58653E+01,2.29034E+01) -- (axis cs:7.58863E+01,2.27043E+01)  -- (axis cs:7.69660E+01,2.27977E+01) -- (axis cs:7.69465E+01,2.29969E+01) -- cycle  ;
\draw [ecliptics-full] (axis cs:7.80291E+01,2.30830E+01) -- (axis cs:7.80470E+01,2.28837E+01)  -- (axis cs:7.91294E+01,2.29624E+01) -- (axis cs:7.91131E+01,2.31619E+01) -- cycle  ;
\draw [ecliptics-full] (axis cs:8.01982E+01,2.32333E+01) -- (axis cs:8.02130E+01,2.30338E+01)  -- (axis cs:8.12976E+01,2.30977E+01) -- (axis cs:8.12845E+01,2.32973E+01) -- cycle  ;
\draw [ecliptics-full] (axis cs:8.23718E+01,2.33538E+01) -- (axis cs:8.23833E+01,2.31541E+01)  -- (axis cs:8.34698E+01,2.32031E+01) -- (axis cs:8.34599E+01,2.34029E+01) -- cycle  ;
\draw [ecliptics-full] (axis cs:8.45488E+01,2.34444E+01) -- (axis cs:8.45570E+01,2.32446E+01)  -- (axis cs:8.56449E+01,2.32785E+01) -- (axis cs:8.56383E+01,2.34785E+01) -- cycle  ;
\draw [ecliptics-full] (axis cs:8.67283E+01,2.35049E+01) -- (axis cs:8.67332E+01,2.33050E+01)  -- (axis cs:8.78219E+01,2.33239E+01) -- (axis cs:8.78186E+01,2.35239E+01) -- cycle  ;
\draw [ecliptics-full] (axis cs:8.89093E+01,2.35352E+01) -- (axis cs:8.89109E+01,2.33352E+01)  -- (axis cs:9.00000E+01,2.33390E+01) -- (axis cs:9.00000E+01,2.35390E+01) -- cycle  ;
\draw [ecliptics-full] (axis cs:9.10907E+01,2.35352E+01) -- (axis cs:9.10891E+01,2.33352E+01)  -- (axis cs:9.21781E+01,2.33239E+01) -- (axis cs:9.21814E+01,2.35239E+01) -- cycle  ;
\draw [ecliptics-full] (axis cs:9.32717E+01,2.35049E+01) -- (axis cs:9.32668E+01,2.33050E+01)  -- (axis cs:9.43551E+01,2.32785E+01) -- (axis cs:9.43617E+01,2.34785E+01) -- cycle  ;
\draw [ecliptics-full] (axis cs:9.54512E+01,2.34444E+01) -- (axis cs:9.54430E+01,2.32446E+01)  -- (axis cs:9.65302E+01,2.32031E+01) -- (axis cs:9.65401E+01,2.34029E+01) -- cycle  ;
\draw [ecliptics-full] (axis cs:9.76282E+01,2.33538E+01) -- (axis cs:9.76167E+01,2.31541E+01)  -- (axis cs:9.87024E+01,2.30977E+01) -- (axis cs:9.87155E+01,2.32973E+01) -- cycle  ;
\draw [ecliptics-full] (axis cs:9.98018E+01,2.32333E+01) -- (axis cs:9.97870E+01,2.30338E+01)  -- (axis cs:1.00871E+02,2.29624E+01) -- (axis cs:1.00887E+02,2.31619E+01) -- cycle  ;
\draw [ecliptics-full] (axis cs:1.01971E+02,2.30830E+01) -- (axis cs:1.01953E+02,2.28837E+01)  -- (axis cs:1.03034E+02,2.27977E+01) -- (axis cs:1.03054E+02,2.29969E+01) -- cycle  ;
\draw [ecliptics-full] (axis cs:1.04135E+02,2.29034E+01) -- (axis cs:1.04114E+02,2.27043E+01)  -- (axis cs:1.05192E+02,2.26037E+01) -- (axis cs:1.05214E+02,2.28026E+01) -- cycle  ;
\draw [ecliptics-full] (axis cs:1.06292E+02,2.26946E+01) -- (axis cs:1.06268E+02,2.24959E+01)  -- (axis cs:1.07343E+02,2.23809E+01) -- (axis cs:1.07369E+02,2.25795E+01) -- cycle  ;
\draw [ecliptics-full] (axis cs:1.08443E+02,2.24572E+01) -- (axis cs:1.08416E+02,2.22587E+01)  -- (axis cs:1.09487E+02,2.21296E+01) -- (axis cs:1.09516E+02,2.23278E+01) -- cycle  ;
\draw [ecliptics-full] (axis cs:1.10586E+02,2.21914E+01) -- (axis cs:1.10556E+02,2.19934E+01)  -- (axis cs:1.11623E+02,2.18502E+01) -- (axis cs:1.11654E+02,2.20481E+01) -- cycle  ;
\draw [ecliptics-full] (axis cs:1.12720E+02,2.18979E+01) -- (axis cs:1.12687E+02,2.17002E+01)  -- (axis cs:1.13750E+02,2.15434E+01) -- (axis cs:1.13784E+02,2.17408E+01) -- cycle  ;
\draw [ecliptics-full] (axis cs:1.14846E+02,2.15771E+01) -- (axis cs:1.14810E+02,2.13799E+01)  -- (axis cs:1.15867E+02,2.12096E+01) -- (axis cs:1.15905E+02,2.14066E+01) -- cycle  ;
\draw [ecliptics-full] (axis cs:1.16961E+02,2.12295E+01) -- (axis cs:1.16922E+02,2.10328E+01)  -- (axis cs:1.17975E+02,2.08495E+01) -- (axis cs:1.18015E+02,2.10459E+01) -- cycle  ;
\draw [ecliptics-full] (axis cs:1.19066E+02,2.08559E+01) -- (axis cs:1.19025E+02,2.06597E+01)  -- (axis cs:1.20072E+02,2.04635E+01) -- (axis cs:1.20115E+02,2.06595E+01) -- cycle  ;
\draw [ecliptics-full] (axis cs:1.21161E+02,2.04568E+01) -- (axis cs:1.21117E+02,2.02610E+01)  -- (axis cs:1.22159E+02,2.00524E+01) -- (axis cs:1.22204E+02,2.02479E+01) -- cycle  ;
\draw [ecliptics-full] (axis cs:1.23244E+02,2.00328E+01) -- (axis cs:1.23198E+02,1.98376E+01)  -- (axis cs:1.24234E+02,1.96168E+01) -- (axis cs:1.24281E+02,1.98118E+01) -- cycle  ;
\draw [ecliptics-full] (axis cs:1.25316E+02,1.95848E+01) -- (axis cs:1.25267E+02,1.93901E+01)  -- (axis cs:1.26297E+02,1.91575E+01) -- (axis cs:1.26347E+02,1.93519E+01) -- cycle  ;
\draw [ecliptics-full] (axis cs:1.27376E+02,1.91133E+01) -- (axis cs:1.27325E+02,1.89192E+01)  -- (axis cs:1.28349E+02,1.86752E+01) -- (axis cs:1.28401E+02,1.88690E+01) -- cycle  ;
\draw [ecliptics-full] (axis cs:1.29424E+02,1.86192E+01) -- (axis cs:1.29371E+02,1.84256E+01)  -- (axis cs:1.30389E+02,1.81706E+01) -- (axis cs:1.30443E+02,1.83638E+01) -- cycle  ;
\draw [ecliptics-full] (axis cs:1.31460E+02,1.81031E+01) -- (axis cs:1.31404E+02,1.79102E+01)  -- (axis cs:1.32417E+02,1.76445E+01) -- (axis cs:1.32473E+02,1.78371E+01) -- cycle  ;
\draw [ecliptics-full] (axis cs:1.33484E+02,1.75660E+01) -- (axis cs:1.33426E+02,1.73736E+01)  -- (axis cs:1.34433E+02,1.70976E+01) -- (axis cs:1.34491E+02,1.72897E+01) -- cycle  ;
\draw [ecliptics-full] (axis cs:1.35495E+02,1.70085E+01) -- (axis cs:1.35436E+02,1.68167E+01)  -- (axis cs:1.36436E+02,1.65309E+01) -- (axis cs:1.36496E+02,1.67224E+01) -- cycle  ;
\draw [ecliptics-full] (axis cs:1.37495E+02,1.64315E+01) -- (axis cs:1.37434E+02,1.62403E+01)  -- (axis cs:1.38428E+02,1.59450E+01) -- (axis cs:1.38490E+02,1.61359E+01) -- cycle  ;
\draw [ecliptics-full] (axis cs:1.39482E+02,1.58358E+01) -- (axis cs:1.39419E+02,1.56451E+01)  -- (axis cs:1.40408E+02,1.53408E+01) -- (axis cs:1.40471E+02,1.55312E+01) -- cycle  ;
\draw [ecliptics-full] (axis cs:1.41458E+02,1.52222E+01) -- (axis cs:1.41393E+02,1.50321E+01)  -- (axis cs:1.42376E+02,1.47192E+01) -- (axis cs:1.42441E+02,1.49090E+01) -- cycle  ;
\draw [ecliptics-full] (axis cs:1.43422E+02,1.45916E+01) -- (axis cs:1.43356E+02,1.44021E+01)  -- (axis cs:1.44333E+02,1.40809E+01) -- (axis cs:1.44400E+02,1.42701E+01) -- cycle  ;
\draw [ecliptics-full] (axis cs:1.45375E+02,1.39448E+01) -- (axis cs:1.45307E+02,1.37558E+01)  -- (axis cs:1.46279E+02,1.34268E+01) -- (axis cs:1.46347E+02,1.36155E+01) -- cycle  ;
\draw [ecliptics-full] (axis cs:1.47317E+02,1.32826E+01) -- (axis cs:1.47248E+02,1.30941E+01)  -- (axis cs:1.48214E+02,1.27578E+01) -- (axis cs:1.48283E+02,1.29460E+01) -- cycle  ;
\draw [ecliptics-full] (axis cs:1.49248E+02,1.26059E+01) -- (axis cs:1.49178E+02,1.24179E+01)  -- (axis cs:1.50139E+02,1.20746E+01) -- (axis cs:1.50209E+02,1.22623E+01) -- cycle  ;
\draw [ecliptics-full] (axis cs:1.51169E+02,1.19155E+01) -- (axis cs:1.51097E+02,1.17280E+01)  -- (axis cs:1.52054E+02,1.13782E+01) -- (axis cs:1.52125E+02,1.15655E+01) -- cycle  ;
\draw [ecliptics-full] (axis cs:1.53080E+02,1.12123E+01) -- (axis cs:1.53007E+02,1.10253E+01)  -- (axis cs:1.53959E+02,1.06694E+01) -- (axis cs:1.54032E+02,1.08562E+01) -- cycle  ;
\draw [ecliptics-full] (axis cs:1.54981E+02,1.04971E+01) -- (axis cs:1.54908E+02,1.03105E+01)  -- (axis cs:1.55855E+02,9.94894E+00) -- (axis cs:1.55929E+02,1.01353E+01) -- cycle  ;
\draw [ecliptics-full] (axis cs:1.56874E+02,9.77078E+00) -- (axis cs:1.56800E+02,9.58464E+00)  -- (axis cs:1.57743E+02,9.21776E+00) -- (axis cs:1.57818E+02,9.40370E+00) -- cycle  ;
\draw [ecliptics-full] (axis cs:1.58759E+02,9.03415E+00) -- (axis cs:1.58684E+02,8.84840E+00)  -- (axis cs:1.59623E+02,8.47667E+00) -- (axis cs:1.59699E+02,8.66224E+00) -- cycle  ;
\draw [ecliptics-full] (axis cs:1.60636E+02,8.28806E+00) -- (axis cs:1.60560E+02,8.10267E+00)  -- (axis cs:1.61496E+02,7.72651E+00) -- (axis cs:1.61572E+02,7.91173E+00) -- cycle  ;
\draw [ecliptics-full] (axis cs:1.62506E+02,7.53334E+00) -- (axis cs:1.62430E+02,7.34829E+00)  -- (axis cs:1.63362E+02,6.96811E+00) -- (axis cs:1.63439E+02,7.15301E+00) -- cycle  ;
\draw [ecliptics-full] (axis cs:1.64370E+02,6.77084E+00) -- (axis cs:1.64292E+02,6.58609E+00)  -- (axis cs:1.65222E+02,6.20231E+00) -- (axis cs:1.65299E+02,6.38692E+00) -- cycle  ;
\draw [ecliptics-full] (axis cs:1.66227E+02,6.00137E+00) -- (axis cs:1.66150E+02,5.81689E+00)  -- (axis cs:1.67076E+02,5.42993E+00) -- (axis cs:1.67154E+02,5.61429E+00) -- cycle  ;
\draw [ecliptics-full] (axis cs:1.68080E+02,5.22577E+00) -- (axis cs:1.68001E+02,5.04154E+00)  -- (axis cs:1.68926E+02,4.65180E+00) -- (axis cs:1.69004E+02,4.83593E+00) -- cycle  ;
\draw [ecliptics-full] (axis cs:1.69927E+02,4.44487E+00) -- (axis cs:1.69849E+02,4.26084E+00)  -- (axis cs:1.70771E+02,3.86875E+00) -- (axis cs:1.70850E+02,4.05268E+00) -- cycle  ;
\draw [ecliptics-full] (axis cs:1.71771E+02,3.65948E+00) -- (axis cs:1.71692E+02,3.47563E+00)  -- (axis cs:1.72613E+02,3.08158E+00) -- (axis cs:1.72692E+02,3.26536E+00) -- cycle  ;
\draw [ecliptics-full] (axis cs:1.73612E+02,2.87043E+00) -- (axis cs:1.73533E+02,2.68672E+00)  -- (axis cs:1.74452E+02,2.29113E+00) -- (axis cs:1.74532E+02,2.47479E+00) -- cycle  ;
\draw [ecliptics-full] (axis cs:1.75450E+02,2.07854E+00) -- (axis cs:1.75371E+02,1.89493E+00)  -- (axis cs:1.76289E+02,1.49822E+00) -- (axis cs:1.76369E+02,1.68179E+00) -- cycle  ;
\draw [ecliptics-full] (axis cs:1.77287E+02,1.28463E+00) -- (axis cs:1.77207E+02,1.10109E+00)  -- (axis cs:1.78125E+02,7.03652E-01) -- (axis cs:1.78205E+02,8.87166E-01) -- cycle  ;
\draw [ecliptics-full] (axis cs:1.79122E+02,4.89506E-01) -- (axis cs:1.79043E+02,3.06005E-01)  -- (axis cs:1.79960E+02,-9.17483E-02) -- (axis cs:1.80040E+02,9.17485E-02) -- cycle ;




\draw [ecliptics-full] (axis cs:3.58125E+02,-7.03651E-01) -- (axis cs:3.58205E+02,-8.87166E-01)  -- (axis cs:3.57287E+02,-1.28463E+00) -- (axis cs:3.57207E+02,-1.10109E+00) -- cycle  ;
\draw [ecliptics-full] (axis cs:3.56289E+02,-1.49822E+00) -- (axis cs:3.56369E+02,-1.68179E+00)  -- (axis cs:3.55450E+02,-2.07854E+00) -- (axis cs:3.55371E+02,-1.89493E+00) -- cycle  ;
\draw [ecliptics-full] (axis cs:3.54452E+02,-2.29113E+00) -- (axis cs:3.54532E+02,-2.47479E+00)  -- (axis cs:3.53612E+02,-2.87043E+00) -- (axis cs:3.53533E+02,-2.68672E+00) -- cycle  ;
\draw [ecliptics-full] (axis cs:3.52613E+02,-3.08158E+00) -- (axis cs:3.52692E+02,-3.26536E+00)  -- (axis cs:3.51771E+02,-3.65948E+00) -- (axis cs:3.51693E+02,-3.47563E+00) -- cycle  ;
\draw [ecliptics-full] (axis cs:3.50771E+02,-3.86875E+00) -- (axis cs:3.50850E+02,-4.05268E+00)  -- (axis cs:3.49927E+02,-4.44487E+00) -- (axis cs:3.49849E+02,-4.26084E+00) -- cycle  ;
\draw [ecliptics-full] (axis cs:3.48926E+02,-4.65180E+00) -- (axis cs:3.49004E+02,-4.83593E+00)  -- (axis cs:3.48080E+02,-5.22577E+00) -- (axis cs:3.48001E+02,-5.04154E+00) -- cycle  ;
\draw [ecliptics-full] (axis cs:3.47076E+02,-5.42993E+00) -- (axis cs:3.47154E+02,-5.61429E+00)  -- (axis cs:3.46227E+02,-6.00137E+00) -- (axis cs:3.46150E+02,-5.81689E+00) -- cycle  ;
\draw [ecliptics-full] (axis cs:3.45222E+02,-6.20231E+00) -- (axis cs:3.45299E+02,-6.38692E+00)  -- (axis cs:3.44370E+02,-6.77084E+00) -- (axis cs:3.44292E+02,-6.58609E+00) -- cycle  ;
\draw [ecliptics-full] (axis cs:3.43362E+02,-6.96811E+00) -- (axis cs:3.43439E+02,-7.15301E+00)  -- (axis cs:3.42506E+02,-7.53334E+00) -- (axis cs:3.42430E+02,-7.34829E+00) -- cycle  ;
\draw [ecliptics-full] (axis cs:3.41496E+02,-7.72651E+00) -- (axis cs:3.41572E+02,-7.91173E+00)  -- (axis cs:3.40636E+02,-8.28806E+00) -- (axis cs:3.40560E+02,-8.10267E+00) -- cycle  ;
\draw [ecliptics-full] (axis cs:3.39623E+02,-8.47667E+00) -- (axis cs:3.39699E+02,-8.66224E+00)  -- (axis cs:3.38759E+02,-9.03415E+00) -- (axis cs:3.38684E+02,-8.84840E+00) -- cycle  ;
\draw [ecliptics-full] (axis cs:3.37743E+02,-9.21776E+00) -- (axis cs:3.37818E+02,-9.40370E+00)  -- (axis cs:3.36874E+02,-9.77078E+00) -- (axis cs:3.36800E+02,-9.58464E+00) -- cycle  ;
\draw [ecliptics-full] (axis cs:3.35855E+02,-9.94894E+00) -- (axis cs:3.35929E+02,-1.01353E+01)  -- (axis cs:3.34981E+02,-1.04971E+01) -- (axis cs:3.34908E+02,-1.03105E+01) -- cycle  ;
\draw [ecliptics-full] (axis cs:3.33959E+02,-1.06694E+01) -- (axis cs:3.34032E+02,-1.08562E+01)  -- (axis cs:3.33080E+02,-1.12123E+01) -- (axis cs:3.33007E+02,-1.10253E+01) -- cycle  ;
\draw [ecliptics-full] (axis cs:3.32054E+02,-1.13782E+01) -- (axis cs:3.32125E+02,-1.15655E+01)  -- (axis cs:3.31169E+02,-1.19155E+01) -- (axis cs:3.31097E+02,-1.17280E+01) -- cycle  ;
\draw [ecliptics-full] (axis cs:3.30139E+02,-1.20746E+01) -- (axis cs:3.30209E+02,-1.22623E+01)  -- (axis cs:3.29248E+02,-1.26059E+01) -- (axis cs:3.29178E+02,-1.24179E+01) -- cycle  ;
\draw [ecliptics-full] (axis cs:3.28214E+02,-1.27578E+01) -- (axis cs:3.28283E+02,-1.29460E+01)  -- (axis cs:3.27316E+02,-1.32826E+01) -- (axis cs:3.27248E+02,-1.30941E+01) -- cycle  ;
\draw [ecliptics-full] (axis cs:3.26279E+02,-1.34268E+01) -- (axis cs:3.26347E+02,-1.36155E+01)  -- (axis cs:3.25375E+02,-1.39448E+01) -- (axis cs:3.25307E+02,-1.37558E+01) -- cycle  ;
\draw [ecliptics-full] (axis cs:3.24333E+02,-1.40809E+01) -- (axis cs:3.24400E+02,-1.42701E+01)  -- (axis cs:3.23422E+02,-1.45916E+01) -- (axis cs:3.23356E+02,-1.44021E+01) -- cycle  ;
\draw [ecliptics-full] (axis cs:3.22376E+02,-1.47192E+01) -- (axis cs:3.22441E+02,-1.49090E+01)  -- (axis cs:3.21458E+02,-1.52222E+01) -- (axis cs:3.21393E+02,-1.50321E+01) -- cycle  ;
\draw [ecliptics-full] (axis cs:3.20408E+02,-1.53408E+01) -- (axis cs:3.20471E+02,-1.55312E+01)  -- (axis cs:3.19482E+02,-1.58358E+01) -- (axis cs:3.19419E+02,-1.56451E+01) -- cycle  ;
\draw [ecliptics-full] (axis cs:3.18428E+02,-1.59450E+01) -- (axis cs:3.18490E+02,-1.61359E+01)  -- (axis cs:3.17495E+02,-1.64315E+01) -- (axis cs:3.17434E+02,-1.62403E+01) -- cycle  ;
\draw [ecliptics-full] (axis cs:3.16436E+02,-1.65309E+01) -- (axis cs:3.16496E+02,-1.67224E+01)  -- (axis cs:3.15495E+02,-1.70085E+01) -- (axis cs:3.15436E+02,-1.68167E+01) -- cycle  ;
\draw [ecliptics-full] (axis cs:3.14433E+02,-1.70976E+01) -- (axis cs:3.14491E+02,-1.72897E+01)  -- (axis cs:3.13484E+02,-1.75660E+01) -- (axis cs:3.13426E+02,-1.73736E+01) -- cycle  ;
\draw [ecliptics-full] (axis cs:3.12417E+02,-1.76445E+01) -- (axis cs:3.12473E+02,-1.78371E+01)  -- (axis cs:3.11460E+02,-1.81031E+01) -- (axis cs:3.11404E+02,-1.79102E+01) -- cycle  ;
\draw [ecliptics-full] (axis cs:3.10389E+02,-1.81706E+01) -- (axis cs:3.10443E+02,-1.83638E+01)  -- (axis cs:3.09424E+02,-1.86192E+01) -- (axis cs:3.09371E+02,-1.84256E+01) -- cycle  ;
\draw [ecliptics-full] (axis cs:3.08349E+02,-1.86752E+01) -- (axis cs:3.08401E+02,-1.88690E+01)  -- (axis cs:3.07376E+02,-1.91133E+01) -- (axis cs:3.07325E+02,-1.89192E+01) -- cycle  ;
\draw [ecliptics-full] (axis cs:3.06297E+02,-1.91575E+01) -- (axis cs:3.06347E+02,-1.93519E+01)  -- (axis cs:3.05316E+02,-1.95848E+01) -- (axis cs:3.05267E+02,-1.93901E+01) -- cycle  ;
\draw [ecliptics-full] (axis cs:3.04234E+02,-1.96168E+01) -- (axis cs:3.04281E+02,-1.98118E+01)  -- (axis cs:3.03244E+02,-2.00328E+01) -- (axis cs:3.03198E+02,-1.98376E+01) -- cycle  ;
\draw [ecliptics-full] (axis cs:3.02159E+02,-2.00524E+01) -- (axis cs:3.02204E+02,-2.02479E+01)  -- (axis cs:3.01161E+02,-2.04568E+01) -- (axis cs:3.01117E+02,-2.02610E+01) -- cycle  ;
\draw [ecliptics-full] (axis cs:3.00072E+02,-2.04635E+01) -- (axis cs:3.00115E+02,-2.06595E+01)  -- (axis cs:2.99066E+02,-2.08559E+01) -- (axis cs:2.99025E+02,-2.06597E+01) -- cycle  ;
\draw [ecliptics-full] (axis cs:2.97975E+02,-2.08495E+01) -- (axis cs:2.98015E+02,-2.10459E+01)  -- (axis cs:2.96961E+02,-2.12295E+01) -- (axis cs:2.96922E+02,-2.10328E+01) -- cycle  ;
\draw [ecliptics-full] (axis cs:2.95867E+02,-2.12096E+01) -- (axis cs:2.95905E+02,-2.14066E+01)  -- (axis cs:2.94846E+02,-2.15771E+01) -- (axis cs:2.94810E+02,-2.13799E+01) -- cycle  ;
\draw [ecliptics-full] (axis cs:2.93750E+02,-2.15434E+01) -- (axis cs:2.93784E+02,-2.17408E+01)  -- (axis cs:2.92720E+02,-2.18979E+01) -- (axis cs:2.92687E+02,-2.17002E+01) -- cycle  ;
\draw [ecliptics-full] (axis cs:2.91623E+02,-2.18502E+01) -- (axis cs:2.91654E+02,-2.20481E+01)  -- (axis cs:2.90586E+02,-2.21914E+01) -- (axis cs:2.90556E+02,-2.19934E+01) -- cycle  ;
\draw [ecliptics-full] (axis cs:2.89487E+02,-2.21296E+01) -- (axis cs:2.89516E+02,-2.23278E+01)  -- (axis cs:2.88443E+02,-2.24572E+01) -- (axis cs:2.88416E+02,-2.22587E+01) -- cycle  ;
\draw [ecliptics-full] (axis cs:2.87343E+02,-2.23809E+01) -- (axis cs:2.87369E+02,-2.25795E+01)  -- (axis cs:2.86292E+02,-2.26946E+01) -- (axis cs:2.86268E+02,-2.24959E+01) -- cycle  ;
\draw [ecliptics-full] (axis cs:2.85192E+02,-2.26037E+01) -- (axis cs:2.85214E+02,-2.28026E+01)  -- (axis cs:2.84135E+02,-2.29034E+01) -- (axis cs:2.84114E+02,-2.27043E+01) -- cycle  ;
\draw [ecliptics-full] (axis cs:2.83034E+02,-2.27977E+01) -- (axis cs:2.83054E+02,-2.29969E+01)  -- (axis cs:2.81971E+02,-2.30830E+01) -- (axis cs:2.81953E+02,-2.28837E+01) -- cycle  ;
\draw [ecliptics-full] (axis cs:2.80871E+02,-2.29624E+01) -- (axis cs:2.80887E+02,-2.31619E+01)  -- (axis cs:2.79802E+02,-2.32333E+01) -- (axis cs:2.79787E+02,-2.30338E+01) -- cycle  ;
\draw [ecliptics-full] (axis cs:2.78702E+02,-2.30977E+01) -- (axis cs:2.78715E+02,-2.32973E+01)  -- (axis cs:2.77628E+02,-2.33538E+01) -- (axis cs:2.77617E+02,-2.31541E+01) -- cycle  ;
\draw [ecliptics-full] (axis cs:2.76530E+02,-2.32031E+01) -- (axis cs:2.76540E+02,-2.34029E+01)  -- (axis cs:2.75451E+02,-2.34444E+01) -- (axis cs:2.75443E+02,-2.32446E+01) -- cycle  ;
\draw [ecliptics-full] (axis cs:2.74355E+02,-2.32785E+01) -- (axis cs:2.74362E+02,-2.34785E+01)  -- (axis cs:2.73272E+02,-2.35049E+01) -- (axis cs:2.73267E+02,-2.33050E+01) -- cycle  ;
\draw [ecliptics-full] (axis cs:2.72178E+02,-2.33239E+01) -- (axis cs:2.72181E+02,-2.35239E+01)  -- (axis cs:2.71091E+02,-2.35352E+01) -- (axis cs:2.71089E+02,-2.33352E+01) -- cycle  ;
\draw [ecliptics-full] (axis cs:2.70000E+02,-2.33390E+01) -- (axis cs:2.70000E+02,-2.35390E+01)  -- (axis cs:2.68909E+02,-2.35352E+01) -- (axis cs:2.68911E+02,-2.33352E+01) -- cycle  ;
\draw [ecliptics-full] (axis cs:2.67822E+02,-2.33239E+01) -- (axis cs:2.67819E+02,-2.35239E+01)  -- (axis cs:2.66728E+02,-2.35049E+01) -- (axis cs:2.66733E+02,-2.33050E+01) -- cycle  ;
\draw [ecliptics-full] (axis cs:2.65645E+02,-2.32785E+01) -- (axis cs:2.65638E+02,-2.34785E+01)  -- (axis cs:2.64549E+02,-2.34444E+01) -- (axis cs:2.64557E+02,-2.32446E+01) -- cycle  ;
\draw [ecliptics-full] (axis cs:2.63470E+02,-2.32031E+01) -- (axis cs:2.63460E+02,-2.34029E+01)  -- (axis cs:2.62372E+02,-2.33538E+01) -- (axis cs:2.62383E+02,-2.31541E+01) -- cycle  ;
\draw [ecliptics-full] (axis cs:2.61298E+02,-2.30977E+01) -- (axis cs:2.61285E+02,-2.32973E+01)  -- (axis cs:2.60198E+02,-2.32333E+01) -- (axis cs:2.60213E+02,-2.30338E+01) -- cycle  ;
\draw [ecliptics-full] (axis cs:2.59129E+02,-2.29624E+01) -- (axis cs:2.59113E+02,-2.31619E+01)  -- (axis cs:2.58029E+02,-2.30830E+01) -- (axis cs:2.58047E+02,-2.28837E+01) -- cycle  ;
\draw [ecliptics-full] (axis cs:2.56966E+02,-2.27977E+01) -- (axis cs:2.56946E+02,-2.29969E+01)  -- (axis cs:2.55865E+02,-2.29034E+01) -- (axis cs:2.55886E+02,-2.27043E+01) -- cycle  ;
\draw [ecliptics-full] (axis cs:2.54808E+02,-2.26037E+01) -- (axis cs:2.54786E+02,-2.28026E+01)  -- (axis cs:2.53708E+02,-2.26946E+01) -- (axis cs:2.53732E+02,-2.24959E+01) -- cycle  ;
\draw [ecliptics-full] (axis cs:2.52657E+02,-2.23809E+01) -- (axis cs:2.52631E+02,-2.25795E+01)  -- (axis cs:2.51557E+02,-2.24572E+01) -- (axis cs:2.51584E+02,-2.22587E+01) -- cycle  ;
\draw [ecliptics-full] (axis cs:2.50513E+02,-2.21296E+01) -- (axis cs:2.50484E+02,-2.23278E+01)  -- (axis cs:2.49414E+02,-2.21914E+01) -- (axis cs:2.49444E+02,-2.19934E+01) -- cycle  ;
\draw [ecliptics-full] (axis cs:2.48377E+02,-2.18502E+01) -- (axis cs:2.48346E+02,-2.20481E+01)  -- (axis cs:2.47280E+02,-2.18979E+01) -- (axis cs:2.47313E+02,-2.17002E+01) -- cycle  ;
\draw [ecliptics-full] (axis cs:2.46250E+02,-2.15434E+01) -- (axis cs:2.46216E+02,-2.17408E+01)  -- (axis cs:2.45154E+02,-2.15771E+01) -- (axis cs:2.45190E+02,-2.13799E+01) -- cycle  ;
\draw [ecliptics-full] (axis cs:2.44133E+02,-2.12096E+01) -- (axis cs:2.44095E+02,-2.14066E+01)  -- (axis cs:2.43039E+02,-2.12295E+01) -- (axis cs:2.43078E+02,-2.10328E+01) -- cycle  ;
\draw [ecliptics-full] (axis cs:2.42025E+02,-2.08495E+01) -- (axis cs:2.41985E+02,-2.10459E+01)  -- (axis cs:2.40934E+02,-2.08559E+01) -- (axis cs:2.40975E+02,-2.06597E+01) -- cycle  ;
\draw [ecliptics-full] (axis cs:2.39928E+02,-2.04635E+01) -- (axis cs:2.39885E+02,-2.06595E+01)  -- (axis cs:2.38839E+02,-2.04568E+01) -- (axis cs:2.38883E+02,-2.02610E+01) -- cycle  ;
\draw [ecliptics-full] (axis cs:2.37841E+02,-2.00524E+01) -- (axis cs:2.37796E+02,-2.02479E+01)  -- (axis cs:2.36756E+02,-2.00328E+01) -- (axis cs:2.36802E+02,-1.98376E+01) -- cycle  ;
\draw [ecliptics-full] (axis cs:2.35766E+02,-1.96168E+01) -- (axis cs:2.35719E+02,-1.98118E+01)  -- (axis cs:2.34684E+02,-1.95848E+01) -- (axis cs:2.34733E+02,-1.93901E+01) -- cycle  ;
\draw [ecliptics-full] (axis cs:2.33703E+02,-1.91575E+01) -- (axis cs:2.33653E+02,-1.93519E+01)  -- (axis cs:2.32624E+02,-1.91133E+01) -- (axis cs:2.32675E+02,-1.89192E+01) -- cycle  ;
\draw [ecliptics-full] (axis cs:2.31651E+02,-1.86752E+01) -- (axis cs:2.31599E+02,-1.88690E+01)  -- (axis cs:2.30576E+02,-1.86192E+01) -- (axis cs:2.30629E+02,-1.84256E+01) -- cycle  ;
\draw [ecliptics-full] (axis cs:2.29611E+02,-1.81706E+01) -- (axis cs:2.29557E+02,-1.83638E+01)  -- (axis cs:2.28540E+02,-1.81031E+01) -- (axis cs:2.28596E+02,-1.79102E+01) -- cycle  ;
\draw [ecliptics-full] (axis cs:2.27583E+02,-1.76445E+01) -- (axis cs:2.27527E+02,-1.78371E+01)  -- (axis cs:2.26516E+02,-1.75660E+01) -- (axis cs:2.26574E+02,-1.73736E+01) -- cycle  ;
\draw [ecliptics-full] (axis cs:2.25567E+02,-1.70976E+01) -- (axis cs:2.25509E+02,-1.72897E+01)  -- (axis cs:2.24505E+02,-1.70085E+01) -- (axis cs:2.24564E+02,-1.68167E+01) -- cycle  ;
\draw [ecliptics-full] (axis cs:2.23564E+02,-1.65309E+01) -- (axis cs:2.23504E+02,-1.67224E+01)  -- (axis cs:2.22505E+02,-1.64315E+01) -- (axis cs:2.22566E+02,-1.62403E+01) -- cycle  ;
\draw [ecliptics-full] (axis cs:2.21572E+02,-1.59450E+01) -- (axis cs:2.21510E+02,-1.61359E+01)  -- (axis cs:2.20518E+02,-1.58358E+01) -- (axis cs:2.20581E+02,-1.56451E+01) -- cycle  ;
\draw [ecliptics-full] (axis cs:2.19592E+02,-1.53408E+01) -- (axis cs:2.19529E+02,-1.55312E+01)  -- (axis cs:2.18542E+02,-1.52222E+01) -- (axis cs:2.18607E+02,-1.50321E+01) -- cycle  ;
\draw [ecliptics-full] (axis cs:2.17624E+02,-1.47192E+01) -- (axis cs:2.17559E+02,-1.49090E+01)  -- (axis cs:2.16578E+02,-1.45916E+01) -- (axis cs:2.16644E+02,-1.44021E+01) -- cycle  ;
\draw [ecliptics-full] (axis cs:2.15667E+02,-1.40809E+01) -- (axis cs:2.15600E+02,-1.42701E+01)  -- (axis cs:2.14625E+02,-1.39448E+01) -- (axis cs:2.14693E+02,-1.37558E+01) -- cycle  ;
\draw [ecliptics-full] (axis cs:2.13721E+02,-1.34268E+01) -- (axis cs:2.13653E+02,-1.36155E+01)  -- (axis cs:2.12683E+02,-1.32826E+01) -- (axis cs:2.12752E+02,-1.30941E+01) -- cycle  ;
\draw [ecliptics-full] (axis cs:2.11786E+02,-1.27578E+01) -- (axis cs:2.11717E+02,-1.29460E+01)  -- (axis cs:2.10752E+02,-1.26059E+01) -- (axis cs:2.10822E+02,-1.24179E+01) -- cycle  ;
\draw [ecliptics-full] (axis cs:2.09861E+02,-1.20746E+01) -- (axis cs:2.09791E+02,-1.22623E+01)  -- (axis cs:2.08831E+02,-1.19155E+01) -- (axis cs:2.08903E+02,-1.17280E+01) -- cycle  ;
\draw [ecliptics-full] (axis cs:2.07946E+02,-1.13782E+01) -- (axis cs:2.07875E+02,-1.15655E+01)  -- (axis cs:2.06920E+02,-1.12123E+01) -- (axis cs:2.06993E+02,-1.10253E+01) -- cycle  ;
\draw [ecliptics-full] (axis cs:2.06041E+02,-1.06694E+01) -- (axis cs:2.05968E+02,-1.08562E+01)  -- (axis cs:2.05019E+02,-1.04971E+01) -- (axis cs:2.05092E+02,-1.03105E+01) -- cycle  ;
\draw [ecliptics-full] (axis cs:2.04145E+02,-9.94894E+00) -- (axis cs:2.04071E+02,-1.01353E+01)  -- (axis cs:2.03126E+02,-9.77078E+00) -- (axis cs:2.03200E+02,-9.58464E+00) -- cycle  ;
\draw [ecliptics-full] (axis cs:2.02257E+02,-9.21776E+00) -- (axis cs:2.02182E+02,-9.40370E+00)  -- (axis cs:2.01241E+02,-9.03415E+00) -- (axis cs:2.01316E+02,-8.84840E+00) -- cycle  ;
\draw [ecliptics-full] (axis cs:2.00377E+02,-8.47667E+00) -- (axis cs:2.00301E+02,-8.66224E+00)  -- (axis cs:1.99364E+02,-8.28806E+00) -- (axis cs:1.99440E+02,-8.10267E+00) -- cycle  ;
\draw [ecliptics-full] (axis cs:1.98504E+02,-7.72651E+00) -- (axis cs:1.98428E+02,-7.91173E+00)  -- (axis cs:1.97494E+02,-7.53334E+00) -- (axis cs:1.97570E+02,-7.34829E+00) -- cycle  ;
\draw [ecliptics-full] (axis cs:1.96638E+02,-6.96811E+00) -- (axis cs:1.96561E+02,-7.15301E+00)  -- (axis cs:1.95630E+02,-6.77084E+00) -- (axis cs:1.95708E+02,-6.58609E+00) -- cycle  ;
\draw [ecliptics-full] (axis cs:1.94778E+02,-6.20231E+00) -- (axis cs:1.94701E+02,-6.38692E+00)  -- (axis cs:1.93773E+02,-6.00137E+00) -- (axis cs:1.93850E+02,-5.81689E+00) -- cycle  ;
\draw [ecliptics-full] (axis cs:1.92924E+02,-5.42993E+00) -- (axis cs:1.92846E+02,-5.61429E+00)  -- (axis cs:1.91920E+02,-5.22577E+00) -- (axis cs:1.91999E+02,-5.04154E+00) -- cycle  ;
\draw [ecliptics-full] (axis cs:1.91074E+02,-4.65180E+00) -- (axis cs:1.90996E+02,-4.83593E+00)  -- (axis cs:1.90073E+02,-4.44487E+00) -- (axis cs:1.90151E+02,-4.26084E+00) -- cycle  ;
\draw [ecliptics-full] (axis cs:1.89229E+02,-3.86875E+00) -- (axis cs:1.89150E+02,-4.05268E+00)  -- (axis cs:1.88229E+02,-3.65948E+00) -- (axis cs:1.88308E+02,-3.47563E+00) -- cycle  ;
\draw [ecliptics-full] (axis cs:1.87387E+02,-3.08158E+00) -- (axis cs:1.87308E+02,-3.26536E+00)  -- (axis cs:1.86388E+02,-2.87043E+00) -- (axis cs:1.86467E+02,-2.68672E+00) -- cycle  ;
\draw [ecliptics-full] (axis cs:1.85548E+02,-2.29113E+00) -- (axis cs:1.85468E+02,-2.47479E+00)  -- (axis cs:1.84550E+02,-2.07854E+00) -- (axis cs:1.84629E+02,-1.89493E+00) -- cycle  ;
\draw [ecliptics-full] (axis cs:1.83711E+02,-1.49822E+00) -- (axis cs:1.83631E+02,-1.68179E+00)  -- (axis cs:1.82713E+02,-1.28463E+00) -- (axis cs:1.82793E+02,-1.10109E+00) -- cycle  ;
\draw [ecliptics-full] (axis cs:1.81875E+02,-7.03652E-01) -- (axis cs:1.81795E+02,-8.87166E-01)  -- (axis cs:1.80878E+02,-4.89506E-01) -- (axis cs:1.80957E+02,-3.06005E-01) -- cycle  ;

\draw [ecliptics-empty] (axis cs:3.59043E+02,-3.06005E-01) -- (axis cs:3.59122E+02,-4.89506E-01)  -- (axis cs:3.58205E+02,-8.87166E-01) -- (axis cs:3.58125E+02,-7.03651E-01) -- cycle  ;
\draw [ecliptics-empty] (axis cs:3.57207E+02,-1.10109E+00) -- (axis cs:3.57287E+02,-1.28463E+00)  -- (axis cs:3.56369E+02,-1.68179E+00) -- (axis cs:3.56289E+02,-1.49822E+00) -- cycle  ;
\draw [ecliptics-empty] (axis cs:3.55371E+02,-1.89493E+00) -- (axis cs:3.55450E+02,-2.07854E+00)  -- (axis cs:3.54532E+02,-2.47479E+00) -- (axis cs:3.54452E+02,-2.29113E+00) -- cycle  ;
\draw [ecliptics-empty] (axis cs:3.53533E+02,-2.68672E+00) -- (axis cs:3.53612E+02,-2.87043E+00)  -- (axis cs:3.52692E+02,-3.26536E+00) -- (axis cs:3.52613E+02,-3.08158E+00) -- cycle  ;
\draw [ecliptics-empty] (axis cs:3.51693E+02,-3.47563E+00) -- (axis cs:3.51771E+02,-3.65948E+00)  -- (axis cs:3.50850E+02,-4.05268E+00) -- (axis cs:3.50771E+02,-3.86875E+00) -- cycle  ;
\draw [ecliptics-empty] (axis cs:3.49849E+02,-4.26084E+00) -- (axis cs:3.49927E+02,-4.44487E+00)  -- (axis cs:3.49004E+02,-4.83593E+00) -- (axis cs:3.48926E+02,-4.65180E+00) -- cycle  ;
\draw [ecliptics-empty] (axis cs:3.48001E+02,-5.04154E+00) -- (axis cs:3.48080E+02,-5.22577E+00)  -- (axis cs:3.47154E+02,-5.61429E+00) -- (axis cs:3.47076E+02,-5.42993E+00) -- cycle  ;
\draw [ecliptics-empty] (axis cs:3.46150E+02,-5.81689E+00) -- (axis cs:3.46227E+02,-6.00137E+00)  -- (axis cs:3.45299E+02,-6.38692E+00) -- (axis cs:3.45222E+02,-6.20231E+00) -- cycle  ;
\draw [ecliptics-empty] (axis cs:3.44292E+02,-6.58609E+00) -- (axis cs:3.44370E+02,-6.77084E+00)  -- (axis cs:3.43439E+02,-7.15301E+00) -- (axis cs:3.43362E+02,-6.96811E+00) -- cycle  ;
\draw [ecliptics-empty] (axis cs:3.42430E+02,-7.34829E+00) -- (axis cs:3.42506E+02,-7.53334E+00)  -- (axis cs:3.41572E+02,-7.91173E+00) -- (axis cs:3.41496E+02,-7.72651E+00) -- cycle  ;
\draw [ecliptics-empty] (axis cs:3.40560E+02,-8.10267E+00) -- (axis cs:3.40636E+02,-8.28806E+00)  -- (axis cs:3.39699E+02,-8.66224E+00) -- (axis cs:3.39623E+02,-8.47667E+00) -- cycle  ;
\draw [ecliptics-empty] (axis cs:3.38684E+02,-8.84840E+00) -- (axis cs:3.38759E+02,-9.03415E+00)  -- (axis cs:3.37818E+02,-9.40370E+00) -- (axis cs:3.37743E+02,-9.21776E+00) -- cycle  ;
\draw [ecliptics-empty] (axis cs:3.36800E+02,-9.58464E+00) -- (axis cs:3.36874E+02,-9.77078E+00)  -- (axis cs:3.35929E+02,-1.01353E+01) -- (axis cs:3.35855E+02,-9.94894E+00) -- cycle  ;
\draw [ecliptics-empty] (axis cs:3.34908E+02,-1.03105E+01) -- (axis cs:3.34981E+02,-1.04971E+01)  -- (axis cs:3.34032E+02,-1.08562E+01) -- (axis cs:3.33959E+02,-1.06694E+01) -- cycle  ;
\draw [ecliptics-empty] (axis cs:3.33007E+02,-1.10253E+01) -- (axis cs:3.33080E+02,-1.12123E+01)  -- (axis cs:3.32125E+02,-1.15655E+01) -- (axis cs:3.32054E+02,-1.13782E+01) -- cycle  ;
\draw [ecliptics-empty] (axis cs:3.31097E+02,-1.17280E+01) -- (axis cs:3.31169E+02,-1.19155E+01)  -- (axis cs:3.30209E+02,-1.22623E+01) -- (axis cs:3.30139E+02,-1.20746E+01) -- cycle  ;
\draw [ecliptics-empty] (axis cs:3.29178E+02,-1.24179E+01) -- (axis cs:3.29248E+02,-1.26059E+01)  -- (axis cs:3.28283E+02,-1.29460E+01) -- (axis cs:3.28214E+02,-1.27578E+01) -- cycle  ;
\draw [ecliptics-empty] (axis cs:3.27248E+02,-1.30941E+01) -- (axis cs:3.27316E+02,-1.32826E+01)  -- (axis cs:3.26347E+02,-1.36155E+01) -- (axis cs:3.26279E+02,-1.34268E+01) -- cycle  ;
\draw [ecliptics-empty] (axis cs:3.25307E+02,-1.37558E+01) -- (axis cs:3.25375E+02,-1.39448E+01)  -- (axis cs:3.24400E+02,-1.42701E+01) -- (axis cs:3.24333E+02,-1.40809E+01) -- cycle  ;
\draw [ecliptics-empty] (axis cs:3.23356E+02,-1.44021E+01) -- (axis cs:3.23422E+02,-1.45916E+01)  -- (axis cs:3.22441E+02,-1.49090E+01) -- (axis cs:3.22376E+02,-1.47192E+01) -- cycle  ;
\draw [ecliptics-empty] (axis cs:3.21393E+02,-1.50321E+01) -- (axis cs:3.21458E+02,-1.52222E+01)  -- (axis cs:3.20471E+02,-1.55312E+01) -- (axis cs:3.20408E+02,-1.53408E+01) -- cycle  ;
\draw [ecliptics-empty] (axis cs:3.19419E+02,-1.56451E+01) -- (axis cs:3.19482E+02,-1.58358E+01)  -- (axis cs:3.18490E+02,-1.61359E+01) -- (axis cs:3.18428E+02,-1.59450E+01) -- cycle  ;
\draw [ecliptics-empty] (axis cs:3.17434E+02,-1.62403E+01) -- (axis cs:3.17495E+02,-1.64315E+01)  -- (axis cs:3.16496E+02,-1.67224E+01) -- (axis cs:3.16436E+02,-1.65309E+01) -- cycle  ;
\draw [ecliptics-empty] (axis cs:3.15436E+02,-1.68167E+01) -- (axis cs:3.15495E+02,-1.70085E+01)  -- (axis cs:3.14491E+02,-1.72897E+01) -- (axis cs:3.14433E+02,-1.70976E+01) -- cycle  ;
\draw [ecliptics-empty] (axis cs:3.13426E+02,-1.73736E+01) -- (axis cs:3.13484E+02,-1.75660E+01)  -- (axis cs:3.12473E+02,-1.78371E+01) -- (axis cs:3.12417E+02,-1.76445E+01) -- cycle  ;
\draw [ecliptics-empty] (axis cs:3.11404E+02,-1.79102E+01) -- (axis cs:3.11460E+02,-1.81031E+01)  -- (axis cs:3.10443E+02,-1.83638E+01) -- (axis cs:3.10389E+02,-1.81706E+01) -- cycle  ;
\draw [ecliptics-empty] (axis cs:3.09371E+02,-1.84256E+01) -- (axis cs:3.09424E+02,-1.86192E+01)  -- (axis cs:3.08401E+02,-1.88690E+01) -- (axis cs:3.08349E+02,-1.86752E+01) -- cycle  ;
\draw [ecliptics-empty] (axis cs:3.07325E+02,-1.89192E+01) -- (axis cs:3.07376E+02,-1.91133E+01)  -- (axis cs:3.06347E+02,-1.93519E+01) -- (axis cs:3.06297E+02,-1.91575E+01) -- cycle  ;
\draw [ecliptics-empty] (axis cs:3.05267E+02,-1.93901E+01) -- (axis cs:3.05316E+02,-1.95848E+01)  -- (axis cs:3.04281E+02,-1.98118E+01) -- (axis cs:3.04234E+02,-1.96168E+01) -- cycle  ;
\draw [ecliptics-empty] (axis cs:3.03198E+02,-1.98376E+01) -- (axis cs:3.03244E+02,-2.00328E+01)  -- (axis cs:3.02204E+02,-2.02479E+01) -- (axis cs:3.02159E+02,-2.00524E+01) -- cycle  ;
\draw [ecliptics-empty] (axis cs:3.01117E+02,-2.02610E+01) -- (axis cs:3.01161E+02,-2.04568E+01)  -- (axis cs:3.00115E+02,-2.06595E+01) -- (axis cs:3.00072E+02,-2.04635E+01) -- cycle  ;
\draw [ecliptics-empty] (axis cs:2.99025E+02,-2.06597E+01) -- (axis cs:2.99066E+02,-2.08559E+01)  -- (axis cs:2.98015E+02,-2.10459E+01) -- (axis cs:2.97975E+02,-2.08495E+01) -- cycle  ;
\draw [ecliptics-empty] (axis cs:2.96922E+02,-2.10328E+01) -- (axis cs:2.96961E+02,-2.12295E+01)  -- (axis cs:2.95905E+02,-2.14066E+01) -- (axis cs:2.95867E+02,-2.12096E+01) -- cycle  ;
\draw [ecliptics-empty] (axis cs:2.94810E+02,-2.13799E+01) -- (axis cs:2.94846E+02,-2.15771E+01)  -- (axis cs:2.93784E+02,-2.17408E+01) -- (axis cs:2.93750E+02,-2.15434E+01) -- cycle  ;
\draw [ecliptics-empty] (axis cs:2.92687E+02,-2.17002E+01) -- (axis cs:2.92720E+02,-2.18979E+01)  -- (axis cs:2.91654E+02,-2.20481E+01) -- (axis cs:2.91623E+02,-2.18502E+01) -- cycle  ;
\draw [ecliptics-empty] (axis cs:2.90556E+02,-2.19934E+01) -- (axis cs:2.90586E+02,-2.21914E+01)  -- (axis cs:2.89516E+02,-2.23278E+01) -- (axis cs:2.89487E+02,-2.21296E+01) -- cycle  ;
\draw [ecliptics-empty] (axis cs:2.88416E+02,-2.22587E+01) -- (axis cs:2.88443E+02,-2.24572E+01)  -- (axis cs:2.87369E+02,-2.25795E+01) -- (axis cs:2.87343E+02,-2.23809E+01) -- cycle  ;
\draw [ecliptics-empty] (axis cs:2.86268E+02,-2.24959E+01) -- (axis cs:2.86292E+02,-2.26946E+01)  -- (axis cs:2.85214E+02,-2.28026E+01) -- (axis cs:2.85192E+02,-2.26037E+01) -- cycle  ;
\draw [ecliptics-empty] (axis cs:2.84114E+02,-2.27043E+01) -- (axis cs:2.84135E+02,-2.29034E+01)  -- (axis cs:2.83054E+02,-2.29969E+01) -- (axis cs:2.83034E+02,-2.27977E+01) -- cycle  ;
\draw [ecliptics-empty] (axis cs:2.81953E+02,-2.28837E+01) -- (axis cs:2.81971E+02,-2.30830E+01)  -- (axis cs:2.80887E+02,-2.31619E+01) -- (axis cs:2.80871E+02,-2.29624E+01) -- cycle  ;
\draw [ecliptics-empty] (axis cs:2.79787E+02,-2.30338E+01) -- (axis cs:2.79802E+02,-2.32333E+01)  -- (axis cs:2.78715E+02,-2.32973E+01) -- (axis cs:2.78702E+02,-2.30977E+01) -- cycle  ;
\draw [ecliptics-empty] (axis cs:2.77617E+02,-2.31541E+01) -- (axis cs:2.77628E+02,-2.33538E+01)  -- (axis cs:2.76540E+02,-2.34029E+01) -- (axis cs:2.76530E+02,-2.32031E+01) -- cycle  ;
\draw [ecliptics-empty] (axis cs:2.75443E+02,-2.32446E+01) -- (axis cs:2.75451E+02,-2.34444E+01)  -- (axis cs:2.74362E+02,-2.34785E+01) -- (axis cs:2.74355E+02,-2.32785E+01) -- cycle  ;
\draw [ecliptics-empty] (axis cs:2.73267E+02,-2.33050E+01) -- (axis cs:2.73272E+02,-2.35049E+01)  -- (axis cs:2.72181E+02,-2.35239E+01) -- (axis cs:2.72178E+02,-2.33239E+01) -- cycle  ;
\draw [ecliptics-empty] (axis cs:2.71089E+02,-2.33352E+01) -- (axis cs:2.71091E+02,-2.35352E+01)  -- (axis cs:2.70000E+02,-2.35390E+01) -- (axis cs:2.70000E+02,-2.33390E+01) -- cycle  ;
\draw [ecliptics-empty] (axis cs:2.68911E+02,-2.33352E+01) -- (axis cs:2.68909E+02,-2.35352E+01)  -- (axis cs:2.67819E+02,-2.35239E+01) -- (axis cs:2.67822E+02,-2.33239E+01) -- cycle  ;
\draw [ecliptics-empty] (axis cs:2.66733E+02,-2.33050E+01) -- (axis cs:2.66728E+02,-2.35049E+01)  -- (axis cs:2.65638E+02,-2.34785E+01) -- (axis cs:2.65645E+02,-2.32785E+01) -- cycle  ;
\draw [ecliptics-empty] (axis cs:2.64557E+02,-2.32446E+01) -- (axis cs:2.64549E+02,-2.34444E+01)  -- (axis cs:2.63460E+02,-2.34029E+01) -- (axis cs:2.63470E+02,-2.32031E+01) -- cycle  ;
\draw [ecliptics-empty] (axis cs:2.62383E+02,-2.31541E+01) -- (axis cs:2.62372E+02,-2.33538E+01)  -- (axis cs:2.61285E+02,-2.32973E+01) -- (axis cs:2.61298E+02,-2.30977E+01) -- cycle  ;
\draw [ecliptics-empty] (axis cs:2.60213E+02,-2.30338E+01) -- (axis cs:2.60198E+02,-2.32333E+01)  -- (axis cs:2.59113E+02,-2.31619E+01) -- (axis cs:2.59129E+02,-2.29624E+01) -- cycle  ;
\draw [ecliptics-empty] (axis cs:2.58047E+02,-2.28837E+01) -- (axis cs:2.58029E+02,-2.30830E+01)  -- (axis cs:2.56946E+02,-2.29969E+01) -- (axis cs:2.56966E+02,-2.27977E+01) -- cycle  ;
\draw [ecliptics-empty] (axis cs:2.55886E+02,-2.27043E+01) -- (axis cs:2.55865E+02,-2.29034E+01)  -- (axis cs:2.54786E+02,-2.28026E+01) -- (axis cs:2.54808E+02,-2.26037E+01) -- cycle  ;
\draw [ecliptics-empty] (axis cs:2.53732E+02,-2.24959E+01) -- (axis cs:2.53708E+02,-2.26946E+01)  -- (axis cs:2.52631E+02,-2.25795E+01) -- (axis cs:2.52657E+02,-2.23809E+01) -- cycle  ;
\draw [ecliptics-empty] (axis cs:2.51584E+02,-2.22587E+01) -- (axis cs:2.51557E+02,-2.24572E+01)  -- (axis cs:2.50484E+02,-2.23278E+01) -- (axis cs:2.50513E+02,-2.21296E+01) -- cycle  ;
\draw [ecliptics-empty] (axis cs:2.49444E+02,-2.19934E+01) -- (axis cs:2.49414E+02,-2.21914E+01)  -- (axis cs:2.48346E+02,-2.20481E+01) -- (axis cs:2.48377E+02,-2.18502E+01) -- cycle  ;
\draw [ecliptics-empty] (axis cs:2.47313E+02,-2.17002E+01) -- (axis cs:2.47280E+02,-2.18979E+01)  -- (axis cs:2.46216E+02,-2.17408E+01) -- (axis cs:2.46250E+02,-2.15434E+01) -- cycle  ;
\draw [ecliptics-empty] (axis cs:2.45190E+02,-2.13799E+01) -- (axis cs:2.45154E+02,-2.15771E+01)  -- (axis cs:2.44095E+02,-2.14066E+01) -- (axis cs:2.44133E+02,-2.12096E+01) -- cycle  ;
\draw [ecliptics-empty] (axis cs:2.43078E+02,-2.10328E+01) -- (axis cs:2.43039E+02,-2.12295E+01)  -- (axis cs:2.41985E+02,-2.10459E+01) -- (axis cs:2.42025E+02,-2.08495E+01) -- cycle  ;
\draw [ecliptics-empty] (axis cs:2.40975E+02,-2.06597E+01) -- (axis cs:2.40934E+02,-2.08559E+01)  -- (axis cs:2.39885E+02,-2.06595E+01) -- (axis cs:2.39928E+02,-2.04635E+01) -- cycle  ;
\draw [ecliptics-empty] (axis cs:2.38883E+02,-2.02610E+01) -- (axis cs:2.38839E+02,-2.04568E+01)  -- (axis cs:2.37796E+02,-2.02479E+01) -- (axis cs:2.37841E+02,-2.00524E+01) -- cycle  ;
\draw [ecliptics-empty] (axis cs:2.36802E+02,-1.98376E+01) -- (axis cs:2.36756E+02,-2.00328E+01)  -- (axis cs:2.35719E+02,-1.98118E+01) -- (axis cs:2.35766E+02,-1.96168E+01) -- cycle  ;
\draw [ecliptics-empty] (axis cs:2.34733E+02,-1.93901E+01) -- (axis cs:2.34684E+02,-1.95848E+01)  -- (axis cs:2.33653E+02,-1.93519E+01) -- (axis cs:2.33703E+02,-1.91575E+01) -- cycle  ;
\draw [ecliptics-empty] (axis cs:2.32675E+02,-1.89192E+01) -- (axis cs:2.32624E+02,-1.91133E+01)  -- (axis cs:2.31599E+02,-1.88690E+01) -- (axis cs:2.31651E+02,-1.86752E+01) -- cycle  ;
\draw [ecliptics-empty] (axis cs:2.30629E+02,-1.84256E+01) -- (axis cs:2.30576E+02,-1.86192E+01)  -- (axis cs:2.29557E+02,-1.83638E+01) -- (axis cs:2.29611E+02,-1.81706E+01) -- cycle  ;
\draw [ecliptics-empty] (axis cs:2.28596E+02,-1.79102E+01) -- (axis cs:2.28540E+02,-1.81031E+01)  -- (axis cs:2.27527E+02,-1.78371E+01) -- (axis cs:2.27583E+02,-1.76445E+01) -- cycle  ;
\draw [ecliptics-empty] (axis cs:2.26574E+02,-1.73736E+01) -- (axis cs:2.26516E+02,-1.75660E+01)  -- (axis cs:2.25509E+02,-1.72897E+01) -- (axis cs:2.25567E+02,-1.70976E+01) -- cycle  ;
\draw [ecliptics-empty] (axis cs:2.24564E+02,-1.68167E+01) -- (axis cs:2.24505E+02,-1.70085E+01)  -- (axis cs:2.23504E+02,-1.67224E+01) -- (axis cs:2.23564E+02,-1.65309E+01) -- cycle  ;
\draw [ecliptics-empty] (axis cs:2.22566E+02,-1.62403E+01) -- (axis cs:2.22505E+02,-1.64315E+01)  -- (axis cs:2.21510E+02,-1.61359E+01) -- (axis cs:2.21572E+02,-1.59450E+01) -- cycle  ;
\draw [ecliptics-empty] (axis cs:2.20581E+02,-1.56451E+01) -- (axis cs:2.20518E+02,-1.58358E+01)  -- (axis cs:2.19529E+02,-1.55312E+01) -- (axis cs:2.19592E+02,-1.53408E+01) -- cycle  ;
\draw [ecliptics-empty] (axis cs:2.18607E+02,-1.50321E+01) -- (axis cs:2.18542E+02,-1.52222E+01)  -- (axis cs:2.17559E+02,-1.49090E+01) -- (axis cs:2.17624E+02,-1.47192E+01) -- cycle  ;
\draw [ecliptics-empty] (axis cs:2.16644E+02,-1.44021E+01) -- (axis cs:2.16578E+02,-1.45916E+01)  -- (axis cs:2.15600E+02,-1.42701E+01) -- (axis cs:2.15667E+02,-1.40809E+01) -- cycle  ;
\draw [ecliptics-empty] (axis cs:2.14693E+02,-1.37558E+01) -- (axis cs:2.14625E+02,-1.39448E+01)  -- (axis cs:2.13653E+02,-1.36155E+01) -- (axis cs:2.13721E+02,-1.34268E+01) -- cycle  ;
\draw [ecliptics-empty] (axis cs:2.12752E+02,-1.30941E+01) -- (axis cs:2.12683E+02,-1.32826E+01)  -- (axis cs:2.11717E+02,-1.29460E+01) -- (axis cs:2.11786E+02,-1.27578E+01) -- cycle  ;
\draw [ecliptics-empty] (axis cs:2.10822E+02,-1.24179E+01) -- (axis cs:2.10752E+02,-1.26059E+01)  -- (axis cs:2.09791E+02,-1.22623E+01) -- (axis cs:2.09861E+02,-1.20746E+01) -- cycle  ;
\draw [ecliptics-empty] (axis cs:2.08903E+02,-1.17280E+01) -- (axis cs:2.08831E+02,-1.19155E+01)  -- (axis cs:2.07875E+02,-1.15655E+01) -- (axis cs:2.07946E+02,-1.13782E+01) -- cycle  ;
\draw [ecliptics-empty] (axis cs:2.06993E+02,-1.10253E+01) -- (axis cs:2.06920E+02,-1.12123E+01)  -- (axis cs:2.05968E+02,-1.08562E+01) -- (axis cs:2.06041E+02,-1.06694E+01) -- cycle  ;
\draw [ecliptics-empty] (axis cs:2.05092E+02,-1.03105E+01) -- (axis cs:2.05019E+02,-1.04971E+01)  -- (axis cs:2.04071E+02,-1.01353E+01) -- (axis cs:2.04145E+02,-9.94894E+00) -- cycle  ;
\draw [ecliptics-empty] (axis cs:2.03200E+02,-9.58464E+00) -- (axis cs:2.03126E+02,-9.77078E+00)  -- (axis cs:2.02182E+02,-9.40370E+00) -- (axis cs:2.02257E+02,-9.21776E+00) -- cycle  ;
\draw [ecliptics-empty] (axis cs:2.01316E+02,-8.84840E+00) -- (axis cs:2.01241E+02,-9.03415E+00)  -- (axis cs:2.00301E+02,-8.66224E+00) -- (axis cs:2.00377E+02,-8.47667E+00) -- cycle  ;
\draw [ecliptics-empty] (axis cs:1.99440E+02,-8.10267E+00) -- (axis cs:1.99364E+02,-8.28806E+00)  -- (axis cs:1.98428E+02,-7.91173E+00) -- (axis cs:1.98504E+02,-7.72651E+00) -- cycle  ;
\draw [ecliptics-empty] (axis cs:1.97570E+02,-7.34829E+00) -- (axis cs:1.97494E+02,-7.53334E+00)  -- (axis cs:1.96561E+02,-7.15301E+00) -- (axis cs:1.96638E+02,-6.96811E+00) -- cycle  ;
\draw [ecliptics-empty] (axis cs:1.95708E+02,-6.58609E+00) -- (axis cs:1.95630E+02,-6.77084E+00)  -- (axis cs:1.94701E+02,-6.38692E+00) -- (axis cs:1.94778E+02,-6.20231E+00) -- cycle  ;
\draw [ecliptics-empty] (axis cs:1.93850E+02,-5.81689E+00) -- (axis cs:1.93773E+02,-6.00137E+00)  -- (axis cs:1.92846E+02,-5.61429E+00) -- (axis cs:1.92924E+02,-5.42993E+00) -- cycle  ;
\draw [ecliptics-empty] (axis cs:1.91999E+02,-5.04154E+00) -- (axis cs:1.91920E+02,-5.22577E+00)  -- (axis cs:1.90996E+02,-4.83593E+00) -- (axis cs:1.91074E+02,-4.65180E+00) -- cycle  ;
\draw [ecliptics-empty] (axis cs:1.90151E+02,-4.26084E+00) -- (axis cs:1.90073E+02,-4.44487E+00)  -- (axis cs:1.89150E+02,-4.05268E+00) -- (axis cs:1.89229E+02,-3.86875E+00) -- cycle  ;
\draw [ecliptics-empty] (axis cs:1.88308E+02,-3.47563E+00) -- (axis cs:1.88229E+02,-3.65948E+00)  -- (axis cs:1.87308E+02,-3.26536E+00) -- (axis cs:1.87387E+02,-3.08158E+00) -- cycle  ;
\draw [ecliptics-empty] (axis cs:1.86467E+02,-2.68672E+00) -- (axis cs:1.86388E+02,-2.87043E+00)  -- (axis cs:1.85468E+02,-2.47479E+00) -- (axis cs:1.85548E+02,-2.29113E+00) -- cycle  ;
\draw [ecliptics-empty] (axis cs:1.84629E+02,-1.89493E+00) -- (axis cs:1.84550E+02,-2.07854E+00)  -- (axis cs:1.83631E+02,-1.68179E+00) -- (axis cs:1.83711E+02,-1.49822E+00) -- cycle  ;
\draw [ecliptics-empty] (axis cs:1.82793E+02,-1.10109E+00) -- (axis cs:1.82713E+02,-1.28463E+00)  -- (axis cs:1.81795E+02,-8.87166E-01) -- (axis cs:1.81875E+02,-7.03652E-01) -- cycle  ;
\draw [ecliptics-empty] (axis cs:1.80957E+02,-3.06005E-01) -- (axis cs:1.80878E+02,-4.89506E-01)  -- (axis cs:1.79960E+02,-9.17483E-02) -- (axis cs:1.80040E+02,9.17485E-02) -- cycle  ;



\draw [ecliptics-empty] (axis cs:-0.4000854E-01,9.17484E-02) -- (axis cs:0.0398,-9.17484E-02)  -- (axis cs:9.57272E-01,3.06005E-01) -- (axis cs:8.77725E-01,4.89506E-01) -- cycle  ;
\draw [ecliptics-empty] (axis cs:1.79532E+00,8.87166E-01) -- (axis cs:1.87485E+00,7.03652E-01)  -- (axis cs:2.79259E+00,1.10109E+00) -- (axis cs:2.71311E+00,1.28463E+00) -- cycle  ;
\draw [ecliptics-empty] (axis cs:3.63117E+00,1.68179E+00) -- (axis cs:3.71059E+00,1.49822E+00)  -- (axis cs:4.62893E+00,1.89493E+00) -- (axis cs:4.54959E+00,2.07854E+00) -- cycle  ;
\draw [ecliptics-empty] (axis cs:5.46845E+00,2.47479E+00) -- (axis cs:5.54771E+00,2.29113E+00)  -- (axis cs:6.46701E+00,2.68672E+00) -- (axis cs:6.38786E+00,2.87043E+00) -- cycle  ;
\draw [ecliptics-empty] (axis cs:7.30789E+00,3.26536E+00) -- (axis cs:7.38691E+00,3.08158E+00)  -- (axis cs:8.30751E+00,3.47563E+00) -- (axis cs:8.22863E+00,3.65948E+00) -- cycle  ;
\draw [ecliptics-empty] (axis cs:9.15017E+00,4.05268E+00) -- (axis cs:9.22889E+00,3.86875E+00)  -- (axis cs:1.01511E+01,4.26084E+00) -- (axis cs:1.00726E+01,4.44487E+00) -- cycle  ;
\draw [ecliptics-empty] (axis cs:1.09960E+01,4.83593E+00) -- (axis cs:1.10743E+01,4.65180E+00)  -- (axis cs:1.19986E+01,5.04154E+00) -- (axis cs:1.19204E+01,5.22577E+00) -- cycle  ;
\draw [ecliptics-empty] (axis cs:1.28460E+01,5.61429E+00) -- (axis cs:1.29239E+01,5.42993E+00)  -- (axis cs:1.38505E+01,5.81689E+00) -- (axis cs:1.37728E+01,6.00137E+00) -- cycle  ;
\draw [ecliptics-empty] (axis cs:1.47009E+01,6.38692E+00) -- (axis cs:1.47783E+01,6.20231E+00)  -- (axis cs:1.57075E+01,6.58609E+00) -- (axis cs:1.56304E+01,6.77084E+00) -- cycle  ;
\draw [ecliptics-empty] (axis cs:1.65614E+01,7.15301E+00) -- (axis cs:1.66382E+01,6.96811E+00)  -- (axis cs:1.75704E+01,7.34829E+00) -- (axis cs:1.74939E+01,7.53334E+00) -- cycle  ;
\draw [ecliptics-empty] (axis cs:1.84280E+01,7.91173E+00) -- (axis cs:1.85041E+01,7.72651E+00)  -- (axis cs:1.94396E+01,8.10267E+00) -- (axis cs:1.93638E+01,8.28806E+00) -- cycle  ;
\draw [ecliptics-empty] (axis cs:2.03014E+01,8.66224E+00) -- (axis cs:2.03768E+01,8.47667E+00)  -- (axis cs:2.13159E+01,8.84840E+00) -- (axis cs:2.12408E+01,9.03415E+00) -- cycle  ;
\draw [ecliptics-empty] (axis cs:2.21822E+01,9.40370E+00) -- (axis cs:2.22568E+01,9.21776E+00)  -- (axis cs:2.31998E+01,9.58464E+00) -- (axis cs:2.31256E+01,9.77078E+00) -- cycle  ;
\draw [ecliptics-empty] (axis cs:2.40710E+01,1.01353E+01) -- (axis cs:2.41448E+01,9.94894E+00)  -- (axis cs:2.50918E+01,1.03105E+01) -- (axis cs:2.50186E+01,1.04971E+01) -- cycle  ;
\draw [ecliptics-empty] (axis cs:2.59683E+01,1.08562E+01) -- (axis cs:2.60411E+01,1.06694E+01)  -- (axis cs:2.69926E+01,1.10253E+01) -- (axis cs:2.69204E+01,1.12123E+01) -- cycle  ;
\draw [ecliptics-empty] (axis cs:2.78747E+01,1.15655E+01) -- (axis cs:2.79465E+01,1.13782E+01)  -- (axis cs:2.89026E+01,1.17280E+01) -- (axis cs:2.88315E+01,1.19155E+01) -- cycle  ;
\draw [ecliptics-empty] (axis cs:2.97907E+01,1.22623E+01) -- (axis cs:2.98613E+01,1.20746E+01)  -- (axis cs:3.08224E+01,1.24179E+01) -- (axis cs:3.07524E+01,1.26059E+01) -- cycle  ;
\draw [ecliptics-empty] (axis cs:3.17166E+01,1.29460E+01) -- (axis cs:3.17860E+01,1.27578E+01)  -- (axis cs:3.27522E+01,1.30941E+01) -- (axis cs:3.26835E+01,1.32826E+01) -- cycle  ;
\draw [ecliptics-empty] (axis cs:3.36530E+01,1.36155E+01) -- (axis cs:3.37211E+01,1.34268E+01)  -- (axis cs:3.46927E+01,1.37558E+01) -- (axis cs:3.46253E+01,1.39448E+01) -- cycle  ;
\draw [ecliptics-empty] (axis cs:3.56002E+01,1.42701E+01) -- (axis cs:3.56669E+01,1.40809E+01)  -- (axis cs:3.66440E+01,1.44021E+01) -- (axis cs:3.65780E+01,1.45916E+01) -- cycle  ;
\draw [ecliptics-empty] (axis cs:3.75586E+01,1.49090E+01) -- (axis cs:3.76238E+01,1.47192E+01)  -- (axis cs:3.86065E+01,1.50321E+01) -- (axis cs:3.85421E+01,1.52222E+01) -- cycle  ;
\draw [ecliptics-empty] (axis cs:3.95285E+01,1.55312E+01) -- (axis cs:3.95921E+01,1.53408E+01)  -- (axis cs:4.05806E+01,1.56451E+01) -- (axis cs:4.05178E+01,1.58358E+01) -- cycle  ;
\draw [ecliptics-empty] (axis cs:4.15101E+01,1.61359E+01) -- (axis cs:4.15720E+01,1.59450E+01)  -- (axis cs:4.25664E+01,1.62403E+01) -- (axis cs:4.25053E+01,1.64315E+01) -- cycle  ;
\draw [ecliptics-empty] (axis cs:4.35036E+01,1.67224E+01) -- (axis cs:4.35638E+01,1.65309E+01)  -- (axis cs:4.45641E+01,1.68167E+01) -- (axis cs:4.45048E+01,1.70085E+01) -- cycle  ;
\draw [ecliptics-empty] (axis cs:4.55091E+01,1.72897E+01) -- (axis cs:4.55675E+01,1.70976E+01)  -- (axis cs:4.65738E+01,1.73736E+01) -- (axis cs:4.65165E+01,1.75660E+01) -- cycle  ;
\draw [ecliptics-empty] (axis cs:4.75268E+01,1.78371E+01) -- (axis cs:4.75832E+01,1.76445E+01)  -- (axis cs:4.85956E+01,1.79102E+01) -- (axis cs:4.85402E+01,1.81031E+01) -- cycle  ;
\draw [ecliptics-empty] (axis cs:4.95567E+01,1.83638E+01) -- (axis cs:4.96110E+01,1.81706E+01)  -- (axis cs:5.06294E+01,1.84256E+01) -- (axis cs:5.05762E+01,1.86192E+01) -- cycle  ;
\draw [ecliptics-empty] (axis cs:5.15987E+01,1.88690E+01) -- (axis cs:5.16509E+01,1.86752E+01)  -- (axis cs:5.26753E+01,1.89192E+01) -- (axis cs:5.26242E+01,1.91133E+01) -- cycle  ;
\draw [ecliptics-empty] (axis cs:5.36527E+01,1.93519E+01) -- (axis cs:5.37027E+01,1.91575E+01)  -- (axis cs:5.47330E+01,1.93901E+01) -- (axis cs:5.46842E+01,1.95848E+01) -- cycle  ;
\draw [ecliptics-empty] (axis cs:5.57187E+01,1.98118E+01) -- (axis cs:5.57662E+01,1.96168E+01)  -- (axis cs:5.68024E+01,1.98376E+01) -- (axis cs:5.67560E+01,2.00328E+01) -- cycle  ;
\draw [ecliptics-empty] (axis cs:5.77962E+01,2.02479E+01) -- (axis cs:5.78414E+01,2.00524E+01)  -- (axis cs:5.88831E+01,2.02610E+01) -- (axis cs:5.88393E+01,2.04568E+01) -- cycle  ;
\draw [ecliptics-empty] (axis cs:5.98851E+01,2.06595E+01) -- (axis cs:5.99277E+01,2.04635E+01)  -- (axis cs:6.09750E+01,2.06597E+01) -- (axis cs:6.09337E+01,2.08559E+01) -- cycle  ;
\draw [ecliptics-empty] (axis cs:6.19850E+01,2.10459E+01) -- (axis cs:6.20250E+01,2.08495E+01)  -- (axis cs:6.30775E+01,2.10328E+01) -- (axis cs:6.30389E+01,2.12295E+01) -- cycle  ;
\draw [ecliptics-empty] (axis cs:6.40954E+01,2.14066E+01) -- (axis cs:6.41327E+01,2.12096E+01)  -- (axis cs:6.51903E+01,2.13799E+01) -- (axis cs:6.51544E+01,2.15771E+01) -- cycle  ;
\draw [ecliptics-empty] (axis cs:6.62158E+01,2.17408E+01) -- (axis cs:6.62503E+01,2.15434E+01)  -- (axis cs:6.73127E+01,2.17002E+01) -- (axis cs:6.72796E+01,2.18979E+01) -- cycle  ;
\draw [ecliptics-empty] (axis cs:6.83457E+01,2.20481E+01) -- (axis cs:6.83773E+01,2.18502E+01)  -- (axis cs:6.94442E+01,2.19934E+01) -- (axis cs:6.94140E+01,2.21914E+01) -- cycle  ;
\draw [ecliptics-empty] (axis cs:7.04845E+01,2.23278E+01) -- (axis cs:7.05131E+01,2.21296E+01)  -- (axis cs:7.15841E+01,2.22587E+01) -- (axis cs:7.15569E+01,2.24572E+01) -- cycle  ;
\draw [ecliptics-empty] (axis cs:7.26313E+01,2.25795E+01) -- (axis cs:7.26570E+01,2.23809E+01)  -- (axis cs:7.37317E+01,2.24959E+01) -- (axis cs:7.37076E+01,2.26946E+01) -- cycle  ;
\draw [ecliptics-empty] (axis cs:7.47856E+01,2.28026E+01) -- (axis cs:7.48082E+01,2.26037E+01)  -- (axis cs:7.58863E+01,2.27043E+01) -- (axis cs:7.58653E+01,2.29034E+01) -- cycle  ;
\draw [ecliptics-empty] (axis cs:7.69465E+01,2.29969E+01) -- (axis cs:7.69660E+01,2.27977E+01)  -- (axis cs:7.80470E+01,2.28837E+01) -- (axis cs:7.80291E+01,2.30830E+01) -- cycle  ;
\draw [ecliptics-empty] (axis cs:7.91131E+01,2.31619E+01) -- (axis cs:7.91294E+01,2.29624E+01)  -- (axis cs:8.02130E+01,2.30338E+01) -- (axis cs:8.01982E+01,2.32333E+01) -- cycle  ;
\draw [ecliptics-empty] (axis cs:8.12845E+01,2.32973E+01) -- (axis cs:8.12976E+01,2.30977E+01)  -- (axis cs:8.23833E+01,2.31541E+01) -- (axis cs:8.23718E+01,2.33538E+01) -- cycle  ;
\draw [ecliptics-empty] (axis cs:8.34599E+01,2.34029E+01) -- (axis cs:8.34698E+01,2.32031E+01)  -- (axis cs:8.45570E+01,2.32446E+01) -- (axis cs:8.45488E+01,2.34444E+01) -- cycle  ;
\draw [ecliptics-empty] (axis cs:8.56383E+01,2.34785E+01) -- (axis cs:8.56449E+01,2.32785E+01)  -- (axis cs:8.67332E+01,2.33050E+01) -- (axis cs:8.67283E+01,2.35049E+01) -- cycle  ;
\draw [ecliptics-empty] (axis cs:8.78186E+01,2.35239E+01) -- (axis cs:8.78219E+01,2.33239E+01)  -- (axis cs:8.89109E+01,2.33352E+01) -- (axis cs:8.89093E+01,2.35352E+01) -- cycle  ;
\draw [ecliptics-empty] (axis cs:9.00000E+01,2.35390E+01) -- (axis cs:9.00000E+01,2.33390E+01)  -- (axis cs:9.10891E+01,2.33352E+01) -- (axis cs:9.10907E+01,2.35352E+01) -- cycle  ;
\draw [ecliptics-empty] (axis cs:9.21814E+01,2.35239E+01) -- (axis cs:9.21781E+01,2.33239E+01)  -- (axis cs:9.32668E+01,2.33050E+01) -- (axis cs:9.32717E+01,2.35049E+01) -- cycle  ;
\draw [ecliptics-empty] (axis cs:9.43617E+01,2.34785E+01) -- (axis cs:9.43551E+01,2.32785E+01)  -- (axis cs:9.54430E+01,2.32446E+01) -- (axis cs:9.54512E+01,2.34444E+01) -- cycle  ;
\draw [ecliptics-empty] (axis cs:9.65401E+01,2.34029E+01) -- (axis cs:9.65302E+01,2.32031E+01)  -- (axis cs:9.76167E+01,2.31541E+01) -- (axis cs:9.76282E+01,2.33538E+01) -- cycle  ;
\draw [ecliptics-empty] (axis cs:9.87155E+01,2.32973E+01) -- (axis cs:9.87024E+01,2.30977E+01)  -- (axis cs:9.97870E+01,2.30338E+01) -- (axis cs:9.98018E+01,2.32333E+01) -- cycle  ;
\draw [ecliptics-empty] (axis cs:1.00887E+02,2.31619E+01) -- (axis cs:1.00871E+02,2.29624E+01)  -- (axis cs:1.01953E+02,2.28837E+01) -- (axis cs:1.01971E+02,2.30830E+01) -- cycle  ;
\draw [ecliptics-empty] (axis cs:1.03054E+02,2.29969E+01) -- (axis cs:1.03034E+02,2.27977E+01)  -- (axis cs:1.04114E+02,2.27043E+01) -- (axis cs:1.04135E+02,2.29034E+01) -- cycle  ;
\draw [ecliptics-empty] (axis cs:1.05214E+02,2.28026E+01) -- (axis cs:1.05192E+02,2.26037E+01)  -- (axis cs:1.06268E+02,2.24959E+01) -- (axis cs:1.06292E+02,2.26946E+01) -- cycle  ;
\draw [ecliptics-empty] (axis cs:1.07369E+02,2.25795E+01) -- (axis cs:1.07343E+02,2.23809E+01)  -- (axis cs:1.08416E+02,2.22587E+01) -- (axis cs:1.08443E+02,2.24572E+01) -- cycle  ;
\draw [ecliptics-empty] (axis cs:1.09516E+02,2.23278E+01) -- (axis cs:1.09487E+02,2.21296E+01)  -- (axis cs:1.10556E+02,2.19934E+01) -- (axis cs:1.10586E+02,2.21914E+01) -- cycle  ;
\draw [ecliptics-empty] (axis cs:1.11654E+02,2.20481E+01) -- (axis cs:1.11623E+02,2.18502E+01)  -- (axis cs:1.12687E+02,2.17002E+01) -- (axis cs:1.12720E+02,2.18979E+01) -- cycle  ;
\draw [ecliptics-empty] (axis cs:1.13784E+02,2.17408E+01) -- (axis cs:1.13750E+02,2.15434E+01)  -- (axis cs:1.14810E+02,2.13799E+01) -- (axis cs:1.14846E+02,2.15771E+01) -- cycle  ;
\draw [ecliptics-empty] (axis cs:1.15905E+02,2.14066E+01) -- (axis cs:1.15867E+02,2.12096E+01)  -- (axis cs:1.16922E+02,2.10328E+01) -- (axis cs:1.16961E+02,2.12295E+01) -- cycle  ;
\draw [ecliptics-empty] (axis cs:1.18015E+02,2.10459E+01) -- (axis cs:1.17975E+02,2.08495E+01)  -- (axis cs:1.19025E+02,2.06597E+01) -- (axis cs:1.19066E+02,2.08559E+01) -- cycle  ;
\draw [ecliptics-empty] (axis cs:1.20115E+02,2.06595E+01) -- (axis cs:1.20072E+02,2.04635E+01)  -- (axis cs:1.21117E+02,2.02610E+01) -- (axis cs:1.21161E+02,2.04568E+01) -- cycle  ;
\draw [ecliptics-empty] (axis cs:1.22204E+02,2.02479E+01) -- (axis cs:1.22159E+02,2.00524E+01)  -- (axis cs:1.23198E+02,1.98376E+01) -- (axis cs:1.23244E+02,2.00328E+01) -- cycle  ;
\draw [ecliptics-empty] (axis cs:1.24281E+02,1.98118E+01) -- (axis cs:1.24234E+02,1.96168E+01)  -- (axis cs:1.25267E+02,1.93901E+01) -- (axis cs:1.25316E+02,1.95848E+01) -- cycle  ;
\draw [ecliptics-empty] (axis cs:1.26347E+02,1.93519E+01) -- (axis cs:1.26297E+02,1.91575E+01)  -- (axis cs:1.27325E+02,1.89192E+01) -- (axis cs:1.27376E+02,1.91133E+01) -- cycle  ;
\draw [ecliptics-empty] (axis cs:1.28401E+02,1.88690E+01) -- (axis cs:1.28349E+02,1.86752E+01)  -- (axis cs:1.29371E+02,1.84256E+01) -- (axis cs:1.29424E+02,1.86192E+01) -- cycle  ;
\draw [ecliptics-empty] (axis cs:1.30443E+02,1.83638E+01) -- (axis cs:1.30389E+02,1.81706E+01)  -- (axis cs:1.31404E+02,1.79102E+01) -- (axis cs:1.31460E+02,1.81031E+01) -- cycle  ;
\draw [ecliptics-empty] (axis cs:1.32473E+02,1.78371E+01) -- (axis cs:1.32417E+02,1.76445E+01)  -- (axis cs:1.33426E+02,1.73736E+01) -- (axis cs:1.33484E+02,1.75660E+01) -- cycle  ;
\draw [ecliptics-empty] (axis cs:1.34491E+02,1.72897E+01) -- (axis cs:1.34433E+02,1.70976E+01)  -- (axis cs:1.35436E+02,1.68167E+01) -- (axis cs:1.35495E+02,1.70085E+01) -- cycle  ;
\draw [ecliptics-empty] (axis cs:1.36496E+02,1.67224E+01) -- (axis cs:1.36436E+02,1.65309E+01)  -- (axis cs:1.37434E+02,1.62403E+01) -- (axis cs:1.37495E+02,1.64315E+01) -- cycle  ;
\draw [ecliptics-empty] (axis cs:1.38490E+02,1.61359E+01) -- (axis cs:1.38428E+02,1.59450E+01)  -- (axis cs:1.39419E+02,1.56451E+01) -- (axis cs:1.39482E+02,1.58358E+01) -- cycle  ;
\draw [ecliptics-empty] (axis cs:1.40471E+02,1.55312E+01) -- (axis cs:1.40408E+02,1.53408E+01)  -- (axis cs:1.41393E+02,1.50321E+01) -- (axis cs:1.41458E+02,1.52222E+01) -- cycle  ;
\draw [ecliptics-empty] (axis cs:1.42441E+02,1.49090E+01) -- (axis cs:1.42376E+02,1.47192E+01)  -- (axis cs:1.43356E+02,1.44021E+01) -- (axis cs:1.43422E+02,1.45916E+01) -- cycle  ;
\draw [ecliptics-empty] (axis cs:1.44400E+02,1.42701E+01) -- (axis cs:1.44333E+02,1.40809E+01)  -- (axis cs:1.45307E+02,1.37558E+01) -- (axis cs:1.45375E+02,1.39448E+01) -- cycle  ;
\draw [ecliptics-empty] (axis cs:1.46347E+02,1.36155E+01) -- (axis cs:1.46279E+02,1.34268E+01)  -- (axis cs:1.47248E+02,1.30941E+01) -- (axis cs:1.47317E+02,1.32826E+01) -- cycle  ;
\draw [ecliptics-empty] (axis cs:1.48283E+02,1.29460E+01) -- (axis cs:1.48214E+02,1.27578E+01)  -- (axis cs:1.49178E+02,1.24179E+01) -- (axis cs:1.49248E+02,1.26059E+01) -- cycle  ;
\draw [ecliptics-empty] (axis cs:1.50209E+02,1.22623E+01) -- (axis cs:1.50139E+02,1.20746E+01)  -- (axis cs:1.51097E+02,1.17280E+01) -- (axis cs:1.51169E+02,1.19155E+01) -- cycle  ;
\draw [ecliptics-empty] (axis cs:1.52125E+02,1.15655E+01) -- (axis cs:1.52054E+02,1.13782E+01)  -- (axis cs:1.53007E+02,1.10253E+01) -- (axis cs:1.53080E+02,1.12123E+01) -- cycle  ;
\draw [ecliptics-empty] (axis cs:1.54032E+02,1.08562E+01) -- (axis cs:1.53959E+02,1.06694E+01)  -- (axis cs:1.54908E+02,1.03105E+01) -- (axis cs:1.54981E+02,1.04971E+01) -- cycle  ;
\draw [ecliptics-empty] (axis cs:1.55929E+02,1.01353E+01) -- (axis cs:1.55855E+02,9.94894E+00)  -- (axis cs:1.56800E+02,9.58464E+00) -- (axis cs:1.56874E+02,9.77078E+00) -- cycle  ;
\draw [ecliptics-empty] (axis cs:1.57818E+02,9.40370E+00) -- (axis cs:1.57743E+02,9.21776E+00)  -- (axis cs:1.58684E+02,8.84840E+00) -- (axis cs:1.58759E+02,9.03415E+00) -- cycle  ;
\draw [ecliptics-empty] (axis cs:1.59699E+02,8.66224E+00) -- (axis cs:1.59623E+02,8.47667E+00)  -- (axis cs:1.60560E+02,8.10267E+00) -- (axis cs:1.60636E+02,8.28806E+00) -- cycle  ;
\draw [ecliptics-empty] (axis cs:1.61572E+02,7.91173E+00) -- (axis cs:1.61496E+02,7.72651E+00)  -- (axis cs:1.62430E+02,7.34829E+00) -- (axis cs:1.62506E+02,7.53334E+00) -- cycle  ;
\draw [ecliptics-empty] (axis cs:1.63439E+02,7.15301E+00) -- (axis cs:1.63362E+02,6.96811E+00)  -- (axis cs:1.64292E+02,6.58609E+00) -- (axis cs:1.64370E+02,6.77084E+00) -- cycle  ;
\draw [ecliptics-empty] (axis cs:1.65299E+02,6.38692E+00) -- (axis cs:1.65222E+02,6.20231E+00)  -- (axis cs:1.66150E+02,5.81689E+00) -- (axis cs:1.66227E+02,6.00137E+00) -- cycle  ;
\draw [ecliptics-empty] (axis cs:1.67154E+02,5.61429E+00) -- (axis cs:1.67076E+02,5.42993E+00)  -- (axis cs:1.68001E+02,5.04154E+00) -- (axis cs:1.68080E+02,5.22577E+00) -- cycle  ;
\draw [ecliptics-empty] (axis cs:1.69004E+02,4.83593E+00) -- (axis cs:1.68926E+02,4.65180E+00)  -- (axis cs:1.69849E+02,4.26084E+00) -- (axis cs:1.69927E+02,4.44487E+00) -- cycle  ;
\draw [ecliptics-empty] (axis cs:1.70850E+02,4.05268E+00) -- (axis cs:1.70771E+02,3.86875E+00)  -- (axis cs:1.71692E+02,3.47563E+00) -- (axis cs:1.71771E+02,3.65948E+00) -- cycle  ;
\draw [ecliptics-empty] (axis cs:1.72692E+02,3.26536E+00) -- (axis cs:1.72613E+02,3.08158E+00)  -- (axis cs:1.73533E+02,2.68672E+00) -- (axis cs:1.73612E+02,2.87043E+00) -- cycle  ;
\draw [ecliptics-empty] (axis cs:1.74532E+02,2.47479E+00) -- (axis cs:1.74452E+02,2.29113E+00)  -- (axis cs:1.75371E+02,1.89493E+00) -- (axis cs:1.75450E+02,2.07854E+00) -- cycle  ;
\draw [ecliptics-empty] (axis cs:1.76369E+02,1.68179E+00) -- (axis cs:1.76289E+02,1.49822E+00)  -- (axis cs:1.77207E+02,1.10109E+00) -- (axis cs:1.77287E+02,1.28463E+00) -- cycle  ;
\draw [ecliptics-empty] (axis cs:1.78205E+02,8.87166E-01) -- (axis cs:1.78125E+02,7.03652E-01)  -- (axis cs:1.79043E+02,3.06005E-01) -- (axis cs:1.79122E+02,4.89506E-01) -- cycle  ;




\draw [ecliptics-empty] (axis cs:-0.4000854E-01+360,9.17484E-02) -- (axis cs:0.0398+360,-9.17484E-02)  -- (axis cs:9.57272E-01+360,3.06005E-01) -- (axis cs:8.77725E-01+360,4.89506E-01) -- cycle  ;
\draw [ecliptics-empty] (axis cs:1.79532E+00+360,8.87166E-01) -- (axis cs:1.87485E+00+360,7.03652E-01)  -- (axis cs:2.79259E+00+360,1.10109E+00) -- (axis cs:2.71311E+00+360,1.28463E+00) -- cycle  ;
\draw [ecliptics-empty] (axis cs:3.63117E+00+360,1.68179E+00) -- (axis cs:3.71059E+00+360,1.49822E+00)  -- (axis cs:4.62893E+00+360,1.89493E+00) -- (axis cs:4.54959E+00+360,2.07854E+00) -- cycle  ;
\draw [ecliptics-empty] (axis cs:5.46845E+00+360,2.47479E+00) -- (axis cs:5.54771E+00+360,2.29113E+00)  -- (axis cs:6.46701E+00+360,2.68672E+00) -- (axis cs:6.38786E+00+360,2.87043E+00) -- cycle  ;
\draw [ecliptics-empty] (axis cs:7.30789E+00+360,3.26536E+00) -- (axis cs:7.38691E+00+360,3.08158E+00)  -- (axis cs:8.30751E+00+360,3.47563E+00) -- (axis cs:8.22863E+00+360,3.65948E+00) -- cycle  ;
\draw [ecliptics-empty] (axis cs:9.15017E+00+360,4.05268E+00) -- (axis cs:9.22889E+00+360,3.86875E+00)  -- (axis cs:1.01511E+01+360,4.26084E+00) -- (axis cs:1.00726E+01+360,4.44487E+00) -- cycle  ;
\draw [ecliptics-empty] (axis cs:1.09960E+01+360,4.83593E+00) -- (axis cs:1.10743E+01+360,4.65180E+00)  -- (axis cs:1.19986E+01+360,5.04154E+00) -- (axis cs:1.19204E+01+360,5.22577E+00) -- cycle  ;
\draw [ecliptics-empty] (axis cs:1.28460E+01+360,5.61429E+00) -- (axis cs:1.29239E+01+360,5.42993E+00)  -- (axis cs:1.38505E+01+360,5.81689E+00) -- (axis cs:1.37728E+01+360,6.00137E+00) -- cycle  ;
\draw [ecliptics-empty] (axis cs:1.47009E+01+360,6.38692E+00) -- (axis cs:1.47783E+01+360,6.20231E+00)  -- (axis cs:1.57075E+01+360,6.58609E+00) -- (axis cs:1.56304E+01+360,6.77084E+00) -- cycle  ;
\draw [ecliptics-empty] (axis cs:1.65614E+01+360,7.15301E+00) -- (axis cs:1.66382E+01+360,6.96811E+00)  -- (axis cs:1.75704E+01+360,7.34829E+00) -- (axis cs:1.74939E+01+360,7.53334E+00) -- cycle  ;
\draw [ecliptics-empty] (axis cs:1.84280E+01+360,7.91173E+00) -- (axis cs:1.85041E+01+360,7.72651E+00)  -- (axis cs:1.94396E+01+360,8.10267E+00) -- (axis cs:1.93638E+01+360,8.28806E+00) -- cycle  ;
\draw [ecliptics-empty] (axis cs:2.03014E+01+360,8.66224E+00) -- (axis cs:2.03768E+01+360,8.47667E+00)  -- (axis cs:2.13159E+01+360,8.84840E+00) -- (axis cs:2.12408E+01+360,9.03415E+00) -- cycle  ;


\draw [ecliptics-full] (axis cs:8.77725E-01+360,{4.89506E-01}) -- (axis cs:9.57272E-01+360,{3.06005E-01})  -- (axis cs:1.87485E+00+360,{7.03652E-01}) -- (axis cs:1.79532E+00+360,{8.87166E-01}) -- cycle  ;
\draw [ecliptics-full] (axis cs:2.71311E+00+360,1.28463E+00) -- (axis cs:2.79259E+00+360,1.10109E+00)  -- (axis cs:3.71059E+00+360,1.49822E+00) -- (axis cs:3.63117E+00+360,1.68179E+00) -- cycle  ;
\draw [ecliptics-full] (axis cs:4.54959E+00+360,2.07854E+00) -- (axis cs:4.62893E+00+360,1.89493E+00)  -- (axis cs:5.54771E+00+360,2.29113E+00) -- (axis cs:5.46845E+00+360,2.47479E+00) -- cycle  ;
\draw [ecliptics-full] (axis cs:6.38786E+00+360,2.87043E+00) -- (axis cs:6.46701E+00+360,2.68672E+00)  -- (axis cs:7.38691E+00+360,3.08158E+00) -- (axis cs:7.30789E+00+360,3.26536E+00) -- cycle  ;
\draw [ecliptics-full] (axis cs:8.22863E+00+360,3.65948E+00) -- (axis cs:8.30751E+00+360,3.47563E+00)  -- (axis cs:9.22889E+00+360,3.86875E+00) -- (axis cs:9.15017E+00+360,4.05268E+00) -- cycle  ;
\draw [ecliptics-full] (axis cs:1.00726E+01+360,4.44487E+00) -- (axis cs:1.01511E+01+360,4.26084E+00)  -- (axis cs:1.10743E+01+360,4.65180E+00) -- (axis cs:1.09960E+01+360,4.83593E+00) -- cycle  ;
\draw [ecliptics-full] (axis cs:1.19204E+01+360,5.22577E+00) -- (axis cs:1.19986E+01+360,5.04154E+00)  -- (axis cs:1.29239E+01+360,5.42993E+00) -- (axis cs:1.28460E+01+360,5.61429E+00) -- cycle  ;
\draw [ecliptics-full] (axis cs:1.37728E+01+360,6.00137E+00) -- (axis cs:1.38505E+01+360,5.81689E+00)  -- (axis cs:1.47783E+01+360,6.20231E+00) -- (axis cs:1.47009E+01+360,6.38692E+00) -- cycle  ;
\draw [ecliptics-full] (axis cs:1.56304E+01+360,6.77084E+00) -- (axis cs:1.57075E+01+360,6.58609E+00)  -- (axis cs:1.66382E+01+360,6.96811E+00) -- (axis cs:1.65614E+01+360,7.15301E+00) -- cycle  ;
\draw [ecliptics-full] (axis cs:1.74939E+01+360,7.53334E+00) -- (axis cs:1.75704E+01+360,7.34829E+00)  -- (axis cs:1.85041E+01+360,7.72651E+00) -- (axis cs:1.84280E+01+360,7.91173E+00) -- cycle  ;
\draw [ecliptics-full] (axis cs:1.93638E+01+360,8.28806E+00) -- (axis cs:1.94396E+01+360,8.10267E+00)  -- (axis cs:2.03768E+01+360,8.47667E+00) -- (axis cs:2.03014E+01+360,8.66224E+00) -- cycle  ;


\draw [ecliptics-empty] (axis cs:-1.59111E-01,3.66993E-01) -- (axis cs:1.59111E-01,-3.66993E-01)   ;
\node[pin={[pin distance=-0.4\onedegree,ecliptics-label]+90:{0$^\circ$}}] at (axis cs:1.59111E-01,3.66993E-01) {} ;
\draw [ecliptics-empty] (axis cs:9.03202E+00,4.32857E+00) -- (axis cs:9.34691E+00,3.59283E+00)   ;
\node[pin={[pin distance=-0.4\onedegree,ecliptics-label]-90:{10$^\circ$}}] at (axis cs:9.03202E+00,4.32857E+00) {} ;
\draw [ecliptics-empty] (axis cs:1.83136E+01,8.18953E+00) -- (axis cs:1.86183E+01,7.44866E+00)   ;
\node[pin={[pin distance=-0.4\onedegree,ecliptics-label]-90:{20$^\circ$}}] at (axis cs:1.83136E+01,8.18953E+00) {} ;
\draw [ecliptics-empty] (axis cs:2.77669E+01,1.18463E+01) -- (axis cs:2.80539E+01,1.10973E+01)   ;
\node[pin={[pin distance=-0.4\onedegree,ecliptics-label]-90:{30$^\circ$}}] at (axis cs:2.77669E+01,1.18463E+01) {} ;
\draw [ecliptics-empty] (axis cs:3.74606E+01,1.51936E+01) -- (axis cs:3.77214E+01,1.44344E+01)   ;
\node[pin={[pin distance=-0.4\onedegree,ecliptics-label]-90:{40$^\circ$}}] at (axis cs:3.74606E+01,1.51936E+01) {} ;
\draw [ecliptics-empty] (axis cs:4.74420E+01,1.81261E+01) -- (axis cs:4.76675E+01,1.73555E+01)   ;
\node[pin={[pin distance=-0.4\onedegree,ecliptics-label]-90:{50$^\circ$}}] at (axis cs:4.74420E+01,1.81261E+01) {} ;
\draw [ecliptics-empty] (axis cs:5.77283E+01,2.05410E+01) -- (axis cs:5.79088E+01,1.97592E+01)   ;
\node[pin={[pin distance=-0.4\onedegree,ecliptics-label]-90:{60$^\circ$}}] at (axis cs:5.77283E+01,2.05410E+01) {} ;
\draw [ecliptics-empty] (axis cs:6.82981E+01,2.23448E+01) -- (axis cs:6.84246E+01,2.15535E+01)   ;
\node[pin={[pin distance=-0.4\onedegree,ecliptics-label]-90:{70$^\circ$}}] at (axis cs:6.82981E+01,2.23448E+01) {} ;
\draw [ecliptics-empty] (axis cs:7.90885E+01,2.34610E+01) -- (axis cs:7.91538E+01,2.26633E+01)   ;
\node[pin={[pin distance=-0.4\onedegree,ecliptics-label]-90:{80$^\circ$}}] at (axis cs:7.90885E+01,2.34610E+01) {} ;
\draw [ecliptics-empty] (axis cs:9.00000E+01,2.38390E+01) -- (axis cs:9.00000E+01,2.30390E+01)   ;
\node[pin={[pin distance=-0.4\onedegree,ecliptics-label]-90:{90$^\circ$}}] at (axis cs:9.00000E+01,2.38390E+01) {} ;
\draw [ecliptics-empty] (axis cs:1.00911E+02,2.34610E+01) -- (axis cs:1.00846E+02,2.26633E+01)   ;
\node[pin={[pin distance=-0.4\onedegree,ecliptics-label]-90:{100$^\circ$}}] at (axis cs:1.00911E+02,2.34610E+01) {} ;
\draw [ecliptics-empty] (axis cs:1.11702E+02,2.23448E+01) -- (axis cs:1.11575E+02,2.15535E+01)   ;
\node[pin={[pin distance=-0.4\onedegree,ecliptics-label]-90:{110$^\circ$}}] at (axis cs:1.11702E+02,2.23448E+01) {} ;
\draw [ecliptics-empty] (axis cs:1.22272E+02,2.05410E+01) -- (axis cs:1.22091E+02,1.97592E+01)   ;
\node[pin={[pin distance=-0.4\onedegree,ecliptics-label]-90:{120$^\circ$}}] at (axis cs:1.22272E+02,2.05410E+01) {} ;
\draw [ecliptics-empty] (axis cs:1.32558E+02,1.81261E+01) -- (axis cs:1.32332E+02,1.73555E+01)   ;
\node[pin={[pin distance=-0.4\onedegree,ecliptics-label]-90:{130$^\circ$}}] at (axis cs:1.32558E+02,1.81261E+01) {} ;
\draw [ecliptics-empty] (axis cs:1.42539E+02,1.51936E+01) -- (axis cs:1.42279E+02,1.44344E+01)   ;
\node[pin={[pin distance=-0.4\onedegree,ecliptics-label]-90:{140$^\circ$}}] at (axis cs:1.42539E+02,1.51936E+01) {} ;
\draw [ecliptics-empty] (axis cs:1.52233E+02,1.18463E+01) -- (axis cs:1.51946E+02,1.10973E+01)   ;
\node[pin={[pin distance=0\onedegree,ecliptics-label]-45:{150$^\circ$}}] at (axis cs:1.52233E+02,1.18463E+01) {} ;
\draw [ecliptics-empty] (axis cs:1.61686E+02,8.18953E+00) -- (axis cs:1.61382E+02,7.44866E+00)   ;
\node[pin={[pin distance=-0.4\onedegree,ecliptics-label]-90:{160$^\circ$}}] at (axis cs:1.61686E+02,8.18953E+00) {} ;
\draw [ecliptics-empty] (axis cs:1.70968E+02,4.32857E+00) -- (axis cs:1.70653E+02,3.59283E+00)   ;
\node[pin={[pin distance=-0.4\onedegree,ecliptics-label]-90:{170$^\circ$}}] at (axis cs:1.70968E+02,4.32857E+00) {} ;
\draw [ecliptics-empty] (axis cs:360-1.79841E+02,3.66993E-01) -- (axis cs:1.79841E+02,-3.66993E-01)   ;
%\node[pin={[pin distance=0.4\onedegree,ecliptics-label]+90:{180$^\circ$}}] at (axis cs:1.79841E+02,3.66993E-01) {} ;

\draw [ecliptics-empty] (axis cs:3.50653E+02,-3.59283E+00) -- (axis cs:3.50968E+02,-4.32857E+00)   ;
\node[pin={[pin distance=-0.4\onedegree,ecliptics-label]-90:{350$^\circ$}}] at (axis cs:3.50653E+02,-3.59283E+00) {} ;
\draw [ecliptics-empty] (axis cs:3.41382E+02,-7.44866E+00) -- (axis cs:3.41686E+02,-8.18953E+00)   ;
\node[pin={[pin distance=-0.4\onedegree,ecliptics-label]-90:{340$^\circ$}}] at (axis cs:3.41382E+02,-7.44866E+00) {} ;
\draw [ecliptics-empty] (axis cs:3.31946E+02,-1.10973E+01) -- (axis cs:3.32233E+02,-1.18463E+01)   ;
\node[pin={[pin distance=-0.4\onedegree,ecliptics-label]-90:{330$^\circ$}}] at (axis cs:3.31946E+02,-1.10973E+01) {} ;
\draw [ecliptics-empty] (axis cs:3.22279E+02,-1.44344E+01) -- (axis cs:3.22539E+02,-1.51936E+01)   ;
\node[pin={[pin distance=-0.4\onedegree,ecliptics-label]-90:{320$^\circ$}}] at (axis cs:3.22279E+02,-1.44344E+01) {} ;
\draw [ecliptics-empty] (axis cs:3.12332E+02,-1.73555E+01) -- (axis cs:3.12558E+02,-1.81261E+01)   ;
\node[pin={[pin distance=-0.4\onedegree,ecliptics-label]-90:{310$^\circ$}}] at (axis cs:3.12332E+02,-1.73555E+01) {} ;
\draw [ecliptics-empty] (axis cs:3.02091E+02,-1.97592E+01) -- (axis cs:3.02272E+02,-2.05410E+01)   ;
\node[pin={[pin distance=-0.4\onedegree,ecliptics-label]-90:{300$^\circ$}}] at (axis cs:3.02091E+02,-1.97592E+01) {} ;
\draw [ecliptics-empty] (axis cs:2.91575E+02,-2.15535E+01) -- (axis cs:2.91702E+02,-2.23448E+01)   ;
\node[pin={[pin distance=-0.4\onedegree,ecliptics-label]-90:{290$^\circ$}}] at (axis cs:2.91575E+02,-2.15535E+01) {} ;
\draw [ecliptics-empty] (axis cs:2.80846E+02,-2.26633E+01) -- (axis cs:2.80911E+02,-2.34610E+01)   ;
\node[pin={[pin distance=-0.4\onedegree,ecliptics-label]-90:{280$^\circ$}}] at (axis cs:2.80846E+02,-2.26633E+01) {} ;
\draw [ecliptics-empty] (axis cs:2.70000E+02,-2.30390E+01) -- (axis cs:2.70000E+02,-2.38390E+01)   ;
\node[pin={[pin distance=-0.4\onedegree,ecliptics-label]-90:{270$^\circ$}}] at (axis cs:2.70000E+02,-2.30390E+01) {} ;
\draw [ecliptics-empty] (axis cs:2.59154E+02,-2.26633E+01) -- (axis cs:2.59089E+02,-2.34610E+01)   ;
\node[pin={[pin distance=-0.4\onedegree,ecliptics-label]-90:{260$^\circ$}}] at (axis cs:2.59154E+02,-2.26633E+01) {} ;
\draw [ecliptics-empty] (axis cs:2.48425E+02,-2.15535E+01) -- (axis cs:2.48298E+02,-2.23448E+01)   ;
\node[pin={[pin distance=-0.4\onedegree,ecliptics-label]-90:{250$^\circ$}}] at (axis cs:2.48425E+02,-2.15535E+01) {} ;
\draw [ecliptics-empty] (axis cs:2.37909E+02,-1.97592E+01) -- (axis cs:2.37728E+02,-2.05410E+01)   ;
\node[pin={[pin distance=-0.4\onedegree,ecliptics-label]-90:{240$^\circ$}}] at (axis cs:2.37909E+02,-1.97592E+01) {} ;
\draw [ecliptics-empty] (axis cs:2.27668E+02,-1.73555E+01) -- (axis cs:2.27442E+02,-1.81261E+01)   ;
\node[pin={[pin distance=-0.4\onedegree,ecliptics-label]-90:{230$^\circ$}}] at (axis cs:2.27668E+02,-1.73555E+01) {} ;
\draw [ecliptics-empty] (axis cs:2.17721E+02,-1.44344E+01) -- (axis cs:2.17461E+02,-1.51936E+01)   ;
\node[pin={[pin distance=-0.4\onedegree,ecliptics-label]-90:{220$^\circ$}}] at (axis cs:2.17721E+02,-1.44344E+01) {} ;
\draw [ecliptics-empty] (axis cs:2.08054E+02,-1.10973E+01) -- (axis cs:2.07767E+02,-1.18463E+01)   ;
\node[pin={[pin distance=-0.4\onedegree,ecliptics-label]-90:{210$^\circ$}}] at (axis cs:2.08054E+02,-1.10973E+01) {} ;
\draw [ecliptics-empty] (axis cs:1.98618E+02,-7.44866E+00) -- (axis cs:1.98314E+02,-8.18953E+00)   ;
\node[pin={[pin distance=-0.4\onedegree,ecliptics-label]-90:{200$^\circ$}}] at (axis cs:1.98618E+02,-7.44866E+00) {} ;
\draw [ecliptics-empty] (axis cs:1.89347E+02,-3.59283E+00) -- (axis cs:1.89032E+02,-4.32857E+00)   ;
\node[pin={[pin distance=-0.4\onedegree,ecliptics-label]-90:{190$^\circ$}}] at (axis cs:1.89347E+02,-3.59283E+00) {} ;

%\draw [ecliptics-empty] (axis cs:-1.59111E-01+360,3.66993E-01) -- (axis cs:1.59111E-01+360,-3.66993E-01)   ;
%\node[pin={[pin distance=-0.4\onedegree,ecliptics-label]+90:{0$^\circ$}}] at (axis cs:1.59111E-01+360,3.66993E-01) {} ;
\draw [ecliptics-empty] (axis cs:9.03202E+00+360,4.32857E+00) -- (axis cs:9.34691E+00+360,3.59283E+00)   ;
\node[pin={[pin distance=-0.4\onedegree,ecliptics-label]-90:{10$^\circ$}}] at (axis cs:9.03202E+00+360,4.32857E+00) {} ;
\draw [ecliptics-empty] (axis cs:1.83136E+01+360,8.18953E+00) -- (axis cs:1.86183E+01+360,7.44866E+00)   ;
\node[pin={[pin distance=-0.4\onedegree,ecliptics-label]-90:{20$^\circ$}}] at (axis cs:1.83136E+01+360,8.18953E+00) {} ;



\end{axis}

% Equatorial coordinate system
%
% Some are commented out because they are surrounded by already plotted borders
% The x-coordinate is in fractional hours, so must be times 15
%

\begin{axis}[name=constellations,at=(base.center),anchor=center,axis lines=none]


\draw[Equator-full] (axis cs:0.00000E+00,1.00000E-01) -- (axis cs:1.00000E+00,1.00000E-01) -- (axis cs:1.00000E+00,-1.00000E-01) -- (axis cs:0.00000E+00,-1.00000E-01) -- cycle ; 
\draw[Equator-full] (axis cs:2.00000E+00,1.00000E-01) -- (axis cs:3.00000E+00,1.00000E-01) -- (axis cs:3.00000E+00,-1.00000E-01) -- (axis cs:2.00000E+00,-1.00000E-01) -- cycle ; 
\draw[Equator-full] (axis cs:4.00000E+00,1.00000E-01) -- (axis cs:5.00000E+00,1.00000E-01) -- (axis cs:5.00000E+00,-1.00000E-01) -- (axis cs:4.00000E+00,-1.00000E-01) -- cycle ; 
\draw[Equator-full] (axis cs:6.00000E+00,1.00000E-01) -- (axis cs:7.00000E+00,1.00000E-01) -- (axis cs:7.00000E+00,-1.00000E-01) -- (axis cs:6.00000E+00,-1.00000E-01) -- cycle ; 
\draw[Equator-full] (axis cs:8.00000E+00,1.00000E-01) -- (axis cs:9.00000E+00,1.00000E-01) -- (axis cs:9.00000E+00,-1.00000E-01) -- (axis cs:8.00000E+00,-1.00000E-01) -- cycle ; 
\draw[Equator-full] (axis cs:1.00000E+01,1.00000E-01) -- (axis cs:1.10000E+01,1.00000E-01) -- (axis cs:1.10000E+01,-1.00000E-01) -- (axis cs:1.00000E+01,-1.00000E-01) -- cycle ; 
\draw[Equator-full] (axis cs:1.20000E+01,1.00000E-01) -- (axis cs:1.30000E+01,1.00000E-01) -- (axis cs:1.30000E+01,-1.00000E-01) -- (axis cs:1.20000E+01,-1.00000E-01) -- cycle ; 
\draw[Equator-full] (axis cs:1.40000E+01,1.00000E-01) -- (axis cs:1.50000E+01,1.00000E-01) -- (axis cs:1.50000E+01,-1.00000E-01) -- (axis cs:1.40000E+01,-1.00000E-01) -- cycle ; 
\draw[Equator-full] (axis cs:1.60000E+01,1.00000E-01) -- (axis cs:1.70000E+01,1.00000E-01) -- (axis cs:1.70000E+01,-1.00000E-01) -- (axis cs:1.60000E+01,-1.00000E-01) -- cycle ; 
\draw[Equator-full] (axis cs:1.80000E+01,1.00000E-01) -- (axis cs:1.90000E+01,1.00000E-01) -- (axis cs:1.90000E+01,-1.00000E-01) -- (axis cs:1.80000E+01,-1.00000E-01) -- cycle ; 
\draw[Equator-full] (axis cs:2.00000E+01,1.00000E-01) -- (axis cs:2.10000E+01,1.00000E-01) -- (axis cs:2.10000E+01,-1.00000E-01) -- (axis cs:2.00000E+01,-1.00000E-01) -- cycle ; 
\draw[Equator-full] (axis cs:2.20000E+01,1.00000E-01) -- (axis cs:2.30000E+01,1.00000E-01) -- (axis cs:2.30000E+01,-1.00000E-01) -- (axis cs:2.20000E+01,-1.00000E-01) -- cycle ; 
\draw[Equator-full] (axis cs:2.40000E+01,1.00000E-01) -- (axis cs:2.50000E+01,1.00000E-01) -- (axis cs:2.50000E+01,-1.00000E-01) -- (axis cs:2.40000E+01,-1.00000E-01) -- cycle ; 
\draw[Equator-full] (axis cs:2.60000E+01,1.00000E-01) -- (axis cs:2.70000E+01,1.00000E-01) -- (axis cs:2.70000E+01,-1.00000E-01) -- (axis cs:2.60000E+01,-1.00000E-01) -- cycle ; 
\draw[Equator-full] (axis cs:2.80000E+01,1.00000E-01) -- (axis cs:2.90000E+01,1.00000E-01) -- (axis cs:2.90000E+01,-1.00000E-01) -- (axis cs:2.80000E+01,-1.00000E-01) -- cycle ; 
\draw[Equator-full] (axis cs:3.00000E+01,1.00000E-01) -- (axis cs:3.10000E+01,1.00000E-01) -- (axis cs:3.10000E+01,-1.00000E-01) -- (axis cs:3.00000E+01,-1.00000E-01) -- cycle ; 
\draw[Equator-full] (axis cs:3.20000E+01,1.00000E-01) -- (axis cs:3.30000E+01,1.00000E-01) -- (axis cs:3.30000E+01,-1.00000E-01) -- (axis cs:3.20000E+01,-1.00000E-01) -- cycle ; 
\draw[Equator-full] (axis cs:3.40000E+01,1.00000E-01) -- (axis cs:3.50000E+01,1.00000E-01) -- (axis cs:3.50000E+01,-1.00000E-01) -- (axis cs:3.40000E+01,-1.00000E-01) -- cycle ; 
\draw[Equator-full] (axis cs:3.60000E+01,1.00000E-01) -- (axis cs:3.70000E+01,1.00000E-01) -- (axis cs:3.70000E+01,-1.00000E-01) -- (axis cs:3.60000E+01,-1.00000E-01) -- cycle ; 
\draw[Equator-full] (axis cs:3.80000E+01,1.00000E-01) -- (axis cs:3.90000E+01,1.00000E-01) -- (axis cs:3.90000E+01,-1.00000E-01) -- (axis cs:3.80000E+01,-1.00000E-01) -- cycle ; 
\draw[Equator-full] (axis cs:4.00000E+01,1.00000E-01) -- (axis cs:4.10000E+01,1.00000E-01) -- (axis cs:4.10000E+01,-1.00000E-01) -- (axis cs:4.00000E+01,-1.00000E-01) -- cycle ; 
\draw[Equator-full] (axis cs:4.20000E+01,1.00000E-01) -- (axis cs:4.30000E+01,1.00000E-01) -- (axis cs:4.30000E+01,-1.00000E-01) -- (axis cs:4.20000E+01,-1.00000E-01) -- cycle ; 
\draw[Equator-full] (axis cs:4.40000E+01,1.00000E-01) -- (axis cs:4.50000E+01,1.00000E-01) -- (axis cs:4.50000E+01,-1.00000E-01) -- (axis cs:4.40000E+01,-1.00000E-01) -- cycle ; 
\draw[Equator-full] (axis cs:4.60000E+01,1.00000E-01) -- (axis cs:4.70000E+01,1.00000E-01) -- (axis cs:4.70000E+01,-1.00000E-01) -- (axis cs:4.60000E+01,-1.00000E-01) -- cycle ; 
\draw[Equator-full] (axis cs:4.80000E+01,1.00000E-01) -- (axis cs:4.90000E+01,1.00000E-01) -- (axis cs:4.90000E+01,-1.00000E-01) -- (axis cs:4.80000E+01,-1.00000E-01) -- cycle ; 
\draw[Equator-full] (axis cs:5.00000E+01,1.00000E-01) -- (axis cs:5.10000E+01,1.00000E-01) -- (axis cs:5.10000E+01,-1.00000E-01) -- (axis cs:5.00000E+01,-1.00000E-01) -- cycle ; 
\draw[Equator-full] (axis cs:5.20000E+01,1.00000E-01) -- (axis cs:5.30000E+01,1.00000E-01) -- (axis cs:5.30000E+01,-1.00000E-01) -- (axis cs:5.20000E+01,-1.00000E-01) -- cycle ; 
\draw[Equator-full] (axis cs:5.40000E+01,1.00000E-01) -- (axis cs:5.50000E+01,1.00000E-01) -- (axis cs:5.50000E+01,-1.00000E-01) -- (axis cs:5.40000E+01,-1.00000E-01) -- cycle ; 
\draw[Equator-full] (axis cs:5.60000E+01,1.00000E-01) -- (axis cs:5.70000E+01,1.00000E-01) -- (axis cs:5.70000E+01,-1.00000E-01) -- (axis cs:5.60000E+01,-1.00000E-01) -- cycle ; 
\draw[Equator-full] (axis cs:5.80000E+01,1.00000E-01) -- (axis cs:5.90000E+01,1.00000E-01) -- (axis cs:5.90000E+01,-1.00000E-01) -- (axis cs:5.80000E+01,-1.00000E-01) -- cycle ; 
\draw[Equator-full] (axis cs:6.00000E+01,1.00000E-01) -- (axis cs:6.10000E+01,1.00000E-01) -- (axis cs:6.10000E+01,-1.00000E-01) -- (axis cs:6.00000E+01,-1.00000E-01) -- cycle ; 
\draw[Equator-full] (axis cs:6.20000E+01,1.00000E-01) -- (axis cs:6.30000E+01,1.00000E-01) -- (axis cs:6.30000E+01,-1.00000E-01) -- (axis cs:6.20000E+01,-1.00000E-01) -- cycle ; 
\draw[Equator-full] (axis cs:6.40000E+01,1.00000E-01) -- (axis cs:6.50000E+01,1.00000E-01) -- (axis cs:6.50000E+01,-1.00000E-01) -- (axis cs:6.40000E+01,-1.00000E-01) -- cycle ; 
\draw[Equator-full] (axis cs:6.60000E+01,1.00000E-01) -- (axis cs:6.70000E+01,1.00000E-01) -- (axis cs:6.70000E+01,-1.00000E-01) -- (axis cs:6.60000E+01,-1.00000E-01) -- cycle ; 
\draw[Equator-full] (axis cs:6.80000E+01,1.00000E-01) -- (axis cs:6.90000E+01,1.00000E-01) -- (axis cs:6.90000E+01,-1.00000E-01) -- (axis cs:6.80000E+01,-1.00000E-01) -- cycle ; 
\draw[Equator-full] (axis cs:7.00000E+01,1.00000E-01) -- (axis cs:7.10000E+01,1.00000E-01) -- (axis cs:7.10000E+01,-1.00000E-01) -- (axis cs:7.00000E+01,-1.00000E-01) -- cycle ; 
\draw[Equator-full] (axis cs:7.20000E+01,1.00000E-01) -- (axis cs:7.30000E+01,1.00000E-01) -- (axis cs:7.30000E+01,-1.00000E-01) -- (axis cs:7.20000E+01,-1.00000E-01) -- cycle ; 
\draw[Equator-full] (axis cs:7.40000E+01,1.00000E-01) -- (axis cs:7.50000E+01,1.00000E-01) -- (axis cs:7.50000E+01,-1.00000E-01) -- (axis cs:7.40000E+01,-1.00000E-01) -- cycle ; 
\draw[Equator-full] (axis cs:7.60000E+01,1.00000E-01) -- (axis cs:7.70000E+01,1.00000E-01) -- (axis cs:7.70000E+01,-1.00000E-01) -- (axis cs:7.60000E+01,-1.00000E-01) -- cycle ; 
\draw[Equator-full] (axis cs:7.80000E+01,1.00000E-01) -- (axis cs:7.90000E+01,1.00000E-01) -- (axis cs:7.90000E+01,-1.00000E-01) -- (axis cs:7.80000E+01,-1.00000E-01) -- cycle ; 
\draw[Equator-full] (axis cs:8.00000E+01,1.00000E-01) -- (axis cs:8.10000E+01,1.00000E-01) -- (axis cs:8.10000E+01,-1.00000E-01) -- (axis cs:8.00000E+01,-1.00000E-01) -- cycle ; 
\draw[Equator-full] (axis cs:8.20000E+01,1.00000E-01) -- (axis cs:8.30000E+01,1.00000E-01) -- (axis cs:8.30000E+01,-1.00000E-01) -- (axis cs:8.20000E+01,-1.00000E-01) -- cycle ; 
\draw[Equator-full] (axis cs:8.40000E+01,1.00000E-01) -- (axis cs:8.50000E+01,1.00000E-01) -- (axis cs:8.50000E+01,-1.00000E-01) -- (axis cs:8.40000E+01,-1.00000E-01) -- cycle ; 
\draw[Equator-full] (axis cs:8.60000E+01,1.00000E-01) -- (axis cs:8.70000E+01,1.00000E-01) -- (axis cs:8.70000E+01,-1.00000E-01) -- (axis cs:8.60000E+01,-1.00000E-01) -- cycle ; 
\draw[Equator-full] (axis cs:8.80000E+01,1.00000E-01) -- (axis cs:8.90000E+01,1.00000E-01) -- (axis cs:8.90000E+01,-1.00000E-01) -- (axis cs:8.80000E+01,-1.00000E-01) -- cycle ; 
\draw[Equator-full] (axis cs:9.00000E+01,1.00000E-01) -- (axis cs:9.10000E+01,1.00000E-01) -- (axis cs:9.10000E+01,-1.00000E-01) -- (axis cs:9.00000E+01,-1.00000E-01) -- cycle ; 
\draw[Equator-full] (axis cs:9.20000E+01,1.00000E-01) -- (axis cs:9.30000E+01,1.00000E-01) -- (axis cs:9.30000E+01,-1.00000E-01) -- (axis cs:9.20000E+01,-1.00000E-01) -- cycle ; 
\draw[Equator-full] (axis cs:9.40000E+01,1.00000E-01) -- (axis cs:9.50000E+01,1.00000E-01) -- (axis cs:9.50000E+01,-1.00000E-01) -- (axis cs:9.40000E+01,-1.00000E-01) -- cycle ; 
\draw[Equator-full] (axis cs:9.60000E+01,1.00000E-01) -- (axis cs:9.70000E+01,1.00000E-01) -- (axis cs:9.70000E+01,-1.00000E-01) -- (axis cs:9.60000E+01,-1.00000E-01) -- cycle ; 
\draw[Equator-full] (axis cs:9.80000E+01,1.00000E-01) -- (axis cs:9.90000E+01,1.00000E-01) -- (axis cs:9.90000E+01,-1.00000E-01) -- (axis cs:9.80000E+01,-1.00000E-01) -- cycle ; 
\draw[Equator-full] (axis cs:1.00000E+02,1.00000E-01) -- (axis cs:1.01000E+02,1.00000E-01) -- (axis cs:1.01000E+02,-1.00000E-01) -- (axis cs:1.00000E+02,-1.00000E-01) -- cycle ; 
\draw[Equator-full] (axis cs:1.02000E+02,1.00000E-01) -- (axis cs:1.03000E+02,1.00000E-01) -- (axis cs:1.03000E+02,-1.00000E-01) -- (axis cs:1.02000E+02,-1.00000E-01) -- cycle ; 
\draw[Equator-full] (axis cs:1.04000E+02,1.00000E-01) -- (axis cs:1.05000E+02,1.00000E-01) -- (axis cs:1.05000E+02,-1.00000E-01) -- (axis cs:1.04000E+02,-1.00000E-01) -- cycle ; 
\draw[Equator-full] (axis cs:1.06000E+02,1.00000E-01) -- (axis cs:1.07000E+02,1.00000E-01) -- (axis cs:1.07000E+02,-1.00000E-01) -- (axis cs:1.06000E+02,-1.00000E-01) -- cycle ; 
\draw[Equator-full] (axis cs:1.08000E+02,1.00000E-01) -- (axis cs:1.09000E+02,1.00000E-01) -- (axis cs:1.09000E+02,-1.00000E-01) -- (axis cs:1.08000E+02,-1.00000E-01) -- cycle ; 
\draw[Equator-full] (axis cs:1.10000E+02,1.00000E-01) -- (axis cs:1.11000E+02,1.00000E-01) -- (axis cs:1.11000E+02,-1.00000E-01) -- (axis cs:1.10000E+02,-1.00000E-01) -- cycle ; 
\draw[Equator-full] (axis cs:1.12000E+02,1.00000E-01) -- (axis cs:1.13000E+02,1.00000E-01) -- (axis cs:1.13000E+02,-1.00000E-01) -- (axis cs:1.12000E+02,-1.00000E-01) -- cycle ; 
\draw[Equator-full] (axis cs:1.14000E+02,1.00000E-01) -- (axis cs:1.15000E+02,1.00000E-01) -- (axis cs:1.15000E+02,-1.00000E-01) -- (axis cs:1.14000E+02,-1.00000E-01) -- cycle ; 
\draw[Equator-full] (axis cs:1.16000E+02,1.00000E-01) -- (axis cs:1.17000E+02,1.00000E-01) -- (axis cs:1.17000E+02,-1.00000E-01) -- (axis cs:1.16000E+02,-1.00000E-01) -- cycle ; 
\draw[Equator-full] (axis cs:1.18000E+02,1.00000E-01) -- (axis cs:1.19000E+02,1.00000E-01) -- (axis cs:1.19000E+02,-1.00000E-01) -- (axis cs:1.18000E+02,-1.00000E-01) -- cycle ; 
\draw[Equator-full] (axis cs:1.20000E+02,1.00000E-01) -- (axis cs:1.21000E+02,1.00000E-01) -- (axis cs:1.21000E+02,-1.00000E-01) -- (axis cs:1.20000E+02,-1.00000E-01) -- cycle ; 
\draw[Equator-full] (axis cs:1.22000E+02,1.00000E-01) -- (axis cs:1.23000E+02,1.00000E-01) -- (axis cs:1.23000E+02,-1.00000E-01) -- (axis cs:1.22000E+02,-1.00000E-01) -- cycle ; 
\draw[Equator-full] (axis cs:1.24000E+02,1.00000E-01) -- (axis cs:1.25000E+02,1.00000E-01) -- (axis cs:1.25000E+02,-1.00000E-01) -- (axis cs:1.24000E+02,-1.00000E-01) -- cycle ; 
\draw[Equator-full] (axis cs:1.26000E+02,1.00000E-01) -- (axis cs:1.27000E+02,1.00000E-01) -- (axis cs:1.27000E+02,-1.00000E-01) -- (axis cs:1.26000E+02,-1.00000E-01) -- cycle ; 
\draw[Equator-full] (axis cs:1.28000E+02,1.00000E-01) -- (axis cs:1.29000E+02,1.00000E-01) -- (axis cs:1.29000E+02,-1.00000E-01) -- (axis cs:1.28000E+02,-1.00000E-01) -- cycle ; 
\draw[Equator-full] (axis cs:1.30000E+02,1.00000E-01) -- (axis cs:1.31000E+02,1.00000E-01) -- (axis cs:1.31000E+02,-1.00000E-01) -- (axis cs:1.30000E+02,-1.00000E-01) -- cycle ; 
\draw[Equator-full] (axis cs:1.32000E+02,1.00000E-01) -- (axis cs:1.33000E+02,1.00000E-01) -- (axis cs:1.33000E+02,-1.00000E-01) -- (axis cs:1.32000E+02,-1.00000E-01) -- cycle ; 
\draw[Equator-full] (axis cs:1.34000E+02,1.00000E-01) -- (axis cs:1.35000E+02,1.00000E-01) -- (axis cs:1.35000E+02,-1.00000E-01) -- (axis cs:1.34000E+02,-1.00000E-01) -- cycle ; 
\draw[Equator-full] (axis cs:1.36000E+02,1.00000E-01) -- (axis cs:1.37000E+02,1.00000E-01) -- (axis cs:1.37000E+02,-1.00000E-01) -- (axis cs:1.36000E+02,-1.00000E-01) -- cycle ; 
\draw[Equator-full] (axis cs:1.38000E+02,1.00000E-01) -- (axis cs:1.39000E+02,1.00000E-01) -- (axis cs:1.39000E+02,-1.00000E-01) -- (axis cs:1.38000E+02,-1.00000E-01) -- cycle ; 
\draw[Equator-full] (axis cs:1.40000E+02,1.00000E-01) -- (axis cs:1.41000E+02,1.00000E-01) -- (axis cs:1.41000E+02,-1.00000E-01) -- (axis cs:1.40000E+02,-1.00000E-01) -- cycle ; 
\draw[Equator-full] (axis cs:1.42000E+02,1.00000E-01) -- (axis cs:1.43000E+02,1.00000E-01) -- (axis cs:1.43000E+02,-1.00000E-01) -- (axis cs:1.42000E+02,-1.00000E-01) -- cycle ; 
\draw[Equator-full] (axis cs:1.44000E+02,1.00000E-01) -- (axis cs:1.45000E+02,1.00000E-01) -- (axis cs:1.45000E+02,-1.00000E-01) -- (axis cs:1.44000E+02,-1.00000E-01) -- cycle ; 
\draw[Equator-full] (axis cs:1.46000E+02,1.00000E-01) -- (axis cs:1.47000E+02,1.00000E-01) -- (axis cs:1.47000E+02,-1.00000E-01) -- (axis cs:1.46000E+02,-1.00000E-01) -- cycle ; 
\draw[Equator-full] (axis cs:1.48000E+02,1.00000E-01) -- (axis cs:1.49000E+02,1.00000E-01) -- (axis cs:1.49000E+02,-1.00000E-01) -- (axis cs:1.48000E+02,-1.00000E-01) -- cycle ; 
\draw[Equator-full] (axis cs:1.50000E+02,1.00000E-01) -- (axis cs:1.51000E+02,1.00000E-01) -- (axis cs:1.51000E+02,-1.00000E-01) -- (axis cs:1.50000E+02,-1.00000E-01) -- cycle ; 
\draw[Equator-full] (axis cs:1.52000E+02,1.00000E-01) -- (axis cs:1.53000E+02,1.00000E-01) -- (axis cs:1.53000E+02,-1.00000E-01) -- (axis cs:1.52000E+02,-1.00000E-01) -- cycle ; 
\draw[Equator-full] (axis cs:1.54000E+02,1.00000E-01) -- (axis cs:1.55000E+02,1.00000E-01) -- (axis cs:1.55000E+02,-1.00000E-01) -- (axis cs:1.54000E+02,-1.00000E-01) -- cycle ; 
\draw[Equator-full] (axis cs:1.56000E+02,1.00000E-01) -- (axis cs:1.57000E+02,1.00000E-01) -- (axis cs:1.57000E+02,-1.00000E-01) -- (axis cs:1.56000E+02,-1.00000E-01) -- cycle ; 
\draw[Equator-full] (axis cs:1.58000E+02,1.00000E-01) -- (axis cs:1.59000E+02,1.00000E-01) -- (axis cs:1.59000E+02,-1.00000E-01) -- (axis cs:1.58000E+02,-1.00000E-01) -- cycle ; 
\draw[Equator-full] (axis cs:1.60000E+02,1.00000E-01) -- (axis cs:1.61000E+02,1.00000E-01) -- (axis cs:1.61000E+02,-1.00000E-01) -- (axis cs:1.60000E+02,-1.00000E-01) -- cycle ; 
\draw[Equator-full] (axis cs:1.62000E+02,1.00000E-01) -- (axis cs:1.63000E+02,1.00000E-01) -- (axis cs:1.63000E+02,-1.00000E-01) -- (axis cs:1.62000E+02,-1.00000E-01) -- cycle ; 
\draw[Equator-full] (axis cs:1.64000E+02,1.00000E-01) -- (axis cs:1.65000E+02,1.00000E-01) -- (axis cs:1.65000E+02,-1.00000E-01) -- (axis cs:1.64000E+02,-1.00000E-01) -- cycle ; 
\draw[Equator-full] (axis cs:1.66000E+02,1.00000E-01) -- (axis cs:1.67000E+02,1.00000E-01) -- (axis cs:1.67000E+02,-1.00000E-01) -- (axis cs:1.66000E+02,-1.00000E-01) -- cycle ; 
\draw[Equator-full] (axis cs:1.68000E+02,1.00000E-01) -- (axis cs:1.69000E+02,1.00000E-01) -- (axis cs:1.69000E+02,-1.00000E-01) -- (axis cs:1.68000E+02,-1.00000E-01) -- cycle ; 
\draw[Equator-full] (axis cs:1.70000E+02,1.00000E-01) -- (axis cs:1.71000E+02,1.00000E-01) -- (axis cs:1.71000E+02,-1.00000E-01) -- (axis cs:1.70000E+02,-1.00000E-01) -- cycle ; 
\draw[Equator-full] (axis cs:1.72000E+02,1.00000E-01) -- (axis cs:1.73000E+02,1.00000E-01) -- (axis cs:1.73000E+02,-1.00000E-01) -- (axis cs:1.72000E+02,-1.00000E-01) -- cycle ; 
\draw[Equator-full] (axis cs:1.74000E+02,1.00000E-01) -- (axis cs:1.75000E+02,1.00000E-01) -- (axis cs:1.75000E+02,-1.00000E-01) -- (axis cs:1.74000E+02,-1.00000E-01) -- cycle ; 
\draw[Equator-full] (axis cs:1.76000E+02,1.00000E-01) -- (axis cs:1.77000E+02,1.00000E-01) -- (axis cs:1.77000E+02,-1.00000E-01) -- (axis cs:1.76000E+02,-1.00000E-01) -- cycle ; 
\draw[Equator-full] (axis cs:1.78000E+02,1.00000E-01) -- (axis cs:1.79000E+02,1.00000E-01) -- (axis cs:1.79000E+02,-1.00000E-01) -- (axis cs:1.78000E+02,-1.00000E-01) -- cycle ; 
\draw[Equator-full] (axis cs:1.80000E+02,1.00000E-01) -- (axis cs:1.81000E+02,1.00000E-01) -- (axis cs:1.81000E+02,-1.00000E-01) -- (axis cs:1.80000E+02,-1.00000E-01) -- cycle ; 
\draw[Equator-full] (axis cs:1.82000E+02,1.00000E-01) -- (axis cs:1.83000E+02,1.00000E-01) -- (axis cs:1.83000E+02,-1.00000E-01) -- (axis cs:1.82000E+02,-1.00000E-01) -- cycle ; 
\draw[Equator-full] (axis cs:1.84000E+02,1.00000E-01) -- (axis cs:1.85000E+02,1.00000E-01) -- (axis cs:1.85000E+02,-1.00000E-01) -- (axis cs:1.84000E+02,-1.00000E-01) -- cycle ; 
\draw[Equator-full] (axis cs:1.86000E+02,1.00000E-01) -- (axis cs:1.87000E+02,1.00000E-01) -- (axis cs:1.87000E+02,-1.00000E-01) -- (axis cs:1.86000E+02,-1.00000E-01) -- cycle ; 
\draw[Equator-full] (axis cs:1.88000E+02,1.00000E-01) -- (axis cs:1.89000E+02,1.00000E-01) -- (axis cs:1.89000E+02,-1.00000E-01) -- (axis cs:1.88000E+02,-1.00000E-01) -- cycle ; 
\draw[Equator-full] (axis cs:1.90000E+02,1.00000E-01) -- (axis cs:1.91000E+02,1.00000E-01) -- (axis cs:1.91000E+02,-1.00000E-01) -- (axis cs:1.90000E+02,-1.00000E-01) -- cycle ; 
\draw[Equator-full] (axis cs:1.92000E+02,1.00000E-01) -- (axis cs:1.93000E+02,1.00000E-01) -- (axis cs:1.93000E+02,-1.00000E-01) -- (axis cs:1.92000E+02,-1.00000E-01) -- cycle ; 
\draw[Equator-full] (axis cs:1.94000E+02,1.00000E-01) -- (axis cs:1.95000E+02,1.00000E-01) -- (axis cs:1.95000E+02,-1.00000E-01) -- (axis cs:1.94000E+02,-1.00000E-01) -- cycle ; 
\draw[Equator-full] (axis cs:1.96000E+02,1.00000E-01) -- (axis cs:1.97000E+02,1.00000E-01) -- (axis cs:1.97000E+02,-1.00000E-01) -- (axis cs:1.96000E+02,-1.00000E-01) -- cycle ; 
\draw[Equator-full] (axis cs:1.98000E+02,1.00000E-01) -- (axis cs:1.99000E+02,1.00000E-01) -- (axis cs:1.99000E+02,-1.00000E-01) -- (axis cs:1.98000E+02,-1.00000E-01) -- cycle ; 
\draw[Equator-full] (axis cs:2.00000E+02,1.00000E-01) -- (axis cs:2.01000E+02,1.00000E-01) -- (axis cs:2.01000E+02,-1.00000E-01) -- (axis cs:2.00000E+02,-1.00000E-01) -- cycle ; 
\draw[Equator-full] (axis cs:2.02000E+02,1.00000E-01) -- (axis cs:2.03000E+02,1.00000E-01) -- (axis cs:2.03000E+02,-1.00000E-01) -- (axis cs:2.02000E+02,-1.00000E-01) -- cycle ; 
\draw[Equator-full] (axis cs:2.04000E+02,1.00000E-01) -- (axis cs:2.05000E+02,1.00000E-01) -- (axis cs:2.05000E+02,-1.00000E-01) -- (axis cs:2.04000E+02,-1.00000E-01) -- cycle ; 
\draw[Equator-full] (axis cs:2.06000E+02,1.00000E-01) -- (axis cs:2.07000E+02,1.00000E-01) -- (axis cs:2.07000E+02,-1.00000E-01) -- (axis cs:2.06000E+02,-1.00000E-01) -- cycle ; 
\draw[Equator-full] (axis cs:2.08000E+02,1.00000E-01) -- (axis cs:2.09000E+02,1.00000E-01) -- (axis cs:2.09000E+02,-1.00000E-01) -- (axis cs:2.08000E+02,-1.00000E-01) -- cycle ; 
\draw[Equator-full] (axis cs:2.10000E+02,1.00000E-01) -- (axis cs:2.11000E+02,1.00000E-01) -- (axis cs:2.11000E+02,-1.00000E-01) -- (axis cs:2.10000E+02,-1.00000E-01) -- cycle ; 
\draw[Equator-full] (axis cs:2.12000E+02,1.00000E-01) -- (axis cs:2.13000E+02,1.00000E-01) -- (axis cs:2.13000E+02,-1.00000E-01) -- (axis cs:2.12000E+02,-1.00000E-01) -- cycle ; 
\draw[Equator-full] (axis cs:2.14000E+02,1.00000E-01) -- (axis cs:2.15000E+02,1.00000E-01) -- (axis cs:2.15000E+02,-1.00000E-01) -- (axis cs:2.14000E+02,-1.00000E-01) -- cycle ; 
\draw[Equator-full] (axis cs:2.16000E+02,1.00000E-01) -- (axis cs:2.17000E+02,1.00000E-01) -- (axis cs:2.17000E+02,-1.00000E-01) -- (axis cs:2.16000E+02,-1.00000E-01) -- cycle ; 
\draw[Equator-full] (axis cs:2.18000E+02,1.00000E-01) -- (axis cs:2.19000E+02,1.00000E-01) -- (axis cs:2.19000E+02,-1.00000E-01) -- (axis cs:2.18000E+02,-1.00000E-01) -- cycle ; 
\draw[Equator-full] (axis cs:2.20000E+02,1.00000E-01) -- (axis cs:2.21000E+02,1.00000E-01) -- (axis cs:2.21000E+02,-1.00000E-01) -- (axis cs:2.20000E+02,-1.00000E-01) -- cycle ; 
\draw[Equator-full] (axis cs:2.22000E+02,1.00000E-01) -- (axis cs:2.23000E+02,1.00000E-01) -- (axis cs:2.23000E+02,-1.00000E-01) -- (axis cs:2.22000E+02,-1.00000E-01) -- cycle ; 
\draw[Equator-full] (axis cs:2.24000E+02,1.00000E-01) -- (axis cs:2.25000E+02,1.00000E-01) -- (axis cs:2.25000E+02,-1.00000E-01) -- (axis cs:2.24000E+02,-1.00000E-01) -- cycle ; 
\draw[Equator-full] (axis cs:2.26000E+02,1.00000E-01) -- (axis cs:2.27000E+02,1.00000E-01) -- (axis cs:2.27000E+02,-1.00000E-01) -- (axis cs:2.26000E+02,-1.00000E-01) -- cycle ; 
\draw[Equator-full] (axis cs:2.28000E+02,1.00000E-01) -- (axis cs:2.29000E+02,1.00000E-01) -- (axis cs:2.29000E+02,-1.00000E-01) -- (axis cs:2.28000E+02,-1.00000E-01) -- cycle ; 
\draw[Equator-full] (axis cs:2.30000E+02,1.00000E-01) -- (axis cs:2.31000E+02,1.00000E-01) -- (axis cs:2.31000E+02,-1.00000E-01) -- (axis cs:2.30000E+02,-1.00000E-01) -- cycle ; 
\draw[Equator-full] (axis cs:2.32000E+02,1.00000E-01) -- (axis cs:2.33000E+02,1.00000E-01) -- (axis cs:2.33000E+02,-1.00000E-01) -- (axis cs:2.32000E+02,-1.00000E-01) -- cycle ; 
\draw[Equator-full] (axis cs:2.34000E+02,1.00000E-01) -- (axis cs:2.35000E+02,1.00000E-01) -- (axis cs:2.35000E+02,-1.00000E-01) -- (axis cs:2.34000E+02,-1.00000E-01) -- cycle ; 
\draw[Equator-full] (axis cs:2.36000E+02,1.00000E-01) -- (axis cs:2.37000E+02,1.00000E-01) -- (axis cs:2.37000E+02,-1.00000E-01) -- (axis cs:2.36000E+02,-1.00000E-01) -- cycle ; 
\draw[Equator-full] (axis cs:2.38000E+02,1.00000E-01) -- (axis cs:2.39000E+02,1.00000E-01) -- (axis cs:2.39000E+02,-1.00000E-01) -- (axis cs:2.38000E+02,-1.00000E-01) -- cycle ; 
\draw[Equator-full] (axis cs:2.40000E+02,1.00000E-01) -- (axis cs:2.41000E+02,1.00000E-01) -- (axis cs:2.41000E+02,-1.00000E-01) -- (axis cs:2.40000E+02,-1.00000E-01) -- cycle ; 
\draw[Equator-full] (axis cs:2.42000E+02,1.00000E-01) -- (axis cs:2.43000E+02,1.00000E-01) -- (axis cs:2.43000E+02,-1.00000E-01) -- (axis cs:2.42000E+02,-1.00000E-01) -- cycle ; 
\draw[Equator-full] (axis cs:2.44000E+02,1.00000E-01) -- (axis cs:2.45000E+02,1.00000E-01) -- (axis cs:2.45000E+02,-1.00000E-01) -- (axis cs:2.44000E+02,-1.00000E-01) -- cycle ; 
\draw[Equator-full] (axis cs:2.46000E+02,1.00000E-01) -- (axis cs:2.47000E+02,1.00000E-01) -- (axis cs:2.47000E+02,-1.00000E-01) -- (axis cs:2.46000E+02,-1.00000E-01) -- cycle ; 
\draw[Equator-full] (axis cs:2.48000E+02,1.00000E-01) -- (axis cs:2.49000E+02,1.00000E-01) -- (axis cs:2.49000E+02,-1.00000E-01) -- (axis cs:2.48000E+02,-1.00000E-01) -- cycle ; 
\draw[Equator-full] (axis cs:2.50000E+02,1.00000E-01) -- (axis cs:2.51000E+02,1.00000E-01) -- (axis cs:2.51000E+02,-1.00000E-01) -- (axis cs:2.50000E+02,-1.00000E-01) -- cycle ; 
\draw[Equator-full] (axis cs:2.52000E+02,1.00000E-01) -- (axis cs:2.53000E+02,1.00000E-01) -- (axis cs:2.53000E+02,-1.00000E-01) -- (axis cs:2.52000E+02,-1.00000E-01) -- cycle ; 
\draw[Equator-full] (axis cs:2.54000E+02,1.00000E-01) -- (axis cs:2.55000E+02,1.00000E-01) -- (axis cs:2.55000E+02,-1.00000E-01) -- (axis cs:2.54000E+02,-1.00000E-01) -- cycle ; 
\draw[Equator-full] (axis cs:2.56000E+02,1.00000E-01) -- (axis cs:2.57000E+02,1.00000E-01) -- (axis cs:2.57000E+02,-1.00000E-01) -- (axis cs:2.56000E+02,-1.00000E-01) -- cycle ; 
\draw[Equator-full] (axis cs:2.58000E+02,1.00000E-01) -- (axis cs:2.59000E+02,1.00000E-01) -- (axis cs:2.59000E+02,-1.00000E-01) -- (axis cs:2.58000E+02,-1.00000E-01) -- cycle ; 
\draw[Equator-full] (axis cs:2.60000E+02,1.00000E-01) -- (axis cs:2.61000E+02,1.00000E-01) -- (axis cs:2.61000E+02,-1.00000E-01) -- (axis cs:2.60000E+02,-1.00000E-01) -- cycle ; 
\draw[Equator-full] (axis cs:2.62000E+02,1.00000E-01) -- (axis cs:2.63000E+02,1.00000E-01) -- (axis cs:2.63000E+02,-1.00000E-01) -- (axis cs:2.62000E+02,-1.00000E-01) -- cycle ; 
\draw[Equator-full] (axis cs:2.64000E+02,1.00000E-01) -- (axis cs:2.65000E+02,1.00000E-01) -- (axis cs:2.65000E+02,-1.00000E-01) -- (axis cs:2.64000E+02,-1.00000E-01) -- cycle ; 
\draw[Equator-full] (axis cs:2.66000E+02,1.00000E-01) -- (axis cs:2.67000E+02,1.00000E-01) -- (axis cs:2.67000E+02,-1.00000E-01) -- (axis cs:2.66000E+02,-1.00000E-01) -- cycle ; 
\draw[Equator-full] (axis cs:2.68000E+02,1.00000E-01) -- (axis cs:2.69000E+02,1.00000E-01) -- (axis cs:2.69000E+02,-1.00000E-01) -- (axis cs:2.68000E+02,-1.00000E-01) -- cycle ; 
\draw[Equator-full] (axis cs:2.70000E+02,1.00000E-01) -- (axis cs:2.71000E+02,1.00000E-01) -- (axis cs:2.71000E+02,-1.00000E-01) -- (axis cs:2.70000E+02,-1.00000E-01) -- cycle ; 
\draw[Equator-full] (axis cs:2.72000E+02,1.00000E-01) -- (axis cs:2.73000E+02,1.00000E-01) -- (axis cs:2.73000E+02,-1.00000E-01) -- (axis cs:2.72000E+02,-1.00000E-01) -- cycle ; 
\draw[Equator-full] (axis cs:2.74000E+02,1.00000E-01) -- (axis cs:2.75000E+02,1.00000E-01) -- (axis cs:2.75000E+02,-1.00000E-01) -- (axis cs:2.74000E+02,-1.00000E-01) -- cycle ; 
\draw[Equator-full] (axis cs:2.76000E+02,1.00000E-01) -- (axis cs:2.77000E+02,1.00000E-01) -- (axis cs:2.77000E+02,-1.00000E-01) -- (axis cs:2.76000E+02,-1.00000E-01) -- cycle ; 
\draw[Equator-full] (axis cs:2.78000E+02,1.00000E-01) -- (axis cs:2.79000E+02,1.00000E-01) -- (axis cs:2.79000E+02,-1.00000E-01) -- (axis cs:2.78000E+02,-1.00000E-01) -- cycle ; 
\draw[Equator-full] (axis cs:2.80000E+02,1.00000E-01) -- (axis cs:2.81000E+02,1.00000E-01) -- (axis cs:2.81000E+02,-1.00000E-01) -- (axis cs:2.80000E+02,-1.00000E-01) -- cycle ; 
\draw[Equator-full] (axis cs:2.82000E+02,1.00000E-01) -- (axis cs:2.83000E+02,1.00000E-01) -- (axis cs:2.83000E+02,-1.00000E-01) -- (axis cs:2.82000E+02,-1.00000E-01) -- cycle ; 
\draw[Equator-full] (axis cs:2.84000E+02,1.00000E-01) -- (axis cs:2.85000E+02,1.00000E-01) -- (axis cs:2.85000E+02,-1.00000E-01) -- (axis cs:2.84000E+02,-1.00000E-01) -- cycle ; 
\draw[Equator-full] (axis cs:2.86000E+02,1.00000E-01) -- (axis cs:2.87000E+02,1.00000E-01) -- (axis cs:2.87000E+02,-1.00000E-01) -- (axis cs:2.86000E+02,-1.00000E-01) -- cycle ; 
\draw[Equator-full] (axis cs:2.88000E+02,1.00000E-01) -- (axis cs:2.89000E+02,1.00000E-01) -- (axis cs:2.89000E+02,-1.00000E-01) -- (axis cs:2.88000E+02,-1.00000E-01) -- cycle ; 
\draw[Equator-full] (axis cs:2.90000E+02,1.00000E-01) -- (axis cs:2.91000E+02,1.00000E-01) -- (axis cs:2.91000E+02,-1.00000E-01) -- (axis cs:2.90000E+02,-1.00000E-01) -- cycle ; 
\draw[Equator-full] (axis cs:2.92000E+02,1.00000E-01) -- (axis cs:2.93000E+02,1.00000E-01) -- (axis cs:2.93000E+02,-1.00000E-01) -- (axis cs:2.92000E+02,-1.00000E-01) -- cycle ; 
\draw[Equator-full] (axis cs:2.94000E+02,1.00000E-01) -- (axis cs:2.95000E+02,1.00000E-01) -- (axis cs:2.95000E+02,-1.00000E-01) -- (axis cs:2.94000E+02,-1.00000E-01) -- cycle ; 
\draw[Equator-full] (axis cs:2.96000E+02,1.00000E-01) -- (axis cs:2.97000E+02,1.00000E-01) -- (axis cs:2.97000E+02,-1.00000E-01) -- (axis cs:2.96000E+02,-1.00000E-01) -- cycle ; 
\draw[Equator-full] (axis cs:2.98000E+02,1.00000E-01) -- (axis cs:2.99000E+02,1.00000E-01) -- (axis cs:2.99000E+02,-1.00000E-01) -- (axis cs:2.98000E+02,-1.00000E-01) -- cycle ; 
\draw[Equator-full] (axis cs:3.00000E+02,1.00000E-01) -- (axis cs:3.01000E+02,1.00000E-01) -- (axis cs:3.01000E+02,-1.00000E-01) -- (axis cs:3.00000E+02,-1.00000E-01) -- cycle ; 
\draw[Equator-full] (axis cs:3.02000E+02,1.00000E-01) -- (axis cs:3.03000E+02,1.00000E-01) -- (axis cs:3.03000E+02,-1.00000E-01) -- (axis cs:3.02000E+02,-1.00000E-01) -- cycle ; 
\draw[Equator-full] (axis cs:3.04000E+02,1.00000E-01) -- (axis cs:3.05000E+02,1.00000E-01) -- (axis cs:3.05000E+02,-1.00000E-01) -- (axis cs:3.04000E+02,-1.00000E-01) -- cycle ; 
\draw[Equator-full] (axis cs:3.06000E+02,1.00000E-01) -- (axis cs:3.07000E+02,1.00000E-01) -- (axis cs:3.07000E+02,-1.00000E-01) -- (axis cs:3.06000E+02,-1.00000E-01) -- cycle ; 
\draw[Equator-full] (axis cs:3.08000E+02,1.00000E-01) -- (axis cs:3.09000E+02,1.00000E-01) -- (axis cs:3.09000E+02,-1.00000E-01) -- (axis cs:3.08000E+02,-1.00000E-01) -- cycle ; 
\draw[Equator-full] (axis cs:3.10000E+02,1.00000E-01) -- (axis cs:3.11000E+02,1.00000E-01) -- (axis cs:3.11000E+02,-1.00000E-01) -- (axis cs:3.10000E+02,-1.00000E-01) -- cycle ; 
\draw[Equator-full] (axis cs:3.12000E+02,1.00000E-01) -- (axis cs:3.13000E+02,1.00000E-01) -- (axis cs:3.13000E+02,-1.00000E-01) -- (axis cs:3.12000E+02,-1.00000E-01) -- cycle ; 
\draw[Equator-full] (axis cs:3.14000E+02,1.00000E-01) -- (axis cs:3.15000E+02,1.00000E-01) -- (axis cs:3.15000E+02,-1.00000E-01) -- (axis cs:3.14000E+02,-1.00000E-01) -- cycle ; 
\draw[Equator-full] (axis cs:3.16000E+02,1.00000E-01) -- (axis cs:3.17000E+02,1.00000E-01) -- (axis cs:3.17000E+02,-1.00000E-01) -- (axis cs:3.16000E+02,-1.00000E-01) -- cycle ; 
\draw[Equator-full] (axis cs:3.18000E+02,1.00000E-01) -- (axis cs:3.19000E+02,1.00000E-01) -- (axis cs:3.19000E+02,-1.00000E-01) -- (axis cs:3.18000E+02,-1.00000E-01) -- cycle ; 
\draw[Equator-full] (axis cs:3.20000E+02,1.00000E-01) -- (axis cs:3.21000E+02,1.00000E-01) -- (axis cs:3.21000E+02,-1.00000E-01) -- (axis cs:3.20000E+02,-1.00000E-01) -- cycle ; 
\draw[Equator-full] (axis cs:3.22000E+02,1.00000E-01) -- (axis cs:3.23000E+02,1.00000E-01) -- (axis cs:3.23000E+02,-1.00000E-01) -- (axis cs:3.22000E+02,-1.00000E-01) -- cycle ; 
\draw[Equator-full] (axis cs:3.24000E+02,1.00000E-01) -- (axis cs:3.25000E+02,1.00000E-01) -- (axis cs:3.25000E+02,-1.00000E-01) -- (axis cs:3.24000E+02,-1.00000E-01) -- cycle ; 
\draw[Equator-full] (axis cs:3.26000E+02,1.00000E-01) -- (axis cs:3.27000E+02,1.00000E-01) -- (axis cs:3.27000E+02,-1.00000E-01) -- (axis cs:3.26000E+02,-1.00000E-01) -- cycle ; 
\draw[Equator-full] (axis cs:3.28000E+02,1.00000E-01) -- (axis cs:3.29000E+02,1.00000E-01) -- (axis cs:3.29000E+02,-1.00000E-01) -- (axis cs:3.28000E+02,-1.00000E-01) -- cycle ; 
\draw[Equator-full] (axis cs:3.30000E+02,1.00000E-01) -- (axis cs:3.31000E+02,1.00000E-01) -- (axis cs:3.31000E+02,-1.00000E-01) -- (axis cs:3.30000E+02,-1.00000E-01) -- cycle ; 
\draw[Equator-full] (axis cs:3.32000E+02,1.00000E-01) -- (axis cs:3.33000E+02,1.00000E-01) -- (axis cs:3.33000E+02,-1.00000E-01) -- (axis cs:3.32000E+02,-1.00000E-01) -- cycle ; 
\draw[Equator-full] (axis cs:3.34000E+02,1.00000E-01) -- (axis cs:3.35000E+02,1.00000E-01) -- (axis cs:3.35000E+02,-1.00000E-01) -- (axis cs:3.34000E+02,-1.00000E-01) -- cycle ; 
\draw[Equator-full] (axis cs:3.36000E+02,1.00000E-01) -- (axis cs:3.37000E+02,1.00000E-01) -- (axis cs:3.37000E+02,-1.00000E-01) -- (axis cs:3.36000E+02,-1.00000E-01) -- cycle ; 
\draw[Equator-full] (axis cs:3.38000E+02,1.00000E-01) -- (axis cs:3.39000E+02,1.00000E-01) -- (axis cs:3.39000E+02,-1.00000E-01) -- (axis cs:3.38000E+02,-1.00000E-01) -- cycle ; 
\draw[Equator-full] (axis cs:3.40000E+02,1.00000E-01) -- (axis cs:3.41000E+02,1.00000E-01) -- (axis cs:3.41000E+02,-1.00000E-01) -- (axis cs:3.40000E+02,-1.00000E-01) -- cycle ; 
\draw[Equator-full] (axis cs:3.42000E+02,1.00000E-01) -- (axis cs:3.43000E+02,1.00000E-01) -- (axis cs:3.43000E+02,-1.00000E-01) -- (axis cs:3.42000E+02,-1.00000E-01) -- cycle ; 
\draw[Equator-full] (axis cs:3.44000E+02,1.00000E-01) -- (axis cs:3.45000E+02,1.00000E-01) -- (axis cs:3.45000E+02,-1.00000E-01) -- (axis cs:3.44000E+02,-1.00000E-01) -- cycle ; 
\draw[Equator-full] (axis cs:3.46000E+02,1.00000E-01) -- (axis cs:3.47000E+02,1.00000E-01) -- (axis cs:3.47000E+02,-1.00000E-01) -- (axis cs:3.46000E+02,-1.00000E-01) -- cycle ; 
\draw[Equator-full] (axis cs:3.48000E+02,1.00000E-01) -- (axis cs:3.49000E+02,1.00000E-01) -- (axis cs:3.49000E+02,-1.00000E-01) -- (axis cs:3.48000E+02,-1.00000E-01) -- cycle ; 
\draw[Equator-full] (axis cs:3.50000E+02,1.00000E-01) -- (axis cs:3.51000E+02,1.00000E-01) -- (axis cs:3.51000E+02,-1.00000E-01) -- (axis cs:3.50000E+02,-1.00000E-01) -- cycle ; 
\draw[Equator-full] (axis cs:3.52000E+02,1.00000E-01) -- (axis cs:3.53000E+02,1.00000E-01) -- (axis cs:3.53000E+02,-1.00000E-01) -- (axis cs:3.52000E+02,-1.00000E-01) -- cycle ; 
\draw[Equator-full] (axis cs:3.54000E+02,1.00000E-01) -- (axis cs:3.55000E+02,1.00000E-01) -- (axis cs:3.55000E+02,-1.00000E-01) -- (axis cs:3.54000E+02,-1.00000E-01) -- cycle ; 
\draw[Equator-full] (axis cs:3.56000E+02,1.00000E-01) -- (axis cs:3.57000E+02,1.00000E-01) -- (axis cs:3.57000E+02,-1.00000E-01) -- (axis cs:3.56000E+02,-1.00000E-01) -- cycle ; 
\draw[Equator-full] (axis cs:3.58000E+02,1.00000E-01) -- (axis cs:3.59000E+02,1.00000E-01) -- (axis cs:3.59000E+02,-1.00000E-01) -- (axis cs:3.58000E+02,-1.00000E-01) -- cycle ; 
\draw[Equator-empty] (axis cs:1.00000E+00,1.00000E-01) -- (axis cs:2.00000E+00,1.00000E-01) -- (axis cs:2.00000E+00,-1.00000E-01) -- (axis cs:1.00000E+00,-1.00000E-01) -- cycle ; 
\draw[Equator-empty] (axis cs:3.00000E+00,1.00000E-01) -- (axis cs:4.00000E+00,1.00000E-01) -- (axis cs:4.00000E+00,-1.00000E-01) -- (axis cs:3.00000E+00,-1.00000E-01) -- cycle ; 
\draw[Equator-empty] (axis cs:5.00000E+00,1.00000E-01) -- (axis cs:6.00000E+00,1.00000E-01) -- (axis cs:6.00000E+00,-1.00000E-01) -- (axis cs:5.00000E+00,-1.00000E-01) -- cycle ; 
\draw[Equator-empty] (axis cs:7.00000E+00,1.00000E-01) -- (axis cs:8.00000E+00,1.00000E-01) -- (axis cs:8.00000E+00,-1.00000E-01) -- (axis cs:7.00000E+00,-1.00000E-01) -- cycle ; 
\draw[Equator-empty] (axis cs:9.00000E+00,1.00000E-01) -- (axis cs:1.00000E+01,1.00000E-01) -- (axis cs:1.00000E+01,-1.00000E-01) -- (axis cs:9.00000E+00,-1.00000E-01) -- cycle ; 
\draw[Equator-empty] (axis cs:1.10000E+01,1.00000E-01) -- (axis cs:1.20000E+01,1.00000E-01) -- (axis cs:1.20000E+01,-1.00000E-01) -- (axis cs:1.10000E+01,-1.00000E-01) -- cycle ; 
\draw[Equator-empty] (axis cs:1.30000E+01,1.00000E-01) -- (axis cs:1.40000E+01,1.00000E-01) -- (axis cs:1.40000E+01,-1.00000E-01) -- (axis cs:1.30000E+01,-1.00000E-01) -- cycle ; 
\draw[Equator-empty] (axis cs:1.50000E+01,1.00000E-01) -- (axis cs:1.60000E+01,1.00000E-01) -- (axis cs:1.60000E+01,-1.00000E-01) -- (axis cs:1.50000E+01,-1.00000E-01) -- cycle ; 
\draw[Equator-empty] (axis cs:1.70000E+01,1.00000E-01) -- (axis cs:1.80000E+01,1.00000E-01) -- (axis cs:1.80000E+01,-1.00000E-01) -- (axis cs:1.70000E+01,-1.00000E-01) -- cycle ; 
\draw[Equator-empty] (axis cs:1.90000E+01,1.00000E-01) -- (axis cs:2.00000E+01,1.00000E-01) -- (axis cs:2.00000E+01,-1.00000E-01) -- (axis cs:1.90000E+01,-1.00000E-01) -- cycle ; 
\draw[Equator-empty] (axis cs:2.10000E+01,1.00000E-01) -- (axis cs:2.20000E+01,1.00000E-01) -- (axis cs:2.20000E+01,-1.00000E-01) -- (axis cs:2.10000E+01,-1.00000E-01) -- cycle ; 
\draw[Equator-empty] (axis cs:2.30000E+01,1.00000E-01) -- (axis cs:2.40000E+01,1.00000E-01) -- (axis cs:2.40000E+01,-1.00000E-01) -- (axis cs:2.30000E+01,-1.00000E-01) -- cycle ; 
\draw[Equator-empty] (axis cs:2.50000E+01,1.00000E-01) -- (axis cs:2.60000E+01,1.00000E-01) -- (axis cs:2.60000E+01,-1.00000E-01) -- (axis cs:2.50000E+01,-1.00000E-01) -- cycle ; 
\draw[Equator-empty] (axis cs:2.70000E+01,1.00000E-01) -- (axis cs:2.80000E+01,1.00000E-01) -- (axis cs:2.80000E+01,-1.00000E-01) -- (axis cs:2.70000E+01,-1.00000E-01) -- cycle ; 
\draw[Equator-empty] (axis cs:2.90000E+01,1.00000E-01) -- (axis cs:3.00000E+01,1.00000E-01) -- (axis cs:3.00000E+01,-1.00000E-01) -- (axis cs:2.90000E+01,-1.00000E-01) -- cycle ; 
\draw[Equator-empty] (axis cs:3.10000E+01,1.00000E-01) -- (axis cs:3.20000E+01,1.00000E-01) -- (axis cs:3.20000E+01,-1.00000E-01) -- (axis cs:3.10000E+01,-1.00000E-01) -- cycle ; 
\draw[Equator-empty] (axis cs:3.30000E+01,1.00000E-01) -- (axis cs:3.40000E+01,1.00000E-01) -- (axis cs:3.40000E+01,-1.00000E-01) -- (axis cs:3.30000E+01,-1.00000E-01) -- cycle ; 
\draw[Equator-empty] (axis cs:3.50000E+01,1.00000E-01) -- (axis cs:3.60000E+01,1.00000E-01) -- (axis cs:3.60000E+01,-1.00000E-01) -- (axis cs:3.50000E+01,-1.00000E-01) -- cycle ; 
\draw[Equator-empty] (axis cs:3.70000E+01,1.00000E-01) -- (axis cs:3.80000E+01,1.00000E-01) -- (axis cs:3.80000E+01,-1.00000E-01) -- (axis cs:3.70000E+01,-1.00000E-01) -- cycle ; 
\draw[Equator-empty] (axis cs:3.90000E+01,1.00000E-01) -- (axis cs:4.00000E+01,1.00000E-01) -- (axis cs:4.00000E+01,-1.00000E-01) -- (axis cs:3.90000E+01,-1.00000E-01) -- cycle ; 
\draw[Equator-empty] (axis cs:4.10000E+01,1.00000E-01) -- (axis cs:4.20000E+01,1.00000E-01) -- (axis cs:4.20000E+01,-1.00000E-01) -- (axis cs:4.10000E+01,-1.00000E-01) -- cycle ; 
\draw[Equator-empty] (axis cs:4.30000E+01,1.00000E-01) -- (axis cs:4.40000E+01,1.00000E-01) -- (axis cs:4.40000E+01,-1.00000E-01) -- (axis cs:4.30000E+01,-1.00000E-01) -- cycle ; 
\draw[Equator-empty] (axis cs:4.50000E+01,1.00000E-01) -- (axis cs:4.60000E+01,1.00000E-01) -- (axis cs:4.60000E+01,-1.00000E-01) -- (axis cs:4.50000E+01,-1.00000E-01) -- cycle ; 
\draw[Equator-empty] (axis cs:4.70000E+01,1.00000E-01) -- (axis cs:4.80000E+01,1.00000E-01) -- (axis cs:4.80000E+01,-1.00000E-01) -- (axis cs:4.70000E+01,-1.00000E-01) -- cycle ; 
\draw[Equator-empty] (axis cs:4.90000E+01,1.00000E-01) -- (axis cs:5.00000E+01,1.00000E-01) -- (axis cs:5.00000E+01,-1.00000E-01) -- (axis cs:4.90000E+01,-1.00000E-01) -- cycle ; 
\draw[Equator-empty] (axis cs:5.10000E+01,1.00000E-01) -- (axis cs:5.20000E+01,1.00000E-01) -- (axis cs:5.20000E+01,-1.00000E-01) -- (axis cs:5.10000E+01,-1.00000E-01) -- cycle ; 
\draw[Equator-empty] (axis cs:5.30000E+01,1.00000E-01) -- (axis cs:5.40000E+01,1.00000E-01) -- (axis cs:5.40000E+01,-1.00000E-01) -- (axis cs:5.30000E+01,-1.00000E-01) -- cycle ; 
\draw[Equator-empty] (axis cs:5.50000E+01,1.00000E-01) -- (axis cs:5.60000E+01,1.00000E-01) -- (axis cs:5.60000E+01,-1.00000E-01) -- (axis cs:5.50000E+01,-1.00000E-01) -- cycle ; 
\draw[Equator-empty] (axis cs:5.70000E+01,1.00000E-01) -- (axis cs:5.80000E+01,1.00000E-01) -- (axis cs:5.80000E+01,-1.00000E-01) -- (axis cs:5.70000E+01,-1.00000E-01) -- cycle ; 
\draw[Equator-empty] (axis cs:5.90000E+01,1.00000E-01) -- (axis cs:6.00000E+01,1.00000E-01) -- (axis cs:6.00000E+01,-1.00000E-01) -- (axis cs:5.90000E+01,-1.00000E-01) -- cycle ; 
\draw[Equator-empty] (axis cs:6.10000E+01,1.00000E-01) -- (axis cs:6.20000E+01,1.00000E-01) -- (axis cs:6.20000E+01,-1.00000E-01) -- (axis cs:6.10000E+01,-1.00000E-01) -- cycle ; 
\draw[Equator-empty] (axis cs:6.30000E+01,1.00000E-01) -- (axis cs:6.40000E+01,1.00000E-01) -- (axis cs:6.40000E+01,-1.00000E-01) -- (axis cs:6.30000E+01,-1.00000E-01) -- cycle ; 
\draw[Equator-empty] (axis cs:6.50000E+01,1.00000E-01) -- (axis cs:6.60000E+01,1.00000E-01) -- (axis cs:6.60000E+01,-1.00000E-01) -- (axis cs:6.50000E+01,-1.00000E-01) -- cycle ; 
\draw[Equator-empty] (axis cs:6.70000E+01,1.00000E-01) -- (axis cs:6.80000E+01,1.00000E-01) -- (axis cs:6.80000E+01,-1.00000E-01) -- (axis cs:6.70000E+01,-1.00000E-01) -- cycle ; 
\draw[Equator-empty] (axis cs:6.90000E+01,1.00000E-01) -- (axis cs:7.00000E+01,1.00000E-01) -- (axis cs:7.00000E+01,-1.00000E-01) -- (axis cs:6.90000E+01,-1.00000E-01) -- cycle ; 
\draw[Equator-empty] (axis cs:7.10000E+01,1.00000E-01) -- (axis cs:7.20000E+01,1.00000E-01) -- (axis cs:7.20000E+01,-1.00000E-01) -- (axis cs:7.10000E+01,-1.00000E-01) -- cycle ; 
\draw[Equator-empty] (axis cs:7.30000E+01,1.00000E-01) -- (axis cs:7.40000E+01,1.00000E-01) -- (axis cs:7.40000E+01,-1.00000E-01) -- (axis cs:7.30000E+01,-1.00000E-01) -- cycle ; 
\draw[Equator-empty] (axis cs:7.50000E+01,1.00000E-01) -- (axis cs:7.60000E+01,1.00000E-01) -- (axis cs:7.60000E+01,-1.00000E-01) -- (axis cs:7.50000E+01,-1.00000E-01) -- cycle ; 
\draw[Equator-empty] (axis cs:7.70000E+01,1.00000E-01) -- (axis cs:7.80000E+01,1.00000E-01) -- (axis cs:7.80000E+01,-1.00000E-01) -- (axis cs:7.70000E+01,-1.00000E-01) -- cycle ; 
\draw[Equator-empty] (axis cs:7.90000E+01,1.00000E-01) -- (axis cs:8.00000E+01,1.00000E-01) -- (axis cs:8.00000E+01,-1.00000E-01) -- (axis cs:7.90000E+01,-1.00000E-01) -- cycle ; 
\draw[Equator-empty] (axis cs:8.10000E+01,1.00000E-01) -- (axis cs:8.20000E+01,1.00000E-01) -- (axis cs:8.20000E+01,-1.00000E-01) -- (axis cs:8.10000E+01,-1.00000E-01) -- cycle ; 
\draw[Equator-empty] (axis cs:8.30000E+01,1.00000E-01) -- (axis cs:8.40000E+01,1.00000E-01) -- (axis cs:8.40000E+01,-1.00000E-01) -- (axis cs:8.30000E+01,-1.00000E-01) -- cycle ; 
\draw[Equator-empty] (axis cs:8.50000E+01,1.00000E-01) -- (axis cs:8.60000E+01,1.00000E-01) -- (axis cs:8.60000E+01,-1.00000E-01) -- (axis cs:8.50000E+01,-1.00000E-01) -- cycle ; 
\draw[Equator-empty] (axis cs:8.70000E+01,1.00000E-01) -- (axis cs:8.80000E+01,1.00000E-01) -- (axis cs:8.80000E+01,-1.00000E-01) -- (axis cs:8.70000E+01,-1.00000E-01) -- cycle ; 
\draw[Equator-empty] (axis cs:8.90000E+01,1.00000E-01) -- (axis cs:9.00000E+01,1.00000E-01) -- (axis cs:9.00000E+01,-1.00000E-01) -- (axis cs:8.90000E+01,-1.00000E-01) -- cycle ; 
\draw[Equator-empty] (axis cs:9.10000E+01,1.00000E-01) -- (axis cs:9.20000E+01,1.00000E-01) -- (axis cs:9.20000E+01,-1.00000E-01) -- (axis cs:9.10000E+01,-1.00000E-01) -- cycle ; 
\draw[Equator-empty] (axis cs:9.30000E+01,1.00000E-01) -- (axis cs:9.40000E+01,1.00000E-01) -- (axis cs:9.40000E+01,-1.00000E-01) -- (axis cs:9.30000E+01,-1.00000E-01) -- cycle ; 
\draw[Equator-empty] (axis cs:9.50000E+01,1.00000E-01) -- (axis cs:9.60000E+01,1.00000E-01) -- (axis cs:9.60000E+01,-1.00000E-01) -- (axis cs:9.50000E+01,-1.00000E-01) -- cycle ; 
\draw[Equator-empty] (axis cs:9.70000E+01,1.00000E-01) -- (axis cs:9.80000E+01,1.00000E-01) -- (axis cs:9.80000E+01,-1.00000E-01) -- (axis cs:9.70000E+01,-1.00000E-01) -- cycle ; 
\draw[Equator-empty] (axis cs:9.90000E+01,1.00000E-01) -- (axis cs:1.00000E+02,1.00000E-01) -- (axis cs:1.00000E+02,-1.00000E-01) -- (axis cs:9.90000E+01,-1.00000E-01) -- cycle ; 
\draw[Equator-empty] (axis cs:1.01000E+02,1.00000E-01) -- (axis cs:1.02000E+02,1.00000E-01) -- (axis cs:1.02000E+02,-1.00000E-01) -- (axis cs:1.01000E+02,-1.00000E-01) -- cycle ; 
\draw[Equator-empty] (axis cs:1.03000E+02,1.00000E-01) -- (axis cs:1.04000E+02,1.00000E-01) -- (axis cs:1.04000E+02,-1.00000E-01) -- (axis cs:1.03000E+02,-1.00000E-01) -- cycle ; 
\draw[Equator-empty] (axis cs:1.05000E+02,1.00000E-01) -- (axis cs:1.06000E+02,1.00000E-01) -- (axis cs:1.06000E+02,-1.00000E-01) -- (axis cs:1.05000E+02,-1.00000E-01) -- cycle ; 
\draw[Equator-empty] (axis cs:1.07000E+02,1.00000E-01) -- (axis cs:1.08000E+02,1.00000E-01) -- (axis cs:1.08000E+02,-1.00000E-01) -- (axis cs:1.07000E+02,-1.00000E-01) -- cycle ; 
\draw[Equator-empty] (axis cs:1.09000E+02,1.00000E-01) -- (axis cs:1.10000E+02,1.00000E-01) -- (axis cs:1.10000E+02,-1.00000E-01) -- (axis cs:1.09000E+02,-1.00000E-01) -- cycle ; 
\draw[Equator-empty] (axis cs:1.11000E+02,1.00000E-01) -- (axis cs:1.12000E+02,1.00000E-01) -- (axis cs:1.12000E+02,-1.00000E-01) -- (axis cs:1.11000E+02,-1.00000E-01) -- cycle ; 
\draw[Equator-empty] (axis cs:1.13000E+02,1.00000E-01) -- (axis cs:1.14000E+02,1.00000E-01) -- (axis cs:1.14000E+02,-1.00000E-01) -- (axis cs:1.13000E+02,-1.00000E-01) -- cycle ; 
\draw[Equator-empty] (axis cs:1.15000E+02,1.00000E-01) -- (axis cs:1.16000E+02,1.00000E-01) -- (axis cs:1.16000E+02,-1.00000E-01) -- (axis cs:1.15000E+02,-1.00000E-01) -- cycle ; 
\draw[Equator-empty] (axis cs:1.17000E+02,1.00000E-01) -- (axis cs:1.18000E+02,1.00000E-01) -- (axis cs:1.18000E+02,-1.00000E-01) -- (axis cs:1.17000E+02,-1.00000E-01) -- cycle ; 
\draw[Equator-empty] (axis cs:1.19000E+02,1.00000E-01) -- (axis cs:1.20000E+02,1.00000E-01) -- (axis cs:1.20000E+02,-1.00000E-01) -- (axis cs:1.19000E+02,-1.00000E-01) -- cycle ; 
\draw[Equator-empty] (axis cs:1.21000E+02,1.00000E-01) -- (axis cs:1.22000E+02,1.00000E-01) -- (axis cs:1.22000E+02,-1.00000E-01) -- (axis cs:1.21000E+02,-1.00000E-01) -- cycle ; 
\draw[Equator-empty] (axis cs:1.23000E+02,1.00000E-01) -- (axis cs:1.24000E+02,1.00000E-01) -- (axis cs:1.24000E+02,-1.00000E-01) -- (axis cs:1.23000E+02,-1.00000E-01) -- cycle ; 
\draw[Equator-empty] (axis cs:1.25000E+02,1.00000E-01) -- (axis cs:1.26000E+02,1.00000E-01) -- (axis cs:1.26000E+02,-1.00000E-01) -- (axis cs:1.25000E+02,-1.00000E-01) -- cycle ; 
\draw[Equator-empty] (axis cs:1.27000E+02,1.00000E-01) -- (axis cs:1.28000E+02,1.00000E-01) -- (axis cs:1.28000E+02,-1.00000E-01) -- (axis cs:1.27000E+02,-1.00000E-01) -- cycle ; 
\draw[Equator-empty] (axis cs:1.29000E+02,1.00000E-01) -- (axis cs:1.30000E+02,1.00000E-01) -- (axis cs:1.30000E+02,-1.00000E-01) -- (axis cs:1.29000E+02,-1.00000E-01) -- cycle ; 
\draw[Equator-empty] (axis cs:1.31000E+02,1.00000E-01) -- (axis cs:1.32000E+02,1.00000E-01) -- (axis cs:1.32000E+02,-1.00000E-01) -- (axis cs:1.31000E+02,-1.00000E-01) -- cycle ; 
\draw[Equator-empty] (axis cs:1.33000E+02,1.00000E-01) -- (axis cs:1.34000E+02,1.00000E-01) -- (axis cs:1.34000E+02,-1.00000E-01) -- (axis cs:1.33000E+02,-1.00000E-01) -- cycle ; 
\draw[Equator-empty] (axis cs:1.35000E+02,1.00000E-01) -- (axis cs:1.36000E+02,1.00000E-01) -- (axis cs:1.36000E+02,-1.00000E-01) -- (axis cs:1.35000E+02,-1.00000E-01) -- cycle ; 
\draw[Equator-empty] (axis cs:1.37000E+02,1.00000E-01) -- (axis cs:1.38000E+02,1.00000E-01) -- (axis cs:1.38000E+02,-1.00000E-01) -- (axis cs:1.37000E+02,-1.00000E-01) -- cycle ; 
\draw[Equator-empty] (axis cs:1.39000E+02,1.00000E-01) -- (axis cs:1.40000E+02,1.00000E-01) -- (axis cs:1.40000E+02,-1.00000E-01) -- (axis cs:1.39000E+02,-1.00000E-01) -- cycle ; 
\draw[Equator-empty] (axis cs:1.41000E+02,1.00000E-01) -- (axis cs:1.42000E+02,1.00000E-01) -- (axis cs:1.42000E+02,-1.00000E-01) -- (axis cs:1.41000E+02,-1.00000E-01) -- cycle ; 
\draw[Equator-empty] (axis cs:1.43000E+02,1.00000E-01) -- (axis cs:1.44000E+02,1.00000E-01) -- (axis cs:1.44000E+02,-1.00000E-01) -- (axis cs:1.43000E+02,-1.00000E-01) -- cycle ; 
\draw[Equator-empty] (axis cs:1.45000E+02,1.00000E-01) -- (axis cs:1.46000E+02,1.00000E-01) -- (axis cs:1.46000E+02,-1.00000E-01) -- (axis cs:1.45000E+02,-1.00000E-01) -- cycle ; 
\draw[Equator-empty] (axis cs:1.47000E+02,1.00000E-01) -- (axis cs:1.48000E+02,1.00000E-01) -- (axis cs:1.48000E+02,-1.00000E-01) -- (axis cs:1.47000E+02,-1.00000E-01) -- cycle ; 
\draw[Equator-empty] (axis cs:1.49000E+02,1.00000E-01) -- (axis cs:1.50000E+02,1.00000E-01) -- (axis cs:1.50000E+02,-1.00000E-01) -- (axis cs:1.49000E+02,-1.00000E-01) -- cycle ; 
\draw[Equator-empty] (axis cs:1.51000E+02,1.00000E-01) -- (axis cs:1.52000E+02,1.00000E-01) -- (axis cs:1.52000E+02,-1.00000E-01) -- (axis cs:1.51000E+02,-1.00000E-01) -- cycle ; 
\draw[Equator-empty] (axis cs:1.53000E+02,1.00000E-01) -- (axis cs:1.54000E+02,1.00000E-01) -- (axis cs:1.54000E+02,-1.00000E-01) -- (axis cs:1.53000E+02,-1.00000E-01) -- cycle ; 
\draw[Equator-empty] (axis cs:1.55000E+02,1.00000E-01) -- (axis cs:1.56000E+02,1.00000E-01) -- (axis cs:1.56000E+02,-1.00000E-01) -- (axis cs:1.55000E+02,-1.00000E-01) -- cycle ; 
\draw[Equator-empty] (axis cs:1.57000E+02,1.00000E-01) -- (axis cs:1.58000E+02,1.00000E-01) -- (axis cs:1.58000E+02,-1.00000E-01) -- (axis cs:1.57000E+02,-1.00000E-01) -- cycle ; 
\draw[Equator-empty] (axis cs:1.59000E+02,1.00000E-01) -- (axis cs:1.60000E+02,1.00000E-01) -- (axis cs:1.60000E+02,-1.00000E-01) -- (axis cs:1.59000E+02,-1.00000E-01) -- cycle ; 
\draw[Equator-empty] (axis cs:1.61000E+02,1.00000E-01) -- (axis cs:1.62000E+02,1.00000E-01) -- (axis cs:1.62000E+02,-1.00000E-01) -- (axis cs:1.61000E+02,-1.00000E-01) -- cycle ; 
\draw[Equator-empty] (axis cs:1.63000E+02,1.00000E-01) -- (axis cs:1.64000E+02,1.00000E-01) -- (axis cs:1.64000E+02,-1.00000E-01) -- (axis cs:1.63000E+02,-1.00000E-01) -- cycle ; 
\draw[Equator-empty] (axis cs:1.65000E+02,1.00000E-01) -- (axis cs:1.66000E+02,1.00000E-01) -- (axis cs:1.66000E+02,-1.00000E-01) -- (axis cs:1.65000E+02,-1.00000E-01) -- cycle ; 
\draw[Equator-empty] (axis cs:1.67000E+02,1.00000E-01) -- (axis cs:1.68000E+02,1.00000E-01) -- (axis cs:1.68000E+02,-1.00000E-01) -- (axis cs:1.67000E+02,-1.00000E-01) -- cycle ; 
\draw[Equator-empty] (axis cs:1.69000E+02,1.00000E-01) -- (axis cs:1.70000E+02,1.00000E-01) -- (axis cs:1.70000E+02,-1.00000E-01) -- (axis cs:1.69000E+02,-1.00000E-01) -- cycle ; 
\draw[Equator-empty] (axis cs:1.71000E+02,1.00000E-01) -- (axis cs:1.72000E+02,1.00000E-01) -- (axis cs:1.72000E+02,-1.00000E-01) -- (axis cs:1.71000E+02,-1.00000E-01) -- cycle ; 
\draw[Equator-empty] (axis cs:1.73000E+02,1.00000E-01) -- (axis cs:1.74000E+02,1.00000E-01) -- (axis cs:1.74000E+02,-1.00000E-01) -- (axis cs:1.73000E+02,-1.00000E-01) -- cycle ; 
\draw[Equator-empty] (axis cs:1.75000E+02,1.00000E-01) -- (axis cs:1.76000E+02,1.00000E-01) -- (axis cs:1.76000E+02,-1.00000E-01) -- (axis cs:1.75000E+02,-1.00000E-01) -- cycle ; 
\draw[Equator-empty] (axis cs:1.77000E+02,1.00000E-01) -- (axis cs:1.78000E+02,1.00000E-01) -- (axis cs:1.78000E+02,-1.00000E-01) -- (axis cs:1.77000E+02,-1.00000E-01) -- cycle ; 
\draw[Equator-empty] (axis cs:1.79000E+02,1.00000E-01) -- (axis cs:1.80000E+02,1.00000E-01) -- (axis cs:1.80000E+02,-1.00000E-01) -- (axis cs:1.79000E+02,-1.00000E-01) -- cycle ; 
\draw[Equator-empty] (axis cs:1.81000E+02,1.00000E-01) -- (axis cs:1.82000E+02,1.00000E-01) -- (axis cs:1.82000E+02,-1.00000E-01) -- (axis cs:1.81000E+02,-1.00000E-01) -- cycle ; 
\draw[Equator-empty] (axis cs:1.83000E+02,1.00000E-01) -- (axis cs:1.84000E+02,1.00000E-01) -- (axis cs:1.84000E+02,-1.00000E-01) -- (axis cs:1.83000E+02,-1.00000E-01) -- cycle ; 
\draw[Equator-empty] (axis cs:1.85000E+02,1.00000E-01) -- (axis cs:1.86000E+02,1.00000E-01) -- (axis cs:1.86000E+02,-1.00000E-01) -- (axis cs:1.85000E+02,-1.00000E-01) -- cycle ; 
\draw[Equator-empty] (axis cs:1.87000E+02,1.00000E-01) -- (axis cs:1.88000E+02,1.00000E-01) -- (axis cs:1.88000E+02,-1.00000E-01) -- (axis cs:1.87000E+02,-1.00000E-01) -- cycle ; 
\draw[Equator-empty] (axis cs:1.89000E+02,1.00000E-01) -- (axis cs:1.90000E+02,1.00000E-01) -- (axis cs:1.90000E+02,-1.00000E-01) -- (axis cs:1.89000E+02,-1.00000E-01) -- cycle ; 
\draw[Equator-empty] (axis cs:1.91000E+02,1.00000E-01) -- (axis cs:1.92000E+02,1.00000E-01) -- (axis cs:1.92000E+02,-1.00000E-01) -- (axis cs:1.91000E+02,-1.00000E-01) -- cycle ; 
\draw[Equator-empty] (axis cs:1.93000E+02,1.00000E-01) -- (axis cs:1.94000E+02,1.00000E-01) -- (axis cs:1.94000E+02,-1.00000E-01) -- (axis cs:1.93000E+02,-1.00000E-01) -- cycle ; 
\draw[Equator-empty] (axis cs:1.95000E+02,1.00000E-01) -- (axis cs:1.96000E+02,1.00000E-01) -- (axis cs:1.96000E+02,-1.00000E-01) -- (axis cs:1.95000E+02,-1.00000E-01) -- cycle ; 
\draw[Equator-empty] (axis cs:1.97000E+02,1.00000E-01) -- (axis cs:1.98000E+02,1.00000E-01) -- (axis cs:1.98000E+02,-1.00000E-01) -- (axis cs:1.97000E+02,-1.00000E-01) -- cycle ; 
\draw[Equator-empty] (axis cs:1.99000E+02,1.00000E-01) -- (axis cs:2.00000E+02,1.00000E-01) -- (axis cs:2.00000E+02,-1.00000E-01) -- (axis cs:1.99000E+02,-1.00000E-01) -- cycle ; 
\draw[Equator-empty] (axis cs:2.01000E+02,1.00000E-01) -- (axis cs:2.02000E+02,1.00000E-01) -- (axis cs:2.02000E+02,-1.00000E-01) -- (axis cs:2.01000E+02,-1.00000E-01) -- cycle ; 
\draw[Equator-empty] (axis cs:2.03000E+02,1.00000E-01) -- (axis cs:2.04000E+02,1.00000E-01) -- (axis cs:2.04000E+02,-1.00000E-01) -- (axis cs:2.03000E+02,-1.00000E-01) -- cycle ; 
\draw[Equator-empty] (axis cs:2.05000E+02,1.00000E-01) -- (axis cs:2.06000E+02,1.00000E-01) -- (axis cs:2.06000E+02,-1.00000E-01) -- (axis cs:2.05000E+02,-1.00000E-01) -- cycle ; 
\draw[Equator-empty] (axis cs:2.07000E+02,1.00000E-01) -- (axis cs:2.08000E+02,1.00000E-01) -- (axis cs:2.08000E+02,-1.00000E-01) -- (axis cs:2.07000E+02,-1.00000E-01) -- cycle ; 
\draw[Equator-empty] (axis cs:2.09000E+02,1.00000E-01) -- (axis cs:2.10000E+02,1.00000E-01) -- (axis cs:2.10000E+02,-1.00000E-01) -- (axis cs:2.09000E+02,-1.00000E-01) -- cycle ; 
\draw[Equator-empty] (axis cs:2.11000E+02,1.00000E-01) -- (axis cs:2.12000E+02,1.00000E-01) -- (axis cs:2.12000E+02,-1.00000E-01) -- (axis cs:2.11000E+02,-1.00000E-01) -- cycle ; 
\draw[Equator-empty] (axis cs:2.13000E+02,1.00000E-01) -- (axis cs:2.14000E+02,1.00000E-01) -- (axis cs:2.14000E+02,-1.00000E-01) -- (axis cs:2.13000E+02,-1.00000E-01) -- cycle ; 
\draw[Equator-empty] (axis cs:2.15000E+02,1.00000E-01) -- (axis cs:2.16000E+02,1.00000E-01) -- (axis cs:2.16000E+02,-1.00000E-01) -- (axis cs:2.15000E+02,-1.00000E-01) -- cycle ; 
\draw[Equator-empty] (axis cs:2.17000E+02,1.00000E-01) -- (axis cs:2.18000E+02,1.00000E-01) -- (axis cs:2.18000E+02,-1.00000E-01) -- (axis cs:2.17000E+02,-1.00000E-01) -- cycle ; 
\draw[Equator-empty] (axis cs:2.19000E+02,1.00000E-01) -- (axis cs:2.20000E+02,1.00000E-01) -- (axis cs:2.20000E+02,-1.00000E-01) -- (axis cs:2.19000E+02,-1.00000E-01) -- cycle ; 
\draw[Equator-empty] (axis cs:2.21000E+02,1.00000E-01) -- (axis cs:2.22000E+02,1.00000E-01) -- (axis cs:2.22000E+02,-1.00000E-01) -- (axis cs:2.21000E+02,-1.00000E-01) -- cycle ; 
\draw[Equator-empty] (axis cs:2.23000E+02,1.00000E-01) -- (axis cs:2.24000E+02,1.00000E-01) -- (axis cs:2.24000E+02,-1.00000E-01) -- (axis cs:2.23000E+02,-1.00000E-01) -- cycle ; 
\draw[Equator-empty] (axis cs:2.25000E+02,1.00000E-01) -- (axis cs:2.26000E+02,1.00000E-01) -- (axis cs:2.26000E+02,-1.00000E-01) -- (axis cs:2.25000E+02,-1.00000E-01) -- cycle ; 
\draw[Equator-empty] (axis cs:2.27000E+02,1.00000E-01) -- (axis cs:2.28000E+02,1.00000E-01) -- (axis cs:2.28000E+02,-1.00000E-01) -- (axis cs:2.27000E+02,-1.00000E-01) -- cycle ; 
\draw[Equator-empty] (axis cs:2.29000E+02,1.00000E-01) -- (axis cs:2.30000E+02,1.00000E-01) -- (axis cs:2.30000E+02,-1.00000E-01) -- (axis cs:2.29000E+02,-1.00000E-01) -- cycle ; 
\draw[Equator-empty] (axis cs:2.31000E+02,1.00000E-01) -- (axis cs:2.32000E+02,1.00000E-01) -- (axis cs:2.32000E+02,-1.00000E-01) -- (axis cs:2.31000E+02,-1.00000E-01) -- cycle ; 
\draw[Equator-empty] (axis cs:2.33000E+02,1.00000E-01) -- (axis cs:2.34000E+02,1.00000E-01) -- (axis cs:2.34000E+02,-1.00000E-01) -- (axis cs:2.33000E+02,-1.00000E-01) -- cycle ; 
\draw[Equator-empty] (axis cs:2.35000E+02,1.00000E-01) -- (axis cs:2.36000E+02,1.00000E-01) -- (axis cs:2.36000E+02,-1.00000E-01) -- (axis cs:2.35000E+02,-1.00000E-01) -- cycle ; 
\draw[Equator-empty] (axis cs:2.37000E+02,1.00000E-01) -- (axis cs:2.38000E+02,1.00000E-01) -- (axis cs:2.38000E+02,-1.00000E-01) -- (axis cs:2.37000E+02,-1.00000E-01) -- cycle ; 
\draw[Equator-empty] (axis cs:2.39000E+02,1.00000E-01) -- (axis cs:2.40000E+02,1.00000E-01) -- (axis cs:2.40000E+02,-1.00000E-01) -- (axis cs:2.39000E+02,-1.00000E-01) -- cycle ; 
\draw[Equator-empty] (axis cs:2.41000E+02,1.00000E-01) -- (axis cs:2.42000E+02,1.00000E-01) -- (axis cs:2.42000E+02,-1.00000E-01) -- (axis cs:2.41000E+02,-1.00000E-01) -- cycle ; 
\draw[Equator-empty] (axis cs:2.43000E+02,1.00000E-01) -- (axis cs:2.44000E+02,1.00000E-01) -- (axis cs:2.44000E+02,-1.00000E-01) -- (axis cs:2.43000E+02,-1.00000E-01) -- cycle ; 
\draw[Equator-empty] (axis cs:2.45000E+02,1.00000E-01) -- (axis cs:2.46000E+02,1.00000E-01) -- (axis cs:2.46000E+02,-1.00000E-01) -- (axis cs:2.45000E+02,-1.00000E-01) -- cycle ; 
\draw[Equator-empty] (axis cs:2.47000E+02,1.00000E-01) -- (axis cs:2.48000E+02,1.00000E-01) -- (axis cs:2.48000E+02,-1.00000E-01) -- (axis cs:2.47000E+02,-1.00000E-01) -- cycle ; 
\draw[Equator-empty] (axis cs:2.49000E+02,1.00000E-01) -- (axis cs:2.50000E+02,1.00000E-01) -- (axis cs:2.50000E+02,-1.00000E-01) -- (axis cs:2.49000E+02,-1.00000E-01) -- cycle ; 
\draw[Equator-empty] (axis cs:2.51000E+02,1.00000E-01) -- (axis cs:2.52000E+02,1.00000E-01) -- (axis cs:2.52000E+02,-1.00000E-01) -- (axis cs:2.51000E+02,-1.00000E-01) -- cycle ; 
\draw[Equator-empty] (axis cs:2.53000E+02,1.00000E-01) -- (axis cs:2.54000E+02,1.00000E-01) -- (axis cs:2.54000E+02,-1.00000E-01) -- (axis cs:2.53000E+02,-1.00000E-01) -- cycle ; 
\draw[Equator-empty] (axis cs:2.55000E+02,1.00000E-01) -- (axis cs:2.56000E+02,1.00000E-01) -- (axis cs:2.56000E+02,-1.00000E-01) -- (axis cs:2.55000E+02,-1.00000E-01) -- cycle ; 
\draw[Equator-empty] (axis cs:2.57000E+02,1.00000E-01) -- (axis cs:2.58000E+02,1.00000E-01) -- (axis cs:2.58000E+02,-1.00000E-01) -- (axis cs:2.57000E+02,-1.00000E-01) -- cycle ; 
\draw[Equator-empty] (axis cs:2.59000E+02,1.00000E-01) -- (axis cs:2.60000E+02,1.00000E-01) -- (axis cs:2.60000E+02,-1.00000E-01) -- (axis cs:2.59000E+02,-1.00000E-01) -- cycle ; 
\draw[Equator-empty] (axis cs:2.61000E+02,1.00000E-01) -- (axis cs:2.62000E+02,1.00000E-01) -- (axis cs:2.62000E+02,-1.00000E-01) -- (axis cs:2.61000E+02,-1.00000E-01) -- cycle ; 
\draw[Equator-empty] (axis cs:2.63000E+02,1.00000E-01) -- (axis cs:2.64000E+02,1.00000E-01) -- (axis cs:2.64000E+02,-1.00000E-01) -- (axis cs:2.63000E+02,-1.00000E-01) -- cycle ; 
\draw[Equator-empty] (axis cs:2.65000E+02,1.00000E-01) -- (axis cs:2.66000E+02,1.00000E-01) -- (axis cs:2.66000E+02,-1.00000E-01) -- (axis cs:2.65000E+02,-1.00000E-01) -- cycle ; 
\draw[Equator-empty] (axis cs:2.67000E+02,1.00000E-01) -- (axis cs:2.68000E+02,1.00000E-01) -- (axis cs:2.68000E+02,-1.00000E-01) -- (axis cs:2.67000E+02,-1.00000E-01) -- cycle ; 
\draw[Equator-empty] (axis cs:2.69000E+02,1.00000E-01) -- (axis cs:2.70000E+02,1.00000E-01) -- (axis cs:2.70000E+02,-1.00000E-01) -- (axis cs:2.69000E+02,-1.00000E-01) -- cycle ; 
\draw[Equator-empty] (axis cs:2.71000E+02,1.00000E-01) -- (axis cs:2.72000E+02,1.00000E-01) -- (axis cs:2.72000E+02,-1.00000E-01) -- (axis cs:2.71000E+02,-1.00000E-01) -- cycle ; 
\draw[Equator-empty] (axis cs:2.73000E+02,1.00000E-01) -- (axis cs:2.74000E+02,1.00000E-01) -- (axis cs:2.74000E+02,-1.00000E-01) -- (axis cs:2.73000E+02,-1.00000E-01) -- cycle ; 
\draw[Equator-empty] (axis cs:2.75000E+02,1.00000E-01) -- (axis cs:2.76000E+02,1.00000E-01) -- (axis cs:2.76000E+02,-1.00000E-01) -- (axis cs:2.75000E+02,-1.00000E-01) -- cycle ; 
\draw[Equator-empty] (axis cs:2.77000E+02,1.00000E-01) -- (axis cs:2.78000E+02,1.00000E-01) -- (axis cs:2.78000E+02,-1.00000E-01) -- (axis cs:2.77000E+02,-1.00000E-01) -- cycle ; 
\draw[Equator-empty] (axis cs:2.79000E+02,1.00000E-01) -- (axis cs:2.80000E+02,1.00000E-01) -- (axis cs:2.80000E+02,-1.00000E-01) -- (axis cs:2.79000E+02,-1.00000E-01) -- cycle ; 
\draw[Equator-empty] (axis cs:2.81000E+02,1.00000E-01) -- (axis cs:2.82000E+02,1.00000E-01) -- (axis cs:2.82000E+02,-1.00000E-01) -- (axis cs:2.81000E+02,-1.00000E-01) -- cycle ; 
\draw[Equator-empty] (axis cs:2.83000E+02,1.00000E-01) -- (axis cs:2.84000E+02,1.00000E-01) -- (axis cs:2.84000E+02,-1.00000E-01) -- (axis cs:2.83000E+02,-1.00000E-01) -- cycle ; 
\draw[Equator-empty] (axis cs:2.85000E+02,1.00000E-01) -- (axis cs:2.86000E+02,1.00000E-01) -- (axis cs:2.86000E+02,-1.00000E-01) -- (axis cs:2.85000E+02,-1.00000E-01) -- cycle ; 
\draw[Equator-empty] (axis cs:2.87000E+02,1.00000E-01) -- (axis cs:2.88000E+02,1.00000E-01) -- (axis cs:2.88000E+02,-1.00000E-01) -- (axis cs:2.87000E+02,-1.00000E-01) -- cycle ; 
\draw[Equator-empty] (axis cs:2.89000E+02,1.00000E-01) -- (axis cs:2.90000E+02,1.00000E-01) -- (axis cs:2.90000E+02,-1.00000E-01) -- (axis cs:2.89000E+02,-1.00000E-01) -- cycle ; 
\draw[Equator-empty] (axis cs:2.91000E+02,1.00000E-01) -- (axis cs:2.92000E+02,1.00000E-01) -- (axis cs:2.92000E+02,-1.00000E-01) -- (axis cs:2.91000E+02,-1.00000E-01) -- cycle ; 
\draw[Equator-empty] (axis cs:2.93000E+02,1.00000E-01) -- (axis cs:2.94000E+02,1.00000E-01) -- (axis cs:2.94000E+02,-1.00000E-01) -- (axis cs:2.93000E+02,-1.00000E-01) -- cycle ; 
\draw[Equator-empty] (axis cs:2.95000E+02,1.00000E-01) -- (axis cs:2.96000E+02,1.00000E-01) -- (axis cs:2.96000E+02,-1.00000E-01) -- (axis cs:2.95000E+02,-1.00000E-01) -- cycle ; 
\draw[Equator-empty] (axis cs:2.97000E+02,1.00000E-01) -- (axis cs:2.98000E+02,1.00000E-01) -- (axis cs:2.98000E+02,-1.00000E-01) -- (axis cs:2.97000E+02,-1.00000E-01) -- cycle ; 
\draw[Equator-empty] (axis cs:2.99000E+02,1.00000E-01) -- (axis cs:3.00000E+02,1.00000E-01) -- (axis cs:3.00000E+02,-1.00000E-01) -- (axis cs:2.99000E+02,-1.00000E-01) -- cycle ; 
\draw[Equator-empty] (axis cs:3.01000E+02,1.00000E-01) -- (axis cs:3.02000E+02,1.00000E-01) -- (axis cs:3.02000E+02,-1.00000E-01) -- (axis cs:3.01000E+02,-1.00000E-01) -- cycle ; 
\draw[Equator-empty] (axis cs:3.03000E+02,1.00000E-01) -- (axis cs:3.04000E+02,1.00000E-01) -- (axis cs:3.04000E+02,-1.00000E-01) -- (axis cs:3.03000E+02,-1.00000E-01) -- cycle ; 
\draw[Equator-empty] (axis cs:3.05000E+02,1.00000E-01) -- (axis cs:3.06000E+02,1.00000E-01) -- (axis cs:3.06000E+02,-1.00000E-01) -- (axis cs:3.05000E+02,-1.00000E-01) -- cycle ; 
\draw[Equator-empty] (axis cs:3.07000E+02,1.00000E-01) -- (axis cs:3.08000E+02,1.00000E-01) -- (axis cs:3.08000E+02,-1.00000E-01) -- (axis cs:3.07000E+02,-1.00000E-01) -- cycle ; 
\draw[Equator-empty] (axis cs:3.09000E+02,1.00000E-01) -- (axis cs:3.10000E+02,1.00000E-01) -- (axis cs:3.10000E+02,-1.00000E-01) -- (axis cs:3.09000E+02,-1.00000E-01) -- cycle ; 
\draw[Equator-empty] (axis cs:3.11000E+02,1.00000E-01) -- (axis cs:3.12000E+02,1.00000E-01) -- (axis cs:3.12000E+02,-1.00000E-01) -- (axis cs:3.11000E+02,-1.00000E-01) -- cycle ; 
\draw[Equator-empty] (axis cs:3.13000E+02,1.00000E-01) -- (axis cs:3.14000E+02,1.00000E-01) -- (axis cs:3.14000E+02,-1.00000E-01) -- (axis cs:3.13000E+02,-1.00000E-01) -- cycle ; 
\draw[Equator-empty] (axis cs:3.15000E+02,1.00000E-01) -- (axis cs:3.16000E+02,1.00000E-01) -- (axis cs:3.16000E+02,-1.00000E-01) -- (axis cs:3.15000E+02,-1.00000E-01) -- cycle ; 
\draw[Equator-empty] (axis cs:3.17000E+02,1.00000E-01) -- (axis cs:3.18000E+02,1.00000E-01) -- (axis cs:3.18000E+02,-1.00000E-01) -- (axis cs:3.17000E+02,-1.00000E-01) -- cycle ; 
\draw[Equator-empty] (axis cs:3.19000E+02,1.00000E-01) -- (axis cs:3.20000E+02,1.00000E-01) -- (axis cs:3.20000E+02,-1.00000E-01) -- (axis cs:3.19000E+02,-1.00000E-01) -- cycle ; 
\draw[Equator-empty] (axis cs:3.21000E+02,1.00000E-01) -- (axis cs:3.22000E+02,1.00000E-01) -- (axis cs:3.22000E+02,-1.00000E-01) -- (axis cs:3.21000E+02,-1.00000E-01) -- cycle ; 
\draw[Equator-empty] (axis cs:3.23000E+02,1.00000E-01) -- (axis cs:3.24000E+02,1.00000E-01) -- (axis cs:3.24000E+02,-1.00000E-01) -- (axis cs:3.23000E+02,-1.00000E-01) -- cycle ; 
\draw[Equator-empty] (axis cs:3.25000E+02,1.00000E-01) -- (axis cs:3.26000E+02,1.00000E-01) -- (axis cs:3.26000E+02,-1.00000E-01) -- (axis cs:3.25000E+02,-1.00000E-01) -- cycle ; 
\draw[Equator-empty] (axis cs:3.27000E+02,1.00000E-01) -- (axis cs:3.28000E+02,1.00000E-01) -- (axis cs:3.28000E+02,-1.00000E-01) -- (axis cs:3.27000E+02,-1.00000E-01) -- cycle ; 
\draw[Equator-empty] (axis cs:3.29000E+02,1.00000E-01) -- (axis cs:3.30000E+02,1.00000E-01) -- (axis cs:3.30000E+02,-1.00000E-01) -- (axis cs:3.29000E+02,-1.00000E-01) -- cycle ; 
\draw[Equator-empty] (axis cs:3.31000E+02,1.00000E-01) -- (axis cs:3.32000E+02,1.00000E-01) -- (axis cs:3.32000E+02,-1.00000E-01) -- (axis cs:3.31000E+02,-1.00000E-01) -- cycle ; 
\draw[Equator-empty] (axis cs:3.33000E+02,1.00000E-01) -- (axis cs:3.34000E+02,1.00000E-01) -- (axis cs:3.34000E+02,-1.00000E-01) -- (axis cs:3.33000E+02,-1.00000E-01) -- cycle ; 
\draw[Equator-empty] (axis cs:3.35000E+02,1.00000E-01) -- (axis cs:3.36000E+02,1.00000E-01) -- (axis cs:3.36000E+02,-1.00000E-01) -- (axis cs:3.35000E+02,-1.00000E-01) -- cycle ; 
\draw[Equator-empty] (axis cs:3.37000E+02,1.00000E-01) -- (axis cs:3.38000E+02,1.00000E-01) -- (axis cs:3.38000E+02,-1.00000E-01) -- (axis cs:3.37000E+02,-1.00000E-01) -- cycle ; 
\draw[Equator-empty] (axis cs:3.39000E+02,1.00000E-01) -- (axis cs:3.40000E+02,1.00000E-01) -- (axis cs:3.40000E+02,-1.00000E-01) -- (axis cs:3.39000E+02,-1.00000E-01) -- cycle ; 
\draw[Equator-empty] (axis cs:3.41000E+02,1.00000E-01) -- (axis cs:3.42000E+02,1.00000E-01) -- (axis cs:3.42000E+02,-1.00000E-01) -- (axis cs:3.41000E+02,-1.00000E-01) -- cycle ; 
\draw[Equator-empty] (axis cs:3.43000E+02,1.00000E-01) -- (axis cs:3.44000E+02,1.00000E-01) -- (axis cs:3.44000E+02,-1.00000E-01) -- (axis cs:3.43000E+02,-1.00000E-01) -- cycle ; 
\draw[Equator-empty] (axis cs:3.45000E+02,1.00000E-01) -- (axis cs:3.46000E+02,1.00000E-01) -- (axis cs:3.46000E+02,-1.00000E-01) -- (axis cs:3.45000E+02,-1.00000E-01) -- cycle ; 
\draw[Equator-empty] (axis cs:3.47000E+02,1.00000E-01) -- (axis cs:3.48000E+02,1.00000E-01) -- (axis cs:3.48000E+02,-1.00000E-01) -- (axis cs:3.47000E+02,-1.00000E-01) -- cycle ; 
\draw[Equator-empty] (axis cs:3.49000E+02,1.00000E-01) -- (axis cs:3.50000E+02,1.00000E-01) -- (axis cs:3.50000E+02,-1.00000E-01) -- (axis cs:3.49000E+02,-1.00000E-01) -- cycle ; 
\draw[Equator-empty] (axis cs:3.51000E+02,1.00000E-01) -- (axis cs:3.52000E+02,1.00000E-01) -- (axis cs:3.52000E+02,-1.00000E-01) -- (axis cs:3.51000E+02,-1.00000E-01) -- cycle ; 
\draw[Equator-empty] (axis cs:3.53000E+02,1.00000E-01) -- (axis cs:3.54000E+02,1.00000E-01) -- (axis cs:3.54000E+02,-1.00000E-01) -- (axis cs:3.53000E+02,-1.00000E-01) -- cycle ; 
\draw[Equator-empty] (axis cs:3.55000E+02,1.00000E-01) -- (axis cs:3.56000E+02,1.00000E-01) -- (axis cs:3.56000E+02,-1.00000E-01) -- (axis cs:3.55000E+02,-1.00000E-01) -- cycle ; 
\draw[Equator-empty] (axis cs:3.57000E+02,1.00000E-01) -- (axis cs:3.58000E+02,1.00000E-01) -- (axis cs:3.58000E+02,-1.00000E-01) -- (axis cs:3.57000E+02,-1.00000E-01) -- cycle ; 
\draw[Equator-empty] (axis cs:3.59000E+02,1.00000E-01) -- (axis cs:3.60000E+02,1.00000E-01) -- (axis cs:3.60000E+02,-1.00000E-01) -- (axis cs:3.59000E+02,-1.00000E-01) -- cycle ; 

\draw[Equator-full] (axis cs:3.60000E+02,1.00000E-01) -- (axis cs:3.61000E+02,1.00000E-01) -- (axis cs:3.61000E+02,-1.00000E-01) -- (axis cs:3.60000E+02,-1.00000E-01) -- cycle ; 
\draw[Equator-full] (axis cs:3.62000E+02,1.00000E-01) -- (axis cs:3.63000E+02,1.00000E-01) -- (axis cs:3.63000E+02,-1.00000E-01) -- (axis cs:3.62000E+02,-1.00000E-01) -- cycle ; 
\draw[Equator-full] (axis cs:3.64000E+02,1.00000E-01) -- (axis cs:3.65000E+02,1.00000E-01) -- (axis cs:3.65000E+02,-1.00000E-01) -- (axis cs:3.64000E+02,-1.00000E-01) -- cycle ; 
\draw[Equator-full] (axis cs:3.66000E+02,1.00000E-01) -- (axis cs:3.67000E+02,1.00000E-01) -- (axis cs:3.67000E+02,-1.00000E-01) -- (axis cs:3.66000E+02,-1.00000E-01) -- cycle ; 
\draw[Equator-full] (axis cs:3.68000E+02,1.00000E-01) -- (axis cs:3.69000E+02,1.00000E-01) -- (axis cs:3.69000E+02,-1.00000E-01) -- (axis cs:3.68000E+02,-1.00000E-01) -- cycle ; 
\draw[Equator-full] (axis cs:3.70000E+02,1.00000E-01) -- (axis cs:3.71000E+02,1.00000E-01) -- (axis cs:3.71000E+02,-1.00000E-01) -- (axis cs:3.70000E+02,-1.00000E-01) -- cycle ; 
\draw[Equator-full] (axis cs:3.72000E+02,1.00000E-01) -- (axis cs:3.73000E+02,1.00000E-01) -- (axis cs:3.73000E+02,-1.00000E-01) -- (axis cs:3.72000E+02,-1.00000E-01) -- cycle ; 
\draw[Equator-full] (axis cs:3.74000E+02,1.00000E-01) -- (axis cs:3.75000E+02,1.00000E-01) -- (axis cs:3.75000E+02,-1.00000E-01) -- (axis cs:3.74000E+02,-1.00000E-01) -- cycle ; 
\draw[Equator-full] (axis cs:3.76000E+02,1.00000E-01) -- (axis cs:3.77000E+02,1.00000E-01) -- (axis cs:3.77000E+02,-1.00000E-01) -- (axis cs:3.76000E+02,-1.00000E-01) -- cycle ; 
\draw[Equator-full] (axis cs:3.78000E+02,1.00000E-01) -- (axis cs:3.79000E+02,1.00000E-01) -- (axis cs:3.79000E+02,-1.00000E-01) -- (axis cs:3.78000E+02,-1.00000E-01) -- cycle ; 

\draw[Equator-empty] (axis cs:3.61000E+02,1.00000E-01) -- (axis cs:3.62000E+02,1.00000E-01) -- (axis cs:3.62000E+02,-1.00000E-01) -- (axis cs:3.61000E+02,-1.00000E-01) -- cycle ; 
\draw[Equator-empty] (axis cs:3.63000E+02,1.00000E-01) -- (axis cs:3.64000E+02,1.00000E-01) -- (axis cs:3.64000E+02,-1.00000E-01) -- (axis cs:3.63000E+02,-1.00000E-01) -- cycle ; 
\draw[Equator-empty] (axis cs:3.65000E+02,1.00000E-01) -- (axis cs:3.66000E+02,1.00000E-01) -- (axis cs:3.66000E+02,-1.00000E-01) -- (axis cs:3.65000E+02,-1.00000E-01) -- cycle ; 
\draw[Equator-empty] (axis cs:3.67000E+02,1.00000E-01) -- (axis cs:3.68000E+02,1.00000E-01) -- (axis cs:3.68000E+02,-1.00000E-01) -- (axis cs:3.67000E+02,-1.00000E-01) -- cycle ; 
\draw[Equator-empty] (axis cs:3.69000E+02,1.00000E-01) -- (axis cs:3.70000E+02,1.00000E-01) -- (axis cs:3.70000E+02,-1.00000E-01) -- (axis cs:3.69000E+02,-1.00000E-01) -- cycle ; 
\draw[Equator-empty] (axis cs:3.71000E+02,1.00000E-01) -- (axis cs:3.72000E+02,1.00000E-01) -- (axis cs:3.72000E+02,-1.00000E-01) -- (axis cs:3.71000E+02,-1.00000E-01) -- cycle ; 
\draw[Equator-empty] (axis cs:3.73000E+02,1.00000E-01) -- (axis cs:3.74000E+02,1.00000E-01) -- (axis cs:3.74000E+02,-1.00000E-01) -- (axis cs:3.73000E+02,-1.00000E-01) -- cycle ; 
\draw[Equator-empty] (axis cs:3.75000E+02,1.00000E-01) -- (axis cs:3.76000E+02,1.00000E-01) -- (axis cs:3.76000E+02,-1.00000E-01) -- (axis cs:3.75000E+02,-1.00000E-01) -- cycle ; 
\draw[Equator-empty] (axis cs:3.77000E+02,1.00000E-01) -- (axis cs:3.78000E+02,1.00000E-01) -- (axis cs:3.78000E+02,-1.00000E-01) -- (axis cs:3.77000E+02,-1.00000E-01) -- cycle ; 
\draw[Equator-empty] (axis cs:3.79000E+02,1.00000E-01) -- (axis cs:3.80000E+02,1.00000E-01) -- (axis cs:3.80000E+02,-1.00000E-01) -- (axis cs:3.79000E+02,-1.00000E-01) -- cycle ; 


\draw [Equator-empty] (axis cs:0.00000E+00,4.00000E-01) -- (axis cs:0.00000E+00,-4.00000E-01)   ;
\node[pin={[pin distance=-0.4\onedegree,Equator-label]90:{0$^\circ$}}] at (axis cs:0.00000E+00,4.00000E-01) {} ;
\draw [Equator-empty] (axis cs:1.00000E+01,4.00000E-01) -- (axis cs:1.00000E+01,-4.00000E-01)   ;
\node[pin={[pin distance=-0.4\onedegree,Equator-label]90:{10$^\circ$}}] at (axis cs:1.00000E+01,4.00000E-01) {} ;
\draw [Equator-empty] (axis cs:2.00000E+01,4.00000E-01) -- (axis cs:2.00000E+01,-4.00000E-01)   ;
\node[pin={[pin distance=-0.4\onedegree,Equator-label]90:{20$^\circ$}}] at (axis cs:2.00000E+01,4.00000E-01) {} ;
\draw [Equator-empty] (axis cs:3.00000E+01,4.00000E-01) -- (axis cs:3.00000E+01,-4.00000E-01)   ;
\node[pin={[pin distance=-0.4\onedegree,Equator-label]90:{30$^\circ$}}] at (axis cs:3.00000E+01,4.00000E-01) {} ;
\draw [Equator-empty] (axis cs:4.00000E+01,4.00000E-01) -- (axis cs:4.00000E+01,-4.00000E-01)   ;
\node[pin={[pin distance=-0.4\onedegree,Equator-label]90:{40$^\circ$}}] at (axis cs:4.00000E+01,4.00000E-01) {} ;
\draw [Equator-empty] (axis cs:5.00000E+01,4.00000E-01) -- (axis cs:5.00000E+01,-4.00000E-01)   ;
\node[pin={[pin distance=-0.4\onedegree,Equator-label]90:{50$^\circ$}}] at (axis cs:5.00000E+01,4.00000E-01) {} ;
\draw [Equator-empty] (axis cs:6.00000E+01,4.00000E-01) -- (axis cs:6.00000E+01,-4.00000E-01)   ;
\node[pin={[pin distance=-0.4\onedegree,Equator-label]90:{60$^\circ$}}] at (axis cs:6.00000E+01,4.00000E-01) {} ;
\draw [Equator-empty] (axis cs:7.00000E+01,4.00000E-01) -- (axis cs:7.00000E+01,-4.00000E-01)   ;
\node[pin={[pin distance=-0.4\onedegree,Equator-label]90:{70$^\circ$}}] at (axis cs:7.00000E+01,4.00000E-01) {} ;
\draw [Equator-empty] (axis cs:8.00000E+01,4.00000E-01) -- (axis cs:8.00000E+01,-4.00000E-01)   ;
\node[pin={[pin distance=-0.4\onedegree,Equator-label]90:{80$^\circ$}}] at (axis cs:8.00000E+01,4.00000E-01) {} ;
\draw [Equator-empty] (axis cs:9.00000E+01,4.00000E-01) -- (axis cs:9.00000E+01,-4.00000E-01)   ;
\node[pin={[pin distance=-0.4\onedegree,Equator-label]90:{90$^\circ$}}] at (axis cs:9.00000E+01,4.00000E-01) {} ;
\draw [Equator-empty] (axis cs:1.00000E+02,4.00000E-01) -- (axis cs:1.00000E+02,-4.00000E-01)   ;
\node[pin={[pin distance=-0.4\onedegree,Equator-label]090:{100$^\circ$}}] at (axis cs:1.00000E+02,4.00000E-01) {} ;
\draw [Equator-empty] (axis cs:1.10000E+02,4.00000E-01) -- (axis cs:1.10000E+02,-4.00000E-01)   ;
\node[pin={[pin distance=-0.4\onedegree,Equator-label]090:{110$^\circ$}}] at (axis cs:1.10000E+02,4.00000E-01) {} ;
\draw [Equator-empty] (axis cs:1.20000E+02,4.00000E-01) -- (axis cs:1.20000E+02,-4.00000E-01)   ;
\node[pin={[pin distance=-0.4\onedegree,Equator-label]090:{120$^\circ$}}] at (axis cs:1.20000E+02,4.00000E-01) {} ;
\draw [Equator-empty] (axis cs:1.30000E+02,4.00000E-01) -- (axis cs:1.30000E+02,-4.00000E-01)   ;
\node[pin={[pin distance=-0.4\onedegree,Equator-label]090:{130$^\circ$}}] at (axis cs:1.30000E+02,4.00000E-01) {} ;
\draw [Equator-empty] (axis cs:1.40000E+02,4.00000E-01) -- (axis cs:1.40000E+02,-4.00000E-01)   ;
\node[pin={[pin distance=-0.4\onedegree,Equator-label]090:{140$^\circ$}}] at (axis cs:1.40000E+02,4.00000E-01) {} ;
\draw [Equator-empty] (axis cs:1.50000E+02,4.00000E-01) -- (axis cs:1.50000E+02,-4.00000E-01)   ;
\node[pin={[pin distance=-0.4\onedegree,Equator-label]090:{150$^\circ$}}] at (axis cs:1.50000E+02,4.00000E-01) {} ;
\draw [Equator-empty] (axis cs:1.60000E+02,4.00000E-01) -- (axis cs:1.60000E+02,-4.00000E-01)   ;
\node[pin={[pin distance=-0.4\onedegree,Equator-label]090:{160$^\circ$}}] at (axis cs:1.60000E+02,4.00000E-01) {} ;
\draw [Equator-empty] (axis cs:1.70000E+02,4.00000E-01) -- (axis cs:1.70000E+02,-4.00000E-01)   ;
\node[pin={[pin distance=-0.4\onedegree,Equator-label]090:{170$^\circ$}}] at (axis cs:1.70000E+02,4.00000E-01) {} ;
\draw [Equator-empty] (axis cs:1.80000E+02,4.00000E-01) -- (axis cs:1.80000E+02,-4.00000E-01)   ;
\node[pin={[pin distance=-0.4\onedegree,Equator-label]090:{180$^\circ$}}] at (axis cs:1.80000E+02,4.00000E-01) {} ;
\draw [Equator-empty] (axis cs:1.90000E+02,4.00000E-01) -- (axis cs:1.90000E+02,-4.00000E-01)   ;
\node[pin={[pin distance=-0.4\onedegree,Equator-label]090:{190$^\circ$}}] at (axis cs:1.90000E+02,4.00000E-01) {} ;
\draw [Equator-empty] (axis cs:2.00000E+02,4.00000E-01) -- (axis cs:2.00000E+02,-4.00000E-01)   ;
\node[pin={[pin distance=-0.4\onedegree,Equator-label]090:{200$^\circ$}}] at (axis cs:2.00000E+02,4.00000E-01) {} ;
\draw [Equator-empty] (axis cs:2.10000E+02,4.00000E-01) -- (axis cs:2.10000E+02,-4.00000E-01)   ;
\node[pin={[pin distance=-0.4\onedegree,Equator-label]090:{210$^\circ$}}] at (axis cs:2.10000E+02,4.00000E-01) {} ;
\draw [Equator-empty] (axis cs:2.20000E+02,4.00000E-01) -- (axis cs:2.20000E+02,-4.00000E-01)   ;
\node[pin={[pin distance=-0.4\onedegree,Equator-label]090:{220$^\circ$}}] at (axis cs:2.20000E+02,4.00000E-01) {} ;
\draw [Equator-empty] (axis cs:2.30000E+02,4.00000E-01) -- (axis cs:2.30000E+02,-4.00000E-01)   ;
\node[pin={[pin distance=-0.4\onedegree,Equator-label]090:{230$^\circ$}}] at (axis cs:2.30000E+02,4.00000E-01) {} ;
\draw [Equator-empty] (axis cs:2.40000E+02,4.00000E-01) -- (axis cs:2.40000E+02,-4.00000E-01)   ;
\node[pin={[pin distance=-0.4\onedegree,Equator-label]090:{240$^\circ$}}] at (axis cs:2.40000E+02,4.00000E-01) {} ;
\draw [Equator-empty] (axis cs:2.50000E+02,4.00000E-01) -- (axis cs:2.50000E+02,-4.00000E-01)   ;
\node[pin={[pin distance=-0.4\onedegree,Equator-label]090:{250$^\circ$}}] at (axis cs:2.50000E+02,4.00000E-01) {} ;
\draw [Equator-empty] (axis cs:2.60000E+02,4.00000E-01) -- (axis cs:2.60000E+02,-4.00000E-01)   ;
\node[pin={[pin distance=-0.4\onedegree,Equator-label]090:{260$^\circ$}}] at (axis cs:2.60000E+02,4.00000E-01) {} ;
\draw [Equator-empty] (axis cs:2.70000E+02,4.00000E-01) -- (axis cs:2.70000E+02,-4.00000E-01)   ;
\node[pin={[pin distance=-0.4\onedegree,Equator-label]090:{270$^\circ$}}] at (axis cs:2.70000E+02,4.00000E-01) {} ;
\draw [Equator-empty] (axis cs:2.80000E+02,4.00000E-01) -- (axis cs:2.80000E+02,-4.00000E-01)   ;
\node[pin={[pin distance=-0.4\onedegree,Equator-label]090:{280$^\circ$}}] at (axis cs:2.80000E+02,4.00000E-01) {} ;
\draw [Equator-empty] (axis cs:2.90000E+02,4.00000E-01) -- (axis cs:2.90000E+02,-4.00000E-01)   ;
\node[pin={[pin distance=-0.4\onedegree,Equator-label]090:{290$^\circ$}}] at (axis cs:2.90000E+02,4.00000E-01) {} ;
\draw [Equator-empty] (axis cs:3.00000E+02,4.00000E-01) -- (axis cs:3.00000E+02,-4.00000E-01)   ;
\node[pin={[pin distance=-0.4\onedegree,Equator-label]090:{300$^\circ$}}] at (axis cs:3.00000E+02,4.00000E-01) {} ;
\draw [Equator-empty] (axis cs:3.10000E+02,4.00000E-01) -- (axis cs:3.10000E+02,-4.00000E-01)   ;
\node[pin={[pin distance=-0.4\onedegree,Equator-label]090:{310$^\circ$}}] at (axis cs:3.10000E+02,4.00000E-01) {} ;
\draw [Equator-empty] (axis cs:3.20000E+02,4.00000E-01) -- (axis cs:3.20000E+02,-4.00000E-01)   ;
\node[pin={[pin distance=-0.4\onedegree,Equator-label]090:{320$^\circ$}}] at (axis cs:3.20000E+02,4.00000E-01) {} ;
\draw [Equator-empty] (axis cs:3.30000E+02,4.00000E-01) -- (axis cs:3.30000E+02,-4.00000E-01)   ;
\node[pin={[pin distance=-0.4\onedegree,Equator-label]090:{330$^\circ$}}] at (axis cs:3.30000E+02,4.00000E-01) {} ;
\draw [Equator-empty] (axis cs:3.40000E+02,4.00000E-01) -- (axis cs:3.40000E+02,-4.00000E-01)   ;
\node[pin={[pin distance=-0.4\onedegree,Equator-label]090:{340$^\circ$}}] at (axis cs:3.40000E+02,4.00000E-01) {} ;
\draw [Equator-empty] (axis cs:3.50000E+02,4.00000E-01) -- (axis cs:3.50000E+02,-4.00000E-01)   ;
\node[pin={[pin distance=-0.4\onedegree,Equator-label]090:{350$^\circ$}}] at (axis cs:3.50000E+02,4.00000E-01) {} ;

\draw [Equator-empty] (axis cs:3.60000E+02,4.00000E-01) -- (axis cs:3.60000E+02,-4.00000E-01)   ;
\node[pin={[pin distance=-0.4\onedegree,Equator-label]090:{0$^\circ$}}] at (axis cs:3.60000E+02,4.00000E-01) {} ;
\draw [Equator-empty] (axis cs:3.70000E+02,4.00000E-01) -- (axis cs:3.70000E+02,-4.00000E-01)   ;
\node[pin={[pin distance=-0.4\onedegree,Equator-label]090:{10$^\circ$}}] at (axis cs:3.70000E+02,4.00000E-01) {} ;




\end{axis}


% Coordinate axes 


% Left and bottom axis with gridlines
  \begin{axis}[at=(base.center),anchor=center,RA_in_hours,color=cAxes] \end{axis}
% Right and top axis, labelling in RA hours (offset)
  \begin{axis}[at=(base.center),anchor=center,yticklabel pos=right,xticklabel pos=right,RA_in_hours
  ,xticklabel style={yshift=2.5ex, anchor=south},color=cAxes]\end{axis}
% Top axis labelling in RA degree (small offset)
  \begin{axis}[at=(base.center),anchor=center,axis y line=none,xticklabel pos=top,RA_in_deg
     ,xticklabel style={font=\footnotesize},xticklabel style={yshift=0.5ex, anchor=south} ,color=cAxes]
  \end{axis}

% Coordinate grid

\begin{axis}[at=(base.center),anchor=center,coordinategrid,RA_in_deg,
    separate axis lines,
    y axis line style= { draw opacity=0 },
    x axis line style= { draw opacity=0 },
    ,xticklabels={},yticklabels={}
] \end{axis}


% Ornamental frame


  \begin{axis}[name=frame,at={($(base.center)+(0cm,+.7\onedegree)$)},anchor=center,width=11.75\tendegree,height=8.53\tendegree,ticks=none
  ,yticklabels={},xticklabels={},axis line style={line width=1pt},color=cFrame] \end{axis}

  \begin{axis}[name=frame2,at={($(frame.center)+(0cm,+0cm)$)},anchor=center,width=11.85\tendegree,height=8.63\tendegree,ticks=none
  ,yticklabels={},xticklabels={},axis line style={line width=3pt},color=cFrame] \end{axis}
  

% Symbol legend


 
 \begin{axis}[name=legend,at=(frame2.south),anchor=north,axis y line=none,axis x
   line=none,xmin=-55,xmax=55,height=0.5\tendegree   ,ymin=-1.5,ymax=1.5,  ,y dir=normal,x dir=reverse, color=cAxes]

\node at (axis cs:54,+1.0) {\small\it 1};                  \node at (axis cs:54,0.3)  {\tikz{\pgfuseplotmark{m1b};}};           
\node at (axis cs:52,+1.0) {\small\it 1{\nicefrac{1}{3}}}; \node at (axis cs:52,0.3)  {\tikz{\pgfuseplotmark{m1c};}};           
\node at (axis cs:53,-0.5)  {\small First};                  

\node at (axis cs:48,+1.0) {\small\it 1{\nicefrac{2}{3}}}; \node at (axis cs:48,0.3)  {\tikz{\pgfuseplotmark{m2a};}};           
\node at (axis cs:46,+1.0) {\small\it 2};                  \node at (axis cs:46,0.3)  {\tikz{\pgfuseplotmark{m2b};}};           
\node at (axis cs:44,+1.0) {\small\it 2{\nicefrac{1}{3}}}; \node at (axis cs:44,0.3)  {\tikz{\pgfuseplotmark{m2c};}};           
\node at (axis cs:46,-0.5)  {\small Second};

\node at (axis cs:40,+1.0) {\small\it 2{\nicefrac{2}{3}}}; \node at (axis cs:40,0.3)  {\tikz{\pgfuseplotmark{m3a};}};           
\node at (axis cs:38,+1.0) {\small\it 3};                  \node at (axis cs:38,0.3)  {\tikz{\pgfuseplotmark{m3b};}};           
\node at (axis cs:36,+1.0) {\small\it 3{\nicefrac{1}{3}}}; \node at (axis cs:36,0.3)  {\tikz{\pgfuseplotmark{m3c};}};           
\node at (axis cs:38,-0.5)  {\small Third};

\node at (axis cs:32,+1.0) {\small\it 3{\nicefrac{2}{3}}}; \node at (axis cs:32,0.3)  {\tikz{\pgfuseplotmark{m4a};}};           
\node at (axis cs:30,+1.0) {\small\it 4};                  \node at (axis cs:30,0.3)  {\tikz{\pgfuseplotmark{m4b};}};           
\node at (axis cs:28,+1.0) {\small\it 4{\nicefrac{1}{3}}}; \node at (axis cs:28,0.3)  {\tikz{\pgfuseplotmark{m4c};}};           
\node at (axis cs:30,-0.5)  {\small Fourth};                                                                      
                                                                                                                  
\node at (axis cs:24,+1.0) {\small\it 4{\nicefrac{2}{3}}}; \node at (axis cs:24,0.3)  {\tikz{\pgfuseplotmark{m5a};}};           
\node at (axis cs:22,+1.0) {\small\it 5};                  \node at (axis cs:22,0.3)  {\tikz{\pgfuseplotmark{m5b};}};           
\node at (axis cs:20,+1.0) {\small\it 5{\nicefrac{1}{3}}}; \node at (axis cs:20,0.3)  {\tikz{\pgfuseplotmark{m5c};}};           
\node at (axis cs:22,-0.5)  {\small Fifth};

\node at (axis cs:16,+1.0) {\small\it 5{\nicefrac{2}{3}}}; \node at (axis cs:16,0.3)  {\tikz{\pgfuseplotmark{m6a};}};           
\node at (axis cs:14,+1.0) {\small\it 6};                  \node at (axis cs:14,0.3)  {\tikz{\pgfuseplotmark{m6b};}};           
\node at (axis cs:12,+1.0) {\small\it 6{\nicefrac{1}{3}}}; \node at (axis cs:12,0.3)  {\tikz{\pgfuseplotmark{m6c};}};           
\node at (axis cs:14,-0.5)  {\small Sixth};                                                                       
                                                                                                                  
                                                                                                                 
\node at  (axis cs:6,+1.0)   {\small\it Variable};  \node at  (axis cs:6,+0.3)       {\tikz{\pgfuseplotmark{m2av};}};           
\node at  (axis cs:2,+.992)    {\small\it Binary};   \node at  (axis cs:2,+0.3)      {\tikz{\pgfuseplotmark{m2ab};}};           
\node at  (axis cs:-2,+1.0)   {\small\it Var. \& Bin.};  \node at  (axis cs:-2,+0.3) {\tikz{\pgfuseplotmark{m2avb};}};           


\node at  (axis cs:-17,+.992)   {\small\it Planetary};  \node at  (axis cs:-18,+0.3)  {\tikz{\pgfuseplotmark{PN};}};           
\node at  (axis cs:-22,+1.0)   {\small\it Emission};   \node at  (axis cs:-22,+0.3)   {\tikz{\pgfuseplotmark{EN};}};           
\node at  (axis cs:-27,+1.0)   {\small\it w/ Cluster};  \node at  (axis cs:-26,+0.3)  {\tikz{\pgfuseplotmark{CN};}};           
\node at (axis cs:-22,-0.5)  {\small Nebulae};


\node at  (axis cs:-32,+.994)   {\small\it Open};      \node at  (axis cs:-32,+0.3)   {\tikz{\pgfuseplotmark{OC};}};           
\node at  (axis cs:-37,+1.0)   {\small\it Globular};   \node at  (axis cs:-37,+0.3)   {\tikz{\pgfuseplotmark{GC};}};           
\node at (axis cs:-34.5,-0.5)  {\small Cluster};

\node at  (axis cs:-42,+.994)   {\small\it Galaxy};      \node at  (axis cs:-42,+0.3) {\tikz{\pgfuseplotmark{GAL};}};         

%
\end{axis}
         


% Constellations
%
% Some are commented out because they are surrounded by already plotted borders
% The x-coordinate is in fractional hours, so must be times 15
%

\begin{axis}[name=constellations,axis lines=none]

\draw[constellation-boundary]   ++ (0,0);  \draw[constellation-boundary]  (axis cs:180.599,    -35.696092) --  (axis cs:181.608,    -35.696028) --  (axis cs:182.617,    -35.695749) --  (axis cs:183.625,    -35.695258) --  (axis cs:184.634,    -35.694552) --  (axis cs:185.39,    -35.693883) ;
  \draw[constellation-boundary]  (axis cs:280.326,    +02.115345) --  (axis cs:280.576,    +02.118339) --  (axis cs:281.576,    +02.130293) --  (axis cs:282.576,    +02.142208) --  (axis cs:283.576,    +02.154078) --  (axis cs:284.576,    +02.165903) ;
  \draw[constellation-boundary]  (axis cs:284.576,    +02.165903) --  (axis cs:284.565,    +03.165833) --  (axis cs:284.553,    +04.165763) --  (axis cs:284.541,    +05.165693) --  (axis cs:284.529,    +06.165623) --  (axis cs:284.526,    +06.415605) ;
  \draw[constellation-boundary]  (axis cs:281.459,    +06.379193) --  (axis cs:281.45,    +07.129139) --  (axis cs:281.437,    +08.129066) --  (axis cs:281.425,    +09.128993) --  (axis cs:281.413,    +10.128920) --  (axis cs:281.4,    +11.128846) --  (axis cs:281.388,    +12.128772) ;
  \draw[constellation-boundary]  (axis cs:284.456,    +12.165194) --  (axis cs:284.444,    +13.165121) --  (axis cs:284.431,    +14.165047) --  (axis cs:284.419,    +15.164972) --  (axis cs:284.406,    +16.164897) --  (axis cs:284.393,    +17.164822) --  (axis cs:284.38,    +18.164745) --  (axis cs:284.367,    +19.164667) --  (axis cs:284.354,    +20.164589) --  (axis cs:284.34,    +21.164509) --  (axis cs:284.327,    +22.164429) --  (axis cs:284.313,    +23.164347) --  (axis cs:284.299,    +24.164264) --  (axis cs:284.284,    +25.164180) --  (axis cs:284.27,    +26.164094) ;
  \draw[constellation-boundary]  (axis cs:284.374,    +18.664706) --  (axis cs:285.374,    +18.676490) --  (axis cs:286.375,    +18.688221) ;
  \draw[constellation-boundary]  (axis cs:286.405,    +16.355063) --  (axis cs:286.395,    +17.188334) --  (axis cs:286.382,    +18.188259) --  (axis cs:286.375,    +18.688221) ;
  \draw[constellation-boundary]  (axis cs:286.405,    +16.355063) --  (axis cs:287.406,    +16.366735) --  (axis cs:288.407,    +16.378346) --  (axis cs:289.408,    +16.389893) --  (axis cs:290.409,    +16.401373) --  (axis cs:291.411,    +16.412781) --  (axis cs:292.412,    +16.424114) --  (axis cs:293.413,    +16.435369) --  (axis cs:294.414,    +16.446543) ;
  \draw[constellation-boundary]  (axis cs:284.744,    -11.866442) --  (axis cs:285.743,    -11.854677) --  (axis cs:286.743,    -11.842967) --  (axis cs:287.742,    -11.831315) --  (axis cs:288.741,    -11.819725) --  (axis cs:289.74,    -11.808200) --  (axis cs:290.739,    -11.796744) --  (axis cs:291.738,    -11.785360) --  (axis cs:292.737,    -11.774052) --  (axis cs:293.736,    -11.762822) --  (axis cs:294.735,    -11.751676) ;
  \draw[constellation-boundary]  (axis cs:280.398,    -03.884224) --  (axis cs:280.386,    -02.884296) --  (axis cs:280.374,    -01.884368) --  (axis cs:280.362,    -00.884439) --  (axis cs:280.35,    +00.115489) --  (axis cs:280.338,    +01.115417) --  (axis cs:280.326,    +02.115345) ;
  \draw[constellation-boundary]  (axis cs:269.773,    -44.016565) --  (axis cs:269.75,    -43.016706) --  (axis cs:269.728,    -42.016841) --  (axis cs:269.706,    -41.016973) --  (axis cs:269.685,    -40.017101) --  (axis cs:269.665,    -39.017224) --  (axis cs:269.645,    -38.017345) --  (axis cs:269.625,    -37.017462) --  (axis cs:269.607,    -36.017576) --  (axis cs:269.588,    -35.017688) --  (axis cs:269.57,    -34.017796) --  (axis cs:269.553,    -33.017902) --  (axis cs:269.536,    -32.018006) --  (axis cs:269.519,    -31.018107) --  (axis cs:269.503,    -30.018207) ;
  \draw[constellation-boundary]  (axis cs:248.537,    -44.267237) --  (axis cs:248.516,    -43.267357) --  (axis cs:248.495,    -42.267474) ;
  \draw[constellation-boundary]  (axis cs:204.064,    +07.360584) --  (axis cs:204.563,    +07.363008) --  (axis cs:205.562,    +07.368000) --  (axis cs:206.56,    +07.373184) --  (axis cs:207.559,    +07.378559) --  (axis cs:208.557,    +07.384123) --  (axis cs:209.556,    +07.389874) --  (axis cs:210.554,    +07.395811) --  (axis cs:211.553,    +07.401932) --  (axis cs:212.551,    +07.408234) --  (axis cs:213.55,    +07.414717) --  (axis cs:214.549,    +07.421378) --  (axis cs:215.547,    +07.428215) --  (axis cs:216.546,    +07.435225) --  (axis cs:217.544,    +07.442408) --  (axis cs:218.543,    +07.449761) --  (axis cs:219.542,    +07.457280) --  (axis cs:220.54,    +07.464965) --  (axis cs:221.539,    +07.472813) --  (axis cs:222.538,    +07.480822) --  (axis cs:223.537,    +07.488988) --  (axis cs:224.535,    +07.497310) --  (axis cs:225.534,    +07.505784) --  (axis cs:226.533,    +07.514410) --  (axis cs:227.532,    +07.523183) --  (axis cs:227.782,    +07.525399) ;
  \draw[constellation-boundary]  (axis cs:204.064,    +07.360584) --  (axis cs:204.059,    +08.360572) --  (axis cs:204.054,    +09.360561) --  (axis cs:204.049,    +10.360549) --  (axis cs:204.044,    +11.360537) --  (axis cs:204.039,    +12.360525) --  (axis cs:204.034,    +13.360513) --  (axis cs:204.029,    +14.360500) --  (axis cs:204.024,    +15.360488) --  (axis cs:204.019,    +16.360475) --  (axis cs:204.013,    +17.360463) --  (axis cs:204.008,    +18.360450) --  (axis cs:204.003,    +19.360437) --  (axis cs:203.997,    +20.360424) --  (axis cs:203.992,    +21.360411) --  (axis cs:203.986,    +22.360397) --  (axis cs:203.98,    +23.360384) --  (axis cs:203.975,    +24.360370) --  (axis cs:203.969,    +25.360356) --  (axis cs:203.963,    +26.360342) --  (axis cs:203.957,    +27.360327) --  (axis cs:203.954,    +27.860320) ;
  \draw[constellation-boundary]  (axis cs:210.789,    +27.897658) --  (axis cs:210.785,    +28.397646) --  (axis cs:210.777,    +29.397621) --  (axis cs:210.771,    +30.147603) ;
  \draw[constellation-boundary]  (axis cs:210.771,    +30.147603) --  (axis cs:211.392,    +30.151432) --  (axis cs:211.889,    +30.154546) ;
  \draw[constellation-boundary]  (axis cs:211.889,    +30.154546) --  (axis cs:211.887,    +30.404539) --  (axis cs:211.878,    +31.404512) --  (axis cs:211.87,    +32.404484) --  (axis cs:211.861,    +33.404456) --  (axis cs:211.851,    +34.404427) --  (axis cs:211.842,    +35.404397) --  (axis cs:211.832,    +36.404367) --  (axis cs:211.823,    +37.404336) --  (axis cs:211.812,    +38.404304) --  (axis cs:211.802,    +39.404271) --  (axis cs:211.791,    +40.404237) --  (axis cs:211.78,    +41.404202) --  (axis cs:211.769,    +42.404166) --  (axis cs:211.757,    +43.404129) --  (axis cs:211.745,    +44.404091) ;
  \draw[constellation-boundary]  (axis cs:237.365,    +39.618913) --  (axis cs:237.348,    +40.618825) --  (axis cs:237.33,    +41.618733) --  (axis cs:237.311,    +42.618638) --  (axis cs:237.292,    +43.618541) --  (axis cs:237.273,    +44.618440) ;
  \draw[constellation-boundary]  (axis cs:232.644,    +39.572120) --  (axis cs:233.141,    +39.576913) --  (axis cs:234.134,    +39.586595) --  (axis cs:235.128,    +39.596402) --  (axis cs:236.123,    +39.606332) --  (axis cs:237.117,    +39.616382) --  (axis cs:238.111,    +39.626550) --  (axis cs:239.106,    +39.636831) --  (axis cs:240.101,    +39.647223) --  (axis cs:241.095,    +39.657722) --  (axis cs:242.09,    +39.668326) --  (axis cs:243.086,    +39.679032) --  (axis cs:244.081,    +39.689835) --  (axis cs:245.076,    +39.700734) --  (axis cs:246.072,    +39.711724) ;
  \draw[constellation-boundary]  (axis cs:232.747,    +32.572617) --  (axis cs:232.733,    +33.572551) --  (axis cs:232.719,    +34.572484) --  (axis cs:232.705,    +35.572415) --  (axis cs:232.69,    +36.572344) --  (axis cs:232.675,    +37.572272) --  (axis cs:232.66,    +38.572197) --  (axis cs:232.644,    +39.572120) ;
  \draw[constellation-boundary]  (axis cs:229.016,    +32.537682) --  (axis cs:229.265,    +32.539951) --  (axis cs:230.259,    +32.549115) --  (axis cs:231.254,    +32.558415) --  (axis cs:232.249,    +32.567850) --  (axis cs:232.747,    +32.572617) ;
  \draw[constellation-boundary]  (axis cs:229.1,    +25.538063) --  (axis cs:229.088,    +26.538012) --  (axis cs:229.077,    +27.537959) --  (axis cs:229.065,    +28.537906) --  (axis cs:229.053,    +29.537852) --  (axis cs:229.041,    +30.537796) --  (axis cs:229.029,    +31.537740) --  (axis cs:229.016,    +32.537682) ;
  \draw[constellation-boundary]  (axis cs:227.606,    +25.524616) --  (axis cs:228.353,    +25.531300) --  (axis cs:229.349,    +25.540335) --  (axis cs:230.345,    +25.549510) --  (axis cs:231.341,    +25.558822) --  (axis cs:232.337,    +25.568269) --  (axis cs:233.334,    +25.577847) --  (axis cs:234.33,    +25.587553) --  (axis cs:235.326,    +25.597385) --  (axis cs:236.323,    +25.607339) --  (axis cs:237.32,    +25.617414) --  (axis cs:238.316,    +25.627604) --  (axis cs:239.313,    +25.637909) --  (axis cs:240.31,    +25.648323) --  (axis cs:241.307,    +25.658845) --  (axis cs:242.304,    +25.669471) --  (axis cs:243.302,    +25.680198) --  (axis cs:243.8,    +25.685599) ;
  \draw[constellation-boundary]  (axis cs:181.594,    +33.303971) --  (axis cs:182.586,    +33.304245) --  (axis cs:183.578,    +33.304728) --  (axis cs:184.57,    +33.305422) --  (axis cs:185.562,    +33.306326) --  (axis cs:186.554,    +33.307439) ;
  \draw[constellation-boundary]  (axis cs:181.591,    +44.303971) --  (axis cs:182.579,    +44.304243) --  (axis cs:182.826,    +44.304344) ;
  \draw[constellation-boundary]  (axis cs:182.826,    +44.304344) ;
  \draw[constellation-boundary]  (axis cs:200.227,    +27.843782) --  (axis cs:200.475,    +27.844798) --  (axis cs:201.469,    +27.848987) --  (axis cs:202.463,    +27.853374) --  (axis cs:203.457,    +27.857956) --  (axis cs:204.451,    +27.862733) --  (axis cs:205.445,    +27.867702) --  (axis cs:206.439,    +27.872864) --  (axis cs:207.433,    +27.878215) --  (axis cs:208.427,    +27.883755) --  (axis cs:209.422,    +27.889482) --  (axis cs:210.416,    +27.895394) --  (axis cs:210.789,    +27.897658) ;
  \draw[constellation-boundary]  (axis cs:200.227,    +27.843782) --  (axis cs:200.224,    +28.343776) --  (axis cs:200.219,    +29.343765) --  (axis cs:200.213,    +30.343754) --  (axis cs:200.208,    +31.343743) ;
  \draw[constellation-boundary]  (axis cs:186.558,    +31.307441) --  (axis cs:187.55,    +31.308765) --  (axis cs:188.543,    +31.310297) --  (axis cs:189.536,    +31.312038) --  (axis cs:190.528,    +31.313987) --  (axis cs:191.521,    +31.316143) --  (axis cs:192.513,    +31.318506) --  (axis cs:193.506,    +31.321075) --  (axis cs:194.499,    +31.323850) --  (axis cs:195.492,    +31.326829) --  (axis cs:196.484,    +31.330012) --  (axis cs:197.477,    +31.333397) --  (axis cs:198.47,    +31.336984) --  (axis cs:199.463,    +31.340772) --  (axis cs:200.208,    +31.343743) ;
  \draw[constellation-boundary]  (axis cs:186.558,    +31.307441) --  (axis cs:186.556,    +32.307440) --  (axis cs:186.554,    +33.307439) ;
  \draw[constellation-boundary]  (axis cs:214.476,    -44.580553) --  (axis cs:214.463,    -43.580596) --  (axis cs:214.45,    -42.580638) ;
  \draw[constellation-boundary]  (axis cs:214.45,    -42.580638) --  (axis cs:214.955,    -42.577249) --  (axis cs:215.964,    -42.570341) --  (axis cs:216.973,    -42.563257) --  (axis cs:217.982,    -42.556000) --  (axis cs:218.991,    -42.548574) --  (axis cs:219.999,    -42.540979) --  (axis cs:221.008,    -42.533218) --  (axis cs:222.016,    -42.525294) --  (axis cs:223.024,    -42.517209) --  (axis cs:224.032,    -42.508966) --  (axis cs:225.04,    -42.500567) --  (axis cs:225.796,    -42.494168) ;
  \draw[constellation-boundary]  (axis cs:225.796,    -42.494168) --  (axis cs:225.781,    -41.494233) --  (axis cs:225.766,    -40.494297) --  (axis cs:225.751,    -39.494359) --  (axis cs:225.737,    -38.494419) --  (axis cs:225.723,    -37.494478) --  (axis cs:225.71,    -36.494535) --  (axis cs:225.697,    -35.494590) --  (axis cs:225.684,    -34.494644) --  (axis cs:225.672,    -33.494697) --  (axis cs:225.66,    -32.494749) --  (axis cs:225.648,    -31.494799) --  (axis cs:225.636,    -30.494849) --  (axis cs:225.625,    -29.494897) --  (axis cs:225.614,    -28.494945) --  (axis cs:225.603,    -27.494991) --  (axis cs:225.592,    -26.495037) --  (axis cs:225.582,    -25.495082) --  (axis cs:225.576,    -24.995104) ;
  \draw[constellation-boundary]  (axis cs:190.417,    -30.186382) --  (axis cs:190.669,    -30.185868) --  (axis cs:191.676,    -30.183680) --  (axis cs:192.683,    -30.181284) --  (axis cs:193.69,    -30.178678) --  (axis cs:194.697,    -30.175864) --  (axis cs:195.703,    -30.172843) --  (axis cs:196.71,    -30.169615) --  (axis cs:197.717,    -30.166183) --  (axis cs:198.723,    -30.162546) --  (axis cs:199.73,    -30.158706) --  (axis cs:200.737,    -30.154664) --  (axis cs:201.743,    -30.150421) --  (axis cs:202.75,    -30.145980) --  (axis cs:203.756,    -30.141340) --  (axis cs:204.762,    -30.136504) --  (axis cs:205.769,    -30.131472) --  (axis cs:206.775,    -30.126248) --  (axis cs:207.781,    -30.120831) --  (axis cs:208.788,    -30.115225) --  (axis cs:209.794,    -30.109430) --  (axis cs:210.8,    -30.103448) --  (axis cs:211.806,    -30.097282) --  (axis cs:212.812,    -30.090933) --  (axis cs:213.817,    -30.084403) --  (axis cs:214.823,    -30.077694) --  (axis cs:215.829,    -30.070808) --  (axis cs:216.835,    -30.063748) --  (axis cs:217.84,    -30.056516) --  (axis cs:218.846,    -30.049113) --  (axis cs:219.851,    -30.041542) --  (axis cs:220.856,    -30.033806) --  (axis cs:221.862,    -30.025907) --  (axis cs:222.867,    -30.017846) --  (axis cs:223.872,    -30.009628) --  (axis cs:224.877,    -30.001254) --  (axis cs:225.882,    -29.992727) --  (axis cs:226.887,    -29.984050) --  (axis cs:227.892,    -29.975224) --  (axis cs:228.896,    -29.966254) --  (axis cs:229.901,    -29.957141) --  (axis cs:230.905,    -29.947889) --  (axis cs:231.91,    -29.938501) --  (axis cs:232.914,    -29.928978) --  (axis cs:233.918,    -29.919325) --  (axis cs:234.922,    -29.909544) --  (axis cs:235.926,    -29.899639) --  (axis cs:236.93,    -29.889612) --  (axis cs:237.934,    -29.879466) --  (axis cs:238.937,    -29.869204) --  (axis cs:239.941,    -29.858830) --  (axis cs:240.944,    -29.848347) --  (axis cs:241.948,    -29.837759) ;
  \draw[constellation-boundary]  (axis cs:190.427,    -33.686372) --  (axis cs:190.424,    -32.686375) --  (axis cs:190.422,    -31.686377) --  (axis cs:190.419,    -30.686380) --  (axis cs:190.417,    -30.186382) ;
  \draw[constellation-boundary]  (axis cs:185.387,    -33.693884) --  (axis cs:185.639,    -33.693635) --  (axis cs:186.647,    -33.692504) --  (axis cs:187.655,    -33.691160) --  (axis cs:188.663,    -33.689604) --  (axis cs:189.671,    -33.687836) --  (axis cs:190.427,    -33.686372) ;
  \draw[constellation-boundary]  (axis cs:185.39,    -35.693883) --  (axis cs:185.389,    -34.693884) --  (axis cs:185.387,    -33.693884) ;
  \draw[constellation-boundary]  (axis cs:180.602,    +13.303908) --  (axis cs:181.599,    +13.303972) --  (axis cs:182.596,    +13.304247) --  (axis cs:183.593,    +13.304733) --  (axis cs:184.59,    +13.305430) --  (axis cs:185.587,    +13.306338) --  (axis cs:186.584,    +13.307457) --  (axis cs:187.581,    +13.308787) --  (axis cs:188.578,    +13.310326) --  (axis cs:189.575,    +13.312075) --  (axis cs:190.572,    +13.314032) --  (axis cs:191.569,    +13.316198) --  (axis cs:192.566,    +13.318572) --  (axis cs:193.564,    +13.321152) --  (axis cs:194.062,    +13.322520) ;
  \draw[constellation-boundary]  (axis cs:181.596,    +28.303971) --  (axis cs:181.595,    +29.303971) --  (axis cs:181.595,    +30.303971) --  (axis cs:181.595,    +31.303971) --  (axis cs:181.595,    +32.303971) --  (axis cs:181.594,    +33.303971) --  (axis cs:181.594,    +34.303971) --  (axis cs:181.594,    +35.303971) --  (axis cs:181.594,    +36.303971) --  (axis cs:181.593,    +37.303971) --  (axis cs:181.593,    +38.303971) --  (axis cs:181.593,    +39.303971) --  (axis cs:181.593,    +40.303971) --  (axis cs:181.592,    +41.303971) --  (axis cs:181.592,    +42.303971) --  (axis cs:181.592,    +43.303971) --  (axis cs:181.591,    +44.303971) ;
  \draw[constellation-boundary]  (axis cs:194.059,    +14.322516) --  (axis cs:194.558,    +14.323935) --  (axis cs:195.554,    +14.326926) --  (axis cs:196.551,    +14.330122) --  (axis cs:197.548,    +14.333522) --  (axis cs:198.545,    +14.337124) --  (axis cs:199.542,    +14.340927) --  (axis cs:200.539,    +14.344930) --  (axis cs:201.536,    +14.349132) --  (axis cs:202.533,    +14.353532) --  (axis cs:203.53,    +14.358129) --  (axis cs:204.029,    +14.360500) ;
  \draw[constellation-boundary]  (axis cs:194.062,    +13.322520) --  (axis cs:194.059,    +14.322516) ;
  \draw[constellation-boundary]  (axis cs:269.625,    -37.017462) --  (axis cs:270.126,    -37.011388) --  (axis cs:271.125,    -36.999240) --  (axis cs:272.125,    -36.987092) --  (axis cs:273.125,    -36.974950) --  (axis cs:274.124,    -36.962816) --  (axis cs:275.124,    -36.950694) --  (axis cs:276.123,    -36.938588) --  (axis cs:277.122,    -36.926502) --  (axis cs:278.121,    -36.914440) --  (axis cs:279.119,    -36.902404) --  (axis cs:280.118,    -36.890399) --  (axis cs:281.116,    -36.878428) --  (axis cs:282.115,    -36.866495) --  (axis cs:283.113,    -36.854604) --  (axis cs:284.11,    -36.842757) --  (axis cs:285.108,    -36.830960) --  (axis cs:286.106,    -36.819214) --  (axis cs:287.103,    -36.807525) --  (axis cs:288.101,    -36.795895) --  (axis cs:289.098,    -36.784328) --  (axis cs:289.596,    -36.778569) ;
  \draw[constellation-boundary]  (axis cs:289.758,    -44.777636) --  (axis cs:289.736,    -43.777765) --  (axis cs:289.714,    -42.777891) --  (axis cs:289.693,    -41.778012) --  (axis cs:289.672,    -40.778130) --  (axis cs:289.652,    -39.778245) --  (axis cs:289.633,    -38.778356) --  (axis cs:289.614,    -37.778464) --  (axis cs:289.596,    -36.778569) ;
  \draw[constellation-boundary]  (axis cs:246.28,    +26.712875) --  (axis cs:246.266,    +27.712798) --  (axis cs:246.252,    +28.712719) --  (axis cs:246.237,    +29.712638) --  (axis cs:246.222,    +30.712556) --  (axis cs:246.207,    +31.712472) --  (axis cs:246.192,    +32.712386) --  (axis cs:246.176,    +33.712298) --  (axis cs:246.159,    +34.712209) --  (axis cs:246.143,    +35.712117) --  (axis cs:246.126,    +36.712022) --  (axis cs:246.108,    +37.711925) --  (axis cs:246.09,    +38.711826) --  (axis cs:246.072,    +39.711724) ;
  \draw[constellation-boundary]  (axis cs:243.787,    +26.685525) --  (axis cs:244.285,    +26.690949) --  (axis cs:245.282,    +26.701867) --  (axis cs:246.28,    +26.712875) ;
  \draw[constellation-boundary]  (axis cs:243.8,    +25.685599) --  (axis cs:243.787,    +26.685525) ;
  \draw[constellation-boundary]  (axis cs:180.6,    -11.696092) --  (axis cs:181.603,    -11.696028) --  (axis cs:182.605,    -11.695752) --  (axis cs:183.608,    -11.695263) --  (axis cs:184.61,    -11.694562) --  (axis cs:185.613,    -11.693649) --  (axis cs:186.615,    -11.692524) --  (axis cs:187.617,    -11.691187) --  (axis cs:188.62,    -11.689640) --  (axis cs:189.622,    -11.687882) --  (axis cs:190.625,    -11.685914) --  (axis cs:191.627,    -11.683736) --  (axis cs:192.63,    -11.681350) --  (axis cs:193.632,    -11.678756) --  (axis cs:194.133,    -11.677381) ;
  \draw[constellation-boundary]  (axis cs:190.405,    -25.186395) --  (axis cs:190.403,    -24.686396) --  (axis cs:190.401,    -23.686398) --  (axis cs:190.399,    -22.686401) ;
  \draw[constellation-boundary]  (axis cs:190.399,    -22.686401) --  (axis cs:190.65,    -22.685888) --  (axis cs:191.655,    -22.683705) --  (axis cs:192.66,    -22.681312) --  (axis cs:193.664,    -22.678712) --  (axis cs:194.669,    -22.675903) --  (axis cs:195.674,    -22.672888) --  (axis cs:196.679,    -22.669667) --  (axis cs:197.684,    -22.666240) --  (axis cs:198.689,    -22.662610) --  (axis cs:199.693,    -22.658778) --  (axis cs:200.698,    -22.654743) --  (axis cs:201.703,    -22.650509) --  (axis cs:202.707,    -22.646075) --  (axis cs:203.712,    -22.641444) --  (axis cs:204.717,    -22.636616) --  (axis cs:205.721,    -22.631594) --  (axis cs:206.726,    -22.626379) --  (axis cs:207.73,    -22.620972) --  (axis cs:208.734,    -22.615375) --  (axis cs:209.739,    -22.609590) --  (axis cs:210.743,    -22.603619) --  (axis cs:211.747,    -22.597463) --  (axis cs:212.752,    -22.591125) --  (axis cs:213.756,    -22.584605) --  (axis cs:214.76,    -22.577908) --  (axis cs:215.513,    -22.572768) ;
  \draw[constellation-boundary]  (axis cs:194.167,    -22.677333) --  (axis cs:194.164,    -21.677338) --  (axis cs:194.16,    -20.677342) --  (axis cs:194.157,    -19.677347) --  (axis cs:194.154,    -18.677351) --  (axis cs:194.151,    -17.677356) --  (axis cs:194.148,    -16.677360) --  (axis cs:194.145,    -15.677364) --  (axis cs:194.142,    -14.677368) --  (axis cs:194.139,    -13.677372) --  (axis cs:194.136,    -12.677377) --  (axis cs:194.133,    -11.677381) ;
  \draw[constellation-boundary]  (axis cs:180.6,    -25.196092) --  (axis cs:181.605,    -25.196028) --  (axis cs:182.611,    -25.195750) --  (axis cs:183.617,    -25.195260) --  (axis cs:184.622,    -25.194557) --  (axis cs:185.628,    -25.193641) --  (axis cs:186.634,    -25.192512) --  (axis cs:187.639,    -25.191171) --  (axis cs:188.645,    -25.189619) --  (axis cs:189.65,    -25.187856) --  (axis cs:190.405,    -25.186395) ;
  \draw[constellation-boundary]  (axis cs:290.133,    +27.732407) --  (axis cs:290.258,    +27.733840) --  (axis cs:291.26,    +27.745259) --  (axis cs:292.262,    +27.756603) --  (axis cs:293.265,    +27.767870) --  (axis cs:294.267,    +27.779056) ;
  \draw[constellation-boundary]  (axis cs:290.095,    +30.232193) --  (axis cs:290.221,    +30.233626) --  (axis cs:291.223,    +30.245048) --  (axis cs:291.599,    +30.249312) ;
  \draw[constellation-boundary]  (axis cs:291.599,    +30.249312) --  (axis cs:291.583,    +31.249225) --  (axis cs:291.568,    +32.249136) --  (axis cs:291.552,    +33.249046) --  (axis cs:291.535,    +34.248953) --  (axis cs:291.519,    +35.248858) --  (axis cs:291.501,    +36.248760) --  (axis cs:291.493,    +36.748711) ;
  \draw[constellation-boundary]  (axis cs:291.493,    +36.748711) --  (axis cs:292.12,    +36.755799) ;
  \draw[constellation-boundary]  (axis cs:292.12,    +36.755799) --  (axis cs:292.111,    +37.255749) --  (axis cs:292.093,    +38.255647) --  (axis cs:292.074,    +39.255543) --  (axis cs:292.055,    +40.255435) --  (axis cs:292.035,    +41.255324) --  (axis cs:292.015,    +42.255210) --  (axis cs:291.994,    +43.255092) --  (axis cs:291.984,    +43.755031) ;
  \draw[constellation-boundary]  (axis cs:288.47,    +43.714937) --  (axis cs:288.972,    +43.720717) --  (axis cs:289.976,    +43.732226) --  (axis cs:290.98,    +43.743666) --  (axis cs:291.984,    +43.755031) ;
  \draw[constellation-boundary]  (axis cs:288.47,    +43.714937) --  (axis cs:288.459,    +44.214873) ;
  \draw[constellation-boundary]  (axis cs:242.81,    +03.673518) --  (axis cs:242.799,    +04.673460) --  (axis cs:242.788,    +05.673402) --  (axis cs:242.777,    +06.673344) --  (axis cs:242.766,    +07.673285) --  (axis cs:242.755,    +08.673226) --  (axis cs:242.744,    +09.673167) --  (axis cs:242.733,    +10.673108) --  (axis cs:242.722,    +11.673048) --  (axis cs:242.711,    +12.672988) --  (axis cs:242.7,    +13.672927) --  (axis cs:242.688,    +14.672866) --  (axis cs:242.677,    +15.672804) ;
  \draw[constellation-boundary]  (axis cs:242.81,    +03.673518) --  (axis cs:243.559,    +03.681592) --  (axis cs:244.559,    +03.692441) --  (axis cs:245.559,    +03.703384) --  (axis cs:246.558,    +03.714417) --  (axis cs:247.558,    +03.725538) --  (axis cs:248.557,    +03.736742) --  (axis cs:249.557,    +03.748026) --  (axis cs:250.557,    +03.759388) --  (axis cs:251.556,    +03.770822) --  (axis cs:252.556,    +03.782326) --  (axis cs:252.806,    +03.785213) ;
  \draw[constellation-boundary]  (axis cs:240.181,    +15.646338) --  (axis cs:240.431,    +15.648955) --  (axis cs:241.429,    +15.659490) --  (axis cs:242.427,    +15.670129) --  (axis cs:242.677,    +15.672804) ;
  \draw[constellation-boundary]  (axis cs:240.181,    +15.646338) --  (axis cs:240.17,    +16.646279) --  (axis cs:240.158,    +17.646219) --  (axis cs:240.147,    +18.646158) --  (axis cs:240.135,    +19.646096) --  (axis cs:240.123,    +20.646034) --  (axis cs:240.111,    +21.645971) ;
  \draw[constellation-boundary]  (axis cs:240.111,    +21.645971) --  (axis cs:240.36,    +21.648586) --  (axis cs:241.358,    +21.659114) --  (axis cs:241.857,    +21.664416) ;
  \draw[constellation-boundary]  (axis cs:241.857,    +21.664416) --  (axis cs:241.844,    +22.664350) --  (axis cs:241.832,    +23.664283) --  (axis cs:241.819,    +24.664215) --  (axis cs:241.806,    +25.664146) ;
  \draw[constellation-boundary]  (axis cs:273.824,    +30.039155) --  (axis cs:273.808,    +31.039056) --  (axis cs:273.791,    +32.038955) --  (axis cs:273.774,    +33.038851) --  (axis cs:273.757,    +34.038745) --  (axis cs:273.739,    +35.038637) --  (axis cs:273.721,    +36.038526) --  (axis cs:273.702,    +37.038412) --  (axis cs:273.683,    +38.038295) --  (axis cs:273.663,    +39.038175) --  (axis cs:273.642,    +40.038051) --  (axis cs:273.621,    +41.037924) --  (axis cs:273.6,    +42.037793) --  (axis cs:273.577,    +43.037657) --  (axis cs:273.554,    +44.037517) ;
  \draw[constellation-boundary]  (axis cs:273.824,    +30.039155) --  (axis cs:274.199,    +30.043704) --  (axis cs:275.2,    +30.055824) --  (axis cs:276.2,    +30.067929) --  (axis cs:276.701,    +30.073974) ;
  \draw[constellation-boundary]  (axis cs:276.763,    +26.074349) --  (axis cs:276.748,    +27.074257) --  (axis cs:276.732,    +28.074165) --  (axis cs:276.717,    +29.074070) --  (axis cs:276.701,    +30.073974) ;
  \draw[constellation-boundary]  (axis cs:276.763,    +26.074349) --  (axis cs:277.263,    +26.080387) --  (axis cs:278.264,    +26.092446) --  (axis cs:279.264,    +26.104478) --  (axis cs:280.265,    +26.116478) --  (axis cs:281.266,    +26.128444) --  (axis cs:282.267,    +26.140371) --  (axis cs:283.269,    +26.152255) --  (axis cs:284.27,    +26.164094) ;
  \draw[constellation-boundary]  (axis cs:284.277,    +25.664137) --  (axis cs:285.278,    +25.675926) --  (axis cs:286.28,    +25.687662) --  (axis cs:287.281,    +25.699341) --  (axis cs:288.283,    +25.710960) --  (axis cs:289.285,    +25.722516) --  (axis cs:290.161,    +25.732571) ;
  \draw[constellation-boundary]  (axis cs:284.339,    +21.247836) --  (axis cs:285.34,    +21.259622) --  (axis cs:286.341,    +21.271354) --  (axis cs:287.343,    +21.283030) --  (axis cs:288.344,    +21.294645) --  (axis cs:289.345,    +21.306197) --  (axis cs:290.096,    +21.314816) ;
  \draw[constellation-boundary]  (axis cs:275.203,    +12.054330) --  (axis cs:275.453,    +12.057357) --  (axis cs:276.453,    +12.069457) --  (axis cs:277.453,    +12.081536) --  (axis cs:278.454,    +12.093590) --  (axis cs:279.454,    +12.105616) --  (axis cs:280.454,    +12.117610) --  (axis cs:281.455,    +12.129568) --  (axis cs:282.455,    +12.141487) --  (axis cs:283.456,    +12.153364) --  (axis cs:284.456,    +12.165194) ;
  \draw[constellation-boundary]  (axis cs:275.203,    +12.054330) --  (axis cs:275.19,    +13.054253) --  (axis cs:275.178,    +14.054175) --  (axis cs:275.173,    +14.387482) ;
  \draw[constellation-boundary]  (axis cs:260.177,    +14.206039) --  (axis cs:260.427,    +14.209026) --  (axis cs:261.426,    +14.220995) --  (axis cs:262.426,    +14.232998) --  (axis cs:263.425,    +14.245032) --  (axis cs:264.425,    +14.257093) --  (axis cs:265.424,    +14.269178) --  (axis cs:266.424,    +14.281283) --  (axis cs:267.424,    +14.293404) --  (axis cs:268.423,    +14.305537) --  (axis cs:269.423,    +14.317679) --  (axis cs:270.423,    +14.329827) --  (axis cs:271.423,    +14.341975) --  (axis cs:272.423,    +14.354122) --  (axis cs:273.423,    +14.366262) --  (axis cs:274.423,    +14.378393) --  (axis cs:275.173,    +14.387482) ;
  \draw[constellation-boundary]  (axis cs:260.196,    +12.706153) --  (axis cs:260.194,    +12.872807) --  (axis cs:260.181,    +13.872731) --  (axis cs:260.177,    +14.206039) ;
  \draw[constellation-boundary]  (axis cs:252.701,    +12.617943) --  (axis cs:253.451,    +12.626622) --  (axis cs:254.45,    +12.638249) --  (axis cs:255.449,    +12.649936) --  (axis cs:256.448,    +12.661679) --  (axis cs:257.448,    +12.673474) --  (axis cs:258.447,    +12.685318) --  (axis cs:259.446,    +12.697208) --  (axis cs:260.196,    +12.706153) ;
  \draw[constellation-boundary]  (axis cs:252.806,    +03.785213) --  (axis cs:252.794,    +04.785146) --  (axis cs:252.783,    +05.785078) --  (axis cs:252.771,    +06.785011) --  (axis cs:252.759,    +07.784943) --  (axis cs:252.747,    +08.784875) --  (axis cs:252.735,    +09.784806) --  (axis cs:252.724,    +10.784737) --  (axis cs:252.712,    +11.784668) --  (axis cs:252.701,    +12.617943) ;
  \draw[constellation-boundary]  (axis cs:215.534,    -25.072697) --  (axis cs:215.785,    -25.070961) --  (axis cs:216.789,    -25.063909) --  (axis cs:217.794,    -25.056684) --  (axis cs:218.798,    -25.049289) --  (axis cs:219.803,    -25.041727) --  (axis cs:220.807,    -25.033998) --  (axis cs:221.811,    -25.026107) --  (axis cs:222.815,    -25.018055) --  (axis cs:223.819,    -25.009845) --  (axis cs:224.824,    -25.001479) --  (axis cs:225.576,    -24.995104) ;
  \draw[constellation-boundary]  (axis cs:215.534,    -25.072697) --  (axis cs:215.529,    -24.572712) --  (axis cs:215.521,    -23.572740) --  (axis cs:215.513,    -22.572768) --  (axis cs:215.505,    -21.572796) --  (axis cs:215.497,    -20.572823) --  (axis cs:215.489,    -19.572850) --  (axis cs:215.482,    -18.572877) --  (axis cs:215.474,    -17.572903) --  (axis cs:215.466,    -16.572929) --  (axis cs:215.459,    -15.572955) --  (axis cs:215.452,    -14.572980) --  (axis cs:215.444,    -13.573005) --  (axis cs:215.437,    -12.573030) --  (axis cs:215.43,    -11.573055) --  (axis cs:215.423,    -10.573080) --  (axis cs:215.416,    -09.573104) --  (axis cs:215.408,    -08.573129) ;
  \draw[constellation-boundary]  (axis cs:180.602,    +28.303908) --  (axis cs:181.596,    +28.303971) ;
  \draw[constellation-boundary]  (axis cs:221.603,    -00.526933) --  (axis cs:222.603,    -00.518915) --  (axis cs:223.603,    -00.510738) --  (axis cs:224.603,    -00.502406) --  (axis cs:225.603,    -00.493921) --  (axis cs:226.603,    -00.485285) --  (axis cs:227.603,    -00.476502) --  (axis cs:227.853,    -00.474283) ;
  \draw[constellation-boundary]  (axis cs:221.667,    -08.526679) --  (axis cs:221.659,    -07.526711) --  (axis cs:221.651,    -06.526743) --  (axis cs:221.643,    -05.526775) --  (axis cs:221.635,    -04.526807) --  (axis cs:221.627,    -03.526839) --  (axis cs:221.619,    -02.526870) --  (axis cs:221.611,    -01.526902) --  (axis cs:221.603,    -00.526933) ;
  \draw[constellation-boundary]  (axis cs:215.408,    -08.573129) --  (axis cs:215.659,    -08.571398) --  (axis cs:216.66,    -08.564368) --  (axis cs:217.662,    -08.557165) --  (axis cs:218.663,    -08.549793) --  (axis cs:219.664,    -08.542253) --  (axis cs:220.666,    -08.534548) --  (axis cs:221.667,    -08.526679) ;
  \draw[constellation-boundary]  (axis cs:236.93,    -29.889612) --  (axis cs:236.923,    -29.389645) --  (axis cs:236.91,    -28.389711) --  (axis cs:236.897,    -27.389775) --  (axis cs:236.885,    -26.389839) --  (axis cs:236.872,    -25.389901) --  (axis cs:236.86,    -24.389963) --  (axis cs:236.848,    -23.390023) --  (axis cs:236.836,    -22.390083) --  (axis cs:236.824,    -21.390141) --  (axis cs:236.813,    -20.390199) ;
  \draw[constellation-boundary]  (axis cs:236.813,    -20.390199) --  (axis cs:237.815,    -20.380067) --  (axis cs:238.818,    -20.369819) --  (axis cs:239.82,    -20.359458) --  (axis cs:240.572,    -20.351616) ;
  \draw[constellation-boundary]  (axis cs:240.572,    -20.351616) --  (axis cs:240.56,    -19.351678) --  (axis cs:240.548,    -18.351739) --  (axis cs:240.537,    -17.351800) --  (axis cs:240.525,    -16.351860) --  (axis cs:240.514,    -15.351919) --  (axis cs:240.503,    -14.351978) --  (axis cs:240.492,    -13.352036) --  (axis cs:240.481,    -12.352093) --  (axis cs:240.47,    -11.352151) --  (axis cs:240.459,    -10.352208) --  (axis cs:240.448,    -09.352264) --  (axis cs:240.437,    -08.352320) --  (axis cs:240.427,    -07.352376) --  (axis cs:240.416,    -06.352432) --  (axis cs:240.405,    -05.352487) --  (axis cs:240.395,    -04.352542) --  (axis cs:240.387,    -03.602584) ;
  \draw[constellation-boundary]  (axis cs:240.437,    -08.352320) --  (axis cs:240.687,    -08.349696) --  (axis cs:241.688,    -08.339134) --  (axis cs:242.689,    -08.328469) --  (axis cs:243.69,    -08.317704) --  (axis cs:244.69,    -08.306842) --  (axis cs:245.691,    -08.295887) ;
  \draw[constellation-boundary]  (axis cs:227.882,    -03.724154) --  (axis cs:227.88,    -03.474164) --  (axis cs:227.871,    -02.474204) --  (axis cs:227.862,    -01.474243) --  (axis cs:227.853,    -00.474283) --  (axis cs:227.844,    +00.525677) --  (axis cs:227.835,    +01.525638) --  (axis cs:227.826,    +02.525598) --  (axis cs:227.817,    +03.525559) --  (axis cs:227.808,    +04.525519) --  (axis cs:227.8,    +05.525479) --  (axis cs:227.791,    +06.525439) --  (axis cs:227.782,    +07.525399) --  (axis cs:227.772,    +08.525359) --  (axis cs:227.763,    +09.525318) --  (axis cs:227.754,    +10.525277) --  (axis cs:227.745,    +11.525236) --  (axis cs:227.736,    +12.525195) --  (axis cs:227.726,    +13.525153) --  (axis cs:227.717,    +14.525111) --  (axis cs:227.707,    +15.525069) --  (axis cs:227.698,    +16.525026) --  (axis cs:227.688,    +17.524983) --  (axis cs:227.678,    +18.524939) --  (axis cs:227.668,    +19.524895) --  (axis cs:227.658,    +20.524850) --  (axis cs:227.648,    +21.524805) --  (axis cs:227.638,    +22.524759) --  (axis cs:227.627,    +23.524712) --  (axis cs:227.616,    +24.524665) --  (axis cs:227.606,    +25.524616) ;
  \draw[constellation-boundary]  (axis cs:227.882,    -03.724154) --  (axis cs:228.632,    -03.717441) --  (axis cs:229.633,    -03.708366) --  (axis cs:230.633,    -03.699151) --  (axis cs:231.634,    -03.689800) --  (axis cs:232.634,    -03.680314) --  (axis cs:233.634,    -03.670697) --  (axis cs:234.635,    -03.660952) --  (axis cs:235.635,    -03.651082) --  (axis cs:236.636,    -03.641090) --  (axis cs:237.636,    -03.630979) --  (axis cs:238.636,    -03.620751) --  (axis cs:239.637,    -03.610411) --  (axis cs:240.637,    -03.599961) --  (axis cs:241.637,    -03.589405) --  (axis cs:242.638,    -03.578745) --  (axis cs:243.638,    -03.567985) --  (axis cs:244.638,    -03.557128) --  (axis cs:245.638,    -03.546177) ;
  \draw[constellation-boundary]  (axis cs:237.163,    -44.388443) --  (axis cs:237.143,    -43.388540) --  (axis cs:237.124,    -42.388634) ;
  \draw[constellation-boundary]  (axis cs:237.124,    -42.388634) --  (axis cs:238.13,    -42.378465) --  (axis cs:239.136,    -42.368181) --  (axis cs:240.142,    -42.357785) --  (axis cs:241.147,    -42.347281) --  (axis cs:242.153,    -42.336671) --  (axis cs:243.158,    -42.325959) --  (axis cs:244.163,    -42.315148) --  (axis cs:245.168,    -42.304241) --  (axis cs:246.172,    -42.293242) --  (axis cs:247.177,    -42.282155) --  (axis cs:248.181,    -42.270982) --  (axis cs:248.495,    -42.267474) ;
  \draw[constellation-boundary]  (axis cs:242.153,    -42.336671) --  (axis cs:242.134,    -41.336772) --  (axis cs:242.115,    -40.336870) --  (axis cs:242.097,    -39.336966) --  (axis cs:242.08,    -38.337059) --  (axis cs:242.063,    -37.337149) --  (axis cs:242.046,    -36.337237) --  (axis cs:242.03,    -35.337322) --  (axis cs:242.014,    -34.337406) --  (axis cs:241.999,    -33.337487) --  (axis cs:241.984,    -32.337567) --  (axis cs:241.969,    -31.337645) --  (axis cs:241.955,    -30.337721) --  (axis cs:241.948,    -29.837759) ;
  \draw[constellation-boundary]  (axis cs:290.161,    +25.732571) --  (axis cs:290.154,    +26.232531) --  (axis cs:290.14,    +27.232449) --  (axis cs:290.125,    +28.232365) --  (axis cs:290.11,    +29.232280) --  (axis cs:290.095,    +30.232193) ;
  \draw[constellation-boundary]  (axis cs:275.275,    +06.304763) --  (axis cs:275.525,    +06.307790) --  (axis cs:276.525,    +06.319888) --  (axis cs:277.525,    +06.331965) --  (axis cs:278.525,    +06.344018) --  (axis cs:279.525,    +06.356041) --  (axis cs:280.525,    +06.368033) --  (axis cs:281.525,    +06.379989) --  (axis cs:282.525,    +06.391905) --  (axis cs:283.526,    +06.403778) --  (axis cs:284.526,    +06.415605) ;
  \draw[constellation-boundary]  (axis cs:275.296,    +04.554892) --  (axis cs:275.29,    +05.054855) --  (axis cs:275.278,    +06.054781) --  (axis cs:275.275,    +06.304763) ;
  \draw[constellation-boundary]  (axis cs:275.296,    +04.554892) --  (axis cs:275.546,    +04.557919) --  (axis cs:276.546,    +04.570017) --  (axis cs:277.546,    +04.582094) --  (axis cs:277.921,    +04.586616) ;
  \draw[constellation-boundary]  (axis cs:277.939,    +03.086726) --  (axis cs:277.927,    +04.086653) --  (axis cs:277.921,    +04.586616) ;
  \draw[constellation-boundary]  (axis cs:275.314,    +03.055003) --  (axis cs:275.564,    +03.058030) --  (axis cs:276.564,    +03.070127) --  (axis cs:277.564,    +03.082204) --  (axis cs:277.939,    +03.086726) ;
  \draw[constellation-boundary]  (axis cs:275.351,    +00.055223) --  (axis cs:275.338,    +01.055150) --  (axis cs:275.326,    +02.055076) --  (axis cs:275.314,    +03.055003) ;
  \draw[constellation-boundary]  (axis cs:269.101,    -00.020646) --  (axis cs:269.601,    -00.014574) --  (axis cs:270.601,    -00.002426) --  (axis cs:271.601,    +00.009722) --  (axis cs:272.601,    +00.021868) --  (axis cs:273.601,    +00.034007) --  (axis cs:274.601,    +00.046135) --  (axis cs:275.351,    +00.055223) ;
  \draw[constellation-boundary]  (axis cs:269.15,    -04.020351) --  (axis cs:269.138,    -03.020425) --  (axis cs:269.125,    -02.020499) --  (axis cs:269.113,    -01.020573) --  (axis cs:269.101,    -00.020646) ;
  \draw[constellation-boundary]  (axis cs:269.15,    -04.020351) --  (axis cs:269.65,    -04.014278) --  (axis cs:270.65,    -04.002130) --  (axis cs:271.15,    -03.996056) ;
  \draw[constellation-boundary]  (axis cs:271.224,    -09.995606) --  (axis cs:271.211,    -08.995682) --  (axis cs:271.199,    -07.995758) --  (axis cs:271.186,    -06.995833) --  (axis cs:271.174,    -05.995907) --  (axis cs:271.162,    -04.995982) --  (axis cs:271.15,    -03.996056) ;
  \draw[constellation-boundary]  (axis cs:266.724,    -10.050234) --  (axis cs:267.724,    -10.038109) --  (axis cs:268.724,    -10.025973) --  (axis cs:269.724,    -10.013829) --  (axis cs:270.724,    -10.001681) --  (axis cs:271.224,    -09.995606) ;
  \draw[constellation-boundary]  (axis cs:266.745,    -11.716774) --  (axis cs:266.736,    -11.050158) --  (axis cs:266.724,    -10.050234) ;
  \draw[constellation-boundary]  (axis cs:265.494,    -11.731909) --  (axis cs:265.744,    -11.728884) --  (axis cs:266.745,    -11.716774) ;
  \draw[constellation-boundary]  (axis cs:265.494,    -11.731909) --  (axis cs:265.486,    -11.065293) --  (axis cs:265.474,    -10.065369) ;
  \draw[constellation-boundary]  (axis cs:259.222,    -10.140437) --  (axis cs:259.722,    -10.134481) --  (axis cs:260.723,    -10.122539) --  (axis cs:261.723,    -10.110560) --  (axis cs:262.723,    -10.098547) --  (axis cs:263.723,    -10.086504) --  (axis cs:264.723,    -10.074435) --  (axis cs:265.474,    -10.065369) ;
  \draw[constellation-boundary]  (axis cs:259.297,    -16.139989) --  (axis cs:259.285,    -15.140065) --  (axis cs:259.272,    -14.140141) --  (axis cs:259.259,    -13.140216) --  (axis cs:259.247,    -12.140290) --  (axis cs:259.234,    -11.140364) --  (axis cs:259.222,    -10.140437) ;
  \draw[constellation-boundary]  (axis cs:259.297,    -16.139989) --  (axis cs:259.798,    -16.134031) --  (axis cs:260.798,    -16.122086) --  (axis cs:261.799,    -16.110104) --  (axis cs:262.799,    -16.098089) --  (axis cs:263.8,    -16.086044) --  (axis cs:264.8,    -16.073974) --  (axis cs:265.8,    -16.061881) --  (axis cs:266.8,    -16.049770) --  (axis cs:267.8,    -16.037644) --  (axis cs:268.801,    -16.025506) --  (axis cs:269.801,    -16.013362) --  (axis cs:270.801,    -16.001214) --  (axis cs:271.8,    -15.989065) --  (axis cs:272.8,    -15.976921) --  (axis cs:273.8,    -15.964784) --  (axis cs:274.8,    -15.952658) --  (axis cs:275.799,    -15.940546) --  (axis cs:276.799,    -15.928454) --  (axis cs:277.799,    -15.916383) --  (axis cs:278.798,    -15.904338) --  (axis cs:279.797,    -15.892323) --  (axis cs:280.797,    -15.880340) --  (axis cs:281.796,    -15.868395) --  (axis cs:282.795,    -15.856490) --  (axis cs:283.795,    -15.844629) --  (axis cs:284.794,    -15.832816) ;
  \draw[constellation-boundary]  (axis cs:266.002,    -30.060661) --  (axis cs:265.986,    -29.060758) --  (axis cs:265.97,    -28.060853) --  (axis cs:265.955,    -27.060946) --  (axis cs:265.94,    -26.061037) --  (axis cs:265.925,    -25.061127) --  (axis cs:265.91,    -24.061216) --  (axis cs:265.896,    -23.061303) --  (axis cs:265.882,    -22.061389) --  (axis cs:265.868,    -21.061473) --  (axis cs:265.854,    -20.061557) --  (axis cs:265.84,    -19.061639) --  (axis cs:265.827,    -18.061721) --  (axis cs:265.813,    -17.061801) --  (axis cs:265.8,    -16.061881) ;
  \draw[constellation-boundary]  (axis cs:253.235,    -30.212305) --  (axis cs:253.22,    -29.212394) --  (axis cs:253.205,    -28.212481) --  (axis cs:253.19,    -27.212566) --  (axis cs:253.176,    -26.212650) --  (axis cs:253.162,    -25.212732) --  (axis cs:253.156,    -24.796099) ;
  \draw[constellation-boundary]  (axis cs:253.235,    -30.212305) --  (axis cs:253.987,    -30.203601) --  (axis cs:254.989,    -30.191941) --  (axis cs:255.991,    -30.180224) --  (axis cs:256.992,    -30.168452) --  (axis cs:257.994,    -30.156629) --  (axis cs:258.995,    -30.144760) --  (axis cs:259.997,    -30.132847) --  (axis cs:260.998,    -30.120894) --  (axis cs:261.999,    -30.108905) --  (axis cs:263,    -30.096883) --  (axis cs:264.001,    -30.084833) --  (axis cs:265.001,    -30.072758) --  (axis cs:266.002,    -30.060661) --  (axis cs:267.002,    -30.048547) --  (axis cs:268.003,    -30.036418) --  (axis cs:269.003,    -30.024279) --  (axis cs:269.503,    -30.018207) ;
  \draw[constellation-boundary]  (axis cs:245.891,    -24.878120) --  (axis cs:246.894,    -24.867057) --  (axis cs:247.896,    -24.855908) --  (axis cs:248.898,    -24.844676) --  (axis cs:249.9,    -24.833365) --  (axis cs:250.902,    -24.821978) --  (axis cs:251.904,    -24.810519) --  (axis cs:252.905,    -24.798991) --  (axis cs:253.156,    -24.796099) ;
  \draw[constellation-boundary]  (axis cs:245.891,    -24.878120) --  (axis cs:245.884,    -24.294829) --  (axis cs:245.871,    -23.294901) --  (axis cs:245.858,    -22.294972) --  (axis cs:245.845,    -21.295042) --  (axis cs:245.832,    -20.295112) --  (axis cs:245.823,    -19.545163) ;
  \draw[constellation-boundary]  (axis cs:245.823,    -19.545163) --  (axis cs:246.825,    -19.534106) --  (axis cs:247.451,    -19.527151) ;
  \draw[constellation-boundary]  (axis cs:247.451,    -19.527151) --  (axis cs:247.448,    -19.277169) --  (axis cs:247.438,    -18.527221) ;
  \draw[constellation-boundary]  (axis cs:245.811,    -18.545231) --  (axis cs:246.812,    -18.534175) --  (axis cs:247.438,    -18.527221) ;
  \draw[constellation-boundary]  (axis cs:245.811,    -18.545231) --  (axis cs:245.808,    -18.295247) --  (axis cs:245.795,    -17.295314) --  (axis cs:245.783,    -16.295380) --  (axis cs:245.772,    -15.295445) --  (axis cs:245.76,    -14.295510) --  (axis cs:245.748,    -13.295574) --  (axis cs:245.737,    -12.295637) --  (axis cs:245.725,    -11.295700) --  (axis cs:245.714,    -10.295763) --  (axis cs:245.702,    -09.295825) --  (axis cs:245.691,    -08.295887) ;
  \draw[constellation-boundary]  (axis cs:245.638,    -03.546177) --  (axis cs:245.636,    -03.296193) --  (axis cs:245.625,    -02.296253) --  (axis cs:245.614,    -01.296314) --  (axis cs:245.603,    -00.296374) --  (axis cs:245.592,    +00.703566) --  (axis cs:245.581,    +01.703505) --  (axis cs:245.57,    +02.703445) --  (axis cs:245.559,    +03.703384) ;
  \draw[constellation-boundary]  (axis cs:290.121,    +19.398292) --  (axis cs:290.111,    +20.231564) --  (axis cs:290.097,    +21.231489) --  (axis cs:290.096,    +21.314816) ;
  \draw[constellation-boundary]  (axis cs:290.121,    +19.398292) --  (axis cs:290.372,    +19.401156) --  (axis cs:291.373,    +19.412567) --  (axis cs:292.374,    +19.423903) --  (axis cs:293.376,    +19.435161) --  (axis cs:294.378,    +19.446338) ;
  \draw[constellation-boundary]  (axis cs:284.794,    -15.832816) --  (axis cs:284.781,    -14.832890) --  (axis cs:284.768,    -13.832965) --  (axis cs:284.756,    -12.833038) --  (axis cs:284.744,    -11.833111) --  (axis cs:284.731,    -10.833183) --  (axis cs:284.719,    -09.833255) --  (axis cs:284.707,    -08.833327) --  (axis cs:284.695,    -07.833398) --  (axis cs:284.683,    -06.833469) --  (axis cs:284.671,    -05.833539) --  (axis cs:284.659,    -04.833609) --  (axis cs:284.647,    -03.833679) ;
  \draw[constellation-boundary]  (axis cs:275.55,    -15.943573) --  (axis cs:275.536,    -14.943652) --  (axis cs:275.524,    -13.943730) --  (axis cs:275.511,    -12.943807) --  (axis cs:275.498,    -11.943884) --  (axis cs:275.485,    -10.943961) --  (axis cs:275.473,    -09.944036) --  (axis cs:275.46,    -08.944112) --  (axis cs:275.448,    -07.944187) --  (axis cs:275.436,    -06.944261) --  (axis cs:275.424,    -05.944336) --  (axis cs:275.411,    -04.944410) --  (axis cs:275.399,    -03.944483) ;
  \draw[constellation-boundary]  (axis cs:275.399,    -03.944483) --  (axis cs:275.649,    -03.941456) --  (axis cs:276.649,    -03.929361) --  (axis cs:277.649,    -03.917286) --  (axis cs:278.649,    -03.905237) --  (axis cs:279.648,    -03.893217) --  (axis cs:280.648,    -03.881230) --  (axis cs:281.648,    -03.869279) --  (axis cs:282.648,    -03.857368) --  (axis cs:283.647,    -03.845500) --  (axis cs:284.647,    -03.833679) ;
  \draw[constellation-boundary]  (axis cs:175.556,    -35.693207) --  (axis cs:176.565,    -35.694211) --  (axis cs:177.573,    -35.695001) --  (axis cs:178.582,    -35.695579) --  (axis cs:179.591,    -35.695942) ;
  \draw[constellation-boundary]  (axis cs:179.604,    +13.304056) ;
  \draw[constellation-boundary]  (axis cs:179.097,    -11.695789) --  (axis cs:179.598,    -11.695943) ;
  \draw[constellation-boundary]  (axis cs:179.091,    -25.195788) --  (axis cs:179.092,    -24.695788) --  (axis cs:179.092,    -23.695788) --  (axis cs:179.092,    -22.695788) --  (axis cs:179.093,    -21.695788) --  (axis cs:179.093,    -20.695788) --  (axis cs:179.094,    -19.695788) --  (axis cs:179.094,    -18.695789) --  (axis cs:179.094,    -17.695789) --  (axis cs:179.095,    -16.695789) --  (axis cs:179.095,    -15.695789) --  (axis cs:179.096,    -14.695789) --  (axis cs:179.096,    -13.695789) --  (axis cs:179.096,    -12.695789) --  (axis cs:179.097,    -11.695789) --  (axis cs:179.097,    -10.695789) --  (axis cs:179.098,    -09.695789) --  (axis cs:179.098,    -08.695789) --  (axis cs:179.098,    -07.695789) --  (axis cs:179.099,    -06.695789) ;
  \draw[constellation-boundary]  (axis cs:175.594,    -06.693228) --  (axis cs:176.595,    -06.694224) --  (axis cs:177.597,    -06.695009) --  (axis cs:178.598,    -06.695582) --  (axis cs:179.099,    -06.695789) ;
  \draw[constellation-boundary]  (axis cs:175.572,    -25.193215) --  (axis cs:176.577,    -25.194216) --  (axis cs:177.583,    -25.195005) --  (axis cs:178.589,    -25.195580) --  (axis cs:179.594,    -25.195943) ;
  \draw[constellation-boundary]  (axis cs:175.635,    +28.306750) --  (axis cs:176.629,    +28.305761) --  (axis cs:177.622,    +28.304982) --  (axis cs:178.615,    +28.304414) --  (axis cs:179.609,    +28.304056) ;
  \draw[constellation-boundary]  (axis cs:179.604,    +10.304056) --  (axis cs:179.604,    +11.304056) --  (axis cs:179.604,    +12.304056) --  (axis cs:179.604,    +13.304056) --  (axis cs:179.605,    +14.304056) --  (axis cs:179.605,    +15.304056) --  (axis cs:179.605,    +16.304056) --  (axis cs:179.606,    +17.304056) --  (axis cs:179.606,    +18.304056) --  (axis cs:179.606,    +19.304056) --  (axis cs:179.606,    +20.304056) --  (axis cs:179.607,    +21.304056) --  (axis cs:179.607,    +22.304056) --  (axis cs:179.607,    +23.304056) --  (axis cs:179.608,    +24.304056) --  (axis cs:179.608,    +25.304056) --  (axis cs:179.608,    +26.304056) --  (axis cs:179.609,    +27.304056) --  (axis cs:179.609,    +28.304056) ;
  \draw[constellation-boundary]  (axis cs:175.613,    +10.306762) --  (axis cs:176.611,    +10.305769) --  (axis cs:177.608,    +10.304987) --  (axis cs:178.606,    +10.304416) --  (axis cs:179.604,    +10.304056) ;
  

\node[constellation-label] at (axis cs:{191,34})  {\large \textls{Canes}}; 
\node[constellation-label] at (axis cs:{191,36})  {\large \textls{Venatici}};

\node[constellation-label] at (axis cs:{188,21.5})  {\large \textls{Coma}}; 
\node[constellation-label] at (axis cs:{188,23.2})  {\large \textls{Berenices}};

\node[constellation-label][rotate=-90] at (axis cs:{180.8,35})  {{Ursa Major}};
\node[constellation-label] at (axis cs:{186,-20})   {\Large \textls{Corvus}};
\node[constellation-label] at (axis cs:{205,-35})   {\Huge \textls{Centaurus}};
\node[constellation-label] at (axis cs:{205,-15})   {\Huge \textls{Virgo}};
\node[constellation-label] at (axis cs:{198,-27})   {\huge \textls{Hydra}};
\node[constellation-label] at (axis cs:{285.8,8})  {\Large \textls{Aquila}};
\node[constellation-label] at (axis cs:{218,23})   {\huge \textls{Bootes}};
\node[constellation-label] at (axis cs:{280,-38})  {{Corona Australis}};
\node[constellation-label] at (axis cs:{238,28.5})    {\large \textls{Corona}}; 
  \node[constellation-label] at (axis cs:{238,31.5})    {\large \textls{Borealis}};
\node[constellation-label] at (axis cs:{260,18.5})   {\Huge \textls{Hercules}};
\node[constellation-label] at (axis cs:{234,-13})   {\huge \textls{Libra}};
\node[constellation-label] at (axis cs:{235,-32})   {\large \textls{Lupus}};
\node[constellation-label] at (axis cs:{283,35})   {\huge \textls{Lyra}};
\node[constellation-label] at (axis cs:{258,2.5})   {\Huge \textls{Ophiuchus}};
\node[constellation-label] at (axis cs:{253,-32})   {\huge \textls{Scorpius}};
\node[constellation-label] at (axis cs:{280,-12})   {\large \textls{Scutum}};
\node[constellation-label] at (axis cs:{279,-28})   {\Large \textls{Sagittarius}};
\node[constellation-label] at (axis cs:{288,19.2})  {{Sagitta}};
\node[constellation-label] at (axis cs:{287,23.8})  {{Vulpecula}};


\node[constellation-label][rotate=60] at (axis cs:{240,5})  {\Large \textls{Caput}};
\node[constellation-label][rotate=60] at (axis cs:{233.5,7})  {\Large \textls{Serpens}};

\node[constellation-label][rotate=-68] at (axis cs:{273.3,-7.3})  {\Large \textls{Serpens Cauda}};
%\node[constellation-label][rotate=-60] at (axis cs:{273,-10})  {\Large \textls{Cauda}};


\end{axis}


% Nebulae and nebulae names


\begin{axis}[name=NGC,axis lines=none]



\node[NGC,pin={[pin distance=-1.2\onedegree,NGC-label]0:{4038}}] at (axis cs:180.475,-18.867) {\tikz\pgfuseplotmark{GAL};}; % 10.7
\node[NGC,pin={[pin distance=-1.2\onedegree,NGC-label]0:{4151}}] at (axis cs:182.625,39.400) {\tikz\pgfuseplotmark{GAL};}; % 10.4
\node[NGC,pin={[pin distance=-1.2\onedegree,NGC-label]90:{4179}}] at (axis cs:183.225,1.300) {\tikz\pgfuseplotmark{GAL};}; % 10.9
\node[NGC,pin={[pin distance=-1.2\onedegree,NGC-label]-90:{4203}}] at (axis cs:183.775,33.200) {\tikz\pgfuseplotmark{GAL};}; % 10.7
\node[NGC,pin={[pin distance=-1.2\onedegree,NGC-label]90:{4214}}] at (axis cs:183.900,36.333) {\tikz\pgfuseplotmark{GAL};}; %  9.7
\node[NGC,pin={[pin distance=-1.2\onedegree,NGC-label]180:{4216}}] at (axis cs:183.975,13.150) {\tikz\pgfuseplotmark{GAL};}; % 10.0
\node[NGC,pin={[pin distance=-1.2\onedegree,NGC-label]90:{4244}}] at (axis cs:184.375,37.817) {\tikz\pgfuseplotmark{GAL};}; % 10.2
\node[NGC,pin={[pin distance=-1.2\onedegree,NGC-label]-90:{4261}}] at (axis cs:184.850,5.817) {\tikz\pgfuseplotmark{GAL};}; % 10.3
\node[NGC,pin={[pin distance=-1.2\onedegree,NGC-label]180:{4267}}] at (axis cs:184.950,12.800) {\tikz\pgfuseplotmark{GAL};}; % 10.9
\node[NGC,pin={[pin distance=-1.2\onedegree,NGC-label]180:{4274}}] at (axis cs:184.950,29.617) {\tikz\pgfuseplotmark{GAL};}; % 10.4
\node[NGC,pin={[pin distance=-1.2\onedegree,NGC-label]0:{4278}}] at (axis cs:185.025,29.283) {\tikz\pgfuseplotmark{GAL};}; % 10.2
\node[NGC,pin={[pin distance=-1.2\onedegree,NGC-label]-90:{4314}}] at (axis cs:185.650,29.883) {\tikz\pgfuseplotmark{GAL};}; % 10.5
\node[NGC,pin={[pin distance=-1.2\onedegree,NGC-label]-90:{4371}}] at (axis cs:186.225,11.700) {\tikz\pgfuseplotmark{GAL};}; % 10.8
\node[NGC,pin={[pin distance=-1.2\onedegree,NGC-label]-90:{4395}}] at (axis cs:186.450,33.550) {\tikz\pgfuseplotmark{GAL};}; % 10.2
\node[NGC,pin={[pin distance=-1.2\onedegree,NGC-label]90:{}}] at (axis cs:186.475,18.217) {\tikz\pgfuseplotmark{GAL};}; % 10.9
\node[NGC,pin={[pin distance=-1.2\onedegree,NGC-label]180:{4414}}] at (axis cs:186.600,31.217) {\tikz\pgfuseplotmark{GAL};}; % 10.3
\node[NGC,pin={[pin distance=-1.2\onedegree,NGC-label]90:{4429}}] at (axis cs:186.850,11.117) {\tikz\pgfuseplotmark{GAL};}; % 10.2
\node[NGC,pin={[pin distance=-1.2\onedegree,NGC-label]90:{}}] at (axis cs:186.925,13.083) {\tikz\pgfuseplotmark{GAL};}; % 10.9
\node[NGC,pin={[pin distance=-1.2\onedegree,NGC-label]90:{}}] at (axis cs:186.950,13.017) {\tikz\pgfuseplotmark{GAL};}; % 10.1
\node[NGC,pin={[pin distance=-1.2\onedegree,NGC-label]90:{4442}}] at (axis cs:187.025,9.800) {\tikz\pgfuseplotmark{GAL};}; % 10.5
\node[NGC,pin={[pin distance=-1.2\onedegree,NGC-label]-90:{4450}}] at (axis cs:187.125,17.083) {\tikz\pgfuseplotmark{GAL};}; % 10.1
\node[NGC,pin={[pin distance=-1.2\onedegree,NGC-label]90:{4457}}] at (axis cs:187.250,3.567) {\tikz\pgfuseplotmark{GAL};}; % 10.8
\node[NGC,pin={[pin distance=-1.2\onedegree,NGC-label]90:{}}] at (axis cs:187.250,13.983) {\tikz\pgfuseplotmark{GAL};}; % 10.4
\node[NGC,pin={[pin distance=-1.2\onedegree,NGC-label]90:{}}] at (axis cs:187.450,13.433) {\tikz\pgfuseplotmark{GAL};}; % 10.2
\node[NGC,pin={[pin distance=-1.2\onedegree,NGC-label]90:{}}] at (axis cs:187.500,13.633) {\tikz\pgfuseplotmark{GAL};}; % 10.4
\node[NGC,pin={[pin distance=-1.2\onedegree,NGC-label]-90:{4494}}] at (axis cs:187.850,25.783) {\tikz\pgfuseplotmark{GAL};}; %  9.9
\node[NGC,pin={[pin distance=-1.2\onedegree,NGC-label]90:{4517}}] at (axis cs:188.200,0.117) {\tikz\pgfuseplotmark{GAL};}; % 10.5
\node[NGC,pin={[pin distance=-1.2\onedegree,NGC-label]0:{4526}}] at (axis cs:188.500,7.700) {\tikz\pgfuseplotmark{GAL};}; %  9.6
\node[NGC,pin={[pin distance=-1.2\onedegree,NGC-label]-90:{4527}}] at (axis cs:188.525,2.650) {\tikz\pgfuseplotmark{GAL};}; % 10.4
\node[NGC,pin={[pin distance=-1.2\onedegree,NGC-label]0:{4535}}] at (axis cs:188.575,8.200) {\tikz\pgfuseplotmark{GAL};}; %  9.8
\node[NGC,pin={[pin distance=-1.2\onedegree,NGC-label]90:{4536}}] at (axis cs:188.625,2.183) {\tikz\pgfuseplotmark{GAL};}; % 10.4
\node[NGC,pin={[pin distance=-1.2\onedegree,NGC-label]90:{4546}}] at (axis cs:188.875,-3.800) {\tikz\pgfuseplotmark{GAL};}; % 10.3
\node[NGC,pin={[pin distance=-1.2\onedegree,NGC-label]90:{4559}}] at (axis cs:189,27.967) {\tikz\pgfuseplotmark{GAL};}; %  9.9
\node[NGC,pin={[pin distance=-1.2\onedegree,NGC-label]90:{4565}}] at (axis cs:189.075,25.983) {\tikz\pgfuseplotmark{GAL};}; %  9.6
\node[NGC,pin={[pin distance=-1.2\onedegree,NGC-label]90:{4568}}] at (axis cs:189.150,11.233) {\tikz\pgfuseplotmark{GAL};}; % 10.8
\node[NGC,pin={[pin distance=-1.2\onedegree,NGC-label]90:{4570}}] at (axis cs:189.225,7.250) {\tikz\pgfuseplotmark{GAL};}; % 10.9
\node[NGC,pin={[pin distance=-1.2\onedegree,NGC-label]90:{4596}}] at (axis cs:189.975,10.183) {\tikz\pgfuseplotmark{GAL};}; % 10.5
\node[NGC,pin={[pin distance=-1.2\onedegree,NGC-label]-90:{4631}}] at (axis cs:190.525,32.533) {\tikz\pgfuseplotmark{GAL};}; %  9.3
\node[NGC,pin={[pin distance=-1.2\onedegree,NGC-label]-90:{4636}}] at (axis cs:190.700,2.683) {\tikz\pgfuseplotmark{GAL};}; %  9.6
\node[NGC,pin={[pin distance=-1.2\onedegree,NGC-label]00:{4643}}] at (axis cs:190.825,1.983) {\tikz\pgfuseplotmark{GAL};}; % 10.6
\node[NGC,pin={[pin distance=-1.2\onedegree,NGC-label]-90:{4651}}] at (axis cs:190.925,16.400) {\tikz\pgfuseplotmark{GAL};}; % 10.7
\node[NGC,pin={[pin distance=-1.2\onedegree,NGC-label]0:{4654}}] at (axis cs:191,13.133) {\tikz\pgfuseplotmark{GAL};}; % 10.5
\node[NGC,pin={[pin distance=-1.2\onedegree,NGC-label]0:{4656}}] at (axis cs:191,32.167) {\tikz\pgfuseplotmark{GAL};}; % 10.4
\node[NGC,pin={[pin distance=-1.2\onedegree,NGC-label]0:{4666}}] at (axis cs:191.275,-0.467) {\tikz\pgfuseplotmark{GAL};}; % 10.8
\node[NGC,pin={[pin distance=-1.2\onedegree,NGC-label]0:{4689}}] at (axis cs:191.950,13.767) {\tikz\pgfuseplotmark{GAL};}; % 10.9
\node[NGC,pin={[pin distance=-1.2\onedegree,NGC-label]90:{4698}}] at (axis cs:192.100,8.483) {\tikz\pgfuseplotmark{GAL};}; % 10.7
\node[NGC,pin={[pin distance=-1.2\onedegree,NGC-label]90:{4697}}] at (axis cs:192.150,-5.800) {\tikz\pgfuseplotmark{GAL};}; %  9.3
\node[NGC,pin={[pin distance=-1.2\onedegree,NGC-label]90:{4699}}] at (axis cs:192.250,-8.667) {\tikz\pgfuseplotmark{GAL};}; %  9.6
\node[NGC,pin={[pin distance=-1.2\onedegree,NGC-label]180:{4725}}] at (axis cs:192.600,25.500) {\tikz\pgfuseplotmark{GAL};}; %  9.2
\node[NGC,pin={[pin distance=-1.2\onedegree,NGC-label]-90:{4754}}] at (axis cs:193.075,11.317) {\tikz\pgfuseplotmark{GAL};}; % 10.6
\node[NGC,pin={[pin distance=-1.2\onedegree,NGC-label]90:{4753}}] at (axis cs:193.100,-1.200) {\tikz\pgfuseplotmark{GAL};}; %  9.9
\node[NGC,pin={[pin distance=-1.2\onedegree,NGC-label]0:{4762}}] at (axis cs:193.225,11.233) {\tikz\pgfuseplotmark{GAL};}; % 10.2
\node[NGC,pin={[pin distance=-1.2\onedegree,NGC-label]90:{4856}}] at (axis cs:194.825,-15.033) {\tikz\pgfuseplotmark{GAL};}; % 10.4
\node[NGC,pin={[pin distance=-1.2\onedegree,NGC-label]90:{4958}}] at (axis cs:196.450,-8.017) {\tikz\pgfuseplotmark{GAL};}; % 10.5
\node[NGC,pin={[pin distance=-1.2\onedegree,NGC-label]180:{5005}}] at (axis cs:197.725,37.050) {\tikz\pgfuseplotmark{GAL};}; %  9.8
\node[NGC,pin={[pin distance=-1.2\onedegree,NGC-label]180:{5018}}] at (axis cs:198.250,-19.517) {\tikz\pgfuseplotmark{GAL};}; % 10.8
\node[NGC,pin={[pin distance=-1.2\onedegree,NGC-label]90:{5033}}] at (axis cs:198.350,36.600) {\tikz\pgfuseplotmark{GAL};}; % 10.1


\node[NGC,pin={[pin distance=-1.2\onedegree,NGC-label]00:{ 5102}}] at (axis cs:200.500,-36.633) {\tikz\pgfuseplotmark{GAL};}; %  9.7
\node[NGC,pin={[pin distance=-1.2\onedegree,NGC-label]-90:{I4296}}] at (axis cs:204.150,-33.967) {\tikz\pgfuseplotmark{GAL};}; % 10.6
\node[NGC,pin={[pin distance=-1.2\onedegree,NGC-label]00:{ 5248}}] at (axis cs:204.375,8.883) {\tikz\pgfuseplotmark{GAL};}; % 10.2
\node[NGC,pin={[pin distance=-1.2\onedegree,NGC-label]90:{ 5247}}] at (axis cs:204.525,-17.883) {\tikz\pgfuseplotmark{GAL};}; % 10.5
\node[NGC,pin={[pin distance=-1.2\onedegree,NGC-label]90:{ 5253}}] at (axis cs:204.975,-31.650) {\tikz\pgfuseplotmark{GAL};}; % 10.6
\node[NGC,pin={[pin distance=-1.2\onedegree,NGC-label]-90:{ 5363}}] at (axis cs:209.025,5.250) {\tikz\pgfuseplotmark{GAL};}; % 10.2
\node[NGC,pin={[pin distance=-1.2\onedegree,NGC-label]90:{ 5364}}] at (axis cs:209.050,5.017) {\tikz\pgfuseplotmark{GAL};}; % 10.4
\node[NGC,pin={[pin distance=-1.2\onedegree,NGC-label]-90:{ 5566}}] at (axis cs:215.075,3.933) {\tikz\pgfuseplotmark{GAL};}; % 10.5
\node[NGC,pin={[pin distance=-1.2\onedegree,NGC-label]90:{ 5576}}] at (axis cs:215.275,3.267) {\tikz\pgfuseplotmark{GAL};}; % 10.9
\node[NGC,pin={[pin distance=-1.2\onedegree,NGC-label]180:{ 5746}}] at (axis cs:221.225,1.950) {\tikz\pgfuseplotmark{GAL};}; % 10.6
\node[NGC,pin={[pin distance=-1.2\onedegree,NGC-label]180:{ 5813}}] at (axis cs:225.300,1.700) {\tikz\pgfuseplotmark{GAL};}; % 10.7
\node[NGC,pin={[pin distance=-1.2\onedegree,NGC-label]-90:{ 5838}}] at (axis cs:226.350,2.100) {\tikz\pgfuseplotmark{GAL};}; % 10.8
\node[NGC,pin={[pin distance=-1.2\onedegree,NGC-label]90:{ 5846}}] at (axis cs:226.600,1.600) {\tikz\pgfuseplotmark{GAL};}; % 10.2
\node[NGC,pin={[pin distance=-1.2\onedegree,NGC-label]00:{ 5921}}] at (axis cs:230.475,5.067) {\tikz\pgfuseplotmark{GAL};}; % 10.8
\node[NGC,pin={[pin distance=-1.2\onedegree,NGC-label]90:{ 6384}}] at (axis cs:263.100,7.067) {\tikz\pgfuseplotmark{GAL};}; % 10.6
\node[NGC,pin={[pin distance=-1.2\onedegree,NGC-label]90:{ 4361}}] at (axis cs:186.125,-18.800) {\tikz\pgfuseplotmark{PN};}; % 10. 
\node[NGC,pin={[pin distance=-1.2\onedegree,NGC-label]90:{ 6210}}] at (axis cs:251.125,23.817) {\tikz\pgfuseplotmark{PN};}; %  9. 
\node[NGC,pin={[pin distance=-1.2\onedegree,NGC-label]90:{ 6572}}] at (axis cs:273.025,6.850) {\tikz\pgfuseplotmark{PN};}; %  9. 
\node[NGC,pin={[pin distance=-1.2\onedegree,NGC-label]90:{ 6281}}] at (axis cs:256.200,-37.900) {\tikz\pgfuseplotmark{CN};}; %  5.4
\node[NGC,pin={[pin distance=-1.2\onedegree,NGC-label]90:{ 6383}}] at (axis cs:263.700,-32.567) {\tikz\pgfuseplotmark{CN};}; %  5.5
\node[NGC,pin={[pin distance=-1.2\onedegree,NGC-label]90:{ 6604}}] at (axis cs:274.525,-12.233) {\tikz\pgfuseplotmark{CN};}; %  6.5
\node[NGC,pin={[pin distance=-1.2\onedegree,NGC-label]-90:{ 6242}}] at (axis cs:253.900,-39.500) {\tikz\pgfuseplotmark{OC};}; %  6.4
\node[NGC,pin={[pin distance=-1.2\onedegree,NGC-label]90:{ 6268}}] at (axis cs:255.600,-39.733) {\tikz\pgfuseplotmark{OC};}; % 10. 
\node[NGC,pin={[pin distance=-1.2\onedegree,NGC-label]180:{ 6374}}] at (axis cs:263.075,-32.600) {\tikz\pgfuseplotmark{OC};}; %  9.0
\node[NGC,pin={[pin distance=-1.2\onedegree,NGC-label]90:{ 6396}}] at (axis cs:264.525,-35) {\tikz\pgfuseplotmark{OC};}; %  8.5
\node[NGC,pin={[pin distance=-1.2\onedegree,NGC-label]-90:{ 6400}}] at (axis cs:265.200,-36.950) {\tikz\pgfuseplotmark{OC};}; %  9. 
\node[NGC,pin={[pin distance=-1.2\onedegree,NGC-label]90:{ 6416}}] at (axis cs:266.100,-32.350) {\tikz\pgfuseplotmark{OC};}; %  5.7
\node[NGC,pin={[pin distance=-1.2\onedegree,NGC-label]-90:{I4665}}] at (axis cs:266.575,5.717) {\tikz\pgfuseplotmark{OC};}; %  4.2
\node[NGC,pin={[pin distance=-1.2\onedegree,NGC-label]-90:{ 6425}}] at (axis cs:266.725,-31.533) {\tikz\pgfuseplotmark{OC};}; %  7.2
\node[NGC,pin={[pin distance=-1.2\onedegree,NGC-label]00:{ 6451}}] at (axis cs:267.675,-30.217) {\tikz\pgfuseplotmark{OC};}; %  8. 
\node[NGC,pin={[pin distance=-1.2\onedegree,NGC-label]-90:{ 6469}}] at (axis cs:268.225,-22.350) {\tikz\pgfuseplotmark{OC};}; %  8. 
\node[NGC,pin={[pin distance=-1.2\onedegree,NGC-label]90:{ 6507}}] at (axis cs:269.900,-17.400) {\tikz\pgfuseplotmark{OC};}; % 10. 
\node[NGC,pin={[pin distance=-1.2\onedegree,NGC-label]90:{ 6520}}] at (axis cs:270.850,-27.900) {\tikz\pgfuseplotmark{OC};}; %  8. 
\node[NGC,pin={[pin distance=-1.2\onedegree,NGC-label]00:{ 6530}}] at (axis cs:271.200,-24.333) {\tikz\pgfuseplotmark{OC};}; %  4.6
\node[NGC,pin={[pin distance=-1.2\onedegree,NGC-label]90:{ 6546}}] at (axis cs:271.800,-23.333) {\tikz\pgfuseplotmark{OC};}; %  8.0
\node[NGC,pin={[pin distance=-1.2\onedegree,NGC-label]90:{ 6568}}] at (axis cs:273.200,-21.600) {\tikz\pgfuseplotmark{OC};}; %  9. 
\node[NGC,pin={[pin distance=-1.2\onedegree,NGC-label]90:{ 6583}}] at (axis cs:273.950,-22.133) {\tikz\pgfuseplotmark{OC};}; % 10. 
\node[NGC,pin={[pin distance=-1.2\onedegree,NGC-label]-90:{ 6595}}] at (axis cs:274.250,-19.883) {\tikz\pgfuseplotmark{OC};}; %  7. 
\node[NGC,pin={[pin distance=-1.2\onedegree,NGC-label]90:{ 6605}}] at (axis cs:274.275,-14.967) {\tikz\pgfuseplotmark{OC};}; %  6. 
\node[NGC,pin={[pin distance=-1.2\onedegree,NGC-label]-135:{ 6625}}] at (axis cs:275.800,-12.050) {\tikz\pgfuseplotmark{OC};}; %  9. 
\node[NGC,pin={[pin distance=-1.2\onedegree,NGC-label]-90:{ 6633}}] at (axis cs:276.925,6.567) {\tikz\pgfuseplotmark{OC};}; %  4.6
\node[NGC,pin={[pin distance=-1.2\onedegree,NGC-label]180:{ 6647}}] at (axis cs:277.875,-17.350) {\tikz\pgfuseplotmark{OC};}; %  8. 
\node[NGC,pin={[pin distance=-1.2\onedegree,NGC-label]-90:{ 6645}}] at (axis cs:278.150,-16.900) {\tikz\pgfuseplotmark{OC};}; %  9. 
\node[NGC,pin={[pin distance=-1.2\onedegree,NGC-label]00:{ 6649}}] at (axis cs:278.375,-10.400) {\tikz\pgfuseplotmark{OC};}; %  8.9
\node[NGC,pin={[pin distance=-1.2\onedegree,NGC-label]90:{ 6664}}] at (axis cs:279.175,-8.217) {\tikz\pgfuseplotmark{OC};}; %  7.8
\node[NGC,pin={[pin distance=-1.2\onedegree,NGC-label]180:{I4756}}] at (axis cs:279.750,5.450) {\tikz\pgfuseplotmark{OC};}; %  5. 
\node[NGC,pin={[pin distance=-1.2\onedegree,NGC-label]00:{ 6683}}] at (axis cs:280.550,-6.283) {\tikz\pgfuseplotmark{OC};}; % 10. 
\node[NGC,pin={[pin distance=-1.2\onedegree,NGC-label]0:{ 6704}}] at (axis cs:282.725,-5.200) {\tikz\pgfuseplotmark{OC};}; %  9.2
\node[NGC,pin={[pin distance=-1.2\onedegree,NGC-label]90:{ 6709}}] at (axis cs:282.875,10.350) {\tikz\pgfuseplotmark{OC};}; %  6.7
\node[NGC,pin={[pin distance=-1.2\onedegree,NGC-label]-90:{ 6716}}] at (axis cs:283.650,-19.883) {\tikz\pgfuseplotmark{OC};}; %  6.9
\node[NGC,pin={[pin distance=-1.2\onedegree,NGC-label]90:{ 6738}}] at (axis cs:285.350,11.600) {\tikz\pgfuseplotmark{OC};}; %  8. 
\node[NGC,pin={[pin distance=-1.2\onedegree,NGC-label]90:{ 6755}}] at (axis cs:286.950,4.233) {\tikz\pgfuseplotmark{OC};}; %  7.5
\node[NGC,pin={[pin distance=-1.2\onedegree,NGC-label]90:{ 4147}}] at (axis cs:182.525,18.550) {\tikz\pgfuseplotmark{GC};}; % 10.3
\node[NGC,pin={[pin distance=-1.2\onedegree,NGC-label]90:{ 5053}}] at (axis cs:199.100,17.700) {\tikz\pgfuseplotmark{GC};}; %  9.8
\node[NGC,pin={[pin distance=-1.2\onedegree,NGC-label]90:{ 5466}}] at (axis cs:211.375,28.533) {\tikz\pgfuseplotmark{GC};}; %  9.1
\node[NGC,pin={[pin distance=-1.2\onedegree,NGC-label]-90:{ 5634}}] at (axis cs:217.400,-5.983) {\tikz\pgfuseplotmark{GC};}; %  9.6
\node[NGC,pin={[pin distance=-1.2\onedegree,NGC-label]00:{ 5694}}] at (axis cs:219.900,-26.533) {\tikz\pgfuseplotmark{GC};}; % 10.2
\node[NGC,pin={[pin distance=-1.2\onedegree,NGC-label]00:{ 5824}}] at (axis cs:226,-33.067) {\tikz\pgfuseplotmark{GC};}; %  9.0
\node[NGC,pin={[pin distance=-1.2\onedegree,NGC-label]90:{ 5897}}] at (axis cs:229.350,-21.017) {\tikz\pgfuseplotmark{GC};}; %  8.6
\node[NGC,pin={[pin distance=-1.2\onedegree,NGC-label]90:{ 5986}}] at (axis cs:236.525,-37.783) {\tikz\pgfuseplotmark{GC};}; %  7.1
\node[NGC,pin={[pin distance=-1.2\onedegree,NGC-label]-90:{ 6144}}] at (axis cs:246.825,-26.033) {\tikz\pgfuseplotmark{GC};}; %  9.1
\node[NGC,pin={[pin distance=-1.2\onedegree,NGC-label]90:{ 6139}}] at (axis cs:246.925,-38.850) {\tikz\pgfuseplotmark{GC};}; %  9.2
\node[NGC,pin={[pin distance=-1.2\onedegree,NGC-label]-90:{ 6235}}] at (axis cs:253.350,-22.183) {\tikz\pgfuseplotmark{GC};}; % 10.2
\node[NGC,pin={[pin distance=-1.2\onedegree,NGC-label]90:{ 6284}}] at (axis cs:256.125,-24.767) {\tikz\pgfuseplotmark{GC};}; %  9.0
\node[NGC,pin={[pin distance=-1.2\onedegree,NGC-label]-90:{ 6287}}] at (axis cs:256.300,-22.700) {\tikz\pgfuseplotmark{GC};}; %  9.2
\node[NGC,pin={[pin distance=-1.2\onedegree,NGC-label]90:{ 6293}}] at (axis cs:257.550,-26.583) {\tikz\pgfuseplotmark{GC};}; %  8.2
\node[NGC,pin={[pin distance=-1.2\onedegree,NGC-label]180:{ 6304}}] at (axis cs:258.625,-29.467) {\tikz\pgfuseplotmark{GC};}; %  8.4
\node[NGC,pin={[pin distance=-1.2\onedegree,NGC-label]90:{ 6316}}] at (axis cs:259.150,-28.133) {\tikz\pgfuseplotmark{GC};}; %  9.0
\node[NGC,pin={[pin distance=-1.2\onedegree,NGC-label]180:{ 6325}}] at (axis cs:259.500,-23.767) {\tikz\pgfuseplotmark{GC};}; % 10.7
\node[NGC,pin={[pin distance=-1.2\onedegree,NGC-label]180:{ 6342}}] at (axis cs:260.300,-19.583) {\tikz\pgfuseplotmark{GC};}; %  9.9
\node[NGC,pin={[pin distance=-1.2\onedegree,NGC-label]-90:{ 6356}}] at (axis cs:260.900,-17.817) {\tikz\pgfuseplotmark{GC};}; %  8.4
\node[NGC,pin={[pin distance=-1.2\onedegree,NGC-label]90:{ 6355}}] at (axis cs:261,-26.350) {\tikz\pgfuseplotmark{GC};}; %  9.6
\node[NGC,pin={[pin distance=-1.2\onedegree,NGC-label]-90:{ 6366}}] at (axis cs:261.925,-5.083) {\tikz\pgfuseplotmark{GC};}; % 10.0
\node[NGC,pin={[pin distance=-1.2\onedegree,NGC-label]90:{ 6401}}] at (axis cs:264.650,-23.917) {\tikz\pgfuseplotmark{GC};}; %  9.5
\node[NGC,pin={[pin distance=-1.2\onedegree,NGC-label]90:{ 6440}}] at (axis cs:267.225,-20.367) {\tikz\pgfuseplotmark{GC};}; %  9.7
\node[NGC,pin={[pin distance=-1.2\onedegree,NGC-label]90:{ 6441}}] at (axis cs:267.550,-37.050) {\tikz\pgfuseplotmark{GC};}; %  7.4
\node[NGC,pin={[pin distance=-1.2\onedegree,NGC-label]180:{ 6453}}] at (axis cs:267.725,-34.600) {\tikz\pgfuseplotmark{GC};}; %  9.9
\node[NGC,pin={[pin distance=-1.2\onedegree,NGC-label]180:{ 6517}}] at (axis cs:270.450,-8.967) {\tikz\pgfuseplotmark{GC};}; % 10.3

%\node[NGC,pin={[pin distance=-1.2\onedegree,NGC-label]90:{ 6522}}] at (axis cs:270.900,-30.033) {\tikz\pgfuseplotmark{GC};}; %  8.6
\node[NGC] at (axis cs:270.900,-30.033) {\tikz\pgfuseplotmark{GC};}; %  8.6

\node[NGC,pin={[pin distance=-1.2\onedegree,NGC-label]90:{ 6535}}] at (axis cs:270.950,-0.300) {\tikz\pgfuseplotmark{GC};}; % 10.6
\node[NGC,pin={[pin distance=-1.6\onedegree,NGC-label]-135:{ 6522/8}}] at (axis cs:271.200,-30.050) {\tikz\pgfuseplotmark{GC};}; %  9.5
\node[NGC,pin={[pin distance=-1.2\onedegree,NGC-label]0:{ 6539}}] at (axis cs:271.200,-7.583) {\tikz\pgfuseplotmark{GC};}; %  9.6
\node[NGC,pin={[pin distance=-1.2\onedegree,NGC-label]0:{ 6544}}] at (axis cs:271.825,-25) {\tikz\pgfuseplotmark{GC};}; %  8.3
\node[NGC,pin={[pin distance=-1.2\onedegree,NGC-label]180:{ 6553}}] at (axis cs:272.325,-25.900) {\tikz\pgfuseplotmark{GC};}; %  8.3
\node[NGC,pin={[pin distance=-1.2\onedegree,NGC-label]90:{ 6569}}] at (axis cs:273.400,-31.833) {\tikz\pgfuseplotmark{GC};}; %  8.7
\node[NGC,pin={[pin distance=-1.2\onedegree,NGC-label]00:{ 6624}}] at (axis cs:275.925,-30.367) {\tikz\pgfuseplotmark{GC};}; %  8.3
\node[NGC,pin={[pin distance=-1.2\onedegree,NGC-label]90:{ 6638}}] at (axis cs:277.725,-25.500) {\tikz\pgfuseplotmark{GC};}; %  9.2
\node[NGC,pin={[pin distance=-1.2\onedegree,NGC-label]-90:{ 6642}}] at (axis cs:277.975,-23.483) {\tikz\pgfuseplotmark{GC};}; %  8.8
\node[NGC,pin={[pin distance=-1.2\onedegree,NGC-label]270:{ 6652}}] at (axis cs:278.950,-32.983) {\tikz\pgfuseplotmark{GC};}; %  8.9
\node[NGC,pin={[pin distance=-1.2\onedegree,NGC-label]90:{ 6712}}] at (axis cs:283.275,-8.700) {\tikz\pgfuseplotmark{GC};}; %  8.2
\node[NGC,pin={[pin distance=-1.2\onedegree,NGC-label]-90:{ 6723}}] at (axis cs:284.900,-36.633) {\tikz\pgfuseplotmark{GC};}; %  7.3
\node[NGC,pin={[pin distance=-1.2\onedegree,NGC-label]90:{ 6760}}] at (axis cs:287.800,1.033) {\tikz\pgfuseplotmark{GC};}; %  9.1



\node[Messier,pin={[pin distance=-0.8\onedegree,Messier-label]90:{M64}}] at (axis cs:194.175,21.683) {\tikz\pgfuseplotmark{GAL};}; %  8.5
\node[Messier,pin={[pin distance=-0.8\onedegree,Messier-label]-90:{M53}}] at (axis cs:198.225,18.167) {\tikz\pgfuseplotmark{GC};}; %  7.7
\node[Messier,pin={[pin distance=-0.8\onedegree,Messier-label]-90:{M85}}] at (axis cs:186.350,18.183) {\tikz\pgfuseplotmark{GAL};}; %  9.2
\node[Messier,pin={[pin distance=-1.2\onedegree,Messier-label]-90:{M100}}] at (axis cs:185.725,15.817) {\tikz\pgfuseplotmark{GAL};}; %  9.4
\node[Messier,pin={[pin distance=-0.8\onedegree,Messier-label]180:{M98}}] at (axis cs:183.450,14.900) {\tikz\pgfuseplotmark{GAL};}; % 10.1
\node[Messier,pin={[pin distance=-0.8\onedegree,Messier-label]180:{M49}}] at (axis cs:187.450,8) {\tikz\pgfuseplotmark{GAL};}; %  8.4
\node[Messier,pin={[pin distance=-0.8\onedegree,Messier-label]180:{M61}}] at (axis cs:185.475,4.467) {\tikz\pgfuseplotmark{GAL};}; %  9.7
\node[Messier,pin={[pin distance=-0.8\onedegree,Messier-label]-90:{M104}}] at (axis cs:190,-11.617) {\tikz\pgfuseplotmark{GAL};}; %  8.3
\node[Messier,pin={[pin distance=-0.8\onedegree,Messier-label]00:{M68}}] at (axis cs:189.875,-26.750) {\tikz\pgfuseplotmark{GC};}; %  8.2




\node[Messier] at (axis cs:188,14.417) {\tikz\pgfuseplotmark{GAL};}; %  9.5
\node[Messier] at (axis cs:188.850,14.500) {\tikz\pgfuseplotmark{GAL};}; % 10.2
\node[Messier] at (axis cs:189.425,11.817) {\tikz\pgfuseplotmark{GAL};}; %  9.8
\node[Messier] at (axis cs:190.500,11.650) {\tikz\pgfuseplotmark{GAL};}; %  9.8
\node[Messier] at (axis cs:190.925,11.550) {\tikz\pgfuseplotmark{GAL};}; %  8.8
\node[Messier] at (axis cs:186.275,12.883) {\tikz\pgfuseplotmark{GAL};}; %  9.3
\node[Messier] at (axis cs:186.550,12.950) {\tikz\pgfuseplotmark{GAL};}; %  9.2
\node[Messier] at (axis cs:187.700,12.400) {\tikz\pgfuseplotmark{GAL};}; %  8.6
\node[Messier] at (axis cs:188.925,12.550) {\tikz\pgfuseplotmark{GAL};}; %  9.8
\node[Messier] at (axis cs:184.700,14.417) {\tikz\pgfuseplotmark{GAL};}; %  9.8
\node[Messier] at (axis cs:189.200,13.167) {\tikz\pgfuseplotmark{GAL};}; %  9.5



\node[pin={[pin distance=0.75\onedegree,Messier-label-crowded]-90:{M88}}] at (axis cs:188,14.417)          [circle,inner sep=0pt,opacity=0]{};
\node[pin={[pin distance=0.5\onedegree,Messier-label-crowded]-35:{M91}}] at (axis cs:188.850,14.500)  [circle,inner sep=0pt,opacity=0]{};
\node[pin={[pin distance=1.1\onedegree,Messier-label-crowded]90:{M46}}] at (axis cs:115.450,-14.817)    [circle,inner sep=0pt,opacity=0]{};
\node[pin={[pin distance=0.65\onedegree,Messier-label-crowded]90:{M47}}] at (axis cs:114.150,-14.500) [circle,inner sep=0pt,opacity=0]{};
\node[pin={[pin distance=0.\onedegree,Messier-label-crowded]180:{M58}}] at (axis cs:189.425,11.817)      [circle,inner sep=0pt,opacity=0]{};
\node[pin={[pin distance=2\onedegree,Messier-label-crowded]-20:{M59}}] at (axis cs:190.500,11.650)     [circle,inner sep=0pt,opacity=0]{};
\node[pin={[pin distance=.5\onedegree,Messier-label-crowded]-285:{M60}}] at (axis cs:190.925,11.550)  [circle,inner sep=0pt,opacity=0]{};
\node[pin={[pin distance=1\onedegree,Messier-label-crowded]-202:{M84}}] at (axis cs:186.275,12.883)  [circle,inner sep=0pt,opacity=0]{};
\node[pin={[pin distance=0.25\onedegree,Messier-label-crowded]-135:{M86}}] at (axis cs:186.550,12.950)   [circle,inner sep=0pt,opacity=0]{};
\node[pin={[pin distance=1.4\onedegree,Messier-label-crowded]135:{M87}}] at (axis cs:187.700,12.400)  [circle,inner sep=0pt,opacity=0]{};
\node[pin={[pin distance=0.3\onedegree,Messier-label-crowded]0:{M89}}] at (axis cs:188.925,12.550)  [circle,inner sep=0pt,opacity=0]{};
\node[pin={[pin distance=0.75\onedegree,Messier-label-crowded]-105:{M99}}] at (axis cs:184.700,14.417)   [circle,inner sep=0pt,opacity=0]{};
\node[pin={[pin distance=0.5\onedegree,Messier-label-crowded]280:{M90}}] at (axis cs:189.200,13.167)     [circle,inner sep=0pt,opacity=0]{};


\node[interest,pin={[pin distance=-1.5\onedegree,interest-label]270:{Coma cluster}}] at (axis cs:195,27.97) {\tikz\draw[dashed]  circle (1.5\onedegree) ;}; 

\node[interest,pin={[pin distance=-1.5\onedegree,interest-label]90:{Virgo cluster}}] at (axis cs:186.6337,+12.7233) {\tikz\draw[dashed]  circle (8\onedegree) ;}; 

\node[Messier,pin={[pin distance=-0.8\onedegree,Messier-label]00:{M13}}] at (axis cs:250.425,36.467) {\tikz\pgfuseplotmark{GC};}; %  5.9
\node[Messier,pin={[pin distance=-0.8\onedegree,Messier-label]90:{M57}}] at (axis cs:283.400,33.033) {\tikz\pgfuseplotmark{PN};}; %  9.0
\node[Messier,pin={[pin distance=-0.8\onedegree,Messier-label]90:{M56}}] at (axis cs:289.150,30.183) {\tikz\pgfuseplotmark{GC};}; %  8.3
\node[Messier,pin={[pin distance=-0.8\onedegree,Messier-label]90:{M3}}] at (axis cs:205.550,28.383) {\tikz\pgfuseplotmark{GC};}; %  6.4
\node[Messier,pin={[pin distance=-0.8\onedegree,Messier-label]180:{M5}}] at (axis cs:229.650,2.083) {\tikz\pgfuseplotmark{GC};}; %  5.8
\node[Messier,pin={[pin distance=-0.8\onedegree,Messier-label]90:{M12}}] at (axis cs:251.800,-1.950) {\tikz\pgfuseplotmark{GC};}; %  6.6
\node[Messier,pin={[pin distance=-0.8\onedegree,Messier-label]90:{M14}}] at (axis cs:264.400,-3.250) {\tikz\pgfuseplotmark{GC};}; %  7.6
\node[Messier,pin={[pin distance=-0.8\onedegree,Messier-label]90:{M10}}] at (axis cs:254.275,-4.100) {\tikz\pgfuseplotmark{GC};}; %  6.6
\node[Messier,pin={[pin distance=-0.8\onedegree,Messier-label]90:{M11}}] at (axis cs:282.775,-6.267) {\tikz\pgfuseplotmark{OC};}; %  5.8
\node[Messier,pin={[pin distance=-0.8\onedegree,Messier-label]90:{M26}}] at (axis cs:281.300,-9.400) {\tikz\pgfuseplotmark{OC};}; %  8.0
\node[Messier,pin={[pin distance=-0.8\onedegree,Messier-label]180:{M16}}] at (axis cs:274.700,-13.783) {\tikz\pgfuseplotmark{CN};}; %  6.0
\node[Messier,pin={[pin distance=-0.8\onedegree,Messier-label]90:{M107}}] at (axis cs:248.125,-13.050) {\tikz\pgfuseplotmark{GC};}; %  8.1
\node[Messier,pin={[pin distance=-0.8\onedegree,Messier-label]0:{M17}}] at (axis cs:275.200,-16.183) {\tikz\pgfuseplotmark{CN};}; %  6.0
\node[Messier,pin={[pin distance=-0.8\onedegree,Messier-label]90:{M18}}] at (axis cs:274.975,-17.133) {\tikz\pgfuseplotmark{OC};}; %  6.9
\node[Messier,pin={[pin distance=-0.8\onedegree,Messier-label]180:{M9}}] at (axis cs:259.800,-18.517) {\tikz\pgfuseplotmark{GC};}; %  7.9
\node[Messier,pin={[pin distance=-0.8\onedegree,Messier-label]90:{M23}}] at (axis cs:269.200,-19.017) {\tikz\pgfuseplotmark{OC};}; %  5.5
\node[Messier,pin={[pin distance=-0.8\onedegree,Messier-label]-90:{M21}}] at (axis cs:271.150,-22.500) {\tikz\pgfuseplotmark{OC};}; %  5.9
\node[Messier,pin={[pin distance=-0.8\onedegree,Messier-label]-90:{M80}}] at (axis cs:244.250,-22.983) {\tikz\pgfuseplotmark{GC};}; %  7.2
\node[Messier,pin={[pin distance=-0.8\onedegree,Messier-label]180:{M20}}] at (axis cs:270.575,-23.033) {\tikz\pgfuseplotmark{CN};}; %  6.3
\node[Messier,pin={[pin distance=-0.8\onedegree,Messier-label]0:{M22}}] at (axis cs:279.100,-23.900) {\tikz\pgfuseplotmark{GC};}; %  5.1
\node[Messier,pin={[pin distance=-0.8\onedegree,Messier-label]180:{M8}}] at (axis cs:270.950,-24.383) {\tikz\pgfuseplotmark{EN};}; %  5.8
\node[Messier,pin={[pin distance=-0.8\onedegree,Messier-label]-90:{M28}}] at (axis cs:276.125,-24.867) {\tikz\pgfuseplotmark{GC};}; %  6.9
\node[Messier,pin={[pin distance=-0.8\onedegree,Messier-label]90:{M4}}] at (axis cs:245.900,-26.533) {\tikz\pgfuseplotmark{GC};}; %  5.9
\node[Messier,pin={[pin distance=-0.8\onedegree,Messier-label]90:{M19}}] at (axis cs:255.650,-26.267) {\tikz\pgfuseplotmark{GC};}; %  7.2
\node[Messier,pin={[pin distance=-0.8\onedegree,Messier-label]-90:{M83}}] at (axis cs:204.250,-29.867) {\tikz\pgfuseplotmark{GAL};}; %  7.6
\node[Messier,pin={[pin distance=-0.8\onedegree,Messier-label]180:{M54}}] at (axis cs:283.775,-30.483) {\tikz\pgfuseplotmark{GC};}; %  7.7
\node[Messier,pin={[pin distance=-0.8\onedegree,Messier-label]90:{M62}}] at (axis cs:255.300,-30.117) {\tikz\pgfuseplotmark{GC};}; %  6.6
\node[Messier,pin={[pin distance=-0.8\onedegree,Messier-label]-90:{M6}}] at (axis cs:265.025,-32.217) {\tikz\pgfuseplotmark{OC};}; %  4.2
\node[Messier,pin={[pin distance=-0.8\onedegree,Messier-label]-90:{M69}}] at (axis cs:277.850,-32.350) {\tikz\pgfuseplotmark{GC};}; %  7.7
\node[Messier,pin={[pin distance=-0.8\onedegree,Messier-label]90:{M70}}] at (axis cs:280.800,-32.300) {\tikz\pgfuseplotmark{GC};}; %  8.1
\node[Messier,pin={[pin distance=-0.8\onedegree,Messier-label]0:{M7}}] at (axis cs:268.475,-34.817) {\tikz\pgfuseplotmark{OC};}; %  3.3
\node[Messier,pin={[pin distance=-0.8\onedegree,Messier-label]90:{M25}}] at (axis cs:277.900,-19.250) {\tikz\pgfuseplotmark{OC};}; %  4.6





\end{axis}

%
% Stars and star names
\begin{axis}[name=stars,axis lines=none]


\node[stars] at (axis cs:{180.176},{-21.837}) {\tikz\pgfuseplotmark{m6c};};
\node[stars] at (axis cs:{180.185},{-10.446}) {\tikz\pgfuseplotmark{m6a};};
\node[stars] at (axis cs:{180.213},{-19.659}) {\tikz\pgfuseplotmark{m5cv};};
\node[stars] at (axis cs:{180.218},{6.614}) {\tikz\pgfuseplotmark{m5a};};
\node[stars] at (axis cs:{180.257},{-1.768}) {\tikz\pgfuseplotmark{m6c};};
\node[stars] at (axis cs:{180.414},{36.042}) {\tikz\pgfuseplotmark{m6a};};
\node[stars] at (axis cs:{180.715},{-7.684}) {\tikz\pgfuseplotmark{m6c};};
\node[stars] at (axis cs:{180.935},{5.558}) {\tikz\pgfuseplotmark{m6c};};
\node[stars] at (axis cs:{181.069},{21.459}) {\tikz\pgfuseplotmark{m6bb};};
\node[stars] at (axis cs:{181.302},{8.733}) {\tikz\pgfuseplotmark{m4b};};
\node[stars] at (axis cs:{181.486},{-35.694}) {\tikz\pgfuseplotmark{m6cv};};
\node[stars] at (axis cs:{181.499},{-3.131}) {\tikz\pgfuseplotmark{m6c};};
\node[stars] at (axis cs:{182.103},{-24.729}) {\tikz\pgfuseplotmark{m4b};};
\node[stars] at (axis cs:{182.422},{1.898}) {\tikz\pgfuseplotmark{m6b};};
\node[stars] at (axis cs:{182.511},{-34.705}) {\tikz\pgfuseplotmark{m6cb};};
\node[stars] at (axis cs:{182.514},{5.807}) {\tikz\pgfuseplotmark{m6a};};
\node[stars] at (axis cs:{182.531},{-22.620}) {\tikz\pgfuseplotmark{m3bv};};
\node[stars] at (axis cs:{182.632},{16.809}) {\tikz\pgfuseplotmark{m6cv};};
\node[stars] at (axis cs:{182.641},{-37.870}) {\tikz\pgfuseplotmark{m6b};};
\node[stars] at (axis cs:{182.692},{27.281}) {\tikz\pgfuseplotmark{m6bv};};
\node[stars] at (axis cs:{182.766},{-23.602}) {\tikz\pgfuseplotmark{m5c};};
\node[stars] at (axis cs:{182.963},{25.870}) {\tikz\pgfuseplotmark{m6a};};
\node[stars] at (axis cs:{183.004},{28.536}) {\tikz\pgfuseplotmark{m6c};};
\node[stars] at (axis cs:{183.039},{20.542}) {\tikz\pgfuseplotmark{m6a};};
\node[stars] at (axis cs:{183.304},{-34.125}) {\tikz\pgfuseplotmark{m6cv};};
\node[stars] at (axis cs:{183.354},{-38.929}) {\tikz\pgfuseplotmark{m6a};};
\node[stars] at (axis cs:{183.358},{10.262}) {\tikz\pgfuseplotmark{m6a};};
\node[stars] at (axis cs:{183.403},{-33.793}) {\tikz\pgfuseplotmark{m6cb};};
\node[stars] at (axis cs:{183.748},{-20.844}) {\tikz\pgfuseplotmark{m6a};};
\node[stars] at (axis cs:{183.794},{-10.312}) {\tikz\pgfuseplotmark{m6b};};
\node[stars] at (axis cs:{183.952},{-17.542}) {\tikz\pgfuseplotmark{m3av};};
\node[stars] at (axis cs:{184.001},{14.899}) {\tikz\pgfuseplotmark{m5b};};
\node[stars] at (axis cs:{184.086},{23.945}) {\tikz\pgfuseplotmark{m5b};};
\node[stars] at (axis cs:{184.126},{33.061}) {\tikz\pgfuseplotmark{m5b};};
\node[stars] at (axis cs:{184.264},{-16.694}) {\tikz\pgfuseplotmark{m6bv};};
\node[stars] at (axis cs:{184.377},{28.937}) {\tikz\pgfuseplotmark{m6a};};
\node[stars] at (axis cs:{184.434},{15.145}) {\tikz\pgfuseplotmark{m6c};};
\node[stars] at (axis cs:{184.447},{-36.094}) {\tikz\pgfuseplotmark{m6c};};
\node[stars] at (axis cs:{184.632},{30.249}) {\tikz\pgfuseplotmark{m6c};};
\node[stars] at (axis cs:{184.668},{-0.787}) {\tikz\pgfuseplotmark{m6b};};
\node[stars] at (axis cs:{184.758},{26.008}) {\tikz\pgfuseplotmark{m6cv};};
\node[stars] at (axis cs:{184.830},{23.035}) {\tikz\pgfuseplotmark{m6c};};
\node[stars] at (axis cs:{184.873},{28.157}) {\tikz\pgfuseplotmark{m6c};};
\node[stars] at (axis cs:{184.976},{-0.667}) {\tikz\pgfuseplotmark{m4bv};};
\node[stars] at (axis cs:{185.045},{-22.176}) {\tikz\pgfuseplotmark{m6b};};
\node[stars] at (axis cs:{185.074},{26.002}) {\tikz\pgfuseplotmark{m6cv};};
\node[stars] at (axis cs:{185.082},{26.619}) {\tikz\pgfuseplotmark{m6av};};
\node[stars] at (axis cs:{185.087},{3.312}) {\tikz\pgfuseplotmark{m5bv};};
\node[stars] at (axis cs:{185.140},{-22.216}) {\tikz\pgfuseplotmark{m5cvb};};
\node[stars] at (axis cs:{185.179},{17.793}) {\tikz\pgfuseplotmark{m5a};};
\node[stars] at (axis cs:{185.203},{15.541}) {\tikz\pgfuseplotmark{m6c};};
\node[stars] at (axis cs:{185.232},{-13.565}) {\tikz\pgfuseplotmark{m5cv};};
\node[stars] at (axis cs:{185.545},{24.774}) {\tikz\pgfuseplotmark{m6c};};
\node[stars] at (axis cs:{185.626},{25.846}) {\tikz\pgfuseplotmark{m5av};};
\node[stars] at (axis cs:{185.633},{5.305}) {\tikz\pgfuseplotmark{m6cb};};
\node[stars] at (axis cs:{185.814},{-4.974}) {\tikz\pgfuseplotmark{m6c};};
\node[stars] at (axis cs:{185.829},{-11.812}) {\tikz\pgfuseplotmark{m6cv};};
\node[stars] at (axis cs:{185.840},{-24.841}) {\tikz\pgfuseplotmark{m6a};};
\node[stars] at (axis cs:{185.898},{-35.412}) {\tikz\pgfuseplotmark{m5c};};
\node[stars] at (axis cs:{185.903},{-39.302}) {\tikz\pgfuseplotmark{m6c};};
\node[stars] at (axis cs:{185.936},{-38.911}) {\tikz\pgfuseplotmark{m6a};};
\node[stars] at (axis cs:{185.949},{-30.335}) {\tikz\pgfuseplotmark{m6c};};
\node[stars] at (axis cs:{186.077},{26.098}) {\tikz\pgfuseplotmark{m5cv};};
\node[stars] at (axis cs:{186.299},{-11.610}) {\tikz\pgfuseplotmark{m6b};};
\node[stars] at (axis cs:{186.313},{23.926}) {\tikz\pgfuseplotmark{m6b};};
\node[stars] at (axis cs:{186.327},{-27.749}) {\tikz\pgfuseplotmark{m6b};};
\node[stars] at (axis cs:{186.341},{-35.186}) {\tikz\pgfuseplotmark{m6av};};
\node[stars] at (axis cs:{186.462},{39.019}) {\tikz\pgfuseplotmark{m5b};};
\node[stars] at (axis cs:{186.600},{27.268}) {\tikz\pgfuseplotmark{m5b};};
\node[stars] at (axis cs:{186.715},{-32.830}) {\tikz\pgfuseplotmark{m6a};};
\node[stars] at (axis cs:{186.734},{28.268}) {\tikz\pgfuseplotmark{m4c};};
\node[stars] at (axis cs:{186.747},{26.826}) {\tikz\pgfuseplotmark{m5b};};
\node[stars] at (axis cs:{186.925},{8.610}) {\tikz\pgfuseplotmark{m6c};};
\node[stars] at (axis cs:{186.956},{-16.632}) {\tikz\pgfuseplotmark{m6c};};
\node[stars] at (axis cs:{186.965},{-4.615}) {\tikz\pgfuseplotmark{m6cv};};
\node[stars] at (axis cs:{187.094},{-39.041}) {\tikz\pgfuseplotmark{m5c};};
\node[stars] at (axis cs:{187.228},{25.913}) {\tikz\pgfuseplotmark{m5cvb};};
\node[stars] at (axis cs:{187.363},{24.109}) {\tikz\pgfuseplotmark{m5c};};
\node[stars] at (axis cs:{187.430},{20.896}) {\tikz\pgfuseplotmark{m6a};};
\node[stars] at (axis cs:{187.466},{-16.515}) {\tikz\pgfuseplotmark{m3bv};};
\node[stars] at (axis cs:{187.520},{-13.393}) {\tikz\pgfuseplotmark{m6cb};};
\node[stars] at (axis cs:{187.573},{-23.696}) {\tikz\pgfuseplotmark{m6a};};
\node[stars] at (axis cs:{187.752},{24.567}) {\tikz\pgfuseplotmark{m5cv};};
\node[stars] at (axis cs:{187.839},{7.604}) {\tikz\pgfuseplotmark{m6b};};
\node[stars] at (axis cs:{187.911},{-5.053}) {\tikz\pgfuseplotmark{m6c};};
\node[stars] at (axis cs:{188.018},{-16.196}) {\tikz\pgfuseplotmark{m4cv};};
\node[stars] at (axis cs:{188.018},{-32.534}) {\tikz\pgfuseplotmark{m6c};};
\node[stars] at (axis cs:{188.137},{-21.212}) {\tikz\pgfuseplotmark{m6c};};
\node[stars] at (axis cs:{188.150},{-13.859}) {\tikz\pgfuseplotmark{m6a};};
\node[stars] at (axis cs:{188.262},{10.295}) {\tikz\pgfuseplotmark{m6c};};
\node[stars] at (axis cs:{188.343},{-19.792}) {\tikz\pgfuseplotmark{m6cv};};
\node[stars] at (axis cs:{188.393},{24.283}) {\tikz\pgfuseplotmark{m6c};};
\node[stars] at (axis cs:{188.393},{-12.830}) {\tikz\pgfuseplotmark{m6a};};
\node[stars] at (axis cs:{188.412},{33.247}) {\tikz\pgfuseplotmark{m5c};};
\node[stars] at (axis cs:{188.445},{-9.452}) {\tikz\pgfuseplotmark{m5c};};
\node[stars] at (axis cs:{188.448},{33.385}) {\tikz\pgfuseplotmark{m6c};};
\node[stars] at (axis cs:{188.597},{-23.397}) {\tikz\pgfuseplotmark{m3av};};
\node[stars] at (axis cs:{188.713},{22.629}) {\tikz\pgfuseplotmark{m5a};};
\node[stars] at (axis cs:{188.782},{18.377}) {\tikz\pgfuseplotmark{m5bvb};};
\node[stars] at (axis cs:{188.784},{21.881}) {\tikz\pgfuseplotmark{m6a};};
\node[stars] at (axis cs:{188.995},{-20.527}) {\tikz\pgfuseplotmark{m6cv};};
\node[stars] at (axis cs:{189.004},{-39.869}) {\tikz\pgfuseplotmark{m6a};};
\node[stars] at (axis cs:{189.197},{-5.832}) {\tikz\pgfuseplotmark{m6b};};
\node[stars] at (axis cs:{189.243},{17.089}) {\tikz\pgfuseplotmark{m6a};};
\node[stars] at (axis cs:{189.426},{-27.139}) {\tikz\pgfuseplotmark{m5c};};
\node[stars] at (axis cs:{189.518},{3.282}) {\tikz\pgfuseplotmark{m6c};};
\node[stars] at (axis cs:{189.593},{1.855}) {\tikz\pgfuseplotmark{m6av};};
\node[stars] at (axis cs:{189.686},{-18.250}) {\tikz\pgfuseplotmark{m6b};};
\node[stars] at (axis cs:{189.759},{22.659}) {\tikz\pgfuseplotmark{m6c};};
\node[stars] at (axis cs:{189.764},{-30.422}) {\tikz\pgfuseplotmark{m6a};};
\node[stars] at (axis cs:{189.780},{21.062}) {\tikz\pgfuseplotmark{m5c};};
\node[stars] at (axis cs:{189.812},{-7.995}) {\tikz\pgfuseplotmark{m5ab};};
\node[stars] at (axis cs:{189.820},{35.952}) {\tikz\pgfuseplotmark{m6cv};};
\node[stars] at (axis cs:{189.969},{-39.987}) {\tikz\pgfuseplotmark{m5av};};
\node[stars] at (axis cs:{190.318},{-13.015}) {\tikz\pgfuseplotmark{m6bvb};};
\node[stars] at (axis cs:{190.393},{10.426}) {\tikz\pgfuseplotmark{m6cvb};};
\node[stars] at (axis cs:{190.415},{-1.449}) {\tikz\pgfuseplotmark{m3cvb};};
\node[stars] at (axis cs:{190.456},{-19.759}) {\tikz\pgfuseplotmark{m6b};};
\node[stars] at (axis cs:{190.471},{10.236}) {\tikz\pgfuseplotmark{m5bv};};
\node[stars] at (axis cs:{190.488},{6.807}) {\tikz\pgfuseplotmark{m6av};};
\node[stars] at (axis cs:{190.909},{-1.577}) {\tikz\pgfuseplotmark{m6b};};
\node[stars] at (axis cs:{190.994},{-36.349}) {\tikz\pgfuseplotmark{m6c};};
\node[stars] at (axis cs:{191.002},{-28.324}) {\tikz\pgfuseplotmark{m5c};};
\node[stars] at (axis cs:{191.248},{39.279}) {\tikz\pgfuseplotmark{m6b};};
\node[stars] at (axis cs:{191.404},{7.673}) {\tikz\pgfuseplotmark{m5cv};};
\node[stars] at (axis cs:{191.594},{9.540}) {\tikz\pgfuseplotmark{m6ab};};
\node[stars] at (axis cs:{191.662},{16.578}) {\tikz\pgfuseplotmark{m5b};};
\node[stars] at (axis cs:{191.693},{-33.315}) {\tikz\pgfuseplotmark{m6b};};
\node[stars] at (axis cs:{191.760},{5.950}) {\tikz\pgfuseplotmark{m6cv};};
\node[stars] at (axis cs:{191.807},{11.958}) {\tikz\pgfuseplotmark{m6bb};};
\node[stars] at (axis cs:{191.889},{-6.302}) {\tikz\pgfuseplotmark{m6c};};
\node[stars] at (axis cs:{191.964},{3.573}) {\tikz\pgfuseplotmark{m6cv};};
\node[stars] at (axis cs:{191.974},{-24.852}) {\tikz\pgfuseplotmark{m6c};};
\node[stars] at (axis cs:{192.060},{13.553}) {\tikz\pgfuseplotmark{m6c};};
\node[stars] at (axis cs:{192.109},{-27.597}) {\tikz\pgfuseplotmark{m6a};};
\node[stars] at (axis cs:{192.196},{24.840}) {\tikz\pgfuseplotmark{m6c};};
\node[stars] at (axis cs:{192.226},{14.122}) {\tikz\pgfuseplotmark{m6a};};
\node[stars] at (axis cs:{192.323},{27.552}) {\tikz\pgfuseplotmark{m6a};};
\node[stars] at (axis cs:{192.545},{37.517}) {\tikz\pgfuseplotmark{m6bv};};
\node[stars] at (axis cs:{192.572},{22.863}) {\tikz\pgfuseplotmark{m6c};};
\node[stars] at (axis cs:{192.672},{-33.999}) {\tikz\pgfuseplotmark{m5b};};
\node[stars] at (axis cs:{192.846},{-10.338}) {\tikz\pgfuseplotmark{m6cvb};};
\node[stars] at (axis cs:{192.904},{3.057}) {\tikz\pgfuseplotmark{m6b};};
\node[stars] at (axis cs:{192.925},{27.541}) {\tikz\pgfuseplotmark{m5bv};};
\node[stars] at (axis cs:{192.987},{-39.680}) {\tikz\pgfuseplotmark{m6bv};};
\node[stars] at (axis cs:{192.991},{-26.738}) {\tikz\pgfuseplotmark{m6cv};};
\node[stars] at (axis cs:{193.051},{17.074}) {\tikz\pgfuseplotmark{m6cb};};
\node[stars] at (axis cs:{193.115},{16.122}) {\tikz\pgfuseplotmark{m6c};};
\node[stars] at (axis cs:{193.296},{-3.553}) {\tikz\pgfuseplotmark{m6b};};
\node[stars] at (axis cs:{193.324},{21.245}) {\tikz\pgfuseplotmark{m5bb};};
\node[stars] at (axis cs:{193.384},{19.481}) {\tikz\pgfuseplotmark{m6c};};
\node[stars] at (axis cs:{193.409},{-4.225}) {\tikz\pgfuseplotmark{m6c};};
\node[stars] at (axis cs:{193.457},{12.418}) {\tikz\pgfuseplotmark{m6c};};
\node[stars] at (axis cs:{193.555},{33.534}) {\tikz\pgfuseplotmark{m6c};};
\node[stars] at (axis cs:{193.578},{-11.648}) {\tikz\pgfuseplotmark{m6b};};
\node[stars] at (axis cs:{193.588},{-9.539}) {\tikz\pgfuseplotmark{m5av};};
\node[stars] at (axis cs:{193.901},{3.397}) {\tikz\pgfuseplotmark{m3cv};};
\node[stars] at (axis cs:{193.972},{-15.327}) {\tikz\pgfuseplotmark{m6c};};
\node[stars] at (axis cs:{194.007},{38.318}) {\tikz\pgfuseplotmark{m3bvb};};
\node[stars] at (axis cs:{194.303},{-12.067}) {\tikz\pgfuseplotmark{m6c};};
\node[stars] at (axis cs:{194.388},{-22.753}) {\tikz\pgfuseplotmark{m6c};};
\node[stars] at (axis cs:{194.731},{17.409}) {\tikz\pgfuseplotmark{m5av};};
\node[stars] at (axis cs:{194.915},{-3.812}) {\tikz\pgfuseplotmark{m6a};};
\node[stars] at (axis cs:{195.069},{30.785}) {\tikz\pgfuseplotmark{m5bv};};
\node[stars] at (axis cs:{195.136},{-33.505}) {\tikz\pgfuseplotmark{m6b};};
\node[stars] at (axis cs:{195.150},{-3.368}) {\tikz\pgfuseplotmark{m6bv};};
\node[stars] at (axis cs:{195.161},{18.373}) {\tikz\pgfuseplotmark{m6cb};};
\node[stars] at (axis cs:{195.290},{17.123}) {\tikz\pgfuseplotmark{m6b};};
\node[stars] at (axis cs:{195.544},{10.959}) {\tikz\pgfuseplotmark{m3av};};
\node[stars] at (axis cs:{195.942},{-20.583}) {\tikz\pgfuseplotmark{m6a};};
\node[stars] at (axis cs:{196.435},{35.799}) {\tikz\pgfuseplotmark{m5cv};};
\node[stars] at (axis cs:{196.588},{21.153}) {\tikz\pgfuseplotmark{m6bb};};
\node[stars] at (axis cs:{196.594},{22.616}) {\tikz\pgfuseplotmark{m6av};};
\node[stars] at (axis cs:{196.726},{-35.862}) {\tikz\pgfuseplotmark{m6a};};
\node[stars] at (axis cs:{196.795},{27.625}) {\tikz\pgfuseplotmark{m5a};};
\node[stars] at (axis cs:{196.973},{27.556}) {\tikz\pgfuseplotmark{m6c};};
\node[stars] at (axis cs:{196.974},{-10.740}) {\tikz\pgfuseplotmark{m5c};};
\node[stars] at (axis cs:{197.135},{-8.984}) {\tikz\pgfuseplotmark{m6a};};
\node[stars] at (axis cs:{197.264},{-23.118}) {\tikz\pgfuseplotmark{m5b};};
\node[stars] at (axis cs:{197.302},{10.022}) {\tikz\pgfuseplotmark{m6a};};
\node[stars] at (axis cs:{197.309},{-9.538}) {\tikz\pgfuseplotmark{m6c};};
\node[stars] at (axis cs:{197.411},{37.423}) {\tikz\pgfuseplotmark{m6b};};
\node[stars] at (axis cs:{197.439},{-10.329}) {\tikz\pgfuseplotmark{m6b};};
\node[stars] at (axis cs:{197.449},{16.848}) {\tikz\pgfuseplotmark{m6b};};
\node[stars] at (axis cs:{197.487},{-5.539}) {\tikz\pgfuseplotmark{m4cb};};
\node[stars] at (axis cs:{197.497},{17.529}) {\tikz\pgfuseplotmark{m4cv};};
\node[stars] at (axis cs:{197.513},{38.499}) {\tikz\pgfuseplotmark{m6bvb};};
\node[stars] at (axis cs:{197.913},{-26.552}) {\tikz\pgfuseplotmark{m6c};};
\node[stars] at (axis cs:{197.968},{27.878}) {\tikz\pgfuseplotmark{m4cv};};
\node[stars] at (axis cs:{198.013},{-37.803}) {\tikz\pgfuseplotmark{m5a};};
\node[stars] at (axis cs:{198.015},{-16.198}) {\tikz\pgfuseplotmark{m5bvb};};
\node[stars] at (axis cs:{198.035},{24.258}) {\tikz\pgfuseplotmark{m6c};};
\node[stars] at (axis cs:{198.137},{11.556}) {\tikz\pgfuseplotmark{m6a};};
\node[stars] at (axis cs:{198.302},{18.727}) {\tikz\pgfuseplotmark{m6b};};
\node[stars] at (axis cs:{198.545},{-19.931}) {\tikz\pgfuseplotmark{m5c};};
\node[stars] at (axis cs:{198.630},{11.331}) {\tikz\pgfuseplotmark{m6av};};
\node[stars] at (axis cs:{198.789},{-36.371}) {\tikz\pgfuseplotmark{m6c};};
\node[stars] at (axis cs:{198.995},{-19.943}) {\tikz\pgfuseplotmark{m5c};};
\node[stars] at (axis cs:{199.060},{19.052}) {\tikz\pgfuseplotmark{m6c};};
\node[stars] at (axis cs:{199.134},{19.785}) {\tikz\pgfuseplotmark{m6c};};
\node[stars] at (axis cs:{199.194},{9.424}) {\tikz\pgfuseplotmark{m5c};};
\node[stars] at (axis cs:{199.221},{-31.506}) {\tikz\pgfuseplotmark{m5b};};
\node[stars] at (axis cs:{199.315},{13.676}) {\tikz\pgfuseplotmark{m5c};};
\node[stars] at (axis cs:{199.374},{-0.676}) {\tikz\pgfuseplotmark{m6cv};};
\node[stars] at (axis cs:{199.401},{5.470}) {\tikz\pgfuseplotmark{m5av};};
\node[stars] at (axis cs:{199.601},{-18.311}) {\tikz\pgfuseplotmark{m5a};};
\node[stars] at (axis cs:{199.616},{34.098}) {\tikz\pgfuseplotmark{m6a};};
\node[stars] at (axis cs:{199.730},{-23.171}) {\tikz\pgfuseplotmark{m3bv};};
\node[stars] at (axis cs:{200.149},{-36.712}) {\tikz\pgfuseplotmark{m3a};};
\node[stars] at (axis cs:{200.173},{2.942}) {\tikz\pgfuseplotmark{m6cb};};
\node[stars] at (axis cs:{200.374},{-19.489}) {\tikz\pgfuseplotmark{m6cv};};
\node[stars] at (axis cs:{200.424},{2.087}) {\tikz\pgfuseplotmark{m6a};};
\node[stars] at (axis cs:{200.540},{5.155}) {\tikz\pgfuseplotmark{m6b};};
\node[stars] at (axis cs:{200.755},{-17.735}) {\tikz\pgfuseplotmark{m5cv};};
\node[stars] at (axis cs:{200.786},{-33.190}) {\tikz\pgfuseplotmark{m6cv};};
\node[stars] at (axis cs:{200.829},{-4.924}) {\tikz\pgfuseplotmark{m6a};};
\node[stars] at (axis cs:{200.975},{37.034}) {\tikz\pgfuseplotmark{m6bv};};
\node[stars] at (axis cs:{200.988},{-20.924}) {\tikz\pgfuseplotmark{m6cb};};
\node[stars] at (axis cs:{201.127},{12.432}) {\tikz\pgfuseplotmark{m6c};};
\node[stars] at (axis cs:{201.138},{-5.164}) {\tikz\pgfuseplotmark{m6a};};
\node[stars] at (axis cs:{201.278},{23.854}) {\tikz\pgfuseplotmark{m6av};};
\node[stars] at (axis cs:{201.298},{-11.161}) {\tikz\pgfuseplotmark{m1bv};};
\node[stars] at (axis cs:{201.532},{-39.755}) {\tikz\pgfuseplotmark{m5b};};
\node[stars] at (axis cs:{201.548},{-1.192}) {\tikz\pgfuseplotmark{m6bv};};
\node[stars] at (axis cs:{201.680},{-12.708}) {\tikz\pgfuseplotmark{m5cv};};
\node[stars] at (axis cs:{201.863},{-15.973}) {\tikz\pgfuseplotmark{m5av};};
\node[stars] at (axis cs:{202.108},{13.779}) {\tikz\pgfuseplotmark{m5bb};};
\node[stars] at (axis cs:{202.304},{10.818}) {\tikz\pgfuseplotmark{m6a};};
\node[stars] at (axis cs:{202.312},{-1.364}) {\tikz\pgfuseplotmark{m6c};};
\node[stars] at (axis cs:{202.428},{-23.281}) {\tikz\pgfuseplotmark{m6cv};};
\node[stars] at (axis cs:{202.500},{7.179}) {\tikz\pgfuseplotmark{m6c};};
\node[stars] at (axis cs:{202.607},{-6.470}) {\tikz\pgfuseplotmark{m6b};};
\node[stars] at (axis cs:{202.761},{-39.407}) {\tikz\pgfuseplotmark{m4b};};
\node[stars] at (axis cs:{202.888},{-28.113}) {\tikz\pgfuseplotmark{m6c};};
\node[stars] at (axis cs:{202.991},{-6.256}) {\tikz\pgfuseplotmark{m5av};};
\node[stars] at (axis cs:{203.012},{-18.729}) {\tikz\pgfuseplotmark{m6bv};};
\node[stars] at (axis cs:{203.023},{-38.399}) {\tikz\pgfuseplotmark{m6c};};
\node[stars] at (axis cs:{203.144},{-29.565}) {\tikz\pgfuseplotmark{m6c};};
\node[stars] at (axis cs:{203.150},{-28.693}) {\tikz\pgfuseplotmark{m6a};};
\node[stars] at (axis cs:{203.201},{24.347}) {\tikz\pgfuseplotmark{m6bb};};
\node[stars] at (axis cs:{203.215},{-15.363}) {\tikz\pgfuseplotmark{m6a};};
\node[stars] at (axis cs:{203.242},{-10.165}) {\tikz\pgfuseplotmark{m5c};};
\node[stars] at (axis cs:{203.533},{3.659}) {\tikz\pgfuseplotmark{m5bv};};
\node[stars] at (axis cs:{203.591},{38.789}) {\tikz\pgfuseplotmark{m6c};};
\node[stars] at (axis cs:{203.669},{-13.214}) {\tikz\pgfuseplotmark{m6b};};
\node[stars] at (axis cs:{203.673},{-0.596}) {\tikz\pgfuseplotmark{m3c};};
\node[stars] at (axis cs:{203.682},{-33.311}) {\tikz\pgfuseplotmark{m6c};};
\node[stars] at (axis cs:{203.699},{37.182}) {\tikz\pgfuseplotmark{m5bv};};
\node[stars] at (axis cs:{203.880},{-5.396}) {\tikz\pgfuseplotmark{m6a};};
\node[stars] at (axis cs:{203.889},{10.205}) {\tikz\pgfuseplotmark{m6cb};};
\node[stars] at (axis cs:{204.202},{-26.495}) {\tikz\pgfuseplotmark{m6ab};};
\node[stars] at (axis cs:{204.211},{-34.468}) {\tikz\pgfuseplotmark{m6c};};
\node[stars] at (axis cs:{204.246},{24.613}) {\tikz\pgfuseplotmark{m6av};};
\node[stars] at (axis cs:{204.365},{36.295}) {\tikz\pgfuseplotmark{m5bb};};
\node[stars] at (axis cs:{204.675},{-29.561}) {\tikz\pgfuseplotmark{m6a};};
\node[stars] at (axis cs:{204.760},{18.265}) {\tikz\pgfuseplotmark{m6c};};
\node[stars] at (axis cs:{204.894},{10.746}) {\tikz\pgfuseplotmark{m6av};};
\node[stars] at (axis cs:{204.920},{-39.747}) {\tikz\pgfuseplotmark{m6cv};};
\node[stars] at (axis cs:{205.065},{31.012}) {\tikz\pgfuseplotmark{m6c};};
\node[stars] at (axis cs:{205.163},{28.065}) {\tikz\pgfuseplotmark{m6c};};
\node[stars] at (axis cs:{205.169},{19.956}) {\tikz\pgfuseplotmark{m6ab};};
\node[stars] at (axis cs:{205.260},{22.496}) {\tikz\pgfuseplotmark{m6a};};
\node[stars] at (axis cs:{205.403},{-8.703}) {\tikz\pgfuseplotmark{m5bv};};
\node[stars] at (axis cs:{205.553},{8.388}) {\tikz\pgfuseplotmark{m6cv};};
\node[stars] at (axis cs:{205.681},{34.989}) {\tikz\pgfuseplotmark{m6b};};
\node[stars] at (axis cs:{205.765},{3.538}) {\tikz\pgfuseplotmark{m5cb};};
\node[stars] at (axis cs:{205.938},{22.700}) {\tikz\pgfuseplotmark{m6b};};
\node[stars] at (axis cs:{206.124},{-16.179}) {\tikz\pgfuseplotmark{m6a};};
\node[stars] at (axis cs:{206.190},{-25.501}) {\tikz\pgfuseplotmark{m6c};};
\node[stars] at (axis cs:{206.306},{10.325}) {\tikz\pgfuseplotmark{m6c};};
\node[stars] at (axis cs:{206.396},{-15.767}) {\tikz\pgfuseplotmark{m6c};};
\node[stars] at (axis cs:{206.404},{-26.116}) {\tikz\pgfuseplotmark{m6a};};
\node[stars] at (axis cs:{206.422},{-33.044}) {\tikz\pgfuseplotmark{m4cv};};
\node[stars] at (axis cs:{206.485},{-12.426}) {\tikz\pgfuseplotmark{m5cb};};
\node[stars] at (axis cs:{206.579},{38.504}) {\tikz\pgfuseplotmark{m6b};};
\node[stars] at (axis cs:{206.681},{25.702}) {\tikz\pgfuseplotmark{m6b};};
\node[stars] at (axis cs:{206.735},{-36.252}) {\tikz\pgfuseplotmark{m5c};};
\node[stars] at (axis cs:{206.738},{6.350}) {\tikz\pgfuseplotmark{m6cb};};
\node[stars] at (axis cs:{206.749},{38.543}) {\tikz\pgfuseplotmark{m5cb};};
\node[stars] at (axis cs:{206.806},{-9.709}) {\tikz\pgfuseplotmark{m6b};};
\node[stars] at (axis cs:{206.816},{17.457}) {\tikz\pgfuseplotmark{m4cv};};
\node[stars] at (axis cs:{206.856},{-17.860}) {\tikz\pgfuseplotmark{m5cv};};
\node[stars] at (axis cs:{207.161},{31.190}) {\tikz\pgfuseplotmark{m6a};};
\node[stars] at (axis cs:{207.361},{-34.451}) {\tikz\pgfuseplotmark{m4cv};};
\node[stars] at (axis cs:{207.369},{15.798}) {\tikz\pgfuseplotmark{m4bv};};
\node[stars] at (axis cs:{207.428},{21.264}) {\tikz\pgfuseplotmark{m5b};};
\node[stars] at (axis cs:{207.437},{36.633}) {\tikz\pgfuseplotmark{m6c};};
\node[stars] at (axis cs:{207.468},{-18.134}) {\tikz\pgfuseplotmark{m5b};};
\node[stars] at (axis cs:{207.527},{-29.081}) {\tikz\pgfuseplotmark{m6cv};};
\node[stars] at (axis cs:{207.581},{-39.901}) {\tikz\pgfuseplotmark{m6c};};
\node[stars] at (axis cs:{207.603},{5.497}) {\tikz\pgfuseplotmark{m6b};};
\node[stars] at (axis cs:{207.788},{34.664}) {\tikz\pgfuseplotmark{m6bv};};
\node[stars] at (axis cs:{207.835},{-24.390}) {\tikz\pgfuseplotmark{m6c};};
\node[stars] at (axis cs:{207.902},{-36.433}) {\tikz\pgfuseplotmark{m6c};};
\node[stars] at (axis cs:{207.948},{34.444}) {\tikz\pgfuseplotmark{m5av};};
\node[stars] at (axis cs:{207.957},{-32.994}) {\tikz\pgfuseplotmark{m5avb};};
\node[stars] at (axis cs:{208.004},{-31.619}) {\tikz\pgfuseplotmark{m6bb};};
\node[stars] at (axis cs:{208.077},{12.165}) {\tikz\pgfuseplotmark{m6b};};
\node[stars] at (axis cs:{208.293},{28.648}) {\tikz\pgfuseplotmark{m6b};};
\node[stars] at (axis cs:{208.302},{-31.927}) {\tikz\pgfuseplotmark{m5ab};};
\node[stars] at (axis cs:{208.304},{17.933}) {\tikz\pgfuseplotmark{m6a};};
\node[stars] at (axis cs:{208.387},{-35.664}) {\tikz\pgfuseplotmark{m6ab};};
\node[stars] at (axis cs:{208.467},{-35.314}) {\tikz\pgfuseplotmark{m6c};};
\node[stars] at (axis cs:{208.569},{-28.570}) {\tikz\pgfuseplotmark{m6bv};};
\node[stars] at (axis cs:{208.671},{18.397}) {\tikz\pgfuseplotmark{m3a};};
\node[stars] at (axis cs:{208.676},{-1.503}) {\tikz\pgfuseplotmark{m5c};};
\node[stars] at (axis cs:{208.743},{-8.059}) {\tikz\pgfuseplotmark{m6cb};};
\node[stars] at (axis cs:{208.958},{14.056}) {\tikz\pgfuseplotmark{m6c};};
\node[stars] at (axis cs:{209.044},{32.032}) {\tikz\pgfuseplotmark{m6c};};
\node[stars] at (axis cs:{209.116},{1.051}) {\tikz\pgfuseplotmark{m6b};};
\node[stars] at (axis cs:{209.142},{27.492}) {\tikz\pgfuseplotmark{m5bv};};
\node[stars] at (axis cs:{209.366},{-23.023}) {\tikz\pgfuseplotmark{m6bv};};
\node[stars] at (axis cs:{209.495},{-25.998}) {\tikz\pgfuseplotmark{m6c};};
\node[stars] at (axis cs:{209.630},{-24.972}) {\tikz\pgfuseplotmark{m5c};};
\node[stars] at (axis cs:{209.662},{21.696}) {\tikz\pgfuseplotmark{m6a};};
\node[stars] at (axis cs:{209.666},{14.649}) {\tikz\pgfuseplotmark{m6b};};
\node[stars] at (axis cs:{209.955},{-3.550}) {\tikz\pgfuseplotmark{m6c};};
\node[stars] at (axis cs:{210.001},{-25.010}) {\tikz\pgfuseplotmark{m6a};};
\node[stars] at (axis cs:{210.294},{27.387}) {\tikz\pgfuseplotmark{m6c};};
\node[stars] at (axis cs:{210.335},{8.895}) {\tikz\pgfuseplotmark{m6b};};
\node[stars] at (axis cs:{210.403},{17.668}) {\tikz\pgfuseplotmark{m6c};};
\node[stars] at (axis cs:{210.412},{1.544}) {\tikz\pgfuseplotmark{m4cb};};
\node[stars] at (axis cs:{210.595},{-27.430}) {\tikz\pgfuseplotmark{m5c};};
\node[stars] at (axis cs:{210.633},{9.686}) {\tikz\pgfuseplotmark{m6cv};};
\node[stars] at (axis cs:{210.757},{-31.684}) {\tikz\pgfuseplotmark{m6cb};};
\node[stars] at (axis cs:{210.767},{-17.367}) {\tikz\pgfuseplotmark{m6c};};
\node[stars] at (axis cs:{210.885},{10.787}) {\tikz\pgfuseplotmark{m6c};};
\node[stars] at (axis cs:{210.903},{7.546}) {\tikz\pgfuseplotmark{m6c};};
\node[stars] at (axis cs:{210.971},{-22.422}) {\tikz\pgfuseplotmark{m6c};};
\node[stars] at (axis cs:{210.982},{4.901}) {\tikz\pgfuseplotmark{m6c};};
\node[stars] at (axis cs:{211.061},{-5.381}) {\tikz\pgfuseplotmark{m6c};};
\node[stars] at (axis cs:{211.112},{-14.972}) {\tikz\pgfuseplotmark{m6c};};
\node[stars] at (axis cs:{211.156},{2.297}) {\tikz\pgfuseplotmark{m6c};};
\node[stars] at (axis cs:{211.593},{-26.682}) {\tikz\pgfuseplotmark{m3c};};
\node[stars] at (axis cs:{211.671},{-36.370}) {\tikz\pgfuseplotmark{m2b};};
\node[stars] at (axis cs:{211.678},{-9.313}) {\tikz\pgfuseplotmark{m5c};};
\node[stars] at (axis cs:{212.252},{-10.335}) {\tikz\pgfuseplotmark{m6c};};
\node[stars] at (axis cs:{212.600},{25.092}) {\tikz\pgfuseplotmark{m5a};};
\node[stars] at (axis cs:{212.710},{-16.302}) {\tikz\pgfuseplotmark{m5bv};};
\node[stars] at (axis cs:{212.813},{32.296}) {\tikz\pgfuseplotmark{m6b};};
\node[stars] at (axis cs:{212.880},{1.362}) {\tikz\pgfuseplotmark{m6c};};
\node[stars] at (axis cs:{213.066},{2.409}) {\tikz\pgfuseplotmark{m5bv};};
\node[stars] at (axis cs:{213.102},{-24.363}) {\tikz\pgfuseplotmark{m6c};};
\node[stars] at (axis cs:{213.192},{-27.261}) {\tikz\pgfuseplotmark{m5b};};
\node[stars] at (axis cs:{213.224},{-10.274}) {\tikz\pgfuseplotmark{m4cv};};
\node[stars] at (axis cs:{213.305},{-26.612}) {\tikz\pgfuseplotmark{m6c};};
\node[stars] at (axis cs:{213.420},{-0.845}) {\tikz\pgfuseplotmark{m6bv};};
\node[stars] at (axis cs:{213.522},{12.959}) {\tikz\pgfuseplotmark{m6av};};
\node[stars] at (axis cs:{213.589},{-5.947}) {\tikz\pgfuseplotmark{m6c};};
\node[stars] at (axis cs:{213.648},{13.975}) {\tikz\pgfuseplotmark{m6c};};
\node[stars] at (axis cs:{213.671},{21.873}) {\tikz\pgfuseplotmark{m6c};};
\node[stars] at (axis cs:{213.712},{10.101}) {\tikz\pgfuseplotmark{m5c};};
\node[stars] at (axis cs:{213.721},{3.336}) {\tikz\pgfuseplotmark{m6cv};};
\node[stars] at (axis cs:{213.755},{-29.282}) {\tikz\pgfuseplotmark{m6b};};
\node[stars] at (axis cs:{213.850},{-18.201}) {\tikz\pgfuseplotmark{m6a};};
\node[stars] at (axis cs:{213.915},{19.182}) {\tikz\pgfuseplotmark{m1bv};};
\node[stars] at (axis cs:{214.004},{-6.001}) {\tikz\pgfuseplotmark{m4bv};};
\node[stars] at (axis cs:{214.017},{18.912}) {\tikz\pgfuseplotmark{m6bv};};
\node[stars] at (axis cs:{214.089},{-6.621}) {\tikz\pgfuseplotmark{m6c};};
\node[stars] at (axis cs:{214.101},{39.745}) {\tikz\pgfuseplotmark{m6c};};
\node[stars] at (axis cs:{214.126},{-3.196}) {\tikz\pgfuseplotmark{m6c};};
\node[stars] at (axis cs:{214.137},{20.122}) {\tikz\pgfuseplotmark{m6cb};};
\node[stars] at (axis cs:{214.203},{-8.884}) {\tikz\pgfuseplotmark{m6c};};
\node[stars] at (axis cs:{214.266},{-18.586}) {\tikz\pgfuseplotmark{m6c};};
\node[stars] at (axis cs:{214.369},{15.263}) {\tikz\pgfuseplotmark{m6av};};
\node[stars] at (axis cs:{214.499},{35.509}) {\tikz\pgfuseplotmark{m5a};};
\node[stars] at (axis cs:{214.503},{-7.542}) {\tikz\pgfuseplotmark{m6c};};
\node[stars] at (axis cs:{214.659},{-18.716}) {\tikz\pgfuseplotmark{m6av};};
\node[stars] at (axis cs:{214.754},{-25.815}) {\tikz\pgfuseplotmark{m6b};};
\node[stars] at (axis cs:{214.777},{-13.371}) {\tikz\pgfuseplotmark{m5av};};
\node[stars] at (axis cs:{214.780},{-27.141}) {\tikz\pgfuseplotmark{m6c};};
\node[stars] at (axis cs:{214.818},{13.004}) {\tikz\pgfuseplotmark{m5c};};
\node[stars] at (axis cs:{214.849},{-37.003}) {\tikz\pgfuseplotmark{m6b};};
\node[stars] at (axis cs:{214.885},{-2.265}) {\tikz\pgfuseplotmark{m5cv};};
\node[stars] at (axis cs:{214.921},{0.384}) {\tikz\pgfuseplotmark{m6cv};};
\node[stars] at (axis cs:{214.938},{16.307}) {\tikz\pgfuseplotmark{m5av};};
\node[stars] at (axis cs:{214.949},{38.794}) {\tikz\pgfuseplotmark{m6c};};
\node[stars] at (axis cs:{214.972},{-6.746}) {\tikz\pgfuseplotmark{m6c};};
\node[stars] at (axis cs:{215.036},{30.429}) {\tikz\pgfuseplotmark{m6cv};};
\node[stars] at (axis cs:{215.139},{-37.885}) {\tikz\pgfuseplotmark{m4bv};};
\node[stars] at (axis cs:{215.582},{-34.787}) {\tikz\pgfuseplotmark{m6a};};
\node[stars] at (axis cs:{215.759},{-39.512}) {\tikz\pgfuseplotmark{m4cv};};
\node[stars] at (axis cs:{215.774},{-27.754}) {\tikz\pgfuseplotmark{m5av};};
\node[stars] at (axis cs:{215.778},{25.338}) {\tikz\pgfuseplotmark{m6c};};
\node[stars] at (axis cs:{215.814},{1.241}) {\tikz\pgfuseplotmark{m6c};};
\node[stars] at (axis cs:{215.845},{8.446}) {\tikz\pgfuseplotmark{m5bb};};
\node[stars] at (axis cs:{215.857},{-11.714}) {\tikz\pgfuseplotmark{m6c};};
\node[stars] at (axis cs:{216.004},{8.244}) {\tikz\pgfuseplotmark{m6bv};};
\node[stars] at (axis cs:{216.008},{27.413}) {\tikz\pgfuseplotmark{m6c};};
\node[stars] at (axis cs:{216.047},{5.820}) {\tikz\pgfuseplotmark{m5b};};
\node[stars] at (axis cs:{216.076},{8.085}) {\tikz\pgfuseplotmark{m6cv};};
\node[stars] at (axis cs:{216.203},{-24.806}) {\tikz\pgfuseplotmark{m5c};};
\node[stars] at (axis cs:{216.371},{38.393}) {\tikz\pgfuseplotmark{m6c};};
\node[stars] at (axis cs:{216.449},{-26.852}) {\tikz\pgfuseplotmark{m6c};};
\node[stars] at (axis cs:{216.614},{19.227}) {\tikz\pgfuseplotmark{m5cv};};
\node[stars] at (axis cs:{216.708},{-39.874}) {\tikz\pgfuseplotmark{m6c};};
\node[stars] at (axis cs:{216.852},{-6.120}) {\tikz\pgfuseplotmark{m6c};};
\node[stars] at (axis cs:{217.043},{-29.492}) {\tikz\pgfuseplotmark{m5bv};};
\node[stars] at (axis cs:{217.051},{-2.228}) {\tikz\pgfuseplotmark{m5av};};
\node[stars] at (axis cs:{217.069},{36.197}) {\tikz\pgfuseplotmark{m6c};};
\node[stars] at (axis cs:{217.174},{-6.901}) {\tikz\pgfuseplotmark{m5c};};
\node[stars] at (axis cs:{217.457},{31.791}) {\tikz\pgfuseplotmark{m6bv};};
\node[stars] at (axis cs:{217.460},{0.829}) {\tikz\pgfuseplotmark{m6b};};
\node[stars] at (axis cs:{217.689},{4.772}) {\tikz\pgfuseplotmark{m6b};};
\node[stars] at (axis cs:{217.795},{-38.870}) {\tikz\pgfuseplotmark{m6b};};
\node[stars] at (axis cs:{217.957},{30.371}) {\tikz\pgfuseplotmark{m4av};};
\node[stars] at (axis cs:{218.019},{38.308}) {\tikz\pgfuseplotmark{m3bv};};
\node[stars] at (axis cs:{218.084},{26.677}) {\tikz\pgfuseplotmark{m6bv};};
\node[stars] at (axis cs:{218.136},{22.260}) {\tikz\pgfuseplotmark{m6b};};
\node[stars] at (axis cs:{218.290},{-30.715}) {\tikz\pgfuseplotmark{m6bb};};
\node[stars] at (axis cs:{218.334},{36.959}) {\tikz\pgfuseplotmark{m6cv};};
\node[stars] at (axis cs:{218.549},{32.534}) {\tikz\pgfuseplotmark{m6cv};};
\node[stars] at (axis cs:{218.660},{36.626}) {\tikz\pgfuseplotmark{m6b};};
\node[stars] at (axis cs:{218.670},{29.745}) {\tikz\pgfuseplotmark{m4cvb};};
\node[stars] at (axis cs:{218.711},{-20.439}) {\tikz\pgfuseplotmark{m6c};};
\node[stars] at (axis cs:{219.029},{23.250}) {\tikz\pgfuseplotmark{m6c};};
\node[stars] at (axis cs:{219.100},{-39.597}) {\tikz\pgfuseplotmark{m6c};};
\node[stars] at (axis cs:{219.249},{-12.305}) {\tikz\pgfuseplotmark{m6c};};
\node[stars] at (axis cs:{219.369},{2.277}) {\tikz\pgfuseplotmark{m6c};};
\node[stars] at (axis cs:{219.558},{18.298}) {\tikz\pgfuseplotmark{m6b};};
\node[stars] at (axis cs:{219.581},{-38.794}) {\tikz\pgfuseplotmark{m6b};};
\node[stars] at (axis cs:{220.091},{21.976}) {\tikz\pgfuseplotmark{m6b};};
\node[stars] at (axis cs:{220.177},{13.534}) {\tikz\pgfuseplotmark{m6b};};
\node[stars] at (axis cs:{220.182},{16.418}) {\tikz\pgfuseplotmark{m5avb};};
\node[stars] at (axis cs:{220.256},{-36.135}) {\tikz\pgfuseplotmark{m6a};};
\node[stars] at (axis cs:{220.287},{13.728}) {\tikz\pgfuseplotmark{m4a};};
\node[stars] at (axis cs:{220.411},{8.162}) {\tikz\pgfuseplotmark{m5a};};
\node[stars] at (axis cs:{220.431},{11.660}) {\tikz\pgfuseplotmark{m6a};};
\node[stars] at (axis cs:{220.463},{-30.933}) {\tikz\pgfuseplotmark{m6cb};};
\node[stars] at (axis cs:{220.476},{21.123}) {\tikz\pgfuseplotmark{m6c};};
\node[stars] at (axis cs:{220.490},{-37.793}) {\tikz\pgfuseplotmark{m4b};};
\node[stars] at (axis cs:{220.765},{-5.658}) {\tikz\pgfuseplotmark{m4b};};
\node[stars] at (axis cs:{220.806},{-24.998}) {\tikz\pgfuseplotmark{m6a};};
\node[stars] at (axis cs:{220.856},{26.528}) {\tikz\pgfuseplotmark{m5av};};
\node[stars] at (axis cs:{220.914},{-35.174}) {\tikz\pgfuseplotmark{m4b};};
\node[stars] at (axis cs:{221.247},{-35.192}) {\tikz\pgfuseplotmark{m5b};};
\node[stars] at (axis cs:{221.247},{27.074}) {\tikz\pgfuseplotmark{m2cb};};
\node[stars] at (axis cs:{221.299},{-1.418}) {\tikz\pgfuseplotmark{m6bv};};
\node[stars] at (axis cs:{221.307},{32.788}) {\tikz\pgfuseplotmark{m6cv};};
\node[stars] at (axis cs:{221.310},{16.964}) {\tikz\pgfuseplotmark{m5a};};
\node[stars] at (axis cs:{221.336},{18.885}) {\tikz\pgfuseplotmark{m6c};};
\node[stars] at (axis cs:{221.376},{0.717}) {\tikz\pgfuseplotmark{m6a};};
\node[stars] at (axis cs:{221.491},{-15.459}) {\tikz\pgfuseplotmark{m6cb};};
\node[stars] at (axis cs:{221.500},{-25.443}) {\tikz\pgfuseplotmark{m5bb};};
\node[stars] at (axis cs:{221.525},{15.132}) {\tikz\pgfuseplotmark{m6bv};};
\node[stars] at (axis cs:{221.528},{-23.153}) {\tikz\pgfuseplotmark{m6a};};
\node[stars] at (axis cs:{221.545},{-21.176}) {\tikz\pgfuseplotmark{m6c};};
\node[stars] at (axis cs:{221.562},{1.893}) {\tikz\pgfuseplotmark{m4av};};
\node[stars] at (axis cs:{221.771},{-38.291}) {\tikz\pgfuseplotmark{m6b};};
\node[stars] at (axis cs:{221.807},{-21.325}) {\tikz\pgfuseplotmark{m6b};};
\node[stars] at (axis cs:{221.844},{-25.624}) {\tikz\pgfuseplotmark{m6a};};
\node[stars] at (axis cs:{221.937},{-26.087}) {\tikz\pgfuseplotmark{m5c};};
\node[stars] at (axis cs:{221.979},{-12.840}) {\tikz\pgfuseplotmark{m6c};};
\node[stars] at (axis cs:{221.990},{-26.646}) {\tikz\pgfuseplotmark{m6a};};
\node[stars] at (axis cs:{222.097},{24.367}) {\tikz\pgfuseplotmark{m6cb};};
\node[stars] at (axis cs:{222.159},{-36.635}) {\tikz\pgfuseplotmark{m6bv};};
\node[stars] at (axis cs:{222.225},{-0.847}) {\tikz\pgfuseplotmark{m6c};};
\node[stars] at (axis cs:{222.278},{37.811}) {\tikz\pgfuseplotmark{m6c};};
\node[stars] at (axis cs:{222.328},{-24.251}) {\tikz\pgfuseplotmark{m6ab};};
\node[stars] at (axis cs:{222.329},{-14.149}) {\tikz\pgfuseplotmark{m6avb};};
\node[stars] at (axis cs:{222.493},{28.616}) {\tikz\pgfuseplotmark{m6a};};
\node[stars] at (axis cs:{222.566},{23.912}) {\tikz\pgfuseplotmark{m6b};};
\node[stars] at (axis cs:{222.572},{-27.960}) {\tikz\pgfuseplotmark{m4c};};
\node[stars] at (axis cs:{222.624},{37.272}) {\tikz\pgfuseplotmark{m5c};};
\node[stars] at (axis cs:{222.720},{-16.042}) {\tikz\pgfuseplotmark{m3avb};};
\node[stars] at (axis cs:{222.750},{-0.257}) {\tikz\pgfuseplotmark{m6cv};};
\node[stars] at (axis cs:{222.754},{-2.299}) {\tikz\pgfuseplotmark{m5b};};
\node[stars] at (axis cs:{222.847},{19.100}) {\tikz\pgfuseplotmark{m5avb};};
\node[stars] at (axis cs:{222.964},{-18.355}) {\tikz\pgfuseplotmark{m6c};};
\node[stars] at (axis cs:{223.138},{-30.577}) {\tikz\pgfuseplotmark{m6c};};
\node[stars] at (axis cs:{223.213},{-37.803}) {\tikz\pgfuseplotmark{m5bv};};
\node[stars] at (axis cs:{223.349},{19.153}) {\tikz\pgfuseplotmark{m6bv};};
\node[stars] at (axis cs:{223.584},{-24.642}) {\tikz\pgfuseplotmark{m5c};};
\node[stars] at (axis cs:{223.595},{-11.898}) {\tikz\pgfuseplotmark{m6a};};
\node[stars] at (axis cs:{223.658},{-33.301}) {\tikz\pgfuseplotmark{m6a};};
\node[stars] at (axis cs:{223.936},{-33.856}) {\tikz\pgfuseplotmark{m5c};};
\node[stars] at (axis cs:{223.994},{32.300}) {\tikz\pgfuseplotmark{m6b};};
\node[stars] at (axis cs:{224.055},{14.446}) {\tikz\pgfuseplotmark{m6b};};
\node[stars] at (axis cs:{224.129},{-32.637}) {\tikz\pgfuseplotmark{m6b};};
\node[stars] at (axis cs:{224.149},{-39.416}) {\tikz\pgfuseplotmark{m6c};};
\node[stars] at (axis cs:{224.192},{-11.410}) {\tikz\pgfuseplotmark{m5c};};
\node[stars] at (axis cs:{224.296},{-4.346}) {\tikz\pgfuseplotmark{m4c};};
\node[stars] at (axis cs:{224.299},{16.388}) {\tikz\pgfuseplotmark{m6a};};
\node[stars] at (axis cs:{224.307},{-29.158}) {\tikz\pgfuseplotmark{m6c};};
\node[stars] at (axis cs:{224.367},{-21.415}) {\tikz\pgfuseplotmark{m6ab};};
\node[stars] at (axis cs:{224.389},{-0.168}) {\tikz\pgfuseplotmark{m6av};};
\node[stars] at (axis cs:{224.652},{-39.907}) {\tikz\pgfuseplotmark{m6c};};
\node[stars] at (axis cs:{224.664},{-27.657}) {\tikz\pgfuseplotmark{m6a};};
\node[stars] at (axis cs:{224.720},{-4.989}) {\tikz\pgfuseplotmark{m6b};};
\node[stars] at (axis cs:{224.723},{-11.144}) {\tikz\pgfuseplotmark{m6ab};};
\node[stars] at (axis cs:{224.808},{-37.881}) {\tikz\pgfuseplotmark{m6c};};
\node[stars] at (axis cs:{224.846},{4.568}) {\tikz\pgfuseplotmark{m6b};};
\node[stars] at (axis cs:{224.904},{39.265}) {\tikz\pgfuseplotmark{m6a};};
\node[stars] at (axis cs:{225.218},{22.045}) {\tikz\pgfuseplotmark{m6c};};
\node[stars] at (axis cs:{225.243},{-8.519}) {\tikz\pgfuseplotmark{m5bvb};};
\node[stars] at (axis cs:{225.305},{-38.058}) {\tikz\pgfuseplotmark{m6b};};
\node[stars] at (axis cs:{225.333},{-2.755}) {\tikz\pgfuseplotmark{m6av};};
\node[stars] at (axis cs:{225.454},{-0.140}) {\tikz\pgfuseplotmark{m6av};};
\node[stars] at (axis cs:{225.492},{-34.359}) {\tikz\pgfuseplotmark{m6c};};
\node[stars] at (axis cs:{225.527},{-28.060}) {\tikz\pgfuseplotmark{m6a};};
\node[stars] at (axis cs:{225.527},{25.008}) {\tikz\pgfuseplotmark{m5a};};
\node[stars] at (axis cs:{225.536},{-7.575}) {\tikz\pgfuseplotmark{m6c};};
\node[stars] at (axis cs:{225.725},{2.091}) {\tikz\pgfuseplotmark{m4c};};
\node[stars] at (axis cs:{225.747},{-32.643}) {\tikz\pgfuseplotmark{m5c};};
\node[stars] at (axis cs:{225.775},{35.206}) {\tikz\pgfuseplotmark{m6a};};
\node[stars] at (axis cs:{226.018},{-25.282}) {\tikz\pgfuseplotmark{m3cv};};
\node[stars] at (axis cs:{226.111},{26.947}) {\tikz\pgfuseplotmark{m5a};};
\node[stars] at (axis cs:{226.391},{-30.551}) {\tikz\pgfuseplotmark{m6c};};
\node[stars] at (axis cs:{226.558},{-36.264}) {\tikz\pgfuseplotmark{m6c};};
\node[stars] at (axis cs:{226.613},{-22.032}) {\tikz\pgfuseplotmark{m6cv};};
\node[stars] at (axis cs:{226.638},{-30.918}) {\tikz\pgfuseplotmark{m6bv};};
\node[stars] at (axis cs:{226.646},{36.456}) {\tikz\pgfuseplotmark{m6c};};
\node[stars] at (axis cs:{226.657},{-16.257}) {\tikz\pgfuseplotmark{m5c};};
\node[stars] at (axis cs:{226.704},{-16.484}) {\tikz\pgfuseplotmark{m6c};};
\node[stars] at (axis cs:{226.825},{24.869}) {\tikz\pgfuseplotmark{m5bvb};};
\node[stars] at (axis cs:{226.835},{18.442}) {\tikz\pgfuseplotmark{m6b};};
\node[stars] at (axis cs:{226.918},{5.498}) {\tikz\pgfuseplotmark{m6c};};
\node[stars] at (axis cs:{227.099},{26.301}) {\tikz\pgfuseplotmark{m6a};};
\node[stars] at (axis cs:{227.148},{25.108}) {\tikz\pgfuseplotmark{m6a};};
\node[stars] at (axis cs:{227.223},{13.235}) {\tikz\pgfuseplotmark{m6c};};
\node[stars] at (axis cs:{227.531},{-38.792}) {\tikz\pgfuseplotmark{m6b};};
\node[stars] at (axis cs:{227.578},{-26.332}) {\tikz\pgfuseplotmark{m6a};};
\node[stars] at (axis cs:{228.018},{18.976}) {\tikz\pgfuseplotmark{m6bv};};
\node[stars] at (axis cs:{228.055},{-19.792}) {\tikz\pgfuseplotmark{m5avb};};
\node[stars] at (axis cs:{228.282},{-36.091}) {\tikz\pgfuseplotmark{m6b};};
\node[stars] at (axis cs:{228.323},{-24.008}) {\tikz\pgfuseplotmark{m6c};};
\node[stars] at (axis cs:{228.330},{-19.647}) {\tikz\pgfuseplotmark{m6b};};
\node[stars] at (axis cs:{228.369},{-25.309}) {\tikz\pgfuseplotmark{m6c};};
\node[stars] at (axis cs:{228.383},{22.983}) {\tikz\pgfuseplotmark{m6c};};
\node[stars] at (axis cs:{228.398},{38.264}) {\tikz\pgfuseplotmark{m6c};};
\node[stars] at (axis cs:{228.472},{-26.193}) {\tikz\pgfuseplotmark{m6a};};
\node[stars] at (axis cs:{228.525},{31.788}) {\tikz\pgfuseplotmark{m6bb};};
\node[stars] at (axis cs:{228.621},{29.164}) {\tikz\pgfuseplotmark{m5c};};
\node[stars] at (axis cs:{228.641},{-17.769}) {\tikz\pgfuseplotmark{m6c};};
\node[stars] at (axis cs:{228.655},{-31.519}) {\tikz\pgfuseplotmark{m5b};};
\node[stars] at (axis cs:{228.711},{-5.503}) {\tikz\pgfuseplotmark{m6c};};
\node[stars] at (axis cs:{228.797},{4.939}) {\tikz\pgfuseplotmark{m5c};};
\node[stars] at (axis cs:{228.876},{33.315}) {\tikz\pgfuseplotmark{m3cv};};
\node[stars] at (axis cs:{228.954},{0.372}) {\tikz\pgfuseplotmark{m6a};};
\node[stars] at (axis cs:{228.973},{6.465}) {\tikz\pgfuseplotmark{m6c};};
\node[stars] at (axis cs:{229.096},{-22.399}) {\tikz\pgfuseplotmark{m6a};};
\node[stars] at (axis cs:{229.252},{-9.383}) {\tikz\pgfuseplotmark{m3av};};
\node[stars] at (axis cs:{229.458},{-30.148}) {\tikz\pgfuseplotmark{m4c};};
\node[stars] at (axis cs:{229.602},{20.573}) {\tikz\pgfuseplotmark{m6av};};
\node[stars] at (axis cs:{229.609},{-0.461}) {\tikz\pgfuseplotmark{m6b};};
\node[stars] at (axis cs:{229.672},{-31.209}) {\tikz\pgfuseplotmark{m6c};};
\node[stars] at (axis cs:{229.828},{1.765}) {\tikz\pgfuseplotmark{m5bv};};
\node[stars] at (axis cs:{229.875},{32.515}) {\tikz\pgfuseplotmark{m6c};};
\node[stars] at (axis cs:{229.882},{-37.097}) {\tikz\pgfuseplotmark{m6c};};
\node[stars] at (axis cs:{230.036},{29.616}) {\tikz\pgfuseplotmark{m6av};};
\node[stars] at (axis cs:{230.196},{-2.413}) {\tikz\pgfuseplotmark{m6c};};
\node[stars] at (axis cs:{230.224},{-18.159}) {\tikz\pgfuseplotmark{m6c};};
\node[stars] at (axis cs:{230.256},{-15.548}) {\tikz\pgfuseplotmark{m6cb};};
\node[stars] at (axis cs:{230.258},{0.715}) {\tikz\pgfuseplotmark{m5cvb};};
\node[stars] at (axis cs:{230.279},{24.958}) {\tikz\pgfuseplotmark{m6c};};
\node[stars] at (axis cs:{230.282},{-5.825}) {\tikz\pgfuseplotmark{m6a};};
\node[stars] at (axis cs:{230.452},{-36.261}) {\tikz\pgfuseplotmark{m4a};};
\node[stars] at (axis cs:{230.452},{32.934}) {\tikz\pgfuseplotmark{m5c};};
\node[stars] at (axis cs:{230.597},{12.568}) {\tikz\pgfuseplotmark{m6c};};
\node[stars] at (axis cs:{230.656},{39.581}) {\tikz\pgfuseplotmark{m6a};};
\node[stars] at (axis cs:{230.757},{-15.134}) {\tikz\pgfuseplotmark{m6cv};};
\node[stars] at (axis cs:{230.789},{-36.858}) {\tikz\pgfuseplotmark{m5a};};
\node[stars] at (axis cs:{230.801},{30.288}) {\tikz\pgfuseplotmark{m5bv};};
\node[stars] at (axis cs:{230.932},{-1.022}) {\tikz\pgfuseplotmark{m6b};};
\node[stars] at (axis cs:{230.968},{-12.369}) {\tikz\pgfuseplotmark{m6a};};
\node[stars] at (axis cs:{231.050},{-10.322}) {\tikz\pgfuseplotmark{m5b};};
\node[stars] at (axis cs:{231.123},{37.377}) {\tikz\pgfuseplotmark{m4cvb};};
\node[stars] at (axis cs:{231.188},{-39.710}) {\tikz\pgfuseplotmark{m5c};};
\node[stars] at (axis cs:{231.334},{-38.734}) {\tikz\pgfuseplotmark{m5a};};
\node[stars] at (axis cs:{231.447},{15.428}) {\tikz\pgfuseplotmark{m5cv};};
\node[stars] at (axis cs:{231.472},{19.480}) {\tikz\pgfuseplotmark{m6cv};};
\node[stars] at (axis cs:{231.572},{34.336}) {\tikz\pgfuseplotmark{m5c};};
\node[stars] at (axis cs:{231.826},{-36.768}) {\tikz\pgfuseplotmark{m5c};};
\node[stars] at (axis cs:{231.912},{25.102}) {\tikz\pgfuseplotmark{m6b};};
\node[stars] at (axis cs:{231.957},{29.106}) {\tikz\pgfuseplotmark{m4av};};
\node[stars] at (axis cs:{232.064},{-16.716}) {\tikz\pgfuseplotmark{m6a};};
\node[stars] at (axis cs:{232.159},{1.842}) {\tikz\pgfuseplotmark{m5c};};
\node[stars] at (axis cs:{232.300},{16.395}) {\tikz\pgfuseplotmark{m6c};};
\node[stars] at (axis cs:{232.330},{-38.635}) {\tikz\pgfuseplotmark{m6cv};};
\node[stars] at (axis cs:{232.595},{31.286}) {\tikz\pgfuseplotmark{m6c};};
\node[stars] at (axis cs:{232.651},{-20.728}) {\tikz\pgfuseplotmark{m6cv};};
\node[stars] at (axis cs:{232.668},{-16.609}) {\tikz\pgfuseplotmark{m6a};};
\node[stars] at (axis cs:{232.843},{36.616}) {\tikz\pgfuseplotmark{m6c};};
\node[stars] at (axis cs:{232.931},{-20.165}) {\tikz\pgfuseplotmark{m6cb};};
\node[stars] at (axis cs:{232.959},{-32.881}) {\tikz\pgfuseplotmark{m6c};};
\node[stars] at (axis cs:{233.017},{-38.622}) {\tikz\pgfuseplotmark{m6c};};
\node[stars] at (axis cs:{233.040},{16.056}) {\tikz\pgfuseplotmark{m6c};};
\node[stars] at (axis cs:{233.153},{-19.670}) {\tikz\pgfuseplotmark{m5c};};
\node[stars] at (axis cs:{233.230},{-16.853}) {\tikz\pgfuseplotmark{m5cv};};
\node[stars] at (axis cs:{233.232},{31.359}) {\tikz\pgfuseplotmark{m4cv};};
\node[stars] at (axis cs:{233.241},{-1.186}) {\tikz\pgfuseplotmark{m6a};};
\node[stars] at (axis cs:{233.545},{-10.064}) {\tikz\pgfuseplotmark{m5a};};
\node[stars] at (axis cs:{233.587},{-39.349}) {\tikz\pgfuseplotmark{m6c};};
\node[stars] at (axis cs:{233.610},{-9.183}) {\tikz\pgfuseplotmark{m5cv};};
\node[stars] at (axis cs:{233.656},{-28.047}) {\tikz\pgfuseplotmark{m5b};};
\node[stars] at (axis cs:{233.672},{26.714}) {\tikz\pgfuseplotmark{m2cv};};
\node[stars] at (axis cs:{233.701},{10.539}) {\tikz\pgfuseplotmark{m4bvb};};
\node[stars] at (axis cs:{233.812},{39.010}) {\tikz\pgfuseplotmark{m5cv};};
\node[stars] at (axis cs:{233.882},{-14.789}) {\tikz\pgfuseplotmark{m4b};};
\node[stars] at (axis cs:{233.888},{17.655}) {\tikz\pgfuseplotmark{m6b};};
\node[stars] at (axis cs:{233.955},{38.374}) {\tikz\pgfuseplotmark{m6c};};
\node[stars] at (axis cs:{233.972},{11.265}) {\tikz\pgfuseplotmark{m6b};};
\node[stars] at (axis cs:{234.047},{-33.093}) {\tikz\pgfuseplotmark{m6c};};
\node[stars] at (axis cs:{234.122},{16.119}) {\tikz\pgfuseplotmark{m6b};};
\node[stars] at (axis cs:{234.123},{10.010}) {\tikz\pgfuseplotmark{m5cv};};
\node[stars] at (axis cs:{234.223},{29.991}) {\tikz\pgfuseplotmark{m6c};};
\node[stars] at (axis cs:{234.256},{-28.135}) {\tikz\pgfuseplotmark{m4a};};
\node[stars] at (axis cs:{234.369},{-26.280}) {\tikz\pgfuseplotmark{m6c};};
\node[stars] at (axis cs:{234.450},{-23.142}) {\tikz\pgfuseplotmark{m6a};};
\node[stars] at (axis cs:{234.566},{-28.207}) {\tikz\pgfuseplotmark{m6c};};
\node[stars] at (axis cs:{234.568},{-21.016}) {\tikz\pgfuseplotmark{m6a};};
\node[stars] at (axis cs:{234.664},{-29.778}) {\tikz\pgfuseplotmark{m4a};};
\node[stars] at (axis cs:{234.667},{-8.791}) {\tikz\pgfuseplotmark{m6cb};};
\node[stars] at (axis cs:{234.676},{-39.128}) {\tikz\pgfuseplotmark{m6b};};
\node[stars] at (axis cs:{234.704},{34.675}) {\tikz\pgfuseplotmark{m6b};};
\node[stars] at (axis cs:{234.727},{-19.302}) {\tikz\pgfuseplotmark{m5c};};
\node[stars] at (axis cs:{234.839},{-23.150}) {\tikz\pgfuseplotmark{m6c};};
\node[stars] at (axis cs:{234.844},{36.636}) {\tikz\pgfuseplotmark{m5bb};};
\node[stars] at (axis cs:{234.942},{-34.412}) {\tikz\pgfuseplotmark{m5a};};
\node[stars] at (axis cs:{235.043},{12.053}) {\tikz\pgfuseplotmark{m6cb};};
\node[stars] at (axis cs:{235.065},{-31.214}) {\tikz\pgfuseplotmark{m6c};};
\node[stars] at (axis cs:{235.070},{-23.818}) {\tikz\pgfuseplotmark{m5b};};
\node[stars] at (axis cs:{235.246},{16.024}) {\tikz\pgfuseplotmark{m6b};};
\node[stars] at (axis cs:{235.388},{19.670}) {\tikz\pgfuseplotmark{m5a};};
\node[stars] at (axis cs:{235.448},{12.847}) {\tikz\pgfuseplotmark{m5cv};};
\node[stars] at (axis cs:{235.478},{18.464}) {\tikz\pgfuseplotmark{m6a};};
\node[stars] at (axis cs:{235.487},{-19.679}) {\tikz\pgfuseplotmark{m5av};};
\node[stars] at (axis cs:{235.660},{-37.425}) {\tikz\pgfuseplotmark{m5c};};
\node[stars] at (axis cs:{235.671},{-34.710}) {\tikz\pgfuseplotmark{m5a};};
\node[stars] at (axis cs:{235.686},{26.295}) {\tikz\pgfuseplotmark{m4av};};
\node[stars] at (axis cs:{235.854},{-15.043}) {\tikz\pgfuseplotmark{m6c};};
\node[stars] at (axis cs:{235.997},{32.516}) {\tikz\pgfuseplotmark{m6a};};
\node[stars] at (axis cs:{236.008},{2.515}) {\tikz\pgfuseplotmark{m6bb};};
\node[stars] at (axis cs:{236.018},{-15.673}) {\tikz\pgfuseplotmark{m5c};};
\node[stars] at (axis cs:{236.067},{6.425}) {\tikz\pgfuseplotmark{m3av};};
\node[stars] at (axis cs:{236.176},{17.264}) {\tikz\pgfuseplotmark{m6c};};
\node[stars] at (axis cs:{236.348},{5.447}) {\tikz\pgfuseplotmark{m6a};};
\node[stars] at (axis cs:{236.415},{0.891}) {\tikz\pgfuseplotmark{m6c};};
\node[stars] at (axis cs:{236.523},{-1.804}) {\tikz\pgfuseplotmark{m5cv};};
\node[stars] at (axis cs:{236.547},{15.422}) {\tikz\pgfuseplotmark{m4av};};
\node[stars] at (axis cs:{236.611},{7.353}) {\tikz\pgfuseplotmark{m4cv};};
\node[stars] at (axis cs:{236.684},{-34.682}) {\tikz\pgfuseplotmark{m6a};};
\node[stars] at (axis cs:{236.689},{-6.120}) {\tikz\pgfuseplotmark{m6c};};
\node[stars] at (axis cs:{236.822},{14.115}) {\tikz\pgfuseplotmark{m6a};};
\node[stars] at (axis cs:{236.871},{-37.916}) {\tikz\pgfuseplotmark{m6b};};
\node[stars] at (axis cs:{237.007},{31.736}) {\tikz\pgfuseplotmark{m6c};};
\node[stars] at (axis cs:{237.055},{13.789}) {\tikz\pgfuseplotmark{m6b};};
\node[stars] at (axis cs:{237.143},{28.157}) {\tikz\pgfuseplotmark{m6bv};};
\node[stars] at (axis cs:{237.185},{18.141}) {\tikz\pgfuseplotmark{m4bv};};
\node[stars] at (axis cs:{237.237},{-3.818}) {\tikz\pgfuseplotmark{m6a};};
\node[stars] at (axis cs:{237.399},{26.068}) {\tikz\pgfuseplotmark{m5av};};
\node[stars] at (axis cs:{237.405},{-3.430}) {\tikz\pgfuseplotmark{m4a};};
\node[stars] at (axis cs:{237.573},{2.196}) {\tikz\pgfuseplotmark{m5c};};
\node[stars] at (axis cs:{237.704},{4.478}) {\tikz\pgfuseplotmark{m4a};};
\node[stars] at (axis cs:{237.740},{-33.627}) {\tikz\pgfuseplotmark{m4b};};
\node[stars] at (axis cs:{237.745},{-25.751}) {\tikz\pgfuseplotmark{m5a};};
\node[stars] at (axis cs:{237.808},{35.657}) {\tikz\pgfuseplotmark{m5a};};
\node[stars] at (axis cs:{237.815},{-3.090}) {\tikz\pgfuseplotmark{m5bv};};
\node[stars] at (axis cs:{237.816},{20.978}) {\tikz\pgfuseplotmark{m5av};};
\node[stars] at (axis cs:{237.910},{-14.134}) {\tikz\pgfuseplotmark{m6c};};
\node[stars] at (axis cs:{238.053},{-29.886}) {\tikz\pgfuseplotmark{m6cv};};
\node[stars] at (axis cs:{238.234},{17.403}) {\tikz\pgfuseplotmark{m6c};};
\node[stars] at (axis cs:{238.300},{13.196}) {\tikz\pgfuseplotmark{m6bv};};
\node[stars] at (axis cs:{238.334},{-20.167}) {\tikz\pgfuseplotmark{m5bv};};
\node[stars] at (axis cs:{238.395},{16.075}) {\tikz\pgfuseplotmark{m6c};};
\node[stars] at (axis cs:{238.403},{-25.327}) {\tikz\pgfuseplotmark{m5ab};};
\node[stars] at (axis cs:{238.456},{-16.729}) {\tikz\pgfuseplotmark{m4b};};
\node[stars] at (axis cs:{238.475},{-24.533}) {\tikz\pgfuseplotmark{m5c};};
\node[stars] at (axis cs:{238.483},{-23.978}) {\tikz\pgfuseplotmark{m5cv};};
\node[stars] at (axis cs:{238.625},{-27.339}) {\tikz\pgfuseplotmark{m6c};};
\node[stars] at (axis cs:{238.644},{20.311}) {\tikz\pgfuseplotmark{m5cv};};
\node[stars] at (axis cs:{238.665},{-25.244}) {\tikz\pgfuseplotmark{m6bv};};
\node[stars] at (axis cs:{238.668},{8.580}) {\tikz\pgfuseplotmark{m6cv};};
\node[stars] at (axis cs:{238.752},{-19.383}) {\tikz\pgfuseplotmark{m6bv};};
\node[stars] at (axis cs:{238.875},{-26.266}) {\tikz\pgfuseplotmark{m6a};};
\node[stars] at (axis cs:{238.877},{-31.084}) {\tikz\pgfuseplotmark{m6c};};
\node[stars] at (axis cs:{238.916},{18.621}) {\tikz\pgfuseplotmark{m6c};};
\node[stars] at (axis cs:{238.948},{37.947}) {\tikz\pgfuseplotmark{m5c};};
\node[stars] at (axis cs:{239.028},{-39.864}) {\tikz\pgfuseplotmark{m6b};};
\node[stars] at (axis cs:{239.058},{-31.786}) {\tikz\pgfuseplotmark{m6c};};
\node[stars] at (axis cs:{239.060},{-14.399}) {\tikz\pgfuseplotmark{m6c};};
\node[stars] at (axis cs:{239.113},{15.661}) {\tikz\pgfuseplotmark{m4avb};};
\node[stars] at (axis cs:{239.139},{-14.829}) {\tikz\pgfuseplotmark{m6bv};};
\node[stars] at (axis cs:{239.221},{-29.214}) {\tikz\pgfuseplotmark{m4b};};
\node[stars] at (axis cs:{239.223},{-33.966}) {\tikz\pgfuseplotmark{m5bb};};
\node[stars] at (axis cs:{239.311},{14.414}) {\tikz\pgfuseplotmark{m6a};};
\node[stars] at (axis cs:{239.339},{-36.185}) {\tikz\pgfuseplotmark{m6a};};
\node[stars] at (axis cs:{239.374},{39.695}) {\tikz\pgfuseplotmark{m6c};};
\node[stars] at (axis cs:{239.397},{26.878}) {\tikz\pgfuseplotmark{m4c};};
\node[stars] at (axis cs:{239.419},{-20.983}) {\tikz\pgfuseplotmark{m6av};};
\node[stars] at (axis cs:{239.547},{-14.279}) {\tikz\pgfuseplotmark{m5bv};};
\node[stars] at (axis cs:{239.628},{-37.504}) {\tikz\pgfuseplotmark{m6c};};
\node[stars] at (axis cs:{239.645},{-24.831}) {\tikz\pgfuseplotmark{m5cv};};
\node[stars] at (axis cs:{239.713},{-26.114}) {\tikz\pgfuseplotmark{m3bvb};};
\node[stars] at (axis cs:{239.740},{36.644}) {\tikz\pgfuseplotmark{m6a};};
\node[stars] at (axis cs:{240.031},{-38.396}) {\tikz\pgfuseplotmark{m3c};};
\node[stars] at (axis cs:{240.082},{-16.533}) {\tikz\pgfuseplotmark{m5c};};
\node[stars] at (axis cs:{240.083},{-22.622}) {\tikz\pgfuseplotmark{m2cv};};
\node[stars] at (axis cs:{240.198},{-8.411}) {\tikz\pgfuseplotmark{m6a};};
\node[stars] at (axis cs:{240.213},{4.427}) {\tikz\pgfuseplotmark{m6a};};
\node[stars] at (axis cs:{240.261},{33.303}) {\tikz\pgfuseplotmark{m5c};};
\node[stars] at (axis cs:{240.310},{17.818}) {\tikz\pgfuseplotmark{m5b};};
\node[stars] at (axis cs:{240.331},{-31.889}) {\tikz\pgfuseplotmark{m6c};};
\node[stars] at (axis cs:{240.361},{29.851}) {\tikz\pgfuseplotmark{m5bv};};
\node[stars] at (axis cs:{240.574},{22.804}) {\tikz\pgfuseplotmark{m5a};};
\node[stars] at (axis cs:{240.664},{-29.136}) {\tikz\pgfuseplotmark{m6b};};
\node[stars] at (axis cs:{240.831},{36.632}) {\tikz\pgfuseplotmark{m6a};};
\node[stars] at (axis cs:{240.836},{-25.865}) {\tikz\pgfuseplotmark{m5b};};
\node[stars] at (axis cs:{240.851},{-38.602}) {\tikz\pgfuseplotmark{m5b};};
\node[stars] at (axis cs:{240.893},{-32.001}) {\tikz\pgfuseplotmark{m6b};};
\node[stars] at (axis cs:{240.940},{4.987}) {\tikz\pgfuseplotmark{m6b};};
\node[stars] at (axis cs:{240.978},{-24.726}) {\tikz\pgfuseplotmark{m6c};};
\node[stars] at (axis cs:{241.074},{-33.213}) {\tikz\pgfuseplotmark{m6bv};};
\node[stars] at (axis cs:{241.092},{-11.373}) {\tikz\pgfuseplotmark{m4cb};};
\node[stars] at (axis cs:{241.153},{-37.863}) {\tikz\pgfuseplotmark{m6b};};
\node[stars] at (axis cs:{241.359},{-19.805}) {\tikz\pgfuseplotmark{m3avb};};
\node[stars] at (axis cs:{241.408},{8.096}) {\tikz\pgfuseplotmark{m6c};};
\node[stars] at (axis cs:{241.436},{-6.291}) {\tikz\pgfuseplotmark{m6cb};};
\node[stars] at (axis cs:{241.499},{-6.140}) {\tikz\pgfuseplotmark{m6c};};
\node[stars] at (axis cs:{241.527},{-23.606}) {\tikz\pgfuseplotmark{m6bv};};
\node[stars] at (axis cs:{241.648},{-36.802}) {\tikz\pgfuseplotmark{m4cv};};
\node[stars] at (axis cs:{241.702},{-20.669}) {\tikz\pgfuseplotmark{m4b};};
\node[stars] at (axis cs:{241.764},{-14.071}) {\tikz\pgfuseplotmark{m6c};};
\node[stars] at (axis cs:{241.818},{-36.756}) {\tikz\pgfuseplotmark{m6a};};
\node[stars] at (axis cs:{241.842},{21.823}) {\tikz\pgfuseplotmark{m6cv};};
\node[stars] at (axis cs:{241.851},{-20.869}) {\tikz\pgfuseplotmark{m4c};};
\node[stars] at (axis cs:{241.902},{-12.745}) {\tikz\pgfuseplotmark{m6a};};
\node[stars] at (axis cs:{241.906},{9.892}) {\tikz\pgfuseplotmark{m6a};};
\node[stars] at (axis cs:{241.966},{-24.462}) {\tikz\pgfuseplotmark{m6c};};
\node[stars] at (axis cs:{242.019},{17.047}) {\tikz\pgfuseplotmark{m5bvb};};
\node[stars] at (axis cs:{242.032},{-26.327}) {\tikz\pgfuseplotmark{m5c};};
\node[stars] at (axis cs:{242.117},{8.534}) {\tikz\pgfuseplotmark{m6av};};
\node[stars] at (axis cs:{242.182},{-23.685}) {\tikz\pgfuseplotmark{m6av};};
\node[stars] at (axis cs:{242.194},{17.206}) {\tikz\pgfuseplotmark{m6bv};};
\node[stars] at (axis cs:{242.243},{36.491}) {\tikz\pgfuseplotmark{m5av};};
\node[stars] at (axis cs:{242.245},{3.454}) {\tikz\pgfuseplotmark{m6b};};
\node[stars] at (axis cs:{242.297},{6.378}) {\tikz\pgfuseplotmark{m6b};};
\node[stars] at (axis cs:{242.460},{-3.467}) {\tikz\pgfuseplotmark{m5cv};};
\node[stars] at (axis cs:{242.469},{-33.546}) {\tikz\pgfuseplotmark{m5c};};
\node[stars] at (axis cs:{242.480},{-18.341}) {\tikz\pgfuseplotmark{m6c};};
\node[stars] at (axis cs:{242.759},{-29.416}) {\tikz\pgfuseplotmark{m5b};};
\node[stars] at (axis cs:{242.870},{16.665}) {\tikz\pgfuseplotmark{m6b};};
\node[stars] at (axis cs:{242.908},{23.495}) {\tikz\pgfuseplotmark{m6av};};
\node[stars] at (axis cs:{242.915},{33.343}) {\tikz\pgfuseplotmark{m6cb};};
\node[stars] at (axis cs:{242.950},{36.425}) {\tikz\pgfuseplotmark{m6a};};
\node[stars] at (axis cs:{242.999},{-19.461}) {\tikz\pgfuseplotmark{m4bb};};
\node[stars] at (axis cs:{243.000},{-10.064}) {\tikz\pgfuseplotmark{m5b};};
\node[stars] at (axis cs:{243.027},{39.055}) {\tikz\pgfuseplotmark{m6c};};
\node[stars] at (axis cs:{243.030},{-8.547}) {\tikz\pgfuseplotmark{m5c};};
\node[stars] at (axis cs:{243.067},{-28.417}) {\tikz\pgfuseplotmark{m6ab};};
\node[stars] at (axis cs:{243.076},{-27.926}) {\tikz\pgfuseplotmark{m5a};};
\node[stars] at (axis cs:{243.189},{26.671}) {\tikz\pgfuseplotmark{m6cb};};
\node[stars] at (axis cs:{243.236},{-4.221}) {\tikz\pgfuseplotmark{m6c};};
\node[stars] at (axis cs:{243.314},{5.021}) {\tikz\pgfuseplotmark{m5c};};
\node[stars] at (axis cs:{243.411},{-1.475}) {\tikz\pgfuseplotmark{m6c};};
\node[stars] at (axis cs:{243.440},{-24.422}) {\tikz\pgfuseplotmark{m6cvb};};
\node[stars] at (axis cs:{243.462},{-11.838}) {\tikz\pgfuseplotmark{m5c};};
\node[stars] at (axis cs:{243.557},{5.902}) {\tikz\pgfuseplotmark{m6c};};
\node[stars] at (axis cs:{243.586},{-3.694}) {\tikz\pgfuseplotmark{m3av};};
\node[stars] at (axis cs:{243.593},{-33.011}) {\tikz\pgfuseplotmark{m6b};};
\node[stars] at (axis cs:{243.620},{-21.108}) {\tikz\pgfuseplotmark{m6c};};
\node[stars] at (axis cs:{243.664},{-18.536}) {\tikz\pgfuseplotmark{m6c};};
\node[stars] at (axis cs:{243.670},{33.858}) {\tikz\pgfuseplotmark{m6avb};};
\node[stars] at (axis cs:{243.723},{-25.477}) {\tikz\pgfuseplotmark{m6b};};
\node[stars] at (axis cs:{243.869},{18.808}) {\tikz\pgfuseplotmark{m6a};};
\node[stars] at (axis cs:{243.905},{-8.369}) {\tikz\pgfuseplotmark{m5cv};};
\node[stars] at (axis cs:{243.947},{27.422}) {\tikz\pgfuseplotmark{m6c};};
\node[stars] at (axis cs:{243.965},{-14.849}) {\tikz\pgfuseplotmark{m6bv};};
\node[stars] at (axis cs:{244.187},{29.150}) {\tikz\pgfuseplotmark{m6avb};};
\node[stars] at (axis cs:{244.230},{-3.953}) {\tikz\pgfuseplotmark{m6c};};
\node[stars] at (axis cs:{244.247},{-20.104}) {\tikz\pgfuseplotmark{m6c};};
\node[stars] at (axis cs:{244.431},{1.496}) {\tikz\pgfuseplotmark{m6c};};
\node[stars] at (axis cs:{244.575},{-28.614}) {\tikz\pgfuseplotmark{m5a};};
\node[stars] at (axis cs:{244.580},{-4.692}) {\tikz\pgfuseplotmark{m3c};};
\node[stars] at (axis cs:{244.752},{-14.873}) {\tikz\pgfuseplotmark{m6b};};
\node[stars] at (axis cs:{244.782},{-20.218}) {\tikz\pgfuseplotmark{m6c};};
\node[stars] at (axis cs:{244.886},{-30.907}) {\tikz\pgfuseplotmark{m5cb};};
\node[stars] at (axis cs:{244.980},{39.708}) {\tikz\pgfuseplotmark{m5c};};
\node[stars] at (axis cs:{245.018},{21.132}) {\tikz\pgfuseplotmark{m6bv};};
\node[stars] at (axis cs:{245.136},{-39.430}) {\tikz\pgfuseplotmark{m6b};};
\node[stars] at (axis cs:{245.159},{-24.169}) {\tikz\pgfuseplotmark{m5av};};
\node[stars] at (axis cs:{245.297},{-25.593}) {\tikz\pgfuseplotmark{m3bv};};
\node[stars] at (axis cs:{245.480},{19.153}) {\tikz\pgfuseplotmark{m4avb};};
\node[stars] at (axis cs:{245.518},{1.029}) {\tikz\pgfuseplotmark{m5a};};
\node[stars] at (axis cs:{245.524},{30.892}) {\tikz\pgfuseplotmark{m5av};};
\node[stars] at (axis cs:{245.589},{33.799}) {\tikz\pgfuseplotmark{m5cvb};};
\node[stars] at (axis cs:{245.662},{-2.080}) {\tikz\pgfuseplotmark{m6c};};
\node[stars] at (axis cs:{245.735},{32.333}) {\tikz\pgfuseplotmark{m6cb};};
\node[stars] at (axis cs:{246.005},{-39.193}) {\tikz\pgfuseplotmark{m5cv};};
\node[stars] at (axis cs:{246.026},{-20.037}) {\tikz\pgfuseplotmark{m4c};};
\node[stars] at (axis cs:{246.045},{6.948}) {\tikz\pgfuseplotmark{m6a};};
\node[stars] at (axis cs:{246.132},{-37.566}) {\tikz\pgfuseplotmark{m5cv};};
\node[stars] at (axis cs:{246.166},{-29.705}) {\tikz\pgfuseplotmark{m5cb};};
\node[stars] at (axis cs:{246.351},{37.394}) {\tikz\pgfuseplotmark{m6a};};
\node[stars] at (axis cs:{246.354},{14.033}) {\tikz\pgfuseplotmark{m5av};};
\node[stars] at (axis cs:{246.396},{-23.447}) {\tikz\pgfuseplotmark{m5ab};};
\node[stars] at (axis cs:{246.547},{11.407}) {\tikz\pgfuseplotmark{m6b};};
\node[stars] at (axis cs:{246.709},{2.348}) {\tikz\pgfuseplotmark{m6b};};
\node[stars] at (axis cs:{246.756},{-18.456}) {\tikz\pgfuseplotmark{m4cv};};
\node[stars] at (axis cs:{246.884},{2.871}) {\tikz\pgfuseplotmark{m6c};};
\node[stars] at (axis cs:{246.931},{-7.598}) {\tikz\pgfuseplotmark{m5cv};};
\node[stars] at (axis cs:{246.951},{-8.372}) {\tikz\pgfuseplotmark{m5a};};
\node[stars] at (axis cs:{247.060},{-37.180}) {\tikz\pgfuseplotmark{m6a};};
\node[stars] at (axis cs:{247.142},{0.665}) {\tikz\pgfuseplotmark{m5c};};
\node[stars] at (axis cs:{247.176},{0.055}) {\tikz\pgfuseplotmark{m6c};};
\node[stars] at (axis cs:{247.352},{-26.432}) {\tikz\pgfuseplotmark{m1bv};};
\node[stars] at (axis cs:{247.446},{-14.551}) {\tikz\pgfuseplotmark{m6a};};
\node[stars] at (axis cs:{247.552},{-25.115}) {\tikz\pgfuseplotmark{m5av};};
\node[stars] at (axis cs:{247.555},{21.490}) {\tikz\pgfuseplotmark{m3av};};
\node[stars] at (axis cs:{247.625},{-7.514}) {\tikz\pgfuseplotmark{m6c};};
\node[stars] at (axis cs:{247.640},{20.479}) {\tikz\pgfuseplotmark{m5c};};
\node[stars] at (axis cs:{247.728},{1.984}) {\tikz\pgfuseplotmark{m4bvb};};
\node[stars] at (axis cs:{247.762},{35.225}) {\tikz\pgfuseplotmark{m6c};};
\node[stars] at (axis cs:{247.785},{-16.613}) {\tikz\pgfuseplotmark{m4c};};
\node[stars] at (axis cs:{247.806},{22.195}) {\tikz\pgfuseplotmark{m6a};};
\node[stars] at (axis cs:{247.845},{-26.538}) {\tikz\pgfuseplotmark{m6b};};
\node[stars] at (axis cs:{247.846},{-34.704}) {\tikz\pgfuseplotmark{m4c};};
\node[stars] at (axis cs:{248.034},{-21.466}) {\tikz\pgfuseplotmark{m4cv};};
\node[stars] at (axis cs:{248.149},{5.521}) {\tikz\pgfuseplotmark{m6a};};
\node[stars] at (axis cs:{248.151},{11.488}) {\tikz\pgfuseplotmark{m5av};};
\node[stars] at (axis cs:{248.860},{17.057}) {\tikz\pgfuseplotmark{m6cb};};
\node[stars] at (axis cs:{248.971},{-28.216}) {\tikz\pgfuseplotmark{m3a};};
\node[stars] at (axis cs:{249.089},{-2.324}) {\tikz\pgfuseplotmark{m6av};};
\node[stars] at (axis cs:{249.094},{-35.255}) {\tikz\pgfuseplotmark{m4cv};};
\node[stars] at (axis cs:{249.179},{15.497}) {\tikz\pgfuseplotmark{m6cv};};
\node[stars] at (axis cs:{249.290},{-10.567}) {\tikz\pgfuseplotmark{m3av};};
\node[stars] at (axis cs:{249.391},{5.277}) {\tikz\pgfuseplotmark{m6c};};
\node[stars] at (axis cs:{249.450},{13.687}) {\tikz\pgfuseplotmark{m6c};};
\node[stars] at (axis cs:{249.506},{-6.538}) {\tikz\pgfuseplotmark{m6b};};
\node[stars] at (axis cs:{249.699},{-8.619}) {\tikz\pgfuseplotmark{m6cv};};
\node[stars] at (axis cs:{249.772},{-37.217}) {\tikz\pgfuseplotmark{m6b};};
\node[stars] at (axis cs:{249.913},{-9.555}) {\tikz\pgfuseplotmark{m6c};};
\node[stars] at (axis cs:{250.144},{-20.409}) {\tikz\pgfuseplotmark{m6c};};
\node[stars] at (axis cs:{250.161},{4.220}) {\tikz\pgfuseplotmark{m6ab};};
\node[stars] at (axis cs:{250.214},{12.395}) {\tikz\pgfuseplotmark{m6b};};
\node[stars] at (axis cs:{250.235},{-8.309}) {\tikz\pgfuseplotmark{m6c};};
\node[stars] at (axis cs:{250.252},{24.859}) {\tikz\pgfuseplotmark{m6b};};
\node[stars] at (axis cs:{250.298},{-1.000}) {\tikz\pgfuseplotmark{m6c};};
\node[stars] at (axis cs:{250.320},{1.245}) {\tikz\pgfuseplotmark{m6c};};
\node[stars] at (axis cs:{250.322},{31.603}) {\tikz\pgfuseplotmark{m3avb};};
\node[stars] at (axis cs:{250.393},{-17.742}) {\tikz\pgfuseplotmark{m5b};};
\node[stars] at (axis cs:{250.401},{-24.468}) {\tikz\pgfuseplotmark{m6b};};
\node[stars] at (axis cs:{250.403},{26.917}) {\tikz\pgfuseplotmark{m6b};};
\node[stars] at (axis cs:{250.427},{1.181}) {\tikz\pgfuseplotmark{m6a};};
\node[stars] at (axis cs:{250.439},{-33.146}) {\tikz\pgfuseplotmark{m6a};};
\node[stars] at (axis cs:{250.474},{-19.924}) {\tikz\pgfuseplotmark{m6a};};
\node[stars] at (axis cs:{250.724},{38.922}) {\tikz\pgfuseplotmark{m3cv};};
\node[stars] at (axis cs:{250.911},{-32.106}) {\tikz\pgfuseplotmark{m6c};};
\node[stars] at (axis cs:{250.948},{-38.156}) {\tikz\pgfuseplotmark{m6b};};
\node[stars] at (axis cs:{250.965},{34.039}) {\tikz\pgfuseplotmark{m6b};};
\node[stars] at (axis cs:{251.251},{-28.509}) {\tikz\pgfuseplotmark{m6bv};};
\node[stars] at (axis cs:{251.344},{15.745}) {\tikz\pgfuseplotmark{m6av};};
\node[stars] at (axis cs:{251.374},{1.020}) {\tikz\pgfuseplotmark{m6bv};};
\node[stars] at (axis cs:{251.458},{8.582}) {\tikz\pgfuseplotmark{m5cb};};
\node[stars] at (axis cs:{251.700},{-39.377}) {\tikz\pgfuseplotmark{m5c};};
\node[stars] at (axis cs:{251.791},{2.064}) {\tikz\pgfuseplotmark{m6bvb};};
\node[stars] at (axis cs:{251.943},{5.247}) {\tikz\pgfuseplotmark{m5cv};};
\node[stars] at (axis cs:{252.037},{13.590}) {\tikz\pgfuseplotmark{m6c};};
\node[stars] at (axis cs:{252.112},{-14.909}) {\tikz\pgfuseplotmark{m6b};};
\node[stars] at (axis cs:{252.365},{-15.668}) {\tikz\pgfuseplotmark{m6cv};};
\node[stars] at (axis cs:{252.394},{13.261}) {\tikz\pgfuseplotmark{m6b};};
\node[stars] at (axis cs:{252.458},{-10.783}) {\tikz\pgfuseplotmark{m5a};};
\node[stars] at (axis cs:{252.541},{-34.293}) {\tikz\pgfuseplotmark{m2cv};};
\node[stars] at (axis cs:{252.581},{7.247}) {\tikz\pgfuseplotmark{m5cv};};
\node[stars] at (axis cs:{252.593},{-2.654}) {\tikz\pgfuseplotmark{m6c};};
\node[stars] at (axis cs:{252.662},{29.806}) {\tikz\pgfuseplotmark{m6a};};
\node[stars] at (axis cs:{252.680},{32.554}) {\tikz\pgfuseplotmark{m6c};};
\node[stars] at (axis cs:{252.750},{-37.514}) {\tikz\pgfuseplotmark{m6cvb};};
\node[stars] at (axis cs:{252.854},{1.216}) {\tikz\pgfuseplotmark{m6a};};
\node[stars] at (axis cs:{252.939},{24.656}) {\tikz\pgfuseplotmark{m5b};};
\node[stars] at (axis cs:{252.968},{-38.047}) {\tikz\pgfuseplotmark{m3bv};};
\node[stars] at (axis cs:{253.084},{-38.018}) {\tikz\pgfuseplotmark{m4a};};
\node[stars] at (axis cs:{253.242},{31.702}) {\tikz\pgfuseplotmark{m5c};};
\node[stars] at (axis cs:{253.355},{-20.415}) {\tikz\pgfuseplotmark{m6b};};
\node[stars] at (axis cs:{253.502},{10.165}) {\tikz\pgfuseplotmark{m4c};};
\node[stars] at (axis cs:{253.544},{-1.612}) {\tikz\pgfuseplotmark{m6c};};
\node[stars] at (axis cs:{253.649},{-6.154}) {\tikz\pgfuseplotmark{m5cv};};
\node[stars] at (axis cs:{253.650},{-30.587}) {\tikz\pgfuseplotmark{m6cv};};
\node[stars] at (axis cs:{253.730},{20.958}) {\tikz\pgfuseplotmark{m5c};};
\node[stars] at (axis cs:{253.759},{25.730}) {\tikz\pgfuseplotmark{m6b};};
\node[stars] at (axis cs:{253.817},{13.620}) {\tikz\pgfuseplotmark{m6cv};};
\node[stars] at (axis cs:{253.842},{18.433}) {\tikz\pgfuseplotmark{m5c};};
\node[stars] at (axis cs:{253.991},{-33.507}) {\tikz\pgfuseplotmark{m6c};};
\node[stars] at (axis cs:{254.008},{-16.806}) {\tikz\pgfuseplotmark{m6c};};
\node[stars] at (axis cs:{254.200},{-23.150}) {\tikz\pgfuseplotmark{m6a};};
\node[stars] at (axis cs:{254.267},{-19.540}) {\tikz\pgfuseplotmark{m6cb};};
\node[stars] at (axis cs:{254.297},{-33.259}) {\tikz\pgfuseplotmark{m5c};};
\node[stars] at (axis cs:{254.358},{-10.963}) {\tikz\pgfuseplotmark{m6c};};
\node[stars] at (axis cs:{254.379},{25.353}) {\tikz\pgfuseplotmark{m6c};};
\node[stars] at (axis cs:{254.383},{13.884}) {\tikz\pgfuseplotmark{m6c};};
\node[stars] at (axis cs:{254.417},{9.375}) {\tikz\pgfuseplotmark{m3cv};};
\node[stars] at (axis cs:{254.426},{24.381}) {\tikz\pgfuseplotmark{m6c};};
\node[stars] at (axis cs:{254.673},{-14.870}) {\tikz\pgfuseplotmark{m6c};};
\node[stars] at (axis cs:{254.718},{-37.620}) {\tikz\pgfuseplotmark{m6bv};};
\node[stars] at (axis cs:{254.990},{-25.092}) {\tikz\pgfuseplotmark{m6b};};
\node[stars] at (axis cs:{255.040},{-24.989}) {\tikz\pgfuseplotmark{m6a};};
\node[stars] at (axis cs:{255.072},{30.926}) {\tikz\pgfuseplotmark{m4b};};
\node[stars] at (axis cs:{255.154},{-35.934}) {\tikz\pgfuseplotmark{m6b};};
\node[stars] at (axis cs:{255.242},{22.632}) {\tikz\pgfuseplotmark{m6a};};
\node[stars] at (axis cs:{255.265},{-4.223}) {\tikz\pgfuseplotmark{m5ab};};
\node[stars] at (axis cs:{255.388},{14.949}) {\tikz\pgfuseplotmark{m6cvb};};
\node[stars] at (axis cs:{255.402},{33.568}) {\tikz\pgfuseplotmark{m5cv};};
\node[stars] at (axis cs:{255.464},{-18.886}) {\tikz\pgfuseplotmark{m6c};};
\node[stars] at (axis cs:{255.469},{-32.143}) {\tikz\pgfuseplotmark{m5b};};
\node[stars] at (axis cs:{255.496},{8.451}) {\tikz\pgfuseplotmark{m6cb};};
\node[stars] at (axis cs:{255.571},{31.884}) {\tikz\pgfuseplotmark{m6c};};
\node[stars] at (axis cs:{255.578},{25.506}) {\tikz\pgfuseplotmark{m6a};};
\node[stars] at (axis cs:{255.783},{14.092}) {\tikz\pgfuseplotmark{m5bv};};
\node[stars] at (axis cs:{255.876},{35.414}) {\tikz\pgfuseplotmark{m6cv};};
\node[stars] at (axis cs:{255.914},{13.605}) {\tikz\pgfuseplotmark{m6bvb};};
\node[stars] at (axis cs:{255.962},{-38.153}) {\tikz\pgfuseplotmark{m6bv};};
\node[stars] at (axis cs:{255.969},{19.690}) {\tikz\pgfuseplotmark{m6cb};};
\node[stars] at (axis cs:{255.973},{34.790}) {\tikz\pgfuseplotmark{m6b};};
\node[stars] at (axis cs:{256.172},{19.599}) {\tikz\pgfuseplotmark{m6cb};};
\node[stars] at (axis cs:{256.189},{-20.495}) {\tikz\pgfuseplotmark{m6c};};
\node[stars] at (axis cs:{256.206},{-34.123}) {\tikz\pgfuseplotmark{m5av};};
\node[stars] at (axis cs:{256.320},{0.703}) {\tikz\pgfuseplotmark{m6bv};};
\node[stars] at (axis cs:{256.345},{12.741}) {\tikz\pgfuseplotmark{m5b};};
\node[stars] at (axis cs:{256.384},{-0.892}) {\tikz\pgfuseplotmark{m6a};};
\node[stars] at (axis cs:{256.540},{9.734}) {\tikz\pgfuseplotmark{m6c};};
\node[stars] at (axis cs:{256.549},{-21.564}) {\tikz\pgfuseplotmark{m6c};};
\node[stars] at (axis cs:{256.554},{10.454}) {\tikz\pgfuseplotmark{m6c};};
\node[stars] at (axis cs:{256.575},{22.084}) {\tikz\pgfuseplotmark{m6a};};
\node[stars] at (axis cs:{256.584},{-37.227}) {\tikz\pgfuseplotmark{m6b};};
\node[stars] at (axis cs:{256.618},{-35.451}) {\tikz\pgfuseplotmark{m6cv};};
\node[stars] at (axis cs:{256.721},{-1.656}) {\tikz\pgfuseplotmark{m6cb};};
\node[stars] at (axis cs:{256.722},{-26.513}) {\tikz\pgfuseplotmark{m6c};};
\node[stars] at (axis cs:{257.003},{31.206}) {\tikz\pgfuseplotmark{m6c};};
\node[stars] at (axis cs:{257.009},{35.935}) {\tikz\pgfuseplotmark{m5c};};
\node[stars] at (axis cs:{257.057},{-1.079}) {\tikz\pgfuseplotmark{m6b};};
\node[stars] at (axis cs:{257.062},{-17.609}) {\tikz\pgfuseplotmark{m6b};};
\node[stars] at (axis cs:{257.198},{-30.404}) {\tikz\pgfuseplotmark{m6b};};
\node[stars] at (axis cs:{257.227},{-3.883}) {\tikz\pgfuseplotmark{m6c};};
\node[stars] at (axis cs:{257.450},{-10.523}) {\tikz\pgfuseplotmark{m5c};};
\node[stars] at (axis cs:{257.595},{-15.725}) {\tikz\pgfuseplotmark{m2c};};
\node[stars] at (axis cs:{257.763},{24.238}) {\tikz\pgfuseplotmark{m6cv};};
\node[stars] at (axis cs:{257.938},{7.895}) {\tikz\pgfuseplotmark{m6c};};
\node[stars] at (axis cs:{258.068},{-39.507}) {\tikz\pgfuseplotmark{m6a};};
\node[stars] at (axis cs:{258.069},{-38.822}) {\tikz\pgfuseplotmark{m6c};};
\node[stars] at (axis cs:{258.104},{-27.762}) {\tikz\pgfuseplotmark{m6b};};
\node[stars] at (axis cs:{258.116},{10.585}) {\tikz\pgfuseplotmark{m5c};};
\node[stars] at (axis cs:{258.244},{-32.438}) {\tikz\pgfuseplotmark{m6b};};
\node[stars] at (axis cs:{258.615},{-39.766}) {\tikz\pgfuseplotmark{m6cvb};};
\node[stars] at (axis cs:{258.662},{14.390}) {\tikz\pgfuseplotmark{m3cvb};};
\node[stars] at (axis cs:{258.758},{24.839}) {\tikz\pgfuseplotmark{m3b};};
\node[stars] at (axis cs:{258.762},{36.809}) {\tikz\pgfuseplotmark{m3cv};};
\node[stars] at (axis cs:{258.830},{-33.548}) {\tikz\pgfuseplotmark{m6av};};
\node[stars] at (axis cs:{258.835},{-14.584}) {\tikz\pgfuseplotmark{m6b};};
\node[stars] at (axis cs:{258.837},{-26.602}) {\tikz\pgfuseplotmark{m5bb};};
\node[stars] at (axis cs:{258.900},{-38.594}) {\tikz\pgfuseplotmark{m6b};};
\node[stars] at (axis cs:{258.923},{23.743}) {\tikz\pgfuseplotmark{m6b};};
\node[stars] at (axis cs:{258.964},{-30.211}) {\tikz\pgfuseplotmark{m6c};};
\node[stars] at (axis cs:{259.059},{2.186}) {\tikz\pgfuseplotmark{m6cv};};
\node[stars] at (axis cs:{259.090},{-35.750}) {\tikz\pgfuseplotmark{m6b};};
\node[stars] at (axis cs:{259.132},{1.210}) {\tikz\pgfuseplotmark{m6bv};};
\node[stars] at (axis cs:{259.153},{-0.445}) {\tikz\pgfuseplotmark{m5a};};
\node[stars] at (axis cs:{259.178},{-6.245}) {\tikz\pgfuseplotmark{m6b};};
\node[stars] at (axis cs:{259.265},{-32.663}) {\tikz\pgfuseplotmark{m6a};};
\node[stars] at (axis cs:{259.332},{33.100}) {\tikz\pgfuseplotmark{m5av};};
\node[stars] at (axis cs:{259.399},{23.091}) {\tikz\pgfuseplotmark{m6c};};
\node[stars] at (axis cs:{259.418},{37.291}) {\tikz\pgfuseplotmark{m5av};};
\node[stars] at (axis cs:{259.503},{-24.287}) {\tikz\pgfuseplotmark{m5bb};};
\node[stars] at (axis cs:{259.521},{17.318}) {\tikz\pgfuseplotmark{m6b};};
\node[stars] at (axis cs:{259.580},{-16.312}) {\tikz\pgfuseplotmark{m6c};};
\node[stars] at (axis cs:{259.585},{-32.553}) {\tikz\pgfuseplotmark{m6cv};};
\node[stars] at (axis cs:{259.597},{38.811}) {\tikz\pgfuseplotmark{m6b};};
\node[stars] at (axis cs:{259.654},{10.864}) {\tikz\pgfuseplotmark{m5b};};
\node[stars] at (axis cs:{259.702},{28.823}) {\tikz\pgfuseplotmark{m6a};};
\node[stars] at (axis cs:{259.738},{-34.990}) {\tikz\pgfuseplotmark{m6bvb};};
\node[stars] at (axis cs:{259.972},{-17.756}) {\tikz\pgfuseplotmark{m6bb};};
\node[stars] at (axis cs:{259.998},{-5.917}) {\tikz\pgfuseplotmark{m6c};};
\node[stars] at (axis cs:{260.041},{25.538}) {\tikz\pgfuseplotmark{m5c};};
\node[stars] at (axis cs:{260.079},{18.057}) {\tikz\pgfuseplotmark{m5bv};};
\node[stars] at (axis cs:{260.165},{32.468}) {\tikz\pgfuseplotmark{m5cvb};};
\node[stars] at (axis cs:{260.207},{-12.847}) {\tikz\pgfuseplotmark{m4c};};
\node[stars] at (axis cs:{260.219},{-10.696}) {\tikz\pgfuseplotmark{m6c};};
\node[stars] at (axis cs:{260.226},{24.499}) {\tikz\pgfuseplotmark{m5bb};};
\node[stars] at (axis cs:{260.252},{-21.113}) {\tikz\pgfuseplotmark{m4c};};
\node[stars] at (axis cs:{260.380},{28.758}) {\tikz\pgfuseplotmark{m6c};};
\node[stars] at (axis cs:{260.389},{16.731}) {\tikz\pgfuseplotmark{m6c};};
\node[stars] at (axis cs:{260.423},{-24.906}) {\tikz\pgfuseplotmark{m6c};};
\node[stars] at (axis cs:{260.432},{39.975}) {\tikz\pgfuseplotmark{m6av};};
\node[stars] at (axis cs:{260.502},{-24.999}) {\tikz\pgfuseplotmark{m3cv};};
\node[stars] at (axis cs:{260.658},{-35.911}) {\tikz\pgfuseplotmark{m6c};};
\node[stars] at (axis cs:{260.663},{-37.805}) {\tikz\pgfuseplotmark{m6cv};};
\node[stars] at (axis cs:{260.714},{-2.388}) {\tikz\pgfuseplotmark{m6c};};
\node[stars] at (axis cs:{260.728},{-37.221}) {\tikz\pgfuseplotmark{m6b};};
\node[stars] at (axis cs:{260.840},{-28.143}) {\tikz\pgfuseplotmark{m5c};};
\node[stars] at (axis cs:{260.921},{37.146}) {\tikz\pgfuseplotmark{m5avb};};
\node[stars] at (axis cs:{260.990},{8.853}) {\tikz\pgfuseplotmark{m6a};};
\node[stars] at (axis cs:{261.027},{22.960}) {\tikz\pgfuseplotmark{m6av};};
\node[stars] at (axis cs:{261.113},{36.952}) {\tikz\pgfuseplotmark{m6cb};};
\node[stars] at (axis cs:{261.131},{16.301}) {\tikz\pgfuseplotmark{m6a};};
\node[stars] at (axis cs:{261.141},{15.606}) {\tikz\pgfuseplotmark{m6cb};};
\node[stars] at (axis cs:{261.154},{-18.446}) {\tikz\pgfuseplotmark{m6c};};
\node[stars] at (axis cs:{261.175},{-21.441}) {\tikz\pgfuseplotmark{m6a};};
\node[stars] at (axis cs:{261.261},{-34.696}) {\tikz\pgfuseplotmark{m6c};};
\node[stars] at (axis cs:{261.276},{-24.244}) {\tikz\pgfuseplotmark{m6c};};
\node[stars] at (axis cs:{261.477},{16.918}) {\tikz\pgfuseplotmark{m6cv};};
\node[stars] at (axis cs:{261.491},{-1.652}) {\tikz\pgfuseplotmark{m6c};};
\node[stars] at (axis cs:{261.504},{26.879}) {\tikz\pgfuseplotmark{m6c};};
\node[stars] at (axis cs:{261.579},{7.595}) {\tikz\pgfuseplotmark{m6b};};
\node[stars] at (axis cs:{261.593},{-24.175}) {\tikz\pgfuseplotmark{m4c};};
\node[stars] at (axis cs:{261.629},{4.140}) {\tikz\pgfuseplotmark{m4c};};
\node[stars] at (axis cs:{261.658},{-5.087}) {\tikz\pgfuseplotmark{m5a};};
\node[stars] at (axis cs:{261.692},{34.696}) {\tikz\pgfuseplotmark{m6b};};
\node[stars] at (axis cs:{261.705},{20.081}) {\tikz\pgfuseplotmark{m6a};};
\node[stars] at (axis cs:{261.730},{-25.943}) {\tikz\pgfuseplotmark{m6c};};
\node[stars] at (axis cs:{261.759},{-12.512}) {\tikz\pgfuseplotmark{m6c};};
\node[stars] at (axis cs:{261.791},{-15.853}) {\tikz\pgfuseplotmark{m6c};};
\node[stars] at (axis cs:{261.839},{-29.867}) {\tikz\pgfuseplotmark{m4c};};
\node[stars] at (axis cs:{261.906},{-29.724}) {\tikz\pgfuseplotmark{m6b};};
\node[stars] at (axis cs:{261.934},{8.442}) {\tikz\pgfuseplotmark{m6cv};};
\node[stars] at (axis cs:{262.010},{-8.208}) {\tikz\pgfuseplotmark{m6c};};
\node[stars] at (axis cs:{262.207},{0.330}) {\tikz\pgfuseplotmark{m5c};};
\node[stars] at (axis cs:{262.234},{-36.778}) {\tikz\pgfuseplotmark{m6b};};
\node[stars] at (axis cs:{262.357},{-38.517}) {\tikz\pgfuseplotmark{m6c};};
\node[stars] at (axis cs:{262.447},{-5.919}) {\tikz\pgfuseplotmark{m6cb};};
\node[stars] at (axis cs:{262.594},{11.924}) {\tikz\pgfuseplotmark{m6c};};
\node[stars] at (axis cs:{262.599},{-1.063}) {\tikz\pgfuseplotmark{m5cb};};
\node[stars] at (axis cs:{262.668},{38.882}) {\tikz\pgfuseplotmark{m6c};};
\node[stars] at (axis cs:{262.685},{26.110}) {\tikz\pgfuseplotmark{m4c};};
\node[stars] at (axis cs:{262.691},{-37.296}) {\tikz\pgfuseplotmark{m3a};};
\node[stars] at (axis cs:{262.731},{31.158}) {\tikz\pgfuseplotmark{m6a};};
\node[stars] at (axis cs:{262.839},{2.724}) {\tikz\pgfuseplotmark{m6a};};
\node[stars] at (axis cs:{262.854},{-23.962}) {\tikz\pgfuseplotmark{m5a};};
\node[stars] at (axis cs:{262.935},{-26.270}) {\tikz\pgfuseplotmark{m6b};};
\node[stars] at (axis cs:{262.947},{-33.703}) {\tikz\pgfuseplotmark{m6c};};
\node[stars] at (axis cs:{262.957},{28.407}) {\tikz\pgfuseplotmark{m6a};};
\node[stars] at (axis cs:{263.062},{11.930}) {\tikz\pgfuseplotmark{m6c};};
\node[stars] at (axis cs:{263.103},{-34.279}) {\tikz\pgfuseplotmark{m6cv};};
\node[stars] at (axis cs:{263.345},{19.257}) {\tikz\pgfuseplotmark{m6a};};
\node[stars] at (axis cs:{263.374},{-5.745}) {\tikz\pgfuseplotmark{m6a};};
\node[stars] at (axis cs:{263.402},{-37.104}) {\tikz\pgfuseplotmark{m2av};};
\node[stars] at (axis cs:{263.414},{16.318}) {\tikz\pgfuseplotmark{m6a};};
\node[stars] at (axis cs:{263.428},{14.841}) {\tikz\pgfuseplotmark{m6cv};};
\node[stars] at (axis cs:{263.653},{9.587}) {\tikz\pgfuseplotmark{m6ab};};
\node[stars] at (axis cs:{263.677},{-32.581}) {\tikz\pgfuseplotmark{m6av};};
\node[stars] at (axis cs:{263.693},{-11.242}) {\tikz\pgfuseplotmark{m6a};};
\node[stars] at (axis cs:{263.734},{12.560}) {\tikz\pgfuseplotmark{m2b};};
\node[stars] at (axis cs:{263.827},{-22.044}) {\tikz\pgfuseplotmark{m6cv};};
\node[stars] at (axis cs:{263.927},{37.302}) {\tikz\pgfuseplotmark{m6b};};
\node[stars] at (axis cs:{263.929},{-37.440}) {\tikz\pgfuseplotmark{m6c};};
\node[stars] at (axis cs:{263.998},{20.996}) {\tikz\pgfuseplotmark{m6bb};};
\node[stars] at (axis cs:{264.034},{28.185}) {\tikz\pgfuseplotmark{m6c};};
\node[stars] at (axis cs:{264.089},{27.566}) {\tikz\pgfuseplotmark{m6cv};};
\node[stars] at (axis cs:{264.137},{-38.635}) {\tikz\pgfuseplotmark{m4c};};
\node[stars] at (axis cs:{264.153},{30.785}) {\tikz\pgfuseplotmark{m6b};};
\node[stars] at (axis cs:{264.362},{-38.066}) {\tikz\pgfuseplotmark{m6c};};
\node[stars] at (axis cs:{264.380},{24.310}) {\tikz\pgfuseplotmark{m6a};};
\node[stars] at (axis cs:{264.397},{-15.398}) {\tikz\pgfuseplotmark{m4a};};
\node[stars] at (axis cs:{264.401},{-15.571}) {\tikz\pgfuseplotmark{m6b};};
\node[stars] at (axis cs:{264.461},{-8.119}) {\tikz\pgfuseplotmark{m5a};};
\node[stars] at (axis cs:{264.540},{-10.926}) {\tikz\pgfuseplotmark{m6a};};
\node[stars] at (axis cs:{264.687},{-21.912}) {\tikz\pgfuseplotmark{m6c};};
\node[stars] at (axis cs:{264.741},{13.329}) {\tikz\pgfuseplotmark{m6c};};
\node[stars] at (axis cs:{264.785},{2.028}) {\tikz\pgfuseplotmark{m6cb};};
\node[stars] at (axis cs:{264.990},{31.202}) {\tikz\pgfuseplotmark{m6b};};
\node[stars] at (axis cs:{265.050},{-2.152}) {\tikz\pgfuseplotmark{m6c};};
\node[stars] at (axis cs:{265.172},{31.287}) {\tikz\pgfuseplotmark{m6cb};};
\node[stars] at (axis cs:{265.244},{-32.214}) {\tikz\pgfuseplotmark{m6bv};};
\node[stars] at (axis cs:{265.273},{24.513}) {\tikz\pgfuseplotmark{m6cb};};
\node[stars] at (axis cs:{265.296},{15.178}) {\tikz\pgfuseplotmark{m6c};};
\node[stars] at (axis cs:{265.354},{-12.875}) {\tikz\pgfuseplotmark{m4cv};};
\node[stars] at (axis cs:{265.385},{6.313}) {\tikz\pgfuseplotmark{m6b};};
\node[stars] at (axis cs:{265.494},{15.952}) {\tikz\pgfuseplotmark{m6ab};};
\node[stars] at (axis cs:{265.618},{24.564}) {\tikz\pgfuseplotmark{m6ab};};
\node[stars] at (axis cs:{265.622},{-39.030}) {\tikz\pgfuseplotmark{m2cv};};
\node[stars] at (axis cs:{265.713},{-36.945}) {\tikz\pgfuseplotmark{m6a};};
\node[stars] at (axis cs:{265.779},{-33.051}) {\tikz\pgfuseplotmark{m6c};};
\node[stars] at (axis cs:{265.824},{-27.884}) {\tikz\pgfuseplotmark{m6c};};
\node[stars] at (axis cs:{265.840},{24.328}) {\tikz\pgfuseplotmark{m6a};};
\node[stars] at (axis cs:{265.842},{14.295}) {\tikz\pgfuseplotmark{m6c};};
\node[stars] at (axis cs:{265.857},{-21.683}) {\tikz\pgfuseplotmark{m5b};};
\node[stars] at (axis cs:{265.868},{4.567}) {\tikz\pgfuseplotmark{m3a};};
\node[stars] at (axis cs:{265.946},{-7.079}) {\tikz\pgfuseplotmark{m6c};};
\node[stars] at (axis cs:{265.952},{-13.509}) {\tikz\pgfuseplotmark{m6c};};
\node[stars] at (axis cs:{266.072},{14.410}) {\tikz\pgfuseplotmark{m6cv};};
\node[stars] at (axis cs:{266.142},{2.579}) {\tikz\pgfuseplotmark{m6cb};};
\node[stars] at (axis cs:{266.418},{31.505}) {\tikz\pgfuseplotmark{m6c};};
\node[stars] at (axis cs:{266.615},{27.721}) {\tikz\pgfuseplotmark{m3cb};};
\node[stars] at (axis cs:{266.780},{-38.112}) {\tikz\pgfuseplotmark{m6c};};
\node[stars] at (axis cs:{266.784},{17.697}) {\tikz\pgfuseplotmark{m6a};};
\node[stars] at (axis cs:{266.890},{-27.831}) {\tikz\pgfuseplotmark{m5av};};
\node[stars] at (axis cs:{266.903},{-14.726}) {\tikz\pgfuseplotmark{m6b};};
\node[stars] at (axis cs:{266.940},{-22.478}) {\tikz\pgfuseplotmark{m6cv};};
\node[stars] at (axis cs:{266.973},{2.707}) {\tikz\pgfuseplotmark{m4a};};
\node[stars] at (axis cs:{267.084},{3.804}) {\tikz\pgfuseplotmark{m6c};};
\node[stars] at (axis cs:{267.103},{20.565}) {\tikz\pgfuseplotmark{m6a};};
\node[stars] at (axis cs:{267.116},{-26.975}) {\tikz\pgfuseplotmark{m6cv};};
\node[stars] at (axis cs:{267.199},{19.255}) {\tikz\pgfuseplotmark{m6c};};
\node[stars] at (axis cs:{267.205},{25.623}) {\tikz\pgfuseplotmark{m5b};};
\node[stars] at (axis cs:{267.294},{-31.703}) {\tikz\pgfuseplotmark{m5a};};
\node[stars] at (axis cs:{267.329},{1.961}) {\tikz\pgfuseplotmark{m6c};};
\node[stars] at (axis cs:{267.465},{-37.043}) {\tikz\pgfuseplotmark{m3c};};
\node[stars] at (axis cs:{267.595},{29.322}) {\tikz\pgfuseplotmark{m6a};};
\node[stars] at (axis cs:{267.682},{11.946}) {\tikz\pgfuseplotmark{m6c};};
\node[stars] at (axis cs:{267.702},{22.316}) {\tikz\pgfuseplotmark{m6c};};
\node[stars] at (axis cs:{267.994},{15.326}) {\tikz\pgfuseplotmark{m6c};};
\node[stars] at (axis cs:{267.998},{-1.237}) {\tikz\pgfuseplotmark{m6c};};
\node[stars] at (axis cs:{268.020},{39.982}) {\tikz\pgfuseplotmark{m6b};};
\node[stars] at (axis cs:{268.057},{-34.799}) {\tikz\pgfuseplotmark{m6bv};};
\node[stars] at (axis cs:{268.082},{-34.417}) {\tikz\pgfuseplotmark{m6a};};
\node[stars] at (axis cs:{268.148},{1.305}) {\tikz\pgfuseplotmark{m6bv};};
\node[stars] at (axis cs:{268.161},{-6.143}) {\tikz\pgfuseplotmark{m6cv};};
\node[stars] at (axis cs:{268.205},{-34.115}) {\tikz\pgfuseplotmark{m6b};};
\node[stars] at (axis cs:{268.240},{-35.624}) {\tikz\pgfuseplotmark{m6b};};
\node[stars] at (axis cs:{268.265},{-10.899}) {\tikz\pgfuseplotmark{m6cv};};
\node[stars] at (axis cs:{268.309},{6.101}) {\tikz\pgfuseplotmark{m6a};};
\node[stars] at (axis cs:{268.332},{-34.731}) {\tikz\pgfuseplotmark{m6b};};
\node[stars] at (axis cs:{268.348},{-34.895}) {\tikz\pgfuseplotmark{m6a};};
\node[stars] at (axis cs:{268.440},{-34.786}) {\tikz\pgfuseplotmark{m6c};};
\node[stars] at (axis cs:{268.478},{-34.753}) {\tikz\pgfuseplotmark{m6bv};};
\node[stars] at (axis cs:{268.492},{-34.831}) {\tikz\pgfuseplotmark{m6cv};};
\node[stars] at (axis cs:{268.559},{11.130}) {\tikz\pgfuseplotmark{m6cv};};
\node[stars] at (axis cs:{268.613},{-34.467}) {\tikz\pgfuseplotmark{m6b};};
\node[stars] at (axis cs:{268.725},{-24.887}) {\tikz\pgfuseplotmark{m6c};};
\node[stars] at (axis cs:{268.783},{-36.475}) {\tikz\pgfuseplotmark{m6b};};
\node[stars] at (axis cs:{268.855},{26.050}) {\tikz\pgfuseplotmark{m5cv};};
\node[stars] at (axis cs:{268.962},{22.464}) {\tikz\pgfuseplotmark{m6a};};
\node[stars] at (axis cs:{268.979},{-18.802}) {\tikz\pgfuseplotmark{m6c};};
\node[stars] at (axis cs:{269.063},{37.251}) {\tikz\pgfuseplotmark{m4av};};
\node[stars] at (axis cs:{269.077},{0.670}) {\tikz\pgfuseplotmark{m6av};};
\node[stars] at (axis cs:{269.079},{-15.812}) {\tikz\pgfuseplotmark{m6bb};};
\node[stars] at (axis cs:{269.174},{-28.065}) {\tikz\pgfuseplotmark{m6a};};
\node[stars] at (axis cs:{269.199},{-4.082}) {\tikz\pgfuseplotmark{m5c};};
\node[stars] at (axis cs:{269.233},{6.488}) {\tikz\pgfuseplotmark{m6cv};};
\node[stars] at (axis cs:{269.268},{0.066}) {\tikz\pgfuseplotmark{m6bv};};
\node[stars] at (axis cs:{269.310},{23.996}) {\tikz\pgfuseplotmark{m6c};};
\node[stars] at (axis cs:{269.362},{11.044}) {\tikz\pgfuseplotmark{m6c};};
\node[stars] at (axis cs:{269.441},{29.248}) {\tikz\pgfuseplotmark{m4av};};
\node[stars] at (axis cs:{269.491},{-39.136}) {\tikz\pgfuseplotmark{m6cb};};
\node[stars] at (axis cs:{269.626},{30.189}) {\tikz\pgfuseplotmark{m4cv};};
\node[stars] at (axis cs:{269.663},{-28.759}) {\tikz\pgfuseplotmark{m6b};};
\node[stars] at (axis cs:{269.676},{36.288}) {\tikz\pgfuseplotmark{m6b};};
\node[stars] at (axis cs:{269.732},{-36.858}) {\tikz\pgfuseplotmark{m6ab};};
\node[stars] at (axis cs:{269.757},{-9.774}) {\tikz\pgfuseplotmark{m3c};};
\node[stars] at (axis cs:{269.772},{-30.253}) {\tikz\pgfuseplotmark{m5cb};};
\node[stars] at (axis cs:{269.903},{-4.821}) {\tikz\pgfuseplotmark{m6a};};
\node[stars] at (axis cs:{269.948},{-23.816}) {\tikz\pgfuseplotmark{m5a};};
\node[stars] at (axis cs:{270.000},{-20.339}) {\tikz\pgfuseplotmark{m6c};};
\node[stars] at (axis cs:{270.014},{16.751}) {\tikz\pgfuseplotmark{m5a};};
\node[stars] at (axis cs:{270.065},{0.629}) {\tikz\pgfuseplotmark{m6cv};};
\node[stars] at (axis cs:{270.066},{4.369}) {\tikz\pgfuseplotmark{m5av};};
\node[stars] at (axis cs:{270.115},{19.506}) {\tikz\pgfuseplotmark{m6cv};};
\node[stars] at (axis cs:{270.121},{-3.690}) {\tikz\pgfuseplotmark{m5a};};
\node[stars] at (axis cs:{270.152},{33.213}) {\tikz\pgfuseplotmark{m6b};};
\node[stars] at (axis cs:{270.161},{2.931}) {\tikz\pgfuseplotmark{m4bb};};
\node[stars] at (axis cs:{270.220},{6.268}) {\tikz\pgfuseplotmark{m6c};};
\node[stars] at (axis cs:{270.239},{15.093}) {\tikz\pgfuseplotmark{m6c};};
\node[stars] at (axis cs:{270.346},{-17.157}) {\tikz\pgfuseplotmark{m6c};};
\node[stars] at (axis cs:{270.377},{21.596}) {\tikz\pgfuseplotmark{m5bb};};
\node[stars] at (axis cs:{270.400},{33.311}) {\tikz\pgfuseplotmark{m6c};};
\node[stars] at (axis cs:{270.438},{1.305}) {\tikz\pgfuseplotmark{m4c};};
\node[stars] at (axis cs:{270.451},{-36.378}) {\tikz\pgfuseplotmark{m6c};};
\node[stars] at (axis cs:{270.477},{-22.780}) {\tikz\pgfuseplotmark{m6a};};
\node[stars] at (axis cs:{270.596},{20.834}) {\tikz\pgfuseplotmark{m5cv};};
\node[stars] at (axis cs:{270.626},{22.923}) {\tikz\pgfuseplotmark{m6c};};
\node[stars] at (axis cs:{270.713},{-24.282}) {\tikz\pgfuseplotmark{m5c};};
\node[stars] at (axis cs:{270.770},{-8.180}) {\tikz\pgfuseplotmark{m5ab};};
\node[stars] at (axis cs:{270.811},{19.613}) {\tikz\pgfuseplotmark{m6c};};
\node[stars] at (axis cs:{270.969},{-24.361}) {\tikz\pgfuseplotmark{m6b};};
\node[stars] at (axis cs:{271.156},{1.919}) {\tikz\pgfuseplotmark{m6cv};};
\node[stars] at (axis cs:{271.168},{23.942}) {\tikz\pgfuseplotmark{m6cv};};
\node[stars] at (axis cs:{271.210},{-35.901}) {\tikz\pgfuseplotmark{m6b};};
\node[stars] at (axis cs:{271.255},{-29.580}) {\tikz\pgfuseplotmark{m5av};};
\node[stars] at (axis cs:{271.364},{2.500}) {\tikz\pgfuseplotmark{m4bvb};};
\node[stars] at (axis cs:{271.376},{21.646}) {\tikz\pgfuseplotmark{m6c};};
\node[stars] at (axis cs:{271.452},{-30.424}) {\tikz\pgfuseplotmark{m4a};};
\node[stars] at (axis cs:{271.457},{32.230}) {\tikz\pgfuseplotmark{m6a};};
\node[stars] at (axis cs:{271.508},{22.219}) {\tikz\pgfuseplotmark{m5bv};};
\node[stars] at (axis cs:{271.531},{-8.324}) {\tikz\pgfuseplotmark{m6a};};
\node[stars] at (axis cs:{271.531},{-0.446}) {\tikz\pgfuseplotmark{m6c};};
\node[stars] at (axis cs:{271.563},{-4.751}) {\tikz\pgfuseplotmark{m6a};};
\node[stars] at (axis cs:{271.599},{-36.020}) {\tikz\pgfuseplotmark{m6b};};
\node[stars] at (axis cs:{271.756},{30.562}) {\tikz\pgfuseplotmark{m5bb};};
\node[stars] at (axis cs:{271.797},{-21.444}) {\tikz\pgfuseplotmark{m6c};};
\node[stars] at (axis cs:{271.827},{8.734}) {\tikz\pgfuseplotmark{m5a};};
\node[stars] at (axis cs:{271.837},{2.481}) {\tikz\pgfuseplotmark{m6c};};
\node[stars] at (axis cs:{271.837},{9.564}) {\tikz\pgfuseplotmark{m4ab};};
\node[stars] at (axis cs:{271.886},{28.762}) {\tikz\pgfuseplotmark{m4av};};
\node[stars] at (axis cs:{271.951},{-17.154}) {\tikz\pgfuseplotmark{m6a};};
\node[stars] at (axis cs:{271.957},{26.101}) {\tikz\pgfuseplotmark{m6ab};};
\node[stars] at (axis cs:{272.009},{36.401}) {\tikz\pgfuseplotmark{m5c};};
\node[stars] at (axis cs:{272.021},{-28.457}) {\tikz\pgfuseplotmark{m5a};};
\node[stars] at (axis cs:{272.141},{14.285}) {\tikz\pgfuseplotmark{m6c};};
\node[stars] at (axis cs:{272.190},{20.814}) {\tikz\pgfuseplotmark{m4c};};
\node[stars] at (axis cs:{272.220},{20.045}) {\tikz\pgfuseplotmark{m5b};};
\node[stars] at (axis cs:{272.293},{30.469}) {\tikz\pgfuseplotmark{m6c};};
\node[stars] at (axis cs:{272.391},{3.993}) {\tikz\pgfuseplotmark{m6av};};
\node[stars] at (axis cs:{272.406},{38.458}) {\tikz\pgfuseplotmark{m6cv};};
\node[stars] at (axis cs:{272.431},{-13.934}) {\tikz\pgfuseplotmark{m6c};};
\node[stars] at (axis cs:{272.475},{3.120}) {\tikz\pgfuseplotmark{m6a};};
\node[stars] at (axis cs:{272.496},{36.466}) {\tikz\pgfuseplotmark{m6a};};
\node[stars] at (axis cs:{272.500},{-32.720}) {\tikz\pgfuseplotmark{m6c};};
\node[stars] at (axis cs:{272.524},{-30.728}) {\tikz\pgfuseplotmark{m6ab};};
\node[stars] at (axis cs:{272.536},{16.476}) {\tikz\pgfuseplotmark{m6cvb};};
\node[stars] at (axis cs:{272.668},{3.324}) {\tikz\pgfuseplotmark{m5c};};
\node[stars] at (axis cs:{272.731},{-33.800}) {\tikz\pgfuseplotmark{m6c};};
\node[stars] at (axis cs:{272.931},{-23.701}) {\tikz\pgfuseplotmark{m5b};};
\node[stars] at (axis cs:{272.938},{33.447}) {\tikz\pgfuseplotmark{m6bv};};
\node[stars] at (axis cs:{272.976},{31.405}) {\tikz\pgfuseplotmark{m5bv};};
\node[stars] at (axis cs:{273.270},{38.773}) {\tikz\pgfuseplotmark{m6b};};
\node[stars] at (axis cs:{273.319},{21.880}) {\tikz\pgfuseplotmark{m6b};};
\node[stars] at (axis cs:{273.441},{-21.059}) {\tikz\pgfuseplotmark{m4avb};};
\node[stars] at (axis cs:{273.468},{2.393}) {\tikz\pgfuseplotmark{m6c};};
\node[stars] at (axis cs:{273.566},{-21.713}) {\tikz\pgfuseplotmark{m5cv};};
\node[stars] at (axis cs:{273.804},{-20.728}) {\tikz\pgfuseplotmark{m5cv};};
\node[stars] at (axis cs:{273.804},{-20.388}) {\tikz\pgfuseplotmark{m6b};};
\node[stars] at (axis cs:{273.878},{-18.661}) {\tikz\pgfuseplotmark{m6bv};};
\node[stars] at (axis cs:{273.991},{-3.618}) {\tikz\pgfuseplotmark{m6c};};
\node[stars] at (axis cs:{274.023},{2.377}) {\tikz\pgfuseplotmark{m6cv};};
\node[stars] at (axis cs:{274.221},{-3.007}) {\tikz\pgfuseplotmark{m6b};};
\node[stars] at (axis cs:{274.298},{-17.374}) {\tikz\pgfuseplotmark{m6a};};
\node[stars] at (axis cs:{274.349},{-28.289}) {\tikz\pgfuseplotmark{m6c};};
\node[stars] at (axis cs:{274.350},{-28.652}) {\tikz\pgfuseplotmark{m6c};};
\node[stars] at (axis cs:{274.351},{-9.758}) {\tikz\pgfuseplotmark{m6cv};};
\node[stars] at (axis cs:{274.401},{-34.107}) {\tikz\pgfuseplotmark{m6bvb};};
\node[stars] at (axis cs:{274.407},{-36.761}) {\tikz\pgfuseplotmark{m3bv};};
\node[stars] at (axis cs:{274.512},{13.777}) {\tikz\pgfuseplotmark{m6c};};
\node[stars] at (axis cs:{274.513},{-27.042}) {\tikz\pgfuseplotmark{m5a};};
\node[stars] at (axis cs:{274.532},{18.131}) {\tikz\pgfuseplotmark{m6c};};
\node[stars] at (axis cs:{274.532},{23.297}) {\tikz\pgfuseplotmark{m6cb};};
\node[stars] at (axis cs:{274.790},{7.260}) {\tikz\pgfuseplotmark{m5c};};
\node[stars] at (axis cs:{274.795},{24.446}) {\tikz\pgfuseplotmark{m5c};};
\node[stars] at (axis cs:{274.965},{36.064}) {\tikz\pgfuseplotmark{m4cv};};
\node[stars] at (axis cs:{274.967},{29.666}) {\tikz\pgfuseplotmark{m6b};};
\node[stars] at (axis cs:{275.037},{-15.831}) {\tikz\pgfuseplotmark{m5c};};
\node[stars] at (axis cs:{275.075},{21.961}) {\tikz\pgfuseplotmark{m5bv};};
\node[stars] at (axis cs:{275.217},{3.377}) {\tikz\pgfuseplotmark{m5av};};
\node[stars] at (axis cs:{275.230},{-37.488}) {\tikz\pgfuseplotmark{m6c};};
\node[stars] at (axis cs:{275.237},{29.859}) {\tikz\pgfuseplotmark{m6a};};
\node[stars] at (axis cs:{275.249},{-29.828}) {\tikz\pgfuseplotmark{m3av};};
\node[stars] at (axis cs:{275.254},{28.870}) {\tikz\pgfuseplotmark{m5b};};
\node[stars] at (axis cs:{275.328},{-2.899}) {\tikz\pgfuseplotmark{m3cv};};
\node[stars] at (axis cs:{275.346},{-18.860}) {\tikz\pgfuseplotmark{m6av};};
\node[stars] at (axis cs:{275.368},{5.436}) {\tikz\pgfuseplotmark{m6cv};};
\node[stars] at (axis cs:{275.381},{-24.915}) {\tikz\pgfuseplotmark{m6cv};};
\node[stars] at (axis cs:{275.501},{-28.430}) {\tikz\pgfuseplotmark{m6c};};
\node[stars] at (axis cs:{275.536},{23.285}) {\tikz\pgfuseplotmark{m5c};};
\node[stars] at (axis cs:{275.577},{-38.657}) {\tikz\pgfuseplotmark{m5b};};
\node[stars] at (axis cs:{275.647},{12.029}) {\tikz\pgfuseplotmark{m6b};};
\node[stars] at (axis cs:{275.704},{17.826}) {\tikz\pgfuseplotmark{m5c};};
\node[stars] at (axis cs:{275.721},{-36.670}) {\tikz\pgfuseplotmark{m5cv};};
\node[stars] at (axis cs:{275.759},{-10.219}) {\tikz\pgfuseplotmark{m6c};};
\node[stars] at (axis cs:{275.762},{16.688}) {\tikz\pgfuseplotmark{m6c};};
\node[stars] at (axis cs:{275.801},{-12.015}) {\tikz\pgfuseplotmark{m6a};};
\node[stars] at (axis cs:{275.870},{-36.238}) {\tikz\pgfuseplotmark{m6a};};
\node[stars] at (axis cs:{275.915},{-8.934}) {\tikz\pgfuseplotmark{m5a};};
\node[stars] at (axis cs:{275.925},{21.770}) {\tikz\pgfuseplotmark{m4av};};
\node[stars] at (axis cs:{275.989},{38.739}) {\tikz\pgfuseplotmark{m6cv};};
\node[stars] at (axis cs:{276.015},{-3.583}) {\tikz\pgfuseplotmark{m6c};};
\node[stars] at (axis cs:{276.043},{-34.385}) {\tikz\pgfuseplotmark{m2a};};
\node[stars] at (axis cs:{276.057},{39.507}) {\tikz\pgfuseplotmark{m5b};};
\node[stars] at (axis cs:{276.176},{-7.076}) {\tikz\pgfuseplotmark{m6c};};
\node[stars] at (axis cs:{276.238},{-1.579}) {\tikz\pgfuseplotmark{m6c};};
\node[stars] at (axis cs:{276.244},{27.395}) {\tikz\pgfuseplotmark{m6cb};};
\node[stars] at (axis cs:{276.256},{-30.756}) {\tikz\pgfuseplotmark{m6a};};
\node[stars] at (axis cs:{276.338},{-20.542}) {\tikz\pgfuseplotmark{m5ab};};
\node[stars] at (axis cs:{276.340},{-35.992}) {\tikz\pgfuseplotmark{m6c};};
\node[stars] at (axis cs:{276.412},{8.032}) {\tikz\pgfuseplotmark{m6av};};
\node[stars] at (axis cs:{276.478},{-33.946}) {\tikz\pgfuseplotmark{m6c};};
\node[stars] at (axis cs:{276.480},{14.966}) {\tikz\pgfuseplotmark{m6c};};
\node[stars] at (axis cs:{276.495},{29.829}) {\tikz\pgfuseplotmark{m6a};};
\node[stars] at (axis cs:{276.802},{0.196}) {\tikz\pgfuseplotmark{m5cvb};};
\node[stars] at (axis cs:{276.932},{-26.635}) {\tikz\pgfuseplotmark{m6c};};
\node[stars] at (axis cs:{276.956},{-29.817}) {\tikz\pgfuseplotmark{m6b};};
\node[stars] at (axis cs:{276.960},{3.748}) {\tikz\pgfuseplotmark{m6b};};
\node[stars] at (axis cs:{276.985},{-17.800}) {\tikz\pgfuseplotmark{m6cv};};
\node[stars] at (axis cs:{276.993},{-25.422}) {\tikz\pgfuseplotmark{m3a};};
\node[stars] at (axis cs:{276.995},{6.194}) {\tikz\pgfuseplotmark{m6a};};
\node[stars] at (axis cs:{277.026},{-26.757}) {\tikz\pgfuseplotmark{m6c};};
\node[stars] at (axis cs:{277.113},{-38.995}) {\tikz\pgfuseplotmark{m6a};};
\node[stars] at (axis cs:{277.299},{-14.566}) {\tikz\pgfuseplotmark{m5a};};
\node[stars] at (axis cs:{277.399},{23.866}) {\tikz\pgfuseplotmark{m6b};};
\node[stars] at (axis cs:{277.421},{-1.985}) {\tikz\pgfuseplotmark{m5c};};
\node[stars] at (axis cs:{277.445},{-14.582}) {\tikz\pgfuseplotmark{m6bv};};
\node[stars] at (axis cs:{277.549},{-18.729}) {\tikz\pgfuseplotmark{m6a};};
\node[stars] at (axis cs:{277.560},{-5.724}) {\tikz\pgfuseplotmark{m6c};};
\node[stars] at (axis cs:{277.674},{20.815}) {\tikz\pgfuseplotmark{m6cv};};
\node[stars] at (axis cs:{277.769},{16.928}) {\tikz\pgfuseplotmark{m6a};};
\node[stars] at (axis cs:{277.770},{-32.989}) {\tikz\pgfuseplotmark{m5c};};
\node[stars] at (axis cs:{277.857},{-10.796}) {\tikz\pgfuseplotmark{m6ab};};
\node[stars] at (axis cs:{277.860},{-18.403}) {\tikz\pgfuseplotmark{m5b};};
\node[stars] at (axis cs:{277.972},{-19.125}) {\tikz\pgfuseplotmark{m6cvb};};
\node[stars] at (axis cs:{277.987},{-1.003}) {\tikz\pgfuseplotmark{m6b};};
\node[stars] at (axis cs:{278.029},{3.659}) {\tikz\pgfuseplotmark{m6cv};};
\node[stars] at (axis cs:{278.086},{-14.644}) {\tikz\pgfuseplotmark{m6c};};
\node[stars] at (axis cs:{278.089},{-39.704}) {\tikz\pgfuseplotmark{m5c};};
\node[stars] at (axis cs:{278.181},{-14.865}) {\tikz\pgfuseplotmark{m5cv};};
\node[stars] at (axis cs:{278.192},{23.617}) {\tikz\pgfuseplotmark{m6a};};
\node[stars] at (axis cs:{278.208},{30.554}) {\tikz\pgfuseplotmark{m5c};};
\node[stars] at (axis cs:{278.254},{-39.892}) {\tikz\pgfuseplotmark{m6c};};
\node[stars] at (axis cs:{278.345},{-5.912}) {\tikz\pgfuseplotmark{m6c};};
\node[stars] at (axis cs:{278.346},{-38.726}) {\tikz\pgfuseplotmark{m6ab};};
\node[stars] at (axis cs:{278.347},{8.268}) {\tikz\pgfuseplotmark{m6cvb};};
\node[stars] at (axis cs:{278.413},{-14.854}) {\tikz\pgfuseplotmark{m6a};};
\node[stars] at (axis cs:{278.473},{-24.032}) {\tikz\pgfuseplotmark{m5c};};
\node[stars] at (axis cs:{278.491},{-33.016}) {\tikz\pgfuseplotmark{m5c};};
\node[stars] at (axis cs:{278.698},{10.892}) {\tikz\pgfuseplotmark{m6c};};
\node[stars] at (axis cs:{278.760},{-10.977}) {\tikz\pgfuseplotmark{m5bv};};
\node[stars] at (axis cs:{278.802},{-8.244}) {\tikz\pgfuseplotmark{m4av};};
\node[stars] at (axis cs:{278.803},{18.203}) {\tikz\pgfuseplotmark{m6a};};
\node[stars] at (axis cs:{278.806},{34.458}) {\tikz\pgfuseplotmark{m6b};};
\node[stars] at (axis cs:{278.839},{-20.840}) {\tikz\pgfuseplotmark{m6cv};};
\node[stars] at (axis cs:{278.877},{23.605}) {\tikz\pgfuseplotmark{m6a};};
\node[stars] at (axis cs:{278.902},{4.936}) {\tikz\pgfuseplotmark{m6cvb};};
\node[stars] at (axis cs:{278.972},{16.976}) {\tikz\pgfuseplotmark{m6cb};};
\node[stars] at (axis cs:{278.999},{-29.699}) {\tikz\pgfuseplotmark{m6c};};
\node[stars] at (axis cs:{279.116},{9.122}) {\tikz\pgfuseplotmark{m5c};};
\node[stars] at (axis cs:{279.156},{33.469}) {\tikz\pgfuseplotmark{m5cv};};
\node[stars] at (axis cs:{279.163},{6.672}) {\tikz\pgfuseplotmark{m5c};};
\node[stars] at (axis cs:{279.235},{38.784}) {\tikz\pgfuseplotmark{m1bvb};};
\node[stars] at (axis cs:{279.288},{16.198}) {\tikz\pgfuseplotmark{m6c};};
\node[stars] at (axis cs:{279.400},{-0.309}) {\tikz\pgfuseplotmark{m6av};};
\node[stars] at (axis cs:{279.477},{-21.397}) {\tikz\pgfuseplotmark{m6b};};
\node[stars] at (axis cs:{279.527},{39.668}) {\tikz\pgfuseplotmark{m6cv};};
\node[stars] at (axis cs:{279.599},{-3.193}) {\tikz\pgfuseplotmark{m6c};};
\node[stars] at (axis cs:{279.628},{-23.505}) {\tikz\pgfuseplotmark{m6a};};
\node[stars] at (axis cs:{279.723},{-21.052}) {\tikz\pgfuseplotmark{m6a};};
\node[stars] at (axis cs:{279.904},{5.264}) {\tikz\pgfuseplotmark{m6c};};
\node[stars] at (axis cs:{279.965},{7.358}) {\tikz\pgfuseplotmark{m6c};};
\node[stars] at (axis cs:{280.002},{-7.791}) {\tikz\pgfuseplotmark{m6a};};
\node[stars] at (axis cs:{280.008},{30.849}) {\tikz\pgfuseplotmark{m6c};};
\node[stars] at (axis cs:{280.051},{38.367}) {\tikz\pgfuseplotmark{m6c};};
\node[stars] at (axis cs:{280.422},{31.618}) {\tikz\pgfuseplotmark{m6c};};
\node[stars] at (axis cs:{280.427},{-14.564}) {\tikz\pgfuseplotmark{m6c};};
\node[stars] at (axis cs:{280.465},{-23.833}) {\tikz\pgfuseplotmark{m6c};};
\node[stars] at (axis cs:{280.534},{34.746}) {\tikz\pgfuseplotmark{m6cb};};
\node[stars] at (axis cs:{280.568},{-9.053}) {\tikz\pgfuseplotmark{m5av};};
\node[stars] at (axis cs:{280.650},{-7.074}) {\tikz\pgfuseplotmark{m6b};};
\node[stars] at (axis cs:{280.730},{-19.284}) {\tikz\pgfuseplotmark{m6cv};};
\node[stars] at (axis cs:{280.819},{39.300}) {\tikz\pgfuseplotmark{m6cb};};
\node[stars] at (axis cs:{280.880},{-8.275}) {\tikz\pgfuseplotmark{m5b};};
\node[stars] at (axis cs:{280.900},{36.556}) {\tikz\pgfuseplotmark{m6b};};
\node[stars] at (axis cs:{280.946},{-38.323}) {\tikz\pgfuseplotmark{m5bv};};
\node[stars] at (axis cs:{280.963},{-6.818}) {\tikz\pgfuseplotmark{m6c};};
\node[stars] at (axis cs:{280.965},{31.926}) {\tikz\pgfuseplotmark{m6a};};
\node[stars] at (axis cs:{281.033},{-36.718}) {\tikz\pgfuseplotmark{m6c};};
\node[stars] at (axis cs:{281.081},{-35.642}) {\tikz\pgfuseplotmark{m5a};};
\node[stars] at (axis cs:{281.085},{39.670}) {\tikz\pgfuseplotmark{m5bb};};
\node[stars] at (axis cs:{281.168},{23.590}) {\tikz\pgfuseplotmark{m6c};};
\node[stars] at (axis cs:{281.193},{37.605}) {\tikz\pgfuseplotmark{m4cvb};};
\node[stars] at (axis cs:{281.207},{-25.011}) {\tikz\pgfuseplotmark{m6a};};
\node[stars] at (axis cs:{281.208},{2.060}) {\tikz\pgfuseplotmark{m5bv};};
\node[stars] at (axis cs:{281.238},{-39.686}) {\tikz\pgfuseplotmark{m5c};};
\node[stars] at (axis cs:{281.328},{-21.002}) {\tikz\pgfuseplotmark{m6c};};
\node[stars] at (axis cs:{281.368},{5.500}) {\tikz\pgfuseplotmark{m6ab};};
\node[stars] at (axis cs:{281.414},{-26.991}) {\tikz\pgfuseplotmark{m3c};};
\node[stars] at (axis cs:{281.416},{20.546}) {\tikz\pgfuseplotmark{m4cv};};
\node[stars] at (axis cs:{281.505},{-19.606}) {\tikz\pgfuseplotmark{m6cv};};
\node[stars] at (axis cs:{281.519},{26.662}) {\tikz\pgfuseplotmark{m5a};};
\node[stars] at (axis cs:{281.586},{-22.392}) {\tikz\pgfuseplotmark{m5c};};
\node[stars] at (axis cs:{281.619},{-0.962}) {\tikz\pgfuseplotmark{m6bvb};};
\node[stars] at (axis cs:{281.672},{18.706}) {\tikz\pgfuseplotmark{m6c};};
\node[stars] at (axis cs:{281.681},{-10.125}) {\tikz\pgfuseplotmark{m6av};};
\node[stars] at (axis cs:{281.755},{18.181}) {\tikz\pgfuseplotmark{m4cvb};};
\node[stars] at (axis cs:{281.794},{-4.748}) {\tikz\pgfuseplotmark{m4c};};
\node[stars] at (axis cs:{281.871},{-5.705}) {\tikz\pgfuseplotmark{m5cv};};
\node[stars] at (axis cs:{281.989},{31.757}) {\tikz\pgfuseplotmark{m6b};};
\node[stars] at (axis cs:{282.011},{4.241}) {\tikz\pgfuseplotmark{m6c};};
\node[stars] at (axis cs:{282.068},{23.515}) {\tikz\pgfuseplotmark{m6c};};
\node[stars] at (axis cs:{282.222},{19.329}) {\tikz\pgfuseplotmark{m6b};};
\node[stars] at (axis cs:{282.405},{0.836}) {\tikz\pgfuseplotmark{m6c};};
\node[stars] at (axis cs:{282.417},{-20.325}) {\tikz\pgfuseplotmark{m5c};};
\node[stars] at (axis cs:{282.421},{-5.913}) {\tikz\pgfuseplotmark{m6bb};};
\node[stars] at (axis cs:{282.441},{32.813}) {\tikz\pgfuseplotmark{m6bvb};};
\node[stars] at (axis cs:{282.470},{32.551}) {\tikz\pgfuseplotmark{m5c};};
\node[stars] at (axis cs:{282.520},{33.362}) {\tikz\pgfuseplotmark{m4avb};};
\node[stars] at (axis cs:{282.630},{-13.572}) {\tikz\pgfuseplotmark{m6c};};
\node[stars] at (axis cs:{282.690},{10.976}) {\tikz\pgfuseplotmark{m6cb};};
\node[stars] at (axis cs:{282.710},{-22.162}) {\tikz\pgfuseplotmark{m6c};};
\node[stars] at (axis cs:{282.744},{-9.774}) {\tikz\pgfuseplotmark{m6a};};
\node[stars] at (axis cs:{282.842},{-3.318}) {\tikz\pgfuseplotmark{m6b};};
\node[stars] at (axis cs:{282.900},{28.784}) {\tikz\pgfuseplotmark{m6c};};
\node[stars] at (axis cs:{282.902},{36.539}) {\tikz\pgfuseplotmark{m6b};};
\node[stars] at (axis cs:{283.008},{13.965}) {\tikz\pgfuseplotmark{m6cv};};
\node[stars] at (axis cs:{283.068},{21.425}) {\tikz\pgfuseplotmark{m5c};};
\node[stars] at (axis cs:{283.118},{-26.651}) {\tikz\pgfuseplotmark{m6c};};
\node[stars] at (axis cs:{283.154},{-29.379}) {\tikz\pgfuseplotmark{m6b};};
\node[stars] at (axis cs:{283.258},{-9.575}) {\tikz\pgfuseplotmark{m6c};};
\node[stars] at (axis cs:{283.431},{36.972}) {\tikz\pgfuseplotmark{m6av};};
\node[stars] at (axis cs:{283.500},{-21.360}) {\tikz\pgfuseplotmark{m6a};};
\node[stars] at (axis cs:{283.542},{-22.745}) {\tikz\pgfuseplotmark{m5a};};
\node[stars] at (axis cs:{283.555},{27.909}) {\tikz\pgfuseplotmark{m6a};};
\node[stars] at (axis cs:{283.626},{36.898}) {\tikz\pgfuseplotmark{m4cvb};};
\node[stars] at (axis cs:{283.680},{-15.603}) {\tikz\pgfuseplotmark{m5b};};
\node[stars] at (axis cs:{283.687},{22.645}) {\tikz\pgfuseplotmark{m5ab};};
\node[stars] at (axis cs:{283.719},{33.969}) {\tikz\pgfuseplotmark{m6bb};};
\node[stars] at (axis cs:{283.780},{-22.671}) {\tikz\pgfuseplotmark{m5b};};
\node[stars] at (axis cs:{283.816},{-26.297}) {\tikz\pgfuseplotmark{m2b};};
\node[stars] at (axis cs:{283.848},{-37.387}) {\tikz\pgfuseplotmark{m6c};};
\node[stars] at (axis cs:{283.864},{6.615}) {\tikz\pgfuseplotmark{m6av};};
\node[stars] at (axis cs:{283.879},{-16.376}) {\tikz\pgfuseplotmark{m6a};};
\node[stars] at (axis cs:{284.003},{-23.174}) {\tikz\pgfuseplotmark{m6b};};
\node[stars] at (axis cs:{284.026},{18.105}) {\tikz\pgfuseplotmark{m6a};};
\node[stars] at (axis cs:{284.055},{4.204}) {\tikz\pgfuseplotmark{m5avb};};
\node[stars] at (axis cs:{284.094},{-1.800}) {\tikz\pgfuseplotmark{m6c};};
\node[stars] at (axis cs:{284.107},{2.471}) {\tikz\pgfuseplotmark{m6cb};};
\node[stars] at (axis cs:{284.113},{-31.689}) {\tikz\pgfuseplotmark{m6c};};
\node[stars] at (axis cs:{284.169},{-37.343}) {\tikz\pgfuseplotmark{m5cv};};
\node[stars] at (axis cs:{284.257},{32.901}) {\tikz\pgfuseplotmark{m5cb};};
\node[stars] at (axis cs:{284.265},{-5.846}) {\tikz\pgfuseplotmark{m5a};};
\node[stars] at (axis cs:{284.319},{2.535}) {\tikz\pgfuseplotmark{m6a};};
\node[stars] at (axis cs:{284.335},{-20.656}) {\tikz\pgfuseplotmark{m5b};};
\node[stars] at (axis cs:{284.393},{-39.823}) {\tikz\pgfuseplotmark{m6c};};
\node[stars] at (axis cs:{284.433},{-21.107}) {\tikz\pgfuseplotmark{m4a};};
\node[stars] at (axis cs:{284.561},{17.361}) {\tikz\pgfuseplotmark{m5cv};};
\node[stars] at (axis cs:{284.589},{-31.036}) {\tikz\pgfuseplotmark{m6b};};
\node[stars] at (axis cs:{284.599},{6.240}) {\tikz\pgfuseplotmark{m6c};};
\node[stars] at (axis cs:{284.603},{-22.529}) {\tikz\pgfuseplotmark{m6c};};
\node[stars] at (axis cs:{284.681},{-37.107}) {\tikz\pgfuseplotmark{m5av};};
\node[stars] at (axis cs:{284.696},{13.907}) {\tikz\pgfuseplotmark{m6bv};};
\node[stars] at (axis cs:{284.736},{32.689}) {\tikz\pgfuseplotmark{m3cv};};
\node[stars] at (axis cs:{284.774},{13.622}) {\tikz\pgfuseplotmark{m5cvb};};
\node[stars] at (axis cs:{284.801},{39.217}) {\tikz\pgfuseplotmark{m6cv};};
\node[stars] at (axis cs:{284.849},{-12.840}) {\tikz\pgfuseplotmark{m6a};};
\node[stars] at (axis cs:{284.862},{-18.566}) {\tikz\pgfuseplotmark{m6c};};
\node[stars] at (axis cs:{284.906},{15.068}) {\tikz\pgfuseplotmark{m4bb};};
\node[stars] at (axis cs:{284.940},{26.230}) {\tikz\pgfuseplotmark{m5cv};};
\node[stars] at (axis cs:{284.992},{22.814}) {\tikz\pgfuseplotmark{m6cv};};
\node[stars] at (axis cs:{285.003},{32.145}) {\tikz\pgfuseplotmark{m5bv};};
\node[stars] at (axis cs:{285.103},{-24.942}) {\tikz\pgfuseplotmark{m6c};};
\node[stars] at (axis cs:{285.230},{33.802}) {\tikz\pgfuseplotmark{m6b};};
\node[stars] at (axis cs:{285.268},{-37.061}) {\tikz\pgfuseplotmark{m6cb};};
\node[stars] at (axis cs:{285.273},{19.309}) {\tikz\pgfuseplotmark{m6c};};
\node[stars] at (axis cs:{285.322},{26.291}) {\tikz\pgfuseplotmark{m6a};};
\node[stars] at (axis cs:{285.390},{-15.282}) {\tikz\pgfuseplotmark{m6c};};
\node[stars] at (axis cs:{285.407},{-22.695}) {\tikz\pgfuseplotmark{m6c};};
\node[stars] at (axis cs:{285.420},{-5.739}) {\tikz\pgfuseplotmark{m4b};};
\node[stars] at (axis cs:{285.452},{33.621}) {\tikz\pgfuseplotmark{m6c};};
\node[stars] at (axis cs:{285.456},{22.264}) {\tikz\pgfuseplotmark{m6c};};
\node[stars] at (axis cs:{285.590},{8.373}) {\tikz\pgfuseplotmark{m6c};};
\node[stars] at (axis cs:{285.615},{-24.847}) {\tikz\pgfuseplotmark{m6a};};
\node[stars] at (axis cs:{285.653},{-29.880}) {\tikz\pgfuseplotmark{m3a};};
\node[stars] at (axis cs:{285.719},{19.661}) {\tikz\pgfuseplotmark{m6b};};
\node[stars] at (axis cs:{285.727},{-3.699}) {\tikz\pgfuseplotmark{m5c};};
\node[stars] at (axis cs:{285.766},{-19.245}) {\tikz\pgfuseplotmark{m6bb};};
\node[stars] at (axis cs:{285.779},{-19.103}) {\tikz\pgfuseplotmark{m6c};};
\node[stars] at (axis cs:{285.824},{-38.253}) {\tikz\pgfuseplotmark{m6av};};
\node[stars] at (axis cs:{285.884},{1.819}) {\tikz\pgfuseplotmark{m6av};};
\node[stars] at (axis cs:{286.104},{-31.047}) {\tikz\pgfuseplotmark{m5c};};
\node[stars] at (axis cs:{286.171},{-21.741}) {\tikz\pgfuseplotmark{m4av};};
\node[stars] at (axis cs:{286.240},{-4.031}) {\tikz\pgfuseplotmark{m5cb};};
\node[stars] at (axis cs:{286.241},{31.744}) {\tikz\pgfuseplotmark{m6a};};
\node[stars] at (axis cs:{286.243},{30.733}) {\tikz\pgfuseplotmark{m6b};};
\node[stars] at (axis cs:{286.353},{13.863}) {\tikz\pgfuseplotmark{m3bv};};
\node[stars] at (axis cs:{286.422},{-15.660}) {\tikz\pgfuseplotmark{m6bv};};
\node[stars] at (axis cs:{286.446},{29.922}) {\tikz\pgfuseplotmark{m6c};};
\node[stars] at (axis cs:{286.562},{-4.882}) {\tikz\pgfuseplotmark{m3c};};
\node[stars] at (axis cs:{286.605},{-37.063}) {\tikz\pgfuseplotmark{m4cb};};
\node[stars] at (axis cs:{286.657},{28.628}) {\tikz\pgfuseplotmark{m6a};};
\node[stars] at (axis cs:{286.660},{24.251}) {\tikz\pgfuseplotmark{m6a};};
\node[stars] at (axis cs:{286.717},{-16.229}) {\tikz\pgfuseplotmark{m6bb};};
\node[stars] at (axis cs:{286.719},{-37.810}) {\tikz\pgfuseplotmark{m6c};};
\node[stars] at (axis cs:{286.735},{-27.670}) {\tikz\pgfuseplotmark{m3c};};
\node[stars] at (axis cs:{286.744},{11.071}) {\tikz\pgfuseplotmark{m5bv};};
\node[stars] at (axis cs:{286.785},{-18.738}) {\tikz\pgfuseplotmark{m6cv};};
\node[stars] at (axis cs:{286.826},{36.100}) {\tikz\pgfuseplotmark{m5cv};};
\node[stars] at (axis cs:{286.857},{32.502}) {\tikz\pgfuseplotmark{m5cb};};
\node[stars] at (axis cs:{286.879},{-28.637}) {\tikz\pgfuseplotmark{m6b};};
\node[stars] at (axis cs:{286.989},{16.853}) {\tikz\pgfuseplotmark{m6b};};
\node[stars] at (axis cs:{287.015},{21.699}) {\tikz\pgfuseplotmark{m6c};};
\node[stars] at (axis cs:{287.061},{-24.657}) {\tikz\pgfuseplotmark{m6c};};
\node[stars] at (axis cs:{287.070},{-19.290}) {\tikz\pgfuseplotmark{m6av};};
\node[stars] at (axis cs:{287.250},{6.073}) {\tikz\pgfuseplotmark{m5c};};
\node[stars] at (axis cs:{287.368},{-37.904}) {\tikz\pgfuseplotmark{m4b};};
\node[stars] at (axis cs:{287.416},{-39.827}) {\tikz\pgfuseplotmark{m6c};};
\node[stars] at (axis cs:{287.441},{-21.024}) {\tikz\pgfuseplotmark{m3bv};};
\node[stars] at (axis cs:{287.451},{-19.804}) {\tikz\pgfuseplotmark{m6b};};
\node[stars] at (axis cs:{287.465},{-0.428}) {\tikz\pgfuseplotmark{m6c};};
\node[stars] at (axis cs:{287.507},{-39.341}) {\tikz\pgfuseplotmark{m4bv};};
\node[stars] at (axis cs:{287.758},{-39.005}) {\tikz\pgfuseplotmark{m6c};};
\node[stars] at (axis cs:{287.828},{-29.502}) {\tikz\pgfuseplotmark{m6c};};
\node[stars] at (axis cs:{287.879},{26.736}) {\tikz\pgfuseplotmark{m6c};};
\node[stars] at (axis cs:{287.942},{31.283}) {\tikz\pgfuseplotmark{m6bv};};
\node[stars] at (axis cs:{288.117},{-21.658}) {\tikz\pgfuseplotmark{m6c};};
\node[stars] at (axis cs:{288.153},{21.554}) {\tikz\pgfuseplotmark{m6b};};
\node[stars] at (axis cs:{288.170},{-7.939}) {\tikz\pgfuseplotmark{m5cv};};
\node[stars] at (axis cs:{288.307},{-25.907}) {\tikz\pgfuseplotmark{m6a};};
\node[stars] at (axis cs:{288.315},{-12.282}) {\tikz\pgfuseplotmark{m5c};};
\node[stars] at (axis cs:{288.428},{2.294}) {\tikz\pgfuseplotmark{m5cv};};
\node[stars] at (axis cs:{288.433},{5.515}) {\tikz\pgfuseplotmark{m6c};};
\node[stars] at (axis cs:{288.440},{39.146}) {\tikz\pgfuseplotmark{m4cv};};
\node[stars] at (axis cs:{288.761},{20.203}) {\tikz\pgfuseplotmark{m6bv};};
\node[stars] at (axis cs:{288.822},{21.232}) {\tikz\pgfuseplotmark{m6a};};
\node[stars] at (axis cs:{288.834},{15.084}) {\tikz\pgfuseplotmark{m6ab};};
\node[stars] at (axis cs:{288.854},{30.526}) {\tikz\pgfuseplotmark{m6bv};};
\node[stars] at (axis cs:{288.885},{-25.257}) {\tikz\pgfuseplotmark{m5b};};
\node[stars] at (axis cs:{288.888},{-24.179}) {\tikz\pgfuseplotmark{m6c};};
\node[stars] at (axis cs:{288.889},{18.516}) {\tikz\pgfuseplotmark{m6c};};
\node[stars] at (axis cs:{288.998},{27.926}) {\tikz\pgfuseplotmark{m6cv};};
\node[stars] at (axis cs:{289.054},{21.390}) {\tikz\pgfuseplotmark{m5av};};
\node[stars] at (axis cs:{289.092},{38.134}) {\tikz\pgfuseplotmark{m4cv};};
\node[stars] at (axis cs:{289.112},{14.544}) {\tikz\pgfuseplotmark{m6ab};};
\node[stars] at (axis cs:{289.129},{4.835}) {\tikz\pgfuseplotmark{m6a};};
\node[stars] at (axis cs:{289.174},{-19.308}) {\tikz\pgfuseplotmark{m6bv};};
\node[stars] at (axis cs:{289.409},{-18.953}) {\tikz\pgfuseplotmark{m5bv};};
\node[stars] at (axis cs:{289.432},{23.025}) {\tikz\pgfuseplotmark{m5cvb};};
\node[stars] at (axis cs:{289.451},{2.031}) {\tikz\pgfuseplotmark{m6c};};
\node[stars] at (axis cs:{289.454},{11.595}) {\tikz\pgfuseplotmark{m5c};};
\node[stars] at (axis cs:{289.635},{1.085}) {\tikz\pgfuseplotmark{m5bvb};};
\node[stars] at (axis cs:{289.712},{0.339}) {\tikz\pgfuseplotmark{m6cb};};
\node[stars] at (axis cs:{289.720},{9.618}) {\tikz\pgfuseplotmark{m6c};};
\node[stars] at (axis cs:{289.750},{-15.537}) {\tikz\pgfuseplotmark{m6b};};
\node[stars] at (axis cs:{289.755},{37.445}) {\tikz\pgfuseplotmark{m6c};};
\node[stars] at (axis cs:{289.912},{37.330}) {\tikz\pgfuseplotmark{m6c};};
\node[stars] at (axis cs:{289.914},{12.375}) {\tikz\pgfuseplotmark{m6avb};};
\node[stars] at (axis cs:{289.917},{-35.421}) {\tikz\pgfuseplotmark{m6a};};
\node[stars] at (axis cs:{289.971},{11.535}) {\tikz\pgfuseplotmark{m6b};};
\end{axis}

\begin{axis}[name=base,axis lines=none]
  \boldmath

  %  tbd:  Sco, Sgr
  
\node[designation-label,anchor=south east]  at (axis cs:{18*15+36/4},{+38+46/ 60  })  {Vega};   
\node[designation-label,anchor=south east]  at (axis cs:{16*15+29/4},{-26-25/ 60  })  {Antares};    
\node[designation-label,anchor=south west]  at (axis cs:{14*15+15/4},{+19+11/ 60  })  {Arcturus};   
\node[designation-label,anchor=south west]  at (axis cs:{17*15+33/4},{-37-06/ 60  })  {Shaula};   
\node[designation-label,anchor=south west]  at (axis cs:{18*15+24/4},{-34-23/ 60  })  {Kaus Australis};   
\node[designation-label,anchor=south west]  at (axis cs:{13*15+25/4},{-11- 9/ 60  })  {Spica};   

\node[interest,pin={[pin distance=-0.4\onedegree,interest-label]180:{Sgr A$^*$}}] at (axis cs:17*15+45.66/4,-29+0.48/60) {\pgfuseplotmark{+}} ;


\node[pin={[pin distance=-0.4\onedegree,Bayer]0:{$\boldsymbol\epsilon$}}] at (axis cs:{284.906},{15.068}) {}; % Aql,  4.02 
\node[pin={[pin distance=-0.4\onedegree,Bayer]00:{$\boldsymbol\lambda$}}] at (axis cs:{286.562},{-4.882}) {}; % Aql,  3.44 
\node[pin={[pin distance=-0.4\onedegree,Bayer]00:{$\boldsymbol\omega^1$}}] at (axis cs:{289.454},{11.595}) {}; % Aql,  5.30 
\node[pin={[pin distance=-0.4\onedegree,Bayer]00:{$\boldsymbol\zeta$}}] at (axis cs:{286.353},{13.863}) {}; % Aql,  2.99 
\node[pin={[pin distance=-0.4\onedegree,Flaamsted]-90:{10}}] at (axis cs:{284.696},{13.907}) {}; % Aql,  5.91 
\node[pin={[pin distance=-0.4\onedegree,Flaamsted]90:{11}}] at (axis cs:{284.774},{13.622}) {}; % Aql,  5.27 
\node[pin={[pin distance=-0.4\onedegree,Flaamsted]00:{12}}] at (axis cs:{285.420},{-5.739}) {}; % Aql,  4.02 
\node[pin={[pin distance=-0.4\onedegree,Flaamsted]-90:{14}}] at (axis cs:{285.727},{-3.699}) {}; % Aql,  5.41 
\node[pin={[pin distance=-0.4\onedegree,Flaamsted]00:{15}}] at (axis cs:{286.240},{-4.031}) {}; % Aql,  5.40 
\node[pin={[pin distance=-0.4\onedegree,Flaamsted]00:{18}}] at (axis cs:{286.744},{11.071}) {}; % Aql,  5.08 
\node[pin={[pin distance=-0.4\onedegree,Flaamsted]00:{19}}] at (axis cs:{287.250},{6.073}) {}; % Aql,  5.23 
\node[pin={[pin distance=-0.4\onedegree,Flaamsted]00:{20}}] at (axis cs:{288.170},{-7.939}) {}; % Aql,  5.36 
\node[pin={[pin distance=-0.4\onedegree,Flaamsted]00:{21}}] at (axis cs:{288.428},{2.294}) {}; % Aql,  5.15 
\node[pin={[pin distance=-0.4\onedegree,Flaamsted]180:{22}}] at (axis cs:{289.129},{4.835}) {}; % Aql,  5.58 
\node[pin={[pin distance=-0.4\onedegree,Flaamsted]00:{23}}] at (axis cs:{289.635},{1.085}) {}; % Aql,  5.10 
\node[pin={[pin distance=-0.4\onedegree,Flaamsted]00:{28}}] at (axis cs:{289.914},{12.375}) {}; % Aql,  5.53 
\node[pin={[pin distance=-0.4\onedegree,Flaamsted]90:{ 4}}] at (axis cs:{281.208},{2.060}) {}; % Aql,  5.02 
\node[pin={[pin distance=-0.4\onedegree,Flaamsted]180:{ 5}}] at (axis cs:{281.619},{-0.962}) {}; % Aql,  5.88

\node[pin={[pin distance=-0.2\onedegree,Bayer]00:{$\boldsymbol\alpha$}}] at (axis cs:{213.915},{19.182}) {}; % Boo,  0.16 
\node[pin={[pin distance=-0.4\onedegree,Bayer]180:{$\boldsymbol\chi$}}] at (axis cs:{228.621},{29.164}) {}; % Boo,  5.28 
\node[pin={[pin distance=-0.4\onedegree,Bayer]00:{$\boldsymbol\delta$}}] at (axis cs:{228.876},{33.315}) {}; % Boo,  3.47 
\node[pin={[pin distance=-0.4\onedegree,Bayer]00:{$\boldsymbol\epsilon$}}] at (axis cs:{221.246},{27.075}) {}; % Boo,  4.66 
\node[pin={[pin distance=-0.4\onedegree,Bayer]00:{$\boldsymbol\epsilon$}}] at (axis cs:{221.247},{27.074}) {}; % Boo,  2.50 
\node[pin={[pin distance=-0.4\onedegree,Bayer]00:{$\boldsymbol\eta$}}] at (axis cs:{208.671},{18.397}) {}; % Boo,  2.68 
\node[pin={[pin distance=-0.4\onedegree,Bayer]00:{$\boldsymbol\gamma$}}] at (axis cs:{218.019},{38.308}) {}; % Boo,  3.04 
\node[pin={[pin distance=-0.4\onedegree,Bayer]00:{$\boldsymbol\mu^1$}}] at (axis cs:{231.123},{37.377}) {}; % Boo,  4.30 
\node[pin={[pin distance=-0.4\onedegree,Bayer]180:{$\boldsymbol\omega$}}] at (axis cs:{225.527},{25.008}) {}; % Boo,  4.79 
\node[pin={[pin distance=-0.4\onedegree,Bayer]00:{$\boldsymbol\omicron$}}] at (axis cs:{221.310},{16.964}) {}; % Boo,  4.60 
\node[pin={[pin distance=-0.4\onedegree,Bayer]180:{$\boldsymbol\pi^{1,2}$}}] at (axis cs:{220.182},{16.418}) {}; % Boo,  4.86 
%\node[pin={[pin distance=-0.4\onedegree,Bayer]00:{$\boldsymbol\pi^2$}}] at (axis cs:{220.183},{16.418}) {}; % Boo,  5.77 
\node[pin={[pin distance=-0.4\onedegree,Bayer]180:{$\boldsymbol\psi$}}] at (axis cs:{226.111},{26.947}) {}; % Boo,  4.51 
\node[pin={[pin distance=-0.4\onedegree,Bayer]-90:{$\boldsymbol\rho$}}] at (axis cs:{217.957},{30.371}) {}; % Boo,  3.57 
\node[pin={[pin distance=-0.4\onedegree,Bayer]90:{$\boldsymbol\sigma$}}] at (axis cs:{218.670},{29.745}) {}; % Boo,  4.47 
\node[pin={[pin distance=-0.4\onedegree,Bayer]00:{$\boldsymbol\tau$}}] at (axis cs:{206.816},{17.457}) {}; % Boo,  4.49 
\node[pin={[pin distance=-0.4\onedegree,Bayer]00:{$\boldsymbol\upsilon$}}] at (axis cs:{207.369},{15.798}) {}; % Boo,  4.04 
\node[pin={[pin distance=-0.4\onedegree,Bayer]180:{$\boldsymbol\xi$}}] at (axis cs:{222.847},{19.100}) {}; % Boo,  4.67 
\node[pin={[pin distance=-0.4\onedegree,Bayer]00:{$\boldsymbol\zeta$}}] at (axis cs:{220.287},{13.728}) {}; % Boo,  3.78 
\node[pin={[pin distance=-0.4\onedegree,Flaamsted]00:{10}}] at (axis cs:{209.662},{21.696}) {}; % Boo,  5.76 
\node[pin={[pin distance=-0.4\onedegree,Flaamsted]00:{12}}] at (axis cs:{212.600},{25.092}) {}; % Boo,  4.82 
\node[pin={[pin distance=-0.4\onedegree,Flaamsted]180:{14}}] at (axis cs:{213.522},{12.959}) {}; % Boo,  5.53 
\node[pin={[pin distance=-0.4\onedegree,Flaamsted]-90:{15}}] at (axis cs:{213.712},{10.101}) {}; % Boo,  5.29 
\node[pin={[pin distance=-0.4\onedegree,Flaamsted]00:{18}}] at (axis cs:{214.818},{13.004}) {}; % Boo,  5.40 
\node[pin={[pin distance=-0.4\onedegree,Flaamsted]90:{ 1}}] at (axis cs:{205.169},{19.956}) {}; % Boo,  5.76 
\node[pin={[pin distance=-0.4\onedegree,Flaamsted]00:{20}}] at (axis cs:{214.938},{16.307}) {}; % Boo,  4.84 
\node[pin={[pin distance=-0.4\onedegree,Flaamsted]00:{22}}] at (axis cs:{216.614},{19.227}) {}; % Boo,  5.40 
\node[pin={[pin distance=-0.4\onedegree,Flaamsted]00:{26}}] at (axis cs:{218.136},{22.260}) {}; % Boo,  5.90 
\node[pin={[pin distance=-0.4\onedegree,Flaamsted]180:{ 2}}] at (axis cs:{205.260},{22.496}) {}; % Boo,  5.63 
\node[pin={[pin distance=-0.4\onedegree,Flaamsted]00:{31}}] at (axis cs:{220.411},{8.162}) {}; % Boo,  4.86 
\node[pin={[pin distance=-0.4\onedegree,Flaamsted]00:{32}}] at (axis cs:{220.431},{11.660}) {}; % Boo,  5.55 
\node[pin={[pin distance=-0.4\onedegree,Flaamsted]90:{34}}] at (axis cs:{220.856},{26.528}) {}; % Boo,  4.81 
\node[pin={[pin distance=-0.4\onedegree,Flaamsted]00:{ 3}}] at (axis cs:{206.681},{25.702}) {}; % Boo,  5.97 
\node[pin={[pin distance=-0.4\onedegree,Flaamsted]00:{40}}] at (axis cs:{224.904},{39.265}) {}; % Boo,  5.64 
\node[pin={[pin distance=-0.4\onedegree,Flaamsted]90:{45}}] at (axis cs:{226.825},{24.869}) {}; % Boo,  4.93 
\node[pin={[pin distance=-0.4\onedegree,Flaamsted]00:{46}}] at (axis cs:{227.099},{26.301}) {}; % Boo,  5.67 
\node[pin={[pin distance=-0.4\onedegree,Flaamsted]00:{50}}] at (axis cs:{230.452},{32.934}) {}; % Boo,  5.38 
\node[pin={[pin distance=-0.4\onedegree,Flaamsted]00:{ 6}}] at (axis cs:{207.428},{21.264}) {}; % Boo,  4.91 
\node[pin={[pin distance=-0.4\onedegree,Flaamsted]90:{ 7}}] at (axis cs:{208.304},{17.933}) {}; % Boo,  5.71 
\node[pin={[pin distance=-0.4\onedegree,Flaamsted]00:{ 9}}] at (axis cs:{209.142},{27.492}) {}; % Boo,  5.01

\node[pin={[pin distance=-0.2\onedegree,Bayer]00:{$\boldsymbol\vartheta$}}] at (axis cs:{211.671},{-36.370}) {}; % Cen,  2.08 
\node[pin={[pin distance=-0.4\onedegree,Bayer]180:{$\boldsymbol\iota$}}] at (axis cs:{200.149},{-36.712}) {}; % Cen,  2.76 
\node[pin={[pin distance=-0.4\onedegree,Bayer]00:{$\boldsymbol\psi$}}] at (axis cs:{215.139},{-37.885}) {}; % Cen,  4.05 
\node[pin={[pin distance=-0.4\onedegree,Flaamsted]00:{ 1}}] at (axis cs:{206.422},{-33.044}) {}; % Cen,  4.23 
\node[pin={[pin distance=-0.4\onedegree,Flaamsted]180:{ 2}}] at (axis cs:{207.361},{-34.451}) {}; % Cen,  4.29 
\node[pin={[pin distance=-0.4\onedegree,Flaamsted]00:{ 3}}] at (axis cs:{207.957},{-32.994}) {}; % Cen,  4.55 
\node[pin={[pin distance=-0.4\onedegree,Flaamsted]00:{ 4}}] at (axis cs:{208.302},{-31.927}) {}; % Cen,  4.74 

\node[pin={[pin distance=-0.4\onedegree,Bayer]00:{$\boldsymbol\alpha$}}] at (axis cs:{197.497},{17.529}) {}; % Com,  4.32 
\node[pin={[pin distance=-0.4\onedegree,Bayer]00:{$\boldsymbol\beta$}}] at (axis cs:{197.968},{27.878}) {}; % Com,  4.25 
\node[pin={[pin distance=-0.4\onedegree,Bayer]00:{$\boldsymbol\gamma$}}] at (axis cs:{186.734},{28.268}) {}; % Com,  4.34 
\node[pin={[pin distance=-0.4\onedegree,Flaamsted]00:{11}}] at (axis cs:{185.179},{17.793}) {}; % Com,  4.73 
\node[pin={[pin distance=-0.4\onedegree,Flaamsted]90:{12}}] at (axis cs:{185.626},{25.846}) {}; % Com,  4.80 
\node[pin={[pin distance=-0.4\onedegree,Flaamsted]-90:{13}}] at (axis cs:{186.077},{26.098}) {}; % Com,  5.15 
%\node[pin={[pin distance=-0.4\onedegree,Flaamsted]90:{14}}] at (axis cs:{186.600},{27.268}) {}; % Com,  4.92 
\node[pin={[pin distance=-0.4\onedegree,Flaamsted]0:{16}}] at (axis cs:{186.747},{26.826}) {}; % Com,  4.96 
\node[pin={[pin distance=-0.4\onedegree,Flaamsted]90:{17}}] at (axis cs:{187.228},{25.913}) {}; % Com,  5.25 
%\node[pin={[pin distance=-0.4\onedegree,Flaamsted]90:{18}}] at (axis cs:{187.363},{24.109}) {}; % Com,  5.48 
\node[pin={[pin distance=-0.4\onedegree,Flaamsted]00:{20}}] at (axis cs:{187.430},{20.896}) {}; % Com,  5.68 
\node[pin={[pin distance=-0.4\onedegree,Flaamsted]0:{21}}] at (axis cs:{187.752},{24.567}) {}; % Com,  5.44 
\node[pin={[pin distance=-0.4\onedegree,Flaamsted]180:{23}}] at (axis cs:{188.713},{22.629}) {}; % Com,  4.79 
\node[pin={[pin distance=-0.4\onedegree,Flaamsted]00:{24}}] at (axis cs:{188.782},{18.377}) {}; % Com,  4.99 
\node[pin={[pin distance=-0.4\onedegree,Flaamsted]00:{25}}] at (axis cs:{189.243},{17.089}) {}; % Com,  5.68 
\node[pin={[pin distance=-0.4\onedegree,Flaamsted]00:{26}}] at (axis cs:{189.780},{21.062}) {}; % Com,  5.48 
\node[pin={[pin distance=-0.4\onedegree,Flaamsted]0:{27}}] at (axis cs:{191.662},{16.578}) {}; % Com,  5.11 
\node[pin={[pin distance=-0.4\onedegree,Flaamsted]00:{29}}] at (axis cs:{192.226},{14.122}) {}; % Com,  5.71 
\node[pin={[pin distance=-0.4\onedegree,Flaamsted]00:{ 2}}] at (axis cs:{181.069},{21.459}) {}; % Com,  5.89 
\node[pin={[pin distance=-0.4\onedegree,Flaamsted]180:{30}}] at (axis cs:{192.323},{27.552}) {}; % Com,  5.76 
\node[pin={[pin distance=-0.4\onedegree,Flaamsted]-90:{31}}] at (axis cs:{192.925},{27.541}) {}; % Com,  4.93 
\node[pin={[pin distance=-0.4\onedegree,Flaamsted]180:{35}}] at (axis cs:{193.324},{21.245}) {}; % Com,  4.92 
\node[pin={[pin distance=-0.4\onedegree,Flaamsted]-90:{36}}] at (axis cs:{194.731},{17.409}) {}; % Com,  4.75 
\node[pin={[pin distance=-0.4\onedegree,Flaamsted]00:{37}}] at (axis cs:{195.069},{30.785}) {}; % Com,  4.89 
\node[pin={[pin distance=-0.4\onedegree,Flaamsted]90:{38}}] at (axis cs:{195.290},{17.123}) {}; % Com,  5.97 
\node[pin={[pin distance=-0.4\onedegree,Flaamsted]00:{40}}] at (axis cs:{196.594},{22.616}) {}; % Com,  5.64 
\node[pin={[pin distance=-0.4\onedegree,Flaamsted]90:{41}}] at (axis cs:{196.795},{27.625}) {}; % Com,  4.78 
\node[pin={[pin distance=-0.4\onedegree,Flaamsted]00:{ 4}}] at (axis cs:{182.963},{25.870}) {}; % Com,  5.65 
\node[pin={[pin distance=-0.4\onedegree,Flaamsted]00:{ 5}}] at (axis cs:{183.039},{20.542}) {}; % Com,  5.60 
\node[pin={[pin distance=-0.4\onedegree,Flaamsted]90:{ 6}}] at (axis cs:{184.001},{14.899}) {}; % Com,  5.10 
\node[pin={[pin distance=-0.4\onedegree,Flaamsted]00:{ 7}}] at (axis cs:{184.086},{23.945}) {}; % Com,  4.93

\node[pin={[pin distance=-0.4\onedegree,Bayer]00:{$\boldsymbol\alpha$}}] at (axis cs:{287.368},{-37.904}) {}; % CrA,  4.10 
\node[pin={[pin distance=-0.4\onedegree,Bayer]00:{$\boldsymbol\beta$}}] at (axis cs:{287.507},{-39.341}) {}; % CrA,  4.11 
\node[pin={[pin distance=-0.4\onedegree,Bayer]90:{$\boldsymbol\epsilon$}}] at (axis cs:{284.681},{-37.107}) {}; % CrA,  4.83 
\node[pin={[pin distance=-0.4\onedegree,Bayer]00:{$\boldsymbol\gamma$}}] at (axis cs:{286.605},{-37.063}) {}; % CrA,  4.23 
\node[pin={[pin distance=-0.4\onedegree,Bayer]00:{$\boldsymbol\kappa^2$}}] at (axis cs:{278.346},{-38.726}) {}; % CrA,  5.61 
\node[pin={[pin distance=-0.4\onedegree,Bayer]90:{$\boldsymbol\lambda$}}] at (axis cs:{280.946},{-38.323}) {}; % CrA,  5.12


\node[pin={[pin distance=-0.2\onedegree,Bayer]00:{$\boldsymbol\alpha$}}] at (axis cs:{233.672},{26.714}) {}; % CrB,  2.22 
\node[pin={[pin distance=-0.4\onedegree,Bayer]00:{$\boldsymbol\beta$}}] at (axis cs:{231.957},{29.106}) {}; % CrB,  3.66 
\node[pin={[pin distance=-0.4\onedegree,Bayer]00:{$\boldsymbol\delta$}}] at (axis cs:{237.399},{26.068}) {}; % CrB,  4.60 
\node[pin={[pin distance=-0.4\onedegree,Bayer]00:{$\boldsymbol\epsilon$}}] at (axis cs:{239.397},{26.878}) {}; % CrB,  4.14 
\node[pin={[pin distance=-0.4\onedegree,Bayer]00:{$\boldsymbol\eta$}}] at (axis cs:{230.801},{30.288}) {}; % CrB,  4.997
\node[pin={[pin distance=-0.4\onedegree,Bayer]00:{$\boldsymbol\eta$}}] at (axis cs:{230.802},{30.288}) {}; % CrB,  4.997
\node[pin={[pin distance=-0.4\onedegree,Bayer]00:{$\boldsymbol\gamma$}}] at (axis cs:{235.686},{26.295}) {}; % CrB,  3.82 
\node[pin={[pin distance=-0.4\onedegree,Bayer]00:{$\boldsymbol\iota$}}] at (axis cs:{240.361},{29.851}) {}; % CrB,  4.98 
\node[pin={[pin distance=-0.4\onedegree,Bayer]00:{$\boldsymbol\kappa$}}] at (axis cs:{237.808},{35.657}) {}; % CrB,  4.81 
\node[pin={[pin distance=-0.4\onedegree,Bayer]00:{$\boldsymbol\lambda$}}] at (axis cs:{238.948},{37.947}) {}; % CrB,  5.44 
\node[pin={[pin distance=-0.4\onedegree,Bayer]00:{$\boldsymbol\mu$}}] at (axis cs:{233.812},{39.010}) {}; % CrB,  5.14 
\node[pin={[pin distance=-0.4\onedegree,Bayer]90:{$\boldsymbol\nu^{1,2}$}}] at (axis cs:{245.589},{33.799}) {}; % CrB,  5.21 
\node[pin={[pin distance=-0.4\onedegree,Bayer]00:{$\boldsymbol\omicron$}}] at (axis cs:{230.036},{29.616}) {}; % CrB,  5.51 
\node[pin={[pin distance=-0.4\onedegree,Bayer]00:{$\boldsymbol\pi$}}] at (axis cs:{235.997},{32.516}) {}; % CrB,  5.57 
\node[pin={[pin distance=-0.4\onedegree,Bayer]00:{$\boldsymbol\rho$}}] at (axis cs:{240.261},{33.303}) {}; % CrB,  5.40 
\node[pin={[pin distance=-0.4\onedegree,Bayer]180:{$\boldsymbol\sigma$}}] at (axis cs:{243.670},{33.858}) {}; % CrB,  5.54 
\node[pin={[pin distance=-0.4\onedegree,Bayer]180:{$\boldsymbol\tau$}}] at (axis cs:{242.243},{36.491}) {}; % CrB,  4.73 
\node[pin={[pin distance=-0.4\onedegree,Bayer]00:{$\boldsymbol\upsilon$}}] at (axis cs:{244.187},{29.150}) {}; % CrB,  5.79 
\node[pin={[pin distance=-0.4\onedegree,Bayer]-90:{$\boldsymbol\vartheta$}}] at (axis cs:{233.232},{31.359}) {}; % CrB,  4.16 
\node[pin={[pin distance=-0.4\onedegree,Bayer]180:{$\boldsymbol\xi$}}] at (axis cs:{245.524},{30.892}) {}; % CrB,  4.85 
\node[pin={[pin distance=-0.4\onedegree,Bayer]00:{$\boldsymbol\zeta^{1,2}$}}] at (axis cs:{234.843},{36.637}) {}; % CrB,



\node[pin={[pin distance=-0.4\onedegree,Bayer]00:{$\boldsymbol\alpha$}}] at (axis cs:{182.103},{-24.729}) {}; % Crv,  4.04 
\node[pin={[pin distance=-0.4\onedegree,Bayer]00:{$\boldsymbol\beta$}}] at (axis cs:{188.597},{-23.397}) {}; % Crv,  2.66 
\node[pin={[pin distance=-0.4\onedegree,Bayer]90:{$\boldsymbol\delta$}}] at (axis cs:{187.466},{-16.515}) {}; % Crv,  2.97 
\node[pin={[pin distance=-0.4\onedegree,Bayer]00:{$\boldsymbol\epsilon$}}] at (axis cs:{182.531},{-22.620}) {}; % Crv,  3.01 
\node[pin={[pin distance=-0.4\onedegree,Bayer]00:{$\boldsymbol\eta$}}] at (axis cs:{188.018},{-16.196}) {}; % Crv,  4.31 
\node[pin={[pin distance=-0.4\onedegree,Bayer]00:{$\boldsymbol\gamma$}}] at (axis cs:{183.952},{-17.542}) {}; % Crv,  2.59 
\node[pin={[pin distance=-0.4\onedegree,Bayer]00:{$\boldsymbol\zeta$}}] at (axis cs:{185.140},{-22.216}) {}; % Crv,  5.21 
\node[pin={[pin distance=-0.4\onedegree,Flaamsted]00:{ 3}}] at (axis cs:{182.766},{-23.602}) {}; % Crv,  5.46 
\node[pin={[pin distance=-0.4\onedegree,Flaamsted]00:{ 6}}] at (axis cs:{185.840},{-24.841}) {}; % Crv,  5.66

\node[pin={[pin distance=-0.4\onedegree,Bayer]00:{$\boldsymbol\alpha^{1,2}$}}] at (axis cs:{194.002},{38.315}) {}; % CVn,  5.48 
\node[pin={[pin distance=-0.4\onedegree,Flaamsted]00:{10}}] at (axis cs:{191.248},{39.279}) {}; % CVn,  5.96 
\node[pin={[pin distance=-0.4\onedegree,Flaamsted]00:{14}}] at (axis cs:{196.435},{35.799}) {}; % CVn,  5.20 
\node[pin={[pin distance=-0.4\onedegree,Flaamsted]00:{17}}] at (axis cs:{197.513},{38.499}) {}; % CVn,  5.93 
\node[pin={[pin distance=-0.4\onedegree,Flaamsted]00:{25}}] at (axis cs:{204.365},{36.295}) {}; % CVn,  4.91 
\node[pin={[pin distance=-0.4\onedegree,Flaamsted]00:{ 6}}] at (axis cs:{186.462},{39.019}) {}; % CVn,  5.02 

\node[pin={[pin distance=-0.4\onedegree,Bayer]180:{$\boldsymbol\alpha^{1,2}$}}] at (axis cs:{258.662},{14.390}) {}; % Her,  3.37 
\node[pin={[pin distance=-0.4\onedegree,Bayer]00:{$\boldsymbol\beta$}}] at (axis cs:{247.555},{21.490}) {}; % Her,  2.78 
\node[pin={[pin distance=-0.4\onedegree,Bayer]00:{$\boldsymbol\delta$}}] at (axis cs:{258.758},{24.839}) {}; % Her,  3.13 
\node[pin={[pin distance=-0.4\onedegree,Bayer]00:{$\boldsymbol\epsilon$}}] at (axis cs:{255.072},{30.926}) {}; % Her,  3.91 
\node[pin={[pin distance=-0.4\onedegree,Bayer]00:{$\boldsymbol\eta$}}] at (axis cs:{250.724},{38.922}) {}; % Her,  3.48 
\node[pin={[pin distance=-0.4\onedegree,Bayer]00:{$\boldsymbol\gamma$}}] at (axis cs:{245.480},{19.153}) {}; % Her,  3.74 
\node[pin={[pin distance=-0.4\onedegree,Bayer]90:{$\boldsymbol\kappa$}}] at (axis cs:{242.019},{17.047}) {}; % Her,  5.00 
\node[pin={[pin distance=-0.4\onedegree,Bayer]00:{$\boldsymbol\lambda$}}] at (axis cs:{262.685},{26.110}) {}; % Her,  4.39 
\node[pin={[pin distance=-0.4\onedegree,Bayer]00:{$\boldsymbol\mu$}}] at (axis cs:{266.615},{27.721}) {}; % Her,  3.42 
\node[pin={[pin distance=-0.4\onedegree,Bayer]180:{$\boldsymbol\nu$}}] at (axis cs:{269.626},{30.189}) {}; % Her,  4.41 
\node[pin={[pin distance=-0.4\onedegree,Bayer]00:{$\boldsymbol\omega$}}] at (axis cs:{246.354},{14.033}) {}; % Her,  4.58 
\node[pin={[pin distance=-0.4\onedegree,Bayer]00:{$\boldsymbol\omicron$}}] at (axis cs:{271.886},{28.762}) {}; % Her,  3.84 
\node[pin={[pin distance=-0.4\onedegree,Bayer]180:{$\boldsymbol\pi$}}] at (axis cs:{258.762},{36.809}) {}; % Her,  3.14 
\node[pin={[pin distance=-0.4\onedegree,Bayer]00:{$\boldsymbol\rho$}}] at (axis cs:{260.920},{37.147}) {}; % Her,  5.40 
\node[pin={[pin distance=-0.4\onedegree,Bayer]00:{$\boldsymbol\rho$}}] at (axis cs:{260.921},{37.146}) {}; % Her,  4.51 
\node[pin={[pin distance=-0.4\onedegree,Bayer]180:{$\boldsymbol\vartheta$}}] at (axis cs:{269.063},{37.251}) {}; % Her,  3.84 
\node[pin={[pin distance=-0.4\onedegree,Bayer]180:{$\boldsymbol\xi$}}] at (axis cs:{269.441},{29.248}) {}; % Her,  3.70 
\node[pin={[pin distance=-0.4\onedegree,Bayer]00:{$\boldsymbol\zeta$}}] at (axis cs:{250.322},{31.603}) {}; % Her,  2.85 
\node[pin={[pin distance=-0.4\onedegree,Flaamsted]00:{100}}] at (axis cs:{271.956},{26.097}) {}; % Her,  5.83 
\node[pin={[pin distance=-0.4\onedegree,Flaamsted]00:{100}}] at (axis cs:{271.957},{26.101}) {}; % Her,  5.86 
\node[pin={[pin distance=-0.4\onedegree,Flaamsted]90:{101}}] at (axis cs:{272.220},{20.045}) {}; % Her,  5.11 
\node[pin={[pin distance=-0.4\onedegree,Flaamsted]00:{102}}] at (axis cs:{272.190},{20.814}) {}; % Her,  4.37 
\node[pin={[pin distance=-0.4\onedegree,Flaamsted]180:{104}}] at (axis cs:{272.976},{31.405}) {}; % Her,  4.98 
\node[pin={[pin distance=-0.4\onedegree,Flaamsted]00:{105}}] at (axis cs:{274.795},{24.446}) {}; % Her,  5.28 
\node[pin={[pin distance=-0.4\onedegree,Flaamsted]180:{106}}] at (axis cs:{275.075},{21.961}) {}; % Her,  4.93 
\node[pin={[pin distance=-0.4\onedegree,Flaamsted]00:{107}}] at (axis cs:{275.254},{28.870}) {}; % Her,  5.13 
\node[pin={[pin distance=-0.4\onedegree,Flaamsted]180:{108}}] at (axis cs:{275.237},{29.859}) {}; % Her,  5.61 
\node[pin={[pin distance=-0.4\onedegree,Flaamsted]-90:{109}}] at (axis cs:{275.925},{21.770}) {}; % Her,  3.85 
\node[pin={[pin distance=-0.4\onedegree,Flaamsted]00:{10}}] at (axis cs:{242.908},{23.495}) {}; % Her,  5.82 
\node[pin={[pin distance=-0.4\onedegree,Flaamsted]00:{110}}] at (axis cs:{281.416},{20.546}) {}; % Her,  4.21 
\node[pin={[pin distance=-0.4\onedegree,Flaamsted]00:{111}}] at (axis cs:{281.755},{18.181}) {}; % Her,  4.35 
\node[pin={[pin distance=-0.4\onedegree,Flaamsted]180:{112}}] at (axis cs:{283.068},{21.425}) {}; % Her,  5.41 
\node[pin={[pin distance=-0.4\onedegree,Flaamsted]180:{113}}] at (axis cs:{283.687},{22.645}) {}; % Her,  4.59 
\node[pin={[pin distance=-0.4\onedegree,Flaamsted]180:{16}}] at (axis cs:{243.869},{18.808}) {}; % Her,  5.71 
\node[pin={[pin distance=-0.4\onedegree,Flaamsted]00:{21}}] at (axis cs:{246.045},{6.948}) {}; % Her,  5.83 
\node[pin={[pin distance=-0.4\onedegree,Flaamsted]180:{25}}] at (axis cs:{246.351},{37.394}) {}; % Her,  5.54 
\node[pin={[pin distance=-0.4\onedegree,Flaamsted]00:{28}}] at (axis cs:{248.149},{5.521}) {}; % Her,  5.63 
\node[pin={[pin distance=-0.4\onedegree,Flaamsted]00:{29}}] at (axis cs:{248.151},{11.488}) {}; % Her,  4.84 
\node[pin={[pin distance=-0.4\onedegree,Flaamsted]00:{37}}] at (axis cs:{250.161},{4.220}) {}; % Her,  5.78 
\node[pin={[pin distance=-0.4\onedegree,Flaamsted]00:{39}}] at (axis cs:{250.403},{26.917}) {}; % Her,  5.92 
\node[pin={[pin distance=-0.4\onedegree,Flaamsted]00:{43}}] at (axis cs:{251.458},{8.582}) {}; % Her,  5.14 
\node[pin={[pin distance=-0.4\onedegree,Flaamsted]180:{45}}] at (axis cs:{251.943},{5.247}) {}; % Her,  5.22 
\node[pin={[pin distance=-0.4\onedegree,Flaamsted]180:{47}}] at (axis cs:{252.581},{7.247}) {}; % Her,  5.48 
\node[pin={[pin distance=-0.4\onedegree,Flaamsted]00:{50}}] at (axis cs:{252.662},{29.806}) {}; % Her,  5.72 
\node[pin={[pin distance=-0.4\onedegree,Flaamsted]00:{51}}] at (axis cs:{252.939},{24.656}) {}; % Her,  5.03 
\node[pin={[pin distance=-0.4\onedegree,Flaamsted]00:{53}}] at (axis cs:{253.242},{31.702}) {}; % Her,  5.34 
\node[pin={[pin distance=-0.4\onedegree,Flaamsted]00:{54}}] at (axis cs:{253.842},{18.433}) {}; % Her,  5.35 
\node[pin={[pin distance=-0.4\onedegree,Flaamsted]00:{59}}] at (axis cs:{255.402},{33.568}) {}; % Her,  5.28 
\node[pin={[pin distance=-0.4\onedegree,Flaamsted]180:{ 5}}] at (axis cs:{240.310},{17.818}) {}; % Her,  5.11 
\node[pin={[pin distance=-0.4\onedegree,Flaamsted]-90:{60}}] at (axis cs:{256.345},{12.741}) {}; % Her,  4.88 
\node[pin={[pin distance=-0.4\onedegree,Flaamsted]180:{68}}] at (axis cs:{259.332},{33.100}) {}; % Her,  4.81 
\node[pin={[pin distance=-0.4\onedegree,Flaamsted]-90:{69}}] at (axis cs:{259.418},{37.291}) {}; % Her,  4.63 
\node[pin={[pin distance=-0.4\onedegree,Flaamsted]00:{70}}] at (axis cs:{260.226},{24.499}) {}; % Her,  5.13 
\node[pin={[pin distance=-0.4\onedegree,Flaamsted]00:{72}}] at (axis cs:{260.165},{32.468}) {}; % Her,  5.39 
\node[pin={[pin distance=-0.4\onedegree,Flaamsted]00:{73}}] at (axis cs:{261.027},{22.960}) {}; % Her,  5.71 
\node[pin={[pin distance=-0.4\onedegree,Flaamsted]00:{78}}] at (axis cs:{262.957},{28.407}) {}; % Her,  5.66 
\node[pin={[pin distance=-0.4\onedegree,Flaamsted]180:{79}}] at (axis cs:{264.380},{24.310}) {}; % Her,  5.76 
\node[pin={[pin distance=-0.4\onedegree,Flaamsted]-90:{83}}] at (axis cs:{265.618},{24.564}) {}; % Her,  5.56 
\node[pin={[pin distance=-0.4\onedegree,Flaamsted]90:{84}}] at (axis cs:{265.840},{24.328}) {}; % Her,  5.73 
\node[pin={[pin distance=-0.4\onedegree,Flaamsted]00:{87}}] at (axis cs:{267.205},{25.623}) {}; % Her,  5.10 
\node[pin={[pin distance=-0.4\onedegree,Flaamsted]00:{89}}] at (axis cs:{268.855},{26.050}) {}; % Her,  5.47 
\node[pin={[pin distance=-0.4\onedegree,Flaamsted]00:{93}}] at (axis cs:{270.014},{16.751}) {}; % Her,  4.67 
\node[pin={[pin distance=-0.4\onedegree,Flaamsted]00:{95}}] at (axis cs:{270.375},{21.595}) {}; % Her,  5.10 
\node[pin={[pin distance=-0.4\onedegree,Flaamsted]00:{96}}] at (axis cs:{270.596},{20.834}) {}; % Her,  5.26 
\node[pin={[pin distance=-0.4\onedegree,Flaamsted]00:{98}}] at (axis cs:{271.508},{22.219}) {}; % Her,  4.99 
\node[pin={[pin distance=-0.4\onedegree,Flaamsted]180:{99}}] at (axis cs:{271.756},{30.562}) {}; % Her,  5.06 
\node[pin={[pin distance=-0.4\onedegree,Flaamsted]00:{ 9}}] at (axis cs:{243.314},{5.021}) {}; % Her,  5.46 

\node[pin={[pin distance=-0.4\onedegree,Bayer]00:{$\boldsymbol\gamma$}}] at (axis cs:{199.730},{-23.171}) {}; % Hya,  3.00 
\node[pin={[pin distance=-0.4\onedegree,Bayer]00:{$\boldsymbol\pi$}}] at (axis cs:{211.593},{-26.682}) {}; % Hya,  3.26 
\node[pin={[pin distance=-0.4\onedegree,Bayer]00:{$\boldsymbol\psi$}}] at (axis cs:{197.264},{-23.118}) {}; % Hya,  4.94 
\node[pin={[pin distance=-0.4\onedegree,Flaamsted]180:{47}}] at (axis cs:{209.630},{-24.972}) {}; % Hya,  5.21 
\node[pin={[pin distance=-0.4\onedegree,Flaamsted]00:{48}}] at (axis cs:{210.001},{-25.010}) {}; % Hya,  5.77 
\node[pin={[pin distance=-0.4\onedegree,Flaamsted]00:{50}}] at (axis cs:{213.192},{-27.261}) {}; % Hya,  5.08 
\node[pin={[pin distance=-0.4\onedegree,Flaamsted]00:{51}}] at (axis cs:{215.774},{-27.754}) {}; % Hya,  4.79 
\node[pin={[pin distance=-0.4\onedegree,Flaamsted]00:{52}}] at (axis cs:{217.043},{-29.492}) {}; % Hya,  4.98 
\node[pin={[pin distance=-0.4\onedegree,Flaamsted]180:{54}}] at (axis cs:{221.500},{-25.443}) {}; % Hya,  5.08 
\node[pin={[pin distance=-0.4\onedegree,Flaamsted]00:{55}}] at (axis cs:{221.844},{-25.624}) {}; % Hya,  5.62 
\node[pin={[pin distance=-0.4\onedegree,Flaamsted]00:{56}}] at (axis cs:{221.937},{-26.087}) {}; % Hya,  5.24 
\node[pin={[pin distance=-0.4\onedegree,Flaamsted]00:{57}}] at (axis cs:{221.990},{-26.646}) {}; % Hya,  5.77 
\node[pin={[pin distance=-0.4\onedegree,Flaamsted]00:{58}}] at (axis cs:{222.572},{-27.960}) {}; % Hya,  4.42 
\node[pin={[pin distance=-0.4\onedegree,Flaamsted]-90:{59}}] at (axis cs:{224.664},{-27.657}) {}; % Hya,  5.67 
\node[pin={[pin distance=-0.4\onedegree,Flaamsted]180:{60}}] at (axis cs:{225.527},{-28.060}) {}; % Hya,  5.83 

\node[pin={[pin distance=-0.4\onedegree,Bayer]00:{$\boldsymbol\alpha^{1,2}$}}] at (axis cs:{222.672},{-15.997}) {}; % Lib,  5.15 
\node[pin={[pin distance=-0.4\onedegree,Bayer]180:{$\boldsymbol\beta$}}] at (axis cs:{229.252},{-9.383}) {}; % Lib,  2.61 
\node[pin={[pin distance=-0.4\onedegree,Bayer]00:{$\boldsymbol\delta$}}] at (axis cs:{225.243},{-8.519}) {}; % Lib,  4.95 
\node[pin={[pin distance=-0.4\onedegree,Bayer]00:{$\boldsymbol\epsilon$}}] at (axis cs:{231.050},{-10.322}) {}; % Lib,  4.92 
\node[pin={[pin distance=-0.4\onedegree,Bayer]00:{$\boldsymbol\eta$}}] at (axis cs:{236.018},{-15.673}) {}; % Lib,  5.42 
\node[pin={[pin distance=-0.4\onedegree,Bayer]00:{$\boldsymbol\gamma$}}] at (axis cs:{233.882},{-14.789}) {}; % Lib,  3.92 
\node[pin={[pin distance=-0.4\onedegree,Bayer]00:{$\boldsymbol\iota^1$}}] at (axis cs:{228.055},{-19.792}) {}; % Lib,  4.56 
\node[pin={[pin distance=-0.4\onedegree,Bayer]90:{$\boldsymbol\kappa$}}] at (axis cs:{235.487},{-19.679}) {}; % Lib,  4.75 
\node[pin={[pin distance=-0.4\onedegree,Bayer]-90:{$\boldsymbol\lambda$}}] at (axis cs:{238.334},{-20.167}) {}; % Lib,  5.03 
\node[pin={[pin distance=-0.4\onedegree,Bayer]00:{$\boldsymbol\mu$}}] at (axis cs:{222.329},{-14.149}) {}; % Lib,  5.60 
\node[pin={[pin distance=-0.4\onedegree,Bayer]-90:{$\boldsymbol\nu$}}] at (axis cs:{226.657},{-16.257}) {}; % Lib,  5.19 
\node[pin={[pin distance=-0.4\onedegree,Bayer]00:{$\boldsymbol\sigma$}}] at (axis cs:{226.018},{-25.282}) {}; % Lib,  3.28 
\node[pin={[pin distance=-0.4\onedegree,Bayer]-90:{$\boldsymbol\tau$}}] at (axis cs:{234.664},{-29.778}) {}; % Lib,  3.66 
\node[pin={[pin distance=-0.4\onedegree,Bayer]-90:{$\boldsymbol\upsilon$}}] at (axis cs:{234.256},{-28.135}) {}; % Lib,  3.60 
\node[pin={[pin distance=-0.4\onedegree,Bayer]180:{$\boldsymbol\vartheta$}}] at (axis cs:{238.456},{-16.729}) {}; % Lib,  4.13 
\node[pin={[pin distance=-0.4\onedegree,Bayer]180:{$\boldsymbol\xi^1$}}] at (axis cs:{223.595},{-11.898}) {}; % Lib,  5.79 
\node[pin={[pin distance=-0.4\onedegree,Bayer]180:{$\boldsymbol\xi^2$}}] at (axis cs:{224.192},{-11.410}) {}; % Lib,  5.46 
\node[pin={[pin distance=-0.4\onedegree,Bayer]180:{$\boldsymbol\zeta^1$}}] at (axis cs:{232.064},{-16.716}) {}; % Lib,  5.64 
\node[pin={[pin distance=-0.4\onedegree,Bayer]-90:{$\boldsymbol\zeta^3$}}] at (axis cs:{232.668},{-16.609}) {}; % Lib,  5.81 
\node[pin={[pin distance=-0.4\onedegree,Bayer]00:{$\boldsymbol\zeta^4$}}] at (axis cs:{233.230},{-16.853}) {}; % Lib,  5.50 
\node[pin={[pin distance=-0.4\onedegree,Flaamsted]00:{11}}] at (axis cs:{222.754},{-2.299}) {}; % Lib,  4.93 
\node[pin={[pin distance=-0.4\onedegree,Flaamsted]00:{12}}] at (axis cs:{223.584},{-24.642}) {}; % Lib,  5.26 
\node[pin={[pin distance=-0.4\onedegree,Flaamsted]00:{16}}] at (axis cs:{224.296},{-4.346}) {}; % Lib,  4.47 
\node[pin={[pin distance=-0.4\onedegree,Flaamsted]00:{18}}] at (axis cs:{224.723},{-11.144}) {}; % Lib,  5.86 
\node[pin={[pin distance=-0.4\onedegree,Flaamsted]90:{36}}] at (axis cs:{233.656},{-28.047}) {}; % Lib,  5.13 
\node[pin={[pin distance=-0.4\onedegree,Flaamsted]90:{37}}] at (axis cs:{233.545},{-10.064}) {}; % Lib,  4.61 
\node[pin={[pin distance=-0.4\onedegree,Flaamsted]-90:{41}}] at (axis cs:{234.727},{-19.302}) {}; % Lib,  5.36 
\node[pin={[pin distance=-0.4\onedegree,Flaamsted]00:{42}}] at (axis cs:{235.070},{-23.818}) {}; % Lib,  4.95 
\node[pin={[pin distance=-0.4\onedegree,Flaamsted]00:{47}}] at (axis cs:{238.752},{-19.383}) {}; % Lib,  5.96 
\node[pin={[pin distance=-0.4\onedegree,Flaamsted]-90:{48}}] at (axis cs:{239.547},{-14.279}) {}; % Lib,  4.95 
\node[pin={[pin distance=-0.4\onedegree,Flaamsted]180:{49}}] at (axis cs:{240.082},{-16.533}) {}; % Lib,  5.46 
\node[pin={[pin distance=-0.4\onedegree,Flaamsted]90:{ 4}}] at (axis cs:{220.806},{-24.998}) {}; % Lib,  5.70 
\node[pin={[pin distance=-0.4\onedegree,Flaamsted]180:{50}}] at (axis cs:{240.198},{-8.411}) {}; % Lib,  5.53 

\node[pin={[pin distance=-0.4\onedegree,Bayer]180:{$\boldsymbol\chi$}}] at (axis cs:{237.740},{-33.627}) {}; % Lup,  3.96 
\node[pin={[pin distance=-0.4\onedegree,Bayer]180:{$\boldsymbol\eta$}}] at (axis cs:{240.031},{-38.396}) {}; % Lup,  3.43 
\node[pin={[pin distance=-0.4\onedegree,Bayer]180:{$\boldsymbol\psi^1$}}] at (axis cs:{234.942},{-34.412}) {}; % Lup,  4.66 
\node[pin={[pin distance=-0.4\onedegree,Bayer]90:{$\boldsymbol\psi^2$}}] at (axis cs:{235.671},{-34.710}) {}; % Lup,  4.73 
\node[pin={[pin distance=-0.4\onedegree,Bayer]00:{$\boldsymbol\upsilon$}}] at (axis cs:{231.188},{-39.710}) {}; % Lup,  5.38 
\node[pin={[pin distance=-0.4\onedegree,Bayer]180:{$\boldsymbol\varphi^1$}}] at (axis cs:{230.452},{-36.261}) {}; % Lup,  3.56 
\node[pin={[pin distance=-0.4\onedegree,Bayer]90:{$\boldsymbol\varphi^2$}}] at (axis cs:{230.789},{-36.858}) {}; % Lup,  4.53 
\node[pin={[pin distance=-0.4\onedegree,Bayer]180:{$\boldsymbol\vartheta$}}] at (axis cs:{241.648},{-36.802}) {}; % Lup,  4.22 
\node[pin={[pin distance=-0.4\onedegree,Bayer]-90:{$\boldsymbol\xi^{1,2}$}}] at (axis cs:{239.223},{-33.966}) {}; % Lup,  5.09 
\node[pin={[pin distance=-0.4\onedegree,Flaamsted]00:{ 1}}] at (axis cs:{228.655},{-31.519}) {}; % Lup,  4.92 
\node[pin={[pin distance=-0.4\onedegree,Flaamsted]90:{ 2}}] at (axis cs:{229.458},{-30.148}) {}; % Lup,  4.34 

\node[pin={[pin distance=-0.2\onedegree,Bayer]00:{$\boldsymbol\alpha$}}] at (axis cs:{279.235},{38.784}) {}; % Lyr,  0.03 
\node[pin={[pin distance=-0.4\onedegree,Bayer]180:{$\boldsymbol\beta$}}] at (axis cs:{282.520},{33.362}) {}; % Lyr,  3.52 
\node[pin={[pin distance=-0.4\onedegree,Bayer]00:{$\boldsymbol\delta^{1,2}$}}] at (axis cs:{283.431},{36.972}) {}; % Lyr,  5.58 
%\node[pin={[pin distance=-0.4\onedegree,Bayer]00:{$\boldsymbol\delta^2$}}] at (axis cs:{283.626},{36.898}) {}; % Lyr,  4.28 
\node[pin={[pin distance=-0.4\onedegree,Bayer]00:{$\boldsymbol\epsilon^{1,2}$}}] at (axis cs:{281.085},{39.670}) {}; % Lyr,  5.01 
\node[pin={[pin distance=-0.4\onedegree,Bayer]00:{$\boldsymbol\eta$}}] at (axis cs:{288.440},{39.146}) {}; % Lyr,  4.41 
\node[pin={[pin distance=-0.4\onedegree,Bayer]-90:{$\boldsymbol\gamma$}}] at (axis cs:{284.736},{32.689}) {}; % Lyr,  3.25 
\node[pin={[pin distance=-0.4\onedegree,Bayer]00:{$\boldsymbol\iota$}}] at (axis cs:{286.826},{36.100}) {}; % Lyr,  5.26 
\node[pin={[pin distance=-0.4\onedegree,Bayer]00:{$\boldsymbol\kappa$}}] at (axis cs:{274.965},{36.064}) {}; % Lyr,  4.32 
\node[pin={[pin distance=-0.4\onedegree,Bayer]90:{$\boldsymbol\lambda$}}] at (axis cs:{285.003},{32.145}) {}; % Lyr,  4.94 
\node[pin={[pin distance=-0.4\onedegree,Bayer]00:{$\boldsymbol\mu$}}] at (axis cs:{276.057},{39.507}) {}; % Lyr,  5.12 
\node[pin={[pin distance=-0.4\onedegree,Bayer]180:{$\boldsymbol\nu^1$}}] at (axis cs:{282.441},{32.813}) {}; % Lyr,  5.93 
\node[pin={[pin distance=-0.4\onedegree,Bayer]90:{$\boldsymbol\nu^2$}}] at (axis cs:{282.470},{32.551}) {}; % Lyr,  5.23 
\node[pin={[pin distance=-0.4\onedegree,Bayer]90:{$\boldsymbol\vartheta$}}] at (axis cs:{289.092},{38.134}) {}; % Lyr,  4.34 
\node[pin={[pin distance=-0.4\onedegree,Bayer]00:{$\boldsymbol\zeta^{1,2}$}}] at (axis cs:{281.193},{37.605}) {}; % Lyr,  4.34 
\node[pin={[pin distance=-0.4\onedegree,Flaamsted]00:{17}}] at (axis cs:{286.857},{32.502}) {}; % Lyr,  5.23 
\node[pin={[pin distance=-0.4\onedegree,Flaamsted]00:{19}}] at (axis cs:{287.942},{31.283}) {}; % Lyr,  5.94 

\node[pin={[pin distance=-0.2\onedegree,Bayer]00:{$\boldsymbol\alpha$}}] at (axis cs:{263.734},{12.560}) {}; % Oph,  2.09 
\node[pin={[pin distance=-0.2\onedegree,Bayer]00:{$\boldsymbol\eta$}}] at (axis cs:{257.595},{-15.725}) {}; % Oph,  2.43 
\node[pin={[pin distance=-0.4\onedegree,Bayer]00:{$\boldsymbol\beta$}}] at (axis cs:{265.868},{4.567}) {}; % Oph,  2.77 
\node[pin={[pin distance=-0.4\onedegree,Bayer]-90:{$\boldsymbol\chi$}}] at (axis cs:{246.756},{-18.456}) {}; % Oph,  4.26 
\node[pin={[pin distance=-0.2\onedegree,Bayer]90:{$\boldsymbol\delta$}}] at (axis cs:{243.586},{-3.694}) {}; % Oph,  2.73 
\node[pin={[pin distance=-0.4\onedegree,Bayer]00:{$\boldsymbol\epsilon$}}] at (axis cs:{244.580},{-4.692}) {}; % Oph,  3.24 
\node[pin={[pin distance=-0.4\onedegree,Bayer]00:{$\boldsymbol\gamma$}}] at (axis cs:{266.973},{2.707}) {}; % Oph,  3.75 
\node[pin={[pin distance=-0.4\onedegree,Bayer]00:{$\boldsymbol\iota$}}] at (axis cs:{253.502},{10.165}) {}; % Oph,  4.37 
\node[pin={[pin distance=-0.4\onedegree,Bayer]00:{$\boldsymbol\kappa$}}] at (axis cs:{254.417},{9.375}) {}; % Oph,  3.19 
\node[pin={[pin distance=-0.4\onedegree,Bayer]00:{$\boldsymbol\lambda$}}] at (axis cs:{247.728},{1.984}) {}; % Oph,  3.90 
\node[pin={[pin distance=-0.4\onedegree,Bayer]00:{$\boldsymbol\mu$}}] at (axis cs:{264.461},{-8.119}) {}; % Oph,  4.62 
\node[pin={[pin distance=-0.4\onedegree,Bayer]180:{$\boldsymbol\nu$}}] at (axis cs:{269.757},{-9.774}) {}; % Oph,  3.32 
\node[pin={[pin distance=-0.4\onedegree,Bayer]00:{$\boldsymbol\omega$}}] at (axis cs:{248.034},{-21.466}) {}; % Oph,  4.44 
\node[pin={[pin distance=-0.4\onedegree,Bayer]180:{$\boldsymbol\omicron$}}] at (axis cs:{259.503},{-24.287}) {}; % Oph,  5.11 
\node[pin={[pin distance=-0.4\onedegree,Bayer]00:{$\boldsymbol\psi$}}] at (axis cs:{246.026},{-20.037}) {}; % Oph,  4.49 
\node[pin={[pin distance=-0.4\onedegree,Bayer]00:{$\boldsymbol\rho$}}] at (axis cs:{246.396},{-23.446}) {}; % Oph,  5.74 
\node[pin={[pin distance=-0.4\onedegree,Bayer]00:{$\boldsymbol\sigma$}}] at (axis cs:{261.629},{4.140}) {}; % Oph,  4.32 
\node[pin={[pin distance=-0.4\onedegree,Bayer]180:{$\boldsymbol\tau$}}] at (axis cs:{270.770},{-8.180}) {}; % Oph,  4.77 
\node[pin={[pin distance=-0.4\onedegree,Bayer]00:{$\boldsymbol\upsilon$}}] at (axis cs:{246.951},{-8.372}) {}; % Oph,  4.64 
\node[pin={[pin distance=-0.4\onedegree,Bayer]00:{$\boldsymbol\varphi$}}] at (axis cs:{247.785},{-16.613}) {}; % Oph,  4.29 
\node[pin={[pin distance=-0.4\onedegree,Bayer]00:{$\boldsymbol\vartheta$}}] at (axis cs:{260.502},{-24.999}) {}; % Oph,  3.26 
\node[pin={[pin distance=-0.4\onedegree,Bayer]018:{$\boldsymbol\xi$}}] at (axis cs:{260.252},{-21.113}) {}; % Oph,  4.39 
\node[pin={[pin distance=-0.4\onedegree,Bayer]180:{$\boldsymbol\zeta$}}] at (axis cs:{249.290},{-10.567}) {}; % Oph,  2.58 
\node[pin={[pin distance=-0.4\onedegree,Flaamsted]00:{12}}] at (axis cs:{249.089},{-2.324}) {}; % Oph,  5.77 
\node[pin={[pin distance=-0.4\onedegree,Flaamsted]90:{14}}] at (axis cs:{250.427},{1.181}) {}; % Oph,  5.73 
\node[pin={[pin distance=-0.4\onedegree,Flaamsted]00:{20}}] at (axis cs:{252.458},{-10.783}) {}; % Oph,  4.64 
\node[pin={[pin distance=-0.4\onedegree,Flaamsted]90:{21}}] at (axis cs:{252.854},{1.216}) {}; % Oph,  5.51 
\node[pin={[pin distance=-0.4\onedegree,Flaamsted]00:{23}}] at (axis cs:{253.649},{-6.154}) {}; % Oph,  5.23 
\node[pin={[pin distance=-0.4\onedegree,Flaamsted]00:{24}}] at (axis cs:{254.200},{-23.150}) {}; % Oph,  5.59 
\node[pin={[pin distance=-0.4\onedegree,Flaamsted]180:{26}}] at (axis cs:{255.040},{-24.989}) {}; % Oph,  5.74 
\node[pin={[pin distance=-0.4\onedegree,Flaamsted]-90:{30}}] at (axis cs:{255.265},{-4.223}) {}; % Oph,  4.81 
\node[pin={[pin distance=-0.4\onedegree,Flaamsted]00:{36}}] at (axis cs:{258.837},{-26.602}) {}; % Oph,  5.07 
\node[pin={[pin distance=-0.4\onedegree,Flaamsted]00:{36}}] at (axis cs:{258.837},{-26.603}) {}; % Oph,  4.33 
\node[pin={[pin distance=-0.4\onedegree,Flaamsted]00:{37}}] at (axis cs:{258.116},{10.585}) {}; % Oph,  5.33 
\node[pin={[pin distance=-0.4\onedegree,Flaamsted]90:{41}}] at (axis cs:{259.153},{-0.445}) {}; % Oph,  4.74 
\node[pin={[pin distance=-0.4\onedegree,Flaamsted]00:{43}}] at (axis cs:{260.840},{-28.143}) {}; % Oph,  5.30 
\node[pin={[pin distance=-0.4\onedegree,Flaamsted]-90:{44}}] at (axis cs:{261.593},{-24.175}) {}; % Oph,  4.16 
\node[pin={[pin distance=-0.4\onedegree,Flaamsted]-90:{45}}] at (axis cs:{261.839},{-29.867}) {}; % Oph,  4.28 
\node[pin={[pin distance=-0.4\onedegree,Flaamsted]90:{51}}] at (axis cs:{262.854},{-23.962}) {}; % Oph,  4.80 
\node[pin={[pin distance=-0.4\onedegree,Flaamsted]00:{53}}] at (axis cs:{263.653},{9.587}) {}; % Oph,  5.80 
\node[pin={[pin distance=-0.4\onedegree,Flaamsted]180:{58}}] at (axis cs:{265.857},{-21.683}) {}; % Oph,  4.87 
\node[pin={[pin distance=-0.4\onedegree,Flaamsted]00:{66}}] at (axis cs:{270.066},{4.369}) {}; % Oph,  4.79 
\node[pin={[pin distance=-0.4\onedegree,Flaamsted]00:{67}}] at (axis cs:{270.161},{2.931}) {}; % Oph,  3.98 
\node[pin={[pin distance=-0.4\onedegree,Flaamsted]00:{68}}] at (axis cs:{270.438},{1.305}) {}; % Oph,  4.44 
\node[pin={[pin distance=-0.4\onedegree,Flaamsted]00:{70}}] at (axis cs:{271.364},{2.500}) {}; % Oph,  4.03 
\node[pin={[pin distance=-0.4\onedegree,Flaamsted]00:{71}}] at (axis cs:{271.827},{8.734}) {}; % Oph,  4.64 
\node[pin={[pin distance=-0.4\onedegree,Flaamsted]00:{72}}] at (axis cs:{271.837},{9.564}) {}; % Oph,  3.72 
\node[pin={[pin distance=-0.4\onedegree,Flaamsted]00:{73}}] at (axis cs:{272.391},{3.993}) {}; % Oph,  5.71 
\node[pin={[pin distance=-0.4\onedegree,Flaamsted]00:{74}}] at (axis cs:{275.217},{3.377}) {}; % Oph,  4.85 

\node[pin={[pin distance=-0.2\onedegree,Bayer]00:{$\boldsymbol\alpha$}}] at (axis cs:{247.352},{-26.432}) {}; % Sco,  1.07 
\node[pin={[pin distance=-0.2\onedegree,Bayer]00:{$\boldsymbol\delta$}}] at (axis cs:{240.083},{-22.622}) {}; % Sco,  2.30 
\node[pin={[pin distance=-0.2\onedegree,Bayer]00:{$\boldsymbol\epsilon$}}] at (axis cs:{252.541},{-34.293}) {}; % Sco,  2.29 
\node[pin={[pin distance=-0.2\onedegree,Bayer]00:{$\boldsymbol\kappa$}}] at (axis cs:{265.622},{-39.030}) {}; % Sco,  2.39 
\node[pin={[pin distance=-0.2\onedegree,Bayer]-90:{$\boldsymbol\lambda$}}] at (axis cs:{263.402},{-37.104}) {}; % Sco,  1.63 
\node[pin={[pin distance=-0.4\onedegree,Bayer]-90:{$\boldsymbol\beta^{1,2}$}}] at (axis cs:{241.359},{-19.805}) {}; % Sco,  2.62 
\node[pin={[pin distance=-0.4\onedegree,Bayer]00:{$\boldsymbol\chi$}}] at (axis cs:{243.462},{-11.838}) {}; % Sco,  5.24 
\node[pin={[pin distance=-0.4\onedegree,Bayer]00:{$\boldsymbol\mu^{1,2}$}}] at (axis cs:{252.968},{-38.047}) {}; % Sco,  3.00 
\node[pin={[pin distance=-0.4\onedegree,Bayer]00:{$\boldsymbol\nu$}}] at (axis cs:{242.999},{-19.461}) {}; % Sco,  4.13 
\node[pin={[pin distance=-0.4\onedegree,Bayer]180:{$\boldsymbol\omega^1$}}] at (axis cs:{241.702},{-20.669}) {}; % Sco,  3.95 
\node[pin={[pin distance=-0.4\onedegree,Bayer]90:{$\boldsymbol\omega^2$}}] at (axis cs:{241.851},{-20.869}) {}; % Sco,  4.32 
\node[pin={[pin distance=-0.4\onedegree,Bayer]180:{$\boldsymbol\omicron$}}] at (axis cs:{245.159},{-24.169}) {}; % Sco,  4.59 
\node[pin={[pin distance=-0.4\onedegree,Bayer]90:{$\boldsymbol\pi$}}] at (axis cs:{239.713},{-26.114}) {}; % Sco,  2.89 
\node[pin={[pin distance=-0.4\onedegree,Bayer]90:{$\boldsymbol\psi$}}] at (axis cs:{243.000},{-10.064}) {}; % Sco,  4.94 
\node[pin={[pin distance=-0.4\onedegree,Bayer]00:{$\boldsymbol\rho$}}] at (axis cs:{239.221},{-29.214}) {}; % Sco,  3.88 
\node[pin={[pin distance=-0.4\onedegree,Bayer]180:{$\boldsymbol\sigma$}}] at (axis cs:{245.297},{-25.593}) {}; % Sco,  2.91 
\node[pin={[pin distance=-0.4\onedegree,Bayer]00:{$\boldsymbol\tau$}}] at (axis cs:{248.971},{-28.216}) {}; % Sco,  2.83 
\node[pin={[pin distance=-0.4\onedegree,Bayer]180:{$\boldsymbol\upsilon$}}] at (axis cs:{262.691},{-37.296}) {}; % Sco,  2.68 
\node[pin={[pin distance=-0.4\onedegree,Bayer]00:{$\boldsymbol\xi$}}] at (axis cs:{241.092},{-11.373}) {}; % Sco,  4.16 
\node[pin={[pin distance=-0.4\onedegree,Flaamsted]00:{11}}] at (axis cs:{241.902},{-12.745}) {}; % Sco,  5.76 
\node[pin={[pin distance=-0.4\onedegree,Flaamsted]180:{12}}] at (axis cs:{243.067},{-28.417}) {}; % Sco,  5.67 
\node[pin={[pin distance=-0.4\onedegree,Flaamsted]180:{13}}] at (axis cs:{243.076},{-27.926}) {}; % Sco,  4.58 
\node[pin={[pin distance=-0.4\onedegree,Flaamsted]180:{16}}] at (axis cs:{243.030},{-8.547}) {}; % Sco,  5.44 
\node[pin={[pin distance=-0.4\onedegree,Flaamsted]90:{18}}] at (axis cs:{243.905},{-8.369}) {}; % Sco,  5.49 
\node[pin={[pin distance=-0.4\onedegree,Flaamsted]90:{ 1}}] at (axis cs:{237.745},{-25.751}) {}; % Sco,  4.65 
\node[pin={[pin distance=-0.4\onedegree,Flaamsted]00:{22}}] at (axis cs:{247.552},{-25.115}) {}; % Sco,  4.79 
\node[pin={[pin distance=-0.4\onedegree,Flaamsted]00:{27}}] at (axis cs:{254.297},{-33.259}) {}; % Sco,  5.48 
\node[pin={[pin distance=-0.4\onedegree,Flaamsted]180:{ 2}}] at (axis cs:{238.403},{-25.327}) {}; % Sco,  4.67 
\node[pin={[pin distance=-0.4\onedegree,Flaamsted]00:{ 3}}] at (axis cs:{238.665},{-25.244}) {}; % Sco,  5.87 
\node[pin={[pin distance=-0.4\onedegree,Flaamsted]180:{ 4}}] at (axis cs:{238.875},{-26.266}) {}; % Sco,  5.63 


\node[pin={[pin distance=-0.4\onedegree,Bayer]180:{$\boldsymbol\alpha$}}] at (axis cs:{278.802},{-8.244}) {}; % Sct,  3.85 
\node[pin={[pin distance=-0.4\onedegree,Bayer]00:{$\boldsymbol\beta$}}] at (axis cs:{281.794},{-4.748}) {}; % Sct,  4.23 
\node[pin={[pin distance=-0.4\onedegree,Bayer]90:{$\boldsymbol\delta$}}] at (axis cs:{280.568},{-9.053}) {}; % Sct,  4.71 
\node[pin={[pin distance=-0.4\onedegree,Bayer]00:{$\boldsymbol\epsilon$}}] at (axis cs:{280.880},{-8.275}) {}; % Sct,  4.90 
\node[pin={[pin distance=-0.4\onedegree,Bayer]180:{$\boldsymbol\eta$}}] at (axis cs:{284.265},{-5.846}) {}; % Sct,  4.82 
\node[pin={[pin distance=-0.4\onedegree,Bayer]180:{$\boldsymbol\gamma$}}] at (axis cs:{277.299},{-14.566}) {}; % Sct,  4.69 
\node[pin={[pin distance=-0.4\onedegree,Bayer]0:{$\boldsymbol\zeta$}}] at (axis cs:{275.915},{-8.934}) {}; % Sct,  4.67 


\node[pin={[pin distance=-0.4\onedegree,Bayer]00:{$\boldsymbol\alpha$}}] at (axis cs:{236.067},{6.425}) {}; % Ser,  2.63 
\node[pin={[pin distance=-0.4\onedegree,Bayer]00:{$\boldsymbol\beta$}}] at (axis cs:{236.547},{15.422}) {}; % Ser,  3.66 
\node[pin={[pin distance=-0.4\onedegree,Bayer]00:{$\boldsymbol\chi$}}] at (axis cs:{235.448},{12.847}) {}; % Ser,  5.33 
\node[pin={[pin distance=-0.4\onedegree,Bayer]00:{$\boldsymbol\delta$}}] at (axis cs:{233.701},{10.538}) {}; % Ser,  5.13 
\node[pin={[pin distance=-0.4\onedegree,Bayer]00:{$\boldsymbol\epsilon$}}] at (axis cs:{237.704},{4.478}) {}; % Ser,  3.71 
\node[pin={[pin distance=-0.4\onedegree,Bayer]0:{$\boldsymbol\eta$}}] at (axis cs:{275.328},{-2.899}) {}; % Ser,  3.25 
\node[pin={[pin distance=-0.4\onedegree,Bayer]00:{$\boldsymbol\gamma$}}] at (axis cs:{239.113},{15.661}) {}; % Ser,  3.85 
\node[pin={[pin distance=-0.4\onedegree,Bayer]00:{$\boldsymbol\iota$}}] at (axis cs:{235.388},{19.670}) {}; % Ser,  4.51 
\node[pin={[pin distance=-0.4\onedegree,Bayer]00:{$\boldsymbol\kappa$}}] at (axis cs:{237.185},{18.141}) {}; % Ser,  4.09 
\node[pin={[pin distance=-0.4\onedegree,Bayer]00:{$\boldsymbol\lambda$}}] at (axis cs:{236.611},{7.353}) {}; % Ser,  4.42 
\node[pin={[pin distance=-0.6\onedegree,Bayer]-135:{$\boldsymbol\mu$}}] at (axis cs:{237.405},{-3.430}) {}; % Ser,  3.55 
\node[pin={[pin distance=-0.4\onedegree,Bayer]-90:{$\boldsymbol\nu$}}] at (axis cs:{260.207},{-12.847}) {}; % Ser,  4.34 
\node[pin={[pin distance=-0.4\onedegree,Bayer]00:{$\boldsymbol\omega$}}] at (axis cs:{237.573},{2.196}) {}; % Ser,  5.23 
\node[pin={[pin distance=-0.4\onedegree,Bayer]00:{$\boldsymbol\omicron$}}] at (axis cs:{265.354},{-12.875}) {}; % Ser,  4.24 
\node[pin={[pin distance=-0.4\onedegree,Bayer]00:{$\boldsymbol\pi$}}] at (axis cs:{240.574},{22.804}) {}; % Ser,  4.82 
\node[pin={[pin distance=-0.4\onedegree,Bayer]00:{$\boldsymbol\psi$}}] at (axis cs:{236.008},{2.515}) {}; % Ser,  5.87 
\node[pin={[pin distance=-0.4\onedegree,Bayer]00:{$\boldsymbol\rho$}}] at (axis cs:{237.816},{20.978}) {}; % Ser,  4.74 
\node[pin={[pin distance=-0.4\onedegree,Bayer]180:{$\boldsymbol\sigma$}}] at (axis cs:{245.518},{1.029}) {}; % Ser,  4.82 
\node[pin={[pin distance=-0.4\onedegree,Bayer]00:{$\boldsymbol\tau^1$}}] at (axis cs:{231.447},{15.428}) {}; % Ser,  5.15 
\node[pin={[pin distance=-0.4\onedegree,Bayer]90:{$\boldsymbol\tau^5$}}] at (axis cs:{234.122},{16.119}) {}; % Ser,  5.94 
\node[pin={[pin distance=-0.4\onedegree,Bayer]90:{$\boldsymbol\tau^6$}}] at (axis cs:{235.246},{16.024}) {}; % Ser,  5.99 
\node[pin={[pin distance=-0.4\onedegree,Bayer]180:{$\boldsymbol\tau^7$}}] at (axis cs:{235.478},{18.464}) {}; % Ser,  5.80 
\node[pin={[pin distance=-0.4\onedegree,Bayer]00:{$\boldsymbol\upsilon$}}] at (axis cs:{236.822},{14.115}) {}; % Ser,  5.71 
\node[pin={[pin distance=-0.4\onedegree,Bayer]00:{$\boldsymbol\varphi$}}] at (axis cs:{239.311},{14.414}) {}; % Ser,  5.54 
\node[pin={[pin distance=-0.4\onedegree,Bayer]180:{$\boldsymbol\vartheta^{1,2}$}}] at (axis cs:{284.055},{4.204}) {}; % Ser,  4.61 
\node[pin={[pin distance=-0.4\onedegree,Bayer]00:{$\boldsymbol\xi$}}] at (axis cs:{264.397},{-15.398}) {}; % Ser,  3.54 
\node[pin={[pin distance=-0.4\onedegree,Bayer]00:{$\boldsymbol\zeta$}}] at (axis cs:{270.121},{-3.690}) {}; % Ser,  4.63 
\node[pin={[pin distance=-0.4\onedegree,Flaamsted]00:{10}}] at (axis cs:{232.159},{1.842}) {}; % Ser,  5.16 
\node[pin={[pin distance=-0.4\onedegree,Flaamsted]00:{11}}] at (axis cs:{233.241},{-1.186}) {}; % Ser,  5.51 
\node[pin={[pin distance=-0.4\onedegree,Flaamsted]90:{16}}] at (axis cs:{234.123},{10.010}) {}; % Ser,  5.26 
\node[pin={[pin distance=-0.4\onedegree,Flaamsted]00:{25}}] at (axis cs:{236.523},{-1.804}) {}; % Ser,  5.39 
\node[pin={[pin distance=-0.4\onedegree,Flaamsted]90:{36}}] at (axis cs:{237.815},{-3.090}) {}; % Ser,  5.10 
\node[pin={[pin distance=-0.4\onedegree,Flaamsted]00:{ 3}}] at (axis cs:{228.797},{4.939}) {}; % Ser,  5.34 
\node[pin={[pin distance=-0.4\onedegree,Flaamsted]180:{45}}] at (axis cs:{241.906},{9.892}) {}; % Ser,  5.64 
\node[pin={[pin distance=-0.4\onedegree,Flaamsted]180:{47}}] at (axis cs:{242.117},{8.534}) {}; % Ser,  5.74 
\node[pin={[pin distance=-0.4\onedegree,Flaamsted]00:{ 4}}] at (axis cs:{228.954},{0.372}) {}; % Ser,  5.63 
\node[pin={[pin distance=-0.6\onedegree,Flaamsted]-30:{59}}] at (axis cs:{276.802},{0.196}) {}; % Ser,  5.20 
\node[pin={[pin distance=-0.4\onedegree,Flaamsted]00:{ 5}}] at (axis cs:{229.828},{1.765}) {}; % Ser,  5.05 
\node[pin={[pin distance=-0.4\onedegree,Flaamsted]00:{60}}] at (axis cs:{277.421},{-1.985}) {}; % Ser,  5.38 
\node[pin={[pin distance=-0.4\onedegree,Flaamsted]00:{61}}] at (axis cs:{277.987},{-1.003}) {}; % Ser,  5.95 
\node[pin={[pin distance=-0.4\onedegree,Flaamsted]180:{64}}] at (axis cs:{284.319},{2.535}) {}; % Ser,  5.57 
\node[pin={[pin distance=-0.4\onedegree,Flaamsted]00:{ 6}}] at (axis cs:{230.258},{0.715}) {}; % Ser,  5.38 

\node[pin={[pin distance=-0.4\onedegree,Flaamsted]90:{ 1}}] at (axis cs:{288.822},{21.232}) {}; % Sge,  5.65 

\node[pin={[pin distance=-0.2\onedegree,Bayer]00:{$\boldsymbol\epsilon$}}] at (axis cs:{276.043},{-34.385}) {}; % Sgr,  1.81 
\node[pin={[pin distance=-0.2\onedegree,Bayer]00:{$\boldsymbol\sigma$}}] at (axis cs:{283.816},{-26.297}) {}; % Sgr,  2.07 
\node[pin={[pin distance=-0.4\onedegree,Bayer]180:{$\boldsymbol\delta$}}] at (axis cs:{275.249},{-29.828}) {}; % Sgr,  2.70 
\node[pin={[pin distance=-0.3\onedegree,Bayer]-90:{$\boldsymbol\eta$}}] at (axis cs:{274.407},{-36.761}) {}; % Sgr,  3.13 
\node[pin={[pin distance=-0.3\onedegree,Bayer]00:{$\boldsymbol\gamma^1$}}] at (axis cs:{271.255},{-29.580}) {}; % Sgr,  4.65 
\node[pin={[pin distance=-0.4\onedegree,Bayer]90:{$\boldsymbol\gamma^2$}}] at (axis cs:{271.452},{-30.424}) {}; % Sgr,  3.63 
\node[pin={[pin distance=-0.4\onedegree,Bayer]180:{$\boldsymbol\lambda$}}] at (axis cs:{276.993},{-25.422}) {}; % Sgr,  2.83 
\node[pin={[pin distance=-0.4\onedegree,Bayer]180:{$\boldsymbol\mu$}}] at (axis cs:{273.441},{-21.059}) {}; % Sgr,  3.84 
\node[pin={[pin distance=-0.4\onedegree,Bayer]180:{$\boldsymbol\nu^1$}}] at (axis cs:{283.542},{-22.745}) {}; % Sgr,  4.85 
\node[pin={[pin distance=-0.4\onedegree,Bayer]00:{$\boldsymbol\nu^2$}}] at (axis cs:{283.780},{-22.671}) {}; % Sgr,  5.00 
\node[pin={[pin distance=-0.4\onedegree,Bayer]00:{$\boldsymbol\omicron$}}] at (axis cs:{286.171},{-21.741}) {}; % Sgr,  3.77 
\node[pin={[pin distance=-0.4\onedegree,Bayer]00:{$\boldsymbol\pi$}}] at (axis cs:{287.441},{-21.024}) {}; % Sgr,  2.90 
\node[pin={[pin distance=-0.4\onedegree,Bayer]00:{$\boldsymbol\psi$}}] at (axis cs:{288.885},{-25.257}) {}; % Sgr,  4.87 
\node[pin={[pin distance=-0.4\onedegree,Bayer]00:{$\boldsymbol\tau$}}] at (axis cs:{286.735},{-27.670}) {}; % Sgr,  3.32 
\node[pin={[pin distance=-0.4\onedegree,Bayer]00:{$\boldsymbol\varphi$}}] at (axis cs:{281.414},{-26.991}) {}; % Sgr,  3.17 
\node[pin={[pin distance=-0.4\onedegree,Bayer]-90:{$\boldsymbol\xi^1$}}] at (axis cs:{284.335},{-20.656}) {}; % Sgr,  5.06 
\node[pin={[pin distance=-0.4\onedegree,Bayer]90:{$\boldsymbol\xi^2$}}] at (axis cs:{284.433},{-21.107}) {}; % Sgr,  3.53 
\node[pin={[pin distance=-0.3\onedegree,Bayer]90:{$\boldsymbol\zeta$}}] at (axis cs:{285.653},{-29.880}) {}; % Sgr,  2.61 
\node[pin={[pin distance=-0.4\onedegree,Flaamsted]00:{11}}] at (axis cs:{272.931},{-23.701}) {}; % Sgr,  4.96 
\node[pin={[pin distance=-0.4\onedegree,Flaamsted]0:{14}}] at (axis cs:{273.566},{-21.713}) {}; % Sgr,  5.48 
\node[pin={[pin distance=-0.4\onedegree,Flaamsted]00:{15}}] at (axis cs:{273.804},{-20.728}) {}; % Sgr,  5.35 
\node[pin={[pin distance=-0.4\onedegree,Flaamsted]-90:{16}}] at (axis cs:{273.804},{-20.388}) {}; % Sgr,  5.98 
\node[pin={[pin distance=-0.4\onedegree,Flaamsted]90:{18}}] at (axis cs:{276.256},{-30.756}) {}; % Sgr,  5.58 
\node[pin={[pin distance=-0.4\onedegree,Flaamsted]90:{21}}] at (axis cs:{276.338},{-20.542}) {}; % Sgr,  4.86 
\node[pin={[pin distance=-0.4\onedegree,Flaamsted]180:{24}}] at (axis cs:{278.473},{-24.032}) {}; % Sgr,  5.50 
\node[pin={[pin distance=-0.4\onedegree,Flaamsted]180:{28}}] at (axis cs:{281.586},{-22.392}) {}; % Sgr,  5.38 
\node[pin={[pin distance=-0.4\onedegree,Flaamsted]00:{29}}] at (axis cs:{282.417},{-20.325}) {}; % Sgr,  5.22 
\node[pin={[pin distance=-0.4\onedegree,Flaamsted]180:{33}}] at (axis cs:{283.500},{-21.360}) {}; % Sgr,  5.68 
\node[pin={[pin distance=-0.4\onedegree,Flaamsted]00:{ 3}}] at (axis cs:{266.890},{-27.831}) {}; % Sgr,  4.56 
\node[pin={[pin distance=-0.4\onedegree,Flaamsted]00:{43}}] at (axis cs:{289.409},{-18.953}) {}; % Sgr,  4.88 
\node[pin={[pin distance=-0.4\onedegree,Flaamsted]180:{ 4}}] at (axis cs:{269.948},{-23.816}) {}; % Sgr,  4.74 
\node[pin={[pin distance=-0.8\onedegree,Flaamsted]135:{ 7}}] at (axis cs:{270.713},{-24.282}) {}; % Sgr,  5.37 
\node[pin={[pin distance=-0.4\onedegree,Flaamsted]90:{ 9}}] at (axis cs:{270.969},{-24.361}) {}; % Sgr,  5.89 


\node[pin={[pin distance=-0.2\onedegree,Bayer]00:{$\boldsymbol\alpha$}}] at (axis cs:{201.298},{-11.161}) {}; % Vir,  1.06 
\node[pin={[pin distance=-0.4\onedegree,Bayer]00:{$\boldsymbol\chi$}}] at (axis cs:{189.812},{-7.995}) {}; % Vir,  4.65 
\node[pin={[pin distance=-0.4\onedegree,Bayer]00:{$\boldsymbol\delta$}}] at (axis cs:{193.901},{3.397}) {}; % Vir,  3.42 
\node[pin={[pin distance=-0.4\onedegree,Bayer]00:{$\boldsymbol\epsilon$}}] at (axis cs:{195.544},{10.959}) {}; % Vir,  2.84 
\node[pin={[pin distance=-0.4\onedegree,Bayer]00:{$\boldsymbol\eta$}}] at (axis cs:{184.976},{-0.667}) {}; % Vir,  3.89 
\node[pin={[pin distance=-0.3\onedegree,Bayer]90:{$\boldsymbol\gamma$}}] at (axis cs:{190.415},{-1.449}) {}; % Vir,  3.48 
\node[pin={[pin distance=-0.4\onedegree,Bayer]00:{$\boldsymbol\gamma$}}] at (axis cs:{190.415},{-1.449}) {}; % Vir,  3.50 
\node[pin={[pin distance=-0.4\onedegree,Bayer]00:{$\boldsymbol\iota$}}] at (axis cs:{214.004},{-6.001}) {}; % Vir,  4.08 
\node[pin={[pin distance=-0.4\onedegree,Bayer]180:{$\boldsymbol\kappa$}}] at (axis cs:{213.224},{-10.274}) {}; % Vir,  4.18 
\node[pin={[pin distance=-0.4\onedegree,Bayer]-90:{$\boldsymbol\lambda$}}] at (axis cs:{214.777},{-13.371}) {}; % Vir,  4.52 
\node[pin={[pin distance=-0.3\onedegree,Bayer]90:{$\boldsymbol\mu$}}] at (axis cs:{220.765},{-5.658}) {}; % Vir,  3.87 
\node[pin={[pin distance=-0.4\onedegree,Bayer]00:{$\boldsymbol\omicron$}}] at (axis cs:{181.302},{8.733}) {}; % Vir,  4.12 
\node[pin={[pin distance=-0.4\onedegree,Bayer]00:{$\boldsymbol\pi$}}] at (axis cs:{180.218},{6.614}) {}; % Vir,  4.66 
\node[pin={[pin distance=-0.4\onedegree,Bayer]00:{$\boldsymbol\psi$}}] at (axis cs:{193.588},{-9.539}) {}; % Vir,  4.79 
\node[pin={[pin distance=-0.4\onedegree,Bayer]0:{$\boldsymbol\rho$}}] at (axis cs:{190.471},{10.236}) {}; % Vir,  4.87 
\node[pin={[pin distance=-0.4\onedegree,Bayer]180:{$\boldsymbol\sigma$}}] at (axis cs:{199.401},{5.470}) {}; % Vir,  4.79 
\node[pin={[pin distance=-0.4\onedegree,Bayer]00:{$\boldsymbol\tau$}}] at (axis cs:{210.412},{1.544}) {}; % Vir,  4.24 
\node[pin={[pin distance=-0.4\onedegree,Bayer]00:{$\boldsymbol\upsilon$}}] at (axis cs:{214.885},{-2.265}) {}; % Vir,  5.15 
\node[pin={[pin distance=-0.4\onedegree,Bayer]00:{$\boldsymbol\varphi$}}] at (axis cs:{217.051},{-2.228}) {}; % Vir,  4.84 
\node[pin={[pin distance=-0.4\onedegree,Bayer]00:{$\boldsymbol\vartheta$}}] at (axis cs:{197.487},{-5.539}) {}; % Vir,  4.39 
\node[pin={[pin distance=-0.4\onedegree,Bayer]00:{$\boldsymbol\zeta$}}] at (axis cs:{203.673},{-0.596}) {}; % Vir,  3.38 
\node[pin={[pin distance=-0.4\onedegree,Flaamsted]00:{106}}] at (axis cs:{217.174},{-6.901}) {}; % Vir,  5.42 
\node[pin={[pin distance=-0.4\onedegree,Flaamsted]00:{108}}] at (axis cs:{221.376},{0.717}) {}; % Vir,  5.68 
\node[pin={[pin distance=-0.4\onedegree,Flaamsted]-90:{109}}] at (axis cs:{221.562},{1.893}) {}; % Vir,  3.73 
\node[pin={[pin distance=-0.4\onedegree,Flaamsted]00:{10}}] at (axis cs:{182.422},{1.898}) {}; % Vir,  5.96 
\node[pin={[pin distance=-0.7\onedegree,Flaamsted]-135:{110}}] at (axis cs:{225.725},{2.091}) {}; % Vir,  4.40 
\node[pin={[pin distance=-0.4\onedegree,Flaamsted]00:{11}}] at (axis cs:{182.514},{5.807}) {}; % Vir,  5.71 
\node[pin={[pin distance=-0.4\onedegree,Flaamsted]00:{12}}] at (axis cs:{183.358},{10.262}) {}; % Vir,  5.85 
\node[pin={[pin distance=-0.4\onedegree,Flaamsted]180:{13}}] at (axis cs:{184.668},{-0.787}) {}; % Vir,  5.91 
\node[pin={[pin distance=-0.4\onedegree,Flaamsted]00:{16}}] at (axis cs:{185.087},{3.312}) {}; % Vir,  4.97 
\node[pin={[pin distance=-0.4\onedegree,Flaamsted]00:{21}}] at (axis cs:{188.445},{-9.452}) {}; % Vir,  5.50 
\node[pin={[pin distance=-0.4\onedegree,Flaamsted]00:{25}}] at (axis cs:{189.197},{-5.832}) {}; % Vir,  5.88 
\node[pin={[pin distance=-0.4\onedegree,Flaamsted]90:{31}}] at (axis cs:{190.488},{6.807}) {}; % Vir,  5.58 
\node[pin={[pin distance=-0.4\onedegree,Flaamsted]90:{32}}] at (axis cs:{191.404},{7.673}) {}; % Vir,  5.21 
%\node[pin={[pin distance=-0.4\onedegree,Flaamsted]00:{33}}] at (axis cs:{191.594},{9.540}) {}; % Vir,  5.66 
\node[pin={[pin distance=-0.4\onedegree,Flaamsted]90:{44}}] at (axis cs:{194.915},{-3.812}) {}; % Vir,  5.80 
\node[pin={[pin distance=-0.4\onedegree,Flaamsted]-90:{46}}] at (axis cs:{195.150},{-3.368}) {}; % Vir,  5.99 
\node[pin={[pin distance=-0.4\onedegree,Flaamsted]90:{49}}] at (axis cs:{196.974},{-10.740}) {}; % Vir,  5.16 
\node[pin={[pin distance=-0.4\onedegree,Flaamsted]00:{50}}] at (axis cs:{197.439},{-10.329}) {}; % Vir,  5.96 
\node[pin={[pin distance=-0.4\onedegree,Flaamsted]00:{53}}] at (axis cs:{198.015},{-16.198}) {}; % Vir,  5.04 
\node[pin={[pin distance=-0.7\onedegree,Flaamsted]135:{55}}] at (axis cs:{198.545},{-19.931}) {}; % Vir,  5.32 
\node[pin={[pin distance=-0.7\onedegree,Flaamsted]45:{57}}] at (axis cs:{198.995},{-19.943}) {}; % Vir,  5.22 
\node[pin={[pin distance=-0.4\onedegree,Flaamsted]180:{59}}] at (axis cs:{199.194},{9.424}) {}; % Vir,  5.20 
\node[pin={[pin distance=-0.4\onedegree,Flaamsted]180:{61}}] at (axis cs:{199.601},{-18.311}) {}; % Vir,  4.74 
\node[pin={[pin distance=-0.4\onedegree,Flaamsted]00:{63}}] at (axis cs:{200.755},{-17.735}) {}; % Vir,  5.36 
\node[pin={[pin distance=-0.4\onedegree,Flaamsted]-90:{64}}] at (axis cs:{200.540},{5.155}) {}; % Vir,  5.88 
\node[pin={[pin distance=-0.4\onedegree,Flaamsted]-90:{65}}] at (axis cs:{200.829},{-4.924}) {}; % Vir,  5.86 
\node[pin={[pin distance=-0.4\onedegree,Flaamsted]90:{66}}] at (axis cs:{201.138},{-5.164}) {}; % Vir,  5.77 
\node[pin={[pin distance=-0.4\onedegree,Flaamsted]00:{68}}] at (axis cs:{201.680},{-12.708}) {}; % Vir,  5.26 
\node[pin={[pin distance=-0.4\onedegree,Flaamsted]00:{69}}] at (axis cs:{201.863},{-15.973}) {}; % Vir,  4.75 
\node[pin={[pin distance=-0.4\onedegree,Flaamsted]00:{70}}] at (axis cs:{202.108},{13.779}) {}; % Vir,  4.97 
\node[pin={[pin distance=-0.4\onedegree,Flaamsted]00:{71}}] at (axis cs:{202.304},{10.818}) {}; % Vir,  5.65 
\node[pin={[pin distance=-0.4\onedegree,Flaamsted]0:{74}}] at (axis cs:{202.991},{-6.256}) {}; % Vir,  4.70 
\node[pin={[pin distance=-0.4\onedegree,Flaamsted]180:{75}}] at (axis cs:{203.215},{-15.363}) {}; % Vir,  5.53 
\node[pin={[pin distance=-0.4\onedegree,Flaamsted]00:{76}}] at (axis cs:{203.242},{-10.165}) {}; % Vir,  5.21 
\node[pin={[pin distance=-0.4\onedegree,Flaamsted]00:{78}}] at (axis cs:{203.533},{3.659}) {}; % Vir,  4.92 
\node[pin={[pin distance=-0.4\onedegree,Flaamsted]00:{80}}] at (axis cs:{203.880},{-5.396}) {}; % Vir,  5.71 
\node[pin={[pin distance=-0.4\onedegree,Flaamsted]00:{82}}] at (axis cs:{205.403},{-8.703}) {}; % Vir,  5.03 
\node[pin={[pin distance=-0.4\onedegree,Flaamsted]00:{83}}] at (axis cs:{206.124},{-16.179}) {}; % Vir,  5.57 
\node[pin={[pin distance=-0.4\onedegree,Flaamsted]00:{84}}] at (axis cs:{205.765},{3.538}) {}; % Vir,  5.35 
\node[pin={[pin distance=-0.4\onedegree,Flaamsted]00:{86}}] at (axis cs:{206.485},{-12.426}) {}; % Vir,  5.50 
\node[pin={[pin distance=-0.4\onedegree,Flaamsted]180:{87}}] at (axis cs:{206.856},{-17.860}) {}; % Vir,  5.42 
\node[pin={[pin distance=-0.4\onedegree,Flaamsted]90:{89}}] at (axis cs:{207.468},{-18.134}) {}; % Vir,  4.96 
\node[pin={[pin distance=-0.4\onedegree,Flaamsted]00:{90}}] at (axis cs:{208.676},{-1.503}) {}; % Vir,  5.16 
\node[pin={[pin distance=-0.4\onedegree,Flaamsted]180:{92}}] at (axis cs:{209.116},{1.051}) {}; % Vir,  5.90 
\node[pin={[pin distance=-0.4\onedegree,Flaamsted]00:{95}}] at (axis cs:{211.678},{-9.313}) {}; % Vir,  5.46

\node[pin={[pin distance=-0.4\onedegree,Flaamsted]-90:{ 1}}] at (axis cs:{289.054},{21.390}) {}; % Vul,  4.77 
\node[pin={[pin distance=-0.4\onedegree,Flaamsted]180:{ 2}}] at (axis cs:{289.432},{23.025}) {}; % Vul,  5.48 
\end{axis}


%
% Comets, own objects etc.
\begin{axis}[name=transients,axis lines=none]
% Format for label:
%\node[transient,pin={[pin distance=-0.8\onedegree,transient-label]90:{LABEL}}] at (axis cs:{ *15 + /4 + /240 },{  + /60 + /3600 })  {+};
%
% Note for Southern positions, if in sexagesimal, decl. signs need to be inverted: 
%   \(axis cs:{ *15 + /4 + /240 },{   - /60 - /3600 })  {+};
%
% Note "08" is interpreted as Octal 8, and triggers a problem in computation

%%%%%%%%%%%%%%%%%%%%%%%%%%%%%%%%%%%%%%%%%%%%%%%%%%%%%%%%%%%%%%%%%%%%%%%%%%%%%%%%%%%%%%%%
% Comet C/2023 A3 (Tsuchinshan-ATLAS) in 2024, Apr-Dec\
% Coords taken from https://in-the-sky.org/ephemeris.php?objtxt=CK23A030
%%%%%%%%%%%%%%%%%%%%%%%%%%%%%%%%%%%%%%%%%%%%%%%%%%%%%%%%%%%%%%%%%%%%%%%%%%%%%%%%%%%%%%%%
%
% \node[transient,pin={[pin distance=-0.8\onedegree,transient-label]90:{Apr\,01}}] at (axis cs:{14 *15 + 32/4 +  30/240 },{ -04  - 59/60 -  6/3600 })  {+};
% \node[transient,pin={[pin distance=-0.8\onedegree,transient-label]90:{Apr\,08}}] at (axis cs:{14 *15 + 19/4 +  22/240 },{ -04  -  8/60 - 35/3600 })  {+};
% \node[transient,pin={[pin distance=-0.8\onedegree,transient-label]90:{Apr\,15}}] at (axis cs:{14 *15 +  4/4 +  04/240 },{ -03  - 11/60 - 56/3600 })  {+};
% \node[transient,pin={[pin distance=-0.8\onedegree,transient-label]90:{Apr\,22}}] at (axis cs:{13 *15 + 46/4 +  53/240 },{ -02  - 10/60 - 50/3600 })  {+};
% \node[transient,pin={[pin distance=-0.8\onedegree,transient-label]90:{May\,01}}] at (axis cs:{13 *15 + 22/4 +  47/240 },{ -00  - 50/60 -  5/3600 })  {+};
% \node[transient,pin={[pin distance=-0.8\onedegree,transient-label]90:{May\,08}}] at (axis cs:{13 *15 +  3/4 +  21/240 },{ +00  + 10/60 +  8/3600 })  {+};
% \node[transient,pin={[pin distance=-0.8\onedegree,transient-label]90:{May\,15}}] at (axis cs:{12 *15 + 44/4 +   8/240 },{ +01  +  4/60 + 17/3600 })  {+};
% \node[transient,pin={[pin distance=-0.8\onedegree,transient-label]90:{May\,22}}] at (axis cs:{12 *15 + 25/4 +  54/240 },{ +01  + 49/60 + 32/3600 })  {+};
% \node[transient,pin={[pin distance=-0.8\onedegree,transient-label]90:{Jun\,01}}] at (axis cs:{12 *15 +  2/4 +  36/240 },{ +02  + 35/60 + 35/3600 })  {+};
% \node[transient,pin={[pin distance=-0.8\onedegree,transient-label]90:{Jun\,08}}] at (axis cs:{11 *15 + 48/4 +  37/240 },{ +02  + 54/60 + 16/3600 })  {+};
% \node[transient,pin={[pin distance=-0.8\onedegree,transient-label]90:{Jun\,15}}] at (axis cs:{11 *15 + 36/4 +  38/240 },{ +03  +  2/60 + 21/3600 })  {+};
% \node[transient,pin={[pin distance=-0.8\onedegree,transient-label]90:{Jun\,22}}] at (axis cs:{11 *15 + 26/4 +  34/240 },{ +03  +  0/60 + 47/3600 })  {+};
% \node[transient,pin={[pin distance=-0.8\onedegree,transient-label]90:{Jul\,01}}] at (axis cs:{11 *15 + 16/4 +  10/240 },{ +02  + 46/60 + 21/3600 })  {+};
% \node[transient,pin={[pin distance=-0.8\onedegree,transient-label]90:{Jul\,08}}] at (axis cs:{11 *15 +  9/4 +  44/240 },{ +02  + 26/60 + 41/3600 })  {+};
% \node[transient,pin={[pin distance=-0.8\onedegree,transient-label]90:{Jul\,15}}] at (axis cs:{11 *15 +  4/4 +  32/240 },{ +02  +  0/60 + 32/3600 })  {+};
% \node[transient,pin={[pin distance=-0.8\onedegree,transient-label]90:{Jul\,22}}] at (axis cs:{11 *15 +  0/4 +  19/240 },{ +01  + 28/60 + 34/3600 })  {+};
% \node[transient,pin={[pin distance=-0.8\onedegree,transient-label]90:{Aug\,01}}] at (axis cs:{10 *15 + 55/4 +  30/240 },{ +00  + 33/60 + 43/3600 })  {+};
% \node[transient,pin={[pin distance=-0.8\onedegree,transient-label]90:{Aug\,08}}] at (axis cs:{10 *15 + 52/4 +  42/240 },{ -00  - 10/60 - 40/3600 })  {+};
% \node[transient,pin={[pin distance=-0.8\onedegree,transient-label]90:{Aug\,15}}] at (axis cs:{10 *15 + 50/4 +  05/240 },{ -00  - 59/60 - 45/3600 })  {+};
% \node[transient,pin={[pin distance=-0.8\onedegree,transient-label]90:{Aug\,22}}] at (axis cs:{10 *15 + 47/4 +  26/240 },{ -01  - 53/60 - 19/3600 })  {+};
% \node[transient,pin={[pin distance=-0.8\onedegree,transient-label]90:{Sep\,01}}] at (axis cs:{10 *15 + 43/4 +  06/240 },{ -03  - 16/60 - 41/3600 })  {+};
% \node[transient,pin={[pin distance=-0.8\onedegree,transient-label]90:{Sep\,08}}] at (axis cs:{10 *15 + 39/4 +  30/240 },{ -04  - 17/60 - 50/3600 })  {+};
% \node[transient,pin={[pin distance=-0.8\onedegree,transient-label]90:{Sep\,15}}] at (axis cs:{10 *15 + 35/4 +  57/240 },{ -05  - 16/60 -  5/3600 })  {+};
% \node[transient,pin={[pin distance=-0.8\onedegree,transient-label]90:{Sep\,22}}] at (axis cs:{10 *15 + 35/4 +  49/240 },{ -05  - 58/60 -  7/3600 })  {+};
% \node[transient,pin={[pin distance=-0.8\onedegree,transient-label]90:{Oct\,01}}] at (axis cs:{11 *15 +  4/4 +  03/240 },{ -05  - 41/60 - 36/3600 })  {+};
% \node[transient,pin={[pin distance=-0.8\onedegree,transient-label]90:{Oct\,08}}] at (axis cs:{12 *15 + 30/4 +  27/240 },{ -03  - 36/60 - 16/3600 })  {+};
% \node[transient,pin={[pin distance=-0.8\onedegree,transient-label]90:{Oct\,15}}] at (axis cs:{14 *15 + 59/4 +  40/240 },{ +00  + 23/60 + 14/3600 })  {+};
% \node[transient,pin={[pin distance=-0.8\onedegree,transient-label]90:{Oct\,22}}] at (axis cs:{16 *15 + 49/4 +  25/240 },{ +02  + 50/60 + 11/3600 })  {+};
% \node[transient,pin={[pin distance=-0.8\onedegree,transient-label]90:{Nov\,01}}] at (axis cs:{17 *15 + 59/4 +  40/240 },{ +03  + 45/60 + 12/3600 })  {+};
% \node[transient,pin={[pin distance=-0.8\onedegree,transient-label]90:{Nov\,08}}] at (axis cs:{18 *15 + 24/4 +  28/240 },{ +03  + 55/60 + 51/3600 })  {+};
% \node[transient,pin={[pin distance=-0.8\onedegree,transient-label]90:{Nov\,15}}] at (axis cs:{18 *15 + 41/4 +  24/240 },{ +04  +  3/60 + 27/3600 })  {+};
% \node[transient,pin={[pin distance=-0.8\onedegree,transient-label]90:{Nov\,22}}] at (axis cs:{18 *15 + 54/4 +  18/240 },{ +04  + 12/60 + 40/3600 })  {+};
% \node[transient,pin={[pin distance=-0.8\onedegree,transient-label]90:{Dec\,01}}] at (axis cs:{19 *15 +  7/4 +  39/240 },{ +04  + 29/60 + 30/3600 })  {+};
% \node[transient,pin={[pin distance=-0.8\onedegree,transient-label]90:{Dec\,08}}] at (axis cs:{19 *15 + 16/4 +  32/240 },{ +04  + 47/60 + 12/3600 })  {+};
% \node[transient,pin={[pin distance=-0.8\onedegree,transient-label]90:{Dec\,15}}] at (axis cs:{19 *15 + 24/4 +  32/240 },{ +05  +  9/60 +  7/3600 })  {+};
% \node[transient,pin={[pin distance=-0.8\onedegree,transient-label]90:{Dec\,22}}] at (axis cs:{19 *15 + 31/4 +  54/240 },{ +05  + 35/60 + 17/3600 })  {+};
% 
% 
% \draw[transient]
% (axis cs:{14 *15 + 32/4 + 30/240 },{ -04  - 59/60 -  6/3600 }) --
% (axis cs:{14 *15 + 19/4 + 22/240 },{ -04  -  8/60 - 35/3600 }) --
% (axis cs:{14 *15 +  4/4 + 04/240 },{ -03  - 11/60 - 56/3600 }) --
% (axis cs:{13 *15 + 46/4 + 53/240 },{ -02  - 10/60 - 50/3600 }) --
% (axis cs:{13 *15 + 22/4 + 47/240 },{ -00  - 50/60 -  5/3600 }) --
% (axis cs:{13 *15 +  3/4 + 21/240 },{ +00  + 10/60 +  8/3600 }) --
% (axis cs:{12 *15 + 44/4 +  8/240 },{ +01  +  4/60 + 17/3600 }) --
% (axis cs:{12 *15 + 25/4 + 54/240 },{ +01  + 49/60 + 32/3600 }) --
% (axis cs:{12 *15 +  2/4 + 36/240 },{ +02  + 35/60 + 35/3600 }) --
% (axis cs:{11 *15 + 48/4 + 37/240 },{ +02  + 54/60 + 16/3600 }) --
% (axis cs:{11 *15 + 36/4 + 38/240 },{ +03  +  2/60 + 21/3600 }) --
% (axis cs:{11 *15 + 26/4 + 34/240 },{ +03  +  0/60 + 47/3600 }) --
% (axis cs:{11 *15 + 16/4 + 10/240 },{ +02  + 46/60 + 21/3600 }) --
% (axis cs:{11 *15 +  9/4 + 44/240 },{ +02  + 26/60 + 41/3600 }) --
% (axis cs:{11 *15 +  4/4 + 32/240 },{ +02  +  0/60 + 32/3600 }) --
% (axis cs:{11 *15 +  0/4 + 19/240 },{ +01  + 28/60 + 34/3600 }) --
% (axis cs:{10 *15 + 55/4 + 30/240 },{ +00  + 33/60 + 43/3600 }) --
% (axis cs:{10 *15 + 52/4 + 42/240 },{ -00  - 10/60 - 40/3600 }) --
% (axis cs:{10 *15 + 50/4 + 05/240 },{ -00  - 59/60 - 45/3600 }) --
% (axis cs:{10 *15 + 47/4 + 26/240 },{ -01  - 53/60 - 19/3600 }) --
% (axis cs:{10 *15 + 43/4 + 06/240 },{ -03  - 16/60 - 41/3600 }) --
% (axis cs:{10 *15 + 39/4 + 30/240 },{ -04  - 17/60 - 50/3600 }) --
% (axis cs:{10 *15 + 35/4 + 57/240 },{ -05  - 16/60 -  5/3600 }) --
% (axis cs:{10 *15 + 35/4 + 49/240 },{ -05  - 58/60 -  7/3600 }) --
% (axis cs:{11 *15 +  4/4 + 03/240 },{ -05  - 41/60 - 36/3600 }) --
% (axis cs:{12 *15 + 30/4 + 27/240 },{ -03  - 36/60 - 16/3600 }) --
% (axis cs:{14 *15 + 59/4 + 40/240 },{ +00  + 23/60 + 14/3600 }) --
% (axis cs:{16 *15 + 49/4 + 25/240 },{ +02  + 50/60 + 11/3600 }) --
% (axis cs:{17 *15 + 59/4 + 40/240 },{ +03  + 45/60 + 12/3600 }) --
% (axis cs:{18 *15 + 24/4 + 28/240 },{ +03  + 55/60 + 51/3600 }) --
% (axis cs:{18 *15 + 41/4 + 24/240 },{ +04  +  3/60 + 27/3600 }) --
% (axis cs:{18 *15 + 54/4 + 18/240 },{ +04  + 12/60 + 40/3600 }) --
% (axis cs:{19 *15 +  7/4 + 39/240 },{ +04  + 29/60 + 30/3600 }) --
% (axis cs:{19 *15 + 16/4 + 32/240 },{ +04  + 47/60 + 12/3600 }) --
% (axis cs:{19 *15 + 24/4 + 32/240 },{ +05  +  9/60 +  7/3600 }) --
% (axis cs:{19 *15 + 31/4 + 54/240 },{ +05  + 35/60 + 17/3600 }) ;
% 
% 

\end{axis}

\end{tikzpicture}


\end{document}

