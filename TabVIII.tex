%%%%%%%%%%%%%%%%%%%%%%%%%%%%%%%%%%%%%%%%%%%%%%%%%%%%%%%%%%%%%%%%%%%%%%
%
% rm TabulaVIII.pdf ; pdflatex -shell-escape  TabVIII.tex 
%
%%%%%%%%%%%%%%%%%%%%%%%%%%%%%%%%%%%%%%%%%%%%%%%%%%%%%%%%%%%%%%%%%%%%%%
\documentclass[10pt,landscape]{article}
\usepackage[margin=1cm,a3paper]{geometry}


\usepackage[margin=1cm,a3paper]{geometry}

\def\pgfsysdriver{pgfsys-pdftex.def}
\usepackage{pgfplots}
\usepgfplotslibrary{external} 
\usetikzlibrary{pgfplots.external}
\usetikzlibrary{calc}
\usetikzlibrary{shapes}
\usetikzlibrary{patterns}
\usetikzlibrary{plotmarks}
\usepgflibrary{arrows}
\usepgfplotslibrary{polar}
\tikzexternalize

\usepackage{times,latexsym,amssymb}
\usepackage{eulervm} % math font
\pagestyle{empty}

\usepackage{amsmath} 
\usepackage{nicefrac} 
%
% This is used because from 1.11 onwards the use of \pgfdeclareplotmark will
% introduce a shift of the stellar markers.
%
\pgfplotsset{compat=1.10}
%
% Use to increase spacing for constellation labels
\usepackage[letterspace=400]{microtype}
%
\usepackage{pagecolor}
\usepackage{natbib}
%


%\pagecolor{black}
\pagecolor{white}


\begin{document} 


%\input{./input/styles_dark.tex}



%%%%%%%%%%%%%%%%%%%%%%%%%%%%%%%%%%%%%%%%%%%%%%%%%%%%%%%%%%%%%%%%%%%%%%%%%%%%%%%%%%%%%%%%%%
%%%%%%%%%%%%%%%%%%%%%%%%%%%%%%%%%%%%%%%%%%%%%%%%%%%%%%%%%%%%%%%%%%%%%%%%%%%%%%%%%%%%%%%%%%
%   Magnitude markers
%%%%%%%%%%%%%%%%%%%%%%%%%%%%%%%%%%%%%%%%%%%%%%%%%%%%%%%%%%%%%%%%%%%%%%%%%%%%%%%%%%%%%%%%%%
%%%%%%%%%%%%%%%%%%%%%%%%%%%%%%%%%%%%%%%%%%%%%%%%%%%%%%%%%%%%%%%%%%%%%%%%%%%%%%%%%%%%%%%%%%

\pgfdeclareplotmark{m1b}{%
\node[scale=\mOnescale*\AtoBscale*\symscale] at (0,0) {\tikz {%
\draw[fill=black,line width=.3pt,even odd rule,rotate=90] 
(0.*25.71428:10pt)  -- (0.5*25.71428:5pt)   -- 
(1.*25.71428:10pt)  -- (1.5*25.71428:5pt)   -- 
(2.*25.71428:10pt)  -- (2.5*25.71428:5pt)   -- 
(3.*25.71428:10pt)  -- (3.5*25.71428:5pt)   -- 
(4.*25.71428:10pt)  -- (4.5*25.71428:5pt)   -- 
(5.*25.71428:10pt)  -- (5.5*25.71428:5pt)   -- 
(6.*25.71428:10pt)  -- (6.5*25.71428:5pt)   -- 
(7.*25.71428:10pt)  -- (7.5*25.71428:5pt)   -- 
(8.*25.71428:10pt)  -- (8.5*25.71428:5pt)   -- 
(9.*25.71428:10pt)  -- (9.5*25.71428:5pt)   -- 
(10.*25.71428:10pt) -- (10.5*25.71428:5pt)  -- 
(11.*25.71428:10pt) -- (11.5*25.71428:5pt)  -- 
(12.*25.71428:10pt) -- (12.5*25.71428:5pt)  -- 
(13.*25.71428:10pt) -- (13.5*25.71428:5pt)  --  cycle;
}}}


\pgfdeclareplotmark{m1bv}{%
\node[scale=\mOnescale*\AtoBscale*\symscale] at (0,0) {\tikz {%
\draw[fill=black,line width=.3pt,even odd rule,rotate=90] 
(0.*25.71428:9pt)  -- (0.5*25.71428:4.5pt)   -- 
(1.*25.71428:9pt)  -- (1.5*25.71428:4.5pt)   -- 
(2.*25.71428:9pt)  -- (2.5*25.71428:4.5pt)   -- 
(3.*25.71428:9pt)  -- (3.5*25.71428:4.5pt)   -- 
(4.*25.71428:9pt)  -- (4.5*25.71428:4.5pt)   -- 
(5.*25.71428:9pt)  -- (5.5*25.71428:4.5pt)   -- 
(6.*25.71428:9pt)  -- (6.5*25.71428:4.5pt)   -- 
(7.*25.71428:9pt)  -- (7.5*25.71428:4.5pt)   -- 
(8.*25.71428:9pt)  -- (8.5*25.71428:4.5pt)   -- 
(9.*25.71428:9pt)  -- (9.5*25.71428:4.5pt)   -- 
(10.*25.71428:9pt) -- (10.5*25.71428:4.5pt)  -- 
(11.*25.71428:9pt) -- (11.5*25.71428:4.5pt)  -- 
(12.*25.71428:9pt) -- (12.5*25.71428:4.5pt)  -- 
(13.*25.71428:9pt) -- (13.5*25.71428:4.5pt)  --cycle;
\draw (0,0) circle (9pt);
%\draw[line width=.2pt,even odd rule,rotate=90] 
%(0.*25.71428:11pt)  -- (0.5*25.71428:6.5pt)   -- 
%(1.*25.71428:11pt)  -- (1.5*25.71428:6.5pt)   -- 
%(2.*25.71428:11pt)  -- (2.5*25.71428:6.5pt)   -- 
%(3.*25.71428:11pt)  -- (3.5*25.71428:6.5pt)   -- 
%(4.*25.71428:11pt)  -- (4.5*25.71428:6.5pt)   -- 
%(5.*25.71428:11pt)  -- (5.5*25.71428:6.5pt)   -- 
%(6.*25.71428:11pt)  -- (6.5*25.71428:6.5pt)   -- 
%(7.*25.71428:11pt)  -- (7.5*25.71428:6.5pt)   -- 
%(8.*25.71428:11pt)  -- (8.5*25.71428:6.5pt)   -- 
%(9.*25.71428:11pt)  -- (9.5*25.71428:6.5pt)   -- 
%(10.*25.71428:11pt) -- (10.5*25.71428:6.5pt)  -- 
%(11.*25.71428:11pt) -- (11.5*25.71428:6.5pt)  -- 
%(12.*25.71428:11pt) -- (12.5*25.71428:6.5pt)  -- 
%(13.*25.71428:11pt) -- (13.5*25.71428:6.5pt)  --  cycle;
}}}



\pgfdeclareplotmark{m1bb}{%
\node[scale=\mOnescale*\AtoBscale*\symscale] at (0,0) {\tikz {%
\draw[fill=black,line width=.3pt,even odd rule,rotate=90] 
(0.*25.71428:10pt)  -- (0.5*25.71428:5pt)   -- 
(1.*25.71428:10pt)  -- (1.5*25.71428:5pt)   -- 
(2.*25.71428:10pt)  -- (2.5*25.71428:5pt)   -- 
(3.*25.71428:10pt)  -- (3.5*25.71428:5pt)   -- 
(4.*25.71428:10pt)  -- (4.5*25.71428:5pt)   -- 
(5.*25.71428:10pt)  -- (5.5*25.71428:5pt)   -- 
(6.*25.71428:10pt)  -- (6.5*25.71428:5pt)   -- 
(7.*25.71428:10pt)  -- (7.5*25.71428:5pt)   -- 
(8.*25.71428:10pt)  -- (8.5*25.71428:5pt)   -- 
(9.*25.71428:10pt)  -- (9.5*25.71428:5pt)   -- 
(10.*25.71428:10pt) -- (10.5*25.71428:5pt)  -- 
(11.*25.71428:10pt) -- (11.5*25.71428:5pt)  -- 
(12.*25.71428:10pt) -- (12.5*25.71428:5pt)  -- 
(13.*25.71428:10pt) -- (13.5*25.71428:5pt)  --   cycle;
\draw[line width=.4pt,even odd rule,rotate=90] 
(0.5*25.71428:9pt)  -- (0.5*25.71428:5pt)   
(1.5*25.71428:9pt)  -- (1.5*25.71428:5pt)   
(2.5*25.71428:9pt)  -- (2.5*25.71428:5pt)   
(3.5*25.71428:9pt)  -- (3.5*25.71428:5pt)   
(4.5*25.71428:9pt)  -- (4.5*25.71428:5pt)   
(5.5*25.71428:9pt)  -- (5.5*25.71428:5pt)   
(6.5*25.71428:9pt)  -- (6.5*25.71428:5pt)   
(7.5*25.71428:9pt)  -- (7.5*25.71428:5pt)   
(8.5*25.71428:9pt)  -- (8.5*25.71428:5pt)   
(9.5*25.71428:9pt)  -- (9.5*25.71428:5pt)   
(10.5*25.71428:9pt) -- (10.5*25.71428:5pt)  
(11.5*25.71428:9pt) -- (11.5*25.71428:5pt)  
(12.5*25.71428:9pt) -- (12.5*25.71428:5pt)  
(13.5*25.71428:9pt) -- (13.5*25.71428:5pt)  ;
 }}}

\pgfdeclareplotmark{m1bvb}{%
\node[scale=\mOnescale*\AtoBscale*\symscale] at (0,0) {\tikz {%
\draw[fill=black,line width=.3pt,even odd rule,rotate=90] 
(0.*25.71428:9pt)  -- (0.5*25.71428:4.5pt)   -- 
(1.*25.71428:9pt)  -- (1.5*25.71428:4.5pt)   -- 
(2.*25.71428:9pt)  -- (2.5*25.71428:4.5pt)   -- 
(3.*25.71428:9pt)  -- (3.5*25.71428:4.5pt)   -- 
(4.*25.71428:9pt)  -- (4.5*25.71428:4.5pt)   -- 
(5.*25.71428:9pt)  -- (5.5*25.71428:4.5pt)   -- 
(6.*25.71428:9pt)  -- (6.5*25.71428:4.5pt)   -- 
(7.*25.71428:9pt)  -- (7.5*25.71428:4.5pt)   -- 
(8.*25.71428:9pt)  -- (8.5*25.71428:4.5pt)   -- 
(9.*25.71428:9pt)  -- (9.5*25.71428:4.5pt)   -- 
(10.*25.71428:9pt) -- (10.5*25.71428:4.5pt)  -- 
(11.*25.71428:9pt) -- (11.5*25.71428:4.5pt)  -- 
(12.*25.71428:9pt) -- (12.5*25.71428:4.5pt)  -- 
(13.*25.71428:9pt) -- (13.5*25.71428:4.5pt)  --  cycle;
\draw (0,0) circle (9pt);
%\draw[line width=.2pt,even odd rule,rotate=90] 
%(0.*25.71428:11pt)  -- (0.5*25.71428:6.5pt)   -- 
%(1.*25.71428:11pt)  -- (1.5*25.71428:6.5pt)   -- 
%(2.*25.71428:11pt)  -- (2.5*25.71428:6.5pt)   -- 
%(3.*25.71428:11pt)  -- (3.5*25.71428:6.5pt)   -- 
%(4.*25.71428:11pt)  -- (4.5*25.71428:6.5pt)   -- 
%(5.*25.71428:11pt)  -- (5.5*25.71428:6.5pt)   -- 
%(6.*25.71428:11pt)  -- (6.5*25.71428:6.5pt)   -- 
%(7.*25.71428:11pt)  -- (7.5*25.71428:6.5pt)   -- 
%(8.*25.71428:11pt)  -- (8.5*25.71428:6.5pt)   -- 
%(9.*25.71428:11pt)  -- (9.5*25.71428:6.5pt)   -- 
%(10.*25.71428:11pt) -- (10.5*25.71428:6.5pt)  -- 
%(11.*25.71428:11pt) -- (11.5*25.71428:6.5pt)  -- 
%(12.*25.71428:11pt) -- (12.5*25.71428:6.5pt)  -- 
%(13.*25.71428:11pt) -- (13.5*25.71428:6.5pt)  --  cycle;
\draw[line width=.4pt,even odd rule,rotate=90] 
(0.5*25.71428:10pt)  -- (0.5*25.71428:6.5pt)   
(1.5*25.71428:10pt)  -- (1.5*25.71428:6.5pt)   
(2.5*25.71428:10pt)  -- (2.5*25.71428:6.5pt)   
(3.5*25.71428:10pt)  -- (3.5*25.71428:6.5pt)   
(4.5*25.71428:10pt)  -- (4.5*25.71428:6.5pt)   
(5.5*25.71428:10pt)  -- (5.5*25.71428:6.5pt)   
(6.5*25.71428:10pt)  -- (6.5*25.71428:6.5pt)   
(7.5*25.71428:10pt)  -- (7.5*25.71428:6.5pt)   
(8.5*25.71428:10pt)  -- (8.5*25.71428:6.5pt)   
(9.5*25.71428:10pt)  -- (9.5*25.71428:6.5pt)   
(10.5*25.71428:10pt) -- (10.5*25.71428:6.5pt)  
(11.5*25.71428:10pt) -- (11.5*25.71428:6.5pt)  
(12.5*25.71428:10pt) -- (12.5*25.71428:6.5pt)  
(13.5*25.71428:10pt) -- (13.5*25.71428:6.5pt)  ;
 }}}



\pgfdeclareplotmark{m1c}{%
\node[scale=\mOnescale*\AtoCscale*\symscale] at (0,0) {\tikz {%
\draw[fill=black,line width=.3pt,even odd rule,rotate=90] 
(0.*25.71428:10pt)  -- (0.5*25.71428:5pt)   -- 
(1.*25.71428:10pt)  -- (1.5*25.71428:5pt)   -- 
(2.*25.71428:10pt)  -- (2.5*25.71428:5pt)   -- 
(3.*25.71428:10pt)  -- (3.5*25.71428:5pt)   -- 
(4.*25.71428:10pt)  -- (4.5*25.71428:5pt)   -- 
(5.*25.71428:10pt)  -- (5.5*25.71428:5pt)   -- 
(6.*25.71428:10pt)  -- (6.5*25.71428:5pt)   -- 
(7.*25.71428:10pt)  -- (7.5*25.71428:5pt)   -- 
(8.*25.71428:10pt)  -- (8.5*25.71428:5pt)   -- 
(9.*25.71428:10pt)  -- (9.5*25.71428:5pt)   -- 
(10.*25.71428:10pt) -- (10.5*25.71428:5pt)  -- 
(11.*25.71428:10pt) -- (11.5*25.71428:5pt)  -- 
(12.*25.71428:10pt) -- (12.5*25.71428:5pt)  -- 
(13.*25.71428:10pt) -- (13.5*25.71428:5pt)  --  cycle
(0,0) circle (2pt);
 }}}


\pgfdeclareplotmark{m1cv}{%
\node[scale=\mOnescale*\AtoCscale*\symscale] at (0,0) {\tikz {%
\draw[fill=black,line width=.3pt,even odd rule,rotate=90] 
(0.*25.71428:9pt)  -- (0.5*25.71428:4.5pt)   -- 
(1.*25.71428:9pt)  -- (1.5*25.71428:4.5pt)   -- 
(2.*25.71428:9pt)  -- (2.5*25.71428:4.5pt)   -- 
(3.*25.71428:9pt)  -- (3.5*25.71428:4.5pt)   -- 
(4.*25.71428:9pt)  -- (4.5*25.71428:4.5pt)   -- 
(5.*25.71428:9pt)  -- (5.5*25.71428:4.5pt)   -- 
(6.*25.71428:9pt)  -- (6.5*25.71428:4.5pt)   -- 
(7.*25.71428:9pt)  -- (7.5*25.71428:4.5pt)   -- 
(8.*25.71428:9pt)  -- (8.5*25.71428:4.5pt)   -- 
(9.*25.71428:9pt)  -- (9.5*25.71428:4.5pt)   -- 
(10.*25.71428:9pt) -- (10.5*25.71428:4.5pt)  -- 
(11.*25.71428:9pt) -- (11.5*25.71428:4.5pt)  -- 
(12.*25.71428:9pt) -- (12.5*25.71428:4.5pt)  -- 
(13.*25.71428:9pt) -- (13.5*25.71428:4.5pt)  -- cycle
(0,0) circle (2pt);
\draw (0,0) circle (9pt);
%\draw[line width=.2pt,even odd rule,rotate=90] 
%(0.*25.71428:11pt)  -- (0.5*25.71428:6.5pt)   -- 
%(1.*25.71428:11pt)  -- (1.5*25.71428:6.5pt)   -- 
%(2.*25.71428:11pt)  -- (2.5*25.71428:6.5pt)   -- 
%(3.*25.71428:11pt)  -- (3.5*25.71428:6.5pt)   -- 
%(4.*25.71428:11pt)  -- (4.5*25.71428:6.5pt)   -- 
%(5.*25.71428:11pt)  -- (5.5*25.71428:6.5pt)   -- 
%(6.*25.71428:11pt)  -- (6.5*25.71428:6.5pt)   -- 
%(7.*25.71428:11pt)  -- (7.5*25.71428:6.5pt)   -- 
%(8.*25.71428:11pt)  -- (8.5*25.71428:6.5pt)   -- 
%(9.*25.71428:11pt)  -- (9.5*25.71428:6.5pt)   -- 
%(10.*25.71428:11pt) -- (10.5*25.71428:6.5pt)  -- 
%(11.*25.71428:11pt) -- (11.5*25.71428:6.5pt)  -- 
%(12.*25.71428:11pt) -- (12.5*25.71428:6.5pt)  -- 
%(13.*25.71428:11pt) -- (13.5*25.71428:6.5pt)  --  cycle;
 }}}



\pgfdeclareplotmark{m1cb}{%
\node[scale=\mOnescale*\AtoCscale*\symscale] at (0,0) {\tikz {%
\draw[fill=black,line width=.3pt,even odd rule,rotate=90] 
(0.*25.71428:10pt)  -- (0.5*25.71428:5pt)   -- 
(1.*25.71428:10pt)  -- (1.5*25.71428:5pt)   -- 
(2.*25.71428:10pt)  -- (2.5*25.71428:5pt)   -- 
(3.*25.71428:10pt)  -- (3.5*25.71428:5pt)   -- 
(4.*25.71428:10pt)  -- (4.5*25.71428:5pt)   -- 
(5.*25.71428:10pt)  -- (5.5*25.71428:5pt)   -- 
(6.*25.71428:10pt)  -- (6.5*25.71428:5pt)   -- 
(7.*25.71428:10pt)  -- (7.5*25.71428:5pt)   -- 
(8.*25.71428:10pt)  -- (8.5*25.71428:5pt)   -- 
(9.*25.71428:10pt)  -- (9.5*25.71428:5pt)   -- 
(10.*25.71428:10pt) -- (10.5*25.71428:5pt)  -- 
(11.*25.71428:10pt) -- (11.5*25.71428:5pt)  -- 
(12.*25.71428:10pt) -- (12.5*25.71428:5pt)  -- 
(13.*25.71428:10pt) -- (13.5*25.71428:5pt)  --  cycle
(0,0) circle (2pt);
\draw[line width=.4pt,even odd rule,rotate=90] 
(0.5*25.71428:9pt)  -- (0.5*25.71428:5pt)   
(1.5*25.71428:9pt)  -- (1.5*25.71428:5pt)   
(2.5*25.71428:9pt)  -- (2.5*25.71428:5pt)   
(3.5*25.71428:9pt)  -- (3.5*25.71428:5pt)   
(4.5*25.71428:9pt)  -- (4.5*25.71428:5pt)   
(5.5*25.71428:9pt)  -- (5.5*25.71428:5pt)   
(6.5*25.71428:9pt)  -- (6.5*25.71428:5pt)   
(7.5*25.71428:9pt)  -- (7.5*25.71428:5pt)   
(8.5*25.71428:9pt)  -- (8.5*25.71428:5pt)   
(9.5*25.71428:9pt)  -- (9.5*25.71428:5pt)   
(10.5*25.71428:9pt) -- (10.5*25.71428:5pt)  
(11.5*25.71428:9pt) -- (11.5*25.71428:5pt)  
(12.5*25.71428:9pt) -- (12.5*25.71428:5pt)  
(13.5*25.71428:9pt) -- (13.5*25.71428:5pt)  ;
 }}}

\pgfdeclareplotmark{m1cvb}{%
\node[scale=\mOnescale*\AtoCscale*\symscale] at (0,0) {\tikz {%
\draw[fill=black,line width=.3pt,even odd rule,rotate=90] 
(0.*25.71428:9pt)  -- (0.5*25.71428:4.5pt)   -- 
(1.*25.71428:9pt)  -- (1.5*25.71428:4.5pt)   -- 
(2.*25.71428:9pt)  -- (2.5*25.71428:4.5pt)   -- 
(3.*25.71428:9pt)  -- (3.5*25.71428:4.5pt)   -- 
(4.*25.71428:9pt)  -- (4.5*25.71428:4.5pt)   -- 
(5.*25.71428:9pt)  -- (5.5*25.71428:4.5pt)   -- 
(6.*25.71428:9pt)  -- (6.5*25.71428:4.5pt)   -- 
(7.*25.71428:9pt)  -- (7.5*25.71428:4.5pt)   -- 
(8.*25.71428:9pt)  -- (8.5*25.71428:4.5pt)   -- 
(9.*25.71428:9pt)  -- (9.5*25.71428:4.5pt)   -- 
(10.*25.71428:9pt) -- (10.5*25.71428:4.5pt)  -- 
(11.*25.71428:9pt) -- (11.5*25.71428:4.5pt)  -- 
(12.*25.71428:9pt) -- (12.5*25.71428:4.5pt)  -- 
(13.*25.71428:9pt) -- (13.5*25.71428:4.5pt)  --  cycle
(0,0) circle (2pt);
\draw (0,0) circle (9pt);
%\draw[line width=.2pt,even odd rule,rotate=90] 
%(0.*25.71428:11pt)  -- (0.5*25.71428:6.5pt)   -- 
%(1.*25.71428:11pt)  -- (1.5*25.71428:6.5pt)   -- 
%(2.*25.71428:11pt)  -- (2.5*25.71428:6.5pt)   -- 
%(3.*25.71428:11pt)  -- (3.5*25.71428:6.5pt)   -- 
%(4.*25.71428:11pt)  -- (4.5*25.71428:6.5pt)   -- 
%(5.*25.71428:11pt)  -- (5.5*25.71428:6.5pt)   -- 
%(6.*25.71428:11pt)  -- (6.5*25.71428:6.5pt)   -- 
%(7.*25.71428:11pt)  -- (7.5*25.71428:6.5pt)   -- 
%(8.*25.71428:11pt)  -- (8.5*25.71428:6.5pt)   -- 
%(9.*25.71428:11pt)  -- (9.5*25.71428:6.5pt)   -- 
%(10.*25.71428:11pt) -- (10.5*25.71428:6.5pt)  -- 
%(11.*25.71428:11pt) -- (11.5*25.71428:6.5pt)  -- 
%(12.*25.71428:11pt) -- (12.5*25.71428:6.5pt)  -- 
%(13.*25.71428:11pt) -- (13.5*25.71428:6.5pt)  --  cycle;
\draw[line width=.4pt,even odd rule,rotate=90] 
(0.5*25.71428:10pt)  -- (0.5*25.71428:6.5pt)   
(1.5*25.71428:10pt)  -- (1.5*25.71428:6.5pt)   
(2.5*25.71428:10pt)  -- (2.5*25.71428:6.5pt)   
(3.5*25.71428:10pt)  -- (3.5*25.71428:6.5pt)   
(4.5*25.71428:10pt)  -- (4.5*25.71428:6.5pt)   
(5.5*25.71428:10pt)  -- (5.5*25.71428:6.5pt)   
(6.5*25.71428:10pt)  -- (6.5*25.71428:6.5pt)   
(7.5*25.71428:10pt)  -- (7.5*25.71428:6.5pt)   
(8.5*25.71428:10pt)  -- (8.5*25.71428:6.5pt)   
(9.5*25.71428:10pt)  -- (9.5*25.71428:6.5pt)   
(10.5*25.71428:10pt) -- (10.5*25.71428:6.5pt)  
(11.5*25.71428:10pt) -- (11.5*25.71428:6.5pt)  
(12.5*25.71428:10pt) -- (12.5*25.71428:6.5pt)  
(13.5*25.71428:10pt) -- (13.5*25.71428:6.5pt)  ;
 }}}



\pgfdeclareplotmark{m2a}{%
\node[scale=\mTwoscale*\symscale] at (0,0) {\tikz {%
\draw[fill=black,line width=.3pt,even odd rule,rotate=90] 
(0.*30.0:10pt)  -- (0.5*30.0:5pt)   -- 
(1.*30.0:10pt)  -- (1.5*30.0:5pt)   -- 
(2.*30.0:10pt)  -- (2.5*30.0:5pt)   -- 
(3.*30.0:10pt)  -- (3.5*30.0:5pt)   -- 
(4.*30.0:10pt)  -- (4.5*30.0:5pt)   -- 
(5.*30.0:10pt)  -- (5.5*30.0:5pt)   -- 
(6.*30.0:10pt)  -- (6.5*30.0:5pt)   -- 
(7.*30.0:10pt)  -- (7.5*30.0:5pt)   -- 
(8.*30.0:10pt)  -- (8.5*30.0:5pt)   -- 
(9.*30.0:10pt)  -- (9.5*30.0:5pt)   -- 
(10.*30.0:10pt) -- (10.5*30.0:5pt)  -- 
(11.*30.0:10pt) -- (11.5*30.0:5pt)  --  cycle;
 }}}


\pgfdeclareplotmark{m2av}{%
\node[scale=\mTwoscale*\symscale] at (0,0) {\tikz {%
\draw[fill=black,line width=.3pt,even odd rule,rotate=90] 
(0.*30.0:9pt)  -- (0.5*30.0:4.5pt)   -- 
(1.*30.0:9pt)  -- (1.5*30.0:4.5pt)   -- 
(2.*30.0:9pt)  -- (2.5*30.0:4.5pt)   -- 
(3.*30.0:9pt)  -- (3.5*30.0:4.5pt)   -- 
(4.*30.0:9pt)  -- (4.5*30.0:4.5pt)   -- 
(5.*30.0:9pt)  -- (5.5*30.0:4.5pt)   -- 
(6.*30.0:9pt)  -- (6.5*30.0:4.5pt)   -- 
(7.*30.0:9pt)  -- (7.5*30.0:4.5pt)   -- 
(8.*30.0:9pt)  -- (8.5*30.0:4.5pt)   -- 
(9.*30.0:9pt)  -- (9.5*30.0:4.5pt)   -- 
(10.*30.0:9pt) -- (10.5*30.0:4.5pt)  -- 
(11.*30.0:9pt) -- (11.5*30.0:4.5pt)  -- cycle;
\draw (0,0) circle (9pt);
%\draw[line width=.2pt,even odd rule,rotate=90] 
%(0.*30.0:11pt)  -- (0.5*30.0:6.5pt)   -- 
%(1.*30.0:11pt)  -- (1.5*30.0:6.5pt)   -- 
%(2.*30.0:11pt)  -- (2.5*30.0:6.5pt)   -- 
%(3.*30.0:11pt)  -- (3.5*30.0:6.5pt)   -- 
%(4.*30.0:11pt)  -- (4.5*30.0:6.5pt)   -- 
%(5.*30.0:11pt)  -- (5.5*30.0:6.5pt)   -- 
%(6.*30.0:11pt)  -- (6.5*30.0:6.5pt)   -- 
%(7.*30.0:11pt)  -- (7.5*30.0:6.5pt)   -- 
%(8.*30.0:11pt)  -- (8.5*30.0:6.5pt)   -- 
%(9.*30.0:11pt)  -- (9.5*30.0:6.5pt)   -- 
%(10.*30.0:11pt) -- (10.5*30.0:6.5pt)  -- 
%(11.*30.0:11pt) -- (11.5*30.0:6.5pt)  --  cycle;
 }}}



\pgfdeclareplotmark{m2ab}{%
\node[scale=\mTwoscale*\symscale] at (0,0) {\tikz {%
\draw[fill=black,line width=.3pt,even odd rule,rotate=90] 
(0.*30.0:10pt)  -- (0.5*30.0:5pt)   -- 
(1.*30.0:10pt)  -- (1.5*30.0:5pt)   -- 
(2.*30.0:10pt)  -- (2.5*30.0:5pt)   -- 
(3.*30.0:10pt)  -- (3.5*30.0:5pt)   -- 
(4.*30.0:10pt)  -- (4.5*30.0:5pt)   -- 
(5.*30.0:10pt)  -- (5.5*30.0:5pt)   -- 
(6.*30.0:10pt)  -- (6.5*30.0:5pt)   -- 
(7.*30.0:10pt)  -- (7.5*30.0:5pt)   -- 
(8.*30.0:10pt)  -- (8.5*30.0:5pt)   -- 
(9.*30.0:10pt)  -- (9.5*30.0:5pt)   -- 
(10.*30.0:10pt) -- (10.5*30.0:5pt)  -- 
(11.*30.0:10pt) -- (11.5*30.0:5pt)  --  cycle;
\draw[line width=.4pt,even odd rule,rotate=90] 
(0.5*30.0:9pt)  -- (0.5*30.0:5pt)   
(1.5*30.0:9pt)  -- (1.5*30.0:5pt)   
(2.5*30.0:9pt)  -- (2.5*30.0:5pt)   
(3.5*30.0:9pt)  -- (3.5*30.0:5pt)   
(4.5*30.0:9pt)  -- (4.5*30.0:5pt)   
(5.5*30.0:9pt)  -- (5.5*30.0:5pt)   
(6.5*30.0:9pt)  -- (6.5*30.0:5pt)   
(7.5*30.0:9pt)  -- (7.5*30.0:5pt)   
(8.5*30.0:9pt)  -- (8.5*30.0:5pt)   
(9.5*30.0:9pt)  -- (9.5*30.0:5pt)   
(10.5*30.0:9pt) -- (10.5*30.0:5pt)  
(11.5*30.0:9pt) -- (11.5*30.0:5pt) ;
 }}}

\pgfdeclareplotmark{m2avb}{%
\node[scale=\mTwoscale*\symscale] at (0,0) {\tikz {%
\draw[fill=black,line width=.3pt,even odd rule,rotate=90] 
(0.*30.0:9pt)  -- (0.5*30.0:4.5pt)   -- 
(1.*30.0:9pt)  -- (1.5*30.0:4.5pt)   -- 
(2.*30.0:9pt)  -- (2.5*30.0:4.5pt)   -- 
(3.*30.0:9pt)  -- (3.5*30.0:4.5pt)   -- 
(4.*30.0:9pt)  -- (4.5*30.0:4.5pt)   -- 
(5.*30.0:9pt)  -- (5.5*30.0:4.5pt)   -- 
(6.*30.0:9pt)  -- (6.5*30.0:4.5pt)   -- 
(7.*30.0:9pt)  -- (7.5*30.0:4.5pt)   -- 
(8.*30.0:9pt)  -- (8.5*30.0:4.5pt)   -- 
(9.*30.0:9pt)  -- (9.5*30.0:4.5pt)   -- 
(10.*30.0:9pt) -- (10.5*30.0:4.5pt)  -- 
(11.*30.0:9pt) -- (11.5*30.0:4.5pt)  --  cycle;
\draw (0,0) circle (9pt);
%\draw[line width=.2pt,even odd rule,rotate=90] 
%(0.*30.0:11pt)  -- (0.5*30.0:6.5pt)   -- 
%(1.*30.0:11pt)  -- (1.5*30.0:6.5pt)   -- 
%(2.*30.0:11pt)  -- (2.5*30.0:6.5pt)   -- 
%(3.*30.0:11pt)  -- (3.5*30.0:6.5pt)   -- 
%(4.*30.0:11pt)  -- (4.5*30.0:6.5pt)   -- 
%(5.*30.0:11pt)  -- (5.5*30.0:6.5pt)   -- 
%(6.*30.0:11pt)  -- (6.5*30.0:6.5pt)   -- 
%(7.*30.0:11pt)  -- (7.5*30.0:6.5pt)   -- 
%(8.*30.0:11pt)  -- (8.5*30.0:6.5pt)   -- 
%(9.*30.0:11pt)  -- (9.5*30.0:6.5pt)   -- 
%(10.*30.0:11pt) -- (10.5*30.0:6.5pt)  -- 
%(11.*30.0:11pt) -- (11.5*30.0:6.5pt)  -- cycle;
\draw[line width=.4pt,even odd rule,rotate=90] 
(0.5*30.0:10pt)  -- (0.5*30.0:6.5pt)   
(1.5*30.0:10pt)  -- (1.5*30.0:6.5pt)   
(2.5*30.0:10pt)  -- (2.5*30.0:6.5pt)   
(3.5*30.0:10pt)  -- (3.5*30.0:6.5pt)   
(4.5*30.0:10pt)  -- (4.5*30.0:6.5pt)   
(5.5*30.0:10pt)  -- (5.5*30.0:6.5pt)   
(6.5*30.0:10pt)  -- (6.5*30.0:6.5pt)   
(7.5*30.0:10pt)  -- (7.5*30.0:6.5pt)   
(8.5*30.0:10pt)  -- (8.5*30.0:6.5pt)   
(9.5*30.0:10pt)  -- (9.5*30.0:6.5pt)   
(10.5*30.0:10pt) -- (10.5*30.0:6.5pt)  
(11.5*30.0:10pt) -- (11.5*30.0:6.5pt)  ;
 }}}



\pgfdeclareplotmark{m2b}{%
\node[scale=\mTwoscale*\AtoBscale*\symscale] at (0,0) {\tikz {%
\draw[fill=black,line width=.3pt,even odd rule,rotate=90] 
(0.*30.0:10pt)  -- (0.5*30.0:5pt)   -- 
(1.*30.0:10pt)  -- (1.5*30.0:5pt)   -- 
(2.*30.0:10pt)  -- (2.5*30.0:5pt)   -- 
(3.*30.0:10pt)  -- (3.5*30.0:5pt)   -- 
(4.*30.0:10pt)  -- (4.5*30.0:5pt)   -- 
(5.*30.0:10pt)  -- (5.5*30.0:5pt)   -- 
(6.*30.0:10pt)  -- (6.5*30.0:5pt)   -- 
(7.*30.0:10pt)  -- (7.5*30.0:5pt)   -- 
(8.*30.0:10pt)  -- (8.5*30.0:5pt)   -- 
(9.*30.0:10pt)  -- (9.5*30.0:5pt)   -- 
(10.*30.0:10pt) -- (10.5*30.0:5pt)  -- 
(11.*30.0:10pt) -- (11.5*30.0:5pt)  -- cycle
(0,0) circle (2pt);
 }}}


\pgfdeclareplotmark{m2bv}{%
\node[scale=\mTwoscale*\AtoBscale*\symscale] at (0,0) {\tikz {%
\draw[fill=black,line width=.3pt,even odd rule,rotate=90] 
(0.*30.0:9pt)  -- (0.5*30.0:4.5pt)   -- 
(1.*30.0:9pt)  -- (1.5*30.0:4.5pt)   -- 
(2.*30.0:9pt)  -- (2.5*30.0:4.5pt)   -- 
(3.*30.0:9pt)  -- (3.5*30.0:4.5pt)   -- 
(4.*30.0:9pt)  -- (4.5*30.0:4.5pt)   -- 
(5.*30.0:9pt)  -- (5.5*30.0:4.5pt)   -- 
(6.*30.0:9pt)  -- (6.5*30.0:4.5pt)   -- 
(7.*30.0:9pt)  -- (7.5*30.0:4.5pt)   -- 
(8.*30.0:9pt)  -- (8.5*30.0:4.5pt)   -- 
(9.*30.0:9pt)  -- (9.5*30.0:4.5pt)   -- 
(10.*30.0:9pt) -- (10.5*30.0:4.5pt)  -- 
(11.*30.0:9pt) -- (11.5*30.0:4.5pt)  --  cycle
(0,0) circle (2pt);
\draw (0,0) circle (9pt);
%\draw[line width=.2pt,even odd rule,rotate=90] 
%(0.*30.0:11pt)  -- (0.5*30.0:6.5pt)   -- 
%(1.*30.0:11pt)  -- (1.5*30.0:6.5pt)   -- 
%(2.*30.0:11pt)  -- (2.5*30.0:6.5pt)   -- 
%(3.*30.0:11pt)  -- (3.5*30.0:6.5pt)   -- 
%(4.*30.0:11pt)  -- (4.5*30.0:6.5pt)   -- 
%(5.*30.0:11pt)  -- (5.5*30.0:6.5pt)   -- 
%(6.*30.0:11pt)  -- (6.5*30.0:6.5pt)   -- 
%(7.*30.0:11pt)  -- (7.5*30.0:6.5pt)   -- 
%(8.*30.0:11pt)  -- (8.5*30.0:6.5pt)   -- 
%(9.*30.0:11pt)  -- (9.5*30.0:6.5pt)   -- 
%(10.*30.0:11pt) -- (10.5*30.0:6.5pt)  -- 
%(11.*30.0:11pt) -- (11.5*30.0:6.5pt)  --  cycle;
 }}}



\pgfdeclareplotmark{m2bb}{%
\node[scale=\mTwoscale*\AtoBscale*\symscale] at (0,0) {\tikz {%
\draw[fill=black,line width=.3pt,even odd rule,rotate=90] 
(0.*30.0:10pt)  -- (0.5*30.0:5pt)   -- 
(1.*30.0:10pt)  -- (1.5*30.0:5pt)   -- 
(2.*30.0:10pt)  -- (2.5*30.0:5pt)   -- 
(3.*30.0:10pt)  -- (3.5*30.0:5pt)   -- 
(4.*30.0:10pt)  -- (4.5*30.0:5pt)   -- 
(5.*30.0:10pt)  -- (5.5*30.0:5pt)   -- 
(6.*30.0:10pt)  -- (6.5*30.0:5pt)   -- 
(7.*30.0:10pt)  -- (7.5*30.0:5pt)   -- 
(8.*30.0:10pt)  -- (8.5*30.0:5pt)   -- 
(9.*30.0:10pt)  -- (9.5*30.0:5pt)   -- 
(10.*30.0:10pt) -- (10.5*30.0:5pt)  -- 
(11.*30.0:10pt) -- (11.5*30.0:5pt)  --  cycle
(0,0) circle (2pt);
\draw[line width=.4pt,even odd rule,rotate=90] 
(0.5*30.0:9pt)  -- (0.5*30.0:5pt)   
(1.5*30.0:9pt)  -- (1.5*30.0:5pt)   
(2.5*30.0:9pt)  -- (2.5*30.0:5pt)   
(3.5*30.0:9pt)  -- (3.5*30.0:5pt)   
(4.5*30.0:9pt)  -- (4.5*30.0:5pt)   
(5.5*30.0:9pt)  -- (5.5*30.0:5pt)   
(6.5*30.0:9pt)  -- (6.5*30.0:5pt)   
(7.5*30.0:9pt)  -- (7.5*30.0:5pt)   
(8.5*30.0:9pt)  -- (8.5*30.0:5pt)   
(9.5*30.0:9pt)  -- (9.5*30.0:5pt)   
(10.5*30.0:9pt) -- (10.5*30.0:5pt)  
(11.5*30.0:9pt) -- (11.5*30.0:5pt)  ;
 }}}

\pgfdeclareplotmark{m2bvb}{%
\node[scale=\mTwoscale*\AtoBscale*\symscale] at (0,0) {\tikz {%
\draw[fill=black,line width=.3pt,even odd rule,rotate=90] 
(0.*30.0:9pt)  -- (0.5*30.0:4.5pt)   -- 
(1.*30.0:9pt)  -- (1.5*30.0:4.5pt)   -- 
(2.*30.0:9pt)  -- (2.5*30.0:4.5pt)   -- 
(3.*30.0:9pt)  -- (3.5*30.0:4.5pt)   -- 
(4.*30.0:9pt)  -- (4.5*30.0:4.5pt)   -- 
(5.*30.0:9pt)  -- (5.5*30.0:4.5pt)   -- 
(6.*30.0:9pt)  -- (6.5*30.0:4.5pt)   -- 
(7.*30.0:9pt)  -- (7.5*30.0:4.5pt)   -- 
(8.*30.0:9pt)  -- (8.5*30.0:4.5pt)   -- 
(9.*30.0:9pt)  -- (9.5*30.0:4.5pt)   -- 
(10.*30.0:9pt) -- (10.5*30.0:4.5pt)  -- 
(11.*30.0:9pt) -- (11.5*30.0:4.5pt)  --  cycle
(0,0) circle (2pt);
\draw (0,0) circle (9pt);
%\draw[line width=.2pt,even odd rule,rotate=90] 
%(0.*30.0:11pt)  -- (0.5*30.0:6.5pt)   -- 
%(1.*30.0:11pt)  -- (1.5*30.0:6.5pt)   -- 
%(2.*30.0:11pt)  -- (2.5*30.0:6.5pt)   -- 
%(3.*30.0:11pt)  -- (3.5*30.0:6.5pt)   -- 
%(4.*30.0:11pt)  -- (4.5*30.0:6.5pt)   -- 
%(5.*30.0:11pt)  -- (5.5*30.0:6.5pt)   -- 
%(6.*30.0:11pt)  -- (6.5*30.0:6.5pt)   -- 
%(7.*30.0:11pt)  -- (7.5*30.0:6.5pt)   -- 
%(8.*30.0:11pt)  -- (8.5*30.0:6.5pt)   -- 
%(9.*30.0:11pt)  -- (9.5*30.0:6.5pt)   -- 
%(10.*30.0:11pt) -- (10.5*30.0:6.5pt)  -- 
%(11.*30.0:11pt) -- (11.5*30.0:6.5pt)  --  cycle;
\draw[line width=.4pt,even odd rule,rotate=90] 
(0.5*30.0:10pt)  -- (0.5*30.0:6.5pt)   
(1.5*30.0:10pt)  -- (1.5*30.0:6.5pt)   
(2.5*30.0:10pt)  -- (2.5*30.0:6.5pt)   
(3.5*30.0:10pt)  -- (3.5*30.0:6.5pt)   
(4.5*30.0:10pt)  -- (4.5*30.0:6.5pt)   
(5.5*30.0:10pt)  -- (5.5*30.0:6.5pt)   
(6.5*30.0:10pt)  -- (6.5*30.0:6.5pt)   
(7.5*30.0:10pt)  -- (7.5*30.0:6.5pt)   
(8.5*30.0:10pt)  -- (8.5*30.0:6.5pt)   
(9.5*30.0:10pt)  -- (9.5*30.0:6.5pt)   
(10.5*30.0:10pt) -- (10.5*30.0:6.5pt)  
(11.5*30.0:10pt) -- (11.5*30.0:6.5pt)  ;
 }}}



\pgfdeclareplotmark{m2c}{%
\node[scale=\mTwoscale*\AtoCscale*\symscale] at (0,0) {\tikz {%
\draw[fill=black,line width=.3pt,even odd rule,rotate=90] 
(0.*30.0:10pt)  -- (0.5*30.0:5pt)   -- 
(1.*30.0:10pt)  -- (1.5*30.0:5pt)   -- 
(2.*30.0:10pt)  -- (2.5*30.0:5pt)   -- 
(3.*30.0:10pt)  -- (3.5*30.0:5pt)   -- 
(4.*30.0:10pt)  -- (4.5*30.0:5pt)   -- 
(5.*30.0:10pt)  -- (5.5*30.0:5pt)   -- 
(6.*30.0:10pt)  -- (6.5*30.0:5pt)   -- 
(7.*30.0:10pt)  -- (7.5*30.0:5pt)   -- 
(8.*30.0:10pt)  -- (8.5*30.0:5pt)   -- 
(9.*30.0:10pt)  -- (9.5*30.0:5pt)   -- 
(10.*30.0:10pt) -- (10.5*30.0:5pt)  -- 
(11.*30.0:10pt) -- (11.5*30.0:5pt)  --cycle
(0,0) circle (3pt);
 }}}


\pgfdeclareplotmark{m2cv}{%
\node[scale=\mTwoscale*\AtoCscale*\symscale] at (0,0) {\tikz {%
\draw[fill=black,line width=.3pt,even odd rule,rotate=90] 
(0.*30.0:9pt)  -- (0.5*30.0:4.5pt)   -- 
(1.*30.0:9pt)  -- (1.5*30.0:4.5pt)   -- 
(2.*30.0:9pt)  -- (2.5*30.0:4.5pt)   -- 
(3.*30.0:9pt)  -- (3.5*30.0:4.5pt)   -- 
(4.*30.0:9pt)  -- (4.5*30.0:4.5pt)   -- 
(5.*30.0:9pt)  -- (5.5*30.0:4.5pt)   -- 
(6.*30.0:9pt)  -- (6.5*30.0:4.5pt)   -- 
(7.*30.0:9pt)  -- (7.5*30.0:4.5pt)   -- 
(8.*30.0:9pt)  -- (8.5*30.0:4.5pt)   -- 
(9.*30.0:9pt)  -- (9.5*30.0:4.5pt)   -- 
(10.*30.0:9pt) -- (10.5*30.0:4.5pt)  -- 
(11.*30.0:9pt) -- (11.5*30.0:4.5pt)  --  cycle
(0,0) circle (3pt);
\draw (0,0) circle (9pt);
%\draw[line width=.2pt,even odd rule,rotate=90] 
%(0.*30.0:11pt)  -- (0.5*30.0:6.5pt)   -- 
%(1.*30.0:11pt)  -- (1.5*30.0:6.5pt)   -- 
%(2.*30.0:11pt)  -- (2.5*30.0:6.5pt)   -- 
%(3.*30.0:11pt)  -- (3.5*30.0:6.5pt)   -- 
%(4.*30.0:11pt)  -- (4.5*30.0:6.5pt)   -- 
%(5.*30.0:11pt)  -- (5.5*30.0:6.5pt)   -- 
%(6.*30.0:11pt)  -- (6.5*30.0:6.5pt)   -- 
%(7.*30.0:11pt)  -- (7.5*30.0:6.5pt)   -- 
%(8.*30.0:11pt)  -- (8.5*30.0:6.5pt)   -- 
%(9.*30.0:11pt)  -- (9.5*30.0:6.5pt)   -- 
%(10.*30.0:11pt) -- (10.5*30.0:6.5pt)  -- 
%(11.*30.0:11pt) -- (11.5*30.0:6.5pt)  --  cycle;
 }}}



\pgfdeclareplotmark{m2cb}{%
\node[scale=\mTwoscale*\AtoCscale*\symscale] at (0,0) {\tikz {%
\draw[fill=black,line width=.3pt,even odd rule,rotate=90] 
(0.*30.0:10pt)  -- (0.5*30.0:5pt)   -- 
(1.*30.0:10pt)  -- (1.5*30.0:5pt)   -- 
(2.*30.0:10pt)  -- (2.5*30.0:5pt)   -- 
(3.*30.0:10pt)  -- (3.5*30.0:5pt)   -- 
(4.*30.0:10pt)  -- (4.5*30.0:5pt)   -- 
(5.*30.0:10pt)  -- (5.5*30.0:5pt)   -- 
(6.*30.0:10pt)  -- (6.5*30.0:5pt)   -- 
(7.*30.0:10pt)  -- (7.5*30.0:5pt)   -- 
(8.*30.0:10pt)  -- (8.5*30.0:5pt)   -- 
(9.*30.0:10pt)  -- (9.5*30.0:5pt)   -- 
(10.*30.0:10pt) -- (10.5*30.0:5pt)  -- 
(11.*30.0:10pt) -- (11.5*30.0:5pt)  --  cycle
(0,0) circle (3pt);
\draw[line width=.4pt,even odd rule,rotate=90] 
(0.5*30.0:9pt)  -- (0.5*30.0:5pt)   
(1.5*30.0:9pt)  -- (1.5*30.0:5pt)   
(2.5*30.0:9pt)  -- (2.5*30.0:5pt)   
(3.5*30.0:9pt)  -- (3.5*30.0:5pt)   
(4.5*30.0:9pt)  -- (4.5*30.0:5pt)   
(5.5*30.0:9pt)  -- (5.5*30.0:5pt)   
(6.5*30.0:9pt)  -- (6.5*30.0:5pt)   
(7.5*30.0:9pt)  -- (7.5*30.0:5pt)   
(8.5*30.0:9pt)  -- (8.5*30.0:5pt)   
(9.5*30.0:9pt)  -- (9.5*30.0:5pt)   
(10.5*30.0:9pt) -- (10.5*30.0:5pt)  
(11.5*30.0:9pt) -- (11.5*30.0:5pt)  ;
 }}}

\pgfdeclareplotmark{m2cvb}{%
\node[scale=\mTwoscale*\AtoCscale*\symscale] at (0,0) {\tikz {%
\draw[fill=black,line width=.3pt,even odd rule,rotate=90] 
(0.*30.0:9pt)  -- (0.5*30.0:4.5pt)   -- 
(1.*30.0:9pt)  -- (1.5*30.0:4.5pt)   -- 
(2.*30.0:9pt)  -- (2.5*30.0:4.5pt)   -- 
(3.*30.0:9pt)  -- (3.5*30.0:4.5pt)   -- 
(4.*30.0:9pt)  -- (4.5*30.0:4.5pt)   -- 
(5.*30.0:9pt)  -- (5.5*30.0:4.5pt)   -- 
(6.*30.0:9pt)  -- (6.5*30.0:4.5pt)   -- 
(7.*30.0:9pt)  -- (7.5*30.0:4.5pt)   -- 
(8.*30.0:9pt)  -- (8.5*30.0:4.5pt)   -- 
(9.*30.0:9pt)  -- (9.5*30.0:4.5pt)   -- 
(10.*30.0:9pt) -- (10.5*30.0:4.5pt)  -- 
(11.*30.0:9pt) -- (11.5*30.0:4.5pt)  --  cycle
(0,0) circle (3pt);
\draw (0,0) circle (9pt);
%\draw[line width=.2pt,even odd rule,rotate=90] 
%(0.*30.0:11pt)  -- (0.5*30.0:6.5pt)   -- 
%(1.*30.0:11pt)  -- (1.5*30.0:6.5pt)   -- 
%(2.*30.0:11pt)  -- (2.5*30.0:6.5pt)   -- 
%(3.*30.0:11pt)  -- (3.5*30.0:6.5pt)   -- 
%(4.*30.0:11pt)  -- (4.5*30.0:6.5pt)   -- 
%(5.*30.0:11pt)  -- (5.5*30.0:6.5pt)   -- 
%(6.*30.0:11pt)  -- (6.5*30.0:6.5pt)   -- 
%(7.*30.0:11pt)  -- (7.5*30.0:6.5pt)   -- 
%(8.*30.0:11pt)  -- (8.5*30.0:6.5pt)   -- 
%(9.*30.0:11pt)  -- (9.5*30.0:6.5pt)   -- 
%(10.*30.0:11pt) -- (10.5*30.0:6.5pt)  -- 
%(11.*30.0:11pt) -- (11.5*30.0:6.5pt)  --  cycle;
\draw[line width=.4pt,even odd rule,rotate=90] 
(0.5*30.0:10pt)  -- (0.5*30.0:6.5pt)   
(1.5*30.0:10pt)  -- (1.5*30.0:6.5pt)   
(2.5*30.0:10pt)  -- (2.5*30.0:6.5pt)   
(3.5*30.0:10pt)  -- (3.5*30.0:6.5pt)   
(4.5*30.0:10pt)  -- (4.5*30.0:6.5pt)   
(5.5*30.0:10pt)  -- (5.5*30.0:6.5pt)   
(6.5*30.0:10pt)  -- (6.5*30.0:6.5pt)   
(7.5*30.0:10pt)  -- (7.5*30.0:6.5pt)   
(8.5*30.0:10pt)  -- (8.5*30.0:6.5pt)   
(9.5*30.0:10pt)  -- (9.5*30.0:6.5pt)   
(10.5*30.0:10pt) -- (10.5*30.0:6.5pt)  
(11.5*30.0:10pt) -- (11.5*30.0:6.5pt)  ;
 }}}



\pgfdeclareplotmark{m3a}{%
\node[scale=\mThreescale*\symscale] at (0,0) {\tikz {%
\draw[fill=black,line width=.3pt,even odd rule,rotate=90] 
(0.*36.0:10pt)  -- (0.5*36.0:5pt)   -- 
(1.*36.0:10pt)  -- (1.5*36.0:5pt)   -- 
(2.*36.0:10pt)  -- (2.5*36.0:5pt)   -- 
(3.*36.0:10pt)  -- (3.5*36.0:5pt)   -- 
(4.*36.0:10pt)  -- (4.5*36.0:5pt)   -- 
(5.*36.0:10pt)  -- (5.5*36.0:5pt)   -- 
(6.*36.0:10pt)  -- (6.5*36.0:5pt)   -- 
(7.*36.0:10pt)  -- (7.5*36.0:5pt)   -- 
(8.*36.0:10pt)  -- (8.5*36.0:5pt)   -- 
(9.*36.0:10pt)  -- (9.5*36.0:5pt)   --  cycle;
 }}}


\pgfdeclareplotmark{m3av}{%
\node[scale=\mThreescale*\symscale] at (0,0) {\tikz {%
\draw[fill=black,line width=.3pt,even odd rule,rotate=90] 
(0.*36.0:9pt)  -- (0.5*36.0:4.5pt)   -- 
(1.*36.0:9pt)  -- (1.5*36.0:4.5pt)   -- 
(2.*36.0:9pt)  -- (2.5*36.0:4.5pt)   -- 
(3.*36.0:9pt)  -- (3.5*36.0:4.5pt)   -- 
(4.*36.0:9pt)  -- (4.5*36.0:4.5pt)   -- 
(5.*36.0:9pt)  -- (5.5*36.0:4.5pt)   -- 
(6.*36.0:9pt)  -- (6.5*36.0:4.5pt)   -- 
(7.*36.0:9pt)  -- (7.5*36.0:4.5pt)   -- 
(8.*36.0:9pt)  -- (8.5*36.0:4.5pt)   -- 
(9.*36.0:9pt)  -- (9.5*36.0:4.5pt)   -- cycle;
\draw (0,0) circle (9pt);
%\draw[line width=.2pt,even odd rule,rotate=90] 
%(0.*36.0:11pt)  -- (0.5*36.0:6.5pt)   -- 
%(1.*36.0:11pt)  -- (1.5*36.0:6.5pt)   -- 
%(2.*36.0:11pt)  -- (2.5*36.0:6.5pt)   -- 
%(3.*36.0:11pt)  -- (3.5*36.0:6.5pt)   -- 
%(4.*36.0:11pt)  -- (4.5*36.0:6.5pt)   -- 
%(5.*36.0:11pt)  -- (5.5*36.0:6.5pt)   -- 
%(6.*36.0:11pt)  -- (6.5*36.0:6.5pt)   -- 
%(7.*36.0:11pt)  -- (7.5*36.0:6.5pt)   -- 
%(8.*36.0:11pt)  -- (8.5*36.0:6.5pt)   -- 
%(9.*36.0:11pt)  -- (9.5*36.0:6.5pt)   --  cycle;
 }}}



\pgfdeclareplotmark{m3ab}{%
\node[scale=\mThreescale*\symscale] at (0,0) {\tikz {%
\draw[fill=black,line width=.3pt,even odd rule,rotate=90] 
(0.*36.0:10pt)  -- (0.5*36.0:5pt)   -- 
(1.*36.0:10pt)  -- (1.5*36.0:5pt)   -- 
(2.*36.0:10pt)  -- (2.5*36.0:5pt)   -- 
(3.*36.0:10pt)  -- (3.5*36.0:5pt)   -- 
(4.*36.0:10pt)  -- (4.5*36.0:5pt)   -- 
(5.*36.0:10pt)  -- (5.5*36.0:5pt)   -- 
(6.*36.0:10pt)  -- (6.5*36.0:5pt)   -- 
(7.*36.0:10pt)  -- (7.5*36.0:5pt)   -- 
(8.*36.0:10pt)  -- (8.5*36.0:5pt)   -- 
(9.*36.0:10pt)  -- (9.5*36.0:5pt)   --  cycle;
\draw[line width=.4pt,even odd rule,rotate=90] 
(0.5*36.0:9pt)  -- (0.5*36.0:5pt)   
(1.5*36.0:9pt)  -- (1.5*36.0:5pt)   
(2.5*36.0:9pt)  -- (2.5*36.0:5pt)   
(3.5*36.0:9pt)  -- (3.5*36.0:5pt)   
(4.5*36.0:9pt)  -- (4.5*36.0:5pt)   
(5.5*36.0:9pt)  -- (5.5*36.0:5pt)   
(6.5*36.0:9pt)  -- (6.5*36.0:5pt)   
(7.5*36.0:9pt)  -- (7.5*36.0:5pt)   
(8.5*36.0:9pt)  -- (8.5*36.0:5pt)   
(9.5*36.0:9pt)  -- (9.5*36.0:5pt)  ;
 }}}

\pgfdeclareplotmark{m3avb}{%
\node[scale=\mThreescale*\symscale] at (0,0) {\tikz {%
\draw[fill=black,line width=.3pt,even odd rule,rotate=90] 
(0.*36.0:9pt)  -- (0.5*36.0:4.5pt)   -- 
(1.*36.0:9pt)  -- (1.5*36.0:4.5pt)   -- 
(2.*36.0:9pt)  -- (2.5*36.0:4.5pt)   -- 
(3.*36.0:9pt)  -- (3.5*36.0:4.5pt)   -- 
(4.*36.0:9pt)  -- (4.5*36.0:4.5pt)   -- 
(5.*36.0:9pt)  -- (5.5*36.0:4.5pt)   -- 
(6.*36.0:9pt)  -- (6.5*36.0:4.5pt)   -- 
(7.*36.0:9pt)  -- (7.5*36.0:4.5pt)   -- 
(8.*36.0:9pt)  -- (8.5*36.0:4.5pt)   -- 
(9.*36.0:9pt)  -- (9.5*36.0:4.5pt)   --  cycle;
\draw (0,0) circle (9pt);
%\draw[line width=.2pt,even odd rule,rotate=90] 
%(0.*36.0:11pt)  -- (0.5*36.0:6.5pt)   -- 
%(1.*36.0:11pt)  -- (1.5*36.0:6.5pt)   -- 
%(2.*36.0:11pt)  -- (2.5*36.0:6.5pt)   -- 
%(3.*36.0:11pt)  -- (3.5*36.0:6.5pt)   -- 
%(4.*36.0:11pt)  -- (4.5*36.0:6.5pt)   -- 
%(5.*36.0:11pt)  -- (5.5*36.0:6.5pt)   -- 
%(6.*36.0:11pt)  -- (6.5*36.0:6.5pt)   -- 
%(7.*36.0:11pt)  -- (7.5*36.0:6.5pt)   -- 
%(8.*36.0:11pt)  -- (8.5*36.0:6.5pt)   -- 
%(9.*36.0:11pt)  -- (9.5*36.0:6.5pt)   --  cycle;
\draw[line width=.4pt,even odd rule,rotate=90] 
(0.5*36.0:10pt)  -- (0.5*36.0:6.5pt)   
(1.5*36.0:10pt)  -- (1.5*36.0:6.5pt)   
(2.5*36.0:10pt)  -- (2.5*36.0:6.5pt)   
(3.5*36.0:10pt)  -- (3.5*36.0:6.5pt)   
(4.5*36.0:10pt)  -- (4.5*36.0:6.5pt)   
(5.5*36.0:10pt)  -- (5.5*36.0:6.5pt)   
(6.5*36.0:10pt)  -- (6.5*36.0:6.5pt)   
(7.5*36.0:10pt)  -- (7.5*36.0:6.5pt)   
(8.5*36.0:10pt)  -- (8.5*36.0:6.5pt)   
(9.5*36.0:10pt)  -- (9.5*36.0:6.5pt)   ;
 }}}



\pgfdeclareplotmark{m3b}{%
\node[scale=\mThreescale*\AtoBscale*\symscale] at (0,0) {\tikz {%
\draw[fill=black,line width=.3pt,even odd rule,rotate=90] 
(0.*36.0:10pt)  -- (0.5*36.0:5pt)   -- 
(1.*36.0:10pt)  -- (1.5*36.0:5pt)   -- 
(2.*36.0:10pt)  -- (2.5*36.0:5pt)   -- 
(3.*36.0:10pt)  -- (3.5*36.0:5pt)   -- 
(4.*36.0:10pt)  -- (4.5*36.0:5pt)   -- 
(5.*36.0:10pt)  -- (5.5*36.0:5pt)   -- 
(6.*36.0:10pt)  -- (6.5*36.0:5pt)   -- 
(7.*36.0:10pt)  -- (7.5*36.0:5pt)   -- 
(8.*36.0:10pt)  -- (8.5*36.0:5pt)   -- 
(9.*36.0:10pt)  -- (9.5*36.0:5pt)   -- cycle
(0,0) circle (2pt);
 }}}


\pgfdeclareplotmark{m3bv}{%
\node[scale=\mThreescale*\AtoBscale*\symscale] at (0,0) {\tikz {%
\draw[fill=black,line width=.3pt,even odd rule,rotate=90] 
(0.*36.0:9pt)  -- (0.5*36.0:4.5pt)   -- 
(1.*36.0:9pt)  -- (1.5*36.0:4.5pt)   -- 
(2.*36.0:9pt)  -- (2.5*36.0:4.5pt)   -- 
(3.*36.0:9pt)  -- (3.5*36.0:4.5pt)   -- 
(4.*36.0:9pt)  -- (4.5*36.0:4.5pt)   -- 
(5.*36.0:9pt)  -- (5.5*36.0:4.5pt)   -- 
(6.*36.0:9pt)  -- (6.5*36.0:4.5pt)   -- 
(7.*36.0:9pt)  -- (7.5*36.0:4.5pt)   -- 
(8.*36.0:9pt)  -- (8.5*36.0:4.5pt)   -- 
(9.*36.0:9pt)  -- (9.5*36.0:4.5pt)   -- cycle
(0,0) circle (2pt);
\draw (0,0) circle (9pt);
%\draw[line width=.2pt,even odd rule,rotate=90] 
%(0.*36.0:11pt)  -- (0.5*36.0:6.5pt)   -- 
%(1.*36.0:11pt)  -- (1.5*36.0:6.5pt)   -- 
%(2.*36.0:11pt)  -- (2.5*36.0:6.5pt)   -- 
%(3.*36.0:11pt)  -- (3.5*36.0:6.5pt)   -- 
%(4.*36.0:11pt)  -- (4.5*36.0:6.5pt)   -- 
%(5.*36.0:11pt)  -- (5.5*36.0:6.5pt)   -- 
%(6.*36.0:11pt)  -- (6.5*36.0:6.5pt)   -- 
%(7.*36.0:11pt)  -- (7.5*36.0:6.5pt)   -- 
%(8.*36.0:11pt)  -- (8.5*36.0:6.5pt)   -- 
%(9.*36.0:11pt)  -- (9.5*36.0:6.5pt)   -- cycle;
 }}}



\pgfdeclareplotmark{m3bb}{%
\node[scale=\mThreescale*\AtoBscale*\symscale] at (0,0) {\tikz {%
\draw[fill=black,line width=.3pt,even odd rule,rotate=90] 
(0.*36.0:10pt)  -- (0.5*36.0:5pt)   -- 
(1.*36.0:10pt)  -- (1.5*36.0:5pt)   -- 
(2.*36.0:10pt)  -- (2.5*36.0:5pt)   -- 
(3.*36.0:10pt)  -- (3.5*36.0:5pt)   -- 
(4.*36.0:10pt)  -- (4.5*36.0:5pt)   -- 
(5.*36.0:10pt)  -- (5.5*36.0:5pt)   -- 
(6.*36.0:10pt)  -- (6.5*36.0:5pt)   -- 
(7.*36.0:10pt)  -- (7.5*36.0:5pt)   -- 
(8.*36.0:10pt)  -- (8.5*36.0:5pt)   -- 
(9.*36.0:10pt)  -- (9.5*36.0:5pt)   --  cycle
(0,0) circle (2pt);
\draw[line width=.4pt,even odd rule,rotate=90] 
(0.5*36.0:9pt)  -- (0.5*36.0:5pt)   
(1.5*36.0:9pt)  -- (1.5*36.0:5pt)   
(2.5*36.0:9pt)  -- (2.5*36.0:5pt)   
(3.5*36.0:9pt)  -- (3.5*36.0:5pt)   
(4.5*36.0:9pt)  -- (4.5*36.0:5pt)   
(5.5*36.0:9pt)  -- (5.5*36.0:5pt)   
(6.5*36.0:9pt)  -- (6.5*36.0:5pt)   
(7.5*36.0:9pt)  -- (7.5*36.0:5pt)   
(8.5*36.0:9pt)  -- (8.5*36.0:5pt)   
(9.5*36.0:9pt)  -- (9.5*36.0:5pt)  ;
 }}}

\pgfdeclareplotmark{m3bvb}{%
\node[scale=\mThreescale*\AtoBscale*\symscale] at (0,0) {\tikz {%
\draw[fill=black,line width=.3pt,even odd rule,rotate=90] 
(0.*36.0:9pt)  -- (0.5*36.0:4.5pt)   -- 
(1.*36.0:9pt)  -- (1.5*36.0:4.5pt)   -- 
(2.*36.0:9pt)  -- (2.5*36.0:4.5pt)   -- 
(3.*36.0:9pt)  -- (3.5*36.0:4.5pt)   -- 
(4.*36.0:9pt)  -- (4.5*36.0:4.5pt)   -- 
(5.*36.0:9pt)  -- (5.5*36.0:4.5pt)   -- 
(6.*36.0:9pt)  -- (6.5*36.0:4.5pt)   -- 
(7.*36.0:9pt)  -- (7.5*36.0:4.5pt)   -- 
(8.*36.0:9pt)  -- (8.5*36.0:4.5pt)   -- 
(9.*36.0:9pt)  -- (9.5*36.0:4.5pt)   --  cycle
(0,0) circle (2pt);
\draw (0,0) circle (9pt);
%\draw[line width=.2pt,even odd rule,rotate=90] 
%(0.*36.0:11pt)  -- (0.5*36.0:6.5pt)   -- 
%(1.*36.0:11pt)  -- (1.5*36.0:6.5pt)   -- 
%(2.*36.0:11pt)  -- (2.5*36.0:6.5pt)   -- 
%(3.*36.0:11pt)  -- (3.5*36.0:6.5pt)   -- 
%(4.*36.0:11pt)  -- (4.5*36.0:6.5pt)   -- 
%(5.*36.0:11pt)  -- (5.5*36.0:6.5pt)   -- 
%(6.*36.0:11pt)  -- (6.5*36.0:6.5pt)   -- 
%(7.*36.0:11pt)  -- (7.5*36.0:6.5pt)   -- 
%(8.*36.0:11pt)  -- (8.5*36.0:6.5pt)   -- 
%(9.*36.0:11pt)  -- (9.5*36.0:6.5pt)   -- cycle;
\draw[line width=.4pt,even odd rule,rotate=90] 
(0.5*36.0:10pt)  -- (0.5*36.0:6.5pt)   
(1.5*36.0:10pt)  -- (1.5*36.0:6.5pt)   
(2.5*36.0:10pt)  -- (2.5*36.0:6.5pt)   
(3.5*36.0:10pt)  -- (3.5*36.0:6.5pt)   
(4.5*36.0:10pt)  -- (4.5*36.0:6.5pt)   
(5.5*36.0:10pt)  -- (5.5*36.0:6.5pt)   
(6.5*36.0:10pt)  -- (6.5*36.0:6.5pt)   
(7.5*36.0:10pt)  -- (7.5*36.0:6.5pt)   
(8.5*36.0:10pt)  -- (8.5*36.0:6.5pt)   
(9.5*36.0:10pt)  -- (9.5*36.0:6.5pt)  ;
 }}}



\pgfdeclareplotmark{m3c}{%
\node[scale=\mThreescale*\AtoCscale*\symscale] at (0,0) {\tikz {%
\draw[fill=black,line width=.3pt,even odd rule,rotate=90] 
(0.*36.0:10pt)  -- (0.5*36.0:5pt)   -- 
(1.*36.0:10pt)  -- (1.5*36.0:5pt)   -- 
(2.*36.0:10pt)  -- (2.5*36.0:5pt)   -- 
(3.*36.0:10pt)  -- (3.5*36.0:5pt)   -- 
(4.*36.0:10pt)  -- (4.5*36.0:5pt)   -- 
(5.*36.0:10pt)  -- (5.5*36.0:5pt)   -- 
(6.*36.0:10pt)  -- (6.5*36.0:5pt)   -- 
(7.*36.0:10pt)  -- (7.5*36.0:5pt)   -- 
(8.*36.0:10pt)  -- (8.5*36.0:5pt)   -- 
(9.*36.0:10pt)  -- (9.5*36.0:5pt)   -- cycle
(0,0) circle (3pt);
 }}}


\pgfdeclareplotmark{m3cv}{%
\node[scale=\mThreescale*\AtoCscale*\symscale] at (0,0) {\tikz {%
\draw[fill=black,line width=.3pt,even odd rule,rotate=90] 
(0.*36.0:9pt)  -- (0.5*36.0:4.5pt)   -- 
(1.*36.0:9pt)  -- (1.5*36.0:4.5pt)   -- 
(2.*36.0:9pt)  -- (2.5*36.0:4.5pt)   -- 
(3.*36.0:9pt)  -- (3.5*36.0:4.5pt)   -- 
(4.*36.0:9pt)  -- (4.5*36.0:4.5pt)   -- 
(5.*36.0:9pt)  -- (5.5*36.0:4.5pt)   -- 
(6.*36.0:9pt)  -- (6.5*36.0:4.5pt)   -- 
(7.*36.0:9pt)  -- (7.5*36.0:4.5pt)   -- 
(8.*36.0:9pt)  -- (8.5*36.0:4.5pt)   -- 
(9.*36.0:9pt)  -- (9.5*36.0:4.5pt)   --  cycle
(0,0) circle (3pt);
\draw (0,0) circle (9pt);
%\draw[line width=.2pt,even odd rule,rotate=90] 
%(0.*36.0:11pt)  -- (0.5*36.0:6.5pt)   -- 
%(1.*36.0:11pt)  -- (1.5*36.0:6.5pt)   -- 
%(2.*36.0:11pt)  -- (2.5*36.0:6.5pt)   -- 
%(3.*36.0:11pt)  -- (3.5*36.0:6.5pt)   -- 
%(4.*36.0:11pt)  -- (4.5*36.0:6.5pt)   -- 
%(5.*36.0:11pt)  -- (5.5*36.0:6.5pt)   -- 
%(6.*36.0:11pt)  -- (6.5*36.0:6.5pt)   -- 
%(7.*36.0:11pt)  -- (7.5*36.0:6.5pt)   -- 
%(8.*36.0:11pt)  -- (8.5*36.0:6.5pt)   -- 
%(9.*36.0:11pt)  -- (9.5*36.0:6.5pt)   -- cycle;
 }}}



\pgfdeclareplotmark{m3cb}{%
\node[scale=\mThreescale*\AtoCscale*\symscale] at (0,0) {\tikz {%
\draw[fill=black,line width=.3pt,even odd rule,rotate=90] 
(0.*36.0:10pt)  -- (0.5*36.0:5pt)   -- 
(1.*36.0:10pt)  -- (1.5*36.0:5pt)   -- 
(2.*36.0:10pt)  -- (2.5*36.0:5pt)   -- 
(3.*36.0:10pt)  -- (3.5*36.0:5pt)   -- 
(4.*36.0:10pt)  -- (4.5*36.0:5pt)   -- 
(5.*36.0:10pt)  -- (5.5*36.0:5pt)   -- 
(6.*36.0:10pt)  -- (6.5*36.0:5pt)   -- 
(7.*36.0:10pt)  -- (7.5*36.0:5pt)   -- 
(8.*36.0:10pt)  -- (8.5*36.0:5pt)   -- 
(9.*36.0:10pt)  -- (9.5*36.0:5pt)   -- cycle
(0,0) circle (3pt);
\draw[line width=.4pt,even odd rule,rotate=90] 
(0.5*36.0:9pt)  -- (0.5*36.0:5pt)   
(1.5*36.0:9pt)  -- (1.5*36.0:5pt)   
(2.5*36.0:9pt)  -- (2.5*36.0:5pt)   
(3.5*36.0:9pt)  -- (3.5*36.0:5pt)   
(4.5*36.0:9pt)  -- (4.5*36.0:5pt)   
(5.5*36.0:9pt)  -- (5.5*36.0:5pt)   
(6.5*36.0:9pt)  -- (6.5*36.0:5pt)   
(7.5*36.0:9pt)  -- (7.5*36.0:5pt)   
(8.5*36.0:9pt)  -- (8.5*36.0:5pt)   
(9.5*36.0:9pt)  -- (9.5*36.0:5pt)   ;
 }}}

\pgfdeclareplotmark{m3cvb}{%
\node[scale=\mThreescale*\AtoCscale*\symscale] at (0,0) {\tikz {%
\draw[fill=black,line width=.3pt,even odd rule,rotate=90] 
(0.*36.0:9pt)  -- (0.5*36.0:4.5pt)   -- 
(1.*36.0:9pt)  -- (1.5*36.0:4.5pt)   -- 
(2.*36.0:9pt)  -- (2.5*36.0:4.5pt)   -- 
(3.*36.0:9pt)  -- (3.5*36.0:4.5pt)   -- 
(4.*36.0:9pt)  -- (4.5*36.0:4.5pt)   -- 
(5.*36.0:9pt)  -- (5.5*36.0:4.5pt)   -- 
(6.*36.0:9pt)  -- (6.5*36.0:4.5pt)   -- 
(7.*36.0:9pt)  -- (7.5*36.0:4.5pt)   -- 
(8.*36.0:9pt)  -- (8.5*36.0:4.5pt)   -- 
(9.*36.0:9pt)  -- (9.5*36.0:4.5pt)   -- cycle
(0,0) circle (3pt);
\draw (0,0) circle (9pt);
%\draw[line width=.2pt,even odd rule,rotate=90] 
%(0.*36.0:11pt)  -- (0.5*36.0:6.5pt)   -- 
%(1.*36.0:11pt)  -- (1.5*36.0:6.5pt)   -- 
%(2.*36.0:11pt)  -- (2.5*36.0:6.5pt)   -- 
%(3.*36.0:11pt)  -- (3.5*36.0:6.5pt)   -- 
%(4.*36.0:11pt)  -- (4.5*36.0:6.5pt)   -- 
%(5.*36.0:11pt)  -- (5.5*36.0:6.5pt)   -- 
%(6.*36.0:11pt)  -- (6.5*36.0:6.5pt)   -- 
%(7.*36.0:11pt)  -- (7.5*36.0:6.5pt)   -- 
%(8.*36.0:11pt)  -- (8.5*36.0:6.5pt)   -- 
%(9.*36.0:11pt)  -- (9.5*36.0:6.5pt)   --  cycle;
\draw[line width=.4pt,even odd rule,rotate=90] 
(0.5*36.0:10pt)  -- (0.5*36.0:6.5pt)   
(1.5*36.0:10pt)  -- (1.5*36.0:6.5pt)   
(2.5*36.0:10pt)  -- (2.5*36.0:6.5pt)   
(3.5*36.0:10pt)  -- (3.5*36.0:6.5pt)   
(4.5*36.0:10pt)  -- (4.5*36.0:6.5pt)   
(5.5*36.0:10pt)  -- (5.5*36.0:6.5pt)   
(6.5*36.0:10pt)  -- (6.5*36.0:6.5pt)   
(7.5*36.0:10pt)  -- (7.5*36.0:6.5pt)   
(8.5*36.0:10pt)  -- (8.5*36.0:6.5pt)   
(9.5*36.0:10pt)  -- (9.5*36.0:6.5pt)   ;
 }}}



\pgfdeclareplotmark{m4a}{%
\node[scale=\mFourscale*\symscale] at (0,0) {\tikz {%
\draw[fill=black,line width=.3pt,even odd rule,rotate=90] 
(0.*45.0:10pt)  -- (0.5*45.0:5pt)   -- 
(1.*45.0:10pt)  -- (1.5*45.0:5pt)   -- 
(2.*45.0:10pt)  -- (2.5*45.0:5pt)   -- 
(3.*45.0:10pt)  -- (3.5*45.0:5pt)   -- 
(4.*45.0:10pt)  -- (4.5*45.0:5pt)   -- 
(5.*45.0:10pt)  -- (5.5*45.0:5pt)   -- 
(6.*45.0:10pt)  -- (6.5*45.0:5pt)   -- 
(7.*45.0:10pt)  -- (7.5*45.0:5pt)   --  cycle;
 }}}


\pgfdeclareplotmark{m4av}{%
\node[scale=\mFourscale*\symscale] at (0,0) {\tikz {%
\draw[fill=black,line width=.3pt,even odd rule,rotate=90] 
(0.*45.0:9pt)  -- (0.5*45.0:4.5pt)   -- 
(1.*45.0:9pt)  -- (1.5*45.0:4.5pt)   -- 
(2.*45.0:9pt)  -- (2.5*45.0:4.5pt)   -- 
(3.*45.0:9pt)  -- (3.5*45.0:4.5pt)   -- 
(4.*45.0:9pt)  -- (4.5*45.0:4.5pt)   -- 
(5.*45.0:9pt)  -- (5.5*45.0:4.5pt)   -- 
(6.*45.0:9pt)  -- (6.5*45.0:4.5pt)   -- 
(7.*45.0:9pt)  -- (7.5*45.0:4.5pt)   -- cycle;
\draw (0,0) circle (9pt);
%\draw[line width=.3pt,even odd rule,rotate=90] 
%(0.*45.0:11pt)  -- (0.5*45.0:6.5pt)   -- 
%(1.*45.0:11pt)  -- (1.5*45.0:6.5pt)   -- 
%(2.*45.0:11pt)  -- (2.5*45.0:6.5pt)   -- 
%(3.*45.0:11pt)  -- (3.5*45.0:6.5pt)   -- 
%(4.*45.0:11pt)  -- (4.5*45.0:6.5pt)   -- 
%(5.*45.0:11pt)  -- (5.5*45.0:6.5pt)   -- 
%(6.*45.0:11pt)  -- (6.5*45.0:6.5pt)   -- 
%(7.*45.0:11pt)  -- (7.5*45.0:6.5pt)   --  cycle;
 }}}



\pgfdeclareplotmark{m4ab}{%
\node[scale=\mFourscale*\symscale] at (0,0) {\tikz {%
\draw[fill=black,line width=.3pt,even odd rule,rotate=90] 
(0.*45.0:10pt)  -- (0.5*45.0:5pt)   -- 
(1.*45.0:10pt)  -- (1.5*45.0:5pt)   -- 
(2.*45.0:10pt)  -- (2.5*45.0:5pt)   -- 
(3.*45.0:10pt)  -- (3.5*45.0:5pt)   -- 
(4.*45.0:10pt)  -- (4.5*45.0:5pt)   -- 
(5.*45.0:10pt)  -- (5.5*45.0:5pt)   -- 
(6.*45.0:10pt)  -- (6.5*45.0:5pt)   -- 
(7.*45.0:10pt)  -- (7.5*45.0:5pt)   --  cycle;
\draw[line width=.4pt,even odd rule,rotate=90] 
(0.5*45.0:9pt)  -- (0.5*45.0:5pt)   
(1.5*45.0:9pt)  -- (1.5*45.0:5pt)   
(2.5*45.0:9pt)  -- (2.5*45.0:5pt)   
(3.5*45.0:9pt)  -- (3.5*45.0:5pt)   
(4.5*45.0:9pt)  -- (4.5*45.0:5pt)   
(5.5*45.0:9pt)  -- (5.5*45.0:5pt)   
(6.5*45.0:9pt)  -- (6.5*45.0:5pt)   
(7.5*45.0:9pt)  -- (7.5*45.0:5pt)  ;
 }}}

\pgfdeclareplotmark{m4avb}{%
\node[scale=\mFourscale*\symscale] at (0,0) {\tikz {%
\draw[fill=black,line width=.3pt,even odd rule,rotate=90] 
(0.*45.0:9pt)  -- (0.5*45.0:4.5pt)   -- 
(1.*45.0:9pt)  -- (1.5*45.0:4.5pt)   -- 
(2.*45.0:9pt)  -- (2.5*45.0:4.5pt)   -- 
(3.*45.0:9pt)  -- (3.5*45.0:4.5pt)   -- 
(4.*45.0:9pt)  -- (4.5*45.0:4.5pt)   -- 
(5.*45.0:9pt)  -- (5.5*45.0:4.5pt)   -- 
(6.*45.0:9pt)  -- (6.5*45.0:4.5pt)   -- 
(7.*45.0:9pt)  -- (7.5*45.0:4.5pt)   --  cycle;
\draw (0,0) circle (9pt);
%\draw[line width=.3pt,even odd rule,rotate=90] 
%(0.*45.0:11pt)  -- (0.5*45.0:6.5pt)   -- 
%(1.*45.0:11pt)  -- (1.5*45.0:6.5pt)   -- 
%(2.*45.0:11pt)  -- (2.5*45.0:6.5pt)   -- 
%(3.*45.0:11pt)  -- (3.5*45.0:6.5pt)   -- 
%(4.*45.0:11pt)  -- (4.5*45.0:6.5pt)   -- 
%(5.*45.0:11pt)  -- (5.5*45.0:6.5pt)   -- 
%(6.*45.0:11pt)  -- (6.5*45.0:6.5pt)   -- 
%(7.*45.0:11pt)  -- (7.5*45.0:6.5pt)   --  cycle;
\draw[line width=.4pt,even odd rule,rotate=90] 
(0.5*45.0:10pt)  -- (0.5*45.0:6.5pt)   
(1.5*45.0:10pt)  -- (1.5*45.0:6.5pt)   
(2.5*45.0:10pt)  -- (2.5*45.0:6.5pt)   
(3.5*45.0:10pt)  -- (3.5*45.0:6.5pt)   
(4.5*45.0:10pt)  -- (4.5*45.0:6.5pt)   
(5.5*45.0:10pt)  -- (5.5*45.0:6.5pt)   
(6.5*45.0:10pt)  -- (6.5*45.0:6.5pt)   
(7.5*45.0:10pt)  -- (7.5*45.0:6.5pt)   ;
 }}}



\pgfdeclareplotmark{m4b}{%
\node[scale=\mFourscale*\AtoBscale*\symscale] at (0,0) {\tikz {%
\draw[fill=black,line width=.3pt,even odd rule,rotate=90] 
(0.*45.0:10pt)  -- (0.5*45.0:5pt)   -- 
(1.*45.0:10pt)  -- (1.5*45.0:5pt)   -- 
(2.*45.0:10pt)  -- (2.5*45.0:5pt)   -- 
(3.*45.0:10pt)  -- (3.5*45.0:5pt)   -- 
(4.*45.0:10pt)  -- (4.5*45.0:5pt)   -- 
(5.*45.0:10pt)  -- (5.5*45.0:5pt)   -- 
(6.*45.0:10pt)  -- (6.5*45.0:5pt)   -- 
(7.*45.0:10pt)  -- (7.5*45.0:5pt)   -- cycle
(0,0) circle (2pt);
 }}}


\pgfdeclareplotmark{m4bv}{%
\node[scale=\mFourscale*\AtoBscale*\symscale] at (0,0) {\tikz {%
\draw[fill=black,line width=.3pt,even odd rule,rotate=90] 
(0.*45.0:9pt)  -- (0.5*45.0:4.5pt)   -- 
(1.*45.0:9pt)  -- (1.5*45.0:4.5pt)   -- 
(2.*45.0:9pt)  -- (2.5*45.0:4.5pt)   -- 
(3.*45.0:9pt)  -- (3.5*45.0:4.5pt)   -- 
(4.*45.0:9pt)  -- (4.5*45.0:4.5pt)   -- 
(5.*45.0:9pt)  -- (5.5*45.0:4.5pt)   -- 
(6.*45.0:9pt)  -- (6.5*45.0:4.5pt)   -- 
(7.*45.0:9pt)  -- (7.5*45.0:4.5pt)   -- cycle
(0,0) circle (2pt);
\draw (0,0) circle (9pt);
%\draw[line width=.3pt,even odd rule,rotate=90] 
%(0.*45.0:11pt)  -- (0.5*45.0:6.5pt)   -- 
%(1.*45.0:11pt)  -- (1.5*45.0:6.5pt)   -- 
%(2.*45.0:11pt)  -- (2.5*45.0:6.5pt)   -- 
%(3.*45.0:11pt)  -- (3.5*45.0:6.5pt)   -- 
%(4.*45.0:11pt)  -- (4.5*45.0:6.5pt)   -- 
%(5.*45.0:11pt)  -- (5.5*45.0:6.5pt)   -- 
%(6.*45.0:11pt)  -- (6.5*45.0:6.5pt)   -- 
%(7.*45.0:11pt)  -- (7.5*45.0:6.5pt)   --cycle;
 }}}



\pgfdeclareplotmark{m4bb}{%
\node[scale=\mFourscale*\AtoBscale*\symscale] at (0,0) {\tikz {%
\draw[fill=black,line width=.3pt,even odd rule,rotate=90] 
(0.*45.0:10pt)  -- (0.5*45.0:5pt)   -- 
(1.*45.0:10pt)  -- (1.5*45.0:5pt)   -- 
(2.*45.0:10pt)  -- (2.5*45.0:5pt)   -- 
(3.*45.0:10pt)  -- (3.5*45.0:5pt)   -- 
(4.*45.0:10pt)  -- (4.5*45.0:5pt)   -- 
(5.*45.0:10pt)  -- (5.5*45.0:5pt)   -- 
(6.*45.0:10pt)  -- (6.5*45.0:5pt)   -- 
(7.*45.0:10pt)  -- (7.5*45.0:5pt)   --  cycle
(0,0) circle (2pt);
\draw[line width=.4pt,even odd rule,rotate=90] 
(0.5*45.0:9pt)  -- (0.5*45.0:5pt)   
(1.5*45.0:9pt)  -- (1.5*45.0:5pt)   
(2.5*45.0:9pt)  -- (2.5*45.0:5pt)   
(3.5*45.0:9pt)  -- (3.5*45.0:5pt)   
(4.5*45.0:9pt)  -- (4.5*45.0:5pt)   
(5.5*45.0:9pt)  -- (5.5*45.0:5pt)   
(6.5*45.0:9pt)  -- (6.5*45.0:5pt)   
(7.5*45.0:9pt)  -- (7.5*45.0:5pt)  ;
 }}}

\pgfdeclareplotmark{m4bvb}{%
\node[scale=\mFourscale*\AtoBscale*\symscale] at (0,0) {\tikz {%
\draw[fill=black,line width=.3pt,even odd rule,rotate=90] 
(0.*45.0:9pt)  -- (0.5*45.0:4.5pt)   -- 
(1.*45.0:9pt)  -- (1.5*45.0:4.5pt)   -- 
(2.*45.0:9pt)  -- (2.5*45.0:4.5pt)   -- 
(3.*45.0:9pt)  -- (3.5*45.0:4.5pt)   -- 
(4.*45.0:9pt)  -- (4.5*45.0:4.5pt)   -- 
(5.*45.0:9pt)  -- (5.5*45.0:4.5pt)   -- 
(6.*45.0:9pt)  -- (6.5*45.0:4.5pt)   -- 
(7.*45.0:9pt)  -- (7.5*45.0:4.5pt)   --  cycle
(0,0) circle (2pt);
\draw (0,0) circle (9pt);
%\draw[line width=.3pt,even odd rule,rotate=90] 
%(0.*45.0:11pt)  -- (0.5*45.0:6.5pt)   -- 
%(1.*45.0:11pt)  -- (1.5*45.0:6.5pt)   -- 
%(2.*45.0:11pt)  -- (2.5*45.0:6.5pt)   -- 
%(3.*45.0:11pt)  -- (3.5*45.0:6.5pt)   -- 
%(4.*45.0:11pt)  -- (4.5*45.0:6.5pt)   -- 
%(5.*45.0:11pt)  -- (5.5*45.0:6.5pt)   -- 
%(6.*45.0:11pt)  -- (6.5*45.0:6.5pt)   -- 
%(7.*45.0:11pt)  -- (7.5*45.0:6.5pt)   -- cycle;
\draw[line width=.4pt,even odd rule,rotate=90] 
(0.5*45.0:10pt)  -- (0.5*45.0:6.5pt)   
(1.5*45.0:10pt)  -- (1.5*45.0:6.5pt)   
(2.5*45.0:10pt)  -- (2.5*45.0:6.5pt)   
(3.5*45.0:10pt)  -- (3.5*45.0:6.5pt)   
(4.5*45.0:10pt)  -- (4.5*45.0:6.5pt)   
(5.5*45.0:10pt)  -- (5.5*45.0:6.5pt)   
(6.5*45.0:10pt)  -- (6.5*45.0:6.5pt)   
(7.5*45.0:10pt)  -- (7.5*45.0:6.5pt)  ;
 }}}



\pgfdeclareplotmark{m4c}{%
\node[scale=\mFourscale*\AtoCscale*\symscale] at (0,0) {\tikz {%
\draw[fill=black,line width=.3pt,even odd rule,rotate=90] 
(0.*45.0:10pt)  -- (0.5*45.0:5pt)   -- 
(1.*45.0:10pt)  -- (1.5*45.0:5pt)   -- 
(2.*45.0:10pt)  -- (2.5*45.0:5pt)   -- 
(3.*45.0:10pt)  -- (3.5*45.0:5pt)   -- 
(4.*45.0:10pt)  -- (4.5*45.0:5pt)   -- 
(5.*45.0:10pt)  -- (5.5*45.0:5pt)   -- 
(6.*45.0:10pt)  -- (6.5*45.0:5pt)   -- 
(7.*45.0:10pt)  -- (7.5*45.0:5pt)   -- cycle
(0,0) circle (3pt);
 }}}


\pgfdeclareplotmark{m4cv}{%
\node[scale=\mFourscale*\AtoCscale*\symscale] at (0,0) {\tikz {%
\draw[fill=black,line width=.3pt,even odd rule,rotate=90] 
(0.*45.0:9pt)  -- (0.5*45.0:4.5pt)   -- 
(1.*45.0:9pt)  -- (1.5*45.0:4.5pt)   -- 
(2.*45.0:9pt)  -- (2.5*45.0:4.5pt)   -- 
(3.*45.0:9pt)  -- (3.5*45.0:4.5pt)   -- 
(4.*45.0:9pt)  -- (4.5*45.0:4.5pt)   -- 
(5.*45.0:9pt)  -- (5.5*45.0:4.5pt)   -- 
(6.*45.0:9pt)  -- (6.5*45.0:4.5pt)   -- 
(7.*45.0:9pt)  -- (7.5*45.0:4.5pt)   --  cycle
(0,0) circle (3pt);
\draw (0,0) circle (9pt);
%\draw[line width=.3pt,even odd rule,rotate=90] 
%(0.*45.0:11pt)  -- (0.5*45.0:6.5pt)   -- 
%(1.*45.0:11pt)  -- (1.5*45.0:6.5pt)   -- 
%(2.*45.0:11pt)  -- (2.5*45.0:6.5pt)   -- 
%(3.*45.0:11pt)  -- (3.5*45.0:6.5pt)   -- 
%(4.*45.0:11pt)  -- (4.5*45.0:6.5pt)   -- 
%(5.*45.0:11pt)  -- (5.5*45.0:6.5pt)   -- 
%(6.*45.0:11pt)  -- (6.5*45.0:6.5pt)   -- 
%(7.*45.0:11pt)  -- (7.5*45.0:6.5pt)   -- cycle;
 }}}



\pgfdeclareplotmark{m4cb}{%
\node[scale=\mFourscale*\AtoCscale*\symscale] at (0,0) {\tikz {%
\draw[fill=black,line width=.3pt,even odd rule,rotate=90] 
(0.*45.0:10pt)  -- (0.5*45.0:5pt)   -- 
(1.*45.0:10pt)  -- (1.5*45.0:5pt)   -- 
(2.*45.0:10pt)  -- (2.5*45.0:5pt)   -- 
(3.*45.0:10pt)  -- (3.5*45.0:5pt)   -- 
(4.*45.0:10pt)  -- (4.5*45.0:5pt)   -- 
(5.*45.0:10pt)  -- (5.5*45.0:5pt)   -- 
(6.*45.0:10pt)  -- (6.5*45.0:5pt)   -- 
(7.*45.0:10pt)  -- (7.5*45.0:5pt)   -- cycle
(0,0) circle (3pt);
\draw[line width=.4pt,even odd rule,rotate=90] 
(0.5*45.0:9pt)  -- (0.5*45.0:5pt)   
(1.5*45.0:9pt)  -- (1.5*45.0:5pt)   
(2.5*45.0:9pt)  -- (2.5*45.0:5pt)   
(3.5*45.0:9pt)  -- (3.5*45.0:5pt)   
(4.5*45.0:9pt)  -- (4.5*45.0:5pt)   
(5.5*45.0:9pt)  -- (5.5*45.0:5pt)   
(6.5*45.0:9pt)  -- (6.5*45.0:5pt)   
(7.5*45.0:9pt)  -- (7.5*45.0:5pt)  ;
 }}}

\pgfdeclareplotmark{m4cvb}{%
\node[scale=\mFourscale*\AtoCscale*\symscale] at (0,0) {\tikz {%
\draw[fill=black,line width=.3pt,even odd rule,rotate=90] 
(0.*45.0:9pt)  -- (0.5*45.0:4.5pt)   -- 
(1.*45.0:9pt)  -- (1.5*45.0:4.5pt)   -- 
(2.*45.0:9pt)  -- (2.5*45.0:4.5pt)   -- 
(3.*45.0:9pt)  -- (3.5*45.0:4.5pt)   -- 
(4.*45.0:9pt)  -- (4.5*45.0:4.5pt)   -- 
(5.*45.0:9pt)  -- (5.5*45.0:4.5pt)   -- 
(6.*45.0:9pt)  -- (6.5*45.0:4.5pt)   -- 
(7.*45.0:9pt)  -- (7.5*45.0:4.5pt)   --  cycle
(0,0) circle (3pt);
\draw (0,0) circle (9pt);
%\draw[line width=.3pt,even odd rule,rotate=90] 
%(0.*45.0:11pt)  -- (0.5*45.0:6.5pt)   -- 
%(1.*45.0:11pt)  -- (1.5*45.0:6.5pt)   -- 
%(2.*45.0:11pt)  -- (2.5*45.0:6.5pt)   -- 
%(3.*45.0:11pt)  -- (3.5*45.0:6.5pt)   -- 
%(4.*45.0:11pt)  -- (4.5*45.0:6.5pt)   -- 
%(5.*45.0:11pt)  -- (5.5*45.0:6.5pt)   -- 
%(6.*45.0:11pt)  -- (6.5*45.0:6.5pt)   -- 
%(7.*45.0:11pt)  -- (7.5*45.0:6.5pt)   --  cycle;
\draw[line width=.4pt,even odd rule,rotate=90] 
(0.5*45.0:10pt)  -- (0.5*45.0:6.5pt)   
(1.5*45.0:10pt)  -- (1.5*45.0:6.5pt)   
(2.5*45.0:10pt)  -- (2.5*45.0:6.5pt)   
(3.5*45.0:10pt)  -- (3.5*45.0:6.5pt)   
(4.5*45.0:10pt)  -- (4.5*45.0:6.5pt)   
(5.5*45.0:10pt)  -- (5.5*45.0:6.5pt)   
(6.5*45.0:10pt)  -- (6.5*45.0:6.5pt)   
(7.5*45.0:10pt)  -- (7.5*45.0:6.5pt)   ;
 }}}



\pgfdeclareplotmark{m5a}{%
\node[scale=\mFivescale*\symscale] at (0,0) {\tikz {%
\draw[fill=black,line width=.3pt,even odd rule,rotate=90] 
(0.*60.0:10pt)  -- (0.5*60.0:1.14*5.0pt)   -- 
(1.*60.0:10pt)  -- (1.5*60.0:1.14*5.0pt)   -- 
(2.*60.0:10pt)  -- (2.5*60.0:1.14*5.0pt)   -- 
(3.*60.0:10pt)  -- (3.5*60.0:1.14*5.0pt)   -- 
(4.*60.0:10pt)  -- (4.5*60.0:1.14*5.0pt)   -- 
(5.*60.0:10pt)  -- (5.5*60.0:1.14*5.0pt)   --  cycle;
 }}}


\pgfdeclareplotmark{m5av}{%
\node[scale=\mFivescale*\symscale] at (0,0) {\tikz {%
\draw[fill=black,line width=.3pt,even odd rule,rotate=90] 
(0.*60.0:9pt)  -- (0.5*60.0:1.14*4.5pt)   -- 
(1.*60.0:9pt)  -- (1.5*60.0:1.14*4.5pt)   -- 
(2.*60.0:9pt)  -- (2.5*60.0:1.14*4.5pt)   -- 
(3.*60.0:9pt)  -- (3.5*60.0:1.14*4.5pt)   -- 
(4.*60.0:9pt)  -- (4.5*60.0:1.14*4.5pt)   -- 
(5.*60.0:9pt)  -- (5.5*60.0:1.14*4.5pt)   -- cycle;
\draw (0,0) circle (9pt);
%\draw[line width=.35pt,even odd rule,rotate=90] 
%(0.*60.0:11pt)  -- (0.5*60.0:1.14*1.00*6.0pt)   -- 
%(1.*60.0:11pt)  -- (1.5*60.0:1.14*1.00*6.0pt)   -- 
%(2.*60.0:11pt)  -- (2.5*60.0:1.14*1.00*6.0pt)   -- 
%(3.*60.0:11pt)  -- (3.5*60.0:1.14*1.00*6.0pt)   -- 
%(4.*60.0:11pt)  -- (4.5*60.0:1.14*1.00*6.0pt)   -- 
%(5.*60.0:11pt)  -- (5.5*60.0:1.14*1.00*6.0pt)   --  cycle;
 }}}



\pgfdeclareplotmark{m5ab}{%
\node[scale=\mFivescale*\symscale] at (0,0) {\tikz {%
\draw[fill=black,line width=.3pt,even odd rule,rotate=90] 
(0.*60.0:10pt)  -- (0.5*60.0:1.14*5.0pt)   -- 
(1.*60.0:10pt)  -- (1.5*60.0:1.14*5.0pt)   -- 
(2.*60.0:10pt)  -- (2.5*60.0:1.14*5.0pt)   -- 
(3.*60.0:10pt)  -- (3.5*60.0:1.14*5.0pt)   -- 
(4.*60.0:10pt)  -- (4.5*60.0:1.14*5.0pt)   -- 
(5.*60.0:10pt)  -- (5.5*60.0:1.14*5.0pt)   --  cycle;
\draw[line width=0.6pt,even odd rule,rotate=90] 
(0.5*60.0:9pt)  -- (0.5*60.0:1.14*5.0pt)   
(1.5*60.0:9pt)  -- (1.5*60.0:1.14*5.0pt)   
(2.5*60.0:9pt)  -- (2.5*60.0:1.14*5.0pt)   
(3.5*60.0:9pt)  -- (3.5*60.0:1.14*5.0pt)   
(4.5*60.0:9pt)  -- (4.5*60.0:1.14*5.0pt)   
(5.5*60.0:9pt)  -- (5.5*60.0:1.14*5.0pt)  ;
 }}}

\pgfdeclareplotmark{m5avb}{%
\node[scale=\mFivescale*\symscale] at (0,0) {\tikz {%
\draw[fill=black,line width=.3pt,even odd rule,rotate=90] 
(0.*60.0:9pt)  -- (0.5*60.0:1.14*4.5pt)   -- 
(1.*60.0:9pt)  -- (1.5*60.0:1.14*4.5pt)   -- 
(2.*60.0:9pt)  -- (2.5*60.0:1.14*4.5pt)   -- 
(3.*60.0:9pt)  -- (3.5*60.0:1.14*4.5pt)   -- 
(4.*60.0:9pt)  -- (4.5*60.0:1.14*4.5pt)   -- 
(5.*60.0:9pt)  -- (5.5*60.0:1.14*4.5pt)   --  cycle;
\draw (0,0) circle (9pt);
%\draw[line width=.35pt,even odd rule,rotate=90] 
%(0.*60.0:11pt)  -- (0.5*60.0:1.14*1.00*6.0pt)   -- 
%(1.*60.0:11pt)  -- (1.5*60.0:1.14*1.00*6.0pt)   -- 
%(2.*60.0:11pt)  -- (2.5*60.0:1.14*1.00*6.0pt)   -- 
%(3.*60.0:11pt)  -- (3.5*60.0:1.14*1.00*6.0pt)   -- 
%(4.*60.0:11pt)  -- (4.5*60.0:1.14*1.00*6.0pt)   -- 
%(5.*60.0:11pt)  -- (5.5*60.0:1.14*1.00*6.0pt)   --  cycle;
\draw[line width=0.6pt,even odd rule,rotate=90] 
(0.5*60.0:10pt)  -- (0.5*60.0:1.14*1.00*6.0pt)   
(1.5*60.0:10pt)  -- (1.5*60.0:1.14*1.00*6.0pt)   
(2.5*60.0:10pt)  -- (2.5*60.0:1.14*1.00*6.0pt)   
(3.5*60.0:10pt)  -- (3.5*60.0:1.14*1.00*6.0pt)   
(4.5*60.0:10pt)  -- (4.5*60.0:1.14*1.00*6.0pt)   
(5.5*60.0:10pt)  -- (5.5*60.0:1.14*1.00*6.0pt)   ;
 }}}



\pgfdeclareplotmark{m5b}{%
\node[scale=\mFivescale*\AtoBscale*\symscale] at (0,0) {\tikz {%
\draw[fill=black,line width=.3pt,even odd rule,rotate=90] 
(0.*60.0:10pt)  -- (0.5*60.0:1.14*5.0pt)   -- 
(1.*60.0:10pt)  -- (1.5*60.0:1.14*5.0pt)   -- 
(2.*60.0:10pt)  -- (2.5*60.0:1.14*5.0pt)   -- 
(3.*60.0:10pt)  -- (3.5*60.0:1.14*5.0pt)   -- 
(4.*60.0:10pt)  -- (4.5*60.0:1.14*5.0pt)   -- 
(5.*60.0:10pt)  -- (5.5*60.0:1.14*5.0pt)   -- cycle
(0,0) circle (2pt);
 }}}


\pgfdeclareplotmark{m5bv}{%
\node[scale=\mFivescale*\AtoBscale*\symscale] at (0,0) {\tikz {%
\draw[fill=black,line width=.3pt,even odd rule,rotate=90] 
(0.*60.0:9pt)  -- (0.5*60.0:1.14*4.5pt)   -- 
(1.*60.0:9pt)  -- (1.5*60.0:1.14*4.5pt)   -- 
(2.*60.0:9pt)  -- (2.5*60.0:1.14*4.5pt)   -- 
(3.*60.0:9pt)  -- (3.5*60.0:1.14*4.5pt)   -- 
(4.*60.0:9pt)  -- (4.5*60.0:1.14*4.5pt)   -- 
(5.*60.0:9pt)  -- (5.5*60.0:1.14*4.5pt)   -- cycle
(0,0) circle (2pt);
\draw (0,0) circle (9pt);
%\draw[line width=.35pt,even odd rule,rotate=90] 
%(0.*60.0:11pt)  -- (0.5*60.0:1.14*1.00*6.0pt)   -- 
%(1.*60.0:11pt)  -- (1.5*60.0:1.14*1.00*6.0pt)   -- 
%(2.*60.0:11pt)  -- (2.5*60.0:1.14*1.00*6.0pt)   -- 
%(3.*60.0:11pt)  -- (3.5*60.0:1.14*1.00*6.0pt)   -- 
%(4.*60.0:11pt)  -- (4.5*60.0:1.14*1.00*6.0pt)   -- 
%(5.*60.0:11pt)  -- (5.5*60.0:1.14*1.00*6.0pt)   --  cycle;
 }}}



\pgfdeclareplotmark{m5bb}{%
\node[scale=\mFivescale*\AtoBscale*\symscale] at (0,0) {\tikz {%
\draw[fill=black,line width=.3pt,even odd rule,rotate=90] 
(0.*60.0:10pt)  -- (0.5*60.0:1.14*5.0pt)   -- 
(1.*60.0:10pt)  -- (1.5*60.0:1.14*5.0pt)   -- 
(2.*60.0:10pt)  -- (2.5*60.0:1.14*5.0pt)   -- 
(3.*60.0:10pt)  -- (3.5*60.0:1.14*5.0pt)   -- 
(4.*60.0:10pt)  -- (4.5*60.0:1.14*5.0pt)   -- 
(5.*60.0:10pt)  -- (5.5*60.0:1.14*5.0pt)   --  cycle
(0,0) circle (2pt);
\draw[line width=0.6pt,even odd rule,rotate=90] 
(0.5*60.0:9pt)  -- (0.5*60.0:1.14*5.0pt)   
(1.5*60.0:9pt)  -- (1.5*60.0:1.14*5.0pt)   
(2.5*60.0:9pt)  -- (2.5*60.0:1.14*5.0pt)   
(3.5*60.0:9pt)  -- (3.5*60.0:1.14*5.0pt)   
(4.5*60.0:9pt)  -- (4.5*60.0:1.14*5.0pt)   
(5.5*60.0:9pt)  -- (5.5*60.0:1.14*5.0pt)  ;
 }}}

\pgfdeclareplotmark{m5bvb}{%
\node[scale=\mFivescale*\AtoBscale*\symscale] at (0,0) {\tikz {%
\draw[fill=black,line width=.3pt,even odd rule,rotate=90] 
(0.*60.0:9pt)  -- (0.5*60.0:1.14*4.5pt)   -- 
(1.*60.0:9pt)  -- (1.5*60.0:1.14*4.5pt)   -- 
(2.*60.0:9pt)  -- (2.5*60.0:1.14*4.5pt)   -- 
(3.*60.0:9pt)  -- (3.5*60.0:1.14*4.5pt)   -- 
(4.*60.0:9pt)  -- (4.5*60.0:1.14*4.5pt)   -- 
(5.*60.0:9pt)  -- (5.5*60.0:1.14*4.5pt)   --  cycle
(0,0) circle (2pt);
\draw (0,0) circle (9pt);
%\draw[line width=.35pt,even odd rule,rotate=90] 
%(0.*60.0:11pt)  -- (0.5*60.0:1.14*1.00*6.0pt)   -- 
%(1.*60.0:11pt)  -- (1.5*60.0:1.14*1.00*6.0pt)   -- 
%(2.*60.0:11pt)  -- (2.5*60.0:1.14*1.00*6.0pt)   -- 
%(3.*60.0:11pt)  -- (3.5*60.0:1.14*1.00*6.0pt)   -- 
%(4.*60.0:11pt)  -- (4.5*60.0:1.14*1.00*6.0pt)   -- 
%(5.*60.0:11pt)  -- (5.5*60.0:1.14*1.00*6.0pt)   -- cycle;
\draw[line width=0.6pt,even odd rule,rotate=90] 
(0.5*60.0:10pt)  -- (0.5*60.0:1.14*1.00*6.0pt)   
(1.5*60.0:10pt)  -- (1.5*60.0:1.14*1.00*6.0pt)   
(2.5*60.0:10pt)  -- (2.5*60.0:1.14*1.00*6.0pt)   
(3.5*60.0:10pt)  -- (3.5*60.0:1.14*1.00*6.0pt)   
(4.5*60.0:10pt)  -- (4.5*60.0:1.14*1.00*6.0pt)   
(5.5*60.0:10pt)  -- (5.5*60.0:1.14*1.00*6.0pt)  ;
 }}}



\pgfdeclareplotmark{m5c}{%
\node[scale=\mFivescale*\AtoCscale*\symscale] at (0,0) {\tikz {%
\draw[fill=black,line width=.3pt,even odd rule,rotate=90] 
(0.*60.0:10pt)  -- (0.5*60.0:1.14*5.0pt)   -- 
(1.*60.0:10pt)  -- (1.5*60.0:1.14*5.0pt)   -- 
(2.*60.0:10pt)  -- (2.5*60.0:1.14*5.0pt)   -- 
(3.*60.0:10pt)  -- (3.5*60.0:1.14*5.0pt)   -- 
(4.*60.0:10pt)  -- (4.5*60.0:1.14*5.0pt)   -- 
(5.*60.0:10pt)  -- (5.5*60.0:1.14*5.0pt)   -- cycle
(0,0) circle (3.25pt);
 }}}


\pgfdeclareplotmark{m5cv}{%
\node[scale=\mFivescale*\AtoCscale*\symscale] at (0,0) {\tikz {%
\draw[fill=black,line width=.3pt,even odd rule,rotate=90] 
(0.*60.0:9pt)  -- (0.5*60.0:1.14*4.5pt)   -- 
(1.*60.0:9pt)  -- (1.5*60.0:1.14*4.5pt)   -- 
(2.*60.0:9pt)  -- (2.5*60.0:1.14*4.5pt)   -- 
(3.*60.0:9pt)  -- (3.5*60.0:1.14*4.5pt)   -- 
(4.*60.0:9pt)  -- (4.5*60.0:1.14*4.5pt)   -- 
(5.*60.0:9pt)  -- (5.5*60.0:1.14*4.5pt)   --  cycle
(0,0) circle (3.25pt);
\draw (0,0) circle (9pt);
%\draw[line width=.35pt,even odd rule,rotate=90] 
%(0.*60.0:11pt)  -- (0.5*60.0:1.14*1.00*6.0pt)   -- 
%(1.*60.0:11pt)  -- (1.5*60.0:1.14*1.00*6.0pt)   -- 
%(2.*60.0:11pt)  -- (2.5*60.0:1.14*1.00*6.0pt)   -- 
%(3.*60.0:11pt)  -- (3.5*60.0:1.14*1.00*6.0pt)   -- 
%(4.*60.0:11pt)  -- (4.5*60.0:1.14*1.00*6.0pt)   -- 
%(5.*60.0:11pt)  -- (5.5*60.0:1.14*1.00*6.0pt)   -- cycle;
 }}}



\pgfdeclareplotmark{m5cb}{%
\node[scale=\mFivescale*\AtoCscale*\symscale] at (0,0) {\tikz {%
\draw[fill=black,line width=.3pt,even odd rule,rotate=90] 
(0.*60.0:10pt)  -- (0.5*60.0:1.14*5.0pt)   -- 
(1.*60.0:10pt)  -- (1.5*60.0:1.14*5.0pt)   -- 
(2.*60.0:10pt)  -- (2.5*60.0:1.14*5.0pt)   -- 
(3.*60.0:10pt)  -- (3.5*60.0:1.14*5.0pt)   -- 
(4.*60.0:10pt)  -- (4.5*60.0:1.14*5.0pt)   -- 
(5.*60.0:10pt)  -- (5.5*60.0:1.14*5.0pt)   -- cycle
(0,0) circle (3.25pt);
\draw[line width=0.6pt,even odd rule,rotate=90] 
(0.5*60.0:9pt)  -- (0.5*60.0:1.14*5.0pt)   
(1.5*60.0:9pt)  -- (1.5*60.0:1.14*5.0pt)   
(2.5*60.0:9pt)  -- (2.5*60.0:1.14*5.0pt)   
(3.5*60.0:9pt)  -- (3.5*60.0:1.14*5.0pt)   
(4.5*60.0:9pt)  -- (4.5*60.0:1.14*5.0pt)   
(5.5*60.0:9pt)  -- (5.5*60.0:1.14*5.0pt)  ;
 }}}

\pgfdeclareplotmark{m5cvb}{%
\node[scale=\mFivescale*\AtoCscale*\symscale] at (0,0) {\tikz {%
\draw[fill=black,line width=.3pt,even odd rule,rotate=90] 
(0.*60.0:9pt)  -- (0.5*60.0:1.14*4.5pt)   -- 
(1.*60.0:9pt)  -- (1.5*60.0:1.14*4.5pt)   -- 
(2.*60.0:9pt)  -- (2.5*60.0:1.14*4.5pt)   -- 
(3.*60.0:9pt)  -- (3.5*60.0:1.14*4.5pt)   -- 
(4.*60.0:9pt)  -- (4.5*60.0:1.14*4.5pt)   -- 
(5.*60.0:9pt)  -- (5.5*60.0:1.14*4.5pt)   --  cycle
(0,0) circle (3.25pt);
\draw (0,0) circle (9pt);
%\draw[line width=.35pt,even odd rule,rotate=90] 
%(0.*60.0:11pt)  -- (0.5*60.0:1.14*1.00*6.0pt)   -- 
%(1.*60.0:11pt)  -- (1.5*60.0:1.14*1.00*6.0pt)   -- 
%(2.*60.0:11pt)  -- (2.5*60.0:1.14*1.00*6.0pt)   -- 
%(3.*60.0:11pt)  -- (3.5*60.0:1.14*1.00*6.0pt)   -- 
%(4.*60.0:11pt)  -- (4.5*60.0:1.14*1.00*6.0pt)   -- 
%(5.*60.0:11pt)  -- (5.5*60.0:1.14*1.00*6.0pt)   -- cycle;
\draw[line width=0.6pt,even odd rule,rotate=90] 
(0.5*60.0:10pt)  -- (0.5*60.0:1.14*1.00*6.0pt)   
(1.5*60.0:10pt)  -- (1.5*60.0:1.14*1.00*6.0pt)   
(2.5*60.0:10pt)  -- (2.5*60.0:1.14*1.00*6.0pt)   
(3.5*60.0:10pt)  -- (3.5*60.0:1.14*1.00*6.0pt)   
(4.5*60.0:10pt)  -- (4.5*60.0:1.14*1.00*6.0pt)   
(5.5*60.0:10pt)  -- (5.5*60.0:1.14*1.00*6.0pt)   ;
 }}}

\pgfdeclareplotmark{m6a}{%
\node[scale=\mSixscale*\symscale] at (0,0) {\tikz {%
\draw[fill,line width=0.2pt,even odd rule,rotate=90] %
(0:10pt) -- (36:3.8pt)    -- 
(72:10pt) -- (108:3.8pt)  -- 
(144:10pt) -- (180:3.8pt) -- 
(216:10pt) -- (252:3.8pt) --
(288:10pt) -- (324:3.8pt) -- cycle
(0pt,0pt) circle (2.5pt);
% The following is needed to center the darn thing.
\draw[opacity=0] (0pt,0pt) circle (13pt) ;
 }}}


\pgfdeclareplotmark{m6av}{%
\node[scale=\mSixscale*\symscale] at (0,0) {\tikz {%
\draw[fill,line width=0.2pt,even odd rule,rotate=90] %
(0:0.9*10pt) -- (36:0.9*3.8pt)    -- 
(72:0.9*10pt) -- (108:0.9*3.8pt)  -- 
(144:0.9*10pt) -- (180:0.9*3.8pt) -- 
(216:0.9*10pt) -- (252:0.9*3.8pt) --
(288:0.9*10pt) -- (324:0.9*3.8pt) -- cycle
(0pt,0pt) circle (2.5pt);
\draw (0,0) circle (10pt);
%\draw[line width=0.6pt,even odd rule,rotate=90] %
%(0:0.9*13pt) --  (36:0.9*6.0pt)    -- 
%(72:0.9*13pt) -- (108:0.9*6.0pt)  -- 
%(144:0.9*13pt) -- (180:0.9*6.0pt) -- 
%(216:0.9*13pt) -- (252:0.9*6.0pt) --
%(288:0.9*13pt) -- (324:0.9*6.0pt) -- cycle ;
% The following is needed to center the darn thing.
\draw[opacity=0] (0pt,0pt) circle (13pt) ;
 }}}

\pgfdeclareplotmark{m6ab}{%
\node[scale=\mSixscale*\symscale] at (0,0) {\tikz {%
\draw[fill,line width=0.2pt,even odd rule,rotate=90] %
(0:10pt) -- (36:3.8pt)    -- 
(72:10pt) -- (108:3.8pt)  -- 
(144:10pt) -- (180:3.8pt) -- 
(216:10pt) -- (252:3.8pt) --
(288:10pt) -- (324:3.8pt) -- cycle
(0pt,0pt) circle (2.5pt);
\draw[line width=0.9pt,even odd rule,rotate=90] %
(36:9pt) -- (36:3.8pt)    
(108:9pt) -- (108:3.8pt)  
(180:9pt) -- (180:3.8pt) 
(252:9pt) -- (252:3.8pt) 
(324:9pt) -- (324:3.8pt) 
(0pt,0pt) circle (2.5pt);
% The following is needed to center the darn thing.
\draw[opacity=0] (0pt,0pt) circle (13pt) ;
 }}}

\pgfdeclareplotmark{m6avb}{%
\node[scale=\mSixscale*\symscale] at (0,0) {\tikz {%
\draw[fill,line width=0.2pt,even odd rule,rotate=90] %
(0:10pt) -- (36:3.8pt)    -- 
(72:10pt) -- (108:3.8pt)  -- 
(144:10pt) -- (180:3.8pt) -- 
(216:10pt) -- (252:3.8pt) --
(288:10pt) -- (324:3.8pt) -- cycle
(0pt,0pt) circle (2.5pt);
\draw (0,0) circle (10pt);
%\draw[line width=0.6pt,even odd rule,rotate=90] %
%(0:0.9*13pt) --  (36:0.9*6.0pt)    -- 
%(72:0.9*13pt) -- (108:0.9*6.0pt)  -- 
%(144:0.9*13pt) -- (180:0.9*6.0pt) -- 
%(216:0.9*13pt) -- (252:0.9*6.0pt) --
%(288:0.9*13pt) -- (324:0.9*6.0pt) -- cycle ;
\draw[line width=0.9pt,even odd rule,rotate=90] %
(36:12pt) -- (36:6.0pt)    
(108:12pt) -- (108:6.0pt)  
(180:12pt) -- (180:6.0pt) 
(252:12pt) -- (252:6.0pt) 
(324:12pt) -- (324:6.0pt) 
(0pt,0pt) circle (2.5pt);
% The following is needed to center the darn thing.
\draw[opacity=0] (0pt,0pt) circle (13pt) ;
 }}}



\pgfdeclareplotmark{m6b}{%
\node[scale=\mSixscale*\AtoBscale*\symscale] at (0,0) {\tikz {%
\draw[fill,line width=0.2pt,even odd rule,rotate=45] %
(0:10pt)   --  (45:5pt) -- 
(90:10pt)  -- (135:5pt) -- 
(180:10pt) -- (225:5pt) --
(270:10pt) -- (315:5pt) -- cycle 
(0pt,0pt) circle (3pt) ;
 }}}


\pgfdeclareplotmark{m6bv}{%
\node[scale=\mSixscale*\AtoBscale*\symscale] at (0,0) {\tikz {%
\draw[fill,line width=0.2pt,even odd rule,rotate=45] %
(0:10pt)   --  (45:5pt) -- 
(90:10pt)  -- (135:5pt) -- 
(180:10pt) -- (225:5pt) --
(270:10pt) -- (315:5pt) -- cycle 
(0pt,0pt) circle (3pt) ;
\draw (0,0) circle (10pt);
%\draw[line width=0.6pt,even odd rule,rotate=45] %
%(0:13pt)   --  (45:7pt) -- 
%(90:13pt)  -- (135:7pt) -- 
%(180:13pt) -- (225:7pt) --
%(270:13pt) -- (315:7pt) -- cycle ;
 }}}

\pgfdeclareplotmark{m6bb}{%
\node[scale=\mSixscale*\AtoBscale*\symscale] at (0,0) {\tikz {%
\draw[fill,line width=0.2pt,even odd rule,rotate=45] %
(0:10pt)   --  (45:5pt) -- 
(90:10pt)  -- (135:5pt) -- 
(180:10pt) -- (225:5pt) --
(270:10pt) -- (315:5pt) -- cycle 
(0pt,0pt) circle (3pt) ;
\draw[line width=1.2pt,even odd rule,rotate=45] %
(45:9pt)   --  (45:5pt) 
(135:9pt)  -- (135:5pt) 
(225:9pt) -- (225:5pt) 
(315:9pt) -- (315:5pt) ;
 }}}

\pgfdeclareplotmark{m6bvb}{%
\node[scale=\mSixscale*\AtoBscale*\symscale] at (0,0) {\tikz {%
\draw[fill,line width=0.2pt,even odd rule,rotate=45] %
(0:10pt)   --  (45:5pt) -- 
(90:10pt)  -- (135:5pt) -- 
(180:10pt) -- (225:5pt) --
(270:10pt) -- (315:5pt) -- cycle 
(0pt,0pt) circle (3pt) ;
\draw (0,0) circle (10pt);
%\draw[line width=0.6pt,even odd rule,rotate=45] %
%(0:13pt)   --  (45:7pt) -- 
%(90:13pt)  -- (135:7pt) -- 
%(180:13pt) -- (225:7pt) --
%(270:13pt) -- (315:7pt) -- cycle ;
\draw[line width=1.2pt,even odd rule,rotate=45] %
(45:11pt)   --  (45:7pt) 
(135:11pt)  -- (135:7pt) 
(225:11pt)  -- (225:7pt) 
(315:11pt)  -- (315:7pt) ;
 }}}


\pgfdeclareplotmark{m6c}{%
\node[scale=\mSixscale*\AtoCscale*\symscale] at (0,0) {\tikz {%
\draw[fill,line width=0.2pt,even odd rule,rotate=90] %
(0pt,0pt) circle (5pt) ;
 }}}


\pgfdeclareplotmark{m6cv}{%
\node[scale=\mSixscale*\AtoCscale*\symscale] at (0,0) {\tikz {%
\draw[fill,line width=0.2pt,even odd rule,rotate=90] %
(0pt,0pt) circle (5pt) ;
\draw[line width=0.6pt,even odd rule,rotate=90] %
(0pt,0pt) circle (7pt) ;
 }}}

\pgfdeclareplotmark{m6cb}{%
\node[scale=\mSixscale*\AtoCscale*\symscale] at (0,0) {\tikz {%
\draw[fill,line width=0.2pt,even odd rule,rotate=90] %
(0pt,0pt) circle (5pt) ;
\draw[line width=2pt,even odd rule] %
(-8pt,0pt) -- (8pt,0pt) ;
 }}}

\pgfdeclareplotmark{m6cvb}{%
\node[scale=\mSixscale*\AtoCscale*\symscale] at (0,0) {\tikz {%
\draw[fill,line width=0.2pt,even odd rule,rotate=90] %
(0pt,0pt) circle (5pt) ;
\draw[line width=0.6pt,even odd rule,rotate=90] %
(0pt,0pt) circle (7pt) ;
\draw[line width=2pt,even odd rule] %
(-9pt,0pt) -- (-7pt,0pt) 
(9pt,0pt) -- (7pt,0pt) ;
 }}}




%%%%%%%%%%%%%%%%%%%%%%%%%%%%%%%%%%%%%%%%%%%%%%%%%%%%%%%%%%%%%%%%%%%%%%%%%%%%%%%%%%%%%%%%%%
%%%%%%%%%%%%%%%%%%%%%%%%%%%%%%%%%%%%%%%%%%%%%%%%%%%%%%%%%%%%%%%%%%%%%%%%%%%%%%%%%%%%%%%%%%
%   Nebulae markers
%%%%%%%%%%%%%%%%%%%%%%%%%%%%%%%%%%%%%%%%%%%%%%%%%%%%%%%%%%%%%%%%%%%%%%%%%%%%%%%%%%%%%%%%%%
%%%%%%%%%%%%%%%%%%%%%%%%%%%%%%%%%%%%%%%%%%%%%%%%%%%%%%%%%%%%%%%%%%%%%%%%%%%%%%%%%%%%%%%%%%


\pgfdeclarepatternformonly{dense crosshatch dots}{\pgfqpoint{-1pt}{-1pt}}{\pgfqpoint{.6pt}{.6pt}}{\pgfqpoint{.8pt}{.8pt}}%
{
    \pgfpathcircle{\pgfqpoint{0pt}{0pt}}{.3pt}
    \pgfpathcircle{\pgfqpoint{1pt}{1pt}}{.3pt}
    \pgfusepath{fill}
}

% Open Cluster
%\pgfdeclareplotmark{OC}{%
%\node at (0,0) {\tikz { 
%\draw[scale=3,draw opacity=0,line width=.2pt,pattern=dense crosshatch dots] %
%(0:1pt) -- (30:0.75pt)  -- (60:1pt) -- (90:0.75pt)  -- (120:1pt) --  (150:0.75pt)  -- (180:1pt) --
%(210:0.75pt) -- (240:1pt)  -- (270:0.75pt) -- (300:1pt) -- (330:0.75pt)  -- (0:1pt) ;
%};};   
%}

\pgfdeclareplotmark{OC}{%
\node at (0,0) {\tikz {%
%\clip (0:1pt) -- (30:0.75pt)  -- (60:1pt) -- (90:0.75pt)  -- (120:1pt) --  (150:0.75pt)  -- (180:1pt) --
%(210:0.75pt) -- (240:1pt)  -- (270:0.75pt) -- (300:1pt) -- (330:0.75pt)  -- (0:1pt) ;
\filldraw[scale=3,draw opacity=1,line width=.2pt]
(0,0) circle (.1pt)
(0:.4pt) circle (.1pt) (60:.4pt) circle (.1pt) (120:.4pt) circle (.1pt)
(180:.4pt) circle (.1pt) (240:.4pt) circle (.1pt) (300:.4pt) circle (.1pt)
%
(30:.75pt) circle (.1pt) (90:.75pt) circle (.1pt) (150:.75pt) circle (.1pt)
(210:.75pt) circle (.1pt) (270:.75pt) circle (.1pt) (330:.75pt) circle (.1pt)
;
}}}

% Globular Cluster
\pgfdeclareplotmark{GC}{%
\node at (0,0) {\tikz { 
\filldraw[scale=3,draw opacity=1,line width=.2pt]
(0,0) circle (.1pt)
(0:.4pt) circle (.1pt) (60:.4pt) circle (.1pt) (120:.4pt) circle (.1pt)
(180:.4pt) circle (.1pt) (240:.4pt) circle (.1pt) (300:.4pt) circle (.1pt)
%
(30:.75pt) circle (.1pt) (90:.75pt) circle (.1pt) (150:.75pt) circle (.1pt)
(210:.75pt) circle (.1pt) (270:.75pt) circle (.1pt) (330:.75pt) circle (.1pt)
;
\draw[scale=3,line width=.3pt] %
circle (1pt) ;
}}}

% Emission Nebula
\pgfdeclareplotmark{EN}{%
\node[opacity=0] at (0,0) {\tikz { 
\clip (0,0) circle (2.5pt);
\draw[opacity=1,line width=.3pt] %
(-4pt,-1pt) -- (4pt,7pt) 
(-4pt,-2.5pt) -- (4pt,5.5pt)  
(-4pt,-4pt) -- (4pt,4pt) 
(-4pt,-5.5pt) -- (4pt,2.5pt) 
(-4pt,-7pt) -- (4pt,1pt) 
;
}}}

% Nebuluous Cluster
\pgfdeclareplotmark{CN}{%
\node[opacity=0] at (0,0) {\tikz { 
\clip (0,0) circle (2.5pt);
\draw[opacity=1,line width=.3pt] %
(-4pt,-1pt) -- (4pt,7pt) 
(-4pt,-2.5pt) -- (4pt,5.5pt)  
(-4pt,-4pt) -- (4pt,4pt) 
(-4pt,-5.5pt) -- (4pt,2.5pt) 
(-4pt,-7pt) -- (4pt,1pt) ;
}}  ;
\node at (0,0) {\tikz { 
\filldraw[scale=3,draw opacity=1,line width=.2pt]
(0,0) circle (.1pt)
(0:.4pt) circle (.1pt) (60:.4pt) circle (.1pt) (120:.4pt) circle (.1pt)
(180:.4pt) circle (.1pt) (240:.4pt) circle (.1pt) (300:.4pt) circle (.1pt)
%
(30:.75pt) circle (.1pt) (90:.75pt) circle (.1pt) (150:.75pt) circle (.1pt)
(210:.75pt) circle (.1pt) (270:.75pt) circle (.1pt) (330:.75pt) circle (.1pt)
;
}}}


% Planetary Nebula
\pgfdeclareplotmark{PN}{%
\node at (0,0) {\tikz { 
\draw[line width=.3pt] %
circle (2.7pt);
\clip (0,0) circle (2.7pt);
\draw[opacity=1,line width=.3pt] %
(-4pt,-1pt) -- (4pt,7pt) 
(-4pt,-2.5pt) -- (4pt,5.5pt)  
(-4pt,-4pt) -- (4pt,4pt) 
(-4pt,-5.5pt) -- (4pt,2.5pt) 
(-4pt,-7pt) -- (4pt,1pt) 
;
\draw[line width=.3pt,fill=white] %
circle (1.2pt) ;
}}}

% Galaxies
\pgfdeclareplotmark{GAL}{%
\node [rotate=-15] at (0,0) {\tikz {
\draw (0,0) ellipse (4pt and 1.5pt);
\clip (0,0) ellipse (4pt and 1.5pt);
\draw[opacity=1,line width=.3pt] %
(-4pt,-1pt) -- (4pt,7pt) 
(-4pt,-2.5pt) -- (4pt,5.5pt)  
(-4pt,-4pt) -- (4pt,4pt) 
(-4pt,-5.5pt) -- (4pt,2.5pt) 
(-4pt,-7pt) -- (4pt,1pt) 
;
}}}  ;


 


% 
% Colors
\definecolor{Burgundy}{rgb}{.75,.25,.25}
\definecolor{IndianRed4}{HTML}{8B3A3A}
\definecolor{IndianRed3}{HTML}{CD5555}
\definecolor{IndianRed2}{HTML}{EE6363}
\definecolor{IndianRed1}{HTML}{FF6A6A}
\definecolor{GoldenRod1}{HTML}{FFC125}
\definecolor{Orchid}{rgb}{0.85, 0.44, 0.84}
\definecolor{Mauve}{HTML}{B784A7}
\colorlet{cConstellation}{IndianRed4}
%\colorlet{cConstellation}{burgundy}
\definecolor{ForestGreen}{rgb}{0.13, 0.55, 0.13}

\colorlet{cFrame}{black}
\colorlet{cStars}{black}
\colorlet{cAxes}{black}


\colorlet{cProper}{IndianRed2}
\colorlet{cBayer}{IndianRed3}
\colorlet{cFlamsteed}{IndianRed4}

\colorlet{cTransient}{ForestGreen}

\colorlet{cNGC}{Orchid}
\colorlet{cMessier}{Orchid}

%\colorlet{cNGC}{Mauve}
%\colorlet{cMessier}{Mauve}

%\colorlet{cNGC}{IndianRed4}
%\colorlet{cMessier}{IndianRed4}

%\colorlet{cNGC}{green!80!black}
%\colorlet{cMessier}{blue}

\colorlet{cEcliptic}{GoldenRod1!80!black}
\colorlet{cEquator}{gray}

\colorlet{cGrid}{black}

\definecolor{DeepSkyBlue}{HTML}{00BFFF}
% Milky Way  color
\colorlet{cMW}{DeepSkyBlue}
\colorlet{cMW0}{white}
\colorlet{cMW1}{DeepSkyBlue!10}
\colorlet{cMW2}{DeepSkyBlue!14}
\colorlet{cMW3}{DeepSkyBlue!18}
\colorlet{cMW4}{DeepSkyBlue!22}
\colorlet{cMW5}{DeepSkyBlue!26}

%\colorlet{cMW1}{DeepSkyBlue!15}
%\colorlet{cMW2}{DeepSkyBlue!30}
%\colorlet{cMW3}{DeepSkyBlue!45}
%\colorlet{cMW4}{DeepSkyBlue!60}
%\colorlet{cMW5}{DeepSkyBlue!75}

%\colorlet{cMW6}{DeepSkyBlue!40}
%\colorlet{cMW7}{DeepSkyBlue!50}
%\colorlet{cMW8}{DeepSkyBlue!60}
\colorlet{cMW9}{DeepSkyBlue!70}

\def\symscale {1.5}

\def\AtoBscale {0.9}
\def\AtoCscale {0.8}

\def\mOnescale   {0.48}
\def\mTwoscale   {0.38}
\def\mThreescale {0.3}
\def\mFourscale  {0.23}
\def\mFivescale  {0.15}
\def\mSixscale   {0.1}


% Styles:
\pgfplotsset{MW0/.style={mark=,color=cMW0,fill=cMW0,opacity=1}}
\pgfplotsset{MW1/.style={mark=,color=cMW1,fill=cMW1,opacity=1}}
\pgfplotsset{MW2/.style={mark=,color=cMW2,fill=cMW2,opacity=1}}
\pgfplotsset{MW3/.style={mark=,color=cMW3,fill=cMW3,opacity=1}}
\pgfplotsset{MW4/.style={mark=,color=cMW4,fill=cMW4,opacity=1}}
\pgfplotsset{MW5/.style={mark=,color=cMW5,fill=cMW5,opacity=1}}

%\pgfplotsset{MW6/.style={mark=,color=cMW6,opacity=1}}
%\pgfplotsset{MW7/.style={mark=,color=cMW7,opacity=1}}
%\pgfplotsset{MW8/.style={mark=,color=cMW8,opacity=1}}
%\pgfplotsset{MW9/.style={mark=,color=cMW9,opacity=1}}
\pgfplotsset{MWX/.style={mark=,opacity=1}}
\pgfplotsset{MWR/.style={mark=,color=red,opacity=1}}


% Coordinate grid overlay
\pgfplotsset{coordinategrid/.style={grid=major,major grid style={thin,color=cGrid}}}
\tikzset{coordinategrid/.style={thin,color=cGrid}}
% Ecliptics
\tikzset{ecliptics-full/.style={color=cEcliptic,line width=.4pt,mark=,fill=cEcliptic,opacity=1}}
\tikzset{ecliptics-empty/.style={color=cEcliptic,line width=.4pt,mark=,fill=,fill opacity=0}}
\tikzset{ecliptics-label/.style={color=black,font=\scriptsize}}
% Galactic Equator
\tikzset{MWE-full/.style={color=cMW,line width=.4pt,mark=,fill=cMW,opacity=1}}
\tikzset{MWE-empty/.style={color=cMW,line width=.4pt,mark=,fill=,fill opacity=0}}
\tikzset{MWE-label/.style={color=black,font=\scriptsize}}
% Equator
\tikzset{Equator-full/.style={color=cEquator,line width=.4pt,mark=,fill=cEquator,opacity=1}}
\tikzset{Equator-empty/.style={color=cEquator,line width=.4pt,mark=,fill=,fill opacity=0}}
\tikzset{Equator-label/.style={color=black,font=\scriptsize}}
% Transients
\tikzset{transient-label/.style={color=cTransient,font=\bfseries\scriptsize}}
\tikzset{transient/.style={color=cTransient,line width=1pt,mark=,opacity=1}}

% To enable consistent input scripting:
\newcommand\omicron{o}

% Scale of panels:
\newlength{\tendegree}
\setlength{\tendegree}{3cm}
\newlength{\onedegree}
\setlength{\onedegree}{0.1\tendegree}

% General style definitions
\pgfkeys{/pgf/number format/.cd,fixed,precision=4}


% For RA tickmark labeling
\newcommand{\fh}{$^{\rm h}$}
\newcommand{\fm}{$^{\rm m}$}



% Style for constellation boundaries and names:
\pgfplotsset{constellation/.style={thin,mark=,color=cConstellation,dashed}}
\tikzset{constellation-label/.style={color=cConstellation!80,font=\bfseries}}
\tikzset{constellation-boundary/.style={thin,mark=,color=cConstellation,dashed}}


\tikzset{designation-label/.style={color=cProper,font=\bfseries\scriptsize}}
 
% Style for Star markers and names
\tikzset{stars/.style={color=black}}

\tikzset{Proper/.style={color=cProper,font=\bfseries\footnotesize},pin edge={draw opacity=0}}
\tikzset{Bayer/.style={color=cBayer,font=\bfseries\scriptsize},pin edge={draw opacity=0}}
\tikzset{Flaamsted/.style={color=cFlamsteed,font=\bfseries\tiny},pin edge={draw opacity=0}}
% Intersting objects label
\tikzset{interest/.style={color=cProper}}
\tikzset{interest-label/.style={color=cProper,font=\bfseries\tiny}}

% Style for Nebulae markers and names
\tikzset{NGC/.style={color=cNGC,pattern color=cNGC}}
\tikzset{NGC-label/.style={color=cNGC,font=\bfseries\tiny},pin edge={draw opacity=0}}

\tikzset{Caldwell/.style={color=cMessier,pattern color=cMessier}}
\tikzset{Caldwell-label/.style={color=cMessier,font=\bfseries\scriptsize},pin edge={draw opacity=0}}

\tikzset{Messier/.style={color=cMessier,pattern color=cMessier}}
\tikzset{Messier-label/.style={color=cMessier,font=\bfseries\scriptsize},pin edge={draw opacity=0}}
% An extra one for the Messier galaxies in the Virgo-group 
\tikzset{Messier-label-crowded/.style={pin edge={draw opacity=1,color=cMessier,thin},color=cMessier,font=\bfseries\footnotesize}}




% General style definitions
\pgfkeys{/pgf/number format/.cd,fixed,precision=4}

\pgfplotsset{tick label style={font=\normalsize}, label style={font=\large}, legend style={font=\large}
  ,every axis/.append style={scale only axis},scaled ticks=false
  % A3paper has size 297 × 420mm
   ,width=14\tendegree,height=14\tendegree,x dir=reverse
%  ,/pgf/number format/set thousands separator={\,}
%
%    ,ytick={-90,-80,...,90},minor y tick num=0
%  ,yticklabels={$-90^\circ$,$-80^\circ$,$-70^\circ$,$-60^\circ$,$-50^\circ$,$-40^\circ$,$-30^\circ$,$-20^\circ$,$-10^\circ$,$0^\circ$,
%    $+10^\circ$,$+20^\circ$,$+30^\circ$,$+40^\circ$,$+50^\circ$,$+60^\circ$,$+70^\circ$,$+80^\circ$,$+90^\circ$}
%  
}



% RA tickmark labels in hours
\pgfplotsset{RA_in_hours/.style={,xtick={0,10,...,360},minor x tick num=0
      ,xticklabels={0\fh{\small 00}\fm,0\fh{\small 40}\fm,1\fh{\small 20}\fm,2\fh{\small
      00}\fm,2\fh{\small 40}\fm,3\fh{\small 20}\fm,4\fh{\small 00}\fm,4\fh{\small
      40}\fm,5\fh{\small 20}\fm,6\fh{\small 00}\fm,6\fh{\small 40}\fm,7\fh{\small
      20}\fm,8\fh{\small 00}\fm,8\fh{\small 40}\fm,9\fh{\small 20}\fm,10\fh{\small
      00}\fm,10\fh{\small 40}\fm,11\fh{\small 20}\fm,12\fh{\small 00}\fm,12\fh{\small
      40}\fm,13\fh{\small 20}\fm,14\fh{\small 00}\fm,14\fh{\small 40}\fm,15\fh{\small
      20}\fm,16\fh{\small 00}\fm,16\fh{\small 40}\fm,17\fh{\small 20}\fm,18\fh{\small
      00}\fm,18\fh{\small 40}\fm,19\fh{\small 20}\fm,20\fh{\small 00}\fm,20\fh{\small
      40}\fm,21\fh{\small 20}\fm,22\fh{\small 00}\fm,22\fh{\small 40}\fm,23\fh{\small
      20}\fm,0\fh{\small 00}\fm,0\fh{\small 40}\fm,1\fh{\small 20}\fm}}}

% RA tickmark labels in degree
\pgfplotsset{RA_in_deg/.style={,xtick={0,10,...,360},minor x tick num=0,x dir=reverse,
      tick label style={font=\footnotesize},xticklabels={$0^\circ$,$10^\circ$,$20^\circ$,$30^\circ$, 
      $40^\circ$,$50^\circ$,$60^\circ$,$70^\circ$,$80^\circ$,$90^\circ$,$100^\circ$,$110^\circ$,$120^\circ$,
      $130^\circ$,$140^\circ$,$150^\circ$,$160^\circ$,$170^\circ$,$180^\circ$,$190^\circ$,$200^\circ$,
      $210^\circ$,$220^\circ$,$230^\circ$,$240^\circ$,$250^\circ$,$260^\circ$,$270^\circ$,$280^\circ$,
      $290^\circ$,$300^\circ$,$310^\circ$,$320^\circ$,$330^\circ$,$340^\circ$,$350^\circ$}  
}} 
% RA tickmark labels in degree
\pgfplotsset{RA_in_deg_pole/.style={,xtick={0,30,...,360},minor x tick num=0,x dir=reverse
}} 

\center

\pgfplotsset{xmin=0,xmax=360,ymin=-90,ymax=-2.5
,y coord trafo/.code=\pgfmathparse{-90-#1}
  ,y coord inv trafo/.code=\pgfmathparse{-90-#1}
,x coord trafo/.code=\pgfmathparse{0-#1}
,x coord inv trafo/.code=\pgfmathparse{0-#1}
}

\tikzsetnextfilename{TabulaVIII}


\begin{tikzpicture}

\begin{polaraxis}[rotate=270,name=base,axis y line=none,axis x line=none
]\end{polaraxis}



% The Milky Way first, as it should not obstruct coordinate systems 


\begin{polaraxis}[rotate=270,name=constellations,at={($(base.center)+(+0.75pt,0pt)$)},anchor=center,axis lines=none]

\clip (6\tendegree,0\tendegree) arc (0:90:6\tendegree) -- 
(-2\tendegree,6\tendegree) -- (-2\tendegree,-6\tendegree) -- (0\tendegree,-6\tendegree)
--  (0\tendegree,-6\tendegree) arc (270:359.9999:6\tendegree) -- cycle ;

\addplot[MW1] table[x index=0,y index=1] {./MW/VIIaVIII_MWbase.dat}  -- cycle ;
\addplot[MW0] table[x index=0,y index=1] {./MW/VIIaVIII_ol1_3.dat}  -- cycle ;
\addplot[MW0] table[x index=0,y index=1] {./MW/VIIaVIII_ol1_4.dat}  -- cycle ;
\addplot[MW0] table[x index=0,y index=1] {./MW/VIIaVIII_ol1_5.dat}  -- cycle ;
\addplot[MW0] table[x index=0,y index=1] {./MW/VIIaVIII_ol1_6.dat}  -- cycle ;
\addplot[MW0] table[x index=0,y index=1] {./MW/VIIaVIII_ol1_7.dat}  -- cycle ;
\addplot[MW0] table[x index=0,y index=1] {./MW/VIIaVIII_ol1_8.dat}  -- cycle ;
\addplot[MW0] table[x index=0,y index=1] {./MW/VIIaVIII_ol1_9.dat}  -- cycle ;
\addplot[MW0] table[x index=0,y index=1] {./MW/VIIaVIII_ol1_10.dat}  -- cycle ;

\addplot[MW2] table[x index=0,y index=1] {./MW/VIIaVIII_ol2_1.dat}  -- cycle ;
\addplot[MW2] table[x index=0,y index=1] {./MW/VIIaVIII_ol2_3.dat}  -- cycle ;
\addplot[MW2] table[x index=0,y index=1] {./MW/VIIaVIII_ol2_4.dat}  -- cycle ;
\addplot[MW2] table[x index=0,y index=1] {./MW/VIIaVIII_ol2_5.dat}  -- cycle ;
\addplot[MW2] table[x index=0,y index=1] {./MW/VIIaVIII_ol2_6.dat}  -- cycle ;
\addplot[MW2] table[x index=0,y index=1] {./MW/VIIaVIII_ol2_7.dat}  -- cycle ;
\addplot[MW2] table[x index=0,y index=1] {./MW/VIIaVIII_ol2_8.dat}  -- cycle ;
\addplot[MW2] table[x index=0,y index=1] {./MW/VIIaVIII_ol2_9.dat}  -- cycle ;
\addplot[MW2] table[x index=0,y index=1] {./MW/VIIaVIII_ol2_10.dat}  -- cycle ;
\addplot[MW2] table[x index=0,y index=1] {./MW/VIIaVIII_ol2_11.dat}  -- cycle ;
\addplot[MW2] table[x index=0,y index=1] {./MW/VIIaVIII_ol2_13.dat}  -- cycle ;
\addplot[MW2] table[x index=0,y index=1] {./MW/VIIaVIII_ol2_14.dat}  -- cycle ;
\addplot[MW2] table[x index=0,y index=1] {./MW/VIIaVIII_ol2_15.dat}  -- cycle ;
\addplot[MW2] table[x index=0,y index=1] {./MW/VIIaVIII_ol2_17.dat}  -- cycle ;
\addplot[MW2] table[x index=0,y index=1] {./MW/VIIaVIII_ol2_18.dat}  -- cycle ;
\addplot[MW2] table[x index=0,y index=1] {./MW/VIIaVIII_ol2_20.dat}  -- cycle ;
\addplot[MW2] table[x index=0,y index=1] {./MW/VIIaVIII_ol2_21.dat}  -- cycle ;
\addplot[MW2] table[x index=0,y index=1] {./MW/VIIaVIII_ol2_22.dat}  -- cycle ;
\addplot[MW2] table[x index=0,y index=1] {./MW/VIIaVIII_ol2_23.dat}  -- cycle ;
\addplot[MW2] table[x index=0,y index=1] {./MW/VIIaVIII_ol2_24.dat}  -- cycle ;
\addplot[MW2] table[x index=0,y index=1] {./MW/VIIaVIII_ol2_25.dat}  -- cycle ;
\addplot[MW2] table[x index=0,y index=1] {./MW/VIIaVIII_ol2_26.dat}  -- cycle ;
\addplot[MW2] table[x index=0,y index=1] {./MW/VIIaVIII_ol2_29.dat}  -- cycle ;
\addplot[MW2] table[x index=0,y index=1] {./MW/VIIaVIII_ol2_33.dat}  -- cycle ;
\addplot[MW2] table[x index=0,y index=1] {./MW/VIIaVIII_ol2_34.dat}  -- cycle ;
\addplot[MW2] table[x index=0,y index=1] {./MW/VIIaVIII_ol2_36.dat}  -- cycle ;
\addplot[MW2] table[x index=0,y index=1] {./MW/VIIaVIII_ol2_37.dat}  -- cycle ;
\addplot[MW2] table[x index=0,y index=1] {./MW/VIIaVIII_ol2_39.dat}  -- cycle ;
\addplot[MW2] table[x index=0,y index=1] {./MW/VIIaVIII_ol2_40.dat}  -- cycle ;
\addplot[MW2] table[x index=0,y index=1] {./MW/VIIaVIII_ol2_41.dat}  -- cycle ;
\addplot[MW2] table[x index=0,y index=1] {./MW/VIIaVIII_ol2_42.dat}  -- cycle ;
\addplot[MW2] table[x index=0,y index=1] {./MW/VIIaVIII_ol2_43.dat}  -- cycle ;
\addplot[MW2] table[x index=0,y index=1] {./MW/VIIaVIII_ol2_46.dat}  -- cycle ;
\addplot[MW2] table[x index=0,y index=1] {./MW/VIIaVIII_ol2_47.dat}  -- cycle ;
\addplot[MW2] table[x index=0,y index=1] {./MW/VIIaVIII_ol2_48.dat}  -- cycle ;
\addplot[MW2] table[x index=0,y index=1] {./MW/VIIaVIII_ol2_49.dat}  -- cycle ;
\addplot[MW2] table[x index=0,y index=1] {./MW/VIIaVIII_ol2_50.dat}  -- cycle ;
\addplot[MW2] table[x index=0,y index=1] {./MW/VIIaVIII_ol2_51.dat}  -- cycle ;
\addplot[MW2] table[x index=0,y index=1] {./MW/VIIaVIII_ol2_52.dat}  -- cycle ;
\addplot[MW2] table[x index=0,y index=1] {./MW/VIIaVIII_ol2_53.dat}  -- cycle ;
\addplot[MW2] table[x index=0,y index=1] {./MW/VIIaVIII_ol2_55.dat}  -- cycle ;
\addplot[MW2] table[x index=0,y index=1] {./MW/VIIaVIII_ol2_56.dat}  -- cycle ;
\addplot[MW2] table[x index=0,y index=1] {./MW/VIIaVIII_ol2_57.dat}  -- cycle ;
\addplot[MW2] table[x index=0,y index=1] {./MW/VIIaVIII_ol2_58.dat}  -- cycle ;
\addplot[MW2] table[x index=0,y index=1] {./MW/VIIaVIII_ol2_59.dat}  -- cycle ;
\addplot[MW2] table[x index=0,y index=1] {./MW/VIIaVIII_ol2_60.dat}  -- cycle ;
\addplot[MW2] table[x index=0,y index=1] {./MW/VIIaVIII_ol2_61.dat}  -- cycle ;
\addplot[MW2] table[x index=0,y index=1] {./MW/VIIaVIII_ol2_62.dat}  -- cycle ;
\addplot[MW2] table[x index=0,y index=1] {./MW/VIIaVIII_ol2_64.dat}  -- cycle ;
\addplot[MW2] table[x index=0,y index=1] {./MW/VIIaVIII_ol2_65.dat}  -- cycle ;
\addplot[MW2] table[x index=0,y index=1] {./MW/VIIaVIII_ol2_68.dat}  -- cycle ;
\addplot[MW2] table[x index=0,y index=1] {./MW/VIIaVIII_ol2_69.dat}  -- cycle ;
\addplot[MW2] table[x index=0,y index=1] {./MW/VIIaVIII_ol2_70.dat}  -- cycle ;
\addplot[MW2] table[x index=0,y index=1] {./MW/VIIaVIII_ol2_71.dat}  -- cycle ;
\addplot[MW2] table[x index=0,y index=1] {./MW/VIIaVIII_ol2_72.dat}  -- cycle ;
\addplot[MW2] table[x index=0,y index=1] {./MW/VIIaVIII_ol2_73.dat}  -- cycle ;
\addplot[MW2] table[x index=0,y index=1] {./MW/VIIaVIII_ol2_75.dat}  -- cycle ;
\addplot[MW2] table[x index=0,y index=1] {./MW/VIIaVIII_ol2_77.dat}  -- cycle ;
\addplot[MW2] table[x index=0,y index=1] {./MW/VIIaVIII_ol2_79.dat}  -- cycle ;
\addplot[MW2] table[x index=0,y index=1] {./MW/VIIaVIII_ol2_80.dat}  -- cycle ;
\addplot[MW2] table[x index=0,y index=1] {./MW/VIIaVIII_ol2_81.dat}  -- cycle ;
\addplot[MW2] table[x index=0,y index=1] {./MW/VIIaVIII_ol2_83.dat}  -- cycle ;
\addplot[MW2] table[x index=0,y index=1] {./MW/VIIaVIII_ol2_84.dat}  -- cycle ;


\addplot[MW2] table[x index=0,y index=1] {./MW/VIIaVIII_ol2_85.dat}  -- cycle ;
\addplot[MW2] table[x index=0,y index=1] {./MW/VIIaVIII_ol2_86.dat}  -- cycle ;
\addplot[MW2] table[x index=0,y index=1] {./MW/VIIaVIII_ol2_89.dat}  -- cycle ;
\addplot[MW2] table[x index=0,y index=1] {./MW/VIIaVIII_ol2_91.dat}  -- cycle ;
\addplot[MW2] table[x index=0,y index=1] {./MW/VIIaVIII_ol2_92.dat}  -- cycle ;
\addplot[MW2] table[x index=0,y index=1] {./MW/VIIaVIII_ol2_93.dat}  -- cycle ;
\addplot[MW2] table[x index=0,y index=1] {./MW/VIIaVIII_ol2_94.dat}  -- cycle ;
\addplot[MW2] table[x index=0,y index=1] {./MW/VIIaVIII_ol2_96.dat}  -- cycle ;
\addplot[MW2] table[x index=0,y index=1] {./MW/VIIaVIII_ol2_97.dat}  -- cycle ;
\addplot[MW2] table[x index=0,y index=1] {./MW/VIIaVIII_ol2_98.dat}  -- cycle ;
\addplot[MW2] table[x index=0,y index=1] {./MW/VIIaVIII_ol2_99.dat}  -- cycle ;
\addplot[MW2] table[x index=0,y index=1] {./MW/VIIaVIII_ol2_100.dat}  -- cycle ;
\addplot[MW2] table[x index=0,y index=1] {./MW/VIIaVIII_ol2_101.dat}  -- cycle ;
\addplot[MW2] table[x index=0,y index=1] {./MW/VIIaVIII_ol2_102.dat}  -- cycle ;
\addplot[MW2] table[x index=0,y index=1] {./MW/VIIaVIII_ol2_103.dat}  -- cycle ;
\addplot[MW2] table[x index=0,y index=1] {./MW/VIIaVIII_ol2_105.dat}  -- cycle ;
\addplot[MW2] table[x index=0,y index=1] {./MW/VIIaVIII_ol2_106.dat}  -- cycle ;
\addplot[MW2] table[x index=0,y index=1] {./MW/VIIaVIII_ol2_107.dat}  -- cycle ;
\addplot[MW2] table[x index=0,y index=1] {./MW/VIIaVIII_ol2_111.dat}  -- cycle ;


\addplot[MW1] table[x index=0,y index=1] {./MW/VIIaVIII_ol2_110.dat}  -- cycle ;
\addplot[MW1] table[x index=0,y index=1] {./MW/VIIaVIII_ol2_113.dat}  -- cycle ;
\addplot[MW1] table[x index=0,y index=1] {./MW/VIIaVIII_ol2_63.dat}  -- cycle ;
\addplot[MW1] table[x index=0,y index=1] {./MW/VIIaVIII_ol2_88.dat}  -- cycle ;

\addplot[MW3] table[x index=0,y index=1] {./MW/VIIaVIII_ol3_1.dat}  -- cycle ;
\addplot[MW3] table[x index=0,y index=1] {./MW/VIIaVIII_ol3_2.dat}  -- cycle ;
\addplot[MW3] table[x index=0,y index=1] {./MW/VIIaVIII_ol3_3.dat}  -- cycle ;
\addplot[MW3] table[x index=0,y index=1] {./MW/VIIaVIII_ol3_5.dat}  -- cycle ;
\addplot[MW3] table[x index=0,y index=1] {./MW/VIIaVIII_ol3_6.dat}  -- cycle ;
\addplot[MW3] table[x index=0,y index=1] {./MW/VIIaVIII_ol3_7.dat}  -- cycle ;
\addplot[MW3] table[x index=0,y index=1] {./MW/VIIaVIII_ol3_8.dat}  -- cycle ;
\addplot[MW3] table[x index=0,y index=1] {./MW/VIIaVIII_ol3_9.dat}  -- cycle ;
\addplot[MW3] table[x index=0,y index=1] {./MW/VIIaVIII_ol3_11.dat}  -- cycle ;
\addplot[MW3] table[x index=0,y index=1] {./MW/VIIaVIII_ol3_12.dat}  -- cycle ;
\addplot[MW3] table[x index=0,y index=1] {./MW/VIIaVIII_ol3_13.dat}  -- cycle ;
\addplot[MW3] table[x index=0,y index=1] {./MW/VIIaVIII_ol3_14.dat}  -- cycle ;
\addplot[MW3] table[x index=0,y index=1] {./MW/VIIaVIII_ol3_15.dat}  -- cycle ;
\addplot[MW3] table[x index=0,y index=1] {./MW/VIIaVIII_ol3_16.dat}  -- cycle ;
\addplot[MW3] table[x index=0,y index=1] {./MW/VIIaVIII_ol3_17.dat}  -- cycle ;
\addplot[MW3] table[x index=0,y index=1] {./MW/VIIaVIII_ol3_18.dat}  -- cycle ;
\addplot[MW3] table[x index=0,y index=1] {./MW/VIIaVIII_ol3_19.dat}  -- cycle ;
\addplot[MW3] table[x index=0,y index=1] {./MW/VIIaVIII_ol3_20.dat}  -- cycle ;
\addplot[MW3] table[x index=0,y index=1] {./MW/VIIaVIII_ol3_21.dat}  -- cycle ;
\addplot[MW3] table[x index=0,y index=1] {./MW/VIIaVIII_ol3_22.dat}  -- cycle ;
\addplot[MW3] table[x index=0,y index=1] {./MW/VIIaVIII_ol3_23.dat}  -- cycle ;
\addplot[MW3] table[x index=0,y index=1] {./MW/VIIaVIII_ol3_24.dat}  -- cycle ;
\addplot[MW3] table[x index=0,y index=1] {./MW/VIIaVIII_ol3_25.dat}  -- cycle ;
\addplot[MW3] table[x index=0,y index=1] {./MW/VIIaVIII_ol3_26.dat}  -- cycle ;
\addplot[MW3] table[x index=0,y index=1] {./MW/VIIaVIII_ol3_27.dat}  -- cycle ;
\addplot[MW3] table[x index=0,y index=1] {./MW/VIIaVIII_ol3_28.dat}  -- cycle ;
\addplot[MW3] table[x index=0,y index=1] {./MW/VIIaVIII_ol3_29.dat}  -- cycle ;
\addplot[MW3] table[x index=0,y index=1] {./MW/VIIaVIII_ol3_30.dat}  -- cycle ;
\addplot[MW3] table[x index=0,y index=1] {./MW/VIIaVIII_ol3_31.dat}  -- cycle ;
\addplot[MW3] table[x index=0,y index=1] {./MW/VIIaVIII_ol3_32.dat}  -- cycle ;
\addplot[MW3] table[x index=0,y index=1] {./MW/VIIaVIII_ol3_33.dat}  -- cycle ;
\addplot[MW3] table[x index=0,y index=1] {./MW/VIIaVIII_ol3_34.dat}  -- cycle ;
\addplot[MW3] table[x index=0,y index=1] {./MW/VIIaVIII_ol3_35.dat}  -- cycle ;
\addplot[MW3] table[x index=0,y index=1] {./MW/VIIaVIII_ol3_37.dat}  -- cycle ;
\addplot[MW3] table[x index=0,y index=1] {./MW/VIIaVIII_ol3_38.dat}  -- cycle ;
\addplot[MW3] table[x index=0,y index=1] {./MW/VIIaVIII_ol3_39.dat}  -- cycle ;
\addplot[MW3] table[x index=0,y index=1] {./MW/VIIaVIII_ol3_40.dat}  -- cycle ;
\addplot[MW3] table[x index=0,y index=1] {./MW/VIIaVIII_ol3_41.dat}  -- cycle ;
\addplot[MW3] table[x index=0,y index=1] {./MW/VIIaVIII_ol3_42.dat}  -- cycle ;
\addplot[MW3] table[x index=0,y index=1] {./MW/VIIaVIII_ol3_43.dat}  -- cycle ;
\addplot[MW3] table[x index=0,y index=1] {./MW/VIIaVIII_ol3_44.dat}  -- cycle ;
\addplot[MW3] table[x index=0,y index=1] {./MW/VIIaVIII_ol3_45.dat}  -- cycle ;

\addplot[MW2] table[x index=0,y index=1] {./MW/VIIaVIII_ol3_46.dat}  -- cycle ;

\addplot[MW4] table[x index=0,y index=1] {./MW/VIIaVIII_ol4_1.dat}  -- cycle ;
\addplot[MW4] table[x index=0,y index=1] {./MW/VIIaVIII_ol4_2.dat}  -- cycle ;
\addplot[MW4] table[x index=0,y index=1] {./MW/VIIaVIII_ol4_3.dat}  -- cycle ;
\addplot[MW4] table[x index=0,y index=1] {./MW/VIIaVIII_ol4_4.dat}  -- cycle ;
\addplot[MW4] table[x index=0,y index=1] {./MW/VIIaVIII_ol4_5.dat}  -- cycle ;
\addplot[MW4] table[x index=0,y index=1] {./MW/VIIaVIII_ol4_6.dat}  -- cycle ;
\addplot[MW4] table[x index=0,y index=1] {./MW/VIIaVIII_ol4_7.dat}  -- cycle ;
\addplot[MW4] table[x index=0,y index=1] {./MW/VIIaVIII_ol4_8.dat}  -- cycle ;
\addplot[MW4] table[x index=0,y index=1] {./MW/VIIaVIII_ol4_9.dat}  -- cycle ;
\addplot[MW4] table[x index=0,y index=1] {./MW/VIIaVIII_ol4_10.dat}  -- cycle ;
\addplot[MW4] table[x index=0,y index=1] {./MW/VIIaVIII_ol4_11.dat}  -- cycle ;
\addplot[MW4] table[x index=0,y index=1] {./MW/VIIaVIII_ol4_12.dat}  -- cycle ;
\addplot[MW4] table[x index=0,y index=1] {./MW/VIIaVIII_ol4_13.dat}  -- cycle ;
\addplot[MW4] table[x index=0,y index=1] {./MW/VIIaVIII_ol4_17.dat}  -- cycle ;
\addplot[MW4] table[x index=0,y index=1] {./MW/VIIaVIII_ol4_18.dat}  -- cycle ;
\addplot[MW4] table[x index=0,y index=1] {./MW/VIIaVIII_ol4_19.dat}  -- cycle ;
\addplot[MW4] table[x index=0,y index=1] {./MW/VIIaVIII_ol4_21.dat}  -- cycle ;
\addplot[MW4] table[x index=0,y index=1] {./MW/VIIaVIII_ol4_22.dat}  -- cycle ;
\addplot[MW4] table[x index=0,y index=1] {./MW/VIIaVIII_ol4_23.dat}  -- cycle ;
\addplot[MW4] table[x index=0,y index=1] {./MW/VIIaVIII_ol4_24.dat}  -- cycle ;
\addplot[MW4] table[x index=0,y index=1] {./MW/VIIaVIII_ol4_25.dat}  -- cycle ;
\addplot[MW4] table[x index=0,y index=1] {./MW/VIIaVIII_ol4_26.dat}  -- cycle ;
\addplot[MW4] table[x index=0,y index=1] {./MW/VIIaVIII_ol4_27.dat}  -- cycle ;


\addplot[MW5] table[x index=0,y index=1] {./MW/VIIaVIII_ol5_1.dat}  -- cycle ;
\addplot[MW5] table[x index=0,y index=1] {./MW/VIIaVIII_ol5_2.dat}  -- cycle ;
\addplot[MW5] table[x index=0,y index=1] {./MW/VIIaVIII_ol5_3.dat}  -- cycle ;
\addplot[MW5] table[x index=0,y index=1] {./MW/VIIaVIII_ol5_4.dat}  -- cycle ;
\addplot[MW5] table[x index=0,y index=1] {./MW/VIIaVIII_ol5_5.dat}  -- cycle ;
\addplot[MW5] table[x index=0,y index=1] {./MW/VIIaVIII_ol5_6.dat}  -- cycle ;

\addplot[MW1]  table[x index=0,y index=1] {./MW/SMC.dat}  -- cycle ;

\addplot[MW1]  table[x index=0,y index=1] {./MW/LMC.dat}  -- cycle ;

\end{polaraxis}




\begin{polaraxis}[rotate=270,at={($(base.center)+(1pt,0pt)$)},anchor=center,y axis line style= { draw opacity=0 },
    axis x line=none,y tick label style= { opacity=0 },y tick style= { opacity=0 },grid=none]

  \begin{scope}
\clip (6\tendegree,0\tendegree) arc (0:90:6\tendegree) -- 
(-2\tendegree,6\tendegree) -- (-2\tendegree,-6\tendegree) -- (0\tendegree,-6\tendegree)
--  (0\tendegree,-6\tendegree) arc (270:359.9999:6\tendegree) -- cycle ;


\draw[coordinategrid] (axis cs:0,-90) circle (1\tendegree);
\draw[coordinategrid] (axis cs:0,-90) circle (2\tendegree);
\draw[coordinategrid] (axis cs:0,-90) circle (3\tendegree);
\draw[coordinategrid] (axis cs:0,-90) circle (4\tendegree);
\draw[coordinategrid] (axis cs:0,-90) circle (5\tendegree);
\draw[coordinategrid] (axis cs:0,-90) circle (6\tendegree);

\draw[coordinategrid] (axis cs:0,-90) --  (axis cs:00,-20) ;
\draw[coordinategrid] (axis cs:0,-90) --  (axis cs:30,-20) ;
\draw[coordinategrid] (axis cs:0,-90) --  (axis cs:60,-20) ;
\draw[coordinategrid] (axis cs:0,-90) --  (axis cs:90,-20) ;
\draw[coordinategrid] (axis cs:0,-90) --  (axis cs:120,-20) ;
\draw[coordinategrid] (axis cs:0,-90) --  (axis cs:150,-20) ;
\draw[coordinategrid] (axis cs:0,-90) --  (axis cs:180,-20) ;
\draw[coordinategrid] (axis cs:0,-90) --  (axis cs:210,-20) ;
\draw[coordinategrid] (axis cs:0,-90) --  (axis cs:240,-20) ;
\draw[coordinategrid] (axis cs:0,-90) --  (axis cs:270,-20) ;
\draw[coordinategrid] (axis cs:0,-90) --  (axis cs:300,-20) ;
\draw[coordinategrid] (axis cs:0,-90) --  (axis cs:330,-20) ;

\draw[coordinategrid] (axis cs:10,-80) --  (axis cs:10,-20) ;
\draw[coordinategrid] (axis cs:20,-80) --  (axis cs:20,-20) ;
\draw[coordinategrid] (axis cs:40,-80) --  (axis cs:40,-20) ;
\draw[coordinategrid] (axis cs:50,-80) --  (axis cs:50,-20) ;
\draw[coordinategrid] (axis cs:70,-80) --  (axis cs:70,-20) ;
\draw[coordinategrid] (axis cs:80,-80) --  (axis cs:80,-20) ;

\draw[coordinategrid] (axis cs:100,-80) --  (axis cs:100,-20) ;
\draw[coordinategrid] (axis cs:110,-80) --  (axis cs:110,-20) ;
\draw[coordinategrid] (axis cs:130,-80) --  (axis cs:130,-20) ;
\draw[coordinategrid] (axis cs:140,-80) --  (axis cs:140,-20) ;
\draw[coordinategrid] (axis cs:160,-80) --  (axis cs:160,-20) ;
\draw[coordinategrid] (axis cs:170,-80) --  (axis cs:170,-20) ;

\draw[coordinategrid] (axis cs:190,-80) --  (axis cs:190,-20) ;
\draw[coordinategrid] (axis cs:200,-80) --  (axis cs:200,-20) ;
\draw[coordinategrid] (axis cs:220,-80) --  (axis cs:220,-20) ;
\draw[coordinategrid] (axis cs:230,-80) --  (axis cs:230,-20) ;
\draw[coordinategrid] (axis cs:250,-80) --  (axis cs:250,-20) ;
\draw[coordinategrid] (axis cs:260,-80) --  (axis cs:260,-20) ;
                                                      
\draw[coordinategrid] (axis cs:280,-80) --  (axis cs:280,-20) ;
\draw[coordinategrid] (axis cs:290,-80) --  (axis cs:290,-20) ;
\draw[coordinategrid] (axis cs:310,-80) --  (axis cs:310,-20) ;
\draw[coordinategrid] (axis cs:320,-80) --  (axis cs:320,-20) ;
\draw[coordinategrid] (axis cs:340,-80) --  (axis cs:340,-20) ;
\draw[coordinategrid] (axis cs:350,-80) --  (axis cs:350,-20) ;

\node[pin={[pin distance=-0.9\onedegree,Equator-label]45:{$-30^\circ$}}] at (axis cs:180,-30) {} ;
\node[pin={[pin distance=-0.9\onedegree,Equator-label]45:{$-40^\circ$}}] at (axis cs:180,-40) {} ;
\node[pin={[pin distance=-0.9\onedegree,Equator-label]135:{$-50^\circ$}}] at (axis cs:180,-50) {} ;
\node[pin={[pin distance=-0.9\onedegree,Equator-label]135:{$-60^\circ$}}] at (axis cs:180,-60) {} ;
\node[pin={[pin distance=-0.9\onedegree,Equator-label]45:{$-70^\circ$}}] at (axis cs:180,-70) {} ;
\node[pin={[pin distance=-0.9\onedegree,Equator-label]45:{$-80^\circ$}}] at (axis cs:180,-80) {} ;

\node[pin={[pin distance=-0.8\onedegree,Equator-label]-45:{$-80^\circ$}}] at (axis cs:0,-80) {} ;
\node[pin={[pin distance=-0.9\onedegree,Equator-label]-45:{$-70^\circ$}}] at (axis cs:0,-70) {} ;
\end{scope}



\end{polaraxis}


\begin{polaraxis}[rotate=270,name=axis1,at=(base.center),anchor=center,color=cAxes,axis
    lines=none,ymin=-90,ymax=-8.8  
    ,width=13\tendegree,height=13\tendegree
  ,x dir = reverse]


%\draw (axis cs:90,0) arc (90:270:6\tendegree) ;


\draw (6\tendegree,0\tendegree) arc (0:90:6\tendegree) -- 
(-2\tendegree,6\tendegree) -- (-2\tendegree,-6\tendegree) -- (0\tendegree,-6\tendegree)
--  (0\tendegree,-6\tendegree) arc (270:359.9999:6\tendegree) -- cycle ;


\node[rotate=-90]  at (axis cs:90,-28.5)  {6\fh{\small 00}\fm};
\node[rotate=-80]  at (axis cs:100,-28.5) {6\fh{\small 40}\fm};
\node[rotate=-70]  at (axis cs:110,-28.5) {7\fh{\small 20}\fm};
\node[rotate=-60]  at (axis cs:120,-28.5) {8\fh{\small 00}\fm};
\node[rotate=-50]  at (axis cs:130,-28.5) {8\fh{\small 40}\fm};
\node[rotate=-40]  at (axis cs:140,-28.5) {9\fh{\small 20}\fm};
\node[rotate=-30]  at (axis cs:150,-28.5) {10\fh{\small 00}\fm};
\node[rotate=-20]  at (axis cs:160,-28.5) {10\fh{\small 40}\fm};
\node[rotate=-10]  at (axis cs:170,-28.5) {11\fh{\small 20}\fm};
\node[rotate=-00]  at (axis cs:180,-28.5) {12\fh{\small 00}\fm};
\node[rotate=10] at (axis cs:190,-28.5) {12\fh{\small 40}\fm};
\node[rotate=20] at (axis cs:200,-28.5) {13\fh{\small 20}\fm};
\node[rotate=30] at (axis cs:210,-28.5) {14\fh{\small 00}\fm};
\node[rotate=40] at (axis cs:220,-28.5) {14\fh{\small 40}\fm};
\node[rotate=50] at (axis cs:230,-28.5) {15\fh{\small 20}\fm};
\node[rotate=60] at (axis cs:240,-28.5) {16\fh{\small 00}\fm};
\node[rotate=70] at (axis cs:250,-28.5) {16\fh{\small 40}\fm};
\node[rotate=80] at (axis cs:260,-28.5) {17\fh{\small 20}\fm};
\node[rotate=90] at (axis cs:270,-28.5) {18\fh{\small 00}\fm};

\node[rotate=-90]  at (axis cs:90,-29.4)  {\footnotesize$90^\circ$};
\node[rotate=-80]  at (axis cs:100,-29.4) {\footnotesize$100^\circ$};
\node[rotate=-70]  at (axis cs:110,-29.4) {\footnotesize$110^\circ$};
\node[rotate=-60]  at (axis cs:120,-29.4) {\footnotesize$120^\circ$};
\node[rotate=-50]  at (axis cs:130,-29.4) {\footnotesize$130^\circ$};
\node[rotate=-40]  at (axis cs:140,-29.4) {\footnotesize$140^\circ$};
\node[rotate=-30]  at (axis cs:150,-29.4) {\footnotesize$150^\circ$};
\node[rotate=-20]  at (axis cs:160,-29.4) {\footnotesize$160^\circ$};
\node[rotate=-10]  at (axis cs:170,-29.4) {\footnotesize$170^\circ$};
\node[rotate=00]  at (axis cs:180,-29.4) {\footnotesize$180^\circ$};
\node[rotate=10] at (axis cs:190,-29.4) {\footnotesize$190^\circ$};
\node[rotate=20] at (axis cs:200,-29.4) {\footnotesize$200^\circ$};
\node[rotate=30] at (axis cs:210,-29.4) {\footnotesize$210^\circ$};
\node[rotate=40] at (axis cs:220,-29.4) {\footnotesize$220^\circ$};
\node[rotate=50] at (axis cs:230,-29.4) {\footnotesize$230^\circ$};
\node[rotate=60] at (axis cs:240,-29.4) {\footnotesize$240^\circ$};
\node[rotate=70] at (axis cs:250,-29.4) {\footnotesize$250^\circ$};
\node[rotate=80] at (axis cs:260,-29.4) {\footnotesize$260^\circ$};
\node[rotate=90] at (axis cs:270,-29.4) {\footnotesize$270^\circ$};

\node[rotate=90] at (axis cs:280,-28.5) {\footnotesize$280^\circ$};
\node[rotate=90] at (axis cs:280,-27.6) {18\fh{\small 40}\fm};

\node[rotate=-90] at (axis cs:80,-28.5) {\footnotesize$80^\circ$};
\node[rotate=-90] at (axis cs:80,-27.6) {5\fh{\small 20}\fm};




\node at (-2.18\tendegree,5.494957\tendegree)  {19\fh{\small 20}\fm};
\node at (-2.18\tendegree,3.464103\tendegree)  {20\fh{\small 00}\fm};
\node at (-2.18\tendegree,2.383508\tendegree)  {20\fh{\small 40}\fm};
\node at (-2.18\tendegree,1.678200\tendegree)  {21\fh{\small 20}\fm};
\node at (-2.18\tendegree,1.154701\tendegree)  {22\fh{\small 00}\fm};
\node at (-2.18\tendegree,0.7279406\tendegree) {22\fh{\small 40}\fm};
\node at (-2.18\tendegree,0.352\tendegree)     {23\fh{\small 20}\fm};
\node at (-2.18\tendegree,0.0\tendegree)       {0\fh{\small 00}\fm};
\node at (-2.18\tendegree,-0.352\tendegree)    {0\fh{\small 40}\fm};
\node at (-2.18\tendegree,-0.7279406\tendegree){1\fh{\small 20}\fm};
\node at (-2.18\tendegree,-1.154701\tendegree) {2\fh{\small 00}\fm};
\node at (-2.18\tendegree,-1.678200\tendegree) {2\fh{\small 40}\fm};
\node at (-2.18\tendegree,-2.383508\tendegree) {3\fh{\small 20}\fm};
\node at (-2.18\tendegree,-3.464103\tendegree) {4\fh{\small 00}\fm};
\node at (-2.18\tendegree,-5.494957\tendegree) {4\fh{\small 40}\fm};

\node at (-2.06\tendegree,5.494957\tendegree)   {\footnotesize$290^\circ$};
\node at (-2.06\tendegree,3.464103\tendegree)   {\footnotesize$300^\circ$};
\node at (-2.06\tendegree,2.383508\tendegree)  {\footnotesize$310^\circ$}; 
\node at (-2.06\tendegree,1.678200\tendegree)   {\footnotesize$320^\circ$};
\node at (-2.06\tendegree,1.154701\tendegree)   {\footnotesize$330^\circ$};
\node at (-2.06\tendegree,0.7279406\tendegree){\footnotesize$340^\circ$}; 
\node at (-2.06\tendegree,0.352\tendegree)      {\footnotesize$350^\circ$};
\node at (-2.06\tendegree,0.0\tendegree) {\footnotesize$0^\circ$};
\node at (-2.06\tendegree,-0.352\tendegree)    {\footnotesize$10^\circ$};
\node at (-2.06\tendegree,-0.7279406\tendegree) {\footnotesize$20^\circ$};
\node at (-2.06\tendegree,-1.154701\tendegree) {\footnotesize$30^\circ$};
\node at (-2.06\tendegree,-1.678200\tendegree) {\footnotesize$40^\circ$};
\node at (-2.06\tendegree,-2.383508\tendegree)  {\footnotesize$50^\circ$};
\node at (-2.06\tendegree,-3.464103\tendegree) {\footnotesize$60^\circ$};
\node at (-2.06\tendegree,-5.494957\tendegree) {\footnotesize$70^\circ$};



\end{polaraxis}


% Left and bottom axis with gridlines
%  \begin{axis}[at=(base.center),anchor=center,RA_in_hours,color=cAxes] \end{axis}
% Right and top axis, labelling in RA hours (offset)
%  \begin{axis}[at=(base.center),anchor=center,yticklabel pos=right,xticklabel pos=right,RA_in_hours,xticklabel style={yshift=2.5ex, anchor=south},color=cAxes]\end{axis}
% Top axis labelling in RA degree (small offset)
%  \begin{axis}[at=(base.center),anchor=center,axis y line=none,xticklabel pos=top,RA_in_deg     ,xticklabel style={font=\footnotesize},xticklabel style={yshift=0.5ex, anchor=south} ,color=cAxes]\end{axis}



  \begin{axis}[name=frame,at={($(base.center)+(-.376cm,-19.5\onedegree)$)},anchor=center,width=12.43\tendegree,height=8.53\tendegree,ticks=none
  ,yticklabels={},xticklabels={},axis line style={line width=1pt},color=cFrame] \end{axis}

  \begin{axis}[name=frame2,at={($(frame.center)+(0cm,+0cm)$)},anchor=center,width=12.53\tendegree,height=8.63\tendegree,ticks=none
  ,yticklabels={},xticklabels={},axis line style={line width=3pt},color=cFrame] \end{axis}
  


\begin{polaraxis}[rotate=270,at=(base.center),anchor=center,y axis line style= { draw opacity=0 },
    axis x line=none,y tick label style= { opacity=0 },y tick style= {opacity=0 }
,grid=none,width=13\tendegree,height=13\tendegree]

\clip (-0\tendegree,-8\tendegree) --  (8\tendegree,-8\tendegree) --
(8\tendegree,8\tendegree) --  (-0\tendegree,8\tendegree) -- cycle ;

\draw[line width=0.6\onedegree,color=white] (0,0) circle (6.245\tendegree);

\draw[line width=1pt] (0,0) circle (6.215\tendegree);
\draw[line width=3pt] (0,0) circle (6.265\tendegree);

  \end{polaraxis}



 
 \begin{axis}[name=legend,at={($(frame2.south)+(2cm,-0cm)$)},anchor=north,axis y line=none,axis x
   line=none,xmin=-55,xmax=85,height=.75\tendegree,width=12.53\tendegree   ,ymin=-3,ymax=3,  ,y dir=normal,x dir=reverse, color=cAxes]

\node at (axis cs:54,+1.0) {\small\it 1};                  \node at (axis cs:54,0.3)  {\tikz\pgfuseplotmark{m1b};};           
\node at (axis cs:52,+1.0) {\small\it 1{\nicefrac{1}{3}}}; \node at (axis cs:52,0.3)  {\tikz\pgfuseplotmark{m1c};};           
\node at (axis cs:53,-0.5)  {\small First};                  

\node at (axis cs:48,+1.0) {\small\it 1{\nicefrac{2}{3}}}; \node at (axis cs:48,0.3)  {\tikz\pgfuseplotmark{m2a};};           
\node at (axis cs:46,+1.0) {\small\it 2};                  \node at (axis cs:46,0.3)  {\tikz\pgfuseplotmark{m2b};};           
\node at (axis cs:44,+1.0) {\small\it 2{\nicefrac{1}{3}}}; \node at (axis cs:44,0.3)  {\tikz\pgfuseplotmark{m2c};};           
\node at (axis cs:46,-0.5)  {\small Second};

\node at (axis cs:40,+1.0) {\small\it 2{\nicefrac{2}{3}}}; \node at (axis cs:40,0.3)  {\tikz\pgfuseplotmark{m3a};};           
\node at (axis cs:38,+1.0) {\small\it 3};                  \node at (axis cs:38,0.3)  {\tikz\pgfuseplotmark{m3b};};           
\node at (axis cs:36,+1.0) {\small\it 3{\nicefrac{1}{3}}}; \node at (axis cs:36,0.3)  {\tikz\pgfuseplotmark{m3c};};           
\node at (axis cs:38,-0.5)  {\small Third};

\node at (axis cs:32,+1.0) {\small\it 3{\nicefrac{2}{3}}}; \node at (axis cs:32,0.3)  {\tikz\pgfuseplotmark{m4a};};           
\node at (axis cs:30,+1.0) {\small\it 4};                  \node at (axis cs:30,0.3)  {\tikz\pgfuseplotmark{m4b};};           
\node at (axis cs:28,+1.0) {\small\it 4{\nicefrac{1}{3}}}; \node at (axis cs:28,0.3)  {\tikz\pgfuseplotmark{m4c};};           
\node at (axis cs:30,-0.5)  {\small Fourth};

\node at (axis cs:24,+1.0) {\small\it 4{\nicefrac{2}{3}}}; \node at (axis cs:24,0.3)  {\tikz\pgfuseplotmark{m5a};};           
\node at (axis cs:22,+1.0) {\small\it 5};                  \node at (axis cs:22,0.3)  {\tikz\pgfuseplotmark{m5b};};           
\node at (axis cs:20,+1.0) {\small\it 5{\nicefrac{1}{3}}}; \node at (axis cs:20,0.3)  {\tikz\pgfuseplotmark{m5c};};           
\node at (axis cs:22,-0.5)  {\small Fifth};

\node at (axis cs:16,+1.0) {\small\it 5{\nicefrac{2}{3}}}; \node at (axis cs:16,0.3)  {\tikz\pgfuseplotmark{m6a};};           
\node at (axis cs:14,+1.0) {\small\it 6};                  \node at (axis cs:14,0.3)  {\tikz\pgfuseplotmark{m6b};};           
\node at (axis cs:12,+1.0) {\small\it 6{\nicefrac{1}{3}}}; \node at (axis cs:12,0.3)  {\tikz\pgfuseplotmark{m6c};};           
\node at (axis cs:14,-0.5)  {\small Sixth};


\node at  (axis cs:6,+1.0)   {\small\it Variable};  \node at  (axis cs:6,+0.3)   {\tikz\pgfuseplotmark{m2av};};           
\node at  (axis cs:2,+.992)    {\small\it Binary};   \node at  (axis cs:2,+0.3)   {\tikz\pgfuseplotmark{m2ab};};           
\node at  (axis cs:-2,+1.0)   {\small\it Var. \& Bin.};  \node at  (axis cs:-2,+0.3)   {\tikz\pgfuseplotmark{m2avb};};           


\node at  (axis cs:-17,+.992)   {\small\it Planetary};  \node at  (axis cs:-18,+0.3)   {\tikz\pgfuseplotmark{PN};};           
\node at  (axis cs:-22,+1.0)   {\small\it Emission};   \node at  (axis cs:-22,+0.3)   {\tikz\pgfuseplotmark{EN};};           
\node at  (axis cs:-27,+1.0)   {\small\it w/ Cluster};  \node at  (axis cs:-26,+0.3)   {\tikz\pgfuseplotmark{CN};};           
\node at (axis cs:-22,-0.5)  {\small Nebulae};


\node at  (axis cs:-32,+.994)   {\small\it Open};      \node at  (axis cs:-32,+0.3)   {\tikz\pgfuseplotmark{OC};};           
\node at  (axis cs:-37,+1.0)   {\small\it Globular};   \node at  (axis cs:-37,+0.3)   {\tikz\pgfuseplotmark{GC};};           
\node at (axis cs:-34.5,-0.5)  {\small Cluster};

\node at  (axis cs:-42,+.994)   {\small\it Galaxy};      \node at  (axis cs:-42,+0.3)   {\tikz\pgfuseplotmark{GAL};};         

%
\end{axis}
         


% The Galactic coordinate system
%
% Some are commented out because they are surrounded by already plotted borders
% The x-coordinate is in fractional hours, so must be times 15
%

\begin{polaraxis}[rotate=270,name=MWcoord,at={($(base.center)+(+0.75pt,0pt)$)},anchor=center,axis lines=none]

\clip (6\tendegree,0\tendegree) arc (0:90:6\tendegree) -- 
(-2\tendegree,6\tendegree) -- (-2\tendegree,-6\tendegree) -- (0\tendegree,-6\tendegree)
--  (0\tendegree,-6\tendegree) arc (270:359.9999:6\tendegree) -- cycle ;

\node[MWE-empty,pin={[pin distance=-0.4\onedegree,MWE-label]090:{NGP}}] at (axis cs:12*15+51.42/4,27+07.8/60) {\pgfuseplotmark{+}} ;
\node[MWE-empty,pin={[pin distance=-0.4\onedegree,MWE-label]000:{SGP}}] at (axis cs:12*15+51.42/4-180,-27+07.8/60) {\pgfuseplotmark{+}} ;

\draw[MWE-full] (axis cs:2.66308E+02,-2.88819E+01) -- (axis cs:2.66899E+02,-2.80275E+01) -- (axis cs:2.67093E+02,-2.81309E+01) -- (axis cs:2.66503E+02,-2.89861E+01) -- cycle ; 
\draw[MWE-full] (axis cs:2.67481E+02,-2.71706E+01) -- (axis cs:2.68054E+02,-2.63113E+01) -- (axis cs:2.68246E+02,-2.64131E+01) -- (axis cs:2.67674E+02,-2.72732E+01) -- cycle ; 
\draw[MWE-full] (axis cs:2.68618E+02,-2.54498E+01) -- (axis cs:2.69174E+02,-2.45861E+01) -- (axis cs:2.69365E+02,-2.46865E+01) -- (axis cs:2.68809E+02,-2.55508E+01) -- cycle ; 
\draw[MWE-full] (axis cs:2.69723E+02,-2.37204E+01) -- (axis cs:2.70264E+02,-2.28528E+01) -- (axis cs:2.70453E+02,-2.29518E+01) -- (axis cs:2.69912E+02,-2.38201E+01) -- cycle ; 
\draw[MWE-full] (axis cs:2.70799E+02,-2.19834E+01) -- (axis cs:2.71327E+02,-2.11122E+01) -- (axis cs:2.71514E+02,-2.12100E+01) -- (axis cs:2.70986E+02,-2.20817E+01) -- cycle ; 
\draw[MWE-full] (axis cs:2.71848E+02,-2.02394E+01) -- (axis cs:2.72364E+02,-1.93650E+01) -- (axis cs:2.72550E+02,-1.94617E+01) -- (axis cs:2.72035E+02,-2.03366E+01) -- cycle ; 
\draw[MWE-full] (axis cs:2.72874E+02,-1.84892E+01) -- (axis cs:2.73379E+02,-1.76121E+01) -- (axis cs:2.73564E+02,-1.77078E+01) -- (axis cs:2.73059E+02,-1.85854E+01) -- cycle ; 
\draw[MWE-full] (axis cs:2.73879E+02,-1.67336E+01) -- (axis cs:2.74375E+02,-1.58540E+01) -- (axis cs:2.74558E+02,-1.59488E+01) -- (axis cs:2.74063E+02,-1.68289E+01) -- cycle ; 
\draw[MWE-full] (axis cs:2.74866E+02,-1.49732E+01) -- (axis cs:2.75353E+02,-1.40913E+01) -- (axis cs:2.75535E+02,-1.41854E+01) -- (axis cs:2.75049E+02,-1.50676E+01) -- cycle ; 
\draw[MWE-full] (axis cs:2.75837E+02,-1.32085E+01) -- (axis cs:2.76316E+02,-1.23248E+01) -- (axis cs:2.76498E+02,-1.24181E+01) -- (axis cs:2.76018E+02,-1.33022E+01) -- cycle ; 
\draw[MWE-full] (axis cs:2.76793E+02,-1.14402E+01) -- (axis cs:2.77267E+02,-1.05548E+01) -- (axis cs:2.77447E+02,-1.06476E+01) -- (axis cs:2.76974E+02,-1.15333E+01) -- cycle ; 
\draw[MWE-full] (axis cs:2.77738E+02,-9.66873E+00) -- (axis cs:2.78206E+02,-8.78200E+00) -- (axis cs:2.78385E+02,-8.87430E+00) -- (axis cs:2.77917E+02,-9.76126E+00) -- cycle ; 
\draw[MWE-full] (axis cs:2.78672E+02,-7.89468E+00) -- (axis cs:2.79136E+02,-7.00682E+00) -- (axis cs:2.79315E+02,-7.09872E+00) -- (axis cs:2.78851E+02,-7.98676E+00) -- cycle ; 
\draw[MWE-full] (axis cs:2.79598E+02,-6.11851E+00) -- (axis cs:2.80059E+02,-5.22978E+00) -- (axis cs:2.80237E+02,-5.32137E+00) -- (axis cs:2.79777E+02,-6.21024E+00) -- cycle ; 
\draw[MWE-full] (axis cs:2.80518E+02,-4.34070E+00) -- (axis cs:2.80977E+02,-3.45134E+00) -- (axis cs:2.81155E+02,-3.54271E+00) -- (axis cs:2.80697E+02,-4.43217E+00) -- cycle ; 
\draw[MWE-full] (axis cs:2.81434E+02,-2.56174E+00) -- (axis cs:2.81891E+02,-1.67196E+00) -- (axis cs:2.82069E+02,-1.76320E+00) -- (axis cs:2.81612E+02,-2.65303E+00) -- cycle ; 
\draw[MWE-full] (axis cs:2.82347E+02,-7.82062E-01) -- (axis cs:2.82917E+02,1.07899E-01) -- (axis cs:2.82981E+02,1.66967E-02) -- (axis cs:2.82525E+02,-8.73274E-01) -- cycle ; 
\draw[MWE-full] (axis cs:2.83259E+02,9.97867E-01) -- (axis cs:2.83715E+02,1.88779E+00) -- (axis cs:2.83893E+02,1.79654E+00) -- (axis cs:2.83437E+02,9.06652E-01) -- cycle ; 
\draw[MWE-full] (axis cs:2.84172E+02,2.77760E+00) -- (axis cs:2.84629E+02,3.66725E+00) -- (axis cs:2.84808E+02,3.57586E+00) -- (axis cs:2.84350E+02,2.68629E+00) -- cycle ; 
\draw[MWE-full] (axis cs:2.85088E+02,4.55668E+00) -- (axis cs:2.85547E+02,5.44583E+00) -- (axis cs:2.85725E+02,5.35422E+00) -- (axis cs:2.85266E+02,4.46519E+00) -- cycle ; 
\draw[MWE-full] (axis cs:2.86008E+02,6.33465E+00) -- (axis cs:2.86470E+02,7.22308E+00) -- (axis cs:2.86649E+02,7.13115E+00) -- (axis cs:2.86186E+02,6.24290E+00) -- cycle ; 
\draw[MWE-full] (axis cs:2.86934E+02,8.11105E+00) -- (axis cs:2.87400E+02,8.99851E+00) -- (axis cs:2.87580E+02,8.90618E+00) -- (axis cs:2.87113E+02,8.01893E+00) -- cycle ; 
\draw[MWE-full] (axis cs:2.87868E+02,9.88539E+00) -- (axis cs:2.88339E+02,1.07716E+01) -- (axis cs:2.88519E+02,1.06788E+01) -- (axis cs:2.88048E+02,9.79283E+00) -- cycle ; 
\draw[MWE-full] (axis cs:2.88813E+02,1.16572E+01) -- (axis cs:2.89289E+02,1.25420E+01) -- (axis cs:2.89470E+02,1.24486E+01) -- (axis cs:2.88993E+02,1.15641E+01) -- cycle ; 
\draw[MWE-full] (axis cs:2.89769E+02,1.34259E+01) -- (axis cs:2.90253E+02,1.43090E+01) -- (axis cs:2.90435E+02,1.42149E+01) -- (axis cs:2.89951E+02,1.33322E+01) -- cycle ; 
\draw[MWE-full] (axis cs:2.90740E+02,1.51910E+01) -- (axis cs:2.91231E+02,1.60721E+01) -- (axis cs:2.91414E+02,1.59772E+01) -- (axis cs:2.90922E+02,1.50966E+01) -- cycle ; 
\draw[MWE-full] (axis cs:2.91727E+02,1.69520E+01) -- (axis cs:2.92227E+02,1.78307E+01) -- (axis cs:2.92411E+02,1.77350E+01) -- (axis cs:2.91910E+02,1.68567E+01) -- cycle ; 
\draw[MWE-full] (axis cs:2.92732E+02,1.87082E+01) -- (axis cs:2.93242E+02,1.95843E+01) -- (axis cs:2.93428E+02,1.94875E+01) -- (axis cs:2.92917E+02,1.86119E+01) -- cycle ; 
\draw[MWE-full] (axis cs:2.93758E+02,2.04590E+01) -- (axis cs:2.94280E+02,2.13321E+01) -- (axis cs:2.94467E+02,2.12342E+01) -- (axis cs:2.93944E+02,2.03617E+01) -- cycle ; 
\draw[MWE-full] (axis cs:2.94808E+02,2.22037E+01) -- (axis cs:2.95342E+02,2.30735E+01) -- (axis cs:2.95531E+02,2.29744E+01) -- (axis cs:2.94996E+02,2.21052E+01) -- cycle ; 
\draw[MWE-full] (axis cs:2.95884E+02,2.39415E+01) -- (axis cs:2.96433E+02,2.48076E+01) -- (axis cs:2.96623E+02,2.47072E+01) -- (axis cs:2.96073E+02,2.38417E+01) -- cycle ; 
\draw[MWE-full] (axis cs:2.96989E+02,2.56717E+01) -- (axis cs:2.97554E+02,2.65337E+01) -- (axis cs:2.97746E+02,2.64318E+01) -- (axis cs:2.97180E+02,2.55705E+01) -- cycle ; 
\draw[MWE-full] (axis cs:2.98127E+02,2.73934E+01) -- (axis cs:2.98709E+02,2.82508E+01) -- (axis cs:2.98903E+02,2.81473E+01) -- (axis cs:2.98320E+02,2.72907E+01) -- cycle ; 
\draw[MWE-full] (axis cs:2.99300E+02,2.91056E+01) -- (axis cs:2.99901E+02,2.99579E+01) -- (axis cs:3.00098E+02,2.98527E+01) -- (axis cs:2.99495E+02,2.90013E+01) -- cycle ; 
\draw[MWE-full] (axis cs:3.00513E+02,3.08074E+01) -- (axis cs:3.01135E+02,3.16540E+01) -- (axis cs:3.01334E+02,3.15469E+01) -- (axis cs:3.00710E+02,3.07013E+01) -- cycle ; 
\draw[MWE-full] (axis cs:3.01769E+02,3.24975E+01) -- (axis cs:3.02415E+02,3.33378E+01) -- (axis cs:3.02616E+02,3.32287E+01) -- (axis cs:3.01969E+02,3.23894E+01) -- cycle ; 
\draw[MWE-full] (axis cs:3.03073E+02,3.41747E+01) -- (axis cs:3.03745E+02,3.50081E+01) -- (axis cs:3.03948E+02,3.48968E+01) -- (axis cs:3.03275E+02,3.40646E+01) -- cycle ; 
\draw[MWE-full] (axis cs:3.04430E+02,3.58377E+01) -- (axis cs:3.05130E+02,3.66633E+01) -- (axis cs:3.05335E+02,3.65497E+01) -- (axis cs:3.04634E+02,3.57252E+01) -- cycle ; 
\draw[MWE-full] (axis cs:3.05844E+02,3.74848E+01) -- (axis cs:3.06575E+02,3.83019E+01) -- (axis cs:3.06782E+02,3.81857E+01) -- (axis cs:3.06051E+02,3.73699E+01) -- cycle ; 
\draw[MWE-full] (axis cs:3.07322E+02,3.91143E+01) -- (axis cs:3.08087E+02,3.99220E+01) -- (axis cs:3.08296E+02,3.98031E+01) -- (axis cs:3.07531E+02,3.89969E+01) -- cycle ; 
\draw[MWE-full] (axis cs:3.08869E+02,4.07244E+01) -- (axis cs:3.09671E+02,4.15215E+01) -- (axis cs:3.09883E+02,4.13998E+01) -- (axis cs:3.09080E+02,4.06042E+01) -- cycle ; 
\draw[MWE-full] (axis cs:3.10492E+02,4.23130E+01) -- (axis cs:3.11335E+02,4.30984E+01) -- (axis cs:3.11548E+02,4.29736E+01) -- (axis cs:3.10705E+02,4.21897E+01) -- cycle ; 
\draw[MWE-full] (axis cs:3.12199E+02,4.38775E+01) -- (axis cs:3.13086E+02,4.46499E+01) -- (axis cs:3.13301E+02,4.45218E+01) -- (axis cs:3.12414E+02,4.37511E+01) -- cycle ; 
\draw[MWE-full] (axis cs:3.13996E+02,4.54153E+01) -- (axis cs:3.14932E+02,4.61733E+01) -- (axis cs:3.15149E+02,4.60418E+01) -- (axis cs:3.14213E+02,4.52855E+01) -- cycle ; 
\draw[MWE-full] (axis cs:3.15893E+02,4.69235E+01) -- (axis cs:3.16882E+02,4.76654E+01) -- (axis cs:3.17100E+02,4.75301E+01) -- (axis cs:3.16111E+02,4.67901E+01) -- cycle ; 
\draw[MWE-full] (axis cs:3.17899E+02,4.83986E+01) -- (axis cs:3.18945E+02,4.91226E+01) -- (axis cs:3.19165E+02,4.89834E+01) -- (axis cs:3.18118E+02,4.82614E+01) -- cycle ; 
\draw[MWE-full] (axis cs:3.20023E+02,4.98368E+01) -- (axis cs:3.21132E+02,5.05409E+01) -- (axis cs:3.21351E+02,5.03976E+01) -- (axis cs:3.20242E+02,4.96956E+01) -- cycle ; 
\draw[MWE-full] (axis cs:3.22274E+02,5.12341E+01) -- (axis cs:3.23452E+02,5.19158E+01) -- (axis cs:3.23670E+02,5.17682E+01) -- (axis cs:3.22493E+02,5.10886E+01) -- cycle ; 
\draw[MWE-full] (axis cs:3.24665E+02,5.25855E+01) -- (axis cs:3.25915E+02,5.32424E+01) -- (axis cs:3.26132E+02,5.30902E+01) -- (axis cs:3.24882E+02,5.24356E+01) -- cycle ; 
\draw[MWE-full] (axis cs:3.27204E+02,5.38858E+01) -- (axis cs:3.28533E+02,5.45151E+01) -- (axis cs:3.28746E+02,5.43582E+01) -- (axis cs:3.27419E+02,5.37314E+01) -- cycle ; 
\draw[MWE-full] (axis cs:3.29903E+02,5.51293E+01) -- (axis cs:3.31315E+02,5.57277E+01) -- (axis cs:3.31523E+02,5.55661E+01) -- (axis cs:3.30114E+02,5.49701E+01) -- cycle ; 
\draw[MWE-full] (axis cs:3.32770E+02,5.63095E+01) -- (axis cs:3.34269E+02,5.68737E+01) -- (axis cs:3.34472E+02,5.67072E+01) -- (axis cs:3.32976E+02,5.61454E+01) -- cycle ; 
\draw[MWE-full] (axis cs:3.35814E+02,5.74194E+01) -- (axis cs:3.37405E+02,5.79457E+01) -- (axis cs:3.37598E+02,5.77743E+01) -- (axis cs:3.36012E+02,5.72505E+01) -- cycle ; 
\draw[MWE-full] (axis cs:3.39042E+02,5.84516E+01) -- (axis cs:3.40725E+02,5.89360E+01) -- (axis cs:3.40908E+02,5.87597E+01) -- (axis cs:3.39230E+02,5.82777E+01) -- cycle ; 
\draw[MWE-full] (axis cs:3.42456E+02,5.93980E+01) -- (axis cs:3.44233E+02,5.98365E+01) -- (axis cs:3.44402E+02,5.96555E+01) -- (axis cs:3.42632E+02,5.92193E+01) -- cycle ; 
\draw[MWE-full] (axis cs:3.46056E+02,6.02505E+01) -- (axis cs:3.47924E+02,6.06388E+01) -- (axis cs:3.48077E+02,6.04533E+01) -- (axis cs:3.46217E+02,6.00672E+01) -- cycle ; 
\draw[MWE-full] (axis cs:3.49837E+02,6.10005E+01) -- (axis cs:3.51791E+02,6.13346E+01) -- (axis cs:3.51924E+02,6.11451E+01) -- (axis cs:3.49979E+02,6.08130E+01) -- cycle ; 
\draw[MWE-full] (axis cs:3.53786E+02,6.16401E+01) -- (axis cs:3.55819E+02,6.19160E+01) -- (axis cs:3.55929E+02,6.17229E+01) -- (axis cs:3.53908E+02,6.14487E+01) -- cycle ; 
\draw[MWE-full] (axis cs:3.57886E+02,6.21614E+01) -- (axis cs:3.59986E+02,6.23756E+01) -- (axis cs:7.05566E-02,6.21796E+01) -- (axis cs:3.57984E+02,6.19668E+01) -- cycle ; 
\draw[MWE-full] (axis cs:2.11273E+00,6.25578E+01) -- (axis cs:4.26373E+00,6.27073E+01) -- (axis cs:4.32138E+00,6.25091E+01) -- (axis cs:2.18430E+00,6.23606E+01) -- cycle ; 
\draw[MWE-full] (axis cs:6.43417E+00,6.28237E+01) -- (axis cs:8.61932E+00,6.29064E+01) -- (axis cs:8.64801E+00,6.27068E+01) -- (axis cs:6.47748E+00,6.26247E+01) -- cycle ; 
\draw[MWE-full] (axis cs:1.08143E+01,6.29552E+01) -- (axis cs:1.30140E+01,6.29699E+01) -- (axis cs:1.30130E+01,6.27699E+01) -- (axis cs:1.08282E+01,6.27553E+01) -- cycle ; 
\draw[MWE-full] (axis cs:1.52134E+01,6.29504E+01) -- (axis cs:1.74073E+01,6.28969E+01) -- (axis cs:1.73765E+01,6.26974E+01) -- (axis cs:1.51974E+01,6.27506E+01) -- cycle ; 
\draw[MWE-full] (axis cs:1.95907E+01,6.28094E+01) -- (axis cs:2.17586E+01,6.26884E+01) -- (axis cs:2.16990E+01,6.24903E+01) -- (axis cs:1.95453E+01,6.26105E+01) -- cycle ; 
\draw[MWE-full] (axis cs:2.39066E+01,6.25342E+01) -- (axis cs:2.60301E+01,6.23475E+01) -- (axis cs:2.59432E+01,6.21517E+01) -- (axis cs:2.38331E+01,6.23372E+01) -- cycle ; 
\draw[MWE-full] (axis cs:2.81251E+01,6.21289E+01) -- (axis cs:3.01880E+01,6.18792E+01) -- (axis cs:3.00762E+01,6.16863E+01) -- (axis cs:2.80255E+01,6.19345E+01) -- cycle ; 
\draw[MWE-full] (axis cs:3.22156E+01,6.15991E+01) -- (axis cs:3.42049E+01,6.12895E+01) -- (axis cs:3.40709E+01,6.11003E+01) -- (axis cs:3.20923E+01,6.14079E+01) -- cycle ; 
\draw[MWE-full] (axis cs:3.61537E+01,6.09515E+01) -- (axis cs:3.80599E+01,6.05860E+01) -- (axis cs:3.79064E+01,6.04008E+01) -- (axis cs:3.60096E+01,6.07642E+01) -- cycle ; 
\draw[MWE-full] (axis cs:3.99219E+01,6.01940E+01) -- (axis cs:4.17386E+01,5.97766E+01) -- (axis cs:4.15687E+01,5.95959E+01) -- (axis cs:3.97599E+01,6.00110E+01) -- cycle ; 
\draw[MWE-full] (axis cs:4.35091E+01,5.93347E+01) -- (axis cs:4.52331E+01,5.88695E+01) -- (axis cs:4.50497E+01,5.86936E+01) -- (axis cs:4.33322E+01,5.91564E+01) -- cycle ; 
\draw[MWE-full] (axis cs:4.69102E+01,5.83820E+01) -- (axis cs:4.85407E+01,5.78732E+01) -- (axis cs:4.83464E+01,5.77022E+01) -- (axis cs:4.67210E+01,5.82085E+01) -- cycle ; 
\draw[MWE-full] (axis cs:5.01248E+01,5.73442E+01) -- (axis cs:5.16632E+01,5.67958E+01) -- (axis cs:5.14604E+01,5.66296E+01) -- (axis cs:4.99260E+01,5.71755E+01) -- cycle ; 
\draw[MWE-full] (axis cs:5.31565E+01,5.62291E+01) -- (axis cs:5.46057E+01,5.56449E+01) -- (axis cs:5.43965E+01,5.54837E+01) -- (axis cs:5.29503E+01,5.60654E+01) -- cycle ; 
\draw[MWE-full] (axis cs:5.60117E+01,5.50442E+01) -- (axis cs:5.73757E+01,5.44279E+01) -- (axis cs:5.71620E+01,5.42714E+01) -- (axis cs:5.58001E+01,5.48854E+01) -- cycle ; 
\draw[MWE-full] (axis cs:5.86988E+01,5.37966E+01) -- (axis cs:5.99822E+01,5.31512E+01) -- (axis cs:5.97655E+01,5.29994E+01) -- (axis cs:5.84834E+01,5.36425E+01) -- cycle ; 
\draw[MWE-full] (axis cs:6.12273E+01,5.24925E+01) -- (axis cs:6.24353E+01,5.18211E+01) -- (axis cs:6.22168E+01,5.16738E+01) -- (axis cs:6.10096E+01,5.23429E+01) -- cycle ; 
\draw[MWE-full] (axis cs:6.36076E+01,5.11377E+01) -- (axis cs:6.47454E+01,5.04429E+01) -- (axis cs:6.45262E+01,5.02999E+01) -- (axis cs:6.33886E+01,5.09926E+01) -- cycle ; 
\draw[MWE-full] (axis cs:6.58501E+01,4.97374E+01) -- (axis cs:6.69230E+01,4.90218E+01) -- (axis cs:6.67038E+01,4.88829E+01) -- (axis cs:6.56308E+01,4.95965E+01) -- cycle ; 
\draw[MWE-full] (axis cs:6.79652E+01,4.82965E+01) -- (axis cs:6.89781E+01,4.75620E+01) -- (axis cs:6.87597E+01,4.74270E+01) -- (axis cs:6.77464E+01,4.81596E+01) -- cycle ; 
\draw[MWE-full] (axis cs:6.99629E+01,4.68189E+01) -- (axis cs:7.09207E+01,4.60676E+01) -- (axis cs:7.07035E+01,4.59363E+01) -- (axis cs:6.97450E+01,4.66858E+01) -- cycle ; 
\draw[MWE-full] (axis cs:7.18527E+01,4.53086E+01) -- (axis cs:7.27599E+01,4.45422E+01) -- (axis cs:7.25443E+01,4.44144E+01) -- (axis cs:7.16362E+01,4.51791E+01) -- cycle ; 
\draw[MWE-full] (axis cs:7.36436E+01,4.37688E+01) -- (axis cs:7.45045E+01,4.29888E+01) -- (axis cs:7.42908E+01,4.28642E+01) -- (axis cs:7.34289E+01,4.36426E+01) -- cycle ; 
\draw[MWE-full] (axis cs:7.53439E+01,4.22025E+01) -- (axis cs:7.61626E+01,4.14103E+01) -- (axis cs:7.59510E+01,4.12888E+01) -- (axis cs:7.51312E+01,4.20795E+01) -- cycle ; 
\draw[MWE-full] (axis cs:7.69615E+01,4.06124E+01) -- (axis cs:7.77415E+01,3.98092E+01) -- (axis cs:7.75321E+01,3.96906E+01) -- (axis cs:7.67510E+01,4.04924E+01) -- cycle ; 
\draw[MWE-full] (axis cs:7.85035E+01,3.90009E+01) -- (axis cs:7.92483E+01,3.81877E+01) -- (axis cs:7.90411E+01,3.80718E+01) -- (axis cs:7.82952E+01,3.88836E+01) -- cycle ; 
\draw[MWE-full] (axis cs:7.99766E+01,3.73700E+01) -- (axis cs:8.06892E+01,3.65480E+01) -- (axis cs:8.04843E+01,3.64345E+01) -- (axis cs:7.97705E+01,3.72554E+01) -- cycle ; 
\draw[MWE-full] (axis cs:8.13867E+01,3.57218E+01) -- (axis cs:8.20700E+01,3.48916E+01) -- (axis cs:8.18674E+01,3.47805E+01) -- (axis cs:8.11830E+01,3.56095E+01) -- cycle ; 
\draw[MWE-full] (axis cs:8.27396E+01,3.40578E+01) -- (axis cs:8.33961E+01,3.32204E+01) -- (axis cs:8.31957E+01,3.31114E+01) -- (axis cs:8.25381E+01,3.39478E+01) -- cycle ; 
\draw[MWE-full] (axis cs:8.40402E+01,3.23796E+01) -- (axis cs:8.46724E+01,3.15356E+01) -- (axis cs:8.44742E+01,3.14287E+01) -- (axis cs:8.38409E+01,3.22717E+01) -- cycle ; 
\draw[MWE-full] (axis cs:8.52933E+01,3.06886E+01) -- (axis cs:8.59034E+01,2.98387E+01) -- (axis cs:8.57074E+01,2.97337E+01) -- (axis cs:8.50962E+01,3.05826E+01) -- cycle ; 
\draw[MWE-full] (axis cs:8.65033E+01,2.89861E+01) -- (axis cs:8.70933E+01,2.81309E+01) -- (axis cs:8.68993E+01,2.80275E+01) -- (axis cs:8.63082E+01,2.88819E+01) -- cycle ; 
\draw[MWE-full] (axis cs:8.76740E+01,2.72732E+01) -- (axis cs:8.82459E+01,2.64131E+01) -- (axis cs:8.80537E+01,2.63113E+01) -- (axis cs:8.74810E+01,2.71706E+01) -- cycle ; 
\draw[MWE-full] (axis cs:8.88093E+01,2.55508E+01) -- (axis cs:8.93646E+01,2.46865E+01) -- (axis cs:8.91743E+01,2.45861E+01) -- (axis cs:8.86180E+01,2.54498E+01) -- cycle ; 
\draw[MWE-full] (axis cs:8.99124E+01,2.38201E+01) -- (axis cs:9.04529E+01,2.29518E+01) -- (axis cs:9.02642E+01,2.28528E+01) -- (axis cs:8.97229E+01,2.37204E+01) -- cycle ; 
\draw[MWE-full] (axis cs:9.09865E+01,2.20817E+01) -- (axis cs:9.15135E+01,2.12100E+01) -- (axis cs:9.13265E+01,2.11122E+01) -- (axis cs:9.07986E+01,2.19834E+01) -- cycle ; 
\draw[MWE-full] (axis cs:9.20345E+01,2.03366E+01) -- (axis cs:9.25496E+01,1.94617E+01) -- (axis cs:9.23640E+01,1.93650E+01) -- (axis cs:9.18481E+01,2.02394E+01) -- cycle ; 
\draw[MWE-full] (axis cs:9.30592E+01,1.85854E+01) -- (axis cs:9.35636E+01,1.77078E+01) -- (axis cs:9.33793E+01,1.76121E+01) -- (axis cs:9.28742E+01,1.84892E+01) -- cycle ; 
\draw[MWE-full] (axis cs:9.40631E+01,1.68289E+01) -- (axis cs:9.45580E+01,1.59488E+01) -- (axis cs:9.43750E+01,1.58540E+01) -- (axis cs:9.38795E+01,1.67336E+01) -- cycle ; 
\draw[MWE-full] (axis cs:9.50487E+01,1.50676E+01) -- (axis cs:9.55353E+01,1.41854E+01) -- (axis cs:9.53533E+01,1.40913E+01) -- (axis cs:9.48661E+01,1.49732E+01) -- cycle ; 
\draw[MWE-full] (axis cs:9.60182E+01,1.33022E+01) -- (axis cs:9.64975E+01,1.24181E+01) -- (axis cs:9.63165E+01,1.23248E+01) -- (axis cs:9.58366E+01,1.32085E+01) -- cycle ; 
\draw[MWE-full] (axis cs:9.69737E+01,1.15333E+01) -- (axis cs:9.74469E+01,1.06476E+01) -- (axis cs:9.72666E+01,1.05548E+01) -- (axis cs:9.67931E+01,1.14402E+01) -- cycle ; 
\draw[MWE-full] (axis cs:9.79174E+01,9.76126E+00) -- (axis cs:9.83854E+01,8.87430E+00) -- (axis cs:9.82058E+01,8.78200E+00) -- (axis cs:9.77375E+01,9.66874E+00) -- cycle ; 
\draw[MWE-full] (axis cs:9.88512E+01,7.98676E+00) -- (axis cs:9.93149E+01,7.09872E+00) -- (axis cs:9.91360E+01,7.00683E+00) -- (axis cs:9.86719E+01,7.89468E+00) -- cycle ; 
\draw[MWE-full] (axis cs:9.97769E+01,6.21024E+00) -- (axis cs:1.00237E+02,5.32137E+00) -- (axis cs:1.00059E+02,5.22978E+00) -- (axis cs:9.95982E+01,6.11851E+00) -- cycle ; 
\draw[MWE-full] (axis cs:1.00697E+02,4.43217E+00) -- (axis cs:1.01155E+02,3.54271E+00) -- (axis cs:1.00977E+02,3.45134E+00) -- (axis cs:1.00518E+02,4.34070E+00) -- cycle ; 
\draw[MWE-full] (axis cs:1.01612E+02,2.65303E+00) -- (axis cs:1.02069E+02,1.76320E+00) -- (axis cs:1.01891E+02,1.67196E+00) -- (axis cs:1.01434E+02,2.56174E+00) -- cycle ; 
\draw[MWE-full] (axis cs:1.02525E+02,8.73274E-01) -- (axis cs:1.02981E+02,-1.66966E-02) -- (axis cs:1.02917E+02,-1.07899E-01) -- (axis cs:1.02347E+02,7.82062E-01) -- cycle ; 
\draw[MWE-full] (axis cs:1.03437E+02,-9.06652E-01) -- (axis cs:1.03893E+02,-1.79654E+00) -- (axis cs:1.03715E+02,-1.88779E+00) -- (axis cs:1.03259E+02,-9.97867E-01) -- cycle ; 
\draw[MWE-full] (axis cs:1.04350E+02,-2.68629E+00) -- (axis cs:1.04808E+02,-3.57586E+00) -- (axis cs:1.04629E+02,-3.66725E+00) -- (axis cs:1.04172E+02,-2.77760E+00) -- cycle ; 
\draw[MWE-full] (axis cs:1.05266E+02,-4.46519E+00) -- (axis cs:1.05726E+02,-5.35422E+00) -- (axis cs:1.05547E+02,-5.44583E+00) -- (axis cs:1.05088E+02,-4.55668E+00) -- cycle ; 
\draw[MWE-full] (axis cs:1.06186E+02,-6.24290E+00) -- (axis cs:1.06649E+02,-7.13115E+00) -- (axis cs:1.06470E+02,-7.22308E+00) -- (axis cs:1.06008E+02,-6.33465E+00) -- cycle ; 
\draw[MWE-full] (axis cs:1.07113E+02,-8.01893E+00) -- (axis cs:1.07580E+02,-8.90618E+00) -- (axis cs:1.07400E+02,-8.99851E+00) -- (axis cs:1.06934E+02,-8.11105E+00) -- cycle ; 
\draw[MWE-full] (axis cs:1.08048E+02,-9.79283E+00) -- (axis cs:1.08520E+02,-1.06788E+01) -- (axis cs:1.08339E+02,-1.07716E+01) -- (axis cs:1.07868E+02,-9.88539E+00) -- cycle ; 
\draw[MWE-full] (axis cs:1.08993E+02,-1.15641E+01) -- (axis cs:1.09471E+02,-1.24486E+01) -- (axis cs:1.09289E+02,-1.25420E+01) -- (axis cs:1.08813E+02,-1.16572E+01) -- cycle ; 
\draw[MWE-full] (axis cs:1.09951E+02,-1.33322E+01) -- (axis cs:1.10435E+02,-1.42149E+01) -- (axis cs:1.10253E+02,-1.43090E+01) -- (axis cs:1.09769E+02,-1.34259E+01) -- cycle ; 
\draw[MWE-full] (axis cs:1.10922E+02,-1.50966E+01) -- (axis cs:1.11414E+02,-1.59772E+01) -- (axis cs:1.11231E+02,-1.60721E+01) -- (axis cs:1.10740E+02,-1.51910E+01) -- cycle ; 
\draw[MWE-full] (axis cs:1.11910E+02,-1.68567E+01) -- (axis cs:1.12411E+02,-1.77350E+01) -- (axis cs:1.12227E+02,-1.78307E+01) -- (axis cs:1.11727E+02,-1.69520E+01) -- cycle ; 
\draw[MWE-full] (axis cs:1.12917E+02,-1.86119E+01) -- (axis cs:1.13428E+02,-1.94875E+01) -- (axis cs:1.13242E+02,-1.95843E+01) -- (axis cs:1.12732E+02,-1.87082E+01) -- cycle ; 
\draw[MWE-full] (axis cs:1.13945E+02,-2.03617E+01) -- (axis cs:1.14467E+02,-2.12342E+01) -- (axis cs:1.14280E+02,-2.13321E+01) -- (axis cs:1.13758E+02,-2.04590E+01) -- cycle ; 
\draw[MWE-full] (axis cs:1.14996E+02,-2.21052E+01) -- (axis cs:1.15531E+02,-2.29744E+01) -- (axis cs:1.15342E+02,-2.30735E+01) -- (axis cs:1.14808E+02,-2.22037E+01) -- cycle ; 
\draw[MWE-full] (axis cs:1.16073E+02,-2.38417E+01) -- (axis cs:1.16623E+02,-2.47072E+01) -- (axis cs:1.16433E+02,-2.48076E+01) -- (axis cs:1.15884E+02,-2.39415E+01) -- cycle ; 
\draw[MWE-full] (axis cs:1.17180E+02,-2.55705E+01) -- (axis cs:1.17746E+02,-2.64318E+01) -- (axis cs:1.17554E+02,-2.65337E+01) -- (axis cs:1.16989E+02,-2.56717E+01) -- cycle ; 
\draw[MWE-full] (axis cs:1.18320E+02,-2.72907E+01) -- (axis cs:1.18903E+02,-2.81473E+01) -- (axis cs:1.18709E+02,-2.82508E+01) -- (axis cs:1.18127E+02,-2.73934E+01) -- cycle ; 
\draw[MWE-full] (axis cs:1.19495E+02,-2.90013E+01) -- (axis cs:1.20098E+02,-2.98527E+01) -- (axis cs:1.19901E+02,-2.99579E+01) -- (axis cs:1.19300E+02,-2.91056E+01) -- cycle ; 
\draw[MWE-full] (axis cs:1.20710E+02,-3.07013E+01) -- (axis cs:1.21334E+02,-3.15469E+01) -- (axis cs:1.21135E+02,-3.16540E+01) -- (axis cs:1.20513E+02,-3.08074E+01) -- cycle ; 
\draw[MWE-full] (axis cs:1.21969E+02,-3.23894E+01) -- (axis cs:1.22616E+02,-3.32287E+01) -- (axis cs:1.22415E+02,-3.33378E+01) -- (axis cs:1.21769E+02,-3.24975E+01) -- cycle ; 
\draw[MWE-full] (axis cs:1.23275E+02,-3.40646E+01) -- (axis cs:1.23948E+02,-3.48968E+01) -- (axis cs:1.23745E+02,-3.50081E+01) -- (axis cs:1.23073E+02,-3.41747E+01) -- cycle ; 
\draw[MWE-full] (axis cs:1.24634E+02,-3.57252E+01) -- (axis cs:1.25335E+02,-3.65497E+01) -- (axis cs:1.25130E+02,-3.66633E+01) -- (axis cs:1.24430E+02,-3.58377E+01) -- cycle ; 
\draw[MWE-full] (axis cs:1.26051E+02,-3.73699E+01) -- (axis cs:1.26782E+02,-3.81857E+01) -- (axis cs:1.26575E+02,-3.83019E+01) -- (axis cs:1.25844E+02,-3.74848E+01) -- cycle ; 
\draw[MWE-full] (axis cs:1.27531E+02,-3.89969E+01) -- (axis cs:1.28296E+02,-3.98031E+01) -- (axis cs:1.28087E+02,-3.99220E+01) -- (axis cs:1.27322E+02,-3.91143E+01) -- cycle ; 
\draw[MWE-full] (axis cs:1.29080E+02,-4.06042E+01) -- (axis cs:1.29883E+02,-4.13998E+01) -- (axis cs:1.29671E+02,-4.15215E+01) -- (axis cs:1.28869E+02,-4.07244E+01) -- cycle ; 
\draw[MWE-full] (axis cs:1.30705E+02,-4.21897E+01) -- (axis cs:1.31549E+02,-4.29736E+01) -- (axis cs:1.31335E+02,-4.30984E+01) -- (axis cs:1.30492E+02,-4.23130E+01) -- cycle ; 
\draw[MWE-full] (axis cs:1.32414E+02,-4.37511E+01) -- (axis cs:1.33301E+02,-4.45218E+01) -- (axis cs:1.33086E+02,-4.46499E+01) -- (axis cs:1.32199E+02,-4.38775E+01) -- cycle ; 
\draw[MWE-full] (axis cs:1.34213E+02,-4.52855E+01) -- (axis cs:1.35149E+02,-4.60418E+01) -- (axis cs:1.34932E+02,-4.61733E+01) -- (axis cs:1.33996E+02,-4.54153E+01) -- cycle ; 
\draw[MWE-full] (axis cs:1.36111E+02,-4.67901E+01) -- (axis cs:1.37100E+02,-4.75301E+01) -- (axis cs:1.36882E+02,-4.76654E+01) -- (axis cs:1.35893E+02,-4.69235E+01) -- cycle ; 
\draw[MWE-full] (axis cs:1.38118E+02,-4.82614E+01) -- (axis cs:1.39165E+02,-4.89834E+01) -- (axis cs:1.38945E+02,-4.91226E+01) -- (axis cs:1.37899E+02,-4.83986E+01) -- cycle ; 
\draw[MWE-full] (axis cs:1.40242E+02,-4.96956E+01) -- (axis cs:1.41351E+02,-5.03976E+01) -- (axis cs:1.41132E+02,-5.05409E+01) -- (axis cs:1.40023E+02,-4.98368E+01) -- cycle ; 
\draw[MWE-full] (axis cs:1.42493E+02,-5.10886E+01) -- (axis cs:1.43670E+02,-5.17682E+01) -- (axis cs:1.43452E+02,-5.19158E+01) -- (axis cs:1.42274E+02,-5.12341E+01) -- cycle ; 
\draw[MWE-full] (axis cs:1.44882E+02,-5.24356E+01) -- (axis cs:1.46132E+02,-5.30902E+01) -- (axis cs:1.45915E+02,-5.32424E+01) -- (axis cs:1.44665E+02,-5.25855E+01) -- cycle ; 
\draw[MWE-full] (axis cs:1.47419E+02,-5.37314E+01) -- (axis cs:1.48746E+02,-5.43582E+01) -- (axis cs:1.48533E+02,-5.45151E+01) -- (axis cs:1.47204E+02,-5.38858E+01) -- cycle ; 
\draw[MWE-full] (axis cs:1.50114E+02,-5.49701E+01) -- (axis cs:1.51523E+02,-5.55661E+01) -- (axis cs:1.51315E+02,-5.57277E+01) -- (axis cs:1.49903E+02,-5.51293E+01) -- cycle ; 
\draw[MWE-full] (axis cs:1.52976E+02,-5.61454E+01) -- (axis cs:1.54472E+02,-5.67072E+01) -- (axis cs:1.54269E+02,-5.68737E+01) -- (axis cs:1.52770E+02,-5.63095E+01) -- cycle ; 
\draw[MWE-full] (axis cs:1.56012E+02,-5.72505E+01) -- (axis cs:1.57598E+02,-5.77743E+01) -- (axis cs:1.57405E+02,-5.79457E+01) -- (axis cs:1.55814E+02,-5.74194E+01) -- cycle ; 
\draw[MWE-full] (axis cs:1.59230E+02,-5.82777E+01) -- (axis cs:1.60908E+02,-5.87597E+01) -- (axis cs:1.60725E+02,-5.89360E+01) -- (axis cs:1.59042E+02,-5.84516E+01) -- cycle ; 
\draw[MWE-full] (axis cs:1.62632E+02,-5.92193E+01) -- (axis cs:1.64402E+02,-5.96555E+01) -- (axis cs:1.64233E+02,-5.98365E+01) -- (axis cs:1.62456E+02,-5.93980E+01) -- cycle ; 
\draw[MWE-full] (axis cs:1.66217E+02,-6.00672E+01) -- (axis cs:1.68077E+02,-6.04533E+01) -- (axis cs:1.67924E+02,-6.06388E+01) -- (axis cs:1.66056E+02,-6.02505E+01) -- cycle ; 
\draw[MWE-full] (axis cs:1.69979E+02,-6.08130E+01) -- (axis cs:1.71924E+02,-6.11451E+01) -- (axis cs:1.71791E+02,-6.13346E+01) -- (axis cs:1.69837E+02,-6.10005E+01) -- cycle ; 
\draw[MWE-full] (axis cs:1.73908E+02,-6.14487E+01) -- (axis cs:1.75929E+02,-6.17229E+01) -- (axis cs:1.75819E+02,-6.19160E+01) -- (axis cs:1.73786E+02,-6.16401E+01) -- cycle ; 
\draw[MWE-full] (axis cs:1.77984E+02,-6.19668E+01) -- (axis cs:1.80071E+02,-6.21796E+01) -- (axis cs:1.79986E+02,-6.23756E+01) -- (axis cs:1.77886E+02,-6.21614E+01) -- cycle ; 
\draw[MWE-full] (axis cs:1.82184E+02,-6.23606E+01) -- (axis cs:1.84321E+02,-6.25091E+01) -- (axis cs:1.84264E+02,-6.27073E+01) -- (axis cs:1.82113E+02,-6.25578E+01) -- cycle ; 
\draw[MWE-full] (axis cs:1.86478E+02,-6.26247E+01) -- (axis cs:1.88648E+02,-6.27068E+01) -- (axis cs:1.88619E+02,-6.29064E+01) -- (axis cs:1.86434E+02,-6.28237E+01) -- cycle ; 
\draw[MWE-full] (axis cs:1.90828E+02,-6.27553E+01) -- (axis cs:1.93013E+02,-6.27699E+01) -- (axis cs:1.93014E+02,-6.29699E+01) -- (axis cs:1.90814E+02,-6.29552E+01) -- cycle ; 
\draw[MWE-full] (axis cs:1.95197E+02,-6.27506E+01) -- (axis cs:1.97377E+02,-6.26974E+01) -- (axis cs:1.97407E+02,-6.28969E+01) -- (axis cs:1.95213E+02,-6.29504E+01) -- cycle ; 
\draw[MWE-full] (axis cs:1.99545E+02,-6.26105E+01) -- (axis cs:2.01699E+02,-6.24903E+01) -- (axis cs:2.01759E+02,-6.26884E+01) -- (axis cs:1.99591E+02,-6.28094E+01) -- cycle ; 
\draw[MWE-full] (axis cs:2.03833E+02,-6.23372E+01) -- (axis cs:2.05943E+02,-6.21517E+01) -- (axis cs:2.06030E+02,-6.23475E+01) -- (axis cs:2.03907E+02,-6.25342E+01) -- cycle ; 
\draw[MWE-full] (axis cs:2.08026E+02,-6.19345E+01) -- (axis cs:2.10076E+02,-6.16863E+01) -- (axis cs:2.10188E+02,-6.18792E+01) -- (axis cs:2.08125E+02,-6.21289E+01) -- cycle ; 
\draw[MWE-full] (axis cs:2.12092E+02,-6.14079E+01) -- (axis cs:2.14071E+02,-6.11003E+01) -- (axis cs:2.14205E+02,-6.12895E+01) -- (axis cs:2.12216E+02,-6.15991E+01) -- cycle ; 
\draw[MWE-full] (axis cs:2.16010E+02,-6.07642E+01) -- (axis cs:2.17906E+02,-6.04008E+01) -- (axis cs:2.18060E+02,-6.05860E+01) -- (axis cs:2.16154E+02,-6.09515E+01) -- cycle ; 
\draw[MWE-full] (axis cs:2.19760E+02,-6.00110E+01) -- (axis cs:2.21569E+02,-5.95959E+01) -- (axis cs:2.21739E+02,-5.97766E+01) -- (axis cs:2.19922E+02,-6.01940E+01) -- cycle ; 
\draw[MWE-full] (axis cs:2.23332E+02,-5.91564E+01) -- (axis cs:2.25050E+02,-5.86936E+01) -- (axis cs:2.25233E+02,-5.88695E+01) -- (axis cs:2.23509E+02,-5.93347E+01) -- cycle ; 
\draw[MWE-full] (axis cs:2.26721E+02,-5.82085E+01) -- (axis cs:2.28346E+02,-5.77022E+01) -- (axis cs:2.28541E+02,-5.78732E+01) -- (axis cs:2.26910E+02,-5.83820E+01) -- cycle ; 
\draw[MWE-full] (axis cs:2.29926E+02,-5.71755E+01) -- (axis cs:2.31460E+02,-5.66296E+01) -- (axis cs:2.31663E+02,-5.67958E+01) -- (axis cs:2.30125E+02,-5.73442E+01) -- cycle ; 
\draw[MWE-full] (axis cs:2.32950E+02,-5.60654E+01) -- (axis cs:2.34397E+02,-5.54837E+01) -- (axis cs:2.34606E+02,-5.56449E+01) -- (axis cs:2.33157E+02,-5.62291E+01) -- cycle ; 
\draw[MWE-full] (axis cs:2.35800E+02,-5.48854E+01) -- (axis cs:2.37162E+02,-5.42714E+01) -- (axis cs:2.37376E+02,-5.44279E+01) -- (axis cs:2.36012E+02,-5.50442E+01) -- cycle ; 
\draw[MWE-full] (axis cs:2.38483E+02,-5.36425E+01) -- (axis cs:2.39766E+02,-5.29994E+01) -- (axis cs:2.39982E+02,-5.31512E+01) -- (axis cs:2.38699E+02,-5.37966E+01) -- cycle ; 
\draw[MWE-full] (axis cs:2.41010E+02,-5.23429E+01) -- (axis cs:2.42217E+02,-5.16738E+01) -- (axis cs:2.42435E+02,-5.18211E+01) -- (axis cs:2.41227E+02,-5.24925E+01) -- cycle ; 
\draw[MWE-full] (axis cs:2.43389E+02,-5.09926E+01) -- (axis cs:2.44526E+02,-5.02999E+01) -- (axis cs:2.44745E+02,-5.04429E+01) -- (axis cs:2.43608E+02,-5.11377E+01) -- cycle ; 
\draw[MWE-full] (axis cs:2.45631E+02,-4.95965E+01) -- (axis cs:2.46704E+02,-4.88829E+01) -- (axis cs:2.46923E+02,-4.90218E+01) -- (axis cs:2.45850E+02,-4.97374E+01) -- cycle ; 
\draw[MWE-full] (axis cs:2.47746E+02,-4.81596E+01) -- (axis cs:2.48760E+02,-4.74270E+01) -- (axis cs:2.48978E+02,-4.75620E+01) -- (axis cs:2.47965E+02,-4.82965E+01) -- cycle ; 
\draw[MWE-full] (axis cs:2.49745E+02,-4.66858E+01) -- (axis cs:2.50704E+02,-4.59363E+01) -- (axis cs:2.50921E+02,-4.60676E+01) -- (axis cs:2.49963E+02,-4.68189E+01) -- cycle ; 
\draw[MWE-full] (axis cs:2.51636E+02,-4.51791E+01) -- (axis cs:2.52544E+02,-4.44144E+01) -- (axis cs:2.52760E+02,-4.45422E+01) -- (axis cs:2.51853E+02,-4.53086E+01) -- cycle ; 
\draw[MWE-full] (axis cs:2.53429E+02,-4.36426E+01) -- (axis cs:2.54291E+02,-4.28642E+01) -- (axis cs:2.54505E+02,-4.29888E+01) -- (axis cs:2.53644E+02,-4.37688E+01) -- cycle ; 
\draw[MWE-full] (axis cs:2.55131E+02,-4.20795E+01) -- (axis cs:2.55951E+02,-4.12888E+01) -- (axis cs:2.56163E+02,-4.14103E+01) -- (axis cs:2.55344E+02,-4.22025E+01) -- cycle ; 
\draw[MWE-full] (axis cs:2.56751E+02,-4.04924E+01) -- (axis cs:2.57532E+02,-3.96906E+01) -- (axis cs:2.57742E+02,-3.98092E+01) -- (axis cs:2.56962E+02,-4.06124E+01) -- cycle ; 
\draw[MWE-full] (axis cs:2.58295E+02,-3.88836E+01) -- (axis cs:2.59041E+02,-3.80718E+01) -- (axis cs:2.59248E+02,-3.81877E+01) -- (axis cs:2.58504E+02,-3.90009E+01) -- cycle ; 
\draw[MWE-full] (axis cs:2.59771E+02,-3.72554E+01) -- (axis cs:2.60484E+02,-3.64345E+01) -- (axis cs:2.60689E+02,-3.65480E+01) -- (axis cs:2.59977E+02,-3.73700E+01) -- cycle ; 
\draw[MWE-full] (axis cs:2.61183E+02,-3.56095E+01) -- (axis cs:2.61867E+02,-3.47805E+01) -- (axis cs:2.62070E+02,-3.48916E+01) -- (axis cs:2.61387E+02,-3.57218E+01) -- cycle ; 
\draw[MWE-full] (axis cs:2.62538E+02,-3.39478E+01) -- (axis cs:2.63196E+02,-3.31114E+01) -- (axis cs:2.63396E+02,-3.32204E+01) -- (axis cs:2.62740E+02,-3.40578E+01) -- cycle ; 
\draw[MWE-full] (axis cs:2.63841E+02,-3.22717E+01) -- (axis cs:2.64474E+02,-3.14287E+01) -- (axis cs:2.64672E+02,-3.15356E+01) -- (axis cs:2.64040E+02,-3.23796E+01) -- cycle ; 
\draw[MWE-full] (axis cs:2.65096E+02,-3.05826E+01) -- (axis cs:2.65707E+02,-2.97337E+01) -- (axis cs:2.65903E+02,-2.98387E+01) -- (axis cs:2.65293E+02,-3.06886E+01) -- cycle ; 
\draw[MWE-full] (axis cs:2.66308E+02,-2.88819E+01) -- (axis cs:2.66899E+02,-2.80275E+01) -- (axis cs:2.67093E+02,-2.81309E+01) -- (axis cs:2.66503E+02,-2.89861E+01) -- cycle ; 
\draw[MWE-empty] (axis cs:2.66899E+02,-2.80275E+01) -- (axis cs:2.67481E+02,-2.71706E+01) -- (axis cs:2.67674E+02,-2.72732E+01) -- (axis cs:2.67093E+02,-2.81309E+01) -- cycle ; 
\draw[MWE-empty] (axis cs:2.68054E+02,-2.63113E+01) -- (axis cs:2.68618E+02,-2.54498E+01) -- (axis cs:2.68809E+02,-2.55508E+01) -- (axis cs:2.68246E+02,-2.64131E+01) -- cycle ; 
\draw[MWE-empty] (axis cs:2.69174E+02,-2.45861E+01) -- (axis cs:2.69723E+02,-2.37204E+01) -- (axis cs:2.69912E+02,-2.38201E+01) -- (axis cs:2.69365E+02,-2.46865E+01) -- cycle ; 
\draw[MWE-empty] (axis cs:2.70264E+02,-2.28528E+01) -- (axis cs:2.70799E+02,-2.19834E+01) -- (axis cs:2.70986E+02,-2.20817E+01) -- (axis cs:2.70453E+02,-2.29518E+01) -- cycle ; 
\draw[MWE-empty] (axis cs:2.71327E+02,-2.11122E+01) -- (axis cs:2.71848E+02,-2.02394E+01) -- (axis cs:2.72035E+02,-2.03366E+01) -- (axis cs:2.71514E+02,-2.12100E+01) -- cycle ; 
\draw[MWE-empty] (axis cs:2.72364E+02,-1.93650E+01) -- (axis cs:2.72874E+02,-1.84892E+01) -- (axis cs:2.73059E+02,-1.85854E+01) -- (axis cs:2.72550E+02,-1.94617E+01) -- cycle ; 
\draw[MWE-empty] (axis cs:2.73379E+02,-1.76121E+01) -- (axis cs:2.73879E+02,-1.67336E+01) -- (axis cs:2.74063E+02,-1.68289E+01) -- (axis cs:2.73564E+02,-1.77078E+01) -- cycle ; 
\draw[MWE-empty] (axis cs:2.74375E+02,-1.58540E+01) -- (axis cs:2.74866E+02,-1.49732E+01) -- (axis cs:2.75049E+02,-1.50676E+01) -- (axis cs:2.74558E+02,-1.59488E+01) -- cycle ; 
\draw[MWE-empty] (axis cs:2.75353E+02,-1.40913E+01) -- (axis cs:2.75837E+02,-1.32085E+01) -- (axis cs:2.76018E+02,-1.33022E+01) -- (axis cs:2.75535E+02,-1.41854E+01) -- cycle ; 
\draw[MWE-empty] (axis cs:2.76316E+02,-1.23248E+01) -- (axis cs:2.76793E+02,-1.14402E+01) -- (axis cs:2.76974E+02,-1.15333E+01) -- (axis cs:2.76498E+02,-1.24181E+01) -- cycle ; 
\draw[MWE-empty] (axis cs:2.77267E+02,-1.05548E+01) -- (axis cs:2.77738E+02,-9.66873E+00) -- (axis cs:2.77917E+02,-9.76126E+00) -- (axis cs:2.77447E+02,-1.06476E+01) -- cycle ; 
\draw[MWE-empty] (axis cs:2.78206E+02,-8.78200E+00) -- (axis cs:2.78672E+02,-7.89468E+00) -- (axis cs:2.78851E+02,-7.98676E+00) -- (axis cs:2.78385E+02,-8.87430E+00) -- cycle ; 
\draw[MWE-empty] (axis cs:2.79136E+02,-7.00682E+00) -- (axis cs:2.79598E+02,-6.11851E+00) -- (axis cs:2.79777E+02,-6.21024E+00) -- (axis cs:2.79315E+02,-7.09872E+00) -- cycle ; 
\draw[MWE-empty] (axis cs:2.80059E+02,-5.22978E+00) -- (axis cs:2.80518E+02,-4.34070E+00) -- (axis cs:2.80697E+02,-4.43217E+00) -- (axis cs:2.80237E+02,-5.32137E+00) -- cycle ; 
\draw[MWE-empty] (axis cs:2.80977E+02,-3.45134E+00) -- (axis cs:2.81434E+02,-2.56174E+00) -- (axis cs:2.81612E+02,-2.65303E+00) -- (axis cs:2.81155E+02,-3.54271E+00) -- cycle ; 
\draw[MWE-empty] (axis cs:2.81891E+02,-1.67196E+00) -- (axis cs:2.82347E+02,-7.82062E-01) -- (axis cs:2.82525E+02,-8.73274E-01) -- (axis cs:2.82069E+02,-1.76320E+00) -- cycle ; 
\draw[MWE-empty] (axis cs:2.82917E+02,1.07899E-01) -- (axis cs:2.83259E+02,9.97867E-01) -- (axis cs:2.83437E+02,9.06652E-01) -- (axis cs:2.82981E+02,1.66967E-02) -- cycle ; 
\draw[MWE-empty] (axis cs:2.83715E+02,1.88779E+00) -- (axis cs:2.84172E+02,2.77760E+00) -- (axis cs:2.84350E+02,2.68629E+00) -- (axis cs:2.83893E+02,1.79654E+00) -- cycle ; 
\draw[MWE-empty] (axis cs:2.84629E+02,3.66725E+00) -- (axis cs:2.85088E+02,4.55668E+00) -- (axis cs:2.85266E+02,4.46519E+00) -- (axis cs:2.84808E+02,3.57586E+00) -- cycle ; 
\draw[MWE-empty] (axis cs:2.85547E+02,5.44583E+00) -- (axis cs:2.86008E+02,6.33465E+00) -- (axis cs:2.86186E+02,6.24290E+00) -- (axis cs:2.85725E+02,5.35422E+00) -- cycle ; 
\draw[MWE-empty] (axis cs:2.86470E+02,7.22308E+00) -- (axis cs:2.86934E+02,8.11105E+00) -- (axis cs:2.87113E+02,8.01893E+00) -- (axis cs:2.86649E+02,7.13115E+00) -- cycle ; 
\draw[MWE-empty] (axis cs:2.87400E+02,8.99851E+00) -- (axis cs:2.87868E+02,9.88539E+00) -- (axis cs:2.88048E+02,9.79283E+00) -- (axis cs:2.87580E+02,8.90618E+00) -- cycle ; 
\draw[MWE-empty] (axis cs:2.88339E+02,1.07716E+01) -- (axis cs:2.88813E+02,1.16572E+01) -- (axis cs:2.88993E+02,1.15641E+01) -- (axis cs:2.88519E+02,1.06788E+01) -- cycle ; 
\draw[MWE-empty] (axis cs:2.89289E+02,1.25420E+01) -- (axis cs:2.89769E+02,1.34259E+01) -- (axis cs:2.89951E+02,1.33322E+01) -- (axis cs:2.89470E+02,1.24486E+01) -- cycle ; 
\draw[MWE-empty] (axis cs:2.90253E+02,1.43090E+01) -- (axis cs:2.90740E+02,1.51910E+01) -- (axis cs:2.90922E+02,1.50966E+01) -- (axis cs:2.90435E+02,1.42149E+01) -- cycle ; 
\draw[MWE-empty] (axis cs:2.91231E+02,1.60721E+01) -- (axis cs:2.91727E+02,1.69520E+01) -- (axis cs:2.91910E+02,1.68567E+01) -- (axis cs:2.91414E+02,1.59772E+01) -- cycle ; 
\draw[MWE-empty] (axis cs:2.92227E+02,1.78307E+01) -- (axis cs:2.92732E+02,1.87082E+01) -- (axis cs:2.92917E+02,1.86119E+01) -- (axis cs:2.92411E+02,1.77350E+01) -- cycle ; 
\draw[MWE-empty] (axis cs:2.93242E+02,1.95843E+01) -- (axis cs:2.93758E+02,2.04590E+01) -- (axis cs:2.93944E+02,2.03617E+01) -- (axis cs:2.93428E+02,1.94875E+01) -- cycle ; 
\draw[MWE-empty] (axis cs:2.94280E+02,2.13321E+01) -- (axis cs:2.94808E+02,2.22037E+01) -- (axis cs:2.94996E+02,2.21052E+01) -- (axis cs:2.94467E+02,2.12342E+01) -- cycle ; 
\draw[MWE-empty] (axis cs:2.95342E+02,2.30735E+01) -- (axis cs:2.95884E+02,2.39415E+01) -- (axis cs:2.96073E+02,2.38417E+01) -- (axis cs:2.95531E+02,2.29744E+01) -- cycle ; 
\draw[MWE-empty] (axis cs:2.96433E+02,2.48076E+01) -- (axis cs:2.96989E+02,2.56717E+01) -- (axis cs:2.97180E+02,2.55705E+01) -- (axis cs:2.96623E+02,2.47072E+01) -- cycle ; 
\draw[MWE-empty] (axis cs:2.97554E+02,2.65337E+01) -- (axis cs:2.98127E+02,2.73934E+01) -- (axis cs:2.98320E+02,2.72907E+01) -- (axis cs:2.97746E+02,2.64318E+01) -- cycle ; 
\draw[MWE-empty] (axis cs:2.98709E+02,2.82508E+01) -- (axis cs:2.99300E+02,2.91056E+01) -- (axis cs:2.99495E+02,2.90013E+01) -- (axis cs:2.98903E+02,2.81473E+01) -- cycle ; 
\draw[MWE-empty] (axis cs:2.99901E+02,2.99579E+01) -- (axis cs:3.00513E+02,3.08074E+01) -- (axis cs:3.00710E+02,3.07013E+01) -- (axis cs:3.00098E+02,2.98527E+01) -- cycle ; 
\draw[MWE-empty] (axis cs:3.01135E+02,3.16540E+01) -- (axis cs:3.01769E+02,3.24975E+01) -- (axis cs:3.01969E+02,3.23894E+01) -- (axis cs:3.01334E+02,3.15469E+01) -- cycle ; 
\draw[MWE-empty] (axis cs:3.02415E+02,3.33378E+01) -- (axis cs:3.03073E+02,3.41747E+01) -- (axis cs:3.03275E+02,3.40646E+01) -- (axis cs:3.02616E+02,3.32287E+01) -- cycle ; 
\draw[MWE-empty] (axis cs:3.03745E+02,3.50081E+01) -- (axis cs:3.04430E+02,3.58377E+01) -- (axis cs:3.04634E+02,3.57252E+01) -- (axis cs:3.03948E+02,3.48968E+01) -- cycle ; 
\draw[MWE-empty] (axis cs:3.05130E+02,3.66633E+01) -- (axis cs:3.05844E+02,3.74848E+01) -- (axis cs:3.06051E+02,3.73699E+01) -- (axis cs:3.05335E+02,3.65497E+01) -- cycle ; 
\draw[MWE-empty] (axis cs:3.06575E+02,3.83019E+01) -- (axis cs:3.07322E+02,3.91143E+01) -- (axis cs:3.07531E+02,3.89969E+01) -- (axis cs:3.06782E+02,3.81857E+01) -- cycle ; 
\draw[MWE-empty] (axis cs:3.08087E+02,3.99220E+01) -- (axis cs:3.08869E+02,4.07244E+01) -- (axis cs:3.09080E+02,4.06042E+01) -- (axis cs:3.08296E+02,3.98031E+01) -- cycle ; 
\draw[MWE-empty] (axis cs:3.09671E+02,4.15215E+01) -- (axis cs:3.10492E+02,4.23130E+01) -- (axis cs:3.10705E+02,4.21897E+01) -- (axis cs:3.09883E+02,4.13998E+01) -- cycle ; 
\draw[MWE-empty] (axis cs:3.11335E+02,4.30984E+01) -- (axis cs:3.12199E+02,4.38775E+01) -- (axis cs:3.12414E+02,4.37511E+01) -- (axis cs:3.11548E+02,4.29736E+01) -- cycle ; 
\draw[MWE-empty] (axis cs:3.13086E+02,4.46499E+01) -- (axis cs:3.13996E+02,4.54153E+01) -- (axis cs:3.14213E+02,4.52855E+01) -- (axis cs:3.13301E+02,4.45218E+01) -- cycle ; 
\draw[MWE-empty] (axis cs:3.14932E+02,4.61733E+01) -- (axis cs:3.15893E+02,4.69235E+01) -- (axis cs:3.16111E+02,4.67901E+01) -- (axis cs:3.15149E+02,4.60418E+01) -- cycle ; 
\draw[MWE-empty] (axis cs:3.16882E+02,4.76654E+01) -- (axis cs:3.17899E+02,4.83986E+01) -- (axis cs:3.18118E+02,4.82614E+01) -- (axis cs:3.17100E+02,4.75301E+01) -- cycle ; 
\draw[MWE-empty] (axis cs:3.18945E+02,4.91226E+01) -- (axis cs:3.20023E+02,4.98368E+01) -- (axis cs:3.20242E+02,4.96956E+01) -- (axis cs:3.19165E+02,4.89834E+01) -- cycle ; 
\draw[MWE-empty] (axis cs:3.21132E+02,5.05409E+01) -- (axis cs:3.22274E+02,5.12341E+01) -- (axis cs:3.22493E+02,5.10886E+01) -- (axis cs:3.21351E+02,5.03976E+01) -- cycle ; 
\draw[MWE-empty] (axis cs:3.23452E+02,5.19158E+01) -- (axis cs:3.24665E+02,5.25855E+01) -- (axis cs:3.24882E+02,5.24356E+01) -- (axis cs:3.23670E+02,5.17682E+01) -- cycle ; 
\draw[MWE-empty] (axis cs:3.25915E+02,5.32424E+01) -- (axis cs:3.27204E+02,5.38858E+01) -- (axis cs:3.27419E+02,5.37314E+01) -- (axis cs:3.26132E+02,5.30902E+01) -- cycle ; 
\draw[MWE-empty] (axis cs:3.28533E+02,5.45151E+01) -- (axis cs:3.29903E+02,5.51293E+01) -- (axis cs:3.30114E+02,5.49701E+01) -- (axis cs:3.28746E+02,5.43582E+01) -- cycle ; 
\draw[MWE-empty] (axis cs:3.31315E+02,5.57277E+01) -- (axis cs:3.32770E+02,5.63095E+01) -- (axis cs:3.32976E+02,5.61454E+01) -- (axis cs:3.31523E+02,5.55661E+01) -- cycle ; 
\draw[MWE-empty] (axis cs:3.34269E+02,5.68737E+01) -- (axis cs:3.35814E+02,5.74194E+01) -- (axis cs:3.36012E+02,5.72505E+01) -- (axis cs:3.34472E+02,5.67072E+01) -- cycle ; 
\draw[MWE-empty] (axis cs:3.37405E+02,5.79457E+01) -- (axis cs:3.39042E+02,5.84516E+01) -- (axis cs:3.39230E+02,5.82777E+01) -- (axis cs:3.37598E+02,5.77743E+01) -- cycle ; 
\draw[MWE-empty] (axis cs:3.40725E+02,5.89360E+01) -- (axis cs:3.42456E+02,5.93980E+01) -- (axis cs:3.42632E+02,5.92193E+01) -- (axis cs:3.40908E+02,5.87597E+01) -- cycle ; 
\draw[MWE-empty] (axis cs:3.44233E+02,5.98365E+01) -- (axis cs:3.46056E+02,6.02505E+01) -- (axis cs:3.46217E+02,6.00672E+01) -- (axis cs:3.44402E+02,5.96555E+01) -- cycle ; 
\draw[MWE-empty] (axis cs:3.47924E+02,6.06388E+01) -- (axis cs:3.49837E+02,6.10005E+01) -- (axis cs:3.49979E+02,6.08130E+01) -- (axis cs:3.48077E+02,6.04533E+01) -- cycle ; 
\draw[MWE-empty] (axis cs:3.51791E+02,6.13346E+01) -- (axis cs:3.53786E+02,6.16401E+01) -- (axis cs:3.53908E+02,6.14487E+01) -- (axis cs:3.51924E+02,6.11451E+01) -- cycle ; 
\draw[MWE-empty] (axis cs:3.55819E+02,6.19160E+01) -- (axis cs:3.57886E+02,6.21614E+01) -- (axis cs:3.57984E+02,6.19668E+01) -- (axis cs:3.55929E+02,6.17229E+01) -- cycle ; 
\draw[MWE-empty] (axis cs:3.59986E+02,6.23756E+01) -- (axis cs:2.11273E+00,6.25578E+01) -- (axis cs:2.18430E+00,6.23606E+01) -- (axis cs:7.05566E-02,6.21796E+01) -- cycle ; 
\draw[MWE-empty] (axis cs:4.26373E+00,6.27073E+01) -- (axis cs:6.43417E+00,6.28237E+01) -- (axis cs:6.47748E+00,6.26247E+01) -- (axis cs:4.32138E+00,6.25091E+01) -- cycle ; 
\draw[MWE-empty] (axis cs:8.61932E+00,6.29064E+01) -- (axis cs:1.08143E+01,6.29552E+01) -- (axis cs:1.08282E+01,6.27553E+01) -- (axis cs:8.64801E+00,6.27068E+01) -- cycle ; 
\draw[MWE-empty] (axis cs:1.30140E+01,6.29699E+01) -- (axis cs:1.52134E+01,6.29504E+01) -- (axis cs:1.51974E+01,6.27506E+01) -- (axis cs:1.30130E+01,6.27699E+01) -- cycle ; 
\draw[MWE-empty] (axis cs:1.74073E+01,6.28969E+01) -- (axis cs:1.95907E+01,6.28094E+01) -- (axis cs:1.95453E+01,6.26105E+01) -- (axis cs:1.73765E+01,6.26974E+01) -- cycle ; 
\draw[MWE-empty] (axis cs:2.17586E+01,6.26884E+01) -- (axis cs:2.39066E+01,6.25342E+01) -- (axis cs:2.38331E+01,6.23372E+01) -- (axis cs:2.16990E+01,6.24903E+01) -- cycle ; 
\draw[MWE-empty] (axis cs:2.60301E+01,6.23475E+01) -- (axis cs:2.81251E+01,6.21289E+01) -- (axis cs:2.80255E+01,6.19345E+01) -- (axis cs:2.59432E+01,6.21517E+01) -- cycle ; 
\draw[MWE-empty] (axis cs:3.01880E+01,6.18792E+01) -- (axis cs:3.22156E+01,6.15991E+01) -- (axis cs:3.20923E+01,6.14079E+01) -- (axis cs:3.00762E+01,6.16863E+01) -- cycle ; 
\draw[MWE-empty] (axis cs:3.42049E+01,6.12895E+01) -- (axis cs:3.61537E+01,6.09515E+01) -- (axis cs:3.60096E+01,6.07642E+01) -- (axis cs:3.40709E+01,6.11003E+01) -- cycle ; 
\draw[MWE-empty] (axis cs:3.80599E+01,6.05860E+01) -- (axis cs:3.99219E+01,6.01940E+01) -- (axis cs:3.97599E+01,6.00110E+01) -- (axis cs:3.79064E+01,6.04008E+01) -- cycle ; 
\draw[MWE-empty] (axis cs:4.17386E+01,5.97766E+01) -- (axis cs:4.35091E+01,5.93347E+01) -- (axis cs:4.33322E+01,5.91564E+01) -- (axis cs:4.15687E+01,5.95959E+01) -- cycle ; 
\draw[MWE-empty] (axis cs:4.52331E+01,5.88695E+01) -- (axis cs:4.69102E+01,5.83820E+01) -- (axis cs:4.67210E+01,5.82085E+01) -- (axis cs:4.50497E+01,5.86936E+01) -- cycle ; 
\draw[MWE-empty] (axis cs:4.85407E+01,5.78732E+01) -- (axis cs:5.01248E+01,5.73442E+01) -- (axis cs:4.99260E+01,5.71755E+01) -- (axis cs:4.83464E+01,5.77022E+01) -- cycle ; 
\draw[MWE-empty] (axis cs:5.16632E+01,5.67958E+01) -- (axis cs:5.31565E+01,5.62291E+01) -- (axis cs:5.29503E+01,5.60654E+01) -- (axis cs:5.14604E+01,5.66296E+01) -- cycle ; 
\draw[MWE-empty] (axis cs:5.46057E+01,5.56449E+01) -- (axis cs:5.60117E+01,5.50442E+01) -- (axis cs:5.58001E+01,5.48854E+01) -- (axis cs:5.43965E+01,5.54837E+01) -- cycle ; 
\draw[MWE-empty] (axis cs:5.73757E+01,5.44279E+01) -- (axis cs:5.86988E+01,5.37966E+01) -- (axis cs:5.84834E+01,5.36425E+01) -- (axis cs:5.71620E+01,5.42714E+01) -- cycle ; 
\draw[MWE-empty] (axis cs:5.99822E+01,5.31512E+01) -- (axis cs:6.12273E+01,5.24925E+01) -- (axis cs:6.10096E+01,5.23429E+01) -- (axis cs:5.97655E+01,5.29994E+01) -- cycle ; 
\draw[MWE-empty] (axis cs:6.24353E+01,5.18211E+01) -- (axis cs:6.36076E+01,5.11377E+01) -- (axis cs:6.33886E+01,5.09926E+01) -- (axis cs:6.22168E+01,5.16738E+01) -- cycle ; 
\draw[MWE-empty] (axis cs:6.47454E+01,5.04429E+01) -- (axis cs:6.58501E+01,4.97374E+01) -- (axis cs:6.56308E+01,4.95965E+01) -- (axis cs:6.45262E+01,5.02999E+01) -- cycle ; 
\draw[MWE-empty] (axis cs:6.69230E+01,4.90218E+01) -- (axis cs:6.79652E+01,4.82965E+01) -- (axis cs:6.77464E+01,4.81596E+01) -- (axis cs:6.67038E+01,4.88829E+01) -- cycle ; 
\draw[MWE-empty] (axis cs:6.89781E+01,4.75620E+01) -- (axis cs:6.99629E+01,4.68189E+01) -- (axis cs:6.97450E+01,4.66858E+01) -- (axis cs:6.87597E+01,4.74270E+01) -- cycle ; 
\draw[MWE-empty] (axis cs:7.09207E+01,4.60676E+01) -- (axis cs:7.18527E+01,4.53086E+01) -- (axis cs:7.16362E+01,4.51791E+01) -- (axis cs:7.07035E+01,4.59363E+01) -- cycle ; 
\draw[MWE-empty] (axis cs:7.27599E+01,4.45422E+01) -- (axis cs:7.36436E+01,4.37688E+01) -- (axis cs:7.34289E+01,4.36426E+01) -- (axis cs:7.25443E+01,4.44144E+01) -- cycle ; 
\draw[MWE-empty] (axis cs:7.45045E+01,4.29888E+01) -- (axis cs:7.53439E+01,4.22025E+01) -- (axis cs:7.51312E+01,4.20795E+01) -- (axis cs:7.42908E+01,4.28642E+01) -- cycle ; 
\draw[MWE-empty] (axis cs:7.61626E+01,4.14103E+01) -- (axis cs:7.69615E+01,4.06124E+01) -- (axis cs:7.67510E+01,4.04924E+01) -- (axis cs:7.59510E+01,4.12888E+01) -- cycle ; 
\draw[MWE-empty] (axis cs:7.77415E+01,3.98092E+01) -- (axis cs:7.85035E+01,3.90009E+01) -- (axis cs:7.82952E+01,3.88836E+01) -- (axis cs:7.75321E+01,3.96906E+01) -- cycle ; 
\draw[MWE-empty] (axis cs:7.92483E+01,3.81877E+01) -- (axis cs:7.99766E+01,3.73700E+01) -- (axis cs:7.97705E+01,3.72554E+01) -- (axis cs:7.90411E+01,3.80718E+01) -- cycle ; 
\draw[MWE-empty] (axis cs:8.06892E+01,3.65480E+01) -- (axis cs:8.13867E+01,3.57218E+01) -- (axis cs:8.11830E+01,3.56095E+01) -- (axis cs:8.04843E+01,3.64345E+01) -- cycle ; 
\draw[MWE-empty] (axis cs:8.20700E+01,3.48916E+01) -- (axis cs:8.27396E+01,3.40578E+01) -- (axis cs:8.25381E+01,3.39478E+01) -- (axis cs:8.18674E+01,3.47805E+01) -- cycle ; 
\draw[MWE-empty] (axis cs:8.33961E+01,3.32204E+01) -- (axis cs:8.40402E+01,3.23796E+01) -- (axis cs:8.38409E+01,3.22717E+01) -- (axis cs:8.31957E+01,3.31114E+01) -- cycle ; 
\draw[MWE-empty] (axis cs:8.46724E+01,3.15356E+01) -- (axis cs:8.52933E+01,3.06886E+01) -- (axis cs:8.50962E+01,3.05826E+01) -- (axis cs:8.44742E+01,3.14287E+01) -- cycle ; 
\draw[MWE-empty] (axis cs:8.59034E+01,2.98387E+01) -- (axis cs:8.65033E+01,2.89861E+01) -- (axis cs:8.63082E+01,2.88819E+01) -- (axis cs:8.57074E+01,2.97337E+01) -- cycle ; 
\draw[MWE-empty] (axis cs:8.70933E+01,2.81309E+01) -- (axis cs:8.76740E+01,2.72732E+01) -- (axis cs:8.74810E+01,2.71706E+01) -- (axis cs:8.68993E+01,2.80275E+01) -- cycle ; 
\draw[MWE-empty] (axis cs:8.82459E+01,2.64131E+01) -- (axis cs:8.88093E+01,2.55508E+01) -- (axis cs:8.86180E+01,2.54498E+01) -- (axis cs:8.80537E+01,2.63113E+01) -- cycle ; 
\draw[MWE-empty] (axis cs:8.93646E+01,2.46865E+01) -- (axis cs:8.99124E+01,2.38201E+01) -- (axis cs:8.97229E+01,2.37204E+01) -- (axis cs:8.91743E+01,2.45861E+01) -- cycle ; 
\draw[MWE-empty] (axis cs:9.04529E+01,2.29518E+01) -- (axis cs:9.09865E+01,2.20817E+01) -- (axis cs:9.07986E+01,2.19834E+01) -- (axis cs:9.02642E+01,2.28528E+01) -- cycle ; 
\draw[MWE-empty] (axis cs:9.15135E+01,2.12100E+01) -- (axis cs:9.20345E+01,2.03366E+01) -- (axis cs:9.18481E+01,2.02394E+01) -- (axis cs:9.13265E+01,2.11122E+01) -- cycle ; 
\draw[MWE-empty] (axis cs:9.25496E+01,1.94617E+01) -- (axis cs:9.30592E+01,1.85854E+01) -- (axis cs:9.28742E+01,1.84892E+01) -- (axis cs:9.23640E+01,1.93650E+01) -- cycle ; 
\draw[MWE-empty] (axis cs:9.35636E+01,1.77078E+01) -- (axis cs:9.40631E+01,1.68289E+01) -- (axis cs:9.38795E+01,1.67336E+01) -- (axis cs:9.33793E+01,1.76121E+01) -- cycle ; 
\draw[MWE-empty] (axis cs:9.45580E+01,1.59488E+01) -- (axis cs:9.50487E+01,1.50676E+01) -- (axis cs:9.48661E+01,1.49732E+01) -- (axis cs:9.43750E+01,1.58540E+01) -- cycle ; 
\draw[MWE-empty] (axis cs:9.55353E+01,1.41854E+01) -- (axis cs:9.60182E+01,1.33022E+01) -- (axis cs:9.58366E+01,1.32085E+01) -- (axis cs:9.53533E+01,1.40913E+01) -- cycle ; 
\draw[MWE-empty] (axis cs:9.64975E+01,1.24181E+01) -- (axis cs:9.69737E+01,1.15333E+01) -- (axis cs:9.67931E+01,1.14402E+01) -- (axis cs:9.63165E+01,1.23248E+01) -- cycle ; 
\draw[MWE-empty] (axis cs:9.74469E+01,1.06476E+01) -- (axis cs:9.79174E+01,9.76126E+00) -- (axis cs:9.77375E+01,9.66874E+00) -- (axis cs:9.72666E+01,1.05548E+01) -- cycle ; 
\draw[MWE-empty] (axis cs:9.83854E+01,8.87430E+00) -- (axis cs:9.88512E+01,7.98676E+00) -- (axis cs:9.86719E+01,7.89468E+00) -- (axis cs:9.82058E+01,8.78200E+00) -- cycle ; 
\draw[MWE-empty] (axis cs:9.93149E+01,7.09872E+00) -- (axis cs:9.97769E+01,6.21024E+00) -- (axis cs:9.95982E+01,6.11851E+00) -- (axis cs:9.91360E+01,7.00683E+00) -- cycle ; 
\draw[MWE-empty] (axis cs:1.00237E+02,5.32137E+00) -- (axis cs:1.00697E+02,4.43217E+00) -- (axis cs:1.00518E+02,4.34070E+00) -- (axis cs:1.00059E+02,5.22978E+00) -- cycle ; 
\draw[MWE-empty] (axis cs:1.01155E+02,3.54271E+00) -- (axis cs:1.01612E+02,2.65303E+00) -- (axis cs:1.01434E+02,2.56174E+00) -- (axis cs:1.00977E+02,3.45134E+00) -- cycle ; 
\draw[MWE-empty] (axis cs:1.02069E+02,1.76320E+00) -- (axis cs:1.02525E+02,8.73274E-01) -- (axis cs:1.02347E+02,7.82062E-01) -- (axis cs:1.01891E+02,1.67196E+00) -- cycle ; 
\draw[MWE-empty] (axis cs:1.02981E+02,-1.66966E-02) -- (axis cs:1.03437E+02,-9.06652E-01) -- (axis cs:1.03259E+02,-9.97867E-01) -- (axis cs:1.02917E+02,-1.07899E-01) -- cycle ; 
\draw[MWE-empty] (axis cs:1.03893E+02,-1.79654E+00) -- (axis cs:1.04350E+02,-2.68629E+00) -- (axis cs:1.04172E+02,-2.77760E+00) -- (axis cs:1.03715E+02,-1.88779E+00) -- cycle ; 
\draw[MWE-empty] (axis cs:1.04808E+02,-3.57586E+00) -- (axis cs:1.05266E+02,-4.46519E+00) -- (axis cs:1.05088E+02,-4.55668E+00) -- (axis cs:1.04629E+02,-3.66725E+00) -- cycle ; 
\draw[MWE-empty] (axis cs:1.05726E+02,-5.35422E+00) -- (axis cs:1.06186E+02,-6.24290E+00) -- (axis cs:1.06008E+02,-6.33465E+00) -- (axis cs:1.05547E+02,-5.44583E+00) -- cycle ; 
\draw[MWE-empty] (axis cs:1.06649E+02,-7.13115E+00) -- (axis cs:1.07113E+02,-8.01893E+00) -- (axis cs:1.06934E+02,-8.11105E+00) -- (axis cs:1.06470E+02,-7.22308E+00) -- cycle ; 
\draw[MWE-empty] (axis cs:1.07580E+02,-8.90618E+00) -- (axis cs:1.08048E+02,-9.79283E+00) -- (axis cs:1.07868E+02,-9.88539E+00) -- (axis cs:1.07400E+02,-8.99851E+00) -- cycle ; 
\draw[MWE-empty] (axis cs:1.08520E+02,-1.06788E+01) -- (axis cs:1.08993E+02,-1.15641E+01) -- (axis cs:1.08813E+02,-1.16572E+01) -- (axis cs:1.08339E+02,-1.07716E+01) -- cycle ; 
\draw[MWE-empty] (axis cs:1.09471E+02,-1.24486E+01) -- (axis cs:1.09951E+02,-1.33322E+01) -- (axis cs:1.09769E+02,-1.34259E+01) -- (axis cs:1.09289E+02,-1.25420E+01) -- cycle ; 
\draw[MWE-empty] (axis cs:1.10435E+02,-1.42149E+01) -- (axis cs:1.10922E+02,-1.50966E+01) -- (axis cs:1.10740E+02,-1.51910E+01) -- (axis cs:1.10253E+02,-1.43090E+01) -- cycle ; 
\draw[MWE-empty] (axis cs:1.11414E+02,-1.59772E+01) -- (axis cs:1.11910E+02,-1.68567E+01) -- (axis cs:1.11727E+02,-1.69520E+01) -- (axis cs:1.11231E+02,-1.60721E+01) -- cycle ; 
\draw[MWE-empty] (axis cs:1.12411E+02,-1.77350E+01) -- (axis cs:1.12917E+02,-1.86119E+01) -- (axis cs:1.12732E+02,-1.87082E+01) -- (axis cs:1.12227E+02,-1.78307E+01) -- cycle ; 
\draw[MWE-empty] (axis cs:1.13428E+02,-1.94875E+01) -- (axis cs:1.13945E+02,-2.03617E+01) -- (axis cs:1.13758E+02,-2.04590E+01) -- (axis cs:1.13242E+02,-1.95843E+01) -- cycle ; 
\draw[MWE-empty] (axis cs:1.14467E+02,-2.12342E+01) -- (axis cs:1.14996E+02,-2.21052E+01) -- (axis cs:1.14808E+02,-2.22037E+01) -- (axis cs:1.14280E+02,-2.13321E+01) -- cycle ; 
\draw[MWE-empty] (axis cs:1.15531E+02,-2.29744E+01) -- (axis cs:1.16073E+02,-2.38417E+01) -- (axis cs:1.15884E+02,-2.39415E+01) -- (axis cs:1.15342E+02,-2.30735E+01) -- cycle ; 
\draw[MWE-empty] (axis cs:1.16623E+02,-2.47072E+01) -- (axis cs:1.17180E+02,-2.55705E+01) -- (axis cs:1.16989E+02,-2.56717E+01) -- (axis cs:1.16433E+02,-2.48076E+01) -- cycle ; 
\draw[MWE-empty] (axis cs:1.17746E+02,-2.64318E+01) -- (axis cs:1.18320E+02,-2.72907E+01) -- (axis cs:1.18127E+02,-2.73934E+01) -- (axis cs:1.17554E+02,-2.65337E+01) -- cycle ; 
\draw[MWE-empty] (axis cs:1.18903E+02,-2.81473E+01) -- (axis cs:1.19495E+02,-2.90013E+01) -- (axis cs:1.19300E+02,-2.91056E+01) -- (axis cs:1.18709E+02,-2.82508E+01) -- cycle ; 
\draw[MWE-empty] (axis cs:1.20098E+02,-2.98527E+01) -- (axis cs:1.20710E+02,-3.07013E+01) -- (axis cs:1.20513E+02,-3.08074E+01) -- (axis cs:1.19901E+02,-2.99579E+01) -- cycle ; 
\draw[MWE-empty] (axis cs:1.21334E+02,-3.15469E+01) -- (axis cs:1.21969E+02,-3.23894E+01) -- (axis cs:1.21769E+02,-3.24975E+01) -- (axis cs:1.21135E+02,-3.16540E+01) -- cycle ; 
\draw[MWE-empty] (axis cs:1.22616E+02,-3.32287E+01) -- (axis cs:1.23275E+02,-3.40646E+01) -- (axis cs:1.23073E+02,-3.41747E+01) -- (axis cs:1.22415E+02,-3.33378E+01) -- cycle ; 
\draw[MWE-empty] (axis cs:1.23948E+02,-3.48968E+01) -- (axis cs:1.24634E+02,-3.57252E+01) -- (axis cs:1.24430E+02,-3.58377E+01) -- (axis cs:1.23745E+02,-3.50081E+01) -- cycle ; 
\draw[MWE-empty] (axis cs:1.25335E+02,-3.65497E+01) -- (axis cs:1.26051E+02,-3.73699E+01) -- (axis cs:1.25844E+02,-3.74848E+01) -- (axis cs:1.25130E+02,-3.66633E+01) -- cycle ; 
\draw[MWE-empty] (axis cs:1.26782E+02,-3.81857E+01) -- (axis cs:1.27531E+02,-3.89969E+01) -- (axis cs:1.27322E+02,-3.91143E+01) -- (axis cs:1.26575E+02,-3.83019E+01) -- cycle ; 
\draw[MWE-empty] (axis cs:1.28296E+02,-3.98031E+01) -- (axis cs:1.29080E+02,-4.06042E+01) -- (axis cs:1.28869E+02,-4.07244E+01) -- (axis cs:1.28087E+02,-3.99220E+01) -- cycle ; 
\draw[MWE-empty] (axis cs:1.29883E+02,-4.13998E+01) -- (axis cs:1.30705E+02,-4.21897E+01) -- (axis cs:1.30492E+02,-4.23130E+01) -- (axis cs:1.29671E+02,-4.15215E+01) -- cycle ; 
\draw[MWE-empty] (axis cs:1.31549E+02,-4.29736E+01) -- (axis cs:1.32414E+02,-4.37511E+01) -- (axis cs:1.32199E+02,-4.38775E+01) -- (axis cs:1.31335E+02,-4.30984E+01) -- cycle ; 
\draw[MWE-empty] (axis cs:1.33301E+02,-4.45218E+01) -- (axis cs:1.34213E+02,-4.52855E+01) -- (axis cs:1.33996E+02,-4.54153E+01) -- (axis cs:1.33086E+02,-4.46499E+01) -- cycle ; 
\draw[MWE-empty] (axis cs:1.35149E+02,-4.60418E+01) -- (axis cs:1.36111E+02,-4.67901E+01) -- (axis cs:1.35893E+02,-4.69235E+01) -- (axis cs:1.34932E+02,-4.61733E+01) -- cycle ; 
\draw[MWE-empty] (axis cs:1.37100E+02,-4.75301E+01) -- (axis cs:1.38118E+02,-4.82614E+01) -- (axis cs:1.37899E+02,-4.83986E+01) -- (axis cs:1.36882E+02,-4.76654E+01) -- cycle ; 
\draw[MWE-empty] (axis cs:1.39165E+02,-4.89834E+01) -- (axis cs:1.40242E+02,-4.96956E+01) -- (axis cs:1.40023E+02,-4.98368E+01) -- (axis cs:1.38945E+02,-4.91226E+01) -- cycle ; 
\draw[MWE-empty] (axis cs:1.41351E+02,-5.03976E+01) -- (axis cs:1.42493E+02,-5.10886E+01) -- (axis cs:1.42274E+02,-5.12341E+01) -- (axis cs:1.41132E+02,-5.05409E+01) -- cycle ; 
\draw[MWE-empty] (axis cs:1.43670E+02,-5.17682E+01) -- (axis cs:1.44882E+02,-5.24356E+01) -- (axis cs:1.44665E+02,-5.25855E+01) -- (axis cs:1.43452E+02,-5.19158E+01) -- cycle ; 
\draw[MWE-empty] (axis cs:1.46132E+02,-5.30902E+01) -- (axis cs:1.47419E+02,-5.37314E+01) -- (axis cs:1.47204E+02,-5.38858E+01) -- (axis cs:1.45915E+02,-5.32424E+01) -- cycle ; 
\draw[MWE-empty] (axis cs:1.48746E+02,-5.43582E+01) -- (axis cs:1.50114E+02,-5.49701E+01) -- (axis cs:1.49903E+02,-5.51293E+01) -- (axis cs:1.48533E+02,-5.45151E+01) -- cycle ; 
\draw[MWE-empty] (axis cs:1.51523E+02,-5.55661E+01) -- (axis cs:1.52976E+02,-5.61454E+01) -- (axis cs:1.52770E+02,-5.63095E+01) -- (axis cs:1.51315E+02,-5.57277E+01) -- cycle ; 
\draw[MWE-empty] (axis cs:1.54472E+02,-5.67072E+01) -- (axis cs:1.56012E+02,-5.72505E+01) -- (axis cs:1.55814E+02,-5.74194E+01) -- (axis cs:1.54269E+02,-5.68737E+01) -- cycle ; 
\draw[MWE-empty] (axis cs:1.57598E+02,-5.77743E+01) -- (axis cs:1.59230E+02,-5.82777E+01) -- (axis cs:1.59042E+02,-5.84516E+01) -- (axis cs:1.57405E+02,-5.79457E+01) -- cycle ; 
\draw[MWE-empty] (axis cs:1.60908E+02,-5.87597E+01) -- (axis cs:1.62632E+02,-5.92193E+01) -- (axis cs:1.62456E+02,-5.93980E+01) -- (axis cs:1.60725E+02,-5.89360E+01) -- cycle ; 
\draw[MWE-empty] (axis cs:1.64402E+02,-5.96555E+01) -- (axis cs:1.66217E+02,-6.00672E+01) -- (axis cs:1.66056E+02,-6.02505E+01) -- (axis cs:1.64233E+02,-5.98365E+01) -- cycle ; 
\draw[MWE-empty] (axis cs:1.68077E+02,-6.04533E+01) -- (axis cs:1.69979E+02,-6.08130E+01) -- (axis cs:1.69837E+02,-6.10005E+01) -- (axis cs:1.67924E+02,-6.06388E+01) -- cycle ; 
\draw[MWE-empty] (axis cs:1.71924E+02,-6.11451E+01) -- (axis cs:1.73908E+02,-6.14487E+01) -- (axis cs:1.73786E+02,-6.16401E+01) -- (axis cs:1.71791E+02,-6.13346E+01) -- cycle ; 
\draw[MWE-empty] (axis cs:1.75929E+02,-6.17229E+01) -- (axis cs:1.77984E+02,-6.19668E+01) -- (axis cs:1.77886E+02,-6.21614E+01) -- (axis cs:1.75819E+02,-6.19160E+01) -- cycle ; 
\draw[MWE-empty] (axis cs:1.80071E+02,-6.21796E+01) -- (axis cs:1.82184E+02,-6.23606E+01) -- (axis cs:1.82113E+02,-6.25578E+01) -- (axis cs:1.79986E+02,-6.23756E+01) -- cycle ; 
\draw[MWE-empty] (axis cs:1.84321E+02,-6.25091E+01) -- (axis cs:1.86478E+02,-6.26247E+01) -- (axis cs:1.86434E+02,-6.28237E+01) -- (axis cs:1.84264E+02,-6.27073E+01) -- cycle ; 
\draw[MWE-empty] (axis cs:1.88648E+02,-6.27068E+01) -- (axis cs:1.90828E+02,-6.27553E+01) -- (axis cs:1.90814E+02,-6.29552E+01) -- (axis cs:1.88619E+02,-6.29064E+01) -- cycle ; 
\draw[MWE-empty] (axis cs:1.93013E+02,-6.27699E+01) -- (axis cs:1.95197E+02,-6.27506E+01) -- (axis cs:1.95213E+02,-6.29504E+01) -- (axis cs:1.93014E+02,-6.29699E+01) -- cycle ; 
\draw[MWE-empty] (axis cs:1.97377E+02,-6.26974E+01) -- (axis cs:1.99545E+02,-6.26105E+01) -- (axis cs:1.99591E+02,-6.28094E+01) -- (axis cs:1.97407E+02,-6.28969E+01) -- cycle ; 
\draw[MWE-empty] (axis cs:2.01699E+02,-6.24903E+01) -- (axis cs:2.03833E+02,-6.23372E+01) -- (axis cs:2.03907E+02,-6.25342E+01) -- (axis cs:2.01759E+02,-6.26884E+01) -- cycle ; 
\draw[MWE-empty] (axis cs:2.05943E+02,-6.21517E+01) -- (axis cs:2.08026E+02,-6.19345E+01) -- (axis cs:2.08125E+02,-6.21289E+01) -- (axis cs:2.06030E+02,-6.23475E+01) -- cycle ; 
\draw[MWE-empty] (axis cs:2.10076E+02,-6.16863E+01) -- (axis cs:2.12092E+02,-6.14079E+01) -- (axis cs:2.12216E+02,-6.15991E+01) -- (axis cs:2.10188E+02,-6.18792E+01) -- cycle ; 
\draw[MWE-empty] (axis cs:2.14071E+02,-6.11003E+01) -- (axis cs:2.16010E+02,-6.07642E+01) -- (axis cs:2.16154E+02,-6.09515E+01) -- (axis cs:2.14205E+02,-6.12895E+01) -- cycle ; 
\draw[MWE-empty] (axis cs:2.17906E+02,-6.04008E+01) -- (axis cs:2.19760E+02,-6.00110E+01) -- (axis cs:2.19922E+02,-6.01940E+01) -- (axis cs:2.18060E+02,-6.05860E+01) -- cycle ; 
\draw[MWE-empty] (axis cs:2.21569E+02,-5.95959E+01) -- (axis cs:2.23332E+02,-5.91564E+01) -- (axis cs:2.23509E+02,-5.93347E+01) -- (axis cs:2.21739E+02,-5.97766E+01) -- cycle ; 
\draw[MWE-empty] (axis cs:2.25050E+02,-5.86936E+01) -- (axis cs:2.26721E+02,-5.82085E+01) -- (axis cs:2.26910E+02,-5.83820E+01) -- (axis cs:2.25233E+02,-5.88695E+01) -- cycle ; 
\draw[MWE-empty] (axis cs:2.28346E+02,-5.77022E+01) -- (axis cs:2.29926E+02,-5.71755E+01) -- (axis cs:2.30125E+02,-5.73442E+01) -- (axis cs:2.28541E+02,-5.78732E+01) -- cycle ; 
\draw[MWE-empty] (axis cs:2.31460E+02,-5.66296E+01) -- (axis cs:2.32950E+02,-5.60654E+01) -- (axis cs:2.33157E+02,-5.62291E+01) -- (axis cs:2.31663E+02,-5.67958E+01) -- cycle ; 
\draw[MWE-empty] (axis cs:2.34397E+02,-5.54837E+01) -- (axis cs:2.35800E+02,-5.48854E+01) -- (axis cs:2.36012E+02,-5.50442E+01) -- (axis cs:2.34606E+02,-5.56449E+01) -- cycle ; 
\draw[MWE-empty] (axis cs:2.37162E+02,-5.42714E+01) -- (axis cs:2.38483E+02,-5.36425E+01) -- (axis cs:2.38699E+02,-5.37966E+01) -- (axis cs:2.37376E+02,-5.44279E+01) -- cycle ; 
\draw[MWE-empty] (axis cs:2.39766E+02,-5.29994E+01) -- (axis cs:2.41010E+02,-5.23429E+01) -- (axis cs:2.41227E+02,-5.24925E+01) -- (axis cs:2.39982E+02,-5.31512E+01) -- cycle ; 
\draw[MWE-empty] (axis cs:2.42217E+02,-5.16738E+01) -- (axis cs:2.43389E+02,-5.09926E+01) -- (axis cs:2.43608E+02,-5.11377E+01) -- (axis cs:2.42435E+02,-5.18211E+01) -- cycle ; 
\draw[MWE-empty] (axis cs:2.44526E+02,-5.02999E+01) -- (axis cs:2.45631E+02,-4.95965E+01) -- (axis cs:2.45850E+02,-4.97374E+01) -- (axis cs:2.44745E+02,-5.04429E+01) -- cycle ; 
\draw[MWE-empty] (axis cs:2.46704E+02,-4.88829E+01) -- (axis cs:2.47746E+02,-4.81596E+01) -- (axis cs:2.47965E+02,-4.82965E+01) -- (axis cs:2.46923E+02,-4.90218E+01) -- cycle ; 
\draw[MWE-empty] (axis cs:2.48760E+02,-4.74270E+01) -- (axis cs:2.49745E+02,-4.66858E+01) -- (axis cs:2.49963E+02,-4.68189E+01) -- (axis cs:2.48978E+02,-4.75620E+01) -- cycle ; 
\draw[MWE-empty] (axis cs:2.50704E+02,-4.59363E+01) -- (axis cs:2.51636E+02,-4.51791E+01) -- (axis cs:2.51853E+02,-4.53086E+01) -- (axis cs:2.50921E+02,-4.60676E+01) -- cycle ; 
\draw[MWE-empty] (axis cs:2.52544E+02,-4.44144E+01) -- (axis cs:2.53429E+02,-4.36426E+01) -- (axis cs:2.53644E+02,-4.37688E+01) -- (axis cs:2.52760E+02,-4.45422E+01) -- cycle ; 
\draw[MWE-empty] (axis cs:2.54291E+02,-4.28642E+01) -- (axis cs:2.55131E+02,-4.20795E+01) -- (axis cs:2.55344E+02,-4.22025E+01) -- (axis cs:2.54505E+02,-4.29888E+01) -- cycle ; 
\draw[MWE-empty] (axis cs:2.55951E+02,-4.12888E+01) -- (axis cs:2.56751E+02,-4.04924E+01) -- (axis cs:2.56962E+02,-4.06124E+01) -- (axis cs:2.56163E+02,-4.14103E+01) -- cycle ; 
\draw[MWE-empty] (axis cs:2.57532E+02,-3.96906E+01) -- (axis cs:2.58295E+02,-3.88836E+01) -- (axis cs:2.58504E+02,-3.90009E+01) -- (axis cs:2.57742E+02,-3.98092E+01) -- cycle ; 
\draw[MWE-empty] (axis cs:2.59041E+02,-3.80718E+01) -- (axis cs:2.59771E+02,-3.72554E+01) -- (axis cs:2.59977E+02,-3.73700E+01) -- (axis cs:2.59248E+02,-3.81877E+01) -- cycle ; 
\draw[MWE-empty] (axis cs:2.60484E+02,-3.64345E+01) -- (axis cs:2.61183E+02,-3.56095E+01) -- (axis cs:2.61387E+02,-3.57218E+01) -- (axis cs:2.60689E+02,-3.65480E+01) -- cycle ; 
\draw[MWE-empty] (axis cs:2.61867E+02,-3.47805E+01) -- (axis cs:2.62538E+02,-3.39478E+01) -- (axis cs:2.62740E+02,-3.40578E+01) -- (axis cs:2.62070E+02,-3.48916E+01) -- cycle ; 
\draw[MWE-empty] (axis cs:2.63196E+02,-3.31114E+01) -- (axis cs:2.63841E+02,-3.22717E+01) -- (axis cs:2.64040E+02,-3.23796E+01) -- (axis cs:2.63396E+02,-3.32204E+01) -- cycle ; 
\draw[MWE-empty] (axis cs:2.64474E+02,-3.14287E+01) -- (axis cs:2.65096E+02,-3.05826E+01) -- (axis cs:2.65293E+02,-3.06886E+01) -- (axis cs:2.64672E+02,-3.15356E+01) -- cycle ; 
\draw[MWE-empty] (axis cs:2.65707E+02,-2.97337E+01) -- (axis cs:2.66308E+02,-2.88819E+01) -- (axis cs:2.66503E+02,-2.89861E+01) -- (axis cs:2.65903E+02,-2.98387E+01) -- cycle ; 

 
\draw [MWE-empty] (axis cs:2.66016E+02,-2.87251E+01) -- (axis cs:2.66797E+02,-2.91419E+01)   ;
\node[pin={[pin distance=0.0\onedegree,MWE-label]90:{0$^\circ$}}] at (axis cs:2.66016E+02,-2.87251E+01) {} ;
\draw [MWE-empty] (axis cs:2.71569E+02,-2.00932E+01) -- (axis cs:2.72314E+02,-2.04821E+01)   ;
\node[pin={[pin distance=0.0\onedegree,MWE-label]090:{10$^\circ$}}] at (axis cs:2.71569E+02,-2.00932E+01) {} ;
\draw [MWE-empty] (axis cs:2.76522E+02,-1.13004E+01) -- (axis cs:2.77245E+02,-1.16726E+01)   ;
\node[pin={[pin distance=0.0\onedegree,MWE-label]090:{20$^\circ$}}] at (axis cs:2.76522E+02,-1.13004E+01) {} ;
\draw [MWE-empty] (axis cs:2.81167E+02,-2.42475E+00) -- (axis cs:2.81879E+02,-2.78993E+00)   ;
\node[pin={[pin distance=0.0\onedegree,MWE-label]090:{30$^\circ$}}] at (axis cs:2.81167E+02,-2.42475E+00) {} ;
\draw [MWE-empty] (axis cs:2.85739E+02,6.47217E+00) -- (axis cs:2.86454E+02,6.10515E+00)   ;
\node[pin={[pin distance=0.0\onedegree,MWE-label]090:{40$^\circ$}}] at (axis cs:2.85739E+02,6.47217E+00) {} ;
\draw [MWE-empty] (axis cs:2.90465E+02,1.53325E+01) -- (axis cs:2.91196E+02,1.49546E+01)   ;
\node[pin={[pin distance=0.0\onedegree,MWE-label]090:{50$^\circ$}}] at (axis cs:2.90465E+02,1.53325E+01) {} ;
\draw [MWE-empty] (axis cs:2.95599E+02,2.40907E+01) -- (axis cs:2.96357E+02,2.36917E+01)   ;
\node[pin={[pin distance=0.0\onedegree,MWE-label]090:{60$^\circ$}}] at (axis cs:2.95599E+02,2.40907E+01) {} ;
\draw [MWE-empty] (axis cs:3.01469E+02,3.26590E+01) -- (axis cs:3.02267E+02,3.22267E+01)   ;
\node[pin={[pin distance=0.0\onedegree,MWE-label]090:{70$^\circ$}}] at (axis cs:3.01469E+02,3.26590E+01) {} ;
\draw [MWE-empty] (axis cs:3.08552E+02,4.09041E+01) -- (axis cs:3.09394E+02,4.04231E+01)   ;
\node[pin={[pin distance=0.0\onedegree,MWE-label]090:{80$^\circ$}}] at (axis cs:3.08552E+02,4.09041E+01) {} ;
\draw [MWE-empty] (axis cs:3.17568E+02,4.86036E+01) -- (axis cs:3.18444E+02,4.80548E+01)   ;
\node[pin={[pin distance=0.0\onedegree,MWE-label]090:{90$^\circ$}}] at (axis cs:3.17568E+02,4.86036E+01) {} ;
\draw [MWE-empty] (axis cs:3.29582E+02,5.53674E+01) -- (axis cs:3.30428E+02,5.47306E+01)   ;
\node[pin={[pin distance=0.0\onedegree,MWE-label]0090:{100$^\circ$}}] at (axis cs:3.29582E+02,5.53674E+01) {} ;
\draw [MWE-empty] (axis cs:3.45811E+02,6.05250E+01) -- (axis cs:3.46455E+02,5.97919E+01)   ;
\node[pin={[pin distance=0.0\onedegree,MWE-label]0090:{110$^\circ$}}] at (axis cs:3.45811E+02,6.05250E+01) {} ;
\draw [MWE-empty] (axis cs:6.36807E+00,6.31222E+01) -- (axis cs:6.54138E+00,6.23261E+01)   ;
\node[pin={[pin distance=0.0\onedegree,MWE-label]0090:{120$^\circ$}}] at (axis cs:6.36807E+00,6.31222E+01) {} ;
\draw [MWE-empty] (axis cs:2.82769E+01,6.24205E+01) -- (axis cs:2.78784E+01,6.16426E+01)   ;
\node[pin={[pin distance=0.0\onedegree,MWE-label]0090:{130$^\circ$}}] at (axis cs:2.82769E+01,6.24205E+01) {} ;
\draw [MWE-empty] (axis cs:4.71975E+01,5.86418E+01) -- (axis cs:4.64407E+01,5.79477E+01)   ;
\node[pin={[pin distance=0.0\onedegree,MWE-label]0090:{140$^\circ$}}] at (axis cs:4.71975E+01,5.86418E+01) {} ;
\draw [MWE-empty] (axis cs:6.15567E+01,5.27160E+01) -- (axis cs:6.06857E+01,5.21179E+01)   ;
\node[pin={[pin distance=0.0\onedegree,MWE-label]0090:{150$^\circ$}}] at (axis cs:6.15567E+01,5.27160E+01) {} ;
\draw [MWE-empty] (axis cs:7.21792E+01,4.55021E+01) -- (axis cs:7.13134E+01,4.49840E+01)   ;
\node[pin={[pin distance=0.0\onedegree,MWE-label]0090:{160$^\circ$}}] at (axis cs:7.21792E+01,4.55021E+01) {} ;
\draw [MWE-empty] (axis cs:8.02868E+01,3.75414E+01) -- (axis cs:7.94627E+01,3.70827E+01)   ;
\node[pin={[pin distance=0.0\onedegree,MWE-label]0090:{170$^\circ$}}] at (axis cs:8.02868E+01,3.75414E+01) {} ;
\draw [MWE-empty] (axis cs:8.67966E+01,2.91419E+01) -- (axis cs:8.60164E+01,2.87251E+01)   ;
\node[pin={[pin distance=0.0\onedegree,MWE-label]0090:{180$^\circ$}}] at (axis cs:8.67966E+01,2.91419E+01) {} ;
\draw [MWE-empty] (axis cs:9.23144E+01,2.04821E+01) -- (axis cs:9.15691E+01,2.00932E+01)   ;
\node[pin={[pin distance=0.0\onedegree,MWE-label]090:{190$^\circ$}}] at (axis cs:9.23144E+01,2.04821E+01) {} ;
\draw [MWE-empty] (axis cs:9.72449E+01,1.16726E+01) -- (axis cs:9.65223E+01,1.13004E+01)   ;
\node[pin={[pin distance=0.0\onedegree,MWE-label]090:{200$^\circ$}}] at (axis cs:9.72449E+01,1.16726E+01) {} ;
\draw [MWE-empty] (axis cs:1.01879E+02,2.78993E+00) -- (axis cs:1.01167E+02,2.42475E+00)   ;
\node[pin={[pin distance=0.0\onedegree,MWE-label]090:{210$^\circ$}}] at (axis cs:1.01879E+02,2.78993E+00) {} ;
\draw [MWE-empty] (axis cs:1.06454E+02,-6.10515E+00) -- (axis cs:1.05739E+02,-6.47217E+00)   ;
\node[pin={[pin distance=0.0\onedegree,MWE-label]090:{220$^\circ$}}] at (axis cs:1.06454E+02,-6.10515E+00) {} ;
\draw [MWE-empty] (axis cs:1.11196E+02,-1.49546E+01) -- (axis cs:1.10466E+02,-1.53325E+01)   ;
\node[pin={[pin distance=0.0\onedegree,MWE-label]090:{230$^\circ$}}] at (axis cs:1.11196E+02,-1.49546E+01) {} ;
\draw [MWE-empty] (axis cs:1.16357E+02,-2.36917E+01) -- (axis cs:1.15599E+02,-2.40907E+01)   ;
\node[pin={[pin distance=0.0\onedegree,MWE-label]090:{240$^\circ$}}] at (axis cs:1.16357E+02,-2.36917E+01) {} ;
\draw [MWE-empty] (axis cs:1.22267E+02,-3.22267E+01) -- (axis cs:1.21469E+02,-3.26590E+01)   ;
\node[pin={[pin distance=0.0\onedegree,MWE-label]090:{250$^\circ$}}] at (axis cs:1.22267E+02,-3.22267E+01) {} ;
\draw [MWE-empty] (axis cs:1.29395E+02,-4.04231E+01) -- (axis cs:1.28552E+02,-4.09041E+01)   ;
\node[pin={[pin distance=0.0\onedegree,MWE-label]090:{260$^\circ$}}] at (axis cs:1.29395E+02,-4.04231E+01) {} ;
\draw [MWE-empty] (axis cs:1.38444E+02,-4.80548E+01) -- (axis cs:1.37568E+02,-4.86036E+01)   ;
\node[pin={[pin distance=0.0\onedegree,MWE-label]0090:{270$^\circ$}}] at (axis cs:1.38444E+02,-4.80548E+01) {} ;
\draw [MWE-empty] (axis cs:1.50428E+02,-5.47306E+01) -- (axis cs:1.49582E+02,-5.53674E+01)   ;
\node[pin={[pin distance=0.0\onedegree,MWE-label]0090:{280$^\circ$}}] at (axis cs:1.50428E+02,-5.47306E+01) {} ;
\draw [MWE-empty] (axis cs:1.66455E+02,-5.97919E+01) -- (axis cs:1.65811E+02,-6.05250E+01)   ;
\node[pin={[pin distance=0.0\onedegree,MWE-label]0090:{290$^\circ$}}] at (axis cs:1.66455E+02,-5.97919E+01) {} ;
\draw [MWE-empty] (axis cs:1.86541E+02,-6.23261E+01) -- (axis cs:1.86368E+02,-6.31222E+01)   ;
\node[pin={[pin distance=0.0\onedegree,MWE-label]0090:{300$^\circ$}}] at (axis cs:1.86541E+02,-6.23261E+01) {} ;
\draw [MWE-empty] (axis cs:2.07878E+02,-6.16426E+01) -- (axis cs:2.08277E+02,-6.24205E+01)   ;
\node[pin={[pin distance=0.0\onedegree,MWE-label]0090:{310$^\circ$}}] at (axis cs:2.07878E+02,-6.16426E+01) {} ;
\draw [MWE-empty] (axis cs:2.26441E+02,-5.79477E+01) -- (axis cs:2.27198E+02,-5.86418E+01)   ;
\node[pin={[pin distance=0.0\onedegree,MWE-label]0090:{320$^\circ$}}] at (axis cs:2.26441E+02,-5.79477E+01) {} ;
\draw [MWE-empty] (axis cs:2.40686E+02,-5.21179E+01) -- (axis cs:2.41557E+02,-5.27160E+01)   ;
\node[pin={[pin distance=0.0\onedegree,MWE-label]0090:{330$^\circ$}}] at (axis cs:2.40686E+02,-5.21179E+01) {} ;
\draw [MWE-empty] (axis cs:2.51313E+02,-4.49840E+01) -- (axis cs:2.52179E+02,-4.55021E+01)   ;
\node[pin={[pin distance=0.0\onedegree,MWE-label]0090:{340$^\circ$}}] at (axis cs:2.51313E+02,-4.49840E+01) {} ;
\draw [MWE-empty] (axis cs:2.59463E+02,-3.70827E+01) -- (axis cs:2.60287E+02,-3.75414E+01)   ;
\node[pin={[pin distance=0.0\onedegree,MWE-label]0090:{350$^\circ$}}] at (axis cs:2.59463E+02,-3.70827E+01) {} ;


\end{polaraxis}



% Constellations
%
%
% Some are commented out because they are surrounded by already plotted borders
% The x-coordinate is in fractional hours, so must be times 15
%
\begin{polaraxis}[rotate=270,name=constellations,at={($(base.center)+(+0.75pt,0pt)$)},anchor=center,axis lines=none]

\clip (6\tendegree,0\tendegree) arc (0:90:6\tendegree) -- 
(-2\tendegree,6\tendegree) -- (-2\tendegree,-6\tendegree) -- (0\tendegree,-6\tendegree)
--  (0\tendegree,-6\tendegree) arc (270:359.9999:6\tendegree) -- cycle ;

%\draw[constellation-boundary]   ++ (0,0);
\draw[constellation-boundary]  (axis cs:180.599,-35.696092) --  (axis cs:181.608,-35.696028) --  (axis cs:182.617,-35.695749) --  (axis cs:183.625,-35.695258) --  (axis cs:184.634,-35.694552) --  (axis cs:185.39,-35.693883) ;
  \draw[constellation-boundary]  (axis cs:209.111,-83.120071) --  (axis cs:208.939,-82.620525) --  (axis cs:208.657,-81.621273) --  (axis cs:208.434,-80.621862) --  (axis cs:208.254,-79.622338) --  (axis cs:208.105,-78.622731) --  (axis cs:207.98,-77.623062) --  (axis cs:207.873,-76.623344) --  (axis cs:207.781,-75.623587) --  (axis cs:207.701,-74.623800) --  (axis cs:207.63,-73.623987) --  (axis cs:207.568,-72.624153) --  (axis cs:207.511,-71.624301) --  (axis cs:207.461,-70.624435) --  (axis cs:207.415,-69.624556) --  (axis cs:207.373,-68.624666) --  (axis cs:207.335,-67.624767) --  (axis cs:207.3,-66.624859) --  (axis cs:207.268,-65.624945) ;
  \draw[constellation-boundary]  (axis cs:276.866,-82.458275) --  (axis cs:276.535,-81.960289) --  (axis cs:275.982,-80.963653) --  (axis cs:275.538,-79.966353) --  (axis cs:275.174,-78.968569) --  (axis cs:274.869,-77.970420) --  (axis cs:274.611,-76.971992) --  (axis cs:274.388,-75.973343) --  (axis cs:274.195,-74.974518) --  (axis cs:274.025,-73.975549) --  (axis cs:273.875,-72.976463) --  (axis cs:273.741,-71.977278) --  (axis cs:273.62,-70.978010) --  (axis cs:273.512,-69.978671) --  (axis cs:273.413,-68.979272) --  (axis cs:273.322,-67.979821) --  (axis cs:273.28,-67.480078) ;
  \draw[constellation-boundary]  (axis cs:274.195,-74.974518) --  (axis cs:275.192,-74.962397) --  (axis cs:276.188,-74.950293) --  (axis cs:277.183,-74.938208) --  (axis cs:278.178,-74.926147) --  (axis cs:279.172,-74.914113) --  (axis cs:280.164,-74.902109) --  (axis cs:281.157,-74.890140) --  (axis cs:282.148,-74.878208) --  (axis cs:283.139,-74.866318) --  (axis cs:284.129,-74.854473) --  (axis cs:285.118,-74.842676) --  (axis cs:286.106,-74.830931) --  (axis cs:287.094,-74.819241) --  (axis cs:288.081,-74.807610) --  (axis cs:289.067,-74.796042) --  (axis cs:290.053,-74.784538) --  (axis cs:291.038,-74.773104) --  (axis cs:292.022,-74.761742) --  (axis cs:293.005,-74.750455) --  (axis cs:293.988,-74.739247) --  (axis cs:294.97,-74.728121) --  (axis cs:295.952,-74.717080) --  (axis cs:296.932,-74.706128) --  (axis cs:297.912,-74.695267) --  (axis cs:298.892,-74.684501) --  (axis cs:299.87,-74.673833) --  (axis cs:300.848,-74.663265) --  (axis cs:301.826,-74.652801) --  (axis cs:302.803,-74.642444) --  (axis cs:303.779,-74.632197) --  (axis cs:304.754,-74.622063) --  (axis cs:305.729,-74.612043) --  (axis cs:306.704,-74.602143) --  (axis cs:307.678,-74.592363) --  (axis cs:308.651,-74.582707) --  (axis cs:309.623,-74.573178) --  (axis cs:310.595,-74.563778) --  (axis cs:311.567,-74.554510) --  (axis cs:312.538,-74.545376) --  (axis cs:313.508,-74.536379) --  (axis cs:314.478,-74.527522) --  (axis cs:315.447,-74.518807) --  (axis cs:316.416,-74.510236) --  (axis cs:317.385,-74.501813) --  (axis cs:318.352,-74.493538) --  (axis cs:319.32,-74.485414) --  (axis cs:320.287,-74.477444) --  (axis cs:321.253,-74.469630) --  (axis cs:322.219,-74.461974) --  (axis cs:323.185,-74.454478) --  (axis cs:324.15,-74.447144) --  (axis cs:325.115,-74.439975) --  (axis cs:326.079,-74.432971) --  (axis cs:327.043,-74.426136) --  (axis cs:328.006,-74.419470) --  (axis cs:328.969,-74.412976) --  (axis cs:329.932,-74.406655) --  (axis cs:330.894,-74.400510) --  (axis cs:331.856,-74.394541) --  (axis cs:332.818,-74.388751) --  (axis cs:333.78,-74.383141) --  (axis cs:334.741,-74.377713) --  (axis cs:335.701,-74.372468) --  (axis cs:336.662,-74.367407) --  (axis cs:337.622,-74.362532) --  (axis cs:338.582,-74.357844) --  (axis cs:339.541,-74.353344) --  (axis cs:340.501,-74.349034) --  (axis cs:341.46,-74.344915) --  (axis cs:342.419,-74.340987) --  (axis cs:343.378,-74.337253) --  (axis cs:344.336,-74.333712) --  (axis cs:345.294,-74.330367) --  (axis cs:346.252,-74.327217) --  (axis cs:347.21,-74.324263) --  (axis cs:348.168,-74.321508) --  (axis cs:349.126,-74.318950) --  (axis cs:350.083,-74.316591) --  (axis cs:351.041,-74.314432) --  (axis cs:351.998,-74.312472) --  (axis cs:352.955,-74.310713) --  (axis cs:353.912,-74.309156) --  (axis cs:354.869,-74.307799) --  (axis cs:355.826,-74.306645) --  (axis cs:356.783,-74.305692) --  (axis cs:357.739,-74.304942) --  (axis cs:358.696,-74.304395) --  (axis cs:359.653,-74.304050) --  (axis cs:360.61,-74.303908) --  (axis cs:361.566,-74.303969) --  (axis cs:362.523,-74.304233) --  (axis cs:363.48,-74.304699) --  (axis cs:364.436,-74.305368) ;
  \draw[constellation-boundary]  (axis cs:265.776,-67.571107) --  (axis cs:265.735,-67.071352) --  (axis cs:265.659,-66.071812) --  (axis cs:265.588,-65.072238) --  (axis cs:265.523,-64.072633) --  (axis cs:265.462,-63.073001) --  (axis cs:265.405,-62.073344) --  (axis cs:265.352,-61.073666) --  (axis cs:265.302,-60.073967) --  (axis cs:265.255,-59.074251) --  (axis cs:265.21,-58.074519) --  (axis cs:265.168,-57.074772) ;
  \draw[constellation-boundary]  (axis cs:258.471,-70.159744) --  (axis cs:258.373,-69.160319) --  (axis cs:258.284,-68.160843) --  (axis cs:258.242,-67.661088) ;
  \draw[constellation-boundary]  (axis cs:224.166,-70.511542) --  (axis cs:224.097,-69.511825) --  (axis cs:224.033,-68.512083) --  (axis cs:224.004,-68.012204) ;
  \draw[constellation-boundary]  (axis cs:207.461,-70.624435) --  (axis cs:208.491,-70.618886) --  (axis cs:209.522,-70.613144) --  (axis cs:210.552,-70.607211) --  (axis cs:211.582,-70.601088) --  (axis cs:212.611,-70.594777) --  (axis cs:213.64,-70.588281) --  (axis cs:214.669,-70.581601) --  (axis cs:215.697,-70.574741) --  (axis cs:216.725,-70.567701) --  (axis cs:217.753,-70.560486) --  (axis cs:218.78,-70.553096) --  (axis cs:219.807,-70.545534) --  (axis cs:220.833,-70.537803) --  (axis cs:221.859,-70.529906) --  (axis cs:222.885,-70.521845) --  (axis cs:223.91,-70.513622) --  (axis cs:224.935,-70.505241) --  (axis cs:225.959,-70.496703) --  (axis cs:226.983,-70.488013) --  (axis cs:228.007,-70.479172) --  (axis cs:229.03,-70.470184) --  (axis cs:230.052,-70.461052) --  (axis cs:231.074,-70.451778) --  (axis cs:232.096,-70.442365) --  (axis cs:233.117,-70.432818) --  (axis cs:234.137,-70.423137) --  (axis cs:235.157,-70.413328) --  (axis cs:236.177,-70.403393) --  (axis cs:237.196,-70.393334) --  (axis cs:238.215,-70.383157) --  (axis cs:239.233,-70.372863) --  (axis cs:240.25,-70.362456) --  (axis cs:241.267,-70.351939) --  (axis cs:242.283,-70.341316) --  (axis cs:243.299,-70.330590) --  (axis cs:244.315,-70.319765) --  (axis cs:245.33,-70.308844) --  (axis cs:246.344,-70.297830) --  (axis cs:247.357,-70.286728) --  (axis cs:248.371,-70.275539) --  (axis cs:249.383,-70.264269) --  (axis cs:250.395,-70.252921) --  (axis cs:251.407,-70.241497) --  (axis cs:252.417,-70.230003) --  (axis cs:253.428,-70.218441) --  (axis cs:254.438,-70.206815) --  (axis cs:255.447,-70.195128) --  (axis cs:256.455,-70.183385) --  (axis cs:257.463,-70.171590) --  (axis cs:258.471,-70.159744) ;
  \draw[constellation-boundary]  (axis cs:248.571,-45.767049) --  (axis cs:249.261,-45.759299) --  (axis cs:250.266,-45.747960) --  (axis cs:251.27,-45.736546) --  (axis cs:252.274,-45.725061) --  (axis cs:253.278,-45.713509) --  (axis cs:254.281,-45.701892) --  (axis cs:255.285,-45.690215) --  (axis cs:256.288,-45.678481) --  (axis cs:257.291,-45.666694) --  (axis cs:258.293,-45.654857) --  (axis cs:259.296,-45.642974) --  (axis cs:260.298,-45.631049) --  (axis cs:261.3,-45.619085) --  (axis cs:262.302,-45.607085) --  (axis cs:263.304,-45.595055) --  (axis cs:264.305,-45.582997) --  (axis cs:265.306,-45.570914) --  (axis cs:266.307,-45.558812) --  (axis cs:267.308,-45.546692) --  (axis cs:268.309,-45.534560) --  (axis cs:269.309,-45.522419) --  (axis cs:270.309,-45.510272) --  (axis cs:271.309,-45.498123) --  (axis cs:272.309,-45.485976) --  (axis cs:273.308,-45.473835) --  (axis cs:274.308,-45.461703) --  (axis cs:275.307,-45.449584) --  (axis cs:276.306,-45.437482) --  (axis cs:277.304,-45.425400) --  (axis cs:278.303,-45.413342) --  (axis cs:279.301,-45.401311) --  (axis cs:280.299,-45.389312) --  (axis cs:281.297,-45.377348) --  (axis cs:282.294,-45.365422) --  (axis cs:283.292,-45.353539) --  (axis cs:284.289,-45.341701) --  (axis cs:285.286,-45.329912) --  (axis cs:286.283,-45.318177) --  (axis cs:287.279,-45.306497) --  (axis cs:288.275,-45.294878) --  (axis cs:289.272,-45.283322) --  (axis cs:290.268,-45.271833) --  (axis cs:291.263,-45.260414) --  (axis cs:292.259,-45.249070) --  (axis cs:293.254,-45.237802) --  (axis cs:294.249,-45.226615) --  (axis cs:295.244,-45.215512) --  (axis cs:296.239,-45.204496) --  (axis cs:297.234,-45.193571) --  (axis cs:298.228,-45.182740) --  (axis cs:299.222,-45.172006) --  (axis cs:300.216,-45.161372) --  (axis cs:301.21,-45.150841) --  (axis cs:302.204,-45.140418) --  (axis cs:303.197,-45.130104) --  (axis cs:304.19,-45.119903) --  (axis cs:305.184,-45.109817) --  (axis cs:306.177,-45.099851) --  (axis cs:307.169,-45.090007) --  (axis cs:308.162,-45.080287) --  (axis cs:309.154,-45.070695) --  (axis cs:310.147,-45.061234) --  (axis cs:311.139,-45.051906) --  (axis cs:312.131,-45.042714) --  (axis cs:313.123,-45.033661) --  (axis cs:314.114,-45.024750) --  (axis cs:315.106,-45.015983) --  (axis cs:316.097,-45.007363) --  (axis cs:317.088,-44.998892) --  (axis cs:318.079,-44.990573) --  (axis cs:319.07,-44.982408) --  (axis cs:320.061,-44.974400) --  (axis cs:321.052,-44.966552) --  (axis cs:322.042,-44.958865) ;
  \draw[constellation-boundary]  (axis cs:269.809,-45.516346) --  (axis cs:269.797,-45.016420) --  (axis cs:269.773,-44.016565) --  (axis cs:269.75,-43.016706) --  (axis cs:269.728,-42.016841) --  (axis cs:269.706,-41.016973) --  (axis cs:269.685,-40.017101) --  (axis cs:269.665,-39.017224) --  (axis cs:269.645,-38.017345) --  (axis cs:269.625,-37.017462) --  (axis cs:269.607,-36.017576) --  (axis cs:269.588,-35.017688) --  (axis cs:269.57,-34.017796) --  (axis cs:269.553,-33.017902) --  (axis cs:269.536,-32.018006) --  (axis cs:269.519,-31.018107) --  (axis cs:269.503,-30.018207) ;
  \draw[constellation-boundary]  (axis cs:272.672,-56.983770) --  (axis cs:272.632,-55.984012) --  (axis cs:272.595,-54.984242) --  (axis cs:272.559,-53.984461) --  (axis cs:272.524,-52.984669) --  (axis cs:272.491,-51.984868) --  (axis cs:272.46,-50.985058) --  (axis cs:272.43,-49.985241) --  (axis cs:272.401,-48.985415) --  (axis cs:272.374,-47.985583) --  (axis cs:272.347,-46.985745) --  (axis cs:272.321,-45.985901) --  (axis cs:272.309,-45.485976) ;
  \draw[constellation-boundary]  (axis cs:265.168,-57.074772) --  (axis cs:265.669,-57.068725) --  (axis cs:266.67,-57.056616) --  (axis cs:267.672,-57.044491) --  (axis cs:268.672,-57.032355) --  (axis cs:269.673,-57.020212) --  (axis cs:270.673,-57.008064) --  (axis cs:271.673,-56.995915) --  (axis cs:272.672,-56.983770) --  (axis cs:273.671,-56.971632) --  (axis cs:274.67,-56.959504) --  (axis cs:275.669,-56.947391) --  (axis cs:276.667,-56.935295) --  (axis cs:277.665,-56.923221) --  (axis cs:278.662,-56.911173) --  (axis cs:279.66,-56.899153) --  (axis cs:280.656,-56.887166) --  (axis cs:281.653,-56.875215) --  (axis cs:282.649,-56.863304) --  (axis cs:283.645,-56.851436) --  (axis cs:284.641,-56.839615) --  (axis cs:285.636,-56.827845) --  (axis cs:286.631,-56.816129) --  (axis cs:287.626,-56.804470) --  (axis cs:288.62,-56.792872) --  (axis cs:289.614,-56.781339) --  (axis cs:290.608,-56.769874) --  (axis cs:291.602,-56.758480) --  (axis cs:292.595,-56.747161) --  (axis cs:293.588,-56.735920) --  (axis cs:294.58,-56.724761) --  (axis cs:295.573,-56.713686) --  (axis cs:296.565,-56.702700) --  (axis cs:297.556,-56.691805) --  (axis cs:298.548,-56.681005) --  (axis cs:299.539,-56.670303) --  (axis cs:300.53,-56.659701) --  (axis cs:301.521,-56.649204) --  (axis cs:302.511,-56.638814) --  (axis cs:303.501,-56.628535) --  (axis cs:304.491,-56.618368) --  (axis cs:305.48,-56.608319) --  (axis cs:306.47,-56.598388) --  (axis cs:307.459,-56.588580) ;
  \draw[constellation-boundary]  (axis cs:255.725,-67.690579) --  (axis cs:256.229,-67.684707) --  (axis cs:257.236,-67.672923) --  (axis cs:258.242,-67.661088) --  (axis cs:259.249,-67.649208) --  (axis cs:260.254,-67.637284) --  (axis cs:261.259,-67.625322) --  (axis cs:262.264,-67.613324) --  (axis cs:263.268,-67.601294) --  (axis cs:264.271,-67.589237) --  (axis cs:265.274,-67.577155) --  (axis cs:266.277,-67.565053) --  (axis cs:267.279,-67.552935) --  (axis cs:268.28,-67.540803) --  (axis cs:269.281,-67.528662) --  (axis cs:270.282,-67.516515) --  (axis cs:271.282,-67.504366) --  (axis cs:272.281,-67.492219) --  (axis cs:273.28,-67.480078) ;
  \draw[constellation-boundary]  (axis cs:255.725,-67.690579) --  (axis cs:255.685,-67.190809) --  (axis cs:255.611,-66.191242) --  (axis cs:255.542,-65.191642) ;
  \draw[constellation-boundary]  (axis cs:254.283,-65.206248) --  (axis cs:255.039,-65.197495) --  (axis cs:255.542,-65.191642) ;
  \draw[constellation-boundary]  (axis cs:254.283,-65.206248) --  (axis cs:254.22,-64.206614) --  (axis cs:254.195,-63.790092) ;
  \draw[constellation-boundary]  (axis cs:251.676,-63.818997) --  (axis cs:251.928,-63.816125) --  (axis cs:252.936,-63.804596) --  (axis cs:253.943,-63.793000) --  (axis cs:254.195,-63.790092) ;
  \draw[constellation-boundary]  (axis cs:251.676,-63.818997) --  (axis cs:251.643,-63.235854) --  (axis cs:251.589,-62.236163) --  (axis cs:251.538,-61.236452) ;
  \draw[constellation-boundary]  (axis cs:249.082,-61.264189) --  (axis cs:249.775,-61.256411) --  (axis cs:250.782,-61.245034) --  (axis cs:251.538,-61.236452) ;
  \draw[constellation-boundary]  (axis cs:249.082,-61.264189) --  (axis cs:249.035,-60.264452) --  (axis cs:248.991,-59.264699) --  (axis cs:248.949,-58.264932) --  (axis cs:248.91,-57.265152) --  (axis cs:248.872,-56.265361) --  (axis cs:248.837,-55.265559) --  (axis cs:248.803,-54.265747) --  (axis cs:248.771,-53.265926) --  (axis cs:248.741,-52.266097) --  (axis cs:248.711,-51.266261) --  (axis cs:248.683,-50.266418) --  (axis cs:248.657,-49.266568) --  (axis cs:248.631,-48.266712) --  (axis cs:248.606,-47.266851) --  (axis cs:248.582,-46.266984) --  (axis cs:248.559,-45.267113) --  (axis cs:248.537,-44.267237) --  (axis cs:248.516,-43.267357) --  (axis cs:248.495,-42.267474) ;
  \draw[constellation-boundary]  (axis cs:301.916,-27.641919) --  (axis cs:301.903,-26.641988) --  (axis cs:301.89,-25.642055) ;
  \draw[constellation-boundary]  (axis cs:301.916,-27.641919) --  (axis cs:302.913,-27.631573) --  (axis cs:303.909,-27.621340) --  (axis cs:304.905,-27.611221) --  (axis cs:305.902,-27.601222) --  (axis cs:306.898,-27.591344) --  (axis cs:307.894,-27.581590) --  (axis cs:308.89,-27.571963) --  (axis cs:309.886,-27.562467) --  (axis cs:310.882,-27.553104) --  (axis cs:311.878,-27.543877) --  (axis cs:312.874,-27.534789) --  (axis cs:313.869,-27.525842) --  (axis cs:314.865,-27.517039) --  (axis cs:315.86,-27.508384) --  (axis cs:316.856,-27.499877) --  (axis cs:317.851,-27.491523) --  (axis cs:318.846,-27.483323) --  (axis cs:319.841,-27.475279) --  (axis cs:320.837,-27.467396) --  (axis cs:321.832,-27.459674) ;
  \draw[constellation-boundary]  (axis cs:180.596,-64.696092) --  (axis cs:181.621,-64.696026) --  (axis cs:182.647,-64.695744) --  (axis cs:183.673,-64.695244) --  (axis cs:184.698,-64.694526) --  (axis cs:185.724,-64.693592) --  (axis cs:186.749,-64.692441) --  (axis cs:187.775,-64.691074) --  (axis cs:188.8,-64.689491) --  (axis cs:189.825,-64.687693) --  (axis cs:190.851,-64.685680) --  (axis cs:191.876,-64.683453) --  (axis cs:192.901,-64.681013) --  (axis cs:193.926,-64.678360) --  (axis cs:194.951,-64.675495) --  (axis cs:195.976,-64.672420) --  (axis cs:197,-64.669135) --  (axis cs:198.025,-64.665642) --  (axis cs:199.049,-64.661941) --  (axis cs:200.073,-64.658034) --  (axis cs:201.097,-64.653922) --  (axis cs:202.121,-64.649606) --  (axis cs:203.145,-64.645088) --  (axis cs:204.169,-64.640369) --  (axis cs:205.192,-64.635451) --  (axis cs:206.215,-64.630335) --  (axis cs:207.238,-64.625024) --  (axis cs:208.261,-64.619518) --  (axis cs:209.283,-64.613820) --  (axis cs:210.306,-64.607931) --  (axis cs:211.328,-64.601854) --  (axis cs:212.35,-64.595590) --  (axis cs:213.371,-64.589141) --  (axis cs:214.393,-64.582510) --  (axis cs:215.414,-64.575699) --  (axis cs:216.434,-64.568709) --  (axis cs:217.455,-64.561544) --  (axis cs:218.475,-64.554205) --  (axis cs:219.495,-64.546694) --  (axis cs:220.515,-64.539015) ;
  \draw[constellation-boundary]  (axis cs:180.597,-55.696092) --  (axis cs:181.615,-55.696027) --  (axis cs:182.633,-55.695746) --  (axis cs:183.651,-55.695250) --  (axis cs:184.668,-55.694538) --  (axis cs:185.686,-55.693611) --  (axis cs:186.704,-55.692469) --  (axis cs:187.721,-55.691113) --  (axis cs:188.739,-55.689542) --  (axis cs:189.756,-55.687757) --  (axis cs:190.774,-55.685759) --  (axis cs:191.791,-55.683549) --  (axis cs:192.809,-55.681127) --  (axis cs:193.826,-55.678494) --  (axis cs:194.335,-55.677099) ;
  \draw[constellation-boundary]  (axis cs:194.438,-64.676954) --  (axis cs:194.424,-63.676975) --  (axis cs:194.41,-62.676994) --  (axis cs:194.397,-61.677012) --  (axis cs:194.385,-60.677028) --  (axis cs:194.374,-59.677044) --  (axis cs:194.363,-58.677059) --  (axis cs:194.353,-57.677073) --  (axis cs:194.344,-56.677086) --  (axis cs:194.335,-55.677099) ;
  \draw[constellation-boundary]  (axis cs:220.515,-64.539015) --  (axis cs:220.475,-63.539166) --  (axis cs:220.438,-62.539307) --  (axis cs:220.404,-61.539439) --  (axis cs:220.371,-60.539563) --  (axis cs:220.341,-59.539679) --  (axis cs:220.312,-58.539789) --  (axis cs:220.285,-57.539892) --  (axis cs:220.259,-56.539990) --  (axis cs:220.234,-55.540083) ;
  \draw[constellation-boundary]  (axis cs:214.657,-55.579949) --  (axis cs:214.636,-54.580017) --  (axis cs:214.617,-53.580081) --  (axis cs:214.599,-52.580143) --  (axis cs:214.581,-51.580202) --  (axis cs:214.564,-50.580258) --  (axis cs:214.548,-49.580312) --  (axis cs:214.532,-48.580364) --  (axis cs:214.517,-47.580414) --  (axis cs:214.503,-46.580462) --  (axis cs:214.489,-45.580508) --  (axis cs:214.476,-44.580553) --  (axis cs:214.463,-43.580596) --  (axis cs:214.45,-42.580638) ;
  \draw[constellation-boundary]  (axis cs:214.657,-55.579949) --  (axis cs:215.164,-55.576542) --  (axis cs:216.179,-55.569597) --  (axis cs:217.193,-55.562475) --  (axis cs:218.207,-55.555181) --  (axis cs:219.221,-55.547716) --  (axis cs:220.234,-55.540083) --  (axis cs:221.248,-55.532284) --  (axis cs:222.261,-55.524321) --  (axis cs:223.274,-55.516197) --  (axis cs:224.287,-55.507914) --  (axis cs:225.3,-55.499476) --  (axis cs:226.312,-55.490885) --  (axis cs:227.324,-55.482143) --  (axis cs:228.336,-55.473254) --  (axis cs:229.348,-55.464220) --  (axis cs:230.36,-55.455044) --  (axis cs:231.371,-55.445728) --  (axis cs:232.382,-55.436277) ;
  \draw[constellation-boundary]  (axis cs:214.45,-42.580638) --  (axis cs:214.955,-42.577249) --  (axis cs:215.964,-42.570341) --  (axis cs:216.973,-42.563257) --  (axis cs:217.982,-42.556000) --  (axis cs:218.991,-42.548574) --  (axis cs:219.999,-42.540979) --  (axis cs:221.008,-42.533218) --  (axis cs:222.016,-42.525294) --  (axis cs:223.024,-42.517209) --  (axis cs:224.032,-42.508966) --  (axis cs:225.04,-42.500567) --  (axis cs:225.796,-42.494168) ;
  \draw[constellation-boundary]  (axis cs:225.796,-42.494168) --  (axis cs:225.781,-41.494233) --  (axis cs:225.766,-40.494297) --  (axis cs:225.751,-39.494359) --  (axis cs:225.737,-38.494419) --  (axis cs:225.723,-37.494478) --  (axis cs:225.71,-36.494535) --  (axis cs:225.697,-35.494590) --  (axis cs:225.684,-34.494644) --  (axis cs:225.672,-33.494697) --  (axis cs:225.66,-32.494749) --  (axis cs:225.648,-31.494799) --  (axis cs:225.636,-30.494849) --  (axis cs:225.625,-29.494897) --  (axis cs:225.614,-28.494945) --  (axis cs:225.603,-27.494991) --  (axis cs:225.592,-26.495037) --  (axis cs:225.582,-25.495082) ;
  \draw[constellation-boundary]  (axis cs:190.417,-30.186382) --  (axis cs:190.669,-30.185868) --  (axis cs:191.676,-30.183680) --  (axis cs:192.683,-30.181284) --  (axis cs:193.69,-30.178678) --  (axis cs:194.697,-30.175864) --  (axis cs:195.703,-30.172843) --  (axis cs:196.71,-30.169615) --  (axis cs:197.717,-30.166183) --  (axis cs:198.723,-30.162546) --  (axis cs:199.73,-30.158706) --  (axis cs:200.737,-30.154664) --  (axis cs:201.743,-30.150421) --  (axis cs:202.75,-30.145980) --  (axis cs:203.756,-30.141340) --  (axis cs:204.762,-30.136504) --  (axis cs:205.769,-30.131472) --  (axis cs:206.775,-30.126248) --  (axis cs:207.781,-30.120831) --  (axis cs:208.788,-30.115225) --  (axis cs:209.794,-30.109430) --  (axis cs:210.8,-30.103448) --  (axis cs:211.806,-30.097282) --  (axis cs:212.812,-30.090933) --  (axis cs:213.817,-30.084403) --  (axis cs:214.823,-30.077694) --  (axis cs:215.829,-30.070808) --  (axis cs:216.835,-30.063748) --  (axis cs:217.84,-30.056516) --  (axis cs:218.846,-30.049113) --  (axis cs:219.851,-30.041542) --  (axis cs:220.856,-30.033806) --  (axis cs:221.862,-30.025907) --  (axis cs:222.867,-30.017846) --  (axis cs:223.872,-30.009628) --  (axis cs:224.877,-30.001254) --  (axis cs:225.882,-29.992727) --  (axis cs:226.887,-29.984050) --  (axis cs:227.892,-29.975224) --  (axis cs:228.896,-29.966254) --  (axis cs:229.901,-29.957141) --  (axis cs:230.905,-29.947889) --  (axis cs:231.91,-29.938501) --  (axis cs:232.914,-29.928978) --  (axis cs:233.918,-29.919325) --  (axis cs:234.922,-29.909544) --  (axis cs:235.926,-29.899639) --  (axis cs:236.93,-29.889612) --  (axis cs:237.934,-29.879466) --  (axis cs:238.937,-29.869204) --  (axis cs:239.941,-29.858830) --  (axis cs:240.944,-29.848347) --  (axis cs:241.948,-29.837759) ;
  \draw[constellation-boundary]  (axis cs:190.427,-33.686372) --  (axis cs:190.424,-32.686375) --  (axis cs:190.422,-31.686377) --  (axis cs:190.419,-30.686380) --  (axis cs:190.417,-30.186382) ;
  \draw[constellation-boundary]  (axis cs:185.387,-33.693884) --  (axis cs:185.639,-33.693635) --  (axis cs:186.647,-33.692504) --  (axis cs:187.655,-33.691160) --  (axis cs:188.663,-33.689604) --  (axis cs:189.671,-33.687836) --  (axis cs:190.427,-33.686372) ;
  \draw[constellation-boundary]  (axis cs:185.39,-35.693883) --  (axis cs:185.389,-34.693884) --  (axis cs:185.387,-33.693884) ;
  \draw[constellation-boundary]  (axis cs:180.581,-83.196091) --  (axis cs:181.682,-83.196021) --  (axis cs:182.784,-83.195717) --  (axis cs:183.886,-83.195180) --  (axis cs:184.987,-83.194410) --  (axis cs:186.089,-83.193407) --  (axis cs:187.19,-83.192171) --  (axis cs:188.291,-83.190703) --  (axis cs:189.391,-83.189004) --  (axis cs:190.492,-83.187074) --  (axis cs:191.592,-83.184914) --  (axis cs:192.691,-83.182525) --  (axis cs:193.79,-83.179908) --  (axis cs:194.889,-83.177064) --  (axis cs:195.987,-83.173994) --  (axis cs:197.085,-83.170699) --  (axis cs:198.182,-83.167181) --  (axis cs:199.278,-83.163441) --  (axis cs:200.374,-83.159481) --  (axis cs:201.469,-83.155301) --  (axis cs:202.563,-83.150905) --  (axis cs:203.657,-83.146293) --  (axis cs:204.749,-83.141468) --  (axis cs:205.841,-83.136431) --  (axis cs:206.932,-83.131185) --  (axis cs:208.022,-83.125731) --  (axis cs:209.111,-83.120071) --  (axis cs:210.199,-83.114208) --  (axis cs:211.286,-83.108144) --  (axis cs:212.372,-83.101882) --  (axis cs:213.457,-83.095424) --  (axis cs:214.541,-83.088772) --  (axis cs:215.624,-83.081929) --  (axis cs:216.705,-83.074898) --  (axis cs:217.786,-83.067681) --  (axis cs:218.865,-83.060281) --  (axis cs:219.943,-83.052701) --  (axis cs:221.019,-83.044944) --  (axis cs:222.094,-83.037013) --  (axis cs:223.168,-83.028911) --  (axis cs:224.241,-83.020640) --  (axis cs:225.312,-83.012205) --  (axis cs:226.382,-83.003607) --  (axis cs:227.451,-82.994851) --  (axis cs:228.518,-82.985940) --  (axis cs:229.584,-82.976876) --  (axis cs:230.648,-82.967663) --  (axis cs:231.71,-82.958306) --  (axis cs:232.772,-82.948806) --  (axis cs:233.831,-82.939168) --  (axis cs:234.889,-82.929394) --  (axis cs:235.946,-82.919489) --  (axis cs:237.001,-82.909457) --  (axis cs:238.055,-82.899299) --  (axis cs:239.106,-82.889021) --  (axis cs:240.157,-82.878626) --  (axis cs:241.205,-82.868118) --  (axis cs:242.252,-82.857500) --  (axis cs:243.298,-82.846776) --  (axis cs:244.342,-82.835949) --  (axis cs:245.384,-82.825024) --  (axis cs:246.424,-82.814005) --  (axis cs:247.463,-82.802894) --  (axis cs:248.5,-82.791696) --  (axis cs:249.536,-82.780415) --  (axis cs:250.57,-82.769054) --  (axis cs:251.602,-82.757617) --  (axis cs:252.633,-82.746109) --  (axis cs:253.661,-82.734532) --  (axis cs:254.689,-82.722891) --  (axis cs:255.714,-82.711190) --  (axis cs:256.738,-82.699432) --  (axis cs:257.76,-82.687622) --  (axis cs:258.781,-82.675762) --  (axis cs:259.799,-82.663858) --  (axis cs:260.816,-82.651913) --  (axis cs:261.832,-82.639930) --  (axis cs:262.846,-82.627913) --  (axis cs:263.858,-82.615867) --  (axis cs:264.868,-82.603795) --  (axis cs:265.877,-82.591701) --  (axis cs:266.884,-82.579588) --  (axis cs:267.889,-82.567461) --  (axis cs:268.893,-82.555323) --  (axis cs:269.896,-82.543178) --  (axis cs:270.896,-82.531029) --  (axis cs:271.895,-82.518881) --  (axis cs:272.892,-82.506737) --  (axis cs:273.888,-82.494601) --  (axis cs:274.882,-82.482476) --  (axis cs:275.875,-82.470366) --  (axis cs:276.866,-82.458275) ;
  \draw[constellation-boundary]  (axis cs:204.707,-65.637869) --  (axis cs:204.68,-64.637935) ;
  \draw[constellation-boundary]  (axis cs:204.707,-65.637869) --  (axis cs:205.22,-65.635383) --  (axis cs:206.244,-65.630262) --  (axis cs:207.268,-65.624945) ;
  \draw[constellation-boundary]  (axis cs:224.004,-68.012204) --  (axis cs:224.77,-68.005922) --  (axis cs:225.791,-67.997410) --  (axis cs:226.557,-67.990925) ;
  \draw[constellation-boundary]  (axis cs:226.557,-67.990925) --  (axis cs:226.527,-67.491052) --  (axis cs:226.472,-66.491290) --  (axis cs:226.42,-65.491509) --  (axis cs:226.372,-64.491713) --  (axis cs:226.353,-64.075126) ;
  \draw[constellation-boundary]  (axis cs:226.353,-64.075126) --  (axis cs:226.608,-64.072953) --  (axis cs:227.625,-64.064168) --  (axis cs:228.642,-64.055235) --  (axis cs:229.658,-64.046157) --  (axis cs:230.166,-64.041565) ;
  \draw[constellation-boundary]  (axis cs:230.166,-64.041565) --  (axis cs:230.139,-63.458354) --  (axis cs:230.096,-62.458554) --  (axis cs:230.054,-61.458740) ;
  \draw[constellation-boundary]  (axis cs:230.054,-61.458740) --  (axis cs:230.562,-61.454120) --  (axis cs:231.576,-61.444777) --  (axis cs:232.59,-61.435298) ;
  \draw[constellation-boundary]  (axis cs:232.59,-61.435298) --  (axis cs:232.55,-60.435486) --  (axis cs:232.512,-59.435663) --  (axis cs:232.477,-58.435829) --  (axis cs:232.444,-57.435987) --  (axis cs:232.412,-56.436136) --  (axis cs:232.382,-55.436277) ;
  \draw[constellation-boundary]  (axis cs:232.55,-60.435486) --  (axis cs:233.563,-60.425880) --  (axis cs:234.576,-60.416143) --  (axis cs:235.588,-60.406280) --  (axis cs:236.6,-60.396292) --  (axis cs:237.612,-60.386185) --  (axis cs:238.623,-60.375959) --  (axis cs:239.635,-60.365620) --  (axis cs:240.645,-60.355170) --  (axis cs:241.656,-60.344612) --  (axis cs:242.666,-60.333949) --  (axis cs:243.676,-60.323186) --  (axis cs:244.685,-60.312325) --  (axis cs:245.694,-60.301370) --  (axis cs:246.703,-60.290324) --  (axis cs:247.712,-60.279191) --  (axis cs:248.72,-60.267974) --  (axis cs:249.035,-60.264452) ;
  \draw[constellation-boundary]  (axis cs:228.083,-55.475490) --  (axis cs:228.057,-54.475608) ;
  \draw[constellation-boundary]  (axis cs:269.625,-37.017462) --  (axis cs:270.126,-37.011388) --  (axis cs:271.125,-36.999240) --  (axis cs:272.125,-36.987092) --  (axis cs:273.125,-36.974950) --  (axis cs:274.124,-36.962816) --  (axis cs:275.124,-36.950694) --  (axis cs:276.123,-36.938588) --  (axis cs:277.122,-36.926502) --  (axis cs:278.121,-36.914440) --  (axis cs:279.119,-36.902404) --  (axis cs:280.118,-36.890399) --  (axis cs:281.116,-36.878428) --  (axis cs:282.115,-36.866495) --  (axis cs:283.113,-36.854604) --  (axis cs:284.11,-36.842757) --  (axis cs:285.108,-36.830960) --  (axis cs:286.106,-36.819214) --  (axis cs:287.103,-36.807525) --  (axis cs:288.101,-36.795895) --  (axis cs:289.098,-36.784328) --  (axis cs:289.596,-36.778569) ;
  \draw[constellation-boundary]  (axis cs:289.77,-45.277569) --  (axis cs:289.758,-44.777636) --  (axis cs:289.736,-43.777765) --  (axis cs:289.714,-42.777891) --  (axis cs:289.693,-41.778012) --  (axis cs:289.672,-40.778130) --  (axis cs:289.652,-39.778245) --  (axis cs:289.633,-38.778356) --  (axis cs:289.614,-37.778464) --  (axis cs:289.596,-36.778569) ;
  \draw[constellation-boundary]  (axis cs:190.405,-25.186395) ;
  \draw[constellation-boundary]  (axis cs:180.6,-25.196092) --  (axis cs:181.605,-25.196028) --  (axis cs:182.611,-25.195750) --  (axis cs:183.617,-25.195260) --  (axis cs:184.622,-25.194557) --  (axis cs:185.628,-25.193641) --  (axis cs:186.634,-25.192512) --  (axis cs:187.639,-25.191171) --  (axis cs:188.645,-25.189619) --  (axis cs:189.65,-25.187856) --  (axis cs:190.405,-25.186395) ;
  \draw[constellation-boundary]  (axis cs:321.928,-36.459303) --  (axis cs:322.921,-36.451761) --  (axis cs:323.914,-36.444385) --  (axis cs:324.906,-36.437177) --  (axis cs:325.899,-36.430140) --  (axis cs:326.892,-36.423276) --  (axis cs:327.884,-36.416586) --  (axis cs:328.876,-36.410073) --  (axis cs:329.869,-36.403738) --  (axis cs:330.861,-36.397584) --  (axis cs:331.853,-36.391612) --  (axis cs:332.845,-36.385824) --  (axis cs:333.837,-36.380222) --  (axis cs:334.829,-36.374807) --  (axis cs:335.821,-36.369582) --  (axis cs:336.813,-36.364547) --  (axis cs:337.804,-36.359704) --  (axis cs:338.796,-36.355055) --  (axis cs:339.788,-36.350601) --  (axis cs:340.779,-36.346343) --  (axis cs:341.771,-36.342282) --  (axis cs:342.762,-36.338421) --  (axis cs:343.754,-36.334759) --  (axis cs:344.745,-36.331298) --  (axis cs:345.736,-36.328039) --  (axis cs:346.727,-36.324983) --  (axis cs:347.719,-36.322131) --  (axis cs:348.71,-36.319484) --  (axis cs:349.701,-36.317042) --  (axis cs:350.692,-36.314806) --  (axis cs:351.683,-36.312778) ;

\draw[constellation-boundary]  (axis cs:322.117,-49.458577) --  (axis cs:322.1,-48.458645) --  (axis cs:322.082,-47.458711) --  (axis cs:322.066,-46.458774) --  (axis cs:322.05,-45.458835) --  (axis cs:322.035,-44.458894) --  (axis cs:322.02,-43.458951) --  (axis cs:322.006,-42.459006) --  (axis cs:321.992,-41.459059) --  (axis cs:321.978,-40.459111) --  (axis cs:321.965,-39.459161) --  (axis cs:321.952,-38.459210) --  (axis cs:321.94,-37.459257) --  (axis cs:321.928,-36.459303) --  (axis cs:321.916,-35.459348) --  (axis cs:321.905,-34.459392) --  (axis cs:321.894,-33.459435) --  (axis cs:321.883,-32.459477) --  (axis cs:321.872,-31.459518) --  (axis cs:321.862,-30.459558) --  (axis cs:321.852,-29.459597) --  (axis cs:321.841,-28.459636) --  (axis cs:321.832,-27.459674) --  (axis cs:321.822,-26.459711) --  (axis cs:321.812,-25.459747) ;


\draw[constellation-boundary]  (axis cs:322.117,-49.458577) --  (axis cs:323.106,-49.451066) --  (axis cs:324.095,-49.443720) --  (axis cs:325.083,-49.436543) --  (axis cs:326.072,-49.429536) --  (axis cs:327.06,-49.422702) --  (axis cs:328.048,-49.416041) --  (axis cs:329.036,-49.409557) --  (axis cs:330.024,-49.403251) --  (axis cs:331.011,-49.397124) --  (axis cs:331.999,-49.391180) ;
  \draw[constellation-boundary] (axis cs:332.114, -56.390839) -- (axis
  cs:333.098, -56.385099) -- (axis cs:334.081, -56.379544) -- (axis
  cs:335.065, -56.374175) -- (axis cs:336.048, -56.368994) -- (axis
  cs:337.031, -56.364002) -- (axis cs:338.015, -56.359201) -- (axis
  cs:338.998, -56.354592) -- (axis cs:339.981, -56.350177) -- (axis
  cs:340.963, -56.345956) -- (axis cs:341.946, -56.341932) -- (axis
  cs:342.929, -56.338105) -- (axis cs:343.911, -56.334476) -- (axis
  cs:344.894, -56.331046) -- (axis cs:345.876, -56.327816) -- (axis
  cs:346.858, -56.324788) -- (axis cs:347.84, -56.321962) -- (axis cs:348.823,
  -56.319339) -- (axis cs:349.805, -56.316920) -- (axis cs:350.787,
  -56.314705) -- (axis cs:351.769, -56.312695) ; \draw[constellation-boundary]
  (axis cs:351.693, -39.312768) -- (axis cs:352.683, -39.310949) -- (axis
  cs:353.673, -39.309338) -- (axis cs:354.663, -39.307934) -- (axis
  cs:355.653, -39.306740) -- (axis cs:356.643, -39.305754) -- (axis
  cs:357.633, -39.304978) -- (axis cs:358.623, -39.304412) -- (axis
  cs:359.613, -39.304055) -- (axis cs:360.603, -39.303908) -- (axis
  cs:361.593, -39.303971) -- (axis cs:362.583, -39.304244) -- (axis
  cs:363.573, -39.304727) -- (axis cs:364.563, -39.305419)
  ; \draw[constellation-boundary] (axis cs:346.727, -36.324983) -- (axis
  cs:346.723, -35.324990) -- (axis cs:346.719, -34.324996) -- (axis
  cs:346.714, -33.325003) -- (axis cs:346.71, -32.325009) -- (axis cs:346.706,
  -31.325015) -- (axis cs:346.702, -30.325021) -- (axis cs:346.698,
  -29.325027) -- (axis cs:346.694, -28.325033) -- (axis cs:346.69, -27.325039)
  -- (axis cs:346.686, -26.325044) -- (axis cs:346.683, -25.325050)
  ; \draw[constellation-boundary] (axis cs:215.534, -25.072697) -- (axis
  cs:215.785, -25.070961) -- (axis cs:216.789, -25.063909) -- (axis
  cs:217.794, -25.056684) -- (axis cs:218.798, -25.049289) -- (axis
  cs:219.803, -25.041727) -- (axis cs:220.807, -25.033998) -- (axis
  cs:221.811, -25.026107) -- (axis cs:222.815, -25.018055) -- (axis
  cs:223.819, -25.009845) -- (axis cs:224.824, -25.001479)
  ; \draw[constellation-boundary] (axis cs:215.534, -25.072697)
  ; \draw[constellation-boundary] (axis cs:361.533, -81.803966) -- (axis
  cs:362.449, -81.804219) -- (axis cs:363.365, -81.804665) -- (axis cs:364.28,
  -81.805305) ; \draw[constellation-boundary] (axis cs:361.533, -81.803966) --
  (axis cs:361.537, -81.303966) -- (axis cs:361.544, -80.303967) -- (axis
  cs:361.549, -79.303968) -- (axis cs:361.554, -78.303968) -- (axis
  cs:361.558, -77.303968) -- (axis cs:361.561, -76.303968) -- (axis
  cs:361.564, -75.303969) -- (axis cs:361.566, -74.303969)
  ; \draw[constellation-boundary] (axis cs:323.185, -74.454478) -- (axis
  cs:323.084, -73.454865) -- (axis cs:322.995, -72.455208) -- (axis
  cs:322.915, -71.455516) -- (axis cs:322.842, -70.455793) -- (axis
  cs:322.777, -69.456044) -- (axis cs:322.717, -68.456273) -- (axis
  cs:322.663, -67.456482) -- (axis cs:322.613, -66.456674) -- (axis
  cs:322.567, -65.456852) -- (axis cs:322.524, -64.457017) -- (axis
  cs:322.484, -63.457170) -- (axis cs:322.447, -62.457312) -- (axis
  cs:322.412, -61.457446) -- (axis cs:322.379, -60.457571) -- (axis
  cs:322.349, -59.457689) ; \draw[constellation-boundary] (axis cs:351.998,
  -74.312472) -- (axis cs:351.973, -73.312496) -- (axis cs:351.951,
  -72.312518) -- (axis cs:351.931, -71.312537) -- (axis cs:351.913,
  -70.312555) -- (axis cs:351.896, -69.312571) -- (axis cs:351.881,
  -68.312585) -- (axis cs:351.868, -67.312598) -- (axis cs:351.861,
  -66.812605) ; \draw[constellation-boundary] (axis cs:332.399, -66.889995) --
  (axis cs:332.38, -66.390051) -- (axis cs:332.344, -65.390156) -- (axis
  cs:332.311, -64.390254) -- (axis cs:332.281, -63.390345) -- (axis
  cs:332.252, -62.390429) -- (axis cs:332.225, -61.390508) -- (axis cs:332.2,
  -60.390583) -- (axis cs:332.177, -59.390652) -- (axis cs:332.155,
  -58.390718) -- (axis cs:332.134, -57.390780) -- (axis cs:332.114,
  -56.390839) -- (axis cs:332.095, -55.390895) -- (axis cs:332.077,
  -54.390948) -- (axis cs:332.06, -53.390999) -- (axis cs:332.044, -52.391047)
  -- (axis cs:332.028, -51.391093) -- (axis cs:332.013, -50.391137) -- (axis
  cs:331.999, -49.391180) ; \draw[constellation-boundary] (axis cs:332.399,
  -66.889995) -- (axis cs:333.373, -66.884308) -- (axis cs:334.348,
  -66.878804) -- (axis cs:335.322, -66.873485) -- (axis cs:336.296,
  -66.868352) -- (axis cs:337.27, -66.863407) -- (axis cs:338.244, -66.858652)
  -- (axis cs:339.218, -66.854087) -- (axis cs:340.191, -66.849714) -- (axis
  cs:341.164, -66.845535) -- (axis cs:342.137, -66.841550) -- (axis cs:343.11,
  -66.837760) -- (axis cs:344.083, -66.834167) -- (axis cs:345.056,
  -66.830771) -- (axis cs:346.028, -66.827574) -- (axis cs:347.001,
  -66.824576) -- (axis cs:347.973, -66.821778) -- (axis cs:348.945,
  -66.819181) -- (axis cs:349.917, -66.816787) -- (axis cs:350.889,
  -66.814594) -- (axis cs:351.861, -66.812605) ; \draw[constellation-boundary]
  (axis cs:307.565, -59.588058) -- (axis cs:307.528, -58.588242) -- (axis
  cs:307.492, -57.588416) -- (axis cs:307.459, -56.588580) -- (axis
  cs:307.427, -55.588736) -- (axis cs:307.397, -54.588884) -- (axis
  cs:307.368, -53.589026) -- (axis cs:307.341, -52.589160) -- (axis
  cs:307.315, -51.589289) -- (axis cs:307.29, -50.589412) -- (axis cs:307.266,
  -49.589530) -- (axis cs:307.243, -48.589643) -- (axis cs:307.221,
  -47.589752) -- (axis cs:307.2, -46.589857) -- (axis cs:307.179, -45.589958)
  -- (axis cs:307.16, -44.590055) -- (axis cs:307.14, -43.590149) -- (axis
  cs:307.122, -42.590240) -- (axis cs:307.104, -41.590328) -- (axis
  cs:307.087, -40.590414) -- (axis cs:307.07, -39.590497) -- (axis cs:307.054,
  -38.590577) -- (axis cs:307.038, -37.590656) -- (axis cs:307.022,
  -36.590732) -- (axis cs:307.007, -35.590806) -- (axis cs:306.992,
  -34.590879) -- (axis cs:306.978, -33.590950) -- (axis cs:306.964,
  -32.591019) -- (axis cs:306.95, -31.591086) -- (axis cs:306.937, -30.591153)
  -- (axis cs:306.924, -29.591218) -- (axis cs:306.911, -28.591281) -- (axis
  cs:306.898, -27.591344) ; \draw[constellation-boundary] (axis cs:307.565,
  -59.588058) -- (axis cs:308.552, -59.578389) -- (axis cs:309.539,
  -59.568847) -- (axis cs:310.526, -59.559437) -- (axis cs:311.513,
  -59.550160) -- (axis cs:312.499, -59.541020) -- (axis cs:313.485,
  -59.532019) -- (axis cs:314.471, -59.523160) -- (axis cs:315.457,
  -59.514445) -- (axis cs:316.442, -59.505877) -- (axis cs:317.427,
  -59.497458) -- (axis cs:318.412, -59.489191) -- (axis cs:319.396,
  -59.481078) -- (axis cs:320.381, -59.473122) -- (axis cs:321.365,
  -59.465325) -- (axis cs:322.349, -59.457689) ; \draw[constellation-boundary]
  (axis cs:236.93, -29.889612) -- (axis cs:236.923, -29.389645) -- (axis
  cs:236.91, -28.389711) -- (axis cs:236.897, -27.389775) -- (axis cs:236.885,
  -26.389839) -- (axis cs:236.872, -25.389901) ; \draw[constellation-boundary]
  (axis cs:228.057, -54.475608) -- (axis cs:228.31, -54.473373) -- (axis
  cs:229.321, -54.464343) -- (axis cs:230.332, -54.455170) -- (axis
  cs:231.343, -54.445859) -- (axis cs:232.353, -54.436411)
  ; \draw[constellation-boundary] (axis cs:232.353, -54.436411) -- (axis
  cs:232.326, -53.436539) -- (axis cs:232.3, -52.436661) -- (axis cs:232.276,
  -51.436778) -- (axis cs:232.252, -50.436890) -- (axis cs:232.229,
  -49.436997) -- (axis cs:232.207, -48.437099) ; \draw[constellation-boundary]
  (axis cs:232.207, -48.437099) -- (axis cs:233.216, -48.427538) -- (axis
  cs:234.224, -48.417846) -- (axis cs:235.232, -48.408026) -- (axis cs:236.24,
  -48.398082) -- (axis cs:237.247, -48.388017) ; \draw[constellation-boundary]
  (axis cs:237.247, -48.388017) -- (axis cs:237.225, -47.388130) -- (axis
  cs:237.203, -46.388238) -- (axis cs:237.183, -45.388342) -- (axis
  cs:237.163, -44.388443) -- (axis cs:237.143, -43.388540) -- (axis
  cs:237.124, -42.388634) ; \draw[constellation-boundary] (axis cs:237.124,
  -42.388634) -- (axis cs:238.13, -42.378465) -- (axis cs:239.136, -42.368181)
  -- (axis cs:240.142, -42.357785) -- (axis cs:241.147, -42.347281) -- (axis
  cs:242.153, -42.336671) -- (axis cs:243.158, -42.325959) -- (axis
  cs:244.163, -42.315148) -- (axis cs:245.168, -42.304241) -- (axis
  cs:246.172, -42.293242) -- (axis cs:247.177, -42.282155) -- (axis
  cs:248.181, -42.270982) -- (axis cs:248.495, -42.267474)
  ; \draw[constellation-boundary] (axis cs:242.153, -42.336671) -- (axis
  cs:242.134, -41.336772) -- (axis cs:242.115, -40.336870) -- (axis
  cs:242.097, -39.336966) -- (axis cs:242.08, -38.337059) -- (axis cs:242.063,
  -37.337149) -- (axis cs:242.046, -36.337237) -- (axis cs:242.03, -35.337322)
  -- (axis cs:242.014, -34.337406) -- (axis cs:241.999, -33.337487) -- (axis
  cs:241.984, -32.337567) -- (axis cs:241.969, -31.337645) -- (axis
  cs:241.955, -30.337721) -- (axis cs:241.948, -29.837759)
  ; \draw[constellation-boundary] (axis cs:180.591, -75.696092) -- (axis
  cs:181.639, -75.696025) -- (axis cs:182.687, -75.695736) -- (axis
  cs:183.734, -75.695225) -- (axis cs:184.782, -75.694493) -- (axis
  cs:185.829, -75.693539) -- (axis cs:186.876, -75.692363) -- (axis
  cs:187.924, -75.690967) -- (axis cs:188.971, -75.689350) -- (axis
  cs:190.018, -75.687514) -- (axis cs:191.065, -75.685459) -- (axis
  cs:192.111, -75.683185) -- (axis cs:193.158, -75.680694) -- (axis
  cs:194.204, -75.677985) -- (axis cs:195.25, -75.675062) -- (axis cs:196.296,
  -75.671923) -- (axis cs:197.342, -75.668570) -- (axis cs:198.387,
  -75.665006) -- (axis cs:199.432, -75.661229) -- (axis cs:200.477,
  -75.657243) -- (axis cs:201.522, -75.653049) -- (axis cs:202.566,
  -75.648647) -- (axis cs:203.61, -75.644040) -- (axis cs:204.653, -75.639228)
  -- (axis cs:205.696, -75.634215) -- (axis cs:206.739, -75.629000) -- (axis
  cs:207.781, -75.623587) ; \draw[constellation-boundary] (axis cs:266.002,
  -30.060661) -- (axis cs:265.986, -29.060758) -- (axis cs:265.97, -28.060853)
  -- (axis cs:265.955, -27.060946) -- (axis cs:265.94, -26.061037) -- (axis
  cs:265.925, -25.061127) ; \draw[constellation-boundary] (axis cs:253.235,
  -30.212305) -- (axis cs:253.22, -29.212394) -- (axis cs:253.205, -28.212481)
  -- (axis cs:253.19, -27.212566) -- (axis cs:253.176, -26.212650) -- (axis
  cs:253.162, -25.212732) ; \draw[constellation-boundary] (axis cs:253.235,
  -30.212305) -- (axis cs:253.987, -30.203601) -- (axis cs:254.989,
  -30.191941) -- (axis cs:255.991, -30.180224) -- (axis cs:256.992,
  -30.168452) -- (axis cs:257.994, -30.156629) -- (axis cs:258.995,
  -30.144760) -- (axis cs:259.997, -30.132847) -- (axis cs:260.998,
  -30.120894) -- (axis cs:261.999, -30.108905) -- (axis cs:263, -30.096883) --
  (axis cs:264.001, -30.084833) -- (axis cs:265.001, -30.072758) -- (axis
  cs:266.002, -30.060661) -- (axis cs:267.002, -30.048547) -- (axis
  cs:268.003, -30.036418) -- (axis cs:269.003, -30.024279) -- (axis
  cs:269.503, -30.018207) ; \draw[constellation-boundary] (axis cs:351.778,
  -57.812685) -- (axis cs:352.759, -57.810883) -- (axis cs:353.74, -57.809286)
  -- (axis cs:354.721, -57.807896) -- (axis cs:355.702, -57.806713) -- (axis
  cs:356.682, -57.805737) -- (axis cs:357.663, -57.804968) -- (axis
  cs:358.644, -57.804407) -- (axis cs:359.624, -57.804054) -- (axis
  cs:360.605, -57.803908) -- (axis cs:361.585, -57.803971) -- (axis
  cs:362.566, -57.804241) -- (axis cs:363.547, -57.804719) -- (axis
  cs:364.527, -57.805405) ; \draw[constellation-boundary] (axis cs:351.778,
  -57.812685) -- (axis cs:351.775, -57.312689) -- (axis cs:351.769,
  -56.312695) -- (axis cs:351.762, -55.312701) -- (axis cs:351.757,
  -54.312706) -- (axis cs:351.751, -53.312712) -- (axis cs:351.746,
  -52.312717) -- (axis cs:351.741, -51.312722) -- (axis cs:351.736,
  -50.312727) -- (axis cs:351.731, -49.312731) -- (axis cs:351.727,
  -48.312736) -- (axis cs:351.722, -47.312740) -- (axis cs:351.718,
  -46.312744) -- (axis cs:351.714, -45.312748) -- (axis cs:351.71, -44.312751)
  -- (axis cs:351.707, -43.312755) -- (axis cs:351.703, -42.312759) -- (axis
  cs:351.699, -41.312762) -- (axis cs:351.696, -40.312765) -- (axis
  cs:351.693, -39.312768) -- (axis cs:351.69, -38.312772) -- (axis cs:351.686,
  -37.312775) -- (axis cs:351.683, -36.312778) ; \draw[constellation-boundary]
  (axis cs:141.734, -40.291866) -- (axis cs:141.744, -39.541903) -- (axis
  cs:141.756, -38.541950) -- (axis cs:141.769, -37.541997) -- (axis cs:141.78,
  -36.542042) -- (axis cs:141.792, -35.542086) -- (axis cs:141.803,
  -34.542129) -- (axis cs:141.814, -33.542171) -- (axis cs:141.825,
  -32.542212) -- (axis cs:141.836, -31.542253) -- (axis cs:141.846,
  -30.542292) -- (axis cs:141.856, -29.542330) -- (axis cs:141.866,
  -28.542368) -- (axis cs:141.876, -27.542405) -- (axis cs:141.886,
  -26.542442) -- (axis cs:141.895, -25.542477) ; \draw[constellation-boundary]
  (axis cs:141.734, -40.291866) -- (axis cs:142.238, -40.295673) -- (axis
  cs:143.246, -40.303162) -- (axis cs:144.254, -40.310482) -- (axis
  cs:145.262, -40.317630) -- (axis cs:146.271, -40.324605) -- (axis
  cs:147.279, -40.331403) -- (axis cs:148.288, -40.338022) -- (axis
  cs:149.297, -40.344462) -- (axis cs:150.305, -40.350719) -- (axis
  cs:151.314, -40.356791) -- (axis cs:152.323, -40.362678) -- (axis
  cs:153.332, -40.368376) -- (axis cs:154.341, -40.373885) -- (axis
  cs:155.351, -40.379202) -- (axis cs:156.36, -40.384326) -- (axis cs:157.369,
  -40.389254) -- (axis cs:158.379, -40.393987) -- (axis cs:159.388,
  -40.398522) -- (axis cs:160.398, -40.402857) -- (axis cs:161.407,
  -40.406992) -- (axis cs:162.417, -40.410925) -- (axis cs:163.427,
  -40.414655) -- (axis cs:164.437, -40.418180) -- (axis cs:165.447,
  -40.421500) -- (axis cs:166.456, -40.424613) ; \draw[constellation-boundary]
  (axis cs:163.959, -35.666490) -- (axis cs:163.964, -34.666499) -- (axis
  cs:163.969, -33.666508) -- (axis cs:163.974, -32.666516) -- (axis
  cs:163.978, -31.833190) ; \draw[constellation-boundary] (axis cs:163.959,
  -35.666490) -- (axis cs:164.463, -35.668224) -- (axis cs:165.471,
  -35.671539) -- (axis cs:166.479, -35.674647) -- (axis cs:167.488,
  -35.677549) -- (axis cs:168.496, -35.680242) -- (axis cs:169.505,
  -35.682726) -- (axis cs:170.513, -35.685001) -- (axis cs:171.522,
  -35.687065) -- (axis cs:172.53, -35.688919) -- (axis cs:173.539, -35.690560)
  -- (axis cs:174.547, -35.691990) -- (axis cs:175.556, -35.693207) -- (axis
  cs:176.565, -35.694211) -- (axis cs:177.573, -35.695001) -- (axis
  cs:178.582, -35.695579) -- (axis cs:179.591, -35.695942)
  ; \draw[constellation-boundary] (axis cs:160.201, -31.818578) -- (axis
  cs:160.453, -31.819640) -- (axis cs:161.46, -31.823764) -- (axis cs:162.467,
  -31.827687) -- (axis cs:163.474, -31.831407) -- (axis cs:163.978,
  -31.833190) ; \draw[constellation-boundary] (axis cs:160.201, -31.818578) --
  (axis cs:160.202, -31.651913) -- (axis cs:160.208, -30.651926) -- (axis
  cs:160.213, -29.818602) ; \draw[constellation-boundary] (axis cs:155.181,
  -29.794774) -- (axis cs:155.433, -29.796081) -- (axis cs:156.439,
  -29.801190) -- (axis cs:157.445, -29.806104) -- (axis cs:158.452,
  -29.810822) -- (axis cs:159.458, -29.815342) -- (axis cs:160.213,
  -29.818602) ; \draw[constellation-boundary] (axis cs:155.181, -29.794774) --
  (axis cs:155.182, -29.628110) -- (axis cs:155.189, -28.628128) -- (axis
  cs:155.196, -27.628146) -- (axis cs:155.199, -27.128155)
  ; \draw[constellation-boundary] (axis cs:147.659, -27.083497) -- (axis
  cs:148.413, -27.088428) -- (axis cs:149.418, -27.094845) -- (axis
  cs:150.424, -27.101080) -- (axis cs:151.429, -27.107131) -- (axis
  cs:152.434, -27.112997) -- (axis cs:153.44, -27.118675) -- (axis cs:154.445,
  -27.124163) -- (axis cs:155.199, -27.128155) ; \draw[constellation-boundary]
  (axis cs:147.659, -27.083497) -- (axis cs:147.663, -26.583510) -- (axis
  cs:147.672, -25.583537) ; \draw[constellation-boundary] (axis cs:-4.174,
  -74.306645) -- (axis cs:-3.217, -74.305692) -- (axis cs:-2.261, -74.304942)
  -- (axis cs:-1.304, -74.304395) -- (axis cs:-0.347, -74.304050) -- (axis
  cs:0.609585, -74.303908) -- (axis cs:1.56628, -74.303969) -- (axis
  cs:2.52298, -74.304233) -- (axis cs:3.47971, -74.304699) -- (axis
  cs:4.43646, -74.305368) -- (axis cs:5.39327, -74.306240) -- (axis
  cs:6.35012, -74.307314) -- (axis cs:7.30704, -74.308590) -- (axis
  cs:8.26404, -74.310067) -- (axis cs:9.22113, -74.311746) -- (axis
  cs:10.1783, -74.313625) -- (axis cs:11.1356, -74.315704) -- (axis cs:12.093,
  -74.317984) -- (axis cs:12.3324, -74.318585) ; \draw[constellation-boundary]
  (axis cs:64.8823, -48.699667) -- (axis cs:64.9071, -47.699803) -- (axis
  cs:64.9309, -46.699934) -- (axis cs:64.9538, -45.700060) -- (axis cs:64.976,
  -44.700182) -- (axis cs:64.9974, -43.700299) -- (axis cs:65.0181,
  -42.700413) -- (axis cs:65.0381, -41.700523) -- (axis cs:65.0575,
  -40.700630) -- (axis cs:65.0764, -39.700734) -- (axis cs:65.0947,
  -38.700834) -- (axis cs:65.1125, -37.700932) -- (axis cs:65.1298,
  -36.701027) ; \draw[constellation-boundary] (axis cs:68.4241, -46.238801) --
  (axis cs:68.9217, -46.244428) -- (axis cs:69.917, -46.255741) -- (axis
  cs:70.9125, -46.267128) -- (axis cs:71.9082, -46.278588) -- (axis
  cs:72.9041, -46.290116) -- (axis cs:73.4021, -46.295905)
  ; \draw[constellation-boundary] (axis cs:73.4021, -46.295905) -- (axis
  cs:73.4142, -45.795975) -- (axis cs:73.4377, -44.796112) -- (axis
  cs:73.4604, -43.796244) -- (axis cs:73.4824, -42.796372)
  ; \draw[constellation-boundary] (axis cs:73.4824, -42.796372) -- (axis
  cs:73.9807, -42.802179) -- (axis cs:74.9775, -42.813838) -- (axis
  cs:75.9744, -42.825554) -- (axis cs:76.9716, -42.837325) -- (axis
  cs:77.9689, -42.849147) -- (axis cs:78.9664, -42.861015) -- (axis
  cs:79.9641, -42.872927) -- (axis cs:80.962, -42.884878) -- (axis cs:81.9601,
  -42.896866) -- (axis cs:82.9584, -42.908886) -- (axis cs:83.9569,
  -42.920935) -- (axis cs:84.9556, -42.933010) -- (axis cs:85.9544,
  -42.945105) -- (axis cs:86.9535, -42.957219) -- (axis cs:87.9528,
  -42.969347) -- (axis cs:88.9522, -42.981486) -- (axis cs:89.9519,
  -42.993631) -- (axis cs:90.9518, -43.005780) -- (axis cs:91.9518,
  -43.017927) -- (axis cs:92.9521, -43.030071) -- (axis cs:93.9525,
  -43.042207) -- (axis cs:94.9532, -43.054331) -- (axis cs:95.954, -43.066440)
  -- (axis cs:96.9551, -43.078529) -- (axis cs:97.9563, -43.090596) -- (axis
  cs:98.9577, -43.102636) -- (axis cs:99.709, -43.111647)
  ; \draw[constellation-boundary] (axis cs:75.9744, -42.825554) -- (axis
  cs:75.996, -41.825681) -- (axis cs:76.0169, -40.825804) -- (axis cs:76.0371,
  -39.825924) -- (axis cs:76.0568, -38.826040) -- (axis cs:76.076, -37.826152)
  -- (axis cs:76.0946, -36.826262) -- (axis cs:76.1127, -35.826369) -- (axis
  cs:76.1304, -34.826473) -- (axis cs:76.1477, -33.826575) -- (axis
  cs:76.1646, -32.826674) -- (axis cs:76.1811, -31.826771) -- (axis
  cs:76.1972, -30.826866) -- (axis cs:76.213, -29.826960) -- (axis cs:76.2285,
  -28.827051) -- (axis cs:76.2437, -27.827140) -- (axis cs:76.2549,
  -27.077206) ; \draw[constellation-boundary] (axis cs:73.7593, -27.047983) --
  (axis cs:73.763, -26.798004) -- (axis cs:73.7775, -25.798089)
  ; \draw[constellation-boundary] (axis cs:71.7633, -27.024881) -- (axis
  cs:72.2623, -27.030632) -- (axis cs:73.2603, -27.042183) -- (axis
  cs:74.2584, -27.053798) -- (axis cs:75.2566, -27.065474) -- (axis
  cs:76.2549, -27.077206) -- (axis cs:77.2533, -27.088992) -- (axis
  cs:78.2519, -27.100827) -- (axis cs:79.2505, -27.112708) -- (axis
  cs:80.2492, -27.124631) -- (axis cs:81.2481, -27.136594) -- (axis cs:82.247,
  -27.148591) -- (axis cs:83.2461, -27.160620) -- (axis cs:84.2452,
  -27.172677) -- (axis cs:85.2445, -27.184757) -- (axis cs:86.2439,
  -27.196859) -- (axis cs:87.2434, -27.208977) -- (axis cs:88.2429,
  -27.221109) -- (axis cs:89.2426, -27.233249) -- (axis cs:90.2425,
  -27.245396) -- (axis cs:91.2424, -27.257545) -- (axis cs:92.2424,
  -27.269692) -- (axis cs:92.9925, -27.278799) ; \draw[constellation-boundary]
  (axis cs:71.7224, -29.774645) -- (axis cs:71.7375, -28.774732) -- (axis
  cs:71.7524, -27.774818) -- (axis cs:71.7633, -27.024881)
  ; \draw[constellation-boundary] (axis cs:69.9767, -29.754662) -- (axis
  cs:70.2261, -29.757503) -- (axis cs:71.2236, -29.768913) -- (axis
  cs:71.7224, -29.774645) ; \draw[constellation-boundary] (axis cs:69.8624,
  -36.754011) -- (axis cs:69.8799, -35.754111) -- (axis cs:69.897, -34.754208)
  -- (axis cs:69.9137, -33.754303) -- (axis cs:69.9299, -32.754396) -- (axis
  cs:69.9459, -31.754486) -- (axis cs:69.9614, -30.754575) -- (axis
  cs:69.9767, -29.754662) ; \draw[constellation-boundary] (axis cs:65.1298,
  -36.701027) -- (axis cs:66.1259, -36.712022) -- (axis cs:67.1221,
  -36.723105) -- (axis cs:68.1184, -36.734273) -- (axis cs:69.1149,
  -36.745523) -- (axis cs:69.8624, -36.754011) ; \draw[constellation-boundary]
  (axis cs:111.677, -33.250465) -- (axis cs:111.693, -32.250556) -- (axis
  cs:111.709, -31.250644) -- (axis cs:111.724, -30.250731) -- (axis
  cs:111.739, -29.250816) -- (axis cs:111.754, -28.250899) -- (axis
  cs:111.768, -27.250981) -- (axis cs:111.783, -26.251062) -- (axis
  cs:111.796, -25.251141) ; \draw[constellation-boundary] (axis cs:92.899,
  -33.028232) -- (axis cs:92.9161, -32.028335) -- (axis cs:92.9328,
  -31.028437) -- (axis cs:92.9491, -30.028536) -- (axis cs:92.9652,
  -29.028633) -- (axis cs:92.9809, -28.028729) -- (axis cs:92.9963,
  -27.028822) -- (axis cs:93.0115, -26.028914) -- (axis cs:93.0264,
  -25.029005) ; \draw[constellation-boundary] (axis cs:92.899, -33.028232) --
  (axis cs:93.1491, -33.031266) -- (axis cs:94.1494, -33.043400) -- (axis
  cs:95.1498, -33.055521) -- (axis cs:96.1504, -33.067627) -- (axis
  cs:97.1512, -33.079712) -- (axis cs:98.152, -33.091774) -- (axis cs:99.153,
  -33.103809) -- (axis cs:100.154, -33.115813) -- (axis cs:101.155,
  -33.127782) -- (axis cs:102.157, -33.139714) -- (axis cs:103.158,
  -33.151603) -- (axis cs:104.16, -33.163447) -- (axis cs:105.162, -33.175242)
  -- (axis cs:106.164, -33.186984) -- (axis cs:107.166, -33.198670) -- (axis
  cs:108.168, -33.210297) -- (axis cs:109.171, -33.221859) -- (axis
  cs:110.173, -33.233355) -- (axis cs:111.176, -33.244780) -- (axis
  cs:111.677, -33.250465) ; \draw[constellation-boundary] (axis cs:166.337,
  -57.174436) -- (axis cs:166.342, -56.674443) -- (axis cs:166.352,
  -55.674457) -- (axis cs:166.361, -54.674471) -- (axis cs:166.37, -53.674484)
  -- (axis cs:166.378, -52.674496) -- (axis cs:166.386, -51.674508) -- (axis
  cs:166.394, -50.674519) -- (axis cs:166.401, -49.674530) -- (axis
  cs:166.408, -48.674541) -- (axis cs:166.414, -47.674551) -- (axis
  cs:166.421, -46.674560) -- (axis cs:166.427, -45.674569) -- (axis
  cs:166.433, -44.674578) -- (axis cs:166.439, -43.674587) -- (axis
  cs:166.445, -42.674595) -- (axis cs:166.45, -41.674603) -- (axis cs:166.455,
  -40.674611) -- (axis cs:166.46, -39.674619) -- (axis cs:166.465, -38.674626)
  -- (axis cs:166.47, -37.674633) -- (axis cs:166.475, -36.674640) -- (axis
  cs:166.479, -35.674647) ; \draw[constellation-boundary] (axis cs:133.324,
  -56.973965) -- (axis cs:133.83, -56.978424) -- (axis cs:134.843, -56.987231)
  -- (axis cs:135.856, -56.995889) -- (axis cs:136.869, -57.004394) -- (axis
  cs:137.883, -57.012746) -- (axis cs:138.896, -57.020939) -- (axis cs:139.91,
  -57.028973) -- (axis cs:140.925, -57.036844) -- (axis cs:141.939,
  -57.044550) -- (axis cs:142.954, -57.052089) -- (axis cs:143.968,
  -57.059457) -- (axis cs:144.983, -57.066654) -- (axis cs:145.999,
  -57.073675) -- (axis cs:147.014, -57.080521) -- (axis cs:148.03, -57.087187)
  -- (axis cs:149.046, -57.093672) -- (axis cs:150.062, -57.099974) -- (axis
  cs:151.078, -57.106090) -- (axis cs:152.094, -57.112020) -- (axis cs:153.11,
  -57.117761) -- (axis cs:154.127, -57.123310) -- (axis cs:155.144,
  -57.128667) -- (axis cs:156.161, -57.133830) -- (axis cs:157.178,
  -57.138797) -- (axis cs:158.195, -57.143566) -- (axis cs:159.213,
  -57.148136) -- (axis cs:160.23, -57.152505) -- (axis cs:161.248, -57.156673)
  -- (axis cs:162.265, -57.160637) -- (axis cs:163.283, -57.164396) -- (axis
  cs:164.301, -57.167950) -- (axis cs:165.319, -57.171297) -- (axis
  cs:166.337, -57.174436) -- (axis cs:167.355, -57.177365) -- (axis
  cs:168.374, -57.180085) -- (axis cs:169.392, -57.182594) -- (axis
  cs:170.156, -57.184336) ; \draw[constellation-boundary] (axis cs:133.324,
  -56.973965) -- (axis cs:133.338, -56.474030) -- (axis cs:133.367,
  -55.474155) -- (axis cs:133.38, -54.974215) ; \draw[constellation-boundary]
  (axis cs:127.567, -54.920469) -- (axis cs:127.82, -54.922898) -- (axis
  cs:128.83, -54.932533) -- (axis cs:129.841, -54.942036) -- (axis cs:130.852,
  -54.951405) -- (axis cs:131.863, -54.960635) -- (axis cs:132.874,
  -54.969724) -- (axis cs:133.38, -54.974215) ; \draw[constellation-boundary]
  (axis cs:127.567, -54.920469) -- (axis cs:127.581, -54.420539) -- (axis
  cs:127.609, -53.420672) ; \draw[constellation-boundary] (axis cs:123.32,
  -53.378214) -- (axis cs:123.824, -53.383322) -- (axis cs:124.833,
  -53.393449) -- (axis cs:125.843, -53.403457) -- (axis cs:126.852,
  -53.413341) -- (axis cs:127.609, -53.420672) ; \draw[constellation-boundary]
  (axis cs:123.32, -53.378214) -- (axis cs:123.348, -52.378356) -- (axis
  cs:123.375, -51.378491) -- (axis cs:123.381, -51.128524)
  ; \draw[constellation-boundary] (axis cs:120.862, -51.102580) -- (axis
  cs:120.881, -50.352682) -- (axis cs:120.906, -49.352813) -- (axis cs:120.93,
  -48.352939) -- (axis cs:120.954, -47.353060) -- (axis cs:120.976,
  -46.353176) -- (axis cs:120.997, -45.353288) -- (axis cs:121.018,
  -44.353397) -- (axis cs:121.038, -43.353501) ; \draw[constellation-boundary]
  (axis cs:90.7489, -50.754547) -- (axis cs:91.7489, -50.766696) -- (axis
  cs:92.7492, -50.778840) -- (axis cs:93.7498, -50.790978) -- (axis
  cs:94.7506, -50.803105) -- (axis cs:95.7517, -50.815217) -- (axis cs:96.753,
  -50.827311) -- (axis cs:97.7546, -50.839382) -- (axis cs:98.7565,
  -50.851428) -- (axis cs:99.7586, -50.863445) -- (axis cs:100.761,
  -50.875428) -- (axis cs:101.764, -50.887375) -- (axis cs:102.767,
  -50.899282) -- (axis cs:103.77, -50.911144) -- (axis cs:104.773, -50.922958)
  -- (axis cs:105.777, -50.934721) -- (axis cs:106.781, -50.946430) -- (axis
  cs:107.785, -50.958079) -- (axis cs:108.789, -50.969666) -- (axis
  cs:109.794, -50.981188) -- (axis cs:110.799, -50.992640) -- (axis
  cs:111.804, -51.004019) -- (axis cs:112.81, -51.015322) -- (axis cs:113.815,
  -51.026545) -- (axis cs:114.821, -51.037685) -- (axis cs:115.827,
  -51.048738) -- (axis cs:116.834, -51.059700) -- (axis cs:117.84, -51.070569)
  -- (axis cs:118.847, -51.081341) -- (axis cs:119.854, -51.092012) -- (axis
  cs:120.862, -51.102580) -- (axis cs:121.869, -51.113041) -- (axis
  cs:122.877, -51.123391) -- (axis cs:123.381, -51.128524)
  ; \draw[constellation-boundary] (axis cs:90.6937, -52.504212) -- (axis
  cs:91.6937, -52.516360) -- (axis cs:92.694, -52.528506) -- (axis cs:93.1943,
  -52.534576) ; \draw[constellation-boundary] (axis cs:90.6937, -52.504212) --
  (axis cs:90.7099, -52.004310) -- (axis cs:90.7413, -51.004501) -- (axis
  cs:90.7713, -50.004683) -- (axis cs:90.8001, -49.004858) -- (axis
  cs:90.8278, -48.005026) -- (axis cs:90.8544, -47.005188) -- (axis cs:90.88,
  -46.005344) -- (axis cs:90.9048, -45.005494) -- (axis cs:90.9287,
  -44.005639) -- (axis cs:90.9518, -43.005780) ; \draw[constellation-boundary]
  (axis cs:93.1073, -55.034048) -- (axis cs:93.1434, -54.034267) -- (axis
  cs:93.1777, -53.034475) -- (axis cs:93.1943, -52.534576)
  ; \draw[constellation-boundary] (axis cs:93.1073, -55.034048) -- (axis
  cs:93.6077, -55.040117) -- (axis cs:94.6086, -55.052245) -- (axis
  cs:95.6099, -55.064360) -- (axis cs:96.6114, -55.076456) -- (axis
  cs:97.6132, -55.088531) -- (axis cs:98.1143, -55.094560)
  ; \draw[constellation-boundary] (axis cs:97.9951, -58.093843) -- (axis
  cs:98.0369, -57.094095) -- (axis cs:98.0766, -56.094333) -- (axis
  cs:98.1143, -55.094560) ; \draw[constellation-boundary] (axis cs:97.9951,
  -58.093843) -- (axis cs:98.4963, -58.099866) -- (axis cs:99.499, -58.111891)
  -- (axis cs:100.502, -58.123883) -- (axis cs:101.505, -58.135840) -- (axis
  cs:102.509, -58.147757) -- (axis cs:103.011, -58.153699)
  ; \draw[constellation-boundary] (axis cs:102.703, -64.151880) -- (axis
  cs:102.763, -63.152235) -- (axis cs:102.819, -62.152566) -- (axis
  cs:102.872, -61.152876) -- (axis cs:102.921, -60.153167) -- (axis
  cs:102.967, -59.153441) -- (axis cs:103.011, -58.153699)
  ; \draw[constellation-boundary] (axis cs:135.244, -75.495465) -- (axis
  cs:135.369, -74.495991) -- (axis cs:135.48, -73.496455) -- (axis cs:135.578,
  -72.496868) -- (axis cs:135.666, -71.497237) -- (axis cs:135.745,
  -70.497569) -- (axis cs:135.817, -69.497871) -- (axis cs:135.882,
  -68.498145) -- (axis cs:135.942, -67.498396) -- (axis cs:135.997,
  -66.498627) -- (axis cs:136.048, -65.498840) -- (axis cs:136.095,
  -64.499037) ; \draw[constellation-boundary] (axis cs:169.857, -75.684007) --
  (axis cs:169.891, -74.684044) -- (axis cs:169.92, -73.684077) -- (axis
  cs:169.947, -72.684106) -- (axis cs:169.97, -71.684132) -- (axis cs:169.991,
  -70.684155) -- (axis cs:170.011, -69.684176) -- (axis cs:170.028,
  -68.684196) -- (axis cs:170.044, -67.684213) -- (axis cs:170.059,
  -66.684229) -- (axis cs:170.072, -65.684244) -- (axis cs:170.085,
  -64.684258) -- (axis cs:170.096, -63.684271) -- (axis cs:170.107,
  -62.684283) -- (axis cs:170.117, -61.684294) -- (axis cs:170.127,
  -60.684304) -- (axis cs:170.136, -59.684314) -- (axis cs:170.144,
  -58.684323) -- (axis cs:170.152, -57.684332) -- (axis cs:170.156,
  -57.184336) ; \draw[constellation-boundary] (axis cs:170.085, -64.684258) --
  (axis cs:170.341, -64.684816) -- (axis cs:171.366, -64.686915) -- (axis
  cs:172.392, -64.688799) -- (axis cs:173.417, -64.690468) -- (axis
  cs:174.442, -64.691921) -- (axis cs:175.468, -64.693158) -- (axis
  cs:176.493, -64.694179) -- (axis cs:177.519, -64.694983) -- (axis
  cs:178.545, -64.695570) -- (axis cs:179.57, -64.695940)
  ; \draw[constellation-boundary] (axis cs:179.057, -64.695782) -- (axis
  cs:179.059, -63.695782) -- (axis cs:179.061, -62.695783) -- (axis
  cs:179.063, -61.695783) -- (axis cs:179.064, -60.695783) -- (axis
  cs:179.066, -59.695783) -- (axis cs:179.067, -58.695784) -- (axis
  cs:179.068, -57.695784) -- (axis cs:179.07, -56.695784) -- (axis cs:179.071,
  -55.695784) ; \draw[constellation-boundary] (axis cs:179.071, -55.695784) --
  (axis cs:179.58, -55.695941) ; \draw[constellation-boundary] (axis
  cs:111.652, -82.775886) -- (axis cs:112.687, -82.787200) -- (axis
  cs:113.724, -82.798432) -- (axis cs:114.762, -82.809578) -- (axis
  cs:115.802, -82.820634) -- (axis cs:116.843, -82.831597) -- (axis
  cs:117.886, -82.842464) -- (axis cs:118.931, -82.853229) -- (axis
  cs:119.977, -82.863890) -- (axis cs:121.026, -82.874443) -- (axis
  cs:122.075, -82.884883) -- (axis cs:123.126, -82.895209) -- (axis
  cs:124.179, -82.905414) -- (axis cs:125.234, -82.915497) -- (axis cs:126.29,
  -82.925454) -- (axis cs:127.347, -82.935280) -- (axis cs:128.406,
  -82.944973) -- (axis cs:129.467, -82.954528) -- (axis cs:130.529,
  -82.963943) -- (axis cs:131.593, -82.973214) -- (axis cs:132.658,
  -82.982338) -- (axis cs:133.724, -82.991310) -- (axis cs:134.792,
  -83.000129) -- (axis cs:135.861, -83.008790) -- (axis cs:136.932,
  -83.017291) -- (axis cs:138.004, -83.025627) -- (axis cs:139.078,
  -83.033797) -- (axis cs:140.152, -83.041797) -- (axis cs:141.229,
  -83.049624) -- (axis cs:142.306, -83.057275) -- (axis cs:143.385,
  -83.064747) -- (axis cs:144.464, -83.072037) -- (axis cs:145.546,
  -83.079143) -- (axis cs:146.628, -83.086062) -- (axis cs:147.711,
  -83.092790) -- (axis cs:148.796, -83.099326) -- (axis cs:149.881,
  -83.105667) -- (axis cs:150.968, -83.111810) -- (axis cs:152.056,
  -83.117754) -- (axis cs:153.144, -83.123495) -- (axis cs:154.234,
  -83.129031) -- (axis cs:155.324, -83.134361) -- (axis cs:156.416,
  -83.139482) -- (axis cs:157.508, -83.144392) -- (axis cs:158.601,
  -83.149089) -- (axis cs:159.695, -83.153572) -- (axis cs:160.79, -83.157838)
  -- (axis cs:161.885, -83.161886) -- (axis cs:162.982, -83.165714) -- (axis
  cs:164.078, -83.169320) -- (axis cs:165.176, -83.172704) -- (axis
  cs:166.274, -83.175864) -- (axis cs:167.372, -83.178799) -- (axis
  cs:168.471, -83.181507) -- (axis cs:169.57, -83.183987) -- (axis cs:170.67,
  -83.186238) -- (axis cs:171.77, -83.188260) -- (axis cs:172.871, -83.190052)
  -- (axis cs:173.972, -83.191612) -- (axis cs:175.073, -83.192941) -- (axis
  cs:176.174, -83.194037) -- (axis cs:177.276, -83.194901) -- (axis
  cs:178.377, -83.195531) -- (axis cs:179.479, -83.195928)
  ; \draw[constellation-boundary] (axis cs:99.709, -43.111647) -- (axis
  cs:99.7311, -42.111780) -- (axis cs:99.7525, -41.111908) -- (axis
  cs:99.7733, -40.112033) -- (axis cs:99.7935, -39.112154) -- (axis
  cs:99.8131, -38.112271) -- (axis cs:99.8322, -37.112386) -- (axis
  cs:99.8508, -36.112497) -- (axis cs:99.8689, -35.112606) -- (axis
  cs:99.8866, -34.112711) -- (axis cs:99.9039, -33.112815)
  ; \draw[constellation-boundary] (axis cs:179.091, -25.195788)
  ; \draw[constellation-boundary] (axis cs:164.008, -25.166576) -- (axis
  cs:164.511, -25.168306) -- (axis cs:165.516, -25.171611) -- (axis
  cs:166.522, -25.174710) -- (axis cs:167.527, -25.177603) -- (axis
  cs:168.533, -25.180289) -- (axis cs:169.538, -25.182766) -- (axis
  cs:170.544, -25.185034) -- (axis cs:171.549, -25.187092) -- (axis
  cs:172.555, -25.188940) -- (axis cs:173.56, -25.190577) -- (axis cs:174.566,
  -25.192002) -- (axis cs:175.572, -25.193215) -- (axis cs:176.577,
  -25.194216) -- (axis cs:177.583, -25.195005) -- (axis cs:178.589,
  -25.195580) -- (axis cs:179.594, -25.195943) ; \draw[constellation-boundary]
  (axis cs:164.008, -25.166576) ; \draw[constellation-boundary] (axis
  cs:58.3188, -52.796845) -- (axis cs:58.3235, -52.630203) -- (axis cs:58.351,
  -51.630345) -- (axis cs:58.3772, -50.630481) ; \draw[constellation-boundary]
  (axis cs:60.6929, -56.155588) -- (axis cs:60.7098, -55.655677) -- (axis
  cs:60.7423, -54.655850) -- (axis cs:60.7732, -53.656014) -- (axis
  cs:60.7979, -52.822811) ; \draw[constellation-boundary] (axis cs:60.6929,
  -56.155588) -- (axis cs:61.6839, -56.166150) -- (axis cs:62.6751,
  -56.176814) -- (axis cs:63.6666, -56.187577) -- (axis cs:64.6584,
  -56.198436) -- (axis cs:65.6505, -56.209388) ; \draw[constellation-boundary]
  (axis cs:65.5546, -58.708857) -- (axis cs:65.5946, -57.709078) -- (axis
  cs:65.6324, -56.709288) -- (axis cs:65.6505, -56.209388)
  ; \draw[constellation-boundary] (axis cs:65.5546, -58.708857) -- (axis
  cs:66.5462, -58.719890) -- (axis cs:67.5381, -58.731009) -- (axis
  cs:68.5302, -58.742211) -- (axis cs:69.2746, -58.750665)
  ; \draw[constellation-boundary] (axis cs:68.5816, -69.746720) -- (axis
  cs:68.8285, -69.749534) -- (axis cs:69.8163, -69.760839) -- (axis
  cs:70.8047, -69.772220) -- (axis cs:71.7936, -69.783671) -- (axis
  cs:72.7831, -69.795191) -- (axis cs:73.7731, -69.806776) -- (axis
  cs:74.7636, -69.818423) -- (axis cs:75.7547, -69.830127) -- (axis
  cs:76.7463, -69.841886) -- (axis cs:77.7384, -69.853697) -- (axis
  cs:78.7312, -69.865554) -- (axis cs:79.7244, -69.877456) -- (axis
  cs:80.7183, -69.889398) -- (axis cs:81.7127, -69.901377) -- (axis
  cs:82.7076, -69.913390) -- (axis cs:83.7032, -69.925432) -- (axis
  cs:84.6993, -69.937500) -- (axis cs:85.6959, -69.949591) -- (axis
  cs:86.6932, -69.961700) -- (axis cs:87.691, -69.973825) -- (axis cs:88.6894,
  -69.985961) -- (axis cs:89.6884, -69.998105) -- (axis cs:90.688, -70.010253)
  -- (axis cs:91.6881, -70.022401) -- (axis cs:92.6889, -70.034547) -- (axis
  cs:93.6902, -70.046685) -- (axis cs:94.6921, -70.058812) -- (axis
  cs:95.6946, -70.070925) -- (axis cs:96.6977, -70.083020) -- (axis
  cs:97.7013, -70.095093) -- (axis cs:98.4545, -70.104131)
  ; \draw[constellation-boundary] (axis cs:90.1736, -64.001053) -- (axis
  cs:91.1735, -64.013201) -- (axis cs:92.1738, -64.025349) -- (axis
  cs:93.1746, -64.037491) -- (axis cs:94.1757, -64.049624) -- (axis
  cs:95.1774, -64.061745) -- (axis cs:96.1794, -64.073850) -- (axis
  cs:97.1819, -64.085935) -- (axis cs:98.1848, -64.097996) -- (axis
  cs:99.1882, -64.110030) -- (axis cs:100.192, -64.122033) -- (axis
  cs:101.196, -64.134001) -- (axis cs:102.201, -64.145930) -- (axis
  cs:103.206, -64.157818) -- (axis cs:104.211, -64.169660) -- (axis
  cs:105.217, -64.181452) -- (axis cs:106.224, -64.193191) -- (axis
  cs:107.231, -64.204873) -- (axis cs:108.238, -64.216495) -- (axis
  cs:109.246, -64.228053) -- (axis cs:110.254, -64.239544) -- (axis
  cs:111.262, -64.250963) -- (axis cs:112.271, -64.262307) -- (axis
  cs:113.281, -64.273573) -- (axis cs:114.29, -64.284757) -- (axis cs:115.3,
  -64.295856) -- (axis cs:116.311, -64.306866) -- (axis cs:117.322,
  -64.317784) -- (axis cs:118.333, -64.328606) -- (axis cs:119.345,
  -64.339329) -- (axis cs:120.358, -64.349949) -- (axis cs:121.37, -64.360463)
  -- (axis cs:122.383, -64.370869) -- (axis cs:123.396, -64.381161) -- (axis
  cs:124.41, -64.391338) -- (axis cs:125.424, -64.401396) -- (axis cs:126.439,
  -64.411332) -- (axis cs:127.454, -64.421142) -- (axis cs:128.469,
  -64.430824) -- (axis cs:129.485, -64.440374) -- (axis cs:130.501,
  -64.449790) -- (axis cs:131.517, -64.459068) -- (axis cs:132.534,
  -64.468205) -- (axis cs:133.551, -64.477199) -- (axis cs:134.568,
  -64.486047) -- (axis cs:135.586, -64.494745) -- (axis cs:136.095,
  -64.499037) ; \draw[constellation-boundary] (axis cs:90.1736, -64.001053) --
  (axis cs:90.2347, -63.001424) -- (axis cs:90.2917, -62.001770) -- (axis
  cs:90.3451, -61.002094) ; \draw[constellation-boundary] (axis cs:82.8576,
  -60.911287) -- (axis cs:83.3561, -60.917307) -- (axis cs:84.3534,
  -60.929366) -- (axis cs:85.3511, -60.941450) -- (axis cs:86.3491,
  -60.953553) -- (axis cs:87.3475, -60.965673) -- (axis cs:88.3463,
  -60.977805) -- (axis cs:89.3455, -60.989947) -- (axis cs:90.3451,
  -61.002094) ; \draw[constellation-boundary] (axis cs:82.8576, -60.911287) --
  (axis cs:82.9071, -59.911586) -- (axis cs:82.9537, -58.911867) -- (axis
  cs:82.9977, -57.912132) -- (axis cs:83.0188, -57.412260)
  ; \draw[constellation-boundary] (axis cs:75.5477, -57.323042) -- (axis
  cs:76.5428, -57.334790) -- (axis cs:77.5383, -57.346590) -- (axis cs:78.534,
  -57.358438) -- (axis cs:79.5301, -57.370331) -- (axis cs:80.5264,
  -57.382266) -- (axis cs:81.5231, -57.394238) -- (axis cs:82.5202,
  -57.406245) -- (axis cs:83.0188, -57.412260) ; \draw[constellation-boundary]
  (axis cs:75.5477, -57.323042) -- (axis cs:75.5677, -56.823159) -- (axis
  cs:75.606, -55.823385) -- (axis cs:75.6424, -54.823599) -- (axis cs:75.677,
  -53.823803) ; \draw[constellation-boundary] (axis cs:68.2177, -53.737636) --
  (axis cs:68.2493, -52.737814) -- (axis cs:68.2794, -51.737984) -- (axis
  cs:68.3082, -50.738147) -- (axis cs:68.3358, -49.738303) -- (axis
  cs:68.3622, -48.738452) -- (axis cs:68.3877, -47.738596) -- (axis
  cs:68.4122, -46.738734) -- (axis cs:68.4241, -46.238801)
  ; \draw[constellation-boundary] (axis cs:68.2177, -53.737636) -- (axis
  cs:68.7146, -53.743255) -- (axis cs:69.7084, -53.754552) -- (axis
  cs:70.7025, -53.765924) -- (axis cs:71.6968, -53.777368) -- (axis
  cs:72.6915, -53.788882) -- (axis cs:73.6864, -53.800462) -- (axis
  cs:74.6816, -53.812103) -- (axis cs:75.677, -53.823803)
  ; \draw[constellation-boundary] (axis cs:62.15, -48.669978) -- (axis
  cs:62.895, -48.678002) -- (axis cs:63.8886, -48.688787) -- (axis cs:64.8823,
  -48.699667) -- (axis cs:65.8763, -48.710640) -- (axis cs:66.8705,
  -48.721701) -- (axis cs:67.8649, -48.732848) -- (axis cs:68.3622,
  -48.738452) ; \draw[constellation-boundary] (axis cs:62.0987, -50.669703) --
  (axis cs:62.1248, -49.669843) -- (axis cs:62.15, -48.669978)
  ; \draw[constellation-boundary] (axis cs:58.3772, -50.630481) -- (axis
  cs:58.8732, -50.635622) -- (axis cs:59.8654, -50.645988) -- (axis
  cs:60.8579, -50.656462) -- (axis cs:61.8505, -50.667042) -- (axis
  cs:62.0987, -50.669703) ; \draw[constellation-boundary] (axis cs:59.1058,
  -39.636831) -- (axis cs:60.1005, -39.647223) -- (axis cs:61.0954,
  -39.657722) -- (axis cs:62.0904, -39.668326) -- (axis cs:63.0856,
  -39.679032) -- (axis cs:64.0809, -39.689835) -- (axis cs:65.0764,
  -39.700734) ; \draw[constellation-boundary] (axis cs:59.0313, -43.636443) --
  (axis cs:59.0508, -42.636545) -- (axis cs:59.0697, -41.636643) -- (axis
  cs:59.088, -40.636738) -- (axis cs:59.1058, -39.636831)
  ; \draw[constellation-boundary] (axis cs:52.3267, -43.569408) -- (axis
  cs:53.0713, -43.576578) -- (axis cs:54.0642, -43.586250) -- (axis
  cs:55.0573, -43.596049) -- (axis cs:56.0506, -43.605971) -- (axis cs:57.044,
  -43.616012) -- (axis cs:58.0376, -43.626171) -- (axis cs:59.0313,
  -43.636443) ; \draw[constellation-boundary] (axis cs:52.2889, -45.569227) --
  (axis cs:52.3082, -44.569319) -- (axis cs:52.3267, -43.569408)
  ; \draw[constellation-boundary] (axis cs:46.0908, -45.512484) -- (axis
  cs:47.0821, -45.521191) -- (axis cs:48.0735, -45.530044) -- (axis
  cs:49.0651, -45.539039) -- (axis cs:50.0569, -45.548175) -- (axis
  cs:51.0488, -45.557447) -- (axis cs:52.0409, -45.566854) -- (axis
  cs:52.2889, -45.569227) ; \draw[constellation-boundary] (axis cs:46.0345,
  -48.512239) -- (axis cs:46.054, -47.512324) -- (axis cs:46.0727, -46.512405)
  -- (axis cs:46.0908, -45.512484) ; \draw[constellation-boundary] (axis
  cs:41.0852, -48.471011) -- (axis cs:42.0748, -48.478946) -- (axis
  cs:43.0645, -48.487039) -- (axis cs:44.0543, -48.495287) -- (axis
  cs:45.0443, -48.503688) -- (axis cs:46.0345, -48.512239)
  ; \draw[constellation-boundary] (axis cs:41.0476, -50.470862) -- (axis
  cs:41.0668, -49.470938) -- (axis cs:41.0852, -48.471011)
  ; \draw[constellation-boundary] (axis cs:37.3411, -50.442573) -- (axis
  cs:38.0823, -50.448045) -- (axis cs:39.0706, -50.455487) -- (axis cs:40.059,
  -50.463093) -- (axis cs:41.0476, -50.470862) ; \draw[constellation-boundary]
  (axis cs:37.2833, -53.442361) -- (axis cs:37.3034, -52.442435) -- (axis
  cs:37.3227, -51.442505) -- (axis cs:37.3411, -50.442573)
  ; \draw[constellation-boundary] (axis cs:33.5841, -53.416474) -- (axis
  cs:34.0772, -53.419785) -- (axis cs:35.0635, -53.426538) -- (axis cs:36.05,
  -53.433464) -- (axis cs:37.0366, -53.440560) -- (axis cs:37.2833,
  -53.442361) ; \draw[constellation-boundary] (axis cs:21.2733, -52.848565) --
  (axis cs:22.2583, -52.852912) -- (axis cs:23.2433, -52.857453) -- (axis
  cs:24.2285, -52.862187) -- (axis cs:24.9674, -52.865864)
  ; \draw[constellation-boundary] (axis cs:24.9674, -52.865864) -- (axis
  cs:24.9742, -52.365881) -- (axis cs:24.9875, -51.365915) -- (axis
  cs:24.9939, -50.865931) ; \draw[constellation-boundary] (axis cs:24.9939,
  -50.865931) -- (axis cs:25.2405, -50.867181) -- (axis cs:26.2268,
  -50.872303) -- (axis cs:27.2133, -50.877613) -- (axis cs:28.1999,
  -50.883111) -- (axis cs:28.6933, -50.885929) ; \draw[constellation-boundary]
  (axis cs:28.6933, -50.885929) -- (axis cs:28.7005, -50.385950) -- (axis
  cs:28.7143, -49.385990) -- (axis cs:28.7277, -48.386028) -- (axis
  cs:28.7384, -47.552726) ; \draw[constellation-boundary] (axis cs:28.7384,
  -47.552726) -- (axis cs:29.2325, -47.555595) -- (axis cs:30.2208,
  -47.561471) -- (axis cs:31.2092, -47.567529) -- (axis cs:32.1977,
  -47.573769) -- (axis cs:33.1863, -47.580187) -- (axis cs:34.1751,
  -47.586782) -- (axis cs:35.164, -47.593553) -- (axis cs:36.153, -47.600496)
  ; \draw[constellation-boundary] (axis cs:36.153, -47.600496) -- (axis
  cs:36.1556, -47.433839) -- (axis cs:36.1708, -46.433893) -- (axis
  cs:36.1855, -45.433945) -- (axis cs:36.1997, -44.433996) -- (axis
  cs:36.2134, -43.434044) -- (axis cs:36.2267, -42.434092) -- (axis
  cs:36.2395, -41.434137) -- (axis cs:36.252, -40.434181) -- (axis cs:36.2641,
  -39.434224) ; \draw[constellation-boundary] (axis cs:46.1872, -39.512904) --
  (axis cs:46.1933, -39.096264) ; \draw[constellation-boundary] (axis
  cs:46.1933, -39.096264) -- (axis cs:47.1864, -39.104987) -- (axis
  cs:48.1795, -39.113854) -- (axis cs:49.1728, -39.122865) -- (axis
  cs:50.1663, -39.132016) -- (axis cs:51.1598, -39.141303) -- (axis
  cs:52.1535, -39.150725) -- (axis cs:53.1473, -39.160279) -- (axis
  cs:53.6443, -39.165104) ; \draw[constellation-boundary] (axis cs:53.6443,
  -39.165104) -- (axis cs:53.6536, -38.581817) -- (axis cs:53.6694,
  -37.581893) -- (axis cs:53.6847, -36.581968) -- (axis cs:53.6996,
  -35.582040) ; \draw[constellation-boundary] (axis cs:53.6996, -35.582040) --
  (axis cs:54.1969, -35.586901) -- (axis cs:55.1917, -35.596716) -- (axis
  cs:56.1867, -35.606654) -- (axis cs:57.1817, -35.616712) -- (axis
  cs:57.4305, -35.619245) ; \draw[constellation-boundary] (axis cs:57.4305,
  -35.619245) -- (axis cs:57.4457, -34.619323) -- (axis cs:57.4606,
  -33.619399) -- (axis cs:57.4751, -32.619473) -- (axis cs:57.4894,
  -31.619545) -- (axis cs:57.5033, -30.619616) -- (axis cs:57.5169,
  -29.619686) -- (axis cs:57.5303, -28.619754) -- (axis cs:57.5434,
  -27.619820) -- (axis cs:57.5562, -26.619886) -- (axis cs:57.5689,
  -25.619951) ; \draw[constellation-boundary] (axis cs:26.3507, -39.372630) --
  (axis cs:26.3594, -38.372653) -- (axis cs:26.3679, -37.372676) -- (axis
  cs:26.3761, -36.372697) -- (axis cs:26.3842, -35.372719) -- (axis cs:26.392,
  -34.372740) -- (axis cs:26.3997, -33.372760) -- (axis cs:26.4072,
  -32.372780) -- (axis cs:26.4145, -31.372799) -- (axis cs:26.4217,
  -30.372818) -- (axis cs:26.4288, -29.372837) -- (axis cs:26.4357,
  -28.372855) -- (axis cs:26.4424, -27.372873) -- (axis cs:26.4491,
  -26.372890) -- (axis cs:26.4556, -25.372908) ; \draw[constellation-boundary]
  (axis cs:-4.347, -39.306740) -- (axis cs:-3.357, -39.305754) -- (axis
  cs:-2.367, -39.304978) -- (axis cs:-1.377, -39.304412) -- (axis cs:-0.387,
  -39.304055) -- (axis cs:0.602937, -39.303908) -- (axis cs:1.59292,
  -39.303971) -- (axis cs:2.5829, -39.304244) -- (axis cs:3.57289, -39.304727)
  -- (axis cs:4.56289, -39.305419) -- (axis cs:5.55289, -39.306321) -- (axis
  cs:6.54292, -39.307432) -- (axis cs:7.53296, -39.308752) -- (axis
  cs:8.52301, -39.310280) -- (axis cs:9.51309, -39.312017) -- (axis
  cs:10.5032, -39.313961) -- (axis cs:11.4933, -39.316112) -- (axis
  cs:12.4835, -39.318469) -- (axis cs:13.4737, -39.321032) -- (axis
  cs:14.4639, -39.323799) -- (axis cs:15.4542, -39.326771) -- (axis
  cs:16.4445, -39.329946) -- (axis cs:17.4349, -39.333323) -- (axis
  cs:18.4253, -39.336901) -- (axis cs:19.4158, -39.340679) -- (axis
  cs:20.4063, -39.344656) -- (axis cs:21.3969, -39.348832) -- (axis
  cs:22.3875, -39.353203) -- (axis cs:23.3782, -39.357771) -- (axis
  cs:24.3689, -39.362532) -- (axis cs:25.3598, -39.367485) -- (axis
  cs:26.3507, -39.372630) -- (axis cs:27.3416, -39.377964) -- (axis
  cs:28.3327, -39.383487) -- (axis cs:29.3238, -39.389195) -- (axis cs:30.315,
  -39.395089) -- (axis cs:31.3063, -39.401165) -- (axis cs:32.2977,
  -39.407422) -- (axis cs:33.2891, -39.413859) -- (axis cs:34.2807,
  -39.420473) -- (axis cs:35.2723, -39.427262) -- (axis cs:36.2641,
  -39.434224) -- (axis cs:37.2559, -39.441358) -- (axis cs:38.2478,
  -39.448661) -- (axis cs:39.2399, -39.456131) -- (axis cs:40.232, -39.463766)
  -- (axis cs:41.2243, -39.471563) -- (axis cs:42.2166, -39.479521) -- (axis
  cs:43.2091, -39.487636) -- (axis cs:44.2017, -39.495907) -- (axis
  cs:45.1944, -39.504330) -- (axis cs:46.1872, -39.512904)
  ; \draw[constellation-boundary] (axis cs:53.365, -52.747077) -- (axis
  cs:53.8601, -52.751917) -- (axis cs:54.8506, -52.761689) -- (axis
  cs:55.8412, -52.771585) -- (axis cs:56.8321, -52.781602) -- (axis
  cs:57.8232, -52.791736) -- (axis cs:58.8145, -52.801984) -- (axis
  cs:59.8061, -52.812343) -- (axis cs:60.7979, -52.822811)
  ; \draw[constellation-boundary] (axis cs:53.2368, -57.079786) -- (axis
  cs:53.2531, -56.579866) -- (axis cs:53.2844, -55.580018) -- (axis
  cs:53.3141, -54.580163) -- (axis cs:53.3424, -53.580301) -- (axis cs:53.365,
  -52.747077) ; \draw[constellation-boundary] (axis cs:48.7911, -57.037787) --
  (axis cs:49.7786, -57.046884) -- (axis cs:50.7664, -57.056118) -- (axis
  cs:51.7544, -57.065487) -- (axis cs:52.7426, -57.074987) -- (axis
  cs:53.2368, -57.079786) ; \draw[constellation-boundary] (axis cs:48.3627,
  -67.035829) -- (axis cs:48.3921, -66.535963) -- (axis cs:48.4475,
  -65.536217) -- (axis cs:48.4989, -64.536451) -- (axis cs:48.5466,
  -63.536669) -- (axis cs:48.5911, -62.536873) -- (axis cs:48.6327,
  -61.537063) -- (axis cs:48.6716, -60.537241) -- (axis cs:48.7083,
  -59.537408) -- (axis cs:48.7428, -58.537566) -- (axis cs:48.7755,
  -57.537715) -- (axis cs:48.7911, -57.037787) ; \draw[constellation-boundary]
  (axis cs:33.2023, -66.915201) -- (axis cs:33.6903, -66.918477) -- (axis
  cs:34.6664, -66.925161) -- (axis cs:35.6428, -66.932017) -- (axis
  cs:36.6194, -66.939042) -- (axis cs:37.5963, -66.946236) -- (axis
  cs:38.5735, -66.953595) -- (axis cs:39.551, -66.961118) -- (axis cs:40.5288,
  -66.968803) -- (axis cs:41.5069, -66.976647) -- (axis cs:42.4854,
  -66.984649) -- (axis cs:43.4641, -66.992805) -- (axis cs:44.4431,
  -67.001114) -- (axis cs:45.4225, -67.009574) -- (axis cs:46.4022,
  -67.018181) -- (axis cs:47.3823, -67.026933) -- (axis cs:48.3627,
  -67.035829) -- (axis cs:49.3434, -67.044864) -- (axis cs:50.3246,
  -67.054038) -- (axis cs:51.306, -67.063346) -- (axis cs:52.2879, -67.072787)
  -- (axis cs:53.2701, -67.082357) -- (axis cs:54.2527, -67.092055) -- (axis
  cs:55.2357, -67.101876) -- (axis cs:56.2191, -67.111819) -- (axis
  cs:57.2029, -67.121880) -- (axis cs:58.1871, -67.132057) -- (axis
  cs:59.1717, -67.142346) -- (axis cs:60.1568, -67.152744) -- (axis
  cs:61.1422, -67.163249) -- (axis cs:62.1281, -67.173858) -- (axis
  cs:63.1143, -67.184567) -- (axis cs:64.1011, -67.195373) -- (axis
  cs:65.0882, -67.206273) -- (axis cs:66.0758, -67.217263) -- (axis
  cs:67.0639, -67.228342) -- (axis cs:68.0524, -67.239504) -- (axis
  cs:68.7941, -67.247930) ; \draw[constellation-boundary] (axis cs:33.2023,
  -66.915201) -- (axis cs:33.2236, -66.415271) -- (axis cs:33.2636,
  -65.415405) -- (axis cs:33.3008, -64.415529) -- (axis cs:33.3353,
  -63.415644) -- (axis cs:33.3675, -62.415751) -- (axis cs:33.3976,
  -61.415852) -- (axis cs:33.4258, -60.415946) -- (axis cs:33.4523,
  -59.416035) -- (axis cs:33.4774, -58.416118) -- (axis cs:33.501, -57.416197)
  -- (axis cs:33.5234, -56.416272) -- (axis cs:33.5447, -55.416342) -- (axis
  cs:33.5648, -54.416410) -- (axis cs:33.5841, -53.416474)
  ; \draw[constellation-boundary] (axis cs:67.9575, -74.743167) -- (axis
  cs:68.1121, -73.744047) -- (axis cs:68.2492, -72.744828) -- (axis
  cs:68.3717, -71.745525) -- (axis cs:68.4819, -70.746152) -- (axis
  cs:68.5816, -69.746720) -- (axis cs:68.6723, -68.747237) -- (axis
  cs:68.7552, -67.747709) -- (axis cs:68.8313, -66.748142) -- (axis
  cs:68.9015, -65.748542) -- (axis cs:68.9665, -64.748912) -- (axis
  cs:69.0268, -63.749255) -- (axis cs:69.083, -62.749575) -- (axis cs:69.1356,
  -61.749874) -- (axis cs:69.1848, -60.750154) -- (axis cs:69.231, -59.750417)
  -- (axis cs:69.2746, -58.750665) ; \draw[constellation-boundary] (axis
  cs:52.0758, -74.574131) -- (axis cs:52.562, -74.578886) -- (axis cs:53.535,
  -74.588491) -- (axis cs:54.5085, -74.598222) -- (axis cs:55.4827,
  -74.608075) -- (axis cs:56.4574, -74.618047) -- (axis cs:57.4327,
  -74.628136) -- (axis cs:58.4086, -74.638338) -- (axis cs:59.3852,
  -74.648652) -- (axis cs:60.3623, -74.659073) -- (axis cs:61.3401,
  -74.669600) -- (axis cs:62.3186, -74.680228) -- (axis cs:63.2977,
  -74.690956) -- (axis cs:64.2775, -74.701779) -- (axis cs:65.2579,
  -74.712695) -- (axis cs:66.239, -74.723701) -- (axis cs:67.2208, -74.734793)
  -- (axis cs:67.9575, -74.743167) ; \draw[constellation-boundary] (axis
  cs:1.53337, -81.803966) -- (axis cs:2.44897, -81.804219) -- (axis cs:3.3646,
  -81.804665) -- (axis cs:4.28029, -81.805305) -- (axis cs:5.19605,
  -81.806140) -- (axis cs:6.1119, -81.807168) -- (axis cs:7.02787, -81.808389)
  -- (axis cs:7.94396, -81.809803) -- (axis cs:8.86022, -81.811410) -- (axis
  cs:9.77664, -81.813210) -- (axis cs:10.6933, -81.815201) -- (axis
  cs:11.6101, -81.817384) -- (axis cs:12.5271, -81.819758) -- (axis
  cs:13.4444, -81.822322) -- (axis cs:14.362, -81.825076) -- (axis cs:15.2799,
  -81.828019) -- (axis cs:16.198, -81.831151) -- (axis cs:17.1165, -81.834470)
  -- (axis cs:18.0354, -81.837977) -- (axis cs:18.9546, -81.841669) -- (axis
  cs:19.8742, -81.845547) -- (axis cs:20.7942, -81.849609) -- (axis
  cs:21.7146, -81.853854) -- (axis cs:22.6355, -81.858282) -- (axis
  cs:23.5568, -81.862891) -- (axis cs:24.4786, -81.867680) -- (axis cs:25.401,
  -81.872648) -- (axis cs:26.3238, -81.877795) -- (axis cs:27.2472,
  -81.883117) -- (axis cs:28.1711, -81.888615) -- (axis cs:29.0956,
  -81.894288) -- (axis cs:30.0208, -81.900132) -- (axis cs:30.9465,
  -81.906148) -- (axis cs:31.8729, -81.912334) -- (axis cs:32.7999,
  -81.918688) -- (axis cs:33.7276, -81.925209) -- (axis cs:34.656, -81.931895)
  -- (axis cs:35.5851, -81.938745) -- (axis cs:36.515, -81.945757) -- (axis
  cs:37.4456, -81.952928) -- (axis cs:38.3769, -81.960258) -- (axis
  cs:39.3091, -81.967745) -- (axis cs:40.2421, -81.975386) -- (axis
  cs:41.1758, -81.983180) -- (axis cs:42.1105, -81.991125) -- (axis cs:43.046,
  -81.999219) -- (axis cs:43.9824, -82.007460) -- (axis cs:44.9197,
  -82.015846) -- (axis cs:45.8579, -82.024374) -- (axis cs:46.797, -82.033043)
  -- (axis cs:47.7371, -82.041851) -- (axis cs:48.6782, -82.050794) -- (axis
  cs:49.6203, -82.059872) -- (axis cs:50.0917, -82.064460)
  ; \draw[constellation-boundary] (axis cs:1.53337, -81.803966) -- (axis
  cs:1.53731, -81.303966) -- (axis cs:1.54398, -80.303967) -- (axis cs:1.5494,
  -79.303968) -- (axis cs:1.55391, -78.303968) -- (axis cs:1.55772,
  -77.303968) -- (axis cs:1.56098, -76.303968) -- (axis cs:1.56381,
  -75.303969) -- (axis cs:1.56628, -74.303969) ; \draw[constellation-boundary]
  (axis cs:12.2954, -75.318538) -- (axis cs:12.3324, -74.318585)
  ; \draw[constellation-boundary] (axis cs:12.2954, -75.318538) -- (axis
  cs:13.0113, -75.320409) -- (axis cs:13.966, -75.323078) -- (axis cs:14.9208,
  -75.325943) -- (axis cs:15.8759, -75.329005) -- (axis cs:16.8311,
  -75.332262) -- (axis cs:17.7865, -75.335714) -- (axis cs:18.7421,
  -75.339360) -- (axis cs:19.698, -75.343199) -- (axis cs:20.6541, -75.347229)
  ; \draw[constellation-boundary] (axis cs:20.6541, -75.347229) -- (axis
  cs:20.7173, -74.347366) -- (axis cs:20.7731, -73.347486) -- (axis
  cs:20.8227, -72.347593) -- (axis cs:20.8672, -71.347689) -- (axis
  cs:20.9073, -70.347776) -- (axis cs:20.9437, -69.347854) -- (axis
  cs:20.9768, -68.347926) -- (axis cs:21.0072, -67.347991) -- (axis
  cs:21.0351, -66.348051) -- (axis cs:21.0609, -65.348107) -- (axis
  cs:21.0848, -64.348158) -- (axis cs:21.107, -63.348206) -- (axis cs:21.1277,
  -62.348251) -- (axis cs:21.1471, -61.348293) -- (axis cs:21.1653,
  -60.348332) -- (axis cs:21.1824, -59.348369) -- (axis cs:21.1985,
  -58.348404) -- (axis cs:21.2138, -57.348437) -- (axis cs:21.2282,
  -56.348468) -- (axis cs:21.2419, -55.348497) -- (axis cs:21.2549,
  -54.348525) -- (axis cs:21.2673, -53.348552) -- (axis cs:21.2733,
  -52.848565) ; \draw[constellation-boundary] (axis cs:109.02, -85.261446) --
  (axis cs:110.346, -84.268724) -- (axis cs:111.281, -83.273852) -- (axis
  cs:111.976, -82.277660) -- (axis cs:112.513, -81.280600) -- (axis cs:112.94,
  -80.282941) -- (axis cs:113.289, -79.284849) -- (axis cs:113.578,
  -78.286435) -- (axis cs:113.823, -77.287775) -- (axis cs:114.033,
  -76.288922) -- (axis cs:114.215, -75.289916) ; \draw[constellation-boundary]
  (axis cs:48.233, -84.555390) -- (axis cs:49.1482, -83.559858) -- (axis
  cs:49.8191, -82.563131) -- (axis cs:50.3322, -81.565633) -- (axis
  cs:50.7374, -80.567608) -- (axis cs:51.0657, -79.569209) -- (axis
  cs:51.3372, -78.570532) -- (axis cs:51.5656, -77.571645) -- (axis
  cs:51.7604, -76.572594) -- (axis cs:51.9288, -75.573415) -- (axis
  cs:52.0758, -74.574131) ; \draw[constellation-boundary] (axis cs:48.233,
  -84.555390) -- (axis cs:48.69, -84.559880) -- (axis cs:49.6051, -84.568958)
  -- (axis cs:50.5215, -84.578163) -- (axis cs:51.4394, -84.587494) -- (axis
  cs:52.3587, -84.596948) -- (axis cs:53.2795, -84.606524) -- (axis
  cs:54.2019, -84.616219) -- (axis cs:55.1257, -84.626030) -- (axis
  cs:56.0511, -84.635956) -- (axis cs:56.9781, -84.645994) -- (axis
  cs:57.9066, -84.656141) -- (axis cs:58.8368, -84.666395) -- (axis
  cs:59.7687, -84.676754) -- (axis cs:60.7023, -84.687214) -- (axis
  cs:61.6375, -84.697774) -- (axis cs:62.5746, -84.708431) -- (axis
  cs:63.5133, -84.719182) -- (axis cs:64.4539, -84.730024) -- (axis
  cs:65.3963, -84.740955) -- (axis cs:66.3405, -84.751971) -- (axis
  cs:67.2866, -84.763071) -- (axis cs:68.2346, -84.774251) -- (axis
  cs:69.1845, -84.785507) -- (axis cs:70.1364, -84.796839) -- (axis
  cs:71.0902, -84.808241) -- (axis cs:72.0461, -84.819712) -- (axis
  cs:73.0039, -84.831248) -- (axis cs:73.9638, -84.842846) -- (axis
  cs:74.9258, -84.854503) -- (axis cs:75.8899, -84.866216) -- (axis
  cs:76.8561, -84.877981) -- (axis cs:77.8244, -84.889796) -- (axis
  cs:78.7949, -84.901657) -- (axis cs:79.7676, -84.913561) -- (axis
  cs:80.7425, -84.925505) -- (axis cs:81.7197, -84.937485) -- (axis
  cs:82.6991, -84.949497) -- (axis cs:83.6808, -84.961539) -- (axis
  cs:84.6648, -84.973606) -- (axis cs:85.6511, -84.985696) -- (axis
  cs:86.6398, -84.997805) -- (axis cs:87.6309, -85.009928) -- (axis
  cs:88.6243, -85.022064) -- (axis cs:89.6201, -85.034207) -- (axis
  cs:90.6184, -85.046355) -- (axis cs:91.6191, -85.058504) -- (axis
  cs:92.6223, -85.070649) -- (axis cs:93.6279, -85.082788) -- (axis cs:94.636,
  -85.094916) -- (axis cs:95.6467, -85.107030) -- (axis cs:96.6599,
  -85.119126) -- (axis cs:97.6756, -85.131200) -- (axis cs:98.6939,
  -85.143248) -- (axis cs:99.7147, -85.155266) -- (axis cs:100.738,
  -85.167251) -- (axis cs:101.764, -85.179198) -- (axis cs:102.793,
  -85.191104) -- (axis cs:103.824, -85.202964) -- (axis cs:104.858,
  -85.214775) -- (axis cs:105.894, -85.226533) -- (axis cs:106.934,
  -85.238233) -- (axis cs:107.975, -85.249872) -- (axis cs:109.02, -85.261446)
  ; \draw[constellation-boundary] (axis cs:97.7708, -75.100034) -- (axis
  cs:98.0222, -75.103048) -- (axis cs:99.0282, -75.115086) -- (axis
  cs:100.035, -75.127094) -- (axis cs:101.043, -75.139068) -- (axis
  cs:102.051, -75.151004) -- (axis cs:103.06, -75.162898) -- (axis cs:104.071,
  -75.174746) -- (axis cs:105.081, -75.186545) -- (axis cs:106.093,
  -75.198292) -- (axis cs:107.106, -75.209981) -- (axis cs:108.119,
  -75.221611) -- (axis cs:109.133, -75.233176) -- (axis cs:110.148,
  -75.244674) -- (axis cs:111.163, -75.256101) -- (axis cs:112.18, -75.267452)
  -- (axis cs:113.197, -75.278726) -- (axis cs:114.215, -75.289916) -- (axis
  cs:115.233, -75.301021) -- (axis cs:116.253, -75.312037) -- (axis
  cs:117.273, -75.322960) -- (axis cs:118.294, -75.333786) -- (axis
  cs:119.316, -75.344512) -- (axis cs:120.338, -75.355135) -- (axis
  cs:121.361, -75.365651) -- (axis cs:122.385, -75.376056) -- (axis cs:123.41,
  -75.386348) -- (axis cs:124.435, -75.396523) -- (axis cs:125.462,
  -75.406577) -- (axis cs:126.488, -75.416507) -- (axis cs:127.516,
  -75.426311) -- (axis cs:128.544, -75.435984) -- (axis cs:129.573,
  -75.445523) -- (axis cs:130.603, -75.454926) -- (axis cs:131.633,
  -75.464189) -- (axis cs:132.664, -75.473309) -- (axis cs:133.695,
  -75.482284) -- (axis cs:134.727, -75.491109) -- (axis cs:135.76, -75.499783)
  -- (axis cs:136.794, -75.508301) -- (axis cs:137.828, -75.516662) -- (axis
  cs:138.862, -75.524863) -- (axis cs:139.898, -75.532900) -- (axis
  cs:140.933, -75.540772) -- (axis cs:141.97, -75.548475) -- (axis cs:143.007,
  -75.556006) -- (axis cs:144.044, -75.563364) -- (axis cs:145.082,
  -75.570546) -- (axis cs:146.121, -75.577548) -- (axis cs:147.16, -75.584370)
  -- (axis cs:148.199, -75.591008) -- (axis cs:149.239, -75.597461) -- (axis
  cs:150.28, -75.603725) -- (axis cs:151.321, -75.609800) -- (axis cs:152.362,
  -75.615682) -- (axis cs:153.404, -75.621370) -- (axis cs:154.446,
  -75.626862) -- (axis cs:155.489, -75.632156) -- (axis cs:156.532,
  -75.637250) -- (axis cs:157.575, -75.642142) -- (axis cs:158.619,
  -75.646831) -- (axis cs:159.663, -75.651315) -- (axis cs:160.707,
  -75.655593) -- (axis cs:161.752, -75.659663) -- (axis cs:162.797,
  -75.663523) -- (axis cs:163.842, -75.667172) -- (axis cs:164.888,
  -75.670609) -- (axis cs:165.933, -75.673834) -- (axis cs:166.979,
  -75.676843) -- (axis cs:168.026, -75.679638) -- (axis cs:169.072,
  -75.682216) -- (axis cs:170.119, -75.684577) -- (axis cs:171.165,
  -75.686719) -- (axis cs:172.212, -75.688643) -- (axis cs:173.259,
  -75.690348) -- (axis cs:174.307, -75.691832) -- (axis cs:175.354,
  -75.693096) -- (axis cs:176.401, -75.694138) -- (axis cs:177.449,
  -75.694959) -- (axis cs:178.496, -75.695559) -- (axis cs:179.544,
  -75.695936) ; \draw[constellation-boundary] (axis cs:97.7708, -75.100034) --
  (axis cs:97.9408, -74.101053) -- (axis cs:98.0913, -73.101955) -- (axis
  cs:98.2254, -72.102759) -- (axis cs:98.3458, -71.103480) -- (axis
  cs:98.4545, -70.104131) -- (axis cs:98.5532, -69.104723) -- (axis
  cs:98.6432, -68.105262) -- (axis cs:98.7258, -67.105757) -- (axis
  cs:98.8019, -66.106213) -- (axis cs:98.8721, -65.106634) -- (axis
  cs:98.9373, -64.107024) ; \draw[constellation-boundary] (axis cs:-4.298,
  -57.806713) -- (axis cs:-3.318, -57.805737) -- (axis cs:-2.337, -57.804968)
  -- (axis cs:-1.356, -57.804407) -- (axis cs:-0.376, -57.804054) -- (axis
  cs:0.604805, -57.803908) -- (axis cs:1.58544, -57.803971) -- (axis
  cs:2.56607, -57.804241) -- (axis cs:3.54672, -57.804719) -- (axis
  cs:4.52738, -57.805405) -- (axis cs:5.50806, -57.806298) -- (axis
  cs:6.48877, -57.807399) -- (axis cs:7.46951, -57.808706) -- (axis
  cs:8.45029, -57.810221) -- (axis cs:9.43111, -57.811941) -- (axis cs:10.412,
  -57.813866) -- (axis cs:11.3929, -57.815997) -- (axis cs:12.3739,
  -57.818333) -- (axis cs:13.3549, -57.820872) -- (axis cs:14.336, -57.823614)
  -- (axis cs:15.3172, -57.826558) -- (axis cs:16.2985, -57.829704) -- (axis
  cs:17.2799, -57.833050) -- (axis cs:18.2613, -57.836596) -- (axis
  cs:19.2429, -57.840341) -- (axis cs:20.2245, -57.844282) -- (axis
  cs:21.2063, -57.848420) -- (axis cs:22.1881, -57.852754) -- (axis
  cs:23.1701, -57.857281) -- (axis cs:24.1522, -57.862000) -- (axis
  cs:25.1345, -57.866911) -- (axis cs:26.1168, -57.872012) -- (axis
  cs:27.0993, -57.877301) -- (axis cs:28.082, -57.882777) -- (axis cs:29.0648,
  -57.888437) -- (axis cs:30.0477, -57.894282) -- (axis cs:31.0309,
  -57.900308) -- (axis cs:32.0141, -57.906515) -- (axis cs:32.9976,
  -57.912899) -- (axis cs:33.4894, -57.916158) ; \draw[constellation-boundary]
  (axis cs:121.038, -43.353501) -- (axis cs:122.044, -43.363943) -- (axis
  cs:123.05, -43.374274) -- (axis cs:124.056, -43.384491) -- (axis cs:125.062,
  -43.394592) -- (axis cs:126.069, -43.404572) -- (axis cs:126.572,
  -43.409516) ; \draw[constellation-boundary] (axis cs:126.572, -43.409516) --
  (axis cs:126.59, -42.409605) -- (axis cs:126.608, -41.409692) -- (axis
  cs:126.625, -40.409776) -- (axis cs:126.642, -39.409857) -- (axis
  cs:126.658, -38.409936) -- (axis cs:126.674, -37.410013) -- (axis
  cs:126.689, -36.410088) -- (axis cs:126.704, -35.410161) -- (axis
  cs:126.719, -34.410233) -- (axis cs:126.733, -33.410302) -- (axis
  cs:126.747, -32.410370) -- (axis cs:126.76, -31.410437) -- (axis cs:126.774,
  -30.410502) -- (axis cs:126.787, -29.410566) -- (axis cs:126.799,
  -28.410628) -- (axis cs:126.812, -27.410690) -- (axis cs:126.824,
  -26.410750) -- (axis cs:126.836, -25.410809) ; \draw[constellation-boundary]
  (axis cs:126.678, -37.160032) -- (axis cs:127.18, -37.164939) -- (axis
  cs:128.186, -37.174656) -- (axis cs:129.191, -37.184244) -- (axis
  cs:130.197, -37.193699) -- (axis cs:131.203, -37.203020) -- (axis
  cs:132.209, -37.212201) -- (axis cs:133.215, -37.221242) -- (axis
  cs:134.221, -37.230139) -- (axis cs:135.228, -37.238889) -- (axis
  cs:136.234, -37.247490) -- (axis cs:137.241, -37.255939) -- (axis
  cs:138.247, -37.264233) -- (axis cs:139.254, -37.272370) -- (axis
  cs:140.261, -37.280348) -- (axis cs:141.268, -37.288162) -- (axis
  cs:141.772, -37.292008) ;
  




% Labels

% Phoenix too small to label



\node[constellation-label] at (axis cs:{240,-53}) {\large\bfseries\textls{Norma}};

\node[constellation-label] at (axis cs:{244,-65.3}) {\normalsize\bfseries\textls{Triangulum}};
\node[constellation-label] at (axis cs:{234,-65}) {\normalsize\bfseries\textls{australe}};

\node[constellation-label] at (axis cs:{224,-64.5}) {\normalsize\bfseries\textls{Circinus}};

\node[constellation-label] at (axis cs:{118.5,-70}) {\large\bfseries\textls{Volans}};
\node[constellation-label] at (axis cs:{194,-70}) {\normalsize\bfseries\textls{Musca}};
\node[constellation-label] at (axis cs:{122.5,-60}) {\Large\bfseries\textls{Carina}};
\node[constellation-label] at (axis cs:{110,-45}) {\Large\bfseries\textls{Puppis}};

\node[constellation-label,rotate=-20] at (axis cs:{160,-78}) {\normalsize\bfseries\textls{Chamaeleon}};
\node[constellation-label,rotate=60] at (axis cs:{240,-75}) {\Large\bfseries\textls{Apus}};
\node[constellation-label,rotate=103] at (axis cs:{103,-31}) {\normalsize\bfseries\textls{Canis Major}};
\node[constellation-label,rotate=85] at (axis cs:{85,-37}) {\large\bfseries\textls{Columba}};
\node[constellation-label] at (axis cs:{85,-53}) {\large\bfseries\textls{Pictor}};
\node[constellation-label,rotate=-45] at (axis cs:{78,-62}) {\large\bfseries\textls{Dorado}};
\node[constellation-label,rotate=0] at (axis cs:{89,-74}) {\large\bfseries\textls{Mensa}};
\node[constellation-label,rotate=-17] at (axis cs:{73,-39}) {\normalsize\bfseries\textls{Caelum}};
\node[constellation-label,rotate=-32] at (axis cs:{57,-59}) {\normalsize\bfseries\textls{Reticulum}};
\node[constellation-label,rotate=45] at (axis cs:{45,-73}) {\large\bfseries\textls{Hydrus}};

\node[constellation-label,rotate=0] at (axis cs:{1,-73.5}) {\small\bfseries\textls{Tucana}};
\node[constellation-label,rotate=0] at (axis cs:{335,-71.5}) {\normalsize\bfseries\textls{Indus}};
\node[constellation-label,rotate=0] at (axis cs:{63.5,-47}) {\small\bfseries\textls{Hor.}};
\node[constellation-label,rotate=-45] at (axis cs:{44,-65}) {\normalsize\bfseries\textls{Hor.}};


\node[constellation-label] at (axis cs:{310,-79}) {\Large\bfseries\textls{Octans}};


\node[constellation-label,rotate=-65] at (axis cs:{295,-65}) {\Large\bfseries\textls{Pavo}};
\node[constellation-label,rotate=0] at (axis cs:{258,-55}) {\Large\bfseries\textls{Ara}};

\node[constellation-label,rotate=-70] at (axis cs:{290,-52}) {\normalsize\bfseries\textls{Telescopium}};

\node[constellation-label,rotate=-80] at (axis cs:{280,-41}) {\normalsize\bfseries\textls{Corona australis}};
\node[constellation-label,rotate=-80] at (axis cs:{280,-34}) {\Large\bfseries\textls{Sagittarius}};
\node[constellation-label,rotate=-0] at (axis cs:{260,-38}) {\LARGE\bfseries\textls{Scorpius}};

\node[constellation-label,rotate=0] at (axis cs:{187,-59.5}) {\large\bfseries\textls{Crux}};
\node[constellation-label,rotate=0] at (axis cs:{180,-33}) {\large\bfseries\textls{Hydra}};
\node[constellation-label,rotate=-0] at (axis cs:{230,-42}) {\Large\bfseries\textls{Lupus}};

\node[constellation-label,rotate=-0] at (axis cs:{191,-45}) {\LARGE\bfseries\textls{Centaurus}};


\node[constellation-label,rotate=-29] at (axis cs:{151,-33}) {\large\bfseries\textls{Antlia}};
\node[constellation-label,rotate=-46] at (axis cs:{134,-33}) {\normalsize\bfseries\textls{Pyxis}};
\node[constellation-label,rotate=-0] at (axis cs:{150,-49}) {\Large\bfseries\textls{Vela}};


\end{polaraxis}


% Nebulae and nebulae names


\begin{polaraxis}[rotate=270,name=stars,at={($(base.center)+(+0.75pt,0pt)$)},anchor=center,axis lines=none]

\clip (6\tendegree,0\tendegree) arc (0:90:6\tendegree) -- 
(-2\tendegree,6\tendegree) -- (-2\tendegree,-6\tendegree) -- (0\tendegree,-6\tendegree)
--  (0\tendegree,-6\tendegree) arc (270:359.9999:6\tendegree) -- cycle ;

\node[Messier] at (axis cs:13,-72) {\tikz\draw[dashed]  circle (2\onedegree) ;}; %  SMC
\node[Messier-label] at (axis cs:22,-72) {SMC}; %  SMC

\node[Messier] at (axis cs:80.7,{-69.45}) {\tikz\draw[dashed]  circle (4\onedegree) ;}; %  LMC
\node[Messier-label] at (axis cs:79,-75) {LMC}; %  LMC

\node[NGC,pin={[pin distance=-1.2\onedegree,Flaamsted]0:{47}}] at (axis cs:6.025,-72.083) {\tikz\pgfuseplotmark{GC};}; %  4.0, NGC 104 =  47 Tuc
\node[NGC,pin={[pin distance=-1.2\onedegree,Bayer]90:{$\omega$}}] at (axis cs:201.700,-47.483) {\tikz\pgfuseplotmark{GC};}; %  3.7 omega Cen

\node[NGC,pin={[pin distance=-1.2\onedegree,NGC-label]90:{   55}}] at (axis cs:3.725,-39.183) {\tikz\pgfuseplotmark{GAL};}; %  8. 
\node[NGC,pin={[pin distance=-1.2\onedegree,NGC-label]90:{  134}}] at (axis cs:7.600,-33.250) {\tikz\pgfuseplotmark{GAL};}; % 10.1
\node[NGC,pin={[pin distance=-1.2\onedegree,NGC-label]90:{  300}}] at (axis cs:13.725,-37.683) {\tikz\pgfuseplotmark{GAL};}; %  9. 
\node[NGC,pin={[pin distance=-1.2\onedegree,NGC-label]90:{  613}}] at (axis cs:23.575,-29.417) {\tikz\pgfuseplotmark{GAL};}; % 10.0
\node[NGC,pin={[pin distance=-1.2\onedegree,NGC-label]90:{ 1097}}] at (axis cs:41.575,-30.283) {\tikz\pgfuseplotmark{GAL};}; %  9.3
\node[NGC,pin={[pin distance=-1.2\onedegree,NGC-label]90:{ 1291}}] at (axis cs:49.325,-41.133) {\tikz\pgfuseplotmark{GAL};}; %  8.5
\node[NGC,pin={[pin distance=-1.2\onedegree,NGC-label]180:{ 1313}}] at (axis cs:49.575,-66.500) {\tikz\pgfuseplotmark{GAL};}; %  9. 
\node[NGC,pin={[pin distance=-1.2\onedegree,NGC-label]90:{ 1316}}] at (axis cs:50.675,-37.200) {\tikz\pgfuseplotmark{GAL};}; %  8.9
\node[NGC,pin={[pin distance=-1.2\onedegree,NGC-label]90:{ 1326}}] at (axis cs:50.975,-36.467) {\tikz\pgfuseplotmark{GAL};}; % 10.5
\node[NGC,pin={[pin distance=-1.2\onedegree,NGC-label]90:{ 1344}}] at (axis cs:52.075,-31.067) {\tikz\pgfuseplotmark{GAL};}; % 10.3
\node[NGC,pin={[pin distance=-1.2\onedegree,NGC-label]90:{ 1350}}] at (axis cs:52.775,-33.633) {\tikz\pgfuseplotmark{GAL};}; % 10.5
\node[NGC,pin={[pin distance=-1.2\onedegree,NGC-label]90:{ 1365}}] at (axis cs:53.400,-36.133) {\tikz\pgfuseplotmark{GAL};}; %  9.5
\node[NGC,pin={[pin distance=-1.2\onedegree,NGC-label]90:{ 1399}}] at (axis cs:54.625,-35.450) {\tikz\pgfuseplotmark{GAL};}; %  9.9
\node[NGC,pin={[pin distance=-1.2\onedegree,NGC-label]90:{ 1404}}] at (axis cs:54.725,-35.583) {\tikz\pgfuseplotmark{GAL};}; % 10.3
\node[NGC,pin={[pin distance=-1.2\onedegree,NGC-label]90:{ 1433}}] at (axis cs:55.500,-47.217) {\tikz\pgfuseplotmark{GAL};}; % 10.0
\node[NGC,pin={[pin distance=-1.2\onedegree,NGC-label]90:{ 1512}}] at (axis cs:60.975,-43.350) {\tikz\pgfuseplotmark{GAL};}; % 10.6
\node[NGC,pin={[pin distance=-1.2\onedegree,NGC-label]0:{ 1533}}] at (axis cs:62.475,-56.117) {\tikz\pgfuseplotmark{GAL};}; % 10.9
\node[NGC,pin={[pin distance=-1.2\onedegree,NGC-label]0:{ 1543}}] at (axis cs:63.200,-57.733) {\tikz\pgfuseplotmark{GAL};}; % 10.6
\node[NGC,pin={[pin distance=-1.8\onedegree,NGC-label]135:{ 1549}}] at (axis cs:63.925,-55.600) {\tikz\pgfuseplotmark{GAL};}; %  9.9
\node[NGC,pin={[pin distance=-1.2\onedegree,NGC-label]0:{ 1553}}] at (axis cs:64.050,-55.783) {\tikz\pgfuseplotmark{GAL};}; %  9.5
\node[NGC,pin={[pin distance=-1.2\onedegree,NGC-label]180:{ 1559}}] at (axis cs:64.400,-62.783) {\tikz\pgfuseplotmark{GAL};}; % 10.5
\node[NGC,pin={[pin distance=-1.2\onedegree,NGC-label]180:{ 1566}}] at (axis cs:65,-54.933) {\tikz\pgfuseplotmark{GAL};}; %  9.4
\node[NGC,pin={[pin distance=-1.2\onedegree,NGC-label]-90:{ 1574}}] at (axis cs:65.500,-56.967) {\tikz\pgfuseplotmark{GAL};}; % 10.5
\node[NGC,pin={[pin distance=-1.2\onedegree,NGC-label]90:{ 1617}}] at (axis cs:67.925,-54.600) {\tikz\pgfuseplotmark{GAL};}; % 10.4
\node[NGC,pin={[pin distance=-1.2\onedegree,NGC-label]90:{ 1792}}] at (axis cs:76.300,-37.983) {\tikz\pgfuseplotmark{GAL};}; % 10.2
\node[NGC,pin={[pin distance=-1.2\onedegree,NGC-label]-90:{ 1808}}] at (axis cs:76.925,-37.517) {\tikz\pgfuseplotmark{GAL};}; %  9.9
\node[NGC,pin={[pin distance=-1.2\onedegree,NGC-label]180:{ 1947}}] at (axis cs:81.700,-63.767) {\tikz\pgfuseplotmark{GAL};}; % 10.8
\node[NGC,pin={[pin distance=-1.2\onedegree,NGC-label]90:{ 2217}}] at (axis cs:95.425,-27.233) {\tikz\pgfuseplotmark{GAL};}; % 10.4
\node[NGC,pin={[pin distance=-1.2\onedegree,NGC-label]90:{ 3557}}] at (axis cs:167.500,-37.533) {\tikz\pgfuseplotmark{GAL};}; % 10.4
\node[NGC,pin={[pin distance=-1.2\onedegree,NGC-label]90:{ 3621}}] at (axis cs:169.575,-32.817) {\tikz\pgfuseplotmark{GAL};}; % 10. 
\node[NGC,pin={[pin distance=-1.2\onedegree,NGC-label]90:{ 3923}}] at (axis cs:177.750,-28.800) {\tikz\pgfuseplotmark{GAL};}; % 10.1
\node[NGC,pin={[pin distance=-1.2\onedegree,NGC-label]90:{ 4696}}] at (axis cs:192.200,-41.317) {\tikz\pgfuseplotmark{GAL};}; % 10.7
\node[NGC,pin={[pin distance=-1.2\onedegree,NGC-label]180:{ 4945}}] at (axis cs:196.350,-49.467) {\tikz\pgfuseplotmark{GAL};}; %  9. 
\node[NGC,pin={[pin distance=-1.8\onedegree,NGC-label]-45:{ 4976}}] at (axis cs:197.150,-49.500) {\tikz\pgfuseplotmark{GAL};}; % 10.2
\node[NGC,pin={[pin distance=-1.2\onedegree,NGC-label]0:{ 5102}}] at (axis cs:200.500,-36.633) {\tikz\pgfuseplotmark{GAL};}; %  9.7
\node[NGC,pin={[pin distance=-1.2\onedegree,NGC-label]90:{ 5128 (Cen A)}}] at (axis cs:201.375,-43.017) {\tikz\pgfuseplotmark{GAL};}; %  7.0
\node[NGC,pin={[pin distance=-1.2\onedegree,NGC-label]180:{I4296}}] at (axis cs:204.150,-33.967) {\tikz\pgfuseplotmark{GAL};}; % 10.6
\node[NGC,pin={[pin distance=-1.2\onedegree,NGC-label]90:{ 5253}}] at (axis cs:204.975,-31.650) {\tikz\pgfuseplotmark{GAL};}; % 10.6
\node[NGC,pin={[pin distance=-1.2\onedegree,NGC-label]90:{ 6684}}] at (axis cs:282.250,-65.183) {\tikz\pgfuseplotmark{GAL};}; % 10.5
\node[NGC,pin={[pin distance=-1.2\onedegree,NGC-label]90:{ 6744}}] at (axis cs:287.450,-63.850) {\tikz\pgfuseplotmark{GAL};}; %  9. 
\node[NGC,pin={[pin distance=-1.2\onedegree,NGC-label]90:{ 7049}}] at (axis cs:319.750,-48.567) {\tikz\pgfuseplotmark{GAL};}; % 10.7
\node[NGC,pin={[pin distance=-1.2\onedegree,NGC-label]90:{ 7144}}] at (axis cs:328.175,-48.250) {\tikz\pgfuseplotmark{GAL};}; % 10.7
\node[NGC,pin={[pin distance=-1.2\onedegree,NGC-label]90:{ 7213}}] at (axis cs:332.325,-47.167) {\tikz\pgfuseplotmark{GAL};}; % 10.5
\node[NGC,pin={[pin distance=-1.2\onedegree,NGC-label]90:{ 7410}}] at (axis cs:343.750,-39.667) {\tikz\pgfuseplotmark{GAL};}; % 10.4
\node[NGC,pin={[pin distance=-1.2\onedegree,NGC-label]90:{I1459}}] at (axis cs:344.300,-36.467) {\tikz\pgfuseplotmark{GAL};}; % 10.0
\node[NGC,pin={[pin distance=-1.2\onedegree,NGC-label]90:{I5267}}] at (axis cs:344.300,-43.400) {\tikz\pgfuseplotmark{GAL};}; % 10.5
\node[NGC,pin={[pin distance=-1.2\onedegree,NGC-label]90:{ 7507}}] at (axis cs:348.025,-28.533) {\tikz\pgfuseplotmark{GAL};}; % 10.4
\node[NGC,pin={[pin distance=-1.2\onedegree,NGC-label]90:{ 7552}}] at (axis cs:349.050,-42.583) {\tikz\pgfuseplotmark{GAL};}; % 10.7
\node[NGC,pin={[pin distance=-1.2\onedegree,NGC-label]90:{ 7582}}] at (axis cs:349.600,-42.367) {\tikz\pgfuseplotmark{GAL};}; % 10.6
\node[NGC,pin={[pin distance=-1.2\onedegree,NGC-label]90:{I5332}}] at (axis cs:353.625,-36.100) {\tikz\pgfuseplotmark{GAL};}; % 10.6
\node[NGC,pin={[pin distance=-1.2\onedegree,NGC-label]90:{ 7793}}] at (axis cs:359.450,-32.583) {\tikz\pgfuseplotmark{GAL};}; %  9.1
\node[NGC,pin={[pin distance=-1.2\onedegree,NGC-label]0:{ 2867}}] at (axis cs:140.350,-58.317) {\tikz\pgfuseplotmark{PN};}; % 10. 
\node[NGC,pin={[pin distance=-1.2\onedegree,NGC-label]90:{ 3132}}] at (axis cs:151.750,-40.433) {\tikz\pgfuseplotmark{PN};}; %  8. 
\node[NGC,pin={[pin distance=-1.2\onedegree,NGC-label]180:{ 3918}}] at (axis cs:177.575,-57.183) {\tikz\pgfuseplotmark{PN};}; %  8. 
\node[NGC,pin={[pin distance=-1.2\onedegree,NGC-label]90:{ 5189}}] at (axis cs:203.375,-65.983) {\tikz\pgfuseplotmark{PN};}; % 10. 
\node[NGC,pin={[pin distance=-1.2\onedegree,NGC-label]90:{  346}}] at (axis cs:14.775,-72.183) {\tikz\pgfuseplotmark{CN};}; % 10.3
\node[NGC,pin={[pin distance=-1.2\onedegree,NGC-label]-90:{ 1770}}] at (axis cs:74.250,-68.417) {\tikz\pgfuseplotmark{CN};}; %  9. 
%\node[NGC,pin={[pin distance=-1.2\onedegree,NGC-label]90:{ 1814}}] at (axis cs:75.950,-67.283) {\tikz\pgfuseplotmark{CN};}; %  9. 
%\node[NGC,pin={[pin distance=-1.2\onedegree,NGC-label]-90:{ 1816}}] at (axis cs:75.950,-67.283) {\tikz\pgfuseplotmark{CN};}; %  9. 
%\node[NGC,pin={[pin distance=-1.2\onedegree,NGC-label]90:{ 1829}}] at (axis cs:76.175,-68.050) {\tikz\pgfuseplotmark{CN};}; %  8. 
%\node[NGC,pin={[pin distance=-1.2\onedegree,NGC-label]90:{ 1955}}] at (axis cs:81.525,-67.483) {\tikz\pgfuseplotmark{CN};}; %  9. 
%\node[NGC,pin={[pin distance=-1.2\onedegree,NGC-label]90:{ 1962}}] at (axis cs:81.625,-68.783) {\tikz\pgfuseplotmark{CN};}; %  8. 
%\node[NGC,pin={[pin distance=-1.2\onedegree,NGC-label]90:{ 1965}}] at (axis cs:81.625,-68.783) {\tikz\pgfuseplotmark{CN};}; %  8. 
%\node[NGC,pin={[pin distance=-1.2\onedegree,NGC-label]90:{ 1970}}] at (axis cs:81.625,-68.783) {\tikz\pgfuseplotmark{CN};}; %  8. 
%\node[NGC,pin={[pin distance=-1.2\onedegree,NGC-label]90:{ 1968}}] at (axis cs:81.800,-67.433) {\tikz\pgfuseplotmark{CN};}; %  9. 
\node[NGC,pin={[pin distance=-1.2\onedegree,NGC-label]90:{ 1983}}] at (axis cs:81.875,-68.950) {\tikz\pgfuseplotmark{CN};}; %  8. 
%\node[NGC,pin={[pin distance=-1.2\onedegree,NGC-label]90:{ 1974}}] at (axis cs:81.975,-67.400) {\tikz\pgfuseplotmark{CN};}; %  9. 
%\node[NGC,pin={[pin distance=-1.2\onedegree,NGC-label]90:{ 2011}}] at (axis cs:83.025,-67.517) {\tikz\pgfuseplotmark{CN};}; %  9. 
\node[NGC,pin={[pin distance=-1.2\onedegree,NGC-label]-90:{ 2014}}] at (axis cs:83.050,-67.667) {\tikz\pgfuseplotmark{CN};}; %  8. 
\node[NGC,pin={[pin distance=-1.2\onedegree,NGC-label]0:{ 2070}}] at (axis cs:84.650,-69.083) {\tikz\pgfuseplotmark{CN};}; %  8.2
\node[NGC,pin={[pin distance=-1.2\onedegree,NGC-label]-90:{ 2579}}] at (axis cs:125.275,-36.183) {\tikz\pgfuseplotmark{CN};}; %  7.5
\node[NGC,pin={[pin distance=-1.2\onedegree,NGC-label]180:{ 3247}}] at (axis cs:156.475,-57.933) {\tikz\pgfuseplotmark{CN};}; %  7.6
\node[NGC,pin={[pin distance=-1.8\onedegree,NGC-label]-45:{ 3293}}] at (axis cs:158.950,-58.233) {\tikz\pgfuseplotmark{CN};}; %  4.7
%\node[NGC,pin={[pin distance=-1.2\onedegree,NGC-label]90:{ 3324}}] at (axis cs:159.325,-58.633) {\tikz\pgfuseplotmark{CN};}; %  6.7
\node[NGC,pin={[pin distance=-1.2\onedegree,NGC-label]-90:{ 3572}}] at (axis cs:167.600,-60.233) {\tikz\pgfuseplotmark{CN};}; %  6.6
\node[NGC,pin={[pin distance=-1.2\onedegree,NGC-label]90:{ 3603}}] at (axis cs:168.775,-61.250) {\tikz\pgfuseplotmark{CN};}; %  9.1
\node[NGC,pin={[pin distance=-1.2\onedegree,NGC-label]90:{I2944}}] at (axis cs:174.150,-63.033) {\tikz\pgfuseplotmark{CN};}; %  4.5
\node[NGC,pin={[pin distance=-1.2\onedegree,NGC-label]-90:{ 6231}}] at (axis cs:253.500,-41.800) {\tikz\pgfuseplotmark{CN};}; %  2.6
\node[NGC,pin={[pin distance=-1.2\onedegree,NGC-label]90:{ 6281}}] at (axis cs:256.200,-37.900) {\tikz\pgfuseplotmark{CN};}; %  5.4
\node[NGC,pin={[pin distance=-1.2\onedegree,NGC-label]90:{ 6383}}] at (axis cs:263.700,-32.567) {\tikz\pgfuseplotmark{CN};}; %  5.5
\node[NGC,pin={[pin distance=-1.2\onedegree,NGC-label]0:{  330}}] at (axis cs:14.050,-72.483) {\tikz\pgfuseplotmark{OC};}; %  9.6
\node[NGC,pin={[pin distance=-1.2\onedegree,NGC-label]0:{ 1711}}] at (axis cs:72.625,-70) {\tikz\pgfuseplotmark{OC};}; % 10. 
\node[NGC,pin={[pin distance=-1.2\onedegree,NGC-label]00:{ 1755}}] at (axis cs:73.750,-68.183) {\tikz\pgfuseplotmark{OC};}; %  9.9
%\node[NGC,pin={[pin distance=-1.2\onedegree,NGC-label]90:{ 1774}}] at (axis cs:74.450,-67.250) {\tikz\pgfuseplotmark{OC};}; % 10. 
\node[NGC,pin={[pin distance=-1.2\onedegree,NGC-label]180:{ 1805}}] at (axis cs:75.550,-66.100) {\tikz\pgfuseplotmark{OC};}; % 10. 
%\node[NGC,pin={[pin distance=-1.2\onedegree,NGC-label]90:{ 1820}}] at (axis cs:75.950,-67.283) {\tikz\pgfuseplotmark{OC};}; %  9. 
\node[NGC,pin={[pin distance=-1.2\onedegree,NGC-label]-90:{ 1818}}] at (axis cs:76.050,-66.400) {\tikz\pgfuseplotmark{OC};}; %  9.8
%\node[NGC,pin={[pin distance=-1.2\onedegree,NGC-label]90:{ 1850}}] at (axis cs:77.125,-68.767) {\tikz\pgfuseplotmark{OC};}; %  9.3
%\node[NGC,pin={[pin distance=-1.2\onedegree,NGC-label]90:{ 1854}}] at (axis cs:77.275,-68.850) {\tikz\pgfuseplotmark{OC};}; % 10. 
\node[NGC,pin={[pin distance=-1.2\onedegree,NGC-label]-90:{ 1856}}] at (axis cs:77.350,-69.133) {\tikz\pgfuseplotmark{OC};}; % 10. 
\node[NGC,pin={[pin distance=-1.2\onedegree,NGC-label]180:{ 1866}}] at (axis cs:78.375,-65.467) {\tikz\pgfuseplotmark{OC};}; %  9.8
\node[NGC,pin={[pin distance=-1.2\onedegree,NGC-label]90:{ 1951}}] at (axis cs:81.475,-66.600) {\tikz\pgfuseplotmark{OC};}; % 10. 
%\node[NGC,pin={[pin distance=-1.2\onedegree,NGC-label]90:{ 2004}}] at (axis cs:82.650,-67.283) {\tikz\pgfuseplotmark{OC};}; %  9.8
\node[NGC,pin={[pin distance=-1.2\onedegree,NGC-label]-90:{ 2041}}] at (axis cs:84.100,-66.967) {\tikz\pgfuseplotmark{OC};}; % 10. 
\node[NGC,pin={[pin distance=-1.2\onedegree,NGC-label]180:{ 2100}}] at (axis cs:85.500,-69.233) {\tikz\pgfuseplotmark{OC};}; %  9.6
\node[NGC,pin={[pin distance=-1.2\onedegree,NGC-label]180:{ 2136}}] at (axis cs:88.200,-69.500) {\tikz\pgfuseplotmark{OC};}; % 10. 
\node[NGC,pin={[pin distance=-1.2\onedegree,NGC-label]0:{ 2157}}] at (axis cs:89.325,-69.183) {\tikz\pgfuseplotmark{OC};}; % 10. 
\node[NGC,pin={[pin distance=-1.2\onedegree,NGC-label]180:{ 2164}}] at (axis cs:89.675,-68.517) {\tikz\pgfuseplotmark{OC};}; % 10. 
\node[NGC,pin={[pin distance=-1.2\onedegree,NGC-label]90:{ 2243}}] at (axis cs:97.450,-31.283) {\tikz\pgfuseplotmark{OC};}; %  9.4
\node[NGC,pin={[pin distance=-1.2\onedegree,NGC-label]90:{ 2439}}] at (axis cs:115.200,-31.650) {\tikz\pgfuseplotmark{OC};}; %  6.9
\node[NGC,pin={[pin distance=-1.2\onedegree,NGC-label]180:{ 2451}}] at (axis cs:116.350,-37.967) {\tikz\pgfuseplotmark{OC};}; %  2.8
\node[NGC,pin={[pin distance=-1.2\onedegree,NGC-label]90:{ 2453}}] at (axis cs:116.950,-27.233) {\tikz\pgfuseplotmark{OC};}; %  8.3
\node[NGC,pin={[pin distance=-1.2\onedegree,NGC-label]180:{ 2477}}] at (axis cs:118.075,-38.550) {\tikz\pgfuseplotmark{OC};}; %  5.8
\node[NGC,pin={[pin distance=-1.2\onedegree,NGC-label]90:{ 2483}}] at (axis cs:118.975,-27.933) {\tikz\pgfuseplotmark{OC};}; %  7.6
%\node[NGC,pin={[pin distance=-1.2\onedegree,NGC-label]90:{ 2489}}] at (axis cs:119.050,-30.067) {\tikz\pgfuseplotmark{OC};}; %  7.9
\node[NGC,pin={[pin distance=-1.2\onedegree,NGC-label]90:{ 2516}}] at (axis cs:119.575,-60.867) {\tikz\pgfuseplotmark{OC};}; %  3.8
\node[NGC,pin={[pin distance=-1.2\onedegree,NGC-label]90:{ 2527}}] at (axis cs:121.325,-28.167) {\tikz\pgfuseplotmark{OC};}; %  6.5
%\node[NGC,pin={[pin distance=-1.2\onedegree,NGC-label]90:{ 2533}}] at (axis cs:121.750,-29.900) {\tikz\pgfuseplotmark{OC};}; %  7.6
\node[NGC,pin={[pin distance=-1.2\onedegree,NGC-label]90:{ 2547}}] at (axis cs:122.675,-49.267) {\tikz\pgfuseplotmark{OC};}; %  4.7
\node[NGC,pin={[pin distance=-1.2\onedegree,NGC-label]0:{ 2546}}] at (axis cs:123.100,-37.633) {\tikz\pgfuseplotmark{OC};}; %  6.3
\node[NGC,pin={[pin distance=-1.2\onedegree,NGC-label]0:{ 2567}}] at (axis cs:124.650,-30.633) {\tikz\pgfuseplotmark{OC};}; %  7.4
%\node[NGC,pin={[pin distance=-1.2\onedegree,NGC-label]90:{ 2571}}] at (axis cs:124.725,-29.733) {\tikz\pgfuseplotmark{OC};}; %  7.0
\node[NGC,pin={[pin distance=-1.2\onedegree,NGC-label]0:{ 2580}}] at (axis cs:125.400,-30.317) {\tikz\pgfuseplotmark{OC};}; % 10. 
%\node[NGC,pin={[pin distance=-1.2\onedegree,NGC-label]90:{ 2587}}] at (axis cs:125.875,-29.500) {\tikz\pgfuseplotmark{OC};}; %  9. 
\node[NGC,pin={[pin distance=-1.2\onedegree,NGC-label]0:{ 2627}}] at (axis cs:129.325,-29.950) {\tikz\pgfuseplotmark{OC};}; %  8. 
\node[NGC,pin={[pin distance=-1.2\onedegree,NGC-label]90:{I2391}}] at (axis cs:130.050,-53.067) {\tikz\pgfuseplotmark{OC};}; %  2.5
\node[NGC,pin={[pin distance=-1.2\onedegree,NGC-label]180:{I2395}}] at (axis cs:130.275,-48.200) {\tikz\pgfuseplotmark{OC};}; %  4.6
\node[NGC,pin={[pin distance=-1.2\onedegree,NGC-label]0:{ 2660}}] at (axis cs:130.550,-47.150) {\tikz\pgfuseplotmark{OC};}; %  8.8
\node[NGC,pin={[pin distance=-1.2\onedegree,NGC-label]180:{ 2659}}] at (axis cs:130.650,-44.950) {\tikz\pgfuseplotmark{OC};}; %  8.6
\node[NGC,pin={[pin distance=-1.2\onedegree,NGC-label]180:{ 2658}}] at (axis cs:130.850,-32.650) {\tikz\pgfuseplotmark{OC};}; %  9. 
\node[NGC,pin={[pin distance=-1.2\onedegree,NGC-label]-90:{ 2669}}] at (axis cs:131.225,-52.967) {\tikz\pgfuseplotmark{OC};}; %  6.1
\node[NGC,pin={[pin distance=-1.2\onedegree,NGC-label]0:{ 2670}}] at (axis cs:131.375,-48.783) {\tikz\pgfuseplotmark{OC};}; %  7.8
\node[NGC,pin={[pin distance=-1.2\onedegree,NGC-label]180:{ 2818}}] at (axis cs:139,-36.617) {\tikz\pgfuseplotmark{OC};}; %  8.2
\node[NGC,pin={[pin distance=-1.2\onedegree,NGC-label]90:{I2488}}] at (axis cs:141.900,-56.983) {\tikz\pgfuseplotmark{OC};}; %  7. 
\node[NGC,pin={[pin distance=-1.2\onedegree,NGC-label]-90:{ 2910}}] at (axis cs:142.600,-52.900) {\tikz\pgfuseplotmark{OC};}; %  7.2
\node[NGC,pin={[pin distance=-1.2\onedegree,NGC-label]90:{ 2925}}] at (axis cs:143.425,-53.433) {\tikz\pgfuseplotmark{OC};}; %  8. 
\node[NGC,pin={[pin distance=-1.2\onedegree,NGC-label]90:{ 2972}}] at (axis cs:145.075,-50.333) {\tikz\pgfuseplotmark{OC};}; %  9.9
\node[NGC,pin={[pin distance=-1.2\onedegree,NGC-label]90:{ 3033}}] at (axis cs:147.200,-56.417) {\tikz\pgfuseplotmark{OC};}; %  8.8
\node[NGC,pin={[pin distance=-1.2\onedegree,NGC-label]0:{ 3105}}] at (axis cs:150.200,-54.767) {\tikz\pgfuseplotmark{OC};}; %  9.7
\node[NGC,pin={[pin distance=-1.2\onedegree,NGC-label]-90:{ 3114}}] at (axis cs:150.675,-60.117) {\tikz\pgfuseplotmark{OC};}; %  4.2
\node[NGC,pin={[pin distance=-1.2\onedegree,NGC-label]90:{ 3228}}] at (axis cs:155.450,-51.717) {\tikz\pgfuseplotmark{OC};}; %  6.0
%\node[NGC,pin={[pin distance=-1.2\onedegree,NGC-label]90:{I2581}}] at (axis cs:156.850,-57.633) {\tikz\pgfuseplotmark{OC};}; %  4.3
\node[NGC,pin={[pin distance=-1.2\onedegree,NGC-label]90:{ 3330}}] at (axis cs:159.650,-54.150) {\tikz\pgfuseplotmark{OC};}; %  7.4
%\node[NGC,pin={[pin distance=-1.2\onedegree,NGC-label]90:{I2602}}] at (axis cs:160.800,-64.400) {\tikz\pgfuseplotmark{OC};}; %  1.9
\node[NGC,pin={[pin distance=-1.2\onedegree,NGC-label]180:{ 3496}}] at (axis cs:164.950,-60.333) {\tikz\pgfuseplotmark{OC};}; %  8.2
\node[NGC,pin={[pin distance=-1.2\onedegree,NGC-label]180:{ 3532}}] at (axis cs:166.600,-58.667) {\tikz\pgfuseplotmark{OC};}; %  3.0
\node[NGC,pin={[pin distance=-1.8\onedegree,NGC-label]-45:{ 3590}}] at (axis cs:168.225,-60.783) {\tikz\pgfuseplotmark{OC};}; %  8.2
\node[NGC,pin={[pin distance=-1.2\onedegree,NGC-label]90:{I2714}}] at (axis cs:169.475,-62.700) {\tikz\pgfuseplotmark{OC};}; %  8. 
\node[NGC,pin={[pin distance=-1.2\onedegree,NGC-label]90:{ 3680}}] at (axis cs:171.425,-43.250) {\tikz\pgfuseplotmark{OC};}; %  7.6
\node[NGC,pin={[pin distance=-1.8\onedegree,NGC-label]135:{ 3766}}] at (axis cs:174.025,-61.617) {\tikz\pgfuseplotmark{OC};}; %  5.3
\node[NGC,pin={[pin distance=-1.2\onedegree,NGC-label]90:{ 3960}}] at (axis cs:177.725,-55.700) {\tikz\pgfuseplotmark{OC};}; %  8.3
\node[NGC,pin={[pin distance=-1.2\onedegree,NGC-label]90:{ 4052}}] at (axis cs:180.475,-63.200) {\tikz\pgfuseplotmark{OC};}; %  9. 
\node[NGC,pin={[pin distance=-1.2\onedegree,NGC-label]90:{ 4103}}] at (axis cs:181.675,-61.250) {\tikz\pgfuseplotmark{OC};}; %  7. 
\node[NGC,pin={[pin distance=-1.2\onedegree,NGC-label]-90:{ 4230}}] at (axis cs:184.325,-55.133) {\tikz\pgfuseplotmark{OC};}; %  9. 
\node[NGC,pin={[pin distance=-1.2\onedegree,NGC-label]0:{ 4337}}] at (axis cs:185.975,-58.133) {\tikz\pgfuseplotmark{OC};}; %  8.9
\node[NGC,pin={[pin distance=-1.2\onedegree,NGC-label]-90:{ 4349}}] at (axis cs:186.125,-61.900) {\tikz\pgfuseplotmark{OC};}; %  7.4
\node[NGC,pin={[pin distance=-1.2\onedegree,NGC-label]90:{ 4439}}] at (axis cs:187.100,-60.100) {\tikz\pgfuseplotmark{OC};}; %  8.4
\node[NGC,pin={[pin distance=-1.2\onedegree,NGC-label]90:{ 4463}}] at (axis cs:187.500,-64.800) {\tikz\pgfuseplotmark{OC};}; %  7.2
\node[NGC,pin={[pin distance=-1.2\onedegree,NGC-label]-90:{ 4609}}] at (axis cs:190.575,-62.967) {\tikz\pgfuseplotmark{OC};}; %  6.9
\node[NGC,pin={[pin distance=-1.2\onedegree,NGC-label]0:{ 4755}}] at (axis cs:193.400,-60.333) {\tikz\pgfuseplotmark{OC};}; %  4.2
\node[NGC,pin={[pin distance=-1.2\onedegree,NGC-label]90:{ 485}}] at (axis cs:194.500,-64.950) {\tikz\pgfuseplotmark{OC};}; %  8.6
\node[NGC,pin={[pin distance=-1.2\onedegree,NGC-label]0:{ 4852}}] at (axis cs:195.025,-59.600) {\tikz\pgfuseplotmark{OC};}; %  9. 
\node[NGC,pin={[pin distance=-1.2\onedegree,NGC-label]90:{ 5138}}] at (axis cs:201.825,-59.017) {\tikz\pgfuseplotmark{OC};}; %  7.6
\node[NGC,pin={[pin distance=-1.2\onedegree,NGC-label]90:{ 5168}}] at (axis cs:202.800,-60.933) {\tikz\pgfuseplotmark{OC};}; %  9.1
\node[NGC,pin={[pin distance=-1.2\onedegree,NGC-label]180:{ 5281}}] at (axis cs:206.650,-62.900) {\tikz\pgfuseplotmark{OC};}; %  5.9
\node[NGC,pin={[pin distance=-1.2\onedegree,NGC-label]-90:{ 5316}}] at (axis cs:208.475,-61.867) {\tikz\pgfuseplotmark{OC};}; %  6.0
\node[NGC,pin={[pin distance=-1.2\onedegree,NGC-label]90:{ 5460}}] at (axis cs:211.900,-48.317) {\tikz\pgfuseplotmark{OC};}; %  5.6
\node[NGC,pin={[pin distance=-1.2\onedegree,NGC-label]0:{ 5606}}] at (axis cs:216.950,-59.633) {\tikz\pgfuseplotmark{OC};}; %  7.7
\node[NGC,pin={[pin distance=-1.2\onedegree,NGC-label]-90:{ 5617}}] at (axis cs:217.450,-60.717) {\tikz\pgfuseplotmark{OC};}; %  6.3
\node[NGC,pin={[pin distance=-1.2\onedegree,NGC-label]90:{ 5662}}] at (axis cs:218.800,-56.550) {\tikz\pgfuseplotmark{OC};}; %  5.5
\node[NGC,pin={[pin distance=-1.2\onedegree,NGC-label]90:{ 5715}}] at (axis cs:220.850,-57.550) {\tikz\pgfuseplotmark{OC};}; % 10. 
\node[NGC,pin={[pin distance=-1.2\onedegree,NGC-label]90:{ 5749}}] at (axis cs:222.225,-54.517) {\tikz\pgfuseplotmark{OC};}; %  9. 
\node[NGC,pin={[pin distance=-1.2\onedegree,NGC-label]90:{ 5822}}] at (axis cs:226.300,-54.350) {\tikz\pgfuseplotmark{OC};}; %  7. 
\node[NGC,pin={[pin distance=-1.2\onedegree,NGC-label]180:{ 5823}}] at (axis cs:226.425,-55.600) {\tikz\pgfuseplotmark{OC};}; %  7.9
\node[NGC,pin={[pin distance=-1.2\onedegree,NGC-label]90:{ 5925}}] at (axis cs:231.925,-54.517) {\tikz\pgfuseplotmark{OC};}; %  8. 
\node[NGC,pin={[pin distance=-1.2\onedegree,NGC-label]90:{ 5999}}] at (axis cs:238.050,-56.467) {\tikz\pgfuseplotmark{OC};}; %  9. 
\node[NGC,pin={[pin distance=-1.2\onedegree,NGC-label]90:{ 6005}}] at (axis cs:238.950,-57.433) {\tikz\pgfuseplotmark{OC};}; % 10.7
\node[NGC,pin={[pin distance=-1.2\onedegree,NGC-label]90:{ 6025}}] at (axis cs:240.925,-60.500) {\tikz\pgfuseplotmark{OC};}; %  5.1
\node[NGC,pin={[pin distance=-1.2\onedegree,NGC-label]180:{ 6031}}] at (axis cs:241.900,-54.067) {\tikz\pgfuseplotmark{OC};}; %  8.5
\node[NGC,pin={[pin distance=-1.2\onedegree,NGC-label]-90:{ 6067}}] at (axis cs:243.300,-54.217) {\tikz\pgfuseplotmark{OC};}; %  5.6
\node[NGC,pin={[pin distance=-1.2\onedegree,NGC-label]90:{ 6087}}] at (axis cs:244.725,-57.900) {\tikz\pgfuseplotmark{OC};}; %  5.4
\node[NGC,pin={[pin distance=-1.2\onedegree,NGC-label]90:{ 6124}}] at (axis cs:246.400,-40.667) {\tikz\pgfuseplotmark{OC};}; %  5.8
\node[NGC,pin={[pin distance=-1.2\onedegree,NGC-label]90:{ 6134}}] at (axis cs:246.925,-49.150) {\tikz\pgfuseplotmark{OC};}; %  7.2
\node[NGC,pin={[pin distance=-1.2\onedegree,NGC-label]90:{ 6152}}] at (axis cs:248.175,-52.617) {\tikz\pgfuseplotmark{OC};}; %  8. 
\node[NGC,pin={[pin distance=-1.2\onedegree,NGC-label]90:{ 6169}}] at (axis cs:248.525,-44.050) {\tikz\pgfuseplotmark{OC};}; %  7. 
\node[NGC,pin={[pin distance=-1.2\onedegree,NGC-label]90:{ 6167}}] at (axis cs:248.600,-49.600) {\tikz\pgfuseplotmark{OC};}; %  6.7
\node[NGC,pin={[pin distance=-1.2\onedegree,NGC-label]90:{ 6178}}] at (axis cs:248.925,-45.633) {\tikz\pgfuseplotmark{OC};}; %  7.2
\node[NGC,pin={[pin distance=-1.2\onedegree,NGC-label]90:{ 6192}}] at (axis cs:250.075,-43.367) {\tikz\pgfuseplotmark{OC};}; %  9. 
\node[NGC,pin={[pin distance=-1.2\onedegree,NGC-label]90:{ 6193}}] at (axis cs:250.325,-48.767) {\tikz\pgfuseplotmark{OC};}; %  5.2
\node[NGC,pin={[pin distance=-1.2\onedegree,NGC-label]-90:{ 6200}}] at (axis cs:251.050,-47.483) {\tikz\pgfuseplotmark{OC};}; %  7.4
\node[NGC,pin={[pin distance=-1.2\onedegree,NGC-label]90:{ 6204}}] at (axis cs:251.625,-47.017) {\tikz\pgfuseplotmark{OC};}; %  8.2
\node[NGC,pin={[pin distance=-1.2\onedegree,NGC-label]-90:{ 6216}}] at (axis cs:252.350,-44.733) {\tikz\pgfuseplotmark{OC};}; % 10. 
\node[NGC,pin={[pin distance=-1.2\onedegree,NGC-label]90:{ 6208}}] at (axis cs:252.375,-53.817) {\tikz\pgfuseplotmark{OC};}; %  7.2
\node[NGC,pin={[pin distance=-1.2\onedegree,NGC-label]90:{ 6222}}] at (axis cs:252.675,-44.750) {\tikz\pgfuseplotmark{OC};}; % 10. 
\node[NGC,pin={[pin distance=-1.2\onedegree,NGC-label]90:{ 6242}}] at (axis cs:253.900,-39.500) {\tikz\pgfuseplotmark{OC};}; %  6.4
\node[NGC,pin={[pin distance=-1.2\onedegree,NGC-label]0:{ 6249}}] at (axis cs:254.400,-44.783) {\tikz\pgfuseplotmark{OC};}; %  8.2
\node[NGC,pin={[pin distance=-1.2\onedegree,NGC-label]90:{ 6250}}] at (axis cs:254.500,-45.800) {\tikz\pgfuseplotmark{OC};}; %  5.9
\node[NGC,pin={[pin distance=-1.2\onedegree,NGC-label]0:{ 6253}}] at (axis cs:254.775,-52.717) {\tikz\pgfuseplotmark{OC};}; % 10. 
\node[NGC,pin={[pin distance=-1.2\onedegree,NGC-label]90:{ 6259}}] at (axis cs:255.175,-44.667) {\tikz\pgfuseplotmark{OC};}; %  8.0
\node[NGC,pin={[pin distance=-1.2\onedegree,NGC-label]90:{ 6268}}] at (axis cs:255.600,-39.733) {\tikz\pgfuseplotmark{OC};}; % 10. 
\node[NGC,pin={[pin distance=-1.2\onedegree,NGC-label]90:{ 6322}}] at (axis cs:259.625,-42.950) {\tikz\pgfuseplotmark{OC};}; %  6.0
\node[NGC,pin={[pin distance=-1.2\onedegree,NGC-label]-90:{I4651}}] at (axis cs:261.175,-49.950) {\tikz\pgfuseplotmark{OC};}; %  6.9
\node[NGC,pin={[pin distance=-1.2\onedegree,NGC-label]-90:{ 6374}}] at (axis cs:263.075,-32.600) {\tikz\pgfuseplotmark{OC};}; %  9.0
\node[NGC,pin={[pin distance=-1.2\onedegree,NGC-label]90:{ 6396}}] at (axis cs:264.525,-35) {\tikz\pgfuseplotmark{OC};}; %  8.5
\node[NGC,pin={[pin distance=-1.2\onedegree,NGC-label]90:{ 6400}}] at (axis cs:265.200,-36.950) {\tikz\pgfuseplotmark{OC};}; %  9. 
\node[NGC,pin={[pin distance=-1.2\onedegree,NGC-label]180:{ 6416}}] at (axis cs:266.100,-32.350) {\tikz\pgfuseplotmark{OC};}; %  5.7
\node[NGC,pin={[pin distance=-1.2\onedegree,NGC-label]0:{ 6425}}] at (axis cs:266.725,-31.533) {\tikz\pgfuseplotmark{OC};}; %  7.2
\node[NGC,pin={[pin distance=-1.2\onedegree,NGC-label]180:{ 6451}}] at (axis cs:267.675,-30.217) {\tikz\pgfuseplotmark{OC};}; %  8. 
\node[NGC,pin={[pin distance=-1.2\onedegree,NGC-label]90:{ 6520}}] at (axis cs:270.850,-27.900) {\tikz\pgfuseplotmark{OC};}; %  8. 
\node[NGC,pin={[pin distance=-1.2\onedegree,NGC-label]0:{  121}}] at (axis cs:6.700,-71.533) {\tikz\pgfuseplotmark{GC};}; % 10.6
\node[NGC,pin={[pin distance=-1.2\onedegree,NGC-label]0:{  362}}] at (axis cs:15.800,-70.850) {\tikz\pgfuseplotmark{GC};}; %  6.6
\node[NGC,pin={[pin distance=-1.2\onedegree,NGC-label]-90:{  419}}] at (axis cs:17.075,-72.883) {\tikz\pgfuseplotmark{GC};}; % 10.0
\node[NGC,pin={[pin distance=-1.2\onedegree,NGC-label]90:{ 1261}}] at (axis cs:48.075,-55.217) {\tikz\pgfuseplotmark{GC};}; %  8.4
\node[NGC,pin={[pin distance=-1.2\onedegree,NGC-label]90:{ 1786}}] at (axis cs:74.775,-67.750) {\tikz\pgfuseplotmark{GC};}; % 10.1
\node[NGC,pin={[pin distance=-1.2\onedegree,NGC-label]0:{ 1835}}] at (axis cs:76.300,-69.400) {\tikz\pgfuseplotmark{GC};}; %  9.8
\node[NGC,pin={[pin distance=-1.2\onedegree,NGC-label]0:{ 1851}}] at (axis cs:78.525,-40.050) {\tikz\pgfuseplotmark{GC};}; %  7.3
\node[NGC,pin={[pin distance=-1.2\onedegree,NGC-label]180:{ 1978}}] at (axis cs:82.150,-66.233) {\tikz\pgfuseplotmark{GC};}; %  9.9
\node[NGC,pin={[pin distance=-1.2\onedegree,NGC-label]-90:{ 2210}}] at (axis cs:92.875,-69.133) {\tikz\pgfuseplotmark{GC};}; % 10.2
\node[NGC,pin={[pin distance=-1.2\onedegree,NGC-label]90:{ 2298}}] at (axis cs:102.250,-36) {\tikz\pgfuseplotmark{GC};}; %  9.4
\node[NGC,pin={[pin distance=-1.2\onedegree,NGC-label]90:{ 2808}}] at (axis cs:138,-64.867) {\tikz\pgfuseplotmark{GC};}; %  6.3
\node[NGC,pin={[pin distance=-1.2\onedegree,NGC-label]90:{ 3201}}] at (axis cs:154.400,-46.417) {\tikz\pgfuseplotmark{GC};}; %  6.8
\node[NGC,pin={[pin distance=-1.2\onedegree,NGC-label]90:{ 4372}}] at (axis cs:186.450,-72.667) {\tikz\pgfuseplotmark{GC};}; %  7.8
\node[NGC,pin={[pin distance=-1.2\onedegree,NGC-label]0:{ 4833}}] at (axis cs:194.900,-70.883) {\tikz\pgfuseplotmark{GC};}; %  7.4

\node[NGC,pin={[pin distance=-1.2\onedegree,NGC-label]90:{ 5286}}] at (axis cs:206.600,-51.367) {\tikz\pgfuseplotmark{GC};}; %  7.6
\node[NGC,pin={[pin distance=-1.2\onedegree,NGC-label]0:{I4499}}] at (axis cs:225.075,-82.217) {\tikz\pgfuseplotmark{GC};}; % 10.6
\node[NGC,pin={[pin distance=-1.2\onedegree,NGC-label]90:{ 5824}}] at (axis cs:226,-33.067) {\tikz\pgfuseplotmark{GC};}; %  9.0
\node[NGC,pin={[pin distance=-1.2\onedegree,NGC-label]180:{ 5927}}] at (axis cs:232,-50.667) {\tikz\pgfuseplotmark{GC};}; %  8.3
\node[NGC,pin={[pin distance=-1.2\onedegree,NGC-label]90:{ 5946}}] at (axis cs:233.875,-50.667) {\tikz\pgfuseplotmark{GC};}; %  9.6
\node[NGC,pin={[pin distance=-1.2\onedegree,NGC-label]90:{ 5986}}] at (axis cs:236.525,-37.783) {\tikz\pgfuseplotmark{GC};}; %  7.1
\node[NGC,pin={[pin distance=-1.2\onedegree,NGC-label]90:{ 6101}}] at (axis cs:246.450,-72.200) {\tikz\pgfuseplotmark{GC};}; %  9.3
\node[NGC,pin={[pin distance=-1.2\onedegree,NGC-label]90:{ 6139}}] at (axis cs:246.925,-38.850) {\tikz\pgfuseplotmark{GC};}; %  9.2
\node[NGC,pin={[pin distance=-1.2\onedegree,NGC-label]90:{ 6304}}] at (axis cs:258.625,-29.467) {\tikz\pgfuseplotmark{GC};}; %  8.4
\node[NGC,pin={[pin distance=-1.2\onedegree,NGC-label]90:{ 6316}}] at (axis cs:259.150,-28.133) {\tikz\pgfuseplotmark{GC};}; %  9.0
\node[NGC,pin={[pin distance=-1.2\onedegree,NGC-label]90:{ 6352}}] at (axis cs:261.375,-48.417) {\tikz\pgfuseplotmark{GC};}; %  8.2
\node[NGC,pin={[pin distance=-1.2\onedegree,NGC-label]90:{ 6362}}] at (axis cs:262.975,-67.050) {\tikz\pgfuseplotmark{GC};}; %  8.3
\node[NGC,pin={[pin distance=-1.2\onedegree,NGC-label]90:{ 6388}}] at (axis cs:264.075,-44.733) {\tikz\pgfuseplotmark{GC};}; %  6.9
\node[NGC,pin={[pin distance=-1.2\onedegree,NGC-label]90:{ 6397}}] at (axis cs:265.175,-53.667) {\tikz\pgfuseplotmark{GC};}; %  5.7
\node[NGC,pin={[pin distance=-1.2\onedegree,NGC-label]90:{ 6441}}] at (axis cs:267.550,-37.050) {\tikz\pgfuseplotmark{GC};}; %  7.4
\node[NGC,pin={[pin distance=-1.2\onedegree,NGC-label]-90:{ 6453}}] at (axis cs:267.725,-34.600) {\tikz\pgfuseplotmark{GC};}; %  9.9
\node[NGC,pin={[pin distance=-1.2\onedegree,NGC-label]90:{ 6496}}] at (axis cs:269.750,-44.267) {\tikz\pgfuseplotmark{GC};}; %  9.2
\node[NGC,pin={[pin distance=-1.2\onedegree,NGC-label]90:{ }}] at (axis cs:270.900,-30.033) {\tikz\pgfuseplotmark{GC};}; %  8.6
\node[NGC,pin={[pin distance=-1.2\onedegree,NGC-label]90:{ }}] at (axis cs:271.200,-30.050) {\tikz\pgfuseplotmark{GC};}; %  9.5
\node[NGC,pin={[pin distance=-1.2\onedegree,NGC-label]90:{ 6541}}] at (axis cs:272,-43.700) {\tikz\pgfuseplotmark{GC};}; %  6.6
\node[NGC,pin={[pin distance=-1.2\onedegree,NGC-label]90:{ 6569}}] at (axis cs:273.400,-31.833) {\tikz\pgfuseplotmark{GC};}; %  8.7
\node[NGC,pin={[pin distance=-1.2\onedegree,NGC-label]90:{ 6584}}] at (axis cs:274.650,-52.217) {\tikz\pgfuseplotmark{GC};}; %  9.2
\node[NGC,pin={[pin distance=-1.2\onedegree,NGC-label]90:{ 6624}}] at (axis cs:275.925,-30.367) {\tikz\pgfuseplotmark{GC};}; %  8.3
\node[NGC,pin={[pin distance=-1.2\onedegree,NGC-label]90:{ 6652}}] at (axis cs:278.950,-32.983) {\tikz\pgfuseplotmark{GC};}; %  8.9
\node[NGC,pin={[pin distance=-1.2\onedegree,NGC-label]0:{ 6723}}] at (axis cs:284.900,-36.633) {\tikz\pgfuseplotmark{GC};}; %  7.3
\node[NGC,pin={[pin distance=-1.2\onedegree,NGC-label]90:{ 6752}}] at (axis cs:287.725,-59.983) {\tikz\pgfuseplotmark{GC};}; %  5.4

%\node[Messier,pin={[pin distance=-0.8\onedegree,Messier-label]90:{M83}}] at (axis cs:204.250,-29.867) {\tikz\pgfuseplotmark{GAL};}; %  7.6
\node[Messier,pin={[pin distance=-0.8\onedegree,Messier-label]90:{M54}}] at (axis cs:283.775,-30.483) {\tikz\pgfuseplotmark{GC};}; %  7.7
\node[Messier,pin={[pin distance=-0.8\onedegree,Messier-label]90:{M55}}] at (axis cs:295,-30.967) {\tikz\pgfuseplotmark{GC};}; %  7.0
\node[Messier,pin={[pin distance=-0.8\onedegree,Messier-label]180:{M62}}] at (axis cs:255.300,-30.117) {\tikz\pgfuseplotmark{GC};}; %  6.6
\node[Messier,pin={[pin distance=-0.8\onedegree,Messier-label]0:{M6}}] at (axis cs:265.025,-32.217) {\tikz\pgfuseplotmark{OC};}; %  4.2
\node[Messier,pin={[pin distance=-0.8\onedegree,Messier-label]-90:{M69}}] at (axis cs:277.850,-32.350) {\tikz\pgfuseplotmark{GC};}; %  7.7
\node[Messier,pin={[pin distance=-0.8\onedegree,Messier-label]90:{M70}}] at (axis cs:280.800,-32.300) {\tikz\pgfuseplotmark{GC};}; %  8.1
\node[Messier,pin={[pin distance=-0.8\onedegree,Messier-label]90:{M7}}] at (axis cs:268.475,-34.817) {\tikz\pgfuseplotmark{OC};}; %  3.3



\end{polaraxis}

% Stars and star names
\begin{polaraxis}[rotate=270,name=stars,at={($(base.center)+(+0.75pt,0pt)$)},anchor=center,axis lines=none]

\clip (6\tendegree,0\tendegree) arc (0:90:6\tendegree) -- 
(-2\tendegree,6\tendegree) -- (-2\tendegree,-6\tendegree) -- (0\tendegree,-6\tendegree)
--  (0\tendegree,-6\tendegree) arc (270:359.9999:6\tendegree) -- cycle ;

\node[stars] at (axis cs:{0.080},{-44.290}) {\tikz\pgfuseplotmark{m6c};};
\node[stars] at (axis cs:{0.269},{-48.810}) {\tikz\pgfuseplotmark{m6a};};
\node[stars] at (axis cs:{0.334},{-50.337}) {\tikz\pgfuseplotmark{m6a};};
\node[stars] at (axis cs:{0.399},{-77.065}) {\tikz\pgfuseplotmark{m5a};};
\node[stars] at (axis cs:{0.583},{-29.720}) {\tikz\pgfuseplotmark{m5b};};
\node[stars] at (axis cs:{1.085},{-29.269}) {\tikz\pgfuseplotmark{m6c};};
\node[stars] at (axis cs:{1.172},{-71.437}) {\tikz\pgfuseplotmark{m6a};};
\node[stars] at (axis cs:{1.580},{-49.075}) {\tikz\pgfuseplotmark{m6a};};
\node[stars] at (axis cs:{2.015},{-33.529}) {\tikz\pgfuseplotmark{m6a};};
\node[stars] at (axis cs:{2.260},{-54.002}) {\tikz\pgfuseplotmark{m6c};};
\node[stars] at (axis cs:{2.338},{-27.988}) {\tikz\pgfuseplotmark{m5cb};};
\node[stars] at (axis cs:{2.353},{-45.747}) {\tikz\pgfuseplotmark{m4b};};
\node[stars] at (axis cs:{2.467},{-62.297}) {\tikz\pgfuseplotmark{m6c};};
\node[stars] at (axis cs:{2.509},{-82.224}) {\tikz\pgfuseplotmark{m5c};};
\node[stars] at (axis cs:{2.893},{-27.799}) {\tikz\pgfuseplotmark{m5c};};
\node[stars] at (axis cs:{2.933},{-35.133}) {\tikz\pgfuseplotmark{m5c};};
\node[stars] at (axis cs:{3.332},{-84.994}) {\tikz\pgfuseplotmark{m6a};};
\node[stars] at (axis cs:{3.743},{-34.904}) {\tikz\pgfuseplotmark{m6c};};
\node[stars] at (axis cs:{3.979},{-75.911}) {\tikz\pgfuseplotmark{m6c};};
\node[stars] at (axis cs:{4.037},{-31.446}) {\tikz\pgfuseplotmark{m6a};};
\node[stars] at (axis cs:{4.677},{-43.236}) {\tikz\pgfuseplotmark{m6c};};
\node[stars] at (axis cs:{5.018},{-64.875}) {\tikz\pgfuseplotmark{m4c};};
\node[stars] at (axis cs:{5.163},{-69.625}) {\tikz\pgfuseplotmark{m5c};};
\node[stars] at (axis cs:{5.370},{-77.427}) {\tikz\pgfuseplotmark{m6b};};
\node[stars] at (axis cs:{5.380},{-28.981}) {\tikz\pgfuseplotmark{m5c};};
\node[stars] at (axis cs:{6.438},{-77.254}) {\tikz\pgfuseplotmark{m3av};};
\node[stars] at (axis cs:{6.551},{-43.680}) {\tikz\pgfuseplotmark{m4b};};
\node[stars] at (axis cs:{6.571},{-42.306}) {\tikz\pgfuseplotmark{m2c};};
\node[stars] at (axis cs:{6.982},{-33.007}) {\tikz\pgfuseplotmark{m5bv};};
\node[stars] at (axis cs:{7.111},{-39.915}) {\tikz\pgfuseplotmark{m5c};};
\node[stars] at (axis cs:{7.180},{-50.533}) {\tikz\pgfuseplotmark{m6c};};
\node[stars] at (axis cs:{7.609},{-48.215}) {\tikz\pgfuseplotmark{m6a};};
\node[stars] at (axis cs:{7.616},{-40.939}) {\tikz\pgfuseplotmark{m6cv};};
\node[stars] at (axis cs:{7.854},{-48.803}) {\tikz\pgfuseplotmark{m5a};};
\node[stars] at (axis cs:{7.886},{-62.958}) {\tikz\pgfuseplotmark{m4cb};};
\node[stars] at (axis cs:{8.183},{-63.031}) {\tikz\pgfuseplotmark{m5b};};
\node[stars] at (axis cs:{8.347},{-71.266}) {\tikz\pgfuseplotmark{m6bv};};
\node[stars] at (axis cs:{8.421},{-29.558}) {\tikz\pgfuseplotmark{m6a};};
\node[stars] at (axis cs:{8.616},{-52.373}) {\tikz\pgfuseplotmark{m6a};};
\node[stars] at (axis cs:{8.889},{-54.822}) {\tikz\pgfuseplotmark{m6b};};
\node[stars] at (axis cs:{8.922},{-48.001}) {\tikz\pgfuseplotmark{m6a};};
\node[stars] at (axis cs:{9.157},{-65.124}) {\tikz\pgfuseplotmark{m6c};};
\node[stars] at (axis cs:{9.326},{-54.394}) {\tikz\pgfuseplotmark{m6c};};
\node[stars] at (axis cs:{9.966},{-44.796}) {\tikz\pgfuseplotmark{m6b};};
\node[stars] at (axis cs:{10.107},{-59.455}) {\tikz\pgfuseplotmark{m6b};};
\node[stars] at (axis cs:{10.331},{-46.085}) {\tikz\pgfuseplotmark{m5a};};
\node[stars] at (axis cs:{10.443},{-56.501}) {\tikz\pgfuseplotmark{m6av};};
\node[stars] at (axis cs:{10.618},{-65.468}) {\tikz\pgfuseplotmark{m5c};};
\node[stars] at (axis cs:{10.675},{-60.263}) {\tikz\pgfuseplotmark{m6b};};
\node[stars] at (axis cs:{10.679},{-38.463}) {\tikz\pgfuseplotmark{m6b};};
\node[stars] at (axis cs:{10.838},{-57.463}) {\tikz\pgfuseplotmark{m4c};};
\node[stars] at (axis cs:{11.050},{-38.422}) {\tikz\pgfuseplotmark{m6b};};
\node[stars] at (axis cs:{11.132},{-62.497}) {\tikz\pgfuseplotmark{m6bb};};
\node[stars] at (axis cs:{11.238},{-42.676}) {\tikz\pgfuseplotmark{m6b};};
\node[stars] at (axis cs:{11.250},{-53.715}) {\tikz\pgfuseplotmark{m6c};};
\node[stars] at (axis cs:{11.440},{-47.552}) {\tikz\pgfuseplotmark{m6a};};
\node[stars] at (axis cs:{12.148},{-74.923}) {\tikz\pgfuseplotmark{m5b};};
\node[stars] at (axis cs:{12.236},{-46.697}) {\tikz\pgfuseplotmark{m6c};};
\node[stars] at (axis cs:{12.516},{-43.395}) {\tikz\pgfuseplotmark{m6cv};};
\node[stars] at (axis cs:{12.672},{-50.987}) {\tikz\pgfuseplotmark{m5cv};};
\node[stars] at (axis cs:{13.408},{-62.871}) {\tikz\pgfuseplotmark{m6av};};
\node[stars] at (axis cs:{13.751},{-69.527}) {\tikz\pgfuseplotmark{m5c};};
\node[stars] at (axis cs:{13.981},{-27.776}) {\tikz\pgfuseplotmark{m6b};};
\node[stars] at (axis cs:{14.593},{-60.696}) {\tikz\pgfuseplotmark{m6c};};
\node[stars] at (axis cs:{14.652},{-29.357}) {\tikz\pgfuseplotmark{m4cv};};
\node[stars] at (axis cs:{15.326},{-38.916}) {\tikz\pgfuseplotmark{m6a};};
\node[stars] at (axis cs:{15.508},{-57.002}) {\tikz\pgfuseplotmark{m6b};};
\node[stars] at (axis cs:{15.610},{-31.552}) {\tikz\pgfuseplotmark{m6av};};
\node[stars] at (axis cs:{15.679},{-65.456}) {\tikz\pgfuseplotmark{m6cv};};
\node[stars] at (axis cs:{15.705},{-46.397}) {\tikz\pgfuseplotmark{m5c};};
\node[stars] at (axis cs:{15.824},{-29.526}) {\tikz\pgfuseplotmark{m6c};};
\node[stars] at (axis cs:{16.136},{-33.533}) {\tikz\pgfuseplotmark{m6cb};};
\node[stars] at (axis cs:{16.521},{-46.718}) {\tikz\pgfuseplotmark{m3cv};};
\node[stars] at (axis cs:{16.828},{-61.775}) {\tikz\pgfuseplotmark{m5cv};};
\node[stars] at (axis cs:{16.949},{-41.487}) {\tikz\pgfuseplotmark{m5c};};
\node[stars] at (axis cs:{17.096},{-55.246}) {\tikz\pgfuseplotmark{m4bv};};
\node[stars] at (axis cs:{17.531},{-57.694}) {\tikz\pgfuseplotmark{m6c};};
\node[stars] at (axis cs:{18.189},{-37.856}) {\tikz\pgfuseplotmark{m6bv};};
\node[stars] at (axis cs:{18.796},{-45.531}) {\tikz\pgfuseplotmark{m5b};};
\node[stars] at (axis cs:{18.942},{-68.876}) {\tikz\pgfuseplotmark{m4cvb};};
\node[stars] at (axis cs:{19.266},{-66.398}) {\tikz\pgfuseplotmark{m6cb};};
\node[stars] at (axis cs:{20.879},{-30.945}) {\tikz\pgfuseplotmark{m6a};};
\node[stars] at (axis cs:{21.170},{-41.492}) {\tikz\pgfuseplotmark{m5c};};
\node[stars] at (axis cs:{21.175},{-44.528}) {\tikz\pgfuseplotmark{m6c};};
\node[stars] at (axis cs:{21.272},{-64.369}) {\tikz\pgfuseplotmark{m6b};};
\node[stars] at (axis cs:{22.091},{-43.318}) {\tikz\pgfuseplotmark{m3cv};};
\node[stars] at (axis cs:{22.377},{-46.756}) {\tikz\pgfuseplotmark{m6cv};};
\node[stars] at (axis cs:{22.813},{-49.073}) {\tikz\pgfuseplotmark{m4b};};
\node[stars] at (axis cs:{22.913},{-45.576}) {\tikz\pgfuseplotmark{m6c};};
\node[stars] at (axis cs:{22.930},{-30.283}) {\tikz\pgfuseplotmark{m6a};};
\node[stars] at (axis cs:{23.152},{-49.728}) {\tikz\pgfuseplotmark{m6c};};
\node[stars] at (axis cs:{23.234},{-36.865}) {\tikz\pgfuseplotmark{m5c};};
\node[stars] at (axis cs:{23.413},{-78.505}) {\tikz\pgfuseplotmark{m6b};};
\node[stars] at (axis cs:{23.712},{-31.892}) {\tikz\pgfuseplotmark{m6bv};};
\node[stars] at (axis cs:{23.813},{-58.139}) {\tikz\pgfuseplotmark{m6b};};
\node[stars] at (axis cs:{23.961},{-39.947}) {\tikz\pgfuseplotmark{m6c};};
\node[stars] at (axis cs:{24.035},{-29.907}) {\tikz\pgfuseplotmark{m6a};};
\node[stars] at (axis cs:{24.187},{-58.271}) {\tikz\pgfuseplotmark{m6c};};
\node[stars] at (axis cs:{24.367},{-84.770}) {\tikz\pgfuseplotmark{m6a};};
\node[stars] at (axis cs:{24.429},{-57.237}) {\tikz\pgfuseplotmark{m1bv};};
\node[stars] at (axis cs:{24.482},{-82.975}) {\tikz\pgfuseplotmark{m6b};};
\node[stars] at (axis cs:{24.615},{-36.528}) {\tikz\pgfuseplotmark{m6b};};
\node[stars] at (axis cs:{24.949},{-56.193}) {\tikz\pgfuseplotmark{m6ab};};
\node[stars] at (axis cs:{25.339},{-79.148}) {\tikz\pgfuseplotmark{m6c};};
\node[stars] at (axis cs:{25.364},{-38.133}) {\tikz\pgfuseplotmark{m6c};};
\node[stars] at (axis cs:{25.450},{-60.789}) {\tikz\pgfuseplotmark{m6a};};
\node[stars] at (axis cs:{25.512},{-36.832}) {\tikz\pgfuseplotmark{m6a};};
\node[stars] at (axis cs:{25.536},{-32.327}) {\tikz\pgfuseplotmark{m5c};};
\node[stars] at (axis cs:{25.622},{-53.741}) {\tikz\pgfuseplotmark{m6a};};
\node[stars] at (axis cs:{26.504},{-27.349}) {\tikz\pgfuseplotmark{m6c};};
\node[stars] at (axis cs:{26.525},{-50.816}) {\tikz\pgfuseplotmark{m5c};};
\node[stars] at (axis cs:{26.526},{-53.522}) {\tikz\pgfuseplotmark{m5b};};
\node[stars] at (axis cs:{26.820},{-41.760}) {\tikz\pgfuseplotmark{m6c};};
\node[stars] at (axis cs:{26.944},{-80.176}) {\tikz\pgfuseplotmark{m6b};};
\node[stars] at (axis cs:{26.949},{-37.159}) {\tikz\pgfuseplotmark{m6c};};
\node[stars] at (axis cs:{27.331},{-31.072}) {\tikz\pgfuseplotmark{m6cv};};
\node[stars] at (axis cs:{27.454},{-38.403}) {\tikz\pgfuseplotmark{m6c};};
\node[stars] at (axis cs:{27.584},{-47.816}) {\tikz\pgfuseplotmark{m6c};};
\node[stars] at (axis cs:{27.727},{-50.206}) {\tikz\pgfuseplotmark{m6bv};};
\node[stars] at (axis cs:{27.860},{-39.836}) {\tikz\pgfuseplotmark{m6c};};
\node[stars] at (axis cs:{28.347},{-38.594}) {\tikz\pgfuseplotmark{m6b};};
\node[stars] at (axis cs:{28.411},{-46.303}) {\tikz\pgfuseplotmark{m4cv};};
\node[stars] at (axis cs:{28.592},{-42.497}) {\tikz\pgfuseplotmark{m5b};};
\node[stars] at (axis cs:{28.734},{-67.647}) {\tikz\pgfuseplotmark{m5a};};
\node[stars] at (axis cs:{28.944},{-60.861}) {\tikz\pgfuseplotmark{m6b};};
\node[stars] at (axis cs:{28.961},{-78.348}) {\tikz\pgfuseplotmark{m6c};};
\node[stars] at (axis cs:{28.989},{-51.609}) {\tikz\pgfuseplotmark{m4av};};
\node[stars] at (axis cs:{29.039},{-49.836}) {\tikz\pgfuseplotmark{m6c};};
\node[stars] at (axis cs:{29.250},{-51.766}) {\tikz\pgfuseplotmark{m6b};};
\node[stars] at (axis cs:{29.292},{-47.385}) {\tikz\pgfuseplotmark{m5a};};
\node[stars] at (axis cs:{29.472},{-65.425}) {\tikz\pgfuseplotmark{m6c};};
\node[stars] at (axis cs:{29.611},{-33.067}) {\tikz\pgfuseplotmark{m6c};};
\node[stars] at (axis cs:{29.692},{-61.570}) {\tikz\pgfuseplotmark{m3a};};
\node[stars] at (axis cs:{29.912},{-42.030}) {\tikz\pgfuseplotmark{m6a};};
\node[stars] at (axis cs:{29.921},{-66.066}) {\tikz\pgfuseplotmark{m6b};};
\node[stars] at (axis cs:{30.311},{-30.002}) {\tikz\pgfuseplotmark{m5c};};
\node[stars] at (axis cs:{30.427},{-44.713}) {\tikz\pgfuseplotmark{m5cv};};
\node[stars] at (axis cs:{30.618},{-29.665}) {\tikz\pgfuseplotmark{m6c};};
\node[stars] at (axis cs:{31.123},{-29.297}) {\tikz\pgfuseplotmark{m5av};};
\node[stars] at (axis cs:{31.146},{-54.881}) {\tikz\pgfuseplotmark{m6c};};
\node[stars] at (axis cs:{32.289},{-43.516}) {\tikz\pgfuseplotmark{m6a};};
\node[stars] at (axis cs:{32.520},{-43.815}) {\tikz\pgfuseplotmark{m6c};};
\node[stars] at (axis cs:{32.608},{-50.824}) {\tikz\pgfuseplotmark{m6b};};
\node[stars] at (axis cs:{33.227},{-30.724}) {\tikz\pgfuseplotmark{m5c};};
\node[stars] at (axis cs:{33.561},{-67.841}) {\tikz\pgfuseplotmark{m6av};};
\node[stars] at (axis cs:{33.633},{-41.167}) {\tikz\pgfuseplotmark{m6b};};
\node[stars] at (axis cs:{33.869},{-67.746}) {\tikz\pgfuseplotmark{m6a};};
\node[stars] at (axis cs:{34.127},{-51.512}) {\tikz\pgfuseplotmark{m4ab};};
\node[stars] at (axis cs:{34.853},{-41.848}) {\tikz\pgfuseplotmark{m6c};};
\node[stars] at (axis cs:{34.976},{-55.945}) {\tikz\pgfuseplotmark{m6a};};
\node[stars] at (axis cs:{35.437},{-68.659}) {\tikz\pgfuseplotmark{m4b};};
\node[stars] at (axis cs:{35.549},{-43.200}) {\tikz\pgfuseplotmark{m6c};};
\node[stars] at (axis cs:{35.718},{-73.646}) {\tikz\pgfuseplotmark{m6bv};};
\node[stars] at (axis cs:{35.728},{-51.092}) {\tikz\pgfuseplotmark{m6b};};
\node[stars] at (axis cs:{36.141},{-40.840}) {\tikz\pgfuseplotmark{m6c};};
\node[stars] at (axis cs:{36.225},{-60.312}) {\tikz\pgfuseplotmark{m5c};};
\node[stars] at (axis cs:{36.360},{-66.494}) {\tikz\pgfuseplotmark{m6cv};};
\node[stars] at (axis cs:{36.746},{-47.704}) {\tikz\pgfuseplotmark{m4c};};
\node[stars] at (axis cs:{37.007},{-33.811}) {\tikz\pgfuseplotmark{m5b};};
\node[stars] at (axis cs:{37.018},{-64.299}) {\tikz\pgfuseplotmark{m6c};};
\node[stars] at (axis cs:{37.148},{-31.102}) {\tikz\pgfuseplotmark{m6b};};
\node[stars] at (axis cs:{37.919},{-79.109}) {\tikz\pgfuseplotmark{m5c};};
\node[stars] at (axis cs:{38.062},{-36.427}) {\tikz\pgfuseplotmark{m6c};};
\node[stars] at (axis cs:{38.279},{-34.650}) {\tikz\pgfuseplotmark{m6b};};
\node[stars] at (axis cs:{38.461},{-28.232}) {\tikz\pgfuseplotmark{m5bb};};
\node[stars] at (axis cs:{38.478},{-51.093}) {\tikz\pgfuseplotmark{m6c};};
\node[stars] at (axis cs:{39.039},{-30.045}) {\tikz\pgfuseplotmark{m6a};};
\node[stars] at (axis cs:{39.244},{-34.578}) {\tikz\pgfuseplotmark{m6a};};
\node[stars] at (axis cs:{39.352},{-52.543}) {\tikz\pgfuseplotmark{m5c};};
\node[stars] at (axis cs:{39.578},{-30.194}) {\tikz\pgfuseplotmark{m6a};};
\node[stars] at (axis cs:{39.603},{-37.990}) {\tikz\pgfuseplotmark{m6c};};
\node[stars] at (axis cs:{39.897},{-68.267}) {\tikz\pgfuseplotmark{m4b};};
\node[stars] at (axis cs:{39.950},{-42.892}) {\tikz\pgfuseplotmark{m5a};};
\node[stars] at (axis cs:{40.165},{-54.550}) {\tikz\pgfuseplotmark{m5c};};
\node[stars] at (axis cs:{40.167},{-39.855}) {\tikz\pgfuseplotmark{m4b};};
\node[stars] at (axis cs:{40.528},{-38.384}) {\tikz\pgfuseplotmark{m6b};};
\node[stars] at (axis cs:{40.535},{-46.524}) {\tikz\pgfuseplotmark{m6b};};
\node[stars] at (axis cs:{40.639},{-50.800}) {\tikz\pgfuseplotmark{m5c};};
\node[stars] at (axis cs:{40.834},{-40.528}) {\tikz\pgfuseplotmark{m6cb};};
\node[stars] at (axis cs:{40.861},{-66.714}) {\tikz\pgfuseplotmark{m6c};};
\node[stars] at (axis cs:{41.044},{-52.571}) {\tikz\pgfuseplotmark{m6c};};
\node[stars] at (axis cs:{41.086},{-32.525}) {\tikz\pgfuseplotmark{m6c};};
\node[stars] at (axis cs:{41.364},{-63.704}) {\tikz\pgfuseplotmark{m6a};};
\node[stars] at (axis cs:{41.386},{-67.616}) {\tikz\pgfuseplotmark{m5a};};
\node[stars] at (axis cs:{42.256},{-62.806}) {\tikz\pgfuseplotmark{m5c};};
\node[stars] at (axis cs:{42.273},{-32.406}) {\tikz\pgfuseplotmark{m4c};};
\node[stars] at (axis cs:{42.476},{-27.942}) {\tikz\pgfuseplotmark{m5c};};
\node[stars] at (axis cs:{42.562},{-35.843}) {\tikz\pgfuseplotmark{m6bb};};
\node[stars] at (axis cs:{42.619},{-75.067}) {\tikz\pgfuseplotmark{m5a};};
\node[stars] at (axis cs:{42.668},{-35.676}) {\tikz\pgfuseplotmark{m5c};};
\node[stars] at (axis cs:{42.699},{-39.932}) {\tikz\pgfuseplotmark{m6c};};
\node[stars] at (axis cs:{42.984},{-30.814}) {\tikz\pgfuseplotmark{m6c};};
\node[stars] at (axis cs:{43.079},{-62.909}) {\tikz\pgfuseplotmark{m6b};};
\node[stars] at (axis cs:{43.228},{-65.455}) {\tikz\pgfuseplotmark{m6c};};
\node[stars] at (axis cs:{43.393},{-38.437}) {\tikz\pgfuseplotmark{m6b};};
\node[stars] at (axis cs:{43.527},{-50.871}) {\tikz\pgfuseplotmark{m6c};};
\node[stars] at (axis cs:{44.304},{-29.855}) {\tikz\pgfuseplotmark{m6c};};
\node[stars] at (axis cs:{44.386},{-38.191}) {\tikz\pgfuseplotmark{m6c};};
\node[stars] at (axis cs:{44.565},{-40.305}) {\tikz\pgfuseplotmark{m3cb};};
\node[stars] at (axis cs:{44.699},{-64.071}) {\tikz\pgfuseplotmark{m5b};};
\node[stars] at (axis cs:{44.777},{-28.907}) {\tikz\pgfuseplotmark{m6c};};
\node[stars] at (axis cs:{44.910},{-32.507}) {\tikz\pgfuseplotmark{m6c};};
\node[stars] at (axis cs:{45.407},{-28.091}) {\tikz\pgfuseplotmark{m6b};};
\node[stars] at (axis cs:{45.564},{-71.902}) {\tikz\pgfuseplotmark{m6a};};
\node[stars] at (axis cs:{45.733},{-46.975}) {\tikz\pgfuseplotmark{m6a};};
\node[stars] at (axis cs:{45.903},{-59.738}) {\tikz\pgfuseplotmark{m5b};};
\node[stars] at (axis cs:{46.884},{-78.989}) {\tikz\pgfuseplotmark{m6avb};};
\node[stars] at (axis cs:{46.954},{-69.265}) {\tikz\pgfuseplotmark{m6b};};
\node[stars] at (axis cs:{46.962},{-27.831}) {\tikz\pgfuseplotmark{m6c};};
\node[stars] at (axis cs:{47.614},{-48.734}) {\tikz\pgfuseplotmark{m6b};};
\node[stars] at (axis cs:{48.019},{-28.988}) {\tikz\pgfuseplotmark{m4ab};};
\node[stars] at (axis cs:{48.107},{-44.420}) {\tikz\pgfuseplotmark{m6bb};};
\node[stars] at (axis cs:{48.138},{-57.321}) {\tikz\pgfuseplotmark{m6av};};
\node[stars] at (axis cs:{48.256},{-35.944}) {\tikz\pgfuseplotmark{m6cv};};
\node[stars] at (axis cs:{48.408},{-29.804}) {\tikz\pgfuseplotmark{m6c};};
\node[stars] at (axis cs:{48.990},{-77.388}) {\tikz\pgfuseplotmark{m6a};};
\node[stars] at (axis cs:{49.361},{-47.752}) {\tikz\pgfuseplotmark{m6a};};
\node[stars] at (axis cs:{49.496},{-66.927}) {\tikz\pgfuseplotmark{m6b};};
\node[stars] at (axis cs:{49.511},{-28.797}) {\tikz\pgfuseplotmark{m6b};};
\node[stars] at (axis cs:{49.553},{-62.506}) {\tikz\pgfuseplotmark{m5cb};};
\node[stars] at (axis cs:{49.982},{-43.070}) {\tikz\pgfuseplotmark{m4c};};
\node[stars] at (axis cs:{50.389},{-47.776}) {\tikz\pgfuseplotmark{m6c};};
\node[stars] at (axis cs:{51.010},{-69.624}) {\tikz\pgfuseplotmark{m6c};};
\node[stars] at (axis cs:{51.401},{-69.336}) {\tikz\pgfuseplotmark{m6b};};
\node[stars] at (axis cs:{51.483},{-35.921}) {\tikz\pgfuseplotmark{m6c};};
\node[stars] at (axis cs:{51.549},{-41.637}) {\tikz\pgfuseplotmark{m6c};};
\node[stars] at (axis cs:{51.594},{-27.317}) {\tikz\pgfuseplotmark{m6b};};
\node[stars] at (axis cs:{51.889},{-35.681}) {\tikz\pgfuseplotmark{m6a};};
\node[stars] at (axis cs:{52.344},{-62.937}) {\tikz\pgfuseplotmark{m5a};};
\node[stars] at (axis cs:{52.480},{-42.634}) {\tikz\pgfuseplotmark{m6a};};
\node[stars] at (axis cs:{52.495},{-78.352}) {\tikz\pgfuseplotmark{m6a};};
\node[stars] at (axis cs:{52.557},{-41.370}) {\tikz\pgfuseplotmark{m6bv};};
\node[stars] at (axis cs:{52.654},{-47.375}) {\tikz\pgfuseplotmark{m6bv};};
\node[stars] at (axis cs:{52.715},{-66.490}) {\tikz\pgfuseplotmark{m6a};};
\node[stars] at (axis cs:{53.145},{-50.378}) {\tikz\pgfuseplotmark{m6a};};
\node[stars] at (axis cs:{53.215},{-61.017}) {\tikz\pgfuseplotmark{m6c};};
\node[stars] at (axis cs:{53.487},{-31.080}) {\tikz\pgfuseplotmark{m6c};};
\node[stars] at (axis cs:{53.640},{-31.875}) {\tikz\pgfuseplotmark{m6cb};};
\node[stars] at (axis cs:{54.126},{-78.323}) {\tikz\pgfuseplotmark{m6c};};
\node[stars] at (axis cs:{54.274},{-40.274}) {\tikz\pgfuseplotmark{m5a};};
\node[stars] at (axis cs:{54.699},{-27.943}) {\tikz\pgfuseplotmark{m6b};};
\node[stars] at (axis cs:{55.562},{-31.938}) {\tikz\pgfuseplotmark{m5b};};
\node[stars] at (axis cs:{55.637},{-85.262}) {\tikz\pgfuseplotmark{m6cb};};
\node[stars] at (axis cs:{55.709},{-37.313}) {\tikz\pgfuseplotmark{m5a};};
\node[stars] at (axis cs:{56.050},{-64.807}) {\tikz\pgfuseplotmark{m4a};};
\node[stars] at (axis cs:{56.141},{-54.274}) {\tikz\pgfuseplotmark{m6cb};};
\node[stars] at (axis cs:{56.210},{-48.061}) {\tikz\pgfuseplotmark{m6c};};
\node[stars] at (axis cs:{56.316},{-47.359}) {\tikz\pgfuseplotmark{m6a};};
\node[stars] at (axis cs:{56.349},{-71.658}) {\tikz\pgfuseplotmark{m6cv};};
\node[stars] at (axis cs:{56.614},{-29.338}) {\tikz\pgfuseplotmark{m6b};};
\node[stars] at (axis cs:{56.810},{-74.239}) {\tikz\pgfuseplotmark{m3cv};};
\node[stars] at (axis cs:{56.957},{-36.106}) {\tikz\pgfuseplotmark{m6c};};
\node[stars] at (axis cs:{56.984},{-30.168}) {\tikz\pgfuseplotmark{m6av};};
\node[stars] at (axis cs:{57.149},{-37.620}) {\tikz\pgfuseplotmark{m5ab};};
\node[stars] at (axis cs:{57.364},{-36.200}) {\tikz\pgfuseplotmark{m4c};};
\node[stars] at (axis cs:{58.389},{-46.894}) {\tikz\pgfuseplotmark{m6b};};
\node[stars] at (axis cs:{58.412},{-34.732}) {\tikz\pgfuseplotmark{m5b};};
\node[stars] at (axis cs:{58.597},{-40.357}) {\tikz\pgfuseplotmark{m6a};};
\node[stars] at (axis cs:{58.642},{-52.690}) {\tikz\pgfuseplotmark{m6cv};};
\node[stars] at (axis cs:{59.017},{-63.463}) {\tikz\pgfuseplotmark{m6c};};
\node[stars] at (axis cs:{59.679},{-57.102}) {\tikz\pgfuseplotmark{m6b};};
\node[stars] at (axis cs:{59.686},{-61.400}) {\tikz\pgfuseplotmark{m5a};};
\node[stars] at (axis cs:{60.169},{-30.491}) {\tikz\pgfuseplotmark{m6b};};
\node[stars] at (axis cs:{60.224},{-62.159}) {\tikz\pgfuseplotmark{m5av};};
\node[stars] at (axis cs:{60.326},{-61.079}) {\tikz\pgfuseplotmark{m5b};};
\node[stars] at (axis cs:{61.406},{-27.652}) {\tikz\pgfuseplotmark{m6a};};
\node[stars] at (axis cs:{61.840},{-64.222}) {\tikz\pgfuseplotmark{m6c};};
\node[stars] at (axis cs:{62.691},{-35.274}) {\tikz\pgfuseplotmark{m6c};};
\node[stars] at (axis cs:{62.711},{-41.993}) {\tikz\pgfuseplotmark{m5b};};
\node[stars] at (axis cs:{63.399},{-40.358}) {\tikz\pgfuseplotmark{m6c};};
\node[stars] at (axis cs:{63.500},{-42.294}) {\tikz\pgfuseplotmark{m4a};};
\node[stars] at (axis cs:{63.606},{-62.474}) {\tikz\pgfuseplotmark{m3c};};
\node[stars] at (axis cs:{63.702},{-62.192}) {\tikz\pgfuseplotmark{m5c};};
\node[stars] at (axis cs:{64.007},{-51.487}) {\tikz\pgfuseplotmark{m4cv};};
\node[stars] at (axis cs:{64.087},{-60.948}) {\tikz\pgfuseplotmark{m6cv};};
\node[stars] at (axis cs:{64.121},{-59.302}) {\tikz\pgfuseplotmark{m4c};};
\node[stars] at (axis cs:{64.418},{-63.255}) {\tikz\pgfuseplotmark{m6bvb};};
\node[stars] at (axis cs:{64.474},{-33.798}) {\tikz\pgfuseplotmark{m4a};};
\node[stars] at (axis cs:{64.497},{-80.214}) {\tikz\pgfuseplotmark{m6av};};
\node[stars] at (axis cs:{64.667},{-52.860}) {\tikz\pgfuseplotmark{m6bb};};
\node[stars] at (axis cs:{64.820},{-44.268}) {\tikz\pgfuseplotmark{m5cb};};
\node[stars] at (axis cs:{65.242},{-81.580}) {\tikz\pgfuseplotmark{m6a};};
\node[stars] at (axis cs:{65.472},{-63.386}) {\tikz\pgfuseplotmark{m5c};};
\node[stars] at (axis cs:{65.782},{-35.545}) {\tikz\pgfuseplotmark{m6c};};
\node[stars] at (axis cs:{66.009},{-34.017}) {\tikz\pgfuseplotmark{m4b};};
\node[stars] at (axis cs:{66.272},{-61.238}) {\tikz\pgfuseplotmark{m6bv};};
\node[stars] at (axis cs:{66.330},{-44.161}) {\tikz\pgfuseplotmark{m6cv};};
\node[stars] at (axis cs:{66.775},{-46.947}) {\tikz\pgfuseplotmark{m6b};};
\node[stars] at (axis cs:{66.942},{-62.521}) {\tikz\pgfuseplotmark{m6a};};
\node[stars] at (axis cs:{67.039},{-41.960}) {\tikz\pgfuseplotmark{m6cv};};
\node[stars] at (axis cs:{67.334},{-46.515}) {\tikz\pgfuseplotmark{m6b};};
\node[stars] at (axis cs:{67.668},{-35.653}) {\tikz\pgfuseplotmark{m6b};};
\node[stars] at (axis cs:{67.709},{-44.954}) {\tikz\pgfuseplotmark{m5bv};};
\node[stars] at (axis cs:{68.377},{-29.766}) {\tikz\pgfuseplotmark{m5a};};
\node[stars] at (axis cs:{68.391},{-62.824}) {\tikz\pgfuseplotmark{m6ab};};
\node[stars] at (axis cs:{68.499},{-55.045}) {\tikz\pgfuseplotmark{m3cv};};
\node[stars] at (axis cs:{68.888},{-30.562}) {\tikz\pgfuseplotmark{m4a};};
\node[stars] at (axis cs:{69.190},{-62.077}) {\tikz\pgfuseplotmark{m5cv};};
\node[stars] at (axis cs:{69.212},{-30.717}) {\tikz\pgfuseplotmark{m6c};};
\node[stars] at (axis cs:{69.330},{-41.873}) {\tikz\pgfuseplotmark{m6c};};
\node[stars] at (axis cs:{69.591},{-77.656}) {\tikz\pgfuseplotmark{m6b};};
\node[stars] at (axis cs:{69.768},{-51.673}) {\tikz\pgfuseplotmark{m6c};};
\node[stars] at (axis cs:{70.140},{-41.864}) {\tikz\pgfuseplotmark{m4cv};};
\node[stars] at (axis cs:{70.515},{-37.144}) {\tikz\pgfuseplotmark{m5b};};
\node[stars] at (axis cs:{70.693},{-50.481}) {\tikz\pgfuseplotmark{m5c};};
\node[stars] at (axis cs:{70.767},{-70.931}) {\tikz\pgfuseplotmark{m6a};};
\node[stars] at (axis cs:{70.789},{-30.765}) {\tikz\pgfuseplotmark{m6a};};
\node[stars] at (axis cs:{70.934},{-41.065}) {\tikz\pgfuseplotmark{m6c};};
\node[stars] at (axis cs:{71.088},{-59.732}) {\tikz\pgfuseplotmark{m5cv};};
\node[stars] at (axis cs:{71.241},{-63.230}) {\tikz\pgfuseplotmark{m6cv};};
\node[stars] at (axis cs:{71.481},{-39.357}) {\tikz\pgfuseplotmark{m6b};};
\node[stars] at (axis cs:{71.607},{-28.087}) {\tikz\pgfuseplotmark{m6cv};};
\node[stars] at (axis cs:{71.957},{-30.020}) {\tikz\pgfuseplotmark{m6c};};
\node[stars] at (axis cs:{72.567},{-41.321}) {\tikz\pgfuseplotmark{m6b};};
\node[stars] at (axis cs:{72.730},{-53.461}) {\tikz\pgfuseplotmark{m6ab};};
\node[stars] at (axis cs:{72.868},{-34.906}) {\tikz\pgfuseplotmark{m6a};};
\node[stars] at (axis cs:{73.274},{-72.408}) {\tikz\pgfuseplotmark{m6c};};
\node[stars] at (axis cs:{73.377},{-66.675}) {\tikz\pgfuseplotmark{m6cv};};
\node[stars] at (axis cs:{73.721},{-58.547}) {\tikz\pgfuseplotmark{m6b};};
\node[stars] at (axis cs:{73.728},{-39.628}) {\tikz\pgfuseplotmark{m6b};};
\node[stars] at (axis cs:{73.797},{-74.937}) {\tikz\pgfuseplotmark{m5cv};};
\node[stars] at (axis cs:{74.712},{-82.470}) {\tikz\pgfuseplotmark{m6a};};
\node[stars] at (axis cs:{75.055},{-78.300}) {\tikz\pgfuseplotmark{m6c};};
\node[stars] at (axis cs:{75.394},{-39.718}) {\tikz\pgfuseplotmark{m6b};};
\node[stars] at (axis cs:{75.595},{-31.771}) {\tikz\pgfuseplotmark{m6b};};
\node[stars] at (axis cs:{75.679},{-71.314}) {\tikz\pgfuseplotmark{m5c};};
\node[stars] at (axis cs:{75.703},{-49.151}) {\tikz\pgfuseplotmark{m5c};};
\node[stars] at (axis cs:{75.975},{-41.745}) {\tikz\pgfuseplotmark{m6c};};
\node[stars] at (axis cs:{76.102},{-35.483}) {\tikz\pgfuseplotmark{m5ab};};
\node[stars] at (axis cs:{76.109},{-35.705}) {\tikz\pgfuseplotmark{m6cv};};
\node[stars] at (axis cs:{76.242},{-49.578}) {\tikz\pgfuseplotmark{m5bv};};
\node[stars] at (axis cs:{76.253},{-54.407}) {\tikz\pgfuseplotmark{m6c};};
\node[stars] at (axis cs:{76.378},{-57.473}) {\tikz\pgfuseplotmark{m5a};};
\node[stars] at (axis cs:{76.539},{-73.038}) {\tikz\pgfuseplotmark{m6c};};
\node[stars] at (axis cs:{76.892},{-63.399}) {\tikz\pgfuseplotmark{m5cv};};
\node[stars] at (axis cs:{78.439},{-67.185}) {\tikz\pgfuseplotmark{m5a};};
\node[stars] at (axis cs:{78.472},{-52.031}) {\tikz\pgfuseplotmark{m6b};};
\node[stars] at (axis cs:{78.620},{-35.977}) {\tikz\pgfuseplotmark{m6a};};
\node[stars] at (axis cs:{78.912},{-52.182}) {\tikz\pgfuseplotmark{m6c};};
\node[stars] at (axis cs:{79.371},{-34.895}) {\tikz\pgfuseplotmark{m5a};};
\node[stars] at (axis cs:{79.842},{-50.606}) {\tikz\pgfuseplotmark{m5c};};
\node[stars] at (axis cs:{79.849},{-27.369}) {\tikz\pgfuseplotmark{m6b};};
\node[stars] at (axis cs:{80.086},{-34.699}) {\tikz\pgfuseplotmark{m6c};};
\node[stars] at (axis cs:{80.320},{-34.345}) {\tikz\pgfuseplotmark{m6bv};};
\node[stars] at (axis cs:{80.592},{-56.134}) {\tikz\pgfuseplotmark{m6b};};
\node[stars] at (axis cs:{80.850},{-39.678}) {\tikz\pgfuseplotmark{m6av};};
\node[stars] at (axis cs:{81.193},{-52.316}) {\tikz\pgfuseplotmark{m6cb};};
\node[stars] at (axis cs:{81.232},{-44.226}) {\tikz\pgfuseplotmark{m6b};};
\node[stars] at (axis cs:{81.580},{-58.912}) {\tikz\pgfuseplotmark{m5b};};
\node[stars] at (axis cs:{81.749},{-68.622}) {\tikz\pgfuseplotmark{m6bb};};
\node[stars] at (axis cs:{81.772},{-40.943}) {\tikz\pgfuseplotmark{m6a};};
\node[stars] at (axis cs:{82.064},{-37.231}) {\tikz\pgfuseplotmark{m6a};};
\node[stars] at (axis cs:{82.539},{-47.077}) {\tikz\pgfuseplotmark{m5cb};};
\node[stars] at (axis cs:{82.558},{-84.785}) {\tikz\pgfuseplotmark{m6cv};};
\node[stars] at (axis cs:{82.566},{-63.928}) {\tikz\pgfuseplotmark{m6c};};
\node[stars] at (axis cs:{82.803},{-35.470}) {\tikz\pgfuseplotmark{m4b};};
\node[stars] at (axis cs:{82.900},{-45.925}) {\tikz\pgfuseplotmark{m6a};};
\node[stars] at (axis cs:{82.971},{-76.341}) {\tikz\pgfuseplotmark{m5c};};
\node[stars] at (axis cs:{83.214},{-38.513}) {\tikz\pgfuseplotmark{m5c};};
\node[stars] at (axis cs:{83.248},{-64.227}) {\tikz\pgfuseplotmark{m5c};};
\node[stars] at (axis cs:{83.281},{-35.139}) {\tikz\pgfuseplotmark{m6a};};
\node[stars] at (axis cs:{83.406},{-62.490}) {\tikz\pgfuseplotmark{m4av};};
\node[stars] at (axis cs:{83.435},{-54.902}) {\tikz\pgfuseplotmark{m6cb};};
\node[stars] at (axis cs:{83.687},{-73.741}) {\tikz\pgfuseplotmark{m6av};};
\node[stars] at (axis cs:{83.739},{-61.176}) {\tikz\pgfuseplotmark{m6c};};
\node[stars] at (axis cs:{83.814},{-33.080}) {\tikz\pgfuseplotmark{m6a};};
\node[stars] at (axis cs:{83.901},{-78.821}) {\tikz\pgfuseplotmark{m6bv};};
\node[stars] at (axis cs:{84.012},{-47.314}) {\tikz\pgfuseplotmark{m6bv};};
\node[stars] at (axis cs:{84.043},{-28.708}) {\tikz\pgfuseplotmark{m6c};};
\node[stars] at (axis cs:{84.229},{-66.560}) {\tikz\pgfuseplotmark{m6cv};};
\node[stars] at (axis cs:{84.291},{-80.469}) {\tikz\pgfuseplotmark{m6a};};
\node[stars] at (axis cs:{84.319},{-27.871}) {\tikz\pgfuseplotmark{m6c};};
\node[stars] at (axis cs:{84.436},{-28.690}) {\tikz\pgfuseplotmark{m5c};};
\node[stars] at (axis cs:{84.573},{-50.641}) {\tikz\pgfuseplotmark{m6c};};
\node[stars] at (axis cs:{84.681},{-40.707}) {\tikz\pgfuseplotmark{m6a};};
\node[stars] at (axis cs:{84.912},{-34.074}) {\tikz\pgfuseplotmark{m3av};};
\node[stars] at (axis cs:{84.958},{-32.629}) {\tikz\pgfuseplotmark{m5c};};
\node[stars] at (axis cs:{85.362},{-33.401}) {\tikz\pgfuseplotmark{m6c};};
\node[stars] at (axis cs:{85.548},{-30.535}) {\tikz\pgfuseplotmark{m6c};};
\node[stars] at (axis cs:{85.563},{-34.668}) {\tikz\pgfuseplotmark{m5cv};};
\node[stars] at (axis cs:{85.876},{-39.407}) {\tikz\pgfuseplotmark{m6c};};
\node[stars] at (axis cs:{85.921},{-45.833}) {\tikz\pgfuseplotmark{m6cv};};
\node[stars] at (axis cs:{86.193},{-65.735}) {\tikz\pgfuseplotmark{m4c};};
\node[stars] at (axis cs:{86.500},{-32.306}) {\tikz\pgfuseplotmark{m5c};};
\node[stars] at (axis cs:{86.614},{-46.597}) {\tikz\pgfuseplotmark{m5c};};
\node[stars] at (axis cs:{86.769},{-28.639}) {\tikz\pgfuseplotmark{m6c};};
\node[stars] at (axis cs:{86.806},{-54.361}) {\tikz\pgfuseplotmark{m6c};};
\node[stars] at (axis cs:{86.821},{-51.066}) {\tikz\pgfuseplotmark{m4av};};
\node[stars] at (axis cs:{86.828},{-35.674}) {\tikz\pgfuseplotmark{m6c};};
\node[stars] at (axis cs:{87.392},{-44.875}) {\tikz\pgfuseplotmark{m6c};};
\node[stars] at (axis cs:{87.457},{-56.166}) {\tikz\pgfuseplotmark{m4c};};
\node[stars] at (axis cs:{87.473},{-66.901}) {\tikz\pgfuseplotmark{m5bv};};
\node[stars] at (axis cs:{87.570},{-79.361}) {\tikz\pgfuseplotmark{m5c};};
\node[stars] at (axis cs:{87.620},{-52.768}) {\tikz\pgfuseplotmark{m6c};};
\node[stars] at (axis cs:{87.722},{-52.109}) {\tikz\pgfuseplotmark{m5c};};
\node[stars] at (axis cs:{87.740},{-35.768}) {\tikz\pgfuseplotmark{m3b};};
\node[stars] at (axis cs:{87.846},{-64.034}) {\tikz\pgfuseplotmark{m6c};};
\node[stars] at (axis cs:{87.998},{-29.449}) {\tikz\pgfuseplotmark{m6c};};
\node[stars] at (axis cs:{88.084},{-57.156}) {\tikz\pgfuseplotmark{m6b};};
\node[stars] at (axis cs:{88.138},{-37.631}) {\tikz\pgfuseplotmark{m6a};};
\node[stars] at (axis cs:{88.279},{-33.801}) {\tikz\pgfuseplotmark{m5bv};};
\node[stars] at (axis cs:{88.525},{-63.090}) {\tikz\pgfuseplotmark{m5a};};
\node[stars] at (axis cs:{88.558},{-29.147}) {\tikz\pgfuseplotmark{m6c};};
\node[stars] at (axis cs:{88.671},{-49.627}) {\tikz\pgfuseplotmark{m6b};};
\node[stars] at (axis cs:{88.709},{-52.635}) {\tikz\pgfuseplotmark{m5c};};
\node[stars] at (axis cs:{88.719},{-39.958}) {\tikz\pgfuseplotmark{m6a};};
\node[stars] at (axis cs:{88.875},{-37.121}) {\tikz\pgfuseplotmark{m5b};};
\node[stars] at (axis cs:{89.087},{-31.382}) {\tikz\pgfuseplotmark{m6a};};
\node[stars] at (axis cs:{89.204},{-31.976}) {\tikz\pgfuseplotmark{m6c};};
\node[stars] at (axis cs:{89.310},{-53.426}) {\tikz\pgfuseplotmark{m6cb};};
\node[stars] at (axis cs:{89.384},{-35.283}) {\tikz\pgfuseplotmark{m4cv};};
\node[stars] at (axis cs:{89.656},{-44.035}) {\tikz\pgfuseplotmark{m6a};};
\node[stars] at (axis cs:{89.787},{-42.815}) {\tikz\pgfuseplotmark{m4b};};
\node[stars] at (axis cs:{90.205},{-51.216}) {\tikz\pgfuseplotmark{m6a};};
\node[stars] at (axis cs:{90.318},{-33.912}) {\tikz\pgfuseplotmark{m6a};};
\node[stars] at (axis cs:{90.539},{-60.097}) {\tikz\pgfuseplotmark{m6cv};};
\node[stars] at (axis cs:{91.084},{-32.172}) {\tikz\pgfuseplotmark{m6av};};
\node[stars] at (axis cs:{91.167},{-45.079}) {\tikz\pgfuseplotmark{m6bb};};
\node[stars] at (axis cs:{91.363},{-35.513}) {\tikz\pgfuseplotmark{m6a};};
\node[stars] at (axis cs:{91.523},{-29.758}) {\tikz\pgfuseplotmark{m6a};};
\node[stars] at (axis cs:{91.539},{-66.040}) {\tikz\pgfuseplotmark{m6a};};
\node[stars] at (axis cs:{91.671},{-42.299}) {\tikz\pgfuseplotmark{m6c};};
\node[stars] at (axis cs:{91.758},{-45.091}) {\tikz\pgfuseplotmark{m6c};};
\node[stars] at (axis cs:{91.764},{-62.155}) {\tikz\pgfuseplotmark{m5bv};};
\node[stars] at (axis cs:{91.765},{-34.312}) {\tikz\pgfuseplotmark{m6av};};
\node[stars] at (axis cs:{91.882},{-37.253}) {\tikz\pgfuseplotmark{m5b};};
\node[stars] at (axis cs:{91.970},{-42.154}) {\tikz\pgfuseplotmark{m5c};};
\node[stars] at (axis cs:{92.144},{-44.356}) {\tikz\pgfuseplotmark{m6c};};
\node[stars] at (axis cs:{92.184},{-68.843}) {\tikz\pgfuseplotmark{m5b};};
\node[stars] at (axis cs:{92.348},{-49.563}) {\tikz\pgfuseplotmark{m6c};};
\node[stars] at (axis cs:{92.543},{-40.354}) {\tikz\pgfuseplotmark{m6av};};
\node[stars] at (axis cs:{92.560},{-74.753}) {\tikz\pgfuseplotmark{m5b};};
\node[stars] at (axis cs:{92.575},{-54.969}) {\tikz\pgfuseplotmark{m5av};};
\node[stars] at (axis cs:{92.645},{-27.154}) {\tikz\pgfuseplotmark{m6a};};
\node[stars] at (axis cs:{92.666},{-45.282}) {\tikz\pgfuseplotmark{m6c};};
\node[stars] at (axis cs:{92.812},{-65.589}) {\tikz\pgfuseplotmark{m5bv};};
\node[stars] at (axis cs:{94.078},{-59.213}) {\tikz\pgfuseplotmark{m6cb};};
\node[stars] at (axis cs:{94.138},{-35.140}) {\tikz\pgfuseplotmark{m4cv};};
\node[stars] at (axis cs:{94.148},{-39.264}) {\tikz\pgfuseplotmark{m6b};};
\node[stars] at (axis cs:{94.255},{-37.737}) {\tikz\pgfuseplotmark{m6a};};
\node[stars] at (axis cs:{94.290},{-37.253}) {\tikz\pgfuseplotmark{m6bv};};
\node[stars] at (axis cs:{94.466},{-52.733}) {\tikz\pgfuseplotmark{m6cv};};
\node[stars] at (axis cs:{94.921},{-34.396}) {\tikz\pgfuseplotmark{m6a};};
\node[stars] at (axis cs:{95.078},{-30.063}) {\tikz\pgfuseplotmark{m3bv};};
\node[stars] at (axis cs:{95.151},{-34.144}) {\tikz\pgfuseplotmark{m6av};};
\node[stars] at (axis cs:{95.528},{-33.436}) {\tikz\pgfuseplotmark{m4a};};
\node[stars] at (axis cs:{95.659},{-69.984}) {\tikz\pgfuseplotmark{m6a};};
\node[stars] at (axis cs:{95.733},{-56.370}) {\tikz\pgfuseplotmark{m6a};};
\node[stars] at (axis cs:{95.756},{-63.683}) {\tikz\pgfuseplotmark{m6c};};
\node[stars] at (axis cs:{95.810},{-31.790}) {\tikz\pgfuseplotmark{m6c};};
\node[stars] at (axis cs:{95.907},{-52.181}) {\tikz\pgfuseplotmark{m6b};};
\node[stars] at (axis cs:{95.946},{-58.544}) {\tikz\pgfuseplotmark{m6c};};
\node[stars] at (axis cs:{95.988},{-52.695}) {\tikz\pgfuseplotmark{m1b};};
\node[stars] at (axis cs:{96.004},{-36.708}) {\tikz\pgfuseplotmark{m6ab};};
\node[stars] at (axis cs:{96.058},{-60.281}) {\tikz\pgfuseplotmark{m6a};};
\node[stars] at (axis cs:{96.111},{-63.429}) {\tikz\pgfuseplotmark{m6c};};
\node[stars] at (axis cs:{96.183},{-28.780}) {\tikz\pgfuseplotmark{m6c};};
\node[stars] at (axis cs:{96.185},{-40.284}) {\tikz\pgfuseplotmark{m6c};};
\node[stars] at (axis cs:{96.232},{-63.828}) {\tikz\pgfuseplotmark{m6c};};
\node[stars] at (axis cs:{96.369},{-69.690}) {\tikz\pgfuseplotmark{m5c};};
\node[stars] at (axis cs:{96.375},{-35.064}) {\tikz\pgfuseplotmark{m6cb};};
\node[stars] at (axis cs:{96.432},{-48.177}) {\tikz\pgfuseplotmark{m6a};};
\node[stars] at (axis cs:{96.767},{-58.002}) {\tikz\pgfuseplotmark{m6a};};
\node[stars] at (axis cs:{96.782},{-37.895}) {\tikz\pgfuseplotmark{m6c};};
\node[stars] at (axis cs:{97.043},{-32.580}) {\tikz\pgfuseplotmark{m4c};};
\node[stars] at (axis cs:{97.164},{-32.371}) {\tikz\pgfuseplotmark{m6avb};};
\node[stars] at (axis cs:{97.177},{-41.075}) {\tikz\pgfuseplotmark{m6c};};
\node[stars] at (axis cs:{97.369},{-56.853}) {\tikz\pgfuseplotmark{m5cv};};
\node[stars] at (axis cs:{97.454},{-50.239}) {\tikz\pgfuseplotmark{m5cvb};};
\node[stars] at (axis cs:{97.513},{-65.568}) {\tikz\pgfuseplotmark{m6c};};
\node[stars] at (axis cs:{97.693},{-27.770}) {\tikz\pgfuseplotmark{m6b};};
\node[stars] at (axis cs:{97.750},{-40.916}) {\tikz\pgfuseplotmark{m6c};};
\node[stars] at (axis cs:{97.794},{-61.879}) {\tikz\pgfuseplotmark{m6cv};};
\node[stars] at (axis cs:{97.805},{-35.259}) {\tikz\pgfuseplotmark{m6a};};
\node[stars] at (axis cs:{97.826},{-51.826}) {\tikz\pgfuseplotmark{m6a};};
\node[stars] at (axis cs:{97.850},{-32.869}) {\tikz\pgfuseplotmark{m6cv};};
\node[stars] at (axis cs:{97.896},{-36.940}) {\tikz\pgfuseplotmark{m6cv};};
\node[stars] at (axis cs:{97.993},{-58.754}) {\tikz\pgfuseplotmark{m6avb};};
\node[stars] at (axis cs:{98.089},{-37.696}) {\tikz\pgfuseplotmark{m5c};};
\node[stars] at (axis cs:{98.162},{-32.030}) {\tikz\pgfuseplotmark{m6ab};};
\node[stars] at (axis cs:{98.293},{-38.625}) {\tikz\pgfuseplotmark{m6c};};
\node[stars] at (axis cs:{98.360},{-52.329}) {\tikz\pgfuseplotmark{m6c};};
\node[stars] at (axis cs:{98.456},{-36.232}) {\tikz\pgfuseplotmark{m5c};};
\node[stars] at (axis cs:{98.647},{-32.716}) {\tikz\pgfuseplotmark{m6a};};
\node[stars] at (axis cs:{98.744},{-52.975}) {\tikz\pgfuseplotmark{m4c};};
\node[stars] at (axis cs:{98.851},{-36.780}) {\tikz\pgfuseplotmark{m6ab};};
\node[stars] at (axis cs:{98.975},{-36.089}) {\tikz\pgfuseplotmark{m6cb};};
\node[stars] at (axis cs:{99.214},{-41.557}) {\tikz\pgfuseplotmark{m6c};};
\node[stars] at (axis cs:{99.258},{-38.146}) {\tikz\pgfuseplotmark{m6b};};
\node[stars] at (axis cs:{99.308},{-36.991}) {\tikz\pgfuseplotmark{m6a};};
\node[stars] at (axis cs:{99.440},{-43.196}) {\tikz\pgfuseplotmark{m3cv};};
\node[stars] at (axis cs:{99.448},{-32.340}) {\tikz\pgfuseplotmark{m5c};};
\node[stars] at (axis cs:{99.502},{-61.533}) {\tikz\pgfuseplotmark{m6cb};};
\node[stars] at (axis cs:{99.657},{-48.220}) {\tikz\pgfuseplotmark{m5bb};};
\node[stars] at (axis cs:{99.928},{-30.470}) {\tikz\pgfuseplotmark{m6a};};
\node[stars] at (axis cs:{100.012},{-80.813}) {\tikz\pgfuseplotmark{m6a};};
\node[stars] at (axis cs:{100.309},{-40.350}) {\tikz\pgfuseplotmark{m6cb};};
\node[stars] at (axis cs:{100.847},{-39.193}) {\tikz\pgfuseplotmark{m6c};};
\node[stars] at (axis cs:{100.903},{-73.118}) {\tikz\pgfuseplotmark{m6c};};
\node[stars] at (axis cs:{101.053},{-54.695}) {\tikz\pgfuseplotmark{m6cb};};
\node[stars] at (axis cs:{101.119},{-31.070}) {\tikz\pgfuseplotmark{m5cv};};
\node[stars] at (axis cs:{101.216},{-27.342}) {\tikz\pgfuseplotmark{m6c};};
\node[stars] at (axis cs:{101.232},{-70.434}) {\tikz\pgfuseplotmark{m6b};};
\node[stars] at (axis cs:{101.346},{-31.793}) {\tikz\pgfuseplotmark{m6b};};
\node[stars] at (axis cs:{101.380},{-30.949}) {\tikz\pgfuseplotmark{m6avb};};
\node[stars] at (axis cs:{101.474},{-52.410}) {\tikz\pgfuseplotmark{m6a};};
\node[stars] at (axis cs:{101.551},{-37.775}) {\tikz\pgfuseplotmark{m6cv};};
\node[stars] at (axis cs:{101.719},{-51.265}) {\tikz\pgfuseplotmark{m5c};};
\node[stars] at (axis cs:{101.744},{-87.025}) {\tikz\pgfuseplotmark{m6c};};
\node[stars] at (axis cs:{101.828},{-55.540}) {\tikz\pgfuseplotmark{m6av};};
\node[stars] at (axis cs:{101.839},{-37.929}) {\tikz\pgfuseplotmark{m5c};};
\node[stars] at (axis cs:{102.048},{-61.941}) {\tikz\pgfuseplotmark{m3c};};
\node[stars] at (axis cs:{102.460},{-32.508}) {\tikz\pgfuseplotmark{m4av};};
\node[stars] at (axis cs:{102.464},{-53.622}) {\tikz\pgfuseplotmark{m4cv};};
\node[stars] at (axis cs:{102.478},{-46.614}) {\tikz\pgfuseplotmark{m5c};};
\node[stars] at (axis cs:{102.484},{-50.614}) {\tikz\pgfuseplotmark{m3b};};
\node[stars] at (axis cs:{102.504},{-60.249}) {\tikz\pgfuseplotmark{m6b};};
\node[stars] at (axis cs:{102.597},{-31.706}) {\tikz\pgfuseplotmark{m6avb};};
\node[stars] at (axis cs:{102.718},{-34.367}) {\tikz\pgfuseplotmark{m5b};};
\node[stars] at (axis cs:{102.862},{-70.963}) {\tikz\pgfuseplotmark{m5c};};
\node[stars] at (axis cs:{102.886},{-48.293}) {\tikz\pgfuseplotmark{m6c};};
\node[stars] at (axis cs:{102.927},{-36.230}) {\tikz\pgfuseplotmark{m6b};};
\node[stars] at (axis cs:{103.189},{-59.341}) {\tikz\pgfuseplotmark{m6c};};
\node[stars] at (axis cs:{103.196},{-43.975}) {\tikz\pgfuseplotmark{m6c};};
\node[stars] at (axis cs:{103.391},{-28.540}) {\tikz\pgfuseplotmark{m6b};};
\node[stars] at (axis cs:{103.509},{-50.611}) {\tikz\pgfuseplotmark{m6c};};
\node[stars] at (axis cs:{103.611},{-42.365}) {\tikz\pgfuseplotmark{m6cv};};
\node[stars] at (axis cs:{103.978},{-31.790}) {\tikz\pgfuseplotmark{m6cv};};
\node[stars] at (axis cs:{104.067},{-48.721}) {\tikz\pgfuseplotmark{m5bv};};
\node[stars] at (axis cs:{104.144},{-79.420}) {\tikz\pgfuseplotmark{m5c};};
\node[stars] at (axis cs:{104.190},{-35.341}) {\tikz\pgfuseplotmark{m6c};};
\node[stars] at (axis cs:{104.323},{-35.507}) {\tikz\pgfuseplotmark{m6cv};};
\node[stars] at (axis cs:{104.428},{-27.537}) {\tikz\pgfuseplotmark{m6cv};};
\node[stars] at (axis cs:{104.531},{-27.164}) {\tikz\pgfuseplotmark{m6cv};};
\node[stars] at (axis cs:{104.605},{-34.112}) {\tikz\pgfuseplotmark{m5b};};
\node[stars] at (axis cs:{104.652},{-56.395}) {\tikz\pgfuseplotmark{m6c};};
\node[stars] at (axis cs:{104.656},{-28.972}) {\tikz\pgfuseplotmark{m2a};};
\node[stars] at (axis cs:{104.666},{-55.729}) {\tikz\pgfuseplotmark{m6c};};
\node[stars] at (axis cs:{104.674},{-45.768}) {\tikz\pgfuseplotmark{m6c};};
\node[stars] at (axis cs:{104.682},{-30.998}) {\tikz\pgfuseplotmark{m6cb};};
\node[stars] at (axis cs:{104.961},{-67.916}) {\tikz\pgfuseplotmark{m5c};};
\node[stars] at (axis cs:{105.177},{-28.489}) {\tikz\pgfuseplotmark{m6c};};
\node[stars] at (axis cs:{105.207},{-33.465}) {\tikz\pgfuseplotmark{m6c};};
\node[stars] at (axis cs:{105.215},{-51.403}) {\tikz\pgfuseplotmark{m5cv};};
\node[stars] at (axis cs:{105.271},{-58.940}) {\tikz\pgfuseplotmark{m6bv};};
\node[stars] at (axis cs:{105.430},{-27.935}) {\tikz\pgfuseplotmark{m3cv};};
\node[stars] at (axis cs:{105.564},{-43.404}) {\tikz\pgfuseplotmark{m6c};};
\node[stars] at (axis cs:{105.813},{-59.178}) {\tikz\pgfuseplotmark{m6avb};};
\node[stars] at (axis cs:{105.973},{-49.584}) {\tikz\pgfuseplotmark{m5b};};
\node[stars] at (axis cs:{105.989},{-43.608}) {\tikz\pgfuseplotmark{m6ab};};
\node[stars] at (axis cs:{106.012},{-42.337}) {\tikz\pgfuseplotmark{m5c};};
\node[stars] at (axis cs:{106.076},{-56.749}) {\tikz\pgfuseplotmark{m5cv};};
\node[stars] at (axis cs:{106.318},{-50.360}) {\tikz\pgfuseplotmark{m6c};};
\node[stars] at (axis cs:{106.383},{-34.778}) {\tikz\pgfuseplotmark{m6cb};};
\node[stars] at (axis cs:{106.502},{-30.656}) {\tikz\pgfuseplotmark{m6cv};};
\node[stars] at (axis cs:{106.509},{-38.383}) {\tikz\pgfuseplotmark{m6b};};
\node[stars] at (axis cs:{106.560},{-68.838}) {\tikz\pgfuseplotmark{m6c};};
\node[stars] at (axis cs:{106.779},{-40.893}) {\tikz\pgfuseplotmark{m6a};};
\node[stars] at (axis cs:{106.805},{-51.968}) {\tikz\pgfuseplotmark{m6b};};
\node[stars] at (axis cs:{106.818},{-59.716}) {\tikz\pgfuseplotmark{m6c};};
\node[stars] at (axis cs:{107.187},{-70.499}) {\tikz\pgfuseplotmark{m4avb};};
\node[stars] at (axis cs:{107.213},{-39.656}) {\tikz\pgfuseplotmark{m5av};};
\node[stars] at (axis cs:{107.581},{-27.491}) {\tikz\pgfuseplotmark{m5c};};
\node[stars] at (axis cs:{107.698},{-48.932}) {\tikz\pgfuseplotmark{m5b};};
\node[stars] at (axis cs:{108.008},{-63.190}) {\tikz\pgfuseplotmark{m6b};};
\node[stars] at (axis cs:{108.017},{-30.821}) {\tikz\pgfuseplotmark{m6b};};
\node[stars] at (axis cs:{108.066},{-40.499}) {\tikz\pgfuseplotmark{m5c};};
\node[stars] at (axis cs:{108.108},{-36.544}) {\tikz\pgfuseplotmark{m6bvb};};
\node[stars] at (axis cs:{108.140},{-46.759}) {\tikz\pgfuseplotmark{m4cv};};
\node[stars] at (axis cs:{108.306},{-45.182}) {\tikz\pgfuseplotmark{m5bv};};
\node[stars] at (axis cs:{108.385},{-44.640}) {\tikz\pgfuseplotmark{m5av};};
\node[stars] at (axis cs:{108.402},{-27.356}) {\tikz\pgfuseplotmark{m6bv};};
\node[stars] at (axis cs:{108.489},{-30.340}) {\tikz\pgfuseplotmark{m6c};};
\node[stars] at (axis cs:{108.659},{-48.272}) {\tikz\pgfuseplotmark{m5a};};
\node[stars] at (axis cs:{108.692},{-46.849}) {\tikz\pgfuseplotmark{m6av};};
\node[stars] at (axis cs:{108.713},{-27.038}) {\tikz\pgfuseplotmark{m6a};};
\node[stars] at (axis cs:{108.738},{-41.426}) {\tikz\pgfuseplotmark{m6b};};
\node[stars] at (axis cs:{108.838},{-30.686}) {\tikz\pgfuseplotmark{m5cv};};
\node[stars] at (axis cs:{108.838},{-52.499}) {\tikz\pgfuseplotmark{m6b};};
\node[stars] at (axis cs:{109.064},{-46.774}) {\tikz\pgfuseplotmark{m6a};};
\node[stars] at (axis cs:{109.133},{-38.319}) {\tikz\pgfuseplotmark{m6a};};
\node[stars] at (axis cs:{109.146},{-27.881}) {\tikz\pgfuseplotmark{m5av};};
\node[stars] at (axis cs:{109.206},{-36.592}) {\tikz\pgfuseplotmark{m5bv};};
\node[stars] at (axis cs:{109.208},{-67.957}) {\tikz\pgfuseplotmark{m4b};};
\node[stars] at (axis cs:{109.238},{-30.897}) {\tikz\pgfuseplotmark{m6cb};};
\node[stars] at (axis cs:{109.286},{-37.097}) {\tikz\pgfuseplotmark{m3av};};
\node[stars] at (axis cs:{109.518},{-43.987}) {\tikz\pgfuseplotmark{m6a};};
\node[stars] at (axis cs:{109.577},{-36.734}) {\tikz\pgfuseplotmark{m5avb};};
\node[stars] at (axis cs:{109.640},{-39.210}) {\tikz\pgfuseplotmark{m5c};};
\node[stars] at (axis cs:{109.807},{-33.727}) {\tikz\pgfuseplotmark{m6cv};};
\node[stars] at (axis cs:{110.089},{-52.311}) {\tikz\pgfuseplotmark{m6bb};};
\node[stars] at (axis cs:{110.162},{-52.086}) {\tikz\pgfuseplotmark{m5c};};
\node[stars] at (axis cs:{110.753},{-31.924}) {\tikz\pgfuseplotmark{m5cv};};
\node[stars] at (axis cs:{110.871},{-27.834}) {\tikz\pgfuseplotmark{m5c};};
\node[stars] at (axis cs:{110.883},{-32.202}) {\tikz\pgfuseplotmark{m5c};};
\node[stars] at (axis cs:{110.993},{-35.838}) {\tikz\pgfuseplotmark{m6c};};
\node[stars] at (axis cs:{111.024},{-29.303}) {\tikz\pgfuseplotmark{m2cv};};
\node[stars] at (axis cs:{111.155},{-71.471}) {\tikz\pgfuseplotmark{m6c};};
\node[stars] at (axis cs:{111.183},{-31.809}) {\tikz\pgfuseplotmark{m5cvb};};
\node[stars] at (axis cs:{111.409},{-79.094}) {\tikz\pgfuseplotmark{m6a};};
\node[stars] at (axis cs:{111.429},{-31.739}) {\tikz\pgfuseplotmark{m6cv};};
\node[stars] at (axis cs:{111.591},{-51.018}) {\tikz\pgfuseplotmark{m5b};};
\node[stars] at (axis cs:{111.677},{-34.141}) {\tikz\pgfuseplotmark{m6b};};
\node[stars] at (axis cs:{111.997},{-29.156}) {\tikz\pgfuseplotmark{m6a};};
\node[stars] at (axis cs:{112.213},{-31.848}) {\tikz\pgfuseplotmark{m6cb};};
\node[stars] at (axis cs:{112.214},{-64.510}) {\tikz\pgfuseplotmark{m6c};};
\node[stars] at (axis cs:{112.270},{-31.456}) {\tikz\pgfuseplotmark{m6a};};
\node[stars] at (axis cs:{112.274},{-38.812}) {\tikz\pgfuseplotmark{m5c};};
\node[stars] at (axis cs:{112.308},{-43.301}) {\tikz\pgfuseplotmark{m3cv};};
\node[stars] at (axis cs:{112.499},{-52.651}) {\tikz\pgfuseplotmark{m6b};};
\node[stars] at (axis cs:{112.629},{-54.399}) {\tikz\pgfuseplotmark{m6b};};
\node[stars] at (axis cs:{112.678},{-30.962}) {\tikz\pgfuseplotmark{m5av};};
\node[stars] at (axis cs:{113.093},{-35.961}) {\tikz\pgfuseplotmark{m6c};};
\node[stars] at (axis cs:{113.305},{-40.058}) {\tikz\pgfuseplotmark{m6cv};};
\node[stars] at (axis cs:{113.463},{-36.338}) {\tikz\pgfuseplotmark{m5cv};};
\node[stars] at (axis cs:{113.553},{-33.463}) {\tikz\pgfuseplotmark{m6b};};
\node[stars] at (axis cs:{113.645},{-27.012}) {\tikz\pgfuseplotmark{m6a};};
\node[stars] at (axis cs:{113.665},{-51.474}) {\tikz\pgfuseplotmark{m6c};};
\node[stars] at (axis cs:{113.840},{-74.275}) {\tikz\pgfuseplotmark{m6cb};};
\node[stars] at (axis cs:{113.845},{-28.369}) {\tikz\pgfuseplotmark{m5ab};};
\node[stars] at (axis cs:{113.916},{-52.534}) {\tikz\pgfuseplotmark{m5bv};};
\node[stars] at (axis cs:{114.007},{-55.888}) {\tikz\pgfuseplotmark{m6c};};
\node[stars] at (axis cs:{114.183},{-48.830}) {\tikz\pgfuseplotmark{m6a};};
\node[stars] at (axis cs:{114.342},{-34.968}) {\tikz\pgfuseplotmark{m5a};};
\node[stars] at (axis cs:{114.438},{-38.010}) {\tikz\pgfuseplotmark{m6c};};
\node[stars] at (axis cs:{114.576},{-48.601}) {\tikz\pgfuseplotmark{m6av};};
\node[stars] at (axis cs:{114.636},{-38.781}) {\tikz\pgfuseplotmark{m6c};};
\node[stars] at (axis cs:{114.683},{-36.497}) {\tikz\pgfuseplotmark{m6a};};
\node[stars] at (axis cs:{114.751},{-53.273}) {\tikz\pgfuseplotmark{m6bv};};
\node[stars] at (axis cs:{114.864},{-38.308}) {\tikz\pgfuseplotmark{m5a};};
\node[stars] at (axis cs:{114.933},{-38.139}) {\tikz\pgfuseplotmark{m6ab};};
\node[stars] at (axis cs:{114.949},{-38.260}) {\tikz\pgfuseplotmark{m6a};};
\node[stars] at (axis cs:{114.992},{-37.579}) {\tikz\pgfuseplotmark{m6bv};};
\node[stars] at (axis cs:{115.316},{-38.534}) {\tikz\pgfuseplotmark{m5c};};
\node[stars] at (axis cs:{115.341},{-44.632}) {\tikz\pgfuseplotmark{m6c};};
\node[stars] at (axis cs:{115.455},{-72.606}) {\tikz\pgfuseplotmark{m4b};};
\node[stars] at (axis cs:{115.543},{-58.631}) {\tikz\pgfuseplotmark{m6c};};
\node[stars] at (axis cs:{115.722},{-58.230}) {\tikz\pgfuseplotmark{m6c};};
\node[stars] at (axis cs:{115.738},{-45.173}) {\tikz\pgfuseplotmark{m5b};};
\node[stars] at (axis cs:{115.800},{-36.050}) {\tikz\pgfuseplotmark{m6a};};
\node[stars] at (axis cs:{115.885},{-28.411}) {\tikz\pgfuseplotmark{m5ab};};
\node[stars] at (axis cs:{115.925},{-40.934}) {\tikz\pgfuseplotmark{m5b};};
\node[stars] at (axis cs:{115.929},{-38.202}) {\tikz\pgfuseplotmark{m6c};};
\node[stars] at (axis cs:{115.952},{-28.955}) {\tikz\pgfuseplotmark{m4b};};
\node[stars] at (axis cs:{116.040},{-36.062}) {\tikz\pgfuseplotmark{m6a};};
\node[stars] at (axis cs:{116.054},{-69.822}) {\tikz\pgfuseplotmark{m6c};};
\node[stars] at (axis cs:{116.142},{-37.943}) {\tikz\pgfuseplotmark{m6b};};
\node[stars] at (axis cs:{116.183},{-66.072}) {\tikz\pgfuseplotmark{m6c};};
\node[stars] at (axis cs:{116.314},{-37.969}) {\tikz\pgfuseplotmark{m4av};};
\node[stars] at (axis cs:{116.325},{-43.753}) {\tikz\pgfuseplotmark{m6b};};
\node[stars] at (axis cs:{116.396},{-34.172}) {\tikz\pgfuseplotmark{m5c};};
\node[stars] at (axis cs:{116.398},{-56.722}) {\tikz\pgfuseplotmark{m6b};};
\node[stars] at (axis cs:{116.544},{-37.934}) {\tikz\pgfuseplotmark{m6bv};};
\node[stars] at (axis cs:{116.639},{-37.772}) {\tikz\pgfuseplotmark{m6c};};
\node[stars] at (axis cs:{116.774},{-39.331}) {\tikz\pgfuseplotmark{m6cv};};
\node[stars] at (axis cs:{116.811},{-36.073}) {\tikz\pgfuseplotmark{m6c};};
\node[stars] at (axis cs:{116.854},{-38.511}) {\tikz\pgfuseplotmark{m5bv};};
\node[stars] at (axis cs:{116.881},{-46.608}) {\tikz\pgfuseplotmark{m5c};};
\node[stars] at (axis cs:{117.036},{-40.652}) {\tikz\pgfuseplotmark{m6c};};
\node[stars] at (axis cs:{117.081},{-56.471}) {\tikz\pgfuseplotmark{m6c};};
\node[stars] at (axis cs:{117.084},{-47.078}) {\tikz\pgfuseplotmark{m5a};};
\node[stars] at (axis cs:{117.278},{-56.410}) {\tikz\pgfuseplotmark{m6a};};
\node[stars] at (axis cs:{117.304},{-46.858}) {\tikz\pgfuseplotmark{m6av};};
\node[stars] at (axis cs:{117.304},{-60.284}) {\tikz\pgfuseplotmark{m6ab};};
\node[stars] at (axis cs:{117.310},{-46.373}) {\tikz\pgfuseplotmark{m4b};};
\node[stars] at (axis cs:{117.311},{-35.243}) {\tikz\pgfuseplotmark{m6bv};};
\node[stars] at (axis cs:{117.368},{-44.751}) {\tikz\pgfuseplotmark{m6c};};
\node[stars] at (axis cs:{117.398},{-33.289}) {\tikz\pgfuseplotmark{m6a};};
\node[stars] at (axis cs:{117.421},{-66.196}) {\tikz\pgfuseplotmark{m6a};};
\node[stars] at (axis cs:{117.599},{-50.509}) {\tikz\pgfuseplotmark{m6b};};
\node[stars] at (axis cs:{117.677},{-44.580}) {\tikz\pgfuseplotmark{m6c};};
\node[stars] at (axis cs:{117.835},{-43.095}) {\tikz\pgfuseplotmark{m6cv};};
\node[stars] at (axis cs:{117.918},{-42.888}) {\tikz\pgfuseplotmark{m6bv};};
\node[stars] at (axis cs:{118.054},{-40.576}) {\tikz\pgfuseplotmark{m4a};};
\node[stars] at (axis cs:{118.065},{-34.705}) {\tikz\pgfuseplotmark{m5bb};};
\node[stars] at (axis cs:{118.124},{-54.367}) {\tikz\pgfuseplotmark{m6av};};
\node[stars] at (axis cs:{118.161},{-38.863}) {\tikz\pgfuseplotmark{m4cv};};
\node[stars] at (axis cs:{118.265},{-36.364}) {\tikz\pgfuseplotmark{m5c};};
\node[stars] at (axis cs:{118.265},{-49.613}) {\tikz\pgfuseplotmark{m5a};};
\node[stars] at (axis cs:{118.326},{-48.103}) {\tikz\pgfuseplotmark{m4cv};};
\node[stars] at (axis cs:{118.546},{-35.877}) {\tikz\pgfuseplotmark{m5cv};};
\node[stars] at (axis cs:{118.666},{-34.846}) {\tikz\pgfuseplotmark{m6b};};
\node[stars] at (axis cs:{118.722},{-57.303}) {\tikz\pgfuseplotmark{m6a};};
\node[stars] at (axis cs:{118.752},{-52.583}) {\tikz\pgfuseplotmark{m6c};};
\node[stars] at (axis cs:{118.808},{-30.918}) {\tikz\pgfuseplotmark{m6c};};
\node[stars] at (axis cs:{118.944},{-43.845}) {\tikz\pgfuseplotmark{m6b};};
\node[stars] at (axis cs:{119.078},{-60.526}) {\tikz\pgfuseplotmark{m6a};};
\node[stars] at (axis cs:{119.095},{-30.285}) {\tikz\pgfuseplotmark{m6cv};};
\node[stars] at (axis cs:{119.195},{-52.982}) {\tikz\pgfuseplotmark{m3cv};};
\node[stars] at (axis cs:{119.241},{-43.500}) {\tikz\pgfuseplotmark{m5cv};};
\node[stars] at (axis cs:{119.241},{-42.406}) {\tikz\pgfuseplotmark{m6b};};
\node[stars] at (axis cs:{119.303},{-63.297}) {\tikz\pgfuseplotmark{m6c};};
\node[stars] at (axis cs:{119.327},{-44.110}) {\tikz\pgfuseplotmark{m5b};};
\node[stars] at (axis cs:{119.334},{-47.891}) {\tikz\pgfuseplotmark{m6cv};};
\node[stars] at (axis cs:{119.417},{-30.335}) {\tikz\pgfuseplotmark{m5a};};
\node[stars] at (axis cs:{119.445},{-60.303}) {\tikz\pgfuseplotmark{m6ab};};
\node[stars] at (axis cs:{119.465},{-45.577}) {\tikz\pgfuseplotmark{m5b};};
\node[stars] at (axis cs:{119.478},{-40.784}) {\tikz\pgfuseplotmark{m6c};};
\node[stars] at (axis cs:{119.560},{-49.245}) {\tikz\pgfuseplotmark{m4cvb};};
\node[stars] at (axis cs:{119.589},{-51.448}) {\tikz\pgfuseplotmark{m6cv};};
\node[stars] at (axis cs:{119.711},{-60.824}) {\tikz\pgfuseplotmark{m6av};};
\node[stars] at (axis cs:{119.758},{-45.216}) {\tikz\pgfuseplotmark{m6b};};
\node[stars] at (axis cs:{119.801},{-49.977}) {\tikz\pgfuseplotmark{m6cb};};
\node[stars] at (axis cs:{119.816},{-73.244}) {\tikz\pgfuseplotmark{m6c};};
\node[stars] at (axis cs:{119.868},{-39.297}) {\tikz\pgfuseplotmark{m5c};};
\node[stars] at (axis cs:{119.906},{-60.587}) {\tikz\pgfuseplotmark{m5cv};};
\node[stars] at (axis cs:{119.917},{-60.207}) {\tikz\pgfuseplotmark{m6c};};
\node[stars] at (axis cs:{120.061},{-48.981}) {\tikz\pgfuseplotmark{m6b};};
\node[stars] at (axis cs:{120.083},{-63.567}) {\tikz\pgfuseplotmark{m5av};};
\node[stars] at (axis cs:{120.121},{-48.872}) {\tikz\pgfuseplotmark{m6bv};};
\node[stars] at (axis cs:{120.208},{-54.151}) {\tikz\pgfuseplotmark{m6av};};
\node[stars] at (axis cs:{120.346},{-54.515}) {\tikz\pgfuseplotmark{m6bvb};};
\node[stars] at (axis cs:{120.382},{-55.455}) {\tikz\pgfuseplotmark{m6c};};
\node[stars] at (axis cs:{120.406},{-37.284}) {\tikz\pgfuseplotmark{m6b};};
\node[stars] at (axis cs:{120.526},{-37.050}) {\tikz\pgfuseplotmark{m6cv};};
\node[stars] at (axis cs:{120.687},{-41.310}) {\tikz\pgfuseplotmark{m6avb};};
\node[stars] at (axis cs:{120.767},{-32.463}) {\tikz\pgfuseplotmark{m6ab};};
\node[stars] at (axis cs:{120.873},{-42.948}) {\tikz\pgfuseplotmark{m6c};};
\node[stars] at (axis cs:{120.896},{-40.003}) {\tikz\pgfuseplotmark{m2cv};};
\node[stars] at (axis cs:{121.067},{-32.675}) {\tikz\pgfuseplotmark{m5cv};};
\node[stars] at (axis cs:{121.177},{-50.590}) {\tikz\pgfuseplotmark{m6b};};
\node[stars] at (axis cs:{121.179},{-62.836}) {\tikz\pgfuseplotmark{m6cvb};};
\node[stars] at (axis cs:{121.265},{-53.108}) {\tikz\pgfuseplotmark{m6a};};
\node[stars] at (axis cs:{121.335},{-46.978}) {\tikz\pgfuseplotmark{m6cv};};
\node[stars] at (axis cs:{121.437},{-33.569}) {\tikz\pgfuseplotmark{m6bb};};
\node[stars] at (axis cs:{121.668},{-45.266}) {\tikz\pgfuseplotmark{m5bv};};
\node[stars] at (axis cs:{121.673},{-48.497}) {\tikz\pgfuseplotmark{m6c};};
\node[stars] at (axis cs:{121.982},{-68.617}) {\tikz\pgfuseplotmark{m4cb};};
\node[stars] at (axis cs:{122.103},{-63.801}) {\tikz\pgfuseplotmark{m6cv};};
\node[stars] at (axis cs:{122.157},{-37.681}) {\tikz\pgfuseplotmark{m6cv};};
\node[stars] at (axis cs:{122.253},{-61.302}) {\tikz\pgfuseplotmark{m5a};};
\node[stars] at (axis cs:{122.290},{-48.684}) {\tikz\pgfuseplotmark{m6av};};
\node[stars] at (axis cs:{122.292},{-35.455}) {\tikz\pgfuseplotmark{m6c};};
\node[stars] at (axis cs:{122.383},{-47.337}) {\tikz\pgfuseplotmark{m2avb};};
\node[stars] at (axis cs:{122.390},{-56.085}) {\tikz\pgfuseplotmark{m6a};};
\node[stars] at (axis cs:{122.400},{-44.123}) {\tikz\pgfuseplotmark{m5cv};};
\node[stars] at (axis cs:{122.430},{-47.937}) {\tikz\pgfuseplotmark{m5cv};};
\node[stars] at (axis cs:{122.586},{-49.237}) {\tikz\pgfuseplotmark{m6c};};
\node[stars] at (axis cs:{122.757},{-37.292}) {\tikz\pgfuseplotmark{m6c};};
\node[stars] at (axis cs:{122.795},{-48.462}) {\tikz\pgfuseplotmark{m6av};};
\node[stars] at (axis cs:{122.840},{-39.619}) {\tikz\pgfuseplotmark{m4cv};};
\node[stars] at (axis cs:{122.858},{-42.987}) {\tikz\pgfuseplotmark{m5av};};
\node[stars] at (axis cs:{123.000},{-46.644}) {\tikz\pgfuseplotmark{m6av};};
\node[stars] at (axis cs:{123.025},{-43.400}) {\tikz\pgfuseplotmark{m6c};};
\node[stars] at (axis cs:{123.128},{-46.264}) {\tikz\pgfuseplotmark{m6b};};
\node[stars] at (axis cs:{123.215},{-37.924}) {\tikz\pgfuseplotmark{m6c};};
\node[stars] at (axis cs:{123.373},{-35.899}) {\tikz\pgfuseplotmark{m5av};};
\node[stars] at (axis cs:{123.393},{-50.196}) {\tikz\pgfuseplotmark{m6av};};
\node[stars] at (axis cs:{123.401},{-46.991}) {\tikz\pgfuseplotmark{m5cv};};
\node[stars] at (axis cs:{123.421},{-33.569}) {\tikz\pgfuseplotmark{m6c};};
\node[stars] at (axis cs:{123.440},{-46.578}) {\tikz\pgfuseplotmark{m6c};};
\node[stars] at (axis cs:{123.493},{-36.322}) {\tikz\pgfuseplotmark{m5bvb};};
\node[stars] at (axis cs:{123.512},{-40.348}) {\tikz\pgfuseplotmark{m4c};};
\node[stars] at (axis cs:{123.546},{-32.141}) {\tikz\pgfuseplotmark{m6b};};
\node[stars] at (axis cs:{123.555},{-35.490}) {\tikz\pgfuseplotmark{m6a};};
\node[stars] at (axis cs:{123.599},{-45.835}) {\tikz\pgfuseplotmark{m6ab};};
\node[stars] at (axis cs:{123.714},{-46.486}) {\tikz\pgfuseplotmark{m6c};};
\node[stars] at (axis cs:{123.816},{-62.915}) {\tikz\pgfuseplotmark{m5cb};};
\node[stars] at (axis cs:{123.847},{-50.449}) {\tikz\pgfuseplotmark{m6c};};
\node[stars] at (axis cs:{123.969},{-30.926}) {\tikz\pgfuseplotmark{m6cb};};
\node[stars] at (axis cs:{123.995},{-35.903}) {\tikz\pgfuseplotmark{m6c};};
\node[stars] at (axis cs:{124.482},{-59.167}) {\tikz\pgfuseplotmark{m6cb};};
\node[stars] at (axis cs:{124.493},{-30.004}) {\tikz\pgfuseplotmark{m6c};};
\node[stars] at (axis cs:{124.572},{-35.452}) {\tikz\pgfuseplotmark{m6a};};
\node[stars] at (axis cs:{124.578},{-65.613}) {\tikz\pgfuseplotmark{m5b};};
\node[stars] at (axis cs:{124.631},{-76.920}) {\tikz\pgfuseplotmark{m4b};};
\node[stars] at (axis cs:{124.639},{-36.659}) {\tikz\pgfuseplotmark{m4c};};
\node[stars] at (axis cs:{124.773},{-48.198}) {\tikz\pgfuseplotmark{m6c};};
\node[stars] at (axis cs:{124.873},{-34.590}) {\tikz\pgfuseplotmark{m6c};};
\node[stars] at (axis cs:{124.954},{-71.515}) {\tikz\pgfuseplotmark{m5cb};};
\node[stars] at (axis cs:{125.161},{-77.484}) {\tikz\pgfuseplotmark{m4c};};
\node[stars] at (axis cs:{125.282},{-64.106}) {\tikz\pgfuseplotmark{m6b};};
\node[stars] at (axis cs:{125.300},{-57.973}) {\tikz\pgfuseplotmark{m6bv};};
\node[stars] at (axis cs:{125.338},{-36.484}) {\tikz\pgfuseplotmark{m5cv};};
\node[stars] at (axis cs:{125.346},{-33.054}) {\tikz\pgfuseplotmark{m5a};};
\node[stars] at (axis cs:{125.351},{-39.621}) {\tikz\pgfuseplotmark{m6c};};
\node[stars] at (axis cs:{125.519},{-73.400}) {\tikz\pgfuseplotmark{m5c};};
\node[stars] at (axis cs:{125.628},{-59.509}) {\tikz\pgfuseplotmark{m2bv};};
\node[stars] at (axis cs:{125.632},{-48.490}) {\tikz\pgfuseplotmark{m5av};};
\node[stars] at (axis cs:{125.730},{-52.124}) {\tikz\pgfuseplotmark{m6b};};
\node[stars] at (axis cs:{125.821},{-38.286}) {\tikz\pgfuseplotmark{m6cv};};
\node[stars] at (axis cs:{126.083},{-80.914}) {\tikz\pgfuseplotmark{m6a};};
\node[stars] at (axis cs:{126.238},{-42.770}) {\tikz\pgfuseplotmark{m6bv};};
\node[stars] at (axis cs:{126.381},{-51.727}) {\tikz\pgfuseplotmark{m5c};};
\node[stars] at (axis cs:{126.434},{-66.137}) {\tikz\pgfuseplotmark{m4av};};
\node[stars] at (axis cs:{126.465},{-64.601}) {\tikz\pgfuseplotmark{m6b};};
\node[stars] at (axis cs:{126.466},{-42.153}) {\tikz\pgfuseplotmark{m5c};};
\node[stars] at (axis cs:{126.605},{-52.807}) {\tikz\pgfuseplotmark{m6b};};
\node[stars] at (axis cs:{126.741},{-52.705}) {\tikz\pgfuseplotmark{m6c};};
\node[stars] at (axis cs:{126.818},{-31.673}) {\tikz\pgfuseplotmark{m6c};};
\node[stars] at (axis cs:{126.820},{-70.093}) {\tikz\pgfuseplotmark{m6a};};
\node[stars] at (axis cs:{126.902},{-53.088}) {\tikz\pgfuseplotmark{m5bv};};
\node[stars] at (axis cs:{126.998},{-35.114}) {\tikz\pgfuseplotmark{m6avb};};
\node[stars] at (axis cs:{127.270},{-47.929}) {\tikz\pgfuseplotmark{m5cb};};
\node[stars] at (axis cs:{127.282},{-44.160}) {\tikz\pgfuseplotmark{m6av};};
\node[stars] at (axis cs:{127.365},{-44.725}) {\tikz\pgfuseplotmark{m5cvb};};
\node[stars] at (axis cs:{127.374},{-54.212}) {\tikz\pgfuseplotmark{m6cv};};
\node[stars] at (axis cs:{127.402},{-55.191}) {\tikz\pgfuseplotmark{m6c};};
\node[stars] at (axis cs:{127.440},{-46.331}) {\tikz\pgfuseplotmark{m6b};};
\node[stars] at (axis cs:{127.619},{-32.159}) {\tikz\pgfuseplotmark{m6av};};
\node[stars] at (axis cs:{127.663},{-44.737}) {\tikz\pgfuseplotmark{m6c};};
\node[stars] at (axis cs:{127.794},{-47.866}) {\tikz\pgfuseplotmark{m6c};};
\node[stars] at (axis cs:{127.853},{-39.064}) {\tikz\pgfuseplotmark{m6cb};};
\node[stars] at (axis cs:{127.873},{-54.394}) {\tikz\pgfuseplotmark{m6c};};
\node[stars] at (axis cs:{128.021},{-53.212}) {\tikz\pgfuseplotmark{m6a};};
\node[stars] at (axis cs:{128.169},{-47.042}) {\tikz\pgfuseplotmark{m6c};};
\node[stars] at (axis cs:{128.174},{-73.357}) {\tikz\pgfuseplotmark{m6b};};
\node[stars] at (axis cs:{128.215},{-31.501}) {\tikz\pgfuseplotmark{m6c};};
\node[stars] at (axis cs:{128.244},{-34.634}) {\tikz\pgfuseplotmark{m6cv};};
\node[stars] at (axis cs:{128.258},{-45.753}) {\tikz\pgfuseplotmark{m6c};};
\node[stars] at (axis cs:{128.333},{-38.371}) {\tikz\pgfuseplotmark{m6cv};};
\node[stars] at (axis cs:{128.376},{-46.971}) {\tikz\pgfuseplotmark{m6c};};
\node[stars] at (axis cs:{128.410},{-38.849}) {\tikz\pgfuseplotmark{m6bv};};
\node[stars] at (axis cs:{128.622},{-37.611}) {\tikz\pgfuseplotmark{m6cvb};};
\node[stars] at (axis cs:{128.624},{-27.098}) {\tikz\pgfuseplotmark{m6cv};};
\node[stars] at (axis cs:{128.633},{-32.598}) {\tikz\pgfuseplotmark{m6c};};
\node[stars] at (axis cs:{128.682},{-49.944}) {\tikz\pgfuseplotmark{m5b};};
\node[stars] at (axis cs:{128.802},{-39.970}) {\tikz\pgfuseplotmark{m6c};};
\node[stars] at (axis cs:{128.815},{-58.225}) {\tikz\pgfuseplotmark{m5c};};
\node[stars] at (axis cs:{128.832},{-58.009}) {\tikz\pgfuseplotmark{m5a};};
\node[stars] at (axis cs:{128.967},{-50.970}) {\tikz\pgfuseplotmark{m6av};};
\node[stars] at (axis cs:{129.328},{-62.853}) {\tikz\pgfuseplotmark{m5c};};
\node[stars] at (axis cs:{129.411},{-42.989}) {\tikz\pgfuseplotmark{m4c};};
\node[stars] at (axis cs:{129.687},{-53.090}) {\tikz\pgfuseplotmark{m6c};};
\node[stars] at (axis cs:{129.772},{-70.387}) {\tikz\pgfuseplotmark{m5c};};
\node[stars] at (axis cs:{129.805},{-60.317}) {\tikz\pgfuseplotmark{m6c};};
\node[stars] at (axis cs:{129.842},{-36.607}) {\tikz\pgfuseplotmark{m6b};};
\node[stars] at (axis cs:{129.849},{-53.440}) {\tikz\pgfuseplotmark{m5cv};};
\node[stars] at (axis cs:{129.927},{-29.561}) {\tikz\pgfuseplotmark{m5b};};
\node[stars] at (axis cs:{129.990},{-53.055}) {\tikz\pgfuseplotmark{m5cvb};};
\node[stars] at (axis cs:{130.026},{-35.308}) {\tikz\pgfuseplotmark{m4b};};
\node[stars] at (axis cs:{130.073},{-53.015}) {\tikz\pgfuseplotmark{m6av};};
\node[stars] at (axis cs:{130.073},{-52.922}) {\tikz\pgfuseplotmark{m4av};};
\node[stars] at (axis cs:{130.080},{-40.264}) {\tikz\pgfuseplotmark{m5cvb};};
\node[stars] at (axis cs:{130.147},{-45.191}) {\tikz\pgfuseplotmark{m6av};};
\node[stars] at (axis cs:{130.154},{-59.761}) {\tikz\pgfuseplotmark{m4cv};};
\node[stars] at (axis cs:{130.157},{-46.648}) {\tikz\pgfuseplotmark{m4av};};
\node[stars] at (axis cs:{130.181},{-57.545}) {\tikz\pgfuseplotmark{m6cb};};
\node[stars] at (axis cs:{130.272},{-48.923}) {\tikz\pgfuseplotmark{m6b};};
\node[stars] at (axis cs:{130.305},{-47.317}) {\tikz\pgfuseplotmark{m5a};};
\node[stars] at (axis cs:{130.331},{-78.963}) {\tikz\pgfuseplotmark{m5c};};
\node[stars] at (axis cs:{130.487},{-45.411}) {\tikz\pgfuseplotmark{m5cv};};
\node[stars] at (axis cs:{130.567},{-48.099}) {\tikz\pgfuseplotmark{m5cv};};
\node[stars] at (axis cs:{130.587},{-55.774}) {\tikz\pgfuseplotmark{m6c};};
\node[stars] at (axis cs:{130.606},{-53.114}) {\tikz\pgfuseplotmark{m5avb};};
\node[stars] at (axis cs:{130.737},{-35.943}) {\tikz\pgfuseplotmark{m6c};};
\node[stars] at (axis cs:{130.801},{-79.070}) {\tikz\pgfuseplotmark{m6bv};};
\node[stars] at (axis cs:{130.898},{-33.186}) {\tikz\pgfuseplotmark{m4av};};
\node[stars] at (axis cs:{130.918},{-49.823}) {\tikz\pgfuseplotmark{m5c};};
\node[stars] at (axis cs:{130.976},{-68.211}) {\tikz\pgfuseplotmark{m6c};};
\node[stars] at (axis cs:{131.100},{-42.649}) {\tikz\pgfuseplotmark{m4b};};
\node[stars] at (axis cs:{131.125},{-65.825}) {\tikz\pgfuseplotmark{m6b};};
\node[stars] at (axis cs:{131.176},{-54.709}) {\tikz\pgfuseplotmark{m2bv};};
\node[stars] at (axis cs:{131.216},{-37.147}) {\tikz\pgfuseplotmark{m6av};};
\node[stars] at (axis cs:{131.480},{-79.504}) {\tikz\pgfuseplotmark{m6a};};
\node[stars] at (axis cs:{131.507},{-46.041}) {\tikz\pgfuseplotmark{m4b};};
\node[stars] at (axis cs:{131.599},{-41.125}) {\tikz\pgfuseplotmark{m6c};};
\node[stars] at (axis cs:{131.627},{-45.912}) {\tikz\pgfuseplotmark{m5cv};};
\node[stars] at (axis cs:{131.677},{-56.770}) {\tikz\pgfuseplotmark{m4cv};};
\node[stars] at (axis cs:{131.705},{-34.623}) {\tikz\pgfuseplotmark{m6cv};};
\node[stars] at (axis cs:{131.829},{-46.155}) {\tikz\pgfuseplotmark{m6a};};
\node[stars] at (axis cs:{131.918},{-41.737}) {\tikz\pgfuseplotmark{m6c};};
\node[stars] at (axis cs:{132.001},{-52.850}) {\tikz\pgfuseplotmark{m6c};};
\node[stars] at (axis cs:{132.037},{-42.463}) {\tikz\pgfuseplotmark{m6cv};};
\node[stars] at (axis cs:{132.413},{-40.320}) {\tikz\pgfuseplotmark{m5c};};
\node[stars] at (axis cs:{132.448},{-45.308}) {\tikz\pgfuseplotmark{m5b};};
\node[stars] at (axis cs:{132.459},{-72.551}) {\tikz\pgfuseplotmark{m6b};};
\node[stars] at (axis cs:{132.465},{-32.780}) {\tikz\pgfuseplotmark{m5c};};
\node[stars] at (axis cs:{132.468},{-39.141}) {\tikz\pgfuseplotmark{m6cv};};
\node[stars] at (axis cs:{132.509},{-29.463}) {\tikz\pgfuseplotmark{m6b};};
\node[stars] at (axis cs:{132.588},{-42.090}) {\tikz\pgfuseplotmark{m6b};};
\node[stars] at (axis cs:{132.590},{-28.618}) {\tikz\pgfuseplotmark{m6c};};
\node[stars] at (axis cs:{132.633},{-27.710}) {\tikz\pgfuseplotmark{m4b};};
\node[stars] at (axis cs:{132.639},{-46.529}) {\tikz\pgfuseplotmark{m5bv};};
\node[stars] at (axis cs:{132.645},{-66.793}) {\tikz\pgfuseplotmark{m5c};};
\node[stars] at (axis cs:{132.902},{-57.634}) {\tikz\pgfuseplotmark{m6a};};
\node[stars] at (axis cs:{132.958},{-44.151}) {\tikz\pgfuseplotmark{m6c};};
\node[stars] at (axis cs:{132.982},{-40.987}) {\tikz\pgfuseplotmark{m6c};};
\node[stars] at (axis cs:{133.109},{-32.509}) {\tikz\pgfuseplotmark{m6c};};
\node[stars] at (axis cs:{133.161},{-48.359}) {\tikz\pgfuseplotmark{m6b};};
\node[stars] at (axis cs:{133.161},{-36.545}) {\tikz\pgfuseplotmark{m6cv};};
\node[stars] at (axis cs:{133.170},{-52.129}) {\tikz\pgfuseplotmark{m6cb};};
\node[stars] at (axis cs:{133.200},{-38.724}) {\tikz\pgfuseplotmark{m6a};};
\node[stars] at (axis cs:{133.266},{-56.649}) {\tikz\pgfuseplotmark{m6b};};
\node[stars] at (axis cs:{133.453},{-60.354}) {\tikz\pgfuseplotmark{m6a};};
\node[stars] at (axis cs:{133.461},{-47.521}) {\tikz\pgfuseplotmark{m5c};};
\node[stars] at (axis cs:{133.461},{-40.447}) {\tikz\pgfuseplotmark{m6c};};
\node[stars] at (axis cs:{133.725},{-58.240}) {\tikz\pgfuseplotmark{m6cv};};
\node[stars] at (axis cs:{133.762},{-60.644}) {\tikz\pgfuseplotmark{m4a};};
\node[stars] at (axis cs:{133.799},{-54.966}) {\tikz\pgfuseplotmark{m6a};};
\node[stars] at (axis cs:{133.830},{-45.042}) {\tikz\pgfuseplotmark{m6cv};};
\node[stars] at (axis cs:{133.882},{-27.682}) {\tikz\pgfuseplotmark{m5b};};
\node[stars] at (axis cs:{134.080},{-52.723}) {\tikz\pgfuseplotmark{m5ab};};
\node[stars] at (axis cs:{134.171},{-85.663}) {\tikz\pgfuseplotmark{m5c};};
\node[stars] at (axis cs:{134.243},{-59.229}) {\tikz\pgfuseplotmark{m5bvb};};
\node[stars] at (axis cs:{134.481},{-48.573}) {\tikz\pgfuseplotmark{m6b};};
\node[stars] at (axis cs:{134.718},{-47.235}) {\tikz\pgfuseplotmark{m5cv};};
\node[stars] at (axis cs:{134.816},{-28.806}) {\tikz\pgfuseplotmark{m6c};};
\node[stars] at (axis cs:{134.851},{-59.084}) {\tikz\pgfuseplotmark{m5cb};};
\node[stars] at (axis cs:{135.023},{-41.254}) {\tikz\pgfuseplotmark{m4cv};};
\node[stars] at (axis cs:{135.093},{-43.174}) {\tikz\pgfuseplotmark{m6bv};};
\node[stars] at (axis cs:{135.191},{-60.964}) {\tikz\pgfuseplotmark{m6a};};
\node[stars] at (axis cs:{135.285},{-68.684}) {\tikz\pgfuseplotmark{m6b};};
\node[stars] at (axis cs:{135.337},{-41.864}) {\tikz\pgfuseplotmark{m6av};};
\node[stars] at (axis cs:{135.436},{-52.189}) {\tikz\pgfuseplotmark{m5cv};};
\node[stars] at (axis cs:{135.527},{-39.402}) {\tikz\pgfuseplotmark{m6c};};
\node[stars] at (axis cs:{135.612},{-66.396}) {\tikz\pgfuseplotmark{m4b};};
\node[stars] at (axis cs:{135.772},{-53.550}) {\tikz\pgfuseplotmark{m6c};};
\node[stars] at (axis cs:{136.024},{-47.441}) {\tikz\pgfuseplotmark{m6cv};};
\node[stars] at (axis cs:{136.039},{-47.098}) {\tikz\pgfuseplotmark{m4a};};
\node[stars] at (axis cs:{136.200},{-57.853}) {\tikz\pgfuseplotmark{m6cb};};
\node[stars] at (axis cs:{136.287},{-72.603}) {\tikz\pgfuseplotmark{m4c};};
\node[stars] at (axis cs:{136.410},{-70.538}) {\tikz\pgfuseplotmark{m5av};};
\node[stars] at (axis cs:{136.532},{-64.500}) {\tikz\pgfuseplotmark{m6c};};
\node[stars] at (axis cs:{136.642},{-55.802}) {\tikz\pgfuseplotmark{m6bv};};
\node[stars] at (axis cs:{136.999},{-43.432}) {\tikz\pgfuseplotmark{m2cv};};
\node[stars] at (axis cs:{137.437},{-49.424}) {\tikz\pgfuseplotmark{m6c};};
\node[stars] at (axis cs:{137.485},{-30.365}) {\tikz\pgfuseplotmark{m6ab};};
\node[stars] at (axis cs:{137.488},{-52.083}) {\tikz\pgfuseplotmark{m6c};};
\node[stars] at (axis cs:{137.742},{-58.967}) {\tikz\pgfuseplotmark{m3cv};};
\node[stars] at (axis cs:{137.768},{-44.868}) {\tikz\pgfuseplotmark{m5bv};};
\node[stars] at (axis cs:{137.820},{-62.317}) {\tikz\pgfuseplotmark{m4bv};};
\node[stars] at (axis cs:{137.889},{-46.584}) {\tikz\pgfuseplotmark{m6a};};
\node[stars] at (axis cs:{137.921},{-39.259}) {\tikz\pgfuseplotmark{m6b};};
\node[stars] at (axis cs:{138.052},{-76.663}) {\tikz\pgfuseplotmark{m6b};};
\node[stars] at (axis cs:{138.127},{-43.613}) {\tikz\pgfuseplotmark{m6avb};};
\node[stars] at (axis cs:{138.211},{-60.917}) {\tikz\pgfuseplotmark{m6c};};
\node[stars] at (axis cs:{138.232},{-59.414}) {\tikz\pgfuseplotmark{m6a};};
\node[stars] at (axis cs:{138.300},{-69.717}) {\tikz\pgfuseplotmark{m2a};};
\node[stars] at (axis cs:{138.328},{-42.273}) {\tikz\pgfuseplotmark{m6c};};
\node[stars] at (axis cs:{138.340},{-29.664}) {\tikz\pgfuseplotmark{m6c};};
\node[stars] at (axis cs:{138.358},{-38.616}) {\tikz\pgfuseplotmark{m6c};};
\node[stars] at (axis cs:{138.394},{-47.338}) {\tikz\pgfuseplotmark{m6b};};
\node[stars] at (axis cs:{138.534},{-44.146}) {\tikz\pgfuseplotmark{m6av};};
\node[stars] at (axis cs:{138.575},{-55.570}) {\tikz\pgfuseplotmark{m5cv};};
\node[stars] at (axis cs:{138.602},{-43.227}) {\tikz\pgfuseplotmark{m5cb};};
\node[stars] at (axis cs:{138.738},{-37.602}) {\tikz\pgfuseplotmark{m6a};};
\node[stars] at (axis cs:{138.811},{-45.555}) {\tikz\pgfuseplotmark{m6c};};
\node[stars] at (axis cs:{138.823},{-58.388}) {\tikz\pgfuseplotmark{m6b};};
\node[stars] at (axis cs:{138.895},{-57.578}) {\tikz\pgfuseplotmark{m6c};};
\node[stars] at (axis cs:{138.903},{-38.570}) {\tikz\pgfuseplotmark{m5b};};
\node[stars] at (axis cs:{138.938},{-37.413}) {\tikz\pgfuseplotmark{m5a};};
\node[stars] at (axis cs:{139.017},{-44.898}) {\tikz\pgfuseplotmark{m6bv};};
\node[stars] at (axis cs:{139.050},{-57.541}) {\tikz\pgfuseplotmark{m4cv};};
\node[stars] at (axis cs:{139.096},{-44.265}) {\tikz\pgfuseplotmark{m5bv};};
\node[stars] at (axis cs:{139.238},{-39.402}) {\tikz\pgfuseplotmark{m5c};};
\node[stars] at (axis cs:{139.273},{-59.275}) {\tikz\pgfuseplotmark{m2cv};};
\node[stars] at (axis cs:{139.322},{-68.690}) {\tikz\pgfuseplotmark{m5c};};
\node[stars] at (axis cs:{139.355},{-74.894}) {\tikz\pgfuseplotmark{m5c};};
\node[stars] at (axis cs:{139.365},{-74.735}) {\tikz\pgfuseplotmark{m6b};};
\node[stars] at (axis cs:{139.426},{-54.495}) {\tikz\pgfuseplotmark{m6c};};
\node[stars] at (axis cs:{139.465},{-67.051}) {\tikz\pgfuseplotmark{m6bv};};
\node[stars] at (axis cs:{139.524},{-51.051}) {\tikz\pgfuseplotmark{m5c};};
\node[stars] at (axis cs:{139.676},{-51.560}) {\tikz\pgfuseplotmark{m6bv};};
\node[stars] at (axis cs:{139.885},{-55.186}) {\tikz\pgfuseplotmark{m6c};};
\node[stars] at (axis cs:{139.950},{-34.103}) {\tikz\pgfuseplotmark{m6c};};
\node[stars] at (axis cs:{140.124},{-37.581}) {\tikz\pgfuseplotmark{m6b};};
\node[stars] at (axis cs:{140.237},{-62.405}) {\tikz\pgfuseplotmark{m5a};};
\node[stars] at (axis cs:{140.459},{-55.514}) {\tikz\pgfuseplotmark{m6a};};
\node[stars] at (axis cs:{140.462},{-42.195}) {\tikz\pgfuseplotmark{m6av};};
\node[stars] at (axis cs:{140.528},{-55.010}) {\tikz\pgfuseplotmark{m2c};};
\node[stars] at (axis cs:{140.600},{-46.047}) {\tikz\pgfuseplotmark{m6av};};
\node[stars] at (axis cs:{140.801},{-28.834}) {\tikz\pgfuseplotmark{m5a};};
\node[stars] at (axis cs:{140.856},{-48.287}) {\tikz\pgfuseplotmark{m6cv};};
\node[stars] at (axis cs:{140.864},{-60.302}) {\tikz\pgfuseplotmark{m6c};};
\node[stars] at (axis cs:{140.916},{-46.909}) {\tikz\pgfuseplotmark{m6c};};
\node[stars] at (axis cs:{140.937},{-37.757}) {\tikz\pgfuseplotmark{m6c};};
\node[stars] at (axis cs:{140.997},{-51.737}) {\tikz\pgfuseplotmark{m6bv};};
\node[stars] at (axis cs:{141.023},{-61.649}) {\tikz\pgfuseplotmark{m6b};};
\node[stars] at (axis cs:{141.038},{-80.787}) {\tikz\pgfuseplotmark{m5c};};
\node[stars] at (axis cs:{141.068},{-39.425}) {\tikz\pgfuseplotmark{m6b};};
\node[stars] at (axis cs:{141.155},{-70.076}) {\tikz\pgfuseplotmark{m6c};};
\node[stars] at (axis cs:{141.283},{-43.976}) {\tikz\pgfuseplotmark{m6cv};};
\node[stars] at (axis cs:{141.363},{-61.950}) {\tikz\pgfuseplotmark{m6ab};};
\node[stars] at (axis cs:{141.575},{-53.379}) {\tikz\pgfuseplotmark{m5b};};
\node[stars] at (axis cs:{141.619},{-40.502}) {\tikz\pgfuseplotmark{m6c};};
\node[stars] at (axis cs:{141.684},{-64.929}) {\tikz\pgfuseplotmark{m6b};};
\node[stars] at (axis cs:{141.687},{-28.788}) {\tikz\pgfuseplotmark{m6bv};};
\node[stars] at (axis cs:{141.776},{-71.602}) {\tikz\pgfuseplotmark{m5c};};
\node[stars] at (axis cs:{141.954},{-71.781}) {\tikz\pgfuseplotmark{m6c};};
\node[stars] at (axis cs:{142.127},{-66.702}) {\tikz\pgfuseplotmark{m6b};};
\node[stars] at (axis cs:{142.196},{-62.273}) {\tikz\pgfuseplotmark{m6b};};
\node[stars] at (axis cs:{142.311},{-35.951}) {\tikz\pgfuseplotmark{m4c};};
\node[stars] at (axis cs:{142.317},{-38.403}) {\tikz\pgfuseplotmark{m6c};};
\node[stars] at (axis cs:{142.521},{-51.517}) {\tikz\pgfuseplotmark{m5c};};
\node[stars] at (axis cs:{142.598},{-58.362}) {\tikz\pgfuseplotmark{m6bv};};
\node[stars] at (axis cs:{142.675},{-40.467}) {\tikz\pgfuseplotmark{m4av};};
\node[stars] at (axis cs:{142.692},{-31.889}) {\tikz\pgfuseplotmark{m6cb};};
\node[stars] at (axis cs:{142.805},{-57.034}) {\tikz\pgfuseplotmark{m3c};};
\node[stars] at (axis cs:{142.884},{-31.872}) {\tikz\pgfuseplotmark{m6b};};
\node[stars] at (axis cs:{142.888},{-66.719}) {\tikz\pgfuseplotmark{m6c};};
\node[stars] at (axis cs:{142.888},{-35.715}) {\tikz\pgfuseplotmark{m6ab};};
\node[stars] at (axis cs:{142.901},{-73.081}) {\tikz\pgfuseplotmark{m5c};};
\node[stars] at (axis cs:{143.077},{-28.628}) {\tikz\pgfuseplotmark{m6cv};};
\node[stars] at (axis cs:{143.080},{-40.649}) {\tikz\pgfuseplotmark{m5c};};
\node[stars] at (axis cs:{143.282},{-39.129}) {\tikz\pgfuseplotmark{m6cb};};
\node[stars] at (axis cs:{143.436},{-49.005}) {\tikz\pgfuseplotmark{m5bb};};
\node[stars] at (axis cs:{143.472},{-80.941}) {\tikz\pgfuseplotmark{m5bv};};
\node[stars] at (axis cs:{143.537},{-51.255}) {\tikz\pgfuseplotmark{m5b};};
\node[stars] at (axis cs:{143.611},{-59.230}) {\tikz\pgfuseplotmark{m4b};};
\node[stars] at (axis cs:{143.799},{-35.824}) {\tikz\pgfuseplotmark{m6c};};
\node[stars] at (axis cs:{144.106},{-48.751}) {\tikz\pgfuseplotmark{m6cv};};
\node[stars] at (axis cs:{144.193},{-52.944}) {\tikz\pgfuseplotmark{m6c};};
\node[stars] at (axis cs:{144.206},{-49.355}) {\tikz\pgfuseplotmark{m4c};};
\node[stars] at (axis cs:{144.291},{-32.178}) {\tikz\pgfuseplotmark{m6a};};
\node[stars] at (axis cs:{144.303},{-53.668}) {\tikz\pgfuseplotmark{m5c};};
\node[stars] at (axis cs:{144.368},{-36.096}) {\tikz\pgfuseplotmark{m6b};};
\node[stars] at (axis cs:{144.506},{-43.191}) {\tikz\pgfuseplotmark{m5c};};
\node[stars] at (axis cs:{144.837},{-61.328}) {\tikz\pgfuseplotmark{m5av};};
\node[stars] at (axis cs:{144.938},{-62.943}) {\tikz\pgfuseplotmark{m6cv};};
\node[stars] at (axis cs:{145.177},{-57.984}) {\tikz\pgfuseplotmark{m5c};};
\node[stars] at (axis cs:{145.259},{-57.259}) {\tikz\pgfuseplotmark{m6a};};
\node[stars] at (axis cs:{145.450},{-55.214}) {\tikz\pgfuseplotmark{m6bv};};
\node[stars] at (axis cs:{145.554},{-66.915}) {\tikz\pgfuseplotmark{m6cvb};};
\node[stars] at (axis cs:{145.672},{-35.502}) {\tikz\pgfuseplotmark{m6c};};
\node[stars] at (axis cs:{145.865},{-51.228}) {\tikz\pgfuseplotmark{m6cv};};
\node[stars] at (axis cs:{145.926},{-53.891}) {\tikz\pgfuseplotmark{m6a};};
\node[stars] at (axis cs:{146.050},{-27.769}) {\tikz\pgfuseplotmark{m5a};};
\node[stars] at (axis cs:{146.312},{-62.508}) {\tikz\pgfuseplotmark{m4av};};
\node[stars] at (axis cs:{146.342},{-30.203}) {\tikz\pgfuseplotmark{m6cv};};
\node[stars] at (axis cs:{146.419},{-57.186}) {\tikz\pgfuseplotmark{m6cv};};
\node[stars] at (axis cs:{146.482},{-58.794}) {\tikz\pgfuseplotmark{m6c};};
\node[stars] at (axis cs:{146.586},{-76.776}) {\tikz\pgfuseplotmark{m5c};};
\node[stars] at (axis cs:{146.627},{-44.755}) {\tikz\pgfuseplotmark{m6av};};
\node[stars] at (axis cs:{146.776},{-65.072}) {\tikz\pgfuseplotmark{m3bb};};
\node[stars] at (axis cs:{147.167},{-56.412}) {\tikz\pgfuseplotmark{m6b};};
\node[stars] at (axis cs:{147.367},{-37.187}) {\tikz\pgfuseplotmark{m6b};};
\node[stars] at (axis cs:{147.464},{-36.268}) {\tikz\pgfuseplotmark{m6c};};
\node[stars] at (axis cs:{147.488},{-45.732}) {\tikz\pgfuseplotmark{m5b};};
\node[stars] at (axis cs:{147.675},{-46.934}) {\tikz\pgfuseplotmark{m6av};};
\node[stars] at (axis cs:{147.732},{-62.745}) {\tikz\pgfuseplotmark{m6a};};
\node[stars] at (axis cs:{147.800},{-59.426}) {\tikz\pgfuseplotmark{m6a};};
\node[stars] at (axis cs:{147.832},{-46.194}) {\tikz\pgfuseplotmark{m6a};};
\node[stars] at (axis cs:{147.919},{-46.547}) {\tikz\pgfuseplotmark{m5av};};
\node[stars] at (axis cs:{148.242},{-27.332}) {\tikz\pgfuseplotmark{m6c};};
\node[stars] at (axis cs:{148.459},{-51.146}) {\tikz\pgfuseplotmark{m6bv};};
\node[stars] at (axis cs:{148.574},{-45.284}) {\tikz\pgfuseplotmark{m6ab};};
\node[stars] at (axis cs:{148.713},{-50.244}) {\tikz\pgfuseplotmark{m6a};};
\node[stars] at (axis cs:{149.022},{-40.824}) {\tikz\pgfuseplotmark{m6c};};
\node[stars] at (axis cs:{149.041},{-71.389}) {\tikz\pgfuseplotmark{m6c};};
\node[stars] at (axis cs:{149.091},{-51.336}) {\tikz\pgfuseplotmark{m6c};};
\node[stars] at (axis cs:{149.148},{-33.418}) {\tikz\pgfuseplotmark{m6a};};
\node[stars] at (axis cs:{149.216},{-54.568}) {\tikz\pgfuseplotmark{m4a};};
\node[stars] at (axis cs:{149.225},{-27.475}) {\tikz\pgfuseplotmark{m6c};};
\node[stars] at (axis cs:{149.249},{-69.102}) {\tikz\pgfuseplotmark{m6cv};};
\node[stars] at (axis cs:{149.296},{-52.639}) {\tikz\pgfuseplotmark{m6bv};};
\node[stars] at (axis cs:{149.427},{-48.414}) {\tikz\pgfuseplotmark{m6b};};
\node[stars] at (axis cs:{149.718},{-35.891}) {\tikz\pgfuseplotmark{m5c};};
\node[stars] at (axis cs:{150.182},{-82.214}) {\tikz\pgfuseplotmark{m6a};};
\node[stars] at (axis cs:{150.300},{-56.096}) {\tikz\pgfuseplotmark{m6c};};
\node[stars] at (axis cs:{150.420},{-53.364}) {\tikz\pgfuseplotmark{m6cv};};
\node[stars] at (axis cs:{150.492},{-57.350}) {\tikz\pgfuseplotmark{m6c};};
\node[stars] at (axis cs:{150.500},{-60.421}) {\tikz\pgfuseplotmark{m6b};};
\node[stars] at (axis cs:{150.706},{-62.157}) {\tikz\pgfuseplotmark{m6cv};};
\node[stars] at (axis cs:{150.750},{-60.178}) {\tikz\pgfuseplotmark{m6cv};};
\node[stars] at (axis cs:{150.836},{-46.636}) {\tikz\pgfuseplotmark{m6b};};
\node[stars] at (axis cs:{150.892},{-61.884}) {\tikz\pgfuseplotmark{m6cb};};
\node[stars] at (axis cs:{151.097},{-39.976}) {\tikz\pgfuseplotmark{m6c};};
\node[stars] at (axis cs:{151.258},{-51.313}) {\tikz\pgfuseplotmark{m6c};};
\node[stars] at (axis cs:{151.314},{-36.384}) {\tikz\pgfuseplotmark{m6c};};
\node[stars] at (axis cs:{151.547},{-47.370}) {\tikz\pgfuseplotmark{m5b};};
\node[stars] at (axis cs:{151.899},{-48.261}) {\tikz\pgfuseplotmark{m6c};};
\node[stars] at (axis cs:{151.986},{-62.221}) {\tikz\pgfuseplotmark{m6cv};};
\node[stars] at (axis cs:{152.007},{-37.333}) {\tikz\pgfuseplotmark{m6c};};
\node[stars] at (axis cs:{152.178},{-65.815}) {\tikz\pgfuseplotmark{m5c};};
\node[stars] at (axis cs:{152.234},{-51.811}) {\tikz\pgfuseplotmark{m5a};};
\node[stars] at (axis cs:{152.341},{-61.549}) {\tikz\pgfuseplotmark{m6cv};};
\node[stars] at (axis cs:{152.376},{-68.683}) {\tikz\pgfuseplotmark{m6a};};
\node[stars] at (axis cs:{152.382},{-35.857}) {\tikz\pgfuseplotmark{m6c};};
\node[stars] at (axis cs:{152.657},{-41.715}) {\tikz\pgfuseplotmark{m6b};};
\node[stars] at (axis cs:{152.898},{-58.828}) {\tikz\pgfuseplotmark{m6bv};};
\node[stars] at (axis cs:{152.944},{-58.060}) {\tikz\pgfuseplotmark{m6av};};
\node[stars] at (axis cs:{153.012},{-28.606}) {\tikz\pgfuseplotmark{m6c};};
\node[stars] at (axis cs:{153.096},{-52.163}) {\tikz\pgfuseplotmark{m6cv};};
\node[stars] at (axis cs:{153.255},{-59.918}) {\tikz\pgfuseplotmark{m6b};};
\node[stars] at (axis cs:{153.331},{-27.029}) {\tikz\pgfuseplotmark{m6c};};
\node[stars] at (axis cs:{153.338},{-61.659}) {\tikz\pgfuseplotmark{m6c};};
\node[stars] at (axis cs:{153.343},{-56.587}) {\tikz\pgfuseplotmark{m6c};};
\node[stars] at (axis cs:{153.345},{-51.233}) {\tikz\pgfuseplotmark{m5cv};};
\node[stars] at (axis cs:{153.353},{-33.031}) {\tikz\pgfuseplotmark{m6c};};
\node[stars] at (axis cs:{153.367},{-51.756}) {\tikz\pgfuseplotmark{m6a};};
\node[stars] at (axis cs:{153.378},{-66.373}) {\tikz\pgfuseplotmark{m5c};};
\node[stars] at (axis cs:{153.434},{-70.038}) {\tikz\pgfuseplotmark{m3c};};
\node[stars] at (axis cs:{153.441},{-40.346}) {\tikz\pgfuseplotmark{m6b};};
\node[stars] at (axis cs:{153.486},{-40.311}) {\tikz\pgfuseplotmark{m6c};};
\node[stars] at (axis cs:{153.684},{-42.122}) {\tikz\pgfuseplotmark{m4a};};
\node[stars] at (axis cs:{153.819},{-54.974}) {\tikz\pgfuseplotmark{m6cv};};
\node[stars] at (axis cs:{153.837},{-36.518}) {\tikz\pgfuseplotmark{m6c};};
\node[stars] at (axis cs:{153.882},{-43.112}) {\tikz\pgfuseplotmark{m6a};};
\node[stars] at (axis cs:{154.013},{-59.903}) {\tikz\pgfuseplotmark{m6c};};
\node[stars] at (axis cs:{154.167},{-51.205}) {\tikz\pgfuseplotmark{m6cv};};
\node[stars] at (axis cs:{154.271},{-61.332}) {\tikz\pgfuseplotmark{m3cv};};
\node[stars] at (axis cs:{154.334},{-46.835}) {\tikz\pgfuseplotmark{m6c};};
\node[stars] at (axis cs:{154.532},{-28.992}) {\tikz\pgfuseplotmark{m6av};};
\node[stars] at (axis cs:{154.618},{-41.668}) {\tikz\pgfuseplotmark{m6b};};
\node[stars] at (axis cs:{154.658},{-36.804}) {\tikz\pgfuseplotmark{m6c};};
\node[stars] at (axis cs:{154.658},{-56.110}) {\tikz\pgfuseplotmark{m6a};};
\node[stars] at (axis cs:{154.770},{-64.676}) {\tikz\pgfuseplotmark{m6cb};};
\node[stars] at (axis cs:{154.903},{-55.029}) {\tikz\pgfuseplotmark{m5av};};
\node[stars] at (axis cs:{154.925},{-65.137}) {\tikz\pgfuseplotmark{m6c};};
\node[stars] at (axis cs:{155.070},{-47.699}) {\tikz\pgfuseplotmark{m6a};};
\node[stars] at (axis cs:{155.228},{-56.043}) {\tikz\pgfuseplotmark{m4cvb};};
\node[stars] at (axis cs:{155.543},{-58.265}) {\tikz\pgfuseplotmark{m6c};};
\node[stars] at (axis cs:{155.582},{-41.650}) {\tikz\pgfuseplotmark{m5a};};
\node[stars] at (axis cs:{155.742},{-66.901}) {\tikz\pgfuseplotmark{m5b};};
\node[stars] at (axis cs:{155.805},{-30.162}) {\tikz\pgfuseplotmark{m6c};};
\node[stars] at (axis cs:{155.872},{-38.010}) {\tikz\pgfuseplotmark{m5cv};};
\node[stars] at (axis cs:{155.919},{-41.953}) {\tikz\pgfuseplotmark{m6c};};
\node[stars] at (axis cs:{155.961},{-57.954}) {\tikz\pgfuseplotmark{m6cv};};
\node[stars] at (axis cs:{156.099},{-74.031}) {\tikz\pgfuseplotmark{m4bv};};
\node[stars] at (axis cs:{156.185},{-73.972}) {\tikz\pgfuseplotmark{m6cv};};
\node[stars] at (axis cs:{156.248},{-58.576}) {\tikz\pgfuseplotmark{m6b};};
\node[stars] at (axis cs:{156.322},{-42.467}) {\tikz\pgfuseplotmark{m6c};};
\node[stars] at (axis cs:{156.541},{-42.738}) {\tikz\pgfuseplotmark{m6b};};
\node[stars] at (axis cs:{156.704},{-54.877}) {\tikz\pgfuseplotmark{m6a};};
\node[stars] at (axis cs:{156.788},{-31.068}) {\tikz\pgfuseplotmark{m4cv};};
\node[stars] at (axis cs:{156.852},{-57.639}) {\tikz\pgfuseplotmark{m5av};};
\node[stars] at (axis cs:{156.855},{-65.705}) {\tikz\pgfuseplotmark{m6b};};
\node[stars] at (axis cs:{156.970},{-58.739}) {\tikz\pgfuseplotmark{m4av};};
\node[stars] at (axis cs:{157.009},{-49.406}) {\tikz\pgfuseplotmark{m6bv};};
\node[stars] at (axis cs:{157.076},{-63.164}) {\tikz\pgfuseplotmark{m6cv};};
\node[stars] at (axis cs:{157.219},{-64.172}) {\tikz\pgfuseplotmark{m5cv};};
\node[stars] at (axis cs:{157.266},{-65.176}) {\tikz\pgfuseplotmark{m6c};};
\node[stars] at (axis cs:{157.371},{-29.664}) {\tikz\pgfuseplotmark{m6av};};
\node[stars] at (axis cs:{157.397},{-30.607}) {\tikz\pgfuseplotmark{m6av};};
\node[stars] at (axis cs:{157.537},{-66.985}) {\tikz\pgfuseplotmark{m6cv};};
\node[stars] at (axis cs:{157.584},{-71.993}) {\tikz\pgfuseplotmark{m5a};};
\node[stars] at (axis cs:{157.664},{-61.356}) {\tikz\pgfuseplotmark{m6cb};};
\node[stars] at (axis cs:{157.759},{-73.221}) {\tikz\pgfuseplotmark{m5bv};};
\node[stars] at (axis cs:{157.841},{-53.715}) {\tikz\pgfuseplotmark{m5bb};};
\node[stars] at (axis cs:{157.953},{-28.237}) {\tikz\pgfuseplotmark{m6b};};
\node[stars] at (axis cs:{157.989},{-45.067}) {\tikz\pgfuseplotmark{m6avb};};
\node[stars] at (axis cs:{158.006},{-61.685}) {\tikz\pgfuseplotmark{m3cv};};
\node[stars] at (axis cs:{158.140},{-44.618}) {\tikz\pgfuseplotmark{m6b};};
\node[stars] at (axis cs:{158.199},{-58.667}) {\tikz\pgfuseplotmark{m6b};};
\node[stars] at (axis cs:{158.237},{-47.003}) {\tikz\pgfuseplotmark{m5bb};};
\node[stars] at (axis cs:{158.356},{-58.190}) {\tikz\pgfuseplotmark{m6cv};};
\node[stars] at (axis cs:{158.553},{-60.988}) {\tikz\pgfuseplotmark{m6cv};};
\node[stars] at (axis cs:{158.793},{-43.664}) {\tikz\pgfuseplotmark{m6b};};
\node[stars] at (axis cs:{158.804},{-39.562}) {\tikz\pgfuseplotmark{m6av};};
\node[stars] at (axis cs:{158.853},{-76.309}) {\tikz\pgfuseplotmark{m6c};};
\node[stars] at (axis cs:{158.867},{-78.608}) {\tikz\pgfuseplotmark{m4bv};};
\node[stars] at (axis cs:{158.897},{-57.558}) {\tikz\pgfuseplotmark{m4cv};};
\node[stars] at (axis cs:{159.085},{-59.564}) {\tikz\pgfuseplotmark{m5b};};
\node[stars] at (axis cs:{159.307},{-27.412}) {\tikz\pgfuseplotmark{m5b};};
\node[stars] at (axis cs:{159.317},{-53.855}) {\tikz\pgfuseplotmark{m6c};};
\node[stars] at (axis cs:{159.326},{-48.225}) {\tikz\pgfuseplotmark{m4a};};
\node[stars] at (axis cs:{159.363},{-58.733}) {\tikz\pgfuseplotmark{m5cv};};
\node[stars] at (axis cs:{159.511},{-57.256}) {\tikz\pgfuseplotmark{m6bv};};
\node[stars] at (axis cs:{159.687},{-59.183}) {\tikz\pgfuseplotmark{m5avb};};
\node[stars] at (axis cs:{159.689},{-59.262}) {\tikz\pgfuseplotmark{m6c};};
\node[stars] at (axis cs:{159.710},{-42.753}) {\tikz\pgfuseplotmark{m6b};};
\node[stars] at (axis cs:{159.748},{-58.817}) {\tikz\pgfuseplotmark{m6bb};};
\node[stars] at (axis cs:{159.816},{-74.493}) {\tikz\pgfuseplotmark{m6b};};
\node[stars] at (axis cs:{159.827},{-55.603}) {\tikz\pgfuseplotmark{m4cb};};
\node[stars] at (axis cs:{159.845},{-64.112}) {\tikz\pgfuseplotmark{m6c};};
\node[stars] at (axis cs:{160.048},{-65.100}) {\tikz\pgfuseplotmark{m6av};};
\node[stars] at (axis cs:{160.215},{-35.742}) {\tikz\pgfuseplotmark{m6c};};
\node[stars] at (axis cs:{160.323},{-59.677}) {\tikz\pgfuseplotmark{m6cv};};
\node[stars] at (axis cs:{160.465},{-79.783}) {\tikz\pgfuseplotmark{m6bv};};
\node[stars] at (axis cs:{160.559},{-64.466}) {\tikz\pgfuseplotmark{m5av};};
\node[stars] at (axis cs:{160.669},{-59.216}) {\tikz\pgfuseplotmark{m5cv};};
\node[stars] at (axis cs:{160.680},{-32.715}) {\tikz\pgfuseplotmark{m6a};};
\node[stars] at (axis cs:{160.739},{-64.394}) {\tikz\pgfuseplotmark{m3a};};
\node[stars] at (axis cs:{160.885},{-60.566}) {\tikz\pgfuseplotmark{m5av};};
\node[stars] at (axis cs:{160.963},{-64.249}) {\tikz\pgfuseplotmark{m6av};};
\node[stars] at (axis cs:{160.968},{-60.118}) {\tikz\pgfuseplotmark{m6c};};
\node[stars] at (axis cs:{161.029},{-63.961}) {\tikz\pgfuseplotmark{m5av};};
\node[stars] at (axis cs:{161.083},{-70.860}) {\tikz\pgfuseplotmark{m6cb};};
\node[stars] at (axis cs:{161.095},{-59.993}) {\tikz\pgfuseplotmark{m6cvb};};
\node[stars] at (axis cs:{161.112},{-72.443}) {\tikz\pgfuseplotmark{m6c};};
\node[stars] at (axis cs:{161.265},{-59.684}) {\tikz\pgfuseplotmark{m6cvb};};
\node[stars] at (axis cs:{161.318},{-80.470}) {\tikz\pgfuseplotmark{m5c};};
\node[stars] at (axis cs:{161.446},{-80.540}) {\tikz\pgfuseplotmark{m4c};};
\node[stars] at (axis cs:{161.569},{-64.514}) {\tikz\pgfuseplotmark{m5cv};};
\node[stars] at (axis cs:{161.571},{-60.603}) {\tikz\pgfuseplotmark{m6c};};
\node[stars] at (axis cs:{161.623},{-64.263}) {\tikz\pgfuseplotmark{m5c};};
\node[stars] at (axis cs:{161.655},{-43.193}) {\tikz\pgfuseplotmark{m6c};};
\node[stars] at (axis cs:{161.692},{-49.420}) {\tikz\pgfuseplotmark{m3ab};};
\node[stars] at (axis cs:{161.700},{-69.210}) {\tikz\pgfuseplotmark{m6c};};
\node[stars] at (axis cs:{161.713},{-64.383}) {\tikz\pgfuseplotmark{m5av};};
\node[stars] at (axis cs:{161.739},{-56.757}) {\tikz\pgfuseplotmark{m5c};};
\node[stars] at (axis cs:{161.878},{-69.438}) {\tikz\pgfuseplotmark{m6c};};
\node[stars] at (axis cs:{161.910},{-57.467}) {\tikz\pgfuseplotmark{m6cv};};
\node[stars] at (axis cs:{161.935},{-59.875}) {\tikz\pgfuseplotmark{m6c};};
\node[stars] at (axis cs:{161.973},{-64.263}) {\tikz\pgfuseplotmark{m6c};};
\node[stars] at (axis cs:{162.023},{-59.919}) {\tikz\pgfuseplotmark{m6bv};};
\node[stars] at (axis cs:{162.059},{-31.688}) {\tikz\pgfuseplotmark{m6b};};
\node[stars] at (axis cs:{162.352},{-59.324}) {\tikz\pgfuseplotmark{m6bb};};
\node[stars] at (axis cs:{162.488},{-34.058}) {\tikz\pgfuseplotmark{m6a};};
\node[stars] at (axis cs:{162.945},{-57.272}) {\tikz\pgfuseplotmark{m6c};};
\node[stars] at (axis cs:{163.119},{-79.559}) {\tikz\pgfuseplotmark{m6c};};
\node[stars] at (axis cs:{163.129},{-57.240}) {\tikz\pgfuseplotmark{m5cv};};
\node[stars] at (axis cs:{163.374},{-58.853}) {\tikz\pgfuseplotmark{m4avb};};
\node[stars] at (axis cs:{163.389},{-56.420}) {\tikz\pgfuseplotmark{m6c};};
\node[stars] at (axis cs:{163.425},{-70.720}) {\tikz\pgfuseplotmark{m6bb};};
\node[stars] at (axis cs:{163.623},{-61.826}) {\tikz\pgfuseplotmark{m6b};};
\node[stars] at (axis cs:{163.754},{-42.251}) {\tikz\pgfuseplotmark{m6bv};};
\node[stars] at (axis cs:{163.822},{-60.517}) {\tikz\pgfuseplotmark{m6b};};
\node[stars] at (axis cs:{163.863},{-43.020}) {\tikz\pgfuseplotmark{m6c};};
\node[stars] at (axis cs:{164.179},{-37.138}) {\tikz\pgfuseplotmark{m5a};};
\node[stars] at (axis cs:{164.283},{-50.765}) {\tikz\pgfuseplotmark{m6bv};};
\node[stars] at (axis cs:{164.316},{-75.099}) {\tikz\pgfuseplotmark{m6b};};
\node[stars] at (axis cs:{164.451},{-59.732}) {\tikz\pgfuseplotmark{m6cv};};
\node[stars] at (axis cs:{164.807},{-33.738}) {\tikz\pgfuseplotmark{m6ab};};
\node[stars] at (axis cs:{164.807},{-84.594}) {\tikz\pgfuseplotmark{m6c};};
\node[stars] at (axis cs:{164.808},{-61.321}) {\tikz\pgfuseplotmark{m6cb};};
\node[stars] at (axis cs:{164.997},{-43.807}) {\tikz\pgfuseplotmark{m6a};};
\node[stars] at (axis cs:{165.034},{-51.818}) {\tikz\pgfuseplotmark{m6c};};
\node[stars] at (axis cs:{165.039},{-42.226}) {\tikz\pgfuseplotmark{m4c};};
\node[stars] at (axis cs:{165.151},{-53.112}) {\tikz\pgfuseplotmark{m6c};};
\node[stars] at (axis cs:{165.170},{-31.839}) {\tikz\pgfuseplotmark{m6b};};
\node[stars] at (axis cs:{165.817},{-31.961}) {\tikz\pgfuseplotmark{m6c};};
\node[stars] at (axis cs:{166.001},{-57.955}) {\tikz\pgfuseplotmark{m6c};};
\node[stars] at (axis cs:{166.130},{-47.679}) {\tikz\pgfuseplotmark{m6a};};
\node[stars] at (axis cs:{166.226},{-35.805}) {\tikz\pgfuseplotmark{m5c};};
\node[stars] at (axis cs:{166.267},{-49.392}) {\tikz\pgfuseplotmark{m6b};};
\node[stars] at (axis cs:{166.333},{-27.293}) {\tikz\pgfuseplotmark{m5bb};};
\node[stars] at (axis cs:{166.490},{-27.288}) {\tikz\pgfuseplotmark{m6av};};
\node[stars] at (axis cs:{166.524},{-51.213}) {\tikz\pgfuseplotmark{m6c};};
\node[stars] at (axis cs:{166.601},{-64.840}) {\tikz\pgfuseplotmark{m6c};};
\node[stars] at (axis cs:{166.615},{-50.957}) {\tikz\pgfuseplotmark{m6c};};
\node[stars] at (axis cs:{166.622},{-58.675}) {\tikz\pgfuseplotmark{m6bv};};
\node[stars] at (axis cs:{166.635},{-62.424}) {\tikz\pgfuseplotmark{m5a};};
\node[stars] at (axis cs:{166.708},{-70.878}) {\tikz\pgfuseplotmark{m6a};};
\node[stars] at (axis cs:{166.820},{-42.639}) {\tikz\pgfuseplotmark{m5cvb};};
\node[stars] at (axis cs:{166.892},{-48.640}) {\tikz\pgfuseplotmark{m6c};};
\node[stars] at (axis cs:{167.066},{-29.972}) {\tikz\pgfuseplotmark{m6c};};
\node[stars] at (axis cs:{167.142},{-61.947}) {\tikz\pgfuseplotmark{m5cv};};
\node[stars] at (axis cs:{167.147},{-58.975}) {\tikz\pgfuseplotmark{m4bv};};
\node[stars] at (axis cs:{167.183},{-28.080}) {\tikz\pgfuseplotmark{m5c};};
\node[stars] at (axis cs:{167.472},{-32.368}) {\tikz\pgfuseplotmark{m6a};};
\node[stars] at (axis cs:{167.873},{-71.436}) {\tikz\pgfuseplotmark{m6c};};
\node[stars] at (axis cs:{168.015},{-58.423}) {\tikz\pgfuseplotmark{m6cv};};
\node[stars] at (axis cs:{168.043},{-46.267}) {\tikz\pgfuseplotmark{m6c};};
\node[stars] at (axis cs:{168.062},{-32.434}) {\tikz\pgfuseplotmark{m6cv};};
\node[stars] at (axis cs:{168.138},{-49.101}) {\tikz\pgfuseplotmark{m5c};};
\node[stars] at (axis cs:{168.150},{-60.318}) {\tikz\pgfuseplotmark{m5av};};
\node[stars] at (axis cs:{168.188},{-64.170}) {\tikz\pgfuseplotmark{m5c};};
\node[stars] at (axis cs:{168.236},{-49.736}) {\tikz\pgfuseplotmark{m6b};};
\node[stars] at (axis cs:{168.311},{-44.372}) {\tikz\pgfuseplotmark{m6a};};
\node[stars] at (axis cs:{168.378},{-59.619}) {\tikz\pgfuseplotmark{m6a};};
\node[stars] at (axis cs:{168.414},{-53.232}) {\tikz\pgfuseplotmark{m6a};};
\node[stars] at (axis cs:{168.726},{-43.734}) {\tikz\pgfuseplotmark{m6c};};
\node[stars] at (axis cs:{168.928},{-57.356}) {\tikz\pgfuseplotmark{m6c};};
\node[stars] at (axis cs:{169.116},{-45.880}) {\tikz\pgfuseplotmark{m6cb};};
\node[stars] at (axis cs:{169.300},{-38.014}) {\tikz\pgfuseplotmark{m6c};};
\node[stars] at (axis cs:{169.310},{-41.934}) {\tikz\pgfuseplotmark{m6c};};
\node[stars] at (axis cs:{169.329},{-67.823}) {\tikz\pgfuseplotmark{m6bv};};
\node[stars] at (axis cs:{169.412},{-34.737}) {\tikz\pgfuseplotmark{m6c};};
\node[stars] at (axis cs:{169.643},{-79.669}) {\tikz\pgfuseplotmark{m6c};};
\node[stars] at (axis cs:{169.818},{-64.582}) {\tikz\pgfuseplotmark{m6b};};
\node[stars] at (axis cs:{169.902},{-75.142}) {\tikz\pgfuseplotmark{m6c};};
\node[stars] at (axis cs:{169.961},{-72.958}) {\tikz\pgfuseplotmark{m6c};};
\node[stars] at (axis cs:{170.017},{-71.994}) {\tikz\pgfuseplotmark{m6c};};
\node[stars] at (axis cs:{170.252},{-54.491}) {\tikz\pgfuseplotmark{m4b};};
\node[stars] at (axis cs:{170.487},{-77.608}) {\tikz\pgfuseplotmark{m6c};};
\node[stars] at (axis cs:{170.596},{-44.646}) {\tikz\pgfuseplotmark{m6b};};
\node[stars] at (axis cs:{170.784},{-56.779}) {\tikz\pgfuseplotmark{m6a};};
\node[stars] at (axis cs:{170.803},{-36.165}) {\tikz\pgfuseplotmark{m5b};};
\node[stars] at (axis cs:{171.046},{-72.256}) {\tikz\pgfuseplotmark{m6a};};
\node[stars] at (axis cs:{171.092},{-42.669}) {\tikz\pgfuseplotmark{m6b};};
\node[stars] at (axis cs:{171.373},{-36.063}) {\tikz\pgfuseplotmark{m5c};};
\node[stars] at (axis cs:{171.388},{-37.747}) {\tikz\pgfuseplotmark{m6b};};
\node[stars] at (axis cs:{171.430},{-63.972}) {\tikz\pgfuseplotmark{m5c};};
\node[stars] at (axis cs:{171.648},{-61.115}) {\tikz\pgfuseplotmark{m5cv};};
\node[stars] at (axis cs:{171.697},{-53.160}) {\tikz\pgfuseplotmark{m6a};};
\node[stars] at (axis cs:{171.994},{-35.329}) {\tikz\pgfuseplotmark{m6cv};};
\node[stars] at (axis cs:{172.077},{-72.474}) {\tikz\pgfuseplotmark{m6b};};
\node[stars] at (axis cs:{172.146},{-42.674}) {\tikz\pgfuseplotmark{m5cb};};
\node[stars] at (axis cs:{172.313},{-63.554}) {\tikz\pgfuseplotmark{m6cv};};
\node[stars] at (axis cs:{172.813},{-61.278}) {\tikz\pgfuseplotmark{m6cv};};
\node[stars] at (axis cs:{172.942},{-59.442}) {\tikz\pgfuseplotmark{m5bv};};
\node[stars] at (axis cs:{172.953},{-59.515}) {\tikz\pgfuseplotmark{m5cv};};
\node[stars] at (axis cs:{173.068},{-29.261}) {\tikz\pgfuseplotmark{m6ab};};
\node[stars] at (axis cs:{173.083},{-66.962}) {\tikz\pgfuseplotmark{m6b};};
\node[stars] at (axis cs:{173.200},{-40.436}) {\tikz\pgfuseplotmark{m6a};};
\node[stars] at (axis cs:{173.226},{-31.087}) {\tikz\pgfuseplotmark{m5cv};};
\node[stars] at (axis cs:{173.250},{-31.858}) {\tikz\pgfuseplotmark{m4a};};
\node[stars] at (axis cs:{173.405},{-40.587}) {\tikz\pgfuseplotmark{m5c};};
\node[stars] at (axis cs:{173.623},{-32.831}) {\tikz\pgfuseplotmark{m6bv};};
\node[stars] at (axis cs:{173.690},{-54.264}) {\tikz\pgfuseplotmark{m5a};};
\node[stars] at (axis cs:{173.737},{-49.136}) {\tikz\pgfuseplotmark{m5c};};
\node[stars] at (axis cs:{173.805},{-47.372}) {\tikz\pgfuseplotmark{m6av};};
\node[stars] at (axis cs:{173.926},{-61.288}) {\tikz\pgfuseplotmark{m6c};};
\node[stars] at (axis cs:{173.945},{-63.020}) {\tikz\pgfuseplotmark{m3b};};
\node[stars] at (axis cs:{173.982},{-47.642}) {\tikz\pgfuseplotmark{m5c};};
\node[stars] at (axis cs:{174.093},{-61.052}) {\tikz\pgfuseplotmark{m6a};};
\node[stars] at (axis cs:{174.146},{-33.570}) {\tikz\pgfuseplotmark{m6ab};};
\node[stars] at (axis cs:{174.170},{-37.237}) {\tikz\pgfuseplotmark{m6c};};
\node[stars] at (axis cs:{174.252},{-61.283}) {\tikz\pgfuseplotmark{m5b};};
\node[stars] at (axis cs:{174.255},{-32.988}) {\tikz\pgfuseplotmark{m6c};};
\node[stars] at (axis cs:{174.315},{-75.896}) {\tikz\pgfuseplotmark{m6a};};
\node[stars] at (axis cs:{174.392},{-47.747}) {\tikz\pgfuseplotmark{m5c};};
\node[stars] at (axis cs:{174.452},{-67.620}) {\tikz\pgfuseplotmark{m6b};};
\node[stars] at (axis cs:{174.530},{-61.826}) {\tikz\pgfuseplotmark{m5c};};
\node[stars] at (axis cs:{174.873},{-65.398}) {\tikz\pgfuseplotmark{m5bv};};
\node[stars] at (axis cs:{175.053},{-34.744}) {\tikz\pgfuseplotmark{m5a};};
\node[stars] at (axis cs:{175.177},{-53.969}) {\tikz\pgfuseplotmark{m6bv};};
\node[stars] at (axis cs:{175.223},{-62.090}) {\tikz\pgfuseplotmark{m5bv};};
\node[stars] at (axis cs:{175.255},{-83.100}) {\tikz\pgfuseplotmark{m6cb};};
\node[stars] at (axis cs:{175.285},{-29.196}) {\tikz\pgfuseplotmark{m6c};};
\node[stars] at (axis cs:{175.332},{-43.095}) {\tikz\pgfuseplotmark{m6a};};
\node[stars] at (axis cs:{175.433},{-32.499}) {\tikz\pgfuseplotmark{m5cb};};
\node[stars] at (axis cs:{175.562},{-75.227}) {\tikz\pgfuseplotmark{m6c};};
\node[stars] at (axis cs:{175.729},{-79.306}) {\tikz\pgfuseplotmark{m6c};};
\node[stars] at (axis cs:{175.863},{-37.190}) {\tikz\pgfuseplotmark{m6b};};
\node[stars] at (axis cs:{175.880},{-62.489}) {\tikz\pgfuseplotmark{m5bv};};
\node[stars] at (axis cs:{175.970},{-62.878}) {\tikz\pgfuseplotmark{m6bv};};
\node[stars] at (axis cs:{176.303},{-49.070}) {\tikz\pgfuseplotmark{m6c};};
\node[stars] at (axis cs:{176.402},{-66.729}) {\tikz\pgfuseplotmark{m4a};};
\node[stars] at (axis cs:{176.433},{-45.690}) {\tikz\pgfuseplotmark{m5cv};};
\node[stars] at (axis cs:{176.628},{-61.178}) {\tikz\pgfuseplotmark{m4bv};};
\node[stars] at (axis cs:{176.629},{-40.500}) {\tikz\pgfuseplotmark{m5b};};
\node[stars] at (axis cs:{176.779},{-35.907}) {\tikz\pgfuseplotmark{m6c};};
\node[stars] at (axis cs:{176.816},{-30.286}) {\tikz\pgfuseplotmark{m6c};};
\node[stars] at (axis cs:{176.830},{-57.696}) {\tikz\pgfuseplotmark{m5cv};};
\node[stars] at (axis cs:{177.061},{-66.815}) {\tikz\pgfuseplotmark{m5av};};
\node[stars] at (axis cs:{177.421},{-63.788}) {\tikz\pgfuseplotmark{m4cv};};
\node[stars] at (axis cs:{177.486},{-70.226}) {\tikz\pgfuseplotmark{m5b};};
\node[stars] at (axis cs:{177.614},{-62.649}) {\tikz\pgfuseplotmark{m6av};};
\node[stars] at (axis cs:{177.655},{-27.278}) {\tikz\pgfuseplotmark{m6c};};
\node[stars] at (axis cs:{177.786},{-45.173}) {\tikz\pgfuseplotmark{m4c};};
\node[stars] at (axis cs:{177.923},{-30.835}) {\tikz\pgfuseplotmark{m6a};};
\node[stars] at (axis cs:{177.963},{-65.206}) {\tikz\pgfuseplotmark{m5bb};};
\node[stars] at (axis cs:{178.043},{-56.988}) {\tikz\pgfuseplotmark{m6ab};};
\node[stars] at (axis cs:{178.227},{-33.908}) {\tikz\pgfuseplotmark{m4cv};};
\node[stars] at (axis cs:{178.362},{-35.066}) {\tikz\pgfuseplotmark{m6c};};
\node[stars] at (axis cs:{178.548},{-57.410}) {\tikz\pgfuseplotmark{m6b};};
\node[stars] at (axis cs:{178.608},{-37.749}) {\tikz\pgfuseplotmark{m6c};};
\node[stars] at (axis cs:{178.686},{-66.376}) {\tikz\pgfuseplotmark{m6c};};
\node[stars] at (axis cs:{178.750},{-63.279}) {\tikz\pgfuseplotmark{m6bv};};
\node[stars] at (axis cs:{178.917},{-28.477}) {\tikz\pgfuseplotmark{m6b};};
\node[stars] at (axis cs:{178.978},{-39.689}) {\tikz\pgfuseplotmark{m6b};};
\node[stars] at (axis cs:{179.183},{-47.072}) {\tikz\pgfuseplotmark{m6c};};
\node[stars] at (axis cs:{179.266},{-33.315}) {\tikz\pgfuseplotmark{m6cv};};
\node[stars] at (axis cs:{179.417},{-62.448}) {\tikz\pgfuseplotmark{m6av};};
\node[stars] at (axis cs:{179.563},{-56.317}) {\tikz\pgfuseplotmark{m5cv};};
\node[stars] at (axis cs:{179.699},{-64.339}) {\tikz\pgfuseplotmark{m6a};};
\node[stars] at (axis cs:{179.795},{-45.832}) {\tikz\pgfuseplotmark{m6c};};
\node[stars] at (axis cs:{179.795},{-51.697}) {\tikz\pgfuseplotmark{m6b};};
\node[stars] at (axis cs:{179.857},{-62.831}) {\tikz\pgfuseplotmark{m6c};};
\node[stars] at (axis cs:{179.907},{-78.222}) {\tikz\pgfuseplotmark{m5bb};};
\node[stars] at (axis cs:{180.372},{-57.504}) {\tikz\pgfuseplotmark{m6c};};
\node[stars] at (axis cs:{180.584},{-85.632}) {\tikz\pgfuseplotmark{m6bb};};
\node[stars] at (axis cs:{180.619},{-71.489}) {\tikz\pgfuseplotmark{m6c};};
\node[stars] at (axis cs:{180.657},{-69.192}) {\tikz\pgfuseplotmark{m6b};};
\node[stars] at (axis cs:{180.756},{-63.313}) {\tikz\pgfuseplotmark{m4c};};
\node[stars] at (axis cs:{180.915},{-42.434}) {\tikz\pgfuseplotmark{m5c};};
\node[stars] at (axis cs:{180.935},{-74.214}) {\tikz\pgfuseplotmark{m6c};};
\node[stars] at (axis cs:{181.070},{-51.472}) {\tikz\pgfuseplotmark{m6c};};
\node[stars] at (axis cs:{181.080},{-63.165}) {\tikz\pgfuseplotmark{m5av};};
\node[stars] at (axis cs:{181.162},{-68.329}) {\tikz\pgfuseplotmark{m5c};};
\node[stars] at (axis cs:{181.189},{-59.253}) {\tikz\pgfuseplotmark{m6c};};
\node[stars] at (axis cs:{181.194},{-76.519}) {\tikz\pgfuseplotmark{m5b};};
\node[stars] at (axis cs:{181.238},{-60.968}) {\tikz\pgfuseplotmark{m6b};};
\node[stars] at (axis cs:{181.473},{-65.547}) {\tikz\pgfuseplotmark{m6c};};
\node[stars] at (axis cs:{181.486},{-35.694}) {\tikz\pgfuseplotmark{m6cv};};
\node[stars] at (axis cs:{181.583},{-68.651}) {\tikz\pgfuseplotmark{m6c};};
\node[stars] at (axis cs:{181.596},{-65.709}) {\tikz\pgfuseplotmark{m6bb};};
\node[stars] at (axis cs:{181.720},{-64.614}) {\tikz\pgfuseplotmark{m4c};};
\node[stars] at (axis cs:{181.958},{-75.367}) {\tikz\pgfuseplotmark{m5c};};
\node[stars] at (axis cs:{182.061},{-48.692}) {\tikz\pgfuseplotmark{m5c};};
\node[stars] at (axis cs:{182.090},{-50.722}) {\tikz\pgfuseplotmark{m3avb};};
\node[stars] at (axis cs:{182.103},{-60.847}) {\tikz\pgfuseplotmark{m6c};};
\node[stars] at (axis cs:{182.224},{-44.326}) {\tikz\pgfuseplotmark{m6av};};
\node[stars] at (axis cs:{182.227},{-41.231}) {\tikz\pgfuseplotmark{m6av};};
\node[stars] at (axis cs:{182.511},{-34.705}) {\tikz\pgfuseplotmark{m6cb};};
\node[stars] at (axis cs:{182.641},{-37.870}) {\tikz\pgfuseplotmark{m6b};};
\node[stars] at (axis cs:{182.755},{-68.261}) {\tikz\pgfuseplotmark{m6c};};
\node[stars] at (axis cs:{182.771},{-61.277}) {\tikz\pgfuseplotmark{m6b};};
\node[stars] at (axis cs:{182.881},{-51.359}) {\tikz\pgfuseplotmark{m6c};};
\node[stars] at (axis cs:{182.913},{-52.368}) {\tikz\pgfuseplotmark{m4b};};
\node[stars] at (axis cs:{183.092},{-62.951}) {\tikz\pgfuseplotmark{m6b};};
\node[stars] at (axis cs:{183.196},{-70.152}) {\tikz\pgfuseplotmark{m6bv};};
\node[stars] at (axis cs:{183.304},{-34.125}) {\tikz\pgfuseplotmark{m6cv};};
\node[stars] at (axis cs:{183.354},{-38.929}) {\tikz\pgfuseplotmark{m6a};};
\node[stars] at (axis cs:{183.403},{-33.793}) {\tikz\pgfuseplotmark{m6cb};};
\node[stars] at (axis cs:{183.485},{-78.574}) {\tikz\pgfuseplotmark{m6c};};
\node[stars] at (axis cs:{183.511},{-45.724}) {\tikz\pgfuseplotmark{m5cb};};
\node[stars] at (axis cs:{183.571},{-64.408}) {\tikz\pgfuseplotmark{m6cv};};
\node[stars] at (axis cs:{183.786},{-58.749}) {\tikz\pgfuseplotmark{m3av};};
\node[stars] at (axis cs:{183.877},{-41.913}) {\tikz\pgfuseplotmark{m6c};};
\node[stars] at (axis cs:{184.099},{-72.614}) {\tikz\pgfuseplotmark{m6c};};
\node[stars] at (axis cs:{184.276},{-65.693}) {\tikz\pgfuseplotmark{m6b};};
\node[stars] at (axis cs:{184.393},{-67.961}) {\tikz\pgfuseplotmark{m4bv};};
\node[stars] at (axis cs:{184.447},{-36.094}) {\tikz\pgfuseplotmark{m6c};};
\node[stars] at (axis cs:{184.587},{-79.312}) {\tikz\pgfuseplotmark{m4cv};};
\node[stars] at (axis cs:{184.609},{-64.003}) {\tikz\pgfuseplotmark{m4b};};
\node[stars] at (axis cs:{184.749},{-55.143}) {\tikz\pgfuseplotmark{m5bv};};
\node[stars] at (axis cs:{185.118},{-65.843}) {\tikz\pgfuseplotmark{m6c};};
\node[stars] at (axis cs:{185.340},{-60.401}) {\tikz\pgfuseplotmark{m4av};};
\node[stars] at (axis cs:{185.489},{-56.374}) {\tikz\pgfuseplotmark{m6b};};
\node[stars] at (axis cs:{185.531},{-67.522}) {\tikz\pgfuseplotmark{m5c};};
\node[stars] at (axis cs:{185.550},{-68.307}) {\tikz\pgfuseplotmark{m6a};};
\node[stars] at (axis cs:{185.706},{-57.676}) {\tikz\pgfuseplotmark{m5cv};};
\node[stars] at (axis cs:{185.808},{-67.631}) {\tikz\pgfuseplotmark{m6c};};
\node[stars] at (axis cs:{185.898},{-35.412}) {\tikz\pgfuseplotmark{m5c};};
\node[stars] at (axis cs:{185.903},{-39.302}) {\tikz\pgfuseplotmark{m6c};};
\node[stars] at (axis cs:{185.936},{-38.911}) {\tikz\pgfuseplotmark{m6a};};
\node[stars] at (axis cs:{185.949},{-30.335}) {\tikz\pgfuseplotmark{m6c};};
\node[stars] at (axis cs:{186.186},{-41.384}) {\tikz\pgfuseplotmark{m6cb};};
\node[stars] at (axis cs:{186.285},{-42.514}) {\tikz\pgfuseplotmark{m6b};};
\node[stars] at (axis cs:{186.323},{-65.770}) {\tikz\pgfuseplotmark{m6c};};
\node[stars] at (axis cs:{186.327},{-27.749}) {\tikz\pgfuseplotmark{m6b};};
\node[stars] at (axis cs:{186.341},{-35.186}) {\tikz\pgfuseplotmark{m6av};};
\node[stars] at (axis cs:{186.409},{-86.151}) {\tikz\pgfuseplotmark{m6c};};
\node[stars] at (axis cs:{186.632},{-51.451}) {\tikz\pgfuseplotmark{m5a};};
\node[stars] at (axis cs:{186.650},{-63.099}) {\tikz\pgfuseplotmark{m1cb};};
\node[stars] at (axis cs:{186.701},{-48.913}) {\tikz\pgfuseplotmark{m6c};};
\node[stars] at (axis cs:{186.715},{-32.830}) {\tikz\pgfuseplotmark{m6a};};
\node[stars] at (axis cs:{186.853},{-63.789}) {\tikz\pgfuseplotmark{m6b};};
\node[stars] at (axis cs:{186.870},{-58.992}) {\tikz\pgfuseplotmark{m5cv};};
\node[stars] at (axis cs:{187.010},{-50.230}) {\tikz\pgfuseplotmark{m4b};};
\node[stars] at (axis cs:{187.080},{-64.341}) {\tikz\pgfuseplotmark{m6b};};
\node[stars] at (axis cs:{187.094},{-39.041}) {\tikz\pgfuseplotmark{m5c};};
\node[stars] at (axis cs:{187.107},{-61.795}) {\tikz\pgfuseplotmark{m6c};};
\node[stars] at (axis cs:{187.140},{-56.408}) {\tikz\pgfuseplotmark{m6c};};
\node[stars] at (axis cs:{187.476},{-56.525}) {\tikz\pgfuseplotmark{m6a};};
\node[stars] at (axis cs:{187.491},{-41.736}) {\tikz\pgfuseplotmark{m6bv};};
\node[stars] at (axis cs:{187.791},{-57.113}) {\tikz\pgfuseplotmark{m2av};};
\node[stars] at (axis cs:{187.918},{-59.424}) {\tikz\pgfuseplotmark{m5cv};};
\node[stars] at (axis cs:{187.984},{-63.506}) {\tikz\pgfuseplotmark{m6b};};
\node[stars] at (axis cs:{188.018},{-32.534}) {\tikz\pgfuseplotmark{m6c};};
\node[stars] at (axis cs:{188.042},{-73.001}) {\tikz\pgfuseplotmark{m6b};};
\node[stars] at (axis cs:{188.117},{-72.133}) {\tikz\pgfuseplotmark{m4av};};
\node[stars] at (axis cs:{188.496},{-49.910}) {\tikz\pgfuseplotmark{m6c};};
\node[stars] at (axis cs:{188.677},{-44.673}) {\tikz\pgfuseplotmark{m6a};};
\node[stars] at (axis cs:{188.727},{-67.757}) {\tikz\pgfuseplotmark{m6cv};};
\node[stars] at (axis cs:{188.873},{-61.841}) {\tikz\pgfuseplotmark{m6c};};
\node[stars] at (axis cs:{188.940},{-41.022}) {\tikz\pgfuseplotmark{m5bv};};
\node[stars] at (axis cs:{189.004},{-39.869}) {\tikz\pgfuseplotmark{m6a};};
\node[stars] at (axis cs:{189.194},{-50.335}) {\tikz\pgfuseplotmark{m6c};};
\node[stars] at (axis cs:{189.296},{-69.136}) {\tikz\pgfuseplotmark{m3av};};
\node[stars] at (axis cs:{189.426},{-48.541}) {\tikz\pgfuseplotmark{m4a};};
\node[stars] at (axis cs:{189.426},{-27.139}) {\tikz\pgfuseplotmark{m5c};};
\node[stars] at (axis cs:{189.718},{-67.193}) {\tikz\pgfuseplotmark{m6cv};};
\node[stars] at (axis cs:{189.764},{-30.422}) {\tikz\pgfuseplotmark{m6a};};
\node[stars] at (axis cs:{189.811},{-75.371}) {\tikz\pgfuseplotmark{m6cb};};
\node[stars] at (axis cs:{189.969},{-39.987}) {\tikz\pgfuseplotmark{m5av};};
\node[stars] at (axis cs:{189.983},{-66.511}) {\tikz\pgfuseplotmark{m6cv};};
\node[stars] at (axis cs:{190.346},{-46.145}) {\tikz\pgfuseplotmark{m6a};};
\node[stars] at (axis cs:{190.379},{-48.960}) {\tikz\pgfuseplotmark{m2cb};};
\node[stars] at (axis cs:{190.486},{-59.686}) {\tikz\pgfuseplotmark{m5bv};};
\node[stars] at (axis cs:{190.521},{-69.407}) {\tikz\pgfuseplotmark{m6cv};};
\node[stars] at (axis cs:{190.648},{-48.813}) {\tikz\pgfuseplotmark{m5a};};
\node[stars] at (axis cs:{190.707},{-55.947}) {\tikz\pgfuseplotmark{m6b};};
\node[stars] at (axis cs:{190.709},{-63.058}) {\tikz\pgfuseplotmark{m5cv};};
\node[stars] at (axis cs:{190.788},{-56.176}) {\tikz\pgfuseplotmark{m6b};};
\node[stars] at (axis cs:{190.859},{-40.178}) {\tikz\pgfuseplotmark{m6c};};
\node[stars] at (axis cs:{190.868},{-58.903}) {\tikz\pgfuseplotmark{m6cb};};
\node[stars] at (axis cs:{190.994},{-36.349}) {\tikz\pgfuseplotmark{m6c};};
\node[stars] at (axis cs:{191.002},{-28.324}) {\tikz\pgfuseplotmark{m5c};};
\node[stars] at (axis cs:{191.259},{-68.831}) {\tikz\pgfuseplotmark{m6cv};};
\node[stars] at (axis cs:{191.409},{-60.981}) {\tikz\pgfuseplotmark{m5a};};
\node[stars] at (axis cs:{191.570},{-68.108}) {\tikz\pgfuseplotmark{m3bb};};
\node[stars] at (axis cs:{191.595},{-56.489}) {\tikz\pgfuseplotmark{m5av};};
\node[stars] at (axis cs:{191.693},{-33.315}) {\tikz\pgfuseplotmark{m6b};};
\node[stars] at (axis cs:{191.930},{-59.689}) {\tikz\pgfuseplotmark{m1cv};};
\node[stars] at (axis cs:{192.109},{-27.597}) {\tikz\pgfuseplotmark{m6a};};
\node[stars] at (axis cs:{192.437},{-71.986}) {\tikz\pgfuseplotmark{m6a};};
\node[stars] at (axis cs:{192.581},{-48.459}) {\tikz\pgfuseplotmark{m6c};};
\node[stars] at (axis cs:{192.672},{-33.999}) {\tikz\pgfuseplotmark{m5b};};
\node[stars] at (axis cs:{192.741},{-52.787}) {\tikz\pgfuseplotmark{m6a};};
\node[stars] at (axis cs:{192.825},{-60.330}) {\tikz\pgfuseplotmark{m6av};};
\node[stars] at (axis cs:{192.987},{-39.680}) {\tikz\pgfuseplotmark{m6bv};};
\node[stars] at (axis cs:{193.023},{-48.094}) {\tikz\pgfuseplotmark{m6c};};
\node[stars] at (axis cs:{193.102},{-53.830}) {\tikz\pgfuseplotmark{m6c};};
\node[stars] at (axis cs:{193.267},{-54.952}) {\tikz\pgfuseplotmark{m6b};};
\node[stars] at (axis cs:{193.279},{-48.943}) {\tikz\pgfuseplotmark{m4c};};
\node[stars] at (axis cs:{193.341},{-60.328}) {\tikz\pgfuseplotmark{m6av};};
\node[stars] at (axis cs:{193.359},{-40.179}) {\tikz\pgfuseplotmark{m4c};};
\node[stars] at (axis cs:{193.454},{-60.376}) {\tikz\pgfuseplotmark{m6b};};
\node[stars] at (axis cs:{193.648},{-57.178}) {\tikz\pgfuseplotmark{m4b};};
\node[stars] at (axis cs:{193.663},{-59.146}) {\tikz\pgfuseplotmark{m5av};};
\node[stars] at (axis cs:{193.744},{-44.152}) {\tikz\pgfuseplotmark{m6b};};
\node[stars] at (axis cs:{193.745},{-85.123}) {\tikz\pgfuseplotmark{m5c};};
\node[stars] at (axis cs:{193.831},{-42.915}) {\tikz\pgfuseplotmark{m5c};};
\node[stars] at (axis cs:{193.988},{-56.836}) {\tikz\pgfuseplotmark{m5cv};};
\node[stars] at (axis cs:{194.132},{-72.185}) {\tikz\pgfuseplotmark{m6b};};
\node[stars] at (axis cs:{194.243},{-54.587}) {\tikz\pgfuseplotmark{m6cv};};
\node[stars] at (axis cs:{194.268},{-51.198}) {\tikz\pgfuseplotmark{m5cv};};
\node[stars] at (axis cs:{195.136},{-33.505}) {\tikz\pgfuseplotmark{m6b};};
\node[stars] at (axis cs:{195.568},{-71.549}) {\tikz\pgfuseplotmark{m4a};};
\node[stars] at (axis cs:{195.772},{-71.476}) {\tikz\pgfuseplotmark{m6bv};};
\node[stars] at (axis cs:{195.889},{-49.527}) {\tikz\pgfuseplotmark{m5a};};
\node[stars] at (axis cs:{196.201},{-41.196}) {\tikz\pgfuseplotmark{m6cv};};
\node[stars] at (axis cs:{196.278},{-47.117}) {\tikz\pgfuseplotmark{m6cv};};
\node[stars] at (axis cs:{196.378},{-52.115}) {\tikz\pgfuseplotmark{m6cv};};
\node[stars] at (axis cs:{196.570},{-48.463}) {\tikz\pgfuseplotmark{m5ab};};
\node[stars] at (axis cs:{196.646},{-41.588}) {\tikz\pgfuseplotmark{m6a};};
\node[stars] at (axis cs:{196.726},{-35.862}) {\tikz\pgfuseplotmark{m6a};};
\node[stars] at (axis cs:{196.728},{-49.906}) {\tikz\pgfuseplotmark{m4c};};
\node[stars] at (axis cs:{196.851},{-59.860}) {\tikz\pgfuseplotmark{m6b};};
\node[stars] at (axis cs:{196.910},{-53.460}) {\tikz\pgfuseplotmark{m6a};};
\node[stars] at (axis cs:{197.030},{-65.306}) {\tikz\pgfuseplotmark{m6avb};};
\node[stars] at (axis cs:{197.085},{-78.446}) {\tikz\pgfuseplotmark{m6c};};
\node[stars] at (axis cs:{197.116},{-67.797}) {\tikz\pgfuseplotmark{m6c};};
\node[stars] at (axis cs:{197.743},{-52.567}) {\tikz\pgfuseplotmark{m6bv};};
\node[stars] at (axis cs:{197.787},{-42.233}) {\tikz\pgfuseplotmark{m6a};};
\node[stars] at (axis cs:{197.847},{-43.369}) {\tikz\pgfuseplotmark{m5c};};
\node[stars] at (axis cs:{197.964},{-69.942}) {\tikz\pgfuseplotmark{m6b};};
\node[stars] at (axis cs:{197.971},{-63.303}) {\tikz\pgfuseplotmark{m6c};};
\node[stars] at (axis cs:{198.013},{-37.803}) {\tikz\pgfuseplotmark{m5a};};
\node[stars] at (axis cs:{198.073},{-59.921}) {\tikz\pgfuseplotmark{m5avb};};
\node[stars] at (axis cs:{198.203},{-66.227}) {\tikz\pgfuseplotmark{m6b};};
\node[stars] at (axis cs:{198.212},{-42.699}) {\tikz\pgfuseplotmark{m6cv};};
\node[stars] at (axis cs:{198.232},{-59.816}) {\tikz\pgfuseplotmark{m6cb};};
\node[stars] at (axis cs:{198.347},{-50.700}) {\tikz\pgfuseplotmark{m6b};};
\node[stars] at (axis cs:{198.490},{-43.139}) {\tikz\pgfuseplotmark{m6c};};
\node[stars] at (axis cs:{198.550},{-58.684}) {\tikz\pgfuseplotmark{m6b};};
\node[stars] at (axis cs:{198.563},{-59.103}) {\tikz\pgfuseplotmark{m5b};};
\node[stars] at (axis cs:{198.572},{-78.447}) {\tikz\pgfuseplotmark{m6a};};
\node[stars] at (axis cs:{198.680},{-48.957}) {\tikz\pgfuseplotmark{m6a};};
\node[stars] at (axis cs:{198.789},{-36.371}) {\tikz\pgfuseplotmark{m6c};};
\node[stars] at (axis cs:{198.812},{-67.894}) {\tikz\pgfuseplotmark{m5avb};};
\node[stars] at (axis cs:{198.857},{-69.679}) {\tikz\pgfuseplotmark{m6c};};
\node[stars] at (axis cs:{199.187},{-65.138}) {\tikz\pgfuseplotmark{m6b};};
\node[stars] at (axis cs:{199.221},{-31.506}) {\tikz\pgfuseplotmark{m5b};};
\node[stars] at (axis cs:{199.304},{-66.783}) {\tikz\pgfuseplotmark{m5av};};
\node[stars] at (axis cs:{199.308},{-43.979}) {\tikz\pgfuseplotmark{m6a};};
\node[stars] at (axis cs:{199.645},{-51.286}) {\tikz\pgfuseplotmark{m6c};};
\node[stars] at (axis cs:{199.701},{-79.976}) {\tikz\pgfuseplotmark{m6c};};
\node[stars] at (axis cs:{199.829},{-72.036}) {\tikz\pgfuseplotmark{m6b};};
\node[stars] at (axis cs:{200.146},{-59.773}) {\tikz\pgfuseplotmark{m6c};};
\node[stars] at (axis cs:{200.149},{-36.712}) {\tikz\pgfuseplotmark{m3a};};
\node[stars] at (axis cs:{200.158},{-52.748}) {\tikz\pgfuseplotmark{m5cv};};
\node[stars] at (axis cs:{200.201},{-55.801}) {\tikz\pgfuseplotmark{m6bv};};
\node[stars] at (axis cs:{200.240},{-46.880}) {\tikz\pgfuseplotmark{m6a};};
\node[stars] at (axis cs:{200.568},{-52.183}) {\tikz\pgfuseplotmark{m6a};};
\node[stars] at (axis cs:{200.658},{-60.988}) {\tikz\pgfuseplotmark{m5ab};};
\node[stars] at (axis cs:{200.719},{-47.943}) {\tikz\pgfuseplotmark{m6c};};
\node[stars] at (axis cs:{200.719},{-72.146}) {\tikz\pgfuseplotmark{m6b};};
\node[stars] at (axis cs:{200.761},{-48.563}) {\tikz\pgfuseplotmark{m6c};};
\node[stars] at (axis cs:{200.786},{-33.190}) {\tikz\pgfuseplotmark{m6cv};};
\node[stars] at (axis cs:{200.967},{-49.823}) {\tikz\pgfuseplotmark{m6c};};
\node[stars] at (axis cs:{201.002},{-64.536}) {\tikz\pgfuseplotmark{m5a};};
\node[stars] at (axis cs:{201.280},{-74.888}) {\tikz\pgfuseplotmark{m5b};};
\node[stars] at (axis cs:{201.308},{-64.485}) {\tikz\pgfuseplotmark{m5c};};
\node[stars] at (axis cs:{201.460},{-70.627}) {\tikz\pgfuseplotmark{m6a};};
\node[stars] at (axis cs:{201.532},{-39.755}) {\tikz\pgfuseplotmark{m5b};};
\node[stars] at (axis cs:{201.734},{-41.497}) {\tikz\pgfuseplotmark{m6a};};
\node[stars] at (axis cs:{201.777},{-49.144}) {\tikz\pgfuseplotmark{m6c};};
\node[stars] at (axis cs:{201.811},{-40.162}) {\tikz\pgfuseplotmark{m6c};};
\node[stars] at (axis cs:{201.836},{-49.381}) {\tikz\pgfuseplotmark{m6c};};
\node[stars] at (axis cs:{202.195},{-69.627}) {\tikz\pgfuseplotmark{m6cv};};
\node[stars] at (axis cs:{202.283},{-64.676}) {\tikz\pgfuseplotmark{m6b};};
\node[stars] at (axis cs:{202.355},{-51.165}) {\tikz\pgfuseplotmark{m5b};};
\node[stars] at (axis cs:{202.761},{-39.407}) {\tikz\pgfuseplotmark{m4b};};
\node[stars] at (axis cs:{202.888},{-28.113}) {\tikz\pgfuseplotmark{m6c};};
\node[stars] at (axis cs:{203.023},{-38.399}) {\tikz\pgfuseplotmark{m6c};};
\node[stars] at (axis cs:{203.144},{-29.565}) {\tikz\pgfuseplotmark{m6c};};
\node[stars] at (axis cs:{203.150},{-28.693}) {\tikz\pgfuseplotmark{m6a};};
\node[stars] at (axis cs:{203.399},{-65.632}) {\tikz\pgfuseplotmark{m6c};};
\node[stars] at (axis cs:{203.621},{-48.272}) {\tikz\pgfuseplotmark{m6c};};
\node[stars] at (axis cs:{203.682},{-33.311}) {\tikz\pgfuseplotmark{m6c};};
\node[stars] at (axis cs:{204.211},{-34.468}) {\tikz\pgfuseplotmark{m6c};};
\node[stars] at (axis cs:{204.275},{-44.143}) {\tikz\pgfuseplotmark{m6b};};
\node[stars] at (axis cs:{204.302},{-61.692}) {\tikz\pgfuseplotmark{m6a};};
\node[stars] at (axis cs:{204.348},{-46.428}) {\tikz\pgfuseplotmark{m6b};};
\node[stars] at (axis cs:{204.532},{-58.414}) {\tikz\pgfuseplotmark{m6cb};};
\node[stars] at (axis cs:{204.675},{-29.561}) {\tikz\pgfuseplotmark{m6a};};
\node[stars] at (axis cs:{204.690},{-70.445}) {\tikz\pgfuseplotmark{m6b};};
\node[stars] at (axis cs:{204.704},{-57.623}) {\tikz\pgfuseplotmark{m6b};};
\node[stars] at (axis cs:{204.800},{-75.684}) {\tikz\pgfuseplotmark{m6cv};};
\node[stars] at (axis cs:{204.920},{-39.747}) {\tikz\pgfuseplotmark{m6cv};};
\node[stars] at (axis cs:{204.952},{-40.052}) {\tikz\pgfuseplotmark{m6a};};
\node[stars] at (axis cs:{204.972},{-53.466}) {\tikz\pgfuseplotmark{m2cv};};
\node[stars] at (axis cs:{204.999},{-49.950}) {\tikz\pgfuseplotmark{m6av};};
\node[stars] at (axis cs:{205.045},{-64.576}) {\tikz\pgfuseplotmark{m6a};};
\node[stars] at (axis cs:{205.231},{-85.786}) {\tikz\pgfuseplotmark{m6a};};
\node[stars] at (axis cs:{205.437},{-54.559}) {\tikz\pgfuseplotmark{m5cb};};
\node[stars] at (axis cs:{205.505},{-58.787}) {\tikz\pgfuseplotmark{m5cv};};
\node[stars] at (axis cs:{205.728},{-50.790}) {\tikz\pgfuseplotmark{m6c};};
\node[stars] at (axis cs:{205.729},{-41.401}) {\tikz\pgfuseplotmark{m6b};};
\node[stars] at (axis cs:{205.734},{-56.768}) {\tikz\pgfuseplotmark{m6bv};};
\node[stars] at (axis cs:{205.917},{-42.068}) {\tikz\pgfuseplotmark{m6b};};
\node[stars] at (axis cs:{206.067},{-51.012}) {\tikz\pgfuseplotmark{m6cv};};
\node[stars] at (axis cs:{206.422},{-33.044}) {\tikz\pgfuseplotmark{m4cv};};
\node[stars] at (axis cs:{206.560},{-54.683}) {\tikz\pgfuseplotmark{m6c};};
\node[stars] at (axis cs:{206.664},{-51.433}) {\tikz\pgfuseplotmark{m5a};};
\node[stars] at (axis cs:{206.735},{-36.252}) {\tikz\pgfuseplotmark{m5c};};
\node[stars] at (axis cs:{206.865},{-50.249}) {\tikz\pgfuseplotmark{m6b};};
\node[stars] at (axis cs:{206.911},{-50.321}) {\tikz\pgfuseplotmark{m5c};};
\node[stars] at (axis cs:{207.361},{-34.451}) {\tikz\pgfuseplotmark{m4cv};};
\node[stars] at (axis cs:{207.376},{-41.688}) {\tikz\pgfuseplotmark{m3cv};};
\node[stars] at (axis cs:{207.404},{-42.474}) {\tikz\pgfuseplotmark{m3cv};};
\node[stars] at (axis cs:{207.527},{-29.081}) {\tikz\pgfuseplotmark{m6cv};};
\node[stars] at (axis cs:{207.581},{-39.901}) {\tikz\pgfuseplotmark{m6c};};
\node[stars] at (axis cs:{207.902},{-36.433}) {\tikz\pgfuseplotmark{m6c};};
\node[stars] at (axis cs:{207.947},{-46.898}) {\tikz\pgfuseplotmark{m6a};};
\node[stars] at (axis cs:{207.948},{-69.401}) {\tikz\pgfuseplotmark{m6a};};
\node[stars] at (axis cs:{207.957},{-32.994}) {\tikz\pgfuseplotmark{m5avb};};
\node[stars] at (axis cs:{208.004},{-31.619}) {\tikz\pgfuseplotmark{m6bb};};
\node[stars] at (axis cs:{208.020},{-52.811}) {\tikz\pgfuseplotmark{m5cb};};
\node[stars] at (axis cs:{208.302},{-31.927}) {\tikz\pgfuseplotmark{m5ab};};
\node[stars] at (axis cs:{208.387},{-35.664}) {\tikz\pgfuseplotmark{m6ab};};
\node[stars] at (axis cs:{208.430},{-53.373}) {\tikz\pgfuseplotmark{m6b};};
\node[stars] at (axis cs:{208.467},{-35.314}) {\tikz\pgfuseplotmark{m6c};};
\node[stars] at (axis cs:{208.489},{-47.128}) {\tikz\pgfuseplotmark{m6av};};
\node[stars] at (axis cs:{208.569},{-28.570}) {\tikz\pgfuseplotmark{m6bv};};
\node[stars] at (axis cs:{208.705},{-67.652}) {\tikz\pgfuseplotmark{m6a};};
\node[stars] at (axis cs:{208.801},{-52.161}) {\tikz\pgfuseplotmark{m6a};};
\node[stars] at (axis cs:{208.885},{-47.288}) {\tikz\pgfuseplotmark{m3a};};
\node[stars] at (axis cs:{208.912},{-82.666}) {\tikz\pgfuseplotmark{m6b};};
\node[stars] at (axis cs:{209.082},{-46.593}) {\tikz\pgfuseplotmark{m6a};};
\node[stars] at (axis cs:{209.083},{-54.132}) {\tikz\pgfuseplotmark{m6cb};};
\node[stars] at (axis cs:{209.137},{-54.705}) {\tikz\pgfuseplotmark{m6b};};
\node[stars] at (axis cs:{209.412},{-63.687}) {\tikz\pgfuseplotmark{m5a};};
\node[stars] at (axis cs:{209.568},{-42.101}) {\tikz\pgfuseplotmark{m4av};};
\node[stars] at (axis cs:{209.630},{-65.801}) {\tikz\pgfuseplotmark{m6cb};};
\node[stars] at (axis cs:{209.670},{-44.804}) {\tikz\pgfuseplotmark{m4av};};
\node[stars] at (axis cs:{209.823},{-50.370}) {\tikz\pgfuseplotmark{m6b};};
\node[stars] at (axis cs:{210.072},{-61.481}) {\tikz\pgfuseplotmark{m6c};};
\node[stars] at (axis cs:{210.137},{-78.590}) {\tikz\pgfuseplotmark{m6b};};
\node[stars] at (axis cs:{210.219},{-66.269}) {\tikz\pgfuseplotmark{m6b};};
\node[stars] at (axis cs:{210.329},{-40.222}) {\tikz\pgfuseplotmark{m6b};};
\node[stars] at (axis cs:{210.431},{-45.603}) {\tikz\pgfuseplotmark{m4c};};
\node[stars] at (axis cs:{210.595},{-27.430}) {\tikz\pgfuseplotmark{m5c};};
\node[stars] at (axis cs:{210.757},{-31.684}) {\tikz\pgfuseplotmark{m6cb};};
\node[stars] at (axis cs:{210.860},{-56.213}) {\tikz\pgfuseplotmark{m6b};};
\node[stars] at (axis cs:{210.864},{-41.423}) {\tikz\pgfuseplotmark{m6bv};};
\node[stars] at (axis cs:{210.956},{-60.373}) {\tikz\pgfuseplotmark{m1bv};};
\node[stars] at (axis cs:{211.333},{-76.797}) {\tikz\pgfuseplotmark{m6av};};
\node[stars] at (axis cs:{211.444},{-54.669}) {\tikz\pgfuseplotmark{m6cv};};
\node[stars] at (axis cs:{211.512},{-41.179}) {\tikz\pgfuseplotmark{m4cv};};
\node[stars] at (axis cs:{211.545},{-43.092}) {\tikz\pgfuseplotmark{m6c};};
\node[stars] at (axis cs:{211.605},{-59.716}) {\tikz\pgfuseplotmark{m6cv};};
\node[stars] at (axis cs:{211.671},{-36.370}) {\tikz\pgfuseplotmark{m2b};};
\node[stars] at (axis cs:{211.920},{-48.704}) {\tikz\pgfuseplotmark{m6c};};
\node[stars] at (axis cs:{211.923},{-49.868}) {\tikz\pgfuseplotmark{m6cb};};
\node[stars] at (axis cs:{212.060},{-63.208}) {\tikz\pgfuseplotmark{m6c};};
\node[stars] at (axis cs:{212.113},{-74.850}) {\tikz\pgfuseplotmark{m6b};};
\node[stars] at (axis cs:{212.216},{-43.471}) {\tikz\pgfuseplotmark{m6cb};};
\node[stars] at (axis cs:{212.234},{-59.277}) {\tikz\pgfuseplotmark{m6cv};};
\node[stars] at (axis cs:{212.396},{-51.505}) {\tikz\pgfuseplotmark{m6bvb};};
\node[stars] at (axis cs:{212.478},{-53.439}) {\tikz\pgfuseplotmark{m5a};};
\node[stars] at (axis cs:{212.629},{-70.305}) {\tikz\pgfuseplotmark{m6b};};
\node[stars] at (axis cs:{212.758},{-69.719}) {\tikz\pgfuseplotmark{m6b};};
\node[stars] at (axis cs:{213.192},{-27.261}) {\tikz\pgfuseplotmark{m5b};};
\node[stars] at (axis cs:{213.318},{-53.665}) {\tikz\pgfuseplotmark{m6a};};
\node[stars] at (axis cs:{213.416},{-54.625}) {\tikz\pgfuseplotmark{m6bv};};
\node[stars] at (axis cs:{213.678},{-41.837}) {\tikz\pgfuseplotmark{m6a};};
\node[stars] at (axis cs:{213.738},{-57.086}) {\tikz\pgfuseplotmark{m5bv};};
\node[stars] at (axis cs:{213.755},{-29.282}) {\tikz\pgfuseplotmark{m6b};};
\node[stars] at (axis cs:{213.839},{-53.510}) {\tikz\pgfuseplotmark{m6c};};
\node[stars] at (axis cs:{213.911},{-45.001}) {\tikz\pgfuseplotmark{m6c};};
\node[stars] at (axis cs:{214.161},{-66.588}) {\tikz\pgfuseplotmark{m6a};};
\node[stars] at (axis cs:{214.230},{-77.664}) {\tikz\pgfuseplotmark{m6cv};};
\node[stars] at (axis cs:{214.558},{-81.008}) {\tikz\pgfuseplotmark{m5b};};
\node[stars] at (axis cs:{214.780},{-27.141}) {\tikz\pgfuseplotmark{m6c};};
\node[stars] at (axis cs:{214.849},{-37.003}) {\tikz\pgfuseplotmark{m6b};};
\node[stars] at (axis cs:{214.851},{-46.058}) {\tikz\pgfuseplotmark{m4av};};
\node[stars] at (axis cs:{214.965},{-61.273}) {\tikz\pgfuseplotmark{m5c};};
\node[stars] at (axis cs:{215.040},{-43.059}) {\tikz\pgfuseplotmark{m6ab};};
\node[stars] at (axis cs:{215.081},{-56.386}) {\tikz\pgfuseplotmark{m4c};};
\node[stars] at (axis cs:{215.139},{-37.885}) {\tikz\pgfuseplotmark{m4bv};};
\node[stars] at (axis cs:{215.177},{-45.187}) {\tikz\pgfuseplotmark{m5a};};
\node[stars] at (axis cs:{215.582},{-34.787}) {\tikz\pgfuseplotmark{m6a};};
\node[stars] at (axis cs:{215.596},{-80.109}) {\tikz\pgfuseplotmark{m5bv};};
\node[stars] at (axis cs:{215.654},{-58.459}) {\tikz\pgfuseplotmark{m5bb};};
\node[stars] at (axis cs:{215.661},{-48.320}) {\tikz\pgfuseplotmark{m6bvb};};
\node[stars] at (axis cs:{215.759},{-39.512}) {\tikz\pgfuseplotmark{m4cv};};
\node[stars] at (axis cs:{215.774},{-27.754}) {\tikz\pgfuseplotmark{m5av};};
\node[stars] at (axis cs:{215.814},{-47.417}) {\tikz\pgfuseplotmark{m6c};};
\node[stars] at (axis cs:{215.834},{-50.772}) {\tikz\pgfuseplotmark{m6bb};};
\node[stars] at (axis cs:{215.952},{-53.176}) {\tikz\pgfuseplotmark{m6b};};
\node[stars] at (axis cs:{216.092},{-82.848}) {\tikz\pgfuseplotmark{m6c};};
\node[stars] at (axis cs:{216.170},{-44.311}) {\tikz\pgfuseplotmark{m6c};};
\node[stars] at (axis cs:{216.227},{-46.529}) {\tikz\pgfuseplotmark{m6c};};
\node[stars] at (axis cs:{216.276},{-68.195}) {\tikz\pgfuseplotmark{m6a};};
\node[stars] at (axis cs:{216.416},{-66.173}) {\tikz\pgfuseplotmark{m6c};};
\node[stars] at (axis cs:{216.534},{-45.221}) {\tikz\pgfuseplotmark{m5av};};
\node[stars] at (axis cs:{216.545},{-45.379}) {\tikz\pgfuseplotmark{m4c};};
\node[stars] at (axis cs:{216.556},{-42.319}) {\tikz\pgfuseplotmark{m6c};};
\node[stars] at (axis cs:{216.708},{-39.874}) {\tikz\pgfuseplotmark{m6c};};
\node[stars] at (axis cs:{216.730},{-83.668}) {\tikz\pgfuseplotmark{m4cv};};
\node[stars] at (axis cs:{216.780},{-65.822}) {\tikz\pgfuseplotmark{m6a};};
\node[stars] at (axis cs:{216.801},{-46.134}) {\tikz\pgfuseplotmark{m6a};};
\node[stars] at (axis cs:{217.043},{-29.492}) {\tikz\pgfuseplotmark{m5bv};};
\node[stars] at (axis cs:{217.181},{-59.197}) {\tikz\pgfuseplotmark{m6c};};
\node[stars] at (axis cs:{217.216},{-47.992}) {\tikz\pgfuseplotmark{m6cv};};
\node[stars] at (axis cs:{217.404},{-76.729}) {\tikz\pgfuseplotmark{m6b};};
\node[stars] at (axis cs:{217.536},{-45.321}) {\tikz\pgfuseplotmark{m6a};};
\node[stars] at (axis cs:{217.587},{-49.519}) {\tikz\pgfuseplotmark{m5c};};
\node[stars] at (axis cs:{217.736},{-40.845}) {\tikz\pgfuseplotmark{m6c};};
\node[stars] at (axis cs:{217.795},{-38.870}) {\tikz\pgfuseplotmark{m6b};};
\node[stars] at (axis cs:{217.819},{-67.717}) {\tikz\pgfuseplotmark{m6a};};
\node[stars] at (axis cs:{218.154},{-50.457}) {\tikz\pgfuseplotmark{m4cv};};
\node[stars] at (axis cs:{218.290},{-30.715}) {\tikz\pgfuseplotmark{m6bb};};
\node[stars] at (axis cs:{218.375},{-52.679}) {\tikz\pgfuseplotmark{m6b};};
\node[stars] at (axis cs:{218.385},{-54.998}) {\tikz\pgfuseplotmark{m6a};};
\node[stars] at (axis cs:{218.821},{-60.015}) {\tikz\pgfuseplotmark{m6c};};
\node[stars] at (axis cs:{218.877},{-42.158}) {\tikz\pgfuseplotmark{m2cv};};
\node[stars] at (axis cs:{218.881},{-41.517}) {\tikz\pgfuseplotmark{m6b};};
\node[stars] at (axis cs:{219.079},{-46.245}) {\tikz\pgfuseplotmark{m6avb};};
\node[stars] at (axis cs:{219.100},{-39.597}) {\tikz\pgfuseplotmark{m6c};};
\node[stars] at (axis cs:{219.184},{-40.211}) {\tikz\pgfuseplotmark{m6av};};
\node[stars] at (axis cs:{219.302},{-53.061}) {\tikz\pgfuseplotmark{m6c};};
\node[stars] at (axis cs:{219.334},{-46.133}) {\tikz\pgfuseplotmark{m5cb};};
\node[stars] at (axis cs:{219.443},{-67.932}) {\tikz\pgfuseplotmark{m6b};};
\node[stars] at (axis cs:{219.472},{-49.426}) {\tikz\pgfuseplotmark{m4bv};};
\node[stars] at (axis cs:{219.581},{-38.794}) {\tikz\pgfuseplotmark{m6b};};
\node[stars] at (axis cs:{219.795},{-46.584}) {\tikz\pgfuseplotmark{m6b};};
\node[stars] at (axis cs:{219.852},{-49.055}) {\tikz\pgfuseplotmark{m6c};};
\node[stars] at (axis cs:{219.902},{-60.834}) {\tikz\pgfuseplotmark{m1bvb};};
\node[stars] at (axis cs:{220.136},{-56.441}) {\tikz\pgfuseplotmark{m6c};};
\node[stars] at (axis cs:{220.256},{-36.135}) {\tikz\pgfuseplotmark{m6a};};
\node[stars] at (axis cs:{220.463},{-30.933}) {\tikz\pgfuseplotmark{m6cb};};
\node[stars] at (axis cs:{220.482},{-47.388}) {\tikz\pgfuseplotmark{m2cv};};
\node[stars] at (axis cs:{220.483},{-58.616}) {\tikz\pgfuseplotmark{m6c};};
\node[stars] at (axis cs:{220.490},{-37.793}) {\tikz\pgfuseplotmark{m4b};};
\node[stars] at (axis cs:{220.499},{-77.011}) {\tikz\pgfuseplotmark{m6c};};
\node[stars] at (axis cs:{220.627},{-64.975}) {\tikz\pgfuseplotmark{m3cv};};
\node[stars] at (axis cs:{220.867},{-62.967}) {\tikz\pgfuseplotmark{m6c};};
\node[stars] at (axis cs:{220.914},{-35.174}) {\tikz\pgfuseplotmark{m4b};};
\node[stars] at (axis cs:{221.231},{-58.477}) {\tikz\pgfuseplotmark{m6b};};
\node[stars] at (axis cs:{221.247},{-35.192}) {\tikz\pgfuseplotmark{m5b};};
\node[stars] at (axis cs:{221.296},{-55.602}) {\tikz\pgfuseplotmark{m6bvb};};
\node[stars] at (axis cs:{221.322},{-62.876}) {\tikz\pgfuseplotmark{m5c};};
\node[stars] at (axis cs:{221.328},{-51.315}) {\tikz\pgfuseplotmark{m6c};};
\node[stars] at (axis cs:{221.621},{-47.441}) {\tikz\pgfuseplotmark{m6a};};
\node[stars] at (axis cs:{221.755},{-52.384}) {\tikz\pgfuseplotmark{m5c};};
\node[stars] at (axis cs:{221.771},{-38.291}) {\tikz\pgfuseplotmark{m6b};};
\node[stars] at (axis cs:{221.802},{-52.205}) {\tikz\pgfuseplotmark{m6b};};
\node[stars] at (axis cs:{221.884},{-43.557}) {\tikz\pgfuseplotmark{m6c};};
\node[stars] at (axis cs:{221.965},{-79.045}) {\tikz\pgfuseplotmark{m4a};};
\node[stars] at (axis cs:{222.159},{-36.635}) {\tikz\pgfuseplotmark{m6bv};};
\node[stars] at (axis cs:{222.186},{-66.593}) {\tikz\pgfuseplotmark{m6bvb};};
\node[stars] at (axis cs:{222.279},{-56.668}) {\tikz\pgfuseplotmark{m6c};};
\node[stars] at (axis cs:{222.572},{-27.960}) {\tikz\pgfuseplotmark{m4c};};
\node[stars] at (axis cs:{222.875},{-77.176}) {\tikz\pgfuseplotmark{m6c};};
\node[stars] at (axis cs:{222.910},{-43.575}) {\tikz\pgfuseplotmark{m4cv};};
\node[stars] at (axis cs:{223.138},{-30.577}) {\tikz\pgfuseplotmark{m6c};};
\node[stars] at (axis cs:{223.147},{-63.810}) {\tikz\pgfuseplotmark{m6bv};};
\node[stars] at (axis cs:{223.213},{-37.803}) {\tikz\pgfuseplotmark{m5bv};};
\node[stars] at (axis cs:{223.307},{-73.190}) {\tikz\pgfuseplotmark{m6ab};};
\node[stars] at (axis cs:{223.658},{-33.301}) {\tikz\pgfuseplotmark{m6a};};
\node[stars] at (axis cs:{223.677},{-65.991}) {\tikz\pgfuseplotmark{m6b};};
\node[stars] at (axis cs:{223.725},{-69.861}) {\tikz\pgfuseplotmark{m6c};};
\node[stars] at (axis cs:{223.894},{-60.114}) {\tikz\pgfuseplotmark{m5c};};
\node[stars] at (axis cs:{223.936},{-33.856}) {\tikz\pgfuseplotmark{m5c};};
\node[stars] at (axis cs:{224.072},{-52.809}) {\tikz\pgfuseplotmark{m5c};};
\node[stars] at (axis cs:{224.129},{-32.637}) {\tikz\pgfuseplotmark{m6b};};
\node[stars] at (axis cs:{224.134},{-47.879}) {\tikz\pgfuseplotmark{m6ab};};
\node[stars] at (axis cs:{224.149},{-39.416}) {\tikz\pgfuseplotmark{m6c};};
\node[stars] at (axis cs:{224.160},{-62.364}) {\tikz\pgfuseplotmark{m6c};};
\node[stars] at (axis cs:{224.183},{-62.781}) {\tikz\pgfuseplotmark{m5bv};};
\node[stars] at (axis cs:{224.307},{-29.158}) {\tikz\pgfuseplotmark{m6c};};
\node[stars] at (axis cs:{224.471},{-76.662}) {\tikz\pgfuseplotmark{m5c};};
\node[stars] at (axis cs:{224.537},{-48.863}) {\tikz\pgfuseplotmark{m6c};};
\node[stars] at (axis cs:{224.633},{-43.134}) {\tikz\pgfuseplotmark{m3a};};
\node[stars] at (axis cs:{224.652},{-39.907}) {\tikz\pgfuseplotmark{m6c};};
\node[stars] at (axis cs:{224.664},{-27.657}) {\tikz\pgfuseplotmark{m6a};};
\node[stars] at (axis cs:{224.790},{-42.104}) {\tikz\pgfuseplotmark{m3bv};};
\node[stars] at (axis cs:{224.808},{-37.881}) {\tikz\pgfuseplotmark{m6c};};
\node[stars] at (axis cs:{224.863},{-43.160}) {\tikz\pgfuseplotmark{m6b};};
\node[stars] at (axis cs:{224.982},{-75.032}) {\tikz\pgfuseplotmark{m6c};};
\node[stars] at (axis cs:{225.047},{-77.160}) {\tikz\pgfuseplotmark{m6b};};
\node[stars] at (axis cs:{225.305},{-38.058}) {\tikz\pgfuseplotmark{m6b};};
\node[stars] at (axis cs:{225.462},{-83.227}) {\tikz\pgfuseplotmark{m6a};};
\node[stars] at (axis cs:{225.492},{-34.359}) {\tikz\pgfuseplotmark{m6c};};
\node[stars] at (axis cs:{225.527},{-28.060}) {\tikz\pgfuseplotmark{m6a};};
\node[stars] at (axis cs:{225.747},{-32.643}) {\tikz\pgfuseplotmark{m5c};};
\node[stars] at (axis cs:{226.179},{-40.861}) {\tikz\pgfuseplotmark{m6cv};};
\node[stars] at (axis cs:{226.196},{-83.038}) {\tikz\pgfuseplotmark{m6a};};
\node[stars] at (axis cs:{226.201},{-64.031}) {\tikz\pgfuseplotmark{m5c};};
\node[stars] at (axis cs:{226.280},{-47.051}) {\tikz\pgfuseplotmark{m4bb};};
\node[stars] at (axis cs:{226.330},{-41.067}) {\tikz\pgfuseplotmark{m5c};};
\node[stars] at (axis cs:{226.391},{-30.551}) {\tikz\pgfuseplotmark{m6c};};
\node[stars] at (axis cs:{226.510},{-48.883}) {\tikz\pgfuseplotmark{m6c};};
\node[stars] at (axis cs:{226.558},{-36.264}) {\tikz\pgfuseplotmark{m6c};};
\node[stars] at (axis cs:{226.638},{-30.918}) {\tikz\pgfuseplotmark{m6bv};};
\node[stars] at (axis cs:{226.733},{-52.030}) {\tikz\pgfuseplotmark{m6c};};
\node[stars] at (axis cs:{226.858},{-49.089}) {\tikz\pgfuseplotmark{m6a};};
\node[stars] at (axis cs:{226.984},{-61.128}) {\tikz\pgfuseplotmark{m6c};};
\node[stars] at (axis cs:{226.987},{-65.275}) {\tikz\pgfuseplotmark{m6c};};
\node[stars] at (axis cs:{227.051},{-40.584}) {\tikz\pgfuseplotmark{m6av};};
\node[stars] at (axis cs:{227.163},{-42.868}) {\tikz\pgfuseplotmark{m6a};};
\node[stars] at (axis cs:{227.211},{-45.280}) {\tikz\pgfuseplotmark{m4b};};
\node[stars] at (axis cs:{227.356},{-63.643}) {\tikz\pgfuseplotmark{m6c};};
\node[stars] at (axis cs:{227.375},{-67.084}) {\tikz\pgfuseplotmark{m6a};};
\node[stars] at (axis cs:{227.531},{-38.792}) {\tikz\pgfuseplotmark{m6b};};
\node[stars] at (axis cs:{227.686},{-61.422}) {\tikz\pgfuseplotmark{m6c};};
\node[stars] at (axis cs:{227.787},{-84.788}) {\tikz\pgfuseplotmark{m6b};};
\node[stars] at (axis cs:{227.817},{-55.346}) {\tikz\pgfuseplotmark{m5c};};
\node[stars] at (axis cs:{227.895},{-45.277}) {\tikz\pgfuseplotmark{m6cb};};
\node[stars] at (axis cs:{227.984},{-48.738}) {\tikz\pgfuseplotmark{m4ab};};
\node[stars] at (axis cs:{228.071},{-52.099}) {\tikz\pgfuseplotmark{m3cb};};
\node[stars] at (axis cs:{228.130},{-48.219}) {\tikz\pgfuseplotmark{m6c};};
\node[stars] at (axis cs:{228.142},{-72.771}) {\tikz\pgfuseplotmark{m6b};};
\node[stars] at (axis cs:{228.207},{-44.500}) {\tikz\pgfuseplotmark{m5av};};
\node[stars] at (axis cs:{228.254},{-61.744}) {\tikz\pgfuseplotmark{m6c};};
\node[stars] at (axis cs:{228.282},{-36.091}) {\tikz\pgfuseplotmark{m6b};};
\node[stars] at (axis cs:{228.497},{-61.343}) {\tikz\pgfuseplotmark{m6cb};};
\node[stars] at (axis cs:{228.580},{-70.079}) {\tikz\pgfuseplotmark{m6av};};
\node[stars] at (axis cs:{228.655},{-31.519}) {\tikz\pgfuseplotmark{m5b};};
\node[stars] at (axis cs:{228.974},{-48.074}) {\tikz\pgfuseplotmark{m6b};};
\node[stars] at (axis cs:{229.017},{-41.491}) {\tikz\pgfuseplotmark{m5c};};
\node[stars] at (axis cs:{229.044},{-43.485}) {\tikz\pgfuseplotmark{m6b};};
\node[stars] at (axis cs:{229.153},{-60.904}) {\tikz\pgfuseplotmark{m6avb};};
\node[stars] at (axis cs:{229.216},{-47.864}) {\tikz\pgfuseplotmark{m6c};};
\node[stars] at (axis cs:{229.229},{-46.202}) {\tikz\pgfuseplotmark{m6c};};
\node[stars] at (axis cs:{229.237},{-60.957}) {\tikz\pgfuseplotmark{m5bv};};
\node[stars] at (axis cs:{229.379},{-58.801}) {\tikz\pgfuseplotmark{m4b};};
\node[stars] at (axis cs:{229.412},{-63.610}) {\tikz\pgfuseplotmark{m5a};};
\node[stars] at (axis cs:{229.458},{-30.148}) {\tikz\pgfuseplotmark{m4c};};
\node[stars] at (axis cs:{229.539},{-41.061}) {\tikz\pgfuseplotmark{m6c};};
\node[stars] at (axis cs:{229.633},{-47.875}) {\tikz\pgfuseplotmark{m4cb};};
\node[stars] at (axis cs:{229.672},{-31.209}) {\tikz\pgfuseplotmark{m6c};};
\node[stars] at (axis cs:{229.705},{-60.496}) {\tikz\pgfuseplotmark{m5c};};
\node[stars] at (axis cs:{229.727},{-68.679}) {\tikz\pgfuseplotmark{m3b};};
\node[stars] at (axis cs:{229.735},{-40.788}) {\tikz\pgfuseplotmark{m6av};};
\node[stars] at (axis cs:{229.882},{-37.097}) {\tikz\pgfuseplotmark{m6c};};
\node[stars] at (axis cs:{230.169},{-67.481}) {\tikz\pgfuseplotmark{m6cvb};};
\node[stars] at (axis cs:{230.343},{-40.647}) {\tikz\pgfuseplotmark{m3cv};};
\node[stars] at (axis cs:{230.397},{-40.749}) {\tikz\pgfuseplotmark{m6c};};
\node[stars] at (axis cs:{230.451},{-48.318}) {\tikz\pgfuseplotmark{m6a};};
\node[stars] at (axis cs:{230.452},{-36.261}) {\tikz\pgfuseplotmark{m4a};};
\node[stars] at (axis cs:{230.534},{-47.928}) {\tikz\pgfuseplotmark{m5b};};
\node[stars] at (axis cs:{230.670},{-44.690}) {\tikz\pgfuseplotmark{m3cv};};
\node[stars] at (axis cs:{230.789},{-36.858}) {\tikz\pgfuseplotmark{m5a};};
\node[stars] at (axis cs:{230.794},{-60.657}) {\tikz\pgfuseplotmark{m6a};};
\node[stars] at (axis cs:{230.844},{-59.321}) {\tikz\pgfuseplotmark{m5av};};
\node[stars] at (axis cs:{231.188},{-39.710}) {\tikz\pgfuseplotmark{m5c};};
\node[stars] at (axis cs:{231.334},{-38.734}) {\tikz\pgfuseplotmark{m5a};};
\node[stars] at (axis cs:{231.561},{-68.309}) {\tikz\pgfuseplotmark{m6b};};
\node[stars] at (axis cs:{231.826},{-36.768}) {\tikz\pgfuseplotmark{m5c};};
\node[stars] at (axis cs:{231.888},{-64.531}) {\tikz\pgfuseplotmark{m6av};};
\node[stars] at (axis cs:{232.080},{-88.133}) {\tikz\pgfuseplotmark{m6cv};};
\node[stars] at (axis cs:{232.113},{-51.598}) {\tikz\pgfuseplotmark{m6b};};
\node[stars] at (axis cs:{232.330},{-38.635}) {\tikz\pgfuseplotmark{m6cv};};
\node[stars] at (axis cs:{232.351},{-46.732}) {\tikz\pgfuseplotmark{m5c};};
\node[stars] at (axis cs:{232.589},{-41.919}) {\tikz\pgfuseplotmark{m6cb};};
\node[stars] at (axis cs:{232.705},{-71.654}) {\tikz\pgfuseplotmark{m6c};};
\node[stars] at (axis cs:{232.878},{-73.389}) {\tikz\pgfuseplotmark{m5cv};};
\node[stars] at (axis cs:{232.959},{-32.881}) {\tikz\pgfuseplotmark{m6c};};
\node[stars] at (axis cs:{233.017},{-38.622}) {\tikz\pgfuseplotmark{m6c};};
\node[stars] at (axis cs:{233.351},{-40.491}) {\tikz\pgfuseplotmark{m6c};};
\node[stars] at (axis cs:{233.507},{-40.066}) {\tikz\pgfuseplotmark{m6av};};
\node[stars] at (axis cs:{233.587},{-39.349}) {\tikz\pgfuseplotmark{m6c};};
\node[stars] at (axis cs:{233.656},{-28.047}) {\tikz\pgfuseplotmark{m5b};};
\node[stars] at (axis cs:{233.785},{-41.167}) {\tikz\pgfuseplotmark{m3av};};
\node[stars] at (axis cs:{233.972},{-44.958}) {\tikz\pgfuseplotmark{m5avb};};
\node[stars] at (axis cs:{234.047},{-33.093}) {\tikz\pgfuseplotmark{m6c};};
\node[stars] at (axis cs:{234.050},{-44.397}) {\tikz\pgfuseplotmark{m5c};};
\node[stars] at (axis cs:{234.180},{-66.317}) {\tikz\pgfuseplotmark{m4b};};
\node[stars] at (axis cs:{234.256},{-28.135}) {\tikz\pgfuseplotmark{m4a};};
\node[stars] at (axis cs:{234.513},{-42.567}) {\tikz\pgfuseplotmark{m4c};};
\node[stars] at (axis cs:{234.566},{-28.207}) {\tikz\pgfuseplotmark{m6c};};
\node[stars] at (axis cs:{234.664},{-29.778}) {\tikz\pgfuseplotmark{m4a};};
\node[stars] at (axis cs:{234.676},{-39.128}) {\tikz\pgfuseplotmark{m6b};};
\node[stars] at (axis cs:{234.706},{-52.373}) {\tikz\pgfuseplotmark{m5cb};};
\node[stars] at (axis cs:{234.827},{-77.918}) {\tikz\pgfuseplotmark{m6c};};
\node[stars] at (axis cs:{234.942},{-34.412}) {\tikz\pgfuseplotmark{m5a};};
\node[stars] at (axis cs:{234.986},{-59.908}) {\tikz\pgfuseplotmark{m6b};};
\node[stars] at (axis cs:{235.048},{-70.228}) {\tikz\pgfuseplotmark{m6c};};
\node[stars] at (axis cs:{235.065},{-31.214}) {\tikz\pgfuseplotmark{m6c};};
\node[stars] at (axis cs:{235.089},{-73.446}) {\tikz\pgfuseplotmark{m6a};};
\node[stars] at (axis cs:{235.243},{-47.736}) {\tikz\pgfuseplotmark{m6c};};
\node[stars] at (axis cs:{235.297},{-44.661}) {\tikz\pgfuseplotmark{m5a};};
\node[stars] at (axis cs:{235.478},{-76.082}) {\tikz\pgfuseplotmark{m6b};};
\node[stars] at (axis cs:{235.507},{-64.180}) {\tikz\pgfuseplotmark{m6c};};
\node[stars] at (axis cs:{235.655},{-49.489}) {\tikz\pgfuseplotmark{m6b};};
\node[stars] at (axis cs:{235.660},{-37.425}) {\tikz\pgfuseplotmark{m5c};};
\node[stars] at (axis cs:{235.671},{-34.710}) {\tikz\pgfuseplotmark{m5a};};
\node[stars] at (axis cs:{235.821},{-84.465}) {\tikz\pgfuseplotmark{m6a};};
\node[stars] at (axis cs:{235.980},{-60.287}) {\tikz\pgfuseplotmark{m6c};};
\node[stars] at (axis cs:{236.094},{-41.819}) {\tikz\pgfuseplotmark{m6bb};};
\node[stars] at (axis cs:{236.684},{-34.682}) {\tikz\pgfuseplotmark{m6a};};
\node[stars] at (axis cs:{236.856},{-40.194}) {\tikz\pgfuseplotmark{m6c};};
\node[stars] at (axis cs:{236.871},{-37.916}) {\tikz\pgfuseplotmark{m6b};};
\node[stars] at (axis cs:{236.971},{-65.442}) {\tikz\pgfuseplotmark{m6bb};};
\node[stars] at (axis cs:{237.210},{-52.438}) {\tikz\pgfuseplotmark{m6b};};
\node[stars] at (axis cs:{237.490},{-48.912}) {\tikz\pgfuseplotmark{m6a};};
\node[stars] at (axis cs:{237.530},{-53.210}) {\tikz\pgfuseplotmark{m6a};};
\node[stars] at (axis cs:{237.568},{-45.402}) {\tikz\pgfuseplotmark{m6b};};
\node[stars] at (axis cs:{237.740},{-33.627}) {\tikz\pgfuseplotmark{m4b};};
\node[stars] at (axis cs:{237.778},{-55.055}) {\tikz\pgfuseplotmark{m6avb};};
\node[stars] at (axis cs:{237.881},{-47.061}) {\tikz\pgfuseplotmark{m6b};};
\node[stars] at (axis cs:{238.053},{-29.886}) {\tikz\pgfuseplotmark{m6cv};};
\node[stars] at (axis cs:{238.346},{-62.606}) {\tikz\pgfuseplotmark{m6cv};};
\node[stars] at (axis cs:{238.625},{-27.339}) {\tikz\pgfuseplotmark{m6c};};
\node[stars] at (axis cs:{238.719},{-60.743}) {\tikz\pgfuseplotmark{m6cvb};};
\node[stars] at (axis cs:{238.786},{-63.430}) {\tikz\pgfuseplotmark{m3a};};
\node[stars] at (axis cs:{238.873},{-68.603}) {\tikz\pgfuseplotmark{m5b};};
\node[stars] at (axis cs:{238.877},{-31.084}) {\tikz\pgfuseplotmark{m6c};};
\node[stars] at (axis cs:{238.885},{-60.177}) {\tikz\pgfuseplotmark{m6ab};};
\node[stars] at (axis cs:{239.025},{-60.482}) {\tikz\pgfuseplotmark{m6a};};
\node[stars] at (axis cs:{239.028},{-39.864}) {\tikz\pgfuseplotmark{m6b};};
\node[stars] at (axis cs:{239.058},{-31.786}) {\tikz\pgfuseplotmark{m6c};};
\node[stars] at (axis cs:{239.221},{-29.214}) {\tikz\pgfuseplotmark{m4b};};
\node[stars] at (axis cs:{239.223},{-33.966}) {\tikz\pgfuseplotmark{m5bb};};
\node[stars] at (axis cs:{239.266},{-48.162}) {\tikz\pgfuseplotmark{m6c};};
\node[stars] at (axis cs:{239.339},{-36.185}) {\tikz\pgfuseplotmark{m6a};};
\node[stars] at (axis cs:{239.628},{-37.504}) {\tikz\pgfuseplotmark{m6c};};
\node[stars] at (axis cs:{239.742},{-65.038}) {\tikz\pgfuseplotmark{m6a};};
\node[stars] at (axis cs:{239.876},{-41.744}) {\tikz\pgfuseplotmark{m5b};};
\node[stars] at (axis cs:{239.976},{-54.021}) {\tikz\pgfuseplotmark{m6b};};
\node[stars] at (axis cs:{239.980},{-78.027}) {\tikz\pgfuseplotmark{m6c};};
\node[stars] at (axis cs:{239.991},{-40.653}) {\tikz\pgfuseplotmark{m6c};};
\node[stars] at (axis cs:{240.031},{-38.396}) {\tikz\pgfuseplotmark{m3c};};
\node[stars] at (axis cs:{240.224},{-40.435}) {\tikz\pgfuseplotmark{m6c};};
\node[stars] at (axis cs:{240.277},{-54.578}) {\tikz\pgfuseplotmark{m6b};};
\node[stars] at (axis cs:{240.295},{-63.776}) {\tikz\pgfuseplotmark{m6cv};};
\node[stars] at (axis cs:{240.331},{-31.889}) {\tikz\pgfuseplotmark{m6c};};
\node[stars] at (axis cs:{240.519},{-51.121}) {\tikz\pgfuseplotmark{m6c};};
\node[stars] at (axis cs:{240.664},{-29.136}) {\tikz\pgfuseplotmark{m6b};};
\node[stars] at (axis cs:{240.719},{-62.541}) {\tikz\pgfuseplotmark{m6c};};
\node[stars] at (axis cs:{240.804},{-49.229}) {\tikz\pgfuseplotmark{m5a};};
\node[stars] at (axis cs:{240.851},{-38.602}) {\tikz\pgfuseplotmark{m5b};};
\node[stars] at (axis cs:{240.884},{-57.775}) {\tikz\pgfuseplotmark{m5ab};};
\node[stars] at (axis cs:{240.893},{-32.001}) {\tikz\pgfuseplotmark{m6b};};
\node[stars] at (axis cs:{241.026},{-59.176}) {\tikz\pgfuseplotmark{m6c};};
\node[stars] at (axis cs:{241.074},{-33.213}) {\tikz\pgfuseplotmark{m6bv};};
\node[stars] at (axis cs:{241.089},{-53.710}) {\tikz\pgfuseplotmark{m6cv};};
\node[stars] at (axis cs:{241.153},{-37.863}) {\tikz\pgfuseplotmark{m6b};};
\node[stars] at (axis cs:{241.483},{-72.401}) {\tikz\pgfuseplotmark{m6a};};
\node[stars] at (axis cs:{241.623},{-45.173}) {\tikz\pgfuseplotmark{m5a};};
\node[stars] at (axis cs:{241.648},{-36.802}) {\tikz\pgfuseplotmark{m4cv};};
\node[stars] at (axis cs:{241.818},{-36.756}) {\tikz\pgfuseplotmark{m6a};};
\node[stars] at (axis cs:{241.850},{-56.191}) {\tikz\pgfuseplotmark{m6c};};
\node[stars] at (axis cs:{242.327},{-57.934}) {\tikz\pgfuseplotmark{m6a};};
\node[stars] at (axis cs:{242.469},{-33.546}) {\tikz\pgfuseplotmark{m5c};};
\node[stars] at (axis cs:{242.759},{-29.416}) {\tikz\pgfuseplotmark{m5b};};
\node[stars] at (axis cs:{242.824},{-41.120}) {\tikz\pgfuseplotmark{m6a};};
\node[stars] at (axis cs:{242.991},{-54.353}) {\tikz\pgfuseplotmark{m6c};};
\node[stars] at (axis cs:{243.067},{-28.417}) {\tikz\pgfuseplotmark{m6ab};};
\node[stars] at (axis cs:{243.076},{-27.926}) {\tikz\pgfuseplotmark{m5a};};
\node[stars] at (axis cs:{243.321},{-53.671}) {\tikz\pgfuseplotmark{m6bv};};
\node[stars] at (axis cs:{243.345},{-55.541}) {\tikz\pgfuseplotmark{m6a};};
\node[stars] at (axis cs:{243.370},{-54.630}) {\tikz\pgfuseplotmark{m5b};};
\node[stars] at (axis cs:{243.593},{-33.011}) {\tikz\pgfuseplotmark{m6b};};
\node[stars] at (axis cs:{243.814},{-47.372}) {\tikz\pgfuseplotmark{m5b};};
\node[stars] at (axis cs:{243.850},{-42.899}) {\tikz\pgfuseplotmark{m6c};};
\node[stars] at (axis cs:{243.859},{-63.686}) {\tikz\pgfuseplotmark{m4a};};
\node[stars] at (axis cs:{243.957},{-57.912}) {\tikz\pgfuseplotmark{m6av};};
\node[stars] at (axis cs:{244.180},{-53.811}) {\tikz\pgfuseplotmark{m5cv};};
\node[stars] at (axis cs:{244.246},{-53.696}) {\tikz\pgfuseplotmark{m6c};};
\node[stars] at (axis cs:{244.254},{-50.068}) {\tikz\pgfuseplotmark{m5b};};
\node[stars] at (axis cs:{244.273},{-67.941}) {\tikz\pgfuseplotmark{m6bv};};
\node[stars] at (axis cs:{244.337},{-53.086}) {\tikz\pgfuseplotmark{m6c};};
\node[stars] at (axis cs:{244.575},{-28.614}) {\tikz\pgfuseplotmark{m5a};};
\node[stars] at (axis cs:{244.823},{-42.674}) {\tikz\pgfuseplotmark{m5c};};
\node[stars] at (axis cs:{244.886},{-30.907}) {\tikz\pgfuseplotmark{m5cb};};
\node[stars] at (axis cs:{244.960},{-50.155}) {\tikz\pgfuseplotmark{m4b};};
\node[stars] at (axis cs:{245.087},{-78.696}) {\tikz\pgfuseplotmark{m5avb};};
\node[stars] at (axis cs:{245.105},{-55.140}) {\tikz\pgfuseplotmark{m6a};};
\node[stars] at (axis cs:{245.112},{-78.667}) {\tikz\pgfuseplotmark{m5c};};
\node[stars] at (axis cs:{245.136},{-39.430}) {\tikz\pgfuseplotmark{m6b};};
\node[stars] at (axis cs:{245.617},{-49.572}) {\tikz\pgfuseplotmark{m5c};};
\node[stars] at (axis cs:{245.621},{-43.912}) {\tikz\pgfuseplotmark{m6bb};};
\node[stars] at (axis cs:{246.005},{-39.193}) {\tikz\pgfuseplotmark{m5cv};};
\node[stars] at (axis cs:{246.132},{-37.566}) {\tikz\pgfuseplotmark{m5cv};};
\node[stars] at (axis cs:{246.166},{-29.705}) {\tikz\pgfuseplotmark{m5cb};};
\node[stars] at (axis cs:{246.226},{-45.349}) {\tikz\pgfuseplotmark{m6c};};
\node[stars] at (axis cs:{246.342},{-63.125}) {\tikz\pgfuseplotmark{m6c};};
\node[stars] at (axis cs:{246.796},{-47.555}) {\tikz\pgfuseplotmark{m5ab};};
\node[stars] at (axis cs:{246.989},{-64.058}) {\tikz\pgfuseplotmark{m5cv};};
\node[stars] at (axis cs:{247.060},{-37.180}) {\tikz\pgfuseplotmark{m6a};};
\node[stars] at (axis cs:{247.063},{-58.600}) {\tikz\pgfuseplotmark{m6a};};
\node[stars] at (axis cs:{247.117},{-70.084}) {\tikz\pgfuseplotmark{m5b};};
\node[stars] at (axis cs:{247.426},{-46.243}) {\tikz\pgfuseplotmark{m5cv};};
\node[stars] at (axis cs:{247.438},{-57.756}) {\tikz\pgfuseplotmark{m6bv};};
\node[stars] at (axis cs:{247.706},{-61.634}) {\tikz\pgfuseplotmark{m5c};};
\node[stars] at (axis cs:{247.846},{-34.704}) {\tikz\pgfuseplotmark{m4c};};
\node[stars] at (axis cs:{247.924},{-41.817}) {\tikz\pgfuseplotmark{m5cv};};
\node[stars] at (axis cs:{248.363},{-78.897}) {\tikz\pgfuseplotmark{m4b};};
\node[stars] at (axis cs:{248.521},{-44.045}) {\tikz\pgfuseplotmark{m5bv};};
\node[stars] at (axis cs:{248.581},{-70.988}) {\tikz\pgfuseplotmark{m5c};};
\node[stars] at (axis cs:{248.782},{-45.244}) {\tikz\pgfuseplotmark{m6cv};};
\node[stars] at (axis cs:{248.937},{-65.495}) {\tikz\pgfuseplotmark{m5c};};
\node[stars] at (axis cs:{248.971},{-28.216}) {\tikz\pgfuseplotmark{m3a};};
\node[stars] at (axis cs:{249.094},{-35.255}) {\tikz\pgfuseplotmark{m4cv};};
\node[stars] at (axis cs:{249.094},{-42.859}) {\tikz\pgfuseplotmark{m5cv};};
\node[stars] at (axis cs:{249.610},{-43.398}) {\tikz\pgfuseplotmark{m6avb};};
\node[stars] at (axis cs:{249.720},{-60.990}) {\tikz\pgfuseplotmark{m6c};};
\node[stars] at (axis cs:{249.772},{-37.217}) {\tikz\pgfuseplotmark{m6b};};
\node[stars] at (axis cs:{250.185},{-51.478}) {\tikz\pgfuseplotmark{m6cv};};
\node[stars] at (axis cs:{250.210},{-60.446}) {\tikz\pgfuseplotmark{m6cb};};
\node[stars] at (axis cs:{250.335},{-48.763}) {\tikz\pgfuseplotmark{m6avb};};
\node[stars] at (axis cs:{250.346},{-68.296}) {\tikz\pgfuseplotmark{m6b};};
\node[stars] at (axis cs:{250.418},{-49.652}) {\tikz\pgfuseplotmark{m6a};};
\node[stars] at (axis cs:{250.439},{-33.146}) {\tikz\pgfuseplotmark{m6a};};
\node[stars] at (axis cs:{250.702},{-62.554}) {\tikz\pgfuseplotmark{m6c};};
\node[stars] at (axis cs:{250.765},{-46.070}) {\tikz\pgfuseplotmark{m6c};};
\node[stars] at (axis cs:{250.769},{-77.517}) {\tikz\pgfuseplotmark{m4c};};
\node[stars] at (axis cs:{250.842},{-67.432}) {\tikz\pgfuseplotmark{m6bv};};
\node[stars] at (axis cs:{250.911},{-32.106}) {\tikz\pgfuseplotmark{m6c};};
\node[stars] at (axis cs:{250.948},{-38.156}) {\tikz\pgfuseplotmark{m6b};};
\node[stars] at (axis cs:{250.975},{-41.113}) {\tikz\pgfuseplotmark{m6bvb};};
\node[stars] at (axis cs:{251.166},{-53.152}) {\tikz\pgfuseplotmark{m6b};};
\node[stars] at (axis cs:{251.177},{-40.840}) {\tikz\pgfuseplotmark{m6a};};
\node[stars] at (axis cs:{251.251},{-28.509}) {\tikz\pgfuseplotmark{m6bv};};
\node[stars] at (axis cs:{251.527},{-46.532}) {\tikz\pgfuseplotmark{m6c};};
\node[stars] at (axis cs:{251.588},{-58.504}) {\tikz\pgfuseplotmark{m6a};};
\node[stars] at (axis cs:{251.667},{-67.110}) {\tikz\pgfuseplotmark{m5bv};};
\node[stars] at (axis cs:{251.700},{-39.377}) {\tikz\pgfuseplotmark{m5c};};
\node[stars] at (axis cs:{251.832},{-58.341}) {\tikz\pgfuseplotmark{m6a};};
\node[stars] at (axis cs:{252.166},{-69.028}) {\tikz\pgfuseplotmark{m2b};};
\node[stars] at (axis cs:{252.446},{-59.041}) {\tikz\pgfuseplotmark{m4a};};
\node[stars] at (axis cs:{252.541},{-34.293}) {\tikz\pgfuseplotmark{m2cv};};
\node[stars] at (axis cs:{252.750},{-37.514}) {\tikz\pgfuseplotmark{m6cvb};};
\node[stars] at (axis cs:{252.891},{-41.230}) {\tikz\pgfuseplotmark{m5cv};};
\node[stars] at (axis cs:{252.968},{-38.047}) {\tikz\pgfuseplotmark{m3bv};};
\node[stars] at (axis cs:{252.975},{-65.375}) {\tikz\pgfuseplotmark{m6b};};
\node[stars] at (axis cs:{253.074},{-67.681}) {\tikz\pgfuseplotmark{m6c};};
\node[stars] at (axis cs:{253.080},{-41.854}) {\tikz\pgfuseplotmark{m6cv};};
\node[stars] at (axis cs:{253.084},{-38.018}) {\tikz\pgfuseplotmark{m4a};};
\node[stars] at (axis cs:{253.114},{-40.723}) {\tikz\pgfuseplotmark{m6cv};};
\node[stars] at (axis cs:{253.427},{-43.051}) {\tikz\pgfuseplotmark{m6bv};};
\node[stars] at (axis cs:{253.495},{-41.994}) {\tikz\pgfuseplotmark{m6cv};};
\node[stars] at (axis cs:{253.499},{-42.362}) {\tikz\pgfuseplotmark{m5av};};
\node[stars] at (axis cs:{253.501},{-57.909}) {\tikz\pgfuseplotmark{m6b};};
\node[stars] at (axis cs:{253.508},{-41.806}) {\tikz\pgfuseplotmark{m5cb};};
\node[stars] at (axis cs:{253.542},{-41.825}) {\tikz\pgfuseplotmark{m6bvb};};
\node[stars] at (axis cs:{253.548},{-41.849}) {\tikz\pgfuseplotmark{m6cv};};
\node[stars] at (axis cs:{253.612},{-42.479}) {\tikz\pgfuseplotmark{m6a};};
\node[stars] at (axis cs:{253.631},{-49.711}) {\tikz\pgfuseplotmark{m6c};};
\node[stars] at (axis cs:{253.646},{-42.361}) {\tikz\pgfuseplotmark{m4av};};
\node[stars] at (axis cs:{253.650},{-30.587}) {\tikz\pgfuseplotmark{m6cv};};
\node[stars] at (axis cs:{253.744},{-41.151}) {\tikz\pgfuseplotmark{m6av};};
\node[stars] at (axis cs:{253.764},{-42.091}) {\tikz\pgfuseplotmark{m6c};};
\node[stars] at (axis cs:{253.853},{-63.270}) {\tikz\pgfuseplotmark{m6b};};
\node[stars] at (axis cs:{253.991},{-33.507}) {\tikz\pgfuseplotmark{m6c};};
\node[stars] at (axis cs:{254.037},{-50.675}) {\tikz\pgfuseplotmark{m6cv};};
\node[stars] at (axis cs:{254.120},{-52.284}) {\tikz\pgfuseplotmark{m6b};};
\node[stars] at (axis cs:{254.150},{-40.823}) {\tikz\pgfuseplotmark{m6cv};};
\node[stars] at (axis cs:{254.254},{-71.111}) {\tikz\pgfuseplotmark{m6c};};
\node[stars] at (axis cs:{254.297},{-33.259}) {\tikz\pgfuseplotmark{m5c};};
\node[stars] at (axis cs:{254.575},{-50.641}) {\tikz\pgfuseplotmark{m6a};};
\node[stars] at (axis cs:{254.655},{-55.990}) {\tikz\pgfuseplotmark{m3b};};
\node[stars] at (axis cs:{254.718},{-37.620}) {\tikz\pgfuseplotmark{m6bv};};
\node[stars] at (axis cs:{254.891},{-69.268}) {\tikz\pgfuseplotmark{m6av};};
\node[stars] at (axis cs:{254.896},{-53.160}) {\tikz\pgfuseplotmark{m4b};};
\node[stars] at (axis cs:{255.026},{-54.597}) {\tikz\pgfuseplotmark{m6a};};
\node[stars] at (axis cs:{255.112},{-48.648}) {\tikz\pgfuseplotmark{m6b};};
\node[stars] at (axis cs:{255.154},{-35.934}) {\tikz\pgfuseplotmark{m6b};};
\node[stars] at (axis cs:{255.244},{-86.364}) {\tikz\pgfuseplotmark{m6bv};};
\node[stars] at (axis cs:{255.279},{-56.555}) {\tikz\pgfuseplotmark{m6cvb};};
\node[stars] at (axis cs:{255.443},{-51.131}) {\tikz\pgfuseplotmark{m6c};};
\node[stars] at (axis cs:{255.447},{-58.958}) {\tikz\pgfuseplotmark{m6cv};};
\node[stars] at (axis cs:{255.469},{-32.143}) {\tikz\pgfuseplotmark{m5b};};
\node[stars] at (axis cs:{255.786},{-53.237}) {\tikz\pgfuseplotmark{m5c};};
\node[stars] at (axis cs:{255.924},{-47.160}) {\tikz\pgfuseplotmark{m6b};};
\node[stars] at (axis cs:{255.962},{-38.153}) {\tikz\pgfuseplotmark{m6bv};};
\node[stars] at (axis cs:{256.103},{-57.712}) {\tikz\pgfuseplotmark{m6a};};
\node[stars] at (axis cs:{256.206},{-34.123}) {\tikz\pgfuseplotmark{m5av};};
\node[stars] at (axis cs:{256.272},{-45.502}) {\tikz\pgfuseplotmark{m6c};};
\node[stars] at (axis cs:{256.452},{-44.105}) {\tikz\pgfuseplotmark{m6c};};
\node[stars] at (axis cs:{256.584},{-37.227}) {\tikz\pgfuseplotmark{m6b};};
\node[stars] at (axis cs:{256.618},{-35.451}) {\tikz\pgfuseplotmark{m6cv};};
\node[stars] at (axis cs:{257.198},{-30.404}) {\tikz\pgfuseplotmark{m6b};};
\node[stars] at (axis cs:{257.526},{-61.675}) {\tikz\pgfuseplotmark{m6c};};
\node[stars] at (axis cs:{257.676},{-44.558}) {\tikz\pgfuseplotmark{m5b};};
\node[stars] at (axis cs:{257.911},{-48.873}) {\tikz\pgfuseplotmark{m6bv};};
\node[stars] at (axis cs:{258.038},{-43.239}) {\tikz\pgfuseplotmark{m3c};};
\node[stars] at (axis cs:{258.068},{-39.507}) {\tikz\pgfuseplotmark{m6a};};
\node[stars] at (axis cs:{258.069},{-38.822}) {\tikz\pgfuseplotmark{m6c};};
\node[stars] at (axis cs:{258.083},{-70.721}) {\tikz\pgfuseplotmark{m6c};};
\node[stars] at (axis cs:{258.104},{-27.762}) {\tikz\pgfuseplotmark{m6b};};
\node[stars] at (axis cs:{258.244},{-32.438}) {\tikz\pgfuseplotmark{m6b};};
\node[stars] at (axis cs:{258.324},{-67.196}) {\tikz\pgfuseplotmark{m6bb};};
\node[stars] at (axis cs:{258.556},{-56.888}) {\tikz\pgfuseplotmark{m6bv};};
\node[stars] at (axis cs:{258.615},{-39.766}) {\tikz\pgfuseplotmark{m6cvb};};
\node[stars] at (axis cs:{258.830},{-33.548}) {\tikz\pgfuseplotmark{m6av};};
\node[stars] at (axis cs:{258.900},{-38.594}) {\tikz\pgfuseplotmark{m6b};};
\node[stars] at (axis cs:{258.964},{-30.211}) {\tikz\pgfuseplotmark{m6c};};
\node[stars] at (axis cs:{259.090},{-35.750}) {\tikz\pgfuseplotmark{m6b};};
\node[stars] at (axis cs:{259.149},{-74.533}) {\tikz\pgfuseplotmark{m6c};};
\node[stars] at (axis cs:{259.265},{-32.663}) {\tikz\pgfuseplotmark{m6a};};
\node[stars] at (axis cs:{259.273},{-44.778}) {\tikz\pgfuseplotmark{m6c};};
\node[stars] at (axis cs:{259.585},{-32.553}) {\tikz\pgfuseplotmark{m6cv};};
\node[stars] at (axis cs:{259.699},{-44.130}) {\tikz\pgfuseplotmark{m6a};};
\node[stars] at (axis cs:{259.738},{-34.990}) {\tikz\pgfuseplotmark{m6bvb};};
\node[stars] at (axis cs:{259.766},{-46.636}) {\tikz\pgfuseplotmark{m5cb};};
\node[stars] at (axis cs:{259.802},{-59.694}) {\tikz\pgfuseplotmark{m6b};};
\node[stars] at (axis cs:{259.877},{-50.064}) {\tikz\pgfuseplotmark{m6c};};
\node[stars] at (axis cs:{260.498},{-67.771}) {\tikz\pgfuseplotmark{m5a};};
\node[stars] at (axis cs:{260.524},{-70.123}) {\tikz\pgfuseplotmark{m5c};};
\node[stars] at (axis cs:{260.658},{-35.911}) {\tikz\pgfuseplotmark{m6c};};
\node[stars] at (axis cs:{260.663},{-37.805}) {\tikz\pgfuseplotmark{m6cv};};
\node[stars] at (axis cs:{260.728},{-37.221}) {\tikz\pgfuseplotmark{m6b};};
\node[stars] at (axis cs:{260.730},{-58.010}) {\tikz\pgfuseplotmark{m6ab};};
\node[stars] at (axis cs:{260.780},{-56.525}) {\tikz\pgfuseplotmark{m6a};};
\node[stars] at (axis cs:{260.817},{-47.468}) {\tikz\pgfuseplotmark{m5cv};};
\node[stars] at (axis cs:{260.840},{-28.143}) {\tikz\pgfuseplotmark{m5c};};
\node[stars] at (axis cs:{261.004},{-62.864}) {\tikz\pgfuseplotmark{m6a};};
\node[stars] at (axis cs:{261.055},{-44.162}) {\tikz\pgfuseplotmark{m5b};};
\node[stars] at (axis cs:{261.078},{-60.674}) {\tikz\pgfuseplotmark{m6a};};
\node[stars] at (axis cs:{261.261},{-34.696}) {\tikz\pgfuseplotmark{m6c};};
\node[stars] at (axis cs:{261.325},{-55.530}) {\tikz\pgfuseplotmark{m3a};};
\node[stars] at (axis cs:{261.349},{-56.378}) {\tikz\pgfuseplotmark{m3c};};
\node[stars] at (axis cs:{261.500},{-50.634}) {\tikz\pgfuseplotmark{m5c};};
\node[stars] at (axis cs:{261.717},{-45.843}) {\tikz\pgfuseplotmark{m6ab};};
\node[stars] at (axis cs:{261.735},{-51.949}) {\tikz\pgfuseplotmark{m6c};};
\node[stars] at (axis cs:{261.839},{-29.867}) {\tikz\pgfuseplotmark{m4c};};
\node[stars] at (axis cs:{261.906},{-29.724}) {\tikz\pgfuseplotmark{m6b};};
\node[stars] at (axis cs:{261.990},{-52.297}) {\tikz\pgfuseplotmark{m6a};};
\node[stars] at (axis cs:{262.033},{-63.036}) {\tikz\pgfuseplotmark{m6c};};
\node[stars] at (axis cs:{262.162},{-55.170}) {\tikz\pgfuseplotmark{m6b};};
\node[stars] at (axis cs:{262.234},{-36.778}) {\tikz\pgfuseplotmark{m6b};};
\node[stars] at (axis cs:{262.254},{-43.974}) {\tikz\pgfuseplotmark{m6cb};};
\node[stars] at (axis cs:{262.357},{-38.517}) {\tikz\pgfuseplotmark{m6c};};
\node[stars] at (axis cs:{262.691},{-37.296}) {\tikz\pgfuseplotmark{m3a};};
\node[stars] at (axis cs:{262.775},{-60.684}) {\tikz\pgfuseplotmark{m4a};};
\node[stars] at (axis cs:{262.847},{-56.921}) {\tikz\pgfuseplotmark{m6bv};};
\node[stars] at (axis cs:{262.864},{-80.859}) {\tikz\pgfuseplotmark{m6bv};};
\node[stars] at (axis cs:{262.947},{-33.703}) {\tikz\pgfuseplotmark{m6c};};
\node[stars] at (axis cs:{262.955},{-46.037}) {\tikz\pgfuseplotmark{m6b};};
\node[stars] at (axis cs:{262.960},{-49.876}) {\tikz\pgfuseplotmark{m3av};};
\node[stars] at (axis cs:{263.103},{-34.279}) {\tikz\pgfuseplotmark{m6cv};};
\node[stars] at (axis cs:{263.281},{-41.173}) {\tikz\pgfuseplotmark{m6a};};
\node[stars] at (axis cs:{263.402},{-37.104}) {\tikz\pgfuseplotmark{m2av};};
\node[stars] at (axis cs:{263.570},{-48.527}) {\tikz\pgfuseplotmark{m6c};};
\node[stars] at (axis cs:{263.677},{-32.581}) {\tikz\pgfuseplotmark{m6av};};
\node[stars] at (axis cs:{263.833},{-53.353}) {\tikz\pgfuseplotmark{m6b};};
\node[stars] at (axis cs:{263.896},{-59.846}) {\tikz\pgfuseplotmark{m6c};};
\node[stars] at (axis cs:{263.915},{-46.506}) {\tikz\pgfuseplotmark{m5a};};
\node[stars] at (axis cs:{263.929},{-37.440}) {\tikz\pgfuseplotmark{m6c};};
\node[stars] at (axis cs:{264.137},{-38.635}) {\tikz\pgfuseplotmark{m4c};};
\node[stars] at (axis cs:{264.330},{-42.998}) {\tikz\pgfuseplotmark{m2a};};
\node[stars] at (axis cs:{264.362},{-38.066}) {\tikz\pgfuseplotmark{m6c};};
\node[stars] at (axis cs:{264.364},{-50.060}) {\tikz\pgfuseplotmark{m6b};};
\node[stars] at (axis cs:{264.523},{-54.500}) {\tikz\pgfuseplotmark{m5c};};
\node[stars] at (axis cs:{264.535},{-42.880}) {\tikz\pgfuseplotmark{m6b};};
\node[stars] at (axis cs:{265.099},{-49.415}) {\tikz\pgfuseplotmark{m5a};};
\node[stars] at (axis cs:{265.185},{-67.854}) {\tikz\pgfuseplotmark{m6c};};
\node[stars] at (axis cs:{265.244},{-32.214}) {\tikz\pgfuseplotmark{m6bv};};
\node[stars] at (axis cs:{265.318},{-46.922}) {\tikz\pgfuseplotmark{m6a};};
\node[stars] at (axis cs:{265.515},{-50.511}) {\tikz\pgfuseplotmark{m6cv};};
\node[stars] at (axis cs:{265.622},{-39.030}) {\tikz\pgfuseplotmark{m2cv};};
\node[stars] at (axis cs:{265.713},{-36.945}) {\tikz\pgfuseplotmark{m6a};};
\node[stars] at (axis cs:{265.779},{-33.051}) {\tikz\pgfuseplotmark{m6c};};
\node[stars] at (axis cs:{265.824},{-27.884}) {\tikz\pgfuseplotmark{m6c};};
\node[stars] at (axis cs:{266.036},{-51.834}) {\tikz\pgfuseplotmark{m5b};};
\node[stars] at (axis cs:{266.175},{-42.729}) {\tikz\pgfuseplotmark{m6b};};
\node[stars] at (axis cs:{266.233},{-57.545}) {\tikz\pgfuseplotmark{m6b};};
\node[stars] at (axis cs:{266.433},{-64.724}) {\tikz\pgfuseplotmark{m4a};};
\node[stars] at (axis cs:{266.780},{-38.112}) {\tikz\pgfuseplotmark{m6c};};
\node[stars] at (axis cs:{266.890},{-27.831}) {\tikz\pgfuseplotmark{m5av};};
\node[stars] at (axis cs:{266.896},{-40.127}) {\tikz\pgfuseplotmark{m3b};};
\node[stars] at (axis cs:{267.159},{-55.402}) {\tikz\pgfuseplotmark{m6b};};
\node[stars] at (axis cs:{267.294},{-31.703}) {\tikz\pgfuseplotmark{m5a};};
\node[stars] at (axis cs:{267.465},{-37.043}) {\tikz\pgfuseplotmark{m3c};};
\node[stars] at (axis cs:{267.471},{-61.714}) {\tikz\pgfuseplotmark{m6c};};
\node[stars] at (axis cs:{267.546},{-40.090}) {\tikz\pgfuseplotmark{m5a};};
\node[stars] at (axis cs:{267.618},{-53.612}) {\tikz\pgfuseplotmark{m6avb};};
\node[stars] at (axis cs:{267.796},{-53.130}) {\tikz\pgfuseplotmark{m6b};};
\node[stars] at (axis cs:{267.886},{-40.772}) {\tikz\pgfuseplotmark{m6bv};};
\node[stars] at (axis cs:{267.898},{-60.164}) {\tikz\pgfuseplotmark{m6ab};};
\node[stars] at (axis cs:{267.936},{-45.601}) {\tikz\pgfuseplotmark{m6b};};
\node[stars] at (axis cs:{268.057},{-34.799}) {\tikz\pgfuseplotmark{m6bv};};
\node[stars] at (axis cs:{268.082},{-34.417}) {\tikz\pgfuseplotmark{m6a};};
\node[stars] at (axis cs:{268.205},{-34.115}) {\tikz\pgfuseplotmark{m6b};};
\node[stars] at (axis cs:{268.219},{-41.996}) {\tikz\pgfuseplotmark{m6c};};
\node[stars] at (axis cs:{268.240},{-35.624}) {\tikz\pgfuseplotmark{m6b};};
\node[stars] at (axis cs:{268.326},{-65.489}) {\tikz\pgfuseplotmark{m6c};};
\node[stars] at (axis cs:{268.332},{-34.731}) {\tikz\pgfuseplotmark{m6b};};
\node[stars] at (axis cs:{268.348},{-34.895}) {\tikz\pgfuseplotmark{m6a};};
\node[stars] at (axis cs:{268.440},{-34.786}) {\tikz\pgfuseplotmark{m6c};};
\node[stars] at (axis cs:{268.478},{-34.753}) {\tikz\pgfuseplotmark{m6bv};};
\node[stars] at (axis cs:{268.492},{-34.831}) {\tikz\pgfuseplotmark{m6cv};};
\node[stars] at (axis cs:{268.613},{-34.467}) {\tikz\pgfuseplotmark{m6b};};
\node[stars] at (axis cs:{268.783},{-36.475}) {\tikz\pgfuseplotmark{m6b};};
\node[stars] at (axis cs:{269.101},{-65.722}) {\tikz\pgfuseplotmark{m6c};};
\node[stars] at (axis cs:{269.174},{-28.065}) {\tikz\pgfuseplotmark{m6a};};
\node[stars] at (axis cs:{269.198},{-44.342}) {\tikz\pgfuseplotmark{m5a};};
\node[stars] at (axis cs:{269.233},{-40.306}) {\tikz\pgfuseplotmark{m6c};};
\node[stars] at (axis cs:{269.424},{-76.178}) {\tikz\pgfuseplotmark{m6b};};
\node[stars] at (axis cs:{269.428},{-56.896}) {\tikz\pgfuseplotmark{m6c};};
\node[stars] at (axis cs:{269.449},{-41.716}) {\tikz\pgfuseplotmark{m5b};};
\node[stars] at (axis cs:{269.491},{-39.136}) {\tikz\pgfuseplotmark{m6cb};};
\node[stars] at (axis cs:{269.663},{-28.759}) {\tikz\pgfuseplotmark{m6b};};
\node[stars] at (axis cs:{269.732},{-36.858}) {\tikz\pgfuseplotmark{m6ab};};
\node[stars] at (axis cs:{269.772},{-30.253}) {\tikz\pgfuseplotmark{m5cb};};
\node[stars] at (axis cs:{270.392},{-85.215}) {\tikz\pgfuseplotmark{m6c};};
\node[stars] at (axis cs:{270.451},{-36.378}) {\tikz\pgfuseplotmark{m6c};};
\node[stars] at (axis cs:{271.210},{-35.901}) {\tikz\pgfuseplotmark{m6b};};
\node[stars] at (axis cs:{271.255},{-29.580}) {\tikz\pgfuseplotmark{m5av};};
\node[stars] at (axis cs:{271.362},{-81.487}) {\tikz\pgfuseplotmark{m6c};};
\node[stars] at (axis cs:{271.452},{-30.424}) {\tikz\pgfuseplotmark{m4a};};
\node[stars] at (axis cs:{271.538},{-51.744}) {\tikz\pgfuseplotmark{m6c};};
\node[stars] at (axis cs:{271.599},{-36.020}) {\tikz\pgfuseplotmark{m6b};};
\node[stars] at (axis cs:{271.658},{-50.091}) {\tikz\pgfuseplotmark{m4a};};
\node[stars] at (axis cs:{271.708},{-43.425}) {\tikz\pgfuseplotmark{m6ab};};
\node[stars] at (axis cs:{271.952},{-64.550}) {\tikz\pgfuseplotmark{m6c};};
\node[stars] at (axis cs:{272.021},{-28.457}) {\tikz\pgfuseplotmark{m5a};};
\node[stars] at (axis cs:{272.125},{-45.767}) {\tikz\pgfuseplotmark{m6cb};};
\node[stars] at (axis cs:{272.145},{-63.669}) {\tikz\pgfuseplotmark{m4c};};
\node[stars] at (axis cs:{272.490},{-59.040}) {\tikz\pgfuseplotmark{m6c};};
\node[stars] at (axis cs:{272.500},{-32.720}) {\tikz\pgfuseplotmark{m6c};};
\node[stars] at (axis cs:{272.524},{-30.728}) {\tikz\pgfuseplotmark{m6ab};};
\node[stars] at (axis cs:{272.609},{-62.002}) {\tikz\pgfuseplotmark{m5c};};
\node[stars] at (axis cs:{272.731},{-33.800}) {\tikz\pgfuseplotmark{m6c};};
\node[stars] at (axis cs:{272.768},{-47.513}) {\tikz\pgfuseplotmark{m6b};};
\node[stars] at (axis cs:{272.773},{-41.359}) {\tikz\pgfuseplotmark{m6a};};
\node[stars] at (axis cs:{272.807},{-45.954}) {\tikz\pgfuseplotmark{m5a};};
\node[stars] at (axis cs:{272.816},{-75.891}) {\tikz\pgfuseplotmark{m6a};};
\node[stars] at (axis cs:{273.142},{-73.672}) {\tikz\pgfuseplotmark{m6bb};};
\node[stars] at (axis cs:{273.303},{-41.336}) {\tikz\pgfuseplotmark{m5cv};};
\node[stars] at (axis cs:{273.568},{-63.690}) {\tikz\pgfuseplotmark{m6c};};
\node[stars] at (axis cs:{273.919},{-63.055}) {\tikz\pgfuseplotmark{m6a};};
\node[stars] at (axis cs:{273.973},{-44.206}) {\tikz\pgfuseplotmark{m5c};};
\node[stars] at (axis cs:{274.254},{-51.068}) {\tikz\pgfuseplotmark{m6b};};
\node[stars] at (axis cs:{274.281},{-56.023}) {\tikz\pgfuseplotmark{m5cv};};
\node[stars] at (axis cs:{274.349},{-28.289}) {\tikz\pgfuseplotmark{m6c};};
\node[stars] at (axis cs:{274.350},{-28.652}) {\tikz\pgfuseplotmark{m6c};};
\node[stars] at (axis cs:{274.401},{-34.107}) {\tikz\pgfuseplotmark{m6bvb};};
\node[stars] at (axis cs:{274.407},{-36.761}) {\tikz\pgfuseplotmark{m3bv};};
\node[stars] at (axis cs:{274.504},{-68.229}) {\tikz\pgfuseplotmark{m6cb};};
\node[stars] at (axis cs:{274.513},{-27.042}) {\tikz\pgfuseplotmark{m5a};};
\node[stars] at (axis cs:{274.667},{-42.288}) {\tikz\pgfuseplotmark{m6c};};
\node[stars] at (axis cs:{274.917},{-63.887}) {\tikz\pgfuseplotmark{m6c};};
\node[stars] at (axis cs:{275.230},{-37.488}) {\tikz\pgfuseplotmark{m6c};};
\node[stars] at (axis cs:{275.249},{-29.828}) {\tikz\pgfuseplotmark{m3av};};
\node[stars] at (axis cs:{275.501},{-28.430}) {\tikz\pgfuseplotmark{m6c};};
\node[stars] at (axis cs:{275.577},{-38.657}) {\tikz\pgfuseplotmark{m5b};};
\node[stars] at (axis cs:{275.721},{-36.670}) {\tikz\pgfuseplotmark{m5cv};};
\node[stars] at (axis cs:{275.807},{-61.494}) {\tikz\pgfuseplotmark{m4cb};};
\node[stars] at (axis cs:{275.870},{-36.238}) {\tikz\pgfuseplotmark{m6a};};
\node[stars] at (axis cs:{275.902},{-75.044}) {\tikz\pgfuseplotmark{m5c};};
\node[stars] at (axis cs:{276.043},{-34.385}) {\tikz\pgfuseplotmark{m2a};};
\node[stars] at (axis cs:{276.076},{-44.110}) {\tikz\pgfuseplotmark{m5cv};};
\node[stars] at (axis cs:{276.138},{-49.653}) {\tikz\pgfuseplotmark{m6c};};
\node[stars] at (axis cs:{276.256},{-30.756}) {\tikz\pgfuseplotmark{m6a};};
\node[stars] at (axis cs:{276.340},{-35.992}) {\tikz\pgfuseplotmark{m6c};};
\node[stars] at (axis cs:{276.382},{-63.021}) {\tikz\pgfuseplotmark{m6bv};};
\node[stars] at (axis cs:{276.478},{-33.946}) {\tikz\pgfuseplotmark{m6c};};
\node[stars] at (axis cs:{276.725},{-48.117}) {\tikz\pgfuseplotmark{m5c};};
\node[stars] at (axis cs:{276.743},{-45.968}) {\tikz\pgfuseplotmark{m3c};};
\node[stars] at (axis cs:{276.768},{-53.110}) {\tikz\pgfuseplotmark{m6b};};
\node[stars] at (axis cs:{276.956},{-29.817}) {\tikz\pgfuseplotmark{m6b};};
\node[stars] at (axis cs:{277.027},{-84.387}) {\tikz\pgfuseplotmark{m6c};};
\node[stars] at (axis cs:{277.113},{-38.995}) {\tikz\pgfuseplotmark{m6a};};
\node[stars] at (axis cs:{277.208},{-49.071}) {\tikz\pgfuseplotmark{m4b};};
\node[stars] at (axis cs:{277.304},{-43.846}) {\tikz\pgfuseplotmark{m6c};};
\node[stars] at (axis cs:{277.333},{-80.232}) {\tikz\pgfuseplotmark{m6b};};
\node[stars] at (axis cs:{277.483},{-47.220}) {\tikz\pgfuseplotmark{m6a};};
\node[stars] at (axis cs:{277.486},{-57.523}) {\tikz\pgfuseplotmark{m6ab};};
\node[stars] at (axis cs:{277.762},{-41.914}) {\tikz\pgfuseplotmark{m6b};};
\node[stars] at (axis cs:{277.770},{-32.989}) {\tikz\pgfuseplotmark{m5c};};
\node[stars] at (axis cs:{277.843},{-62.278}) {\tikz\pgfuseplotmark{m5av};};
\node[stars] at (axis cs:{277.939},{-45.915}) {\tikz\pgfuseplotmark{m5b};};
\node[stars] at (axis cs:{277.984},{-43.507}) {\tikz\pgfuseplotmark{m6a};};
\node[stars] at (axis cs:{278.008},{-45.757}) {\tikz\pgfuseplotmark{m5b};};
\node[stars] at (axis cs:{278.089},{-39.704}) {\tikz\pgfuseplotmark{m5c};};
\node[stars] at (axis cs:{278.231},{-73.965}) {\tikz\pgfuseplotmark{m6b};};
\node[stars] at (axis cs:{278.254},{-39.892}) {\tikz\pgfuseplotmark{m6c};};
\node[stars] at (axis cs:{278.346},{-38.726}) {\tikz\pgfuseplotmark{m6ab};};
\node[stars] at (axis cs:{278.373},{-58.709}) {\tikz\pgfuseplotmark{m6c};};
\node[stars] at (axis cs:{278.376},{-42.312}) {\tikz\pgfuseplotmark{m5a};};
\node[stars] at (axis cs:{278.491},{-33.016}) {\tikz\pgfuseplotmark{m5c};};
\node[stars] at (axis cs:{278.630},{-52.891}) {\tikz\pgfuseplotmark{m6c};};
\node[stars] at (axis cs:{278.999},{-29.699}) {\tikz\pgfuseplotmark{m6c};};
\node[stars] at (axis cs:{279.810},{-47.910}) {\tikz\pgfuseplotmark{m6a};};
\node[stars] at (axis cs:{279.896},{-43.186}) {\tikz\pgfuseplotmark{m5cv};};
\node[stars] at (axis cs:{280.378},{-48.094}) {\tikz\pgfuseplotmark{m6cv};};
\node[stars] at (axis cs:{280.560},{-81.808}) {\tikz\pgfuseplotmark{m6cv};};
\node[stars] at (axis cs:{280.594},{-64.643}) {\tikz\pgfuseplotmark{m6c};};
\node[stars] at (axis cs:{280.759},{-71.428}) {\tikz\pgfuseplotmark{m4b};};
\node[stars] at (axis cs:{280.906},{-64.551}) {\tikz\pgfuseplotmark{m6a};};
\node[stars] at (axis cs:{280.946},{-38.323}) {\tikz\pgfuseplotmark{m5bv};};
\node[stars] at (axis cs:{281.033},{-36.718}) {\tikz\pgfuseplotmark{m6c};};
\node[stars] at (axis cs:{281.081},{-35.642}) {\tikz\pgfuseplotmark{m5a};};
\node[stars] at (axis cs:{281.238},{-39.686}) {\tikz\pgfuseplotmark{m5c};};
\node[stars] at (axis cs:{281.298},{-61.095}) {\tikz\pgfuseplotmark{m6b};};
\node[stars] at (axis cs:{281.350},{-56.882}) {\tikz\pgfuseplotmark{m6c};};
\node[stars] at (axis cs:{281.362},{-64.871}) {\tikz\pgfuseplotmark{m5a};};
\node[stars] at (axis cs:{281.483},{-50.873}) {\tikz\pgfuseplotmark{m6c};};
\node[stars] at (axis cs:{281.936},{-40.406}) {\tikz\pgfuseplotmark{m5c};};
\node[stars] at (axis cs:{281.955},{-77.868}) {\tikz\pgfuseplotmark{m6cv};};
\node[stars] at (axis cs:{282.158},{-65.077}) {\tikz\pgfuseplotmark{m6a};};
\node[stars] at (axis cs:{282.210},{-43.680}) {\tikz\pgfuseplotmark{m5c};};
\node[stars] at (axis cs:{282.364},{-45.810}) {\tikz\pgfuseplotmark{m6a};};
\node[stars] at (axis cs:{282.396},{-43.434}) {\tikz\pgfuseplotmark{m6a};};
\node[stars] at (axis cs:{282.432},{-72.996}) {\tikz\pgfuseplotmark{m6bb};};
\node[stars] at (axis cs:{282.744},{-41.062}) {\tikz\pgfuseplotmark{m6cb};};
\node[stars] at (axis cs:{283.050},{-41.709}) {\tikz\pgfuseplotmark{m6c};};
\node[stars] at (axis cs:{283.054},{-62.188}) {\tikz\pgfuseplotmark{m4cv};};
\node[stars] at (axis cs:{283.113},{-46.595}) {\tikz\pgfuseplotmark{m6a};};
\node[stars] at (axis cs:{283.154},{-29.379}) {\tikz\pgfuseplotmark{m6b};};
\node[stars] at (axis cs:{283.165},{-52.107}) {\tikz\pgfuseplotmark{m5c};};
\node[stars] at (axis cs:{283.250},{-46.586}) {\tikz\pgfuseplotmark{m6c};};
\node[stars] at (axis cs:{283.260},{-48.361}) {\tikz\pgfuseplotmark{m6c};};
\node[stars] at (axis cs:{283.300},{-51.931}) {\tikz\pgfuseplotmark{m6c};};
\node[stars] at (axis cs:{283.696},{-87.606}) {\tikz\pgfuseplotmark{m5c};};
\node[stars] at (axis cs:{283.848},{-37.387}) {\tikz\pgfuseplotmark{m6c};};
\node[stars] at (axis cs:{284.071},{-42.711}) {\tikz\pgfuseplotmark{m5c};};
\node[stars] at (axis cs:{284.113},{-31.689}) {\tikz\pgfuseplotmark{m6c};};
\node[stars] at (axis cs:{284.169},{-37.343}) {\tikz\pgfuseplotmark{m5cv};};
\node[stars] at (axis cs:{284.228},{-62.802}) {\tikz\pgfuseplotmark{m6c};};
\node[stars] at (axis cs:{284.238},{-67.233}) {\tikz\pgfuseplotmark{m4cv};};
\node[stars] at (axis cs:{284.310},{-43.917}) {\tikz\pgfuseplotmark{m6c};};
\node[stars] at (axis cs:{284.393},{-39.823}) {\tikz\pgfuseplotmark{m6c};};
\node[stars] at (axis cs:{284.542},{-83.422}) {\tikz\pgfuseplotmark{m6c};};
\node[stars] at (axis cs:{284.589},{-31.036}) {\tikz\pgfuseplotmark{m6b};};
\node[stars] at (axis cs:{284.616},{-52.939}) {\tikz\pgfuseplotmark{m5a};};
\node[stars] at (axis cs:{284.652},{-60.201}) {\tikz\pgfuseplotmark{m5b};};
\node[stars] at (axis cs:{284.681},{-37.107}) {\tikz\pgfuseplotmark{m5av};};
\node[stars] at (axis cs:{284.872},{-70.469}) {\tikz\pgfuseplotmark{m6c};};
\node[stars] at (axis cs:{285.015},{-66.654}) {\tikz\pgfuseplotmark{m6b};};
\node[stars] at (axis cs:{285.268},{-37.061}) {\tikz\pgfuseplotmark{m6cb};};
\node[stars] at (axis cs:{285.536},{-41.910}) {\tikz\pgfuseplotmark{m6c};};
\node[stars] at (axis cs:{285.653},{-29.880}) {\tikz\pgfuseplotmark{m3a};};
\node[stars] at (axis cs:{285.779},{-42.095}) {\tikz\pgfuseplotmark{m5a};};
\node[stars] at (axis cs:{285.824},{-38.253}) {\tikz\pgfuseplotmark{m6av};};
\node[stars] at (axis cs:{285.874},{-68.755}) {\tikz\pgfuseplotmark{m6b};};
\node[stars] at (axis cs:{285.990},{-51.019}) {\tikz\pgfuseplotmark{m6b};};
\node[stars] at (axis cs:{286.004},{-57.960}) {\tikz\pgfuseplotmark{m6c};};
\node[stars] at (axis cs:{286.104},{-31.047}) {\tikz\pgfuseplotmark{m5c};};
\node[stars] at (axis cs:{286.583},{-52.341}) {\tikz\pgfuseplotmark{m5c};};
\node[stars] at (axis cs:{286.605},{-37.063}) {\tikz\pgfuseplotmark{m4cb};};
\node[stars] at (axis cs:{286.719},{-37.810}) {\tikz\pgfuseplotmark{m6c};};
\node[stars] at (axis cs:{286.728},{-50.323}) {\tikz\pgfuseplotmark{m6c};};
\node[stars] at (axis cs:{286.732},{-48.299}) {\tikz\pgfuseplotmark{m6b};};
\node[stars] at (axis cs:{286.735},{-27.670}) {\tikz\pgfuseplotmark{m3c};};
\node[stars] at (axis cs:{286.879},{-28.637}) {\tikz\pgfuseplotmark{m6b};};
\node[stars] at (axis cs:{287.087},{-40.496}) {\tikz\pgfuseplotmark{m5a};};
\node[stars] at (axis cs:{287.218},{-55.720}) {\tikz\pgfuseplotmark{m6c};};
\node[stars] at (axis cs:{287.368},{-37.904}) {\tikz\pgfuseplotmark{m4b};};
\node[stars] at (axis cs:{287.416},{-39.827}) {\tikz\pgfuseplotmark{m6c};};
\node[stars] at (axis cs:{287.470},{-68.424}) {\tikz\pgfuseplotmark{m5c};};
\node[stars] at (axis cs:{287.490},{-41.892}) {\tikz\pgfuseplotmark{m6a};};
\node[stars] at (axis cs:{287.507},{-39.341}) {\tikz\pgfuseplotmark{m4bv};};
\node[stars] at (axis cs:{287.758},{-39.005}) {\tikz\pgfuseplotmark{m6c};};
\node[stars] at (axis cs:{287.828},{-29.502}) {\tikz\pgfuseplotmark{m6c};};
\node[stars] at (axis cs:{288.192},{-50.486}) {\tikz\pgfuseplotmark{m6b};};
\node[stars] at (axis cs:{288.665},{-45.193}) {\tikz\pgfuseplotmark{m6b};};
\node[stars] at (axis cs:{289.091},{-45.466}) {\tikz\pgfuseplotmark{m5c};};
\node[stars] at (axis cs:{289.119},{-69.191}) {\tikz\pgfuseplotmark{m6c};};
\node[stars] at (axis cs:{289.252},{-65.228}) {\tikz\pgfuseplotmark{m6c};};
\node[stars] at (axis cs:{289.301},{-66.661}) {\tikz\pgfuseplotmark{m6a};};
\node[stars] at (axis cs:{289.541},{-53.387}) {\tikz\pgfuseplotmark{m6c};};
\node[stars] at (axis cs:{289.917},{-35.421}) {\tikz\pgfuseplotmark{m6a};};
\node[stars] at (axis cs:{290.374},{-34.983}) {\tikz\pgfuseplotmark{m6cv};};
\node[stars] at (axis cs:{290.540},{-42.016}) {\tikz\pgfuseplotmark{m6c};};
\node[stars] at (axis cs:{290.657},{-51.231}) {\tikz\pgfuseplotmark{m6c};};
\node[stars] at (axis cs:{290.660},{-44.459}) {\tikz\pgfuseplotmark{m4bb};};
\node[stars] at (axis cs:{290.713},{-54.424}) {\tikz\pgfuseplotmark{m5b};};
\node[stars] at (axis cs:{290.805},{-44.800}) {\tikz\pgfuseplotmark{m4c};};
\node[stars] at (axis cs:{290.972},{-40.616}) {\tikz\pgfuseplotmark{m4b};};
\node[stars] at (axis cs:{291.023},{-68.371}) {\tikz\pgfuseplotmark{m6c};};
\node[stars] at (axis cs:{291.089},{-43.722}) {\tikz\pgfuseplotmark{m6c};};
\node[stars] at (axis cs:{291.126},{-27.866}) {\tikz\pgfuseplotmark{m6b};};
\node[stars] at (axis cs:{291.267},{-29.309}) {\tikz\pgfuseplotmark{m6b};};
\node[stars] at (axis cs:{291.735},{-29.743}) {\tikz\pgfuseplotmark{m6a};};
\node[stars] at (axis cs:{291.950},{-54.325}) {\tikz\pgfuseplotmark{m6ab};};
\node[stars] at (axis cs:{292.349},{-43.445}) {\tikz\pgfuseplotmark{m6a};};
\node[stars] at (axis cs:{292.469},{-55.442}) {\tikz\pgfuseplotmark{m6c};};
\node[stars] at (axis cs:{292.644},{-55.110}) {\tikz\pgfuseplotmark{m6c};};
\node[stars] at (axis cs:{292.796},{-68.434}) {\tikz\pgfuseplotmark{m6b};};
\node[stars] at (axis cs:{293.058},{-44.546}) {\tikz\pgfuseplotmark{m6c};};
\node[stars] at (axis cs:{293.224},{-53.186}) {\tikz\pgfuseplotmark{m6a};};
\node[stars] at (axis cs:{293.340},{-45.272}) {\tikz\pgfuseplotmark{m6av};};
\node[stars] at (axis cs:{293.535},{-40.035}) {\tikz\pgfuseplotmark{m6bv};};
\node[stars] at (axis cs:{293.804},{-48.099}) {\tikz\pgfuseplotmark{m5b};};
\node[stars] at (axis cs:{294.608},{-57.983}) {\tikz\pgfuseplotmark{m6c};};
\node[stars] at (axis cs:{294.924},{-45.278}) {\tikz\pgfuseplotmark{m6cv};};
\node[stars] at (axis cs:{294.967},{-66.685}) {\tikz\pgfuseplotmark{m6c};};
\node[stars] at (axis cs:{295.078},{-54.418}) {\tikz\pgfuseplotmark{m6c};};
\node[stars] at (axis cs:{295.405},{-65.854}) {\tikz\pgfuseplotmark{m6b};};
\node[stars] at (axis cs:{295.907},{-37.539}) {\tikz\pgfuseplotmark{m6c};};
\node[stars] at (axis cs:{296.505},{-31.908}) {\tikz\pgfuseplotmark{m6a};};
\node[stars] at (axis cs:{297.005},{-56.362}) {\tikz\pgfuseplotmark{m5c};};
\node[stars] at (axis cs:{297.230},{-52.888}) {\tikz\pgfuseplotmark{m6c};};
\node[stars] at (axis cs:{297.298},{-28.789}) {\tikz\pgfuseplotmark{m6b};};
\node[stars] at (axis cs:{297.355},{-72.503}) {\tikz\pgfuseplotmark{m5c};};
\node[stars] at (axis cs:{297.473},{-66.813}) {\tikz\pgfuseplotmark{m6c};};
\node[stars] at (axis cs:{297.559},{-47.557}) {\tikz\pgfuseplotmark{m6bv};};
\node[stars] at (axis cs:{297.591},{-61.061}) {\tikz\pgfuseplotmark{m6c};};
\node[stars] at (axis cs:{297.687},{-59.193}) {\tikz\pgfuseplotmark{m5cb};};
\node[stars] at (axis cs:{297.755},{-65.605}) {\tikz\pgfuseplotmark{m6bv};};
\node[stars] at (axis cs:{297.961},{-39.874}) {\tikz\pgfuseplotmark{m5cv};};
\node[stars] at (axis cs:{298.157},{-54.971}) {\tikz\pgfuseplotmark{m6ab};};
\node[stars] at (axis cs:{298.669},{-61.171}) {\tikz\pgfuseplotmark{m6c};};
\node[stars] at (axis cs:{298.771},{-33.046}) {\tikz\pgfuseplotmark{m6c};};
\node[stars] at (axis cs:{298.815},{-41.868}) {\tikz\pgfuseplotmark{m4b};};
\node[stars] at (axis cs:{299.007},{-81.350}) {\tikz\pgfuseplotmark{m6c};};
\node[stars] at (axis cs:{299.237},{-27.170}) {\tikz\pgfuseplotmark{m5av};};
\node[stars] at (axis cs:{299.276},{-58.901}) {\tikz\pgfuseplotmark{m5c};};
\node[stars] at (axis cs:{299.672},{-69.164}) {\tikz\pgfuseplotmark{m6a};};
\node[stars] at (axis cs:{299.721},{-68.762}) {\tikz\pgfuseplotmark{m6c};};
\node[stars] at (axis cs:{299.735},{-30.538}) {\tikz\pgfuseplotmark{m6c};};
\node[stars] at (axis cs:{299.934},{-35.276}) {\tikz\pgfuseplotmark{m4c};};
\node[stars] at (axis cs:{299.964},{-34.698}) {\tikz\pgfuseplotmark{m5cb};};
\node[stars] at (axis cs:{300.066},{-37.702}) {\tikz\pgfuseplotmark{m6b};};
\node[stars] at (axis cs:{300.084},{-33.703}) {\tikz\pgfuseplotmark{m6av};};
\node[stars] at (axis cs:{300.096},{-66.949}) {\tikz\pgfuseplotmark{m6a};};
\node[stars] at (axis cs:{300.106},{-49.351}) {\tikz\pgfuseplotmark{m6c};};
\node[stars] at (axis cs:{300.110},{-43.043}) {\tikz\pgfuseplotmark{m6b};};
\node[stars] at (axis cs:{300.148},{-72.910}) {\tikz\pgfuseplotmark{m4bv};};
\node[stars] at (axis cs:{300.201},{-45.113}) {\tikz\pgfuseplotmark{m6a};};
\node[stars] at (axis cs:{300.362},{-40.814}) {\tikz\pgfuseplotmark{m6c};};
\node[stars] at (axis cs:{300.436},{-59.376}) {\tikz\pgfuseplotmark{m5bv};};
\node[stars] at (axis cs:{300.469},{-66.944}) {\tikz\pgfuseplotmark{m5c};};
\node[stars] at (axis cs:{300.665},{-27.710}) {\tikz\pgfuseplotmark{m4cv};};
\node[stars] at (axis cs:{300.889},{-37.941}) {\tikz\pgfuseplotmark{m5a};};
\node[stars] at (axis cs:{301.082},{-32.056}) {\tikz\pgfuseplotmark{m5bv};};
\node[stars] at (axis cs:{301.387},{-67.321}) {\tikz\pgfuseplotmark{m6b};};
\node[stars] at (axis cs:{301.846},{-52.881}) {\tikz\pgfuseplotmark{m5bv};};
\node[stars] at (axis cs:{301.896},{-55.016}) {\tikz\pgfuseplotmark{m6c};};
\node[stars] at (axis cs:{302.040},{-52.578}) {\tikz\pgfuseplotmark{m6c};};
\node[stars] at (axis cs:{302.085},{-66.355}) {\tikz\pgfuseplotmark{m6c};};
\node[stars] at (axis cs:{302.182},{-66.182}) {\tikz\pgfuseplotmark{m4av};};
\node[stars] at (axis cs:{302.266},{-47.068}) {\tikz\pgfuseplotmark{m6c};};
\node[stars] at (axis cs:{302.781},{-57.524}) {\tikz\pgfuseplotmark{m6c};};
\node[stars] at (axis cs:{302.800},{-36.101}) {\tikz\pgfuseplotmark{m5cv};};
\node[stars] at (axis cs:{303.099},{-42.780}) {\tikz\pgfuseplotmark{m6c};};
\node[stars] at (axis cs:{303.385},{-47.713}) {\tikz\pgfuseplotmark{m6c};};
\node[stars] at (axis cs:{303.579},{-52.446}) {\tikz\pgfuseplotmark{m6a};};
\node[stars] at (axis cs:{303.612},{-63.416}) {\tikz\pgfuseplotmark{m6b};};
\node[stars] at (axis cs:{303.822},{-27.033}) {\tikz\pgfuseplotmark{m6av};};
\node[stars] at (axis cs:{303.961},{-30.005}) {\tikz\pgfuseplotmark{m6c};};
\node[stars] at (axis cs:{304.098},{-36.454}) {\tikz\pgfuseplotmark{m6cv};};
\node[stars] at (axis cs:{304.733},{-47.711}) {\tikz\pgfuseplotmark{m6c};};
\node[stars] at (axis cs:{304.762},{-63.231}) {\tikz\pgfuseplotmark{m6c};};
\node[stars] at (axis cs:{304.824},{-47.580}) {\tikz\pgfuseplotmark{m6b};};
\node[stars] at (axis cs:{305.135},{-55.051}) {\tikz\pgfuseplotmark{m6c};};
\node[stars] at (axis cs:{305.216},{-35.674}) {\tikz\pgfuseplotmark{m6c};};
\node[stars] at (axis cs:{305.421},{-49.999}) {\tikz\pgfuseplotmark{m6c};};
\node[stars] at (axis cs:{305.615},{-42.049}) {\tikz\pgfuseplotmark{m6a};};
\node[stars] at (axis cs:{305.972},{-42.423}) {\tikz\pgfuseplotmark{m6a};};
\node[stars] at (axis cs:{306.229},{-83.310}) {\tikz\pgfuseplotmark{m6c};};
\node[stars] at (axis cs:{306.362},{-28.663}) {\tikz\pgfuseplotmark{m6a};};
\node[stars] at (axis cs:{306.412},{-56.735}) {\tikz\pgfuseplotmark{m2bvb};};
\node[stars] at (axis cs:{306.450},{-40.796}) {\tikz\pgfuseplotmark{m6b};};
\node[stars] at (axis cs:{306.721},{-37.403}) {\tikz\pgfuseplotmark{m6cb};};
\node[stars] at (axis cs:{307.195},{-35.596}) {\tikz\pgfuseplotmark{m6b};};
\node[stars] at (axis cs:{307.737},{-29.112}) {\tikz\pgfuseplotmark{m6c};};
\node[stars] at (axis cs:{308.324},{-80.965}) {\tikz\pgfuseplotmark{m6a};};
\node[stars] at (axis cs:{308.479},{-44.516}) {\tikz\pgfuseplotmark{m5b};};
\node[stars] at (axis cs:{308.697},{-30.473}) {\tikz\pgfuseplotmark{m6c};};
\node[stars] at (axis cs:{308.731},{-38.090}) {\tikz\pgfuseplotmark{m6c};};
\node[stars] at (axis cs:{308.776},{-71.189}) {\tikz\pgfuseplotmark{m6c};};
\node[stars] at (axis cs:{308.895},{-60.582}) {\tikz\pgfuseplotmark{m5a};};
\node[stars] at (axis cs:{308.949},{-44.338}) {\tikz\pgfuseplotmark{m6c};};
\node[stars] at (axis cs:{308.966},{-69.610}) {\tikz\pgfuseplotmark{m6b};};
\node[stars] at (axis cs:{309.160},{-63.121}) {\tikz\pgfuseplotmark{m6c};};
\node[stars] at (axis cs:{309.392},{-47.291}) {\tikz\pgfuseplotmark{m3b};};
\node[stars] at (axis cs:{309.397},{-61.530}) {\tikz\pgfuseplotmark{m5av};};
\node[stars] at (axis cs:{309.578},{-81.289}) {\tikz\pgfuseplotmark{m6b};};
\node[stars] at (axis cs:{309.966},{-62.908}) {\tikz\pgfuseplotmark{m6c};};
\node[stars] at (axis cs:{310.011},{-60.549}) {\tikz\pgfuseplotmark{m5b};};
\node[stars] at (axis cs:{310.083},{-33.432}) {\tikz\pgfuseplotmark{m5c};};
\node[stars] at (axis cs:{310.232},{-42.396}) {\tikz\pgfuseplotmark{m6cb};};
\node[stars] at (axis cs:{310.349},{-31.598}) {\tikz\pgfuseplotmark{m6a};};
\node[stars] at (axis cs:{310.353},{-42.134}) {\tikz\pgfuseplotmark{m6cv};};
\node[stars] at (axis cs:{310.434},{-75.351}) {\tikz\pgfuseplotmark{m6cb};};
\node[stars] at (axis cs:{310.488},{-66.761}) {\tikz\pgfuseplotmark{m5c};};
\node[stars] at (axis cs:{310.512},{-76.180}) {\tikz\pgfuseplotmark{m6b};};
\node[stars] at (axis cs:{310.721},{-39.559}) {\tikz\pgfuseplotmark{m6c};};
\node[stars] at (axis cs:{311.010},{-51.921}) {\tikz\pgfuseplotmark{m5a};};
\node[stars] at (axis cs:{311.240},{-66.203}) {\tikz\pgfuseplotmark{m3c};};
\node[stars] at (axis cs:{311.578},{-36.120}) {\tikz\pgfuseplotmark{m6c};};
\node[stars] at (axis cs:{311.584},{-39.199}) {\tikz\pgfuseplotmark{m5c};};
\node[stars] at (axis cs:{312.077},{-44.197}) {\tikz\pgfuseplotmark{m6c};};
\node[stars] at (axis cs:{312.121},{-43.988}) {\tikz\pgfuseplotmark{m5b};};
\node[stars] at (axis cs:{312.326},{-68.776}) {\tikz\pgfuseplotmark{m5c};};
\node[stars] at (axis cs:{312.371},{-46.227}) {\tikz\pgfuseplotmark{m5b};};
\node[stars] at (axis cs:{312.492},{-33.780}) {\tikz\pgfuseplotmark{m5bv};};
\node[stars] at (axis cs:{312.696},{-32.055}) {\tikz\pgfuseplotmark{m6c};};
\node[stars] at (axis cs:{312.753},{-37.913}) {\tikz\pgfuseplotmark{m6a};};
\node[stars] at (axis cs:{312.875},{-51.608}) {\tikz\pgfuseplotmark{m5b};};
\node[stars] at (axis cs:{312.910},{-62.429}) {\tikz\pgfuseplotmark{m6cb};};
\node[stars] at (axis cs:{312.995},{-33.178}) {\tikz\pgfuseplotmark{m6b};};
\node[stars] at (axis cs:{313.354},{-30.719}) {\tikz\pgfuseplotmark{m6c};};
\node[stars] at (axis cs:{313.417},{-39.810}) {\tikz\pgfuseplotmark{m5c};};
\node[stars] at (axis cs:{313.527},{-27.925}) {\tikz\pgfuseplotmark{m6cv};};
\node[stars] at (axis cs:{313.646},{-50.727}) {\tikz\pgfuseplotmark{m6c};};
\node[stars] at (axis cs:{313.703},{-58.454}) {\tikz\pgfuseplotmark{m4a};};
\node[stars] at (axis cs:{314.999},{-36.129}) {\tikz\pgfuseplotmark{m6b};};
\node[stars] at (axis cs:{315.090},{-51.265}) {\tikz\pgfuseplotmark{m6a};};
\node[stars] at (axis cs:{315.181},{-53.739}) {\tikz\pgfuseplotmark{m6c};};
\node[stars] at (axis cs:{315.323},{-32.258}) {\tikz\pgfuseplotmark{m5a};};
\node[stars] at (axis cs:{315.367},{-68.209}) {\tikz\pgfuseplotmark{m6c};};
\node[stars] at (axis cs:{315.613},{-38.531}) {\tikz\pgfuseplotmark{m6b};};
\node[stars] at (axis cs:{315.741},{-38.631}) {\tikz\pgfuseplotmark{m5c};};
\node[stars] at (axis cs:{315.792},{-27.731}) {\tikz\pgfuseplotmark{m6c};};
\node[stars] at (axis cs:{316.179},{-77.024}) {\tikz\pgfuseplotmark{m5bv};};
\node[stars] at (axis cs:{316.309},{-54.727}) {\tikz\pgfuseplotmark{m5c};};
\node[stars] at (axis cs:{316.505},{-30.125}) {\tikz\pgfuseplotmark{m6a};};
\node[stars] at (axis cs:{316.603},{-32.341}) {\tikz\pgfuseplotmark{m5c};};
\node[stars] at (axis cs:{316.606},{-41.386}) {\tikz\pgfuseplotmark{m6ab};};
\node[stars] at (axis cs:{316.936},{-45.379}) {\tikz\pgfuseplotmark{m6c};};
\node[stars] at (axis cs:{317.136},{-63.928}) {\tikz\pgfuseplotmark{m6a};};
\node[stars] at (axis cs:{317.195},{-88.956}) {\tikz\pgfuseplotmark{m5cv};};
\node[stars] at (axis cs:{317.343},{-36.706}) {\tikz\pgfuseplotmark{m6c};};
\node[stars] at (axis cs:{317.344},{-73.173}) {\tikz\pgfuseplotmark{m6a};};
\node[stars] at (axis cs:{317.837},{-72.544}) {\tikz\pgfuseplotmark{m6c};};
\node[stars] at (axis cs:{318.057},{-40.269}) {\tikz\pgfuseplotmark{m6av};};
\node[stars] at (axis cs:{318.263},{-39.425}) {\tikz\pgfuseplotmark{m5c};};
\node[stars] at (axis cs:{318.322},{-27.619}) {\tikz\pgfuseplotmark{m5c};};
\node[stars] at (axis cs:{318.329},{-36.423}) {\tikz\pgfuseplotmark{m6b};};
\node[stars] at (axis cs:{318.335},{-70.126}) {\tikz\pgfuseplotmark{m5bv};};
\node[stars] at (axis cs:{318.810},{-40.506}) {\tikz\pgfuseplotmark{m6c};};
\node[stars] at (axis cs:{318.941},{-53.263}) {\tikz\pgfuseplotmark{m6a};};
\node[stars] at (axis cs:{318.945},{-36.211}) {\tikz\pgfuseplotmark{m6b};};
\node[stars] at (axis cs:{319.485},{-32.172}) {\tikz\pgfuseplotmark{m5a};};
\node[stars] at (axis cs:{319.502},{-64.681}) {\tikz\pgfuseplotmark{m6cv};};
\node[stars] at (axis cs:{319.726},{-28.766}) {\tikz\pgfuseplotmark{m6c};};
\node[stars] at (axis cs:{319.967},{-53.449}) {\tikz\pgfuseplotmark{m4cb};};
\node[stars] at (axis cs:{320.040},{-45.022}) {\tikz\pgfuseplotmark{m6bv};};
\node[stars] at (axis cs:{320.190},{-40.809}) {\tikz\pgfuseplotmark{m5av};};
\node[stars] at (axis cs:{320.319},{-49.938}) {\tikz\pgfuseplotmark{m6c};};
\node[stars] at (axis cs:{321.070},{-69.734}) {\tikz\pgfuseplotmark{m6cv};};
\node[stars] at (axis cs:{321.087},{-46.614}) {\tikz\pgfuseplotmark{m6c};};
\node[stars] at (axis cs:{321.103},{-41.007}) {\tikz\pgfuseplotmark{m6a};};
\node[stars] at (axis cs:{321.325},{-71.799}) {\tikz\pgfuseplotmark{m6b};};
\node[stars] at (axis cs:{321.564},{-54.660}) {\tikz\pgfuseplotmark{m6b};};
\node[stars] at (axis cs:{321.595},{-37.829}) {\tikz\pgfuseplotmark{m6a};};
\node[stars] at (axis cs:{321.611},{-65.366}) {\tikz\pgfuseplotmark{m4cv};};
\node[stars] at (axis cs:{321.757},{-42.548}) {\tikz\pgfuseplotmark{m5cb};};
\node[stars] at (axis cs:{322.187},{-69.505}) {\tikz\pgfuseplotmark{m6av};};
\node[stars] at (axis cs:{322.251},{-53.706}) {\tikz\pgfuseplotmark{m6c};};
\node[stars] at (axis cs:{323.013},{-84.810}) {\tikz\pgfuseplotmark{m6cv};};
\node[stars] at (axis cs:{323.024},{-41.179}) {\tikz\pgfuseplotmark{m5c};};
\node[stars] at (axis cs:{323.061},{-33.944}) {\tikz\pgfuseplotmark{m6b};};
\node[stars] at (axis cs:{323.323},{-52.738}) {\tikz\pgfuseplotmark{m6c};};
\node[stars] at (axis cs:{323.338},{-80.039}) {\tikz\pgfuseplotmark{m6cb};};
\node[stars] at (axis cs:{323.348},{-44.848}) {\tikz\pgfuseplotmark{m6a};};
\node[stars] at (axis cs:{323.477},{-82.683}) {\tikz\pgfuseplotmark{m6c};};
\node[stars] at (axis cs:{323.571},{-42.925}) {\tikz\pgfuseplotmark{m6c};};
\node[stars] at (axis cs:{323.721},{-29.696}) {\tikz\pgfuseplotmark{m6c};};
\node[stars] at (axis cs:{324.203},{-33.048}) {\tikz\pgfuseplotmark{m6b};};
\node[stars] at (axis cs:{324.512},{-64.824}) {\tikz\pgfuseplotmark{m6c};};
\node[stars] at (axis cs:{324.735},{-79.442}) {\tikz\pgfuseplotmark{m6c};};
\node[stars] at (axis cs:{324.775},{-33.679}) {\tikz\pgfuseplotmark{m6c};};
\node[stars] at (axis cs:{324.999},{-52.359}) {\tikz\pgfuseplotmark{m6cv};};
\node[stars] at (axis cs:{325.140},{-55.738}) {\tikz\pgfuseplotmark{m6c};};
\node[stars] at (axis cs:{325.369},{-77.390}) {\tikz\pgfuseplotmark{m4a};};
\node[stars] at (axis cs:{326.123},{-38.552}) {\tikz\pgfuseplotmark{m6c};};
\node[stars] at (axis cs:{326.237},{-33.026}) {\tikz\pgfuseplotmark{m4c};};
\node[stars] at (axis cs:{326.328},{-49.499}) {\tikz\pgfuseplotmark{m6c};};
\node[stars] at (axis cs:{326.370},{-71.009}) {\tikz\pgfuseplotmark{m6b};};
\node[stars] at (axis cs:{326.934},{-30.898}) {\tikz\pgfuseplotmark{m5b};};
\node[stars] at (axis cs:{327.066},{-47.304}) {\tikz\pgfuseplotmark{m6ab};};
\node[stars] at (axis cs:{327.501},{-64.712}) {\tikz\pgfuseplotmark{m6a};};
\node[stars] at (axis cs:{327.697},{-69.629}) {\tikz\pgfuseplotmark{m6a};};
\node[stars] at (axis cs:{327.727},{-82.719}) {\tikz\pgfuseplotmark{m5cb};};
\node[stars] at (axis cs:{328.482},{-37.365}) {\tikz\pgfuseplotmark{m3b};};
\node[stars] at (axis cs:{328.797},{-61.887}) {\tikz\pgfuseplotmark{m6bb};};
\node[stars] at (axis cs:{328.982},{-30.606}) {\tikz\pgfuseplotmark{m6c};};
\node[stars] at (axis cs:{329.014},{-52.464}) {\tikz\pgfuseplotmark{m6c};};
\node[stars] at (axis cs:{329.059},{-57.899}) {\tikz\pgfuseplotmark{m6c};};
\node[stars] at (axis cs:{329.095},{-37.254}) {\tikz\pgfuseplotmark{m5c};};
\node[stars] at (axis cs:{329.259},{-37.747}) {\tikz\pgfuseplotmark{m6cv};};
\node[stars] at (axis cs:{329.479},{-54.992}) {\tikz\pgfuseplotmark{m4c};};
\node[stars] at (axis cs:{329.625},{-59.012}) {\tikz\pgfuseplotmark{m6cv};};
\node[stars] at (axis cs:{329.825},{-38.395}) {\tikz\pgfuseplotmark{m5c};};
\node[stars] at (axis cs:{330.101},{-55.883}) {\tikz\pgfuseplotmark{m6b};};
\node[stars] at (axis cs:{330.209},{-28.454}) {\tikz\pgfuseplotmark{m5cvb};};
\node[stars] at (axis cs:{330.469},{-77.662}) {\tikz\pgfuseplotmark{m6c};};
\node[stars] at (axis cs:{330.766},{-76.118}) {\tikz\pgfuseplotmark{m6b};};
\node[stars] at (axis cs:{330.840},{-56.786}) {\tikz\pgfuseplotmark{m5a};};
\node[stars] at (axis cs:{331.099},{-29.916}) {\tikz\pgfuseplotmark{m6c};};
\node[stars] at (axis cs:{331.463},{-59.636}) {\tikz\pgfuseplotmark{m6a};};
\node[stars] at (axis cs:{331.529},{-39.543}) {\tikz\pgfuseplotmark{m4c};};
\node[stars] at (axis cs:{332.058},{-46.961}) {\tikz\pgfuseplotmark{m2av};};
\node[stars] at (axis cs:{332.096},{-32.988}) {\tikz\pgfuseplotmark{m4cv};};
\node[stars] at (axis cs:{332.108},{-34.044}) {\tikz\pgfuseplotmark{m5b};};
\node[stars] at (axis cs:{332.178},{-33.126}) {\tikz\pgfuseplotmark{m6c};};
\node[stars] at (axis cs:{332.482},{-34.015}) {\tikz\pgfuseplotmark{m5c};};
\node[stars] at (axis cs:{332.491},{-48.108}) {\tikz\pgfuseplotmark{m6c};};
\node[stars] at (axis cs:{332.501},{-28.292}) {\tikz\pgfuseplotmark{m6c};};
\node[stars] at (axis cs:{332.537},{-32.548}) {\tikz\pgfuseplotmark{m5b};};
\node[stars] at (axis cs:{332.979},{-76.116}) {\tikz\pgfuseplotmark{m6b};};
\node[stars] at (axis cs:{333.578},{-27.767}) {\tikz\pgfuseplotmark{m5c};};
\node[stars] at (axis cs:{333.661},{-41.381}) {\tikz\pgfuseplotmark{m6c};};
\node[stars] at (axis cs:{333.896},{-44.451}) {\tikz\pgfuseplotmark{m6b};};
\node[stars] at (axis cs:{333.904},{-41.346}) {\tikz\pgfuseplotmark{m5a};};
\node[stars] at (axis cs:{334.111},{-41.627}) {\tikz\pgfuseplotmark{m5b};};
\node[stars] at (axis cs:{334.461},{-77.511}) {\tikz\pgfuseplotmark{m5c};};
\node[stars] at (axis cs:{334.565},{-53.627}) {\tikz\pgfuseplotmark{m5c};};
\node[stars] at (axis cs:{334.625},{-60.259}) {\tikz\pgfuseplotmark{m3a};};
\node[stars] at (axis cs:{335.007},{-80.440}) {\tikz\pgfuseplotmark{m5cv};};
\node[stars] at (axis cs:{335.151},{-57.510}) {\tikz\pgfuseplotmark{m6c};};
\node[stars] at (axis cs:{335.684},{-45.948}) {\tikz\pgfuseplotmark{m6bv};};
\node[stars] at (axis cs:{335.783},{-45.928}) {\tikz\pgfuseplotmark{m6a};};
\node[stars] at (axis cs:{336.154},{-72.255}) {\tikz\pgfuseplotmark{m5c};};
\node[stars] at (axis cs:{336.235},{-57.797}) {\tikz\pgfuseplotmark{m5c};};
\node[stars] at (axis cs:{336.294},{-70.431}) {\tikz\pgfuseplotmark{m6a};};
\node[stars] at (axis cs:{336.463},{-75.015}) {\tikz\pgfuseplotmark{m6bb};};
\node[stars] at (axis cs:{336.833},{-64.966}) {\tikz\pgfuseplotmark{m4cv};};
\node[stars] at (axis cs:{337.157},{-67.489}) {\tikz\pgfuseplotmark{m6a};};
\node[stars] at (axis cs:{337.163},{-39.132}) {\tikz\pgfuseplotmark{m5c};};
\node[stars] at (axis cs:{337.317},{-43.495}) {\tikz\pgfuseplotmark{m4b};};
\node[stars] at (axis cs:{337.439},{-43.749}) {\tikz\pgfuseplotmark{m4cv};};
\node[stars] at (axis cs:{337.442},{-27.107}) {\tikz\pgfuseplotmark{m6bv};};
\node[stars] at (axis cs:{337.876},{-32.346}) {\tikz\pgfuseplotmark{m4cb};};
\node[stars] at (axis cs:{337.906},{-85.967}) {\tikz\pgfuseplotmark{m6a};};
\node[stars] at (axis cs:{338.250},{-61.982}) {\tikz\pgfuseplotmark{m5bv};};
\node[stars] at (axis cs:{338.861},{-78.772}) {\tikz\pgfuseplotmark{m6c};};
\node[stars] at (axis cs:{338.971},{-57.884}) {\tikz\pgfuseplotmark{m6c};};
\node[stars] at (axis cs:{339.122},{-40.582}) {\tikz\pgfuseplotmark{m6c};};
\node[stars] at (axis cs:{339.148},{-31.664}) {\tikz\pgfuseplotmark{m6ab};};
\node[stars] at (axis cs:{339.245},{-40.591}) {\tikz\pgfuseplotmark{m6bb};};
\node[stars] at (axis cs:{339.686},{-28.748}) {\tikz\pgfuseplotmark{m6c};};
\node[stars] at (axis cs:{339.714},{-33.081}) {\tikz\pgfuseplotmark{m6av};};
\node[stars] at (axis cs:{339.934},{-28.326}) {\tikz\pgfuseplotmark{m6cb};};
\node[stars] at (axis cs:{340.093},{-30.659}) {\tikz\pgfuseplotmark{m6bv};};
\node[stars] at (axis cs:{340.164},{-27.043}) {\tikz\pgfuseplotmark{m4cv};};
\node[stars] at (axis cs:{340.204},{-57.422}) {\tikz\pgfuseplotmark{m6b};};
\node[stars] at (axis cs:{340.592},{-29.361}) {\tikz\pgfuseplotmark{m6c};};
\node[stars] at (axis cs:{340.654},{-47.211}) {\tikz\pgfuseplotmark{m6bb};};
\node[stars] at (axis cs:{340.667},{-46.885}) {\tikz\pgfuseplotmark{m2bv};};
\node[stars] at (axis cs:{340.680},{-44.248}) {\tikz\pgfuseplotmark{m6b};};
\node[stars] at (axis cs:{340.875},{-41.414}) {\tikz\pgfuseplotmark{m5a};};
\node[stars] at (axis cs:{341.069},{-60.499}) {\tikz\pgfuseplotmark{m6c};};
\node[stars] at (axis cs:{341.408},{-53.500}) {\tikz\pgfuseplotmark{m5a};};
\node[stars] at (axis cs:{341.420},{-46.547}) {\tikz\pgfuseplotmark{m5c};};
\node[stars] at (axis cs:{341.515},{-81.381}) {\tikz\pgfuseplotmark{m4c};};
\node[stars] at (axis cs:{341.830},{-34.162}) {\tikz\pgfuseplotmark{m6c};};
\node[stars] at (axis cs:{341.899},{-65.560}) {\tikz\pgfuseplotmark{m6c};};
\node[stars] at (axis cs:{342.089},{-61.684}) {\tikz\pgfuseplotmark{m6c};};
\node[stars] at (axis cs:{342.139},{-51.317}) {\tikz\pgfuseplotmark{m3cv};};
\node[stars] at (axis cs:{342.323},{-70.348}) {\tikz\pgfuseplotmark{m6c};};
\node[stars] at (axis cs:{342.497},{-32.805}) {\tikz\pgfuseplotmark{m6c};};
\node[stars] at (axis cs:{342.595},{-80.124}) {\tikz\pgfuseplotmark{m5cv};};
\node[stars] at (axis cs:{342.759},{-39.157}) {\tikz\pgfuseplotmark{m5c};};
\node[stars] at (axis cs:{342.837},{-29.536}) {\tikz\pgfuseplotmark{m6b};};
\node[stars] at (axis cs:{342.937},{-59.881}) {\tikz\pgfuseplotmark{m6c};};
\node[stars] at (axis cs:{343.042},{-63.189}) {\tikz\pgfuseplotmark{m6c};};
\node[stars] at (axis cs:{343.131},{-32.875}) {\tikz\pgfuseplotmark{m5avb};};
\node[stars] at (axis cs:{343.408},{-48.598}) {\tikz\pgfuseplotmark{m6b};};
\node[stars] at (axis cs:{343.665},{-70.074}) {\tikz\pgfuseplotmark{m6b};};
\node[stars] at (axis cs:{343.812},{-36.388}) {\tikz\pgfuseplotmark{m6c};};
\node[stars] at (axis cs:{343.965},{-31.633}) {\tikz\pgfuseplotmark{m6b};};
\node[stars] at (axis cs:{343.987},{-32.540}) {\tikz\pgfuseplotmark{m4c};};
\node[stars] at (axis cs:{344.100},{-31.565}) {\tikz\pgfuseplotmark{m6cv};};
\node[stars] at (axis cs:{344.199},{-47.969}) {\tikz\pgfuseplotmark{m6a};};
\node[stars] at (axis cs:{344.413},{-29.622}) {\tikz\pgfuseplotmark{m1cv};};
\node[stars] at (axis cs:{344.646},{-35.523}) {\tikz\pgfuseplotmark{m6c};};
\node[stars] at (axis cs:{344.899},{-29.462}) {\tikz\pgfuseplotmark{m6av};};
\node[stars] at (axis cs:{345.220},{-52.754}) {\tikz\pgfuseplotmark{m4b};};
\node[stars] at (axis cs:{345.282},{-50.950}) {\tikz\pgfuseplotmark{m6a};};
\node[stars] at (axis cs:{345.331},{-28.854}) {\tikz\pgfuseplotmark{m6a};};
\node[stars] at (axis cs:{345.642},{-36.421}) {\tikz\pgfuseplotmark{m6cb};};
\node[stars] at (axis cs:{345.874},{-34.749}) {\tikz\pgfuseplotmark{m5b};};
\node[stars] at (axis cs:{345.998},{-41.479}) {\tikz\pgfuseplotmark{m6a};};
\node[stars] at (axis cs:{346.165},{-53.965}) {\tikz\pgfuseplotmark{m5c};};
\node[stars] at (axis cs:{346.218},{-68.820}) {\tikz\pgfuseplotmark{m6a};};
\node[stars] at (axis cs:{346.720},{-43.520}) {\tikz\pgfuseplotmark{m4cb};};
\node[stars] at (axis cs:{346.723},{-38.892}) {\tikz\pgfuseplotmark{m6a};};
\node[stars] at (axis cs:{346.789},{-49.606}) {\tikz\pgfuseplotmark{m6c};};
\node[stars] at (axis cs:{346.812},{-50.687}) {\tikz\pgfuseplotmark{m6cb};};
\node[stars] at (axis cs:{347.088},{-28.824}) {\tikz\pgfuseplotmark{m6a};};
\node[stars] at (axis cs:{347.099},{-79.480}) {\tikz\pgfuseplotmark{m6b};};
\node[stars] at (axis cs:{347.148},{-73.586}) {\tikz\pgfuseplotmark{m6c};};
\node[stars] at (axis cs:{347.436},{-28.088}) {\tikz\pgfuseplotmark{m6b};};
\node[stars] at (axis cs:{347.436},{-67.876}) {\tikz\pgfuseplotmark{m6c};};
\node[stars] at (axis cs:{347.489},{-42.861}) {\tikz\pgfuseplotmark{m6a};};
\node[stars] at (axis cs:{347.541},{-40.591}) {\tikz\pgfuseplotmark{m6bv};};
\node[stars] at (axis cs:{347.551},{-66.858}) {\tikz\pgfuseplotmark{m6c};};
\node[stars] at (axis cs:{347.590},{-45.246}) {\tikz\pgfuseplotmark{m4b};};
\node[stars] at (axis cs:{348.052},{-80.912}) {\tikz\pgfuseplotmark{m6c};};
\node[stars] at (axis cs:{348.527},{-62.700}) {\tikz\pgfuseplotmark{m6b};};
\node[stars] at (axis cs:{348.744},{-41.105}) {\tikz\pgfuseplotmark{m6a};};
\node[stars] at (axis cs:{349.166},{-44.489}) {\tikz\pgfuseplotmark{m6b};};
\node[stars] at (axis cs:{349.208},{-41.194}) {\tikz\pgfuseplotmark{m6c};};
\node[stars] at (axis cs:{349.240},{-62.001}) {\tikz\pgfuseplotmark{m6a};};
\node[stars] at (axis cs:{349.286},{-28.438}) {\tikz\pgfuseplotmark{m6cv};};
\node[stars] at (axis cs:{349.357},{-58.236}) {\tikz\pgfuseplotmark{m4b};};
\node[stars] at (axis cs:{349.541},{-40.824}) {\tikz\pgfuseplotmark{m6a};};
\node[stars] at (axis cs:{349.583},{-67.471}) {\tikz\pgfuseplotmark{m6c};};
\node[stars] at (axis cs:{349.706},{-32.532}) {\tikz\pgfuseplotmark{m4c};};
\node[stars] at (axis cs:{349.786},{-79.472}) {\tikz\pgfuseplotmark{m6c};};
\node[stars] at (axis cs:{349.930},{-33.708}) {\tikz\pgfuseplotmark{m6c};};
\node[stars] at (axis cs:{350.209},{-50.306}) {\tikz\pgfuseplotmark{m6bb};};
\node[stars] at (axis cs:{350.737},{-60.056}) {\tikz\pgfuseplotmark{m6bv};};
\node[stars] at (axis cs:{350.939},{-43.124}) {\tikz\pgfuseplotmark{m6b};};
\node[stars] at (axis cs:{350.977},{-53.809}) {\tikz\pgfuseplotmark{m6bvb};};
\node[stars] at (axis cs:{351.055},{-51.891}) {\tikz\pgfuseplotmark{m6a};};
\node[stars] at (axis cs:{351.331},{-56.849}) {\tikz\pgfuseplotmark{m6a};};
\node[stars] at (axis cs:{351.652},{-52.722}) {\tikz\pgfuseplotmark{m6a};};
\node[stars] at (axis cs:{351.781},{-66.581}) {\tikz\pgfuseplotmark{m6c};};
\node[stars] at (axis cs:{351.788},{-50.157}) {\tikz\pgfuseplotmark{m6c};};
\node[stars] at (axis cs:{351.812},{-58.476}) {\tikz\pgfuseplotmark{m6a};};
\node[stars] at (axis cs:{352.003},{-35.544}) {\tikz\pgfuseplotmark{m6c};};
\node[stars] at (axis cs:{352.016},{-87.482}) {\tikz\pgfuseplotmark{m5c};};
\node[stars] at (axis cs:{352.253},{-44.498}) {\tikz\pgfuseplotmark{m6c};};
\node[stars] at (axis cs:{352.254},{-63.110}) {\tikz\pgfuseplotmark{m6av};};
\node[stars] at (axis cs:{352.862},{-44.843}) {\tikz\pgfuseplotmark{m6b};};
\node[stars] at (axis cs:{353.243},{-37.818}) {\tikz\pgfuseplotmark{m4cv};};
\node[stars] at (axis cs:{353.332},{-77.385}) {\tikz\pgfuseplotmark{m6a};};
\node[stars] at (axis cs:{353.769},{-42.615}) {\tikz\pgfuseplotmark{m5avb};};
\node[stars] at (axis cs:{354.462},{-45.492}) {\tikz\pgfuseplotmark{m5a};};
\node[stars] at (axis cs:{354.600},{-76.870}) {\tikz\pgfuseplotmark{m6b};};
\node[stars] at (axis cs:{354.866},{-46.638}) {\tikz\pgfuseplotmark{m6cb};};
\node[stars] at (axis cs:{355.159},{-32.073}) {\tikz\pgfuseplotmark{m5cv};};
\node[stars] at (axis cs:{356.006},{-45.083}) {\tikz\pgfuseplotmark{m6b};};
\node[stars] at (axis cs:{356.050},{-64.404}) {\tikz\pgfuseplotmark{m6a};};
\node[stars] at (axis cs:{356.106},{-70.490}) {\tikz\pgfuseplotmark{m6b};};
\node[stars] at (axis cs:{356.170},{-78.791}) {\tikz\pgfuseplotmark{m6a};};
\node[stars] at (axis cs:{356.505},{-40.182}) {\tikz\pgfuseplotmark{m6c};};
\node[stars] at (axis cs:{356.817},{-50.226}) {\tikz\pgfuseplotmark{m5c};};
\node[stars] at (axis cs:{357.231},{-28.130}) {\tikz\pgfuseplotmark{m5a};};
\node[stars] at (axis cs:{357.648},{-47.377}) {\tikz\pgfuseplotmark{m6c};};
\node[stars] at (axis cs:{358.027},{-82.019}) {\tikz\pgfuseplotmark{m5b};};
\node[stars] at (axis cs:{358.661},{-40.300}) {\tikz\pgfuseplotmark{m6b};};
\node[stars] at (axis cs:{358.819},{-31.921}) {\tikz\pgfuseplotmark{m6bv};};
\node[stars] at (axis cs:{359.333},{-62.956}) {\tikz\pgfuseplotmark{m6b};};
\node[stars] at (axis cs:{359.387},{-82.170}) {\tikz\pgfuseplotmark{m6a};};
\node[stars] at (axis cs:{359.396},{-64.298}) {\tikz\pgfuseplotmark{m5b};};
\node[stars] at (axis cs:{359.732},{-52.746}) {\tikz\pgfuseplotmark{m5b};};
\node[stars] at (axis cs:{359.866},{-29.485}) {\tikz\pgfuseplotmark{m6av};};
\node[stars] at (axis cs:{359.979},{-65.577}) {\tikz\pgfuseplotmark{m4c};};



\end{polaraxis}

\begin{polaraxis}[rotate=270,name=stars,at={($(base.center)+(+0.75pt,0pt)$)},anchor=center,axis lines=none]
\boldmath

\clip (6\tendegree,0\tendegree) arc (0:90:6\tendegree) -- 
(-2\tendegree,6\tendegree) -- (-2\tendegree,-6\tendegree) -- (0\tendegree,-6\tendegree)
--  (0\tendegree,-6\tendegree) arc (270:359.9999:6\tendegree) -- cycle ;

\node[ecliptics-empty,pin={[pin distance=-0.4\onedegree,interest-label]-90:{NEP}}] at (axis cs:{18*15},{+66+33/ 60  }) {\pgfuseplotmark{+}} ; %  ecliptic pole
\node[ecliptics-empty,pin={[pin distance=-0.4\onedegree,interest-label]90:{SEP}}] at (axis cs:{6*15},{-66-33/ 60  }) {\pgfuseplotmark{+}} ; %  ecliptic pole

\node[designation-label,anchor=south west]  at (axis cs:{06*15-23/4},{-52+41/ 60  })  {Canopus};
\node[designation-label,anchor=north west]  at (axis cs:{ 8*15- 9/4},{-47+20/ 60  })  {Suhail al Muhlif};
\node[designation-label,anchor=north west]  at (axis cs:{ 8*15-22/4},{-59+30/ 60  })  {Avior};
\node[designation-label,anchor=north west]  at (axis cs:{ 9*15-13/4},{-69+43/ 60  })  {Miaplacidus};
\node[designation-label,anchor=south west]  at (axis cs:{12*15-26/4},{-63+05/ 60  })  {Acrux};
\node[designation-label,anchor=north east]  at (axis cs:{12*15-31/4},{-57+06/ 60  })  {Gacrux};
%\node[designation-label,anchor=south west]  at (axis cs:{12*15-47/4},{-59+41/ 60  })  {Becrux};
\node[designation-label,anchor=south west]  at (axis cs:{14*15- 3/4},{-60+22/ 60  })  {Agena};
\node[designation-label,anchor=south east]  at (axis cs:{14*15-39/4},{-60+50/ 60  })  {Rigel Kentaurus};
\node[designation-label,anchor=south west]  at (axis cs:{17*15-33/4},{-37+06/ 60  })  {Shaula};
\node[designation-label,anchor=south west]  at (axis cs:{17*15-37/4},{-42+59/ 60  })  {Sargas};
\node[designation-label,anchor=south east]  at (axis cs:{18*15-24/4},{-34+23/ 60  })  {Kaus Australis};
\node[designation-label,anchor=north west]  at (axis cs:{20*15-25/4},{-56+44/ 60  })  {Peacock};


\node[pin={[pin distance=-0.4\onedegree,Bayer]180:{$\alpha$}}] at (axis cs:{156.788},{-31.068}) {}; % Ant,  4.27 
\node[pin={[pin distance=-0.6\onedegree,Bayer]00:{$\delta$}}] at (axis cs:{157.397},{-30.607}) {}; % Ant,  5.58 
\node[pin={[pin distance=-0.6\onedegree,Bayer]00:{$\epsilon$}}] at (axis cs:{142.311},{-35.951}) {}; % Ant,  4.49 
\node[pin={[pin distance=-0.6\onedegree,Bayer]00:{$\eta$}}] at (axis cs:{149.718},{-35.891}) {}; % Ant,  5.23 
\node[pin={[pin distance=-0.6\onedegree,Bayer]00:{$\iota$}}] at (axis cs:{164.179},{-37.138}) {}; % Ant,  4.60 
\node[pin={[pin distance=-0.6\onedegree,Bayer]00:{$\vartheta$}}] at (axis cs:{146.050},{-27.769}) {}; % Ant,  4.79 
%\node[pin={[pin distance=-0.6\onedegree,Bayer]00:{$\zeta^1$}}] at (axis cs:{142.692},{-31.889}) {}; % Ant,  6.18 
\node[pin={[pin distance=-0.6\onedegree,Bayer]00:{$\zeta^{1,2}$}}] at (axis cs:{142.884},{-31.872}) {}; % Ant,  5.91

\node[pin={[pin distance=-0.6\onedegree,Bayer]00:{$\alpha$}}] at (axis cs:{221.965},{-79.045}) {}; % Aps,  3.81 
\node[pin={[pin distance=-0.6\onedegree,Bayer]00:{$\beta$}}] at (axis cs:{250.769},{-77.517}) {}; % Aps,  4.23 
\node[pin={[pin distance=-0.6\onedegree,Bayer]00:{$\delta^{1,2}$}}] at (axis cs:{245.087},{-78.696}) {}; % Aps,  4.74 
%\node[pin={[pin distance=-0.6\onedegree,Bayer]00:{$\delta^2$}}] at (axis cs:{245.112},{-78.667}) {}; % Aps,  5.26 
\node[pin={[pin distance=-0.4\onedegree,Bayer]-90:{$\epsilon$}}] at (axis cs:{215.596},{-80.109}) {}; % Aps,  5.06 
\node[pin={[pin distance=-0.6\onedegree,Bayer]90:{$\eta$}}] at (axis cs:{214.558},{-81.008}) {}; % Aps,  4.90 
\node[pin={[pin distance=-0.4\onedegree,Bayer]90:{$\gamma$}}] at (axis cs:{248.363},{-78.897}) {}; % Aps,  3.87 
\node[pin={[pin distance=-0.6\onedegree,Bayer]00:{$\iota$}}] at (axis cs:{260.524},{-70.123}) {}; % Aps,  5.40 
\node[pin={[pin distance=-0.6\onedegree,Bayer]00:{$\kappa^1$}}] at (axis cs:{232.878},{-73.389}) {}; % Aps,  5.40 
\node[pin={[pin distance=-0.4\onedegree,Bayer]90:{$\kappa^2$}}] at (axis cs:{235.089},{-73.446}) {}; % Aps,  5.65 
\node[pin={[pin distance=-0.6\onedegree,Bayer]00:{$\vartheta$}}] at (axis cs:{211.333},{-76.797}) {}; % Aps,  5.69 
\node[pin={[pin distance=-0.6\onedegree,Bayer]180:{$\zeta$}}] at (axis cs:{260.498},{-67.771}) {}; % Aps,  4.76

\node[pin={[pin distance=-0.3\onedegree,Bayer]00:{$\alpha$}}] at (axis cs:{262.960},{-49.876}) {}; % Ara,  2.85 
\node[pin={[pin distance=-0.4\onedegree,Bayer]00:{$\beta$}}] at (axis cs:{261.325},{-55.530}) {}; % Ara,  2.82 
\node[pin={[pin distance=-0.6\onedegree,Bayer]00:{$\delta$}}] at (axis cs:{262.775},{-60.684}) {}; % Ara,  3.60 
\node[pin={[pin distance=-0.6\onedegree,Bayer]180:{$\epsilon^1$}}] at (axis cs:{254.896},{-53.160}) {}; % Ara,  4.05 
\node[pin={[pin distance=-0.6\onedegree,Bayer]00:{$\epsilon^2$}}] at (axis cs:{255.786},{-53.237}) {}; % Ara,  5.28 
\node[pin={[pin distance=-0.4\onedegree,Bayer]00:{$\eta$}}] at (axis cs:{252.446},{-59.041}) {}; % Ara,  3.76 
\node[pin={[pin distance=-0.4\onedegree,Bayer]180:{$\gamma$}}] at (axis cs:{261.349},{-56.378}) {}; % Ara,  3.32 
\node[pin={[pin distance=-0.6\onedegree,Bayer]00:{$\iota$}}] at (axis cs:{260.817},{-47.468}) {}; % Ara,  5.26 
\node[pin={[pin distance=-0.6\onedegree,Bayer]180:{$\kappa$}}] at (axis cs:{261.500},{-50.634}) {}; % Ara,  5.20 
\node[pin={[pin distance=-0.6\onedegree,Bayer]00:{$\lambda$}}] at (axis cs:{265.099},{-49.415}) {}; % Ara,  4.76 
\node[pin={[pin distance=-0.6\onedegree,Bayer]00:{$\mu$}}] at (axis cs:{266.036},{-51.834}) {}; % Ara,  5.12 
\node[pin={[pin distance=-0.6\onedegree,Bayer]180:{$\pi$}}] at (axis cs:{264.523},{-54.500}) {}; % Ara,  5.25 
\node[pin={[pin distance=-0.6\onedegree,Bayer]00:{$\sigma$}}] at (axis cs:{263.915},{-46.506}) {}; % Ara,  4.58 
\node[pin={[pin distance=-0.6\onedegree,Bayer]00:{$\vartheta$}}] at (axis cs:{271.658},{-50.091}) {}; % Ara,  3.67 
\node[pin={[pin distance=-0.4\onedegree,Bayer]00:{$\zeta$}}] at (axis cs:{254.655},{-55.990}) {}; % Ara,  3.11

\node[pin={[pin distance=-0.4\onedegree,Bayer]-90:{$\alpha$}}] at (axis cs:{70.140},{-41.864}) {}; % Cae,  4.45 
\node[pin={[pin distance=-0.6\onedegree,Bayer]00:{$\beta$}}] at (axis cs:{70.515},{-37.144}) {}; % Cae,  5.04 
\node[pin={[pin distance=-0.6\onedegree,Bayer]00:{$\delta$}}] at (axis cs:{67.709},{-44.954}) {}; % Cae,  5.07 
\node[pin={[pin distance=-0.4\onedegree,Bayer]90:{$\gamma^{1,2}$}}] at (axis cs:{76.102},{-35.483}) {}; % Cae,  4.57 
%\node[pin={[pin distance=-0.6\onedegree,Bayer]00:{$\gamma^2$}}] at (axis cs:{76.109},{-35.705}) {}; % Cae,  6.33 
\node[pin={[pin distance=-0.6\onedegree,Bayer]00:{$\zeta$}}] at (axis cs:{71.957},{-30.020}) {}; % Cae,  6.35

\node[pin={[pin distance=-0.\onedegree,Bayer]00:{$\alpha$}}] at (axis cs:{95.988},{-52.695}) {}; % Car, -0.63 
\node[pin={[pin distance=-0.5\onedegree,Bayer]45:{$\beta$}}] at (axis cs:{138.300},{-69.717}) {}; % Car,  1.67 
\node[pin={[pin distance=-0.3\onedegree,Bayer]180:{$\epsilon$}}] at (axis cs:{125.628},{-59.509}) {}; % Car,  1.95 
\node[pin={[pin distance=-0.2\onedegree,Bayer]90:{$\iota$}}] at (axis cs:{139.273},{-59.275}) {}; % Car,  2.25 
\node[pin={[pin distance=-0.4\onedegree,Bayer]90:{$\chi$}}] at (axis cs:{119.195},{-52.982}) {}; % Car,  3.46 
\node[interest,pin={[pin distance=-0.6\onedegree,Bayer]-90:{$\eta$}}] at (axis cs:{161.265},{-59.684}) {+}; % Car,  6.21 
\node[pin={[pin distance=-0.4\onedegree,Bayer]00:{$\omega$}}] at (axis cs:{153.434},{-70.038}) {}; % Car,  3.30 
\node[pin={[pin distance=-0.4\onedegree,Bayer]00:{$\upsilon$}}] at (axis cs:{146.776},{-65.072}) {}; % Car,  3.01 
\node[pin={[pin distance=-0.2\onedegree,Bayer]00:{$\vartheta$}}] at (axis cs:{160.739},{-64.394}) {}; % Car,  2.74 

\node[pin={[pin distance=-0.2\onedegree,Bayer]180:{$\alpha^1$}}] at (axis cs:{219.902},{-60.834}) {}; % Cen, -0.01 
\node[pin={[pin distance=-0.2\onedegree,Bayer]00:{$\beta$}}] at (axis cs:{210.956},{-60.373}) {}; % Cen,  0.64 
\node[pin={[pin distance=-0.4\onedegree,Bayer]00:{$\epsilon$}}] at (axis cs:{204.972},{-53.466}) {}; % Cen,  2.28 
\node[pin={[pin distance=-0.4\onedegree,Bayer]00:{$\eta$}}] at (axis cs:{218.877},{-42.158}) {}; % Cen,  2.34 
\node[pin={[pin distance=-0.4\onedegree,Bayer]00:{$\gamma$}}] at (axis cs:{190.379},{-48.960}) {}; % Cen,  2.18 
\node[pin={[pin distance=-0.4\onedegree,Bayer]00:{$\vartheta$}}] at (axis cs:{211.671},{-36.370}) {}; % Cen,  2.08 
\node[pin={[pin distance=-0.6\onedegree,Bayer]00:{$\chi$}}] at (axis cs:{211.512},{-41.179}) {}; % Cen,  4.36 
\node[pin={[pin distance=-0.4\onedegree,Bayer]00:{$\delta$}}] at (axis cs:{182.090},{-50.722}) {}; % Cen,  2.58 
\node[pin={[pin distance=-0.4\onedegree,Bayer]180:{$\iota$}}] at (axis cs:{200.149},{-36.712}) {}; % Cen,  2.76 
\node[pin={[pin distance=-0.3\onedegree,Bayer]-90:{$\kappa$}}] at (axis cs:{224.790},{-42.104}) {}; % Cen,  3.13 
\node[pin={[pin distance=-0.3\onedegree,Bayer]-90:{$\lambda$}}] at (axis cs:{173.945},{-63.020}) {}; % Cen,  3.12 
\node[pin={[pin distance=-0.4\onedegree,Bayer]00:{$\mu$}}] at (axis cs:{207.404},{-42.474}) {}; % Cen,  3.47 
\node[pin={[pin distance=-0.6\onedegree,Bayer]00:{$\nu$}}] at (axis cs:{207.376},{-41.688}) {}; % Cen,  3.40 
\node[pin={[pin distance=-0.6\onedegree,Bayer]00:{$\omicron^{1,2}$}}] at (axis cs:{172.942},{-59.442}) {}; % Cen,  5.09 
%\node[pin={[pin distance=-0.6\onedegree,Bayer]00:{$\omicron^2$}}] at (axis cs:{172.953},{-59.515}) {}; % Cen,  5.19 
\node[pin={[pin distance=-0.6\onedegree,Bayer]00:{$\pi$}}] at (axis cs:{170.252},{-54.491}) {}; % Cen,  3.89 
\node[pin={[pin distance=-0.6\onedegree,Bayer]00:{$\psi$}}] at (axis cs:{215.139},{-37.885}) {}; % Cen,  4.05 
\node[pin={[pin distance=-0.6\onedegree,Bayer]00:{$\rho$}}] at (axis cs:{182.913},{-52.368}) {}; % Cen,  3.96 
\node[pin={[pin distance=-0.6\onedegree,Bayer]00:{$\sigma$}}] at (axis cs:{187.010},{-50.230}) {}; % Cen,  3.91 
\node[pin={[pin distance=-0.4\onedegree,Bayer]180:{$\tau$}}] at (axis cs:{189.426},{-48.541}) {}; % Cen,  3.85 
\node[pin={[pin distance=-0.5\onedegree,Bayer]00:{$\upsilon^1$}}] at (axis cs:{209.670},{-44.804}) {}; % Cen,  3.86 
\node[pin={[pin distance=-0.5\onedegree,Bayer]00:{$\upsilon^2$}}] at (axis cs:{210.431},{-45.603}) {}; % Cen,  4.34 
\node[pin={[pin distance=-0.4\onedegree,Bayer]-90:{$\varphi$}}] at (axis cs:{209.568},{-42.101}) {}; % Cen,  3.82 
\node[pin={[pin distance=-0.4\onedegree,Bayer]90:{$\xi^1$}}] at (axis cs:{195.889},{-49.527}) {}; % Cen,  4.83 
\node[pin={[pin distance=-0.4\onedegree,Bayer]45:{$\xi^2$}}] at (axis cs:{196.728},{-49.906}) {}; % Cen,  4.28 
\node[pin={[pin distance=-0.4\onedegree,Bayer]00:{$\zeta$}}] at (axis cs:{208.885},{-47.288}) {}; % Cen,  2.53 
\node[pin={[pin distance=-0.5\onedegree,Flaamsted]00:{$  1$}}] at (axis cs:{206.422},{-33.044}) {}; % Cen,  4.23 
\node[pin={[pin distance=-0.5\onedegree,Flaamsted]00:{$  2$}}] at (axis cs:{207.361},{-34.451}) {}; % Cen,  4.29 
\node[pin={[pin distance=-0.5\onedegree,Flaamsted]00:{$  3$}}] at (axis cs:{207.957},{-32.994}) {}; % Cen,  4.55 
\node[pin={[pin distance=-0.5\onedegree,Flaamsted]00:{$  4$}}] at (axis cs:{208.302},{-31.927}) {}; % Cen,  4.74


\node[pin={[pin distance=-0.4\onedegree,Bayer]-90:{$\alpha$}}] at (axis cs:{124.631},{-76.920}) {}; % Cha,  4.06 
\node[pin={[pin distance=-0.6\onedegree,Bayer]00:{$\beta$}}] at (axis cs:{184.587},{-79.312}) {}; % Cha,  4.25 
\node[pin={[pin distance=-0.6\onedegree,Bayer]00:{$\delta^{1,2}$}}] at (axis cs:{161.318},{-80.470}) {}; % Cha,  5.50 
%\node[pin={[pin distance=-0.6\onedegree,Bayer]00:{$\delta^2$}}] at (axis cs:{161.446},{-80.540}) {}; % Cha,  4.45 
\node[pin={[pin distance=-0.6\onedegree,Bayer]00:{$\epsilon$}}] at (axis cs:{179.907},{-78.222}) {}; % Cha,  4.91 
\node[pin={[pin distance=-0.4\onedegree,Bayer]-90:{$\eta$}}] at (axis cs:{130.331},{-78.963}) {}; % Cha,  5.46 
\node[pin={[pin distance=-0.6\onedegree,Bayer]00:{$\gamma$}}] at (axis cs:{158.867},{-78.608}) {}; % Cha,  4.10 
\node[pin={[pin distance=-0.6\onedegree,Bayer]180:{$\iota$}}] at (axis cs:{141.038},{-80.787}) {}; % Cha,  5.34 
\node[pin={[pin distance=-0.6\onedegree,Bayer]00:{$\kappa$}}] at (axis cs:{181.194},{-76.519}) {}; % Cha,  5.03 
\node[pin={[pin distance=-0.6\onedegree,Bayer]00:{$\mu^1$}}] at (axis cs:{150.182},{-82.214}) {}; % Cha,  5.52 
\node[pin={[pin distance=-0.6\onedegree,Bayer]00:{$\nu$}}] at (axis cs:{146.586},{-76.776}) {}; % Cha,  5.44 
\node[pin={[pin distance=-0.6\onedegree,Bayer]00:{$\pi$}}] at (axis cs:{174.315},{-75.896}) {}; % Cha,  5.65 
\node[pin={[pin distance=-0.6\onedegree,Bayer]00:{$\vartheta$}}] at (axis cs:{125.161},{-77.484}) {}; % Cha,  4.34 
\node[pin={[pin distance=-0.6\onedegree,Bayer]00:{$\zeta$}}] at (axis cs:{143.472},{-80.941}) {}; % Cha,  5.07

\node[pin={[pin distance=-0.5\onedegree,Bayer]00:{$\alpha$}}] at (axis cs:{220.627},{-64.975}) {}; % Cir,  3.18 
\node[pin={[pin distance=-0.6\onedegree,Bayer]00:{$\beta$}}] at (axis cs:{229.379},{-58.801}) {}; % Cir,  4.07 
\node[pin={[pin distance=-0.6\onedegree,Bayer]-90:{$\delta$}}] at (axis cs:{229.237},{-60.957}) {}; % Cir,  5.08 
\node[pin={[pin distance=-0.4\onedegree,Bayer]-90:{$\epsilon$}}] at (axis cs:{229.412},{-63.610}) {}; % Cir,  4.85 
\node[pin={[pin distance=-0.6\onedegree,Bayer]00:{$\eta$}}] at (axis cs:{226.201},{-64.031}) {}; % Cir,  5.17 
\node[pin={[pin distance=-0.6\onedegree,Bayer]00:{$\gamma$}}] at (axis cs:{230.844},{-59.321}) {}; % Cir,  4.51 
\node[pin={[pin distance=-0.6\onedegree,Bayer]00:{$\vartheta$}}] at (axis cs:{224.183},{-62.781}) {}; % Cir,  5.08 
\node[pin={[pin distance=-0.6\onedegree,Bayer]00:{$\zeta$}}] at (axis cs:{223.677},{-65.991}) {}; % Cir,  6.08

%\node[pin={[pin distance=-0.4\onedegree,Bayer]00:{$\epsilon$}}] at (axis cs:{104.656},{-28.972}) {}; % CMa,  1.53
%\node[pin={[pin distance=-0.4\onedegree,Bayer]00:{$\eta$}}] at (axis cs:{111.024},{-29.303}) {}; % CMa,  2.46 
\node[pin={[pin distance=-0.4\onedegree,Bayer]-90:{$\kappa$}}] at (axis cs:{102.460},{-32.508}) {}; % CMa,  3.53 
\node[pin={[pin distance=-0.4\onedegree,Bayer]90:{$\lambda$}}] at (axis cs:{97.043},{-32.580}) {}; % CMa,  4.47 
%\node[pin={[pin distance=-0.6\onedegree,Bayer]00:{$\sigma$}}] at (axis cs:{105.430},{-27.935}) {}; % CMa,  3.47 
\node[pin={[pin distance=-0.4\onedegree,Bayer]00:{$\zeta$}}] at (axis cs:{95.078},{-30.063}) {}; % CMa,  3.02 
\node[pin={[pin distance=-0.6\onedegree,Flaamsted]00:{$ 10$}}] at (axis cs:{101.119},{-31.070}) {}; % CMa,  5.23

\node[pin={[pin distance=-0.4\onedegree,Bayer]00:{$\alpha$}}] at (axis cs:{84.912},{-34.074}) {}; % Col,  2.66 
\node[pin={[pin distance=-0.5\onedegree,Bayer]180:{$\beta$}}] at (axis cs:{87.740},{-35.768}) {}; % Col,  3.11 
\node[pin={[pin distance=-0.5\onedegree,Bayer]00:{$\delta$}}] at (axis cs:{95.528},{-33.436}) {}; % Col,  3.85 
\node[pin={[pin distance=-0.6\onedegree,Bayer]00:{$\epsilon$}}] at (axis cs:{82.803},{-35.470}) {}; % Col,  3.87 
\node[pin={[pin distance=-0.6\onedegree,Bayer]180:{$\eta$}}] at (axis cs:{89.787},{-42.815}) {}; % Col,  3.94 
\node[pin={[pin distance=-0.6\onedegree,Bayer]00:{$\gamma$}}] at (axis cs:{89.384},{-35.283}) {}; % Col,  4.36 
\node[pin={[pin distance=-0.6\onedegree,Bayer]00:{$\kappa$}}] at (axis cs:{94.138},{-35.140}) {}; % Col,  4.37 
\node[pin={[pin distance=-0.5\onedegree,Bayer]180:{$\lambda$}}] at (axis cs:{88.279},{-33.801}) {}; % Col,  4.87 
\node[pin={[pin distance=-0.6\onedegree,Bayer]00:{$\mu$}}] at (axis cs:{86.500},{-32.306}) {}; % Col,  5.16 
\node[pin={[pin distance=-0.6\onedegree,Bayer]00:{$\nu^1$}}] at (axis cs:{84.319},{-27.871}) {}; % Col,  6.15 
\node[pin={[pin distance=-0.6\onedegree,Bayer]00:{$\nu^2$}}] at (axis cs:{84.436},{-28.690}) {}; % Col,  5.28 
\node[pin={[pin distance=-0.6\onedegree,Bayer]00:{$\omicron$}}] at (axis cs:{79.371},{-34.895}) {}; % Col,  4.82 
\node[pin={[pin distance=-0.5\onedegree,Bayer]90:{$\pi^1$}}] at (axis cs:{91.671},{-42.299}) {}; % Col,  6.16 
\node[pin={[pin distance=-0.5\onedegree,Bayer]-90:{$\pi^2$}}] at (axis cs:{91.970},{-42.154}) {}; % Col,  5.50 
\node[pin={[pin distance=-0.6\onedegree,Bayer]180:{$\sigma$}}] at (axis cs:{89.087},{-31.382}) {}; % Col,  5.52 
\node[pin={[pin distance=-0.6\onedegree,Bayer]00:{$\vartheta$}}] at (axis cs:{91.882},{-37.253}) {}; % Col,  5.00 
\node[pin={[pin distance=-0.6\onedegree,Bayer]00:{$\xi$}}] at (axis cs:{88.875},{-37.121}) {}; % Col,  4.96

\node[pin={[pin distance=-0.4\onedegree,Bayer]90:{$\alpha$}}] at (axis cs:{287.368},{-37.904}) {}; % CrA,  4.10 
\node[pin={[pin distance=-0.4\onedegree,Bayer]90:{$\beta$}}] at (axis cs:{287.507},{-39.341}) {}; % CrA,  4.11 
\node[pin={[pin distance=-0.6\onedegree,Bayer]00:{$\delta$}}] at (axis cs:{287.087},{-40.496}) {}; % CrA,  4.58 
\node[pin={[pin distance=-0.6\onedegree,Bayer]180:{$\epsilon$}}] at (axis cs:{284.681},{-37.107}) {}; % CrA,  4.83 
\node[pin={[pin distance=-0.6\onedegree,Bayer]180:{$\eta^1$}}] at (axis cs:{282.210},{-43.680}) {}; % CrA,  5.46 
\node[pin={[pin distance=-0.6\onedegree,Bayer]00:{$\eta^2$}}] at (axis cs:{282.396},{-43.434}) {}; % CrA,  5.60 
\node[pin={[pin distance=-0.4\onedegree,Bayer]-90:{$\gamma$}}] at (axis cs:{286.605},{-37.063}) {}; % CrA,  4.23 
\node[pin={[pin distance=-0.6\onedegree,Bayer]00:{$\kappa^2$}}] at (axis cs:{278.346},{-38.726}) {}; % CrA,  5.61 
\node[pin={[pin distance=-0.6\onedegree,Bayer]00:{$\lambda$}}] at (axis cs:{280.946},{-38.323}) {}; % CrA,  5.12 
\node[pin={[pin distance=-0.6\onedegree,Bayer]00:{$\mu$}}] at (axis cs:{281.936},{-40.406}) {}; % CrA,  5.21 
\node[pin={[pin distance=-0.6\onedegree,Bayer]00:{$\vartheta$}}] at (axis cs:{278.376},{-42.312}) {}; % CrA,  4.62 
\node[pin={[pin distance=-0.6\onedegree,Bayer]00:{$\zeta$}}] at (axis cs:{285.779},{-42.095}) {}; % CrA,  4.74


\node[pin={[pin distance=-0.4\onedegree,Bayer]00:{$\alpha^1$}}] at (axis cs:{186.650},{-63.099}) {}; % Cru,  1.28 
\node[pin={[pin distance=-0.4\onedegree,Bayer]00:{$\beta$}}] at (axis cs:{191.930},{-59.689}) {}; % Cru,  1.31 
\node[pin={[pin distance=-0.4\onedegree,Bayer]00:{$\gamma$}}] at (axis cs:{187.791},{-57.113}) {}; % Cru,  1.65 
\node[pin={[pin distance=-0.6\onedegree,Bayer]00:{$\delta$}}] at (axis cs:{183.786},{-58.749}) {}; % Cru,  2.79 
\node[pin={[pin distance=-0.6\onedegree,Bayer]00:{$\epsilon$}}] at (axis cs:{185.340},{-60.401}) {}; % Cru,  3.59 
\node[pin={[pin distance=-0.4\onedegree,Bayer]-90:{$\eta$}}] at (axis cs:{181.720},{-64.614}) {}; % Cru,  4.14 
\node[pin={[pin distance=-0.6\onedegree,Bayer]180:{$\iota$}}] at (axis cs:{191.409},{-60.981}) {}; % Cru,  4.69 
\node[pin={[pin distance=-0.4\onedegree,Bayer]90:{$\kappa$}}] at (axis cs:{193.454},{-60.376}) {}; % Cru,  5.90 
\node[pin={[pin distance=-0.4\onedegree,Bayer]-90:{$\lambda$}}] at (axis cs:{193.663},{-59.146}) {}; % Cru,  4.61 
\node[pin={[pin distance=-0.6\onedegree,Bayer]180:{$\mu^1$}}] at (axis cs:{193.648},{-57.178}) {}; % Cru,  4.00 
%\node[pin={[pin distance=-0.6\onedegree,Bayer]00:{$\vartheta^1$}}] at (axis cs:{180.756},{-63.313}) {}; % Cru,  4.33 
\node[pin={[pin distance=-0.6\onedegree,Bayer]00:{$\vartheta^{1,2}$}}] at (axis cs:{181.080},{-63.165}) {}; % Cru,  4.73 
\node[pin={[pin distance=-0.6\onedegree,Bayer]00:{$\zeta$}}] at (axis cs:{184.609},{-64.003}) {}; % Cru,  4.05

\node[pin={[pin distance=-0.6\onedegree,Bayer]00:{$\alpha$}}] at (axis cs:{68.499},{-55.045}) {}; % Dor,  3.26 
\node[pin={[pin distance=-0.6\onedegree,Bayer]00:{$\beta$}}] at (axis cs:{83.406},{-62.490}) {}; % Dor,  3.76 
\node[pin={[pin distance=-0.6\onedegree,Bayer]00:{$\delta$}}] at (axis cs:{86.193},{-65.735}) {}; % Dor,  4.34 
\node[pin={[pin distance=-0.6\onedegree,Bayer]00:{$\epsilon$}}] at (axis cs:{87.473},{-66.901}) {}; % Dor,  5.10 
\node[pin={[pin distance=-0.6\onedegree,Bayer]00:{$\eta^1$}}] at (axis cs:{91.539},{-66.040}) {}; % Dor,  5.70 
\node[pin={[pin distance=-0.4\onedegree,Bayer]-90:{$\eta^2$}}] at (axis cs:{92.812},{-65.589}) {}; % Dor,  5.02 
\node[pin={[pin distance=-0.6\onedegree,Bayer]00:{$\gamma$}}] at (axis cs:{64.007},{-51.487}) {}; % Dor,  4.26 
\node[pin={[pin distance=-0.6\onedegree,Bayer]00:{$\kappa$}}] at (axis cs:{71.088},{-59.732}) {}; % Dor,  5.28 
\node[pin={[pin distance=-0.6\onedegree,Bayer]00:{$\lambda$}}] at (axis cs:{81.580},{-58.912}) {}; % Dor,  5.13 
\node[pin={[pin distance=-0.6\onedegree,Bayer]00:{$\nu$}}] at (axis cs:{92.184},{-68.843}) {}; % Dor,  5.05 
\node[pin={[pin distance=-0.6\onedegree,Bayer]00:{$\pi^1$}}] at (axis cs:{95.659},{-69.984}) {}; % Dor,  5.55 
\node[pin={[pin distance=-0.6\onedegree,Bayer]180:{$\pi^2$}}] at (axis cs:{96.369},{-69.690}) {}; % Dor,  5.38 
\node[pin={[pin distance=-0.6\onedegree,Bayer]00:{$\vartheta$}}] at (axis cs:{78.439},{-67.185}) {}; % Dor,  4.80 
\node[pin={[pin distance=-0.6\onedegree,Bayer]00:{$\zeta$}}] at (axis cs:{76.378},{-57.473}) {}; % Dor,  4.71

\node[pin={[pin distance=-0.4\onedegree,Bayer]00:{$\alpha$}}] at (axis cs:{24.429},{-57.237}) {}; % Eri,  0.54 
\node[pin={[pin distance=-0.6\onedegree,Bayer]00:{$\chi$}}] at (axis cs:{28.989},{-51.609}) {}; % Eri,  3.72 
\node[pin={[pin distance=-0.6\onedegree,Bayer]00:{$\iota$}}] at (axis cs:{40.167},{-39.855}) {}; % Eri,  4.12 
\node[pin={[pin distance=-0.6\onedegree,Bayer]00:{$\kappa$}}] at (axis cs:{36.746},{-47.704}) {}; % Eri,  4.25 
\node[pin={[pin distance=-0.6\onedegree,Bayer]00:{$\upsilon^1$}}] at (axis cs:{68.377},{-29.766}) {}; % Eri,  4.51 
\node[pin={[pin distance=-0.6\onedegree,Bayer]00:{$\upsilon^2$}}] at (axis cs:{68.888},{-30.562}) {}; % Eri,  3.81 
\node[pin={[pin distance=-0.6\onedegree,Bayer]00:{$\upsilon^4$}}] at (axis cs:{64.474},{-33.798}) {}; % Eri,  3.56 
\node[pin={[pin distance=-0.6\onedegree,Bayer]00:{$\varphi$}}] at (axis cs:{34.127},{-51.512}) {}; % Eri,  3.55 
\node[pin={[pin distance=-0.6\onedegree,Bayer]00:{$\vartheta^1$}}] at (axis cs:{44.565},{-40.305}) {}; % Eri,  3.22 
\node[pin={[pin distance=-0.6\onedegree,Flaamsted]00:{$ 43$}}] at (axis cs:{66.009},{-34.017}) {}; % Eri,  3.96 
\node[pin={[pin distance=-0.6\onedegree,Bayer]00:{$\alpha$}}] at (axis cs:{48.019},{-28.988}) {}; % For,  3.80 
\node[pin={[pin distance=-0.6\onedegree,Bayer]00:{$\beta$}}] at (axis cs:{42.273},{-32.406}) {}; % For,  4.46 
\node[pin={[pin distance=-0.6\onedegree,Bayer]00:{$\chi^1$}}] at (axis cs:{51.483},{-35.921}) {}; % For,  6.39 
\node[pin={[pin distance=-0.6\onedegree,Bayer]00:{$\chi^2$}}] at (axis cs:{51.889},{-35.681}) {}; % For,  5.70 
\node[pin={[pin distance=-0.6\onedegree,Bayer]00:{$\delta$}}] at (axis cs:{55.562},{-31.938}) {}; % For,  4.98 
\node[pin={[pin distance=-0.6\onedegree,Bayer]00:{$\epsilon$}}] at (axis cs:{45.407},{-28.091}) {}; % For,  5.88 
\node[pin={[pin distance=-0.6\onedegree,Bayer]00:{$\eta^2$}}] at (axis cs:{42.562},{-35.843}) {}; % For,  5.93 
\node[pin={[pin distance=-0.6\onedegree,Bayer]00:{$\eta^3$}}] at (axis cs:{42.668},{-35.676}) {}; % For,  5.48 
\node[pin={[pin distance=-0.6\onedegree,Bayer]00:{$\gamma^2$}}] at (axis cs:{42.476},{-27.942}) {}; % For,  5.39 
\node[pin={[pin distance=-0.6\onedegree,Bayer]00:{$\iota^1$}}] at (axis cs:{39.039},{-30.045}) {}; % For,  5.73 
\node[pin={[pin distance=-0.6\onedegree,Bayer]00:{$\iota^2$}}] at (axis cs:{39.578},{-30.194}) {}; % For,  5.84 
\node[pin={[pin distance=-0.6\onedegree,Bayer]00:{$\lambda^1$}}] at (axis cs:{38.279},{-34.650}) {}; % For,  5.90 
\node[pin={[pin distance=-0.6\onedegree,Bayer]00:{$\lambda^2$}}] at (axis cs:{39.244},{-34.578}) {}; % For,  5.78 
\node[pin={[pin distance=-0.6\onedegree,Bayer]00:{$\mu$}}] at (axis cs:{33.227},{-30.724}) {}; % For,  5.27 
\node[pin={[pin distance=-0.6\onedegree,Bayer]00:{$\nu$}}] at (axis cs:{31.123},{-29.297}) {}; % For,  4.70 
\node[pin={[pin distance=-0.6\onedegree,Bayer]00:{$\omega$}}] at (axis cs:{38.461},{-28.232}) {}; % For,  4.97 
\node[pin={[pin distance=-0.6\onedegree,Bayer]00:{$\pi$}}] at (axis cs:{30.311},{-30.002}) {}; % For,  5.36 
\node[pin={[pin distance=-0.6\onedegree,Bayer]00:{$\psi$}}] at (axis cs:{43.393},{-38.437}) {}; % For,  5.93 
\node[pin={[pin distance=-0.6\onedegree,Bayer]00:{$\rho$}}] at (axis cs:{56.984},{-30.168}) {}; % For,  5.53 
\node[pin={[pin distance=-0.6\onedegree,Bayer]00:{$\sigma$}}] at (axis cs:{56.614},{-29.338}) {}; % For,  5.92 
\node[pin={[pin distance=-0.6\onedegree,Bayer]00:{$\tau$}}] at (axis cs:{54.699},{-27.943}) {}; % For,  6.01 
\node[pin={[pin distance=-0.6\onedegree,Bayer]00:{$\varphi$}}] at (axis cs:{37.007},{-33.811}) {}; % For,  5.13 
\node[pin={[pin distance=-0.4\onedegree,Bayer]00:{$\alpha$}}] at (axis cs:{332.058},{-46.961}) {}; % Gru,  1.77 
\node[pin={[pin distance=-0.4\onedegree,Bayer]00:{$\beta$}}] at (axis cs:{340.667},{-46.885}) {}; % Gru,  2.12 
\node[pin={[pin distance=-0.6\onedegree,Bayer]00:{$\delta^1$}}] at (axis cs:{337.317},{-43.495}) {}; % Gru,  3.96 
\node[pin={[pin distance=-0.6\onedegree,Bayer]00:{$\delta^2$}}] at (axis cs:{337.439},{-43.749}) {}; % Gru,  4.15 
\node[pin={[pin distance=-0.6\onedegree,Bayer]00:{$\epsilon$}}] at (axis cs:{342.139},{-51.317}) {}; % Gru,  3.49 
\node[pin={[pin distance=-0.6\onedegree,Bayer]00:{$\eta$}}] at (axis cs:{341.408},{-53.500}) {}; % Gru,  4.85 
\node[pin={[pin distance=-0.6\onedegree,Bayer]00:{$\gamma$}}] at (axis cs:{328.482},{-37.365}) {}; % Gru,  3.01 
\node[pin={[pin distance=-0.6\onedegree,Bayer]00:{$\iota$}}] at (axis cs:{347.590},{-45.246}) {}; % Gru,  3.88 
\node[pin={[pin distance=-0.6\onedegree,Bayer]00:{$\kappa$}}] at (axis cs:{346.165},{-53.965}) {}; % Gru,  5.37 
\node[pin={[pin distance=-0.6\onedegree,Bayer]00:{$\lambda$}}] at (axis cs:{331.529},{-39.543}) {}; % Gru,  4.46 
\node[pin={[pin distance=-0.6\onedegree,Bayer]00:{$\mu^1$}}] at (axis cs:{333.904},{-41.346}) {}; % Gru,  4.81 
\node[pin={[pin distance=-0.6\onedegree,Bayer]00:{$\mu^2$}}] at (axis cs:{334.111},{-41.627}) {}; % Gru,  5.11 
\node[pin={[pin distance=-0.6\onedegree,Bayer]00:{$\nu$}}] at (axis cs:{337.163},{-39.132}) {}; % Gru,  5.47 
\node[pin={[pin distance=-0.6\onedegree,Bayer]00:{$\omicron$}}] at (axis cs:{351.652},{-52.722}) {}; % Gru,  5.53 
\node[pin={[pin distance=-0.6\onedegree,Bayer]00:{$\pi^1$}}] at (axis cs:{335.684},{-45.948}) {}; % Gru,  6.13 
\node[pin={[pin distance=-0.6\onedegree,Bayer]00:{$\pi^2$}}] at (axis cs:{335.783},{-45.928}) {}; % Gru,  5.63 
\node[pin={[pin distance=-0.6\onedegree,Bayer]00:{$\rho$}}] at (axis cs:{340.875},{-41.414}) {}; % Gru,  4.84 
\node[pin={[pin distance=-0.6\onedegree,Bayer]00:{$\sigma^1$}}] at (axis cs:{339.122},{-40.582}) {}; % Gru,  6.28 
\node[pin={[pin distance=-0.6\onedegree,Bayer]00:{$\sigma^2$}}] at (axis cs:{339.245},{-40.591}) {}; % Gru,  5.88 
\node[pin={[pin distance=-0.6\onedegree,Bayer]00:{$\tau^1$}}] at (axis cs:{343.408},{-48.598}) {}; % Gru,  6.03 
\node[pin={[pin distance=-0.6\onedegree,Bayer]00:{$\tau^3$}}] at (axis cs:{344.199},{-47.969}) {}; % Gru,  5.71 
\node[pin={[pin distance=-0.6\onedegree,Bayer]00:{$\upsilon$}}] at (axis cs:{346.723},{-38.892}) {}; % Gru,  5.63 
\node[pin={[pin distance=-0.6\onedegree,Bayer]00:{$\varphi$}}] at (axis cs:{349.541},{-40.824}) {}; % Gru,  5.54 
\node[pin={[pin distance=-0.6\onedegree,Bayer]00:{$\vartheta$}}] at (axis cs:{346.720},{-43.520}) {}; % Gru,  4.33 
\node[pin={[pin distance=-0.6\onedegree,Bayer]00:{$\xi$}}] at (axis cs:{323.024},{-41.179}) {}; % Gru,  5.29 
\node[pin={[pin distance=-0.6\onedegree,Bayer]00:{$\zeta$}}] at (axis cs:{345.220},{-52.754}) {}; % Gru,  4.12

\node[pin={[pin distance=-0.6\onedegree,Bayer]00:{$\alpha$}}] at (axis cs:{63.500},{-42.294}) {}; % Hor,  3.86 
\node[pin={[pin distance=-0.6\onedegree,Bayer]-90:{$\beta$}}] at (axis cs:{44.699},{-64.071}) {}; % Hor,  4.98 
\node[pin={[pin distance=-0.6\onedegree,Bayer]00:{$\delta$}}] at (axis cs:{62.711},{-41.993}) {}; % Hor,  4.93 
\node[pin={[pin distance=-0.6\onedegree,Bayer]00:{$\eta$}}] at (axis cs:{39.352},{-52.543}) {}; % Hor,  5.30 
\node[pin={[pin distance=-0.8\onedegree,Bayer]-45:{$\gamma$}}] at (axis cs:{41.364},{-63.704}) {}; % Hor,  5.75 
\node[pin={[pin distance=-0.6\onedegree,Bayer]00:{$\iota$}}] at (axis cs:{40.639},{-50.800}) {}; % Hor,  5.40 
\node[pin={[pin distance=-0.6\onedegree,Bayer]00:{$\lambda$}}] at (axis cs:{36.225},{-60.312}) {}; % Hor,  5.37 
\node[pin={[pin distance=-0.6\onedegree,Bayer]00:{$\mu$}}] at (axis cs:{45.903},{-59.738}) {}; % Hor,  5.12 
\node[pin={[pin distance=-0.6\onedegree,Bayer]00:{$\nu$}}] at (axis cs:{42.256},{-62.806}) {}; % Hor,  5.26 
\node[pin={[pin distance=-0.6\onedegree,Bayer]00:{$\zeta$}}] at (axis cs:{40.165},{-54.550}) {}; % Hor,  5.22

\node[pin={[pin distance=-0.6\onedegree,Bayer]00:{$\beta$}}] at (axis cs:{178.227},{-33.908}) {}; % Hya,  4.29 
\node[pin={[pin distance=-0.6\onedegree,Bayer]00:{$\chi^1$}}] at (axis cs:{166.333},{-27.293}) {}; % Hya,  4.93 
\node[pin={[pin distance=-0.6\onedegree,Bayer]00:{$\chi^2$}}] at (axis cs:{166.490},{-27.288}) {}; % Hya,  5.71 
\node[pin={[pin distance=-0.6\onedegree,Bayer]00:{$\omicron$}}] at (axis cs:{175.053},{-34.744}) {}; % Hya,  4.70 
\node[pin={[pin distance=-0.6\onedegree,Bayer]00:{$\xi$}}] at (axis cs:{173.250},{-31.858}) {}; % Hya,  3.54 
\node[pin={[pin distance=-0.6\onedegree,Flaamsted]00:{$ 50$}}] at (axis cs:{213.192},{-27.261}) {}; % Hya,  5.08 
\node[pin={[pin distance=-0.6\onedegree,Flaamsted]00:{$ 51$}}] at (axis cs:{215.774},{-27.754}) {}; % Hya,  4.79 
\node[pin={[pin distance=-0.6\onedegree,Flaamsted]00:{$ 52$}}] at (axis cs:{217.043},{-29.492}) {}; % Hya,  4.98 
\node[pin={[pin distance=-0.6\onedegree,Flaamsted]00:{$ 58$}}] at (axis cs:{222.572},{-27.960}) {}; % Hya,  4.42 
\node[pin={[pin distance=-0.6\onedegree,Flaamsted]00:{$ 59$}}] at (axis cs:{224.664},{-27.657}) {}; % Hya,  5.67 
\node[pin={[pin distance=-0.6\onedegree,Flaamsted]00:{$ 60$}}] at (axis cs:{225.527},{-28.060}) {}; % Hya,  5.83

\node[pin={[pin distance=-0.6\onedegree,Bayer]00:{$\alpha$}}] at (axis cs:{29.692},{-61.570}) {}; % Hyi,  2.86 
\node[pin={[pin distance=-0.6\onedegree,Bayer]00:{$\beta$}}] at (axis cs:{6.438},{-77.254}) {}; % Hyi,  2.82 
\node[pin={[pin distance=-0.6\onedegree,Bayer]00:{$\delta$}}] at (axis cs:{35.437},{-68.659}) {}; % Hyi,  4.08 
\node[pin={[pin distance=-0.6\onedegree,Bayer]00:{$\epsilon$}}] at (axis cs:{39.897},{-68.267}) {}; % Hyi,  4.11 
\node[pin={[pin distance=-0.8\onedegree,Bayer]-30:{$\eta^2$}}] at (axis cs:{28.734},{-67.647}) {}; % Hyi,  4.69 
\node[pin={[pin distance=-0.6\onedegree,Bayer]180:{$\gamma$}}] at (axis cs:{56.810},{-74.239}) {}; % Hyi,  3.26 
\node[pin={[pin distance=-0.6\onedegree,Bayer]00:{$\iota$}}] at (axis cs:{48.990},{-77.388}) {}; % Hyi,  5.51 
\node[pin={[pin distance=-0.6\onedegree,Bayer]00:{$\kappa$}}] at (axis cs:{35.718},{-73.646}) {}; % Hyi,  5.99 
\node[pin={[pin distance=-0.6\onedegree,Bayer]00:{$\lambda$}}] at (axis cs:{12.148},{-74.923}) {}; % Hyi,  5.08 
\node[pin={[pin distance=-0.6\onedegree,Bayer]00:{$\mu$}}] at (axis cs:{37.919},{-79.109}) {}; % Hyi,  5.27 
\node[pin={[pin distance=-0.6\onedegree,Bayer]00:{$\nu$}}] at (axis cs:{42.619},{-75.067}) {}; % Hyi,  4.74 
\node[pin={[pin distance=-0.6\onedegree,Bayer]00:{$\pi^{1,2}$}}] at (axis cs:{33.561},{-67.841}) {}; % Hyi,  5.57 
%\node[pin={[pin distance=-0.6\onedegree,Bayer]00:{$\pi^2$}}] at (axis cs:{33.869},{-67.746}) {}; % Hyi,  5.67 
\node[pin={[pin distance=-0.6\onedegree,Bayer]90:{$\sigma$}}] at (axis cs:{28.961},{-78.348}) {}; % Hyi,  6.15 
\node[pin={[pin distance=-0.6\onedegree,Bayer]00:{$\tau^1$}}] at (axis cs:{25.339},{-79.148}) {}; % Hyi,  6.35 
\node[pin={[pin distance=-0.6\onedegree,Bayer]00:{$\tau^2$}}] at (axis cs:{26.944},{-80.176}) {}; % Hyi,  6.05 
\node[pin={[pin distance=-0.6\onedegree,Bayer]90:{$\vartheta$}}] at (axis cs:{45.564},{-71.902}) {}; % Hyi,  5.51 
\node[pin={[pin distance=-0.4\onedegree,Bayer]-90:{$\zeta$}}] at (axis cs:{41.386},{-67.616}) {}; % Hyi,  4.84

\node[pin={[pin distance=-0.6\onedegree,Bayer]00:{$\alpha$}}] at (axis cs:{309.392},{-47.291}) {}; % Ind,  3.11 
\node[pin={[pin distance=-0.4\onedegree,Bayer]00:{$\beta$}}] at (axis cs:{313.703},{-58.454}) {}; % Ind,  3.65 
\node[pin={[pin distance=-0.6\onedegree,Bayer]00:{$\delta$}}] at (axis cs:{329.479},{-54.992}) {}; % Ind,  4.41 
\node[pin={[pin distance=-0.6\onedegree,Bayer]00:{$\epsilon$}}] at (axis cs:{330.840},{-56.786}) {}; % Ind,  4.72 
\node[pin={[pin distance=-0.6\onedegree,Bayer]00:{$\eta$}}] at (axis cs:{311.010},{-51.921}) {}; % Ind,  4.52 
\node[pin={[pin distance=-0.6\onedegree,Bayer]00:{$\gamma$}}] at (axis cs:{321.564},{-54.660}) {}; % Ind,  6.09 
\node[pin={[pin distance=-0.6\onedegree,Bayer]00:{$\iota$}}] at (axis cs:{312.875},{-51.608}) {}; % Ind,  5.06 
\node[pin={[pin distance=-0.6\onedegree,Bayer]00:{$\kappa^1$}}] at (axis cs:{329.625},{-59.012}) {}; % Ind,  6.14 
\node[pin={[pin distance=-0.6\onedegree,Bayer]00:{$\kappa^2$}}] at (axis cs:{331.463},{-59.636}) {}; % Ind,  5.62 
\node[pin={[pin distance=-0.6\onedegree,Bayer]00:{$\mu$}}] at (axis cs:{316.309},{-54.727}) {}; % Ind,  5.17 
\node[pin={[pin distance=-0.6\onedegree,Bayer]-90:{$\nu$}}] at (axis cs:{336.154},{-72.255}) {}; % Ind,  5.29 
\node[pin={[pin distance=-0.6\onedegree,Bayer]00:{$\omicron$}}] at (axis cs:{327.697},{-69.629}) {}; % Ind,  5.51 
\node[pin={[pin distance=-0.6\onedegree,Bayer]00:{$\pi$}}] at (axis cs:{329.059},{-57.899}) {}; % Ind,  6.18 
\node[pin={[pin distance=-0.6\onedegree,Bayer]00:{$\rho$}}] at (axis cs:{343.665},{-70.074}) {}; % Ind,  6.06 
\node[pin={[pin distance=-0.6\onedegree,Bayer]00:{$\vartheta$}}] at (axis cs:{319.967},{-53.449}) {}; % Ind,  4.50 
\node[pin={[pin distance=-0.6\onedegree,Bayer]00:{$\zeta$}}] at (axis cs:{312.371},{-46.227}) {}; % Ind,  4.89

\node[pin={[pin distance=-0.4\onedegree,Bayer]00:{$\alpha$}}] at (axis cs:{220.482},{-47.388}) {}; % Lup,  2.29 
\node[pin={[pin distance=-0.4\onedegree,Bayer]180:{$\beta$}}] at (axis cs:{224.633},{-43.134}) {}; % Lup,  2.68 
\node[pin={[pin distance=-0.6\onedegree,Bayer]00:{$\chi$}}] at (axis cs:{237.740},{-33.627}) {}; % Lup,  3.96 
\node[pin={[pin distance=-0.4\onedegree,Bayer]-90:{$\delta$}}] at (axis cs:{230.343},{-40.647}) {}; % Lup,  3.22 
\node[pin={[pin distance=-0.6\onedegree,Bayer]00:{$\epsilon$}}] at (axis cs:{230.670},{-44.690}) {}; % Lup,  3.38 
\node[pin={[pin distance=-0.4\onedegree,Bayer]00:{$\eta$}}] at (axis cs:{240.031},{-38.396}) {}; % Lup,  3.43 
\node[pin={[pin distance=-0.4\onedegree,Bayer]00:{$\gamma$}}] at (axis cs:{233.785},{-41.167}) {}; % Lup,  2.78 
\node[pin={[pin distance=-0.4\onedegree,Bayer]00:{$\iota$}}] at (axis cs:{214.851},{-46.058}) {}; % Lup,  3.55 
\node[pin={[pin distance=-0.4\onedegree,Bayer]-90:{$\kappa^1$}}] at (axis cs:{227.984},{-48.738}) {}; % Lup,  3.86 
\node[pin={[pin distance=-0.6\onedegree,Bayer]00:{$\lambda$}}] at (axis cs:{227.211},{-45.280}) {}; % Lup,  4.06 
\node[pin={[pin distance=-0.4\onedegree,Bayer]-90:{$\mu$}}] at (axis cs:{229.633},{-47.875}) {}; % Lup,  4.28 
\node[pin={[pin distance=-0.6\onedegree,Bayer]00:{$\nu^1$}}] at (axis cs:{230.534},{-47.928}) {}; % Lup,  4.99 
\node[pin={[pin distance=-0.4\onedegree,Bayer]90:{$\nu^2$}}] at (axis cs:{230.451},{-48.318}) {}; % Lup,  5.66 
\node[pin={[pin distance=-0.6\onedegree,Bayer]00:{$\omega$}}] at (axis cs:{234.513},{-42.567}) {}; % Lup,  4.34 
\node[pin={[pin distance=-0.6\onedegree,Bayer]180:{$\omicron$}}] at (axis cs:{222.910},{-43.575}) {}; % Lup,  4.33 
\node[pin={[pin distance=-0.6\onedegree,Bayer]00:{$\pi$}}] at (axis cs:{226.280},{-47.051}) {}; % Lup,  3.91 
\node[pin={[pin distance=-0.6\onedegree,Bayer]00:{$\psi^1$}}] at (axis cs:{234.942},{-34.412}) {}; % Lup,  4.66 
\node[pin={[pin distance=-0.6\onedegree,Bayer]00:{$\psi^2$}}] at (axis cs:{235.671},{-34.710}) {}; % Lup,  4.73 
\node[pin={[pin distance=-0.4\onedegree,Bayer]-90:{$\rho$}}] at (axis cs:{219.472},{-49.426}) {}; % Lup,  4.05 
\node[pin={[pin distance=-0.4\onedegree,Bayer]90:{$\sigma$}}] at (axis cs:{218.154},{-50.457}) {}; % Lup,  4.43 
\node[pin={[pin distance=-0.6\onedegree,Bayer]00:{$\tau^{1,2}$}}] at (axis cs:{216.534},{-45.221}) {}; % Lup,  4.56 
%\node[pin={[pin distance=-0.6\onedegree,Bayer]00:{$\tau^2$}}] at (axis cs:{216.545},{-45.379}) {}; % Lup,  4.35 
\node[pin={[pin distance=-0.6\onedegree,Bayer]00:{$\upsilon$}}] at (axis cs:{231.188},{-39.710}) {}; % Lup,  5.38 
\node[pin={[pin distance=-0.6\onedegree,Bayer]00:{$\varphi^1$}}] at (axis cs:{230.452},{-36.261}) {}; % Lup,  3.56 
\node[pin={[pin distance=-0.6\onedegree,Bayer]00:{$\varphi^2$}}] at (axis cs:{230.789},{-36.858}) {}; % Lup,  4.53 
\node[pin={[pin distance=-0.4\onedegree,Bayer]-90:{$\vartheta$}}] at (axis cs:{241.648},{-36.802}) {}; % Lup,  4.22 
\node[pin={[pin distance=-0.6\onedegree,Bayer]00:{$\xi^1$}}] at (axis cs:{239.223},{-33.966}) {}; % Lup,  5.09 
\node[pin={[pin distance=-0.6\onedegree,Bayer]00:{$\zeta$}}] at (axis cs:{228.071},{-52.099}) {}; % Lup,  3.41 
\node[pin={[pin distance=-0.6\onedegree,Flaamsted]00:{$  1$}}] at (axis cs:{228.655},{-31.519}) {}; % Lup,  4.92 
\node[pin={[pin distance=-0.6\onedegree,Flaamsted]00:{$  2$}}] at (axis cs:{229.458},{-30.148}) {}; % Lup,  4.34

\node[pin={[pin distance=-0.6\onedegree,Bayer]00:{$\alpha$}}] at (axis cs:{92.560},{-74.753}) {}; % Men,  5.07 
\node[pin={[pin distance=-0.6\onedegree,Bayer]00:{$\beta$}}] at (axis cs:{75.679},{-71.314}) {}; % Men,  5.30 
\node[pin={[pin distance=-0.6\onedegree,Bayer]00:{$\delta$}}] at (axis cs:{64.497},{-80.214}) {}; % Men,  5.69 
\node[pin={[pin distance=-0.6\onedegree,Bayer]00:{$\epsilon$}}] at (axis cs:{111.409},{-79.094}) {}; % Men,  5.53 
\node[pin={[pin distance=-0.6\onedegree,Bayer]00:{$\eta$}}] at (axis cs:{73.797},{-74.937}) {}; % Men,  5.46 
\node[pin={[pin distance=-0.6\onedegree,Bayer]00:{$\gamma$}}] at (axis cs:{82.971},{-76.341}) {}; % Men,  5.18 
\node[pin={[pin distance=-0.6\onedegree,Bayer]00:{$\iota$}}] at (axis cs:{83.901},{-78.821}) {}; % Men,  6.05 
\node[pin={[pin distance=-0.6\onedegree,Bayer]00:{$\kappa$}}] at (axis cs:{87.570},{-79.361}) {}; % Men,  5.46 
\node[pin={[pin distance=-0.6\onedegree,Bayer]00:{$\mu$}}] at (axis cs:{70.767},{-70.931}) {}; % Men,  5.53 
\node[pin={[pin distance=-0.6\onedegree,Bayer]180:{$\nu$}}] at (axis cs:{65.242},{-81.580}) {}; % Men,  5.77 
\node[pin={[pin distance=-0.6\onedegree,Bayer]00:{$\pi$}}] at (axis cs:{84.291},{-80.469}) {}; % Men,  5.65 
\node[pin={[pin distance=-0.6\onedegree,Bayer]00:{$\vartheta$}}] at (axis cs:{104.144},{-79.420}) {}; % Men,  5.46 
\node[pin={[pin distance=-0.6\onedegree,Bayer]00:{$\xi$}}] at (axis cs:{74.712},{-82.470}) {}; % Men,  5.84 
\node[pin={[pin distance=-0.6\onedegree,Bayer]00:{$\zeta$}}] at (axis cs:{100.012},{-80.813}) {}; % Men,  5.61

\node[pin={[pin distance=-0.6\onedegree,Bayer]00:{$\alpha$}}] at (axis cs:{312.492},{-33.780}) {}; % Mic,  4.90 
\node[pin={[pin distance=-0.6\onedegree,Bayer]00:{$\beta$}}] at (axis cs:{312.995},{-33.178}) {}; % Mic,  6.06 
\node[pin={[pin distance=-0.6\onedegree,Bayer]00:{$\delta$}}] at (axis cs:{316.505},{-30.125}) {}; % Mic,  5.70 
\node[pin={[pin distance=-0.6\onedegree,Bayer]00:{$\epsilon$}}] at (axis cs:{319.485},{-32.172}) {}; % Mic,  4.72 
\node[pin={[pin distance=-0.6\onedegree,Bayer]00:{$\eta$}}] at (axis cs:{316.606},{-41.386}) {}; % Mic,  5.52 
\node[pin={[pin distance=-0.6\onedegree,Bayer]00:{$\gamma$}}] at (axis cs:{315.323},{-32.258}) {}; % Mic,  4.67 
\node[pin={[pin distance=-0.6\onedegree,Bayer]00:{$\iota$}}] at (axis cs:{312.121},{-43.988}) {}; % Mic,  5.11 
\node[pin={[pin distance=-0.6\onedegree,Bayer]00:{$\nu$}}] at (axis cs:{308.479},{-44.516}) {}; % Mic,  5.12 
\node[pin={[pin distance=-0.6\onedegree,Bayer]00:{$\vartheta^1$}}] at (axis cs:{320.190},{-40.809}) {}; % Mic,  4.81 
\node[pin={[pin distance=-0.6\onedegree,Bayer]00:{$\vartheta^2$}}] at (axis cs:{321.103},{-41.007}) {}; % Mic,  5.78 
\node[pin={[pin distance=-0.6\onedegree,Bayer]00:{$\zeta$}}] at (axis cs:{315.741},{-38.631}) {}; % Mic,  5.33

\node[pin={[pin distance=-0.4\onedegree,Bayer]180:{$\alpha$}}] at (axis cs:{189.296},{-69.136}) {}; % Mus,  2.69 
\node[pin={[pin distance=-0.4\onedegree,Bayer]00:{$\beta$}}] at (axis cs:{191.570},{-68.108}) {}; % Mus,  3.08 
\node[pin={[pin distance=-0.6\onedegree,Bayer]00:{$\delta$}}] at (axis cs:{195.568},{-71.549}) {}; % Mus,  3.61 
\node[pin={[pin distance=-0.5\onedegree,Bayer]180:{$\epsilon$}}] at (axis cs:{184.393},{-67.961}) {}; % Mus,  4.06 
\node[pin={[pin distance=-0.6\onedegree,Bayer]00:{$\eta$}}] at (axis cs:{198.812},{-67.894}) {}; % Mus,  4.78 
\node[pin={[pin distance=-0.6\onedegree,Bayer]00:{$\gamma$}}] at (axis cs:{188.117},{-72.133}) {}; % Mus,  3.84 
\node[pin={[pin distance=-0.6\onedegree,Bayer]00:{$\iota^1$}}] at (axis cs:{201.280},{-74.888}) {}; % Mus,  5.04 
\node[pin={[pin distance=-0.5\onedegree,Bayer]180:{$\lambda$}}] at (axis cs:{176.402},{-66.729}) {}; % Mus,  3.63 
\node[pin={[pin distance=-0.6\onedegree,Bayer]00:{$\mu$}}] at (axis cs:{177.061},{-66.815}) {}; % Mus,  4.74 
\node[pin={[pin distance=-0.6\onedegree,Bayer]00:{$\vartheta$}}] at (axis cs:{197.030},{-65.306}) {}; % Mus,  5.66 
\node[pin={[pin distance=-0.6\onedegree,Bayer]00:{$\zeta^1$}}] at (axis cs:{185.550},{-68.307}) {}; % Mus,  5.74 
\node[pin={[pin distance=-0.6\onedegree,Bayer]00:{$\zeta^2$}}] at (axis cs:{185.531},{-67.522}) {}; % Mus,  5.16

\node[pin={[pin distance=-0.6\onedegree,Bayer]00:{$\delta$}}] at (axis cs:{241.623},{-45.173}) {}; % Nor,  4.72 
\node[pin={[pin distance=-0.6\onedegree,Bayer]00:{$\epsilon$}}] at (axis cs:{246.796},{-47.555}) {}; % Nor,  4.53 
\node[pin={[pin distance=-0.6\onedegree,Bayer]00:{$\eta$}}] at (axis cs:{240.804},{-49.229}) {}; % Nor,  4.64 
\node[pin={[pin distance=-0.6\onedegree,Bayer]00:{$\gamma^1$}}] at (axis cs:{244.254},{-50.068}) {}; % Nor,  4.98 
\node[pin={[pin distance=-0.4\onedegree,Bayer]90:{$\gamma^2$}}] at (axis cs:{244.960},{-50.155}) {}; % Nor,  4.01 
\node[pin={[pin distance=-0.6\onedegree,Bayer]00:{$\iota^1$}}] at (axis cs:{240.884},{-57.775}) {}; % Nor,  4.66 
\node[pin={[pin distance=-0.6\onedegree,Bayer]180:{$\iota^2$}}] at (axis cs:{242.327},{-57.934}) {}; % Nor,  5.58 
\node[pin={[pin distance=-0.4\onedegree,Bayer]90:{$\kappa$}}] at (axis cs:{243.370},{-54.630}) {}; % Nor,  4.95 
\node[pin={[pin distance=-0.5\onedegree,Bayer]180:{$\lambda$}}] at (axis cs:{244.823},{-42.674}) {}; % Nor,  5.44 
\node[pin={[pin distance=-0.4\onedegree,Bayer]-90:{$\mu$}}] at (axis cs:{248.521},{-44.045}) {}; % Nor,  4.92 
\node[pin={[pin distance=-0.6\onedegree,Bayer]00:{$\vartheta$}}] at (axis cs:{243.814},{-47.372}) {}; % Nor,  5.13 
\node[pin={[pin distance=-0.4\onedegree,Bayer]90:{$\zeta$}}] at (axis cs:{243.345},{-55.541}) {}; % Nor,  5.79

\node[pin={[pin distance=-0.3\onedegree,Bayer]-90:{$\alpha$}}] at (axis cs:{316.179},{-77.024}) {}; % Oct,  5.13 
\node[pin={[pin distance=-0.6\onedegree,Bayer]180:{$\beta$}}] at (axis cs:{341.515},{-81.381}) {}; % Oct,  4.14 
\node[pin={[pin distance=-0.6\onedegree,Bayer]00:{$\chi$}}] at (axis cs:{283.696},{-87.606}) {}; % Oct,  5.28 
\node[pin={[pin distance=-0.6\onedegree,Bayer]00:{$\delta$}}] at (axis cs:{216.730},{-83.668}) {}; % Oct,  4.31 
\node[pin={[pin distance=-0.6\onedegree,Bayer]00:{$\epsilon$}}] at (axis cs:{335.007},{-80.440}) {}; % Oct,  5.18 
\node[pin={[pin distance=-0.6\onedegree,Bayer]00:{$\eta$}}] at (axis cs:{164.807},{-84.594}) {}; % Oct,  6.19 
%\node[pin={[pin distance=-0.6\onedegree,Bayer]00:{$\gamma^1$}}] at (axis cs:{358.027},{-82.019}) {}; % Oct,  5.11 
\node[pin={[pin distance=-0.6\onedegree,Bayer]00:{$\gamma^{1,2}$}}] at (axis cs:{359.387},{-82.170}) {}; % Oct,  5.73 
\node[pin={[pin distance=-0.4\onedegree,Bayer]-90:{$\gamma^3$}}] at (axis cs:{2.509},{-82.224}) {}; % Oct,  5.28 
\node[pin={[pin distance=-0.6\onedegree,Bayer]00:{$\iota$}}] at (axis cs:{193.745},{-85.123}) {}; % Oct,  5.46 
\node[pin={[pin distance=-0.6\onedegree,Bayer]180:{$\kappa$}}] at (axis cs:{205.231},{-85.786}) {}; % Oct,  5.56 
\node[pin={[pin distance=-0.6\onedegree,Bayer]00:{$\lambda$}}] at (axis cs:{327.727},{-82.719}) {}; % Oct,  5.29 
\node[pin={[pin distance=-0.6\onedegree,Bayer]00:{$\mu^1$}}] at (axis cs:{310.512},{-76.180}) {}; % Oct,  5.99 
\node[pin={[pin distance=-0.6\onedegree,Bayer]180:{$\mu^2$}}] at (axis cs:{310.434},{-75.351}) {}; % Oct,  6.44 
\node[pin={[pin distance=-0.4\onedegree,Bayer]180:{$\nu$}}] at (axis cs:{325.369},{-77.390}) {}; % Oct,  3.74 
\node[pin={[pin distance=-0.6\onedegree,Bayer]180:{$\omega$}}] at (axis cs:{227.787},{-84.788}) {}; % Oct,  5.88 
\node[pin={[pin distance=-0.6\onedegree,Bayer]90:{$\pi^{1,2}$}}] at (axis cs:{225.462},{-83.227}) {}; % Oct,  5.65 
%\node[pin={[pin distance=-0.6\onedegree,Bayer]00:{$\pi^2$}}] at (axis cs:{226.196},{-83.038}) {}; % Oct,  5.64 
\node[pin={[pin distance=-0.6\onedegree,Bayer]00:{$\psi$}}] at (axis cs:{334.461},{-77.511}) {}; % Oct,  5.48 
\node[pin={[pin distance=-0.6\onedegree,Bayer]00:{$\rho$}}] at (axis cs:{235.821},{-84.465}) {}; % Oct,  5.57 
\node[pin={[pin distance=-0.6\onedegree,Bayer]00:{$\sigma$}}] at (axis cs:{317.195},{-88.956}) {}; % Oct,  5.45 
\node[pin={[pin distance=-0.6\onedegree,Bayer]00:{$\tau$}}] at (axis cs:{352.016},{-87.482}) {}; % Oct,  5.50 
\node[pin={[pin distance=-0.6\onedegree,Bayer]00:{$\upsilon$}}] at (axis cs:{337.906},{-85.967}) {}; % Oct,  5.76 
\node[pin={[pin distance=-0.6\onedegree,Bayer]180:{$\varphi$}}] at (axis cs:{275.902},{-75.044}) {}; % Oct,  5.47 
\node[pin={[pin distance=-0.6\onedegree,Bayer]00:{$\vartheta$}}] at (axis cs:{0.399},{-77.065}) {}; % Oct,  4.79 
\node[pin={[pin distance=-0.6\onedegree,Bayer]90:{$\xi$}}] at (axis cs:{342.595},{-80.124}) {}; % Oct,  5.32 
\node[pin={[pin distance=-0.6\onedegree,Bayer]00:{$\zeta$}}] at (axis cs:{134.171},{-85.663}) {}; % Oct,  5.43

%\node[pin={[pin distance=-0.6\onedegree,Flaamsted]00:{$ 43$}}] at (axis cs:{260.840},{-28.143}) {}; % Oph,  5.30 
%\node[pin={[pin distance=-0.6\onedegree,Flaamsted]00:{$ 45$}}] at (axis cs:{261.839},{-29.867}) {}; % Oph,  4.28

\node[pin={[pin distance=-0.4\onedegree,Bayer]-135:{$\alpha$}}] at (axis cs:{306.412},{-56.735}) {}; % Pav,  1.92 
\node[pin={[pin distance=-0.6\onedegree,Bayer]00:{$\beta$}}] at (axis cs:{311.240},{-66.203}) {}; % Pav,  3.42 
\node[pin={[pin distance=-0.6\onedegree,Bayer]00:{$\delta$}}] at (axis cs:{302.182},{-66.182}) {}; % Pav,  3.55 
\node[pin={[pin distance=-0.6\onedegree,Bayer]00:{$\epsilon$}}] at (axis cs:{300.148},{-72.910}) {}; % Pav,  3.95 
\node[pin={[pin distance=-0.5\onedegree,Bayer]00:{$\eta$}}] at (axis cs:{266.433},{-64.724}) {}; % Pav,  3.59 
\node[pin={[pin distance=-0.6\onedegree,Bayer]00:{$\gamma$}}] at (axis cs:{321.611},{-65.366}) {}; % Pav,  4.23 
\node[pin={[pin distance=-0.6\onedegree,Bayer]00:{$\iota$}}] at (axis cs:{272.609},{-62.002}) {}; % Pav,  5.48 
\node[pin={[pin distance=-0.6\onedegree,Bayer]00:{$\kappa$}}] at (axis cs:{284.238},{-67.233}) {}; % Pav,  4.40 
\node[pin={[pin distance=-0.6\onedegree,Bayer]00:{$\lambda$}}] at (axis cs:{283.054},{-62.188}) {}; % Pav,  4.22 
\node[pin={[pin distance=-0.4\onedegree,Bayer]-90:{$\mu^{1,2}$}}] at (axis cs:{300.096},{-66.949}) {}; % Pav,  5.76 
%\node[pin={[pin distance=-0.6\onedegree,Bayer]00:{$\mu^2$}}] at (axis cs:{300.469},{-66.944}) {}; % Pav,  5.33 
\node[pin={[pin distance=-0.6\onedegree,Bayer]00:{$\nu$}}] at (axis cs:{277.843},{-62.278}) {}; % Pav,  4.62 
\node[pin={[pin distance=-0.6\onedegree,Bayer]00:{$\omega$}}] at (axis cs:{284.652},{-60.201}) {}; % Pav,  5.13 
\node[pin={[pin distance=-0.6\onedegree,Bayer]00:{$\omicron$}}] at (axis cs:{318.335},{-70.126}) {}; % Pav,  5.08 
\node[pin={[pin distance=-0.5\onedegree,Bayer]00:{$\pi$}}] at (axis cs:{272.145},{-63.669}) {}; % Pav,  4.33 
\node[pin={[pin distance=-0.6\onedegree,Bayer]-90:{$\rho$}}] at (axis cs:{309.397},{-61.530}) {}; % Pav,  4.86 
\node[pin={[pin distance=-0.6\onedegree,Bayer]00:{$\sigma$}}] at (axis cs:{312.326},{-68.776}) {}; % Pav,  5.41 
\node[pin={[pin distance=-0.6\onedegree,Bayer]00:{$\tau$}}] at (axis cs:{289.119},{-69.191}) {}; % Pav,  6.26 
\node[pin={[pin distance=-0.6\onedegree,Bayer]-90:{$\upsilon$}}] at (axis cs:{310.488},{-66.761}) {}; % Pav,  5.15 
\node[pin={[pin distance=-0.6\onedegree,Bayer]00:{$\varphi^1$}}] at (axis cs:{308.895},{-60.582}) {}; % Pav,  4.76 
\node[pin={[pin distance=-0.6\onedegree,Bayer]180:{$\varphi^2$}}] at (axis cs:{310.011},{-60.549}) {}; % Pav,  5.12 
%\node[pin={[pin distance=-0.4\onedegree,Bayer]-90:{$\vartheta$}}] at (axis cs:{282.158},{-65.077}) {}; % Pav,  5.71 
\node[pin={[pin distance=-0.6\onedegree,Bayer]00:{$\xi$}}] at (axis cs:{275.807},{-61.494}) {}; % Pav,  4.36 
\node[pin={[pin distance=-0.6\onedegree,Bayer]00:{$\zeta$}}] at (axis cs:{280.759},{-71.428}) {}; % Pav,  4.01

\node[pin={[pin distance=-0.4\onedegree,Bayer]00:{$\alpha$}}] at (axis cs:{6.571},{-42.306}) {}; % Phe,  2.40 
\node[pin={[pin distance=-0.6\onedegree,Bayer]00:{$\beta$}}] at (axis cs:{16.521},{-46.718}) {}; % Phe,  3.32 
\node[pin={[pin distance=-0.6\onedegree,Bayer]00:{$\chi$}}] at (axis cs:{30.427},{-44.713}) {}; % Phe,  5.14 
\node[pin={[pin distance=-0.6\onedegree,Bayer]00:{$\delta$}}] at (axis cs:{22.813},{-49.073}) {}; % Phe,  3.94 
\node[pin={[pin distance=-0.6\onedegree,Bayer]00:{$\epsilon$}}] at (axis cs:{2.353},{-45.747}) {}; % Phe,  3.88 
\node[pin={[pin distance=-0.6\onedegree,Bayer]00:{$\eta$}}] at (axis cs:{10.838},{-57.463}) {}; % Phe,  4.37 
\node[pin={[pin distance=-0.6\onedegree,Bayer]00:{$\gamma$}}] at (axis cs:{22.091},{-43.318}) {}; % Phe,  3.44 
\node[pin={[pin distance=-0.6\onedegree,Bayer]00:{$\iota$}}] at (axis cs:{353.769},{-42.615}) {}; % Phe,  4.71 
\node[pin={[pin distance=-0.6\onedegree,Bayer]00:{$\kappa$}}] at (axis cs:{6.551},{-43.680}) {}; % Phe,  3.95 
\node[pin={[pin distance=-0.6\onedegree,Bayer]00:{$\lambda^1$}}] at (axis cs:{7.854},{-48.803}) {}; % Phe,  4.76 
\node[pin={[pin distance=-0.6\onedegree,Bayer]00:{$\lambda^2$}}] at (axis cs:{8.922},{-48.001}) {}; % Phe,  5.52 
\node[pin={[pin distance=-0.6\onedegree,Bayer]00:{$\mu$}}] at (axis cs:{10.331},{-46.085}) {}; % Phe,  4.60 
\node[pin={[pin distance=-0.6\onedegree,Bayer]00:{$\nu$}}] at (axis cs:{18.796},{-45.531}) {}; % Phe,  4.97 
\node[pin={[pin distance=-0.6\onedegree,Bayer]00:{$\omega$}}] at (axis cs:{15.508},{-57.002}) {}; % Phe,  6.10 
\node[pin={[pin distance=-0.6\onedegree,Bayer]00:{$\pi$}}] at (axis cs:{359.732},{-52.746}) {}; % Phe,  5.13 
\node[pin={[pin distance=-0.6\onedegree,Bayer]00:{$\psi$}}] at (axis cs:{28.411},{-46.303}) {}; % Phe,  4.43 
\node[pin={[pin distance=-0.6\onedegree,Bayer]00:{$\rho$}}] at (axis cs:{12.672},{-50.987}) {}; % Phe,  5.23 
\node[pin={[pin distance=-0.6\onedegree,Bayer]00:{$\sigma$}}] at (axis cs:{356.817},{-50.226}) {}; % Phe,  5.17 
\node[pin={[pin distance=-0.6\onedegree,Bayer]00:{$\tau$}}] at (axis cs:{0.269},{-48.810}) {}; % Phe,  5.71 
\node[pin={[pin distance=-0.6\onedegree,Bayer]00:{$\upsilon$}}] at (axis cs:{16.949},{-41.487}) {}; % Phe,  5.21 
\node[pin={[pin distance=-0.6\onedegree,Bayer]00:{$\varphi$}}] at (axis cs:{28.592},{-42.497}) {}; % Phe,  5.12 
\node[pin={[pin distance=-0.6\onedegree,Bayer]00:{$\vartheta$}}] at (axis cs:{354.866},{-46.638}) {}; % Phe,  6.42 
\node[pin={[pin distance=-0.6\onedegree,Bayer]00:{$\xi$}}] at (axis cs:{10.443},{-56.501}) {}; % Phe,  5.71 
\node[pin={[pin distance=-0.6\onedegree,Bayer]00:{$\zeta$}}] at (axis cs:{17.096},{-55.246}) {}; % Phe,  3.98

\node[pin={[pin distance=-0.5\onedegree,Bayer]00:{$\alpha$}}] at (axis cs:{102.048},{-61.941}) {}; % Pic,  3.25 
\node[pin={[pin distance=-0.5\onedegree,Bayer]00:{$\beta$}}] at (axis cs:{86.821},{-51.066}) {}; % Pic,  3.86 
\node[pin={[pin distance=-0.6\onedegree,Bayer]00:{$\delta$}}] at (axis cs:{92.575},{-54.969}) {}; % Pic,  4.72 
\node[pin={[pin distance=-0.6\onedegree,Bayer]180:{$\eta^1$}}] at (axis cs:{75.703},{-49.151}) {}; % Pic,  5.38 
\node[pin={[pin distance=-0.6\onedegree,Bayer]00:{$\eta^2$}}] at (axis cs:{76.242},{-49.578}) {}; % Pic,  5.04 
\node[pin={[pin distance=-0.6\onedegree,Bayer]00:{$\gamma$}}] at (axis cs:{87.457},{-56.166}) {}; % Pic,  4.50 
\node[pin={[pin distance=-0.6\onedegree,Bayer]180:{$\iota$}}] at (axis cs:{72.730},{-53.461}) {}; % Pic,  5.62 
\node[pin={[pin distance=-0.6\onedegree,Bayer]00:{$\kappa$}}] at (axis cs:{80.592},{-56.134}) {}; % Pic,  6.10 
\node[pin={[pin distance=-0.6\onedegree,Bayer]00:{$\lambda$}}] at (axis cs:{70.693},{-50.481}) {}; % Pic,  5.30 
\node[pin={[pin distance=-0.6\onedegree,Bayer]00:{$\mu$}}] at (axis cs:{97.993},{-58.754}) {}; % Pic,  5.72 
\node[pin={[pin distance=-0.6\onedegree,Bayer]00:{$\nu$}}] at (axis cs:{95.733},{-56.370}) {}; % Pic,  5.61 
\node[pin={[pin distance=-0.6\onedegree,Bayer]00:{$\vartheta$}}] at (axis cs:{81.193},{-52.316}) {}; % Pic,  6.24 
\node[pin={[pin distance=-0.6\onedegree,Bayer]90:{$\zeta$}}] at (axis cs:{79.842},{-50.606}) {}; % Pic,  5.44

\node[pin={[pin distance=-0.4\onedegree,Bayer]00:{$\alpha$}}] at (axis cs:{344.413},{-29.622}) {}; % PsA,  1.23 
\node[pin={[pin distance=-0.6\onedegree,Bayer]00:{$\beta$}}] at (axis cs:{337.876},{-32.346}) {}; % PsA,  4.29 
\node[pin={[pin distance=-0.6\onedegree,Bayer]00:{$\delta$}}] at (axis cs:{343.987},{-32.540}) {}; % PsA,  4.22 
\node[pin={[pin distance=-0.6\onedegree,Bayer]00:{$\epsilon$}}] at (axis cs:{340.164},{-27.043}) {}; % PsA,  4.19 
\node[pin={[pin distance=-0.6\onedegree,Bayer]00:{$\eta$}}] at (axis cs:{330.209},{-28.454}) {}; % PsA,  5.43 
\node[pin={[pin distance=-0.6\onedegree,Bayer]00:{$\gamma$}}] at (axis cs:{343.131},{-32.875}) {}; % PsA,  4.51 
\node[pin={[pin distance=-0.6\onedegree,Bayer]00:{$\iota$}}] at (axis cs:{326.237},{-33.026}) {}; % PsA,  4.34 
\node[pin={[pin distance=-0.6\onedegree,Bayer]00:{$\lambda$}}] at (axis cs:{333.578},{-27.767}) {}; % PsA,  5.44 
\node[pin={[pin distance=-0.6\onedegree,Bayer]00:{$\mu$}}] at (axis cs:{332.096},{-32.988}) {}; % PsA,  4.50 
\node[pin={[pin distance=-0.6\onedegree,Bayer]00:{$\pi$}}] at (axis cs:{345.874},{-34.749}) {}; % PsA,  5.13 
\node[pin={[pin distance=-0.6\onedegree,Bayer]00:{$\tau$}}] at (axis cs:{332.537},{-32.548}) {}; % PsA,  4.94 
\node[pin={[pin distance=-0.6\onedegree,Bayer]00:{$\upsilon$}}] at (axis cs:{332.108},{-34.044}) {}; % PsA,  4.97 
\node[pin={[pin distance=-0.6\onedegree,Bayer]00:{$\vartheta$}}] at (axis cs:{326.934},{-30.898}) {}; % PsA,  5.02 
\node[pin={[pin distance=-0.6\onedegree,Flaamsted]00:{$ 13$}}] at (axis cs:{331.099},{-29.916}) {}; % PsA,  6.45 
\node[pin={[pin distance=-0.6\onedegree,Flaamsted]00:{$ 19$}}] at (axis cs:{340.592},{-29.361}) {}; % PsA,  6.17 
\node[pin={[pin distance=-0.6\onedegree,Flaamsted]00:{$ 21$}}] at (axis cs:{342.837},{-29.536}) {}; % PsA,  5.99 
\node[pin={[pin distance=-0.6\onedegree,Flaamsted]00:{$  6$}}] at (axis cs:{323.061},{-33.944}) {}; % PsA,  5.97 
\node[pin={[pin distance=-0.6\onedegree,Flaamsted]00:{$  7$}}] at (axis cs:{324.203},{-33.048}) {}; % PsA,  6.11

\node[pin={[pin distance=-0.4\onedegree,Bayer]00:{$\zeta$}}] at (axis cs:{120.896},{-40.003}) {}; % Pup,  2.22 
\node[pin={[pin distance=-0.5\onedegree,Bayer]00:{$\nu$}}] at (axis cs:{99.440},{-43.196}) {}; % Pup,  3.18 
\node[pin={[pin distance=-0.4\onedegree,Bayer]00:{$\pi$}}] at (axis cs:{109.286},{-37.097}) {}; % Pup,  2.71 
\node[pin={[pin distance=-0.5\onedegree,Bayer]00:{$\sigma$}}] at (axis cs:{112.308},{-43.301}) {}; % Pup,  3.26 
\node[pin={[pin distance=-0.2\onedegree,Bayer]180:{$\tau$}}] at (axis cs:{102.484},{-50.614}) {}; % Pup,  2.94 


\node[pin={[pin distance=-0.5\onedegree,Bayer]00:{$\alpha$}}] at (axis cs:{130.898},{-33.186}) {}; % Pyx,  3.69 
\node[pin={[pin distance=-0.5\onedegree,Bayer]00:{$\beta$}}] at (axis cs:{130.026},{-35.308}) {}; % Pyx,  3.97 
\node[pin={[pin distance=-0.6\onedegree,Bayer]00:{$\epsilon$}}] at (axis cs:{137.485},{-30.365}) {}; % Pyx,  5.60 


\node[pin={[pin distance=-0.4\onedegree,Bayer]00:{$\alpha$}}] at (axis cs:{63.606},{-62.474}) {}; % Ret,  3.34 
\node[pin={[pin distance=-0.5\onedegree,Bayer]00:{$\beta$}}] at (axis cs:{56.050},{-64.807}) {}; % Ret,  3.84 
\node[pin={[pin distance=-0.6\onedegree,Bayer]00:{$\delta$}}] at (axis cs:{59.686},{-61.400}) {}; % Ret,  4.56 
\node[pin={[pin distance=-0.6\onedegree,Bayer]00:{$\epsilon$}}] at (axis cs:{64.121},{-59.302}) {}; % Ret,  4.44 
\node[pin={[pin distance=-0.6\onedegree,Bayer]-90:{$\eta$}}] at (axis cs:{65.472},{-63.386}) {}; % Ret,  5.24 
\node[pin={[pin distance=-0.6\onedegree,Bayer]00:{$\gamma$}}] at (axis cs:{60.224},{-62.159}) {}; % Ret,  4.51 
\node[pin={[pin distance=-0.6\onedegree,Bayer]180:{$\iota$}}] at (axis cs:{60.326},{-61.079}) {}; % Ret,  4.96 
\node[pin={[pin distance=-0.6\onedegree,Bayer]00:{$\kappa$}}] at (axis cs:{52.344},{-62.937}) {}; % Ret,  4.71 
\node[pin={[pin distance=-0.6\onedegree,Bayer]00:{$\vartheta$}}] at (axis cs:{64.418},{-63.255}) {}; % Ret,  6.05 
\node[pin={[pin distance=-0.6\onedegree,Bayer]180:{$\zeta^2$}}] at (axis cs:{49.553},{-62.506}) {}; % Ret,  5.24

\node[pin={[pin distance=-0.6\onedegree,Bayer]00:{$\alpha$}}] at (axis cs:{14.652},{-29.357}) {}; % Scl,  4.31 
\node[pin={[pin distance=-0.6\onedegree,Bayer]00:{$\beta$}}] at (axis cs:{353.243},{-37.818}) {}; % Scl,  4.38 
\node[pin={[pin distance=-0.6\onedegree,Bayer]00:{$\delta$}}] at (axis cs:{357.231},{-28.130}) {}; % Scl,  4.58 
\node[pin={[pin distance=-0.6\onedegree,Bayer]00:{$\eta$}}] at (axis cs:{6.982},{-33.007}) {}; % Scl,  4.87 
\node[pin={[pin distance=-0.6\onedegree,Bayer]00:{$\gamma$}}] at (axis cs:{349.706},{-32.532}) {}; % Scl,  4.41 
\node[pin={[pin distance=-0.6\onedegree,Bayer]00:{$\iota$}}] at (axis cs:{5.380},{-28.981}) {}; % Scl,  5.17 
\node[pin={[pin distance=-0.6\onedegree,Bayer]00:{$\kappa^1$}}] at (axis cs:{2.338},{-27.988}) {}; % Scl,  5.44 
\node[pin={[pin distance=-0.6\onedegree,Bayer]00:{$\kappa^2$}}] at (axis cs:{2.893},{-27.799}) {}; % Scl,  5.41 
\node[pin={[pin distance=-0.6\onedegree,Bayer]00:{$\lambda^1$}}] at (axis cs:{10.679},{-38.463}) {}; % Scl,  6.08 
\node[pin={[pin distance=-0.6\onedegree,Bayer]00:{$\lambda^2$}}] at (axis cs:{11.050},{-38.422}) {}; % Scl,  5.90 
\node[pin={[pin distance=-0.6\onedegree,Bayer]00:{$\mu$}}] at (axis cs:{355.159},{-32.073}) {}; % Scl,  5.31 
\node[pin={[pin distance=-0.6\onedegree,Bayer]00:{$\pi$}}] at (axis cs:{25.536},{-32.327}) {}; % Scl,  5.26 
\node[pin={[pin distance=-0.6\onedegree,Bayer]00:{$\sigma$}}] at (axis cs:{15.610},{-31.552}) {}; % Scl,  5.51 
\node[pin={[pin distance=-0.6\onedegree,Bayer]00:{$\tau$}}] at (axis cs:{24.035},{-29.907}) {}; % Scl,  5.70 
\node[pin={[pin distance=-0.6\onedegree,Bayer]00:{$\vartheta$}}] at (axis cs:{2.933},{-35.133}) {}; % Scl,  5.24 
\node[pin={[pin distance=-0.6\onedegree,Bayer]00:{$\xi$}}] at (axis cs:{15.326},{-38.916}) {}; % Scl,  5.59 
\node[pin={[pin distance=-0.6\onedegree,Bayer]00:{$\zeta$}}] at (axis cs:{0.583},{-29.720}) {}; % Scl,  5.04

\node[pin={[pin distance=-0.4\onedegree,Bayer]00:{$\epsilon$}}] at (axis cs:{252.541},{-34.293}) {}; % Sco,  2.29 
\node[pin={[pin distance=-0.4\onedegree,Bayer]00:{$\kappa$}}] at (axis cs:{265.622},{-39.030}) {}; % Sco,  2.39 
\node[pin={[pin distance=-0.2\onedegree,Bayer]180:{$\lambda$}}] at (axis cs:{263.402},{-37.104}) {}; % Sco,  1.63 
\node[pin={[pin distance=-0.3\onedegree,Bayer]00:{$\vartheta$}}] at (axis cs:{264.330},{-42.998}) {}; % Sco,  1.86 
\node[pin={[pin distance=-0.4\onedegree,Bayer]00:{$\eta$}}] at (axis cs:{258.038},{-43.239}) {}; % Sco,  3.32 
\node[pin={[pin distance=-0.4\onedegree,Bayer]180:{$\iota^1$}}] at (axis cs:{266.896},{-40.127}) {}; % Sco,  3.01 
\node[pin={[pin distance=-0.6\onedegree,Bayer]00:{$\iota^2$}}] at (axis cs:{267.546},{-40.090}) {}; % Sco,  4.81 
%\node[pin={[pin distance=-0.6\onedegree,Bayer]00:{$\mu^1$}}] at (axis cs:{252.968},{-38.047}) {}; % Sco,  3.00 
\node[pin={[pin distance=-0.4\onedegree,Bayer]00:{$\mu^{1,2}$}}] at (axis cs:{253.084},{-38.018}) {}; % Sco,  3.56 
\node[pin={[pin distance=-0.6\onedegree,Bayer]00:{$\rho$}}] at (axis cs:{239.221},{-29.214}) {}; % Sco,  3.88 
\node[pin={[pin distance=-0.6\onedegree,Bayer]00:{$\tau$}}] at (axis cs:{248.971},{-28.216}) {}; % Sco,  2.83 
\node[pin={[pin distance=-0.2\onedegree,Bayer]-90:{$\upsilon$}}] at (axis cs:{262.691},{-37.296}) {}; % Sco,  2.68 
\node[pin={[pin distance=-0.4\onedegree,Bayer]30:{$\zeta^{1,2}$}}] at (axis cs:{253.499},{-42.362}) {}; % Sco,  4.77 
%\node[pin={[pin distance=-0.6\onedegree,Bayer]00:{$\zeta^2$}}] at (axis cs:{253.646},{-42.361}) {}; % Sco,  3.61 
%\node[pin={[pin distance=-0.6\onedegree,Flaamsted]00:{$ 12$}}] at (axis cs:{243.067},{-28.417}) {}; % Sco,  5.67 
%\node[pin={[pin distance=-0.6\onedegree,Flaamsted]00:{$ 13$}}] at (axis cs:{243.076},{-27.926}) {}; % Sco,  4.58 
\node[pin={[pin distance=-0.6\onedegree,Flaamsted]90:{$ 27$}}] at (axis cs:{254.297},{-33.259}) {}; % Sco,  5.48

\node[pin={[pin distance=-0.2\onedegree,Bayer]-90:{$\epsilon$}}] at (axis cs:{276.043},{-34.385}) {}; % Sgr,  1.81 
\node[pin={[pin distance=-0.4\onedegree,Bayer]00:{$\alpha$}}] at (axis cs:{290.972},{-40.616}) {}; % Sgr,  3.96 
\node[pin={[pin distance=-0.4\onedegree,Bayer]00:{$\beta^{1,2}$}}] at (axis cs:{290.660},{-44.459}) {}; % Sgr,  4.01 
%\node[pin={[pin distance=-0.6\onedegree,Bayer]00:{$\beta^2$}}] at (axis cs:{290.805},{-44.800}) {}; % Sgr,  4.28 
\node[pin={[pin distance=-0.6\onedegree,Bayer]00:{$\delta$}}] at (axis cs:{275.249},{-29.828}) {}; % Sgr,  2.70 
\node[pin={[pin distance=-0.4\onedegree,Bayer]00:{$\eta$}}] at (axis cs:{274.407},{-36.761}) {}; % Sgr,  3.13 
%\node[pin={[pin distance=-0.6\onedegree,Bayer]00:{$\gamma^1$}}] at (axis cs:{271.255},{-29.580}) {}; % Sgr,  4.65 
\node[pin={[pin distance=-0.6\onedegree,Bayer]180:{$\gamma^2$}}] at (axis cs:{271.452},{-30.424}) {}; % Sgr,  3.63 
\node[pin={[pin distance=-0.6\onedegree,Bayer]00:{$\iota$}}] at (axis cs:{298.815},{-41.868}) {}; % Sgr,  4.12 
\node[pin={[pin distance=-0.6\onedegree,Bayer]00:{$\kappa^1$}}] at (axis cs:{305.615},{-42.049}) {}; % Sgr,  5.58 
\node[pin={[pin distance=-0.6\onedegree,Bayer]00:{$\kappa^2$}}] at (axis cs:{305.972},{-42.423}) {}; % Sgr,  5.64 
\node[pin={[pin distance=-0.6\onedegree,Bayer]00:{$\tau$}}] at (axis cs:{286.735},{-27.670}) {}; % Sgr,  3.32 
\node[pin={[pin distance=-0.6\onedegree,Bayer]00:{$\vartheta^1$}}] at (axis cs:{299.934},{-35.276}) {}; % Sgr,  4.37 
\node[pin={[pin distance=-0.6\onedegree,Bayer]00:{$\vartheta^2$}}] at (axis cs:{299.964},{-34.698}) {}; % Sgr,  5.31 
\node[pin={[pin distance=-0.4\onedegree,Bayer]00:{$\zeta$}}] at (axis cs:{285.653},{-29.880}) {}; % Sgr,  2.61 
\node[pin={[pin distance=-0.6\onedegree,Flaamsted]180:{$ 18$}}] at (axis cs:{276.256},{-30.756}) {}; % Sgr,  5.58 
\node[pin={[pin distance=-0.6\onedegree,Flaamsted]00:{$  3$}}] at (axis cs:{266.890},{-27.831}) {}; % Sgr,  4.56 
\node[pin={[pin distance=-0.6\onedegree,Flaamsted]00:{$ 59$}}] at (axis cs:{299.237},{-27.170}) {}; % Sgr,  4.53 
\node[pin={[pin distance=-0.6\onedegree,Flaamsted]00:{$ 62$}}] at (axis cs:{300.665},{-27.710}) {}; % Sgr,  4.49

\node[pin={[pin distance=-0.4\onedegree,Bayer]180:{$\alpha$}}] at (axis cs:{276.743},{-45.968}) {}; % Tel,  3.49 
\node[pin={[pin distance=-0.4\onedegree,Bayer]180:{$\delta^{1,2}$}}] at (axis cs:{277.939},{-45.915}) {}; % Tel,  4.93 
%\node[pin={[pin distance=-0.6\onedegree,Bayer]00:{$\delta^2$}}] at (axis cs:{278.008},{-45.757}) {}; % Tel,  5.08 
\node[pin={[pin distance=-0.4\onedegree,Bayer]90:{$\epsilon$}}] at (axis cs:{272.807},{-45.954}) {}; % Tel,  4.52 
\node[pin={[pin distance=-0.6\onedegree,Bayer]00:{$\eta$}}] at (axis cs:{290.713},{-54.424}) {}; % Tel,  5.03 
\node[pin={[pin distance=-0.6\onedegree,Bayer]00:{$\iota$}}] at (axis cs:{293.804},{-48.099}) {}; % Tel,  4.89 
\node[pin={[pin distance=-0.6\onedegree,Bayer]135:{$\kappa$}}] at (axis cs:{283.165},{-52.107}) {}; % Tel,  5.18 
\node[pin={[pin distance=-0.6\onedegree,Bayer]180:{$\lambda$}}] at (axis cs:{284.616},{-52.939}) {}; % Tel,  4.85 
\node[pin={[pin distance=-0.4\onedegree,Bayer]00:{$\mu$}}] at (axis cs:{292.644},{-55.110}) {}; % Tel,  6.30 
\node[pin={[pin distance=-0.6\onedegree,Bayer]00:{$\nu$}}] at (axis cs:{297.005},{-56.362}) {}; % Tel,  5.34 
\node[pin={[pin distance=-0.6\onedegree,Bayer]180:{$\rho$}}] at (axis cs:{286.583},{-52.341}) {}; % Tel,  5.18 
\node[pin={[pin distance=-0.6\onedegree,Bayer]00:{$\xi$}}] at (axis cs:{301.846},{-52.881}) {}; % Tel,  4.92 
\node[pin={[pin distance=-0.4\onedegree,Bayer]180:{$\zeta$}}] at (axis cs:{277.208},{-49.071}) {}; % Tel,  4.11

\node[pin={[pin distance=-0.4\onedegree,Bayer]00:{$\alpha$}}] at (axis cs:{252.166},{-69.028}) {}; % TrA,  1.91 
\node[pin={[pin distance=-0.4\onedegree,Bayer]00:{$\beta$}}] at (axis cs:{238.786},{-63.430}) {}; % TrA,  2.84 
\node[pin={[pin distance=-0.5\onedegree,Bayer]00:{$\delta$}}] at (axis cs:{243.859},{-63.686}) {}; % TrA,  3.85 
\node[pin={[pin distance=-0.6\onedegree,Bayer]00:{$\epsilon$}}] at (axis cs:{234.180},{-66.317}) {}; % TrA,  4.11 
\node[pin={[pin distance=-0.4\onedegree,Bayer]-90:{$\eta^1$}}] at (axis cs:{250.346},{-68.296}) {}; % TrA,  5.89 
\node[pin={[pin distance=-0.4\onedegree,Bayer]90:{$\gamma$}}] at (axis cs:{229.727},{-68.679}) {}; % TrA,  2.88 
\node[pin={[pin distance=-0.6\onedegree,Bayer]00:{$\iota$}}] at (axis cs:{246.989},{-64.058}) {}; % TrA,  5.29 
\node[pin={[pin distance=-0.6\onedegree,Bayer]00:{$\kappa$}}] at (axis cs:{238.873},{-68.603}) {}; % TrA,  5.10 
\node[pin={[pin distance=-0.6\onedegree,Bayer]00:{$\vartheta$}}] at (axis cs:{248.937},{-65.495}) {}; % TrA,  5.50 
\node[pin={[pin distance=-0.6\onedegree,Bayer]00:{$\zeta$}}] at (axis cs:{247.117},{-70.084}) {}; % TrA,  4.91

\node[pin={[pin distance=-0.6\onedegree,Bayer]00:{$\alpha$}}] at (axis cs:{334.625},{-60.259}) {}; % Tuc,  2.86 
\node[pin={[pin distance=-0.6\onedegree,Bayer]00:{$\beta^1$}}] at (axis cs:{7.886},{-62.958}) {}; % Tuc,  4.34 
\node[pin={[pin distance=-0.6\onedegree,Bayer]00:{$\beta^3$}}] at (axis cs:{8.183},{-63.031}) {}; % Tuc,  5.07 
\node[pin={[pin distance=-0.6\onedegree,Bayer]00:{$\delta$}}] at (axis cs:{336.833},{-64.966}) {}; % Tuc,  4.50 
\node[pin={[pin distance=-0.6\onedegree,Bayer]00:{$\epsilon$}}] at (axis cs:{359.979},{-65.577}) {}; % Tuc,  4.50 
\node[pin={[pin distance=-0.6\onedegree,Bayer]00:{$\eta$}}] at (axis cs:{359.396},{-64.298}) {}; % Tuc,  5.00 
\node[pin={[pin distance=-0.6\onedegree,Bayer]00:{$\gamma$}}] at (axis cs:{349.357},{-58.236}) {}; % Tuc,  4.00 
\node[pin={[pin distance=-0.6\onedegree,Bayer]00:{$\iota$}}] at (axis cs:{16.828},{-61.775}) {}; % Tuc,  5.35 
\node[pin={[pin distance=-0.4\onedegree,Bayer]-90:{$\kappa$}}] at (axis cs:{18.942},{-68.876}) {}; % Tuc,  4.25 
\node[pin={[pin distance=-0.6\onedegree,Bayer]-90:{$\lambda^2$}}] at (axis cs:{13.751},{-69.527}) {}; % Tuc,  5.46 
\node[pin={[pin distance=-0.6\onedegree,Bayer]00:{$\nu$}}] at (axis cs:{338.250},{-61.982}) {}; % Tuc,  4.95 
\node[pin={[pin distance=-0.6\onedegree,Bayer]00:{$\pi$}}] at (axis cs:{5.163},{-69.625}) {}; % Tuc,  5.50 
\node[pin={[pin distance=-0.6\onedegree,Bayer]00:{$\rho$}}] at (axis cs:{10.618},{-65.468}) {}; % Tuc,  5.39 
\node[pin={[pin distance=-0.6\onedegree,Bayer]90:{$\vartheta$}}] at (axis cs:{8.347},{-71.266}) {}; % Tuc,  6.12 
\node[pin={[pin distance=-0.6\onedegree,Bayer]00:{$\zeta$}}] at (axis cs:{5.018},{-64.875}) {}; % Tuc,  4.23

\node[pin={[pin distance=-0.2\onedegree,Bayer]-90:{$\delta$}}] at (axis cs:{131.176},{-54.709}) {}; % Vel,  1.94 
\node[pin={[pin distance=-0.2\onedegree,Bayer]180:{$\gamma^2$}}] at (axis cs:{122.383},{-47.337}) {}; % Vel,  1.79 
\node[pin={[pin distance=-0.3\onedegree,Bayer]00:{$\kappa$}}] at (axis cs:{140.528},{-55.010}) {}; % Vel,  2.48 
\node[pin={[pin distance=-0.3\onedegree,Bayer]00:{$\lambda$}}] at (axis cs:{136.999},{-43.432}) {}; % Vel,  2.21 
\node[pin={[pin distance=-0.4\onedegree,Bayer]00:{$\mu$}}] at (axis cs:{161.692},{-49.420}) {}; % Vel,  2.72 
\node[pin={[pin distance=-0.2\onedegree,Bayer]90:{$\psi$}}] at (axis cs:{142.675},{-40.467}) {}; % Vel,  3.59 
\node[pin={[pin distance=-0.3\onedegree,Bayer]-90:{$\varphi$}}] at (axis cs:{149.216},{-54.568}) {}; % Vel,  3.54

\node[pin={[pin distance=-0.46\onedegree,Bayer]180:{$\alpha$}}] at (axis cs:{135.612},{-66.396}) {}; % Vol,  4.00 
\node[pin={[pin distance=-0.4\onedegree,Bayer]00:{$\beta$}}] at (axis cs:{126.434},{-66.137}) {}; % Vol,  3.77 
\node[pin={[pin distance=-0.6\onedegree,Bayer]00:{$\delta$}}] at (axis cs:{109.208},{-67.957}) {}; % Vol,  3.97 
\node[pin={[pin distance=-0.6\onedegree,Bayer]00:{$\epsilon$}}] at (axis cs:{121.982},{-68.617}) {}; % Vol,  4.40 
\node[pin={[pin distance=-0.6\onedegree,Bayer]00:{$\eta$}}] at (axis cs:{125.519},{-73.400}) {}; % Vol,  5.29 
\node[pin={[pin distance=-0.6\onedegree,Bayer]00:{$\gamma^2$}}] at (axis cs:{107.187},{-70.499}) {}; % Vol,  3.76 
\node[pin={[pin distance=-0.6\onedegree,Bayer]00:{$\iota$}}] at (axis cs:{102.862},{-70.963}) {}; % Vol,  5.40 
\node[pin={[pin distance=-0.4\onedegree,Bayer]-90:{$\kappa^1$}}] at (axis cs:{124.954},{-71.515}) {}; % Vol,  5.34 
\node[pin={[pin distance=-0.6\onedegree,Bayer]00:{$\vartheta$}}] at (axis cs:{129.772},{-70.387}) {}; % Vol,  5.19 
\node[pin={[pin distance=-0.6\onedegree,Bayer]00:{$\zeta$}}] at (axis cs:{115.455},{-72.606}) {}; % Vol,  3.95 




\end{polaraxis}


%
\end{tikzpicture}

\end{document}
