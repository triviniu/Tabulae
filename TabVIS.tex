
\documentclass[10pt,landscape]{article}


\usepackage[margin=1cm,a3paper]{geometry}

\def\pgfsysdriver{pgfsys-pdftex.def}
\usepackage{pgfplots}
\usepgfplotslibrary{external} 
\usetikzlibrary{pgfplots.external}
\usetikzlibrary{calc}
\usetikzlibrary{shapes}
\usetikzlibrary{patterns}
\usetikzlibrary{plotmarks}
\usepgflibrary{arrows}
\usepgfplotslibrary{polar}
\tikzexternalize

\usepackage{times,latexsym,amssymb}
\usepackage{eulervm} % math font
\pagestyle{empty}

\usepackage{amsmath} 
\usepackage{nicefrac} 
%
% This is used because from 1.11 onwards the use of \pgfdeclareplotmark will
% introduce a shift of the stellar markers.
%
\pgfplotsset{compat=1.10}
%
% Use to increase spacing for constellation labels
\usepackage[letterspace=400]{microtype}
%
\usepackage{pagecolor}
\usepackage{natbib}
%


%\pagecolor{black}
\pagecolor{white}

%%%%%%%%%%%%%%%%%%%%%%%%%%%%%%%%%%%%%%%%%%%%%%%%%%%%%%%%%%%%%%%%%%%%%%
\begin{document}

\input{./input/styles_eq_S.tex}
\center

\tikzsetnextfilename{TabulaVI_S}
%
% Coordinate limit of tab to plot
\pgfplotsset{xmin=270,xmax=380,ymin=-40,ymax=40,y dir=reverse,x dir=normal}

\begin{tikzpicture}
%
% Empty plot to establish anchor points
\begin{axis}[name=base,axis lines=none]\end{axis}

% The Milky Way first, as it should not obstruct coordinate grid 


\begin{axis}[name=milkyWay,axis lines=none]
%
% Broad band
\addplot[MW1] table[x index=0,y index=1] {./MW/VI_MWbase.dat}  -- cycle ;
%
% Light areas in broad band
\addplot[MW0] table[x index=0,y index=1] {./MW/VI_ol1_4.dat}  -- cycle ;
\addplot[MW0] table[x index=0,y index=1] {./MW/VI_ol1_8.dat}  -- cycle ;
%
% Darker level I
\addplot[MW2] table[x index=0,y index=1] {./MW/VI_ol2_105.dat}  -- cycle ;
\addplot[MW2] table[x index=0,y index=1] {./MW/VI_ol2_10.dat}  -- cycle ;
\addplot[MW2] table[x index=0,y index=1] {./MW/VI_ol2_111.dat}  -- cycle ;
\addplot[MW2] table[x index=0,y index=1] {./MW/VI_ol2_24.dat}  -- cycle ;
\addplot[MW2] table[x index=0,y index=1] {./MW/VI_ol2_29.dat}  -- cycle ;
\addplot[MW2] table[x index=0,y index=1] {./MW/VI_ol2_36.dat}  -- cycle ;
\addplot[MW2] table[x index=0,y index=1] {./MW/VI_ol2_3.dat}  -- cycle ;
\addplot[MW2] table[x index=0,y index=1] {./MW/VI_ol2_41.dat}  -- cycle ;
\addplot[MW2] table[x index=0,y index=1] {./MW/VI_ol2_42.dat}  -- cycle ;
\addplot[MW2] table[x index=0,y index=1] {./MW/VI_ol2_47.dat}  -- cycle ;
\addplot[MW2] table[x index=0,y index=1] {./MW/VI_ol2_4.dat}  -- cycle ;
\addplot[MW2] table[x index=0,y index=1] {./MW/VI_ol2_58.dat}  -- cycle ;
\addplot[MW2] table[x index=0,y index=1] {./MW/VI_ol2_60.dat}  -- cycle ;
\addplot[MW2] table[x index=0,y index=1] {./MW/VI_ol2_64.dat}  -- cycle ;
\addplot[MW2] table[x index=0,y index=1] {./MW/VI_ol2_70.dat}  -- cycle ;
\addplot[MW2] table[x index=0,y index=1] {./MW/VI_ol2_72.dat}  -- cycle ;
\addplot[MW2] table[x index=0,y index=1] {./MW/VI_ol2_81.dat}  -- cycle ;
\addplot[MW2] table[x index=0,y index=1] {./MW/VI_ol2_99.dat}  -- cycle ;
\addplot[MW2] table[x index=0,y index=1] {./MW/VI_ol2_104.dat}  -- cycle ;
\addplot[MW2] table[x index=0,y index=1] {./MW/VI_ol2_108.dat}  -- cycle ;
\addplot[MW2] table[x index=0,y index=1] {./MW/VI_ol2_109.dat}  -- cycle ;
\addplot[MW2] table[x index=0,y index=1] {./MW/VI_ol2_112.dat}  -- cycle ;
\addplot[MW2] table[x index=0,y index=1] {./MW/VI_ol2_11.dat}  -- cycle ;
\addplot[MW2] table[x index=0,y index=1] {./MW/VI_ol2_12.dat}  -- cycle ;
\addplot[MW2] table[x index=0,y index=1] {./MW/VI_ol2_19.dat}  -- cycle ;
\addplot[MW2] table[x index=0,y index=1] {./MW/VI_ol2_27.dat}  -- cycle ;
\addplot[MW2] table[x index=0,y index=1] {./MW/VI_ol2_28.dat}  -- cycle ;
\addplot[MW2] table[x index=0,y index=1] {./MW/VI_ol2_2.dat}  -- cycle ;
\addplot[MW2] table[x index=0,y index=1] {./MW/VI_ol2_31.dat}  -- cycle ;
\addplot[MW2] table[x index=0,y index=1] {./MW/VI_ol2_35.dat}  -- cycle ;
\addplot[MW2] table[x index=0,y index=1] {./MW/VI_ol2_38.dat}  -- cycle ;
\addplot[MW2] table[x index=0,y index=1] {./MW/VI_ol2_44.dat}  -- cycle ;
\addplot[MW2] table[x index=0,y index=1] {./MW/VI_ol2_45.dat}  -- cycle ;
\addplot[MW2] table[x index=0,y index=1] {./MW/VI_ol2_49.dat}  -- cycle ;
\addplot[MW2] table[x index=0,y index=1] {./MW/VI_ol2_54.dat}  -- cycle ;
\addplot[MW2] table[x index=0,y index=1] {./MW/VI_ol2_67.dat}  -- cycle ;
\addplot[MW2] table[x index=0,y index=1] {./MW/VI_ol2_74.dat}  -- cycle ;
\addplot[MW2] table[x index=0,y index=1] {./MW/VI_ol2_78.dat}  -- cycle ;
\addplot[MW2] table[x index=0,y index=1] {./MW/VI_ol2_87.dat}  -- cycle ;
\addplot[MW2] table[x index=0,y index=1] {./MW/VI_ol2_95.dat}  -- cycle ;
\addplot[MW2] table[x index=0,y index=1] {./MW/VI_ol2_96.dat}  -- cycle ;
\addplot[MW2] table[x index=0,y index=1] {./MW/VI_ol2_97.dat}  -- cycle ;
%
% Light areas in Darker level I
\addplot[MW1] table[x index=0,y index=1] {./MW/VI_ol2_91.dat}  -- cycle ;
\addplot[MW1] table[x index=0,y index=1] {./MW/VI_ol2_73.dat}  -- cycle ;
\addplot[MW1] table[x index=0,y index=1] {./MW/VI_ol2_75.dat}  -- cycle ;
\addplot[MW1] table[x index=0,y index=1] {./MW/VI_ol2_68.dat}  -- cycle ;
\addplot[MW1] table[x index=0,y index=1] {./MW/VI_ol2_94.dat}  -- cycle ;
%
% Darker level II
\addplot[MW3] table[x index=0,y index=1] {./MW/VI_ol3_10.dat}  -- cycle ;
\addplot[MW3] table[x index=0,y index=1] {./MW/VI_ol3_18.dat}  -- cycle ;
\addplot[MW3] table[x index=0,y index=1] {./MW/VI_ol3_25.dat}  -- cycle ;
\addplot[MW3] table[x index=0,y index=1] {./MW/VI_ol3_36.dat}  -- cycle ;
\addplot[MW3] table[x index=0,y index=1] {./MW/VI_ol3_37.dat}  -- cycle ;
\addplot[MW3] table[x index=0,y index=1] {./MW/VI_ol3_3.dat}  -- cycle ;
\addplot[MW3] table[x index=0,y index=1] {./MW/VI_ol3_44.dat}  -- cycle ;
\addplot[MW3] table[x index=0,y index=1] {./MW/VI_ol3_4.dat}  -- cycle ;
\addplot[MW3] table[x index=0,y index=1] {./MW/VI_ol3_8.dat}  -- cycle ;
\addplot[MW3] table[x index=0,y index=1] {./MW/VI_ol3_9.dat}  -- cycle ;
\addplot[MW3] table[x index=0,y index=1] {./MW/VI_ol3_11.dat}  -- cycle ;
\addplot[MW3] table[x index=0,y index=1] {./MW/VI_ol3_12.dat}  -- cycle ;
\addplot[MW3] table[x index=0,y index=1] {./MW/VI_ol3_17.dat}  -- cycle ;
\addplot[MW3] table[x index=0,y index=1] {./MW/VI_ol3_19.dat}  -- cycle ;
\addplot[MW3] table[x index=0,y index=1] {./MW/VI_ol3_1.dat}  -- cycle ;
\addplot[MW3] table[x index=0,y index=1] {./MW/VI_ol3_20.dat}  -- cycle ;
\addplot[MW3] table[x index=0,y index=1] {./MW/VI_ol3_2.dat}  -- cycle ;
\addplot[MW3] table[x index=0,y index=1] {./MW/VI_ol3_34.dat}  -- cycle ;
\addplot[MW3] table[x index=0,y index=1] {./MW/VI_ol3_45.dat}  -- cycle ;
\addplot[MW3] table[x index=0,y index=1] {./MW/VI_ol3_8.dat}  -- cycle ;
%
% Light areas in Darker level II
\addplot[MW2] table[x index=0,y index=1] {./MW/VI_ol3_43.dat}  -- cycle ;
\addplot[MW2] table[x index=0,y index=1] {./MW/VI_ol3_41.dat}  -- cycle ;
\addplot[MW2] table[x index=0,y index=1] {./MW/VI_ol3_26.dat}  -- cycle ;
%
% Darker level III
\addplot[MW4] table[x index=0,y index=1] {./MW/VI_ol4_14.dat}  -- cycle ;
\addplot[MW4] table[x index=0,y index=1] {./MW/VI_ol4_15.dat}  -- cycle ;
\addplot[MW4] table[x index=0,y index=1] {./MW/VI_ol4_16.dat}  -- cycle ;
\addplot[MW4] table[x index=0,y index=1] {./MW/VI_ol4_18.dat}  -- cycle ;
\addplot[MW4] table[x index=0,y index=1] {./MW/VI_ol4_19.dat}  -- cycle ;
\addplot[MW4] table[x index=0,y index=1] {./MW/VI_ol4_1.dat}  -- cycle ;
\addplot[MW4] table[x index=0,y index=1] {./MW/VI_ol4_20.dat}  -- cycle ;
\addplot[MW4] table[x index=0,y index=1] {./MW/VI_ol4_2.dat}  -- cycle ;
\addplot[MW4] table[x index=0,y index=1] {./MW/VI_ol4_6.dat}  -- cycle ;
\addplot[MW4] table[x index=0,y index=1] {./MW/VI_ol4_9.dat}  -- cycle ;
\addplot[MW4] table[x index=0,y index=1] {./MW/VI_ol4_10.dat}  -- cycle ;
\addplot[MW4] table[x index=0,y index=1] {./MW/VI_ol4_17.dat}  -- cycle ;
\addplot[MW4] table[x index=0,y index=1] {./MW/VI_ol4_23.dat}  -- cycle ;
\addplot[MW4] table[x index=0,y index=1] {./MW/VI_ol4_25.dat}  -- cycle ;
\addplot[MW4] table[x index=0,y index=1] {./MW/VI_ol4_2.dat}  -- cycle ;
%
% Darker level IV
\addplot[MW5] table[x index=0,y index=1] {./MW/VI_ol5_1.dat}  -- cycle ;
\addplot[MW5] table[x index=0,y index=1] {./MW/VI_ol5_2.dat}  -- cycle ;
\addplot[MW5] table[x index=0,y index=1] {./MW/VI_ol5_3.dat}  -- cycle ;
\addplot[MW5] table[x index=0,y index=1] {./MW/VI_ol5_6.dat}  -- cycle ;




\end{axis}


% The Galactic coordinate system
\input{./input/MW.tex}
% Ecliptic coordinate system
\input{./input/Ecliptics.tex}
% Equatorial coordinate system
%
% Some are commented out because they are surrounded by already plotted borders
% The x-coordinate is in fractional hours, so must be times 15
%

\begin{axis}[name=constellations,at=(base.center),anchor=center,axis lines=none]


\draw[Equator-full] (axis cs:0.00000E+00,1.00000E-01) -- (axis cs:1.00000E+00,1.00000E-01) -- (axis cs:1.00000E+00,-1.00000E-01) -- (axis cs:0.00000E+00,-1.00000E-01) -- cycle ; 
\draw[Equator-full] (axis cs:2.00000E+00,1.00000E-01) -- (axis cs:3.00000E+00,1.00000E-01) -- (axis cs:3.00000E+00,-1.00000E-01) -- (axis cs:2.00000E+00,-1.00000E-01) -- cycle ; 
\draw[Equator-full] (axis cs:4.00000E+00,1.00000E-01) -- (axis cs:5.00000E+00,1.00000E-01) -- (axis cs:5.00000E+00,-1.00000E-01) -- (axis cs:4.00000E+00,-1.00000E-01) -- cycle ; 
\draw[Equator-full] (axis cs:6.00000E+00,1.00000E-01) -- (axis cs:7.00000E+00,1.00000E-01) -- (axis cs:7.00000E+00,-1.00000E-01) -- (axis cs:6.00000E+00,-1.00000E-01) -- cycle ; 
\draw[Equator-full] (axis cs:8.00000E+00,1.00000E-01) -- (axis cs:9.00000E+00,1.00000E-01) -- (axis cs:9.00000E+00,-1.00000E-01) -- (axis cs:8.00000E+00,-1.00000E-01) -- cycle ; 
\draw[Equator-full] (axis cs:1.00000E+01,1.00000E-01) -- (axis cs:1.10000E+01,1.00000E-01) -- (axis cs:1.10000E+01,-1.00000E-01) -- (axis cs:1.00000E+01,-1.00000E-01) -- cycle ; 
\draw[Equator-full] (axis cs:1.20000E+01,1.00000E-01) -- (axis cs:1.30000E+01,1.00000E-01) -- (axis cs:1.30000E+01,-1.00000E-01) -- (axis cs:1.20000E+01,-1.00000E-01) -- cycle ; 
\draw[Equator-full] (axis cs:1.40000E+01,1.00000E-01) -- (axis cs:1.50000E+01,1.00000E-01) -- (axis cs:1.50000E+01,-1.00000E-01) -- (axis cs:1.40000E+01,-1.00000E-01) -- cycle ; 
\draw[Equator-full] (axis cs:1.60000E+01,1.00000E-01) -- (axis cs:1.70000E+01,1.00000E-01) -- (axis cs:1.70000E+01,-1.00000E-01) -- (axis cs:1.60000E+01,-1.00000E-01) -- cycle ; 
\draw[Equator-full] (axis cs:1.80000E+01,1.00000E-01) -- (axis cs:1.90000E+01,1.00000E-01) -- (axis cs:1.90000E+01,-1.00000E-01) -- (axis cs:1.80000E+01,-1.00000E-01) -- cycle ; 
\draw[Equator-full] (axis cs:2.00000E+01,1.00000E-01) -- (axis cs:2.10000E+01,1.00000E-01) -- (axis cs:2.10000E+01,-1.00000E-01) -- (axis cs:2.00000E+01,-1.00000E-01) -- cycle ; 
\draw[Equator-full] (axis cs:2.20000E+01,1.00000E-01) -- (axis cs:2.30000E+01,1.00000E-01) -- (axis cs:2.30000E+01,-1.00000E-01) -- (axis cs:2.20000E+01,-1.00000E-01) -- cycle ; 
\draw[Equator-full] (axis cs:2.40000E+01,1.00000E-01) -- (axis cs:2.50000E+01,1.00000E-01) -- (axis cs:2.50000E+01,-1.00000E-01) -- (axis cs:2.40000E+01,-1.00000E-01) -- cycle ; 
\draw[Equator-full] (axis cs:2.60000E+01,1.00000E-01) -- (axis cs:2.70000E+01,1.00000E-01) -- (axis cs:2.70000E+01,-1.00000E-01) -- (axis cs:2.60000E+01,-1.00000E-01) -- cycle ; 
\draw[Equator-full] (axis cs:2.80000E+01,1.00000E-01) -- (axis cs:2.90000E+01,1.00000E-01) -- (axis cs:2.90000E+01,-1.00000E-01) -- (axis cs:2.80000E+01,-1.00000E-01) -- cycle ; 
\draw[Equator-full] (axis cs:3.00000E+01,1.00000E-01) -- (axis cs:3.10000E+01,1.00000E-01) -- (axis cs:3.10000E+01,-1.00000E-01) -- (axis cs:3.00000E+01,-1.00000E-01) -- cycle ; 
\draw[Equator-full] (axis cs:3.20000E+01,1.00000E-01) -- (axis cs:3.30000E+01,1.00000E-01) -- (axis cs:3.30000E+01,-1.00000E-01) -- (axis cs:3.20000E+01,-1.00000E-01) -- cycle ; 
\draw[Equator-full] (axis cs:3.40000E+01,1.00000E-01) -- (axis cs:3.50000E+01,1.00000E-01) -- (axis cs:3.50000E+01,-1.00000E-01) -- (axis cs:3.40000E+01,-1.00000E-01) -- cycle ; 
\draw[Equator-full] (axis cs:3.60000E+01,1.00000E-01) -- (axis cs:3.70000E+01,1.00000E-01) -- (axis cs:3.70000E+01,-1.00000E-01) -- (axis cs:3.60000E+01,-1.00000E-01) -- cycle ; 
\draw[Equator-full] (axis cs:3.80000E+01,1.00000E-01) -- (axis cs:3.90000E+01,1.00000E-01) -- (axis cs:3.90000E+01,-1.00000E-01) -- (axis cs:3.80000E+01,-1.00000E-01) -- cycle ; 
\draw[Equator-full] (axis cs:4.00000E+01,1.00000E-01) -- (axis cs:4.10000E+01,1.00000E-01) -- (axis cs:4.10000E+01,-1.00000E-01) -- (axis cs:4.00000E+01,-1.00000E-01) -- cycle ; 
\draw[Equator-full] (axis cs:4.20000E+01,1.00000E-01) -- (axis cs:4.30000E+01,1.00000E-01) -- (axis cs:4.30000E+01,-1.00000E-01) -- (axis cs:4.20000E+01,-1.00000E-01) -- cycle ; 
\draw[Equator-full] (axis cs:4.40000E+01,1.00000E-01) -- (axis cs:4.50000E+01,1.00000E-01) -- (axis cs:4.50000E+01,-1.00000E-01) -- (axis cs:4.40000E+01,-1.00000E-01) -- cycle ; 
\draw[Equator-full] (axis cs:4.60000E+01,1.00000E-01) -- (axis cs:4.70000E+01,1.00000E-01) -- (axis cs:4.70000E+01,-1.00000E-01) -- (axis cs:4.60000E+01,-1.00000E-01) -- cycle ; 
\draw[Equator-full] (axis cs:4.80000E+01,1.00000E-01) -- (axis cs:4.90000E+01,1.00000E-01) -- (axis cs:4.90000E+01,-1.00000E-01) -- (axis cs:4.80000E+01,-1.00000E-01) -- cycle ; 
\draw[Equator-full] (axis cs:5.00000E+01,1.00000E-01) -- (axis cs:5.10000E+01,1.00000E-01) -- (axis cs:5.10000E+01,-1.00000E-01) -- (axis cs:5.00000E+01,-1.00000E-01) -- cycle ; 
\draw[Equator-full] (axis cs:5.20000E+01,1.00000E-01) -- (axis cs:5.30000E+01,1.00000E-01) -- (axis cs:5.30000E+01,-1.00000E-01) -- (axis cs:5.20000E+01,-1.00000E-01) -- cycle ; 
\draw[Equator-full] (axis cs:5.40000E+01,1.00000E-01) -- (axis cs:5.50000E+01,1.00000E-01) -- (axis cs:5.50000E+01,-1.00000E-01) -- (axis cs:5.40000E+01,-1.00000E-01) -- cycle ; 
\draw[Equator-full] (axis cs:5.60000E+01,1.00000E-01) -- (axis cs:5.70000E+01,1.00000E-01) -- (axis cs:5.70000E+01,-1.00000E-01) -- (axis cs:5.60000E+01,-1.00000E-01) -- cycle ; 
\draw[Equator-full] (axis cs:5.80000E+01,1.00000E-01) -- (axis cs:5.90000E+01,1.00000E-01) -- (axis cs:5.90000E+01,-1.00000E-01) -- (axis cs:5.80000E+01,-1.00000E-01) -- cycle ; 
\draw[Equator-full] (axis cs:6.00000E+01,1.00000E-01) -- (axis cs:6.10000E+01,1.00000E-01) -- (axis cs:6.10000E+01,-1.00000E-01) -- (axis cs:6.00000E+01,-1.00000E-01) -- cycle ; 
\draw[Equator-full] (axis cs:6.20000E+01,1.00000E-01) -- (axis cs:6.30000E+01,1.00000E-01) -- (axis cs:6.30000E+01,-1.00000E-01) -- (axis cs:6.20000E+01,-1.00000E-01) -- cycle ; 
\draw[Equator-full] (axis cs:6.40000E+01,1.00000E-01) -- (axis cs:6.50000E+01,1.00000E-01) -- (axis cs:6.50000E+01,-1.00000E-01) -- (axis cs:6.40000E+01,-1.00000E-01) -- cycle ; 
\draw[Equator-full] (axis cs:6.60000E+01,1.00000E-01) -- (axis cs:6.70000E+01,1.00000E-01) -- (axis cs:6.70000E+01,-1.00000E-01) -- (axis cs:6.60000E+01,-1.00000E-01) -- cycle ; 
\draw[Equator-full] (axis cs:6.80000E+01,1.00000E-01) -- (axis cs:6.90000E+01,1.00000E-01) -- (axis cs:6.90000E+01,-1.00000E-01) -- (axis cs:6.80000E+01,-1.00000E-01) -- cycle ; 
\draw[Equator-full] (axis cs:7.00000E+01,1.00000E-01) -- (axis cs:7.10000E+01,1.00000E-01) -- (axis cs:7.10000E+01,-1.00000E-01) -- (axis cs:7.00000E+01,-1.00000E-01) -- cycle ; 
\draw[Equator-full] (axis cs:7.20000E+01,1.00000E-01) -- (axis cs:7.30000E+01,1.00000E-01) -- (axis cs:7.30000E+01,-1.00000E-01) -- (axis cs:7.20000E+01,-1.00000E-01) -- cycle ; 
\draw[Equator-full] (axis cs:7.40000E+01,1.00000E-01) -- (axis cs:7.50000E+01,1.00000E-01) -- (axis cs:7.50000E+01,-1.00000E-01) -- (axis cs:7.40000E+01,-1.00000E-01) -- cycle ; 
\draw[Equator-full] (axis cs:7.60000E+01,1.00000E-01) -- (axis cs:7.70000E+01,1.00000E-01) -- (axis cs:7.70000E+01,-1.00000E-01) -- (axis cs:7.60000E+01,-1.00000E-01) -- cycle ; 
\draw[Equator-full] (axis cs:7.80000E+01,1.00000E-01) -- (axis cs:7.90000E+01,1.00000E-01) -- (axis cs:7.90000E+01,-1.00000E-01) -- (axis cs:7.80000E+01,-1.00000E-01) -- cycle ; 
\draw[Equator-full] (axis cs:8.00000E+01,1.00000E-01) -- (axis cs:8.10000E+01,1.00000E-01) -- (axis cs:8.10000E+01,-1.00000E-01) -- (axis cs:8.00000E+01,-1.00000E-01) -- cycle ; 
\draw[Equator-full] (axis cs:8.20000E+01,1.00000E-01) -- (axis cs:8.30000E+01,1.00000E-01) -- (axis cs:8.30000E+01,-1.00000E-01) -- (axis cs:8.20000E+01,-1.00000E-01) -- cycle ; 
\draw[Equator-full] (axis cs:8.40000E+01,1.00000E-01) -- (axis cs:8.50000E+01,1.00000E-01) -- (axis cs:8.50000E+01,-1.00000E-01) -- (axis cs:8.40000E+01,-1.00000E-01) -- cycle ; 
\draw[Equator-full] (axis cs:8.60000E+01,1.00000E-01) -- (axis cs:8.70000E+01,1.00000E-01) -- (axis cs:8.70000E+01,-1.00000E-01) -- (axis cs:8.60000E+01,-1.00000E-01) -- cycle ; 
\draw[Equator-full] (axis cs:8.80000E+01,1.00000E-01) -- (axis cs:8.90000E+01,1.00000E-01) -- (axis cs:8.90000E+01,-1.00000E-01) -- (axis cs:8.80000E+01,-1.00000E-01) -- cycle ; 
\draw[Equator-full] (axis cs:9.00000E+01,1.00000E-01) -- (axis cs:9.10000E+01,1.00000E-01) -- (axis cs:9.10000E+01,-1.00000E-01) -- (axis cs:9.00000E+01,-1.00000E-01) -- cycle ; 
\draw[Equator-full] (axis cs:9.20000E+01,1.00000E-01) -- (axis cs:9.30000E+01,1.00000E-01) -- (axis cs:9.30000E+01,-1.00000E-01) -- (axis cs:9.20000E+01,-1.00000E-01) -- cycle ; 
\draw[Equator-full] (axis cs:9.40000E+01,1.00000E-01) -- (axis cs:9.50000E+01,1.00000E-01) -- (axis cs:9.50000E+01,-1.00000E-01) -- (axis cs:9.40000E+01,-1.00000E-01) -- cycle ; 
\draw[Equator-full] (axis cs:9.60000E+01,1.00000E-01) -- (axis cs:9.70000E+01,1.00000E-01) -- (axis cs:9.70000E+01,-1.00000E-01) -- (axis cs:9.60000E+01,-1.00000E-01) -- cycle ; 
\draw[Equator-full] (axis cs:9.80000E+01,1.00000E-01) -- (axis cs:9.90000E+01,1.00000E-01) -- (axis cs:9.90000E+01,-1.00000E-01) -- (axis cs:9.80000E+01,-1.00000E-01) -- cycle ; 
\draw[Equator-full] (axis cs:1.00000E+02,1.00000E-01) -- (axis cs:1.01000E+02,1.00000E-01) -- (axis cs:1.01000E+02,-1.00000E-01) -- (axis cs:1.00000E+02,-1.00000E-01) -- cycle ; 
\draw[Equator-full] (axis cs:1.02000E+02,1.00000E-01) -- (axis cs:1.03000E+02,1.00000E-01) -- (axis cs:1.03000E+02,-1.00000E-01) -- (axis cs:1.02000E+02,-1.00000E-01) -- cycle ; 
\draw[Equator-full] (axis cs:1.04000E+02,1.00000E-01) -- (axis cs:1.05000E+02,1.00000E-01) -- (axis cs:1.05000E+02,-1.00000E-01) -- (axis cs:1.04000E+02,-1.00000E-01) -- cycle ; 
\draw[Equator-full] (axis cs:1.06000E+02,1.00000E-01) -- (axis cs:1.07000E+02,1.00000E-01) -- (axis cs:1.07000E+02,-1.00000E-01) -- (axis cs:1.06000E+02,-1.00000E-01) -- cycle ; 
\draw[Equator-full] (axis cs:1.08000E+02,1.00000E-01) -- (axis cs:1.09000E+02,1.00000E-01) -- (axis cs:1.09000E+02,-1.00000E-01) -- (axis cs:1.08000E+02,-1.00000E-01) -- cycle ; 
\draw[Equator-full] (axis cs:1.10000E+02,1.00000E-01) -- (axis cs:1.11000E+02,1.00000E-01) -- (axis cs:1.11000E+02,-1.00000E-01) -- (axis cs:1.10000E+02,-1.00000E-01) -- cycle ; 
\draw[Equator-full] (axis cs:1.12000E+02,1.00000E-01) -- (axis cs:1.13000E+02,1.00000E-01) -- (axis cs:1.13000E+02,-1.00000E-01) -- (axis cs:1.12000E+02,-1.00000E-01) -- cycle ; 
\draw[Equator-full] (axis cs:1.14000E+02,1.00000E-01) -- (axis cs:1.15000E+02,1.00000E-01) -- (axis cs:1.15000E+02,-1.00000E-01) -- (axis cs:1.14000E+02,-1.00000E-01) -- cycle ; 
\draw[Equator-full] (axis cs:1.16000E+02,1.00000E-01) -- (axis cs:1.17000E+02,1.00000E-01) -- (axis cs:1.17000E+02,-1.00000E-01) -- (axis cs:1.16000E+02,-1.00000E-01) -- cycle ; 
\draw[Equator-full] (axis cs:1.18000E+02,1.00000E-01) -- (axis cs:1.19000E+02,1.00000E-01) -- (axis cs:1.19000E+02,-1.00000E-01) -- (axis cs:1.18000E+02,-1.00000E-01) -- cycle ; 
\draw[Equator-full] (axis cs:1.20000E+02,1.00000E-01) -- (axis cs:1.21000E+02,1.00000E-01) -- (axis cs:1.21000E+02,-1.00000E-01) -- (axis cs:1.20000E+02,-1.00000E-01) -- cycle ; 
\draw[Equator-full] (axis cs:1.22000E+02,1.00000E-01) -- (axis cs:1.23000E+02,1.00000E-01) -- (axis cs:1.23000E+02,-1.00000E-01) -- (axis cs:1.22000E+02,-1.00000E-01) -- cycle ; 
\draw[Equator-full] (axis cs:1.24000E+02,1.00000E-01) -- (axis cs:1.25000E+02,1.00000E-01) -- (axis cs:1.25000E+02,-1.00000E-01) -- (axis cs:1.24000E+02,-1.00000E-01) -- cycle ; 
\draw[Equator-full] (axis cs:1.26000E+02,1.00000E-01) -- (axis cs:1.27000E+02,1.00000E-01) -- (axis cs:1.27000E+02,-1.00000E-01) -- (axis cs:1.26000E+02,-1.00000E-01) -- cycle ; 
\draw[Equator-full] (axis cs:1.28000E+02,1.00000E-01) -- (axis cs:1.29000E+02,1.00000E-01) -- (axis cs:1.29000E+02,-1.00000E-01) -- (axis cs:1.28000E+02,-1.00000E-01) -- cycle ; 
\draw[Equator-full] (axis cs:1.30000E+02,1.00000E-01) -- (axis cs:1.31000E+02,1.00000E-01) -- (axis cs:1.31000E+02,-1.00000E-01) -- (axis cs:1.30000E+02,-1.00000E-01) -- cycle ; 
\draw[Equator-full] (axis cs:1.32000E+02,1.00000E-01) -- (axis cs:1.33000E+02,1.00000E-01) -- (axis cs:1.33000E+02,-1.00000E-01) -- (axis cs:1.32000E+02,-1.00000E-01) -- cycle ; 
\draw[Equator-full] (axis cs:1.34000E+02,1.00000E-01) -- (axis cs:1.35000E+02,1.00000E-01) -- (axis cs:1.35000E+02,-1.00000E-01) -- (axis cs:1.34000E+02,-1.00000E-01) -- cycle ; 
\draw[Equator-full] (axis cs:1.36000E+02,1.00000E-01) -- (axis cs:1.37000E+02,1.00000E-01) -- (axis cs:1.37000E+02,-1.00000E-01) -- (axis cs:1.36000E+02,-1.00000E-01) -- cycle ; 
\draw[Equator-full] (axis cs:1.38000E+02,1.00000E-01) -- (axis cs:1.39000E+02,1.00000E-01) -- (axis cs:1.39000E+02,-1.00000E-01) -- (axis cs:1.38000E+02,-1.00000E-01) -- cycle ; 
\draw[Equator-full] (axis cs:1.40000E+02,1.00000E-01) -- (axis cs:1.41000E+02,1.00000E-01) -- (axis cs:1.41000E+02,-1.00000E-01) -- (axis cs:1.40000E+02,-1.00000E-01) -- cycle ; 
\draw[Equator-full] (axis cs:1.42000E+02,1.00000E-01) -- (axis cs:1.43000E+02,1.00000E-01) -- (axis cs:1.43000E+02,-1.00000E-01) -- (axis cs:1.42000E+02,-1.00000E-01) -- cycle ; 
\draw[Equator-full] (axis cs:1.44000E+02,1.00000E-01) -- (axis cs:1.45000E+02,1.00000E-01) -- (axis cs:1.45000E+02,-1.00000E-01) -- (axis cs:1.44000E+02,-1.00000E-01) -- cycle ; 
\draw[Equator-full] (axis cs:1.46000E+02,1.00000E-01) -- (axis cs:1.47000E+02,1.00000E-01) -- (axis cs:1.47000E+02,-1.00000E-01) -- (axis cs:1.46000E+02,-1.00000E-01) -- cycle ; 
\draw[Equator-full] (axis cs:1.48000E+02,1.00000E-01) -- (axis cs:1.49000E+02,1.00000E-01) -- (axis cs:1.49000E+02,-1.00000E-01) -- (axis cs:1.48000E+02,-1.00000E-01) -- cycle ; 
\draw[Equator-full] (axis cs:1.50000E+02,1.00000E-01) -- (axis cs:1.51000E+02,1.00000E-01) -- (axis cs:1.51000E+02,-1.00000E-01) -- (axis cs:1.50000E+02,-1.00000E-01) -- cycle ; 
\draw[Equator-full] (axis cs:1.52000E+02,1.00000E-01) -- (axis cs:1.53000E+02,1.00000E-01) -- (axis cs:1.53000E+02,-1.00000E-01) -- (axis cs:1.52000E+02,-1.00000E-01) -- cycle ; 
\draw[Equator-full] (axis cs:1.54000E+02,1.00000E-01) -- (axis cs:1.55000E+02,1.00000E-01) -- (axis cs:1.55000E+02,-1.00000E-01) -- (axis cs:1.54000E+02,-1.00000E-01) -- cycle ; 
\draw[Equator-full] (axis cs:1.56000E+02,1.00000E-01) -- (axis cs:1.57000E+02,1.00000E-01) -- (axis cs:1.57000E+02,-1.00000E-01) -- (axis cs:1.56000E+02,-1.00000E-01) -- cycle ; 
\draw[Equator-full] (axis cs:1.58000E+02,1.00000E-01) -- (axis cs:1.59000E+02,1.00000E-01) -- (axis cs:1.59000E+02,-1.00000E-01) -- (axis cs:1.58000E+02,-1.00000E-01) -- cycle ; 
\draw[Equator-full] (axis cs:1.60000E+02,1.00000E-01) -- (axis cs:1.61000E+02,1.00000E-01) -- (axis cs:1.61000E+02,-1.00000E-01) -- (axis cs:1.60000E+02,-1.00000E-01) -- cycle ; 
\draw[Equator-full] (axis cs:1.62000E+02,1.00000E-01) -- (axis cs:1.63000E+02,1.00000E-01) -- (axis cs:1.63000E+02,-1.00000E-01) -- (axis cs:1.62000E+02,-1.00000E-01) -- cycle ; 
\draw[Equator-full] (axis cs:1.64000E+02,1.00000E-01) -- (axis cs:1.65000E+02,1.00000E-01) -- (axis cs:1.65000E+02,-1.00000E-01) -- (axis cs:1.64000E+02,-1.00000E-01) -- cycle ; 
\draw[Equator-full] (axis cs:1.66000E+02,1.00000E-01) -- (axis cs:1.67000E+02,1.00000E-01) -- (axis cs:1.67000E+02,-1.00000E-01) -- (axis cs:1.66000E+02,-1.00000E-01) -- cycle ; 
\draw[Equator-full] (axis cs:1.68000E+02,1.00000E-01) -- (axis cs:1.69000E+02,1.00000E-01) -- (axis cs:1.69000E+02,-1.00000E-01) -- (axis cs:1.68000E+02,-1.00000E-01) -- cycle ; 
\draw[Equator-full] (axis cs:1.70000E+02,1.00000E-01) -- (axis cs:1.71000E+02,1.00000E-01) -- (axis cs:1.71000E+02,-1.00000E-01) -- (axis cs:1.70000E+02,-1.00000E-01) -- cycle ; 
\draw[Equator-full] (axis cs:1.72000E+02,1.00000E-01) -- (axis cs:1.73000E+02,1.00000E-01) -- (axis cs:1.73000E+02,-1.00000E-01) -- (axis cs:1.72000E+02,-1.00000E-01) -- cycle ; 
\draw[Equator-full] (axis cs:1.74000E+02,1.00000E-01) -- (axis cs:1.75000E+02,1.00000E-01) -- (axis cs:1.75000E+02,-1.00000E-01) -- (axis cs:1.74000E+02,-1.00000E-01) -- cycle ; 
\draw[Equator-full] (axis cs:1.76000E+02,1.00000E-01) -- (axis cs:1.77000E+02,1.00000E-01) -- (axis cs:1.77000E+02,-1.00000E-01) -- (axis cs:1.76000E+02,-1.00000E-01) -- cycle ; 
\draw[Equator-full] (axis cs:1.78000E+02,1.00000E-01) -- (axis cs:1.79000E+02,1.00000E-01) -- (axis cs:1.79000E+02,-1.00000E-01) -- (axis cs:1.78000E+02,-1.00000E-01) -- cycle ; 
\draw[Equator-full] (axis cs:1.80000E+02,1.00000E-01) -- (axis cs:1.81000E+02,1.00000E-01) -- (axis cs:1.81000E+02,-1.00000E-01) -- (axis cs:1.80000E+02,-1.00000E-01) -- cycle ; 
\draw[Equator-full] (axis cs:1.82000E+02,1.00000E-01) -- (axis cs:1.83000E+02,1.00000E-01) -- (axis cs:1.83000E+02,-1.00000E-01) -- (axis cs:1.82000E+02,-1.00000E-01) -- cycle ; 
\draw[Equator-full] (axis cs:1.84000E+02,1.00000E-01) -- (axis cs:1.85000E+02,1.00000E-01) -- (axis cs:1.85000E+02,-1.00000E-01) -- (axis cs:1.84000E+02,-1.00000E-01) -- cycle ; 
\draw[Equator-full] (axis cs:1.86000E+02,1.00000E-01) -- (axis cs:1.87000E+02,1.00000E-01) -- (axis cs:1.87000E+02,-1.00000E-01) -- (axis cs:1.86000E+02,-1.00000E-01) -- cycle ; 
\draw[Equator-full] (axis cs:1.88000E+02,1.00000E-01) -- (axis cs:1.89000E+02,1.00000E-01) -- (axis cs:1.89000E+02,-1.00000E-01) -- (axis cs:1.88000E+02,-1.00000E-01) -- cycle ; 
\draw[Equator-full] (axis cs:1.90000E+02,1.00000E-01) -- (axis cs:1.91000E+02,1.00000E-01) -- (axis cs:1.91000E+02,-1.00000E-01) -- (axis cs:1.90000E+02,-1.00000E-01) -- cycle ; 
\draw[Equator-full] (axis cs:1.92000E+02,1.00000E-01) -- (axis cs:1.93000E+02,1.00000E-01) -- (axis cs:1.93000E+02,-1.00000E-01) -- (axis cs:1.92000E+02,-1.00000E-01) -- cycle ; 
\draw[Equator-full] (axis cs:1.94000E+02,1.00000E-01) -- (axis cs:1.95000E+02,1.00000E-01) -- (axis cs:1.95000E+02,-1.00000E-01) -- (axis cs:1.94000E+02,-1.00000E-01) -- cycle ; 
\draw[Equator-full] (axis cs:1.96000E+02,1.00000E-01) -- (axis cs:1.97000E+02,1.00000E-01) -- (axis cs:1.97000E+02,-1.00000E-01) -- (axis cs:1.96000E+02,-1.00000E-01) -- cycle ; 
\draw[Equator-full] (axis cs:1.98000E+02,1.00000E-01) -- (axis cs:1.99000E+02,1.00000E-01) -- (axis cs:1.99000E+02,-1.00000E-01) -- (axis cs:1.98000E+02,-1.00000E-01) -- cycle ; 
\draw[Equator-full] (axis cs:2.00000E+02,1.00000E-01) -- (axis cs:2.01000E+02,1.00000E-01) -- (axis cs:2.01000E+02,-1.00000E-01) -- (axis cs:2.00000E+02,-1.00000E-01) -- cycle ; 
\draw[Equator-full] (axis cs:2.02000E+02,1.00000E-01) -- (axis cs:2.03000E+02,1.00000E-01) -- (axis cs:2.03000E+02,-1.00000E-01) -- (axis cs:2.02000E+02,-1.00000E-01) -- cycle ; 
\draw[Equator-full] (axis cs:2.04000E+02,1.00000E-01) -- (axis cs:2.05000E+02,1.00000E-01) -- (axis cs:2.05000E+02,-1.00000E-01) -- (axis cs:2.04000E+02,-1.00000E-01) -- cycle ; 
\draw[Equator-full] (axis cs:2.06000E+02,1.00000E-01) -- (axis cs:2.07000E+02,1.00000E-01) -- (axis cs:2.07000E+02,-1.00000E-01) -- (axis cs:2.06000E+02,-1.00000E-01) -- cycle ; 
\draw[Equator-full] (axis cs:2.08000E+02,1.00000E-01) -- (axis cs:2.09000E+02,1.00000E-01) -- (axis cs:2.09000E+02,-1.00000E-01) -- (axis cs:2.08000E+02,-1.00000E-01) -- cycle ; 
\draw[Equator-full] (axis cs:2.10000E+02,1.00000E-01) -- (axis cs:2.11000E+02,1.00000E-01) -- (axis cs:2.11000E+02,-1.00000E-01) -- (axis cs:2.10000E+02,-1.00000E-01) -- cycle ; 
\draw[Equator-full] (axis cs:2.12000E+02,1.00000E-01) -- (axis cs:2.13000E+02,1.00000E-01) -- (axis cs:2.13000E+02,-1.00000E-01) -- (axis cs:2.12000E+02,-1.00000E-01) -- cycle ; 
\draw[Equator-full] (axis cs:2.14000E+02,1.00000E-01) -- (axis cs:2.15000E+02,1.00000E-01) -- (axis cs:2.15000E+02,-1.00000E-01) -- (axis cs:2.14000E+02,-1.00000E-01) -- cycle ; 
\draw[Equator-full] (axis cs:2.16000E+02,1.00000E-01) -- (axis cs:2.17000E+02,1.00000E-01) -- (axis cs:2.17000E+02,-1.00000E-01) -- (axis cs:2.16000E+02,-1.00000E-01) -- cycle ; 
\draw[Equator-full] (axis cs:2.18000E+02,1.00000E-01) -- (axis cs:2.19000E+02,1.00000E-01) -- (axis cs:2.19000E+02,-1.00000E-01) -- (axis cs:2.18000E+02,-1.00000E-01) -- cycle ; 
\draw[Equator-full] (axis cs:2.20000E+02,1.00000E-01) -- (axis cs:2.21000E+02,1.00000E-01) -- (axis cs:2.21000E+02,-1.00000E-01) -- (axis cs:2.20000E+02,-1.00000E-01) -- cycle ; 
\draw[Equator-full] (axis cs:2.22000E+02,1.00000E-01) -- (axis cs:2.23000E+02,1.00000E-01) -- (axis cs:2.23000E+02,-1.00000E-01) -- (axis cs:2.22000E+02,-1.00000E-01) -- cycle ; 
\draw[Equator-full] (axis cs:2.24000E+02,1.00000E-01) -- (axis cs:2.25000E+02,1.00000E-01) -- (axis cs:2.25000E+02,-1.00000E-01) -- (axis cs:2.24000E+02,-1.00000E-01) -- cycle ; 
\draw[Equator-full] (axis cs:2.26000E+02,1.00000E-01) -- (axis cs:2.27000E+02,1.00000E-01) -- (axis cs:2.27000E+02,-1.00000E-01) -- (axis cs:2.26000E+02,-1.00000E-01) -- cycle ; 
\draw[Equator-full] (axis cs:2.28000E+02,1.00000E-01) -- (axis cs:2.29000E+02,1.00000E-01) -- (axis cs:2.29000E+02,-1.00000E-01) -- (axis cs:2.28000E+02,-1.00000E-01) -- cycle ; 
\draw[Equator-full] (axis cs:2.30000E+02,1.00000E-01) -- (axis cs:2.31000E+02,1.00000E-01) -- (axis cs:2.31000E+02,-1.00000E-01) -- (axis cs:2.30000E+02,-1.00000E-01) -- cycle ; 
\draw[Equator-full] (axis cs:2.32000E+02,1.00000E-01) -- (axis cs:2.33000E+02,1.00000E-01) -- (axis cs:2.33000E+02,-1.00000E-01) -- (axis cs:2.32000E+02,-1.00000E-01) -- cycle ; 
\draw[Equator-full] (axis cs:2.34000E+02,1.00000E-01) -- (axis cs:2.35000E+02,1.00000E-01) -- (axis cs:2.35000E+02,-1.00000E-01) -- (axis cs:2.34000E+02,-1.00000E-01) -- cycle ; 
\draw[Equator-full] (axis cs:2.36000E+02,1.00000E-01) -- (axis cs:2.37000E+02,1.00000E-01) -- (axis cs:2.37000E+02,-1.00000E-01) -- (axis cs:2.36000E+02,-1.00000E-01) -- cycle ; 
\draw[Equator-full] (axis cs:2.38000E+02,1.00000E-01) -- (axis cs:2.39000E+02,1.00000E-01) -- (axis cs:2.39000E+02,-1.00000E-01) -- (axis cs:2.38000E+02,-1.00000E-01) -- cycle ; 
\draw[Equator-full] (axis cs:2.40000E+02,1.00000E-01) -- (axis cs:2.41000E+02,1.00000E-01) -- (axis cs:2.41000E+02,-1.00000E-01) -- (axis cs:2.40000E+02,-1.00000E-01) -- cycle ; 
\draw[Equator-full] (axis cs:2.42000E+02,1.00000E-01) -- (axis cs:2.43000E+02,1.00000E-01) -- (axis cs:2.43000E+02,-1.00000E-01) -- (axis cs:2.42000E+02,-1.00000E-01) -- cycle ; 
\draw[Equator-full] (axis cs:2.44000E+02,1.00000E-01) -- (axis cs:2.45000E+02,1.00000E-01) -- (axis cs:2.45000E+02,-1.00000E-01) -- (axis cs:2.44000E+02,-1.00000E-01) -- cycle ; 
\draw[Equator-full] (axis cs:2.46000E+02,1.00000E-01) -- (axis cs:2.47000E+02,1.00000E-01) -- (axis cs:2.47000E+02,-1.00000E-01) -- (axis cs:2.46000E+02,-1.00000E-01) -- cycle ; 
\draw[Equator-full] (axis cs:2.48000E+02,1.00000E-01) -- (axis cs:2.49000E+02,1.00000E-01) -- (axis cs:2.49000E+02,-1.00000E-01) -- (axis cs:2.48000E+02,-1.00000E-01) -- cycle ; 
\draw[Equator-full] (axis cs:2.50000E+02,1.00000E-01) -- (axis cs:2.51000E+02,1.00000E-01) -- (axis cs:2.51000E+02,-1.00000E-01) -- (axis cs:2.50000E+02,-1.00000E-01) -- cycle ; 
\draw[Equator-full] (axis cs:2.52000E+02,1.00000E-01) -- (axis cs:2.53000E+02,1.00000E-01) -- (axis cs:2.53000E+02,-1.00000E-01) -- (axis cs:2.52000E+02,-1.00000E-01) -- cycle ; 
\draw[Equator-full] (axis cs:2.54000E+02,1.00000E-01) -- (axis cs:2.55000E+02,1.00000E-01) -- (axis cs:2.55000E+02,-1.00000E-01) -- (axis cs:2.54000E+02,-1.00000E-01) -- cycle ; 
\draw[Equator-full] (axis cs:2.56000E+02,1.00000E-01) -- (axis cs:2.57000E+02,1.00000E-01) -- (axis cs:2.57000E+02,-1.00000E-01) -- (axis cs:2.56000E+02,-1.00000E-01) -- cycle ; 
\draw[Equator-full] (axis cs:2.58000E+02,1.00000E-01) -- (axis cs:2.59000E+02,1.00000E-01) -- (axis cs:2.59000E+02,-1.00000E-01) -- (axis cs:2.58000E+02,-1.00000E-01) -- cycle ; 
\draw[Equator-full] (axis cs:2.60000E+02,1.00000E-01) -- (axis cs:2.61000E+02,1.00000E-01) -- (axis cs:2.61000E+02,-1.00000E-01) -- (axis cs:2.60000E+02,-1.00000E-01) -- cycle ; 
\draw[Equator-full] (axis cs:2.62000E+02,1.00000E-01) -- (axis cs:2.63000E+02,1.00000E-01) -- (axis cs:2.63000E+02,-1.00000E-01) -- (axis cs:2.62000E+02,-1.00000E-01) -- cycle ; 
\draw[Equator-full] (axis cs:2.64000E+02,1.00000E-01) -- (axis cs:2.65000E+02,1.00000E-01) -- (axis cs:2.65000E+02,-1.00000E-01) -- (axis cs:2.64000E+02,-1.00000E-01) -- cycle ; 
\draw[Equator-full] (axis cs:2.66000E+02,1.00000E-01) -- (axis cs:2.67000E+02,1.00000E-01) -- (axis cs:2.67000E+02,-1.00000E-01) -- (axis cs:2.66000E+02,-1.00000E-01) -- cycle ; 
\draw[Equator-full] (axis cs:2.68000E+02,1.00000E-01) -- (axis cs:2.69000E+02,1.00000E-01) -- (axis cs:2.69000E+02,-1.00000E-01) -- (axis cs:2.68000E+02,-1.00000E-01) -- cycle ; 
\draw[Equator-full] (axis cs:2.70000E+02,1.00000E-01) -- (axis cs:2.71000E+02,1.00000E-01) -- (axis cs:2.71000E+02,-1.00000E-01) -- (axis cs:2.70000E+02,-1.00000E-01) -- cycle ; 
\draw[Equator-full] (axis cs:2.72000E+02,1.00000E-01) -- (axis cs:2.73000E+02,1.00000E-01) -- (axis cs:2.73000E+02,-1.00000E-01) -- (axis cs:2.72000E+02,-1.00000E-01) -- cycle ; 
\draw[Equator-full] (axis cs:2.74000E+02,1.00000E-01) -- (axis cs:2.75000E+02,1.00000E-01) -- (axis cs:2.75000E+02,-1.00000E-01) -- (axis cs:2.74000E+02,-1.00000E-01) -- cycle ; 
\draw[Equator-full] (axis cs:2.76000E+02,1.00000E-01) -- (axis cs:2.77000E+02,1.00000E-01) -- (axis cs:2.77000E+02,-1.00000E-01) -- (axis cs:2.76000E+02,-1.00000E-01) -- cycle ; 
\draw[Equator-full] (axis cs:2.78000E+02,1.00000E-01) -- (axis cs:2.79000E+02,1.00000E-01) -- (axis cs:2.79000E+02,-1.00000E-01) -- (axis cs:2.78000E+02,-1.00000E-01) -- cycle ; 
\draw[Equator-full] (axis cs:2.80000E+02,1.00000E-01) -- (axis cs:2.81000E+02,1.00000E-01) -- (axis cs:2.81000E+02,-1.00000E-01) -- (axis cs:2.80000E+02,-1.00000E-01) -- cycle ; 
\draw[Equator-full] (axis cs:2.82000E+02,1.00000E-01) -- (axis cs:2.83000E+02,1.00000E-01) -- (axis cs:2.83000E+02,-1.00000E-01) -- (axis cs:2.82000E+02,-1.00000E-01) -- cycle ; 
\draw[Equator-full] (axis cs:2.84000E+02,1.00000E-01) -- (axis cs:2.85000E+02,1.00000E-01) -- (axis cs:2.85000E+02,-1.00000E-01) -- (axis cs:2.84000E+02,-1.00000E-01) -- cycle ; 
\draw[Equator-full] (axis cs:2.86000E+02,1.00000E-01) -- (axis cs:2.87000E+02,1.00000E-01) -- (axis cs:2.87000E+02,-1.00000E-01) -- (axis cs:2.86000E+02,-1.00000E-01) -- cycle ; 
\draw[Equator-full] (axis cs:2.88000E+02,1.00000E-01) -- (axis cs:2.89000E+02,1.00000E-01) -- (axis cs:2.89000E+02,-1.00000E-01) -- (axis cs:2.88000E+02,-1.00000E-01) -- cycle ; 
\draw[Equator-full] (axis cs:2.90000E+02,1.00000E-01) -- (axis cs:2.91000E+02,1.00000E-01) -- (axis cs:2.91000E+02,-1.00000E-01) -- (axis cs:2.90000E+02,-1.00000E-01) -- cycle ; 
\draw[Equator-full] (axis cs:2.92000E+02,1.00000E-01) -- (axis cs:2.93000E+02,1.00000E-01) -- (axis cs:2.93000E+02,-1.00000E-01) -- (axis cs:2.92000E+02,-1.00000E-01) -- cycle ; 
\draw[Equator-full] (axis cs:2.94000E+02,1.00000E-01) -- (axis cs:2.95000E+02,1.00000E-01) -- (axis cs:2.95000E+02,-1.00000E-01) -- (axis cs:2.94000E+02,-1.00000E-01) -- cycle ; 
\draw[Equator-full] (axis cs:2.96000E+02,1.00000E-01) -- (axis cs:2.97000E+02,1.00000E-01) -- (axis cs:2.97000E+02,-1.00000E-01) -- (axis cs:2.96000E+02,-1.00000E-01) -- cycle ; 
\draw[Equator-full] (axis cs:2.98000E+02,1.00000E-01) -- (axis cs:2.99000E+02,1.00000E-01) -- (axis cs:2.99000E+02,-1.00000E-01) -- (axis cs:2.98000E+02,-1.00000E-01) -- cycle ; 
\draw[Equator-full] (axis cs:3.00000E+02,1.00000E-01) -- (axis cs:3.01000E+02,1.00000E-01) -- (axis cs:3.01000E+02,-1.00000E-01) -- (axis cs:3.00000E+02,-1.00000E-01) -- cycle ; 
\draw[Equator-full] (axis cs:3.02000E+02,1.00000E-01) -- (axis cs:3.03000E+02,1.00000E-01) -- (axis cs:3.03000E+02,-1.00000E-01) -- (axis cs:3.02000E+02,-1.00000E-01) -- cycle ; 
\draw[Equator-full] (axis cs:3.04000E+02,1.00000E-01) -- (axis cs:3.05000E+02,1.00000E-01) -- (axis cs:3.05000E+02,-1.00000E-01) -- (axis cs:3.04000E+02,-1.00000E-01) -- cycle ; 
\draw[Equator-full] (axis cs:3.06000E+02,1.00000E-01) -- (axis cs:3.07000E+02,1.00000E-01) -- (axis cs:3.07000E+02,-1.00000E-01) -- (axis cs:3.06000E+02,-1.00000E-01) -- cycle ; 
\draw[Equator-full] (axis cs:3.08000E+02,1.00000E-01) -- (axis cs:3.09000E+02,1.00000E-01) -- (axis cs:3.09000E+02,-1.00000E-01) -- (axis cs:3.08000E+02,-1.00000E-01) -- cycle ; 
\draw[Equator-full] (axis cs:3.10000E+02,1.00000E-01) -- (axis cs:3.11000E+02,1.00000E-01) -- (axis cs:3.11000E+02,-1.00000E-01) -- (axis cs:3.10000E+02,-1.00000E-01) -- cycle ; 
\draw[Equator-full] (axis cs:3.12000E+02,1.00000E-01) -- (axis cs:3.13000E+02,1.00000E-01) -- (axis cs:3.13000E+02,-1.00000E-01) -- (axis cs:3.12000E+02,-1.00000E-01) -- cycle ; 
\draw[Equator-full] (axis cs:3.14000E+02,1.00000E-01) -- (axis cs:3.15000E+02,1.00000E-01) -- (axis cs:3.15000E+02,-1.00000E-01) -- (axis cs:3.14000E+02,-1.00000E-01) -- cycle ; 
\draw[Equator-full] (axis cs:3.16000E+02,1.00000E-01) -- (axis cs:3.17000E+02,1.00000E-01) -- (axis cs:3.17000E+02,-1.00000E-01) -- (axis cs:3.16000E+02,-1.00000E-01) -- cycle ; 
\draw[Equator-full] (axis cs:3.18000E+02,1.00000E-01) -- (axis cs:3.19000E+02,1.00000E-01) -- (axis cs:3.19000E+02,-1.00000E-01) -- (axis cs:3.18000E+02,-1.00000E-01) -- cycle ; 
\draw[Equator-full] (axis cs:3.20000E+02,1.00000E-01) -- (axis cs:3.21000E+02,1.00000E-01) -- (axis cs:3.21000E+02,-1.00000E-01) -- (axis cs:3.20000E+02,-1.00000E-01) -- cycle ; 
\draw[Equator-full] (axis cs:3.22000E+02,1.00000E-01) -- (axis cs:3.23000E+02,1.00000E-01) -- (axis cs:3.23000E+02,-1.00000E-01) -- (axis cs:3.22000E+02,-1.00000E-01) -- cycle ; 
\draw[Equator-full] (axis cs:3.24000E+02,1.00000E-01) -- (axis cs:3.25000E+02,1.00000E-01) -- (axis cs:3.25000E+02,-1.00000E-01) -- (axis cs:3.24000E+02,-1.00000E-01) -- cycle ; 
\draw[Equator-full] (axis cs:3.26000E+02,1.00000E-01) -- (axis cs:3.27000E+02,1.00000E-01) -- (axis cs:3.27000E+02,-1.00000E-01) -- (axis cs:3.26000E+02,-1.00000E-01) -- cycle ; 
\draw[Equator-full] (axis cs:3.28000E+02,1.00000E-01) -- (axis cs:3.29000E+02,1.00000E-01) -- (axis cs:3.29000E+02,-1.00000E-01) -- (axis cs:3.28000E+02,-1.00000E-01) -- cycle ; 
\draw[Equator-full] (axis cs:3.30000E+02,1.00000E-01) -- (axis cs:3.31000E+02,1.00000E-01) -- (axis cs:3.31000E+02,-1.00000E-01) -- (axis cs:3.30000E+02,-1.00000E-01) -- cycle ; 
\draw[Equator-full] (axis cs:3.32000E+02,1.00000E-01) -- (axis cs:3.33000E+02,1.00000E-01) -- (axis cs:3.33000E+02,-1.00000E-01) -- (axis cs:3.32000E+02,-1.00000E-01) -- cycle ; 
\draw[Equator-full] (axis cs:3.34000E+02,1.00000E-01) -- (axis cs:3.35000E+02,1.00000E-01) -- (axis cs:3.35000E+02,-1.00000E-01) -- (axis cs:3.34000E+02,-1.00000E-01) -- cycle ; 
\draw[Equator-full] (axis cs:3.36000E+02,1.00000E-01) -- (axis cs:3.37000E+02,1.00000E-01) -- (axis cs:3.37000E+02,-1.00000E-01) -- (axis cs:3.36000E+02,-1.00000E-01) -- cycle ; 
\draw[Equator-full] (axis cs:3.38000E+02,1.00000E-01) -- (axis cs:3.39000E+02,1.00000E-01) -- (axis cs:3.39000E+02,-1.00000E-01) -- (axis cs:3.38000E+02,-1.00000E-01) -- cycle ; 
\draw[Equator-full] (axis cs:3.40000E+02,1.00000E-01) -- (axis cs:3.41000E+02,1.00000E-01) -- (axis cs:3.41000E+02,-1.00000E-01) -- (axis cs:3.40000E+02,-1.00000E-01) -- cycle ; 
\draw[Equator-full] (axis cs:3.42000E+02,1.00000E-01) -- (axis cs:3.43000E+02,1.00000E-01) -- (axis cs:3.43000E+02,-1.00000E-01) -- (axis cs:3.42000E+02,-1.00000E-01) -- cycle ; 
\draw[Equator-full] (axis cs:3.44000E+02,1.00000E-01) -- (axis cs:3.45000E+02,1.00000E-01) -- (axis cs:3.45000E+02,-1.00000E-01) -- (axis cs:3.44000E+02,-1.00000E-01) -- cycle ; 
\draw[Equator-full] (axis cs:3.46000E+02,1.00000E-01) -- (axis cs:3.47000E+02,1.00000E-01) -- (axis cs:3.47000E+02,-1.00000E-01) -- (axis cs:3.46000E+02,-1.00000E-01) -- cycle ; 
\draw[Equator-full] (axis cs:3.48000E+02,1.00000E-01) -- (axis cs:3.49000E+02,1.00000E-01) -- (axis cs:3.49000E+02,-1.00000E-01) -- (axis cs:3.48000E+02,-1.00000E-01) -- cycle ; 
\draw[Equator-full] (axis cs:3.50000E+02,1.00000E-01) -- (axis cs:3.51000E+02,1.00000E-01) -- (axis cs:3.51000E+02,-1.00000E-01) -- (axis cs:3.50000E+02,-1.00000E-01) -- cycle ; 
\draw[Equator-full] (axis cs:3.52000E+02,1.00000E-01) -- (axis cs:3.53000E+02,1.00000E-01) -- (axis cs:3.53000E+02,-1.00000E-01) -- (axis cs:3.52000E+02,-1.00000E-01) -- cycle ; 
\draw[Equator-full] (axis cs:3.54000E+02,1.00000E-01) -- (axis cs:3.55000E+02,1.00000E-01) -- (axis cs:3.55000E+02,-1.00000E-01) -- (axis cs:3.54000E+02,-1.00000E-01) -- cycle ; 
\draw[Equator-full] (axis cs:3.56000E+02,1.00000E-01) -- (axis cs:3.57000E+02,1.00000E-01) -- (axis cs:3.57000E+02,-1.00000E-01) -- (axis cs:3.56000E+02,-1.00000E-01) -- cycle ; 
\draw[Equator-full] (axis cs:3.58000E+02,1.00000E-01) -- (axis cs:3.59000E+02,1.00000E-01) -- (axis cs:3.59000E+02,-1.00000E-01) -- (axis cs:3.58000E+02,-1.00000E-01) -- cycle ; 
\draw[Equator-empty] (axis cs:1.00000E+00,1.00000E-01) -- (axis cs:2.00000E+00,1.00000E-01) -- (axis cs:2.00000E+00,-1.00000E-01) -- (axis cs:1.00000E+00,-1.00000E-01) -- cycle ; 
\draw[Equator-empty] (axis cs:3.00000E+00,1.00000E-01) -- (axis cs:4.00000E+00,1.00000E-01) -- (axis cs:4.00000E+00,-1.00000E-01) -- (axis cs:3.00000E+00,-1.00000E-01) -- cycle ; 
\draw[Equator-empty] (axis cs:5.00000E+00,1.00000E-01) -- (axis cs:6.00000E+00,1.00000E-01) -- (axis cs:6.00000E+00,-1.00000E-01) -- (axis cs:5.00000E+00,-1.00000E-01) -- cycle ; 
\draw[Equator-empty] (axis cs:7.00000E+00,1.00000E-01) -- (axis cs:8.00000E+00,1.00000E-01) -- (axis cs:8.00000E+00,-1.00000E-01) -- (axis cs:7.00000E+00,-1.00000E-01) -- cycle ; 
\draw[Equator-empty] (axis cs:9.00000E+00,1.00000E-01) -- (axis cs:1.00000E+01,1.00000E-01) -- (axis cs:1.00000E+01,-1.00000E-01) -- (axis cs:9.00000E+00,-1.00000E-01) -- cycle ; 
\draw[Equator-empty] (axis cs:1.10000E+01,1.00000E-01) -- (axis cs:1.20000E+01,1.00000E-01) -- (axis cs:1.20000E+01,-1.00000E-01) -- (axis cs:1.10000E+01,-1.00000E-01) -- cycle ; 
\draw[Equator-empty] (axis cs:1.30000E+01,1.00000E-01) -- (axis cs:1.40000E+01,1.00000E-01) -- (axis cs:1.40000E+01,-1.00000E-01) -- (axis cs:1.30000E+01,-1.00000E-01) -- cycle ; 
\draw[Equator-empty] (axis cs:1.50000E+01,1.00000E-01) -- (axis cs:1.60000E+01,1.00000E-01) -- (axis cs:1.60000E+01,-1.00000E-01) -- (axis cs:1.50000E+01,-1.00000E-01) -- cycle ; 
\draw[Equator-empty] (axis cs:1.70000E+01,1.00000E-01) -- (axis cs:1.80000E+01,1.00000E-01) -- (axis cs:1.80000E+01,-1.00000E-01) -- (axis cs:1.70000E+01,-1.00000E-01) -- cycle ; 
\draw[Equator-empty] (axis cs:1.90000E+01,1.00000E-01) -- (axis cs:2.00000E+01,1.00000E-01) -- (axis cs:2.00000E+01,-1.00000E-01) -- (axis cs:1.90000E+01,-1.00000E-01) -- cycle ; 
\draw[Equator-empty] (axis cs:2.10000E+01,1.00000E-01) -- (axis cs:2.20000E+01,1.00000E-01) -- (axis cs:2.20000E+01,-1.00000E-01) -- (axis cs:2.10000E+01,-1.00000E-01) -- cycle ; 
\draw[Equator-empty] (axis cs:2.30000E+01,1.00000E-01) -- (axis cs:2.40000E+01,1.00000E-01) -- (axis cs:2.40000E+01,-1.00000E-01) -- (axis cs:2.30000E+01,-1.00000E-01) -- cycle ; 
\draw[Equator-empty] (axis cs:2.50000E+01,1.00000E-01) -- (axis cs:2.60000E+01,1.00000E-01) -- (axis cs:2.60000E+01,-1.00000E-01) -- (axis cs:2.50000E+01,-1.00000E-01) -- cycle ; 
\draw[Equator-empty] (axis cs:2.70000E+01,1.00000E-01) -- (axis cs:2.80000E+01,1.00000E-01) -- (axis cs:2.80000E+01,-1.00000E-01) -- (axis cs:2.70000E+01,-1.00000E-01) -- cycle ; 
\draw[Equator-empty] (axis cs:2.90000E+01,1.00000E-01) -- (axis cs:3.00000E+01,1.00000E-01) -- (axis cs:3.00000E+01,-1.00000E-01) -- (axis cs:2.90000E+01,-1.00000E-01) -- cycle ; 
\draw[Equator-empty] (axis cs:3.10000E+01,1.00000E-01) -- (axis cs:3.20000E+01,1.00000E-01) -- (axis cs:3.20000E+01,-1.00000E-01) -- (axis cs:3.10000E+01,-1.00000E-01) -- cycle ; 
\draw[Equator-empty] (axis cs:3.30000E+01,1.00000E-01) -- (axis cs:3.40000E+01,1.00000E-01) -- (axis cs:3.40000E+01,-1.00000E-01) -- (axis cs:3.30000E+01,-1.00000E-01) -- cycle ; 
\draw[Equator-empty] (axis cs:3.50000E+01,1.00000E-01) -- (axis cs:3.60000E+01,1.00000E-01) -- (axis cs:3.60000E+01,-1.00000E-01) -- (axis cs:3.50000E+01,-1.00000E-01) -- cycle ; 
\draw[Equator-empty] (axis cs:3.70000E+01,1.00000E-01) -- (axis cs:3.80000E+01,1.00000E-01) -- (axis cs:3.80000E+01,-1.00000E-01) -- (axis cs:3.70000E+01,-1.00000E-01) -- cycle ; 
\draw[Equator-empty] (axis cs:3.90000E+01,1.00000E-01) -- (axis cs:4.00000E+01,1.00000E-01) -- (axis cs:4.00000E+01,-1.00000E-01) -- (axis cs:3.90000E+01,-1.00000E-01) -- cycle ; 
\draw[Equator-empty] (axis cs:4.10000E+01,1.00000E-01) -- (axis cs:4.20000E+01,1.00000E-01) -- (axis cs:4.20000E+01,-1.00000E-01) -- (axis cs:4.10000E+01,-1.00000E-01) -- cycle ; 
\draw[Equator-empty] (axis cs:4.30000E+01,1.00000E-01) -- (axis cs:4.40000E+01,1.00000E-01) -- (axis cs:4.40000E+01,-1.00000E-01) -- (axis cs:4.30000E+01,-1.00000E-01) -- cycle ; 
\draw[Equator-empty] (axis cs:4.50000E+01,1.00000E-01) -- (axis cs:4.60000E+01,1.00000E-01) -- (axis cs:4.60000E+01,-1.00000E-01) -- (axis cs:4.50000E+01,-1.00000E-01) -- cycle ; 
\draw[Equator-empty] (axis cs:4.70000E+01,1.00000E-01) -- (axis cs:4.80000E+01,1.00000E-01) -- (axis cs:4.80000E+01,-1.00000E-01) -- (axis cs:4.70000E+01,-1.00000E-01) -- cycle ; 
\draw[Equator-empty] (axis cs:4.90000E+01,1.00000E-01) -- (axis cs:5.00000E+01,1.00000E-01) -- (axis cs:5.00000E+01,-1.00000E-01) -- (axis cs:4.90000E+01,-1.00000E-01) -- cycle ; 
\draw[Equator-empty] (axis cs:5.10000E+01,1.00000E-01) -- (axis cs:5.20000E+01,1.00000E-01) -- (axis cs:5.20000E+01,-1.00000E-01) -- (axis cs:5.10000E+01,-1.00000E-01) -- cycle ; 
\draw[Equator-empty] (axis cs:5.30000E+01,1.00000E-01) -- (axis cs:5.40000E+01,1.00000E-01) -- (axis cs:5.40000E+01,-1.00000E-01) -- (axis cs:5.30000E+01,-1.00000E-01) -- cycle ; 
\draw[Equator-empty] (axis cs:5.50000E+01,1.00000E-01) -- (axis cs:5.60000E+01,1.00000E-01) -- (axis cs:5.60000E+01,-1.00000E-01) -- (axis cs:5.50000E+01,-1.00000E-01) -- cycle ; 
\draw[Equator-empty] (axis cs:5.70000E+01,1.00000E-01) -- (axis cs:5.80000E+01,1.00000E-01) -- (axis cs:5.80000E+01,-1.00000E-01) -- (axis cs:5.70000E+01,-1.00000E-01) -- cycle ; 
\draw[Equator-empty] (axis cs:5.90000E+01,1.00000E-01) -- (axis cs:6.00000E+01,1.00000E-01) -- (axis cs:6.00000E+01,-1.00000E-01) -- (axis cs:5.90000E+01,-1.00000E-01) -- cycle ; 
\draw[Equator-empty] (axis cs:6.10000E+01,1.00000E-01) -- (axis cs:6.20000E+01,1.00000E-01) -- (axis cs:6.20000E+01,-1.00000E-01) -- (axis cs:6.10000E+01,-1.00000E-01) -- cycle ; 
\draw[Equator-empty] (axis cs:6.30000E+01,1.00000E-01) -- (axis cs:6.40000E+01,1.00000E-01) -- (axis cs:6.40000E+01,-1.00000E-01) -- (axis cs:6.30000E+01,-1.00000E-01) -- cycle ; 
\draw[Equator-empty] (axis cs:6.50000E+01,1.00000E-01) -- (axis cs:6.60000E+01,1.00000E-01) -- (axis cs:6.60000E+01,-1.00000E-01) -- (axis cs:6.50000E+01,-1.00000E-01) -- cycle ; 
\draw[Equator-empty] (axis cs:6.70000E+01,1.00000E-01) -- (axis cs:6.80000E+01,1.00000E-01) -- (axis cs:6.80000E+01,-1.00000E-01) -- (axis cs:6.70000E+01,-1.00000E-01) -- cycle ; 
\draw[Equator-empty] (axis cs:6.90000E+01,1.00000E-01) -- (axis cs:7.00000E+01,1.00000E-01) -- (axis cs:7.00000E+01,-1.00000E-01) -- (axis cs:6.90000E+01,-1.00000E-01) -- cycle ; 
\draw[Equator-empty] (axis cs:7.10000E+01,1.00000E-01) -- (axis cs:7.20000E+01,1.00000E-01) -- (axis cs:7.20000E+01,-1.00000E-01) -- (axis cs:7.10000E+01,-1.00000E-01) -- cycle ; 
\draw[Equator-empty] (axis cs:7.30000E+01,1.00000E-01) -- (axis cs:7.40000E+01,1.00000E-01) -- (axis cs:7.40000E+01,-1.00000E-01) -- (axis cs:7.30000E+01,-1.00000E-01) -- cycle ; 
\draw[Equator-empty] (axis cs:7.50000E+01,1.00000E-01) -- (axis cs:7.60000E+01,1.00000E-01) -- (axis cs:7.60000E+01,-1.00000E-01) -- (axis cs:7.50000E+01,-1.00000E-01) -- cycle ; 
\draw[Equator-empty] (axis cs:7.70000E+01,1.00000E-01) -- (axis cs:7.80000E+01,1.00000E-01) -- (axis cs:7.80000E+01,-1.00000E-01) -- (axis cs:7.70000E+01,-1.00000E-01) -- cycle ; 
\draw[Equator-empty] (axis cs:7.90000E+01,1.00000E-01) -- (axis cs:8.00000E+01,1.00000E-01) -- (axis cs:8.00000E+01,-1.00000E-01) -- (axis cs:7.90000E+01,-1.00000E-01) -- cycle ; 
\draw[Equator-empty] (axis cs:8.10000E+01,1.00000E-01) -- (axis cs:8.20000E+01,1.00000E-01) -- (axis cs:8.20000E+01,-1.00000E-01) -- (axis cs:8.10000E+01,-1.00000E-01) -- cycle ; 
\draw[Equator-empty] (axis cs:8.30000E+01,1.00000E-01) -- (axis cs:8.40000E+01,1.00000E-01) -- (axis cs:8.40000E+01,-1.00000E-01) -- (axis cs:8.30000E+01,-1.00000E-01) -- cycle ; 
\draw[Equator-empty] (axis cs:8.50000E+01,1.00000E-01) -- (axis cs:8.60000E+01,1.00000E-01) -- (axis cs:8.60000E+01,-1.00000E-01) -- (axis cs:8.50000E+01,-1.00000E-01) -- cycle ; 
\draw[Equator-empty] (axis cs:8.70000E+01,1.00000E-01) -- (axis cs:8.80000E+01,1.00000E-01) -- (axis cs:8.80000E+01,-1.00000E-01) -- (axis cs:8.70000E+01,-1.00000E-01) -- cycle ; 
\draw[Equator-empty] (axis cs:8.90000E+01,1.00000E-01) -- (axis cs:9.00000E+01,1.00000E-01) -- (axis cs:9.00000E+01,-1.00000E-01) -- (axis cs:8.90000E+01,-1.00000E-01) -- cycle ; 
\draw[Equator-empty] (axis cs:9.10000E+01,1.00000E-01) -- (axis cs:9.20000E+01,1.00000E-01) -- (axis cs:9.20000E+01,-1.00000E-01) -- (axis cs:9.10000E+01,-1.00000E-01) -- cycle ; 
\draw[Equator-empty] (axis cs:9.30000E+01,1.00000E-01) -- (axis cs:9.40000E+01,1.00000E-01) -- (axis cs:9.40000E+01,-1.00000E-01) -- (axis cs:9.30000E+01,-1.00000E-01) -- cycle ; 
\draw[Equator-empty] (axis cs:9.50000E+01,1.00000E-01) -- (axis cs:9.60000E+01,1.00000E-01) -- (axis cs:9.60000E+01,-1.00000E-01) -- (axis cs:9.50000E+01,-1.00000E-01) -- cycle ; 
\draw[Equator-empty] (axis cs:9.70000E+01,1.00000E-01) -- (axis cs:9.80000E+01,1.00000E-01) -- (axis cs:9.80000E+01,-1.00000E-01) -- (axis cs:9.70000E+01,-1.00000E-01) -- cycle ; 
\draw[Equator-empty] (axis cs:9.90000E+01,1.00000E-01) -- (axis cs:1.00000E+02,1.00000E-01) -- (axis cs:1.00000E+02,-1.00000E-01) -- (axis cs:9.90000E+01,-1.00000E-01) -- cycle ; 
\draw[Equator-empty] (axis cs:1.01000E+02,1.00000E-01) -- (axis cs:1.02000E+02,1.00000E-01) -- (axis cs:1.02000E+02,-1.00000E-01) -- (axis cs:1.01000E+02,-1.00000E-01) -- cycle ; 
\draw[Equator-empty] (axis cs:1.03000E+02,1.00000E-01) -- (axis cs:1.04000E+02,1.00000E-01) -- (axis cs:1.04000E+02,-1.00000E-01) -- (axis cs:1.03000E+02,-1.00000E-01) -- cycle ; 
\draw[Equator-empty] (axis cs:1.05000E+02,1.00000E-01) -- (axis cs:1.06000E+02,1.00000E-01) -- (axis cs:1.06000E+02,-1.00000E-01) -- (axis cs:1.05000E+02,-1.00000E-01) -- cycle ; 
\draw[Equator-empty] (axis cs:1.07000E+02,1.00000E-01) -- (axis cs:1.08000E+02,1.00000E-01) -- (axis cs:1.08000E+02,-1.00000E-01) -- (axis cs:1.07000E+02,-1.00000E-01) -- cycle ; 
\draw[Equator-empty] (axis cs:1.09000E+02,1.00000E-01) -- (axis cs:1.10000E+02,1.00000E-01) -- (axis cs:1.10000E+02,-1.00000E-01) -- (axis cs:1.09000E+02,-1.00000E-01) -- cycle ; 
\draw[Equator-empty] (axis cs:1.11000E+02,1.00000E-01) -- (axis cs:1.12000E+02,1.00000E-01) -- (axis cs:1.12000E+02,-1.00000E-01) -- (axis cs:1.11000E+02,-1.00000E-01) -- cycle ; 
\draw[Equator-empty] (axis cs:1.13000E+02,1.00000E-01) -- (axis cs:1.14000E+02,1.00000E-01) -- (axis cs:1.14000E+02,-1.00000E-01) -- (axis cs:1.13000E+02,-1.00000E-01) -- cycle ; 
\draw[Equator-empty] (axis cs:1.15000E+02,1.00000E-01) -- (axis cs:1.16000E+02,1.00000E-01) -- (axis cs:1.16000E+02,-1.00000E-01) -- (axis cs:1.15000E+02,-1.00000E-01) -- cycle ; 
\draw[Equator-empty] (axis cs:1.17000E+02,1.00000E-01) -- (axis cs:1.18000E+02,1.00000E-01) -- (axis cs:1.18000E+02,-1.00000E-01) -- (axis cs:1.17000E+02,-1.00000E-01) -- cycle ; 
\draw[Equator-empty] (axis cs:1.19000E+02,1.00000E-01) -- (axis cs:1.20000E+02,1.00000E-01) -- (axis cs:1.20000E+02,-1.00000E-01) -- (axis cs:1.19000E+02,-1.00000E-01) -- cycle ; 
\draw[Equator-empty] (axis cs:1.21000E+02,1.00000E-01) -- (axis cs:1.22000E+02,1.00000E-01) -- (axis cs:1.22000E+02,-1.00000E-01) -- (axis cs:1.21000E+02,-1.00000E-01) -- cycle ; 
\draw[Equator-empty] (axis cs:1.23000E+02,1.00000E-01) -- (axis cs:1.24000E+02,1.00000E-01) -- (axis cs:1.24000E+02,-1.00000E-01) -- (axis cs:1.23000E+02,-1.00000E-01) -- cycle ; 
\draw[Equator-empty] (axis cs:1.25000E+02,1.00000E-01) -- (axis cs:1.26000E+02,1.00000E-01) -- (axis cs:1.26000E+02,-1.00000E-01) -- (axis cs:1.25000E+02,-1.00000E-01) -- cycle ; 
\draw[Equator-empty] (axis cs:1.27000E+02,1.00000E-01) -- (axis cs:1.28000E+02,1.00000E-01) -- (axis cs:1.28000E+02,-1.00000E-01) -- (axis cs:1.27000E+02,-1.00000E-01) -- cycle ; 
\draw[Equator-empty] (axis cs:1.29000E+02,1.00000E-01) -- (axis cs:1.30000E+02,1.00000E-01) -- (axis cs:1.30000E+02,-1.00000E-01) -- (axis cs:1.29000E+02,-1.00000E-01) -- cycle ; 
\draw[Equator-empty] (axis cs:1.31000E+02,1.00000E-01) -- (axis cs:1.32000E+02,1.00000E-01) -- (axis cs:1.32000E+02,-1.00000E-01) -- (axis cs:1.31000E+02,-1.00000E-01) -- cycle ; 
\draw[Equator-empty] (axis cs:1.33000E+02,1.00000E-01) -- (axis cs:1.34000E+02,1.00000E-01) -- (axis cs:1.34000E+02,-1.00000E-01) -- (axis cs:1.33000E+02,-1.00000E-01) -- cycle ; 
\draw[Equator-empty] (axis cs:1.35000E+02,1.00000E-01) -- (axis cs:1.36000E+02,1.00000E-01) -- (axis cs:1.36000E+02,-1.00000E-01) -- (axis cs:1.35000E+02,-1.00000E-01) -- cycle ; 
\draw[Equator-empty] (axis cs:1.37000E+02,1.00000E-01) -- (axis cs:1.38000E+02,1.00000E-01) -- (axis cs:1.38000E+02,-1.00000E-01) -- (axis cs:1.37000E+02,-1.00000E-01) -- cycle ; 
\draw[Equator-empty] (axis cs:1.39000E+02,1.00000E-01) -- (axis cs:1.40000E+02,1.00000E-01) -- (axis cs:1.40000E+02,-1.00000E-01) -- (axis cs:1.39000E+02,-1.00000E-01) -- cycle ; 
\draw[Equator-empty] (axis cs:1.41000E+02,1.00000E-01) -- (axis cs:1.42000E+02,1.00000E-01) -- (axis cs:1.42000E+02,-1.00000E-01) -- (axis cs:1.41000E+02,-1.00000E-01) -- cycle ; 
\draw[Equator-empty] (axis cs:1.43000E+02,1.00000E-01) -- (axis cs:1.44000E+02,1.00000E-01) -- (axis cs:1.44000E+02,-1.00000E-01) -- (axis cs:1.43000E+02,-1.00000E-01) -- cycle ; 
\draw[Equator-empty] (axis cs:1.45000E+02,1.00000E-01) -- (axis cs:1.46000E+02,1.00000E-01) -- (axis cs:1.46000E+02,-1.00000E-01) -- (axis cs:1.45000E+02,-1.00000E-01) -- cycle ; 
\draw[Equator-empty] (axis cs:1.47000E+02,1.00000E-01) -- (axis cs:1.48000E+02,1.00000E-01) -- (axis cs:1.48000E+02,-1.00000E-01) -- (axis cs:1.47000E+02,-1.00000E-01) -- cycle ; 
\draw[Equator-empty] (axis cs:1.49000E+02,1.00000E-01) -- (axis cs:1.50000E+02,1.00000E-01) -- (axis cs:1.50000E+02,-1.00000E-01) -- (axis cs:1.49000E+02,-1.00000E-01) -- cycle ; 
\draw[Equator-empty] (axis cs:1.51000E+02,1.00000E-01) -- (axis cs:1.52000E+02,1.00000E-01) -- (axis cs:1.52000E+02,-1.00000E-01) -- (axis cs:1.51000E+02,-1.00000E-01) -- cycle ; 
\draw[Equator-empty] (axis cs:1.53000E+02,1.00000E-01) -- (axis cs:1.54000E+02,1.00000E-01) -- (axis cs:1.54000E+02,-1.00000E-01) -- (axis cs:1.53000E+02,-1.00000E-01) -- cycle ; 
\draw[Equator-empty] (axis cs:1.55000E+02,1.00000E-01) -- (axis cs:1.56000E+02,1.00000E-01) -- (axis cs:1.56000E+02,-1.00000E-01) -- (axis cs:1.55000E+02,-1.00000E-01) -- cycle ; 
\draw[Equator-empty] (axis cs:1.57000E+02,1.00000E-01) -- (axis cs:1.58000E+02,1.00000E-01) -- (axis cs:1.58000E+02,-1.00000E-01) -- (axis cs:1.57000E+02,-1.00000E-01) -- cycle ; 
\draw[Equator-empty] (axis cs:1.59000E+02,1.00000E-01) -- (axis cs:1.60000E+02,1.00000E-01) -- (axis cs:1.60000E+02,-1.00000E-01) -- (axis cs:1.59000E+02,-1.00000E-01) -- cycle ; 
\draw[Equator-empty] (axis cs:1.61000E+02,1.00000E-01) -- (axis cs:1.62000E+02,1.00000E-01) -- (axis cs:1.62000E+02,-1.00000E-01) -- (axis cs:1.61000E+02,-1.00000E-01) -- cycle ; 
\draw[Equator-empty] (axis cs:1.63000E+02,1.00000E-01) -- (axis cs:1.64000E+02,1.00000E-01) -- (axis cs:1.64000E+02,-1.00000E-01) -- (axis cs:1.63000E+02,-1.00000E-01) -- cycle ; 
\draw[Equator-empty] (axis cs:1.65000E+02,1.00000E-01) -- (axis cs:1.66000E+02,1.00000E-01) -- (axis cs:1.66000E+02,-1.00000E-01) -- (axis cs:1.65000E+02,-1.00000E-01) -- cycle ; 
\draw[Equator-empty] (axis cs:1.67000E+02,1.00000E-01) -- (axis cs:1.68000E+02,1.00000E-01) -- (axis cs:1.68000E+02,-1.00000E-01) -- (axis cs:1.67000E+02,-1.00000E-01) -- cycle ; 
\draw[Equator-empty] (axis cs:1.69000E+02,1.00000E-01) -- (axis cs:1.70000E+02,1.00000E-01) -- (axis cs:1.70000E+02,-1.00000E-01) -- (axis cs:1.69000E+02,-1.00000E-01) -- cycle ; 
\draw[Equator-empty] (axis cs:1.71000E+02,1.00000E-01) -- (axis cs:1.72000E+02,1.00000E-01) -- (axis cs:1.72000E+02,-1.00000E-01) -- (axis cs:1.71000E+02,-1.00000E-01) -- cycle ; 
\draw[Equator-empty] (axis cs:1.73000E+02,1.00000E-01) -- (axis cs:1.74000E+02,1.00000E-01) -- (axis cs:1.74000E+02,-1.00000E-01) -- (axis cs:1.73000E+02,-1.00000E-01) -- cycle ; 
\draw[Equator-empty] (axis cs:1.75000E+02,1.00000E-01) -- (axis cs:1.76000E+02,1.00000E-01) -- (axis cs:1.76000E+02,-1.00000E-01) -- (axis cs:1.75000E+02,-1.00000E-01) -- cycle ; 
\draw[Equator-empty] (axis cs:1.77000E+02,1.00000E-01) -- (axis cs:1.78000E+02,1.00000E-01) -- (axis cs:1.78000E+02,-1.00000E-01) -- (axis cs:1.77000E+02,-1.00000E-01) -- cycle ; 
\draw[Equator-empty] (axis cs:1.79000E+02,1.00000E-01) -- (axis cs:1.80000E+02,1.00000E-01) -- (axis cs:1.80000E+02,-1.00000E-01) -- (axis cs:1.79000E+02,-1.00000E-01) -- cycle ; 
\draw[Equator-empty] (axis cs:1.81000E+02,1.00000E-01) -- (axis cs:1.82000E+02,1.00000E-01) -- (axis cs:1.82000E+02,-1.00000E-01) -- (axis cs:1.81000E+02,-1.00000E-01) -- cycle ; 
\draw[Equator-empty] (axis cs:1.83000E+02,1.00000E-01) -- (axis cs:1.84000E+02,1.00000E-01) -- (axis cs:1.84000E+02,-1.00000E-01) -- (axis cs:1.83000E+02,-1.00000E-01) -- cycle ; 
\draw[Equator-empty] (axis cs:1.85000E+02,1.00000E-01) -- (axis cs:1.86000E+02,1.00000E-01) -- (axis cs:1.86000E+02,-1.00000E-01) -- (axis cs:1.85000E+02,-1.00000E-01) -- cycle ; 
\draw[Equator-empty] (axis cs:1.87000E+02,1.00000E-01) -- (axis cs:1.88000E+02,1.00000E-01) -- (axis cs:1.88000E+02,-1.00000E-01) -- (axis cs:1.87000E+02,-1.00000E-01) -- cycle ; 
\draw[Equator-empty] (axis cs:1.89000E+02,1.00000E-01) -- (axis cs:1.90000E+02,1.00000E-01) -- (axis cs:1.90000E+02,-1.00000E-01) -- (axis cs:1.89000E+02,-1.00000E-01) -- cycle ; 
\draw[Equator-empty] (axis cs:1.91000E+02,1.00000E-01) -- (axis cs:1.92000E+02,1.00000E-01) -- (axis cs:1.92000E+02,-1.00000E-01) -- (axis cs:1.91000E+02,-1.00000E-01) -- cycle ; 
\draw[Equator-empty] (axis cs:1.93000E+02,1.00000E-01) -- (axis cs:1.94000E+02,1.00000E-01) -- (axis cs:1.94000E+02,-1.00000E-01) -- (axis cs:1.93000E+02,-1.00000E-01) -- cycle ; 
\draw[Equator-empty] (axis cs:1.95000E+02,1.00000E-01) -- (axis cs:1.96000E+02,1.00000E-01) -- (axis cs:1.96000E+02,-1.00000E-01) -- (axis cs:1.95000E+02,-1.00000E-01) -- cycle ; 
\draw[Equator-empty] (axis cs:1.97000E+02,1.00000E-01) -- (axis cs:1.98000E+02,1.00000E-01) -- (axis cs:1.98000E+02,-1.00000E-01) -- (axis cs:1.97000E+02,-1.00000E-01) -- cycle ; 
\draw[Equator-empty] (axis cs:1.99000E+02,1.00000E-01) -- (axis cs:2.00000E+02,1.00000E-01) -- (axis cs:2.00000E+02,-1.00000E-01) -- (axis cs:1.99000E+02,-1.00000E-01) -- cycle ; 
\draw[Equator-empty] (axis cs:2.01000E+02,1.00000E-01) -- (axis cs:2.02000E+02,1.00000E-01) -- (axis cs:2.02000E+02,-1.00000E-01) -- (axis cs:2.01000E+02,-1.00000E-01) -- cycle ; 
\draw[Equator-empty] (axis cs:2.03000E+02,1.00000E-01) -- (axis cs:2.04000E+02,1.00000E-01) -- (axis cs:2.04000E+02,-1.00000E-01) -- (axis cs:2.03000E+02,-1.00000E-01) -- cycle ; 
\draw[Equator-empty] (axis cs:2.05000E+02,1.00000E-01) -- (axis cs:2.06000E+02,1.00000E-01) -- (axis cs:2.06000E+02,-1.00000E-01) -- (axis cs:2.05000E+02,-1.00000E-01) -- cycle ; 
\draw[Equator-empty] (axis cs:2.07000E+02,1.00000E-01) -- (axis cs:2.08000E+02,1.00000E-01) -- (axis cs:2.08000E+02,-1.00000E-01) -- (axis cs:2.07000E+02,-1.00000E-01) -- cycle ; 
\draw[Equator-empty] (axis cs:2.09000E+02,1.00000E-01) -- (axis cs:2.10000E+02,1.00000E-01) -- (axis cs:2.10000E+02,-1.00000E-01) -- (axis cs:2.09000E+02,-1.00000E-01) -- cycle ; 
\draw[Equator-empty] (axis cs:2.11000E+02,1.00000E-01) -- (axis cs:2.12000E+02,1.00000E-01) -- (axis cs:2.12000E+02,-1.00000E-01) -- (axis cs:2.11000E+02,-1.00000E-01) -- cycle ; 
\draw[Equator-empty] (axis cs:2.13000E+02,1.00000E-01) -- (axis cs:2.14000E+02,1.00000E-01) -- (axis cs:2.14000E+02,-1.00000E-01) -- (axis cs:2.13000E+02,-1.00000E-01) -- cycle ; 
\draw[Equator-empty] (axis cs:2.15000E+02,1.00000E-01) -- (axis cs:2.16000E+02,1.00000E-01) -- (axis cs:2.16000E+02,-1.00000E-01) -- (axis cs:2.15000E+02,-1.00000E-01) -- cycle ; 
\draw[Equator-empty] (axis cs:2.17000E+02,1.00000E-01) -- (axis cs:2.18000E+02,1.00000E-01) -- (axis cs:2.18000E+02,-1.00000E-01) -- (axis cs:2.17000E+02,-1.00000E-01) -- cycle ; 
\draw[Equator-empty] (axis cs:2.19000E+02,1.00000E-01) -- (axis cs:2.20000E+02,1.00000E-01) -- (axis cs:2.20000E+02,-1.00000E-01) -- (axis cs:2.19000E+02,-1.00000E-01) -- cycle ; 
\draw[Equator-empty] (axis cs:2.21000E+02,1.00000E-01) -- (axis cs:2.22000E+02,1.00000E-01) -- (axis cs:2.22000E+02,-1.00000E-01) -- (axis cs:2.21000E+02,-1.00000E-01) -- cycle ; 
\draw[Equator-empty] (axis cs:2.23000E+02,1.00000E-01) -- (axis cs:2.24000E+02,1.00000E-01) -- (axis cs:2.24000E+02,-1.00000E-01) -- (axis cs:2.23000E+02,-1.00000E-01) -- cycle ; 
\draw[Equator-empty] (axis cs:2.25000E+02,1.00000E-01) -- (axis cs:2.26000E+02,1.00000E-01) -- (axis cs:2.26000E+02,-1.00000E-01) -- (axis cs:2.25000E+02,-1.00000E-01) -- cycle ; 
\draw[Equator-empty] (axis cs:2.27000E+02,1.00000E-01) -- (axis cs:2.28000E+02,1.00000E-01) -- (axis cs:2.28000E+02,-1.00000E-01) -- (axis cs:2.27000E+02,-1.00000E-01) -- cycle ; 
\draw[Equator-empty] (axis cs:2.29000E+02,1.00000E-01) -- (axis cs:2.30000E+02,1.00000E-01) -- (axis cs:2.30000E+02,-1.00000E-01) -- (axis cs:2.29000E+02,-1.00000E-01) -- cycle ; 
\draw[Equator-empty] (axis cs:2.31000E+02,1.00000E-01) -- (axis cs:2.32000E+02,1.00000E-01) -- (axis cs:2.32000E+02,-1.00000E-01) -- (axis cs:2.31000E+02,-1.00000E-01) -- cycle ; 
\draw[Equator-empty] (axis cs:2.33000E+02,1.00000E-01) -- (axis cs:2.34000E+02,1.00000E-01) -- (axis cs:2.34000E+02,-1.00000E-01) -- (axis cs:2.33000E+02,-1.00000E-01) -- cycle ; 
\draw[Equator-empty] (axis cs:2.35000E+02,1.00000E-01) -- (axis cs:2.36000E+02,1.00000E-01) -- (axis cs:2.36000E+02,-1.00000E-01) -- (axis cs:2.35000E+02,-1.00000E-01) -- cycle ; 
\draw[Equator-empty] (axis cs:2.37000E+02,1.00000E-01) -- (axis cs:2.38000E+02,1.00000E-01) -- (axis cs:2.38000E+02,-1.00000E-01) -- (axis cs:2.37000E+02,-1.00000E-01) -- cycle ; 
\draw[Equator-empty] (axis cs:2.39000E+02,1.00000E-01) -- (axis cs:2.40000E+02,1.00000E-01) -- (axis cs:2.40000E+02,-1.00000E-01) -- (axis cs:2.39000E+02,-1.00000E-01) -- cycle ; 
\draw[Equator-empty] (axis cs:2.41000E+02,1.00000E-01) -- (axis cs:2.42000E+02,1.00000E-01) -- (axis cs:2.42000E+02,-1.00000E-01) -- (axis cs:2.41000E+02,-1.00000E-01) -- cycle ; 
\draw[Equator-empty] (axis cs:2.43000E+02,1.00000E-01) -- (axis cs:2.44000E+02,1.00000E-01) -- (axis cs:2.44000E+02,-1.00000E-01) -- (axis cs:2.43000E+02,-1.00000E-01) -- cycle ; 
\draw[Equator-empty] (axis cs:2.45000E+02,1.00000E-01) -- (axis cs:2.46000E+02,1.00000E-01) -- (axis cs:2.46000E+02,-1.00000E-01) -- (axis cs:2.45000E+02,-1.00000E-01) -- cycle ; 
\draw[Equator-empty] (axis cs:2.47000E+02,1.00000E-01) -- (axis cs:2.48000E+02,1.00000E-01) -- (axis cs:2.48000E+02,-1.00000E-01) -- (axis cs:2.47000E+02,-1.00000E-01) -- cycle ; 
\draw[Equator-empty] (axis cs:2.49000E+02,1.00000E-01) -- (axis cs:2.50000E+02,1.00000E-01) -- (axis cs:2.50000E+02,-1.00000E-01) -- (axis cs:2.49000E+02,-1.00000E-01) -- cycle ; 
\draw[Equator-empty] (axis cs:2.51000E+02,1.00000E-01) -- (axis cs:2.52000E+02,1.00000E-01) -- (axis cs:2.52000E+02,-1.00000E-01) -- (axis cs:2.51000E+02,-1.00000E-01) -- cycle ; 
\draw[Equator-empty] (axis cs:2.53000E+02,1.00000E-01) -- (axis cs:2.54000E+02,1.00000E-01) -- (axis cs:2.54000E+02,-1.00000E-01) -- (axis cs:2.53000E+02,-1.00000E-01) -- cycle ; 
\draw[Equator-empty] (axis cs:2.55000E+02,1.00000E-01) -- (axis cs:2.56000E+02,1.00000E-01) -- (axis cs:2.56000E+02,-1.00000E-01) -- (axis cs:2.55000E+02,-1.00000E-01) -- cycle ; 
\draw[Equator-empty] (axis cs:2.57000E+02,1.00000E-01) -- (axis cs:2.58000E+02,1.00000E-01) -- (axis cs:2.58000E+02,-1.00000E-01) -- (axis cs:2.57000E+02,-1.00000E-01) -- cycle ; 
\draw[Equator-empty] (axis cs:2.59000E+02,1.00000E-01) -- (axis cs:2.60000E+02,1.00000E-01) -- (axis cs:2.60000E+02,-1.00000E-01) -- (axis cs:2.59000E+02,-1.00000E-01) -- cycle ; 
\draw[Equator-empty] (axis cs:2.61000E+02,1.00000E-01) -- (axis cs:2.62000E+02,1.00000E-01) -- (axis cs:2.62000E+02,-1.00000E-01) -- (axis cs:2.61000E+02,-1.00000E-01) -- cycle ; 
\draw[Equator-empty] (axis cs:2.63000E+02,1.00000E-01) -- (axis cs:2.64000E+02,1.00000E-01) -- (axis cs:2.64000E+02,-1.00000E-01) -- (axis cs:2.63000E+02,-1.00000E-01) -- cycle ; 
\draw[Equator-empty] (axis cs:2.65000E+02,1.00000E-01) -- (axis cs:2.66000E+02,1.00000E-01) -- (axis cs:2.66000E+02,-1.00000E-01) -- (axis cs:2.65000E+02,-1.00000E-01) -- cycle ; 
\draw[Equator-empty] (axis cs:2.67000E+02,1.00000E-01) -- (axis cs:2.68000E+02,1.00000E-01) -- (axis cs:2.68000E+02,-1.00000E-01) -- (axis cs:2.67000E+02,-1.00000E-01) -- cycle ; 
\draw[Equator-empty] (axis cs:2.69000E+02,1.00000E-01) -- (axis cs:2.70000E+02,1.00000E-01) -- (axis cs:2.70000E+02,-1.00000E-01) -- (axis cs:2.69000E+02,-1.00000E-01) -- cycle ; 
\draw[Equator-empty] (axis cs:2.71000E+02,1.00000E-01) -- (axis cs:2.72000E+02,1.00000E-01) -- (axis cs:2.72000E+02,-1.00000E-01) -- (axis cs:2.71000E+02,-1.00000E-01) -- cycle ; 
\draw[Equator-empty] (axis cs:2.73000E+02,1.00000E-01) -- (axis cs:2.74000E+02,1.00000E-01) -- (axis cs:2.74000E+02,-1.00000E-01) -- (axis cs:2.73000E+02,-1.00000E-01) -- cycle ; 
\draw[Equator-empty] (axis cs:2.75000E+02,1.00000E-01) -- (axis cs:2.76000E+02,1.00000E-01) -- (axis cs:2.76000E+02,-1.00000E-01) -- (axis cs:2.75000E+02,-1.00000E-01) -- cycle ; 
\draw[Equator-empty] (axis cs:2.77000E+02,1.00000E-01) -- (axis cs:2.78000E+02,1.00000E-01) -- (axis cs:2.78000E+02,-1.00000E-01) -- (axis cs:2.77000E+02,-1.00000E-01) -- cycle ; 
\draw[Equator-empty] (axis cs:2.79000E+02,1.00000E-01) -- (axis cs:2.80000E+02,1.00000E-01) -- (axis cs:2.80000E+02,-1.00000E-01) -- (axis cs:2.79000E+02,-1.00000E-01) -- cycle ; 
\draw[Equator-empty] (axis cs:2.81000E+02,1.00000E-01) -- (axis cs:2.82000E+02,1.00000E-01) -- (axis cs:2.82000E+02,-1.00000E-01) -- (axis cs:2.81000E+02,-1.00000E-01) -- cycle ; 
\draw[Equator-empty] (axis cs:2.83000E+02,1.00000E-01) -- (axis cs:2.84000E+02,1.00000E-01) -- (axis cs:2.84000E+02,-1.00000E-01) -- (axis cs:2.83000E+02,-1.00000E-01) -- cycle ; 
\draw[Equator-empty] (axis cs:2.85000E+02,1.00000E-01) -- (axis cs:2.86000E+02,1.00000E-01) -- (axis cs:2.86000E+02,-1.00000E-01) -- (axis cs:2.85000E+02,-1.00000E-01) -- cycle ; 
\draw[Equator-empty] (axis cs:2.87000E+02,1.00000E-01) -- (axis cs:2.88000E+02,1.00000E-01) -- (axis cs:2.88000E+02,-1.00000E-01) -- (axis cs:2.87000E+02,-1.00000E-01) -- cycle ; 
\draw[Equator-empty] (axis cs:2.89000E+02,1.00000E-01) -- (axis cs:2.90000E+02,1.00000E-01) -- (axis cs:2.90000E+02,-1.00000E-01) -- (axis cs:2.89000E+02,-1.00000E-01) -- cycle ; 
\draw[Equator-empty] (axis cs:2.91000E+02,1.00000E-01) -- (axis cs:2.92000E+02,1.00000E-01) -- (axis cs:2.92000E+02,-1.00000E-01) -- (axis cs:2.91000E+02,-1.00000E-01) -- cycle ; 
\draw[Equator-empty] (axis cs:2.93000E+02,1.00000E-01) -- (axis cs:2.94000E+02,1.00000E-01) -- (axis cs:2.94000E+02,-1.00000E-01) -- (axis cs:2.93000E+02,-1.00000E-01) -- cycle ; 
\draw[Equator-empty] (axis cs:2.95000E+02,1.00000E-01) -- (axis cs:2.96000E+02,1.00000E-01) -- (axis cs:2.96000E+02,-1.00000E-01) -- (axis cs:2.95000E+02,-1.00000E-01) -- cycle ; 
\draw[Equator-empty] (axis cs:2.97000E+02,1.00000E-01) -- (axis cs:2.98000E+02,1.00000E-01) -- (axis cs:2.98000E+02,-1.00000E-01) -- (axis cs:2.97000E+02,-1.00000E-01) -- cycle ; 
\draw[Equator-empty] (axis cs:2.99000E+02,1.00000E-01) -- (axis cs:3.00000E+02,1.00000E-01) -- (axis cs:3.00000E+02,-1.00000E-01) -- (axis cs:2.99000E+02,-1.00000E-01) -- cycle ; 
\draw[Equator-empty] (axis cs:3.01000E+02,1.00000E-01) -- (axis cs:3.02000E+02,1.00000E-01) -- (axis cs:3.02000E+02,-1.00000E-01) -- (axis cs:3.01000E+02,-1.00000E-01) -- cycle ; 
\draw[Equator-empty] (axis cs:3.03000E+02,1.00000E-01) -- (axis cs:3.04000E+02,1.00000E-01) -- (axis cs:3.04000E+02,-1.00000E-01) -- (axis cs:3.03000E+02,-1.00000E-01) -- cycle ; 
\draw[Equator-empty] (axis cs:3.05000E+02,1.00000E-01) -- (axis cs:3.06000E+02,1.00000E-01) -- (axis cs:3.06000E+02,-1.00000E-01) -- (axis cs:3.05000E+02,-1.00000E-01) -- cycle ; 
\draw[Equator-empty] (axis cs:3.07000E+02,1.00000E-01) -- (axis cs:3.08000E+02,1.00000E-01) -- (axis cs:3.08000E+02,-1.00000E-01) -- (axis cs:3.07000E+02,-1.00000E-01) -- cycle ; 
\draw[Equator-empty] (axis cs:3.09000E+02,1.00000E-01) -- (axis cs:3.10000E+02,1.00000E-01) -- (axis cs:3.10000E+02,-1.00000E-01) -- (axis cs:3.09000E+02,-1.00000E-01) -- cycle ; 
\draw[Equator-empty] (axis cs:3.11000E+02,1.00000E-01) -- (axis cs:3.12000E+02,1.00000E-01) -- (axis cs:3.12000E+02,-1.00000E-01) -- (axis cs:3.11000E+02,-1.00000E-01) -- cycle ; 
\draw[Equator-empty] (axis cs:3.13000E+02,1.00000E-01) -- (axis cs:3.14000E+02,1.00000E-01) -- (axis cs:3.14000E+02,-1.00000E-01) -- (axis cs:3.13000E+02,-1.00000E-01) -- cycle ; 
\draw[Equator-empty] (axis cs:3.15000E+02,1.00000E-01) -- (axis cs:3.16000E+02,1.00000E-01) -- (axis cs:3.16000E+02,-1.00000E-01) -- (axis cs:3.15000E+02,-1.00000E-01) -- cycle ; 
\draw[Equator-empty] (axis cs:3.17000E+02,1.00000E-01) -- (axis cs:3.18000E+02,1.00000E-01) -- (axis cs:3.18000E+02,-1.00000E-01) -- (axis cs:3.17000E+02,-1.00000E-01) -- cycle ; 
\draw[Equator-empty] (axis cs:3.19000E+02,1.00000E-01) -- (axis cs:3.20000E+02,1.00000E-01) -- (axis cs:3.20000E+02,-1.00000E-01) -- (axis cs:3.19000E+02,-1.00000E-01) -- cycle ; 
\draw[Equator-empty] (axis cs:3.21000E+02,1.00000E-01) -- (axis cs:3.22000E+02,1.00000E-01) -- (axis cs:3.22000E+02,-1.00000E-01) -- (axis cs:3.21000E+02,-1.00000E-01) -- cycle ; 
\draw[Equator-empty] (axis cs:3.23000E+02,1.00000E-01) -- (axis cs:3.24000E+02,1.00000E-01) -- (axis cs:3.24000E+02,-1.00000E-01) -- (axis cs:3.23000E+02,-1.00000E-01) -- cycle ; 
\draw[Equator-empty] (axis cs:3.25000E+02,1.00000E-01) -- (axis cs:3.26000E+02,1.00000E-01) -- (axis cs:3.26000E+02,-1.00000E-01) -- (axis cs:3.25000E+02,-1.00000E-01) -- cycle ; 
\draw[Equator-empty] (axis cs:3.27000E+02,1.00000E-01) -- (axis cs:3.28000E+02,1.00000E-01) -- (axis cs:3.28000E+02,-1.00000E-01) -- (axis cs:3.27000E+02,-1.00000E-01) -- cycle ; 
\draw[Equator-empty] (axis cs:3.29000E+02,1.00000E-01) -- (axis cs:3.30000E+02,1.00000E-01) -- (axis cs:3.30000E+02,-1.00000E-01) -- (axis cs:3.29000E+02,-1.00000E-01) -- cycle ; 
\draw[Equator-empty] (axis cs:3.31000E+02,1.00000E-01) -- (axis cs:3.32000E+02,1.00000E-01) -- (axis cs:3.32000E+02,-1.00000E-01) -- (axis cs:3.31000E+02,-1.00000E-01) -- cycle ; 
\draw[Equator-empty] (axis cs:3.33000E+02,1.00000E-01) -- (axis cs:3.34000E+02,1.00000E-01) -- (axis cs:3.34000E+02,-1.00000E-01) -- (axis cs:3.33000E+02,-1.00000E-01) -- cycle ; 
\draw[Equator-empty] (axis cs:3.35000E+02,1.00000E-01) -- (axis cs:3.36000E+02,1.00000E-01) -- (axis cs:3.36000E+02,-1.00000E-01) -- (axis cs:3.35000E+02,-1.00000E-01) -- cycle ; 
\draw[Equator-empty] (axis cs:3.37000E+02,1.00000E-01) -- (axis cs:3.38000E+02,1.00000E-01) -- (axis cs:3.38000E+02,-1.00000E-01) -- (axis cs:3.37000E+02,-1.00000E-01) -- cycle ; 
\draw[Equator-empty] (axis cs:3.39000E+02,1.00000E-01) -- (axis cs:3.40000E+02,1.00000E-01) -- (axis cs:3.40000E+02,-1.00000E-01) -- (axis cs:3.39000E+02,-1.00000E-01) -- cycle ; 
\draw[Equator-empty] (axis cs:3.41000E+02,1.00000E-01) -- (axis cs:3.42000E+02,1.00000E-01) -- (axis cs:3.42000E+02,-1.00000E-01) -- (axis cs:3.41000E+02,-1.00000E-01) -- cycle ; 
\draw[Equator-empty] (axis cs:3.43000E+02,1.00000E-01) -- (axis cs:3.44000E+02,1.00000E-01) -- (axis cs:3.44000E+02,-1.00000E-01) -- (axis cs:3.43000E+02,-1.00000E-01) -- cycle ; 
\draw[Equator-empty] (axis cs:3.45000E+02,1.00000E-01) -- (axis cs:3.46000E+02,1.00000E-01) -- (axis cs:3.46000E+02,-1.00000E-01) -- (axis cs:3.45000E+02,-1.00000E-01) -- cycle ; 
\draw[Equator-empty] (axis cs:3.47000E+02,1.00000E-01) -- (axis cs:3.48000E+02,1.00000E-01) -- (axis cs:3.48000E+02,-1.00000E-01) -- (axis cs:3.47000E+02,-1.00000E-01) -- cycle ; 
\draw[Equator-empty] (axis cs:3.49000E+02,1.00000E-01) -- (axis cs:3.50000E+02,1.00000E-01) -- (axis cs:3.50000E+02,-1.00000E-01) -- (axis cs:3.49000E+02,-1.00000E-01) -- cycle ; 
\draw[Equator-empty] (axis cs:3.51000E+02,1.00000E-01) -- (axis cs:3.52000E+02,1.00000E-01) -- (axis cs:3.52000E+02,-1.00000E-01) -- (axis cs:3.51000E+02,-1.00000E-01) -- cycle ; 
\draw[Equator-empty] (axis cs:3.53000E+02,1.00000E-01) -- (axis cs:3.54000E+02,1.00000E-01) -- (axis cs:3.54000E+02,-1.00000E-01) -- (axis cs:3.53000E+02,-1.00000E-01) -- cycle ; 
\draw[Equator-empty] (axis cs:3.55000E+02,1.00000E-01) -- (axis cs:3.56000E+02,1.00000E-01) -- (axis cs:3.56000E+02,-1.00000E-01) -- (axis cs:3.55000E+02,-1.00000E-01) -- cycle ; 
\draw[Equator-empty] (axis cs:3.57000E+02,1.00000E-01) -- (axis cs:3.58000E+02,1.00000E-01) -- (axis cs:3.58000E+02,-1.00000E-01) -- (axis cs:3.57000E+02,-1.00000E-01) -- cycle ; 
\draw[Equator-empty] (axis cs:3.59000E+02,1.00000E-01) -- (axis cs:3.60000E+02,1.00000E-01) -- (axis cs:3.60000E+02,-1.00000E-01) -- (axis cs:3.59000E+02,-1.00000E-01) -- cycle ; 

\draw[Equator-full] (axis cs:3.60000E+02,1.00000E-01) -- (axis cs:3.61000E+02,1.00000E-01) -- (axis cs:3.61000E+02,-1.00000E-01) -- (axis cs:3.60000E+02,-1.00000E-01) -- cycle ; 
\draw[Equator-full] (axis cs:3.62000E+02,1.00000E-01) -- (axis cs:3.63000E+02,1.00000E-01) -- (axis cs:3.63000E+02,-1.00000E-01) -- (axis cs:3.62000E+02,-1.00000E-01) -- cycle ; 
\draw[Equator-full] (axis cs:3.64000E+02,1.00000E-01) -- (axis cs:3.65000E+02,1.00000E-01) -- (axis cs:3.65000E+02,-1.00000E-01) -- (axis cs:3.64000E+02,-1.00000E-01) -- cycle ; 
\draw[Equator-full] (axis cs:3.66000E+02,1.00000E-01) -- (axis cs:3.67000E+02,1.00000E-01) -- (axis cs:3.67000E+02,-1.00000E-01) -- (axis cs:3.66000E+02,-1.00000E-01) -- cycle ; 
\draw[Equator-full] (axis cs:3.68000E+02,1.00000E-01) -- (axis cs:3.69000E+02,1.00000E-01) -- (axis cs:3.69000E+02,-1.00000E-01) -- (axis cs:3.68000E+02,-1.00000E-01) -- cycle ; 
\draw[Equator-full] (axis cs:3.70000E+02,1.00000E-01) -- (axis cs:3.71000E+02,1.00000E-01) -- (axis cs:3.71000E+02,-1.00000E-01) -- (axis cs:3.70000E+02,-1.00000E-01) -- cycle ; 
\draw[Equator-full] (axis cs:3.72000E+02,1.00000E-01) -- (axis cs:3.73000E+02,1.00000E-01) -- (axis cs:3.73000E+02,-1.00000E-01) -- (axis cs:3.72000E+02,-1.00000E-01) -- cycle ; 
\draw[Equator-full] (axis cs:3.74000E+02,1.00000E-01) -- (axis cs:3.75000E+02,1.00000E-01) -- (axis cs:3.75000E+02,-1.00000E-01) -- (axis cs:3.74000E+02,-1.00000E-01) -- cycle ; 
\draw[Equator-full] (axis cs:3.76000E+02,1.00000E-01) -- (axis cs:3.77000E+02,1.00000E-01) -- (axis cs:3.77000E+02,-1.00000E-01) -- (axis cs:3.76000E+02,-1.00000E-01) -- cycle ; 
\draw[Equator-full] (axis cs:3.78000E+02,1.00000E-01) -- (axis cs:3.79000E+02,1.00000E-01) -- (axis cs:3.79000E+02,-1.00000E-01) -- (axis cs:3.78000E+02,-1.00000E-01) -- cycle ; 

\draw[Equator-empty] (axis cs:3.61000E+02,1.00000E-01) -- (axis cs:3.62000E+02,1.00000E-01) -- (axis cs:3.62000E+02,-1.00000E-01) -- (axis cs:3.61000E+02,-1.00000E-01) -- cycle ; 
\draw[Equator-empty] (axis cs:3.63000E+02,1.00000E-01) -- (axis cs:3.64000E+02,1.00000E-01) -- (axis cs:3.64000E+02,-1.00000E-01) -- (axis cs:3.63000E+02,-1.00000E-01) -- cycle ; 
\draw[Equator-empty] (axis cs:3.65000E+02,1.00000E-01) -- (axis cs:3.66000E+02,1.00000E-01) -- (axis cs:3.66000E+02,-1.00000E-01) -- (axis cs:3.65000E+02,-1.00000E-01) -- cycle ; 
\draw[Equator-empty] (axis cs:3.67000E+02,1.00000E-01) -- (axis cs:3.68000E+02,1.00000E-01) -- (axis cs:3.68000E+02,-1.00000E-01) -- (axis cs:3.67000E+02,-1.00000E-01) -- cycle ; 
\draw[Equator-empty] (axis cs:3.69000E+02,1.00000E-01) -- (axis cs:3.70000E+02,1.00000E-01) -- (axis cs:3.70000E+02,-1.00000E-01) -- (axis cs:3.69000E+02,-1.00000E-01) -- cycle ; 
\draw[Equator-empty] (axis cs:3.71000E+02,1.00000E-01) -- (axis cs:3.72000E+02,1.00000E-01) -- (axis cs:3.72000E+02,-1.00000E-01) -- (axis cs:3.71000E+02,-1.00000E-01) -- cycle ; 
\draw[Equator-empty] (axis cs:3.73000E+02,1.00000E-01) -- (axis cs:3.74000E+02,1.00000E-01) -- (axis cs:3.74000E+02,-1.00000E-01) -- (axis cs:3.73000E+02,-1.00000E-01) -- cycle ; 
\draw[Equator-empty] (axis cs:3.75000E+02,1.00000E-01) -- (axis cs:3.76000E+02,1.00000E-01) -- (axis cs:3.76000E+02,-1.00000E-01) -- (axis cs:3.75000E+02,-1.00000E-01) -- cycle ; 
\draw[Equator-empty] (axis cs:3.77000E+02,1.00000E-01) -- (axis cs:3.78000E+02,1.00000E-01) -- (axis cs:3.78000E+02,-1.00000E-01) -- (axis cs:3.77000E+02,-1.00000E-01) -- cycle ; 
\draw[Equator-empty] (axis cs:3.79000E+02,1.00000E-01) -- (axis cs:3.80000E+02,1.00000E-01) -- (axis cs:3.80000E+02,-1.00000E-01) -- (axis cs:3.79000E+02,-1.00000E-01) -- cycle ; 


\draw [Equator-empty] (axis cs:0.00000E+00,4.00000E-01) -- (axis cs:0.00000E+00,-4.00000E-01)   ;
\node[pin={[pin distance=-0.4\onedegree,Equator-label]90:{0$^\circ$}}] at (axis cs:0.00000E+00,4.00000E-01) {} ;
\draw [Equator-empty] (axis cs:1.00000E+01,4.00000E-01) -- (axis cs:1.00000E+01,-4.00000E-01)   ;
\node[pin={[pin distance=-0.4\onedegree,Equator-label]90:{10$^\circ$}}] at (axis cs:1.00000E+01,4.00000E-01) {} ;
\draw [Equator-empty] (axis cs:2.00000E+01,4.00000E-01) -- (axis cs:2.00000E+01,-4.00000E-01)   ;
\node[pin={[pin distance=-0.4\onedegree,Equator-label]90:{20$^\circ$}}] at (axis cs:2.00000E+01,4.00000E-01) {} ;
\draw [Equator-empty] (axis cs:3.00000E+01,4.00000E-01) -- (axis cs:3.00000E+01,-4.00000E-01)   ;
\node[pin={[pin distance=-0.4\onedegree,Equator-label]90:{30$^\circ$}}] at (axis cs:3.00000E+01,4.00000E-01) {} ;
\draw [Equator-empty] (axis cs:4.00000E+01,4.00000E-01) -- (axis cs:4.00000E+01,-4.00000E-01)   ;
\node[pin={[pin distance=-0.4\onedegree,Equator-label]90:{40$^\circ$}}] at (axis cs:4.00000E+01,4.00000E-01) {} ;
\draw [Equator-empty] (axis cs:5.00000E+01,4.00000E-01) -- (axis cs:5.00000E+01,-4.00000E-01)   ;
\node[pin={[pin distance=-0.4\onedegree,Equator-label]90:{50$^\circ$}}] at (axis cs:5.00000E+01,4.00000E-01) {} ;
\draw [Equator-empty] (axis cs:6.00000E+01,4.00000E-01) -- (axis cs:6.00000E+01,-4.00000E-01)   ;
\node[pin={[pin distance=-0.4\onedegree,Equator-label]90:{60$^\circ$}}] at (axis cs:6.00000E+01,4.00000E-01) {} ;
\draw [Equator-empty] (axis cs:7.00000E+01,4.00000E-01) -- (axis cs:7.00000E+01,-4.00000E-01)   ;
\node[pin={[pin distance=-0.4\onedegree,Equator-label]90:{70$^\circ$}}] at (axis cs:7.00000E+01,4.00000E-01) {} ;
\draw [Equator-empty] (axis cs:8.00000E+01,4.00000E-01) -- (axis cs:8.00000E+01,-4.00000E-01)   ;
\node[pin={[pin distance=-0.4\onedegree,Equator-label]90:{80$^\circ$}}] at (axis cs:8.00000E+01,4.00000E-01) {} ;
\draw [Equator-empty] (axis cs:9.00000E+01,4.00000E-01) -- (axis cs:9.00000E+01,-4.00000E-01)   ;
\node[pin={[pin distance=-0.4\onedegree,Equator-label]90:{90$^\circ$}}] at (axis cs:9.00000E+01,4.00000E-01) {} ;
\draw [Equator-empty] (axis cs:1.00000E+02,4.00000E-01) -- (axis cs:1.00000E+02,-4.00000E-01)   ;
\node[pin={[pin distance=-0.4\onedegree,Equator-label]090:{100$^\circ$}}] at (axis cs:1.00000E+02,4.00000E-01) {} ;
\draw [Equator-empty] (axis cs:1.10000E+02,4.00000E-01) -- (axis cs:1.10000E+02,-4.00000E-01)   ;
\node[pin={[pin distance=-0.4\onedegree,Equator-label]090:{110$^\circ$}}] at (axis cs:1.10000E+02,4.00000E-01) {} ;
\draw [Equator-empty] (axis cs:1.20000E+02,4.00000E-01) -- (axis cs:1.20000E+02,-4.00000E-01)   ;
\node[pin={[pin distance=-0.4\onedegree,Equator-label]090:{120$^\circ$}}] at (axis cs:1.20000E+02,4.00000E-01) {} ;
\draw [Equator-empty] (axis cs:1.30000E+02,4.00000E-01) -- (axis cs:1.30000E+02,-4.00000E-01)   ;
\node[pin={[pin distance=-0.4\onedegree,Equator-label]090:{130$^\circ$}}] at (axis cs:1.30000E+02,4.00000E-01) {} ;
\draw [Equator-empty] (axis cs:1.40000E+02,4.00000E-01) -- (axis cs:1.40000E+02,-4.00000E-01)   ;
\node[pin={[pin distance=-0.4\onedegree,Equator-label]090:{140$^\circ$}}] at (axis cs:1.40000E+02,4.00000E-01) {} ;
\draw [Equator-empty] (axis cs:1.50000E+02,4.00000E-01) -- (axis cs:1.50000E+02,-4.00000E-01)   ;
\node[pin={[pin distance=-0.4\onedegree,Equator-label]090:{150$^\circ$}}] at (axis cs:1.50000E+02,4.00000E-01) {} ;
\draw [Equator-empty] (axis cs:1.60000E+02,4.00000E-01) -- (axis cs:1.60000E+02,-4.00000E-01)   ;
\node[pin={[pin distance=-0.4\onedegree,Equator-label]090:{160$^\circ$}}] at (axis cs:1.60000E+02,4.00000E-01) {} ;
\draw [Equator-empty] (axis cs:1.70000E+02,4.00000E-01) -- (axis cs:1.70000E+02,-4.00000E-01)   ;
\node[pin={[pin distance=-0.4\onedegree,Equator-label]090:{170$^\circ$}}] at (axis cs:1.70000E+02,4.00000E-01) {} ;
\draw [Equator-empty] (axis cs:1.80000E+02,4.00000E-01) -- (axis cs:1.80000E+02,-4.00000E-01)   ;
\node[pin={[pin distance=-0.4\onedegree,Equator-label]090:{180$^\circ$}}] at (axis cs:1.80000E+02,4.00000E-01) {} ;
\draw [Equator-empty] (axis cs:1.90000E+02,4.00000E-01) -- (axis cs:1.90000E+02,-4.00000E-01)   ;
\node[pin={[pin distance=-0.4\onedegree,Equator-label]090:{190$^\circ$}}] at (axis cs:1.90000E+02,4.00000E-01) {} ;
\draw [Equator-empty] (axis cs:2.00000E+02,4.00000E-01) -- (axis cs:2.00000E+02,-4.00000E-01)   ;
\node[pin={[pin distance=-0.4\onedegree,Equator-label]090:{200$^\circ$}}] at (axis cs:2.00000E+02,4.00000E-01) {} ;
\draw [Equator-empty] (axis cs:2.10000E+02,4.00000E-01) -- (axis cs:2.10000E+02,-4.00000E-01)   ;
\node[pin={[pin distance=-0.4\onedegree,Equator-label]090:{210$^\circ$}}] at (axis cs:2.10000E+02,4.00000E-01) {} ;
\draw [Equator-empty] (axis cs:2.20000E+02,4.00000E-01) -- (axis cs:2.20000E+02,-4.00000E-01)   ;
\node[pin={[pin distance=-0.4\onedegree,Equator-label]090:{220$^\circ$}}] at (axis cs:2.20000E+02,4.00000E-01) {} ;
\draw [Equator-empty] (axis cs:2.30000E+02,4.00000E-01) -- (axis cs:2.30000E+02,-4.00000E-01)   ;
\node[pin={[pin distance=-0.4\onedegree,Equator-label]090:{230$^\circ$}}] at (axis cs:2.30000E+02,4.00000E-01) {} ;
\draw [Equator-empty] (axis cs:2.40000E+02,4.00000E-01) -- (axis cs:2.40000E+02,-4.00000E-01)   ;
\node[pin={[pin distance=-0.4\onedegree,Equator-label]090:{240$^\circ$}}] at (axis cs:2.40000E+02,4.00000E-01) {} ;
\draw [Equator-empty] (axis cs:2.50000E+02,4.00000E-01) -- (axis cs:2.50000E+02,-4.00000E-01)   ;
\node[pin={[pin distance=-0.4\onedegree,Equator-label]090:{250$^\circ$}}] at (axis cs:2.50000E+02,4.00000E-01) {} ;
\draw [Equator-empty] (axis cs:2.60000E+02,4.00000E-01) -- (axis cs:2.60000E+02,-4.00000E-01)   ;
\node[pin={[pin distance=-0.4\onedegree,Equator-label]090:{260$^\circ$}}] at (axis cs:2.60000E+02,4.00000E-01) {} ;
\draw [Equator-empty] (axis cs:2.70000E+02,4.00000E-01) -- (axis cs:2.70000E+02,-4.00000E-01)   ;
\node[pin={[pin distance=-0.4\onedegree,Equator-label]090:{270$^\circ$}}] at (axis cs:2.70000E+02,4.00000E-01) {} ;
\draw [Equator-empty] (axis cs:2.80000E+02,4.00000E-01) -- (axis cs:2.80000E+02,-4.00000E-01)   ;
\node[pin={[pin distance=-0.4\onedegree,Equator-label]090:{280$^\circ$}}] at (axis cs:2.80000E+02,4.00000E-01) {} ;
\draw [Equator-empty] (axis cs:2.90000E+02,4.00000E-01) -- (axis cs:2.90000E+02,-4.00000E-01)   ;
\node[pin={[pin distance=-0.4\onedegree,Equator-label]090:{290$^\circ$}}] at (axis cs:2.90000E+02,4.00000E-01) {} ;
\draw [Equator-empty] (axis cs:3.00000E+02,4.00000E-01) -- (axis cs:3.00000E+02,-4.00000E-01)   ;
\node[pin={[pin distance=-0.4\onedegree,Equator-label]090:{300$^\circ$}}] at (axis cs:3.00000E+02,4.00000E-01) {} ;
\draw [Equator-empty] (axis cs:3.10000E+02,4.00000E-01) -- (axis cs:3.10000E+02,-4.00000E-01)   ;
\node[pin={[pin distance=-0.4\onedegree,Equator-label]090:{310$^\circ$}}] at (axis cs:3.10000E+02,4.00000E-01) {} ;
\draw [Equator-empty] (axis cs:3.20000E+02,4.00000E-01) -- (axis cs:3.20000E+02,-4.00000E-01)   ;
\node[pin={[pin distance=-0.4\onedegree,Equator-label]090:{320$^\circ$}}] at (axis cs:3.20000E+02,4.00000E-01) {} ;
\draw [Equator-empty] (axis cs:3.30000E+02,4.00000E-01) -- (axis cs:3.30000E+02,-4.00000E-01)   ;
\node[pin={[pin distance=-0.4\onedegree,Equator-label]090:{330$^\circ$}}] at (axis cs:3.30000E+02,4.00000E-01) {} ;
\draw [Equator-empty] (axis cs:3.40000E+02,4.00000E-01) -- (axis cs:3.40000E+02,-4.00000E-01)   ;
\node[pin={[pin distance=-0.4\onedegree,Equator-label]090:{340$^\circ$}}] at (axis cs:3.40000E+02,4.00000E-01) {} ;
\draw [Equator-empty] (axis cs:3.50000E+02,4.00000E-01) -- (axis cs:3.50000E+02,-4.00000E-01)   ;
\node[pin={[pin distance=-0.4\onedegree,Equator-label]090:{350$^\circ$}}] at (axis cs:3.50000E+02,4.00000E-01) {} ;

\draw [Equator-empty] (axis cs:3.60000E+02,4.00000E-01) -- (axis cs:3.60000E+02,-4.00000E-01)   ;
\node[pin={[pin distance=-0.4\onedegree,Equator-label]090:{0$^\circ$}}] at (axis cs:3.60000E+02,4.00000E-01) {} ;
\draw [Equator-empty] (axis cs:3.70000E+02,4.00000E-01) -- (axis cs:3.70000E+02,-4.00000E-01)   ;
\node[pin={[pin distance=-0.4\onedegree,Equator-label]090:{10$^\circ$}}] at (axis cs:3.70000E+02,4.00000E-01) {} ;




\end{axis}


% Coordinate axes 


% Left and bottom axis with gridlines
  \begin{axis}[at=(base.center),anchor=center,RA_in_hours,color=cAxes] \end{axis}
% Right and top axis, labelling in RA hours (offset)
  \begin{axis}[at=(base.center),anchor=center,yticklabel pos=right,xticklabel pos=right,RA_in_hours
  ,xticklabel style={yshift=2.5ex, anchor=south},color=cAxes]\end{axis}
% Top axis labelling in RA degree (small offset)
  \begin{axis}[at=(base.center),anchor=center,axis y line=none,xticklabel pos=top,RA_in_deg
     ,xticklabel style={font=\footnotesize},xticklabel style={yshift=0.5ex, anchor=south} ,color=cAxes]
  \end{axis}

% Coordinate grid

\begin{axis}[at=(base.center),anchor=center,coordinategrid,RA_in_deg,
    separate axis lines,
    y axis line style= { draw opacity=0 },
    x axis line style= { draw opacity=0 },
    ,xticklabels={},yticklabels={}
] \end{axis}


% Ornamental frame


  \begin{axis}[name=frame,at={($(base.center)+(0cm,+.7\onedegree)$)},anchor=center,width=11.75\tendegree,height=8.53\tendegree,ticks=none
  ,yticklabels={},xticklabels={},axis line style={line width=1pt},color=cFrame] \end{axis}

  \begin{axis}[name=frame2,at={($(frame.center)+(0cm,+0cm)$)},anchor=center,width=11.85\tendegree,height=8.63\tendegree,ticks=none
  ,yticklabels={},xticklabels={},axis line style={line width=3pt},color=cFrame] \end{axis}
  

% Symbol legend


 
 \begin{axis}[name=legend,at=(frame2.south),anchor=north,axis y line=none,axis x
   line=none,xmin=-55,xmax=55,height=0.5\tendegree   ,ymin=-1.5,ymax=1.5,  ,y dir=normal,x dir=reverse, color=cAxes]

\node at (axis cs:54,+1.0) {\small\it 1};                  \node at (axis cs:54,0.3)  {\tikz{\pgfuseplotmark{m1b};}};           
\node at (axis cs:52,+1.0) {\small\it 1{\nicefrac{1}{3}}}; \node at (axis cs:52,0.3)  {\tikz{\pgfuseplotmark{m1c};}};           
\node at (axis cs:53,-0.5)  {\small First};                  

\node at (axis cs:48,+1.0) {\small\it 1{\nicefrac{2}{3}}}; \node at (axis cs:48,0.3)  {\tikz{\pgfuseplotmark{m2a};}};           
\node at (axis cs:46,+1.0) {\small\it 2};                  \node at (axis cs:46,0.3)  {\tikz{\pgfuseplotmark{m2b};}};           
\node at (axis cs:44,+1.0) {\small\it 2{\nicefrac{1}{3}}}; \node at (axis cs:44,0.3)  {\tikz{\pgfuseplotmark{m2c};}};           
\node at (axis cs:46,-0.5)  {\small Second};

\node at (axis cs:40,+1.0) {\small\it 2{\nicefrac{2}{3}}}; \node at (axis cs:40,0.3)  {\tikz{\pgfuseplotmark{m3a};}};           
\node at (axis cs:38,+1.0) {\small\it 3};                  \node at (axis cs:38,0.3)  {\tikz{\pgfuseplotmark{m3b};}};           
\node at (axis cs:36,+1.0) {\small\it 3{\nicefrac{1}{3}}}; \node at (axis cs:36,0.3)  {\tikz{\pgfuseplotmark{m3c};}};           
\node at (axis cs:38,-0.5)  {\small Third};

\node at (axis cs:32,+1.0) {\small\it 3{\nicefrac{2}{3}}}; \node at (axis cs:32,0.3)  {\tikz{\pgfuseplotmark{m4a};}};           
\node at (axis cs:30,+1.0) {\small\it 4};                  \node at (axis cs:30,0.3)  {\tikz{\pgfuseplotmark{m4b};}};           
\node at (axis cs:28,+1.0) {\small\it 4{\nicefrac{1}{3}}}; \node at (axis cs:28,0.3)  {\tikz{\pgfuseplotmark{m4c};}};           
\node at (axis cs:30,-0.5)  {\small Fourth};                                                                      
                                                                                                                  
\node at (axis cs:24,+1.0) {\small\it 4{\nicefrac{2}{3}}}; \node at (axis cs:24,0.3)  {\tikz{\pgfuseplotmark{m5a};}};           
\node at (axis cs:22,+1.0) {\small\it 5};                  \node at (axis cs:22,0.3)  {\tikz{\pgfuseplotmark{m5b};}};           
\node at (axis cs:20,+1.0) {\small\it 5{\nicefrac{1}{3}}}; \node at (axis cs:20,0.3)  {\tikz{\pgfuseplotmark{m5c};}};           
\node at (axis cs:22,-0.5)  {\small Fifth};

\node at (axis cs:16,+1.0) {\small\it 5{\nicefrac{2}{3}}}; \node at (axis cs:16,0.3)  {\tikz{\pgfuseplotmark{m6a};}};           
\node at (axis cs:14,+1.0) {\small\it 6};                  \node at (axis cs:14,0.3)  {\tikz{\pgfuseplotmark{m6b};}};           
\node at (axis cs:12,+1.0) {\small\it 6{\nicefrac{1}{3}}}; \node at (axis cs:12,0.3)  {\tikz{\pgfuseplotmark{m6c};}};           
\node at (axis cs:14,-0.5)  {\small Sixth};                                                                       
                                                                                                                  
                                                                                                                 
\node at  (axis cs:6,+1.0)   {\small\it Variable};  \node at  (axis cs:6,+0.3)       {\tikz{\pgfuseplotmark{m2av};}};           
\node at  (axis cs:2,+.992)    {\small\it Binary};   \node at  (axis cs:2,+0.3)      {\tikz{\pgfuseplotmark{m2ab};}};           
\node at  (axis cs:-2,+1.0)   {\small\it Var. \& Bin.};  \node at  (axis cs:-2,+0.3) {\tikz{\pgfuseplotmark{m2avb};}};           


\node at  (axis cs:-17,+.992)   {\small\it Planetary};  \node at  (axis cs:-18,+0.3)  {\tikz{\pgfuseplotmark{PN};}};           
\node at  (axis cs:-22,+1.0)   {\small\it Emission};   \node at  (axis cs:-22,+0.3)   {\tikz{\pgfuseplotmark{EN};}};           
\node at  (axis cs:-27,+1.0)   {\small\it w/ Cluster};  \node at  (axis cs:-26,+0.3)  {\tikz{\pgfuseplotmark{CN};}};           
\node at (axis cs:-22,-0.5)  {\small Nebulae};


\node at  (axis cs:-32,+.994)   {\small\it Open};      \node at  (axis cs:-32,+0.3)   {\tikz{\pgfuseplotmark{OC};}};           
\node at  (axis cs:-37,+1.0)   {\small\it Globular};   \node at  (axis cs:-37,+0.3)   {\tikz{\pgfuseplotmark{GC};}};           
\node at (axis cs:-34.5,-0.5)  {\small Cluster};

\node at  (axis cs:-42,+.994)   {\small\it Galaxy};      \node at  (axis cs:-42,+0.3) {\tikz{\pgfuseplotmark{GAL};}};         

%
\end{axis}
         


% Constellations
%
% Some are commented out because they are surrounded by already plotted borders
% The x-coordinate is in fractional hours, so must be times 15
%

\begin{axis}[name=constellations,at=(base.center),anchor=center,axis lines=none]

\draw[constellation-boundary]  (axis cs:344.465,    +35.168228) --  (axis cs:344.463,    +35.668224) --  (axis cs:344.457,    +36.668215) --  (axis cs:344.452,    +37.668206) --  (axis cs:344.447,    +38.668197) --  (axis cs:344.441,    +39.668187) --  (axis cs:344.435,    +40.668177) --  (axis cs:344.429,    +41.668167) --  (axis cs:344.423,    +42.668157) --  (axis cs:344.417,    +43.668146) --  (axis cs:344.41,    +44.668135) ;
  \draw[constellation-boundary]  (axis cs:382.911,    +35.645330) --  (axis cs:383.793,    +35.641253) --  (axis cs:384.801,    +35.636410) ;
  \draw[constellation-boundary]  (axis cs:372.443,    +33.681888) --  (axis cs:372.695,    +33.681269) --  (axis cs:373.703,    +33.678660) --  (axis cs:374.711,    +33.675843) --  (axis cs:375.718,    +33.672819) --  (axis cs:376.726,    +33.669589) --  (axis cs:377.734,    +33.666153) --  (axis cs:378.741,    +33.662512) --  (axis cs:379.749,    +33.658669) --  (axis cs:380.757,    +33.654623) --  (axis cs:381.764,    +33.650376) --  (axis cs:382.772,    +33.645930) --  (axis cs:382.898,    +33.645360) ;
  \draw[constellation-boundary]  (axis cs:372.413,    +24.431924) --  (axis cs:372.414,    +24.681923) --  (axis cs:372.417,    +25.681919) --  (axis cs:372.42,    +26.681916) --  (axis cs:372.423,    +27.681912) --  (axis cs:372.426,    +28.681908) --  (axis cs:372.429,    +29.681904) --  (axis cs:372.433,    +30.681900) --  (axis cs:372.436,    +31.681896) --  (axis cs:372.44,    +32.681892) --  (axis cs:372.443,    +33.681888) ;
  \draw[constellation-boundary]  (axis cs:372.413,    +24.431924) --  (axis cs:372.665,    +24.431306) --  (axis cs:373.67,    +24.428704) --  (axis cs:374.424,    +24.426616) ;
  \draw[constellation-boundary]  (axis cs:374.415,    +21.676629) --  (axis cs:374.418,    +22.676625) --  (axis cs:374.422,    +23.676620) --  (axis cs:374.424,    +24.426616) ;
  \draw[constellation-boundary]  (axis cs:363.74,    +21.695185) --  (axis cs:364.619,    +21.694558) --  (axis cs:365.624,    +21.693643) --  (axis cs:366.628,    +21.692515) --  (axis cs:367.633,    +21.691176) --  (axis cs:368.638,    +21.689625) --  (axis cs:369.643,    +21.687863) --  (axis cs:370.647,    +21.685890) --  (axis cs:371.652,    +21.683708) --  (axis cs:372.657,    +21.681316) --  (axis cs:373.661,    +21.678716) --  (axis cs:374.415,    +21.676629) ;
  \draw[constellation-boundary]  (axis cs:362.61,    +22.695751) --  (axis cs:363.615,    +22.695261) --  (axis cs:363.741,    +22.695184) ;
  \draw[constellation-boundary]  (axis cs:362.61,    +22.695751) --  (axis cs:362.61,    +23.695751) --  (axis cs:362.611,    +24.695751) --  (axis cs:362.611,    +25.695750) --  (axis cs:362.612,    +26.695750) --  (axis cs:362.612,    +27.695750) --  (axis cs:362.613,    +28.695750) ;
  \draw[constellation-boundary]  (axis cs:361.606,    +28.696028) --  (axis cs:362.613,    +28.695750) ;
  \draw[constellation-boundary]  (axis cs:361.606,    +28.696028) --  (axis cs:361.606,    +29.696028) --  (axis cs:361.607,    +30.696028) --  (axis cs:361.607,    +31.696028) --  (axis cs:361.607,    +32.029361) ;
  \draw[constellation-boundary]  (axis cs:357.829,    +32.028500) --  (axis cs:358.584,    +32.028913) --  (axis cs:359.592,    +32.029276) --  (axis cs:360.599,    +32.029425) --  (axis cs:361.607,    +32.029361) ;
  \draw[constellation-boundary]  (axis cs:357.829,    +32.028500) --  (axis cs:357.828,    +32.695167) --  (axis cs:357.828,    +32.778500) ;
  \draw[constellation-boundary]  (axis cs:354.049,    +32.774639) --  (axis cs:354.553,    +32.775327) --  (axis cs:355.561,    +32.776543) --  (axis cs:356.568,    +32.777546) --  (axis cs:357.576,    +32.778336) --  (axis cs:357.828,    +32.778500) ;
  \draw[constellation-boundary]  (axis cs:354.049,    +32.774639) --  (axis cs:354.047,    +33.691305) --  (axis cs:354.045,    +34.691303) --  (axis cs:354.044,    +35.191302) ;
  \draw[constellation-boundary]  (axis cs:314.081,    +02.477314) --  (axis cs:314.072,    +03.477274) --  (axis cs:314.063,    +04.477235) --  (axis cs:314.054,    +05.477196) --  (axis cs:314.046,    +06.477156) ;
  \draw[constellation-boundary]  (axis cs:321.583,    +02.539374) --  (axis cs:321.576,    +03.539344) --  (axis cs:321.568,    +04.539315) --  (axis cs:321.56,    +05.539285) --  (axis cs:321.553,    +06.539255) --  (axis cs:321.545,    +07.539225) --  (axis cs:321.537,    +08.539195) --  (axis cs:321.529,    +09.539165) --  (axis cs:321.521,    +10.539134) --  (axis cs:321.513,    +11.539104) --  (axis cs:321.505,    +12.539073) --  (axis cs:321.501,    +13.039057) ;
  \draw[constellation-boundary]  (axis cs:323.584,    +02.554404) --  (axis cs:323.579,    +03.304384) ;
  \draw[constellation-boundary]  (axis cs:323.579,    +03.304384) --  (axis cs:324.579,    +03.311648) --  (axis cs:325.58,    +03.318741) --  (axis cs:326.58,    +03.325661) ;
  \draw[constellation-boundary]  (axis cs:326.587,    +02.325684) --  (axis cs:326.585,    +02.575678) --  (axis cs:326.58,    +03.325661) ;
  \draw[constellation-boundary]  (axis cs:326.587,    +02.325684) --  (axis cs:327.587,    +02.332427) --  (axis cs:328.588,    +02.338993) --  (axis cs:329.588,    +02.345379) --  (axis cs:330.588,    +02.351584) --  (axis cs:331.589,    +02.357605) ;
  \draw[constellation-boundary]  (axis cs:331.589,    +02.357605) --  (axis cs:331.587,    +02.607601) ;
  \draw[constellation-boundary]  (axis cs:331.587,    +02.607601) --  (axis cs:332.588,    +02.613437) --  (axis cs:333.588,    +02.619087) --  (axis cs:334.589,    +02.624548) --  (axis cs:335.589,    +02.629818) --  (axis cs:336.589,    +02.634897) --  (axis cs:337.59,    +02.639782) --  (axis cs:338.59,    +02.644473) --  (axis cs:339.591,    +02.648967) --  (axis cs:340.591,    +02.653263) --  (axis cs:341.592,    +02.657360) --  (axis cs:342.592,    +02.661257) --  (axis cs:342.842,    +02.662200) ;
  \draw[constellation-boundary]  (axis cs:342.865,    -03.337758) --  (axis cs:343.614,    -03.335009) --  (axis cs:344.613,    -03.331521) --  (axis cs:345.613,    -03.328236) --  (axis cs:346.612,    -03.325155) --  (axis cs:347.611,    -03.322280) --  (axis cs:348.61,    -03.319611) --  (axis cs:349.61,    -03.317150) --  (axis cs:350.609,    -03.314896) --  (axis cs:351.608,    -03.312851) --  (axis cs:352.607,    -03.311015) --  (axis cs:353.607,    -03.309388) --  (axis cs:354.606,    -03.307972) --  (axis cs:355.605,    -03.306766) --  (axis cs:356.604,    -03.305772) --  (axis cs:357.603,    -03.304988) --  (axis cs:358.603,    -03.304417) --  (axis cs:359.102,    -03.304210) ;
  \draw[constellation-boundary]  (axis cs:342.865,    -03.337758) --  (axis cs:342.861,    -02.337765) --  (axis cs:342.857,    -01.337772) --  (axis cs:342.853,    -00.337779) --  (axis cs:342.85,    +00.662214) --  (axis cs:342.846,    +01.662207) --  (axis cs:342.842,    +02.662200) --  (axis cs:342.838,    +03.662193) --  (axis cs:342.835,    +04.662186) --  (axis cs:342.831,    +05.662178) --  (axis cs:342.827,    +06.662171) --  (axis cs:342.823,    +07.662164) --  (axis cs:342.821,    +08.162161) ;
  \draw[constellation-boundary]  (axis cs:359.103,    -06.304210) --  (axis cs:359.603,    -06.304056) --  (axis cs:360.601,    -06.303908) --  (axis cs:361.6,    -06.303972) --  (axis cs:362.598,    -06.304247) --  (axis cs:363.597,    -06.304734) --  (axis cs:364.596,    -06.305432) --  (axis cs:365.594,    -06.306342) --  (axis cs:366.593,    -06.307463) ;
  \draw[constellation-boundary]  (axis cs:359.111,    -24.804209) --  (axis cs:359.11,    -24.304209) --  (axis cs:359.11,    -23.304209) --  (axis cs:359.11,    -22.304209) --  (axis cs:359.109,    -21.304209) --  (axis cs:359.109,    -20.304209) --  (axis cs:359.108,    -19.304209) --  (axis cs:359.108,    -18.304209) --  (axis cs:359.107,    -17.304209) --  (axis cs:359.107,    -16.304209) --  (axis cs:359.107,    -15.304209) --  (axis cs:359.106,    -14.304209) --  (axis cs:359.106,    -13.304209) --  (axis cs:359.106,    -12.304209) --  (axis cs:359.105,    -11.304209) --  (axis cs:359.105,    -10.304210) --  (axis cs:359.104,    -09.304210) --  (axis cs:359.104,    -08.304210) --  (axis cs:359.104,    -07.304210) --  (axis cs:359.103,    -06.304210) --  (axis cs:359.103,    -05.304210) --  (axis cs:359.103,    -04.304210) --  (axis cs:359.102,    -03.304210) ;
  \draw[constellation-boundary]  (axis cs:329.77,    -24.904048) --  (axis cs:329.766,    -24.404060) --  (axis cs:329.759,    -23.404084) --  (axis cs:329.751,    -22.404107) --  (axis cs:329.744,    -21.404130) --  (axis cs:329.737,    -20.404153) --  (axis cs:329.73,    -19.404175) --  (axis cs:329.723,    -18.404197) --  (axis cs:329.716,    -17.404219) --  (axis cs:329.709,    -16.404241) --  (axis cs:329.702,    -15.404262) --  (axis cs:329.695,    -14.404283) --  (axis cs:329.689,    -13.404304) --  (axis cs:329.682,    -12.404325) --  (axis cs:329.676,    -11.404346) --  (axis cs:329.669,    -10.404366) --  (axis cs:329.663,    -09.404387) --  (axis cs:329.656,    -08.404407) ;
  \draw[constellation-boundary]  (axis cs:321.668,    -08.460300) --  (axis cs:322.667,    -08.452715) --  (axis cs:323.665,    -08.445297) --  (axis cs:324.664,    -08.438048) --  (axis cs:325.662,    -08.430969) --  (axis cs:326.661,    -08.424064) --  (axis cs:327.659,    -08.417334) --  (axis cs:328.658,    -08.410781) --  (axis cs:329.656,    -08.404407) ;
  \draw[constellation-boundary]  (axis cs:321.716,    -14.460116) --  (axis cs:321.708,    -13.460147) --  (axis cs:321.7,    -12.460178) --  (axis cs:321.692,    -11.460209) --  (axis cs:321.684,    -10.460240) --  (axis cs:321.676,    -09.460270) --  (axis cs:321.668,    -08.460300) ;
  \draw[constellation-boundary]  (axis cs:309.744,    -14.563141) --  (axis cs:309.734,    -13.563188) --  (axis cs:309.724,    -12.563235) --  (axis cs:309.714,    -11.563282) --  (axis cs:309.704,    -10.563329) --  (axis cs:309.694,    -09.563375) --  (axis cs:309.685,    -08.563421) --  (axis cs:309.675,    -07.563467) --  (axis cs:309.665,    -06.563512) --  (axis cs:309.656,    -05.563558) --  (axis cs:309.646,    -04.563603) --  (axis cs:309.637,    -03.563648) --  (axis cs:309.627,    -02.563693) --  (axis cs:309.618,    -01.563738) --  (axis cs:309.608,    -00.563782) --  (axis cs:309.599,    +00.436173) --  (axis cs:309.589,    +01.436128) --  (axis cs:309.58,    +02.436083) ;
  \draw[constellation-boundary]  (axis cs:309.744,    -14.563141) --  (axis cs:310.742,    -14.553758) --  (axis cs:311.74,    -14.544512) --  (axis cs:312.738,    -14.535405) --  (axis cs:313.735,    -14.526439) --  (axis cs:314.733,    -14.517617) --  (axis cs:315.731,    -14.508941) --  (axis cs:316.729,    -14.500415) --  (axis cs:317.726,    -14.492042) --  (axis cs:318.724,    -14.483822) --  (axis cs:319.721,    -14.475760) --  (axis cs:320.719,    -14.467857) --  (axis cs:321.716,    -14.460116) ;
  \draw[constellation-boundary]  (axis cs:280.326,    +02.115345) --  (axis cs:280.576,    +02.118339) --  (axis cs:281.576,    +02.130293) --  (axis cs:282.576,    +02.142208) --  (axis cs:283.576,    +02.154078) --  (axis cs:284.576,    +02.165903) ;
  \draw[constellation-boundary]  (axis cs:284.576,    +02.165903) --  (axis cs:284.565,    +03.165833) --  (axis cs:284.553,    +04.165763) --  (axis cs:284.541,    +05.165693) --  (axis cs:284.529,    +06.165623) --  (axis cs:284.526,    +06.415605) ;
  \draw[constellation-boundary]  (axis cs:281.459,    +06.379193) --  (axis cs:281.45,    +07.129139) --  (axis cs:281.437,    +08.129066) --  (axis cs:281.425,    +09.128993) --  (axis cs:281.413,    +10.128920) --  (axis cs:281.4,    +11.128846) --  (axis cs:281.388,    +12.128772) ;
  \draw[constellation-boundary]  (axis cs:284.456,    +12.165194) --  (axis cs:284.444,    +13.165121) --  (axis cs:284.431,    +14.165047) --  (axis cs:284.419,    +15.164972) --  (axis cs:284.406,    +16.164897) --  (axis cs:284.393,    +17.164822) --  (axis cs:284.38,    +18.164745) --  (axis cs:284.367,    +19.164667) --  (axis cs:284.354,    +20.164589) --  (axis cs:284.34,    +21.164509) --  (axis cs:284.327,    +22.164429) --  (axis cs:284.313,    +23.164347) --  (axis cs:284.299,    +24.164264) --  (axis cs:284.284,    +25.164180) --  (axis cs:284.27,    +26.164094) ;
  \draw[constellation-boundary]  (axis cs:284.374,    +18.664706) --  (axis cs:285.374,    +18.676490) --  (axis cs:286.375,    +18.688221) ;
  \draw[constellation-boundary]  (axis cs:286.405,    +16.355063) --  (axis cs:286.395,    +17.188334) --  (axis cs:286.382,    +18.188259) --  (axis cs:286.375,    +18.688221) ;
  \draw[constellation-boundary]  (axis cs:286.405,    +16.355063) --  (axis cs:287.406,    +16.366735) --  (axis cs:288.407,    +16.378346) --  (axis cs:289.408,    +16.389893) --  (axis cs:290.409,    +16.401373) --  (axis cs:291.411,    +16.412781) --  (axis cs:292.412,    +16.424114) --  (axis cs:293.413,    +16.435369) --  (axis cs:294.414,    +16.446543) --  (axis cs:295.416,    +16.457631) --  (axis cs:296.417,    +16.468631) --  (axis cs:297.419,    +16.479540) --  (axis cs:298.42,    +16.490353) --  (axis cs:298.921,    +16.495723) ;
  \draw[constellation-boundary]  (axis cs:298.926,    +16.079082) --  (axis cs:298.923,    +16.329066) --  (axis cs:298.921,    +16.495723) ;
  \draw[constellation-boundary]  (axis cs:298.926,    +16.079082) --  (axis cs:299.427,    +16.084426) --  (axis cs:300.428,    +16.095039) --  (axis cs:301.43,    +16.105546) --  (axis cs:302.432,    +16.115945) --  (axis cs:303.434,    +16.126233) --  (axis cs:304.436,    +16.136406) --  (axis cs:305.187,    +16.143959) ;
  \draw[constellation-boundary]  (axis cs:303.637,    +08.877907) --  (axis cs:303.631,    +09.377881) --  (axis cs:303.621,    +10.377827) --  (axis cs:303.61,    +11.377773) --  (axis cs:303.6,    +12.377719) --  (axis cs:303.589,    +13.377664) --  (axis cs:303.578,    +14.377609) --  (axis cs:303.567,    +15.377553) --  (axis cs:303.559,    +16.127511) ;
  \draw[constellation-boundary]  (axis cs:303.637,    +08.877907) --  (axis cs:304.512,    +08.886795) --  (axis cs:305.513,    +08.896841) --  (axis cs:306.014,    +08.901819) ;
  \draw[constellation-boundary]  (axis cs:306.079,    +02.402142) --  (axis cs:306.579,    +02.407086) --  (axis cs:307.579,    +02.416880) --  (axis cs:308.58,    +02.426546) --  (axis cs:309.58,    +02.436083) --  (axis cs:310.58,    +02.445487) --  (axis cs:311.58,    +02.454755) --  (axis cs:312.581,    +02.463885) --  (axis cs:313.581,    +02.472873) --  (axis cs:314.581,    +02.481718) --  (axis cs:315.582,    +02.490415) --  (axis cs:316.582,    +02.498963) --  (axis cs:317.582,    +02.507360) --  (axis cs:318.582,    +02.515601) --  (axis cs:319.583,    +02.523686) --  (axis cs:320.583,    +02.531611) --  (axis cs:321.583,    +02.539374) --  (axis cs:322.584,    +02.546972) --  (axis cs:323.584,    +02.554404) ;
  \draw[constellation-boundary]  (axis cs:306.079,    +02.402142) --  (axis cs:306.069,    +03.402093) --  (axis cs:306.059,    +04.402044) --  (axis cs:306.049,    +05.401994) --  (axis cs:306.039,    +06.401944) --  (axis cs:306.029,    +07.401895) --  (axis cs:306.019,    +08.401844) --  (axis cs:306.014,    +08.901819) ;
  \draw[constellation-boundary]  (axis cs:301.694,    -08.643079) --  (axis cs:302.693,    -08.632708) --  (axis cs:303.692,    -08.622450) --  (axis cs:304.691,    -08.612306) --  (axis cs:305.689,    -08.602281) --  (axis cs:306.688,    -08.592377) --  (axis cs:307.687,    -08.582597) --  (axis cs:308.686,    -08.572944) --  (axis cs:309.685,    -08.563421) ;
  \draw[constellation-boundary]  (axis cs:284.744,    -11.866442) --  (axis cs:285.743,    -11.854677) --  (axis cs:286.743,    -11.842967) --  (axis cs:287.742,    -11.831315) --  (axis cs:288.741,    -11.819725) --  (axis cs:289.74,    -11.808200) --  (axis cs:290.739,    -11.796744) --  (axis cs:291.738,    -11.785360) --  (axis cs:292.737,    -11.774052) --  (axis cs:293.736,    -11.762822) --  (axis cs:294.735,    -11.751676) --  (axis cs:295.734,    -11.740615) --  (axis cs:296.733,    -11.729644) --  (axis cs:297.732,    -11.718765) --  (axis cs:298.73,    -11.707982) --  (axis cs:299.729,    -11.697298) --  (axis cs:300.728,    -11.686717) --  (axis cs:301.726,    -11.676242) ;
  \draw[constellation-boundary]  (axis cs:280.398,    -03.884224) --  (axis cs:280.386,    -02.884296) --  (axis cs:280.374,    -01.884368) --  (axis cs:280.362,    -00.884439) --  (axis cs:280.35,    +00.115489) --  (axis cs:280.338,    +01.115417) --  (axis cs:280.326,    +02.115345) ;
  \draw[constellation-boundary]  (axis cs:317.088,    -44.998892) --  (axis cs:318.079,    -44.990573) --  (axis cs:319.07,    -44.982408) --  (axis cs:320.061,    -44.974400) --  (axis cs:321.052,    -44.966552) --  (axis cs:322.042,    -44.958865) ;
  \draw[constellation-boundary]  (axis cs:269.773,    -44.016565) --  (axis cs:269.75,    -43.016706) --  (axis cs:269.728,    -42.016841) --  (axis cs:269.706,    -41.016973) --  (axis cs:269.685,    -40.017101) --  (axis cs:269.665,    -39.017224) --  (axis cs:269.645,    -38.017345) --  (axis cs:269.625,    -37.017462) --  (axis cs:269.607,    -36.017576) --  (axis cs:269.588,    -35.017688) --  (axis cs:269.57,    -34.017796) --  (axis cs:269.553,    -33.017902) --  (axis cs:269.536,    -32.018006) --  (axis cs:269.519,    -31.018107) --  (axis cs:269.503,    -30.018207) ;
  \draw[constellation-boundary]  (axis cs:301.916,    -27.641919) --  (axis cs:301.903,    -26.641988) --  (axis cs:301.89,    -25.642055) --  (axis cs:301.877,    -24.642122) --  (axis cs:301.865,    -23.642187) --  (axis cs:301.852,    -22.642252) --  (axis cs:301.84,    -21.642315) --  (axis cs:301.828,    -20.642378) --  (axis cs:301.816,    -19.642439) --  (axis cs:301.805,    -18.642500) --  (axis cs:301.793,    -17.642561) --  (axis cs:301.782,    -16.642620) --  (axis cs:301.77,    -15.642679) --  (axis cs:301.759,    -14.642738) --  (axis cs:301.748,    -13.642796) --  (axis cs:301.737,    -12.642853) --  (axis cs:301.726,    -11.642910) --  (axis cs:301.715,    -10.642967) --  (axis cs:301.704,    -09.643023) --  (axis cs:301.694,    -08.643079) ;
  \draw[constellation-boundary]  (axis cs:301.916,    -27.641919) --  (axis cs:302.913,    -27.631573) --  (axis cs:303.909,    -27.621340) --  (axis cs:304.905,    -27.611221) --  (axis cs:305.902,    -27.601222) --  (axis cs:306.898,    -27.591344) --  (axis cs:307.894,    -27.581590) --  (axis cs:308.89,    -27.571963) --  (axis cs:309.886,    -27.562467) --  (axis cs:310.882,    -27.553104) --  (axis cs:311.878,    -27.543877) --  (axis cs:312.874,    -27.534789) --  (axis cs:313.869,    -27.525842) --  (axis cs:314.865,    -27.517039) --  (axis cs:315.86,    -27.508384) --  (axis cs:316.856,    -27.499877) --  (axis cs:317.851,    -27.491523) --  (axis cs:318.846,    -27.483323) --  (axis cs:319.841,    -27.475279) --  (axis cs:320.837,    -27.467396) --  (axis cs:321.832,    -27.459674) ;
  \draw[constellation-boundary]  (axis cs:321.808,    -24.959765) --  (axis cs:322.803,    -24.952203) --  (axis cs:323.799,    -24.944808) --  (axis cs:324.794,    -24.937581) --  (axis cs:325.789,    -24.930524) --  (axis cs:326.785,    -24.923641) --  (axis cs:327.78,    -24.916932) --  (axis cs:328.775,    -24.910401) --  (axis cs:329.77,    -24.904048) --  (axis cs:330.765,    -24.897876) --  (axis cs:331.76,    -24.891887) --  (axis cs:332.755,    -24.886082) --  (axis cs:333.75,    -24.880463) --  (axis cs:334.745,    -24.875032) --  (axis cs:335.74,    -24.869791) --  (axis cs:336.735,    -24.864741) --  (axis cs:337.73,    -24.859883) --  (axis cs:338.724,    -24.855220) --  (axis cs:339.719,    -24.850752) --  (axis cs:340.714,    -24.846480) --  (axis cs:341.708,    -24.842407) --  (axis cs:342.703,    -24.838533) --  (axis cs:343.697,    -24.834860) --  (axis cs:344.692,    -24.831388) --  (axis cs:345.686,    -24.828118) --  (axis cs:346.681,    -24.825052) --  (axis cs:347.675,    -24.822191) --  (axis cs:348.67,    -24.819535) --  (axis cs:349.664,    -24.817085) --  (axis cs:350.659,    -24.814842) --  (axis cs:351.653,    -24.812807) --  (axis cs:352.647,    -24.810980) --  (axis cs:353.642,    -24.809361) --  (axis cs:354.636,    -24.807952) --  (axis cs:355.63,    -24.806752) --  (axis cs:356.625,    -24.805762) --  (axis cs:357.619,    -24.804983) --  (axis cs:358.613,    -24.804414) --  (axis cs:359.608,    -24.804056) --  (axis cs:360.602,    -24.803908) --  (axis cs:361.596,    -24.803972) --  (axis cs:362.591,    -24.804246) --  (axis cs:363.585,    -24.804730) --  (axis cs:364.579,    -24.805426) --  (axis cs:365.574,    -24.806332) --  (axis cs:366.568,    -24.807448) --  (axis cs:367.562,    -24.808773) --  (axis cs:368.557,    -24.810308) --  (axis cs:369.551,    -24.812052) --  (axis cs:370.545,    -24.814004) --  (axis cs:371.54,    -24.816165) --  (axis cs:372.534,    -24.818532) --  (axis cs:373.529,    -24.821106) --  (axis cs:374.523,    -24.823885) --  (axis cs:375.518,    -24.826869) --  (axis cs:376.512,    -24.830057) --  (axis cs:377.507,    -24.833449) --  (axis cs:378.501,    -24.837042) --  (axis cs:379.496,    -24.840836) --  (axis cs:380.49,    -24.844830) --  (axis cs:381.485,    -24.849022) --  (axis cs:382.48,    -24.853412) --  (axis cs:383.475,    -24.857997) --  (axis cs:384.469,    -24.862777) ;
  \draw[constellation-boundary]  (axis cs:366.604,    +02.692530) --  (axis cs:367.604,    +02.691197) --  (axis cs:368.605,    +02.689652) --  (axis cs:369.605,    +02.687897) --  (axis cs:370.606,    +02.685933) --  (axis cs:371.606,    +02.683760) --  (axis cs:372.607,    +02.681378) --  (axis cs:373.607,    +02.678789) --  (axis cs:374.608,    +02.675992) --  (axis cs:375.608,    +02.672990) --  (axis cs:376.609,    +02.669783) --  (axis cs:377.609,    +02.666372) --  (axis cs:378.61,    +02.662757) --  (axis cs:379.61,    +02.658941) --  (axis cs:380.611,    +02.654923) --  (axis cs:381.611,    +02.650706) --  (axis cs:382.611,    +02.646291) --  (axis cs:383.612,    +02.641679) --  (axis cs:384.612,    +02.636872) ;
  \draw[constellation-boundary]  (axis cs:366.593,    -06.307463) --  (axis cs:366.594,    -05.307464) --  (axis cs:366.595,    -04.307464) --  (axis cs:366.596,    -03.307465) --  (axis cs:366.598,    -02.307466) --  (axis cs:366.599,    -01.307467) --  (axis cs:366.6,    -00.307467) --  (axis cs:366.601,    +00.692532) --  (axis cs:366.603,    +01.692531) --  (axis cs:366.604,    +02.692530) ;
  \draw[constellation-boundary]  (axis cs:269.625,    -37.017462) --  (axis cs:270.126,    -37.011388) --  (axis cs:271.125,    -36.999240) --  (axis cs:272.125,    -36.987092) --  (axis cs:273.125,    -36.974950) --  (axis cs:274.124,    -36.962816) --  (axis cs:275.124,    -36.950694) --  (axis cs:276.123,    -36.938588) --  (axis cs:277.122,    -36.926502) --  (axis cs:278.121,    -36.914440) --  (axis cs:279.119,    -36.902404) --  (axis cs:280.118,    -36.890399) --  (axis cs:281.116,    -36.878428) --  (axis cs:282.115,    -36.866495) --  (axis cs:283.113,    -36.854604) --  (axis cs:284.11,    -36.842757) --  (axis cs:285.108,    -36.830960) --  (axis cs:286.106,    -36.819214) --  (axis cs:287.103,    -36.807525) --  (axis cs:288.101,    -36.795895) --  (axis cs:289.098,    -36.784328) --  (axis cs:289.596,    -36.778569) ;
  \draw[constellation-boundary]  (axis cs:289.758,    -44.777636) --  (axis cs:289.736,    -43.777765) --  (axis cs:289.714,    -42.777891) --  (axis cs:289.693,    -41.778012) --  (axis cs:289.672,    -40.778130) --  (axis cs:289.652,    -39.778245) --  (axis cs:289.633,    -38.778356) --  (axis cs:289.614,    -37.778464) --  (axis cs:289.596,    -36.778569) ;
  \draw[constellation-boundary]  (axis cs:290.133,    +27.732407) --  (axis cs:290.258,    +27.733840) --  (axis cs:291.26,    +27.745259) --  (axis cs:292.262,    +27.756603) --  (axis cs:293.265,    +27.767870) --  (axis cs:294.267,    +27.779056) --  (axis cs:295.27,    +27.790158) --  (axis cs:296.272,    +27.801171) ;
  \draw[constellation-boundary]  (axis cs:290.095,    +30.232193) --  (axis cs:290.221,    +30.233626) --  (axis cs:291.223,    +30.245048) --  (axis cs:291.599,    +30.249312) ;
  \draw[constellation-boundary]  (axis cs:291.599,    +30.249312) --  (axis cs:291.583,    +31.249225) --  (axis cs:291.568,    +32.249136) --  (axis cs:291.552,    +33.249046) --  (axis cs:291.535,    +34.248953) --  (axis cs:291.519,    +35.248858) --  (axis cs:291.501,    +36.248760) --  (axis cs:291.493,    +36.748711) ;
  \draw[constellation-boundary]  (axis cs:291.493,    +36.748711) --  (axis cs:292.12,    +36.755799) ;
  \draw[constellation-boundary]  (axis cs:292.12,    +36.755799) --  (axis cs:292.111,    +37.255749) --  (axis cs:292.093,    +38.255647) --  (axis cs:292.074,    +39.255543) --  (axis cs:292.055,    +40.255435) --  (axis cs:292.035,    +41.255324) --  (axis cs:292.015,    +42.255210) --  (axis cs:291.994,    +43.255092) --  (axis cs:291.984,    +43.755031) ;
  \draw[constellation-boundary]  (axis cs:288.47,    +43.714937) --  (axis cs:288.972,    +43.720717) --  (axis cs:289.976,    +43.732226) --  (axis cs:290.98,    +43.743666) --  (axis cs:291.984,    +43.755031) ;
  \draw[constellation-boundary]  (axis cs:288.47,    +43.714937) --  (axis cs:288.459,    +44.214873) ;
  \draw[constellation-boundary]  (axis cs:330.763,    +44.603636) ;
  \draw[constellation-boundary]  (axis cs:329.879,    +44.598244) --  (axis cs:330.257,    +44.600572) --  (axis cs:330.763,    +44.603636) ;
  \draw[constellation-boundary]  (axis cs:329.882,    +44.348254) --  (axis cs:329.879,    +44.598244) ;
  \draw[constellation-boundary]  (axis cs:329.377,    +44.345110) --  (axis cs:329.882,    +44.348254) ;
  \draw[constellation-boundary]  (axis cs:329.461,    +36.595374) --  (axis cs:329.451,    +37.595343) --  (axis cs:329.441,    +38.595311) --  (axis cs:329.431,    +39.595279) --  (axis cs:329.42,    +40.595245) --  (axis cs:329.409,    +41.595211) --  (axis cs:329.397,    +42.595175) --  (axis cs:329.386,    +43.595138) --  (axis cs:329.377,    +44.345110) ;
  \draw[constellation-boundary]  (axis cs:327.32,    +36.581538) --  (axis cs:328.327,    +36.588151) --  (axis cs:329.335,    +36.594583) --  (axis cs:330.343,    +36.600833) --  (axis cs:331.35,    +36.606899) ;
  \draw[constellation-boundary]  (axis cs:327.395,    +28.581788) --  (axis cs:327.387,    +29.581759) --  (axis cs:327.378,    +30.581730) --  (axis cs:327.369,    +31.581699) --  (axis cs:327.359,    +32.581669) --  (axis cs:327.35,    +33.581637) --  (axis cs:327.34,    +34.581605) --  (axis cs:327.33,    +35.581572) --  (axis cs:327.32,    +36.581538) ;
  \draw[constellation-boundary]  (axis cs:315.084,    +28.487183) --  (axis cs:315.335,    +28.489352) --  (axis cs:316.34,    +28.497937) --  (axis cs:317.344,    +28.506370) --  (axis cs:318.349,    +28.514648) --  (axis cs:319.354,    +28.522770) --  (axis cs:320.359,    +28.530731) --  (axis cs:321.364,    +28.538530) --  (axis cs:322.369,    +28.546165) --  (axis cs:323.374,    +28.553632) --  (axis cs:324.379,    +28.560931) --  (axis cs:325.384,    +28.568058) --  (axis cs:326.39,    +28.575011) --  (axis cs:327.395,    +28.581788) ;
  \draw[constellation-boundary]  (axis cs:315.084,    +28.487183) --  (axis cs:315.073,    +29.487134) ;
  \draw[constellation-boundary]  (axis cs:296.251,    +29.301054) --  (axis cs:297.254,    +29.311978) --  (axis cs:298.257,    +29.322807) --  (axis cs:299.26,    +29.333538) --  (axis cs:300.263,    +29.344167) --  (axis cs:301.266,    +29.354692) --  (axis cs:302.27,    +29.365109) --  (axis cs:303.273,    +29.375415) --  (axis cs:304.277,    +29.385607) --  (axis cs:305.281,    +29.395681) --  (axis cs:306.285,    +29.405635) --  (axis cs:307.289,    +29.415465) --  (axis cs:308.293,    +29.425169) --  (axis cs:309.297,    +29.434743) --  (axis cs:310.301,    +29.444185) --  (axis cs:311.305,    +29.453491) --  (axis cs:312.31,    +29.462658) --  (axis cs:313.314,    +29.471685) --  (axis cs:314.319,    +29.480568) --  (axis cs:315.073,    +29.487134) ;
  \draw[constellation-boundary]  (axis cs:296.272,    +27.801171) --  (axis cs:296.265,    +28.301132) --  (axis cs:296.251,    +29.301054) ;
  \draw[constellation-boundary]  (axis cs:305.187,    +16.143959) --  (axis cs:305.184,    +16.393946) --  (axis cs:305.173,    +17.393891) --  (axis cs:305.162,    +18.393836) --  (axis cs:305.151,    +19.393780) --  (axis cs:305.14,    +20.393723) --  (axis cs:305.128,    +21.393666) --  (axis cs:305.125,    +21.643651) ;
  \draw[constellation-boundary]  (axis cs:305.134,    +20.893695) --  (axis cs:305.385,    +20.896199) --  (axis cs:306.387,    +20.906140) --  (axis cs:307.39,    +20.915958) --  (axis cs:308.393,    +20.925649) --  (axis cs:309.395,    +20.935210) --  (axis cs:309.897,    +20.939941) ;
  \draw[constellation-boundary]  (axis cs:309.908,    +19.939991) --  (axis cs:309.902,    +20.439966) --  (axis cs:309.897,    +20.939941) ;
  \draw[constellation-boundary]  (axis cs:309.908,    +19.939991) --  (axis cs:310.409,    +19.944688) --  (axis cs:311.412,    +19.953979) --  (axis cs:312.415,    +19.963132) --  (axis cs:313.417,    +19.972144) --  (axis cs:314.42,    +19.981012) --  (axis cs:315.423,    +19.989733) --  (axis cs:316.426,    +19.998305) --  (axis cs:317.43,    +20.006725) --  (axis cs:318.433,    +20.014990) --  (axis cs:319.436,    +20.023098) --  (axis cs:320.188,    +20.029074) ;
  \draw[constellation-boundary]  (axis cs:317.248,    +12.338258) --  (axis cs:317.247,    +12.504919) --  (axis cs:317.238,    +13.504882) --  (axis cs:317.229,    +14.504845) --  (axis cs:317.22,    +15.504808) --  (axis cs:317.211,    +16.504770) --  (axis cs:317.202,    +17.504732) --  (axis cs:317.193,    +18.504693) --  (axis cs:317.183,    +19.504654) --  (axis cs:317.179,    +20.004634) ;
  \draw[constellation-boundary]  (axis cs:314.619,    +12.315762) --  (axis cs:315.495,    +12.323376) --  (axis cs:316.497,    +12.331937) --  (axis cs:317.499,    +12.340346) --  (axis cs:318.25,    +12.346552) ;
  \draw[constellation-boundary]  (axis cs:314.671,    +06.482659) --  (axis cs:314.662,    +07.482620) --  (axis cs:314.653,    +08.482581) --  (axis cs:314.644,    +09.482542) --  (axis cs:314.635,    +10.482502) --  (axis cs:314.626,    +11.482462) --  (axis cs:314.619,    +12.315762) ;
  \draw[constellation-boundary]  (axis cs:314.046,    +06.477156) --  (axis cs:314.546,    +06.481563) --  (axis cs:314.671,    +06.482659) ;
  \draw[constellation-boundary]  (axis cs:318.25,    +12.346552) --  (axis cs:318.249,    +12.513212) --  (axis cs:318.244,    +13.013195) ;
  \draw[constellation-boundary]  (axis cs:318.244,    +13.013195) --  (axis cs:318.495,    +13.015244) --  (axis cs:319.497,    +13.023342) --  (axis cs:320.499,    +13.031281) --  (axis cs:321.501,    +13.039057) ;
  \draw[constellation-boundary]  (axis cs:321.928,    -36.459303) --  (axis cs:322.921,    -36.451761) --  (axis cs:323.914,    -36.444385) --  (axis cs:324.906,    -36.437177) --  (axis cs:325.899,    -36.430140) --  (axis cs:326.892,    -36.423276) --  (axis cs:327.884,    -36.416586) --  (axis cs:328.876,    -36.410073) --  (axis cs:329.869,    -36.403738) --  (axis cs:330.861,    -36.397584) --  (axis cs:331.853,    -36.391612) --  (axis cs:332.845,    -36.385824) --  (axis cs:333.837,    -36.380222) --  (axis cs:334.829,    -36.374807) --  (axis cs:335.821,    -36.369582) --  (axis cs:336.813,    -36.364547) --  (axis cs:337.804,    -36.359704) --  (axis cs:338.796,    -36.355055) --  (axis cs:339.788,    -36.350601) --  (axis cs:340.779,    -36.346343) --  (axis cs:341.771,    -36.342282) --  (axis cs:342.762,    -36.338421) --  (axis cs:343.754,    -36.334759) --  (axis cs:344.745,    -36.331298) --  (axis cs:345.736,    -36.328039) --  (axis cs:346.727,    -36.324983) --  (axis cs:347.719,    -36.322131) --  (axis cs:348.71,    -36.319484) --  (axis cs:349.701,    -36.317042) --  (axis cs:350.692,    -36.314806) --  (axis cs:351.683,    -36.312778) ;
  \draw[constellation-boundary]  (axis cs:322.035,    -44.458894) --  (axis cs:322.02,    -43.458951) --  (axis cs:322.006,    -42.459006) --  (axis cs:321.992,    -41.459059) --  (axis cs:321.978,    -40.459111) --  (axis cs:321.965,    -39.459161) --  (axis cs:321.952,    -38.459210) --  (axis cs:321.94,    -37.459257) --  (axis cs:321.928,    -36.459303) --  (axis cs:321.916,    -35.459348) --  (axis cs:321.905,    -34.459392) --  (axis cs:321.894,    -33.459435) --  (axis cs:321.883,    -32.459477) --  (axis cs:321.872,    -31.459518) --  (axis cs:321.862,    -30.459558) --  (axis cs:321.852,    -29.459597) --  (axis cs:321.841,    -28.459636) --  (axis cs:321.832,    -27.459674) --  (axis cs:321.822,    -26.459711) --  (axis cs:321.812,    -25.459747) --  (axis cs:321.808,    -24.959765) ;
  \draw[constellation-boundary]  (axis cs:351.693,    -39.312768) --  (axis cs:352.683,    -39.310949) --  (axis cs:353.673,    -39.309338) --  (axis cs:354.663,    -39.307934) --  (axis cs:355.653,    -39.306740) --  (axis cs:356.643,    -39.305754) --  (axis cs:357.633,    -39.304978) --  (axis cs:358.623,    -39.304412) --  (axis cs:359.613,    -39.304055) --  (axis cs:360.603,    -39.303908) --  (axis cs:361.593,    -39.303971) --  (axis cs:362.583,    -39.304244) --  (axis cs:363.573,    -39.304727) --  (axis cs:364.563,    -39.305419) --  (axis cs:365.553,    -39.306321) --  (axis cs:366.543,    -39.307432) --  (axis cs:367.533,    -39.308752) --  (axis cs:368.523,    -39.310280) --  (axis cs:369.513,    -39.312017) --  (axis cs:370.503,    -39.313961) --  (axis cs:371.493,    -39.316112) --  (axis cs:372.483,    -39.318469) --  (axis cs:373.474,    -39.321032) --  (axis cs:374.464,    -39.323799) --  (axis cs:375.454,    -39.326771) --  (axis cs:376.445,    -39.329946) --  (axis cs:377.435,    -39.333323) --  (axis cs:378.425,    -39.336901) --  (axis cs:379.416,    -39.340679) --  (axis cs:380.406,    -39.344656) --  (axis cs:381.397,    -39.348832) --  (axis cs:382.387,    -39.353203) --  (axis cs:383.378,    -39.357771) --  (axis cs:384.369,    -39.362532) ;
  \draw[constellation-boundary]  (axis cs:346.727,    -36.324983) --  (axis cs:346.723,    -35.324990) --  (axis cs:346.719,    -34.324996) --  (axis cs:346.714,    -33.325003) --  (axis cs:346.71,    -32.325009) --  (axis cs:346.706,    -31.325015) --  (axis cs:346.702,    -30.325021) --  (axis cs:346.698,    -29.325027) --  (axis cs:346.694,    -28.325033) --  (axis cs:346.69,    -27.325039) --  (axis cs:346.686,    -26.325044) --  (axis cs:346.683,    -25.325050) --  (axis cs:346.681,    -24.825052) ;
  \draw[constellation-boundary]  (axis cs:273.824,    +30.039155) --  (axis cs:273.808,    +31.039056) --  (axis cs:273.791,    +32.038955) --  (axis cs:273.774,    +33.038851) --  (axis cs:273.757,    +34.038745) --  (axis cs:273.739,    +35.038637) --  (axis cs:273.721,    +36.038526) --  (axis cs:273.702,    +37.038412) --  (axis cs:273.683,    +38.038295) --  (axis cs:273.663,    +39.038175) --  (axis cs:273.642,    +40.038051) --  (axis cs:273.621,    +41.037924) --  (axis cs:273.6,    +42.037793) --  (axis cs:273.577,    +43.037657) --  (axis cs:273.554,    +44.037517) ;
  \draw[constellation-boundary]  (axis cs:273.824,    +30.039155) --  (axis cs:274.199,    +30.043704) --  (axis cs:275.2,    +30.055824) --  (axis cs:276.2,    +30.067929) --  (axis cs:276.701,    +30.073974) ;
  \draw[constellation-boundary]  (axis cs:276.763,    +26.074349) --  (axis cs:276.748,    +27.074257) --  (axis cs:276.732,    +28.074165) --  (axis cs:276.717,    +29.074070) --  (axis cs:276.701,    +30.073974) ;
  \draw[constellation-boundary]  (axis cs:276.763,    +26.074349) --  (axis cs:277.263,    +26.080387) --  (axis cs:278.264,    +26.092446) --  (axis cs:279.264,    +26.104478) --  (axis cs:280.265,    +26.116478) --  (axis cs:281.266,    +26.128444) --  (axis cs:282.267,    +26.140371) --  (axis cs:283.269,    +26.152255) --  (axis cs:284.27,    +26.164094) ;
  \draw[constellation-boundary]  (axis cs:284.277,    +25.664137) --  (axis cs:285.278,    +25.675926) --  (axis cs:286.28,    +25.687662) --  (axis cs:287.281,    +25.699341) --  (axis cs:288.283,    +25.710960) --  (axis cs:289.285,    +25.722516) --  (axis cs:290.161,    +25.732571) ;
  \draw[constellation-boundary]  (axis cs:284.339,    +21.247836) --  (axis cs:285.34,    +21.259622) --  (axis cs:286.341,    +21.271354) --  (axis cs:287.343,    +21.283030) --  (axis cs:288.344,    +21.294645) --  (axis cs:289.345,    +21.306197) --  (axis cs:290.096,    +21.314816) ;
  \draw[constellation-boundary]  (axis cs:275.203,    +12.054330) --  (axis cs:275.453,    +12.057357) --  (axis cs:276.453,    +12.069457) --  (axis cs:277.453,    +12.081536) --  (axis cs:278.454,    +12.093590) --  (axis cs:279.454,    +12.105616) --  (axis cs:280.454,    +12.117610) --  (axis cs:281.455,    +12.129568) --  (axis cs:282.455,    +12.141487) --  (axis cs:283.456,    +12.153364) --  (axis cs:284.456,    +12.165194) ;
  \draw[constellation-boundary]  (axis cs:275.203,    +12.054330) --  (axis cs:275.19,    +13.054253) --  (axis cs:275.178,    +14.054175) --  (axis cs:275.173,    +14.387482) ;
  \draw[constellation-boundary]  (axis cs:265.424,    +14.269178) --  (axis cs:266.424,    +14.281283) --  (axis cs:267.424,    +14.293404) --  (axis cs:268.423,    +14.305537) --  (axis cs:269.423,    +14.317679) --  (axis cs:270.423,    +14.329827) --  (axis cs:271.423,    +14.341975) --  (axis cs:272.423,    +14.354122) --  (axis cs:273.423,    +14.366262) --  (axis cs:274.423,    +14.378393) --  (axis cs:275.173,    +14.387482) ;
  \draw[constellation-boundary]  (axis cs:307.16,    -44.590055) --  (axis cs:307.14,    -43.590149) --  (axis cs:307.122,    -42.590240) --  (axis cs:307.104,    -41.590328) --  (axis cs:307.087,    -40.590414) --  (axis cs:307.07,    -39.590497) --  (axis cs:307.054,    -38.590577) --  (axis cs:307.038,    -37.590656) --  (axis cs:307.022,    -36.590732) --  (axis cs:307.007,    -35.590806) --  (axis cs:306.992,    -34.590879) --  (axis cs:306.978,    -33.590950) --  (axis cs:306.964,    -32.591019) --  (axis cs:306.95,    -31.591086) --  (axis cs:306.937,    -30.591153) --  (axis cs:306.924,    -29.591218) --  (axis cs:306.911,    -28.591281) --  (axis cs:306.898,    -27.591344) ;
  \draw[constellation-boundary]  (axis cs:343.709,    +35.165608) --  (axis cs:344.465,    +35.168228) --  (axis cs:345.473,    +35.171543) --  (axis cs:346.482,    +35.174651) --  (axis cs:347.49,    +35.177552) --  (axis cs:348.498,    +35.180244) --  (axis cs:349.506,    +35.182728) --  (axis cs:350.515,    +35.185003) --  (axis cs:351.523,    +35.187067) --  (axis cs:352.532,    +35.188920) --  (axis cs:353.54,    +35.190561) --  (axis cs:354.044,    +35.191302) ;
  \draw[constellation-boundary]  (axis cs:343.709,    +35.165608) --  (axis cs:343.706,    +35.665603) ;
  \draw[constellation-boundary]  (axis cs:331.36,    +35.606926) --  (axis cs:332.367,    +35.612804) --  (axis cs:333.375,    +35.618494) --  (axis cs:334.382,    +35.623995) --  (axis cs:335.39,    +35.629304) --  (axis cs:336.398,    +35.634421) --  (axis cs:337.406,    +35.639342) --  (axis cs:338.414,    +35.644068) --  (axis cs:339.422,    +35.648596) --  (axis cs:340.43,    +35.652925) --  (axis cs:341.438,    +35.657053) --  (axis cs:342.446,    +35.660980) --  (axis cs:343.454,    +35.664704) --  (axis cs:343.706,    +35.665603) ;
  \draw[constellation-boundary]  (axis cs:331.36,    +35.606926) --  (axis cs:331.35,    +36.606899) ;
  \draw[constellation-boundary]  (axis cs:290.161,    +25.732571) --  (axis cs:290.154,    +26.232531) --  (axis cs:290.14,    +27.232449) --  (axis cs:290.125,    +28.232365) --  (axis cs:290.11,    +29.232280) --  (axis cs:290.095,    +30.232193) ;
  \draw[constellation-boundary]  (axis cs:275.275,    +06.304763) --  (axis cs:275.525,    +06.307790) --  (axis cs:276.525,    +06.319888) --  (axis cs:277.525,    +06.331965) --  (axis cs:278.525,    +06.344018) --  (axis cs:279.525,    +06.356041) --  (axis cs:280.525,    +06.368033) --  (axis cs:281.525,    +06.379989) --  (axis cs:282.525,    +06.391905) --  (axis cs:283.526,    +06.403778) --  (axis cs:284.526,    +06.415605) ;
  \draw[constellation-boundary]  (axis cs:275.296,    +04.554892) --  (axis cs:275.29,    +05.054855) --  (axis cs:275.278,    +06.054781) --  (axis cs:275.275,    +06.304763) ;
  \draw[constellation-boundary]  (axis cs:275.296,    +04.554892) --  (axis cs:275.546,    +04.557919) --  (axis cs:276.546,    +04.570017) --  (axis cs:277.546,    +04.582094) --  (axis cs:277.921,    +04.586616) ;
  \draw[constellation-boundary]  (axis cs:277.939,    +03.086726) --  (axis cs:277.927,    +04.086653) --  (axis cs:277.921,    +04.586616) ;
  \draw[constellation-boundary]  (axis cs:275.314,    +03.055003) --  (axis cs:275.564,    +03.058030) --  (axis cs:276.564,    +03.070127) --  (axis cs:277.564,    +03.082204) --  (axis cs:277.939,    +03.086726) ;
  \draw[constellation-boundary]  (axis cs:275.351,    +00.055223) --  (axis cs:275.338,    +01.055150) --  (axis cs:275.326,    +02.055076) --  (axis cs:275.314,    +03.055003) ;
  \draw[constellation-boundary]  (axis cs:269.101,    -00.020646) --  (axis cs:269.601,    -00.014574) --  (axis cs:270.601,    -00.002426) --  (axis cs:271.601,    +00.009722) --  (axis cs:272.601,    +00.021868) --  (axis cs:273.601,    +00.034007) --  (axis cs:274.601,    +00.046135) --  (axis cs:275.351,    +00.055223) ;
  \draw[constellation-boundary]  (axis cs:269.15,    -04.020351) --  (axis cs:269.138,    -03.020425) --  (axis cs:269.125,    -02.020499) --  (axis cs:269.113,    -01.020573) --  (axis cs:269.101,    -00.020646) ;
  \draw[constellation-boundary]  (axis cs:269.15,    -04.020351) --  (axis cs:269.65,    -04.014278) --  (axis cs:270.65,    -04.002130) --  (axis cs:271.15,    -03.996056) ;
  \draw[constellation-boundary]  (axis cs:271.224,    -09.995606) --  (axis cs:271.211,    -08.995682) --  (axis cs:271.199,    -07.995758) --  (axis cs:271.186,    -06.995833) --  (axis cs:271.174,    -05.995907) --  (axis cs:271.162,    -04.995982) --  (axis cs:271.15,    -03.996056) ;
  \draw[constellation-boundary]  (axis cs:266.724,    -10.050234) --  (axis cs:267.724,    -10.038109) --  (axis cs:268.724,    -10.025973) --  (axis cs:269.724,    -10.013829) --  (axis cs:270.724,    -10.001681) --  (axis cs:271.224,    -09.995606) ;
  \draw[constellation-boundary]  (axis cs:266.745,    -11.716774) --  (axis cs:266.736,    -11.050158) --  (axis cs:266.724,    -10.050234) ;
  \draw[constellation-boundary]  (axis cs:265.494,    -11.731909) --  (axis cs:265.744,    -11.728884) --  (axis cs:266.745,    -11.716774) ;
  \draw[constellation-boundary]  (axis cs:265.494,    -11.731909) --  (axis cs:265.486,    -11.065293) --  (axis cs:265.474,    -10.065369) ;
  \draw[constellation-boundary]  (axis cs:265.474,    -10.065369) ;
  \draw[constellation-boundary]  (axis cs:265.8,    -16.061881) --  (axis cs:266.8,    -16.049770) --  (axis cs:267.8,    -16.037644) --  (axis cs:268.801,    -16.025506) --  (axis cs:269.801,    -16.013362) --  (axis cs:270.801,    -16.001214) --  (axis cs:271.8,    -15.989065) --  (axis cs:272.8,    -15.976921) --  (axis cs:273.8,    -15.964784) --  (axis cs:274.8,    -15.952658) --  (axis cs:275.799,    -15.940546) --  (axis cs:276.799,    -15.928454) --  (axis cs:277.799,    -15.916383) --  (axis cs:278.798,    -15.904338) --  (axis cs:279.797,    -15.892323) --  (axis cs:280.797,    -15.880340) --  (axis cs:281.796,    -15.868395) --  (axis cs:282.795,    -15.856490) --  (axis cs:283.795,    -15.844629) --  (axis cs:284.794,    -15.832816) ;
  \draw[constellation-boundary]  (axis cs:266.002,    -30.060661) --  (axis cs:265.986,    -29.060758) --  (axis cs:265.97,    -28.060853) --  (axis cs:265.955,    -27.060946) --  (axis cs:265.94,    -26.061037) --  (axis cs:265.925,    -25.061127) --  (axis cs:265.91,    -24.061216) --  (axis cs:265.896,    -23.061303) --  (axis cs:265.882,    -22.061389) --  (axis cs:265.868,    -21.061473) --  (axis cs:265.854,    -20.061557) --  (axis cs:265.84,    -19.061639) --  (axis cs:265.827,    -18.061721) --  (axis cs:265.813,    -17.061801) --  (axis cs:265.8,    -16.061881) ;
  \draw[constellation-boundary]  (axis cs:265.001,    -30.072758) --  (axis cs:266.002,    -30.060661) --  (axis cs:267.002,    -30.048547) --  (axis cs:268.003,    -30.036418) --  (axis cs:269.003,    -30.024279) --  (axis cs:269.503,    -30.018207) ;
  \draw[constellation-boundary]  (axis cs:320.188,    +20.029074) --  (axis cs:320.184,    +20.529057) --  (axis cs:320.175,    +21.529021) --  (axis cs:320.166,    +22.528985) --  (axis cs:320.156,    +23.528948) --  (axis cs:320.152,    +24.028930) ;
  \draw[constellation-boundary]  (axis cs:320.152,    +24.028930) --  (axis cs:320.403,    +24.030903) --  (axis cs:321.407,    +24.038695) --  (axis cs:322.411,    +24.046323) --  (axis cs:322.662,    +24.048204) ;
  \draw[constellation-boundary]  (axis cs:322.662,    +24.048204) --  (axis cs:322.657,    +24.548187) --  (axis cs:322.648,    +25.548153) --  (axis cs:322.639,    +26.548118) --  (axis cs:322.63,    +27.548083) --  (axis cs:322.62,    +28.548047) ;
  \draw[constellation-boundary]  (axis cs:363.734,    +13.195186) --  (axis cs:363.734,    +13.695186) --  (axis cs:363.735,    +14.695186) --  (axis cs:363.736,    +15.695186) --  (axis cs:363.736,    +16.695186) --  (axis cs:363.737,    +17.695186) --  (axis cs:363.738,    +18.695185) --  (axis cs:363.738,    +19.695185) --  (axis cs:363.739,    +20.695185) --  (axis cs:363.74,    +21.695185) --  (axis cs:363.741,    +22.695184) ;
  \draw[constellation-boundary]  (axis cs:361.603,    +13.196028) --  (axis cs:362.606,    +13.195751) --  (axis cs:363.609,    +13.195263) --  (axis cs:363.734,    +13.195186) ;
  \draw[constellation-boundary]  (axis cs:361.603,    +10.696028) --  (axis cs:361.603,    +11.696028) --  (axis cs:361.603,    +12.696028) --  (axis cs:361.603,    +13.196028) ;
  \draw[constellation-boundary]  (axis cs:359.097,    +10.695789) --  (axis cs:359.598,    +10.695943) --  (axis cs:360.6,    +10.696092) --  (axis cs:361.603,    +10.696028) ;
  \draw[constellation-boundary]  (axis cs:359.098,    +08.195789) --  (axis cs:359.098,    +08.695789) --  (axis cs:359.098,    +09.695789) --  (axis cs:359.097,    +10.695789) ;
  \draw[constellation-boundary]  (axis cs:342.821,    +08.162161) --  (axis cs:343.573,    +08.164916) --  (axis cs:344.574,    +08.168413) --  (axis cs:345.576,    +08.171706) --  (axis cs:346.577,    +08.174793) --  (axis cs:347.579,    +08.177675) --  (axis cs:348.581,    +08.180350) --  (axis cs:349.582,    +08.182818) --  (axis cs:350.584,    +08.185077) --  (axis cs:351.586,    +08.187128) --  (axis cs:352.587,    +08.188968) --  (axis cs:353.589,    +08.190599) --  (axis cs:354.591,    +08.192018) --  (axis cs:355.592,    +08.193227) --  (axis cs:356.594,    +08.194224) --  (axis cs:357.596,    +08.195009) --  (axis cs:358.597,    +08.195582) --  (axis cs:359.098,    +08.195789) ;
  \draw[constellation-boundary]  (axis cs:351.71,    -44.312751) --  (axis cs:351.707,    -43.312755) --  (axis cs:351.703,    -42.312759) --  (axis cs:351.699,    -41.312762) --  (axis cs:351.696,    -40.312765) --  (axis cs:351.693,    -39.312768) --  (axis cs:351.69,    -38.312772) --  (axis cs:351.686,    -37.312775) --  (axis cs:351.683,    -36.312778) ;
  \draw[constellation-boundary]  (axis cs:382.866,    +28.645431) --  (axis cs:382.872,    +29.645417) --  (axis cs:382.879,    +30.645404) --  (axis cs:382.885,    +31.645390) --  (axis cs:382.891,    +32.645375) --  (axis cs:382.898,    +33.645360) --  (axis cs:382.904,    +34.645345) --  (axis cs:382.911,    +35.645330) ;
  \draw[constellation-boundary]  (axis cs:382.866,    +28.645431) --  (axis cs:383.747,    +28.641362) --  (axis cs:384.753,    +28.636527) ;
  \draw[constellation-boundary]  (axis cs:290.121,    +19.398292) --  (axis cs:290.111,    +20.231564) --  (axis cs:290.097,    +21.231489) --  (axis cs:290.096,    +21.314816) ;
  \draw[constellation-boundary]  (axis cs:290.121,    +19.398292) --  (axis cs:290.372,    +19.401156) --  (axis cs:291.373,    +19.412567) --  (axis cs:292.374,    +19.423903) --  (axis cs:293.376,    +19.435161) --  (axis cs:294.378,    +19.446338) --  (axis cs:295.379,    +19.457430) --  (axis cs:296.381,    +19.468433) --  (axis cs:297.383,    +19.479345) --  (axis cs:298.385,    +19.490161) --  (axis cs:298.886,    +19.495533) ;
  \draw[constellation-boundary]  (axis cs:298.886,    +19.495533) --  (axis cs:298.876,    +20.328812) --  (axis cs:298.863,    +21.328747) --  (axis cs:298.86,    +21.578730) ;
  \draw[constellation-boundary]  (axis cs:298.86,    +21.578730) --  (axis cs:299.361,    +21.584078) --  (axis cs:300.364,    +21.594697) --  (axis cs:301.366,    +21.605211) --  (axis cs:302.368,    +21.615618) --  (axis cs:303.371,    +21.625913) --  (axis cs:304.373,    +21.636093) --  (axis cs:305.125,    +21.643651) ;
  \draw[constellation-boundary]  (axis cs:284.794,    -15.832816) --  (axis cs:284.781,    -14.832890) --  (axis cs:284.768,    -13.832965) --  (axis cs:284.756,    -12.833038) --  (axis cs:284.744,    -11.833111) --  (axis cs:284.731,    -10.833183) --  (axis cs:284.719,    -09.833255) --  (axis cs:284.707,    -08.833327) --  (axis cs:284.695,    -07.833398) --  (axis cs:284.683,    -06.833469) --  (axis cs:284.671,    -05.833539) --  (axis cs:284.659,    -04.833609) --  (axis cs:284.647,    -03.833679) ;
  \draw[constellation-boundary]  (axis cs:275.55,    -15.943573) --  (axis cs:275.536,    -14.943652) --  (axis cs:275.524,    -13.943730) --  (axis cs:275.511,    -12.943807) --  (axis cs:275.498,    -11.943884) --  (axis cs:275.485,    -10.943961) --  (axis cs:275.473,    -09.944036) --  (axis cs:275.46,    -08.944112) --  (axis cs:275.448,    -07.944187) --  (axis cs:275.436,    -06.944261) --  (axis cs:275.424,    -05.944336) --  (axis cs:275.411,    -04.944410) --  (axis cs:275.399,    -03.944483) ;
  \draw[constellation-boundary]  (axis cs:275.399,    -03.944483) --  (axis cs:275.649,    -03.941456) --  (axis cs:276.649,    -03.929361) --  (axis cs:277.649,    -03.917286) --  (axis cs:278.649,    -03.905237) --  (axis cs:279.648,    -03.893217) --  (axis cs:280.648,    -03.881230) --  (axis cs:281.648,    -03.869279) --  (axis cs:282.648,    -03.857368) --  (axis cs:283.647,    -03.845500) --  (axis cs:284.647,    -03.833679) ;
  

\node[constellation-label] at (axis cs:{283,35})   {\huge \textls{Lyra}};
\node[constellation-label] at (axis cs:{280,-12})   {\large \textls{Scutum}};
\node[constellation-label][rotate=-68] at (axis cs:{273.3,-7.3})  {\Large \textls{Serpens Cauda}};
\node[constellation-label] at (axis cs:{280,-38})  {{Corona Australis}};

\node[constellation-label] at (axis cs:{310,23})  {\large \textls{Vulpecula}};

\node[constellation-label] at (axis cs:{287,-28})   {\huge \textls{Sagittarius}};
\node[constellation-label,rotate=-90] at (axis cs:{273,8})   {\textls{Ophiuchus}};
\node[constellation-label] at (axis cs:{295,3})  {\huge \textls{Aquila}};
\node[constellation-label] at (axis cs:{275,19})   {\large \textls{Hercules}};
\node[constellation-label] at (axis cs:{291,18})  {\large \textls{Sagitta}};
\node[constellation-label] at (axis cs:{370,-15})  {\huge \textls{Cetus}};
\node[constellation-label] at (axis cs:{365,6})  {\Huge \textls{Pisces}};
\node[constellation-label] at (axis cs:{370,-35}) {\LARGE \textls{Sculptor}};
\node[constellation-label] at (axis cs:{362,37})  {\Large \textls{Andromeda}};  
\node[constellation-label] at (axis cs:{342,22})  {\Huge \textls{Pegasus}};

%
\node[constellation-label] at (axis cs:{314,-22})  {\huge \textls{Capricornus}};  
\node[constellation-label] at (axis cs:{339,-15})  {\huge \textls{Aquarius}};  
\node[constellation-label] at (axis cs:{310,35})  {\huge \textls{Cygnus}};  
\node[constellation-label] at (axis cs:{337,38})  {\large \textls{Lacerta}};  
\node[constellation-label][rotate=-75] at (axis cs:{318,7.5})  {\large \textls{Equuleus}};  
\node[constellation-label] at (axis cs:{312,17.5})  {\large \textls{Delphinus}};  

\node[constellation-label] at (axis cs:{340,-28})  {\Large \textls{Austrinus}};  
  \node[constellation-label] at (axis cs:{335,-31})  {\Large \textls{Piscis}};  

  \node[constellation-label][rotate=-75] at (axis cs:{315,-35})  {\Large {Microscopium}};  
\node[constellation-label] at (axis cs:{335,-38})  {\large \textls{Grus}};  




\end{axis}


% Nebulae and nebulae names


\begin{axis}[name=NGC,axis lines=none]

%\node[NGC,pin={[pin distance=-1.2\onedegree,NGC-label]90:{ 6522}}] at (axis cs:270.900,-30.033) {\tikz\pgfuseplotmark{GC};}; %  8.6
\node[NGC] at (axis cs:270.900,-30.033) {\tikz\pgfuseplotmark{GC};}; %  8.6



\node[NGC,pin={[pin distance=-1.2\onedegree,NGC-label]180:{ 6572}}] at (axis cs:273.025,6.850) {\tikz\pgfuseplotmark{PN};}; %  9. 
\node[NGC,pin={[pin distance=-1.2\onedegree,NGC-label]90:{ 6604}}] at (axis cs:274.525,-12.233) {\tikz\pgfuseplotmark{CN};}; %  6.5
\node[NGC,pin={[pin distance=-1.2\onedegree,NGC-label]0:{ 6507}}] at (axis cs:269.900,-17.400) {\tikz\pgfuseplotmark{OC};}; % 10. 
\node[NGC,pin={[pin distance=-1.2\onedegree,NGC-label]-90:{ 6520}}] at (axis cs:270.850,-27.900) {\tikz\pgfuseplotmark{OC};}; %  8. 
\node[NGC,pin={[pin distance=-1.2\onedegree,NGC-label]00:{ 6530}}] at (axis cs:271.200,-24.333) {\tikz\pgfuseplotmark{OC};}; %  4.6
\node[NGC,pin={[pin distance=-1.2\onedegree,NGC-label]90:{ 6546}}] at (axis cs:271.800,-23.333) {\tikz\pgfuseplotmark{OC};}; %  8.0
\node[NGC,pin={[pin distance=-1.2\onedegree,NGC-label]90:{ 6568}}] at (axis cs:273.200,-21.600) {\tikz\pgfuseplotmark{OC};}; %  9. 
\node[NGC,pin={[pin distance=-1.2\onedegree,NGC-label]90:{ 6583}}] at (axis cs:273.950,-22.133) {\tikz\pgfuseplotmark{OC};}; % 10. 
\node[NGC,pin={[pin distance=-1.2\onedegree,NGC-label]-90:{ 6595}}] at (axis cs:274.250,-19.883) {\tikz\pgfuseplotmark{OC};}; %  7. 
\node[NGC,pin={[pin distance=-1.2\onedegree,NGC-label]90:{ 6605}}] at (axis cs:274.275,-14.967) {\tikz\pgfuseplotmark{OC};}; %  6. 
\node[NGC,pin={[pin distance=-1.2\onedegree,NGC-label]-90:{ 6625}}] at (axis cs:275.800,-12.050) {\tikz\pgfuseplotmark{OC};}; %  9. 
\node[NGC,pin={[pin distance=-1.2\onedegree,NGC-label]-90:{ 6633}}] at (axis cs:276.925,6.567) {\tikz\pgfuseplotmark{OC};}; %  4.6
\node[NGC,pin={[pin distance=-1.2\onedegree,NGC-label]180:{ 6647}}] at (axis cs:277.875,-17.350) {\tikz\pgfuseplotmark{OC};}; %  8. 
\node[NGC,pin={[pin distance=-1.2\onedegree,NGC-label]-90:{ 6645}}] at (axis cs:278.150,-16.900) {\tikz\pgfuseplotmark{OC};}; %  9. 
\node[NGC,pin={[pin distance=-1.2\onedegree,NGC-label]00:{ 6649}}] at (axis cs:278.375,-10.400) {\tikz\pgfuseplotmark{OC};}; %  8.9
\node[NGC,pin={[pin distance=-1.2\onedegree,NGC-label]90:{ 6664}}] at (axis cs:279.175,-8.217) {\tikz\pgfuseplotmark{OC};}; %  7.8
\node[NGC,pin={[pin distance=-1.2\onedegree,NGC-label]180:{I4756}}] at (axis cs:279.750,5.450) {\tikz\pgfuseplotmark{OC};}; %  5. 
\node[NGC,pin={[pin distance=-1.2\onedegree,NGC-label]00:{ 6683}}] at (axis cs:280.550,-6.283) {\tikz\pgfuseplotmark{OC};}; % 10. 
\node[NGC,pin={[pin distance=-1.2\onedegree,NGC-label]0:{ 6704}}] at (axis cs:282.725,-5.200) {\tikz\pgfuseplotmark{OC};}; %  9.2
\node[NGC,pin={[pin distance=-1.2\onedegree,NGC-label]90:{ 6709}}] at (axis cs:282.875,10.350) {\tikz\pgfuseplotmark{OC};}; %  6.7
\node[NGC,pin={[pin distance=-1.2\onedegree,NGC-label]-90:{ 6716}}] at (axis cs:283.650,-19.883) {\tikz\pgfuseplotmark{OC};}; %  6.9
\node[NGC,pin={[pin distance=-1.2\onedegree,NGC-label]90:{ 6738}}] at (axis cs:285.350,11.600) {\tikz\pgfuseplotmark{OC};}; %  8. 
\node[NGC,pin={[pin distance=-1.2\onedegree,NGC-label]90:{ 6755}}] at (axis cs:286.950,4.233) {\tikz\pgfuseplotmark{OC};}; %  7.5
\node[NGC,pin={[pin distance=-1.2\onedegree,NGC-label]00:{ 6517}}] at (axis cs:270.450,-8.967) {\tikz\pgfuseplotmark{GC};}; % 10.3

\node[NGC] at (axis cs:270.900,-30.033) {\tikz\pgfuseplotmark{GC};}; %  8.6

\node[NGC,pin={[pin distance=-1.2\onedegree,NGC-label]90:{ 6535}}] at (axis cs:270.950,-0.300) {\tikz\pgfuseplotmark{GC};}; % 10.6
\node[NGC,pin={[pin distance=-0.8\onedegree,NGC-label]0:{ 6522/8}}] at (axis cs:271.200,-30.050) {\tikz\pgfuseplotmark{GC};}; %  9.5
\node[NGC,pin={[pin distance=-1.2\onedegree,NGC-label]0:{ 6539}}] at (axis cs:271.200,-7.583) {\tikz\pgfuseplotmark{GC};}; %  9.6
\node[NGC,pin={[pin distance=-1.2\onedegree,NGC-label]0:{ 6544}}] at (axis cs:271.825,-25) {\tikz\pgfuseplotmark{GC};}; %  8.3
\node[NGC,pin={[pin distance=-1.2\onedegree,NGC-label]0:{ 6553}}] at (axis cs:272.325,-25.900) {\tikz\pgfuseplotmark{GC};}; %  8.3
\node[NGC,pin={[pin distance=-1.2\onedegree,NGC-label]90:{ 6569}}] at (axis cs:273.400,-31.833) {\tikz\pgfuseplotmark{GC};}; %  8.7
\node[NGC,pin={[pin distance=-1.2\onedegree,NGC-label]00:{ 6624}}] at (axis cs:275.925,-30.367) {\tikz\pgfuseplotmark{GC};}; %  8.3
\node[NGC,pin={[pin distance=-1.2\onedegree,NGC-label]90:{ 6638}}] at (axis cs:277.725,-25.500) {\tikz\pgfuseplotmark{GC};}; %  9.2
\node[NGC,pin={[pin distance=-1.2\onedegree,NGC-label]-90:{ 6642}}] at (axis cs:277.975,-23.483) {\tikz\pgfuseplotmark{GC};}; %  8.8
\node[NGC,pin={[pin distance=-1.2\onedegree,NGC-label]270:{ 6652}}] at (axis cs:278.950,-32.983) {\tikz\pgfuseplotmark{GC};}; %  8.9
\node[NGC,pin={[pin distance=-1.2\onedegree,NGC-label]90:{ 6712}}] at (axis cs:283.275,-8.700) {\tikz\pgfuseplotmark{GC};}; %  8.2
\node[NGC,pin={[pin distance=-1.2\onedegree,NGC-label]-90:{ 6723}}] at (axis cs:284.900,-36.633) {\tikz\pgfuseplotmark{GC};}; %  7.3
\node[NGC,pin={[pin distance=-1.2\onedegree,NGC-label]90:{ 6760}}] at (axis cs:287.800,1.033) {\tikz\pgfuseplotmark{GC};}; %  9.1


\node[NGC,pin={[pin distance=-1.2\onedegree,NGC-label]180:{ 6822}}] at (axis cs:296.225,-14.800) {\tikz\pgfuseplotmark{GAL};}; %  9. 
%\node[NGC,pin={[pin distance=-1.2\onedegree,NGC-label]90:{I4895}}] at (axis cs:296.250,-14.800) {\tikz\pgfuseplotmark{GAL};}; %  8. 
\node[NGC,pin={[pin distance=-1.2\onedegree,NGC-label]90:{ 7217}}] at (axis cs:331.975,31.367) {\tikz\pgfuseplotmark{GAL};}; % 10.2
\node[NGC,pin={[pin distance=-1.2\onedegree,NGC-label]90:{ 7314}}] at (axis cs:338.950,-26.050) {\tikz\pgfuseplotmark{GAL};}; % 10.9
\node[NGC,pin={[pin distance=-1.2\onedegree,NGC-label]90:{ 7331}}] at (axis cs:339.275,34.417) {\tikz\pgfuseplotmark{GAL};}; %  9.5
\node[NGC,pin={[pin distance=-1.2\onedegree,NGC-label]-90:{ 7410}}] at (axis cs:343.750,-39.667) {\tikz\pgfuseplotmark{GAL};}; % 10.4
\node[NGC,pin={[pin distance=-1.2\onedegree,NGC-label]90:{I1459}}] at (axis cs:344.300,-36.467) {\tikz\pgfuseplotmark{GAL};}; % 10.0
\node[NGC,pin={[pin distance=-1.2\onedegree,NGC-label]90:{ 7457}}] at (axis cs:345.250,30.150) {\tikz\pgfuseplotmark{GAL};}; % 10.8
\node[NGC,pin={[pin distance=-1.2\onedegree,NGC-label]90:{ 7507}}] at (axis cs:348.025,-28.533) {\tikz\pgfuseplotmark{GAL};}; % 10.4
\node[NGC,pin={[pin distance=-1.2\onedegree,NGC-label]00:{ 7606}}] at (axis cs:349.775,-8.483) {\tikz\pgfuseplotmark{GAL};}; % 10.8
\node[NGC,pin={[pin distance=-1.2\onedegree,NGC-label]90:{I5332}}] at (axis cs:353.625,-36.100) {\tikz\pgfuseplotmark{GAL};}; % 10.6
\node[NGC,pin={[pin distance=-1.2\onedegree,NGC-label]180:{ 7727}}] at (axis cs:354.975,-12.300) {\tikz\pgfuseplotmark{GAL};}; % 10.7
\node[NGC,pin={[pin distance=-1.2\onedegree,NGC-label]90:{ 7793}}] at (axis cs:359.450,-32.583) {\tikz\pgfuseplotmark{GAL};}; %  9.1
\node[NGC,pin={[pin distance=-1.2\onedegree,NGC-label]180:{ 6790}}] at (axis cs:290.800,1.517) {\tikz\pgfuseplotmark{PN};}; % 10. 
\node[NGC,pin={[pin distance=-1.2\onedegree,NGC-label]-90:{ 6818}}] at (axis cs:296,-14.150) {\tikz\pgfuseplotmark{PN};}; % 10. 
\node[NGC,pin={[pin distance=-1.2\onedegree,NGC-label]90:{ 7009}}] at (axis cs:316.050,-11.367) {\tikz\pgfuseplotmark{PN};}; %  8.
\node[NGC,pin={[pin distance=-1.2\onedegree,NGC-label]00:{ 6791}}] at (axis cs:290.175,37.850) {\tikz\pgfuseplotmark{OC};}; %  9.5
\node[NGC,pin={[pin distance=-1.2\onedegree,NGC-label]90:{ 6802}}] at (axis cs:292.650,20.267) {\tikz\pgfuseplotmark{OC};}; %  8.8
\node[NGC,pin={[pin distance=-1.2\onedegree,NGC-label]180:{ 6823}}] at (axis cs:295.775,23.300) {\tikz\pgfuseplotmark{OC};}; %  7.1
\node[NGC,pin={[pin distance=-1.2\onedegree,NGC-label]180:{ 6830}}] at (axis cs:297.750,23.067) {\tikz\pgfuseplotmark{OC};}; %  7.9
\node[NGC,pin={[pin distance=-1.2\onedegree,NGC-label]90:{ 6834}}] at (axis cs:298.050,29.417) {\tikz\pgfuseplotmark{OC};}; %  7.8
\node[NGC,pin={[pin distance=-1.2\onedegree,NGC-label]90:{ 6871}}] at (axis cs:301.475,35.783) {\tikz\pgfuseplotmark{OC};}; %  5.2
\node[NGC,pin={[pin distance=-1.2\onedegree,NGC-label]90:{ 6883}}] at (axis cs:302.825,35.850) {\tikz\pgfuseplotmark{OC};}; %  8. 
\node[NGC,pin={[pin distance=-1.2\onedegree,NGC-label]90:{ 6882/5}}] at (axis cs:302.925,26.550) {\tikz\pgfuseplotmark{OC};}; %  8.1
%\node[NGC,pin={[pin distance=-1.2\onedegree,NGC-label]90:{ 6885}}] at (axis cs:303,26.483) {\tikz\pgfuseplotmark{OC};}; %  6. 
\node[NGC] at (axis cs:303,26.483) {\tikz\pgfuseplotmark{OC};}; %  6. 
\node[NGC,pin={[pin distance=-1.2\onedegree,NGC-label]180:{I4996}}] at (axis cs:304.125,37.633) {\tikz\pgfuseplotmark{OC};}; %  7.3
\node[NGC,pin={[pin distance=-1.2\onedegree,NGC-label]90:{ 6940}}] at (axis cs:308.650,28.300) {\tikz\pgfuseplotmark{OC};}; %  6.3
\node[NGC,pin={[pin distance=-1.2\onedegree,NGC-label]90:{ 7063}}] at (axis cs:321.100,36.500) {\tikz\pgfuseplotmark{OC};}; %  7.0
\node[NGC,pin={[pin distance=-1.2\onedegree,NGC-label]90:{ 6934}}] at (axis cs:308.550,7.400) {\tikz\pgfuseplotmark{GC};}; %  8.9
\node[NGC,pin={[pin distance=-1.2\onedegree,NGC-label]90:{ 7006}}] at (axis cs:315.375,16.183) {\tikz\pgfuseplotmark{GC};}; % 10.6











\node[Messier,pin={[pin distance=-0.8\onedegree,Messier-label]90:{M29}}] at (axis cs:305.975,38.533) {\tikz\pgfuseplotmark{OC};}; %  6.6
\node[Messier,pin={[pin distance=-0.8\onedegree,Messier-label]90:{M27}}] at (axis cs:299.900,22.717) {\tikz\pgfuseplotmark{PN};}; %  8.1
\node[Messier,pin={[pin distance=-0.8\onedegree,Messier-label]90:{M71}}] at (axis cs:298.450,18.783) {\tikz\pgfuseplotmark{GC};}; %  8.3
\node[Messier,pin={[pin distance=-0.8\onedegree,Messier-label]90:{M15}}] at (axis cs:322.500,12.167) {\tikz\pgfuseplotmark{GC};}; %  6.4
\node[Messier,pin={[pin distance=-0.8\onedegree,Messier-label]90:{M2}}] at (axis cs:323.375,-0.817) {\tikz\pgfuseplotmark{GC};}; %  6.5
\node[Messier,pin={[pin distance=-0.8\onedegree,Messier-label]-90:{M72}}] at (axis cs:313.375,-12.533) {\tikz\pgfuseplotmark{GC};}; %  9.4
\node[Messier,pin={[pin distance=-0.8\onedegree,Messier-label]90:{M73}}] at (axis cs:314.750,-12.633) {\tikz\pgfuseplotmark{OC};}; %  9. 
\node[Messier,pin={[pin distance=-0.8\onedegree,Messier-label]90:{M75}}] at (axis cs:301.525,-21.917) {\tikz\pgfuseplotmark{GC};}; %  8.6
\node[Messier,pin={[pin distance=-0.8\onedegree,Messier-label]90:{M30}}] at (axis cs:325.100,-23.183) {\tikz\pgfuseplotmark{GC};}; %  7.5
\node[Messier,pin={[pin distance=-0.8\onedegree,Messier-label]90:{M55}}] at (axis cs:295,-30.967) {\tikz\pgfuseplotmark{GC};}; %  7.0



\node[Messier,pin={[pin distance=-0.8\onedegree,Messier-label]90:{M57}}] at (axis cs:283.400,33.033) {\tikz\pgfuseplotmark{PN};}; %  9.0
\node[Messier,pin={[pin distance=-0.8\onedegree,Messier-label]90:{M56}}] at (axis cs:289.150,30.183) {\tikz\pgfuseplotmark{GC};}; %  8.3
\node[Messier,pin={[pin distance=-0.8\onedegree,Messier-label]90:{M11}}] at (axis cs:282.775,-6.267) {\tikz\pgfuseplotmark{OC};}; %  5.8
\node[Messier,pin={[pin distance=-0.8\onedegree,Messier-label]90:{M26}}] at (axis cs:281.300,-9.400) {\tikz\pgfuseplotmark{OC};}; %  8.0
\node[Messier,pin={[pin distance=-0.8\onedegree,Messier-label]180:{M16}}] at (axis cs:274.700,-13.783) {\tikz\pgfuseplotmark{CN};}; %  6.0
\node[Messier,pin={[pin distance=-0.8\onedegree,Messier-label]0:{M17}}] at (axis cs:275.200,-16.183) {\tikz\pgfuseplotmark{CN};}; %  6.0
\node[Messier,pin={[pin distance=-0.8\onedegree,Messier-label]90:{M18}}] at (axis cs:274.975,-17.133) {\tikz\pgfuseplotmark{OC};}; %  6.9
\node[Messier,pin={[pin distance=-0.8\onedegree,Messier-label]180:{M9}}] at (axis cs:259.800,-18.517) {\tikz\pgfuseplotmark{GC};}; %  7.9
\node[Messier,pin={[pin distance=-0.8\onedegree,Messier-label]90:{M23}}] at (axis cs:269.200,-19.017) {\tikz\pgfuseplotmark{OC};}; %  5.5
\node[Messier,pin={[pin distance=-0.8\onedegree,Messier-label]-90:{M21}}] at (axis cs:271.150,-22.500) {\tikz\pgfuseplotmark{OC};}; %  5.9
\node[Messier,pin={[pin distance=-0.8\onedegree,Messier-label]180:{M20}}] at (axis cs:270.575,-23.033) {\tikz\pgfuseplotmark{CN};}; %  6.3
\node[Messier,pin={[pin distance=-0.8\onedegree,Messier-label]0:{M22}}] at (axis cs:279.100,-23.900) {\tikz\pgfuseplotmark{GC};}; %  5.1
\node[Messier,pin={[pin distance=-0.8\onedegree,Messier-label]90:{M8}}] at (axis cs:270.950,-24.383) {\tikz\pgfuseplotmark{EN};}; %  5.8
\node[Messier,pin={[pin distance=-0.8\onedegree,Messier-label]-90:{M28}}] at (axis cs:276.125,-24.867) {\tikz\pgfuseplotmark{GC};}; %  6.9
\node[Messier,pin={[pin distance=-0.8\onedegree,Messier-label]180:{M54}}] at (axis cs:283.775,-30.483) {\tikz\pgfuseplotmark{GC};}; %  7.7
\node[Messier,pin={[pin distance=-0.8\onedegree,Messier-label]-90:{M69}}] at (axis cs:277.850,-32.350) {\tikz\pgfuseplotmark{GC};}; %  7.7
\node[Messier,pin={[pin distance=-0.8\onedegree,Messier-label]90:{M70}}] at (axis cs:280.800,-32.300) {\tikz\pgfuseplotmark{GC};}; %  8.1
\node[Messier,pin={[pin distance=-0.8\onedegree,Messier-label]90:{M25}}] at (axis cs:277.900,-19.250) {\tikz\pgfuseplotmark{OC};}; %  4.6





\end{axis}

% Stars and star names
\begin{axis}[name=stars,axis lines=none]
\node[stars] at (axis cs:{360.100},{26.918}) {\tikz\pgfuseplotmark{m6c};};
\node[stars] at (axis cs:{360.456},{-3.027}) {\tikz\pgfuseplotmark{m5b};};
\node[stars] at (axis cs:{360.490},{-6.014}) {\tikz\pgfuseplotmark{m4cv};};
\node[stars] at (axis cs:{360.542},{27.082}) {\tikz\pgfuseplotmark{m6av};};
\node[stars] at (axis cs:{360.583},{-29.720}) {\tikz\pgfuseplotmark{m5b};};
\node[stars] at (axis cs:{360.601},{8.957}) {\tikz\pgfuseplotmark{m6c};};
\node[stars] at (axis cs:{360.624},{8.485}) {\tikz\pgfuseplotmark{m6a};};
\node[stars] at (axis cs:{360.740},{-20.046}) {\tikz\pgfuseplotmark{m6c};};
\node[stars] at (axis cs:{360.783},{-24.145}) {\tikz\pgfuseplotmark{m6c};};
\node[stars] at (axis cs:{360.935},{-17.336}) {\tikz\pgfuseplotmark{m5a};};
\node[stars] at (axis cs:{361.082},{-16.529}) {\tikz\pgfuseplotmark{m6a};};
\node[stars] at (axis cs:{361.085},{-29.269}) {\tikz\pgfuseplotmark{m6c};};
\node[stars] at (axis cs:{361.125},{-10.509}) {\tikz\pgfuseplotmark{m5bv};};
\node[stars] at (axis cs:{361.224},{34.660}) {\tikz\pgfuseplotmark{m6b};};
\node[stars] at (axis cs:{361.233},{26.649}) {\tikz\pgfuseplotmark{m6c};};
\node[stars] at (axis cs:{361.255},{27.675}) {\tikz\pgfuseplotmark{m6c};};
\node[stars] at (axis cs:{361.266},{-0.503}) {\tikz\pgfuseplotmark{m6c};};
\node[stars] at (axis cs:{361.334},{-5.707}) {\tikz\pgfuseplotmark{m5av};};
\node[stars] at (axis cs:{361.425},{13.396}) {\tikz\pgfuseplotmark{m6a};};
\node[stars] at (axis cs:{361.653},{29.021}) {\tikz\pgfuseplotmark{m6bvb};};
\node[stars] at (axis cs:{361.709},{-23.107}) {\tikz\pgfuseplotmark{m6c};};
\node[stars] at (axis cs:{361.826},{-17.387}) {\tikz\pgfuseplotmark{m6c};};
\node[stars] at (axis cs:{361.934},{-2.548}) {\tikz\pgfuseplotmark{m6cv};};
\node[stars] at (axis cs:{361.945},{-22.508}) {\tikz\pgfuseplotmark{m6b};};
\node[stars] at (axis cs:{362.015},{-33.529}) {\tikz\pgfuseplotmark{m6a};};
\node[stars] at (axis cs:{362.050},{-2.447}) {\tikz\pgfuseplotmark{m6cv};};
\node[stars] at (axis cs:{362.073},{-8.824}) {\tikz\pgfuseplotmark{m6b};};
\node[stars] at (axis cs:{362.097},{29.090}) {\tikz\pgfuseplotmark{m2bv};};
\node[stars] at (axis cs:{362.140},{-17.578}) {\tikz\pgfuseplotmark{m6bv};};
\node[stars] at (axis cs:{362.171},{36.627}) {\tikz\pgfuseplotmark{m6c};};
\node[stars] at (axis cs:{362.217},{25.463}) {\tikz\pgfuseplotmark{m6c};};
\node[stars] at (axis cs:{362.251},{28.247}) {\tikz\pgfuseplotmark{m6c};};
\node[stars] at (axis cs:{362.260},{18.212}) {\tikz\pgfuseplotmark{m6a};};
\node[stars] at (axis cs:{362.338},{-27.988}) {\tikz\pgfuseplotmark{m5cb};};
\node[stars] at (axis cs:{362.509},{11.146}) {\tikz\pgfuseplotmark{m6avb};};
\node[stars] at (axis cs:{362.579},{-5.248}) {\tikz\pgfuseplotmark{m6a};};
\node[stars] at (axis cs:{362.678},{-12.580}) {\tikz\pgfuseplotmark{m6b};};
\node[stars] at (axis cs:{362.816},{-15.468}) {\tikz\pgfuseplotmark{m5b};};
\node[stars] at (axis cs:{362.893},{-27.799}) {\tikz\pgfuseplotmark{m5c};};
\node[stars] at (axis cs:{362.933},{-35.133}) {\tikz\pgfuseplotmark{m5c};};
\node[stars] at (axis cs:{363.042},{-17.938}) {\tikz\pgfuseplotmark{m5c};};
\node[stars] at (axis cs:{363.309},{15.184}) {\tikz\pgfuseplotmark{m3avb};};
\node[stars] at (axis cs:{363.350},{26.987}) {\tikz\pgfuseplotmark{m6c};};
\node[stars] at (axis cs:{363.426},{-26.022}) {\tikz\pgfuseplotmark{m6bv};};
\node[stars] at (axis cs:{363.435},{-26.285}) {\tikz\pgfuseplotmark{m6b};};
\node[stars] at (axis cs:{363.510},{33.206}) {\tikz\pgfuseplotmark{m6c};};
\node[stars] at (axis cs:{363.615},{-7.780}) {\tikz\pgfuseplotmark{m5cv};};
\node[stars] at (axis cs:{363.651},{20.207}) {\tikz\pgfuseplotmark{m5av};};
\node[stars] at (axis cs:{363.660},{-18.933}) {\tikz\pgfuseplotmark{m4cv};};
\node[stars] at (axis cs:{363.727},{-9.570}) {\tikz\pgfuseplotmark{m6a};};
\node[stars] at (axis cs:{363.733},{22.284}) {\tikz\pgfuseplotmark{m6cv};};
\node[stars] at (axis cs:{363.743},{-34.904}) {\tikz\pgfuseplotmark{m6c};};
\node[stars] at (axis cs:{363.745},{8.821}) {\tikz\pgfuseplotmark{m6bvb};};
\node[stars] at (axis cs:{363.779},{31.536}) {\tikz\pgfuseplotmark{m6c};};
\node[stars] at (axis cs:{363.794},{27.283}) {\tikz\pgfuseplotmark{m6c};};
\node[stars] at (axis cs:{364.037},{-31.446}) {\tikz\pgfuseplotmark{m6a};};
\node[stars] at (axis cs:{364.142},{8.240}) {\tikz\pgfuseplotmark{m6b};};
\node[stars] at (axis cs:{364.273},{38.681}) {\tikz\pgfuseplotmark{m5av};};
\node[stars] at (axis cs:{364.386},{-19.051}) {\tikz\pgfuseplotmark{m6c};};
\node[stars] at (axis cs:{364.449},{1.689}) {\tikz\pgfuseplotmark{m6c};};
\node[stars] at (axis cs:{364.572},{11.206}) {\tikz\pgfuseplotmark{m6b};};
\node[stars] at (axis cs:{364.582},{36.785}) {\tikz\pgfuseplotmark{m5av};};
\node[stars] at (axis cs:{364.659},{31.517}) {\tikz\pgfuseplotmark{m6b};};
\node[stars] at (axis cs:{364.674},{-8.053}) {\tikz\pgfuseplotmark{m6c};};
\node[stars] at (axis cs:{364.857},{-8.824}) {\tikz\pgfuseplotmark{m4av};};
\node[stars] at (axis cs:{365.102},{30.936}) {\tikz\pgfuseplotmark{m6b};};
\node[stars] at (axis cs:{365.149},{8.190}) {\tikz\pgfuseplotmark{m5c};};
\node[stars] at (axis cs:{365.190},{32.911}) {\tikz\pgfuseplotmark{m6a};};
\node[stars] at (axis cs:{365.280},{37.969}) {\tikz\pgfuseplotmark{m5c};};
\node[stars] at (axis cs:{365.380},{-28.981}) {\tikz\pgfuseplotmark{m5c};};
\node[stars] at (axis cs:{365.443},{-20.058}) {\tikz\pgfuseplotmark{m6av};};
\node[stars] at (axis cs:{365.606},{13.482}) {\tikz\pgfuseplotmark{m6c};};
\node[stars] at (axis cs:{365.716},{-12.209}) {\tikz\pgfuseplotmark{m6cv};};
\node[stars] at (axis cs:{365.768},{-15.942}) {\tikz\pgfuseplotmark{m6c};};
\node[stars] at (axis cs:{366.124},{-2.219}) {\tikz\pgfuseplotmark{m6b};};
\node[stars] at (axis cs:{366.159},{14.315}) {\tikz\pgfuseplotmark{m6c};};
\node[stars] at (axis cs:{366.351},{1.940}) {\tikz\pgfuseplotmark{m6av};};
\node[stars] at (axis cs:{366.591},{-18.693}) {\tikz\pgfuseplotmark{m6cv};};
\node[stars] at (axis cs:{366.656},{-0.049}) {\tikz\pgfuseplotmark{m6c};};
\node[stars] at (axis cs:{366.811},{-25.547}) {\tikz\pgfuseplotmark{m6b};};
\node[stars] at (axis cs:{366.879},{16.025}) {\tikz\pgfuseplotmark{m6cb};};
\node[stars] at (axis cs:{366.982},{-33.007}) {\tikz\pgfuseplotmark{m5bv};};
\node[stars] at (axis cs:{366.994},{19.514}) {\tikz\pgfuseplotmark{m6c};};
\node[stars] at (axis cs:{367.012},{17.893}) {\tikz\pgfuseplotmark{m5bv};};
\node[stars] at (axis cs:{367.053},{16.445}) {\tikz\pgfuseplotmark{m6b};};
\node[stars] at (axis cs:{367.084},{10.190}) {\tikz\pgfuseplotmark{m6b};};
\node[stars] at (axis cs:{367.088},{-20.335}) {\tikz\pgfuseplotmark{m6c};};
\node[stars] at (axis cs:{367.111},{-39.915}) {\tikz\pgfuseplotmark{m5c};};
\node[stars] at (axis cs:{367.236},{36.899}) {\tikz\pgfuseplotmark{m6c};};
\node[stars] at (axis cs:{367.466},{-14.864}) {\tikz\pgfuseplotmark{m6c};};
\node[stars] at (axis cs:{367.510},{-3.957}) {\tikz\pgfuseplotmark{m6a};};
\node[stars] at (axis cs:{367.531},{29.752}) {\tikz\pgfuseplotmark{m5cv};};
\node[stars] at (axis cs:{367.594},{-23.788}) {\tikz\pgfuseplotmark{m5c};};
\node[stars] at (axis cs:{367.857},{33.581}) {\tikz\pgfuseplotmark{m6bb};};
\node[stars] at (axis cs:{368.099},{6.955}) {\tikz\pgfuseplotmark{m6avb};};
\node[stars] at (axis cs:{368.148},{20.294}) {\tikz\pgfuseplotmark{m5c};};
\node[stars] at (axis cs:{368.205},{28.280}) {\tikz\pgfuseplotmark{m6c};};
\node[stars] at (axis cs:{368.421},{-29.558}) {\tikz\pgfuseplotmark{m6a};};
\node[stars] at (axis cs:{368.731},{13.371}) {\tikz\pgfuseplotmark{m6c};};
\node[stars] at (axis cs:{368.812},{-3.593}) {\tikz\pgfuseplotmark{m5cv};};
\node[stars] at (axis cs:{368.887},{-0.506}) {\tikz\pgfuseplotmark{m6b};};
\node[stars] at (axis cs:{368.978},{13.207}) {\tikz\pgfuseplotmark{m6c};};
\node[stars] at (axis cs:{369.013},{-14.973}) {\tikz\pgfuseplotmark{m6c};};
\node[stars] at (axis cs:{369.029},{-22.842}) {\tikz\pgfuseplotmark{m6bv};};
\node[stars] at (axis cs:{369.197},{15.231}) {\tikz\pgfuseplotmark{m6bv};};
\node[stars] at (axis cs:{369.220},{33.719}) {\tikz\pgfuseplotmark{m4cvb};};
\node[stars] at (axis cs:{369.280},{24.014}) {\tikz\pgfuseplotmark{m6c};};
\node[stars] at (axis cs:{369.336},{-24.767}) {\tikz\pgfuseplotmark{m6a};};
\node[stars] at (axis cs:{369.338},{35.399}) {\tikz\pgfuseplotmark{m5c};};
\node[stars] at (axis cs:{369.377},{3.135}) {\tikz\pgfuseplotmark{m6c};};
\node[stars] at (axis cs:{369.639},{29.312}) {\tikz\pgfuseplotmark{m4c};};
\node[stars] at (axis cs:{369.703},{-25.108}) {\tikz\pgfuseplotmark{m6cb};};
\node[stars] at (axis cs:{369.832},{30.861}) {\tikz\pgfuseplotmark{m3c};};
\node[stars] at (axis cs:{369.841},{21.250}) {\tikz\pgfuseplotmark{m6bv};};
\node[stars] at (axis cs:{369.982},{21.438}) {\tikz\pgfuseplotmark{m5cb};};
\node[stars] at (axis cs:{370.119},{-16.517}) {\tikz\pgfuseplotmark{m6cb};};
\node[stars] at (axis cs:{370.137},{-23.805}) {\tikz\pgfuseplotmark{m6c};};
\node[stars] at (axis cs:{370.177},{-4.352}) {\tikz\pgfuseplotmark{m6bb};};
\node[stars] at (axis cs:{370.280},{39.459}) {\tikz\pgfuseplotmark{m5c};};
\node[stars] at (axis cs:{370.400},{24.629}) {\tikz\pgfuseplotmark{m6bv};};
\node[stars] at (axis cs:{370.679},{-38.463}) {\tikz\pgfuseplotmark{m6b};};
\node[stars] at (axis cs:{370.897},{-17.987}) {\tikz\pgfuseplotmark{m2bv};};
\node[stars] at (axis cs:{370.959},{-12.012}) {\tikz\pgfuseplotmark{m6b};};
\node[stars] at (axis cs:{371.048},{-10.609}) {\tikz\pgfuseplotmark{m5av};};
\node[stars] at (axis cs:{371.050},{-38.422}) {\tikz\pgfuseplotmark{m6b};};
\node[stars] at (axis cs:{371.185},{-22.006}) {\tikz\pgfuseplotmark{m5c};};
\node[stars] at (axis cs:{371.351},{-4.629}) {\tikz\pgfuseplotmark{m6c};};
\node[stars] at (axis cs:{371.370},{-12.881}) {\tikz\pgfuseplotmark{m6c};};
\node[stars] at (axis cs:{371.549},{-22.522}) {\tikz\pgfuseplotmark{m5c};};
\node[stars] at (axis cs:{371.637},{15.475}) {\tikz\pgfuseplotmark{m5cv};};
\node[stars] at (axis cs:{371.756},{11.974}) {\tikz\pgfuseplotmark{m6a};};
\node[stars] at (axis cs:{371.807},{19.579}) {\tikz\pgfuseplotmark{m6bv};};
\node[stars] at (axis cs:{371.835},{24.267}) {\tikz\pgfuseplotmark{m4bv};};
\node[stars] at (axis cs:{371.848},{6.741}) {\tikz\pgfuseplotmark{m6b};};
\node[stars] at (axis cs:{371.930},{-18.061}) {\tikz\pgfuseplotmark{m6a};};
\node[stars] at (axis cs:{372.004},{-21.722}) {\tikz\pgfuseplotmark{m6a};};
\node[stars] at (axis cs:{372.073},{7.300}) {\tikz\pgfuseplotmark{m6b};};
\node[stars] at (axis cs:{372.096},{5.280}) {\tikz\pgfuseplotmark{m6ab};};
\node[stars] at (axis cs:{372.171},{7.585}) {\tikz\pgfuseplotmark{m4c};};
\node[stars] at (axis cs:{372.245},{16.941}) {\tikz\pgfuseplotmark{m5b};};
\node[stars] at (axis cs:{372.308},{-24.137}) {\tikz\pgfuseplotmark{m6b};};
\node[stars] at (axis cs:{372.357},{-13.561}) {\tikz\pgfuseplotmark{m6a};};
\node[stars] at (axis cs:{372.389},{-23.362}) {\tikz\pgfuseplotmark{m6c};};
\node[stars] at (axis cs:{372.472},{27.710}) {\tikz\pgfuseplotmark{m6ab};};
\node[stars] at (axis cs:{372.532},{-10.644}) {\tikz\pgfuseplotmark{m5cv};};
\node[stars] at (axis cs:{372.826},{3.385}) {\tikz\pgfuseplotmark{m6c};};
\node[stars] at (axis cs:{373.169},{-24.006}) {\tikz\pgfuseplotmark{m5c};};
\node[stars] at (axis cs:{373.252},{-1.144}) {\tikz\pgfuseplotmark{m5a};};
\node[stars] at (axis cs:{373.368},{37.418}) {\tikz\pgfuseplotmark{m6b};};
\node[stars] at (axis cs:{373.573},{-8.741}) {\tikz\pgfuseplotmark{m6c};};
\node[stars] at (axis cs:{373.647},{19.188}) {\tikz\pgfuseplotmark{m6a};};
\node[stars] at (axis cs:{373.742},{23.628}) {\tikz\pgfuseplotmark{m5cvb};};
\node[stars] at (axis cs:{373.811},{24.557}) {\tikz\pgfuseplotmark{m6cv};};
\node[stars] at (axis cs:{373.927},{-7.347}) {\tikz\pgfuseplotmark{m6b};};
\node[stars] at (axis cs:{373.981},{-27.776}) {\tikz\pgfuseplotmark{m6b};};
\node[stars] at (axis cs:{373.994},{27.209}) {\tikz\pgfuseplotmark{m6b};};
\node[stars] at (axis cs:{374.006},{-11.266}) {\tikz\pgfuseplotmark{m5c};};
\node[stars] at (axis cs:{374.038},{13.952}) {\tikz\pgfuseplotmark{m6c};};
\node[stars] at (axis cs:{374.188},{38.499}) {\tikz\pgfuseplotmark{m4b};};
\node[stars] at (axis cs:{374.302},{23.418}) {\tikz\pgfuseplotmark{m4c};};
\node[stars] at (axis cs:{374.459},{28.992}) {\tikz\pgfuseplotmark{m5c};};
\node[stars] at (axis cs:{374.477},{13.696}) {\tikz\pgfuseplotmark{m6c};};
\node[stars] at (axis cs:{374.559},{33.951}) {\tikz\pgfuseplotmark{m6b};};
\node[stars] at (axis cs:{374.579},{21.404}) {\tikz\pgfuseplotmark{m6c};};
\node[stars] at (axis cs:{374.652},{-29.357}) {\tikz\pgfuseplotmark{m4cv};};
\node[stars] at (axis cs:{374.683},{-11.380}) {\tikz\pgfuseplotmark{m6a};};
\node[stars] at (axis cs:{374.957},{6.483}) {\tikz\pgfuseplotmark{m6cv};};
\node[stars] at (axis cs:{375.326},{-38.916}) {\tikz\pgfuseplotmark{m6a};};
\node[stars] at (axis cs:{375.411},{-16.265}) {\tikz\pgfuseplotmark{m6c};};
\node[stars] at (axis cs:{375.610},{-31.552}) {\tikz\pgfuseplotmark{m6av};};
\node[stars] at (axis cs:{375.705},{31.804}) {\tikz\pgfuseplotmark{m5c};};
\node[stars] at (axis cs:{375.736},{7.890}) {\tikz\pgfuseplotmark{m4c};};
\node[stars] at (axis cs:{375.761},{-4.837}) {\tikz\pgfuseplotmark{m5c};};
\node[stars] at (axis cs:{375.824},{-29.526}) {\tikz\pgfuseplotmark{m6c};};
\node[stars] at (axis cs:{375.954},{1.367}) {\tikz\pgfuseplotmark{m6bb};};
\node[stars] at (axis cs:{376.115},{29.659}) {\tikz\pgfuseplotmark{m6c};};
\node[stars] at (axis cs:{376.136},{-33.533}) {\tikz\pgfuseplotmark{m6cb};};
\node[stars] at (axis cs:{376.219},{5.656}) {\tikz\pgfuseplotmark{m6b};};
\node[stars] at (axis cs:{376.272},{14.946}) {\tikz\pgfuseplotmark{m6a};};
\node[stars] at (axis cs:{376.404},{-9.979}) {\tikz\pgfuseplotmark{m6b};};
\node[stars] at (axis cs:{376.421},{21.473}) {\tikz\pgfuseplotmark{m5cvb};};
\node[stars] at (axis cs:{376.455},{4.908}) {\tikz\pgfuseplotmark{m6cb};};
\node[stars] at (axis cs:{376.521},{-9.839}) {\tikz\pgfuseplotmark{m6a};};
\node[stars] at (axis cs:{376.532},{-23.992}) {\tikz\pgfuseplotmark{m6b};};
\node[stars] at (axis cs:{376.547},{32.181}) {\tikz\pgfuseplotmark{m6c};};
\node[stars] at (axis cs:{376.640},{12.956}) {\tikz\pgfuseplotmark{m6c};};
\node[stars] at (axis cs:{376.804},{-23.996}) {\tikz\pgfuseplotmark{m6c};};
\node[stars] at (axis cs:{376.943},{-9.786}) {\tikz\pgfuseplotmark{m6a};};
\node[stars] at (axis cs:{376.988},{20.739}) {\tikz\pgfuseplotmark{m6a};};
\node[stars] at (axis cs:{376.999},{1.993}) {\tikz\pgfuseplotmark{m6c};};
\node[stars] at (axis cs:{377.006},{32.012}) {\tikz\pgfuseplotmark{m6c};};
\node[stars] at (axis cs:{377.092},{5.650}) {\tikz\pgfuseplotmark{m6av};};
\node[stars] at (axis cs:{377.147},{-10.182}) {\tikz\pgfuseplotmark{m3c};};
\node[stars] at (axis cs:{377.433},{35.620}) {\tikz\pgfuseplotmark{m2bvb};};
\node[stars] at (axis cs:{377.455},{19.658}) {\tikz\pgfuseplotmark{m6a};};
\node[stars] at (axis cs:{377.548},{15.674}) {\tikz\pgfuseplotmark{m6b};};
\node[stars] at (axis cs:{377.550},{-8.906}) {\tikz\pgfuseplotmark{m6c};};
\node[stars] at (axis cs:{377.581},{25.458}) {\tikz\pgfuseplotmark{m6a};};
\node[stars] at (axis cs:{377.640},{2.445}) {\tikz\pgfuseplotmark{m6bv};};
\node[stars] at (axis cs:{377.778},{31.425}) {\tikz\pgfuseplotmark{m5c};};
\node[stars] at (axis cs:{377.793},{37.724}) {\tikz\pgfuseplotmark{m6a};};
\node[stars] at (axis cs:{377.863},{21.035}) {\tikz\pgfuseplotmark{m5a};};
\node[stars] at (axis cs:{377.915},{30.090}) {\tikz\pgfuseplotmark{m5a};};
\node[stars] at (axis cs:{377.931},{-2.251}) {\tikz\pgfuseplotmark{m6b};};
\node[stars] at (axis cs:{378.189},{-37.856}) {\tikz\pgfuseplotmark{m6bv};};
\node[stars] at (axis cs:{378.248},{30.064}) {\tikz\pgfuseplotmark{m6c};};
\node[stars] at (axis cs:{378.433},{7.575}) {\tikz\pgfuseplotmark{m5cb};};
\node[stars] at (axis cs:{378.437},{24.584}) {\tikz\pgfuseplotmark{m5a};};
\node[stars] at (axis cs:{378.520},{28.529}) {\tikz\pgfuseplotmark{m6cv};};
\node[stars] at (axis cs:{378.532},{16.133}) {\tikz\pgfuseplotmark{m6bv};};
\node[stars] at (axis cs:{378.600},{-7.923}) {\tikz\pgfuseplotmark{m5cvb};};
\node[stars] at (axis cs:{378.677},{6.995}) {\tikz\pgfuseplotmark{m6b};};
\node[stars] at (axis cs:{378.705},{-0.974}) {\tikz\pgfuseplotmark{m6a};};
\node[stars] at (axis cs:{379.079},{33.114}) {\tikz\pgfuseplotmark{m6b};};
\node[stars] at (axis cs:{379.151},{-2.500}) {\tikz\pgfuseplotmark{m5cv};};
\node[stars] at (axis cs:{379.350},{31.744}) {\tikz\pgfuseplotmark{m6c};};
\node[stars] at (axis cs:{379.450},{3.614}) {\tikz\pgfuseplotmark{m5c};};
\node[stars] at (axis cs:{379.696},{37.386}) {\tikz\pgfuseplotmark{m6cb};};
\node[stars] at (axis cs:{379.867},{27.264}) {\tikz\pgfuseplotmark{m5a};};
\node[stars] at (axis cs:{379.951},{-0.509}) {\tikz\pgfuseplotmark{m6cv};};
\end{axis}
\begin{axis}[name=stars,axis lines=none]
\node[stars] at (axis cs:{270.014},{16.751}) {\tikz\pgfuseplotmark{m5a};};
\node[stars] at (axis cs:{270.065},{0.629}) {\tikz\pgfuseplotmark{m6cv};};
\node[stars] at (axis cs:{270.066},{4.369}) {\tikz\pgfuseplotmark{m5av};};
\node[stars] at (axis cs:{270.115},{19.506}) {\tikz\pgfuseplotmark{m6cv};};
\node[stars] at (axis cs:{270.121},{-3.690}) {\tikz\pgfuseplotmark{m5a};};
\node[stars] at (axis cs:{270.152},{33.213}) {\tikz\pgfuseplotmark{m6b};};
\node[stars] at (axis cs:{270.161},{2.931}) {\tikz\pgfuseplotmark{m4bb};};
\node[stars] at (axis cs:{270.220},{6.268}) {\tikz\pgfuseplotmark{m6c};};
\node[stars] at (axis cs:{270.239},{15.093}) {\tikz\pgfuseplotmark{m6c};};
\node[stars] at (axis cs:{270.346},{-17.157}) {\tikz\pgfuseplotmark{m6c};};
\node[stars] at (axis cs:{270.377},{21.596}) {\tikz\pgfuseplotmark{m5bb};};
\node[stars] at (axis cs:{270.400},{33.311}) {\tikz\pgfuseplotmark{m6c};};
\node[stars] at (axis cs:{270.438},{1.305}) {\tikz\pgfuseplotmark{m4c};};
\node[stars] at (axis cs:{270.451},{-36.378}) {\tikz\pgfuseplotmark{m6c};};
\node[stars] at (axis cs:{270.477},{-22.780}) {\tikz\pgfuseplotmark{m6a};};
\node[stars] at (axis cs:{270.596},{20.834}) {\tikz\pgfuseplotmark{m5cv};};
\node[stars] at (axis cs:{270.626},{22.923}) {\tikz\pgfuseplotmark{m6c};};
\node[stars] at (axis cs:{270.713},{-24.282}) {\tikz\pgfuseplotmark{m5c};};
\node[stars] at (axis cs:{270.770},{-8.180}) {\tikz\pgfuseplotmark{m5ab};};
\node[stars] at (axis cs:{270.811},{19.613}) {\tikz\pgfuseplotmark{m6c};};
\node[stars] at (axis cs:{270.969},{-24.361}) {\tikz\pgfuseplotmark{m6b};};
\node[stars] at (axis cs:{271.156},{1.919}) {\tikz\pgfuseplotmark{m6cv};};
\node[stars] at (axis cs:{271.168},{23.942}) {\tikz\pgfuseplotmark{m6cv};};
\node[stars] at (axis cs:{271.210},{-35.901}) {\tikz\pgfuseplotmark{m6b};};
\node[stars] at (axis cs:{271.255},{-29.580}) {\tikz\pgfuseplotmark{m5av};};
\node[stars] at (axis cs:{271.364},{2.500}) {\tikz\pgfuseplotmark{m4bvb};};
\node[stars] at (axis cs:{271.376},{21.646}) {\tikz\pgfuseplotmark{m6c};};
\node[stars] at (axis cs:{271.452},{-30.424}) {\tikz\pgfuseplotmark{m4a};};
\node[stars] at (axis cs:{271.457},{32.230}) {\tikz\pgfuseplotmark{m6a};};
\node[stars] at (axis cs:{271.508},{22.219}) {\tikz\pgfuseplotmark{m5bv};};
\node[stars] at (axis cs:{271.531},{-8.324}) {\tikz\pgfuseplotmark{m6a};};
\node[stars] at (axis cs:{271.531},{-0.446}) {\tikz\pgfuseplotmark{m6c};};
\node[stars] at (axis cs:{271.563},{-4.751}) {\tikz\pgfuseplotmark{m6a};};
\node[stars] at (axis cs:{271.599},{-36.020}) {\tikz\pgfuseplotmark{m6b};};
\node[stars] at (axis cs:{271.756},{30.562}) {\tikz\pgfuseplotmark{m5bb};};
\node[stars] at (axis cs:{271.797},{-21.444}) {\tikz\pgfuseplotmark{m6c};};
\node[stars] at (axis cs:{271.827},{8.734}) {\tikz\pgfuseplotmark{m5a};};
\node[stars] at (axis cs:{271.837},{2.481}) {\tikz\pgfuseplotmark{m6c};};
\node[stars] at (axis cs:{271.837},{9.564}) {\tikz\pgfuseplotmark{m4ab};};
\node[stars] at (axis cs:{271.886},{28.762}) {\tikz\pgfuseplotmark{m4av};};
\node[stars] at (axis cs:{271.951},{-17.154}) {\tikz\pgfuseplotmark{m6a};};
\node[stars] at (axis cs:{271.957},{26.101}) {\tikz\pgfuseplotmark{m6ab};};
\node[stars] at (axis cs:{272.009},{36.401}) {\tikz\pgfuseplotmark{m5c};};
\node[stars] at (axis cs:{272.021},{-28.457}) {\tikz\pgfuseplotmark{m5a};};
\node[stars] at (axis cs:{272.141},{14.285}) {\tikz\pgfuseplotmark{m6c};};
\node[stars] at (axis cs:{272.190},{20.814}) {\tikz\pgfuseplotmark{m4c};};
\node[stars] at (axis cs:{272.220},{20.045}) {\tikz\pgfuseplotmark{m5b};};
\node[stars] at (axis cs:{272.293},{30.469}) {\tikz\pgfuseplotmark{m6c};};
\node[stars] at (axis cs:{272.391},{3.993}) {\tikz\pgfuseplotmark{m6av};};
\node[stars] at (axis cs:{272.406},{38.458}) {\tikz\pgfuseplotmark{m6cv};};
\node[stars] at (axis cs:{272.431},{-13.934}) {\tikz\pgfuseplotmark{m6c};};
\node[stars] at (axis cs:{272.475},{3.120}) {\tikz\pgfuseplotmark{m6a};};
\node[stars] at (axis cs:{272.496},{36.466}) {\tikz\pgfuseplotmark{m6a};};
\node[stars] at (axis cs:{272.500},{-32.720}) {\tikz\pgfuseplotmark{m6c};};
\node[stars] at (axis cs:{272.524},{-30.728}) {\tikz\pgfuseplotmark{m6ab};};
\node[stars] at (axis cs:{272.536},{16.476}) {\tikz\pgfuseplotmark{m6cvb};};
\node[stars] at (axis cs:{272.668},{3.324}) {\tikz\pgfuseplotmark{m5c};};
\node[stars] at (axis cs:{272.731},{-33.800}) {\tikz\pgfuseplotmark{m6c};};
\node[stars] at (axis cs:{272.931},{-23.701}) {\tikz\pgfuseplotmark{m5b};};
\node[stars] at (axis cs:{272.938},{33.447}) {\tikz\pgfuseplotmark{m6bv};};
\node[stars] at (axis cs:{272.976},{31.405}) {\tikz\pgfuseplotmark{m5bv};};
\node[stars] at (axis cs:{273.270},{38.773}) {\tikz\pgfuseplotmark{m6b};};
\node[stars] at (axis cs:{273.319},{21.880}) {\tikz\pgfuseplotmark{m6b};};
\node[stars] at (axis cs:{273.441},{-21.059}) {\tikz\pgfuseplotmark{m4avb};};
\node[stars] at (axis cs:{273.468},{2.393}) {\tikz\pgfuseplotmark{m6c};};
\node[stars] at (axis cs:{273.566},{-21.713}) {\tikz\pgfuseplotmark{m5cv};};
\node[stars] at (axis cs:{273.804},{-20.728}) {\tikz\pgfuseplotmark{m5cv};};
\node[stars] at (axis cs:{273.804},{-20.388}) {\tikz\pgfuseplotmark{m6b};};
\node[stars] at (axis cs:{273.878},{-18.661}) {\tikz\pgfuseplotmark{m6bv};};
\node[stars] at (axis cs:{273.991},{-3.618}) {\tikz\pgfuseplotmark{m6c};};
\node[stars] at (axis cs:{274.023},{2.377}) {\tikz\pgfuseplotmark{m6cv};};
\node[stars] at (axis cs:{274.221},{-3.007}) {\tikz\pgfuseplotmark{m6b};};
\node[stars] at (axis cs:{274.298},{-17.374}) {\tikz\pgfuseplotmark{m6a};};
\node[stars] at (axis cs:{274.349},{-28.289}) {\tikz\pgfuseplotmark{m6c};};
\node[stars] at (axis cs:{274.350},{-28.652}) {\tikz\pgfuseplotmark{m6c};};
\node[stars] at (axis cs:{274.351},{-9.758}) {\tikz\pgfuseplotmark{m6cv};};
\node[stars] at (axis cs:{274.401},{-34.107}) {\tikz\pgfuseplotmark{m6bvb};};
\node[stars] at (axis cs:{274.407},{-36.761}) {\tikz\pgfuseplotmark{m3bv};};
\node[stars] at (axis cs:{274.512},{13.777}) {\tikz\pgfuseplotmark{m6c};};
\node[stars] at (axis cs:{274.513},{-27.042}) {\tikz\pgfuseplotmark{m5a};};
\node[stars] at (axis cs:{274.532},{18.131}) {\tikz\pgfuseplotmark{m6c};};
\node[stars] at (axis cs:{274.532},{23.297}) {\tikz\pgfuseplotmark{m6cb};};
\node[stars] at (axis cs:{274.790},{7.260}) {\tikz\pgfuseplotmark{m5c};};
\node[stars] at (axis cs:{274.795},{24.446}) {\tikz\pgfuseplotmark{m5c};};
\node[stars] at (axis cs:{274.965},{36.064}) {\tikz\pgfuseplotmark{m4cv};};
\node[stars] at (axis cs:{274.967},{29.666}) {\tikz\pgfuseplotmark{m6b};};
\node[stars] at (axis cs:{275.037},{-15.831}) {\tikz\pgfuseplotmark{m5c};};
\node[stars] at (axis cs:{275.075},{21.961}) {\tikz\pgfuseplotmark{m5bv};};
\node[stars] at (axis cs:{275.217},{3.377}) {\tikz\pgfuseplotmark{m5av};};
\node[stars] at (axis cs:{275.230},{-37.488}) {\tikz\pgfuseplotmark{m6c};};
\node[stars] at (axis cs:{275.237},{29.859}) {\tikz\pgfuseplotmark{m6a};};
\node[stars] at (axis cs:{275.249},{-29.828}) {\tikz\pgfuseplotmark{m3av};};
\node[stars] at (axis cs:{275.254},{28.870}) {\tikz\pgfuseplotmark{m5b};};
\node[stars] at (axis cs:{275.328},{-2.899}) {\tikz\pgfuseplotmark{m3cv};};
\node[stars] at (axis cs:{275.346},{-18.860}) {\tikz\pgfuseplotmark{m6av};};
\node[stars] at (axis cs:{275.368},{5.436}) {\tikz\pgfuseplotmark{m6cv};};
\node[stars] at (axis cs:{275.381},{-24.915}) {\tikz\pgfuseplotmark{m6cv};};
\node[stars] at (axis cs:{275.501},{-28.430}) {\tikz\pgfuseplotmark{m6c};};
\node[stars] at (axis cs:{275.536},{23.285}) {\tikz\pgfuseplotmark{m5c};};
\node[stars] at (axis cs:{275.577},{-38.657}) {\tikz\pgfuseplotmark{m5b};};
\node[stars] at (axis cs:{275.647},{12.029}) {\tikz\pgfuseplotmark{m6b};};
\node[stars] at (axis cs:{275.704},{17.826}) {\tikz\pgfuseplotmark{m5c};};
\node[stars] at (axis cs:{275.721},{-36.670}) {\tikz\pgfuseplotmark{m5cv};};
\node[stars] at (axis cs:{275.759},{-10.219}) {\tikz\pgfuseplotmark{m6c};};
\node[stars] at (axis cs:{275.762},{16.688}) {\tikz\pgfuseplotmark{m6c};};
\node[stars] at (axis cs:{275.801},{-12.015}) {\tikz\pgfuseplotmark{m6a};};
\node[stars] at (axis cs:{275.870},{-36.238}) {\tikz\pgfuseplotmark{m6a};};
\node[stars] at (axis cs:{275.915},{-8.934}) {\tikz\pgfuseplotmark{m5a};};
\node[stars] at (axis cs:{275.925},{21.770}) {\tikz\pgfuseplotmark{m4av};};
\node[stars] at (axis cs:{275.989},{38.739}) {\tikz\pgfuseplotmark{m6cv};};
\node[stars] at (axis cs:{276.015},{-3.583}) {\tikz\pgfuseplotmark{m6c};};
\node[stars] at (axis cs:{276.043},{-34.385}) {\tikz\pgfuseplotmark{m2a};};
\node[stars] at (axis cs:{276.057},{39.507}) {\tikz\pgfuseplotmark{m5b};};
\node[stars] at (axis cs:{276.176},{-7.076}) {\tikz\pgfuseplotmark{m6c};};
\node[stars] at (axis cs:{276.238},{-1.579}) {\tikz\pgfuseplotmark{m6c};};
\node[stars] at (axis cs:{276.244},{27.395}) {\tikz\pgfuseplotmark{m6cb};};
\node[stars] at (axis cs:{276.256},{-30.756}) {\tikz\pgfuseplotmark{m6a};};
\node[stars] at (axis cs:{276.338},{-20.542}) {\tikz\pgfuseplotmark{m5ab};};
\node[stars] at (axis cs:{276.340},{-35.992}) {\tikz\pgfuseplotmark{m6c};};
\node[stars] at (axis cs:{276.412},{8.032}) {\tikz\pgfuseplotmark{m6av};};
\node[stars] at (axis cs:{276.478},{-33.946}) {\tikz\pgfuseplotmark{m6c};};
\node[stars] at (axis cs:{276.480},{14.966}) {\tikz\pgfuseplotmark{m6c};};
\node[stars] at (axis cs:{276.495},{29.829}) {\tikz\pgfuseplotmark{m6a};};
\node[stars] at (axis cs:{276.802},{0.196}) {\tikz\pgfuseplotmark{m5cvb};};
\node[stars] at (axis cs:{276.932},{-26.635}) {\tikz\pgfuseplotmark{m6c};};
\node[stars] at (axis cs:{276.956},{-29.817}) {\tikz\pgfuseplotmark{m6b};};
\node[stars] at (axis cs:{276.960},{3.748}) {\tikz\pgfuseplotmark{m6b};};
\node[stars] at (axis cs:{276.985},{-17.800}) {\tikz\pgfuseplotmark{m6cv};};
\node[stars] at (axis cs:{276.993},{-25.422}) {\tikz\pgfuseplotmark{m3a};};
\node[stars] at (axis cs:{276.995},{6.194}) {\tikz\pgfuseplotmark{m6a};};
\node[stars] at (axis cs:{277.026},{-26.757}) {\tikz\pgfuseplotmark{m6c};};
\node[stars] at (axis cs:{277.113},{-38.995}) {\tikz\pgfuseplotmark{m6a};};
\node[stars] at (axis cs:{277.299},{-14.566}) {\tikz\pgfuseplotmark{m5a};};
\node[stars] at (axis cs:{277.399},{23.866}) {\tikz\pgfuseplotmark{m6b};};
\node[stars] at (axis cs:{277.421},{-1.985}) {\tikz\pgfuseplotmark{m5c};};
\node[stars] at (axis cs:{277.445},{-14.582}) {\tikz\pgfuseplotmark{m6bv};};
\node[stars] at (axis cs:{277.549},{-18.729}) {\tikz\pgfuseplotmark{m6a};};
\node[stars] at (axis cs:{277.560},{-5.724}) {\tikz\pgfuseplotmark{m6c};};
\node[stars] at (axis cs:{277.674},{20.815}) {\tikz\pgfuseplotmark{m6cv};};
\node[stars] at (axis cs:{277.769},{16.928}) {\tikz\pgfuseplotmark{m6a};};
\node[stars] at (axis cs:{277.770},{-32.989}) {\tikz\pgfuseplotmark{m5c};};
\node[stars] at (axis cs:{277.857},{-10.796}) {\tikz\pgfuseplotmark{m6ab};};
\node[stars] at (axis cs:{277.860},{-18.403}) {\tikz\pgfuseplotmark{m5b};};
\node[stars] at (axis cs:{277.972},{-19.125}) {\tikz\pgfuseplotmark{m6cvb};};
\node[stars] at (axis cs:{277.987},{-1.003}) {\tikz\pgfuseplotmark{m6b};};
\node[stars] at (axis cs:{278.029},{3.659}) {\tikz\pgfuseplotmark{m6cv};};
\node[stars] at (axis cs:{278.086},{-14.644}) {\tikz\pgfuseplotmark{m6c};};
\node[stars] at (axis cs:{278.089},{-39.704}) {\tikz\pgfuseplotmark{m5c};};
\node[stars] at (axis cs:{278.181},{-14.865}) {\tikz\pgfuseplotmark{m5cv};};
\node[stars] at (axis cs:{278.192},{23.617}) {\tikz\pgfuseplotmark{m6a};};
\node[stars] at (axis cs:{278.208},{30.554}) {\tikz\pgfuseplotmark{m5c};};
\node[stars] at (axis cs:{278.254},{-39.892}) {\tikz\pgfuseplotmark{m6c};};
\node[stars] at (axis cs:{278.345},{-5.912}) {\tikz\pgfuseplotmark{m6c};};
\node[stars] at (axis cs:{278.346},{-38.726}) {\tikz\pgfuseplotmark{m6ab};};
\node[stars] at (axis cs:{278.347},{8.268}) {\tikz\pgfuseplotmark{m6cvb};};
\node[stars] at (axis cs:{278.413},{-14.854}) {\tikz\pgfuseplotmark{m6a};};
\node[stars] at (axis cs:{278.473},{-24.032}) {\tikz\pgfuseplotmark{m5c};};
\node[stars] at (axis cs:{278.491},{-33.016}) {\tikz\pgfuseplotmark{m5c};};
\node[stars] at (axis cs:{278.698},{10.892}) {\tikz\pgfuseplotmark{m6c};};
\node[stars] at (axis cs:{278.760},{-10.977}) {\tikz\pgfuseplotmark{m5bv};};
\node[stars] at (axis cs:{278.802},{-8.244}) {\tikz\pgfuseplotmark{m4av};};
\node[stars] at (axis cs:{278.803},{18.203}) {\tikz\pgfuseplotmark{m6a};};
\node[stars] at (axis cs:{278.806},{34.458}) {\tikz\pgfuseplotmark{m6b};};
\node[stars] at (axis cs:{278.839},{-20.840}) {\tikz\pgfuseplotmark{m6cv};};
\node[stars] at (axis cs:{278.877},{23.605}) {\tikz\pgfuseplotmark{m6a};};
\node[stars] at (axis cs:{278.902},{4.936}) {\tikz\pgfuseplotmark{m6cvb};};
\node[stars] at (axis cs:{278.972},{16.976}) {\tikz\pgfuseplotmark{m6cb};};
\node[stars] at (axis cs:{278.999},{-29.699}) {\tikz\pgfuseplotmark{m6c};};
\node[stars] at (axis cs:{279.116},{9.122}) {\tikz\pgfuseplotmark{m5c};};
\node[stars] at (axis cs:{279.156},{33.469}) {\tikz\pgfuseplotmark{m5cv};};
\node[stars] at (axis cs:{279.163},{6.672}) {\tikz\pgfuseplotmark{m5c};};
\node[stars] at (axis cs:{279.235},{38.784}) {\tikz\pgfuseplotmark{m1bvb};};
\node[stars] at (axis cs:{279.288},{16.198}) {\tikz\pgfuseplotmark{m6c};};
\node[stars] at (axis cs:{279.400},{-0.309}) {\tikz\pgfuseplotmark{m6av};};
\node[stars] at (axis cs:{279.477},{-21.397}) {\tikz\pgfuseplotmark{m6b};};
\node[stars] at (axis cs:{279.527},{39.668}) {\tikz\pgfuseplotmark{m6cv};};
\node[stars] at (axis cs:{279.599},{-3.193}) {\tikz\pgfuseplotmark{m6c};};
\node[stars] at (axis cs:{279.628},{-23.505}) {\tikz\pgfuseplotmark{m6a};};
\node[stars] at (axis cs:{279.723},{-21.052}) {\tikz\pgfuseplotmark{m6a};};
\node[stars] at (axis cs:{279.904},{5.264}) {\tikz\pgfuseplotmark{m6c};};
\node[stars] at (axis cs:{279.965},{7.358}) {\tikz\pgfuseplotmark{m6c};};
\node[stars] at (axis cs:{280.002},{-7.791}) {\tikz\pgfuseplotmark{m6a};};
\node[stars] at (axis cs:{280.008},{30.849}) {\tikz\pgfuseplotmark{m6c};};
\node[stars] at (axis cs:{280.051},{38.367}) {\tikz\pgfuseplotmark{m6c};};
\node[stars] at (axis cs:{280.422},{31.618}) {\tikz\pgfuseplotmark{m6c};};
\node[stars] at (axis cs:{280.427},{-14.564}) {\tikz\pgfuseplotmark{m6c};};
\node[stars] at (axis cs:{280.465},{-23.833}) {\tikz\pgfuseplotmark{m6c};};
\node[stars] at (axis cs:{280.534},{34.746}) {\tikz\pgfuseplotmark{m6cb};};
\node[stars] at (axis cs:{280.568},{-9.053}) {\tikz\pgfuseplotmark{m5av};};
\node[stars] at (axis cs:{280.650},{-7.074}) {\tikz\pgfuseplotmark{m6b};};
\node[stars] at (axis cs:{280.730},{-19.284}) {\tikz\pgfuseplotmark{m6cv};};
\node[stars] at (axis cs:{280.819},{39.300}) {\tikz\pgfuseplotmark{m6cb};};
\node[stars] at (axis cs:{280.880},{-8.275}) {\tikz\pgfuseplotmark{m5b};};
\node[stars] at (axis cs:{280.900},{36.556}) {\tikz\pgfuseplotmark{m6b};};
\node[stars] at (axis cs:{280.946},{-38.323}) {\tikz\pgfuseplotmark{m5bv};};
\node[stars] at (axis cs:{280.963},{-6.818}) {\tikz\pgfuseplotmark{m6c};};
\node[stars] at (axis cs:{280.965},{31.926}) {\tikz\pgfuseplotmark{m6a};};
\node[stars] at (axis cs:{281.033},{-36.718}) {\tikz\pgfuseplotmark{m6c};};
\node[stars] at (axis cs:{281.081},{-35.642}) {\tikz\pgfuseplotmark{m5a};};
\node[stars] at (axis cs:{281.085},{39.670}) {\tikz\pgfuseplotmark{m5bb};};
\node[stars] at (axis cs:{281.168},{23.590}) {\tikz\pgfuseplotmark{m6c};};
\node[stars] at (axis cs:{281.193},{37.605}) {\tikz\pgfuseplotmark{m4cvb};};
\node[stars] at (axis cs:{281.207},{-25.011}) {\tikz\pgfuseplotmark{m6a};};
\node[stars] at (axis cs:{281.208},{2.060}) {\tikz\pgfuseplotmark{m5bv};};
\node[stars] at (axis cs:{281.238},{-39.686}) {\tikz\pgfuseplotmark{m5c};};
\node[stars] at (axis cs:{281.328},{-21.002}) {\tikz\pgfuseplotmark{m6c};};
\node[stars] at (axis cs:{281.368},{5.500}) {\tikz\pgfuseplotmark{m6ab};};
\node[stars] at (axis cs:{281.414},{-26.991}) {\tikz\pgfuseplotmark{m3c};};
\node[stars] at (axis cs:{281.416},{20.546}) {\tikz\pgfuseplotmark{m4cv};};
\node[stars] at (axis cs:{281.505},{-19.606}) {\tikz\pgfuseplotmark{m6cv};};
\node[stars] at (axis cs:{281.519},{26.662}) {\tikz\pgfuseplotmark{m5a};};
\node[stars] at (axis cs:{281.586},{-22.392}) {\tikz\pgfuseplotmark{m5c};};
\node[stars] at (axis cs:{281.619},{-0.962}) {\tikz\pgfuseplotmark{m6bvb};};
\node[stars] at (axis cs:{281.672},{18.706}) {\tikz\pgfuseplotmark{m6c};};
\node[stars] at (axis cs:{281.681},{-10.125}) {\tikz\pgfuseplotmark{m6av};};
\node[stars] at (axis cs:{281.755},{18.181}) {\tikz\pgfuseplotmark{m4cvb};};
\node[stars] at (axis cs:{281.794},{-4.748}) {\tikz\pgfuseplotmark{m4c};};
\node[stars] at (axis cs:{281.871},{-5.705}) {\tikz\pgfuseplotmark{m5cv};};
\node[stars] at (axis cs:{281.989},{31.757}) {\tikz\pgfuseplotmark{m6b};};
\node[stars] at (axis cs:{282.011},{4.241}) {\tikz\pgfuseplotmark{m6c};};
\node[stars] at (axis cs:{282.068},{23.515}) {\tikz\pgfuseplotmark{m6c};};
\node[stars] at (axis cs:{282.222},{19.329}) {\tikz\pgfuseplotmark{m6b};};
\node[stars] at (axis cs:{282.405},{0.836}) {\tikz\pgfuseplotmark{m6c};};
\node[stars] at (axis cs:{282.417},{-20.325}) {\tikz\pgfuseplotmark{m5c};};
\node[stars] at (axis cs:{282.421},{-5.913}) {\tikz\pgfuseplotmark{m6bb};};
\node[stars] at (axis cs:{282.441},{32.813}) {\tikz\pgfuseplotmark{m6bvb};};
\node[stars] at (axis cs:{282.470},{32.551}) {\tikz\pgfuseplotmark{m5c};};
\node[stars] at (axis cs:{282.520},{33.362}) {\tikz\pgfuseplotmark{m4avb};};
\node[stars] at (axis cs:{282.630},{-13.572}) {\tikz\pgfuseplotmark{m6c};};
\node[stars] at (axis cs:{282.690},{10.976}) {\tikz\pgfuseplotmark{m6cb};};
\node[stars] at (axis cs:{282.710},{-22.162}) {\tikz\pgfuseplotmark{m6c};};
\node[stars] at (axis cs:{282.744},{-9.774}) {\tikz\pgfuseplotmark{m6a};};
\node[stars] at (axis cs:{282.842},{-3.318}) {\tikz\pgfuseplotmark{m6b};};
\node[stars] at (axis cs:{282.900},{28.784}) {\tikz\pgfuseplotmark{m6c};};
\node[stars] at (axis cs:{282.902},{36.539}) {\tikz\pgfuseplotmark{m6b};};
\node[stars] at (axis cs:{283.008},{13.965}) {\tikz\pgfuseplotmark{m6cv};};
\node[stars] at (axis cs:{283.068},{21.425}) {\tikz\pgfuseplotmark{m5c};};
\node[stars] at (axis cs:{283.118},{-26.651}) {\tikz\pgfuseplotmark{m6c};};
\node[stars] at (axis cs:{283.154},{-29.379}) {\tikz\pgfuseplotmark{m6b};};
\node[stars] at (axis cs:{283.258},{-9.575}) {\tikz\pgfuseplotmark{m6c};};
\node[stars] at (axis cs:{283.431},{36.972}) {\tikz\pgfuseplotmark{m6av};};
\node[stars] at (axis cs:{283.500},{-21.360}) {\tikz\pgfuseplotmark{m6a};};
\node[stars] at (axis cs:{283.542},{-22.745}) {\tikz\pgfuseplotmark{m5a};};
\node[stars] at (axis cs:{283.555},{27.909}) {\tikz\pgfuseplotmark{m6a};};
\node[stars] at (axis cs:{283.626},{36.898}) {\tikz\pgfuseplotmark{m4cvb};};
\node[stars] at (axis cs:{283.680},{-15.603}) {\tikz\pgfuseplotmark{m5b};};
\node[stars] at (axis cs:{283.687},{22.645}) {\tikz\pgfuseplotmark{m5ab};};
\node[stars] at (axis cs:{283.719},{33.969}) {\tikz\pgfuseplotmark{m6bb};};
\node[stars] at (axis cs:{283.780},{-22.671}) {\tikz\pgfuseplotmark{m5b};};
\node[stars] at (axis cs:{283.816},{-26.297}) {\tikz\pgfuseplotmark{m2b};};
\node[stars] at (axis cs:{283.848},{-37.387}) {\tikz\pgfuseplotmark{m6c};};
\node[stars] at (axis cs:{283.864},{6.615}) {\tikz\pgfuseplotmark{m6av};};
\node[stars] at (axis cs:{283.879},{-16.376}) {\tikz\pgfuseplotmark{m6a};};
\node[stars] at (axis cs:{284.003},{-23.174}) {\tikz\pgfuseplotmark{m6b};};
\node[stars] at (axis cs:{284.026},{18.105}) {\tikz\pgfuseplotmark{m6a};};
\node[stars] at (axis cs:{284.055},{4.204}) {\tikz\pgfuseplotmark{m5avb};};
\node[stars] at (axis cs:{284.094},{-1.800}) {\tikz\pgfuseplotmark{m6c};};
\node[stars] at (axis cs:{284.107},{2.471}) {\tikz\pgfuseplotmark{m6cb};};
\node[stars] at (axis cs:{284.113},{-31.689}) {\tikz\pgfuseplotmark{m6c};};
\node[stars] at (axis cs:{284.169},{-37.343}) {\tikz\pgfuseplotmark{m5cv};};
\node[stars] at (axis cs:{284.257},{32.901}) {\tikz\pgfuseplotmark{m5cb};};
\node[stars] at (axis cs:{284.265},{-5.846}) {\tikz\pgfuseplotmark{m5a};};
\node[stars] at (axis cs:{284.319},{2.535}) {\tikz\pgfuseplotmark{m6a};};
\node[stars] at (axis cs:{284.335},{-20.656}) {\tikz\pgfuseplotmark{m5b};};
\node[stars] at (axis cs:{284.393},{-39.823}) {\tikz\pgfuseplotmark{m6c};};
\node[stars] at (axis cs:{284.433},{-21.107}) {\tikz\pgfuseplotmark{m4a};};
\node[stars] at (axis cs:{284.561},{17.361}) {\tikz\pgfuseplotmark{m5cv};};
\node[stars] at (axis cs:{284.589},{-31.036}) {\tikz\pgfuseplotmark{m6b};};
\node[stars] at (axis cs:{284.599},{6.240}) {\tikz\pgfuseplotmark{m6c};};
\node[stars] at (axis cs:{284.603},{-22.529}) {\tikz\pgfuseplotmark{m6c};};
\node[stars] at (axis cs:{284.681},{-37.107}) {\tikz\pgfuseplotmark{m5av};};
\node[stars] at (axis cs:{284.696},{13.907}) {\tikz\pgfuseplotmark{m6bv};};
\node[stars] at (axis cs:{284.736},{32.689}) {\tikz\pgfuseplotmark{m3cv};};
\node[stars] at (axis cs:{284.774},{13.622}) {\tikz\pgfuseplotmark{m5cvb};};
\node[stars] at (axis cs:{284.801},{39.217}) {\tikz\pgfuseplotmark{m6cv};};
\node[stars] at (axis cs:{284.849},{-12.840}) {\tikz\pgfuseplotmark{m6a};};
\node[stars] at (axis cs:{284.862},{-18.566}) {\tikz\pgfuseplotmark{m6c};};
\node[stars] at (axis cs:{284.906},{15.068}) {\tikz\pgfuseplotmark{m4bb};};
\node[stars] at (axis cs:{284.940},{26.230}) {\tikz\pgfuseplotmark{m5cv};};
\node[stars] at (axis cs:{284.992},{22.814}) {\tikz\pgfuseplotmark{m6cv};};
\node[stars] at (axis cs:{285.003},{32.145}) {\tikz\pgfuseplotmark{m5bv};};
\node[stars] at (axis cs:{285.103},{-24.942}) {\tikz\pgfuseplotmark{m6c};};
\node[stars] at (axis cs:{285.230},{33.802}) {\tikz\pgfuseplotmark{m6b};};
\node[stars] at (axis cs:{285.268},{-37.061}) {\tikz\pgfuseplotmark{m6cb};};
\node[stars] at (axis cs:{285.273},{19.309}) {\tikz\pgfuseplotmark{m6c};};
\node[stars] at (axis cs:{285.322},{26.291}) {\tikz\pgfuseplotmark{m6a};};
\node[stars] at (axis cs:{285.390},{-15.282}) {\tikz\pgfuseplotmark{m6c};};
\node[stars] at (axis cs:{285.407},{-22.695}) {\tikz\pgfuseplotmark{m6c};};
\node[stars] at (axis cs:{285.420},{-5.739}) {\tikz\pgfuseplotmark{m4b};};
\node[stars] at (axis cs:{285.452},{33.621}) {\tikz\pgfuseplotmark{m6c};};
\node[stars] at (axis cs:{285.456},{22.264}) {\tikz\pgfuseplotmark{m6c};};
\node[stars] at (axis cs:{285.590},{8.373}) {\tikz\pgfuseplotmark{m6c};};
\node[stars] at (axis cs:{285.615},{-24.847}) {\tikz\pgfuseplotmark{m6a};};
\node[stars] at (axis cs:{285.653},{-29.880}) {\tikz\pgfuseplotmark{m3a};};
\node[stars] at (axis cs:{285.719},{19.661}) {\tikz\pgfuseplotmark{m6b};};
\node[stars] at (axis cs:{285.727},{-3.699}) {\tikz\pgfuseplotmark{m5c};};
\node[stars] at (axis cs:{285.766},{-19.245}) {\tikz\pgfuseplotmark{m6bb};};
\node[stars] at (axis cs:{285.779},{-19.103}) {\tikz\pgfuseplotmark{m6c};};
\node[stars] at (axis cs:{285.824},{-38.253}) {\tikz\pgfuseplotmark{m6av};};
\node[stars] at (axis cs:{285.884},{1.819}) {\tikz\pgfuseplotmark{m6av};};
\node[stars] at (axis cs:{286.104},{-31.047}) {\tikz\pgfuseplotmark{m5c};};
\node[stars] at (axis cs:{286.171},{-21.741}) {\tikz\pgfuseplotmark{m4av};};
\node[stars] at (axis cs:{286.240},{-4.031}) {\tikz\pgfuseplotmark{m5cb};};
\node[stars] at (axis cs:{286.241},{31.744}) {\tikz\pgfuseplotmark{m6a};};
\node[stars] at (axis cs:{286.243},{30.733}) {\tikz\pgfuseplotmark{m6b};};
\node[stars] at (axis cs:{286.353},{13.863}) {\tikz\pgfuseplotmark{m3bv};};
\node[stars] at (axis cs:{286.422},{-15.660}) {\tikz\pgfuseplotmark{m6bv};};
\node[stars] at (axis cs:{286.446},{29.922}) {\tikz\pgfuseplotmark{m6c};};
\node[stars] at (axis cs:{286.562},{-4.882}) {\tikz\pgfuseplotmark{m3c};};
\node[stars] at (axis cs:{286.605},{-37.063}) {\tikz\pgfuseplotmark{m4cb};};
\node[stars] at (axis cs:{286.657},{28.628}) {\tikz\pgfuseplotmark{m6a};};
\node[stars] at (axis cs:{286.660},{24.251}) {\tikz\pgfuseplotmark{m6a};};
\node[stars] at (axis cs:{286.717},{-16.229}) {\tikz\pgfuseplotmark{m6bb};};
\node[stars] at (axis cs:{286.719},{-37.810}) {\tikz\pgfuseplotmark{m6c};};
\node[stars] at (axis cs:{286.735},{-27.670}) {\tikz\pgfuseplotmark{m3c};};
\node[stars] at (axis cs:{286.744},{11.071}) {\tikz\pgfuseplotmark{m5bv};};
\node[stars] at (axis cs:{286.785},{-18.738}) {\tikz\pgfuseplotmark{m6cv};};
\node[stars] at (axis cs:{286.826},{36.100}) {\tikz\pgfuseplotmark{m5cv};};
\node[stars] at (axis cs:{286.857},{32.502}) {\tikz\pgfuseplotmark{m5cb};};
\node[stars] at (axis cs:{286.879},{-28.637}) {\tikz\pgfuseplotmark{m6b};};
\node[stars] at (axis cs:{286.989},{16.853}) {\tikz\pgfuseplotmark{m6b};};
\node[stars] at (axis cs:{287.015},{21.699}) {\tikz\pgfuseplotmark{m6c};};
\node[stars] at (axis cs:{287.061},{-24.657}) {\tikz\pgfuseplotmark{m6c};};
\node[stars] at (axis cs:{287.070},{-19.290}) {\tikz\pgfuseplotmark{m6av};};
\node[stars] at (axis cs:{287.250},{6.073}) {\tikz\pgfuseplotmark{m5c};};
\node[stars] at (axis cs:{287.368},{-37.904}) {\tikz\pgfuseplotmark{m4b};};
\node[stars] at (axis cs:{287.416},{-39.827}) {\tikz\pgfuseplotmark{m6c};};
\node[stars] at (axis cs:{287.441},{-21.024}) {\tikz\pgfuseplotmark{m3bv};};
\node[stars] at (axis cs:{287.451},{-19.804}) {\tikz\pgfuseplotmark{m6b};};
\node[stars] at (axis cs:{287.465},{-0.428}) {\tikz\pgfuseplotmark{m6c};};
\node[stars] at (axis cs:{287.507},{-39.341}) {\tikz\pgfuseplotmark{m4bv};};
\node[stars] at (axis cs:{287.758},{-39.005}) {\tikz\pgfuseplotmark{m6c};};
\node[stars] at (axis cs:{287.828},{-29.502}) {\tikz\pgfuseplotmark{m6c};};
\node[stars] at (axis cs:{287.879},{26.736}) {\tikz\pgfuseplotmark{m6c};};
\node[stars] at (axis cs:{287.942},{31.283}) {\tikz\pgfuseplotmark{m6bv};};
\node[stars] at (axis cs:{288.117},{-21.658}) {\tikz\pgfuseplotmark{m6c};};
\node[stars] at (axis cs:{288.153},{21.554}) {\tikz\pgfuseplotmark{m6b};};
\node[stars] at (axis cs:{288.170},{-7.939}) {\tikz\pgfuseplotmark{m5cv};};
\node[stars] at (axis cs:{288.307},{-25.907}) {\tikz\pgfuseplotmark{m6a};};
\node[stars] at (axis cs:{288.315},{-12.282}) {\tikz\pgfuseplotmark{m5c};};
\node[stars] at (axis cs:{288.428},{2.294}) {\tikz\pgfuseplotmark{m5cv};};
\node[stars] at (axis cs:{288.433},{5.515}) {\tikz\pgfuseplotmark{m6c};};
\node[stars] at (axis cs:{288.440},{39.146}) {\tikz\pgfuseplotmark{m4cv};};
\node[stars] at (axis cs:{288.761},{20.203}) {\tikz\pgfuseplotmark{m6bv};};
\node[stars] at (axis cs:{288.822},{21.232}) {\tikz\pgfuseplotmark{m6a};};
\node[stars] at (axis cs:{288.834},{15.084}) {\tikz\pgfuseplotmark{m6ab};};
\node[stars] at (axis cs:{288.854},{30.526}) {\tikz\pgfuseplotmark{m6bv};};
\node[stars] at (axis cs:{288.885},{-25.257}) {\tikz\pgfuseplotmark{m5b};};
\node[stars] at (axis cs:{288.888},{-24.179}) {\tikz\pgfuseplotmark{m6c};};
\node[stars] at (axis cs:{288.889},{18.516}) {\tikz\pgfuseplotmark{m6c};};
\node[stars] at (axis cs:{288.998},{27.926}) {\tikz\pgfuseplotmark{m6cv};};
\node[stars] at (axis cs:{289.054},{21.390}) {\tikz\pgfuseplotmark{m5av};};
\node[stars] at (axis cs:{289.092},{38.134}) {\tikz\pgfuseplotmark{m4cv};};
\node[stars] at (axis cs:{289.112},{14.544}) {\tikz\pgfuseplotmark{m6ab};};
\node[stars] at (axis cs:{289.129},{4.835}) {\tikz\pgfuseplotmark{m6a};};
\node[stars] at (axis cs:{289.174},{-19.308}) {\tikz\pgfuseplotmark{m6bv};};
\node[stars] at (axis cs:{289.409},{-18.953}) {\tikz\pgfuseplotmark{m5bv};};
\node[stars] at (axis cs:{289.432},{23.025}) {\tikz\pgfuseplotmark{m5cvb};};
\node[stars] at (axis cs:{289.451},{2.031}) {\tikz\pgfuseplotmark{m6c};};
\node[stars] at (axis cs:{289.454},{11.595}) {\tikz\pgfuseplotmark{m5c};};
\node[stars] at (axis cs:{289.635},{1.085}) {\tikz\pgfuseplotmark{m5bvb};};
\node[stars] at (axis cs:{289.712},{0.339}) {\tikz\pgfuseplotmark{m6cb};};
\node[stars] at (axis cs:{289.720},{9.618}) {\tikz\pgfuseplotmark{m6c};};
\node[stars] at (axis cs:{289.750},{-15.537}) {\tikz\pgfuseplotmark{m6b};};
\node[stars] at (axis cs:{289.755},{37.445}) {\tikz\pgfuseplotmark{m6c};};
\node[stars] at (axis cs:{289.912},{37.330}) {\tikz\pgfuseplotmark{m6c};};
\node[stars] at (axis cs:{289.914},{12.375}) {\tikz\pgfuseplotmark{m6avb};};
\node[stars] at (axis cs:{289.917},{-35.421}) {\tikz\pgfuseplotmark{m6a};};
\node[stars] at (axis cs:{289.971},{11.535}) {\tikz\pgfuseplotmark{m6b};};
\node[stars] at (axis cs:{290.137},{-5.416}) {\tikz\pgfuseplotmark{m5b};};
\node[stars] at (axis cs:{290.138},{35.186}) {\tikz\pgfuseplotmark{m6cb};};
\node[stars] at (axis cs:{290.149},{-0.892}) {\tikz\pgfuseplotmark{m5c};};
\node[stars] at (axis cs:{290.159},{-22.402}) {\tikz\pgfuseplotmark{m6a};};
\node[stars] at (axis cs:{290.374},{-34.983}) {\tikz\pgfuseplotmark{m6cv};};
\node[stars] at (axis cs:{290.405},{-19.234}) {\tikz\pgfuseplotmark{m6cvb};};
\node[stars] at (axis cs:{290.418},{-17.847}) {\tikz\pgfuseplotmark{m4bv};};
\node[stars] at (axis cs:{290.432},{-15.955}) {\tikz\pgfuseplotmark{m5av};};
\node[stars] at (axis cs:{290.462},{-18.308}) {\tikz\pgfuseplotmark{m6a};};
\node[stars] at (axis cs:{290.590},{-0.252}) {\tikz\pgfuseplotmark{m6a};};
\node[stars] at (axis cs:{290.639},{33.518}) {\tikz\pgfuseplotmark{m6b};};
\node[stars] at (axis cs:{290.701},{9.913}) {\tikz\pgfuseplotmark{m6c};};
\node[stars] at (axis cs:{290.712},{26.262}) {\tikz\pgfuseplotmark{m5cv};};
\node[stars] at (axis cs:{290.769},{-7.401}) {\tikz\pgfuseplotmark{m6c};};
\node[stars] at (axis cs:{290.892},{33.222}) {\tikz\pgfuseplotmark{m6c};};
\node[stars] at (axis cs:{290.946},{20.264}) {\tikz\pgfuseplotmark{m6c};};
\node[stars] at (axis cs:{291.025},{36.452}) {\tikz\pgfuseplotmark{m6c};};
\node[stars] at (axis cs:{291.032},{29.621}) {\tikz\pgfuseplotmark{m5b};};
\node[stars] at (axis cs:{291.092},{16.938}) {\tikz\pgfuseplotmark{m6cb};};
\node[stars] at (axis cs:{291.126},{-27.866}) {\tikz\pgfuseplotmark{m6b};};
\node[stars] at (axis cs:{291.243},{11.944}) {\tikz\pgfuseplotmark{m5cvb};};
\node[stars] at (axis cs:{291.267},{-29.309}) {\tikz\pgfuseplotmark{m6b};};
\node[stars] at (axis cs:{291.319},{-24.508}) {\tikz\pgfuseplotmark{m5b};};
\node[stars] at (axis cs:{291.340},{-13.897}) {\tikz\pgfuseplotmark{m6a};};
\node[stars] at (axis cs:{291.343},{20.272}) {\tikz\pgfuseplotmark{m6c};};
\node[stars] at (axis cs:{291.357},{24.913}) {\tikz\pgfuseplotmark{m6c};};
\node[stars] at (axis cs:{291.369},{19.798}) {\tikz\pgfuseplotmark{m5c};};
\node[stars] at (axis cs:{291.374},{-23.962}) {\tikz\pgfuseplotmark{m5cv};};
\node[stars] at (axis cs:{291.375},{3.115}) {\tikz\pgfuseplotmark{m3cv};};
\node[stars] at (axis cs:{291.538},{36.318}) {\tikz\pgfuseplotmark{m5cv};};
\node[stars] at (axis cs:{291.546},{-15.053}) {\tikz\pgfuseplotmark{m6a};};
\node[stars] at (axis cs:{291.555},{20.097}) {\tikz\pgfuseplotmark{m6a};};
\node[stars] at (axis cs:{291.580},{-21.777}) {\tikz\pgfuseplotmark{m6a};};
\node[stars] at (axis cs:{291.601},{13.024}) {\tikz\pgfuseplotmark{m6a};};
\node[stars] at (axis cs:{291.620},{19.891}) {\tikz\pgfuseplotmark{m6a};};
\node[stars] at (axis cs:{291.630},{0.338}) {\tikz\pgfuseplotmark{m5av};};
\node[stars] at (axis cs:{291.735},{-29.743}) {\tikz\pgfuseplotmark{m6a};};
\node[stars] at (axis cs:{291.891},{14.282}) {\tikz\pgfuseplotmark{m6cv};};
\node[stars] at (axis cs:{291.902},{37.941}) {\tikz\pgfuseplotmark{m6cv};};
\node[stars] at (axis cs:{292.087},{2.930}) {\tikz\pgfuseplotmark{m6a};};
\node[stars] at (axis cs:{292.176},{24.665}) {\tikz\pgfuseplotmark{m4cvb};};
\node[stars] at (axis cs:{292.254},{1.950}) {\tikz\pgfuseplotmark{m6av};};
\node[stars] at (axis cs:{292.325},{0.246}) {\tikz\pgfuseplotmark{m6c};};
\node[stars] at (axis cs:{292.337},{20.280}) {\tikz\pgfuseplotmark{m6c};};
\node[stars] at (axis cs:{292.339},{-7.044}) {\tikz\pgfuseplotmark{m6cv};};
\node[stars] at (axis cs:{292.342},{14.596}) {\tikz\pgfuseplotmark{m6a};};
\node[stars] at (axis cs:{292.467},{-26.986}) {\tikz\pgfuseplotmark{m5cb};};
\node[stars] at (axis cs:{292.544},{2.904}) {\tikz\pgfuseplotmark{m6bb};};
\node[stars] at (axis cs:{292.638},{3.444}) {\tikz\pgfuseplotmark{m6bv};};
\node[stars] at (axis cs:{292.666},{-2.789}) {\tikz\pgfuseplotmark{m5bv};};
\node[stars] at (axis cs:{292.680},{27.960}) {\tikz\pgfuseplotmark{m3bvb};};
\node[stars] at (axis cs:{292.695},{36.228}) {\tikz\pgfuseplotmark{m6c};};
\node[stars] at (axis cs:{292.725},{-21.312}) {\tikz\pgfuseplotmark{m6b};};
\node[stars] at (axis cs:{292.840},{26.617}) {\tikz\pgfuseplotmark{m6b};};
\node[stars] at (axis cs:{292.943},{34.453}) {\tikz\pgfuseplotmark{m5a};};
\node[stars] at (axis cs:{293.282},{5.029}) {\tikz\pgfuseplotmark{m6cv};};
\node[stars] at (axis cs:{293.442},{5.465}) {\tikz\pgfuseplotmark{m6cv};};
\node[stars] at (axis cs:{293.522},{7.379}) {\tikz\pgfuseplotmark{m4cvb};};
\node[stars] at (axis cs:{293.641},{-23.859}) {\tikz\pgfuseplotmark{m6cv};};
\node[stars] at (axis cs:{293.645},{19.773}) {\tikz\pgfuseplotmark{m5bv};};
\node[stars] at (axis cs:{293.712},{29.463}) {\tikz\pgfuseplotmark{m5cv};};
\node[stars] at (axis cs:{293.780},{-10.560}) {\tikz\pgfuseplotmark{m5b};};
\node[stars] at (axis cs:{293.855},{2.913}) {\tikz\pgfuseplotmark{m6c};};
\node[stars] at (axis cs:{293.874},{-7.460}) {\tikz\pgfuseplotmark{m6c};};
\node[stars] at (axis cs:{293.889},{-12.253}) {\tikz\pgfuseplotmark{m6c};};
\node[stars] at (axis cs:{293.951},{36.944}) {\tikz\pgfuseplotmark{m6b};};
\node[stars] at (axis cs:{294.007},{-24.719}) {\tikz\pgfuseplotmark{m6av};};
\node[stars] at (axis cs:{294.035},{22.586}) {\tikz\pgfuseplotmark{m6cv};};
\node[stars] at (axis cs:{294.065},{14.391}) {\tikz\pgfuseplotmark{m6c};};
\node[stars] at (axis cs:{294.109},{-18.853}) {\tikz\pgfuseplotmark{m6bv};};
\node[stars] at (axis cs:{294.177},{-24.884}) {\tikz\pgfuseplotmark{m5a};};
\node[stars] at (axis cs:{294.180},{-1.287}) {\tikz\pgfuseplotmark{m4c};};
\node[stars] at (axis cs:{294.219},{11.273}) {\tikz\pgfuseplotmark{m6bb};};
\node[stars] at (axis cs:{294.223},{-7.027}) {\tikz\pgfuseplotmark{m5bv};};
\node[stars] at (axis cs:{294.264},{-18.231}) {\tikz\pgfuseplotmark{m6a};};
\node[stars] at (axis cs:{294.290},{29.334}) {\tikz\pgfuseplotmark{m6c};};
\node[stars] at (axis cs:{294.322},{16.463}) {\tikz\pgfuseplotmark{m6avb};};
\node[stars] at (axis cs:{294.393},{-14.302}) {\tikz\pgfuseplotmark{m5c};};
\node[stars] at (axis cs:{294.420},{35.022}) {\tikz\pgfuseplotmark{m6c};};
\node[stars] at (axis cs:{294.447},{-4.647}) {\tikz\pgfuseplotmark{m5c};};
\node[stars] at (axis cs:{294.704},{3.381}) {\tikz\pgfuseplotmark{m6c};};
\node[stars] at (axis cs:{294.799},{5.398}) {\tikz\pgfuseplotmark{m5cv};};
\node[stars] at (axis cs:{294.844},{30.153}) {\tikz\pgfuseplotmark{m5a};};
\node[stars] at (axis cs:{294.856},{16.571}) {\tikz\pgfuseplotmark{m6cvb};};
\node[stars] at (axis cs:{294.937},{33.979}) {\tikz\pgfuseplotmark{m6bb};};
\node[stars] at (axis cs:{294.956},{-23.428}) {\tikz\pgfuseplotmark{m6c};};
\node[stars] at (axis cs:{295.024},{18.014}) {\tikz\pgfuseplotmark{m4c};};
\node[stars] at (axis cs:{295.030},{-23.429}) {\tikz\pgfuseplotmark{m6b};};
\node[stars] at (axis cs:{295.181},{-0.621}) {\tikz\pgfuseplotmark{m6a};};
\node[stars] at (axis cs:{295.181},{-16.293}) {\tikz\pgfuseplotmark{m5cvb};};
\node[stars] at (axis cs:{295.262},{17.476}) {\tikz\pgfuseplotmark{m4c};};
\node[stars] at (axis cs:{295.273},{13.815}) {\tikz\pgfuseplotmark{m6bv};};
\node[stars] at (axis cs:{295.311},{22.453}) {\tikz\pgfuseplotmark{m6c};};
\node[stars] at (axis cs:{295.512},{-9.193}) {\tikz\pgfuseplotmark{m6c};};
\node[stars] at (axis cs:{295.553},{12.193}) {\tikz\pgfuseplotmark{m6c};};
\node[stars] at (axis cs:{295.630},{-16.124}) {\tikz\pgfuseplotmark{m5b};};
\node[stars] at (axis cs:{295.642},{11.826}) {\tikz\pgfuseplotmark{m5cb};};
\node[stars] at (axis cs:{295.686},{32.427}) {\tikz\pgfuseplotmark{m6b};};
\node[stars] at (axis cs:{295.790},{30.678}) {\tikz\pgfuseplotmark{m6b};};
\node[stars] at (axis cs:{295.795},{38.672}) {\tikz\pgfuseplotmark{m6c};};
\node[stars] at (axis cs:{295.890},{-15.470}) {\tikz\pgfuseplotmark{m5c};};
\node[stars] at (axis cs:{295.907},{-37.539}) {\tikz\pgfuseplotmark{m6c};};
\node[stars] at (axis cs:{295.929},{25.772}) {\tikz\pgfuseplotmark{m5c};};
\node[stars] at (axis cs:{295.964},{34.163}) {\tikz\pgfuseplotmark{m6b};};
\node[stars] at (axis cs:{295.983},{27.135}) {\tikz\pgfuseplotmark{m6cv};};
\node[stars] at (axis cs:{296.069},{37.354}) {\tikz\pgfuseplotmark{m5b};};
\node[stars] at (axis cs:{296.142},{13.303}) {\tikz\pgfuseplotmark{m6c};};
\node[stars] at (axis cs:{296.159},{34.414}) {\tikz\pgfuseplotmark{m6cv};};
\node[stars] at (axis cs:{296.172},{8.726}) {\tikz\pgfuseplotmark{m6c};};
\node[stars] at (axis cs:{296.415},{36.091}) {\tikz\pgfuseplotmark{m6cb};};
\node[stars] at (axis cs:{296.416},{7.613}) {\tikz\pgfuseplotmark{m6b};};
\node[stars] at (axis cs:{296.464},{35.013}) {\tikz\pgfuseplotmark{m6bvb};};
\node[stars] at (axis cs:{296.468},{-2.883}) {\tikz\pgfuseplotmark{m6c};};
\node[stars] at (axis cs:{296.505},{-31.908}) {\tikz\pgfuseplotmark{m6a};};
\node[stars] at (axis cs:{296.565},{10.613}) {\tikz\pgfuseplotmark{m3av};};
\node[stars] at (axis cs:{296.591},{-19.761}) {\tikz\pgfuseplotmark{m5b};};
\node[stars] at (axis cs:{296.607},{33.727}) {\tikz\pgfuseplotmark{m5bb};};
\node[stars] at (axis cs:{296.646},{32.889}) {\tikz\pgfuseplotmark{m6cvb};};
\node[stars] at (axis cs:{296.847},{18.534}) {\tikz\pgfuseplotmark{m4av};};
\node[stars] at (axis cs:{296.866},{38.407}) {\tikz\pgfuseplotmark{m6a};};
\node[stars] at (axis cs:{296.952},{25.384}) {\tikz\pgfuseplotmark{m6b};};
\node[stars] at (axis cs:{297.013},{-13.704}) {\tikz\pgfuseplotmark{m6b};};
\node[stars] at (axis cs:{297.127},{10.694}) {\tikz\pgfuseplotmark{m6cv};};
\node[stars] at (axis cs:{297.175},{11.816}) {\tikz\pgfuseplotmark{m6ab};};
\node[stars] at (axis cs:{297.209},{-12.319}) {\tikz\pgfuseplotmark{m6cv};};
\node[stars] at (axis cs:{297.211},{33.437}) {\tikz\pgfuseplotmark{m6cv};};
\node[stars] at (axis cs:{297.244},{19.142}) {\tikz\pgfuseplotmark{m5bb};};
\node[stars] at (axis cs:{297.259},{-10.871}) {\tikz\pgfuseplotmark{m6b};};
\node[stars] at (axis cs:{297.298},{-28.789}) {\tikz\pgfuseplotmark{m6b};};
\node[stars] at (axis cs:{297.365},{38.710}) {\tikz\pgfuseplotmark{m6bvb};};
\node[stars] at (axis cs:{297.478},{28.440}) {\tikz\pgfuseplotmark{m6cv};};
\node[stars] at (axis cs:{297.483},{27.085}) {\tikz\pgfuseplotmark{m6c};};
\node[stars] at (axis cs:{297.573},{7.902}) {\tikz\pgfuseplotmark{m6cv};};
\node[stars] at (axis cs:{297.642},{38.722}) {\tikz\pgfuseplotmark{m5cvb};};
\node[stars] at (axis cs:{297.695},{-10.763}) {\tikz\pgfuseplotmark{m5c};};
\node[stars] at (axis cs:{297.695},{37.826}) {\tikz\pgfuseplotmark{m6cv};};
\node[stars] at (axis cs:{297.696},{8.868}) {\tikz\pgfuseplotmark{m1bv};};
\node[stars] at (axis cs:{297.757},{10.415}) {\tikz\pgfuseplotmark{m5bv};};
\node[stars] at (axis cs:{297.767},{22.610}) {\tikz\pgfuseplotmark{m5bv};};
\node[stars] at (axis cs:{297.796},{-2.461}) {\tikz\pgfuseplotmark{m6b};};
\node[stars] at (axis cs:{297.824},{9.630}) {\tikz\pgfuseplotmark{m6c};};
\node[stars] at (axis cs:{297.862},{4.089}) {\tikz\pgfuseplotmark{m6cb};};
\node[stars] at (axis cs:{297.961},{-39.874}) {\tikz\pgfuseplotmark{m5cv};};
\node[stars] at (axis cs:{298.007},{24.992}) {\tikz\pgfuseplotmark{m6av};};
\node[stars] at (axis cs:{298.014},{11.628}) {\tikz\pgfuseplotmark{m6cv};};
\node[stars] at (axis cs:{298.050},{-19.045}) {\tikz\pgfuseplotmark{m6b};};
\node[stars] at (axis cs:{298.068},{36.432}) {\tikz\pgfuseplotmark{m6b};};
\node[stars] at (axis cs:{298.091},{18.672}) {\tikz\pgfuseplotmark{m6cv};};
\node[stars] at (axis cs:{298.118},{1.006}) {\tikz\pgfuseplotmark{m4bv};};
\node[stars] at (axis cs:{298.277},{-14.603}) {\tikz\pgfuseplotmark{m6cv};};
\node[stars] at (axis cs:{298.328},{-3.114}) {\tikz\pgfuseplotmark{m6av};};
\node[stars] at (axis cs:{298.365},{24.079}) {\tikz\pgfuseplotmark{m5avb};};
\node[stars] at (axis cs:{298.534},{-8.574}) {\tikz\pgfuseplotmark{m6a};};
\node[stars] at (axis cs:{298.562},{8.461}) {\tikz\pgfuseplotmark{m5a};};
\node[stars] at (axis cs:{298.574},{-23.941}) {\tikz\pgfuseplotmark{m6cv};};
\node[stars] at (axis cs:{298.629},{24.319}) {\tikz\pgfuseplotmark{m6a};};
\node[stars] at (axis cs:{298.657},{-8.227}) {\tikz\pgfuseplotmark{m6ab};};
\node[stars] at (axis cs:{298.668},{7.140}) {\tikz\pgfuseplotmark{m6c};};
\node[stars] at (axis cs:{298.687},{0.274}) {\tikz\pgfuseplotmark{m6a};};
\node[stars] at (axis cs:{298.701},{36.995}) {\tikz\pgfuseplotmark{m6a};};
\node[stars] at (axis cs:{298.771},{-33.046}) {\tikz\pgfuseplotmark{m6c};};
\node[stars] at (axis cs:{298.828},{6.407}) {\tikz\pgfuseplotmark{m4av};};
\node[stars] at (axis cs:{298.960},{-26.299}) {\tikz\pgfuseplotmark{m5a};};
\node[stars] at (axis cs:{298.966},{38.487}) {\tikz\pgfuseplotmark{m5b};};
\node[stars] at (axis cs:{299.005},{16.635}) {\tikz\pgfuseplotmark{m6av};};
\node[stars] at (axis cs:{299.059},{11.424}) {\tikz\pgfuseplotmark{m5cv};};
\node[stars] at (axis cs:{299.077},{35.083}) {\tikz\pgfuseplotmark{m4bvb};};
\node[stars] at (axis cs:{299.184},{36.251}) {\tikz\pgfuseplotmark{m6b};};
\node[stars] at (axis cs:{299.237},{-27.170}) {\tikz\pgfuseplotmark{m5av};};
\node[stars] at (axis cs:{299.439},{16.789}) {\tikz\pgfuseplotmark{m6a};};
\node[stars] at (axis cs:{299.488},{-15.491}) {\tikz\pgfuseplotmark{m5b};};
\node[stars] at (axis cs:{299.658},{30.984}) {\tikz\pgfuseplotmark{m6a};};
\node[stars] at (axis cs:{299.689},{19.492}) {\tikz\pgfuseplotmark{m4av};};
\node[stars] at (axis cs:{299.735},{-30.538}) {\tikz\pgfuseplotmark{m6c};};
\node[stars] at (axis cs:{299.738},{-26.196}) {\tikz\pgfuseplotmark{m5a};};
\node[stars] at (axis cs:{299.794},{23.101}) {\tikz\pgfuseplotmark{m6a};};
\node[stars] at (axis cs:{299.844},{1.378}) {\tikz\pgfuseplotmark{m6c};};
\node[stars] at (axis cs:{299.934},{-35.276}) {\tikz\pgfuseplotmark{m4c};};
\node[stars] at (axis cs:{299.947},{-9.958}) {\tikz\pgfuseplotmark{m6a};};
\node[stars] at (axis cs:{299.964},{-34.698}) {\tikz\pgfuseplotmark{m5cb};};
\node[stars] at (axis cs:{299.980},{37.043}) {\tikz\pgfuseplotmark{m5cv};};
\node[stars] at (axis cs:{300.014},{17.516}) {\tikz\pgfuseplotmark{m5cvb};};
\node[stars] at (axis cs:{300.066},{-37.702}) {\tikz\pgfuseplotmark{m6b};};
\node[stars] at (axis cs:{300.084},{-33.703}) {\tikz\pgfuseplotmark{m6av};};
\node[stars] at (axis cs:{300.246},{8.558}) {\tikz\pgfuseplotmark{m6b};};
\node[stars] at (axis cs:{300.275},{27.754}) {\tikz\pgfuseplotmark{m5av};};
\node[stars] at (axis cs:{300.314},{37.099}) {\tikz\pgfuseplotmark{m6c};};
\node[stars] at (axis cs:{300.349},{-22.737}) {\tikz\pgfuseplotmark{m6b};};
\node[stars] at (axis cs:{300.436},{24.800}) {\tikz\pgfuseplotmark{m6b};};
\node[stars] at (axis cs:{300.494},{-13.637}) {\tikz\pgfuseplotmark{m6a};};
\node[stars] at (axis cs:{300.506},{24.938}) {\tikz\pgfuseplotmark{m5c};};
\node[stars] at (axis cs:{300.665},{-27.710}) {\tikz\pgfuseplotmark{m4cv};};
\node[stars] at (axis cs:{300.703},{31.959}) {\tikz\pgfuseplotmark{m6c};};
\node[stars] at (axis cs:{300.818},{18.501}) {\tikz\pgfuseplotmark{m6b};};
\node[stars] at (axis cs:{300.875},{16.031}) {\tikz\pgfuseplotmark{m6a};};
\node[stars] at (axis cs:{300.889},{-37.941}) {\tikz\pgfuseplotmark{m5a};};
\node[stars] at (axis cs:{300.906},{29.897}) {\tikz\pgfuseplotmark{m6a};};
\node[stars] at (axis cs:{300.912},{22.941}) {\tikz\pgfuseplotmark{m6c};};
\node[stars] at (axis cs:{300.935},{-22.595}) {\tikz\pgfuseplotmark{m6c};};
\node[stars] at (axis cs:{301.026},{17.070}) {\tikz\pgfuseplotmark{m6avb};};
\node[stars] at (axis cs:{301.035},{7.278}) {\tikz\pgfuseplotmark{m6a};};
\node[stars] at (axis cs:{301.082},{-32.056}) {\tikz\pgfuseplotmark{m5bv};};
\node[stars] at (axis cs:{301.096},{-0.709}) {\tikz\pgfuseplotmark{m6av};};
\node[stars] at (axis cs:{301.151},{32.219}) {\tikz\pgfuseplotmark{m6av};};
\node[stars] at (axis cs:{301.273},{-11.599}) {\tikz\pgfuseplotmark{m6cvb};};
\node[stars] at (axis cs:{301.290},{19.991}) {\tikz\pgfuseplotmark{m5b};};
\node[stars] at (axis cs:{301.291},{38.478}) {\tikz\pgfuseplotmark{m6c};};
\node[stars] at (axis cs:{301.361},{15.500}) {\tikz\pgfuseplotmark{m6cv};};
\node[stars] at (axis cs:{301.551},{-4.078}) {\tikz\pgfuseplotmark{m6c};};
\node[stars] at (axis cs:{301.591},{35.972}) {\tikz\pgfuseplotmark{m5cv};};
\node[stars] at (axis cs:{301.723},{23.614}) {\tikz\pgfuseplotmark{m5b};};
\node[stars] at (axis cs:{301.923},{34.423}) {\tikz\pgfuseplotmark{m6c};};
\node[stars] at (axis cs:{302.008},{-0.678}) {\tikz\pgfuseplotmark{m6b};};
\node[stars] at (axis cs:{302.027},{16.664}) {\tikz\pgfuseplotmark{m6cv};};
\node[stars] at (axis cs:{302.130},{-10.062}) {\tikz\pgfuseplotmark{m6cv};};
\node[stars] at (axis cs:{302.159},{10.726}) {\tikz\pgfuseplotmark{m6c};};
\node[stars] at (axis cs:{302.357},{36.840}) {\tikz\pgfuseplotmark{m5bv};};
\node[stars] at (axis cs:{302.486},{20.915}) {\tikz\pgfuseplotmark{m6cb};};
\node[stars] at (axis cs:{302.640},{26.904}) {\tikz\pgfuseplotmark{m6av};};
\node[stars] at (axis cs:{302.765},{21.135}) {\tikz\pgfuseplotmark{m6c};};
\node[stars] at (axis cs:{302.792},{-8.842}) {\tikz\pgfuseplotmark{m6cv};};
\node[stars] at (axis cs:{302.800},{-36.101}) {\tikz\pgfuseplotmark{m5cv};};
\node[stars] at (axis cs:{302.826},{-0.821}) {\tikz\pgfuseplotmark{m3cv};};
\node[stars] at (axis cs:{302.838},{21.875}) {\tikz\pgfuseplotmark{m6c};};
\node[stars] at (axis cs:{302.950},{26.809}) {\tikz\pgfuseplotmark{m5c};};
\node[stars] at (axis cs:{302.991},{-12.392}) {\tikz\pgfuseplotmark{m6c};};
\node[stars] at (axis cs:{303.003},{26.479}) {\tikz\pgfuseplotmark{m6bv};};
\node[stars] at (axis cs:{303.108},{-12.617}) {\tikz\pgfuseplotmark{m6a};};
\node[stars] at (axis cs:{303.146},{0.867}) {\tikz\pgfuseplotmark{m6cb};};
\node[stars] at (axis cs:{303.268},{-0.331}) {\tikz\pgfuseplotmark{m6cv};};
\node[stars] at (axis cs:{303.308},{-1.009}) {\tikz\pgfuseplotmark{m5c};};
\node[stars] at (axis cs:{303.520},{36.605}) {\tikz\pgfuseplotmark{m6cv};};
\node[stars] at (axis cs:{303.561},{28.695}) {\tikz\pgfuseplotmark{m5cv};};
\node[stars] at (axis cs:{303.569},{15.197}) {\tikz\pgfuseplotmark{m5b};};
\node[stars] at (axis cs:{303.633},{36.806}) {\tikz\pgfuseplotmark{m5bvb};};
\node[stars] at (axis cs:{303.816},{25.592}) {\tikz\pgfuseplotmark{m5av};};
\node[stars] at (axis cs:{303.822},{-27.033}) {\tikz\pgfuseplotmark{m6av};};
\node[stars] at (axis cs:{303.849},{33.729}) {\tikz\pgfuseplotmark{m6a};};
\node[stars] at (axis cs:{303.876},{23.509}) {\tikz\pgfuseplotmark{m5cv};};
\node[stars] at (axis cs:{303.942},{27.814}) {\tikz\pgfuseplotmark{m5a};};
\node[stars] at (axis cs:{303.961},{-30.005}) {\tikz\pgfuseplotmark{m6c};};
\node[stars] at (axis cs:{304.014},{38.898}) {\tikz\pgfuseplotmark{m6c};};
\node[stars] at (axis cs:{304.025},{4.581}) {\tikz\pgfuseplotmark{m6c};};
\node[stars] at (axis cs:{304.082},{21.598}) {\tikz\pgfuseplotmark{m6b};};
\node[stars] at (axis cs:{304.095},{-12.337}) {\tikz\pgfuseplotmark{m6cb};};
\node[stars] at (axis cs:{304.098},{-36.454}) {\tikz\pgfuseplotmark{m6cv};};
\node[stars] at (axis cs:{304.117},{37.056}) {\tikz\pgfuseplotmark{m6c};};
\node[stars] at (axis cs:{304.196},{24.671}) {\tikz\pgfuseplotmark{m5c};};
\node[stars] at (axis cs:{304.381},{29.147}) {\tikz\pgfuseplotmark{m6c};};
\node[stars] at (axis cs:{304.412},{-12.508}) {\tikz\pgfuseplotmark{m4cv};};
\node[stars] at (axis cs:{304.447},{38.033}) {\tikz\pgfuseplotmark{m5av};};
\node[stars] at (axis cs:{304.506},{-21.810}) {\tikz\pgfuseplotmark{m6a};};
\node[stars] at (axis cs:{304.514},{-12.545}) {\tikz\pgfuseplotmark{m4avb};};
\node[stars] at (axis cs:{304.619},{37.000}) {\tikz\pgfuseplotmark{m6a};};
\node[stars] at (axis cs:{304.663},{34.983}) {\tikz\pgfuseplotmark{m5cv};};
\node[stars] at (axis cs:{304.740},{39.004}) {\tikz\pgfuseplotmark{m6cb};};
\node[stars] at (axis cs:{304.848},{-19.119}) {\tikz\pgfuseplotmark{m5c};};
\node[stars] at (axis cs:{304.872},{13.217}) {\tikz\pgfuseplotmark{m6c};};
\node[stars] at (axis cs:{304.930},{-1.078}) {\tikz\pgfuseplotmark{m6b};};
\node[stars] at (axis cs:{305.001},{13.548}) {\tikz\pgfuseplotmark{m6b};};
\node[stars] at (axis cs:{305.063},{39.403}) {\tikz\pgfuseplotmark{m6cb};};
\node[stars] at (axis cs:{305.086},{14.569}) {\tikz\pgfuseplotmark{m6c};};
\node[stars] at (axis cs:{305.089},{17.793}) {\tikz\pgfuseplotmark{m6a};};
\node[stars] at (axis cs:{305.166},{-12.759}) {\tikz\pgfuseplotmark{m5a};};
\node[stars] at (axis cs:{305.216},{-35.674}) {\tikz\pgfuseplotmark{m6c};};
\node[stars] at (axis cs:{305.253},{-14.781}) {\tikz\pgfuseplotmark{m3bb};};
\node[stars] at (axis cs:{305.514},{24.446}) {\tikz\pgfuseplotmark{m6av};};
\node[stars] at (axis cs:{305.655},{31.265}) {\tikz\pgfuseplotmark{m6b};};
\node[stars] at (axis cs:{305.718},{14.551}) {\tikz\pgfuseplotmark{m6c};};
\node[stars] at (axis cs:{305.753},{-9.655}) {\tikz\pgfuseplotmark{m6c};};
\node[stars] at (axis cs:{305.795},{5.343}) {\tikz\pgfuseplotmark{m5c};};
\node[stars] at (axis cs:{305.935},{37.476}) {\tikz\pgfuseplotmark{m6bv};};
\node[stars] at (axis cs:{305.965},{32.190}) {\tikz\pgfuseplotmark{m4c};};
\node[stars] at (axis cs:{306.156},{1.068}) {\tikz\pgfuseplotmark{m6c};};
\node[stars] at (axis cs:{306.174},{1.369}) {\tikz\pgfuseplotmark{m6cv};};
\node[stars] at (axis cs:{306.362},{-28.663}) {\tikz\pgfuseplotmark{m6a};};
\node[stars] at (axis cs:{306.419},{21.409}) {\tikz\pgfuseplotmark{m6a};};
\node[stars] at (axis cs:{306.427},{-2.800}) {\tikz\pgfuseplotmark{m6b};};
\node[stars] at (axis cs:{306.434},{10.056}) {\tikz\pgfuseplotmark{m6c};};
\node[stars] at (axis cs:{306.505},{19.865}) {\tikz\pgfuseplotmark{m6c};};
\node[stars] at (axis cs:{306.507},{13.911}) {\tikz\pgfuseplotmark{m6c};};
\node[stars] at (axis cs:{306.596},{17.315}) {\tikz\pgfuseplotmark{m6c};};
\node[stars] at (axis cs:{306.721},{-37.403}) {\tikz\pgfuseplotmark{m6cb};};
\node[stars] at (axis cs:{306.782},{34.329}) {\tikz\pgfuseplotmark{m6c};};
\node[stars] at (axis cs:{306.809},{20.476}) {\tikz\pgfuseplotmark{m6c};};
\node[stars] at (axis cs:{306.830},{-18.212}) {\tikz\pgfuseplotmark{m5cb};};
\node[stars] at (axis cs:{306.893},{38.440}) {\tikz\pgfuseplotmark{m6a};};
\node[stars] at (axis cs:{307.031},{8.437}) {\tikz\pgfuseplotmark{m6c};};
\node[stars] at (axis cs:{307.070},{2.937}) {\tikz\pgfuseplotmark{m6c};};
\node[stars] at (axis cs:{307.104},{-3.358}) {\tikz\pgfuseplotmark{m6b};};
\node[stars] at (axis cs:{307.182},{-15.741}) {\tikz\pgfuseplotmark{m6c};};
\node[stars] at (axis cs:{307.195},{-35.596}) {\tikz\pgfuseplotmark{m6b};};
\node[stars] at (axis cs:{307.215},{-17.813}) {\tikz\pgfuseplotmark{m5ab};};
\node[stars] at (axis cs:{307.335},{36.455}) {\tikz\pgfuseplotmark{m6b};};
\node[stars] at (axis cs:{307.349},{30.369}) {\tikz\pgfuseplotmark{m4bv};};
\node[stars] at (axis cs:{307.381},{-22.391}) {\tikz\pgfuseplotmark{m6c};};
\node[stars] at (axis cs:{307.413},{-2.885}) {\tikz\pgfuseplotmark{m5b};};
\node[stars] at (axis cs:{307.475},{-18.583}) {\tikz\pgfuseplotmark{m6bb};};
\node[stars] at (axis cs:{307.575},{10.896}) {\tikz\pgfuseplotmark{m6bv};};
\node[stars] at (axis cs:{307.737},{-29.112}) {\tikz\pgfuseplotmark{m6c};};
\node[stars] at (axis cs:{307.742},{20.606}) {\tikz\pgfuseplotmark{m6c};};
\node[stars] at (axis cs:{307.747},{36.936}) {\tikz\pgfuseplotmark{m6cb};};
\node[stars] at (axis cs:{307.768},{-15.056}) {\tikz\pgfuseplotmark{m6b};};
\node[stars] at (axis cs:{307.901},{34.330}) {\tikz\pgfuseplotmark{m6c};};
\node[stars] at (axis cs:{307.993},{25.805}) {\tikz\pgfuseplotmark{m6c};};
\node[stars] at (axis cs:{308.085},{2.133}) {\tikz\pgfuseplotmark{m6c};};
\node[stars] at (axis cs:{308.099},{-9.853}) {\tikz\pgfuseplotmark{m6a};};
\node[stars] at (axis cs:{308.218},{-24.944}) {\tikz\pgfuseplotmark{m6c};};
\node[stars] at (axis cs:{308.303},{11.303}) {\tikz\pgfuseplotmark{m4bv};};
\node[stars] at (axis cs:{308.476},{35.251}) {\tikz\pgfuseplotmark{m5av};};
\node[stars] at (axis cs:{308.488},{13.027}) {\tikz\pgfuseplotmark{m5cv};};
\node[stars] at (axis cs:{308.500},{4.898}) {\tikz\pgfuseplotmark{m6c};};
\node[stars] at (axis cs:{308.542},{20.985}) {\tikz\pgfuseplotmark{m6c};};
\node[stars] at (axis cs:{308.549},{-13.721}) {\tikz\pgfuseplotmark{m6b};};
\node[stars] at (axis cs:{308.697},{-30.473}) {\tikz\pgfuseplotmark{m6c};};
\node[stars] at (axis cs:{308.731},{-38.090}) {\tikz\pgfuseplotmark{m6c};};
\node[stars] at (axis cs:{308.827},{14.674}) {\tikz\pgfuseplotmark{m5av};};
\node[stars] at (axis cs:{308.884},{-16.526}) {\tikz\pgfuseplotmark{m6c};};
\node[stars] at (axis cs:{309.035},{25.882}) {\tikz\pgfuseplotmark{m6c};};
\node[stars] at (axis cs:{309.182},{-2.550}) {\tikz\pgfuseplotmark{m5b};};
\node[stars] at (axis cs:{309.269},{26.462}) {\tikz\pgfuseplotmark{m6a};};
\node[stars] at (axis cs:{309.327},{0.097}) {\tikz\pgfuseplotmark{m6c};};
\node[stars] at (axis cs:{309.348},{38.328}) {\tikz\pgfuseplotmark{m6c};};
\node[stars] at (axis cs:{309.382},{31.572}) {\tikz\pgfuseplotmark{m6cb};};
\node[stars] at (axis cs:{309.387},{14.595}) {\tikz\pgfuseplotmark{m4a};};
\node[stars] at (axis cs:{309.455},{11.377}) {\tikz\pgfuseplotmark{m5c};};
\node[stars] at (axis cs:{309.468},{-25.109}) {\tikz\pgfuseplotmark{m6c};};
\node[stars] at (axis cs:{309.478},{18.269}) {\tikz\pgfuseplotmark{m6cv};};
\node[stars] at (axis cs:{309.585},{-1.105}) {\tikz\pgfuseplotmark{m4cv};};
\node[stars] at (axis cs:{309.631},{21.201}) {\tikz\pgfuseplotmark{m5a};};
\node[stars] at (axis cs:{309.633},{24.116}) {\tikz\pgfuseplotmark{m5b};};
\node[stars] at (axis cs:{309.646},{23.680}) {\tikz\pgfuseplotmark{m6b};};
\node[stars] at (axis cs:{309.683},{13.315}) {\tikz\pgfuseplotmark{m6a};};
\node[stars] at (axis cs:{309.748},{30.334}) {\tikz\pgfuseplotmark{m6a};};
\node[stars] at (axis cs:{309.771},{15.838}) {\tikz\pgfuseplotmark{m6b};};
\node[stars] at (axis cs:{309.774},{-4.929}) {\tikz\pgfuseplotmark{m6c};};
\node[stars] at (axis cs:{309.782},{10.086}) {\tikz\pgfuseplotmark{m5bb};};
\node[stars] at (axis cs:{309.794},{21.817}) {\tikz\pgfuseplotmark{m6bv};};
\node[stars] at (axis cs:{309.805},{-2.413}) {\tikz\pgfuseplotmark{m6c};};
\node[stars] at (axis cs:{309.818},{-14.955}) {\tikz\pgfuseplotmark{m5c};};
\node[stars] at (axis cs:{309.854},{0.486}) {\tikz\pgfuseplotmark{m5cb};};
\node[stars] at (axis cs:{309.910},{15.912}) {\tikz\pgfuseplotmark{m4av};};
\node[stars] at (axis cs:{309.966},{11.249}) {\tikz\pgfuseplotmark{m6cb};};
\node[stars] at (axis cs:{310.012},{-18.139}) {\tikz\pgfuseplotmark{m5c};};
\node[stars] at (axis cs:{310.049},{-23.774}) {\tikz\pgfuseplotmark{m6c};};
\node[stars] at (axis cs:{310.083},{-33.432}) {\tikz\pgfuseplotmark{m5c};};
\node[stars] at (axis cs:{310.135},{-16.124}) {\tikz\pgfuseplotmark{m6a};};
\node[stars] at (axis cs:{310.151},{29.805}) {\tikz\pgfuseplotmark{m6b};};
\node[stars] at (axis cs:{310.188},{19.936}) {\tikz\pgfuseplotmark{m6cb};};
\node[stars] at (axis cs:{310.261},{32.307}) {\tikz\pgfuseplotmark{m6ab};};
\node[stars] at (axis cs:{310.318},{14.583}) {\tikz\pgfuseplotmark{m6b};};
\node[stars] at (axis cs:{310.349},{-31.598}) {\tikz\pgfuseplotmark{m6a};};
\node[stars] at (axis cs:{310.351},{-26.000}) {\tikz\pgfuseplotmark{m6c};};
\node[stars] at (axis cs:{310.492},{17.521}) {\tikz\pgfuseplotmark{m6cv};};
\node[stars] at (axis cs:{310.721},{-39.559}) {\tikz\pgfuseplotmark{m6c};};
\node[stars] at (axis cs:{310.851},{35.588}) {\tikz\pgfuseplotmark{m6cv};};
\node[stars] at (axis cs:{310.865},{15.074}) {\tikz\pgfuseplotmark{m4cv};};
\node[stars] at (axis cs:{311.219},{25.271}) {\tikz\pgfuseplotmark{m5b};};
\node[stars] at (axis cs:{311.416},{30.720}) {\tikz\pgfuseplotmark{m4c};};
\node[stars] at (axis cs:{311.524},{-25.271}) {\tikz\pgfuseplotmark{m4c};};
\node[stars] at (axis cs:{311.542},{-21.514}) {\tikz\pgfuseplotmark{m6b};};
\node[stars] at (axis cs:{311.553},{33.970}) {\tikz\pgfuseplotmark{m2c};};
\node[stars] at (axis cs:{311.578},{-36.120}) {\tikz\pgfuseplotmark{m6c};};
\node[stars] at (axis cs:{311.584},{-39.199}) {\tikz\pgfuseplotmark{m5c};};
\node[stars] at (axis cs:{311.665},{16.124}) {\tikz\pgfuseplotmark{m4cb};};
\node[stars] at (axis cs:{311.765},{-2.487}) {\tikz\pgfuseplotmark{m6c};};
\node[stars] at (axis cs:{311.795},{34.374}) {\tikz\pgfuseplotmark{m5bv};};
\node[stars] at (axis cs:{311.852},{36.491}) {\tikz\pgfuseplotmark{m5avb};};
\node[stars] at (axis cs:{311.919},{-9.496}) {\tikz\pgfuseplotmark{m4a};};
\node[stars] at (axis cs:{311.934},{-5.028}) {\tikz\pgfuseplotmark{m4cv};};
\node[stars] at (axis cs:{311.949},{3.306}) {\tikz\pgfuseplotmark{m6c};};
\node[stars] at (axis cs:{311.951},{6.008}) {\tikz\pgfuseplotmark{m6ab};};
\node[stars] at (axis cs:{312.322},{-0.563}) {\tikz\pgfuseplotmark{m6cv};};
\node[stars] at (axis cs:{312.323},{-25.781}) {\tikz\pgfuseplotmark{m6a};};
\node[stars] at (axis cs:{312.336},{-18.036}) {\tikz\pgfuseplotmark{m6c};};
\node[stars] at (axis cs:{312.407},{12.545}) {\tikz\pgfuseplotmark{m6b};};
\node[stars] at (axis cs:{312.451},{7.864}) {\tikz\pgfuseplotmark{m6c};};
\node[stars] at (axis cs:{312.492},{-33.780}) {\tikz\pgfuseplotmark{m5bv};};
\node[stars] at (axis cs:{312.496},{5.544}) {\tikz\pgfuseplotmark{m6cb};};
\node[stars] at (axis cs:{312.674},{-12.545}) {\tikz\pgfuseplotmark{m6b};};
\node[stars] at (axis cs:{312.696},{-32.055}) {\tikz\pgfuseplotmark{m6c};};
\node[stars] at (axis cs:{312.753},{-37.913}) {\tikz\pgfuseplotmark{m6a};};
\node[stars] at (axis cs:{312.857},{-5.626}) {\tikz\pgfuseplotmark{m6bb};};
\node[stars] at (axis cs:{312.868},{28.250}) {\tikz\pgfuseplotmark{m6av};};
\node[stars] at (axis cs:{312.955},{-26.919}) {\tikz\pgfuseplotmark{m4bv};};
\node[stars] at (axis cs:{312.995},{-33.178}) {\tikz\pgfuseplotmark{m6b};};
\node[stars] at (axis cs:{313.002},{32.849}) {\tikz\pgfuseplotmark{m6c};};
\node[stars] at (axis cs:{313.032},{27.097}) {\tikz\pgfuseplotmark{m5av};};
\node[stars] at (axis cs:{313.036},{-5.507}) {\tikz\pgfuseplotmark{m6a};};
\node[stars] at (axis cs:{313.163},{-8.983}) {\tikz\pgfuseplotmark{m5a};};
\node[stars] at (axis cs:{313.255},{-23.783}) {\tikz\pgfuseplotmark{m6cb};};
\node[stars] at (axis cs:{313.273},{-11.574}) {\tikz\pgfuseplotmark{m6c};};
\node[stars] at (axis cs:{313.281},{29.649}) {\tikz\pgfuseplotmark{m6c};};
\node[stars] at (axis cs:{313.354},{-30.719}) {\tikz\pgfuseplotmark{m6c};};
\node[stars] at (axis cs:{313.417},{-39.810}) {\tikz\pgfuseplotmark{m5c};};
\node[stars] at (axis cs:{313.475},{33.438}) {\tikz\pgfuseplotmark{m5c};};
\node[stars] at (axis cs:{313.493},{-6.889}) {\tikz\pgfuseplotmark{m6c};};
\node[stars] at (axis cs:{313.527},{-27.925}) {\tikz\pgfuseplotmark{m6cv};};
\node[stars] at (axis cs:{313.640},{28.057}) {\tikz\pgfuseplotmark{m5bv};};
\node[stars] at (axis cs:{313.699},{-17.923}) {\tikz\pgfuseplotmark{m6a};};
\node[stars] at (axis cs:{313.903},{13.721}) {\tikz\pgfuseplotmark{m5cv};};
\node[stars] at (axis cs:{313.911},{12.569}) {\tikz\pgfuseplotmark{m6a};};
\node[stars] at (axis cs:{313.919},{4.532}) {\tikz\pgfuseplotmark{m6cb};};
\node[stars] at (axis cs:{314.197},{-26.296}) {\tikz\pgfuseplotmark{m6av};};
\node[stars] at (axis cs:{314.225},{-9.697}) {\tikz\pgfuseplotmark{m5cv};};
\node[stars] at (axis cs:{314.294},{0.464}) {\tikz\pgfuseplotmark{m6b};};
\node[stars] at (axis cs:{314.419},{-16.031}) {\tikz\pgfuseplotmark{m6b};};
\node[stars] at (axis cs:{314.568},{22.326}) {\tikz\pgfuseplotmark{m5c};};
\node[stars] at (axis cs:{314.608},{10.839}) {\tikz\pgfuseplotmark{m6ab};};
\node[stars] at (axis cs:{314.674},{-14.483}) {\tikz\pgfuseplotmark{m6bv};};
\node[stars] at (axis cs:{314.768},{4.293}) {\tikz\pgfuseplotmark{m5cb};};
\node[stars] at (axis cs:{314.901},{-19.035}) {\tikz\pgfuseplotmark{m6cv};};
\node[stars] at (axis cs:{314.999},{-36.129}) {\tikz\pgfuseplotmark{m6b};};
\node[stars] at (axis cs:{315.017},{7.516}) {\tikz\pgfuseplotmark{m6b};};
\node[stars] at (axis cs:{315.115},{19.329}) {\tikz\pgfuseplotmark{m6av};};
\node[stars] at (axis cs:{315.141},{-4.730}) {\tikz\pgfuseplotmark{m6c};};
\node[stars] at (axis cs:{315.216},{-17.531}) {\tikz\pgfuseplotmark{m6b};};
\node[stars] at (axis cs:{315.304},{36.026}) {\tikz\pgfuseplotmark{m6b};};
\node[stars] at (axis cs:{315.323},{-32.258}) {\tikz\pgfuseplotmark{m5a};};
\node[stars] at (axis cs:{315.439},{-26.881}) {\tikz\pgfuseplotmark{m6b};};
\node[stars] at (axis cs:{315.587},{39.509}) {\tikz\pgfuseplotmark{m6cb};};
\node[stars] at (axis cs:{315.613},{-38.531}) {\tikz\pgfuseplotmark{m6b};};
\node[stars] at (axis cs:{315.741},{-38.631}) {\tikz\pgfuseplotmark{m5c};};
\node[stars] at (axis cs:{315.757},{14.730}) {\tikz\pgfuseplotmark{m6c};};
\node[stars] at (axis cs:{315.763},{1.532}) {\tikz\pgfuseplotmark{m6cb};};
\node[stars] at (axis cs:{315.770},{38.657}) {\tikz\pgfuseplotmark{m6b};};
\node[stars] at (axis cs:{315.792},{-27.731}) {\tikz\pgfuseplotmark{m6c};};
\node[stars] at (axis cs:{316.020},{-5.823}) {\tikz\pgfuseplotmark{m6ab};};
\node[stars] at (axis cs:{316.101},{-19.855}) {\tikz\pgfuseplotmark{m5a};};
\node[stars] at (axis cs:{316.144},{5.503}) {\tikz\pgfuseplotmark{m6a};};
\node[stars] at (axis cs:{316.173},{2.942}) {\tikz\pgfuseplotmark{m6c};};
\node[stars] at (axis cs:{316.189},{2.270}) {\tikz\pgfuseplotmark{m6c};};
\node[stars] at (axis cs:{316.361},{5.958}) {\tikz\pgfuseplotmark{m6b};};
\node[stars] at (axis cs:{316.487},{-17.233}) {\tikz\pgfuseplotmark{m4b};};
\node[stars] at (axis cs:{316.505},{-30.125}) {\tikz\pgfuseplotmark{m6a};};
\node[stars] at (axis cs:{316.598},{26.924}) {\tikz\pgfuseplotmark{m6b};};
\node[stars] at (axis cs:{316.603},{-32.341}) {\tikz\pgfuseplotmark{m5c};};
\node[stars] at (axis cs:{316.626},{31.185}) {\tikz\pgfuseplotmark{m6av};};
\node[stars] at (axis cs:{316.666},{3.803}) {\tikz\pgfuseplotmark{m6c};};
\node[stars] at (axis cs:{316.725},{38.749}) {\tikz\pgfuseplotmark{m5cvb};};
\node[stars] at (axis cs:{316.782},{-25.006}) {\tikz\pgfuseplotmark{m5a};};
\node[stars] at (axis cs:{316.890},{15.659}) {\tikz\pgfuseplotmark{m6c};};
\node[stars] at (axis cs:{316.936},{-17.456}) {\tikz\pgfuseplotmark{m6c};};
\node[stars] at (axis cs:{317.117},{6.989}) {\tikz\pgfuseplotmark{m6c};};
\node[stars] at (axis cs:{317.140},{-21.194}) {\tikz\pgfuseplotmark{m5cb};};
\node[stars] at (axis cs:{317.162},{30.206}) {\tikz\pgfuseplotmark{m6avb};};
\node[stars] at (axis cs:{317.343},{-36.706}) {\tikz\pgfuseplotmark{m6c};};
\node[stars] at (axis cs:{317.387},{-20.557}) {\tikz\pgfuseplotmark{m6c};};
\node[stars] at (axis cs:{317.399},{-11.372}) {\tikz\pgfuseplotmark{m5a};};
\node[stars] at (axis cs:{317.493},{2.944}) {\tikz\pgfuseplotmark{m6c};};
\node[stars] at (axis cs:{317.585},{10.131}) {\tikz\pgfuseplotmark{m5avb};};
\node[stars] at (axis cs:{317.696},{-9.354}) {\tikz\pgfuseplotmark{m6c};};
\node[stars] at (axis cs:{317.922},{-14.472}) {\tikz\pgfuseplotmark{m6cv};};
\node[stars] at (axis cs:{318.234},{30.227}) {\tikz\pgfuseplotmark{m3c};};
\node[stars] at (axis cs:{318.263},{-39.425}) {\tikz\pgfuseplotmark{m5c};};
\node[stars] at (axis cs:{318.322},{-27.619}) {\tikz\pgfuseplotmark{m5c};};
\node[stars] at (axis cs:{318.329},{-36.423}) {\tikz\pgfuseplotmark{m6b};};
\node[stars] at (axis cs:{318.360},{36.633}) {\tikz\pgfuseplotmark{m6b};};
\node[stars] at (axis cs:{318.370},{15.982}) {\tikz\pgfuseplotmark{m6c};};
\node[stars] at (axis cs:{318.543},{29.901}) {\tikz\pgfuseplotmark{m6c};};
\node[stars] at (axis cs:{318.620},{10.007}) {\tikz\pgfuseplotmark{m4c};};
\node[stars] at (axis cs:{318.654},{0.092}) {\tikz\pgfuseplotmark{m6cv};};
\node[stars] at (axis cs:{318.698},{38.045}) {\tikz\pgfuseplotmark{m4avb};};
\node[stars] at (axis cs:{318.778},{-17.345}) {\tikz\pgfuseplotmark{m6bv};};
\node[stars] at (axis cs:{318.908},{-20.652}) {\tikz\pgfuseplotmark{m5c};};
\node[stars] at (axis cs:{318.937},{-15.171}) {\tikz\pgfuseplotmark{m5c};};
\node[stars] at (axis cs:{318.945},{-36.211}) {\tikz\pgfuseplotmark{m6b};};
\node[stars] at (axis cs:{318.956},{5.248}) {\tikz\pgfuseplotmark{m4b};};
\node[stars] at (axis cs:{319.074},{-9.214}) {\tikz\pgfuseplotmark{m6cv};};
\node[stars] at (axis cs:{319.164},{-1.608}) {\tikz\pgfuseplotmark{m6c};};
\node[stars] at (axis cs:{319.306},{-13.279}) {\tikz\pgfuseplotmark{m6c};};
\node[stars] at (axis cs:{319.354},{39.394}) {\tikz\pgfuseplotmark{m4cv};};
\node[stars] at (axis cs:{319.479},{34.897}) {\tikz\pgfuseplotmark{m4cvb};};
\node[stars] at (axis cs:{319.485},{-32.172}) {\tikz\pgfuseplotmark{m5a};};
\node[stars] at (axis cs:{319.489},{-17.985}) {\tikz\pgfuseplotmark{m5c};};
\node[stars] at (axis cs:{319.546},{-4.519}) {\tikz\pgfuseplotmark{m6a};};
\node[stars] at (axis cs:{319.717},{11.203}) {\tikz\pgfuseplotmark{m6bv};};
\node[stars] at (axis cs:{319.726},{-28.766}) {\tikz\pgfuseplotmark{m6c};};
\node[stars] at (axis cs:{319.842},{38.237}) {\tikz\pgfuseplotmark{m6bv};};
\node[stars] at (axis cs:{320.059},{22.026}) {\tikz\pgfuseplotmark{m6cv};};
\node[stars] at (axis cs:{320.268},{-4.560}) {\tikz\pgfuseplotmark{m6a};};
\node[stars] at (axis cs:{320.268},{23.856}) {\tikz\pgfuseplotmark{m6a};};
\node[stars] at (axis cs:{320.270},{7.354}) {\tikz\pgfuseplotmark{m6a};};
\node[stars] at (axis cs:{320.341},{32.613}) {\tikz\pgfuseplotmark{m6b};};
\node[stars] at (axis cs:{320.522},{19.804}) {\tikz\pgfuseplotmark{m4b};};
\node[stars] at (axis cs:{320.562},{-16.835}) {\tikz\pgfuseplotmark{m4cv};};
\node[stars] at (axis cs:{320.675},{30.310}) {\tikz\pgfuseplotmark{m6b};};
\node[stars] at (axis cs:{320.723},{6.811}) {\tikz\pgfuseplotmark{m5cb};};
\node[stars] at (axis cs:{320.734},{-9.319}) {\tikz\pgfuseplotmark{m6b};};
\node[stars] at (axis cs:{320.752},{-22.669}) {\tikz\pgfuseplotmark{m6a};};
\node[stars] at (axis cs:{320.952},{37.351}) {\tikz\pgfuseplotmark{m6cb};};
\node[stars] at (axis cs:{320.983},{-25.202}) {\tikz\pgfuseplotmark{m6c};};
\node[stars] at (axis cs:{320.995},{24.274}) {\tikz\pgfuseplotmark{m6a};};
\node[stars] at (axis cs:{321.031},{25.312}) {\tikz\pgfuseplotmark{m6c};};
\node[stars] at (axis cs:{321.033},{-22.747}) {\tikz\pgfuseplotmark{m6c};};
\node[stars] at (axis cs:{321.040},{-20.852}) {\tikz\pgfuseplotmark{m5cv};};
\node[stars] at (axis cs:{321.048},{-12.878}) {\tikz\pgfuseplotmark{m5c};};
\node[stars] at (axis cs:{321.096},{24.528}) {\tikz\pgfuseplotmark{m6c};};
\node[stars] at (axis cs:{321.102},{10.174}) {\tikz\pgfuseplotmark{m6c};};
\node[stars] at (axis cs:{321.142},{26.174}) {\tikz\pgfuseplotmark{m6a};};
\node[stars] at (axis cs:{321.215},{-3.398}) {\tikz\pgfuseplotmark{m6c};};
\node[stars] at (axis cs:{321.304},{-9.748}) {\tikz\pgfuseplotmark{m6a};};
\node[stars] at (axis cs:{321.321},{-3.557}) {\tikz\pgfuseplotmark{m5c};};
\node[stars] at (axis cs:{321.446},{36.667}) {\tikz\pgfuseplotmark{m6bvb};};
\node[stars] at (axis cs:{321.465},{0.534}) {\tikz\pgfuseplotmark{m6c};};
\node[stars] at (axis cs:{321.595},{-37.829}) {\tikz\pgfuseplotmark{m6a};};
\node[stars] at (axis cs:{321.611},{19.375}) {\tikz\pgfuseplotmark{m6bv};};
\node[stars] at (axis cs:{321.617},{1.103}) {\tikz\pgfuseplotmark{m6b};};
\node[stars] at (axis cs:{321.667},{-22.411}) {\tikz\pgfuseplotmark{m4av};};
\node[stars] at (axis cs:{321.812},{-21.196}) {\tikz\pgfuseplotmark{m6a};};
\node[stars] at (axis cs:{321.839},{37.117}) {\tikz\pgfuseplotmark{m5c};};
\node[stars] at (axis cs:{321.917},{27.608}) {\tikz\pgfuseplotmark{m5c};};
\node[stars] at (axis cs:{322.034},{32.225}) {\tikz\pgfuseplotmark{m6a};};
\node[stars] at (axis cs:{322.103},{8.195}) {\tikz\pgfuseplotmark{m6cv};};
\node[stars] at (axis cs:{322.181},{-21.807}) {\tikz\pgfuseplotmark{m5a};};
\node[stars] at (axis cs:{322.249},{22.179}) {\tikz\pgfuseplotmark{m6bv};};
\node[stars] at (axis cs:{322.250},{17.906}) {\tikz\pgfuseplotmark{m6c};};
\node[stars] at (axis cs:{322.394},{6.581}) {\tikz\pgfuseplotmark{m6c};};
\node[stars] at (axis cs:{322.487},{23.639}) {\tikz\pgfuseplotmark{m5a};};
\node[stars] at (axis cs:{322.790},{12.137}) {\tikz\pgfuseplotmark{m6b};};
\node[stars] at (axis cs:{322.890},{-5.571}) {\tikz\pgfuseplotmark{m3b};};
\node[stars] at (axis cs:{322.959},{23.845}) {\tikz\pgfuseplotmark{m6c};};
\node[stars] at (axis cs:{323.061},{-33.944}) {\tikz\pgfuseplotmark{m6b};};
\node[stars] at (axis cs:{323.139},{-24.590}) {\tikz\pgfuseplotmark{m6c};};
\node[stars] at (axis cs:{323.642},{22.754}) {\tikz\pgfuseplotmark{m6c};};
\node[stars] at (axis cs:{323.678},{1.829}) {\tikz\pgfuseplotmark{m6cv};};
\node[stars] at (axis cs:{323.691},{18.329}) {\tikz\pgfuseplotmark{m6c};};
\node[stars] at (axis cs:{323.694},{38.534}) {\tikz\pgfuseplotmark{m5b};};
\node[stars] at (axis cs:{323.713},{-20.084}) {\tikz\pgfuseplotmark{m6a};};
\node[stars] at (axis cs:{323.721},{-29.696}) {\tikz\pgfuseplotmark{m6c};};
\node[stars] at (axis cs:{323.816},{-23.454}) {\tikz\pgfuseplotmark{m6c};};
\node[stars] at (axis cs:{323.823},{-3.983}) {\tikz\pgfuseplotmark{m6a};};
\node[stars] at (axis cs:{323.830},{28.197}) {\tikz\pgfuseplotmark{m6c};};
\node[stars] at (axis cs:{323.863},{24.452}) {\tikz\pgfuseplotmark{m6c};};
\node[stars] at (axis cs:{324.046},{-26.171}) {\tikz\pgfuseplotmark{m6a};};
\node[stars] at (axis cs:{324.058},{30.055}) {\tikz\pgfuseplotmark{m6cb};};
\node[stars] at (axis cs:{324.203},{-33.048}) {\tikz\pgfuseplotmark{m6b};};
\node[stars] at (axis cs:{324.270},{-19.466}) {\tikz\pgfuseplotmark{m5av};};
\node[stars] at (axis cs:{324.391},{-0.390}) {\tikz\pgfuseplotmark{m6cb};};
\node[stars] at (axis cs:{324.432},{6.618}) {\tikz\pgfuseplotmark{m6cb};};
\node[stars] at (axis cs:{324.438},{-7.854}) {\tikz\pgfuseplotmark{m5a};};
\node[stars] at (axis cs:{324.439},{19.319}) {\tikz\pgfuseplotmark{m5cv};};
\node[stars] at (axis cs:{324.633},{5.772}) {\tikz\pgfuseplotmark{m6a};};
\node[stars] at (axis cs:{324.688},{25.499}) {\tikz\pgfuseplotmark{m6c};};
\node[stars] at (axis cs:{324.755},{20.265}) {\tikz\pgfuseplotmark{m6av};};
\node[stars] at (axis cs:{324.775},{-33.679}) {\tikz\pgfuseplotmark{m6c};};
\node[stars] at (axis cs:{324.867},{-10.577}) {\tikz\pgfuseplotmark{m6b};};
\node[stars] at (axis cs:{324.889},{2.243}) {\tikz\pgfuseplotmark{m5b};};
\node[stars] at (axis cs:{325.023},{-16.662}) {\tikz\pgfuseplotmark{m4av};};
\node[stars] at (axis cs:{325.387},{-14.047}) {\tikz\pgfuseplotmark{m5cv};};
\node[stars] at (axis cs:{325.503},{-23.263}) {\tikz\pgfuseplotmark{m5c};};
\node[stars] at (axis cs:{325.505},{35.510}) {\tikz\pgfuseplotmark{m6bv};};
\node[stars] at (axis cs:{325.542},{1.285}) {\tikz\pgfuseplotmark{m6a};};
\node[stars] at (axis cs:{325.564},{5.680}) {\tikz\pgfuseplotmark{m5cv};};
\node[stars] at (axis cs:{325.637},{10.824}) {\tikz\pgfuseplotmark{m6bv};};
\node[stars] at (axis cs:{325.665},{-18.866}) {\tikz\pgfuseplotmark{m5a};};
\node[stars] at (axis cs:{325.768},{-14.399}) {\tikz\pgfuseplotmark{m6b};};
\node[stars] at (axis cs:{325.806},{-19.621}) {\tikz\pgfuseplotmark{m6c};};
\node[stars] at (axis cs:{325.857},{38.284}) {\tikz\pgfuseplotmark{m6ab};};
\node[stars] at (axis cs:{325.916},{7.528}) {\tikz\pgfuseplotmark{m6c};};
\node[stars] at (axis cs:{325.992},{22.815}) {\tikz\pgfuseplotmark{m6c};};
\node[stars] at (axis cs:{326.004},{-14.749}) {\tikz\pgfuseplotmark{m6b};};
\node[stars] at (axis cs:{326.036},{28.742}) {\tikz\pgfuseplotmark{m5ab};};
\node[stars] at (axis cs:{326.046},{9.875}) {\tikz\pgfuseplotmark{m2cv};};
\node[stars] at (axis cs:{326.123},{-38.552}) {\tikz\pgfuseplotmark{m6c};};
\node[stars] at (axis cs:{326.128},{17.350}) {\tikz\pgfuseplotmark{m4cv};};
\node[stars] at (axis cs:{326.131},{14.772}) {\tikz\pgfuseplotmark{m6bv};};
\node[stars] at (axis cs:{326.161},{25.645}) {\tikz\pgfuseplotmark{m4cv};};
\node[stars] at (axis cs:{326.237},{-33.026}) {\tikz\pgfuseplotmark{m4c};};
\node[stars] at (axis cs:{326.251},{-9.082}) {\tikz\pgfuseplotmark{m5b};};
\node[stars] at (axis cs:{326.435},{35.857}) {\tikz\pgfuseplotmark{m6c};};
\node[stars] at (axis cs:{326.518},{22.949}) {\tikz\pgfuseplotmark{m5c};};
\node[stars] at (axis cs:{326.568},{-9.276}) {\tikz\pgfuseplotmark{m6bv};};
\node[stars] at (axis cs:{326.600},{25.563}) {\tikz\pgfuseplotmark{m6c};};
\node[stars] at (axis cs:{326.634},{-11.366}) {\tikz\pgfuseplotmark{m6a};};
\node[stars] at (axis cs:{326.760},{-16.127}) {\tikz\pgfuseplotmark{m3av};};
\node[stars] at (axis cs:{326.770},{17.194}) {\tikz\pgfuseplotmark{m6cv};};
\node[stars] at (axis cs:{326.808},{2.686}) {\tikz\pgfuseplotmark{m6a};};
\node[stars] at (axis cs:{326.909},{-5.917}) {\tikz\pgfuseplotmark{m6c};};
\node[stars] at (axis cs:{326.934},{-30.898}) {\tikz\pgfuseplotmark{m5b};};
\node[stars] at (axis cs:{327.035},{36.580}) {\tikz\pgfuseplotmark{m6c};};
\node[stars] at (axis cs:{327.122},{38.648}) {\tikz\pgfuseplotmark{m6b};};
\node[stars] at (axis cs:{327.362},{20.462}) {\tikz\pgfuseplotmark{m6c};};
\node[stars] at (axis cs:{327.421},{-12.723}) {\tikz\pgfuseplotmark{m6c};};
\node[stars] at (axis cs:{327.461},{30.174}) {\tikz\pgfuseplotmark{m5b};};
\node[stars] at (axis cs:{327.536},{17.286}) {\tikz\pgfuseplotmark{m5cv};};
\node[stars] at (axis cs:{327.554},{-16.845}) {\tikz\pgfuseplotmark{m6c};};
\node[stars] at (axis cs:{327.771},{39.537}) {\tikz\pgfuseplotmark{m6cv};};
\node[stars] at (axis cs:{327.893},{19.827}) {\tikz\pgfuseplotmark{m6ab};};
\node[stars] at (axis cs:{327.924},{-18.623}) {\tikz\pgfuseplotmark{m6c};};
\node[stars] at (axis cs:{328.076},{21.273}) {\tikz\pgfuseplotmark{m6cv};};
\node[stars] at (axis cs:{328.125},{28.793}) {\tikz\pgfuseplotmark{m6a};};
\node[stars] at (axis cs:{328.266},{25.925}) {\tikz\pgfuseplotmark{m5bv};};
\node[stars] at (axis cs:{328.324},{-13.552}) {\tikz\pgfuseplotmark{m5bv};};
\node[stars] at (axis cs:{328.406},{19.668}) {\tikz\pgfuseplotmark{m6a};};
\node[stars] at (axis cs:{328.482},{-37.365}) {\tikz\pgfuseplotmark{m3b};};
\node[stars] at (axis cs:{328.491},{6.865}) {\tikz\pgfuseplotmark{m6cv};};
\node[stars] at (axis cs:{328.543},{-4.276}) {\tikz\pgfuseplotmark{m6a};};
\node[stars] at (axis cs:{328.573},{19.718}) {\tikz\pgfuseplotmark{m6cb};};
\node[stars] at (axis cs:{328.650},{-3.301}) {\tikz\pgfuseplotmark{m6cb};};
\node[stars] at (axis cs:{328.982},{-30.606}) {\tikz\pgfuseplotmark{m6c};};
\node[stars] at (axis cs:{329.095},{-37.254}) {\tikz\pgfuseplotmark{m5c};};
\node[stars] at (axis cs:{329.100},{21.240}) {\tikz\pgfuseplotmark{m6cv};};
\node[stars] at (axis cs:{329.235},{12.076}) {\tikz\pgfuseplotmark{m6a};};
\node[stars] at (axis cs:{329.259},{-37.747}) {\tikz\pgfuseplotmark{m6cv};};
\node[stars] at (axis cs:{329.555},{-5.425}) {\tikz\pgfuseplotmark{m6c};};
\node[stars] at (axis cs:{329.682},{-21.183}) {\tikz\pgfuseplotmark{m6bv};};
\node[stars] at (axis cs:{329.729},{-4.373}) {\tikz\pgfuseplotmark{m6c};};
\node[stars] at (axis cs:{329.825},{-38.395}) {\tikz\pgfuseplotmark{m5c};};
\node[stars] at (axis cs:{330.033},{6.717}) {\tikz\pgfuseplotmark{m6b};};
\node[stars] at (axis cs:{330.111},{33.006}) {\tikz\pgfuseplotmark{m6c};};
\node[stars] at (axis cs:{330.209},{-28.454}) {\tikz\pgfuseplotmark{m5cvb};};
\node[stars] at (axis cs:{330.271},{0.605}) {\tikz\pgfuseplotmark{m6a};};
\node[stars] at (axis cs:{330.272},{13.120}) {\tikz\pgfuseplotmark{m6a};};
\node[stars] at (axis cs:{330.288},{8.257}) {\tikz\pgfuseplotmark{m6a};};
\node[stars] at (axis cs:{330.506},{10.974}) {\tikz\pgfuseplotmark{m6c};};
\node[stars] at (axis cs:{330.549},{-17.903}) {\tikz\pgfuseplotmark{m6c};};
\node[stars] at (axis cs:{330.819},{-6.522}) {\tikz\pgfuseplotmark{m6a};};
\node[stars] at (axis cs:{330.829},{-2.155}) {\tikz\pgfuseplotmark{m5av};};
\node[stars] at (axis cs:{330.829},{11.386}) {\tikz\pgfuseplotmark{m6a};};
\node[stars] at (axis cs:{331.099},{-29.916}) {\tikz\pgfuseplotmark{m6c};};
\node[stars] at (axis cs:{331.144},{32.942}) {\tikz\pgfuseplotmark{m6c};};
\node[stars] at (axis cs:{331.153},{-26.822}) {\tikz\pgfuseplotmark{m6bv};};
\node[stars] at (axis cs:{331.198},{-0.906}) {\tikz\pgfuseplotmark{m5c};};
\node[stars] at (axis cs:{331.264},{14.816}) {\tikz\pgfuseplotmark{m6cv};};
\node[stars] at (axis cs:{331.297},{26.674}) {\tikz\pgfuseplotmark{m6a};};
\node[stars] at (axis cs:{331.394},{28.964}) {\tikz\pgfuseplotmark{m6a};};
\node[stars] at (axis cs:{331.420},{5.058}) {\tikz\pgfuseplotmark{m5av};};
\node[stars] at (axis cs:{331.446},{-0.320}) {\tikz\pgfuseplotmark{m3bb};};
\node[stars] at (axis cs:{331.529},{-39.543}) {\tikz\pgfuseplotmark{m4c};};
\node[stars] at (axis cs:{331.609},{-13.870}) {\tikz\pgfuseplotmark{m4c};};
\node[stars] at (axis cs:{331.753},{25.345}) {\tikz\pgfuseplotmark{m4av};};
\node[stars] at (axis cs:{331.869},{19.475}) {\tikz\pgfuseplotmark{m6a};};
\node[stars] at (axis cs:{331.875},{18.001}) {\tikz\pgfuseplotmark{m6cv};};
\node[stars] at (axis cs:{331.960},{21.703}) {\tikz\pgfuseplotmark{m6a};};
\node[stars] at (axis cs:{332.072},{25.543}) {\tikz\pgfuseplotmark{m6b};};
\node[stars] at (axis cs:{332.096},{-32.988}) {\tikz\pgfuseplotmark{m4cv};};
\node[stars] at (axis cs:{332.108},{-34.044}) {\tikz\pgfuseplotmark{m5b};};
\node[stars] at (axis cs:{332.178},{-33.126}) {\tikz\pgfuseplotmark{m6c};};
\node[stars] at (axis cs:{332.246},{-18.520}) {\tikz\pgfuseplotmark{m6a};};
\node[stars] at (axis cs:{332.307},{33.172}) {\tikz\pgfuseplotmark{m6ab};};
\node[stars] at (axis cs:{332.371},{-23.659}) {\tikz\pgfuseplotmark{m6c};};
\node[stars] at (axis cs:{332.482},{-34.015}) {\tikz\pgfuseplotmark{m5c};};
\node[stars] at (axis cs:{332.497},{33.178}) {\tikz\pgfuseplotmark{m4c};};
\node[stars] at (axis cs:{332.501},{-28.292}) {\tikz\pgfuseplotmark{m6c};};
\node[stars] at (axis cs:{332.537},{-32.548}) {\tikz\pgfuseplotmark{m5b};};
\node[stars] at (axis cs:{332.550},{6.198}) {\tikz\pgfuseplotmark{m4av};};
\node[stars] at (axis cs:{332.579},{19.616}) {\tikz\pgfuseplotmark{m6c};};
\node[stars] at (axis cs:{332.588},{-3.894}) {\tikz\pgfuseplotmark{m6c};};
\node[stars] at (axis cs:{332.592},{14.630}) {\tikz\pgfuseplotmark{m6c};};
\node[stars] at (axis cs:{332.626},{20.978}) {\tikz\pgfuseplotmark{m6c};};
\node[stars] at (axis cs:{332.641},{-4.267}) {\tikz\pgfuseplotmark{m6b};};
\node[stars] at (axis cs:{332.656},{11.624}) {\tikz\pgfuseplotmark{m6a};};
\node[stars] at (axis cs:{332.656},{-11.565}) {\tikz\pgfuseplotmark{m5c};};
\node[stars] at (axis cs:{332.715},{30.553}) {\tikz\pgfuseplotmark{m6c};};
\node[stars] at (axis cs:{332.760},{-21.233}) {\tikz\pgfuseplotmark{m6b};};
\node[stars] at (axis cs:{332.964},{16.040}) {\tikz\pgfuseplotmark{m6b};};
\node[stars] at (axis cs:{333.034},{24.951}) {\tikz\pgfuseplotmark{m6b};};
\node[stars] at (axis cs:{333.107},{-14.194}) {\tikz\pgfuseplotmark{m6bv};};
\node[stars] at (axis cs:{333.183},{-4.721}) {\tikz\pgfuseplotmark{m6c};};
\node[stars] at (axis cs:{333.199},{34.605}) {\tikz\pgfuseplotmark{m5c};};
\node[stars] at (axis cs:{333.240},{-26.328}) {\tikz\pgfuseplotmark{m6c};};
\node[stars] at (axis cs:{333.411},{28.608}) {\tikz\pgfuseplotmark{m6b};};
\node[stars] at (axis cs:{333.435},{-25.181}) {\tikz\pgfuseplotmark{m6a};};
\node[stars] at (axis cs:{333.470},{39.715}) {\tikz\pgfuseplotmark{m4cv};};
\node[stars] at (axis cs:{333.575},{-21.074}) {\tikz\pgfuseplotmark{m5cb};};
\node[stars] at (axis cs:{333.578},{-27.767}) {\tikz\pgfuseplotmark{m5c};};
\node[stars] at (axis cs:{333.992},{37.748}) {\tikz\pgfuseplotmark{m4b};};
\node[stars] at (axis cs:{333.999},{8.549}) {\tikz\pgfuseplotmark{m6c};};
\node[stars] at (axis cs:{334.124},{27.804}) {\tikz\pgfuseplotmark{m6c};};
\node[stars] at (axis cs:{334.140},{-1.596}) {\tikz\pgfuseplotmark{m6c};};
\node[stars] at (axis cs:{334.156},{-25.898}) {\tikz\pgfuseplotmark{m6c};};
\node[stars] at (axis cs:{334.200},{-12.831}) {\tikz\pgfuseplotmark{m5c};};
\node[stars] at (axis cs:{334.208},{-7.783}) {\tikz\pgfuseplotmark{m4c};};
\node[stars] at (axis cs:{334.219},{-9.040}) {\tikz\pgfuseplotmark{m6a};};
\node[stars] at (axis cs:{334.249},{-23.140}) {\tikz\pgfuseplotmark{m6c};};
\node[stars] at (axis cs:{334.277},{-5.387}) {\tikz\pgfuseplotmark{m6a};};
\node[stars] at (axis cs:{334.518},{-0.238}) {\tikz\pgfuseplotmark{m6c};};
\node[stars] at (axis cs:{334.734},{37.769}) {\tikz\pgfuseplotmark{m6cb};};
\node[stars] at (axis cs:{334.753},{-13.305}) {\tikz\pgfuseplotmark{m6b};};
\node[stars] at (axis cs:{335.050},{-7.821}) {\tikz\pgfuseplotmark{m5c};};
\node[stars] at (axis cs:{335.115},{5.789}) {\tikz\pgfuseplotmark{m5cv};};
\node[stars] at (axis cs:{335.232},{8.187}) {\tikz\pgfuseplotmark{m6c};};
\node[stars] at (axis cs:{335.331},{28.330}) {\tikz\pgfuseplotmark{m5ab};};
\node[stars] at (axis cs:{335.379},{12.205}) {\tikz\pgfuseplotmark{m5av};};
\node[stars] at (axis cs:{335.398},{-21.598}) {\tikz\pgfuseplotmark{m5b};};
\node[stars] at (axis cs:{335.414},{-1.387}) {\tikz\pgfuseplotmark{m4av};};
\node[stars] at (axis cs:{335.710},{36.659}) {\tikz\pgfuseplotmark{m6c};};
\node[stars] at (axis cs:{335.879},{-24.762}) {\tikz\pgfuseplotmark{m6a};};
\node[stars] at (axis cs:{335.884},{-7.194}) {\tikz\pgfuseplotmark{m6b};};
\node[stars] at (axis cs:{335.915},{20.848}) {\tikz\pgfuseplotmark{m6cb};};
\node[stars] at (axis cs:{335.976},{38.573}) {\tikz\pgfuseplotmark{m6cv};};
\node[stars] at (axis cs:{336.029},{-4.837}) {\tikz\pgfuseplotmark{m6a};};
\node[stars] at (axis cs:{336.113},{-13.529}) {\tikz\pgfuseplotmark{m6a};};
\node[stars] at (axis cs:{336.319},{1.377}) {\tikz\pgfuseplotmark{m5av};};
\node[stars] at (axis cs:{336.420},{18.444}) {\tikz\pgfuseplotmark{m6c};};
\node[stars] at (axis cs:{336.545},{-23.682}) {\tikz\pgfuseplotmark{m6c};};
\node[stars] at (axis cs:{336.643},{-16.742}) {\tikz\pgfuseplotmark{m6c};};
\node[stars] at (axis cs:{336.656},{4.394}) {\tikz\pgfuseplotmark{m6a};};
\node[stars] at (axis cs:{336.690},{37.444}) {\tikz\pgfuseplotmark{m6cb};};
\node[stars] at (axis cs:{336.860},{39.810}) {\tikz\pgfuseplotmark{m6c};};
\node[stars] at (axis cs:{336.943},{31.840}) {\tikz\pgfuseplotmark{m6b};};
\node[stars] at (axis cs:{336.965},{4.695}) {\tikz\pgfuseplotmark{m5ab};};
\node[stars] at (axis cs:{337.046},{27.019}) {\tikz\pgfuseplotmark{m6c};};
\node[stars] at (axis cs:{337.163},{-39.132}) {\tikz\pgfuseplotmark{m5c};};
\node[stars] at (axis cs:{337.208},{-0.020}) {\tikz\pgfuseplotmark{m4bvb};};
\node[stars] at (axis cs:{337.283},{9.129}) {\tikz\pgfuseplotmark{m6a};};
\node[stars] at (axis cs:{337.293},{26.763}) {\tikz\pgfuseplotmark{m6a};};
\node[stars] at (axis cs:{337.442},{-27.107}) {\tikz\pgfuseplotmark{m6bv};};
\node[stars] at (axis cs:{337.491},{4.431}) {\tikz\pgfuseplotmark{m6a};};
\node[stars] at (axis cs:{337.506},{-12.915}) {\tikz\pgfuseplotmark{m6c};};
\node[stars] at (axis cs:{337.508},{32.573}) {\tikz\pgfuseplotmark{m6a};};
\node[stars] at (axis cs:{337.572},{-14.586}) {\tikz\pgfuseplotmark{m6c};};
\node[stars] at (axis cs:{337.662},{-10.678}) {\tikz\pgfuseplotmark{m5a};};
\node[stars] at (axis cs:{337.724},{-26.074}) {\tikz\pgfuseplotmark{m6c};};
\node[stars] at (axis cs:{337.826},{-6.555}) {\tikz\pgfuseplotmark{m6c};};
\node[stars] at (axis cs:{337.827},{-2.911}) {\tikz\pgfuseplotmark{m6c};};
\node[stars] at (axis cs:{337.876},{-32.346}) {\tikz\pgfuseplotmark{m4cb};};
\node[stars] at (axis cs:{337.893},{29.543}) {\tikz\pgfuseplotmark{m6cv};};
\node[stars] at (axis cs:{337.922},{-10.905}) {\tikz\pgfuseplotmark{m6c};};
\node[stars] at (axis cs:{338.110},{39.780}) {\tikz\pgfuseplotmark{m6bb};};
\node[stars] at (axis cs:{338.148},{20.230}) {\tikz\pgfuseplotmark{m6c};};
\node[stars] at (axis cs:{338.195},{15.863}) {\tikz\pgfuseplotmark{m6c};};
\node[stars] at (axis cs:{338.512},{-1.574}) {\tikz\pgfuseplotmark{m6b};};
\node[stars] at (axis cs:{338.673},{-20.708}) {\tikz\pgfuseplotmark{m5c};};
\node[stars] at (axis cs:{338.839},{-0.117}) {\tikz\pgfuseplotmark{m4b};};
\node[stars] at (axis cs:{338.902},{-23.991}) {\tikz\pgfuseplotmark{m6b};};
\node[stars] at (axis cs:{338.953},{-17.460}) {\tikz\pgfuseplotmark{m6c};};
\node[stars] at (axis cs:{338.968},{39.634}) {\tikz\pgfuseplotmark{m6avb};};
\node[stars] at (axis cs:{339.033},{35.577}) {\tikz\pgfuseplotmark{m6b};};
\node[stars] at (axis cs:{339.148},{-31.664}) {\tikz\pgfuseplotmark{m6ab};};
\node[stars] at (axis cs:{339.151},{11.696}) {\tikz\pgfuseplotmark{m6c};};
\node[stars] at (axis cs:{339.203},{35.652}) {\tikz\pgfuseplotmark{m6c};};
\node[stars] at (axis cs:{339.270},{12.577}) {\tikz\pgfuseplotmark{m6c};};
\node[stars] at (axis cs:{339.439},{-4.228}) {\tikz\pgfuseplotmark{m5b};};
\node[stars] at (axis cs:{339.592},{-7.897}) {\tikz\pgfuseplotmark{m6c};};
\node[stars] at (axis cs:{339.686},{-28.748}) {\tikz\pgfuseplotmark{m6c};};
\node[stars] at (axis cs:{339.714},{-33.081}) {\tikz\pgfuseplotmark{m6av};};
\node[stars] at (axis cs:{339.719},{19.522}) {\tikz\pgfuseplotmark{m6a};};
\node[stars] at (axis cs:{339.815},{39.050}) {\tikz\pgfuseplotmark{m5bv};};
\node[stars] at (axis cs:{339.893},{37.593}) {\tikz\pgfuseplotmark{m6b};};
\node[stars] at (axis cs:{339.934},{-28.326}) {\tikz\pgfuseplotmark{m6cb};};
\node[stars] at (axis cs:{339.946},{19.681}) {\tikz\pgfuseplotmark{m6cv};};
\node[stars] at (axis cs:{340.093},{-30.659}) {\tikz\pgfuseplotmark{m6bv};};
\node[stars] at (axis cs:{340.164},{-27.043}) {\tikz\pgfuseplotmark{m4cv};};
\node[stars] at (axis cs:{340.200},{-3.554}) {\tikz\pgfuseplotmark{m6c};};
\node[stars] at (axis cs:{340.220},{14.549}) {\tikz\pgfuseplotmark{m6a};};
\node[stars] at (axis cs:{340.366},{10.831}) {\tikz\pgfuseplotmark{m3c};};
\node[stars] at (axis cs:{340.380},{30.966}) {\tikz\pgfuseplotmark{m6c};};
\node[stars] at (axis cs:{340.439},{29.308}) {\tikz\pgfuseplotmark{m5a};};
\node[stars] at (axis cs:{340.489},{14.516}) {\tikz\pgfuseplotmark{m6b};};
\node[stars] at (axis cs:{340.525},{-5.102}) {\tikz\pgfuseplotmark{m6cvb};};
\node[stars] at (axis cs:{340.592},{-29.361}) {\tikz\pgfuseplotmark{m6c};};
\node[stars] at (axis cs:{340.731},{37.803}) {\tikz\pgfuseplotmark{m6c};};
\node[stars] at (axis cs:{340.751},{30.221}) {\tikz\pgfuseplotmark{m3bvb};};
\node[stars] at (axis cs:{340.809},{-6.963}) {\tikz\pgfuseplotmark{m6c};};
\node[stars] at (axis cs:{340.897},{-18.830}) {\tikz\pgfuseplotmark{m5av};};
\node[stars] at (axis cs:{341.022},{39.465}) {\tikz\pgfuseplotmark{m6bb};};
\node[stars] at (axis cs:{341.221},{39.201}) {\tikz\pgfuseplotmark{m6c};};
\node[stars] at (axis cs:{341.367},{19.366}) {\tikz\pgfuseplotmark{m6c};};
\node[stars] at (axis cs:{341.394},{30.442}) {\tikz\pgfuseplotmark{m6cb};};
\node[stars] at (axis cs:{341.633},{23.565}) {\tikz\pgfuseplotmark{m4b};};
\node[stars] at (axis cs:{341.673},{12.173}) {\tikz\pgfuseplotmark{m4cv};};
\node[stars] at (axis cs:{341.830},{-34.162}) {\tikz\pgfuseplotmark{m6c};};
\node[stars] at (axis cs:{341.888},{-19.613}) {\tikz\pgfuseplotmark{m5c};};
\node[stars] at (axis cs:{341.928},{-14.056}) {\tikz\pgfuseplotmark{m6ab};};
\node[stars] at (axis cs:{341.984},{-25.912}) {\tikz\pgfuseplotmark{m6c};};
\node[stars] at (axis cs:{342.046},{37.417}) {\tikz\pgfuseplotmark{m6av};};
\node[stars] at (axis cs:{342.126},{-10.555}) {\tikz\pgfuseplotmark{m6cv};};
\node[stars] at (axis cs:{342.398},{-13.592}) {\tikz\pgfuseplotmark{m4bv};};
\node[stars] at (axis cs:{342.497},{-32.805}) {\tikz\pgfuseplotmark{m6c};};
\node[stars] at (axis cs:{342.501},{24.602}) {\tikz\pgfuseplotmark{m4a};};
\node[stars] at (axis cs:{342.759},{-39.157}) {\tikz\pgfuseplotmark{m5c};};
\node[stars] at (axis cs:{342.837},{-29.536}) {\tikz\pgfuseplotmark{m6b};};
\node[stars] at (axis cs:{343.100},{9.836}) {\tikz\pgfuseplotmark{m5c};};
\node[stars] at (axis cs:{343.131},{-32.875}) {\tikz\pgfuseplotmark{m5avb};};
\node[stars] at (axis cs:{343.154},{-7.579}) {\tikz\pgfuseplotmark{m4av};};
\node[stars] at (axis cs:{343.259},{16.841}) {\tikz\pgfuseplotmark{m6bv};};
\node[stars] at (axis cs:{343.370},{-11.616}) {\tikz\pgfuseplotmark{m6av};};
\node[stars] at (axis cs:{343.524},{-19.175}) {\tikz\pgfuseplotmark{m6c};};
\node[stars] at (axis cs:{343.642},{-7.205}) {\tikz\pgfuseplotmark{m6c};};
\node[stars] at (axis cs:{343.648},{16.942}) {\tikz\pgfuseplotmark{m6cv};};
\node[stars] at (axis cs:{343.663},{-15.821}) {\tikz\pgfuseplotmark{m3cv};};
\node[stars] at (axis cs:{343.689},{-16.272}) {\tikz\pgfuseplotmark{m6av};};
\node[stars] at (axis cs:{343.748},{1.065}) {\tikz\pgfuseplotmark{m6b};};
\node[stars] at (axis cs:{343.761},{37.077}) {\tikz\pgfuseplotmark{m6b};};
\node[stars] at (axis cs:{343.796},{-4.988}) {\tikz\pgfuseplotmark{m6av};};
\node[stars] at (axis cs:{343.807},{8.816}) {\tikz\pgfuseplotmark{m5b};};
\node[stars] at (axis cs:{343.812},{-36.388}) {\tikz\pgfuseplotmark{m6c};};
\node[stars] at (axis cs:{343.877},{-20.139}) {\tikz\pgfuseplotmark{m6c};};
\node[stars] at (axis cs:{343.935},{36.351}) {\tikz\pgfuseplotmark{m6ab};};
\node[stars] at (axis cs:{343.965},{-31.633}) {\tikz\pgfuseplotmark{m6b};};
\node[stars] at (axis cs:{343.987},{-32.540}) {\tikz\pgfuseplotmark{m4c};};
\node[stars] at (axis cs:{344.100},{-31.565}) {\tikz\pgfuseplotmark{m6cv};};
\node[stars] at (axis cs:{344.322},{-4.810}) {\tikz\pgfuseplotmark{m6c};};
\node[stars] at (axis cs:{344.367},{20.769}) {\tikz\pgfuseplotmark{m5cv};};
\node[stars] at (axis cs:{344.387},{3.810}) {\tikz\pgfuseplotmark{m6c};};
\node[stars] at (axis cs:{344.413},{-29.622}) {\tikz\pgfuseplotmark{m1cv};};
\node[stars] at (axis cs:{344.420},{39.309}) {\tikz\pgfuseplotmark{m6c};};
\node[stars] at (axis cs:{344.565},{-2.395}) {\tikz\pgfuseplotmark{m6c};};
\node[stars] at (axis cs:{344.599},{-1.410}) {\tikz\pgfuseplotmark{m6c};};
\node[stars] at (axis cs:{344.646},{-35.523}) {\tikz\pgfuseplotmark{m6c};};
\node[stars] at (axis cs:{344.646},{9.357}) {\tikz\pgfuseplotmark{m6cvb};};
\node[stars] at (axis cs:{344.678},{7.340}) {\tikz\pgfuseplotmark{m6c};};
\node[stars] at (axis cs:{344.799},{11.729}) {\tikz\pgfuseplotmark{m6a};};
\node[stars] at (axis cs:{344.864},{0.963}) {\tikz\pgfuseplotmark{m5c};};
\node[stars] at (axis cs:{344.899},{-13.071}) {\tikz\pgfuseplotmark{m6b};};
\node[stars] at (axis cs:{344.899},{-29.462}) {\tikz\pgfuseplotmark{m6av};};
\node[stars] at (axis cs:{345.024},{-25.164}) {\tikz\pgfuseplotmark{m6a};};
\node[stars] at (axis cs:{345.103},{-25.627}) {\tikz\pgfuseplotmark{m6c};};
\node[stars] at (axis cs:{345.158},{0.186}) {\tikz\pgfuseplotmark{m6c};};
\node[stars] at (axis cs:{345.179},{3.012}) {\tikz\pgfuseplotmark{m6a};};
\node[stars] at (axis cs:{345.331},{-28.854}) {\tikz\pgfuseplotmark{m6a};};
\node[stars] at (axis cs:{345.346},{-22.791}) {\tikz\pgfuseplotmark{m6c};};
\node[stars] at (axis cs:{345.349},{-7.061}) {\tikz\pgfuseplotmark{m6c};};
\node[stars] at (axis cs:{345.382},{-4.711}) {\tikz\pgfuseplotmark{m6b};};
\node[stars] at (axis cs:{345.432},{3.531}) {\tikz\pgfuseplotmark{m6c};};
\node[stars] at (axis cs:{345.636},{-6.574}) {\tikz\pgfuseplotmark{m6c};};
\node[stars] at (axis cs:{345.642},{-36.421}) {\tikz\pgfuseplotmark{m6cb};};
\node[stars] at (axis cs:{345.684},{-20.871}) {\tikz\pgfuseplotmark{m6b};};
\node[stars] at (axis cs:{345.874},{-34.749}) {\tikz\pgfuseplotmark{m5b};};
\node[stars] at (axis cs:{345.944},{28.083}) {\tikz\pgfuseplotmark{m2cv};};
\node[stars] at (axis cs:{345.969},{3.820}) {\tikz\pgfuseplotmark{m4cv};};
\node[stars] at (axis cs:{346.003},{6.616}) {\tikz\pgfuseplotmark{m6c};};
\node[stars] at (axis cs:{346.190},{15.205}) {\tikz\pgfuseplotmark{m2cv};};
\node[stars] at (axis cs:{346.276},{16.563}) {\tikz\pgfuseplotmark{m6c};};
\node[stars] at (axis cs:{346.291},{-7.694}) {\tikz\pgfuseplotmark{m5cb};};
\node[stars] at (axis cs:{346.304},{-17.079}) {\tikz\pgfuseplotmark{m6c};};
\node[stars] at (axis cs:{346.323},{1.307}) {\tikz\pgfuseplotmark{m6c};};
\node[stars] at (axis cs:{346.576},{18.518}) {\tikz\pgfuseplotmark{m6c};};
\node[stars] at (axis cs:{346.633},{19.911}) {\tikz\pgfuseplotmark{m6cv};};
\node[stars] at (axis cs:{346.670},{-23.743}) {\tikz\pgfuseplotmark{m4c};};
\node[stars] at (axis cs:{346.723},{-38.892}) {\tikz\pgfuseplotmark{m6a};};
\node[stars] at (axis cs:{346.751},{9.409}) {\tikz\pgfuseplotmark{m5av};};
\node[stars] at (axis cs:{346.771},{35.636}) {\tikz\pgfuseplotmark{m6cv};};
\node[stars] at (axis cs:{346.778},{25.468}) {\tikz\pgfuseplotmark{m5av};};
\node[stars] at (axis cs:{346.866},{32.825}) {\tikz\pgfuseplotmark{m6cvb};};
\node[stars] at (axis cs:{346.870},{21.134}) {\tikz\pgfuseplotmark{m6bv};};
\node[stars] at (axis cs:{347.088},{-28.824}) {\tikz\pgfuseplotmark{m6a};};
\node[stars] at (axis cs:{347.171},{2.128}) {\tikz\pgfuseplotmark{m5c};};
\node[stars] at (axis cs:{347.362},{-21.172}) {\tikz\pgfuseplotmark{m4a};};
\node[stars] at (axis cs:{347.381},{8.677}) {\tikz\pgfuseplotmark{m5bv};};
\node[stars] at (axis cs:{347.436},{-28.088}) {\tikz\pgfuseplotmark{m6b};};
\node[stars] at (axis cs:{347.456},{-14.510}) {\tikz\pgfuseplotmark{m6c};};
\node[stars] at (axis cs:{347.479},{-22.457}) {\tikz\pgfuseplotmark{m5a};};
\node[stars] at (axis cs:{347.506},{9.822}) {\tikz\pgfuseplotmark{m5c};};
\node[stars] at (axis cs:{347.678},{17.594}) {\tikz\pgfuseplotmark{m6a};};
\node[stars] at (axis cs:{347.934},{8.720}) {\tikz\pgfuseplotmark{m5cv};};
\node[stars] at (axis cs:{347.955},{26.847}) {\tikz\pgfuseplotmark{m6cb};};
\node[stars] at (axis cs:{348.267},{29.441}) {\tikz\pgfuseplotmark{m6c};};
\node[stars] at (axis cs:{348.360},{11.065}) {\tikz\pgfuseplotmark{m6a};};
\node[stars] at (axis cs:{348.497},{19.634}) {\tikz\pgfuseplotmark{m6c};};
\node[stars] at (axis cs:{348.581},{-6.049}) {\tikz\pgfuseplotmark{m4c};};
\node[stars] at (axis cs:{348.590},{29.772}) {\tikz\pgfuseplotmark{m6cb};};
\node[stars] at (axis cs:{348.652},{24.103}) {\tikz\pgfuseplotmark{m6c};};
\node[stars] at (axis cs:{348.667},{-10.689}) {\tikz\pgfuseplotmark{m6b};};
\node[stars] at (axis cs:{348.893},{-3.496}) {\tikz\pgfuseplotmark{m6a};};
\node[stars] at (axis cs:{348.973},{-9.088}) {\tikz\pgfuseplotmark{m4c};};
\node[stars] at (axis cs:{349.212},{-7.726}) {\tikz\pgfuseplotmark{m5bv};};
\node[stars] at (axis cs:{349.286},{-28.438}) {\tikz\pgfuseplotmark{m6cv};};
\node[stars] at (axis cs:{349.291},{3.282}) {\tikz\pgfuseplotmark{m4a};};
\node[stars] at (axis cs:{349.417},{-11.713}) {\tikz\pgfuseplotmark{m6c};};
\node[stars] at (axis cs:{349.476},{-9.182}) {\tikz\pgfuseplotmark{m4cv};};
\node[stars] at (axis cs:{349.706},{-32.532}) {\tikz\pgfuseplotmark{m4c};};
\node[stars] at (axis cs:{349.740},{-9.611}) {\tikz\pgfuseplotmark{m5bv};};
\node[stars] at (axis cs:{349.778},{-13.458}) {\tikz\pgfuseplotmark{m5cb};};
\node[stars] at (axis cs:{349.850},{-5.124}) {\tikz\pgfuseplotmark{m6a};};
\node[stars] at (axis cs:{349.850},{-18.075}) {\tikz\pgfuseplotmark{m6b};};
\node[stars] at (axis cs:{349.864},{34.793}) {\tikz\pgfuseplotmark{m6c};};
\node[stars] at (axis cs:{349.930},{-33.708}) {\tikz\pgfuseplotmark{m6c};};
\node[stars] at (axis cs:{350.086},{5.381}) {\tikz\pgfuseplotmark{m5b};};
\node[stars] at (axis cs:{350.159},{23.740}) {\tikz\pgfuseplotmark{m5av};};
\node[stars] at (axis cs:{350.170},{-5.908}) {\tikz\pgfuseplotmark{m6cv};};
\node[stars] at (axis cs:{350.206},{30.415}) {\tikz\pgfuseplotmark{m6a};};
\node[stars] at (axis cs:{350.222},{38.182}) {\tikz\pgfuseplotmark{m6ab};};
\node[stars] at (axis cs:{350.315},{-26.987}) {\tikz\pgfuseplotmark{m6a};};
\node[stars] at (axis cs:{350.479},{31.812}) {\tikz\pgfuseplotmark{m5c};};
\node[stars] at (axis cs:{350.619},{25.918}) {\tikz\pgfuseplotmark{m6c};};
\node[stars] at (axis cs:{350.663},{-15.039}) {\tikz\pgfuseplotmark{m5c};};
\node[stars] at (axis cs:{350.669},{20.829}) {\tikz\pgfuseplotmark{m6c};};
\node[stars] at (axis cs:{350.743},{-20.101}) {\tikz\pgfuseplotmark{m4b};};
\node[stars] at (axis cs:{350.769},{12.314}) {\tikz\pgfuseplotmark{m5b};};
\node[stars] at (axis cs:{350.883},{0.291}) {\tikz\pgfuseplotmark{m6c};};
\node[stars] at (axis cs:{351.033},{-18.688}) {\tikz\pgfuseplotmark{m6c};};
\node[stars] at (axis cs:{351.212},{32.385}) {\tikz\pgfuseplotmark{m6a};};
\node[stars] at (axis cs:{351.345},{23.404}) {\tikz\pgfuseplotmark{m4c};};
\node[stars] at (axis cs:{351.512},{-20.642}) {\tikz\pgfuseplotmark{m4cv};};
\node[stars] at (axis cs:{351.648},{-21.741}) {\tikz\pgfuseplotmark{m6c};};
\node[stars] at (axis cs:{351.733},{1.256}) {\tikz\pgfuseplotmark{m5bvb};};
\node[stars] at (axis cs:{351.812},{1.123}) {\tikz\pgfuseplotmark{m6cv};};
\node[stars] at (axis cs:{351.918},{25.167}) {\tikz\pgfuseplotmark{m6bv};};
\node[stars] at (axis cs:{351.992},{6.379}) {\tikz\pgfuseplotmark{m4cv};};
\node[stars] at (axis cs:{352.003},{-35.544}) {\tikz\pgfuseplotmark{m6c};};
\node[stars] at (axis cs:{352.022},{-11.449}) {\tikz\pgfuseplotmark{m6c};};
\node[stars] at (axis cs:{352.253},{-9.266}) {\tikz\pgfuseplotmark{m6c};};
\node[stars] at (axis cs:{352.274},{23.048}) {\tikz\pgfuseplotmark{m6c};};
\node[stars] at (axis cs:{352.289},{12.760}) {\tikz\pgfuseplotmark{m5a};};
\node[stars] at (axis cs:{352.362},{-1.791}) {\tikz\pgfuseplotmark{m6c};};
\node[stars] at (axis cs:{352.384},{-4.532}) {\tikz\pgfuseplotmark{m6c};};
\node[stars] at (axis cs:{352.665},{38.662}) {\tikz\pgfuseplotmark{m6bv};};
\node[stars] at (axis cs:{352.755},{-6.288}) {\tikz\pgfuseplotmark{m6c};};
\node[stars] at (axis cs:{352.823},{39.236}) {\tikz\pgfuseplotmark{m5cv};};
\node[stars] at (axis cs:{352.881},{-4.087}) {\tikz\pgfuseplotmark{m6c};};
\node[stars] at (axis cs:{352.925},{-21.369}) {\tikz\pgfuseplotmark{m6c};};
\node[stars] at (axis cs:{352.929},{28.403}) {\tikz\pgfuseplotmark{m6cv};};
\node[stars] at (axis cs:{352.990},{-1.086}) {\tikz\pgfuseplotmark{m6c};};
\node[stars] at (axis cs:{353.121},{23.843}) {\tikz\pgfuseplotmark{m6c};};
\node[stars] at (axis cs:{353.243},{-37.818}) {\tikz\pgfuseplotmark{m4cv};};
\node[stars] at (axis cs:{353.319},{-20.914}) {\tikz\pgfuseplotmark{m5a};};
\node[stars] at (axis cs:{353.367},{22.499}) {\tikz\pgfuseplotmark{m5cv};};
\node[stars] at (axis cs:{353.481},{20.841}) {\tikz\pgfuseplotmark{m6cv};};
\node[stars] at (axis cs:{353.488},{31.325}) {\tikz\pgfuseplotmark{m5bv};};
\node[stars] at (axis cs:{353.538},{-1.247}) {\tikz\pgfuseplotmark{m6b};};
\node[stars] at (axis cs:{353.659},{33.497}) {\tikz\pgfuseplotmark{m6a};};
\node[stars] at (axis cs:{353.695},{38.024}) {\tikz\pgfuseplotmark{m6c};};
\node[stars] at (axis cs:{353.706},{-15.246}) {\tikz\pgfuseplotmark{m6b};};
\node[stars] at (axis cs:{353.869},{1.313}) {\tikz\pgfuseplotmark{m6c};};
\node[stars] at (axis cs:{353.884},{-7.464}) {\tikz\pgfuseplotmark{m6c};};
\node[stars] at (axis cs:{353.983},{24.561}) {\tikz\pgfuseplotmark{m6c};};
\node[stars] at (axis cs:{354.097},{2.102}) {\tikz\pgfuseplotmark{m6a};};
\node[stars] at (axis cs:{354.127},{32.904}) {\tikz\pgfuseplotmark{m6c};};
\node[stars] at (axis cs:{354.415},{-13.060}) {\tikz\pgfuseplotmark{m6ab};};
\node[stars] at (axis cs:{354.416},{16.825}) {\tikz\pgfuseplotmark{m6cb};};
\node[stars] at (axis cs:{354.487},{18.401}) {\tikz\pgfuseplotmark{m5cv};};
\node[stars] at (axis cs:{354.513},{-15.095}) {\tikz\pgfuseplotmark{m6c};};
\node[stars] at (axis cs:{354.946},{-14.222}) {\tikz\pgfuseplotmark{m5b};};
\node[stars] at (axis cs:{354.979},{9.677}) {\tikz\pgfuseplotmark{m6b};};
\node[stars] at (axis cs:{354.988},{5.626}) {\tikz\pgfuseplotmark{m4cv};};
\node[stars] at (axis cs:{355.159},{-32.073}) {\tikz\pgfuseplotmark{m5cv};};
\node[stars] at (axis cs:{355.169},{36.721}) {\tikz\pgfuseplotmark{m6c};};
\node[stars] at (axis cs:{355.287},{-11.680}) {\tikz\pgfuseplotmark{m6b};};
\node[stars] at (axis cs:{355.394},{-18.027}) {\tikz\pgfuseplotmark{m5c};};
\node[stars] at (axis cs:{355.441},{-17.816}) {\tikz\pgfuseplotmark{m5avb};};
\node[stars] at (axis cs:{355.486},{7.251}) {\tikz\pgfuseplotmark{m6bv};};
\node[stars] at (axis cs:{355.512},{1.780}) {\tikz\pgfuseplotmark{m4c};};
\node[stars] at (axis cs:{355.616},{-15.448}) {\tikz\pgfuseplotmark{m5c};};
\node[stars] at (axis cs:{355.681},{-14.545}) {\tikz\pgfuseplotmark{m4c};};
\node[stars] at (axis cs:{355.683},{16.336}) {\tikz\pgfuseplotmark{m6c};};
\node[stars] at (axis cs:{355.843},{10.331}) {\tikz\pgfuseplotmark{m5bv};};
\node[stars] at (axis cs:{355.998},{29.361}) {\tikz\pgfuseplotmark{m5b};};
\node[stars] at (axis cs:{356.050},{-18.277}) {\tikz\pgfuseplotmark{m5c};};
\node[stars] at (axis cs:{356.121},{-26.246}) {\tikz\pgfuseplotmark{m6cb};};
\node[stars] at (axis cs:{356.504},{-18.678}) {\tikz\pgfuseplotmark{m6ab};};
\node[stars] at (axis cs:{356.598},{3.487}) {\tikz\pgfuseplotmark{m5bv};};
\node[stars] at (axis cs:{356.816},{-11.911}) {\tikz\pgfuseplotmark{m6a};};
\node[stars] at (axis cs:{356.986},{-2.761}) {\tikz\pgfuseplotmark{m5c};};
\node[stars] at (axis cs:{357.135},{-6.380}) {\tikz\pgfuseplotmark{m6bv};};
\node[stars] at (axis cs:{357.206},{2.214}) {\tikz\pgfuseplotmark{m6c};};
\node[stars] at (axis cs:{357.231},{-28.130}) {\tikz\pgfuseplotmark{m5a};};
\node[stars] at (axis cs:{357.364},{1.076}) {\tikz\pgfuseplotmark{m6a};};
\node[stars] at (axis cs:{357.382},{-15.861}) {\tikz\pgfuseplotmark{m6c};};
\node[stars] at (axis cs:{357.414},{28.842}) {\tikz\pgfuseplotmark{m6b};};
\node[stars] at (axis cs:{357.421},{36.425}) {\tikz\pgfuseplotmark{m6bv};};
\node[stars] at (axis cs:{357.457},{-25.331}) {\tikz\pgfuseplotmark{m6c};};
\node[stars] at (axis cs:{357.561},{-9.974}) {\tikz\pgfuseplotmark{m6b};};
\node[stars] at (axis cs:{357.639},{-14.401}) {\tikz\pgfuseplotmark{m6a};};
\node[stars] at (axis cs:{357.839},{9.313}) {\tikz\pgfuseplotmark{m6av};};
\node[stars] at (axis cs:{357.839},{-18.909}) {\tikz\pgfuseplotmark{m5cv};};
\node[stars] at (axis cs:{357.991},{2.930}) {\tikz\pgfuseplotmark{m6a};};
\node[stars] at (axis cs:{358.097},{21.671}) {\tikz\pgfuseplotmark{m6bv};};
\node[stars] at (axis cs:{358.122},{19.120}) {\tikz\pgfuseplotmark{m5bv};};
\node[stars] at (axis cs:{358.125},{-14.251}) {\tikz\pgfuseplotmark{m6a};};
\node[stars] at (axis cs:{358.155},{10.947}) {\tikz\pgfuseplotmark{m5cv};};
\node[stars] at (axis cs:{358.211},{-8.997}) {\tikz\pgfuseplotmark{m6a};};
\node[stars] at (axis cs:{358.232},{-3.155}) {\tikz\pgfuseplotmark{m6b};};
\node[stars] at (axis cs:{358.270},{2.091}) {\tikz\pgfuseplotmark{m6c};};
\node[stars] at (axis cs:{358.337},{-24.229}) {\tikz\pgfuseplotmark{m6c};};
\node[stars] at (axis cs:{358.694},{0.109}) {\tikz\pgfuseplotmark{m6av};};
\node[stars] at (axis cs:{358.782},{7.071}) {\tikz\pgfuseplotmark{m6c};};
\node[stars] at (axis cs:{358.819},{-31.921}) {\tikz\pgfuseplotmark{m6bv};};
\node[stars] at (axis cs:{359.125},{-24.737}) {\tikz\pgfuseplotmark{m6c};};
\node[stars] at (axis cs:{359.173},{22.648}) {\tikz\pgfuseplotmark{m6c};};
\node[stars] at (axis cs:{359.284},{-26.623}) {\tikz\pgfuseplotmark{m6cv};};
\node[stars] at (axis cs:{359.292},{-20.834}) {\tikz\pgfuseplotmark{m6c};};
\node[stars] at (axis cs:{359.440},{25.141}) {\tikz\pgfuseplotmark{m5av};};
\node[stars] at (axis cs:{359.588},{-15.847}) {\tikz\pgfuseplotmark{m6c};};
\node[stars] at (axis cs:{359.668},{-3.556}) {\tikz\pgfuseplotmark{m5bv};};
\node[stars] at (axis cs:{359.828},{6.863}) {\tikz\pgfuseplotmark{m4bv};};
\node[stars] at (axis cs:{359.866},{-29.485}) {\tikz\pgfuseplotmark{m6av};};
\node[stars] at (axis cs:{359.872},{33.724}) {\tikz\pgfuseplotmark{m6cb};};
\end{axis}

\input{./input/VIS_Names.tex}

%
% Comets, own objects etc.
\begin{axis}[name=transients,axis lines=none]
% Format for label:
%\node[transient,pin={[pin distance=-0.8\onedegree,transient-label]90:{LABEL}}] at (axis cs:{ *15 + /4 + /240 },{  + /60 + /3600 })  {+};
%
% Note for Southern positions, if in sexagesimal, decl. signs need to be inverted: 
%   \(axis cs:{ *15 + /4 + /240 },{   - /60 - /3600 })  {+};
%
% Note "08" is interpreted as Octal 8, and triggers a problem in computation

%%%%%%%%%%%%%%%%%%%%%%%%%%%%%%%%%%%%%%%%%%%%%%%%%%%%%%%%%%%%%%%%%%%%%%%%%%%%%%%%%%%%%%%%
% Comet C/2023 A3 (Tsuchinshan-ATLAS) in 2024, Apr-Dec\
% Coords taken from https://in-the-sky.org/ephemeris.php?objtxt=CK23A030
%%%%%%%%%%%%%%%%%%%%%%%%%%%%%%%%%%%%%%%%%%%%%%%%%%%%%%%%%%%%%%%%%%%%%%%%%%%%%%%%%%%%%%%%
%
% \node[transient,pin={[pin distance=-0.8\onedegree,transient-label]90:{Apr\,01}}] at (axis cs:{14 *15 + 32/4 +  30/240 },{ -04  - 59/60 -  6/3600 })  {+};
% \node[transient,pin={[pin distance=-0.8\onedegree,transient-label]90:{Apr\,08}}] at (axis cs:{14 *15 + 19/4 +  22/240 },{ -04  -  8/60 - 35/3600 })  {+};
% \node[transient,pin={[pin distance=-0.8\onedegree,transient-label]90:{Apr\,15}}] at (axis cs:{14 *15 +  4/4 +  04/240 },{ -03  - 11/60 - 56/3600 })  {+};
% \node[transient,pin={[pin distance=-0.8\onedegree,transient-label]90:{Apr\,22}}] at (axis cs:{13 *15 + 46/4 +  53/240 },{ -02  - 10/60 - 50/3600 })  {+};
% \node[transient,pin={[pin distance=-0.8\onedegree,transient-label]90:{May\,01}}] at (axis cs:{13 *15 + 22/4 +  47/240 },{ -00  - 50/60 -  5/3600 })  {+};
% \node[transient,pin={[pin distance=-0.8\onedegree,transient-label]90:{May\,08}}] at (axis cs:{13 *15 +  3/4 +  21/240 },{ +00  + 10/60 +  8/3600 })  {+};
% \node[transient,pin={[pin distance=-0.8\onedegree,transient-label]90:{May\,15}}] at (axis cs:{12 *15 + 44/4 +   8/240 },{ +01  +  4/60 + 17/3600 })  {+};
% \node[transient,pin={[pin distance=-0.8\onedegree,transient-label]90:{May\,22}}] at (axis cs:{12 *15 + 25/4 +  54/240 },{ +01  + 49/60 + 32/3600 })  {+};
% \node[transient,pin={[pin distance=-0.8\onedegree,transient-label]90:{Jun\,01}}] at (axis cs:{12 *15 +  2/4 +  36/240 },{ +02  + 35/60 + 35/3600 })  {+};
% \node[transient,pin={[pin distance=-0.8\onedegree,transient-label]90:{Jun\,08}}] at (axis cs:{11 *15 + 48/4 +  37/240 },{ +02  + 54/60 + 16/3600 })  {+};
% \node[transient,pin={[pin distance=-0.8\onedegree,transient-label]90:{Jun\,15}}] at (axis cs:{11 *15 + 36/4 +  38/240 },{ +03  +  2/60 + 21/3600 })  {+};
% \node[transient,pin={[pin distance=-0.8\onedegree,transient-label]90:{Jun\,22}}] at (axis cs:{11 *15 + 26/4 +  34/240 },{ +03  +  0/60 + 47/3600 })  {+};
% \node[transient,pin={[pin distance=-0.8\onedegree,transient-label]90:{Jul\,01}}] at (axis cs:{11 *15 + 16/4 +  10/240 },{ +02  + 46/60 + 21/3600 })  {+};
% \node[transient,pin={[pin distance=-0.8\onedegree,transient-label]90:{Jul\,08}}] at (axis cs:{11 *15 +  9/4 +  44/240 },{ +02  + 26/60 + 41/3600 })  {+};
% \node[transient,pin={[pin distance=-0.8\onedegree,transient-label]90:{Jul\,15}}] at (axis cs:{11 *15 +  4/4 +  32/240 },{ +02  +  0/60 + 32/3600 })  {+};
% \node[transient,pin={[pin distance=-0.8\onedegree,transient-label]90:{Jul\,22}}] at (axis cs:{11 *15 +  0/4 +  19/240 },{ +01  + 28/60 + 34/3600 })  {+};
% \node[transient,pin={[pin distance=-0.8\onedegree,transient-label]90:{Aug\,01}}] at (axis cs:{10 *15 + 55/4 +  30/240 },{ +00  + 33/60 + 43/3600 })  {+};
% \node[transient,pin={[pin distance=-0.8\onedegree,transient-label]90:{Aug\,08}}] at (axis cs:{10 *15 + 52/4 +  42/240 },{ -00  - 10/60 - 40/3600 })  {+};
% \node[transient,pin={[pin distance=-0.8\onedegree,transient-label]90:{Aug\,15}}] at (axis cs:{10 *15 + 50/4 +  05/240 },{ -00  - 59/60 - 45/3600 })  {+};
% \node[transient,pin={[pin distance=-0.8\onedegree,transient-label]90:{Aug\,22}}] at (axis cs:{10 *15 + 47/4 +  26/240 },{ -01  - 53/60 - 19/3600 })  {+};
% \node[transient,pin={[pin distance=-0.8\onedegree,transient-label]90:{Sep\,01}}] at (axis cs:{10 *15 + 43/4 +  06/240 },{ -03  - 16/60 - 41/3600 })  {+};
% \node[transient,pin={[pin distance=-0.8\onedegree,transient-label]90:{Sep\,08}}] at (axis cs:{10 *15 + 39/4 +  30/240 },{ -04  - 17/60 - 50/3600 })  {+};
% \node[transient,pin={[pin distance=-0.8\onedegree,transient-label]90:{Sep\,15}}] at (axis cs:{10 *15 + 35/4 +  57/240 },{ -05  - 16/60 -  5/3600 })  {+};
% \node[transient,pin={[pin distance=-0.8\onedegree,transient-label]90:{Sep\,22}}] at (axis cs:{10 *15 + 35/4 +  49/240 },{ -05  - 58/60 -  7/3600 })  {+};
% \node[transient,pin={[pin distance=-0.8\onedegree,transient-label]90:{Oct\,01}}] at (axis cs:{11 *15 +  4/4 +  03/240 },{ -05  - 41/60 - 36/3600 })  {+};
% \node[transient,pin={[pin distance=-0.8\onedegree,transient-label]90:{Oct\,08}}] at (axis cs:{12 *15 + 30/4 +  27/240 },{ -03  - 36/60 - 16/3600 })  {+};
% \node[transient,pin={[pin distance=-0.8\onedegree,transient-label]90:{Oct\,15}}] at (axis cs:{14 *15 + 59/4 +  40/240 },{ +00  + 23/60 + 14/3600 })  {+};
% \node[transient,pin={[pin distance=-0.8\onedegree,transient-label]90:{Oct\,22}}] at (axis cs:{16 *15 + 49/4 +  25/240 },{ +02  + 50/60 + 11/3600 })  {+};
% \node[transient,pin={[pin distance=-0.8\onedegree,transient-label]90:{Nov\,01}}] at (axis cs:{17 *15 + 59/4 +  40/240 },{ +03  + 45/60 + 12/3600 })  {+};
% \node[transient,pin={[pin distance=-0.8\onedegree,transient-label]90:{Nov\,08}}] at (axis cs:{18 *15 + 24/4 +  28/240 },{ +03  + 55/60 + 51/3600 })  {+};
% \node[transient,pin={[pin distance=-0.8\onedegree,transient-label]90:{Nov\,15}}] at (axis cs:{18 *15 + 41/4 +  24/240 },{ +04  +  3/60 + 27/3600 })  {+};
% \node[transient,pin={[pin distance=-0.8\onedegree,transient-label]90:{Nov\,22}}] at (axis cs:{18 *15 + 54/4 +  18/240 },{ +04  + 12/60 + 40/3600 })  {+};
% \node[transient,pin={[pin distance=-0.8\onedegree,transient-label]90:{Dec\,01}}] at (axis cs:{19 *15 +  7/4 +  39/240 },{ +04  + 29/60 + 30/3600 })  {+};
% \node[transient,pin={[pin distance=-0.8\onedegree,transient-label]90:{Dec\,08}}] at (axis cs:{19 *15 + 16/4 +  32/240 },{ +04  + 47/60 + 12/3600 })  {+};
% \node[transient,pin={[pin distance=-0.8\onedegree,transient-label]90:{Dec\,15}}] at (axis cs:{19 *15 + 24/4 +  32/240 },{ +05  +  9/60 +  7/3600 })  {+};
% \node[transient,pin={[pin distance=-0.8\onedegree,transient-label]90:{Dec\,22}}] at (axis cs:{19 *15 + 31/4 +  54/240 },{ +05  + 35/60 + 17/3600 })  {+};
% 
% 
% \draw[transient]
% (axis cs:{14 *15 + 32/4 + 30/240 },{ -04  - 59/60 -  6/3600 }) --
% (axis cs:{14 *15 + 19/4 + 22/240 },{ -04  -  8/60 - 35/3600 }) --
% (axis cs:{14 *15 +  4/4 + 04/240 },{ -03  - 11/60 - 56/3600 }) --
% (axis cs:{13 *15 + 46/4 + 53/240 },{ -02  - 10/60 - 50/3600 }) --
% (axis cs:{13 *15 + 22/4 + 47/240 },{ -00  - 50/60 -  5/3600 }) --
% (axis cs:{13 *15 +  3/4 + 21/240 },{ +00  + 10/60 +  8/3600 }) --
% (axis cs:{12 *15 + 44/4 +  8/240 },{ +01  +  4/60 + 17/3600 }) --
% (axis cs:{12 *15 + 25/4 + 54/240 },{ +01  + 49/60 + 32/3600 }) --
% (axis cs:{12 *15 +  2/4 + 36/240 },{ +02  + 35/60 + 35/3600 }) --
% (axis cs:{11 *15 + 48/4 + 37/240 },{ +02  + 54/60 + 16/3600 }) --
% (axis cs:{11 *15 + 36/4 + 38/240 },{ +03  +  2/60 + 21/3600 }) --
% (axis cs:{11 *15 + 26/4 + 34/240 },{ +03  +  0/60 + 47/3600 }) --
% (axis cs:{11 *15 + 16/4 + 10/240 },{ +02  + 46/60 + 21/3600 }) --
% (axis cs:{11 *15 +  9/4 + 44/240 },{ +02  + 26/60 + 41/3600 }) --
% (axis cs:{11 *15 +  4/4 + 32/240 },{ +02  +  0/60 + 32/3600 }) --
% (axis cs:{11 *15 +  0/4 + 19/240 },{ +01  + 28/60 + 34/3600 }) --
% (axis cs:{10 *15 + 55/4 + 30/240 },{ +00  + 33/60 + 43/3600 }) --
% (axis cs:{10 *15 + 52/4 + 42/240 },{ -00  - 10/60 - 40/3600 }) --
% (axis cs:{10 *15 + 50/4 + 05/240 },{ -00  - 59/60 - 45/3600 }) --
% (axis cs:{10 *15 + 47/4 + 26/240 },{ -01  - 53/60 - 19/3600 }) --
% (axis cs:{10 *15 + 43/4 + 06/240 },{ -03  - 16/60 - 41/3600 }) --
% (axis cs:{10 *15 + 39/4 + 30/240 },{ -04  - 17/60 - 50/3600 }) --
% (axis cs:{10 *15 + 35/4 + 57/240 },{ -05  - 16/60 -  5/3600 }) --
% (axis cs:{10 *15 + 35/4 + 49/240 },{ -05  - 58/60 -  7/3600 }) --
% (axis cs:{11 *15 +  4/4 + 03/240 },{ -05  - 41/60 - 36/3600 }) --
% (axis cs:{12 *15 + 30/4 + 27/240 },{ -03  - 36/60 - 16/3600 }) --
% (axis cs:{14 *15 + 59/4 + 40/240 },{ +00  + 23/60 + 14/3600 }) --
% (axis cs:{16 *15 + 49/4 + 25/240 },{ +02  + 50/60 + 11/3600 }) --
% (axis cs:{17 *15 + 59/4 + 40/240 },{ +03  + 45/60 + 12/3600 }) --
% (axis cs:{18 *15 + 24/4 + 28/240 },{ +03  + 55/60 + 51/3600 }) --
% (axis cs:{18 *15 + 41/4 + 24/240 },{ +04  +  3/60 + 27/3600 }) --
% (axis cs:{18 *15 + 54/4 + 18/240 },{ +04  + 12/60 + 40/3600 }) --
% (axis cs:{19 *15 +  7/4 + 39/240 },{ +04  + 29/60 + 30/3600 }) --
% (axis cs:{19 *15 + 16/4 + 32/240 },{ +04  + 47/60 + 12/3600 }) --
% (axis cs:{19 *15 + 24/4 + 32/240 },{ +05  +  9/60 +  7/3600 }) --
% (axis cs:{19 *15 + 31/4 + 54/240 },{ +05  + 35/60 + 17/3600 }) ;
% 
% 

\end{axis}

%
\end{tikzpicture}


\end{document}

