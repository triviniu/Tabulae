%%%%%%%%%%%%%%%%%%%%%%%%%%%%%%%%%%%%%%%%%%%%%%%%%%%%%%%%%%%%%%%%%%%%%%
%
% rm TabulaI.pdf ; pdflatex -shell-escape  TabI.tex 
%
%%%%%%%%%%%%%%%%%%%%%%%%%%%%%%%%%%%%%%%%%%%%%%%%%%%%%%%%%%%%%%%%%%%%%%
\documentclass[10pt,landscape]{article}
\usepackage[margin=1cm,a3paper]{geometry}


\usepackage[margin=1cm,a3paper]{geometry}

\def\pgfsysdriver{pgfsys-pdftex.def}
\usepackage{pgfplots}
\usepgfplotslibrary{external} 
\usetikzlibrary{pgfplots.external}
\usetikzlibrary{calc}
\usetikzlibrary{shapes}
\usetikzlibrary{patterns}
\usetikzlibrary{plotmarks}
\usepgflibrary{arrows}
\usepgfplotslibrary{polar}
\tikzexternalize

\usepackage{times,latexsym,amssymb}
\usepackage{eulervm} % math font
\pagestyle{empty}

\usepackage{amsmath} 
\usepackage{nicefrac} 
%
% This is used because from 1.11 onwards the use of \pgfdeclareplotmark will
% introduce a shift of the stellar markers.
%
\pgfplotsset{compat=1.10}
%
% Use to increase spacing for constellation labels
\usepackage[letterspace=400]{microtype}
%
\usepackage{pagecolor}
\usepackage{natbib}
%


%\pagecolor{black}
\pagecolor{white}


\begin{document} 


%\input{./input/styles_dark.tex}
\input{./input/styles_po.tex}
\center

\pgfplotsset{xmin=0,xmax=360,ymin=90,ymax=20
,y coord trafo/.code=\pgfmathparse{90-#1}
,y coord inv trafo/.code=\pgfmathparse{90-#1}
}

\tikzsetnextfilename{TabulaI}


\begin{tikzpicture}

\begin{polaraxis}[rotate=90,name=base,axis y line=none,axis x line=none]\end{polaraxis}



% The Milky Way first, as it should not obstruct coordinate systems 
\input{./input/I_MW.tex}

\input{./input/IaII_Grid.tex}

\begin{polaraxis}[rotate=90,name=axis1,at=(base.center),anchor=center
    ,axis lines=none
    ,color=cAxes,ymin=90,ymax=25  
    ,width=13\tendegree,height=13\tendegree]


%\draw (axis cs:90,0) arc (90:270:6\tendegree) ;


  \draw (0\tendegree,-6\tendegree) arc (270:90:6\tendegree)
  -- (2\tendegree,6\tendegree)  -- (2\tendegree,-6\tendegree)
   -- cycle ;

\node[rotate=90]  at (axis cs:270,28.5)  {18\fh{\small 00}\fm};
\node[rotate=80]  at (axis cs:280,28.5)  {18\fh{\small 40}\fm};
\node[rotate=70]  at (axis cs:290,28.5)  {19\fh{\small 20}\fm};
\node[rotate=60]  at (axis cs:300,28.5)  {20\fh{\small 00}\fm};
\node[rotate=50]  at (axis cs:310,28.5)  {20\fh{\small 40}\fm};
\node[rotate=40]  at (axis cs:320,28.5)  {21\fh{\small 20}\fm};
\node[rotate=30]  at (axis cs:330,28.5)  {22\fh{\small 00}\fm};
\node[rotate=20]  at (axis cs:340,28.5)  {22\fh{\small 40}\fm};
\node[rotate=10]  at (axis cs:350,28.5)  {23\fh{\small 20}\fm};
\node[rotate=00]  at (axis cs:0,28.5)    {0\fh{\small 00}\fm};
\node[rotate=-10]   at (axis cs:10,28.5) {0\fh{\small 40}\fm};
\node[rotate=-20]   at (axis cs:20,28.5) {1\fh{\small 20}\fm};
\node[rotate=-30]   at (axis cs:30,28.5) {2\fh{\small 00}\fm};
\node[rotate=-40]   at (axis cs:40,28.5) {2\fh{\small 40}\fm};
\node[rotate=-50]   at (axis cs:50,28.5) {3\fh{\small 20}\fm};
\node[rotate=-60]   at (axis cs:60,28.5) {4\fh{\small 00}\fm};
\node[rotate=-70]   at (axis cs:70,28.5) {4\fh{\small 40}\fm};
\node[rotate=-80]   at (axis cs:80,28.5) {5\fh{\small 20}\fm};
\node[rotate=-90]   at (axis cs:90,28.5) {6\fh{\small 00}\fm};

\node[rotate=270]  at (axis cs:90,29.4) {\footnotesize$90^\circ$};
\node[rotate=280]  at (axis cs:80,29.4) {\footnotesize$80^\circ$};
\node[rotate=290]  at (axis cs:70,29.4) {\footnotesize$70^\circ$};
\node[rotate=300]  at (axis cs:60,29.4) {\footnotesize$60^\circ$};
\node[rotate=310]  at (axis cs:50,29.4) {\footnotesize$50^\circ$};
\node[rotate=320]  at (axis cs:40,29.4) {\footnotesize$40^\circ$};
\node[rotate=330]  at (axis cs:30,29.4) {\footnotesize$30^\circ$};
\node[rotate=340]  at (axis cs:20,29.4) {\footnotesize$20^\circ$};
\node[rotate=350]  at (axis cs:10,29.4) {\footnotesize$10^\circ$};
\node[rotate=360]  at (axis cs:0,29.4) {\footnotesize$00^\circ$};
\node[rotate=10] at (axis cs:350,29.4) {\footnotesize$350^\circ$};
\node[rotate=20] at (axis cs:340,29.4) {\footnotesize$340^\circ$};
\node[rotate=30] at (axis cs:330,29.4) {\footnotesize$330^\circ$};
\node[rotate=40] at (axis cs:320,29.4) {\footnotesize$320^\circ$};
\node[rotate=50] at (axis cs:310,29.4) {\footnotesize$310^\circ$};
\node[rotate=60] at (axis cs:300,29.4) {\footnotesize$300^\circ$};
\node[rotate=70] at (axis cs:290,29.4) {\footnotesize$290^\circ$};
\node[rotate=80] at (axis cs:280,29.4) {\footnotesize$280^\circ$};
\node[rotate=90] at (axis cs:270,29.4) {\footnotesize$270^\circ$};

\node[rotate=-90] at (axis cs:100,28.5) {\footnotesize$100^\circ$};
\node[rotate=-90] at (axis cs:100,27.6) {6\fh{\small 40}\fm};

\node[rotate=90] at (axis cs:260,28.5) {\footnotesize$260^\circ$};
\node[rotate=90] at (axis cs:260,27.6) {17\fh{\small 20}\fm};

\node at (2.06\tendegree,-5.494957\tendegree) {\footnotesize$250^\circ$};
\node at (2.06\tendegree,-3.464103\tendegree) {\footnotesize$240^\circ$};
\node at (2.06\tendegree,-2.383508\tendegree) {\footnotesize$230^\circ$};
\node at (2.06\tendegree,-1.678200\tendegree) {\footnotesize$220^\circ$};
\node at (2.06\tendegree,-1.154701\tendegree) {\footnotesize$210^\circ$};
\node at (2.06\tendegree,-0.7279406\tendegree) {\footnotesize$200^\circ$};
\node at (2.06\tendegree,-0.352\tendegree) {\footnotesize$190^\circ$};
\node at (2.06\tendegree,0.0\tendegree) {\footnotesize$180^\circ$};
\node at (2.06\tendegree,0.352\tendegree) {\footnotesize$170^\circ$};
\node at (2.06\tendegree,0.7279406\tendegree) {\footnotesize$160^\circ$};
\node at (2.06\tendegree,1.154701\tendegree) {\footnotesize$150^\circ$};
\node at (2.06\tendegree,1.678200\tendegree) {\footnotesize$140^\circ$};
\node at (2.06\tendegree,2.383508\tendegree) {\footnotesize$130^\circ$};
\node at (2.06\tendegree,3.464103\tendegree) {\footnotesize$120^\circ$};
\node at (2.06\tendegree,5.494957\tendegree) {\footnotesize$110^\circ$};

\node at (2.18\tendegree,-5.494957\tendegree)  {16\fh{\small 40}\fm};
\node at (2.18\tendegree,-3.464103\tendegree)  {16\fh{\small 00}\fm};
\node at (2.18\tendegree,-2.383508\tendegree)  {15\fh{\small 20}\fm};
\node at (2.18\tendegree,-1.678200\tendegree)  {14\fh{\small 40}\fm};
\node at (2.18\tendegree,-1.154701\tendegree)  {14\fh{\small 00}\fm};
\node at (2.18\tendegree,-0.7279406\tendegree) {13\fh{\small 20}\fm};
\node at (2.18\tendegree,-0.352\tendegree)     {12\fh{\small 40}\fm};
\node at (2.18\tendegree,0.0\tendegree)       {12\fh{\small 00}\fm};
\node at (2.18\tendegree,0.352\tendegree)    {11\fh{\small 20}\fm};
\node at (2.18\tendegree,0.7279406\tendegree){10\fh{\small 40}\fm};
\node at (2.18\tendegree,1.154701\tendegree) {10\fh{\small 00}\fm};
\node at (2.18\tendegree,1.678200\tendegree) {9\fh{\small 20}\fm};
\node at (2.18\tendegree,2.383508\tendegree) {8\fh{\small 40}\fm};
\node at (2.18\tendegree,3.464103\tendegree) {8\fh{\small 00}\fm};
\node at (2.18\tendegree,5.494957\tendegree) {7\fh{\small 20}\fm};


\end{polaraxis}


% Left and bottom axis with gridlines
%  \begin{axis}[at=(base.center),anchor=center,RA_in_hours,color=cAxes] \end{axis}
% Right and top axis, labelling in RA hours (offset)
%  \begin{axis}[at=(base.center),anchor=center,yticklabel pos=right,xticklabel pos=right,RA_in_hours,xticklabel style={yshift=2.5ex, anchor=south},color=cAxes]\end{axis}
% Top axis labelling in RA degree (small offset)
%  \begin{axis}[at=(base.center),anchor=center,axis y line=none,xticklabel pos=top,RA_in_deg     ,xticklabel style={font=\footnotesize},xticklabel style={yshift=0.5ex, anchor=south} ,color=cAxes]\end{axis}

\input{./input/IaII_Frame.tex}
\input{./input/IaII_Legend.tex}

% The Galactic coordinate system
%
% Some are commented out because they are surrounded by already plotted borders
% The x-coordinate is in fractional hours, so must be times 15
%

\begin{polaraxis}[rotate=90,name=MWcoord,at=(base.center),anchor=center,axis lines=none]


  \clip (0\tendegree,-6\tendegree) arc (270:90:6\tendegree)
  -- (2\tendegree,6\tendegree)  -- (2\tendegree,-6\tendegree)
   -- cycle ;
  
\node[MWE-empty,pin={[pin distance=-0.4\onedegree,MWE-label]090:{NGP}}] at (axis cs:12*15+51.42/4,27+07.8/60) {\pgfuseplotmark{+}} ;
\node[MWE-empty,pin={[pin distance=-0.4\onedegree,MWE-label]000:{SGP}}] at (axis cs:12*15+51.42/4-180,-27+07.8/60) {\pgfuseplotmark{+}} ;

\draw[MWE-full] (axis cs:2.66308E+02,-2.88819E+01) -- (axis cs:2.66899E+02,-2.80275E+01) -- (axis cs:2.67093E+02,-2.81309E+01) -- (axis cs:2.66503E+02,-2.89861E+01) -- cycle ; 
\draw[MWE-full] (axis cs:2.67481E+02,-2.71706E+01) -- (axis cs:2.68054E+02,-2.63113E+01) -- (axis cs:2.68246E+02,-2.64131E+01) -- (axis cs:2.67674E+02,-2.72732E+01) -- cycle ; 
\draw[MWE-full] (axis cs:2.68618E+02,-2.54498E+01) -- (axis cs:2.69174E+02,-2.45861E+01) -- (axis cs:2.69365E+02,-2.46865E+01) -- (axis cs:2.68809E+02,-2.55508E+01) -- cycle ; 
\draw[MWE-full] (axis cs:2.69723E+02,-2.37204E+01) -- (axis cs:2.70264E+02,-2.28528E+01) -- (axis cs:2.70453E+02,-2.29518E+01) -- (axis cs:2.69912E+02,-2.38201E+01) -- cycle ; 
\draw[MWE-full] (axis cs:2.70799E+02,-2.19834E+01) -- (axis cs:2.71327E+02,-2.11122E+01) -- (axis cs:2.71514E+02,-2.12100E+01) -- (axis cs:2.70986E+02,-2.20817E+01) -- cycle ; 
\draw[MWE-full] (axis cs:2.71848E+02,-2.02394E+01) -- (axis cs:2.72364E+02,-1.93650E+01) -- (axis cs:2.72550E+02,-1.94617E+01) -- (axis cs:2.72035E+02,-2.03366E+01) -- cycle ; 
\draw[MWE-full] (axis cs:2.72874E+02,-1.84892E+01) -- (axis cs:2.73379E+02,-1.76121E+01) -- (axis cs:2.73564E+02,-1.77078E+01) -- (axis cs:2.73059E+02,-1.85854E+01) -- cycle ; 
\draw[MWE-full] (axis cs:2.73879E+02,-1.67336E+01) -- (axis cs:2.74375E+02,-1.58540E+01) -- (axis cs:2.74558E+02,-1.59488E+01) -- (axis cs:2.74063E+02,-1.68289E+01) -- cycle ; 
\draw[MWE-full] (axis cs:2.74866E+02,-1.49732E+01) -- (axis cs:2.75353E+02,-1.40913E+01) -- (axis cs:2.75535E+02,-1.41854E+01) -- (axis cs:2.75049E+02,-1.50676E+01) -- cycle ; 
\draw[MWE-full] (axis cs:2.75837E+02,-1.32085E+01) -- (axis cs:2.76316E+02,-1.23248E+01) -- (axis cs:2.76498E+02,-1.24181E+01) -- (axis cs:2.76018E+02,-1.33022E+01) -- cycle ; 
\draw[MWE-full] (axis cs:2.76793E+02,-1.14402E+01) -- (axis cs:2.77267E+02,-1.05548E+01) -- (axis cs:2.77447E+02,-1.06476E+01) -- (axis cs:2.76974E+02,-1.15333E+01) -- cycle ; 
\draw[MWE-full] (axis cs:2.77738E+02,-9.66873E+00) -- (axis cs:2.78206E+02,-8.78200E+00) -- (axis cs:2.78385E+02,-8.87430E+00) -- (axis cs:2.77917E+02,-9.76126E+00) -- cycle ; 
\draw[MWE-full] (axis cs:2.78672E+02,-7.89468E+00) -- (axis cs:2.79136E+02,-7.00682E+00) -- (axis cs:2.79315E+02,-7.09872E+00) -- (axis cs:2.78851E+02,-7.98676E+00) -- cycle ; 
\draw[MWE-full] (axis cs:2.79598E+02,-6.11851E+00) -- (axis cs:2.80059E+02,-5.22978E+00) -- (axis cs:2.80237E+02,-5.32137E+00) -- (axis cs:2.79777E+02,-6.21024E+00) -- cycle ; 
\draw[MWE-full] (axis cs:2.80518E+02,-4.34070E+00) -- (axis cs:2.80977E+02,-3.45134E+00) -- (axis cs:2.81155E+02,-3.54271E+00) -- (axis cs:2.80697E+02,-4.43217E+00) -- cycle ; 
\draw[MWE-full] (axis cs:2.81434E+02,-2.56174E+00) -- (axis cs:2.81891E+02,-1.67196E+00) -- (axis cs:2.82069E+02,-1.76320E+00) -- (axis cs:2.81612E+02,-2.65303E+00) -- cycle ; 
\draw[MWE-full] (axis cs:2.82347E+02,-7.82062E-01) -- (axis cs:2.82917E+02,1.07899E-01) -- (axis cs:2.82981E+02,1.66967E-02) -- (axis cs:2.82525E+02,-8.73274E-01) -- cycle ; 
\draw[MWE-full] (axis cs:2.83259E+02,9.97867E-01) -- (axis cs:2.83715E+02,1.88779E+00) -- (axis cs:2.83893E+02,1.79654E+00) -- (axis cs:2.83437E+02,9.06652E-01) -- cycle ; 
\draw[MWE-full] (axis cs:2.84172E+02,2.77760E+00) -- (axis cs:2.84629E+02,3.66725E+00) -- (axis cs:2.84808E+02,3.57586E+00) -- (axis cs:2.84350E+02,2.68629E+00) -- cycle ; 
\draw[MWE-full] (axis cs:2.85088E+02,4.55668E+00) -- (axis cs:2.85547E+02,5.44583E+00) -- (axis cs:2.85725E+02,5.35422E+00) -- (axis cs:2.85266E+02,4.46519E+00) -- cycle ; 
\draw[MWE-full] (axis cs:2.86008E+02,6.33465E+00) -- (axis cs:2.86470E+02,7.22308E+00) -- (axis cs:2.86649E+02,7.13115E+00) -- (axis cs:2.86186E+02,6.24290E+00) -- cycle ; 
\draw[MWE-full] (axis cs:2.86934E+02,8.11105E+00) -- (axis cs:2.87400E+02,8.99851E+00) -- (axis cs:2.87580E+02,8.90618E+00) -- (axis cs:2.87113E+02,8.01893E+00) -- cycle ; 
\draw[MWE-full] (axis cs:2.87868E+02,9.88539E+00) -- (axis cs:2.88339E+02,1.07716E+01) -- (axis cs:2.88519E+02,1.06788E+01) -- (axis cs:2.88048E+02,9.79283E+00) -- cycle ; 
\draw[MWE-full] (axis cs:2.88813E+02,1.16572E+01) -- (axis cs:2.89289E+02,1.25420E+01) -- (axis cs:2.89470E+02,1.24486E+01) -- (axis cs:2.88993E+02,1.15641E+01) -- cycle ; 
\draw[MWE-full] (axis cs:2.89769E+02,1.34259E+01) -- (axis cs:2.90253E+02,1.43090E+01) -- (axis cs:2.90435E+02,1.42149E+01) -- (axis cs:2.89951E+02,1.33322E+01) -- cycle ; 
\draw[MWE-full] (axis cs:2.90740E+02,1.51910E+01) -- (axis cs:2.91231E+02,1.60721E+01) -- (axis cs:2.91414E+02,1.59772E+01) -- (axis cs:2.90922E+02,1.50966E+01) -- cycle ; 
\draw[MWE-full] (axis cs:2.91727E+02,1.69520E+01) -- (axis cs:2.92227E+02,1.78307E+01) -- (axis cs:2.92411E+02,1.77350E+01) -- (axis cs:2.91910E+02,1.68567E+01) -- cycle ; 
\draw[MWE-full] (axis cs:2.92732E+02,1.87082E+01) -- (axis cs:2.93242E+02,1.95843E+01) -- (axis cs:2.93428E+02,1.94875E+01) -- (axis cs:2.92917E+02,1.86119E+01) -- cycle ; 
\draw[MWE-full] (axis cs:2.93758E+02,2.04590E+01) -- (axis cs:2.94280E+02,2.13321E+01) -- (axis cs:2.94467E+02,2.12342E+01) -- (axis cs:2.93944E+02,2.03617E+01) -- cycle ; 
\draw[MWE-full] (axis cs:2.94808E+02,2.22037E+01) -- (axis cs:2.95342E+02,2.30735E+01) -- (axis cs:2.95531E+02,2.29744E+01) -- (axis cs:2.94996E+02,2.21052E+01) -- cycle ; 
\draw[MWE-full] (axis cs:2.95884E+02,2.39415E+01) -- (axis cs:2.96433E+02,2.48076E+01) -- (axis cs:2.96623E+02,2.47072E+01) -- (axis cs:2.96073E+02,2.38417E+01) -- cycle ; 
\draw[MWE-full] (axis cs:2.96989E+02,2.56717E+01) -- (axis cs:2.97554E+02,2.65337E+01) -- (axis cs:2.97746E+02,2.64318E+01) -- (axis cs:2.97180E+02,2.55705E+01) -- cycle ; 
\draw[MWE-full] (axis cs:2.98127E+02,2.73934E+01) -- (axis cs:2.98709E+02,2.82508E+01) -- (axis cs:2.98903E+02,2.81473E+01) -- (axis cs:2.98320E+02,2.72907E+01) -- cycle ; 
\draw[MWE-full] (axis cs:2.99300E+02,2.91056E+01) -- (axis cs:2.99901E+02,2.99579E+01) -- (axis cs:3.00098E+02,2.98527E+01) -- (axis cs:2.99495E+02,2.90013E+01) -- cycle ; 
\draw[MWE-full] (axis cs:3.00513E+02,3.08074E+01) -- (axis cs:3.01135E+02,3.16540E+01) -- (axis cs:3.01334E+02,3.15469E+01) -- (axis cs:3.00710E+02,3.07013E+01) -- cycle ; 
\draw[MWE-full] (axis cs:3.01769E+02,3.24975E+01) -- (axis cs:3.02415E+02,3.33378E+01) -- (axis cs:3.02616E+02,3.32287E+01) -- (axis cs:3.01969E+02,3.23894E+01) -- cycle ; 
\draw[MWE-full] (axis cs:3.03073E+02,3.41747E+01) -- (axis cs:3.03745E+02,3.50081E+01) -- (axis cs:3.03948E+02,3.48968E+01) -- (axis cs:3.03275E+02,3.40646E+01) -- cycle ; 
\draw[MWE-full] (axis cs:3.04430E+02,3.58377E+01) -- (axis cs:3.05130E+02,3.66633E+01) -- (axis cs:3.05335E+02,3.65497E+01) -- (axis cs:3.04634E+02,3.57252E+01) -- cycle ; 
\draw[MWE-full] (axis cs:3.05844E+02,3.74848E+01) -- (axis cs:3.06575E+02,3.83019E+01) -- (axis cs:3.06782E+02,3.81857E+01) -- (axis cs:3.06051E+02,3.73699E+01) -- cycle ; 
\draw[MWE-full] (axis cs:3.07322E+02,3.91143E+01) -- (axis cs:3.08087E+02,3.99220E+01) -- (axis cs:3.08296E+02,3.98031E+01) -- (axis cs:3.07531E+02,3.89969E+01) -- cycle ; 
\draw[MWE-full] (axis cs:3.08869E+02,4.07244E+01) -- (axis cs:3.09671E+02,4.15215E+01) -- (axis cs:3.09883E+02,4.13998E+01) -- (axis cs:3.09080E+02,4.06042E+01) -- cycle ; 
\draw[MWE-full] (axis cs:3.10492E+02,4.23130E+01) -- (axis cs:3.11335E+02,4.30984E+01) -- (axis cs:3.11548E+02,4.29736E+01) -- (axis cs:3.10705E+02,4.21897E+01) -- cycle ; 
\draw[MWE-full] (axis cs:3.12199E+02,4.38775E+01) -- (axis cs:3.13086E+02,4.46499E+01) -- (axis cs:3.13301E+02,4.45218E+01) -- (axis cs:3.12414E+02,4.37511E+01) -- cycle ; 
\draw[MWE-full] (axis cs:3.13996E+02,4.54153E+01) -- (axis cs:3.14932E+02,4.61733E+01) -- (axis cs:3.15149E+02,4.60418E+01) -- (axis cs:3.14213E+02,4.52855E+01) -- cycle ; 
\draw[MWE-full] (axis cs:3.15893E+02,4.69235E+01) -- (axis cs:3.16882E+02,4.76654E+01) -- (axis cs:3.17100E+02,4.75301E+01) -- (axis cs:3.16111E+02,4.67901E+01) -- cycle ; 
\draw[MWE-full] (axis cs:3.17899E+02,4.83986E+01) -- (axis cs:3.18945E+02,4.91226E+01) -- (axis cs:3.19165E+02,4.89834E+01) -- (axis cs:3.18118E+02,4.82614E+01) -- cycle ; 
\draw[MWE-full] (axis cs:3.20023E+02,4.98368E+01) -- (axis cs:3.21132E+02,5.05409E+01) -- (axis cs:3.21351E+02,5.03976E+01) -- (axis cs:3.20242E+02,4.96956E+01) -- cycle ; 
\draw[MWE-full] (axis cs:3.22274E+02,5.12341E+01) -- (axis cs:3.23452E+02,5.19158E+01) -- (axis cs:3.23670E+02,5.17682E+01) -- (axis cs:3.22493E+02,5.10886E+01) -- cycle ; 
\draw[MWE-full] (axis cs:3.24665E+02,5.25855E+01) -- (axis cs:3.25915E+02,5.32424E+01) -- (axis cs:3.26132E+02,5.30902E+01) -- (axis cs:3.24882E+02,5.24356E+01) -- cycle ; 
\draw[MWE-full] (axis cs:3.27204E+02,5.38858E+01) -- (axis cs:3.28533E+02,5.45151E+01) -- (axis cs:3.28746E+02,5.43582E+01) -- (axis cs:3.27419E+02,5.37314E+01) -- cycle ; 
\draw[MWE-full] (axis cs:3.29903E+02,5.51293E+01) -- (axis cs:3.31315E+02,5.57277E+01) -- (axis cs:3.31523E+02,5.55661E+01) -- (axis cs:3.30114E+02,5.49701E+01) -- cycle ; 
\draw[MWE-full] (axis cs:3.32770E+02,5.63095E+01) -- (axis cs:3.34269E+02,5.68737E+01) -- (axis cs:3.34472E+02,5.67072E+01) -- (axis cs:3.32976E+02,5.61454E+01) -- cycle ; 
\draw[MWE-full] (axis cs:3.35814E+02,5.74194E+01) -- (axis cs:3.37405E+02,5.79457E+01) -- (axis cs:3.37598E+02,5.77743E+01) -- (axis cs:3.36012E+02,5.72505E+01) -- cycle ; 
\draw[MWE-full] (axis cs:3.39042E+02,5.84516E+01) -- (axis cs:3.40725E+02,5.89360E+01) -- (axis cs:3.40908E+02,5.87597E+01) -- (axis cs:3.39230E+02,5.82777E+01) -- cycle ; 
\draw[MWE-full] (axis cs:3.42456E+02,5.93980E+01) -- (axis cs:3.44233E+02,5.98365E+01) -- (axis cs:3.44402E+02,5.96555E+01) -- (axis cs:3.42632E+02,5.92193E+01) -- cycle ; 
\draw[MWE-full] (axis cs:3.46056E+02,6.02505E+01) -- (axis cs:3.47924E+02,6.06388E+01) -- (axis cs:3.48077E+02,6.04533E+01) -- (axis cs:3.46217E+02,6.00672E+01) -- cycle ; 
\draw[MWE-full] (axis cs:3.49837E+02,6.10005E+01) -- (axis cs:3.51791E+02,6.13346E+01) -- (axis cs:3.51924E+02,6.11451E+01) -- (axis cs:3.49979E+02,6.08130E+01) -- cycle ; 
\draw[MWE-full] (axis cs:3.53786E+02,6.16401E+01) -- (axis cs:3.55819E+02,6.19160E+01) -- (axis cs:3.55929E+02,6.17229E+01) -- (axis cs:3.53908E+02,6.14487E+01) -- cycle ; 
\draw[MWE-full] (axis cs:3.57886E+02,6.21614E+01) -- (axis cs:3.59986E+02,6.23756E+01) -- (axis cs:7.05566E-02,6.21796E+01) -- (axis cs:3.57984E+02,6.19668E+01) -- cycle ; 
\draw[MWE-full] (axis cs:2.11273E+00,6.25578E+01) -- (axis cs:4.26373E+00,6.27073E+01) -- (axis cs:4.32138E+00,6.25091E+01) -- (axis cs:2.18430E+00,6.23606E+01) -- cycle ; 
\draw[MWE-full] (axis cs:6.43417E+00,6.28237E+01) -- (axis cs:8.61932E+00,6.29064E+01) -- (axis cs:8.64801E+00,6.27068E+01) -- (axis cs:6.47748E+00,6.26247E+01) -- cycle ; 
\draw[MWE-full] (axis cs:1.08143E+01,6.29552E+01) -- (axis cs:1.30140E+01,6.29699E+01) -- (axis cs:1.30130E+01,6.27699E+01) -- (axis cs:1.08282E+01,6.27553E+01) -- cycle ; 
\draw[MWE-full] (axis cs:1.52134E+01,6.29504E+01) -- (axis cs:1.74073E+01,6.28969E+01) -- (axis cs:1.73765E+01,6.26974E+01) -- (axis cs:1.51974E+01,6.27506E+01) -- cycle ; 
\draw[MWE-full] (axis cs:1.95907E+01,6.28094E+01) -- (axis cs:2.17586E+01,6.26884E+01) -- (axis cs:2.16990E+01,6.24903E+01) -- (axis cs:1.95453E+01,6.26105E+01) -- cycle ; 
\draw[MWE-full] (axis cs:2.39066E+01,6.25342E+01) -- (axis cs:2.60301E+01,6.23475E+01) -- (axis cs:2.59432E+01,6.21517E+01) -- (axis cs:2.38331E+01,6.23372E+01) -- cycle ; 
\draw[MWE-full] (axis cs:2.81251E+01,6.21289E+01) -- (axis cs:3.01880E+01,6.18792E+01) -- (axis cs:3.00762E+01,6.16863E+01) -- (axis cs:2.80255E+01,6.19345E+01) -- cycle ; 
\draw[MWE-full] (axis cs:3.22156E+01,6.15991E+01) -- (axis cs:3.42049E+01,6.12895E+01) -- (axis cs:3.40709E+01,6.11003E+01) -- (axis cs:3.20923E+01,6.14079E+01) -- cycle ; 
\draw[MWE-full] (axis cs:3.61537E+01,6.09515E+01) -- (axis cs:3.80599E+01,6.05860E+01) -- (axis cs:3.79064E+01,6.04008E+01) -- (axis cs:3.60096E+01,6.07642E+01) -- cycle ; 
\draw[MWE-full] (axis cs:3.99219E+01,6.01940E+01) -- (axis cs:4.17386E+01,5.97766E+01) -- (axis cs:4.15687E+01,5.95959E+01) -- (axis cs:3.97599E+01,6.00110E+01) -- cycle ; 
\draw[MWE-full] (axis cs:4.35091E+01,5.93347E+01) -- (axis cs:4.52331E+01,5.88695E+01) -- (axis cs:4.50497E+01,5.86936E+01) -- (axis cs:4.33322E+01,5.91564E+01) -- cycle ; 
\draw[MWE-full] (axis cs:4.69102E+01,5.83820E+01) -- (axis cs:4.85407E+01,5.78732E+01) -- (axis cs:4.83464E+01,5.77022E+01) -- (axis cs:4.67210E+01,5.82085E+01) -- cycle ; 
\draw[MWE-full] (axis cs:5.01248E+01,5.73442E+01) -- (axis cs:5.16632E+01,5.67958E+01) -- (axis cs:5.14604E+01,5.66296E+01) -- (axis cs:4.99260E+01,5.71755E+01) -- cycle ; 
\draw[MWE-full] (axis cs:5.31565E+01,5.62291E+01) -- (axis cs:5.46057E+01,5.56449E+01) -- (axis cs:5.43965E+01,5.54837E+01) -- (axis cs:5.29503E+01,5.60654E+01) -- cycle ; 
\draw[MWE-full] (axis cs:5.60117E+01,5.50442E+01) -- (axis cs:5.73757E+01,5.44279E+01) -- (axis cs:5.71620E+01,5.42714E+01) -- (axis cs:5.58001E+01,5.48854E+01) -- cycle ; 
\draw[MWE-full] (axis cs:5.86988E+01,5.37966E+01) -- (axis cs:5.99822E+01,5.31512E+01) -- (axis cs:5.97655E+01,5.29994E+01) -- (axis cs:5.84834E+01,5.36425E+01) -- cycle ; 
\draw[MWE-full] (axis cs:6.12273E+01,5.24925E+01) -- (axis cs:6.24353E+01,5.18211E+01) -- (axis cs:6.22168E+01,5.16738E+01) -- (axis cs:6.10096E+01,5.23429E+01) -- cycle ; 
\draw[MWE-full] (axis cs:6.36076E+01,5.11377E+01) -- (axis cs:6.47454E+01,5.04429E+01) -- (axis cs:6.45262E+01,5.02999E+01) -- (axis cs:6.33886E+01,5.09926E+01) -- cycle ; 
\draw[MWE-full] (axis cs:6.58501E+01,4.97374E+01) -- (axis cs:6.69230E+01,4.90218E+01) -- (axis cs:6.67038E+01,4.88829E+01) -- (axis cs:6.56308E+01,4.95965E+01) -- cycle ; 
\draw[MWE-full] (axis cs:6.79652E+01,4.82965E+01) -- (axis cs:6.89781E+01,4.75620E+01) -- (axis cs:6.87597E+01,4.74270E+01) -- (axis cs:6.77464E+01,4.81596E+01) -- cycle ; 
\draw[MWE-full] (axis cs:6.99629E+01,4.68189E+01) -- (axis cs:7.09207E+01,4.60676E+01) -- (axis cs:7.07035E+01,4.59363E+01) -- (axis cs:6.97450E+01,4.66858E+01) -- cycle ; 
\draw[MWE-full] (axis cs:7.18527E+01,4.53086E+01) -- (axis cs:7.27599E+01,4.45422E+01) -- (axis cs:7.25443E+01,4.44144E+01) -- (axis cs:7.16362E+01,4.51791E+01) -- cycle ; 
\draw[MWE-full] (axis cs:7.36436E+01,4.37688E+01) -- (axis cs:7.45045E+01,4.29888E+01) -- (axis cs:7.42908E+01,4.28642E+01) -- (axis cs:7.34289E+01,4.36426E+01) -- cycle ; 
\draw[MWE-full] (axis cs:7.53439E+01,4.22025E+01) -- (axis cs:7.61626E+01,4.14103E+01) -- (axis cs:7.59510E+01,4.12888E+01) -- (axis cs:7.51312E+01,4.20795E+01) -- cycle ; 
\draw[MWE-full] (axis cs:7.69615E+01,4.06124E+01) -- (axis cs:7.77415E+01,3.98092E+01) -- (axis cs:7.75321E+01,3.96906E+01) -- (axis cs:7.67510E+01,4.04924E+01) -- cycle ; 
\draw[MWE-full] (axis cs:7.85035E+01,3.90009E+01) -- (axis cs:7.92483E+01,3.81877E+01) -- (axis cs:7.90411E+01,3.80718E+01) -- (axis cs:7.82952E+01,3.88836E+01) -- cycle ; 
\draw[MWE-full] (axis cs:7.99766E+01,3.73700E+01) -- (axis cs:8.06892E+01,3.65480E+01) -- (axis cs:8.04843E+01,3.64345E+01) -- (axis cs:7.97705E+01,3.72554E+01) -- cycle ; 
\draw[MWE-full] (axis cs:8.13867E+01,3.57218E+01) -- (axis cs:8.20700E+01,3.48916E+01) -- (axis cs:8.18674E+01,3.47805E+01) -- (axis cs:8.11830E+01,3.56095E+01) -- cycle ; 
\draw[MWE-full] (axis cs:8.27396E+01,3.40578E+01) -- (axis cs:8.33961E+01,3.32204E+01) -- (axis cs:8.31957E+01,3.31114E+01) -- (axis cs:8.25381E+01,3.39478E+01) -- cycle ; 
\draw[MWE-full] (axis cs:8.40402E+01,3.23796E+01) -- (axis cs:8.46724E+01,3.15356E+01) -- (axis cs:8.44742E+01,3.14287E+01) -- (axis cs:8.38409E+01,3.22717E+01) -- cycle ; 
\draw[MWE-full] (axis cs:8.52933E+01,3.06886E+01) -- (axis cs:8.59034E+01,2.98387E+01) -- (axis cs:8.57074E+01,2.97337E+01) -- (axis cs:8.50962E+01,3.05826E+01) -- cycle ; 
\draw[MWE-full] (axis cs:8.65033E+01,2.89861E+01) -- (axis cs:8.70933E+01,2.81309E+01) -- (axis cs:8.68993E+01,2.80275E+01) -- (axis cs:8.63082E+01,2.88819E+01) -- cycle ; 
\draw[MWE-full] (axis cs:8.76740E+01,2.72732E+01) -- (axis cs:8.82459E+01,2.64131E+01) -- (axis cs:8.80537E+01,2.63113E+01) -- (axis cs:8.74810E+01,2.71706E+01) -- cycle ; 
\draw[MWE-full] (axis cs:8.88093E+01,2.55508E+01) -- (axis cs:8.93646E+01,2.46865E+01) -- (axis cs:8.91743E+01,2.45861E+01) -- (axis cs:8.86180E+01,2.54498E+01) -- cycle ; 
\draw[MWE-full] (axis cs:8.99124E+01,2.38201E+01) -- (axis cs:9.04529E+01,2.29518E+01) -- (axis cs:9.02642E+01,2.28528E+01) -- (axis cs:8.97229E+01,2.37204E+01) -- cycle ; 
\draw[MWE-full] (axis cs:9.09865E+01,2.20817E+01) -- (axis cs:9.15135E+01,2.12100E+01) -- (axis cs:9.13265E+01,2.11122E+01) -- (axis cs:9.07986E+01,2.19834E+01) -- cycle ; 
\draw[MWE-full] (axis cs:9.20345E+01,2.03366E+01) -- (axis cs:9.25496E+01,1.94617E+01) -- (axis cs:9.23640E+01,1.93650E+01) -- (axis cs:9.18481E+01,2.02394E+01) -- cycle ; 
\draw[MWE-full] (axis cs:9.30592E+01,1.85854E+01) -- (axis cs:9.35636E+01,1.77078E+01) -- (axis cs:9.33793E+01,1.76121E+01) -- (axis cs:9.28742E+01,1.84892E+01) -- cycle ; 
\draw[MWE-full] (axis cs:9.40631E+01,1.68289E+01) -- (axis cs:9.45580E+01,1.59488E+01) -- (axis cs:9.43750E+01,1.58540E+01) -- (axis cs:9.38795E+01,1.67336E+01) -- cycle ; 
\draw[MWE-full] (axis cs:9.50487E+01,1.50676E+01) -- (axis cs:9.55353E+01,1.41854E+01) -- (axis cs:9.53533E+01,1.40913E+01) -- (axis cs:9.48661E+01,1.49732E+01) -- cycle ; 
\draw[MWE-full] (axis cs:9.60182E+01,1.33022E+01) -- (axis cs:9.64975E+01,1.24181E+01) -- (axis cs:9.63165E+01,1.23248E+01) -- (axis cs:9.58366E+01,1.32085E+01) -- cycle ; 
\draw[MWE-full] (axis cs:9.69737E+01,1.15333E+01) -- (axis cs:9.74469E+01,1.06476E+01) -- (axis cs:9.72666E+01,1.05548E+01) -- (axis cs:9.67931E+01,1.14402E+01) -- cycle ; 
\draw[MWE-full] (axis cs:9.79174E+01,9.76126E+00) -- (axis cs:9.83854E+01,8.87430E+00) -- (axis cs:9.82058E+01,8.78200E+00) -- (axis cs:9.77375E+01,9.66874E+00) -- cycle ; 
\draw[MWE-full] (axis cs:9.88512E+01,7.98676E+00) -- (axis cs:9.93149E+01,7.09872E+00) -- (axis cs:9.91360E+01,7.00683E+00) -- (axis cs:9.86719E+01,7.89468E+00) -- cycle ; 
\draw[MWE-full] (axis cs:9.97769E+01,6.21024E+00) -- (axis cs:1.00237E+02,5.32137E+00) -- (axis cs:1.00059E+02,5.22978E+00) -- (axis cs:9.95982E+01,6.11851E+00) -- cycle ; 
\draw[MWE-full] (axis cs:1.00697E+02,4.43217E+00) -- (axis cs:1.01155E+02,3.54271E+00) -- (axis cs:1.00977E+02,3.45134E+00) -- (axis cs:1.00518E+02,4.34070E+00) -- cycle ; 
\draw[MWE-full] (axis cs:1.01612E+02,2.65303E+00) -- (axis cs:1.02069E+02,1.76320E+00) -- (axis cs:1.01891E+02,1.67196E+00) -- (axis cs:1.01434E+02,2.56174E+00) -- cycle ; 
\draw[MWE-full] (axis cs:1.02525E+02,8.73274E-01) -- (axis cs:1.02981E+02,-1.66966E-02) -- (axis cs:1.02917E+02,-1.07899E-01) -- (axis cs:1.02347E+02,7.82062E-01) -- cycle ; 
\draw[MWE-full] (axis cs:1.03437E+02,-9.06652E-01) -- (axis cs:1.03893E+02,-1.79654E+00) -- (axis cs:1.03715E+02,-1.88779E+00) -- (axis cs:1.03259E+02,-9.97867E-01) -- cycle ; 
\draw[MWE-full] (axis cs:1.04350E+02,-2.68629E+00) -- (axis cs:1.04808E+02,-3.57586E+00) -- (axis cs:1.04629E+02,-3.66725E+00) -- (axis cs:1.04172E+02,-2.77760E+00) -- cycle ; 
\draw[MWE-full] (axis cs:1.05266E+02,-4.46519E+00) -- (axis cs:1.05726E+02,-5.35422E+00) -- (axis cs:1.05547E+02,-5.44583E+00) -- (axis cs:1.05088E+02,-4.55668E+00) -- cycle ; 
\draw[MWE-full] (axis cs:1.06186E+02,-6.24290E+00) -- (axis cs:1.06649E+02,-7.13115E+00) -- (axis cs:1.06470E+02,-7.22308E+00) -- (axis cs:1.06008E+02,-6.33465E+00) -- cycle ; 
\draw[MWE-full] (axis cs:1.07113E+02,-8.01893E+00) -- (axis cs:1.07580E+02,-8.90618E+00) -- (axis cs:1.07400E+02,-8.99851E+00) -- (axis cs:1.06934E+02,-8.11105E+00) -- cycle ; 
\draw[MWE-full] (axis cs:1.08048E+02,-9.79283E+00) -- (axis cs:1.08520E+02,-1.06788E+01) -- (axis cs:1.08339E+02,-1.07716E+01) -- (axis cs:1.07868E+02,-9.88539E+00) -- cycle ; 
\draw[MWE-full] (axis cs:1.08993E+02,-1.15641E+01) -- (axis cs:1.09471E+02,-1.24486E+01) -- (axis cs:1.09289E+02,-1.25420E+01) -- (axis cs:1.08813E+02,-1.16572E+01) -- cycle ; 
\draw[MWE-full] (axis cs:1.09951E+02,-1.33322E+01) -- (axis cs:1.10435E+02,-1.42149E+01) -- (axis cs:1.10253E+02,-1.43090E+01) -- (axis cs:1.09769E+02,-1.34259E+01) -- cycle ; 
\draw[MWE-full] (axis cs:1.10922E+02,-1.50966E+01) -- (axis cs:1.11414E+02,-1.59772E+01) -- (axis cs:1.11231E+02,-1.60721E+01) -- (axis cs:1.10740E+02,-1.51910E+01) -- cycle ; 
\draw[MWE-full] (axis cs:1.11910E+02,-1.68567E+01) -- (axis cs:1.12411E+02,-1.77350E+01) -- (axis cs:1.12227E+02,-1.78307E+01) -- (axis cs:1.11727E+02,-1.69520E+01) -- cycle ; 
\draw[MWE-full] (axis cs:1.12917E+02,-1.86119E+01) -- (axis cs:1.13428E+02,-1.94875E+01) -- (axis cs:1.13242E+02,-1.95843E+01) -- (axis cs:1.12732E+02,-1.87082E+01) -- cycle ; 
\draw[MWE-full] (axis cs:1.13945E+02,-2.03617E+01) -- (axis cs:1.14467E+02,-2.12342E+01) -- (axis cs:1.14280E+02,-2.13321E+01) -- (axis cs:1.13758E+02,-2.04590E+01) -- cycle ; 
\draw[MWE-full] (axis cs:1.14996E+02,-2.21052E+01) -- (axis cs:1.15531E+02,-2.29744E+01) -- (axis cs:1.15342E+02,-2.30735E+01) -- (axis cs:1.14808E+02,-2.22037E+01) -- cycle ; 
\draw[MWE-full] (axis cs:1.16073E+02,-2.38417E+01) -- (axis cs:1.16623E+02,-2.47072E+01) -- (axis cs:1.16433E+02,-2.48076E+01) -- (axis cs:1.15884E+02,-2.39415E+01) -- cycle ; 
\draw[MWE-full] (axis cs:1.17180E+02,-2.55705E+01) -- (axis cs:1.17746E+02,-2.64318E+01) -- (axis cs:1.17554E+02,-2.65337E+01) -- (axis cs:1.16989E+02,-2.56717E+01) -- cycle ; 
\draw[MWE-full] (axis cs:1.18320E+02,-2.72907E+01) -- (axis cs:1.18903E+02,-2.81473E+01) -- (axis cs:1.18709E+02,-2.82508E+01) -- (axis cs:1.18127E+02,-2.73934E+01) -- cycle ; 
\draw[MWE-full] (axis cs:1.19495E+02,-2.90013E+01) -- (axis cs:1.20098E+02,-2.98527E+01) -- (axis cs:1.19901E+02,-2.99579E+01) -- (axis cs:1.19300E+02,-2.91056E+01) -- cycle ; 
\draw[MWE-full] (axis cs:1.20710E+02,-3.07013E+01) -- (axis cs:1.21334E+02,-3.15469E+01) -- (axis cs:1.21135E+02,-3.16540E+01) -- (axis cs:1.20513E+02,-3.08074E+01) -- cycle ; 
\draw[MWE-full] (axis cs:1.21969E+02,-3.23894E+01) -- (axis cs:1.22616E+02,-3.32287E+01) -- (axis cs:1.22415E+02,-3.33378E+01) -- (axis cs:1.21769E+02,-3.24975E+01) -- cycle ; 
\draw[MWE-full] (axis cs:1.23275E+02,-3.40646E+01) -- (axis cs:1.23948E+02,-3.48968E+01) -- (axis cs:1.23745E+02,-3.50081E+01) -- (axis cs:1.23073E+02,-3.41747E+01) -- cycle ; 
\draw[MWE-full] (axis cs:1.24634E+02,-3.57252E+01) -- (axis cs:1.25335E+02,-3.65497E+01) -- (axis cs:1.25130E+02,-3.66633E+01) -- (axis cs:1.24430E+02,-3.58377E+01) -- cycle ; 
\draw[MWE-full] (axis cs:1.26051E+02,-3.73699E+01) -- (axis cs:1.26782E+02,-3.81857E+01) -- (axis cs:1.26575E+02,-3.83019E+01) -- (axis cs:1.25844E+02,-3.74848E+01) -- cycle ; 
\draw[MWE-full] (axis cs:1.27531E+02,-3.89969E+01) -- (axis cs:1.28296E+02,-3.98031E+01) -- (axis cs:1.28087E+02,-3.99220E+01) -- (axis cs:1.27322E+02,-3.91143E+01) -- cycle ; 
\draw[MWE-full] (axis cs:1.29080E+02,-4.06042E+01) -- (axis cs:1.29883E+02,-4.13998E+01) -- (axis cs:1.29671E+02,-4.15215E+01) -- (axis cs:1.28869E+02,-4.07244E+01) -- cycle ; 
\draw[MWE-full] (axis cs:1.30705E+02,-4.21897E+01) -- (axis cs:1.31549E+02,-4.29736E+01) -- (axis cs:1.31335E+02,-4.30984E+01) -- (axis cs:1.30492E+02,-4.23130E+01) -- cycle ; 
\draw[MWE-full] (axis cs:1.32414E+02,-4.37511E+01) -- (axis cs:1.33301E+02,-4.45218E+01) -- (axis cs:1.33086E+02,-4.46499E+01) -- (axis cs:1.32199E+02,-4.38775E+01) -- cycle ; 
\draw[MWE-full] (axis cs:1.34213E+02,-4.52855E+01) -- (axis cs:1.35149E+02,-4.60418E+01) -- (axis cs:1.34932E+02,-4.61733E+01) -- (axis cs:1.33996E+02,-4.54153E+01) -- cycle ; 
\draw[MWE-full] (axis cs:1.36111E+02,-4.67901E+01) -- (axis cs:1.37100E+02,-4.75301E+01) -- (axis cs:1.36882E+02,-4.76654E+01) -- (axis cs:1.35893E+02,-4.69235E+01) -- cycle ; 
\draw[MWE-full] (axis cs:1.38118E+02,-4.82614E+01) -- (axis cs:1.39165E+02,-4.89834E+01) -- (axis cs:1.38945E+02,-4.91226E+01) -- (axis cs:1.37899E+02,-4.83986E+01) -- cycle ; 
\draw[MWE-full] (axis cs:1.40242E+02,-4.96956E+01) -- (axis cs:1.41351E+02,-5.03976E+01) -- (axis cs:1.41132E+02,-5.05409E+01) -- (axis cs:1.40023E+02,-4.98368E+01) -- cycle ; 
\draw[MWE-full] (axis cs:1.42493E+02,-5.10886E+01) -- (axis cs:1.43670E+02,-5.17682E+01) -- (axis cs:1.43452E+02,-5.19158E+01) -- (axis cs:1.42274E+02,-5.12341E+01) -- cycle ; 
\draw[MWE-full] (axis cs:1.44882E+02,-5.24356E+01) -- (axis cs:1.46132E+02,-5.30902E+01) -- (axis cs:1.45915E+02,-5.32424E+01) -- (axis cs:1.44665E+02,-5.25855E+01) -- cycle ; 
\draw[MWE-full] (axis cs:1.47419E+02,-5.37314E+01) -- (axis cs:1.48746E+02,-5.43582E+01) -- (axis cs:1.48533E+02,-5.45151E+01) -- (axis cs:1.47204E+02,-5.38858E+01) -- cycle ; 
\draw[MWE-full] (axis cs:1.50114E+02,-5.49701E+01) -- (axis cs:1.51523E+02,-5.55661E+01) -- (axis cs:1.51315E+02,-5.57277E+01) -- (axis cs:1.49903E+02,-5.51293E+01) -- cycle ; 
\draw[MWE-full] (axis cs:1.52976E+02,-5.61454E+01) -- (axis cs:1.54472E+02,-5.67072E+01) -- (axis cs:1.54269E+02,-5.68737E+01) -- (axis cs:1.52770E+02,-5.63095E+01) -- cycle ; 
\draw[MWE-full] (axis cs:1.56012E+02,-5.72505E+01) -- (axis cs:1.57598E+02,-5.77743E+01) -- (axis cs:1.57405E+02,-5.79457E+01) -- (axis cs:1.55814E+02,-5.74194E+01) -- cycle ; 
\draw[MWE-full] (axis cs:1.59230E+02,-5.82777E+01) -- (axis cs:1.60908E+02,-5.87597E+01) -- (axis cs:1.60725E+02,-5.89360E+01) -- (axis cs:1.59042E+02,-5.84516E+01) -- cycle ; 
\draw[MWE-full] (axis cs:1.62632E+02,-5.92193E+01) -- (axis cs:1.64402E+02,-5.96555E+01) -- (axis cs:1.64233E+02,-5.98365E+01) -- (axis cs:1.62456E+02,-5.93980E+01) -- cycle ; 
\draw[MWE-full] (axis cs:1.66217E+02,-6.00672E+01) -- (axis cs:1.68077E+02,-6.04533E+01) -- (axis cs:1.67924E+02,-6.06388E+01) -- (axis cs:1.66056E+02,-6.02505E+01) -- cycle ; 
\draw[MWE-full] (axis cs:1.69979E+02,-6.08130E+01) -- (axis cs:1.71924E+02,-6.11451E+01) -- (axis cs:1.71791E+02,-6.13346E+01) -- (axis cs:1.69837E+02,-6.10005E+01) -- cycle ; 
\draw[MWE-full] (axis cs:1.73908E+02,-6.14487E+01) -- (axis cs:1.75929E+02,-6.17229E+01) -- (axis cs:1.75819E+02,-6.19160E+01) -- (axis cs:1.73786E+02,-6.16401E+01) -- cycle ; 
\draw[MWE-full] (axis cs:1.77984E+02,-6.19668E+01) -- (axis cs:1.80071E+02,-6.21796E+01) -- (axis cs:1.79986E+02,-6.23756E+01) -- (axis cs:1.77886E+02,-6.21614E+01) -- cycle ; 
\draw[MWE-full] (axis cs:1.82184E+02,-6.23606E+01) -- (axis cs:1.84321E+02,-6.25091E+01) -- (axis cs:1.84264E+02,-6.27073E+01) -- (axis cs:1.82113E+02,-6.25578E+01) -- cycle ; 
\draw[MWE-full] (axis cs:1.86478E+02,-6.26247E+01) -- (axis cs:1.88648E+02,-6.27068E+01) -- (axis cs:1.88619E+02,-6.29064E+01) -- (axis cs:1.86434E+02,-6.28237E+01) -- cycle ; 
\draw[MWE-full] (axis cs:1.90828E+02,-6.27553E+01) -- (axis cs:1.93013E+02,-6.27699E+01) -- (axis cs:1.93014E+02,-6.29699E+01) -- (axis cs:1.90814E+02,-6.29552E+01) -- cycle ; 
\draw[MWE-full] (axis cs:1.95197E+02,-6.27506E+01) -- (axis cs:1.97377E+02,-6.26974E+01) -- (axis cs:1.97407E+02,-6.28969E+01) -- (axis cs:1.95213E+02,-6.29504E+01) -- cycle ; 
\draw[MWE-full] (axis cs:1.99545E+02,-6.26105E+01) -- (axis cs:2.01699E+02,-6.24903E+01) -- (axis cs:2.01759E+02,-6.26884E+01) -- (axis cs:1.99591E+02,-6.28094E+01) -- cycle ; 
\draw[MWE-full] (axis cs:2.03833E+02,-6.23372E+01) -- (axis cs:2.05943E+02,-6.21517E+01) -- (axis cs:2.06030E+02,-6.23475E+01) -- (axis cs:2.03907E+02,-6.25342E+01) -- cycle ; 
\draw[MWE-full] (axis cs:2.08026E+02,-6.19345E+01) -- (axis cs:2.10076E+02,-6.16863E+01) -- (axis cs:2.10188E+02,-6.18792E+01) -- (axis cs:2.08125E+02,-6.21289E+01) -- cycle ; 
\draw[MWE-full] (axis cs:2.12092E+02,-6.14079E+01) -- (axis cs:2.14071E+02,-6.11003E+01) -- (axis cs:2.14205E+02,-6.12895E+01) -- (axis cs:2.12216E+02,-6.15991E+01) -- cycle ; 
\draw[MWE-full] (axis cs:2.16010E+02,-6.07642E+01) -- (axis cs:2.17906E+02,-6.04008E+01) -- (axis cs:2.18060E+02,-6.05860E+01) -- (axis cs:2.16154E+02,-6.09515E+01) -- cycle ; 
\draw[MWE-full] (axis cs:2.19760E+02,-6.00110E+01) -- (axis cs:2.21569E+02,-5.95959E+01) -- (axis cs:2.21739E+02,-5.97766E+01) -- (axis cs:2.19922E+02,-6.01940E+01) -- cycle ; 
\draw[MWE-full] (axis cs:2.23332E+02,-5.91564E+01) -- (axis cs:2.25050E+02,-5.86936E+01) -- (axis cs:2.25233E+02,-5.88695E+01) -- (axis cs:2.23509E+02,-5.93347E+01) -- cycle ; 
\draw[MWE-full] (axis cs:2.26721E+02,-5.82085E+01) -- (axis cs:2.28346E+02,-5.77022E+01) -- (axis cs:2.28541E+02,-5.78732E+01) -- (axis cs:2.26910E+02,-5.83820E+01) -- cycle ; 
\draw[MWE-full] (axis cs:2.29926E+02,-5.71755E+01) -- (axis cs:2.31460E+02,-5.66296E+01) -- (axis cs:2.31663E+02,-5.67958E+01) -- (axis cs:2.30125E+02,-5.73442E+01) -- cycle ; 
\draw[MWE-full] (axis cs:2.32950E+02,-5.60654E+01) -- (axis cs:2.34397E+02,-5.54837E+01) -- (axis cs:2.34606E+02,-5.56449E+01) -- (axis cs:2.33157E+02,-5.62291E+01) -- cycle ; 
\draw[MWE-full] (axis cs:2.35800E+02,-5.48854E+01) -- (axis cs:2.37162E+02,-5.42714E+01) -- (axis cs:2.37376E+02,-5.44279E+01) -- (axis cs:2.36012E+02,-5.50442E+01) -- cycle ; 
\draw[MWE-full] (axis cs:2.38483E+02,-5.36425E+01) -- (axis cs:2.39766E+02,-5.29994E+01) -- (axis cs:2.39982E+02,-5.31512E+01) -- (axis cs:2.38699E+02,-5.37966E+01) -- cycle ; 
\draw[MWE-full] (axis cs:2.41010E+02,-5.23429E+01) -- (axis cs:2.42217E+02,-5.16738E+01) -- (axis cs:2.42435E+02,-5.18211E+01) -- (axis cs:2.41227E+02,-5.24925E+01) -- cycle ; 
\draw[MWE-full] (axis cs:2.43389E+02,-5.09926E+01) -- (axis cs:2.44526E+02,-5.02999E+01) -- (axis cs:2.44745E+02,-5.04429E+01) -- (axis cs:2.43608E+02,-5.11377E+01) -- cycle ; 
\draw[MWE-full] (axis cs:2.45631E+02,-4.95965E+01) -- (axis cs:2.46704E+02,-4.88829E+01) -- (axis cs:2.46923E+02,-4.90218E+01) -- (axis cs:2.45850E+02,-4.97374E+01) -- cycle ; 
\draw[MWE-full] (axis cs:2.47746E+02,-4.81596E+01) -- (axis cs:2.48760E+02,-4.74270E+01) -- (axis cs:2.48978E+02,-4.75620E+01) -- (axis cs:2.47965E+02,-4.82965E+01) -- cycle ; 
\draw[MWE-full] (axis cs:2.49745E+02,-4.66858E+01) -- (axis cs:2.50704E+02,-4.59363E+01) -- (axis cs:2.50921E+02,-4.60676E+01) -- (axis cs:2.49963E+02,-4.68189E+01) -- cycle ; 
\draw[MWE-full] (axis cs:2.51636E+02,-4.51791E+01) -- (axis cs:2.52544E+02,-4.44144E+01) -- (axis cs:2.52760E+02,-4.45422E+01) -- (axis cs:2.51853E+02,-4.53086E+01) -- cycle ; 
\draw[MWE-full] (axis cs:2.53429E+02,-4.36426E+01) -- (axis cs:2.54291E+02,-4.28642E+01) -- (axis cs:2.54505E+02,-4.29888E+01) -- (axis cs:2.53644E+02,-4.37688E+01) -- cycle ; 
\draw[MWE-full] (axis cs:2.55131E+02,-4.20795E+01) -- (axis cs:2.55951E+02,-4.12888E+01) -- (axis cs:2.56163E+02,-4.14103E+01) -- (axis cs:2.55344E+02,-4.22025E+01) -- cycle ; 
\draw[MWE-full] (axis cs:2.56751E+02,-4.04924E+01) -- (axis cs:2.57532E+02,-3.96906E+01) -- (axis cs:2.57742E+02,-3.98092E+01) -- (axis cs:2.56962E+02,-4.06124E+01) -- cycle ; 
\draw[MWE-full] (axis cs:2.58295E+02,-3.88836E+01) -- (axis cs:2.59041E+02,-3.80718E+01) -- (axis cs:2.59248E+02,-3.81877E+01) -- (axis cs:2.58504E+02,-3.90009E+01) -- cycle ; 
\draw[MWE-full] (axis cs:2.59771E+02,-3.72554E+01) -- (axis cs:2.60484E+02,-3.64345E+01) -- (axis cs:2.60689E+02,-3.65480E+01) -- (axis cs:2.59977E+02,-3.73700E+01) -- cycle ; 
\draw[MWE-full] (axis cs:2.61183E+02,-3.56095E+01) -- (axis cs:2.61867E+02,-3.47805E+01) -- (axis cs:2.62070E+02,-3.48916E+01) -- (axis cs:2.61387E+02,-3.57218E+01) -- cycle ; 
\draw[MWE-full] (axis cs:2.62538E+02,-3.39478E+01) -- (axis cs:2.63196E+02,-3.31114E+01) -- (axis cs:2.63396E+02,-3.32204E+01) -- (axis cs:2.62740E+02,-3.40578E+01) -- cycle ; 
\draw[MWE-full] (axis cs:2.63841E+02,-3.22717E+01) -- (axis cs:2.64474E+02,-3.14287E+01) -- (axis cs:2.64672E+02,-3.15356E+01) -- (axis cs:2.64040E+02,-3.23796E+01) -- cycle ; 
\draw[MWE-full] (axis cs:2.65096E+02,-3.05826E+01) -- (axis cs:2.65707E+02,-2.97337E+01) -- (axis cs:2.65903E+02,-2.98387E+01) -- (axis cs:2.65293E+02,-3.06886E+01) -- cycle ; 
\draw[MWE-full] (axis cs:2.66308E+02,-2.88819E+01) -- (axis cs:2.66899E+02,-2.80275E+01) -- (axis cs:2.67093E+02,-2.81309E+01) -- (axis cs:2.66503E+02,-2.89861E+01) -- cycle ; 
\draw[MWE-empty] (axis cs:2.66899E+02,-2.80275E+01) -- (axis cs:2.67481E+02,-2.71706E+01) -- (axis cs:2.67674E+02,-2.72732E+01) -- (axis cs:2.67093E+02,-2.81309E+01) -- cycle ; 
\draw[MWE-empty] (axis cs:2.68054E+02,-2.63113E+01) -- (axis cs:2.68618E+02,-2.54498E+01) -- (axis cs:2.68809E+02,-2.55508E+01) -- (axis cs:2.68246E+02,-2.64131E+01) -- cycle ; 
\draw[MWE-empty] (axis cs:2.69174E+02,-2.45861E+01) -- (axis cs:2.69723E+02,-2.37204E+01) -- (axis cs:2.69912E+02,-2.38201E+01) -- (axis cs:2.69365E+02,-2.46865E+01) -- cycle ; 
\draw[MWE-empty] (axis cs:2.70264E+02,-2.28528E+01) -- (axis cs:2.70799E+02,-2.19834E+01) -- (axis cs:2.70986E+02,-2.20817E+01) -- (axis cs:2.70453E+02,-2.29518E+01) -- cycle ; 
\draw[MWE-empty] (axis cs:2.71327E+02,-2.11122E+01) -- (axis cs:2.71848E+02,-2.02394E+01) -- (axis cs:2.72035E+02,-2.03366E+01) -- (axis cs:2.71514E+02,-2.12100E+01) -- cycle ; 
\draw[MWE-empty] (axis cs:2.72364E+02,-1.93650E+01) -- (axis cs:2.72874E+02,-1.84892E+01) -- (axis cs:2.73059E+02,-1.85854E+01) -- (axis cs:2.72550E+02,-1.94617E+01) -- cycle ; 
\draw[MWE-empty] (axis cs:2.73379E+02,-1.76121E+01) -- (axis cs:2.73879E+02,-1.67336E+01) -- (axis cs:2.74063E+02,-1.68289E+01) -- (axis cs:2.73564E+02,-1.77078E+01) -- cycle ; 
\draw[MWE-empty] (axis cs:2.74375E+02,-1.58540E+01) -- (axis cs:2.74866E+02,-1.49732E+01) -- (axis cs:2.75049E+02,-1.50676E+01) -- (axis cs:2.74558E+02,-1.59488E+01) -- cycle ; 
\draw[MWE-empty] (axis cs:2.75353E+02,-1.40913E+01) -- (axis cs:2.75837E+02,-1.32085E+01) -- (axis cs:2.76018E+02,-1.33022E+01) -- (axis cs:2.75535E+02,-1.41854E+01) -- cycle ; 
\draw[MWE-empty] (axis cs:2.76316E+02,-1.23248E+01) -- (axis cs:2.76793E+02,-1.14402E+01) -- (axis cs:2.76974E+02,-1.15333E+01) -- (axis cs:2.76498E+02,-1.24181E+01) -- cycle ; 
\draw[MWE-empty] (axis cs:2.77267E+02,-1.05548E+01) -- (axis cs:2.77738E+02,-9.66873E+00) -- (axis cs:2.77917E+02,-9.76126E+00) -- (axis cs:2.77447E+02,-1.06476E+01) -- cycle ; 
\draw[MWE-empty] (axis cs:2.78206E+02,-8.78200E+00) -- (axis cs:2.78672E+02,-7.89468E+00) -- (axis cs:2.78851E+02,-7.98676E+00) -- (axis cs:2.78385E+02,-8.87430E+00) -- cycle ; 
\draw[MWE-empty] (axis cs:2.79136E+02,-7.00682E+00) -- (axis cs:2.79598E+02,-6.11851E+00) -- (axis cs:2.79777E+02,-6.21024E+00) -- (axis cs:2.79315E+02,-7.09872E+00) -- cycle ; 
\draw[MWE-empty] (axis cs:2.80059E+02,-5.22978E+00) -- (axis cs:2.80518E+02,-4.34070E+00) -- (axis cs:2.80697E+02,-4.43217E+00) -- (axis cs:2.80237E+02,-5.32137E+00) -- cycle ; 
\draw[MWE-empty] (axis cs:2.80977E+02,-3.45134E+00) -- (axis cs:2.81434E+02,-2.56174E+00) -- (axis cs:2.81612E+02,-2.65303E+00) -- (axis cs:2.81155E+02,-3.54271E+00) -- cycle ; 
\draw[MWE-empty] (axis cs:2.81891E+02,-1.67196E+00) -- (axis cs:2.82347E+02,-7.82062E-01) -- (axis cs:2.82525E+02,-8.73274E-01) -- (axis cs:2.82069E+02,-1.76320E+00) -- cycle ; 
\draw[MWE-empty] (axis cs:2.82917E+02,1.07899E-01) -- (axis cs:2.83259E+02,9.97867E-01) -- (axis cs:2.83437E+02,9.06652E-01) -- (axis cs:2.82981E+02,1.66967E-02) -- cycle ; 
\draw[MWE-empty] (axis cs:2.83715E+02,1.88779E+00) -- (axis cs:2.84172E+02,2.77760E+00) -- (axis cs:2.84350E+02,2.68629E+00) -- (axis cs:2.83893E+02,1.79654E+00) -- cycle ; 
\draw[MWE-empty] (axis cs:2.84629E+02,3.66725E+00) -- (axis cs:2.85088E+02,4.55668E+00) -- (axis cs:2.85266E+02,4.46519E+00) -- (axis cs:2.84808E+02,3.57586E+00) -- cycle ; 
\draw[MWE-empty] (axis cs:2.85547E+02,5.44583E+00) -- (axis cs:2.86008E+02,6.33465E+00) -- (axis cs:2.86186E+02,6.24290E+00) -- (axis cs:2.85725E+02,5.35422E+00) -- cycle ; 
\draw[MWE-empty] (axis cs:2.86470E+02,7.22308E+00) -- (axis cs:2.86934E+02,8.11105E+00) -- (axis cs:2.87113E+02,8.01893E+00) -- (axis cs:2.86649E+02,7.13115E+00) -- cycle ; 
\draw[MWE-empty] (axis cs:2.87400E+02,8.99851E+00) -- (axis cs:2.87868E+02,9.88539E+00) -- (axis cs:2.88048E+02,9.79283E+00) -- (axis cs:2.87580E+02,8.90618E+00) -- cycle ; 
\draw[MWE-empty] (axis cs:2.88339E+02,1.07716E+01) -- (axis cs:2.88813E+02,1.16572E+01) -- (axis cs:2.88993E+02,1.15641E+01) -- (axis cs:2.88519E+02,1.06788E+01) -- cycle ; 
\draw[MWE-empty] (axis cs:2.89289E+02,1.25420E+01) -- (axis cs:2.89769E+02,1.34259E+01) -- (axis cs:2.89951E+02,1.33322E+01) -- (axis cs:2.89470E+02,1.24486E+01) -- cycle ; 
\draw[MWE-empty] (axis cs:2.90253E+02,1.43090E+01) -- (axis cs:2.90740E+02,1.51910E+01) -- (axis cs:2.90922E+02,1.50966E+01) -- (axis cs:2.90435E+02,1.42149E+01) -- cycle ; 
\draw[MWE-empty] (axis cs:2.91231E+02,1.60721E+01) -- (axis cs:2.91727E+02,1.69520E+01) -- (axis cs:2.91910E+02,1.68567E+01) -- (axis cs:2.91414E+02,1.59772E+01) -- cycle ; 
\draw[MWE-empty] (axis cs:2.92227E+02,1.78307E+01) -- (axis cs:2.92732E+02,1.87082E+01) -- (axis cs:2.92917E+02,1.86119E+01) -- (axis cs:2.92411E+02,1.77350E+01) -- cycle ; 
\draw[MWE-empty] (axis cs:2.93242E+02,1.95843E+01) -- (axis cs:2.93758E+02,2.04590E+01) -- (axis cs:2.93944E+02,2.03617E+01) -- (axis cs:2.93428E+02,1.94875E+01) -- cycle ; 
\draw[MWE-empty] (axis cs:2.94280E+02,2.13321E+01) -- (axis cs:2.94808E+02,2.22037E+01) -- (axis cs:2.94996E+02,2.21052E+01) -- (axis cs:2.94467E+02,2.12342E+01) -- cycle ; 
\draw[MWE-empty] (axis cs:2.95342E+02,2.30735E+01) -- (axis cs:2.95884E+02,2.39415E+01) -- (axis cs:2.96073E+02,2.38417E+01) -- (axis cs:2.95531E+02,2.29744E+01) -- cycle ; 
\draw[MWE-empty] (axis cs:2.96433E+02,2.48076E+01) -- (axis cs:2.96989E+02,2.56717E+01) -- (axis cs:2.97180E+02,2.55705E+01) -- (axis cs:2.96623E+02,2.47072E+01) -- cycle ; 
\draw[MWE-empty] (axis cs:2.97554E+02,2.65337E+01) -- (axis cs:2.98127E+02,2.73934E+01) -- (axis cs:2.98320E+02,2.72907E+01) -- (axis cs:2.97746E+02,2.64318E+01) -- cycle ; 
\draw[MWE-empty] (axis cs:2.98709E+02,2.82508E+01) -- (axis cs:2.99300E+02,2.91056E+01) -- (axis cs:2.99495E+02,2.90013E+01) -- (axis cs:2.98903E+02,2.81473E+01) -- cycle ; 
\draw[MWE-empty] (axis cs:2.99901E+02,2.99579E+01) -- (axis cs:3.00513E+02,3.08074E+01) -- (axis cs:3.00710E+02,3.07013E+01) -- (axis cs:3.00098E+02,2.98527E+01) -- cycle ; 
\draw[MWE-empty] (axis cs:3.01135E+02,3.16540E+01) -- (axis cs:3.01769E+02,3.24975E+01) -- (axis cs:3.01969E+02,3.23894E+01) -- (axis cs:3.01334E+02,3.15469E+01) -- cycle ; 
\draw[MWE-empty] (axis cs:3.02415E+02,3.33378E+01) -- (axis cs:3.03073E+02,3.41747E+01) -- (axis cs:3.03275E+02,3.40646E+01) -- (axis cs:3.02616E+02,3.32287E+01) -- cycle ; 
\draw[MWE-empty] (axis cs:3.03745E+02,3.50081E+01) -- (axis cs:3.04430E+02,3.58377E+01) -- (axis cs:3.04634E+02,3.57252E+01) -- (axis cs:3.03948E+02,3.48968E+01) -- cycle ; 
\draw[MWE-empty] (axis cs:3.05130E+02,3.66633E+01) -- (axis cs:3.05844E+02,3.74848E+01) -- (axis cs:3.06051E+02,3.73699E+01) -- (axis cs:3.05335E+02,3.65497E+01) -- cycle ; 
\draw[MWE-empty] (axis cs:3.06575E+02,3.83019E+01) -- (axis cs:3.07322E+02,3.91143E+01) -- (axis cs:3.07531E+02,3.89969E+01) -- (axis cs:3.06782E+02,3.81857E+01) -- cycle ; 
\draw[MWE-empty] (axis cs:3.08087E+02,3.99220E+01) -- (axis cs:3.08869E+02,4.07244E+01) -- (axis cs:3.09080E+02,4.06042E+01) -- (axis cs:3.08296E+02,3.98031E+01) -- cycle ; 
\draw[MWE-empty] (axis cs:3.09671E+02,4.15215E+01) -- (axis cs:3.10492E+02,4.23130E+01) -- (axis cs:3.10705E+02,4.21897E+01) -- (axis cs:3.09883E+02,4.13998E+01) -- cycle ; 
\draw[MWE-empty] (axis cs:3.11335E+02,4.30984E+01) -- (axis cs:3.12199E+02,4.38775E+01) -- (axis cs:3.12414E+02,4.37511E+01) -- (axis cs:3.11548E+02,4.29736E+01) -- cycle ; 
\draw[MWE-empty] (axis cs:3.13086E+02,4.46499E+01) -- (axis cs:3.13996E+02,4.54153E+01) -- (axis cs:3.14213E+02,4.52855E+01) -- (axis cs:3.13301E+02,4.45218E+01) -- cycle ; 
\draw[MWE-empty] (axis cs:3.14932E+02,4.61733E+01) -- (axis cs:3.15893E+02,4.69235E+01) -- (axis cs:3.16111E+02,4.67901E+01) -- (axis cs:3.15149E+02,4.60418E+01) -- cycle ; 
\draw[MWE-empty] (axis cs:3.16882E+02,4.76654E+01) -- (axis cs:3.17899E+02,4.83986E+01) -- (axis cs:3.18118E+02,4.82614E+01) -- (axis cs:3.17100E+02,4.75301E+01) -- cycle ; 
\draw[MWE-empty] (axis cs:3.18945E+02,4.91226E+01) -- (axis cs:3.20023E+02,4.98368E+01) -- (axis cs:3.20242E+02,4.96956E+01) -- (axis cs:3.19165E+02,4.89834E+01) -- cycle ; 
\draw[MWE-empty] (axis cs:3.21132E+02,5.05409E+01) -- (axis cs:3.22274E+02,5.12341E+01) -- (axis cs:3.22493E+02,5.10886E+01) -- (axis cs:3.21351E+02,5.03976E+01) -- cycle ; 
\draw[MWE-empty] (axis cs:3.23452E+02,5.19158E+01) -- (axis cs:3.24665E+02,5.25855E+01) -- (axis cs:3.24882E+02,5.24356E+01) -- (axis cs:3.23670E+02,5.17682E+01) -- cycle ; 
\draw[MWE-empty] (axis cs:3.25915E+02,5.32424E+01) -- (axis cs:3.27204E+02,5.38858E+01) -- (axis cs:3.27419E+02,5.37314E+01) -- (axis cs:3.26132E+02,5.30902E+01) -- cycle ; 
\draw[MWE-empty] (axis cs:3.28533E+02,5.45151E+01) -- (axis cs:3.29903E+02,5.51293E+01) -- (axis cs:3.30114E+02,5.49701E+01) -- (axis cs:3.28746E+02,5.43582E+01) -- cycle ; 
\draw[MWE-empty] (axis cs:3.31315E+02,5.57277E+01) -- (axis cs:3.32770E+02,5.63095E+01) -- (axis cs:3.32976E+02,5.61454E+01) -- (axis cs:3.31523E+02,5.55661E+01) -- cycle ; 
\draw[MWE-empty] (axis cs:3.34269E+02,5.68737E+01) -- (axis cs:3.35814E+02,5.74194E+01) -- (axis cs:3.36012E+02,5.72505E+01) -- (axis cs:3.34472E+02,5.67072E+01) -- cycle ; 
\draw[MWE-empty] (axis cs:3.37405E+02,5.79457E+01) -- (axis cs:3.39042E+02,5.84516E+01) -- (axis cs:3.39230E+02,5.82777E+01) -- (axis cs:3.37598E+02,5.77743E+01) -- cycle ; 
\draw[MWE-empty] (axis cs:3.40725E+02,5.89360E+01) -- (axis cs:3.42456E+02,5.93980E+01) -- (axis cs:3.42632E+02,5.92193E+01) -- (axis cs:3.40908E+02,5.87597E+01) -- cycle ; 
\draw[MWE-empty] (axis cs:3.44233E+02,5.98365E+01) -- (axis cs:3.46056E+02,6.02505E+01) -- (axis cs:3.46217E+02,6.00672E+01) -- (axis cs:3.44402E+02,5.96555E+01) -- cycle ; 
\draw[MWE-empty] (axis cs:3.47924E+02,6.06388E+01) -- (axis cs:3.49837E+02,6.10005E+01) -- (axis cs:3.49979E+02,6.08130E+01) -- (axis cs:3.48077E+02,6.04533E+01) -- cycle ; 
\draw[MWE-empty] (axis cs:3.51791E+02,6.13346E+01) -- (axis cs:3.53786E+02,6.16401E+01) -- (axis cs:3.53908E+02,6.14487E+01) -- (axis cs:3.51924E+02,6.11451E+01) -- cycle ; 
\draw[MWE-empty] (axis cs:3.55819E+02,6.19160E+01) -- (axis cs:3.57886E+02,6.21614E+01) -- (axis cs:3.57984E+02,6.19668E+01) -- (axis cs:3.55929E+02,6.17229E+01) -- cycle ; 
\draw[MWE-empty] (axis cs:3.59986E+02,6.23756E+01) -- (axis cs:2.11273E+00,6.25578E+01) -- (axis cs:2.18430E+00,6.23606E+01) -- (axis cs:7.05566E-02,6.21796E+01) -- cycle ; 
\draw[MWE-empty] (axis cs:4.26373E+00,6.27073E+01) -- (axis cs:6.43417E+00,6.28237E+01) -- (axis cs:6.47748E+00,6.26247E+01) -- (axis cs:4.32138E+00,6.25091E+01) -- cycle ; 
\draw[MWE-empty] (axis cs:8.61932E+00,6.29064E+01) -- (axis cs:1.08143E+01,6.29552E+01) -- (axis cs:1.08282E+01,6.27553E+01) -- (axis cs:8.64801E+00,6.27068E+01) -- cycle ; 
\draw[MWE-empty] (axis cs:1.30140E+01,6.29699E+01) -- (axis cs:1.52134E+01,6.29504E+01) -- (axis cs:1.51974E+01,6.27506E+01) -- (axis cs:1.30130E+01,6.27699E+01) -- cycle ; 
\draw[MWE-empty] (axis cs:1.74073E+01,6.28969E+01) -- (axis cs:1.95907E+01,6.28094E+01) -- (axis cs:1.95453E+01,6.26105E+01) -- (axis cs:1.73765E+01,6.26974E+01) -- cycle ; 
\draw[MWE-empty] (axis cs:2.17586E+01,6.26884E+01) -- (axis cs:2.39066E+01,6.25342E+01) -- (axis cs:2.38331E+01,6.23372E+01) -- (axis cs:2.16990E+01,6.24903E+01) -- cycle ; 
\draw[MWE-empty] (axis cs:2.60301E+01,6.23475E+01) -- (axis cs:2.81251E+01,6.21289E+01) -- (axis cs:2.80255E+01,6.19345E+01) -- (axis cs:2.59432E+01,6.21517E+01) -- cycle ; 
\draw[MWE-empty] (axis cs:3.01880E+01,6.18792E+01) -- (axis cs:3.22156E+01,6.15991E+01) -- (axis cs:3.20923E+01,6.14079E+01) -- (axis cs:3.00762E+01,6.16863E+01) -- cycle ; 
\draw[MWE-empty] (axis cs:3.42049E+01,6.12895E+01) -- (axis cs:3.61537E+01,6.09515E+01) -- (axis cs:3.60096E+01,6.07642E+01) -- (axis cs:3.40709E+01,6.11003E+01) -- cycle ; 
\draw[MWE-empty] (axis cs:3.80599E+01,6.05860E+01) -- (axis cs:3.99219E+01,6.01940E+01) -- (axis cs:3.97599E+01,6.00110E+01) -- (axis cs:3.79064E+01,6.04008E+01) -- cycle ; 
\draw[MWE-empty] (axis cs:4.17386E+01,5.97766E+01) -- (axis cs:4.35091E+01,5.93347E+01) -- (axis cs:4.33322E+01,5.91564E+01) -- (axis cs:4.15687E+01,5.95959E+01) -- cycle ; 
\draw[MWE-empty] (axis cs:4.52331E+01,5.88695E+01) -- (axis cs:4.69102E+01,5.83820E+01) -- (axis cs:4.67210E+01,5.82085E+01) -- (axis cs:4.50497E+01,5.86936E+01) -- cycle ; 
\draw[MWE-empty] (axis cs:4.85407E+01,5.78732E+01) -- (axis cs:5.01248E+01,5.73442E+01) -- (axis cs:4.99260E+01,5.71755E+01) -- (axis cs:4.83464E+01,5.77022E+01) -- cycle ; 
\draw[MWE-empty] (axis cs:5.16632E+01,5.67958E+01) -- (axis cs:5.31565E+01,5.62291E+01) -- (axis cs:5.29503E+01,5.60654E+01) -- (axis cs:5.14604E+01,5.66296E+01) -- cycle ; 
\draw[MWE-empty] (axis cs:5.46057E+01,5.56449E+01) -- (axis cs:5.60117E+01,5.50442E+01) -- (axis cs:5.58001E+01,5.48854E+01) -- (axis cs:5.43965E+01,5.54837E+01) -- cycle ; 
\draw[MWE-empty] (axis cs:5.73757E+01,5.44279E+01) -- (axis cs:5.86988E+01,5.37966E+01) -- (axis cs:5.84834E+01,5.36425E+01) -- (axis cs:5.71620E+01,5.42714E+01) -- cycle ; 
\draw[MWE-empty] (axis cs:5.99822E+01,5.31512E+01) -- (axis cs:6.12273E+01,5.24925E+01) -- (axis cs:6.10096E+01,5.23429E+01) -- (axis cs:5.97655E+01,5.29994E+01) -- cycle ; 
\draw[MWE-empty] (axis cs:6.24353E+01,5.18211E+01) -- (axis cs:6.36076E+01,5.11377E+01) -- (axis cs:6.33886E+01,5.09926E+01) -- (axis cs:6.22168E+01,5.16738E+01) -- cycle ; 
\draw[MWE-empty] (axis cs:6.47454E+01,5.04429E+01) -- (axis cs:6.58501E+01,4.97374E+01) -- (axis cs:6.56308E+01,4.95965E+01) -- (axis cs:6.45262E+01,5.02999E+01) -- cycle ; 
\draw[MWE-empty] (axis cs:6.69230E+01,4.90218E+01) -- (axis cs:6.79652E+01,4.82965E+01) -- (axis cs:6.77464E+01,4.81596E+01) -- (axis cs:6.67038E+01,4.88829E+01) -- cycle ; 
\draw[MWE-empty] (axis cs:6.89781E+01,4.75620E+01) -- (axis cs:6.99629E+01,4.68189E+01) -- (axis cs:6.97450E+01,4.66858E+01) -- (axis cs:6.87597E+01,4.74270E+01) -- cycle ; 
\draw[MWE-empty] (axis cs:7.09207E+01,4.60676E+01) -- (axis cs:7.18527E+01,4.53086E+01) -- (axis cs:7.16362E+01,4.51791E+01) -- (axis cs:7.07035E+01,4.59363E+01) -- cycle ; 
\draw[MWE-empty] (axis cs:7.27599E+01,4.45422E+01) -- (axis cs:7.36436E+01,4.37688E+01) -- (axis cs:7.34289E+01,4.36426E+01) -- (axis cs:7.25443E+01,4.44144E+01) -- cycle ; 
\draw[MWE-empty] (axis cs:7.45045E+01,4.29888E+01) -- (axis cs:7.53439E+01,4.22025E+01) -- (axis cs:7.51312E+01,4.20795E+01) -- (axis cs:7.42908E+01,4.28642E+01) -- cycle ; 
\draw[MWE-empty] (axis cs:7.61626E+01,4.14103E+01) -- (axis cs:7.69615E+01,4.06124E+01) -- (axis cs:7.67510E+01,4.04924E+01) -- (axis cs:7.59510E+01,4.12888E+01) -- cycle ; 
\draw[MWE-empty] (axis cs:7.77415E+01,3.98092E+01) -- (axis cs:7.85035E+01,3.90009E+01) -- (axis cs:7.82952E+01,3.88836E+01) -- (axis cs:7.75321E+01,3.96906E+01) -- cycle ; 
\draw[MWE-empty] (axis cs:7.92483E+01,3.81877E+01) -- (axis cs:7.99766E+01,3.73700E+01) -- (axis cs:7.97705E+01,3.72554E+01) -- (axis cs:7.90411E+01,3.80718E+01) -- cycle ; 
\draw[MWE-empty] (axis cs:8.06892E+01,3.65480E+01) -- (axis cs:8.13867E+01,3.57218E+01) -- (axis cs:8.11830E+01,3.56095E+01) -- (axis cs:8.04843E+01,3.64345E+01) -- cycle ; 
\draw[MWE-empty] (axis cs:8.20700E+01,3.48916E+01) -- (axis cs:8.27396E+01,3.40578E+01) -- (axis cs:8.25381E+01,3.39478E+01) -- (axis cs:8.18674E+01,3.47805E+01) -- cycle ; 
\draw[MWE-empty] (axis cs:8.33961E+01,3.32204E+01) -- (axis cs:8.40402E+01,3.23796E+01) -- (axis cs:8.38409E+01,3.22717E+01) -- (axis cs:8.31957E+01,3.31114E+01) -- cycle ; 
\draw[MWE-empty] (axis cs:8.46724E+01,3.15356E+01) -- (axis cs:8.52933E+01,3.06886E+01) -- (axis cs:8.50962E+01,3.05826E+01) -- (axis cs:8.44742E+01,3.14287E+01) -- cycle ; 
\draw[MWE-empty] (axis cs:8.59034E+01,2.98387E+01) -- (axis cs:8.65033E+01,2.89861E+01) -- (axis cs:8.63082E+01,2.88819E+01) -- (axis cs:8.57074E+01,2.97337E+01) -- cycle ; 
\draw[MWE-empty] (axis cs:8.70933E+01,2.81309E+01) -- (axis cs:8.76740E+01,2.72732E+01) -- (axis cs:8.74810E+01,2.71706E+01) -- (axis cs:8.68993E+01,2.80275E+01) -- cycle ; 
\draw[MWE-empty] (axis cs:8.82459E+01,2.64131E+01) -- (axis cs:8.88093E+01,2.55508E+01) -- (axis cs:8.86180E+01,2.54498E+01) -- (axis cs:8.80537E+01,2.63113E+01) -- cycle ; 
\draw[MWE-empty] (axis cs:8.93646E+01,2.46865E+01) -- (axis cs:8.99124E+01,2.38201E+01) -- (axis cs:8.97229E+01,2.37204E+01) -- (axis cs:8.91743E+01,2.45861E+01) -- cycle ; 
\draw[MWE-empty] (axis cs:9.04529E+01,2.29518E+01) -- (axis cs:9.09865E+01,2.20817E+01) -- (axis cs:9.07986E+01,2.19834E+01) -- (axis cs:9.02642E+01,2.28528E+01) -- cycle ; 
\draw[MWE-empty] (axis cs:9.15135E+01,2.12100E+01) -- (axis cs:9.20345E+01,2.03366E+01) -- (axis cs:9.18481E+01,2.02394E+01) -- (axis cs:9.13265E+01,2.11122E+01) -- cycle ; 
\draw[MWE-empty] (axis cs:9.25496E+01,1.94617E+01) -- (axis cs:9.30592E+01,1.85854E+01) -- (axis cs:9.28742E+01,1.84892E+01) -- (axis cs:9.23640E+01,1.93650E+01) -- cycle ; 
\draw[MWE-empty] (axis cs:9.35636E+01,1.77078E+01) -- (axis cs:9.40631E+01,1.68289E+01) -- (axis cs:9.38795E+01,1.67336E+01) -- (axis cs:9.33793E+01,1.76121E+01) -- cycle ; 
\draw[MWE-empty] (axis cs:9.45580E+01,1.59488E+01) -- (axis cs:9.50487E+01,1.50676E+01) -- (axis cs:9.48661E+01,1.49732E+01) -- (axis cs:9.43750E+01,1.58540E+01) -- cycle ; 
\draw[MWE-empty] (axis cs:9.55353E+01,1.41854E+01) -- (axis cs:9.60182E+01,1.33022E+01) -- (axis cs:9.58366E+01,1.32085E+01) -- (axis cs:9.53533E+01,1.40913E+01) -- cycle ; 
\draw[MWE-empty] (axis cs:9.64975E+01,1.24181E+01) -- (axis cs:9.69737E+01,1.15333E+01) -- (axis cs:9.67931E+01,1.14402E+01) -- (axis cs:9.63165E+01,1.23248E+01) -- cycle ; 
\draw[MWE-empty] (axis cs:9.74469E+01,1.06476E+01) -- (axis cs:9.79174E+01,9.76126E+00) -- (axis cs:9.77375E+01,9.66874E+00) -- (axis cs:9.72666E+01,1.05548E+01) -- cycle ; 
\draw[MWE-empty] (axis cs:9.83854E+01,8.87430E+00) -- (axis cs:9.88512E+01,7.98676E+00) -- (axis cs:9.86719E+01,7.89468E+00) -- (axis cs:9.82058E+01,8.78200E+00) -- cycle ; 
\draw[MWE-empty] (axis cs:9.93149E+01,7.09872E+00) -- (axis cs:9.97769E+01,6.21024E+00) -- (axis cs:9.95982E+01,6.11851E+00) -- (axis cs:9.91360E+01,7.00683E+00) -- cycle ; 
\draw[MWE-empty] (axis cs:1.00237E+02,5.32137E+00) -- (axis cs:1.00697E+02,4.43217E+00) -- (axis cs:1.00518E+02,4.34070E+00) -- (axis cs:1.00059E+02,5.22978E+00) -- cycle ; 
\draw[MWE-empty] (axis cs:1.01155E+02,3.54271E+00) -- (axis cs:1.01612E+02,2.65303E+00) -- (axis cs:1.01434E+02,2.56174E+00) -- (axis cs:1.00977E+02,3.45134E+00) -- cycle ; 
\draw[MWE-empty] (axis cs:1.02069E+02,1.76320E+00) -- (axis cs:1.02525E+02,8.73274E-01) -- (axis cs:1.02347E+02,7.82062E-01) -- (axis cs:1.01891E+02,1.67196E+00) -- cycle ; 
\draw[MWE-empty] (axis cs:1.02981E+02,-1.66966E-02) -- (axis cs:1.03437E+02,-9.06652E-01) -- (axis cs:1.03259E+02,-9.97867E-01) -- (axis cs:1.02917E+02,-1.07899E-01) -- cycle ; 
\draw[MWE-empty] (axis cs:1.03893E+02,-1.79654E+00) -- (axis cs:1.04350E+02,-2.68629E+00) -- (axis cs:1.04172E+02,-2.77760E+00) -- (axis cs:1.03715E+02,-1.88779E+00) -- cycle ; 
\draw[MWE-empty] (axis cs:1.04808E+02,-3.57586E+00) -- (axis cs:1.05266E+02,-4.46519E+00) -- (axis cs:1.05088E+02,-4.55668E+00) -- (axis cs:1.04629E+02,-3.66725E+00) -- cycle ; 
\draw[MWE-empty] (axis cs:1.05726E+02,-5.35422E+00) -- (axis cs:1.06186E+02,-6.24290E+00) -- (axis cs:1.06008E+02,-6.33465E+00) -- (axis cs:1.05547E+02,-5.44583E+00) -- cycle ; 
\draw[MWE-empty] (axis cs:1.06649E+02,-7.13115E+00) -- (axis cs:1.07113E+02,-8.01893E+00) -- (axis cs:1.06934E+02,-8.11105E+00) -- (axis cs:1.06470E+02,-7.22308E+00) -- cycle ; 
\draw[MWE-empty] (axis cs:1.07580E+02,-8.90618E+00) -- (axis cs:1.08048E+02,-9.79283E+00) -- (axis cs:1.07868E+02,-9.88539E+00) -- (axis cs:1.07400E+02,-8.99851E+00) -- cycle ; 
\draw[MWE-empty] (axis cs:1.08520E+02,-1.06788E+01) -- (axis cs:1.08993E+02,-1.15641E+01) -- (axis cs:1.08813E+02,-1.16572E+01) -- (axis cs:1.08339E+02,-1.07716E+01) -- cycle ; 
\draw[MWE-empty] (axis cs:1.09471E+02,-1.24486E+01) -- (axis cs:1.09951E+02,-1.33322E+01) -- (axis cs:1.09769E+02,-1.34259E+01) -- (axis cs:1.09289E+02,-1.25420E+01) -- cycle ; 
\draw[MWE-empty] (axis cs:1.10435E+02,-1.42149E+01) -- (axis cs:1.10922E+02,-1.50966E+01) -- (axis cs:1.10740E+02,-1.51910E+01) -- (axis cs:1.10253E+02,-1.43090E+01) -- cycle ; 
\draw[MWE-empty] (axis cs:1.11414E+02,-1.59772E+01) -- (axis cs:1.11910E+02,-1.68567E+01) -- (axis cs:1.11727E+02,-1.69520E+01) -- (axis cs:1.11231E+02,-1.60721E+01) -- cycle ; 
\draw[MWE-empty] (axis cs:1.12411E+02,-1.77350E+01) -- (axis cs:1.12917E+02,-1.86119E+01) -- (axis cs:1.12732E+02,-1.87082E+01) -- (axis cs:1.12227E+02,-1.78307E+01) -- cycle ; 
\draw[MWE-empty] (axis cs:1.13428E+02,-1.94875E+01) -- (axis cs:1.13945E+02,-2.03617E+01) -- (axis cs:1.13758E+02,-2.04590E+01) -- (axis cs:1.13242E+02,-1.95843E+01) -- cycle ; 
\draw[MWE-empty] (axis cs:1.14467E+02,-2.12342E+01) -- (axis cs:1.14996E+02,-2.21052E+01) -- (axis cs:1.14808E+02,-2.22037E+01) -- (axis cs:1.14280E+02,-2.13321E+01) -- cycle ; 
\draw[MWE-empty] (axis cs:1.15531E+02,-2.29744E+01) -- (axis cs:1.16073E+02,-2.38417E+01) -- (axis cs:1.15884E+02,-2.39415E+01) -- (axis cs:1.15342E+02,-2.30735E+01) -- cycle ; 
\draw[MWE-empty] (axis cs:1.16623E+02,-2.47072E+01) -- (axis cs:1.17180E+02,-2.55705E+01) -- (axis cs:1.16989E+02,-2.56717E+01) -- (axis cs:1.16433E+02,-2.48076E+01) -- cycle ; 
\draw[MWE-empty] (axis cs:1.17746E+02,-2.64318E+01) -- (axis cs:1.18320E+02,-2.72907E+01) -- (axis cs:1.18127E+02,-2.73934E+01) -- (axis cs:1.17554E+02,-2.65337E+01) -- cycle ; 
\draw[MWE-empty] (axis cs:1.18903E+02,-2.81473E+01) -- (axis cs:1.19495E+02,-2.90013E+01) -- (axis cs:1.19300E+02,-2.91056E+01) -- (axis cs:1.18709E+02,-2.82508E+01) -- cycle ; 
\draw[MWE-empty] (axis cs:1.20098E+02,-2.98527E+01) -- (axis cs:1.20710E+02,-3.07013E+01) -- (axis cs:1.20513E+02,-3.08074E+01) -- (axis cs:1.19901E+02,-2.99579E+01) -- cycle ; 
\draw[MWE-empty] (axis cs:1.21334E+02,-3.15469E+01) -- (axis cs:1.21969E+02,-3.23894E+01) -- (axis cs:1.21769E+02,-3.24975E+01) -- (axis cs:1.21135E+02,-3.16540E+01) -- cycle ; 
\draw[MWE-empty] (axis cs:1.22616E+02,-3.32287E+01) -- (axis cs:1.23275E+02,-3.40646E+01) -- (axis cs:1.23073E+02,-3.41747E+01) -- (axis cs:1.22415E+02,-3.33378E+01) -- cycle ; 
\draw[MWE-empty] (axis cs:1.23948E+02,-3.48968E+01) -- (axis cs:1.24634E+02,-3.57252E+01) -- (axis cs:1.24430E+02,-3.58377E+01) -- (axis cs:1.23745E+02,-3.50081E+01) -- cycle ; 
\draw[MWE-empty] (axis cs:1.25335E+02,-3.65497E+01) -- (axis cs:1.26051E+02,-3.73699E+01) -- (axis cs:1.25844E+02,-3.74848E+01) -- (axis cs:1.25130E+02,-3.66633E+01) -- cycle ; 
\draw[MWE-empty] (axis cs:1.26782E+02,-3.81857E+01) -- (axis cs:1.27531E+02,-3.89969E+01) -- (axis cs:1.27322E+02,-3.91143E+01) -- (axis cs:1.26575E+02,-3.83019E+01) -- cycle ; 
\draw[MWE-empty] (axis cs:1.28296E+02,-3.98031E+01) -- (axis cs:1.29080E+02,-4.06042E+01) -- (axis cs:1.28869E+02,-4.07244E+01) -- (axis cs:1.28087E+02,-3.99220E+01) -- cycle ; 
\draw[MWE-empty] (axis cs:1.29883E+02,-4.13998E+01) -- (axis cs:1.30705E+02,-4.21897E+01) -- (axis cs:1.30492E+02,-4.23130E+01) -- (axis cs:1.29671E+02,-4.15215E+01) -- cycle ; 
\draw[MWE-empty] (axis cs:1.31549E+02,-4.29736E+01) -- (axis cs:1.32414E+02,-4.37511E+01) -- (axis cs:1.32199E+02,-4.38775E+01) -- (axis cs:1.31335E+02,-4.30984E+01) -- cycle ; 
\draw[MWE-empty] (axis cs:1.33301E+02,-4.45218E+01) -- (axis cs:1.34213E+02,-4.52855E+01) -- (axis cs:1.33996E+02,-4.54153E+01) -- (axis cs:1.33086E+02,-4.46499E+01) -- cycle ; 
\draw[MWE-empty] (axis cs:1.35149E+02,-4.60418E+01) -- (axis cs:1.36111E+02,-4.67901E+01) -- (axis cs:1.35893E+02,-4.69235E+01) -- (axis cs:1.34932E+02,-4.61733E+01) -- cycle ; 
\draw[MWE-empty] (axis cs:1.37100E+02,-4.75301E+01) -- (axis cs:1.38118E+02,-4.82614E+01) -- (axis cs:1.37899E+02,-4.83986E+01) -- (axis cs:1.36882E+02,-4.76654E+01) -- cycle ; 
\draw[MWE-empty] (axis cs:1.39165E+02,-4.89834E+01) -- (axis cs:1.40242E+02,-4.96956E+01) -- (axis cs:1.40023E+02,-4.98368E+01) -- (axis cs:1.38945E+02,-4.91226E+01) -- cycle ; 
\draw[MWE-empty] (axis cs:1.41351E+02,-5.03976E+01) -- (axis cs:1.42493E+02,-5.10886E+01) -- (axis cs:1.42274E+02,-5.12341E+01) -- (axis cs:1.41132E+02,-5.05409E+01) -- cycle ; 
\draw[MWE-empty] (axis cs:1.43670E+02,-5.17682E+01) -- (axis cs:1.44882E+02,-5.24356E+01) -- (axis cs:1.44665E+02,-5.25855E+01) -- (axis cs:1.43452E+02,-5.19158E+01) -- cycle ; 
\draw[MWE-empty] (axis cs:1.46132E+02,-5.30902E+01) -- (axis cs:1.47419E+02,-5.37314E+01) -- (axis cs:1.47204E+02,-5.38858E+01) -- (axis cs:1.45915E+02,-5.32424E+01) -- cycle ; 
\draw[MWE-empty] (axis cs:1.48746E+02,-5.43582E+01) -- (axis cs:1.50114E+02,-5.49701E+01) -- (axis cs:1.49903E+02,-5.51293E+01) -- (axis cs:1.48533E+02,-5.45151E+01) -- cycle ; 
\draw[MWE-empty] (axis cs:1.51523E+02,-5.55661E+01) -- (axis cs:1.52976E+02,-5.61454E+01) -- (axis cs:1.52770E+02,-5.63095E+01) -- (axis cs:1.51315E+02,-5.57277E+01) -- cycle ; 
\draw[MWE-empty] (axis cs:1.54472E+02,-5.67072E+01) -- (axis cs:1.56012E+02,-5.72505E+01) -- (axis cs:1.55814E+02,-5.74194E+01) -- (axis cs:1.54269E+02,-5.68737E+01) -- cycle ; 
\draw[MWE-empty] (axis cs:1.57598E+02,-5.77743E+01) -- (axis cs:1.59230E+02,-5.82777E+01) -- (axis cs:1.59042E+02,-5.84516E+01) -- (axis cs:1.57405E+02,-5.79457E+01) -- cycle ; 
\draw[MWE-empty] (axis cs:1.60908E+02,-5.87597E+01) -- (axis cs:1.62632E+02,-5.92193E+01) -- (axis cs:1.62456E+02,-5.93980E+01) -- (axis cs:1.60725E+02,-5.89360E+01) -- cycle ; 
\draw[MWE-empty] (axis cs:1.64402E+02,-5.96555E+01) -- (axis cs:1.66217E+02,-6.00672E+01) -- (axis cs:1.66056E+02,-6.02505E+01) -- (axis cs:1.64233E+02,-5.98365E+01) -- cycle ; 
\draw[MWE-empty] (axis cs:1.68077E+02,-6.04533E+01) -- (axis cs:1.69979E+02,-6.08130E+01) -- (axis cs:1.69837E+02,-6.10005E+01) -- (axis cs:1.67924E+02,-6.06388E+01) -- cycle ; 
\draw[MWE-empty] (axis cs:1.71924E+02,-6.11451E+01) -- (axis cs:1.73908E+02,-6.14487E+01) -- (axis cs:1.73786E+02,-6.16401E+01) -- (axis cs:1.71791E+02,-6.13346E+01) -- cycle ; 
\draw[MWE-empty] (axis cs:1.75929E+02,-6.17229E+01) -- (axis cs:1.77984E+02,-6.19668E+01) -- (axis cs:1.77886E+02,-6.21614E+01) -- (axis cs:1.75819E+02,-6.19160E+01) -- cycle ; 
\draw[MWE-empty] (axis cs:1.80071E+02,-6.21796E+01) -- (axis cs:1.82184E+02,-6.23606E+01) -- (axis cs:1.82113E+02,-6.25578E+01) -- (axis cs:1.79986E+02,-6.23756E+01) -- cycle ; 
\draw[MWE-empty] (axis cs:1.84321E+02,-6.25091E+01) -- (axis cs:1.86478E+02,-6.26247E+01) -- (axis cs:1.86434E+02,-6.28237E+01) -- (axis cs:1.84264E+02,-6.27073E+01) -- cycle ; 
\draw[MWE-empty] (axis cs:1.88648E+02,-6.27068E+01) -- (axis cs:1.90828E+02,-6.27553E+01) -- (axis cs:1.90814E+02,-6.29552E+01) -- (axis cs:1.88619E+02,-6.29064E+01) -- cycle ; 
\draw[MWE-empty] (axis cs:1.93013E+02,-6.27699E+01) -- (axis cs:1.95197E+02,-6.27506E+01) -- (axis cs:1.95213E+02,-6.29504E+01) -- (axis cs:1.93014E+02,-6.29699E+01) -- cycle ; 
\draw[MWE-empty] (axis cs:1.97377E+02,-6.26974E+01) -- (axis cs:1.99545E+02,-6.26105E+01) -- (axis cs:1.99591E+02,-6.28094E+01) -- (axis cs:1.97407E+02,-6.28969E+01) -- cycle ; 
\draw[MWE-empty] (axis cs:2.01699E+02,-6.24903E+01) -- (axis cs:2.03833E+02,-6.23372E+01) -- (axis cs:2.03907E+02,-6.25342E+01) -- (axis cs:2.01759E+02,-6.26884E+01) -- cycle ; 
\draw[MWE-empty] (axis cs:2.05943E+02,-6.21517E+01) -- (axis cs:2.08026E+02,-6.19345E+01) -- (axis cs:2.08125E+02,-6.21289E+01) -- (axis cs:2.06030E+02,-6.23475E+01) -- cycle ; 
\draw[MWE-empty] (axis cs:2.10076E+02,-6.16863E+01) -- (axis cs:2.12092E+02,-6.14079E+01) -- (axis cs:2.12216E+02,-6.15991E+01) -- (axis cs:2.10188E+02,-6.18792E+01) -- cycle ; 
\draw[MWE-empty] (axis cs:2.14071E+02,-6.11003E+01) -- (axis cs:2.16010E+02,-6.07642E+01) -- (axis cs:2.16154E+02,-6.09515E+01) -- (axis cs:2.14205E+02,-6.12895E+01) -- cycle ; 
\draw[MWE-empty] (axis cs:2.17906E+02,-6.04008E+01) -- (axis cs:2.19760E+02,-6.00110E+01) -- (axis cs:2.19922E+02,-6.01940E+01) -- (axis cs:2.18060E+02,-6.05860E+01) -- cycle ; 
\draw[MWE-empty] (axis cs:2.21569E+02,-5.95959E+01) -- (axis cs:2.23332E+02,-5.91564E+01) -- (axis cs:2.23509E+02,-5.93347E+01) -- (axis cs:2.21739E+02,-5.97766E+01) -- cycle ; 
\draw[MWE-empty] (axis cs:2.25050E+02,-5.86936E+01) -- (axis cs:2.26721E+02,-5.82085E+01) -- (axis cs:2.26910E+02,-5.83820E+01) -- (axis cs:2.25233E+02,-5.88695E+01) -- cycle ; 
\draw[MWE-empty] (axis cs:2.28346E+02,-5.77022E+01) -- (axis cs:2.29926E+02,-5.71755E+01) -- (axis cs:2.30125E+02,-5.73442E+01) -- (axis cs:2.28541E+02,-5.78732E+01) -- cycle ; 
\draw[MWE-empty] (axis cs:2.31460E+02,-5.66296E+01) -- (axis cs:2.32950E+02,-5.60654E+01) -- (axis cs:2.33157E+02,-5.62291E+01) -- (axis cs:2.31663E+02,-5.67958E+01) -- cycle ; 
\draw[MWE-empty] (axis cs:2.34397E+02,-5.54837E+01) -- (axis cs:2.35800E+02,-5.48854E+01) -- (axis cs:2.36012E+02,-5.50442E+01) -- (axis cs:2.34606E+02,-5.56449E+01) -- cycle ; 
\draw[MWE-empty] (axis cs:2.37162E+02,-5.42714E+01) -- (axis cs:2.38483E+02,-5.36425E+01) -- (axis cs:2.38699E+02,-5.37966E+01) -- (axis cs:2.37376E+02,-5.44279E+01) -- cycle ; 
\draw[MWE-empty] (axis cs:2.39766E+02,-5.29994E+01) -- (axis cs:2.41010E+02,-5.23429E+01) -- (axis cs:2.41227E+02,-5.24925E+01) -- (axis cs:2.39982E+02,-5.31512E+01) -- cycle ; 
\draw[MWE-empty] (axis cs:2.42217E+02,-5.16738E+01) -- (axis cs:2.43389E+02,-5.09926E+01) -- (axis cs:2.43608E+02,-5.11377E+01) -- (axis cs:2.42435E+02,-5.18211E+01) -- cycle ; 
\draw[MWE-empty] (axis cs:2.44526E+02,-5.02999E+01) -- (axis cs:2.45631E+02,-4.95965E+01) -- (axis cs:2.45850E+02,-4.97374E+01) -- (axis cs:2.44745E+02,-5.04429E+01) -- cycle ; 
\draw[MWE-empty] (axis cs:2.46704E+02,-4.88829E+01) -- (axis cs:2.47746E+02,-4.81596E+01) -- (axis cs:2.47965E+02,-4.82965E+01) -- (axis cs:2.46923E+02,-4.90218E+01) -- cycle ; 
\draw[MWE-empty] (axis cs:2.48760E+02,-4.74270E+01) -- (axis cs:2.49745E+02,-4.66858E+01) -- (axis cs:2.49963E+02,-4.68189E+01) -- (axis cs:2.48978E+02,-4.75620E+01) -- cycle ; 
\draw[MWE-empty] (axis cs:2.50704E+02,-4.59363E+01) -- (axis cs:2.51636E+02,-4.51791E+01) -- (axis cs:2.51853E+02,-4.53086E+01) -- (axis cs:2.50921E+02,-4.60676E+01) -- cycle ; 
\draw[MWE-empty] (axis cs:2.52544E+02,-4.44144E+01) -- (axis cs:2.53429E+02,-4.36426E+01) -- (axis cs:2.53644E+02,-4.37688E+01) -- (axis cs:2.52760E+02,-4.45422E+01) -- cycle ; 
\draw[MWE-empty] (axis cs:2.54291E+02,-4.28642E+01) -- (axis cs:2.55131E+02,-4.20795E+01) -- (axis cs:2.55344E+02,-4.22025E+01) -- (axis cs:2.54505E+02,-4.29888E+01) -- cycle ; 
\draw[MWE-empty] (axis cs:2.55951E+02,-4.12888E+01) -- (axis cs:2.56751E+02,-4.04924E+01) -- (axis cs:2.56962E+02,-4.06124E+01) -- (axis cs:2.56163E+02,-4.14103E+01) -- cycle ; 
\draw[MWE-empty] (axis cs:2.57532E+02,-3.96906E+01) -- (axis cs:2.58295E+02,-3.88836E+01) -- (axis cs:2.58504E+02,-3.90009E+01) -- (axis cs:2.57742E+02,-3.98092E+01) -- cycle ; 
\draw[MWE-empty] (axis cs:2.59041E+02,-3.80718E+01) -- (axis cs:2.59771E+02,-3.72554E+01) -- (axis cs:2.59977E+02,-3.73700E+01) -- (axis cs:2.59248E+02,-3.81877E+01) -- cycle ; 
\draw[MWE-empty] (axis cs:2.60484E+02,-3.64345E+01) -- (axis cs:2.61183E+02,-3.56095E+01) -- (axis cs:2.61387E+02,-3.57218E+01) -- (axis cs:2.60689E+02,-3.65480E+01) -- cycle ; 
\draw[MWE-empty] (axis cs:2.61867E+02,-3.47805E+01) -- (axis cs:2.62538E+02,-3.39478E+01) -- (axis cs:2.62740E+02,-3.40578E+01) -- (axis cs:2.62070E+02,-3.48916E+01) -- cycle ; 
\draw[MWE-empty] (axis cs:2.63196E+02,-3.31114E+01) -- (axis cs:2.63841E+02,-3.22717E+01) -- (axis cs:2.64040E+02,-3.23796E+01) -- (axis cs:2.63396E+02,-3.32204E+01) -- cycle ; 
\draw[MWE-empty] (axis cs:2.64474E+02,-3.14287E+01) -- (axis cs:2.65096E+02,-3.05826E+01) -- (axis cs:2.65293E+02,-3.06886E+01) -- (axis cs:2.64672E+02,-3.15356E+01) -- cycle ; 
\draw[MWE-empty] (axis cs:2.65707E+02,-2.97337E+01) -- (axis cs:2.66308E+02,-2.88819E+01) -- (axis cs:2.66503E+02,-2.89861E+01) -- (axis cs:2.65903E+02,-2.98387E+01) -- cycle ; 

 
\draw [MWE-empty] (axis cs:2.66016E+02,-2.87251E+01) -- (axis cs:2.66797E+02,-2.91419E+01)   ;
\node[pin={[pin distance=0.0\onedegree,MWE-label]90:{0$^\circ$}}] at (axis cs:2.66016E+02,-2.87251E+01) {} ;
\draw [MWE-empty] (axis cs:2.71569E+02,-2.00932E+01) -- (axis cs:2.72314E+02,-2.04821E+01)   ;
\node[pin={[pin distance=0.0\onedegree,MWE-label]090:{10$^\circ$}}] at (axis cs:2.71569E+02,-2.00932E+01) {} ;
\draw [MWE-empty] (axis cs:2.76522E+02,-1.13004E+01) -- (axis cs:2.77245E+02,-1.16726E+01)   ;
\node[pin={[pin distance=0.0\onedegree,MWE-label]090:{20$^\circ$}}] at (axis cs:2.76522E+02,-1.13004E+01) {} ;
\draw [MWE-empty] (axis cs:2.81167E+02,-2.42475E+00) -- (axis cs:2.81879E+02,-2.78993E+00)   ;
\node[pin={[pin distance=0.0\onedegree,MWE-label]090:{30$^\circ$}}] at (axis cs:2.81167E+02,-2.42475E+00) {} ;
\draw [MWE-empty] (axis cs:2.85739E+02,6.47217E+00) -- (axis cs:2.86454E+02,6.10515E+00)   ;
\node[pin={[pin distance=0.0\onedegree,MWE-label]090:{40$^\circ$}}] at (axis cs:2.85739E+02,6.47217E+00) {} ;
\draw [MWE-empty] (axis cs:2.90465E+02,1.53325E+01) -- (axis cs:2.91196E+02,1.49546E+01)   ;
\node[pin={[pin distance=0.0\onedegree,MWE-label]090:{50$^\circ$}}] at (axis cs:2.90465E+02,1.53325E+01) {} ;
\draw [MWE-empty] (axis cs:2.95599E+02,2.40907E+01) -- (axis cs:2.96357E+02,2.36917E+01)   ;
\node[pin={[pin distance=0.0\onedegree,MWE-label]090:{60$^\circ$}}] at (axis cs:2.95599E+02,2.40907E+01) {} ;
\draw [MWE-empty] (axis cs:3.01469E+02,3.26590E+01) -- (axis cs:3.02267E+02,3.22267E+01)   ;
\node[pin={[pin distance=0.0\onedegree,MWE-label]090:{70$^\circ$}}] at (axis cs:3.01469E+02,3.26590E+01) {} ;
\draw [MWE-empty] (axis cs:3.08552E+02,4.09041E+01) -- (axis cs:3.09394E+02,4.04231E+01)   ;
\node[pin={[pin distance=0.0\onedegree,MWE-label]090:{80$^\circ$}}] at (axis cs:3.08552E+02,4.09041E+01) {} ;
\draw [MWE-empty] (axis cs:3.17568E+02,4.86036E+01) -- (axis cs:3.18444E+02,4.80548E+01)   ;
\node[pin={[pin distance=0.0\onedegree,MWE-label]090:{90$^\circ$}}] at (axis cs:3.17568E+02,4.86036E+01) {} ;
\draw [MWE-empty] (axis cs:3.29582E+02,5.53674E+01) -- (axis cs:3.30428E+02,5.47306E+01)   ;
\node[pin={[pin distance=0.0\onedegree,MWE-label]0090:{100$^\circ$}}] at (axis cs:3.29582E+02,5.53674E+01) {} ;
\draw [MWE-empty] (axis cs:3.45811E+02,6.05250E+01) -- (axis cs:3.46455E+02,5.97919E+01)   ;
\node[pin={[pin distance=0.0\onedegree,MWE-label]0090:{110$^\circ$}}] at (axis cs:3.45811E+02,6.05250E+01) {} ;
\draw [MWE-empty] (axis cs:6.36807E+00,6.31222E+01) -- (axis cs:6.54138E+00,6.23261E+01)   ;
\node[pin={[pin distance=0.0\onedegree,MWE-label]0090:{120$^\circ$}}] at (axis cs:6.36807E+00,6.31222E+01) {} ;
\draw [MWE-empty] (axis cs:2.82769E+01,6.24205E+01) -- (axis cs:2.78784E+01,6.16426E+01)   ;
\node[pin={[pin distance=0.0\onedegree,MWE-label]0090:{130$^\circ$}}] at (axis cs:2.82769E+01,6.24205E+01) {} ;
\draw [MWE-empty] (axis cs:4.71975E+01,5.86418E+01) -- (axis cs:4.64407E+01,5.79477E+01)   ;
\node[pin={[pin distance=0.0\onedegree,MWE-label]0090:{140$^\circ$}}] at (axis cs:4.71975E+01,5.86418E+01) {} ;
\draw [MWE-empty] (axis cs:6.15567E+01,5.27160E+01) -- (axis cs:6.06857E+01,5.21179E+01)   ;
\node[pin={[pin distance=0.0\onedegree,MWE-label]0090:{150$^\circ$}}] at (axis cs:6.15567E+01,5.27160E+01) {} ;
\draw [MWE-empty] (axis cs:7.21792E+01,4.55021E+01) -- (axis cs:7.13134E+01,4.49840E+01)   ;
\node[pin={[pin distance=0.0\onedegree,MWE-label]0090:{160$^\circ$}}] at (axis cs:7.21792E+01,4.55021E+01) {} ;
\draw [MWE-empty] (axis cs:8.02868E+01,3.75414E+01) -- (axis cs:7.94627E+01,3.70827E+01)   ;
\node[pin={[pin distance=0.0\onedegree,MWE-label]0090:{170$^\circ$}}] at (axis cs:8.02868E+01,3.75414E+01) {} ;
\draw [MWE-empty] (axis cs:8.67966E+01,2.91419E+01) -- (axis cs:8.60164E+01,2.87251E+01)   ;
\node[pin={[pin distance=0.0\onedegree,MWE-label]0090:{180$^\circ$}}] at (axis cs:8.67966E+01,2.91419E+01) {} ;
\draw [MWE-empty] (axis cs:9.23144E+01,2.04821E+01) -- (axis cs:9.15691E+01,2.00932E+01)   ;
\node[pin={[pin distance=0.0\onedegree,MWE-label]090:{190$^\circ$}}] at (axis cs:9.23144E+01,2.04821E+01) {} ;
\draw [MWE-empty] (axis cs:9.72449E+01,1.16726E+01) -- (axis cs:9.65223E+01,1.13004E+01)   ;
\node[pin={[pin distance=0.0\onedegree,MWE-label]090:{200$^\circ$}}] at (axis cs:9.72449E+01,1.16726E+01) {} ;
\draw [MWE-empty] (axis cs:1.01879E+02,2.78993E+00) -- (axis cs:1.01167E+02,2.42475E+00)   ;
\node[pin={[pin distance=0.0\onedegree,MWE-label]090:{210$^\circ$}}] at (axis cs:1.01879E+02,2.78993E+00) {} ;
\draw [MWE-empty] (axis cs:1.06454E+02,-6.10515E+00) -- (axis cs:1.05739E+02,-6.47217E+00)   ;
\node[pin={[pin distance=0.0\onedegree,MWE-label]090:{220$^\circ$}}] at (axis cs:1.06454E+02,-6.10515E+00) {} ;
\draw [MWE-empty] (axis cs:1.11196E+02,-1.49546E+01) -- (axis cs:1.10466E+02,-1.53325E+01)   ;
\node[pin={[pin distance=0.0\onedegree,MWE-label]090:{230$^\circ$}}] at (axis cs:1.11196E+02,-1.49546E+01) {} ;
\draw [MWE-empty] (axis cs:1.16357E+02,-2.36917E+01) -- (axis cs:1.15599E+02,-2.40907E+01)   ;
\node[pin={[pin distance=0.0\onedegree,MWE-label]090:{240$^\circ$}}] at (axis cs:1.16357E+02,-2.36917E+01) {} ;
\draw [MWE-empty] (axis cs:1.22267E+02,-3.22267E+01) -- (axis cs:1.21469E+02,-3.26590E+01)   ;
\node[pin={[pin distance=0.0\onedegree,MWE-label]090:{250$^\circ$}}] at (axis cs:1.22267E+02,-3.22267E+01) {} ;
\draw [MWE-empty] (axis cs:1.29395E+02,-4.04231E+01) -- (axis cs:1.28552E+02,-4.09041E+01)   ;
\node[pin={[pin distance=0.0\onedegree,MWE-label]090:{260$^\circ$}}] at (axis cs:1.29395E+02,-4.04231E+01) {} ;
\draw [MWE-empty] (axis cs:1.38444E+02,-4.80548E+01) -- (axis cs:1.37568E+02,-4.86036E+01)   ;
\node[pin={[pin distance=0.0\onedegree,MWE-label]0090:{270$^\circ$}}] at (axis cs:1.38444E+02,-4.80548E+01) {} ;
\draw [MWE-empty] (axis cs:1.50428E+02,-5.47306E+01) -- (axis cs:1.49582E+02,-5.53674E+01)   ;
\node[pin={[pin distance=0.0\onedegree,MWE-label]0090:{280$^\circ$}}] at (axis cs:1.50428E+02,-5.47306E+01) {} ;
\draw [MWE-empty] (axis cs:1.66455E+02,-5.97919E+01) -- (axis cs:1.65811E+02,-6.05250E+01)   ;
\node[pin={[pin distance=0.0\onedegree,MWE-label]0090:{290$^\circ$}}] at (axis cs:1.66455E+02,-5.97919E+01) {} ;
\draw [MWE-empty] (axis cs:1.86541E+02,-6.23261E+01) -- (axis cs:1.86368E+02,-6.31222E+01)   ;
\node[pin={[pin distance=0.0\onedegree,MWE-label]0090:{300$^\circ$}}] at (axis cs:1.86541E+02,-6.23261E+01) {} ;
\draw [MWE-empty] (axis cs:2.07878E+02,-6.16426E+01) -- (axis cs:2.08277E+02,-6.24205E+01)   ;
\node[pin={[pin distance=0.0\onedegree,MWE-label]0090:{310$^\circ$}}] at (axis cs:2.07878E+02,-6.16426E+01) {} ;
\draw [MWE-empty] (axis cs:2.26441E+02,-5.79477E+01) -- (axis cs:2.27198E+02,-5.86418E+01)   ;
\node[pin={[pin distance=0.0\onedegree,MWE-label]0090:{320$^\circ$}}] at (axis cs:2.26441E+02,-5.79477E+01) {} ;
\draw [MWE-empty] (axis cs:2.40686E+02,-5.21179E+01) -- (axis cs:2.41557E+02,-5.27160E+01)   ;
\node[pin={[pin distance=0.0\onedegree,MWE-label]0090:{330$^\circ$}}] at (axis cs:2.40686E+02,-5.21179E+01) {} ;
\draw [MWE-empty] (axis cs:2.51313E+02,-4.49840E+01) -- (axis cs:2.52179E+02,-4.55021E+01)   ;
\node[pin={[pin distance=0.0\onedegree,MWE-label]0090:{340$^\circ$}}] at (axis cs:2.51313E+02,-4.49840E+01) {} ;
\draw [MWE-empty] (axis cs:2.59463E+02,-3.70827E+01) -- (axis cs:2.60287E+02,-3.75414E+01)   ;
\node[pin={[pin distance=0.0\onedegree,MWE-label]0090:{350$^\circ$}}] at (axis cs:2.59463E+02,-3.70827E+01) {} ;


\end{polaraxis}



% Constellations
%
%
% Some are commented out because they are surrounded by already plotted borders
% The x-coordinate is in fractional hours, so must be times 15
%
\begin{polaraxis}[rotate=90,name=constellations,at=(base.center),anchor=center,axis lines=none]

  \clip (0\tendegree,-6\tendegree) arc (270:90:6\tendegree)
  -- (2\tendegree,6\tendegree)  -- (2\tendegree,-6\tendegree)
   -- cycle ;
  
\draw[constellation-boundary]  (axis cs:344.465,    +35.168228) --  (axis cs:344.463,    +35.668224) --  (axis cs:344.457,    +36.668215) --  (axis cs:344.452,    +37.668206) --  (axis cs:344.447,    +38.668197) --  (axis cs:344.441,    +39.668187) --  (axis cs:344.435,    +40.668177) --  (axis cs:344.429,    +41.668167) --  (axis cs:344.423,    +42.668157) --  (axis cs:344.417,    +43.668146) --  (axis cs:344.41,    +44.668135) --  (axis cs:344.403,    +45.668123) --  (axis cs:344.396,    +46.668111) --  (axis cs:344.389,    +47.668099) --  (axis cs:344.381,    +48.668086) --  (axis cs:344.373,    +49.668073) --  (axis cs:344.365,    +50.668059) --  (axis cs:344.356,    +51.668044) --  (axis cs:344.347,    +52.668029) --  (axis cs:344.338,    +53.668013) --  (axis cs:344.328,    +54.667996) --  (axis cs:344.318,    +55.667978) --  (axis cs:344.307,    +56.667960) --  (axis cs:344.295,    +57.667940) --  (axis cs:344.283,    +58.667920) --  (axis cs:344.27,    +59.667898) --  (axis cs:344.269,    +59.751229) ;
  \draw[constellation-boundary]  (axis cs:344.343,    +53.168021) --  (axis cs:345.358,    +53.171359) --  (axis cs:346.374,    +53.174490) --  (axis cs:347.39,    +53.177412) --  (axis cs:348.405,    +53.180125) --  (axis cs:349.421,    +53.182628) --  (axis cs:350.437,    +53.184919) --  (axis cs:351.453,    +53.186999) ;
  \draw[constellation-boundary]  (axis cs:351.466,    +50.687011) --  (axis cs:351.461,    +51.687006) --  (axis cs:351.456,    +52.687001) --  (axis cs:351.453,    +53.186999) ;
  \draw[constellation-boundary]  (axis cs:351.466,    +50.687011) --  (axis cs:352.48,    +50.688875) --  (axis cs:353.495,    +50.690527) --  (axis cs:354.51,    +50.691965) --  (axis cs:355.271,    +50.692903) ;
  \draw[constellation-boundary]  (axis cs:355.276,    +48.692907) --  (axis cs:355.273,    +49.692905) --  (axis cs:355.271,    +50.692903) ;
  \draw[constellation-boundary]  (axis cs:355.276,    +48.692907) --  (axis cs:355.53,    +48.693192) --  (axis cs:356.543,    +48.694201) --  (axis cs:357.557,    +48.694996) --  (axis cs:358.571,    +48.695576) --  (axis cs:359.584,    +48.695941) --  (axis cs:360.598,    +48.696092) --  (axis cs:361.612,    +48.696027) --  (axis cs:362.626,    +48.695748) --  (axis cs:363.639,    +48.695253) --  (axis cs:364.146,    +48.694926) ;
  \draw[constellation-boundary]  (axis cs:364.143,    +46.694927) --  (axis cs:364.145,    +47.694926) --  (axis cs:364.146,    +48.694926) ;
  \draw[constellation-boundary]  (axis cs:364.143,    +46.694927) --  (axis cs:364.65,    +46.694546) ;
  \draw[constellation-boundary]  (axis cs:362.611,    +25.695750) --  (axis cs:362.612,    +26.695750) --  (axis cs:362.612,    +27.695750) --  (axis cs:362.613,    +28.695750) ;
  \draw[constellation-boundary]  (axis cs:361.606,    +28.696028) --  (axis cs:362.613,    +28.695750) ;
  \draw[constellation-boundary]  (axis cs:361.606,    +28.696028) --  (axis cs:361.606,    +29.696028) --  (axis cs:361.607,    +30.696028) --  (axis cs:361.607,    +31.696028) --  (axis cs:361.607,    +32.029361) ;
  \draw[constellation-boundary]  (axis cs:357.829,    +32.028500) --  (axis cs:358.584,    +32.028913) --  (axis cs:359.592,    +32.029276) --  (axis cs:360.599,    +32.029425) --  (axis cs:361.607,    +32.029361) ;
  \draw[constellation-boundary]  (axis cs:357.829,    +32.028500) --  (axis cs:357.828,    +32.695167) --  (axis cs:357.828,    +32.778500) ;
  \draw[constellation-boundary]  (axis cs:354.049,    +32.774639) --  (axis cs:354.553,    +32.775327) --  (axis cs:355.561,    +32.776543) --  (axis cs:356.568,    +32.777546) --  (axis cs:357.576,    +32.778336) --  (axis cs:357.828,    +32.778500) ;
  \draw[constellation-boundary]  (axis cs:354.049,    +32.774639) --  (axis cs:354.047,    +33.691305) --  (axis cs:354.045,    +34.691303) --  (axis cs:354.044,    +35.191302) ;
  \draw[constellation-boundary]  (axis cs:284.284,    +25.164180) --  (axis cs:284.27,    +26.164094) ;
  \draw[constellation-boundary]  (axis cs:203.969,    +25.360356) --  (axis cs:203.963,    +26.360342) --  (axis cs:203.957,    +27.360327) --  (axis cs:203.954,    +27.860320) ;
  \draw[constellation-boundary]  (axis cs:210.789,    +27.897658) --  (axis cs:210.785,    +28.397646) --  (axis cs:210.777,    +29.397621) --  (axis cs:210.771,    +30.147603) ;
  \draw[constellation-boundary]  (axis cs:210.771,    +30.147603) --  (axis cs:211.392,    +30.151432) --  (axis cs:211.889,    +30.154546) ;
  \draw[constellation-boundary]  (axis cs:211.889,    +30.154546) --  (axis cs:211.887,    +30.404539) --  (axis cs:211.878,    +31.404512) --  (axis cs:211.87,    +32.404484) --  (axis cs:211.861,    +33.404456) --  (axis cs:211.851,    +34.404427) --  (axis cs:211.842,    +35.404397) --  (axis cs:211.832,    +36.404367) --  (axis cs:211.823,    +37.404336) --  (axis cs:211.812,    +38.404304) --  (axis cs:211.802,    +39.404271) --  (axis cs:211.791,    +40.404237) --  (axis cs:211.78,    +41.404202) --  (axis cs:211.769,    +42.404166) --  (axis cs:211.757,    +43.404129) --  (axis cs:211.745,    +44.404091) --  (axis cs:211.732,    +45.404051) --  (axis cs:211.719,    +46.404010) --  (axis cs:211.706,    +47.403967) --  (axis cs:211.692,    +48.403923) --  (axis cs:211.677,    +49.403877) --  (axis cs:211.662,    +50.403829) --  (axis cs:211.646,    +51.403779) --  (axis cs:211.629,    +52.403726) --  (axis cs:211.612,    +53.403672) --  (axis cs:211.594,    +54.403614) --  (axis cs:211.584,    +54.903584) ;
  \draw[constellation-boundary]  (axis cs:211.584,    +54.903584) --  (axis cs:212.077,    +54.906716) --  (axis cs:213.062,    +54.913112) --  (axis cs:214.048,    +54.919685) --  (axis cs:215.033,    +54.926433) --  (axis cs:216.019,    +54.933353) --  (axis cs:217.005,    +54.940444) --  (axis cs:217.991,    +54.947704) --  (axis cs:218.977,    +54.955131) --  (axis cs:219.963,    +54.962721) --  (axis cs:220.95,    +54.970474) --  (axis cs:221.937,    +54.978387) --  (axis cs:222.924,    +54.986458) --  (axis cs:223.911,    +54.994684) --  (axis cs:224.898,    +55.003062) --  (axis cs:225.886,    +55.011591) --  (axis cs:226.874,    +55.020268) --  (axis cs:227.862,    +55.029090) --  (axis cs:228.85,    +55.038055) --  (axis cs:229.591,    +55.044871) ;
  \draw[constellation-boundary]  (axis cs:217.251,    +54.942243) --  (axis cs:217.24,    +55.442202) --  (axis cs:217.217,    +56.442117) --  (axis cs:217.192,    +57.442027) --  (axis cs:217.166,    +58.441932) --  (axis cs:217.139,    +59.441831) --  (axis cs:217.109,    +60.441724) --  (axis cs:217.078,    +61.441611) --  (axis cs:217.045,    +62.441489) ;
  \draw[constellation-boundary]  (axis cs:229.657,    +52.545177) --  (axis cs:229.632,    +53.545059) --  (axis cs:229.605,    +54.544935) --  (axis cs:229.591,    +55.044871) ;
  \draw[constellation-boundary]  (axis cs:229.657,    +52.545177) --  (axis cs:229.905,    +52.547469) --  (axis cs:230.894,    +52.556720) --  (axis cs:231.884,    +52.566107) --  (axis cs:232.874,    +52.575625) --  (axis cs:233.865,    +52.585272) --  (axis cs:234.855,    +52.595045) --  (axis cs:235.846,    +52.604942) --  (axis cs:236.837,    +52.614958) --  (axis cs:237.084,    +52.617481) ;
  \draw[constellation-boundary]  (axis cs:237.125,    +51.117686) --  (axis cs:237.868,    +51.125301) --  (axis cs:238.86,    +51.135554) --  (axis cs:239.852,    +51.145919) --  (axis cs:240.845,    +51.156392) --  (axis cs:241.837,    +51.166970) --  (axis cs:242.83,    +51.177649) --  (axis cs:243.823,    +51.188428) --  (axis cs:244.816,    +51.199302) --  (axis cs:245.809,    +51.210268) --  (axis cs:246.803,    +51.221324) --  (axis cs:247.797,    +51.232465) --  (axis cs:248.791,    +51.243688) --  (axis cs:249.785,    +51.254991) --  (axis cs:250.78,    +51.266369) --  (axis cs:251.775,    +51.277819) --  (axis cs:252.77,    +51.289338) --  (axis cs:253.765,    +51.300922) --  (axis cs:254.761,    +51.312568) --  (axis cs:255.757,    +51.324273) ;
  \draw[constellation-boundary]  (axis cs:237.365,    +39.618913) --  (axis cs:237.348,    +40.618825) --  (axis cs:237.33,    +41.618733) --  (axis cs:237.311,    +42.618638) --  (axis cs:237.292,    +43.618541) --  (axis cs:237.273,    +44.618440) --  (axis cs:237.252,    +45.618336) --  (axis cs:237.231,    +46.618228) --  (axis cs:237.209,    +47.618115) --  (axis cs:237.186,    +48.617999) --  (axis cs:237.162,    +49.617877) --  (axis cs:237.137,    +50.617751) --  (axis cs:237.112,    +51.617619) --  (axis cs:237.084,    +52.617481) ;
  \draw[constellation-boundary]  (axis cs:232.644,    +39.572120) --  (axis cs:233.141,    +39.576913) --  (axis cs:234.134,    +39.586595) --  (axis cs:235.128,    +39.596402) --  (axis cs:236.123,    +39.606332) --  (axis cs:237.117,    +39.616382) --  (axis cs:238.111,    +39.626550) --  (axis cs:239.106,    +39.636831) --  (axis cs:240.101,    +39.647223) --  (axis cs:241.095,    +39.657722) --  (axis cs:242.09,    +39.668326) --  (axis cs:243.086,    +39.679032) --  (axis cs:244.081,    +39.689835) --  (axis cs:245.076,    +39.700734) --  (axis cs:246.072,    +39.711724) ;
  \draw[constellation-boundary]  (axis cs:232.747,    +32.572617) --  (axis cs:232.733,    +33.572551) --  (axis cs:232.719,    +34.572484) --  (axis cs:232.705,    +35.572415) --  (axis cs:232.69,    +36.572344) --  (axis cs:232.675,    +37.572272) --  (axis cs:232.66,    +38.572197) --  (axis cs:232.644,    +39.572120) ;
  \draw[constellation-boundary]  (axis cs:229.016,    +32.537682) --  (axis cs:229.265,    +32.539951) --  (axis cs:230.259,    +32.549115) --  (axis cs:231.254,    +32.558415) --  (axis cs:232.249,    +32.567850) --  (axis cs:232.747,    +32.572617) ;
  \draw[constellation-boundary]  (axis cs:229.1,    +25.538063) --  (axis cs:229.088,    +26.538012) --  (axis cs:229.077,    +27.537959) --  (axis cs:229.065,    +28.537906) --  (axis cs:229.053,    +29.537852) --  (axis cs:229.041,    +30.537796) --  (axis cs:229.029,    +31.537740) --  (axis cs:229.016,    +32.537682) ;
  \draw[constellation-boundary]  (axis cs:227.606,    +25.524616) --  (axis cs:228.353,    +25.531300) --  (axis cs:229.349,    +25.540335) --  (axis cs:230.345,    +25.549510) --  (axis cs:231.341,    +25.558822) --  (axis cs:232.337,    +25.568269) --  (axis cs:233.334,    +25.577847) --  (axis cs:234.33,    +25.587553) --  (axis cs:235.326,    +25.597385) --  (axis cs:236.323,    +25.607339) --  (axis cs:237.32,    +25.617414) --  (axis cs:238.316,    +25.627604) --  (axis cs:239.313,    +25.637909) --  (axis cs:240.31,    +25.648323) --  (axis cs:241.307,    +25.658845) --  (axis cs:242.304,    +25.669471) --  (axis cs:243.302,    +25.680198) --  (axis cs:243.8,    +25.685599) ;
  \draw[constellation-boundary]  (axis cs:180.634,    +85.803908) --  (axis cs:181.468,    +85.803961) --  (axis cs:182.303,    +85.804191) --  (axis cs:183.137,    +85.804598) --  (axis cs:183.972,    +85.805181) --  (axis cs:184.806,    +85.805942) --  (axis cs:185.641,    +85.806878) --  (axis cs:186.476,    +85.807992) --  (axis cs:187.311,    +85.809281) --  (axis cs:188.146,    +85.810747) --  (axis cs:188.982,    +85.812388) --  (axis cs:189.818,    +85.814204) --  (axis cs:190.654,    +85.816196) --  (axis cs:191.49,    +85.818362) --  (axis cs:192.327,    +85.820702) --  (axis cs:193.164,    +85.823217) --  (axis cs:194.002,    +85.825905) --  (axis cs:194.84,    +85.828766) --  (axis cs:195.679,    +85.831799) --  (axis cs:196.518,    +85.835004) --  (axis cs:197.357,    +85.838381) --  (axis cs:198.197,    +85.841928) --  (axis cs:199.038,    +85.845645) --  (axis cs:199.88,    +85.849532) --  (axis cs:200.722,    +85.853587) --  (axis cs:201.564,    +85.857810) --  (axis cs:202.408,    +85.862201) --  (axis cs:203.252,    +85.866757) --  (axis cs:204.097,    +85.871480) --  (axis cs:204.943,    +85.876367) --  (axis cs:205.789,    +85.881417) --  (axis cs:206.637,    +85.886631) --  (axis cs:207.485,    +85.892006) --  (axis cs:208.334,    +85.897542) --  (axis cs:209.184,    +85.903237) --  (axis cs:210.036,    +85.909092) --  (axis cs:210.888,    +85.915104) --  (axis cs:211.741,    +85.921272) --  (axis cs:212.595,    +85.927596) --  (axis cs:213.023,    +85.930816) ;
  \draw[constellation-boundary]  (axis cs:216.783,    +79.444991) --  (axis cs:216.535,    +80.444059) --  (axis cs:216.23,    +81.442911) --  (axis cs:215.846,    +82.441461) --  (axis cs:215.345,    +83.439573) --  (axis cs:214.665,    +84.437011) --  (axis cs:213.689,    +85.433331) --  (axis cs:213.023,    +85.930816) ;
  \draw[constellation-boundary]  (axis cs:203.809,    +79.362941) --  (axis cs:204.044,    +79.364133) --  (axis cs:204.985,    +79.369019) --  (axis cs:205.926,    +79.374087) --  (axis cs:206.868,    +79.379336) --  (axis cs:207.81,    +79.384764) --  (axis cs:208.752,    +79.390370) --  (axis cs:209.695,    +79.396152) --  (axis cs:210.638,    +79.402110) --  (axis cs:211.582,    +79.408241) --  (axis cs:212.526,    +79.414544) --  (axis cs:213.471,    +79.421017) --  (axis cs:214.417,    +79.427659) --  (axis cs:215.363,    +79.434469) --  (axis cs:216.309,    +79.441443) --  (axis cs:216.783,    +79.444991) ;
  \draw[constellation-boundary]  (axis cs:204.157,    +76.363819) --  (axis cs:204.059,    +77.363572) --  (axis cs:203.945,    +78.363283) --  (axis cs:203.809,    +79.362941) ;
  \draw[constellation-boundary]  (axis cs:180.611,    +76.303908) --  (axis cs:181.561,    +76.303968) --  (axis cs:182.511,    +76.304230) --  (axis cs:183.461,    +76.304694) --  (axis cs:184.411,    +76.305358) --  (axis cs:185.362,    +76.306224) --  (axis cs:186.312,    +76.307290) --  (axis cs:187.262,    +76.308557) --  (axis cs:188.213,    +76.310024) --  (axis cs:189.163,    +76.311692) --  (axis cs:190.114,    +76.313558) --  (axis cs:191.064,    +76.315623) --  (axis cs:192.015,    +76.317887) --  (axis cs:192.966,    +76.320349) --  (axis cs:193.918,    +76.323008) --  (axis cs:194.869,    +76.325863) --  (axis cs:195.821,    +76.328914) --  (axis cs:196.772,    +76.332159) --  (axis cs:197.724,    +76.335599) --  (axis cs:198.677,    +76.339232) --  (axis cs:199.629,    +76.343057) --  (axis cs:200.582,    +76.347073) --  (axis cs:201.535,    +76.351280) --  (axis cs:202.488,    +76.355675) --  (axis cs:203.442,    +76.360259) --  (axis cs:204.157,    +76.363819) ;
  \draw[constellation-boundary]  (axis cs:181.594,    +33.303971) --  (axis cs:182.586,    +33.304245) --  (axis cs:183.578,    +33.304728) --  (axis cs:184.57,    +33.305422) --  (axis cs:185.562,    +33.306326) --  (axis cs:186.554,    +33.307439) ;
  \draw[constellation-boundary]  (axis cs:181.591,    +44.303971) --  (axis cs:182.579,    +44.304243) --  (axis cs:182.826,    +44.304344) ;
  \draw[constellation-boundary]  (axis cs:182.826,    +44.304344) --  (axis cs:182.826,    +45.304344) --  (axis cs:182.825,    +46.304344) --  (axis cs:182.824,    +47.304344) --  (axis cs:182.823,    +48.304344) --  (axis cs:182.822,    +49.304343) --  (axis cs:182.821,    +50.304343) --  (axis cs:182.82,    +51.304343) --  (axis cs:182.819,    +52.304343) ;
  \draw[constellation-boundary]  (axis cs:182.819,    +52.304343) --  (axis cs:183.557,    +52.304722) --  (axis cs:184.541,    +52.305410) --  (axis cs:185.525,    +52.306307) --  (axis cs:186.509,    +52.307412) --  (axis cs:187.494,    +52.308724) --  (axis cs:188.478,    +52.310243) --  (axis cs:189.462,    +52.311970) --  (axis cs:190.447,    +52.313903) --  (axis cs:191.431,    +52.316041) --  (axis cs:192.416,    +52.318385) --  (axis cs:193.4,    +52.320933) --  (axis cs:194.385,    +52.323685) --  (axis cs:195.37,    +52.326640) --  (axis cs:196.354,    +52.329796) --  (axis cs:197.339,    +52.333154) --  (axis cs:198.324,    +52.336713) --  (axis cs:199.309,    +52.340470) --  (axis cs:200.294,    +52.344425) --  (axis cs:201.279,    +52.348578) --  (axis cs:202.264,    +52.352926) --  (axis cs:203.25,    +52.357468) --  (axis cs:203.742,    +52.359812) ;
  \draw[constellation-boundary]  (axis cs:203.795,    +47.859938) --  (axis cs:204.289,    +47.862335) --  (axis cs:205.277,    +47.867273) --  (axis cs:206.264,    +47.872402) --  (axis cs:207.252,    +47.877719) --  (axis cs:208.24,    +47.883224) --  (axis cs:209.228,    +47.888915) --  (axis cs:210.216,    +47.894791) --  (axis cs:211.205,    +47.900849) --  (axis cs:211.699,    +47.903945) ;
  \draw[constellation-boundary]  (axis cs:203.795,    +47.859938) --  (axis cs:203.79,    +48.359926) --  (axis cs:203.779,    +49.359899) --  (axis cs:203.767,    +50.359871) --  (axis cs:203.755,    +51.359842) --  (axis cs:203.742,    +52.359812) ;
  \draw[constellation-boundary]  (axis cs:200.227,    +27.843782) --  (axis cs:200.475,    +27.844798) --  (axis cs:201.469,    +27.848987) --  (axis cs:202.463,    +27.853374) --  (axis cs:203.457,    +27.857956) --  (axis cs:204.451,    +27.862733) --  (axis cs:205.445,    +27.867702) --  (axis cs:206.439,    +27.872864) --  (axis cs:207.433,    +27.878215) --  (axis cs:208.427,    +27.883755) --  (axis cs:209.422,    +27.889482) --  (axis cs:210.416,    +27.895394) --  (axis cs:210.789,    +27.897658) ;
  \draw[constellation-boundary]  (axis cs:200.227,    +27.843782) --  (axis cs:200.224,    +28.343776) --  (axis cs:200.219,    +29.343765) --  (axis cs:200.213,    +30.343754) --  (axis cs:200.208,    +31.343743) ;
  \draw[constellation-boundary]  (axis cs:186.558,    +31.307441) --  (axis cs:187.55,    +31.308765) --  (axis cs:188.543,    +31.310297) --  (axis cs:189.536,    +31.312038) --  (axis cs:190.528,    +31.313987) --  (axis cs:191.521,    +31.316143) --  (axis cs:192.513,    +31.318506) --  (axis cs:193.506,    +31.321075) --  (axis cs:194.499,    +31.323850) --  (axis cs:195.492,    +31.326829) --  (axis cs:196.484,    +31.330012) --  (axis cs:197.477,    +31.333397) --  (axis cs:198.47,    +31.336984) --  (axis cs:199.463,    +31.340772) --  (axis cs:200.208,    +31.343743) ;
  \draw[constellation-boundary]  (axis cs:186.558,    +31.307441) --  (axis cs:186.556,    +32.307440) --  (axis cs:186.554,    +33.307439) ;
  \draw[constellation-boundary]  (axis cs:344.269,    +59.751229) --  (axis cs:345.289,    +59.754582) --  (axis cs:346.309,    +59.757727) --  (axis cs:347.329,    +59.760662) --  (axis cs:348.349,    +59.763387) --  (axis cs:348.86,    +59.764670) ;
  \draw[constellation-boundary]  (axis cs:348.86,    +59.764670) --  (axis cs:348.85,    +60.681326) --  (axis cs:348.84,    +61.681313) --  (axis cs:348.829,    +62.681299) --  (axis cs:348.816,    +63.681284) ;
  \draw[constellation-boundary]  (axis cs:348.816,    +63.681284) --  (axis cs:349.328,    +63.682518) --  (axis cs:350.352,    +63.684828) --  (axis cs:351.377,    +63.686925) --  (axis cs:352.401,    +63.688807) --  (axis cs:353.425,    +63.690474) --  (axis cs:354.449,    +63.691926) --  (axis cs:355.218,    +63.692873) ;
  \draw[constellation-boundary]  (axis cs:355.218,    +63.692873) --  (axis cs:355.212,    +64.692869) --  (axis cs:355.205,    +65.692865) --  (axis cs:355.198,    +66.692861) ;
  \draw[constellation-boundary]  (axis cs:355.198,    +66.692861) --  (axis cs:355.455,    +66.693151) --  (axis cs:356.483,    +66.694174) --  (axis cs:357.511,    +66.694980) --  (axis cs:358.539,    +66.695569) --  (axis cs:359.567,    +66.695939) --  (axis cs:360.595,    +66.696092) --  (axis cs:361.623,    +66.696026) --  (axis cs:362.652,    +66.695743) --  (axis cs:363.68,    +66.695241) --  (axis cs:364.708,    +66.694522) ;
  \draw[constellation-boundary]  (axis cs:300.573,    +59.851075) --  (axis cs:300.552,    +60.350965) --  (axis cs:300.508,    +61.350736) --  (axis cs:300.485,    +61.850615) ;
  \draw[constellation-boundary]  (axis cs:300.485,    +61.850615) --  (axis cs:301.497,    +61.861116) --  (axis cs:302.508,    +61.871508) --  (axis cs:303.52,    +61.881786) --  (axis cs:304.533,    +61.891949) --  (axis cs:305.545,    +61.901992) --  (axis cs:306.558,    +61.911913) --  (axis cs:306.812,    +61.914374) ;
  \draw[constellation-boundary]  (axis cs:306.812,    +61.914374) --  (axis cs:306.79,    +62.414266) --  (axis cs:306.743,    +63.414039) --  (axis cs:306.693,    +64.413796) --  (axis cs:306.639,    +65.413535) --  (axis cs:306.58,    +66.413252) --  (axis cs:306.517,    +67.412946) ;
  \draw[constellation-boundary]  (axis cs:306.517,    +67.412946) --  (axis cs:307.28,    +67.420309) --  (axis cs:308.298,    +67.430012) --  (axis cs:309.316,    +67.439585) --  (axis cs:310.334,    +67.449023) ;
  \draw[constellation-boundary]  (axis cs:310.334,    +67.449023) --  (axis cs:310.269,    +68.448723) --  (axis cs:310.197,    +69.448395) --  (axis cs:310.119,    +70.448035) --  (axis cs:310.033,    +71.447638) --  (axis cs:309.937,    +72.447197) --  (axis cs:309.83,    +73.446705) --  (axis cs:309.709,    +74.446151) --  (axis cs:309.573,    +75.445523) ;
  \draw[constellation-boundary]  (axis cs:301.873,    +75.370868) --  (axis cs:302.385,    +75.376056) --  (axis cs:303.41,    +75.386348) --  (axis cs:304.435,    +75.396523) --  (axis cs:305.462,    +75.406577) --  (axis cs:306.488,    +75.416507) --  (axis cs:307.516,    +75.426311) --  (axis cs:308.544,    +75.435984) --  (axis cs:309.573,    +75.445523) ;
  \draw[constellation-boundary]  (axis cs:301.873,    +75.370868) --  (axis cs:301.703,    +76.370001) --  (axis cs:301.506,    +77.369000) --  (axis cs:301.275,    +78.367830) --  (axis cs:301.002,    +79.366444) --  (axis cs:300.674,    +80.364773) ;
  \draw[constellation-boundary]  (axis cs:300.674,    +80.364773) --  (axis cs:301.192,    +80.370027) --  (axis cs:302.228,    +80.380450) --  (axis cs:303.266,    +80.390759) --  (axis cs:304.305,    +80.400949) --  (axis cs:305.345,    +80.411018) --  (axis cs:306.387,    +80.420962) --  (axis cs:307.429,    +80.430777) --  (axis cs:308.473,    +80.440460) --  (axis cs:309.517,    +80.450008) --  (axis cs:310.563,    +80.459417) --  (axis cs:311.609,    +80.468685) --  (axis cs:312.657,    +80.477807) --  (axis cs:313.706,    +80.486781) ;
  \draw[constellation-boundary]  (axis cs:313.706,    +80.486781) --  (axis cs:313.36,    +81.485314) --  (axis cs:312.923,    +82.483461) --  (axis cs:312.352,    +83.481043) --  (axis cs:311.577,    +84.477755) --  (axis cs:310.462,    +85.473021) --  (axis cs:308.721,    +86.465621) --  (axis cs:308.331,    +86.630629) ;
  \draw[constellation-boundary]  (axis cs:308.331,    +86.630629) --  (axis cs:309.459,    +86.640190) --  (axis cs:310.59,    +86.649601) --  (axis cs:311.724,    +86.658859) --  (axis cs:312.862,    +86.667959) --  (axis cs:314.004,    +86.676897) --  (axis cs:315.149,    +86.685669) --  (axis cs:316.297,    +86.694271) --  (axis cs:317.448,    +86.702699) --  (axis cs:318.603,    +86.710950) --  (axis cs:319.762,    +86.719019) --  (axis cs:320.923,    +86.726902) --  (axis cs:322.087,    +86.734596) --  (axis cs:323.255,    +86.742097) --  (axis cs:324.426,    +86.749401) --  (axis cs:325.599,    +86.756506) --  (axis cs:326.776,    +86.763407) --  (axis cs:327.955,    +86.770101) --  (axis cs:329.137,    +86.776584) --  (axis cs:330.322,    +86.782854) --  (axis cs:331.51,    +86.788907) --  (axis cs:332.7,    +86.794741) --  (axis cs:333.892,    +86.800351) --  (axis cs:335.087,    +86.805736) --  (axis cs:336.284,    +86.810892) --  (axis cs:337.484,    +86.815817) --  (axis cs:338.685,    +86.820509) --  (axis cs:339.889,    +86.824964) --  (axis cs:341.094,    +86.829181) --  (axis cs:342.302,    +86.833156) --  (axis cs:343.511,    +86.836889) ;
  \draw[constellation-boundary]  (axis cs:343.511,    +86.836889) --  (axis cs:342.404,    +87.668572) --  (axis cs:339.261,    +88.663876) ;
  \draw[constellation-boundary]  (axis cs:339.261,    +88.663876) --  (axis cs:340.749,    +88.668190) --  (axis cs:342.243,    +88.672206) --  (axis cs:343.742,    +88.675922) --  (axis cs:345.247,    +88.679333) --  (axis cs:346.757,    +88.682436) --  (axis cs:348.272,    +88.685228) --  (axis cs:349.79,    +88.687706) --  (axis cs:351.313,    +88.689867) --  (axis cs:352.838,    +88.691710) --  (axis cs:354.366,    +88.693233) --  (axis cs:355.896,    +88.694433) --  (axis cs:357.428,    +88.695310) --  (axis cs:358.961,    +88.695862) --  (axis cs:360.494,    +88.696090) --  (axis cs:362.028,    +88.695992) --  (axis cs:363.561,    +88.695569) ;
  \draw[constellation-boundary]  (axis cs:335.911,    +56.882568) --  (axis cs:336.165,    +56.883841) --  (axis cs:337.182,    +56.888807) --  (axis cs:338.199,    +56.893575) --  (axis cs:339.216,    +56.898144) --  (axis cs:340.234,    +56.902513) --  (axis cs:341.251,    +56.906680) --  (axis cs:342.269,    +56.910643) --  (axis cs:343.286,    +56.914402) --  (axis cs:344.304,    +56.917955) ;
  \draw[constellation-boundary]  (axis cs:335.931,    +55.632620) --  (axis cs:335.915,    +56.632579) --  (axis cs:335.911,    +56.882568) ;
  \draw[constellation-boundary]  (axis cs:333.138,    +55.617836) --  (axis cs:334.153,    +55.623381) --  (axis cs:335.169,    +55.628733) --  (axis cs:335.931,    +55.632620) ;
  \draw[constellation-boundary]  (axis cs:333.175,    +53.367939) --  (axis cs:333.171,    +53.617928) --  (axis cs:333.155,    +54.617883) --  (axis cs:333.138,    +55.617836) ;
  \draw[constellation-boundary]  (axis cs:330.639,    +53.353265) --  (axis cs:331.146,    +53.356294) --  (axis cs:332.16,    +53.362211) --  (axis cs:333.175,    +53.367939) ;
  \draw[constellation-boundary]  (axis cs:309.831,    +55.275325) --  (axis cs:310.842,    +55.284695) --  (axis cs:311.854,    +55.293926) --  (axis cs:312.865,    +55.303017) --  (axis cs:313.877,    +55.311964) --  (axis cs:314.889,    +55.320764) --  (axis cs:315.901,    +55.329415) --  (axis cs:316.914,    +55.337914) --  (axis cs:317.927,    +55.346258) --  (axis cs:318.939,    +55.354445) --  (axis cs:319.953,    +55.362472) --  (axis cs:320.966,    +55.370336) --  (axis cs:321.979,    +55.378035) --  (axis cs:322.993,    +55.385567) --  (axis cs:324.007,    +55.392929) --  (axis cs:325.021,    +55.400119) --  (axis cs:326.036,    +55.407135) --  (axis cs:327.05,    +55.413973) --  (axis cs:328.065,    +55.420633) --  (axis cs:329.08,    +55.427112) --  (axis cs:330.095,    +55.433408) --  (axis cs:330.602,    +55.436487) ;
  \draw[constellation-boundary]  (axis cs:309.831,    +55.275325) --  (axis cs:309.827,    +55.441970) --  (axis cs:309.797,    +56.441831) --  (axis cs:309.765,    +57.441684) --  (axis cs:309.732,    +58.441530) --  (axis cs:309.697,    +59.441366) --  (axis cs:309.66,    +60.441192) --  (axis cs:309.624,    +61.357690) ;
  \draw[constellation-boundary]  (axis cs:308.661,    +61.348638) --  (axis cs:309.624,    +61.357690) ;
  \draw[constellation-boundary]  (axis cs:308.717,    +59.932235) --  (axis cs:308.697,    +60.432145) --  (axis cs:308.661,    +61.348638) ;
  \draw[constellation-boundary]  (axis cs:181.596,    +28.303971) --  (axis cs:181.595,    +29.303971) --  (axis cs:181.595,    +30.303971) --  (axis cs:181.595,    +31.303971) --  (axis cs:181.595,    +32.303971) --  (axis cs:181.594,    +33.303971) --  (axis cs:181.594,    +34.303971) --  (axis cs:181.594,    +35.303971) --  (axis cs:181.594,    +36.303971) --  (axis cs:181.593,    +37.303971) --  (axis cs:181.593,    +38.303971) --  (axis cs:181.593,    +39.303971) --  (axis cs:181.593,    +40.303971) --  (axis cs:181.592,    +41.303971) --  (axis cs:181.592,    +42.303971) --  (axis cs:181.592,    +43.303971) --  (axis cs:181.591,    +44.303971) ;
  \draw[constellation-boundary]  (axis cs:246.28,    +26.712875) --  (axis cs:246.266,    +27.712798) --  (axis cs:246.252,    +28.712719) --  (axis cs:246.237,    +29.712638) --  (axis cs:246.222,    +30.712556) --  (axis cs:246.207,    +31.712472) --  (axis cs:246.192,    +32.712386) --  (axis cs:246.176,    +33.712298) --  (axis cs:246.159,    +34.712209) --  (axis cs:246.143,    +35.712117) --  (axis cs:246.126,    +36.712022) --  (axis cs:246.108,    +37.711925) --  (axis cs:246.09,    +38.711826) --  (axis cs:246.072,    +39.711724) ;
  \draw[constellation-boundary]  (axis cs:243.787,    +26.685525) --  (axis cs:244.285,    +26.690949) --  (axis cs:245.282,    +26.701867) --  (axis cs:246.28,    +26.712875) ;
  \draw[constellation-boundary]  (axis cs:243.8,    +25.685599) --  (axis cs:243.787,    +26.685525) ;
  \draw[constellation-boundary]  (axis cs:290.133,    +27.732407) --  (axis cs:290.258,    +27.733840) --  (axis cs:291.26,    +27.745259) --  (axis cs:292.262,    +27.756603) --  (axis cs:293.265,    +27.767870) --  (axis cs:294.267,    +27.779056) --  (axis cs:295.27,    +27.790158) --  (axis cs:296.272,    +27.801171) ;
  \draw[constellation-boundary]  (axis cs:290.095,    +30.232193) --  (axis cs:290.221,    +30.233626) --  (axis cs:291.223,    +30.245048) --  (axis cs:291.599,    +30.249312) ;
  \draw[constellation-boundary]  (axis cs:291.599,    +30.249312) --  (axis cs:291.583,    +31.249225) --  (axis cs:291.568,    +32.249136) --  (axis cs:291.552,    +33.249046) --  (axis cs:291.535,    +34.248953) --  (axis cs:291.519,    +35.248858) --  (axis cs:291.501,    +36.248760) --  (axis cs:291.493,    +36.748711) ;
  \draw[constellation-boundary]  (axis cs:291.493,    +36.748711) --  (axis cs:292.12,    +36.755799) ;
  \draw[constellation-boundary]  (axis cs:292.12,    +36.755799) --  (axis cs:292.111,    +37.255749) --  (axis cs:292.093,    +38.255647) --  (axis cs:292.074,    +39.255543) --  (axis cs:292.055,    +40.255435) --  (axis cs:292.035,    +41.255324) --  (axis cs:292.015,    +42.255210) --  (axis cs:291.994,    +43.255092) --  (axis cs:291.984,    +43.755031) ;
  \draw[constellation-boundary]  (axis cs:288.47,    +43.714937) --  (axis cs:288.972,    +43.720717) --  (axis cs:289.976,    +43.732226) --  (axis cs:290.98,    +43.743666) --  (axis cs:291.984,    +43.755031) ;
  \draw[constellation-boundary]  (axis cs:288.47,    +43.714937) --  (axis cs:288.459,    +44.214873) --  (axis cs:288.436,    +45.214741) --  (axis cs:288.413,    +46.214605) --  (axis cs:288.388,    +47.214463) --  (axis cs:288.375,    +47.714391) ;
  \draw[constellation-boundary]  (axis cs:287.121,    +47.699864) --  (axis cs:287.108,    +48.199789) --  (axis cs:287.081,    +49.199634) --  (axis cs:287.053,    +50.199473) --  (axis cs:287.024,    +51.199305) --  (axis cs:286.994,    +52.199129) --  (axis cs:286.962,    +53.198945) --  (axis cs:286.929,    +54.198753) --  (axis cs:286.895,    +55.198551) --  (axis cs:286.877,    +55.698446) ;
  \draw[constellation-boundary]  (axis cs:286.877,    +55.698446) --  (axis cs:287.63,    +55.707185) --  (axis cs:288.635,    +55.718782) --  (axis cs:289.641,    +55.730313) --  (axis cs:290.647,    +55.741776) --  (axis cs:291.653,    +55.753167) --  (axis cs:291.905,    +55.756003) ;
  \draw[constellation-boundary]  (axis cs:291.905,    +55.756003) --  (axis cs:291.887,    +56.255901) --  (axis cs:291.849,    +57.255690) --  (axis cs:291.809,    +58.255468) ;
  \draw[constellation-boundary]  (axis cs:291.809,    +58.255468) --  (axis cs:292.565,    +58.263952) --  (axis cs:293.572,    +58.275194) --  (axis cs:294.58,    +58.286354) --  (axis cs:295.588,    +58.297428) --  (axis cs:296.596,    +58.308412) --  (axis cs:297.1,    +58.313870) ;
  \draw[constellation-boundary]  (axis cs:297.1,    +58.313870) --  (axis cs:297.06,    +59.313653) --  (axis cs:297.039,    +59.813539) ;
  \draw[constellation-boundary]  (axis cs:297.039,    +59.813539) --  (axis cs:297.544,    +59.818976) --  (axis cs:298.553,    +59.829776) --  (axis cs:299.563,    +59.840477) --  (axis cs:300.573,    +59.851075) --  (axis cs:301.584,    +59.861567) --  (axis cs:302.595,    +59.871948) --  (axis cs:303.606,    +59.882217) --  (axis cs:304.617,    +59.892370) --  (axis cs:305.629,    +59.902403) --  (axis cs:306.641,    +59.912314) --  (axis cs:307.653,    +59.922099) --  (axis cs:308.666,    +59.931756) --  (axis cs:308.717,    +59.932235) ;
  \draw[constellation-boundary]  (axis cs:330.763,    +44.603636) --  (axis cs:330.751,    +45.603600) --  (axis cs:330.738,    +46.603562) --  (axis cs:330.725,    +47.603523) --  (axis cs:330.712,    +48.603483) --  (axis cs:330.697,    +49.603440) --  (axis cs:330.683,    +50.603396) --  (axis cs:330.668,    +51.603350) --  (axis cs:330.652,    +52.603302) --  (axis cs:330.635,    +53.603252) --  (axis cs:330.617,    +54.603199) --  (axis cs:330.602,    +55.436487) ;
  \draw[constellation-boundary]  (axis cs:329.879,    +44.598244) --  (axis cs:330.257,    +44.600572) --  (axis cs:330.763,    +44.603636) ;
  \draw[constellation-boundary]  (axis cs:329.882,    +44.348254) --  (axis cs:329.879,    +44.598244) ;
  \draw[constellation-boundary]  (axis cs:329.377,    +44.345110) --  (axis cs:329.882,    +44.348254) ;
  \draw[constellation-boundary]  (axis cs:329.461,    +36.595374) --  (axis cs:329.451,    +37.595343) --  (axis cs:329.441,    +38.595311) --  (axis cs:329.431,    +39.595279) --  (axis cs:329.42,    +40.595245) --  (axis cs:329.409,    +41.595211) --  (axis cs:329.397,    +42.595175) --  (axis cs:329.386,    +43.595138) --  (axis cs:329.377,    +44.345110) ;
  \draw[constellation-boundary]  (axis cs:327.32,    +36.581538) --  (axis cs:328.327,    +36.588151) --  (axis cs:329.335,    +36.594583) --  (axis cs:330.343,    +36.600833) --  (axis cs:331.35,    +36.606899) ;
  \draw[constellation-boundary]  (axis cs:327.395,    +28.581788) --  (axis cs:327.387,    +29.581759) --  (axis cs:327.378,    +30.581730) --  (axis cs:327.369,    +31.581699) --  (axis cs:327.359,    +32.581669) --  (axis cs:327.35,    +33.581637) --  (axis cs:327.34,    +34.581605) --  (axis cs:327.33,    +35.581572) --  (axis cs:327.32,    +36.581538) ;
  \draw[constellation-boundary]  (axis cs:315.084,    +28.487183) --  (axis cs:315.335,    +28.489352) --  (axis cs:316.34,    +28.497937) --  (axis cs:317.344,    +28.506370) --  (axis cs:318.349,    +28.514648) --  (axis cs:319.354,    +28.522770) --  (axis cs:320.359,    +28.530731) --  (axis cs:321.364,    +28.538530) --  (axis cs:322.369,    +28.546165) --  (axis cs:323.374,    +28.553632) --  (axis cs:324.379,    +28.560931) --  (axis cs:325.384,    +28.568058) --  (axis cs:326.39,    +28.575011) --  (axis cs:327.395,    +28.581788) ;
  \draw[constellation-boundary]  (axis cs:315.084,    +28.487183) --  (axis cs:315.073,    +29.487134) ;
  \draw[constellation-boundary]  (axis cs:296.251,    +29.301054) --  (axis cs:297.254,    +29.311978) --  (axis cs:298.257,    +29.322807) --  (axis cs:299.26,    +29.333538) --  (axis cs:300.263,    +29.344167) --  (axis cs:301.266,    +29.354692) --  (axis cs:302.27,    +29.365109) --  (axis cs:303.273,    +29.375415) --  (axis cs:304.277,    +29.385607) --  (axis cs:305.281,    +29.395681) --  (axis cs:306.285,    +29.405635) --  (axis cs:307.289,    +29.415465) --  (axis cs:308.293,    +29.425169) --  (axis cs:309.297,    +29.434743) --  (axis cs:310.301,    +29.444185) --  (axis cs:311.305,    +29.453491) --  (axis cs:312.31,    +29.462658) --  (axis cs:313.314,    +29.471685) --  (axis cs:314.319,    +29.480568) --  (axis cs:315.073,    +29.487134) ;
  \draw[constellation-boundary]  (axis cs:296.272,    +27.801171) --  (axis cs:296.265,    +28.301132) --  (axis cs:296.251,    +29.301054) ;
  \draw[constellation-boundary]  (axis cs:196.097,    +69.329372) --  (axis cs:196.07,    +70.329325) --  (axis cs:196.039,    +71.329275) --  (axis cs:196.005,    +72.329218) --  (axis cs:195.967,    +73.329156) --  (axis cs:195.924,    +74.329085) --  (axis cs:195.876,    +75.329005) --  (axis cs:195.821,    +76.328914) ;
  \draw[constellation-boundary]  (axis cs:196.097,    +69.329372) --  (axis cs:197.066,    +69.332676) --  (axis cs:198.035,    +69.336177) --  (axis cs:199.005,    +69.339874) --  (axis cs:199.974,    +69.343767) --  (axis cs:200.944,    +69.347854) --  (axis cs:201.913,    +69.352134) --  (axis cs:202.883,    +69.356606) --  (axis cs:203.854,    +69.361268) --  (axis cs:204.824,    +69.366120) --  (axis cs:205.794,    +69.371160) --  (axis cs:206.765,    +69.376386) --  (axis cs:207.736,    +69.381797) --  (axis cs:208.708,    +69.387392) --  (axis cs:209.679,    +69.393169) --  (axis cs:210.651,    +69.399126) ;
  \draw[constellation-boundary]  (axis cs:210.821,    +65.399654) --  (axis cs:210.783,    +66.399538) --  (axis cs:210.743,    +67.399412) --  (axis cs:210.699,    +68.399275) --  (axis cs:210.651,    +69.399126) ;
  \draw[constellation-boundary]  (axis cs:210.821,    +65.399654) --  (axis cs:211.798,    +65.405821) --  (axis cs:212.775,    +65.412167) --  (axis cs:213.752,    +65.418687) --  (axis cs:214.73,    +65.425382) --  (axis cs:215.708,    +65.432249) --  (axis cs:216.686,    +65.439286) --  (axis cs:217.665,    +65.446491) --  (axis cs:218.644,    +65.453862) --  (axis cs:219.623,    +65.461397) --  (axis cs:220.602,    +65.469094) --  (axis cs:221.582,    +65.476950) --  (axis cs:222.562,    +65.484964) --  (axis cs:223.542,    +65.493132) --  (axis cs:224.522,    +65.501453) --  (axis cs:225.503,    +65.509925) --  (axis cs:226.484,    +65.518544) --  (axis cs:227.466,    +65.527309) --  (axis cs:228.448,    +65.536217) --  (axis cs:229.43,    +65.545264) --  (axis cs:230.412,    +65.554450) --  (axis cs:231.395,    +65.563770) --  (axis cs:232.378,    +65.573223) --  (axis cs:233.361,    +65.582805) --  (axis cs:234.345,    +65.592514) --  (axis cs:235.33,    +65.602348) ;
  \draw[constellation-boundary]  (axis cs:235.33,    +65.602348) --  (axis cs:235.268,    +66.602040) --  (axis cs:235.202,    +67.601706) --  (axis cs:235.13,    +68.601343) --  (axis cs:235.051,    +69.600946) ;
  \draw[constellation-boundary]  (axis cs:235.051,    +69.600946) --  (axis cs:236.032,    +69.610866) --  (axis cs:237.013,    +69.620905) --  (axis cs:237.995,    +69.631059) --  (axis cs:238.978,    +69.641326) --  (axis cs:239.96,    +69.651703) --  (axis cs:240.944,    +69.662187) --  (axis cs:241.928,    +69.672775) --  (axis cs:242.912,    +69.683463) --  (axis cs:243.897,    +69.694250) --  (axis cs:244.882,    +69.705130) --  (axis cs:245.868,    +69.716102) --  (axis cs:246.854,    +69.727163) --  (axis cs:247.841,    +69.738308) ;
  \draw[constellation-boundary]  (axis cs:247.841,    +69.738308) --  (axis cs:247.742,    +70.737746) --  (axis cs:247.632,    +71.737125) --  (axis cs:247.511,    +72.736435) --  (axis cs:247.374,    +73.735663) --  (axis cs:247.221,    +74.734793) ;
  \draw[constellation-boundary]  (axis cs:247.221,    +74.734793) --  (axis cs:248.203,    +74.745969) --  (axis cs:249.186,    +74.757224) --  (axis cs:250.17,    +74.768557) --  (axis cs:251.155,    +74.779963) --  (axis cs:252.14,    +74.791439) --  (axis cs:253.126,    +74.802982) --  (axis cs:254.113,    +74.814589) --  (axis cs:255.1,    +74.826255) --  (axis cs:256.088,    +74.837979) --  (axis cs:257.077,    +74.849755) --  (axis cs:258.067,    +74.861582) --  (axis cs:259.057,    +74.873454) --  (axis cs:260.048,    +74.885370) --  (axis cs:261.04,    +74.897324) --  (axis cs:261.537,    +74.903315) ;
  \draw[constellation-boundary]  (axis cs:261.537,    +74.903315) --  (axis cs:261.346,    +75.902167) --  (axis cs:261.128,    +76.900848) --  (axis cs:260.874,    +77.899314) --  (axis cs:260.575,    +78.897510) --  (axis cs:260.218,    +79.895353) ;
  \draw[constellation-boundary]  (axis cs:260.218,    +79.895353) --  (axis cs:260.712,    +79.901329) --  (axis cs:261.7,    +79.913307) --  (axis cs:262.69,    +79.925319) --  (axis cs:263.681,    +79.937361) --  (axis cs:264.673,    +79.949429) --  (axis cs:265.666,    +79.961519) --  (axis cs:266.66,    +79.973628) --  (axis cs:267.656,    +79.985752) ;
  \draw[constellation-boundary]  (axis cs:267.656,    +79.985752) --  (axis cs:267.21,    +80.983041) --  (axis cs:266.655,    +81.979661) --  (axis cs:265.943,    +82.975328) --  (axis cs:264.997,    +83.969568) --  (axis cs:263.681,    +84.961539) --  (axis cs:261.722,    +85.949570) ;
  \draw[constellation-boundary]  (axis cs:261.722,    +85.949570) --  (axis cs:262.697,    +85.961583) --  (axis cs:263.674,    +85.973624) --  (axis cs:264.654,    +85.985691) --  (axis cs:265.636,    +85.997781) --  (axis cs:266.622,    +86.009889) --  (axis cs:267.611,    +86.022013) --  (axis cs:268.603,    +86.034148) --  (axis cs:269.597,    +86.046292) --  (axis cs:270.595,    +86.058439) --  (axis cs:271.596,    +86.070588) --  (axis cs:272.6,    +86.082734) --  (axis cs:273.607,    +86.094873) --  (axis cs:274.617,    +86.107001) --  (axis cs:275.631,    +86.119115) --  (axis cs:276.647,    +86.131211) --  (axis cs:277.667,    +86.143285) --  (axis cs:278.69,    +86.155333) --  (axis cs:279.717,    +86.167352) --  (axis cs:280.746,    +86.179336) --  (axis cs:281.779,    +86.191283) --  (axis cs:282.816,    +86.203188) --  (axis cs:283.855,    +86.215047) --  (axis cs:284.898,    +86.226856) --  (axis cs:285.945,    +86.238612) --  (axis cs:286.995,    +86.250309) --  (axis cs:288.048,    +86.261944) --  (axis cs:289.105,    +86.273513) --  (axis cs:290.165,    +86.285011) --  (axis cs:291.229,    +86.296434) --  (axis cs:292.296,    +86.307779) --  (axis cs:293.367,    +86.319041) --  (axis cs:294.441,    +86.330215) --  (axis cs:295.518,    +86.341298) --  (axis cs:296.6,    +86.352285) --  (axis cs:297.684,    +86.363172) --  (axis cs:298.772,    +86.373956) --  (axis cs:299.864,    +86.384630) --  (axis cs:300.959,    +86.395192) --  (axis cs:302.057,    +86.405638) --  (axis cs:303.159,    +86.415962) --  (axis cs:304.265,    +86.426161) --  (axis cs:305.374,    +86.436230) --  (axis cs:306.486,    +86.446166) --  (axis cs:307.602,    +86.455964) --  (axis cs:308.721,    +86.465621) ;
  \draw[constellation-boundary]  (axis cs:274.342,    +47.547602) --  (axis cs:274.329,    +48.047520) --  (axis cs:274.301,    +49.047353) --  (axis cs:274.273,    +50.047179) --  (axis cs:274.258,    +50.547089) ;
  \draw[constellation-boundary]  (axis cs:255.786,    +50.324447) --  (axis cs:256.783,    +50.336208) --  (axis cs:257.779,    +50.348020) --  (axis cs:258.776,    +50.359879) --  (axis cs:259.773,    +50.371783) --  (axis cs:260.77,    +50.383727) --  (axis cs:261.768,    +50.395708) --  (axis cs:262.765,    +50.407722) --  (axis cs:263.763,    +50.419766) --  (axis cs:264.762,    +50.431836) --  (axis cs:265.76,    +50.443928) --  (axis cs:266.759,    +50.456038) --  (axis cs:267.758,    +50.468164) --  (axis cs:268.757,    +50.480301) --  (axis cs:269.757,    +50.492445) --  (axis cs:270.756,    +50.504593) --  (axis cs:271.756,    +50.516741) --  (axis cs:272.757,    +50.528886) --  (axis cs:273.757,    +50.541024) --  (axis cs:274.258,    +50.547089) ;
  \draw[constellation-boundary]  (axis cs:255.786,    +50.324447) --  (axis cs:255.772,    +50.824361) --  (axis cs:255.757,    +51.324273) ;
  \draw[constellation-boundary]  (axis cs:203.574,    +62.359406) --  (axis cs:204.063,    +62.361782) --  (axis cs:205.042,    +62.366675) --  (axis cs:206.02,    +62.371757) --  (axis cs:206.999,    +62.377027) --  (axis cs:207.979,    +62.382484) --  (axis cs:208.958,    +62.388125) --  (axis cs:209.938,    +62.393949) --  (axis cs:210.917,    +62.399955) --  (axis cs:211.897,    +62.406140) --  (axis cs:212.877,    +62.412504) --  (axis cs:213.858,    +62.419043) --  (axis cs:214.838,    +62.425757) --  (axis cs:215.819,    +62.432643) --  (axis cs:216.8,    +62.439699) --  (axis cs:217.045,    +62.441489) ;
  \draw[constellation-boundary]  (axis cs:203.574,    +62.359406) --  (axis cs:203.551,    +63.359351) ;
  \draw[constellation-boundary]  (axis cs:181.582,    +63.303970) --  (axis cs:182.557,    +63.304239) --  (axis cs:183.533,    +63.304715) --  (axis cs:184.509,    +63.305398) --  (axis cs:185.485,    +63.306286) --  (axis cs:186.461,    +63.307382) --  (axis cs:187.436,    +63.308683) --  (axis cs:188.412,    +63.310189) --  (axis cs:189.388,    +63.311901) --  (axis cs:190.365,    +63.313817) --  (axis cs:191.341,    +63.315938) --  (axis cs:192.317,    +63.318262) --  (axis cs:193.293,    +63.320789) --  (axis cs:194.269,    +63.323517) --  (axis cs:195.246,    +63.326448) --  (axis cs:196.222,    +63.329578) --  (axis cs:197.199,    +63.332909) --  (axis cs:198.176,    +63.336438) --  (axis cs:199.153,    +63.340164) --  (axis cs:200.13,    +63.344088) --  (axis cs:201.107,    +63.348206) --  (axis cs:202.084,    +63.352520) --  (axis cs:203.062,    +63.357026) --  (axis cs:203.551,    +63.359351) ;
  \draw[constellation-boundary]  (axis cs:181.582,    +63.303970) --  (axis cs:181.581,    +64.303970) --  (axis cs:181.58,    +65.303970) --  (axis cs:181.579,    +65.803970) ;
  \draw[constellation-boundary]  (axis cs:180.606,    +65.803908) --  (axis cs:181.579,    +65.803970) ;
  \draw[constellation-boundary]  (axis cs:241.806,    +25.664146) ;
  \draw[constellation-boundary]  (axis cs:273.467,    +47.536987) --  (axis cs:273.842,    +47.541537) --  (axis cs:274.843,    +47.553663) --  (axis cs:275.844,    +47.565773) --  (axis cs:276.845,    +47.577865) --  (axis cs:277.846,    +47.589935) --  (axis cs:278.848,    +47.601978) --  (axis cs:279.85,    +47.613992) --  (axis cs:280.852,    +47.625972) --  (axis cs:281.855,    +47.637916) --  (axis cs:282.857,    +47.649818) --  (axis cs:283.86,    +47.661676) --  (axis cs:284.863,    +47.673486) --  (axis cs:285.866,    +47.685245) --  (axis cs:286.87,    +47.696948) --  (axis cs:287.874,    +47.708592) --  (axis cs:288.375,    +47.714391) ;
  \draw[constellation-boundary]  (axis cs:273.824,    +30.039155) --  (axis cs:273.808,    +31.039056) --  (axis cs:273.791,    +32.038955) --  (axis cs:273.774,    +33.038851) --  (axis cs:273.757,    +34.038745) --  (axis cs:273.739,    +35.038637) --  (axis cs:273.721,    +36.038526) --  (axis cs:273.702,    +37.038412) --  (axis cs:273.683,    +38.038295) --  (axis cs:273.663,    +39.038175) --  (axis cs:273.642,    +40.038051) --  (axis cs:273.621,    +41.037924) --  (axis cs:273.6,    +42.037793) --  (axis cs:273.577,    +43.037657) --  (axis cs:273.554,    +44.037517) --  (axis cs:273.53,    +45.037372) --  (axis cs:273.506,    +46.037223) --  (axis cs:273.48,    +47.037067) --  (axis cs:273.467,    +47.536987) ;
  \draw[constellation-boundary]  (axis cs:273.824,    +30.039155) --  (axis cs:274.199,    +30.043704) --  (axis cs:275.2,    +30.055824) --  (axis cs:276.2,    +30.067929) --  (axis cs:276.701,    +30.073974) ;
  \draw[constellation-boundary]  (axis cs:276.763,    +26.074349) --  (axis cs:276.748,    +27.074257) --  (axis cs:276.732,    +28.074165) --  (axis cs:276.717,    +29.074070) --  (axis cs:276.701,    +30.073974) ;
  \draw[constellation-boundary]  (axis cs:276.763,    +26.074349) --  (axis cs:277.263,    +26.080387) --  (axis cs:278.264,    +26.092446) --  (axis cs:279.264,    +26.104478) --  (axis cs:280.265,    +26.116478) --  (axis cs:281.266,    +26.128444) --  (axis cs:282.267,    +26.140371) --  (axis cs:283.269,    +26.152255) --  (axis cs:284.27,    +26.164094) ;
  \draw[constellation-boundary]  (axis cs:284.277,    +25.664137) --  (axis cs:285.278,    +25.675926) --  (axis cs:286.28,    +25.687662) --  (axis cs:287.281,    +25.699341) --  (axis cs:288.283,    +25.710960) --  (axis cs:289.285,    +25.722516) --  (axis cs:290.161,    +25.732571) ;
  \draw[constellation-boundary]  (axis cs:343.709,    +35.165608) --  (axis cs:344.465,    +35.168228) --  (axis cs:345.473,    +35.171543) --  (axis cs:346.482,    +35.174651) --  (axis cs:347.49,    +35.177552) --  (axis cs:348.498,    +35.180244) --  (axis cs:349.506,    +35.182728) --  (axis cs:350.515,    +35.185003) --  (axis cs:351.523,    +35.187067) --  (axis cs:352.532,    +35.188920) --  (axis cs:353.54,    +35.190561) --  (axis cs:354.044,    +35.191302) ;
  \draw[constellation-boundary]  (axis cs:343.709,    +35.165608) --  (axis cs:343.706,    +35.665603) ;
  \draw[constellation-boundary]  (axis cs:331.36,    +35.606926) --  (axis cs:332.367,    +35.612804) --  (axis cs:333.375,    +35.618494) --  (axis cs:334.382,    +35.623995) --  (axis cs:335.39,    +35.629304) --  (axis cs:336.398,    +35.634421) --  (axis cs:337.406,    +35.639342) --  (axis cs:338.414,    +35.644068) --  (axis cs:339.422,    +35.648596) --  (axis cs:340.43,    +35.652925) --  (axis cs:341.438,    +35.657053) --  (axis cs:342.446,    +35.660980) --  (axis cs:343.454,    +35.664704) --  (axis cs:343.706,    +35.665603) ;
  \draw[constellation-boundary]  (axis cs:331.36,    +35.606926) --  (axis cs:331.35,    +36.606899) ;
  \draw[constellation-boundary]  (axis cs:180.602,    +28.303908) --  (axis cs:181.596,    +28.303971) ;
  \draw[constellation-boundary]  (axis cs:227.606,    +25.524616) ;
  \draw[constellation-boundary]  (axis cs:290.161,    +25.732571) --  (axis cs:290.154,    +26.232531) --  (axis cs:290.14,    +27.232449) --  (axis cs:290.125,    +28.232365) --  (axis cs:290.11,    +29.232280) --  (axis cs:290.095,    +30.232193) ;
  \draw[constellation-boundary]  (axis cs:322.648,    +25.548153) --  (axis cs:322.639,    +26.548118) --  (axis cs:322.63,    +27.548083) --  (axis cs:322.62,    +28.548047) ;
  \draw[constellation-boundary]  (axis cs:-4.729,    +50.692903) ;
  \draw[constellation-boundary]  (axis cs:-4.724,    +48.692907) --  (axis cs:-4.727,    +49.692905) --  (axis cs:-4.729,    +50.692903) ;
  \draw[constellation-boundary]  (axis cs:-4.724,    +48.692907) --  (axis cs:-4.47,    +48.693192) --  (axis cs:-3.457,    +48.694201) --  (axis cs:-2.443,    +48.694996) --  (axis cs:-1.429,    +48.695576) --  (axis cs:-0.416,    +48.695941) --  (axis cs:0.59819,    +48.696092) --  (axis cs:1.61194,    +48.696027) --  (axis cs:2.62569,    +48.695748) --  (axis cs:3.63943,    +48.695253) --  (axis cs:4.1463,    +48.694926) ;
  \draw[constellation-boundary]  (axis cs:4.14321,    +46.694927) --  (axis cs:4.14472,    +47.694926) --  (axis cs:4.1463,    +48.694926) ;
  \draw[constellation-boundary]  (axis cs:4.14321,    +46.694927) --  (axis cs:4.64961,    +46.694546) --  (axis cs:5.66238,    +46.693623) --  (axis cs:6.67514,    +46.692487) --  (axis cs:7.68788,    +46.691136) --  (axis cs:8.70058,    +46.689573) --  (axis cs:9.71326,    +46.687797) --  (axis cs:10.7259,    +46.685809) --  (axis cs:11.7385,    +46.683609) --  (axis cs:12.751,    +46.681199) --  (axis cs:13.7636,    +46.678578) --  (axis cs:14.776,    +46.675749) ;
  \draw[constellation-boundary]  (axis cs:14.776,    +46.675749) --  (axis cs:14.7823,    +47.675740) --  (axis cs:14.7888,    +48.675730) ;
  \draw[constellation-boundary]  (axis cs:14.7888,    +48.675730) --  (axis cs:15.8021,    +48.672689) --  (axis cs:16.8153,    +48.669441) --  (axis cs:17.8285,    +48.665986) --  (axis cs:18.5883,    +48.663260) ;
  \draw[constellation-boundary]  (axis cs:18.5883,    +48.663260) --  (axis cs:18.5969,    +49.663245) --  (axis cs:18.6058,    +50.663228) ;
  \draw[constellation-boundary]  (axis cs:18.6058,    +50.663228) --  (axis cs:18.8593,    +50.662293) --  (axis cs:19.8733,    +50.658425) --  (axis cs:20.8872,    +50.654354) --  (axis cs:21.901,    +50.650081) --  (axis cs:22.9147,    +50.645607) --  (axis cs:23.9283,    +50.640934) --  (axis cs:24.9418,    +50.636064) --  (axis cs:25.9552,    +50.630998) --  (axis cs:26.9685,    +50.625737) ;
  \draw[constellation-boundary]  (axis cs:22.4079,    +50.647869) --  (axis cs:22.4191,    +51.647844) --  (axis cs:22.4308,    +52.647818) --  (axis cs:22.4431,    +53.647791) --  (axis cs:22.4559,    +54.647763) ;
  \draw[constellation-boundary]  (axis cs:26.9314,    +47.625835) --  (axis cs:26.9432,    +48.625803) --  (axis cs:26.9556,    +49.625771) --  (axis cs:26.9685,    +50.625737) ;
  \draw[constellation-boundary]  (axis cs:26.9314,    +47.625835) --  (axis cs:27.9432,    +47.620388) --  (axis cs:28.9549,    +47.614750) --  (axis cs:29.9665,    +47.608924) --  (axis cs:30.978,    +47.602910) --  (axis cs:31.9894,    +47.596710) --  (axis cs:32.6214,    +47.592743) ;
  \draw[constellation-boundary]  (axis cs:32.6214,    +47.592743) --  (axis cs:32.6356,    +48.592698) --  (axis cs:32.6504,    +49.592651) --  (axis cs:32.6658,    +50.592602) --  (axis cs:32.6737,    +51.092577) ;
  \draw[constellation-boundary]  (axis cs:32.6737,    +51.092577) --  (axis cs:33.0535,    +51.090158) --  (axis cs:34.0661,    +51.083585) --  (axis cs:35.0785,    +51.076831) --  (axis cs:36.0908,    +51.069901) --  (axis cs:37.1029,    +51.062795) --  (axis cs:38.1149,    +51.055516) --  (axis cs:39.1267,    +51.048067) --  (axis cs:39.8854,    +51.042369) ;
  \draw[constellation-boundary]  (axis cs:39.6793,    +37.293149) --  (axis cs:39.6823,    +37.543138) --  (axis cs:39.6945,    +38.543092) --  (axis cs:39.707,    +39.543044) --  (axis cs:39.7199,    +40.542995) --  (axis cs:39.7333,    +41.542945) --  (axis cs:39.747,    +42.542893) --  (axis cs:39.7612,    +43.542839) --  (axis cs:39.7758,    +44.542784) --  (axis cs:39.791,    +45.542727) --  (axis cs:39.8067,    +46.542667) --  (axis cs:39.823,    +47.542605) --  (axis cs:39.8399,    +48.542541) --  (axis cs:39.8576,    +49.542474) --  (axis cs:39.8759,    +50.542405) --  (axis cs:39.8854,    +51.042369) ;
  \draw[constellation-boundary]  (axis cs:31.8711,    +37.347078) --  (axis cs:32.8789,    +37.340717) --  (axis cs:33.8866,    +37.334175) --  (axis cs:34.8943,    +37.327454) --  (axis cs:35.9018,    +37.320556) --  (axis cs:36.9093,    +37.313483) --  (axis cs:37.9166,    +37.306237) --  (axis cs:38.9239,    +37.298822) --  (axis cs:39.931,    +37.291238) --  (axis cs:40.4346,    +37.287384) ;
  \draw[constellation-boundary]  (axis cs:31.8542,    +35.597131) --  (axis cs:31.8638,    +36.597101) --  (axis cs:31.8711,    +37.347078) ;
  \draw[constellation-boundary]  (axis cs:22.9109,    +35.645330) --  (axis cs:23.7928,    +35.641253) --  (axis cs:24.8008,    +35.636410) --  (axis cs:25.8086,    +35.631371) --  (axis cs:26.8164,    +35.626139) --  (axis cs:27.8241,    +35.620714) --  (axis cs:28.8317,    +35.615099) --  (axis cs:29.8393,    +35.609296) --  (axis cs:30.8468,    +35.603306) --  (axis cs:31.8542,    +35.597131) ;
  \draw[constellation-boundary]  (axis cs:12.443,    +33.681888) --  (axis cs:12.695,    +33.681269) --  (axis cs:13.7028,    +33.678660) --  (axis cs:14.7106,    +33.675843) --  (axis cs:15.7184,    +33.672819) --  (axis cs:16.7261,    +33.669589) --  (axis cs:17.7338,    +33.666153) --  (axis cs:18.7414,    +33.662512) --  (axis cs:19.749,    +33.658669) --  (axis cs:20.7566,    +33.654623) --  (axis cs:21.7641,    +33.650376) --  (axis cs:22.7715,    +33.645930) --  (axis cs:22.8975,    +33.645360) ;
  \draw[constellation-boundary]  (axis cs:12.4171,    +25.681919) --  (axis cs:12.4201,    +26.681916) --  (axis cs:12.4232,    +27.681912) --  (axis cs:12.4264,    +28.681908) --  (axis cs:12.4295,    +29.681904) --  (axis cs:12.4328,    +30.681900) --  (axis cs:12.4361,    +31.681896) --  (axis cs:12.4395,    +32.681892) --  (axis cs:12.443,    +33.681888) ;
  \draw[constellation-boundary]  (axis cs:2.61133,    +25.695750) --  (axis cs:2.6118,    +26.695750) --  (axis cs:2.61228,    +27.695750) --  (axis cs:2.61277,    +28.695750) ;
  \draw[constellation-boundary]  (axis cs:1.6062,    +28.696028) --  (axis cs:2.61277,    +28.695750) ;
  \draw[constellation-boundary]  (axis cs:1.6062,    +28.696028) --  (axis cs:1.60642,    +29.696028) --  (axis cs:1.60665,    +30.696028) --  (axis cs:1.60688,    +31.696028) --  (axis cs:1.60696,    +32.029361) ;
  \draw[constellation-boundary]  (axis cs:-2.171,    +32.028500) --  (axis cs:-1.416,    +32.028913) --  (axis cs:-0.408,    +32.029276) --  (axis cs:0.599433,    +32.029425) --  (axis cs:1.60696,    +32.029361) ;
  \draw[constellation-boundary]  (axis cs:-2.171,    +32.028500) --  (axis cs:-2.172,    +32.695167) --  (axis cs:-2.172,    +32.778500) ;
  \draw[constellation-boundary]  (axis cs:-4.439,    +32.776543) --  (axis cs:-3.432,    +32.777546) --  (axis cs:-2.424,    +32.778336) --  (axis cs:-2.172,    +32.778500) ;
  \draw[constellation-boundary]  (axis cs:26.7446,    +25.626328) --  (axis cs:26.7512,    +26.626311) --  (axis cs:26.7578,    +27.626293) --  (axis cs:26.7646,    +28.626275) ;
  \draw[constellation-boundary]  (axis cs:26.7446,    +25.626328) --  (axis cs:27.7497,    +25.620918) --  (axis cs:28.7548,    +25.615317) --  (axis cs:29.7599,    +25.609528) --  (axis cs:30.5136,    +25.605064) ;
  \draw[constellation-boundary]  (axis cs:30.5136,    +25.605064) --  (axis cs:30.5211,    +26.605042) --  (axis cs:30.5286,    +27.605019) --  (axis cs:30.5305,    +27.855014) ;
  \draw[constellation-boundary]  (axis cs:30.5305,    +27.855014) --  (axis cs:30.7819,    +27.853502) --  (axis cs:31.7874,    +27.847339) --  (axis cs:32.7928,    +27.840993) --  (axis cs:33.7981,    +27.834466) --  (axis cs:34.8034,    +27.827761) --  (axis cs:35.8086,    +27.820879) --  (axis cs:36.8137,    +27.813822) --  (axis cs:37.8188,    +27.806593) --  (axis cs:38.0701,    +27.804759) ;
  \draw[constellation-boundary]  (axis cs:38.0701,    +27.804759) --  (axis cs:38.0771,    +28.554734) --  (axis cs:38.0867,    +29.554699) --  (axis cs:38.0965,    +30.554663) --  (axis cs:38.1031,    +31.221305) ;
  \draw[constellation-boundary]  (axis cs:38.1031,    +31.221305) --  (axis cs:38.8574,    +31.215736) --  (axis cs:39.8631,    +31.208163) --  (axis cs:40.8687,    +31.200425) --  (axis cs:41.8742,    +31.192524) --  (axis cs:42.8797,    +31.184461) --  (axis cs:43.885,    +31.176241) --  (axis cs:44.8902,    +31.167865) --  (axis cs:45.8954,    +31.159336) --  (axis cs:46.9005,    +31.150657) --  (axis cs:47.9054,    +31.141829) --  (axis cs:48.9103,    +31.132857) --  (axis cs:49.9151,    +31.123742) --  (axis cs:50.9198,    +31.114488) --  (axis cs:51.9243,    +31.105098) --  (axis cs:52.9288,    +31.095573) --  (axis cs:53.9332,    +31.085918) --  (axis cs:54.9375,    +31.076136) --  (axis cs:55.9416,    +31.066228) --  (axis cs:56.9457,    +31.056199) --  (axis cs:57.9496,    +31.046051) --  (axis cs:58.9535,    +31.035788) --  (axis cs:59.9572,    +31.025412) --  (axis cs:60.9608,    +31.014928) --  (axis cs:61.9643,    +31.004337) --  (axis cs:62.9677,    +30.993644) --  (axis cs:63.971,    +30.982851) --  (axis cs:64.9741,    +30.971963) --  (axis cs:65.9772,    +30.960982) --  (axis cs:66.9801,    +30.949911) --  (axis cs:67.9829,    +30.938755) --  (axis cs:68.9856,    +30.927516) --  (axis cs:69.4869,    +30.921867) ;
  \draw[constellation-boundary]  (axis cs:42.6283,    +31.186492) --  (axis cs:42.632,    +31.519810) --  (axis cs:42.6432,    +32.519765) --  (axis cs:42.6547,    +33.519719) --  (axis cs:42.6664,    +34.519672) ;
  \draw[constellation-boundary]  (axis cs:52.3572,    +25.434014) --  (axis cs:52.369,    +26.433959) --  (axis cs:52.3809,    +27.433902) --  (axis cs:52.393,    +28.433845) --  (axis cs:52.4054,    +29.433786) --  (axis cs:52.418,    +30.433726) --  (axis cs:52.4266,    +31.100352) ;
  \draw[constellation-boundary]  (axis cs:69.5738,    +36.254710) --  (axis cs:70.0754,    +36.249038) --  (axis cs:71.0784,    +36.237638) --  (axis cs:72.0812,    +36.226166) --  (axis cs:72.4573,    +36.221847) ;
  \draw[constellation-boundary]  (axis cs:72.4573,    +36.221847) --  (axis cs:72.4752,    +37.221744) --  (axis cs:72.4935,    +38.221638) --  (axis cs:72.5124,    +39.221530) --  (axis cs:72.5319,    +40.221418) --  (axis cs:72.5519,    +41.221303) --  (axis cs:72.5725,    +42.221184) --  (axis cs:72.5939,    +43.221062) --  (axis cs:72.6159,    +44.220935) --  (axis cs:72.6386,    +45.220804) --  (axis cs:72.6622,    +46.220668) --  (axis cs:72.6867,    +47.220528) --  (axis cs:72.7121,    +48.220382) --  (axis cs:72.7385,    +49.220230) --  (axis cs:72.7661,    +50.220071) --  (axis cs:72.7947,    +51.219906) --  (axis cs:72.8247,    +52.219734) --  (axis cs:72.8402,    +52.719645) ;
  \draw[constellation-boundary]  (axis cs:77.4847,    +52.665551) --  (axis cs:77.5009,    +53.165456) --  (axis cs:77.5345,    +54.165258) --  (axis cs:77.5697,    +55.165051) --  (axis cs:77.6067,    +56.164833) ;
  \draw[constellation-boundary]  (axis cs:77.6067,    +56.164833) --  (axis cs:78.6106,    +56.152981) --  (axis cs:79.6142,    +56.141084) --  (axis cs:80.6175,    +56.129146) --  (axis cs:81.6205,    +56.117170) --  (axis cs:82.6231,    +56.105161) --  (axis cs:83.6255,    +56.093121) --  (axis cs:84.6275,    +56.081054) --  (axis cs:85.6292,    +56.068965) --  (axis cs:86.6306,    +56.056857) --  (axis cs:87.6317,    +56.044733) --  (axis cs:88.6325,    +56.032597) --  (axis cs:89.6329,    +56.020454) --  (axis cs:90.6331,    +56.008306) --  (axis cs:91.6329,    +55.996157) --  (axis cs:92.6324,    +55.984012) --  (axis cs:93.6316,    +55.971873) --  (axis cs:94.1311,    +55.965808) ;
  \draw[constellation-boundary]  (axis cs:94.0573,    +53.966255) --  (axis cs:94.5568,    +53.960192) --  (axis cs:95.5555,    +53.948077) --  (axis cs:96.5539,    +53.935979) --  (axis cs:97.552,    +53.923903) --  (axis cs:98.5498,    +53.911851) --  (axis cs:99.5473,    +53.899828) --  (axis cs:100.046,    +53.893829) ;
  \draw[constellation-boundary]  (axis cs:94.0573,    +53.966255) --  (axis cs:94.0933,    +54.966037) --  (axis cs:94.1311,    +55.965808) --  (axis cs:94.1708,    +56.965566) --  (axis cs:94.2128,    +57.965311) --  (axis cs:94.2572,    +58.965042) --  (axis cs:94.3042,    +59.964756) --  (axis cs:94.3542,    +60.964453) --  (axis cs:94.4074,    +61.964130) ;
  \draw[constellation-boundary]  (axis cs:99.9194,    +49.894589) --  (axis cs:99.949,    +50.894412) --  (axis cs:99.9799,    +51.894226) --  (axis cs:100.012,    +52.894032) --  (axis cs:100.046,    +53.893829) ;
  \draw[constellation-boundary]  (axis cs:99.9194,    +49.894589) --  (axis cs:100.418,    +49.888596) --  (axis cs:101.416,    +49.876636) --  (axis cs:102.413,    +49.864715) --  (axis cs:103.41,    +49.852837) --  (axis cs:104.406,    +49.841005) ;
  \draw[constellation-boundary]  (axis cs:104.265,    +44.341840) --  (axis cs:104.277,    +44.841771) --  (axis cs:104.301,    +45.841629) --  (axis cs:104.326,    +46.841482) --  (axis cs:104.352,    +47.841329) --  (axis cs:104.378,    +48.841170) --  (axis cs:104.406,    +49.841005) ;
  \draw[constellation-boundary]  (axis cs:104.265,    +44.341840) --  (axis cs:105.262,    +44.330051) --  (axis cs:106.259,    +44.318314) --  (axis cs:107.256,    +44.306633) --  (axis cs:108.252,    +44.295012) --  (axis cs:109.249,    +44.283455) --  (axis cs:110.245,    +44.271964) --  (axis cs:111.241,    +44.260544) --  (axis cs:112.236,    +44.249198) --  (axis cs:112.734,    +44.243553) ;
  \draw[constellation-boundary]  (axis cs:112.561,    +35.244534) --  (axis cs:112.569,    +35.744486) --  (axis cs:112.587,    +36.744388) --  (axis cs:112.604,    +37.744286) --  (axis cs:112.623,    +38.744183) --  (axis cs:112.642,    +39.744076) --  (axis cs:112.661,    +40.743966) --  (axis cs:112.681,    +41.743853) --  (axis cs:112.702,    +42.743736) --  (axis cs:112.723,    +43.743615) --  (axis cs:112.734,    +44.243553) ;
  \draw[constellation-boundary]  (axis cs:100.09,    +35.390565) --  (axis cs:101.089,    +35.378593) --  (axis cs:102.087,    +35.366659) --  (axis cs:103.085,    +35.354766) --  (axis cs:104.083,    +35.342919) --  (axis cs:105.081,    +35.331120) --  (axis cs:106.079,    +35.319373) --  (axis cs:107.076,    +35.307682) --  (axis cs:108.074,    +35.296050) --  (axis cs:109.071,    +35.284481) --  (axis cs:110.068,    +35.272978) --  (axis cs:111.065,    +35.261545) --  (axis cs:112.062,    +35.250186) --  (axis cs:113.059,    +35.238903) --  (axis cs:114.056,    +35.227700) --  (axis cs:115.052,    +35.216580) --  (axis cs:116.048,    +35.205547) --  (axis cs:117.045,    +35.194605) --  (axis cs:118.041,    +35.183755) --  (axis cs:118.29,    +35.181058) ;
  \draw[constellation-boundary]  (axis cs:99.9657,    +27.891313) --  (axis cs:99.9812,    +28.891220) --  (axis cs:99.997,    +29.891125) --  (axis cs:100.013,    +30.891028) --  (axis cs:100.03,    +31.890929) --  (axis cs:100.046,    +32.890828) --  (axis cs:100.064,    +33.890724) --  (axis cs:100.081,    +34.890618) --  (axis cs:100.09,    +35.390565) ;
  \draw[constellation-boundary]  (axis cs:90.2211,    +28.009289) --  (axis cs:90.9711,    +28.000178) --  (axis cs:91.9709,    +27.988030) --  (axis cs:92.9707,    +27.975886) --  (axis cs:93.9703,    +27.963751) --  (axis cs:94.9698,    +27.951627) --  (axis cs:95.9692,    +27.939519) --  (axis cs:96.9685,    +27.927429) --  (axis cs:97.9677,    +27.915363) --  (axis cs:98.9667,    +27.903323) --  (axis cs:99.9657,    +27.891313) ;
  \draw[constellation-boundary]  (axis cs:73.2125,    +28.712437) --  (axis cs:73.9639,    +28.703734) --  (axis cs:74.9657,    +28.692076) --  (axis cs:75.9674,    +28.680359) --  (axis cs:76.9689,    +28.668589) --  (axis cs:77.9704,    +28.656768) --  (axis cs:78.9717,    +28.644899) --  (axis cs:79.973,    +28.632987) --  (axis cs:80.9741,    +28.621035) --  (axis cs:81.9751,    +28.609047) --  (axis cs:82.976,    +28.597026) --  (axis cs:83.9767,    +28.584977) --  (axis cs:84.9774,    +28.572902) --  (axis cs:85.9779,    +28.560806) --  (axis cs:86.9783,    +28.548691) --  (axis cs:87.9787,    +28.536563) --  (axis cs:88.9788,    +28.524424) --  (axis cs:89.9789,    +28.512279) --  (axis cs:90.2289,    +28.509242) ;
  \draw[constellation-boundary]  (axis cs:73.2125,    +28.712437) --  (axis cs:73.22,    +29.212394) --  (axis cs:73.2353,    +30.212305) ;
  \draw[constellation-boundary]  (axis cs:69.4768,    +30.255257) --  (axis cs:69.492,    +31.255171) --  (axis cs:69.5077,    +32.255083) --  (axis cs:69.5236,    +33.254993) --  (axis cs:69.54,    +34.254901) --  (axis cs:69.5567,    +35.254806) --  (axis cs:69.5738,    +36.254710) ;
  \draw[constellation-boundary]  (axis cs:69.4768,    +30.255257) --  (axis cs:69.978,    +30.249589) --  (axis cs:70.9804,    +30.238197) --  (axis cs:71.9827,    +30.226732) --  (axis cs:72.9848,    +30.215199) --  (axis cs:73.2353,    +30.212305) ;
  \draw[constellation-boundary]  (axis cs:52.3131,    +52.936601) --  (axis cs:53.3229,    +52.927025) --  (axis cs:54.3325,    +52.917320) --  (axis cs:55.3419,    +52.907486) --  (axis cs:56.351,    +52.897529) --  (axis cs:57.3599,    +52.887451) --  (axis cs:58.3686,    +52.877255) --  (axis cs:59.377,    +52.866944) --  (axis cs:60.3852,    +52.856522) --  (axis cs:61.3931,    +52.845991) --  (axis cs:62.4007,    +52.835356) --  (axis cs:63.4082,    +52.824619) --  (axis cs:64.4153,    +52.813784) --  (axis cs:65.4223,    +52.802854) --  (axis cs:66.4289,    +52.791832) --  (axis cs:67.4353,    +52.780722) --  (axis cs:68.4415,    +52.769528) --  (axis cs:69.4473,    +52.758252) --  (axis cs:70.4529,    +52.746899) --  (axis cs:71.4583,    +52.735472) --  (axis cs:72.4634,    +52.723974) --  (axis cs:73.4682,    +52.712409) --  (axis cs:74.4727,    +52.700781) --  (axis cs:75.477,    +52.689093) --  (axis cs:76.481,    +52.677348) --  (axis cs:77.4847,    +52.665551) ;
  \draw[constellation-boundary]  (axis cs:52.3131,    +52.936601) --  (axis cs:52.3262,    +53.436539) --  (axis cs:52.3534,    +54.436411) --  (axis cs:52.3819,    +55.436277) ;
  \draw[constellation-boundary]  (axis cs:49.8539,    +55.459649) --  (axis cs:50.3597,    +55.455044) --  (axis cs:51.3709,    +55.445728) --  (axis cs:52.3819,    +55.436277) ;
  \draw[constellation-boundary]  (axis cs:49.8539,    +55.459649) --  (axis cs:49.8829,    +56.459518) --  (axis cs:49.9134,    +57.459380) ;
  \draw[constellation-boundary]  (axis cs:48.9009,    +57.468493) --  (axis cs:48.9326,    +58.468352) --  (axis cs:48.9663,    +59.468202) --  (axis cs:49.0019,    +60.468042) --  (axis cs:49.0398,    +61.467873) --  (axis cs:49.0803,    +62.467693) --  (axis cs:49.1235,    +63.467500) --  (axis cs:49.1699,    +64.467293) --  (axis cs:49.2198,    +65.467070) --  (axis cs:49.2736,    +66.466829) --  (axis cs:49.332,    +67.466569) --  (axis cs:49.3954,    +68.466285) ;
  \draw[constellation-boundary]  (axis cs:49.3954,    +68.466285) --  (axis cs:49.9055,    +68.461712) --  (axis cs:50.9254,    +68.452458) --  (axis cs:51.9448,    +68.443065) --  (axis cs:52.9638,    +68.433537) --  (axis cs:53.9824,    +68.423877) --  (axis cs:54.2369,    +68.421442) ;
  \draw[constellation-boundary]  (axis cs:54.2369,    +68.421442) --  (axis cs:54.3111,    +69.421087) --  (axis cs:54.3924,    +70.420697) --  (axis cs:54.4821,    +71.420267) --  (axis cs:54.5816,    +72.419790) --  (axis cs:54.6928,    +73.419258) --  (axis cs:54.8178,    +74.418659) --  (axis cs:54.9594,    +75.417980) --  (axis cs:55.1214,    +76.417204) --  (axis cs:55.3086,    +77.416307) ;
  \draw[constellation-boundary]  (axis cs:56.7262,    +77.402587) --  (axis cs:56.9487,    +78.401501) --  (axis cs:57.2125,    +79.400213) --  (axis cs:57.5305,    +80.398661) ;
  \draw[constellation-boundary]  (axis cs:57.5305,    +80.398661) --  (axis cs:57.9202,    +80.394843) --  (axis cs:58.9585,    +80.384582) --  (axis cs:59.9957,    +80.374204) --  (axis cs:61.0317,    +80.363713) --  (axis cs:62.0666,    +80.353113) --  (axis cs:63.1003,    +80.342408) --  (axis cs:64.1328,    +80.331602) --  (axis cs:65.1641,    +80.320697) --  (axis cs:66.1943,    +80.309697) --  (axis cs:67.2232,    +80.298607) --  (axis cs:68.251,    +80.287429) --  (axis cs:69.2775,    +80.276168) --  (axis cs:70.3029,    +80.264827) --  (axis cs:71.327,    +80.253410) --  (axis cs:72.3499,    +80.241920) --  (axis cs:73.3715,    +80.230362) --  (axis cs:74.392,    +80.218739) --  (axis cs:75.4112,    +80.207055) --  (axis cs:76.4292,    +80.195314) --  (axis cs:77.4459,    +80.183519) --  (axis cs:78.4615,    +80.171675) --  (axis cs:79.4758,    +80.159784) --  (axis cs:80.4888,    +80.147851) ;
  \draw[constellation-boundary]  (axis cs:80.4888,    +80.147851) --  (axis cs:80.936,    +81.145214) --  (axis cs:81.4956,    +82.141914) --  (axis cs:82.2165,    +83.137659) --  (axis cs:83.1807,    +84.131965) --  (axis cs:84.536,    +85.123951) ;
  \draw[constellation-boundary]  (axis cs:84.536,    +85.123951) --  (axis cs:85.5502,    +85.111864) --  (axis cs:86.5619,    +85.099757) --  (axis cs:87.571,    +85.087634) --  (axis cs:88.5777,    +85.075499) --  (axis cs:89.5818,    +85.063355) --  (axis cs:90.5835,    +85.051208) --  (axis cs:91.5827,    +85.039059) --  (axis cs:92.5795,    +85.026913) --  (axis cs:93.5739,    +85.014774) --  (axis cs:94.5659,    +85.002645) --  (axis cs:95.5555,    +84.990530) --  (axis cs:96.5428,    +84.978432) --  (axis cs:97.5277,    +84.966356) --  (axis cs:98.5103,    +84.954303) --  (axis cs:99.4907,    +84.942279) --  (axis cs:100.469,    +84.930285) --  (axis cs:101.445,    +84.918327) --  (axis cs:102.418,    +84.906407) --  (axis cs:103.389,    +84.894528) --  (axis cs:104.359,    +84.882694) --  (axis cs:105.326,    +84.870909) --  (axis cs:106.291,    +84.859174) --  (axis cs:107.253,    +84.847495) --  (axis cs:108.214,    +84.835873) --  (axis cs:109.173,    +84.824312) --  (axis cs:110.13,    +84.812814) --  (axis cs:111.084,    +84.801384) --  (axis cs:112.037,    +84.790024) --  (axis cs:112.987,    +84.778738) --  (axis cs:113.936,    +84.767527) --  (axis cs:114.883,    +84.756395) --  (axis cs:115.828,    +84.745345) --  (axis cs:116.771,    +84.734379) --  (axis cs:117.712,    +84.723502) --  (axis cs:118.652,    +84.712714) --  (axis cs:119.59,    +84.702019) --  (axis cs:120.526,    +84.691420) --  (axis cs:121.46,    +84.680920) --  (axis cs:122.392,    +84.670520) --  (axis cs:123.323,    +84.660224) --  (axis cs:124.253,    +84.650033) --  (axis cs:125.18,    +84.639952) --  (axis cs:126.106,    +84.629981) --  (axis cs:127.031,    +84.620124) --  (axis cs:127.954,    +84.610382) ;
  \draw[constellation-boundary]  (axis cs:127.954,    +84.610382) --  (axis cs:129.403,    +85.602792) --  (axis cs:131.692,    +86.590773) --  (axis cs:135.832,    +87.568922) ;
  \draw[constellation-boundary]  (axis cs:130.403,    +86.097546) --  (axis cs:131.288,    +86.088246) --  (axis cs:132.172,    +86.079068) --  (axis cs:133.054,    +86.070013) --  (axis cs:133.934,    +86.061083) --  (axis cs:134.812,    +86.052282) --  (axis cs:135.689,    +86.043609) --  (axis cs:136.564,    +86.035068) --  (axis cs:137.438,    +86.026659) --  (axis cs:138.31,    +86.018385) --  (axis cs:139.18,    +86.010247) --  (axis cs:140.049,    +86.002247) --  (axis cs:140.917,    +85.994387) --  (axis cs:141.783,    +85.986667) --  (axis cs:142.648,    +85.979090) --  (axis cs:143.511,    +85.971657) --  (axis cs:144.373,    +85.964370) --  (axis cs:145.234,    +85.957230) --  (axis cs:146.093,    +85.950238) --  (axis cs:146.951,    +85.943395) --  (axis cs:147.808,    +85.936704) --  (axis cs:148.664,    +85.930165) --  (axis cs:149.519,    +85.923779) --  (axis cs:150.373,    +85.917549) --  (axis cs:151.225,    +85.911474) --  (axis cs:152.077,    +85.905557) --  (axis cs:152.928,    +85.899797) --  (axis cs:153.777,    +85.894198) --  (axis cs:154.626,    +85.888758) --  (axis cs:155.474,    +85.883480) --  (axis cs:156.321,    +85.878364) --  (axis cs:157.167,    +85.873412) --  (axis cs:158.012,    +85.868624) --  (axis cs:158.856,    +85.864001) --  (axis cs:159.7,    +85.859544) --  (axis cs:160.543,    +85.855254) --  (axis cs:161.385,    +85.851131) --  (axis cs:162.227,    +85.847177) --  (axis cs:163.068,    +85.843392) --  (axis cs:163.909,    +85.839777) --  (axis cs:164.748,    +85.836332) --  (axis cs:165.588,    +85.833059) --  (axis cs:166.426,    +85.829957) --  (axis cs:167.265,    +85.827027) --  (axis cs:168.102,    +85.824270) --  (axis cs:168.94,    +85.821686) --  (axis cs:169.777,    +85.819276) --  (axis cs:170.613,    +85.817040) --  (axis cs:171.45,    +85.814979) --  (axis cs:172.286,    +85.813092) --  (axis cs:173.121,    +85.811381) --  (axis cs:173.957,    +85.809845) --  (axis cs:174.792,    +85.808486) --  (axis cs:175.627,    +85.807302) --  (axis cs:176.462,    +85.806295) --  (axis cs:177.296,    +85.805464) --  (axis cs:178.131,    +85.804809) --  (axis cs:178.965,    +85.804332) --  (axis cs:179.8,    +85.804031) ;
  \draw[constellation-boundary]  (axis cs:174.435,    +76.308421) --  (axis cs:174.91,    +76.307772) --  (axis cs:175.86,    +76.306626) --  (axis cs:176.81,    +76.305680) --  (axis cs:177.761,    +76.304935) --  (axis cs:178.711,    +76.304391) --  (axis cs:179.661,    +76.304049) ;
  \draw[constellation-boundary]  (axis cs:174.435,    +76.308421) --  (axis cs:174.462,    +77.308401) --  (axis cs:174.494,    +78.308379) --  (axis cs:174.532,    +79.308352) ;
  \draw[constellation-boundary]  (axis cs:162.819,    +79.340189) --  (axis cs:163.757,    +79.336533) --  (axis cs:164.695,    +79.333068) --  (axis cs:165.633,    +79.329793) --  (axis cs:166.57,    +79.326710) --  (axis cs:167.508,    +79.323820) --  (axis cs:168.445,    +79.321124) --  (axis cs:169.382,    +79.318621) --  (axis cs:170.318,    +79.316314) --  (axis cs:171.255,    +79.314201) --  (axis cs:172.191,    +79.312284) --  (axis cs:173.127,    +79.310564) --  (axis cs:174.064,    +79.309040) --  (axis cs:174.532,    +79.308352) ;
  \draw[constellation-boundary]  (axis cs:162.819,    +79.340189) --  (axis cs:162.947,    +80.339932) --  (axis cs:163.105,    +81.339616) ;
  \draw[constellation-boundary]  (axis cs:142.191,    +81.467773) --  (axis cs:142.659,    +81.464002) --  (axis cs:143.595,    +81.456577) --  (axis cs:144.53,    +81.449310) --  (axis cs:145.465,    +81.442203) --  (axis cs:146.398,    +81.435258) --  (axis cs:147.332,    +81.428475) --  (axis cs:148.264,    +81.421858) --  (axis cs:149.196,    +81.415408) --  (axis cs:150.127,    +81.409125) --  (axis cs:151.057,    +81.403013) --  (axis cs:151.987,    +81.397071) --  (axis cs:152.916,    +81.391303) --  (axis cs:153.845,    +81.385708) --  (axis cs:154.773,    +81.380289) --  (axis cs:155.701,    +81.375046) --  (axis cs:156.628,    +81.369982) --  (axis cs:157.555,    +81.365097) --  (axis cs:158.481,    +81.360393) --  (axis cs:159.407,    +81.355870) --  (axis cs:160.332,    +81.351530) --  (axis cs:161.257,    +81.347373) --  (axis cs:162.181,    +81.343402) --  (axis cs:163.105,    +81.339616) ;
  \draw[constellation-boundary]  (axis cs:140.615,    +72.974143) --  (axis cs:140.664,    +73.473947) --  (axis cs:140.77,    +74.473518) --  (axis cs:140.89,    +75.473031) --  (axis cs:141.028,    +76.472475) --  (axis cs:141.187,    +77.471831) --  (axis cs:141.374,    +78.471077) --  (axis cs:141.595,    +79.470183) --  (axis cs:141.862,    +80.469103) --  (axis cs:142.191,    +81.467773) ;
  \draw[constellation-boundary]  (axis cs:123.086,    +73.138381) --  (axis cs:123.575,    +73.133260) --  (axis cs:124.553,    +73.123101) --  (axis cs:125.531,    +73.113058) --  (axis cs:126.508,    +73.103133) --  (axis cs:127.484,    +73.093329) --  (axis cs:128.46,    +73.083649) --  (axis cs:129.435,    +73.074095) --  (axis cs:130.41,    +73.064669) --  (axis cs:131.384,    +73.055376) --  (axis cs:132.358,    +73.046217) --  (axis cs:133.331,    +73.037195) --  (axis cs:134.304,    +73.028312) --  (axis cs:135.276,    +73.019572) --  (axis cs:136.248,    +73.010976) --  (axis cs:137.219,    +73.002526) --  (axis cs:138.19,    +72.994226) --  (axis cs:139.16,    +72.986077) --  (axis cs:140.131,    +72.978082) --  (axis cs:141.1,    +72.970243) --  (axis cs:142.069,    +72.962562) --  (axis cs:143.038,    +72.955041) --  (axis cs:144.007,    +72.947683) --  (axis cs:144.975,    +72.940489) --  (axis cs:145.942,    +72.933461) --  (axis cs:146.909,    +72.926602) --  (axis cs:147.876,    +72.919913) --  (axis cs:148.843,    +72.913396) --  (axis cs:149.809,    +72.907053) --  (axis cs:150.775,    +72.900886) --  (axis cs:151.741,    +72.894895) --  (axis cs:152.706,    +72.889084) --  (axis cs:153.671,    +72.883453) --  (axis cs:154.636,    +72.878005) --  (axis cs:155.6,    +72.872740) --  (axis cs:156.564,    +72.867660) --  (axis cs:157.528,    +72.862766) --  (axis cs:158.492,    +72.858060) --  (axis cs:159.455,    +72.853543) --  (axis cs:160.418,    +72.849216) --  (axis cs:161.381,    +72.845080) --  (axis cs:162.344,    +72.841138) --  (axis cs:163.306,    +72.837388) --  (axis cs:164.268,    +72.833834) --  (axis cs:165.231,    +72.830475) --  (axis cs:166.193,    +72.827312) --  (axis cs:167.154,    +72.824347) --  (axis cs:168.116,    +72.821580) --  (axis cs:169.078,    +72.819012) --  (axis cs:170.039,    +72.816643) --  (axis cs:171,    +72.814475) --  (axis cs:171.961,    +72.812508) ;
  \draw[constellation-boundary]  (axis cs:122.129,    +59.643402) --  (axis cs:122.171,    +60.643180) --  (axis cs:122.216,    +61.642943) --  (axis cs:122.264,    +62.642691) --  (axis cs:122.316,    +63.642422) --  (axis cs:122.371,    +64.642132) --  (axis cs:122.43,    +65.641821) --  (axis cs:122.495,    +66.641484) --  (axis cs:122.564,    +67.641120) --  (axis cs:122.64,    +68.640723) --  (axis cs:122.723,    +69.640288) --  (axis cs:122.814,    +70.639811) --  (axis cs:122.914,    +71.639284) --  (axis cs:123.026,    +72.638699) --  (axis cs:123.086,    +73.138381) ;
  \draw[constellation-boundary]  (axis cs:107.753,    +59.803726) --  (axis cs:107.801,    +60.803448) --  (axis cs:107.852,    +61.803151) ;
  \draw[constellation-boundary]  (axis cs:107.753,    +59.803726) --  (axis cs:108.747,    +59.792136) --  (axis cs:109.74,    +59.780612) --  (axis cs:110.733,    +59.769155) --  (axis cs:111.726,    +59.757771) --  (axis cs:112.718,    +59.746461) --  (axis cs:113.71,    +59.735230) --  (axis cs:114.702,    +59.724081) --  (axis cs:115.693,    +59.713017) --  (axis cs:116.684,    +59.702042) --  (axis cs:117.675,    +59.691158) --  (axis cs:118.665,    +59.680369) --  (axis cs:119.655,    +59.669678) --  (axis cs:120.645,    +59.659089) --  (axis cs:121.634,    +59.648604) --  (axis cs:122.624,    +59.638227) --  (axis cs:123.612,    +59.627960) --  (axis cs:124.601,    +59.617806) --  (axis cs:125.589,    +59.607770) --  (axis cs:126.577,    +59.597853) --  (axis cs:127.565,    +59.588058) --  (axis cs:128.552,    +59.578389) --  (axis cs:128.799,    +59.575991) ;
  \draw[constellation-boundary]  (axis cs:94.4074,    +61.964130) --  (axis cs:94.9066,    +61.958069) --  (axis cs:95.9047,    +61.945960) --  (axis cs:96.9024,    +61.933869) --  (axis cs:97.8997,    +61.921801) --  (axis cs:98.8966,    +61.909759) --  (axis cs:99.8932,    +61.897747) --  (axis cs:100.889,    +61.885768) --  (axis cs:101.885,    +61.873826) --  (axis cs:102.88,    +61.861925) --  (axis cs:103.875,    +61.850067) --  (axis cs:104.87,    +61.838258) --  (axis cs:105.864,    +61.826500) --  (axis cs:106.858,    +61.814796) --  (axis cs:107.852,    +61.803151) ;
  \draw[constellation-boundary]  (axis cs:140.568,    +25.469429) --  (axis cs:140.578,    +26.469391) --  (axis cs:140.588,    +27.469352) --  (axis cs:140.598,    +28.469312) --  (axis cs:140.608,    +29.469271) --  (axis cs:140.619,    +30.469230) --  (axis cs:140.629,    +31.469187) --  (axis cs:140.64,    +32.469144) --  (axis cs:140.652,    +33.469100) --  (axis cs:140.663,    +34.469055) --  (axis cs:140.675,    +35.469009) --  (axis cs:140.687,    +36.468961) --  (axis cs:140.699,    +37.468913) --  (axis cs:140.712,    +38.468863) --  (axis cs:140.722,    +39.218824) ;
  \draw[constellation-boundary]  (axis cs:120.145,    +25.660426) --  (axis cs:120.158,    +26.660356) --  (axis cs:120.172,    +27.660285) ;
  \draw[constellation-boundary]  (axis cs:120.172,    +27.660285) --  (axis cs:120.919,    +27.652374) --  (axis cs:121.916,    +27.641919) ;
  \draw[constellation-boundary]  (axis cs:121.916,    +27.641919) --  (axis cs:121.929,    +28.641849) --  (axis cs:121.943,    +29.641778) --  (axis cs:121.957,    +30.641705) --  (axis cs:121.971,    +31.641630) --  (axis cs:121.986,    +32.641554) --  (axis cs:121.993,    +33.141516) ;
  \draw[constellation-boundary]  (axis cs:-4.782,    +63.692873) ;
  \draw[constellation-boundary]  (axis cs:-4.782,    +63.692873) --  (axis cs:-4.788,    +64.692869) --  (axis cs:-4.795,    +65.692865) --  (axis cs:-4.802,    +66.692861) ;
  \draw[constellation-boundary]  (axis cs:-4.802,    +66.692861) --  (axis cs:-4.545,    +66.693151) --  (axis cs:-3.517,    +66.694174) --  (axis cs:-2.489,    +66.694980) --  (axis cs:-1.461,    +66.695569) --  (axis cs:-0.433,    +66.695939) --  (axis cs:0.595318,    +66.696092) --  (axis cs:1.62345,    +66.696026) --  (axis cs:2.65157,    +66.695743) --  (axis cs:3.67968,    +66.695241) --  (axis cs:4.70775,    +66.694522) --  (axis cs:5.7358,    +66.693586) --  (axis cs:6.7638,    +66.692432) ;
  \draw[constellation-boundary]  (axis cs:6.7638,    +66.692432) --  (axis cs:6.77197,    +67.692427) --  (axis cs:6.78087,    +68.692422) --  (axis cs:6.79061,    +69.692416) --  (axis cs:6.8013,    +70.692409) --  (axis cs:6.81313,    +71.692402) --  (axis cs:6.82626,    +72.692394) --  (axis cs:6.84096,    +73.692385) --  (axis cs:6.85752,    +74.692375) --  (axis cs:6.87634,    +75.692363) --  (axis cs:6.89792,    +76.692350) --  (axis cs:6.92295,    +77.692335) ;
  \draw[constellation-boundary]  (axis cs:6.92295,    +77.692335) --  (axis cs:7.97818,    +77.690928) --  (axis cs:9.03327,    +77.689299) --  (axis cs:10.0882,    +77.687449) --  (axis cs:11.143,    +77.685378) --  (axis cs:12.1976,    +77.683087) --  (axis cs:13.2519,    +77.680577) --  (axis cs:14.306,    +77.677848) --  (axis cs:15.3599,    +77.674903) --  (axis cs:16.4135,    +77.671741) --  (axis cs:17.4668,    +77.668364) --  (axis cs:18.5198,    +77.664773) --  (axis cs:19.5725,    +77.660969) --  (axis cs:20.6249,    +77.656954) --  (axis cs:21.6768,    +77.652729) --  (axis cs:22.7284,    +77.648296) --  (axis cs:23.7796,    +77.643656) --  (axis cs:24.8304,    +77.638811) --  (axis cs:25.8807,    +77.633762) --  (axis cs:26.9306,    +77.628512) --  (axis cs:27.98,    +77.623062) --  (axis cs:29.0289,    +77.617414) --  (axis cs:30.0774,    +77.611570) --  (axis cs:31.1253,    +77.605532) --  (axis cs:32.1727,    +77.599303) --  (axis cs:33.2195,    +77.592884) --  (axis cs:34.2657,    +77.586278) --  (axis cs:35.3114,    +77.579486) --  (axis cs:36.3565,    +77.572512) --  (axis cs:37.401,    +77.565358) --  (axis cs:38.4448,    +77.558026) --  (axis cs:39.488,    +77.550519) --  (axis cs:40.5306,    +77.542839) --  (axis cs:41.5724,    +77.534989) --  (axis cs:42.6137,    +77.526972) --  (axis cs:43.6542,    +77.518791) --  (axis cs:44.694,    +77.510448) --  (axis cs:45.7331,    +77.501946) --  (axis cs:46.7714,    +77.493288) --  (axis cs:47.809,    +77.484477) --  (axis cs:48.8459,    +77.475516) --  (axis cs:49.882,    +77.466409) --  (axis cs:50.9174,    +77.457157) --  (axis cs:51.9519,    +77.447765) --  (axis cs:52.9857,    +77.438235) --  (axis cs:54.0186,    +77.428571) --  (axis cs:55.0507,    +77.418775) --  (axis cs:56.0821,    +77.408852) --  (axis cs:56.7262,    +77.402587) ;
  \draw[constellation-boundary]  (axis cs:38.7624,    +57.551291) --  (axis cs:39.2698,    +57.547534) --  (axis cs:40.2846,    +57.539892) --  (axis cs:41.2992,    +57.532084) --  (axis cs:42.3135,    +57.524113) --  (axis cs:43.3276,    +57.515981) --  (axis cs:44.3415,    +57.507690) --  (axis cs:45.3551,    +57.499244) --  (axis cs:46.3685,    +57.490644) --  (axis cs:47.3816,    +57.481894) --  (axis cs:48.3945,    +57.472997) --  (axis cs:49.4072,    +57.463954) --  (axis cs:49.9134,    +57.459380) ;
  \draw[constellation-boundary]  (axis cs:38.7624,    +57.551291) --  (axis cs:38.7887,    +58.551195) --  (axis cs:38.8024,    +59.051144) ;
  \draw[constellation-boundary]  (axis cs:30.7956,    +59.104599) --  (axis cs:31.1772,    +59.102308) --  (axis cs:32.1945,    +59.096072) --  (axis cs:33.2117,    +59.089652) --  (axis cs:34.2287,    +59.083050) --  (axis cs:35.2454,    +59.076267) --  (axis cs:36.262,    +59.069307) --  (axis cs:37.2783,    +59.062172) --  (axis cs:38.2944,    +59.054863) --  (axis cs:38.8024,    +59.051144) ;
  \draw[constellation-boundary]  (axis cs:30.7737,    +58.104665) --  (axis cs:30.7845,    +58.604632) --  (axis cs:30.7956,    +59.104599) ;
  \draw[constellation-boundary]  (axis cs:27.5952,    +58.122710) --  (axis cs:28.1039,    +58.119948) --  (axis cs:29.1211,    +58.114280) --  (axis cs:30.1381,    +58.108422) --  (axis cs:30.7737,    +58.104665) ;
  \draw[constellation-boundary]  (axis cs:27.5336,    +54.622876) --  (axis cs:27.5501,    +55.622832) --  (axis cs:27.5674,    +56.622785) --  (axis cs:27.5857,    +57.622736) --  (axis cs:27.5952,    +58.122710) ;
  \draw[constellation-boundary]  (axis cs:22.4559,    +54.647763) --  (axis cs:22.9638,    +54.645496) --  (axis cs:23.9796,    +54.640814) --  (axis cs:24.9952,    +54.635933) --  (axis cs:26.0107,    +54.630856) --  (axis cs:27.026,    +54.625584) --  (axis cs:27.5336,    +54.622876) ;
  \draw[constellation-boundary]  (axis cs:-4.104,    +88.694433) --  (axis cs:-2.572,    +88.695310) --  (axis cs:-1.039,    +88.695862) --  (axis cs:0.49436,    +88.696090) --  (axis cs:2.02799,    +88.695992) --  (axis cs:3.56121,    +88.695569) --  (axis cs:5.0935,    +88.694822) --  (axis cs:6.62436,    +88.693751) --  (axis cs:8.15327,    +88.692357) --  (axis cs:9.67974,    +88.690642) --  (axis cs:11.2033,    +88.688607) --  (axis cs:12.7234,    +88.686255) --  (axis cs:14.2396,    +88.683588) --  (axis cs:15.7515,    +88.680609) --  (axis cs:17.2586,    +88.677321) --  (axis cs:18.7604,    +88.673726) --  (axis cs:20.2566,    +88.669830) --  (axis cs:21.7468,    +88.665635) --  (axis cs:23.2306,    +88.661145) --  (axis cs:24.7075,    +88.656365) --  (axis cs:26.1773,    +88.651300) --  (axis cs:27.6397,    +88.645954) --  (axis cs:29.0943,    +88.640331) --  (axis cs:30.5409,    +88.634438) --  (axis cs:31.9791,    +88.628280) --  (axis cs:33.4088,    +88.621861) --  (axis cs:34.8297,    +88.615188) --  (axis cs:36.2416,    +88.608266) --  (axis cs:37.6443,    +88.601100) --  (axis cs:39.0376,    +88.593698) --  (axis cs:40.4215,    +88.586064) --  (axis cs:41.7957,    +88.578205) --  (axis cs:43.1602,    +88.570127) --  (axis cs:44.5148,    +88.561836) --  (axis cs:45.8595,    +88.553338) --  (axis cs:47.1943,    +88.544639) --  (axis cs:48.519,    +88.535746) --  (axis cs:49.8336,    +88.526665) --  (axis cs:51.1382,    +88.517401) --  (axis cs:52.4327,    +88.507962) --  (axis cs:53.7171,    +88.498353) --  (axis cs:54.9914,    +88.488580) --  (axis cs:56.2557,    +88.478650) --  (axis cs:57.5101,    +88.468569) --  (axis cs:58.7545,    +88.458342) --  (axis cs:59.989,    +88.447975) --  (axis cs:61.2137,    +88.437475) --  (axis cs:62.4286,    +88.426848) --  (axis cs:63.634,    +88.416098) --  (axis cs:64.8297,    +88.405232) --  (axis cs:66.016,    +88.394256) --  (axis cs:67.193,    +88.383175) --  (axis cs:68.3607,    +88.371994) --  (axis cs:69.5192,    +88.360719) --  (axis cs:70.6687,    +88.349355) --  (axis cs:71.8093,    +88.337907) --  (axis cs:72.9411,    +88.326381) --  (axis cs:74.0642,    +88.314782) --  (axis cs:75.1788,    +88.303114) --  (axis cs:76.285,    +88.291383) --  (axis cs:77.3829,    +88.279594) --  (axis cs:78.4726,    +88.267750) --  (axis cs:79.5543,    +88.255858) --  (axis cs:80.6281,    +88.243921) --  (axis cs:81.6942,    +88.231944) --  (axis cs:82.7526,    +88.219931) --  (axis cs:83.8035,    +88.207887) --  (axis cs:84.8471,    +88.195815) --  (axis cs:85.8834,    +88.183721) --  (axis cs:86.9127,    +88.171608) --  (axis cs:87.9349,    +88.159481) --  (axis cs:88.9504,    +88.147342) --  (axis cs:89.9591,    +88.135197) --  (axis cs:90.9612,    +88.123048) --  (axis cs:91.9569,    +88.110901) --  (axis cs:92.9462,    +88.098757) --  (axis cs:93.9294,    +88.086621) --  (axis cs:94.9064,    +88.074497) --  (axis cs:95.8775,    +88.062387) --  (axis cs:96.8428,    +88.050295) --  (axis cs:97.8023,    +88.038225) --  (axis cs:98.7562,    +88.026180) --  (axis cs:99.7046,    +88.014162) --  (axis cs:100.648,    +88.002176) --  (axis cs:101.585,    +87.990224) --  (axis cs:102.518,    +87.978308) --  (axis cs:103.445,    +87.966433) --  (axis cs:104.368,    +87.954601) --  (axis cs:105.286,    +87.942814) --  (axis cs:106.199,    +87.931076) --  (axis cs:107.107,    +87.919390) --  (axis cs:108.011,    +87.907757) --  (axis cs:108.91,    +87.896181) --  (axis cs:109.805,    +87.884664) --  (axis cs:110.696,    +87.873208) --  (axis cs:111.582,    +87.861817) --  (axis cs:112.465,    +87.850492) --  (axis cs:113.343,    +87.839236) --  (axis cs:114.218,    +87.828052) --  (axis cs:115.089,    +87.816941) --  (axis cs:115.955,    +87.805905) --  (axis cs:116.819,    +87.794948) --  (axis cs:117.679,    +87.784071) --  (axis cs:118.535,    +87.773276) --  (axis cs:119.388,    +87.762565) --  (axis cs:120.237,    +87.751941) --  (axis cs:121.083,    +87.741405) --  (axis cs:121.926,    +87.730959) --  (axis cs:122.766,    +87.720606) --  (axis cs:123.603,    +87.710346) --  (axis cs:124.437,    +87.700183) --  (axis cs:125.268,    +87.690117) --  (axis cs:126.096,    +87.680151) --  (axis cs:126.921,    +87.670286) --  (axis cs:127.744,    +87.660525) --  (axis cs:128.564,    +87.650867) --  (axis cs:129.381,    +87.641316) --  (axis cs:130.196,    +87.631873) --  (axis cs:131.008,    +87.622540) --  (axis cs:131.818,    +87.613317) --  (axis cs:132.625,    +87.604207) --  (axis cs:133.43,    +87.595211) --  (axis cs:134.233,    +87.586331) --  (axis cs:135.034,    +87.577567) --  (axis cs:135.832,    +87.568922) ;
  \draw[constellation-boundary]  (axis cs:171.849,    +65.812616) --  (axis cs:172.823,    +65.810828) --  (axis cs:173.796,    +65.809244) --  (axis cs:174.769,    +65.807865) --  (axis cs:175.742,    +65.806691) --  (axis cs:176.715,    +65.805722) --  (axis cs:177.688,    +65.804960) --  (axis cs:178.661,    +65.804403) --  (axis cs:179.633,    +65.804052) ;
  \draw[constellation-boundary]  (axis cs:171.849,    +65.812616) --  (axis cs:171.855,    +66.312611) --  (axis cs:171.868,    +67.312598) --  (axis cs:171.881,    +68.312585) --  (axis cs:171.896,    +69.312571) --  (axis cs:171.913,    +70.312555) --  (axis cs:171.931,    +71.312537) --  (axis cs:171.951,    +72.312518) --  (axis cs:171.961,    +72.812508) ;
  \draw[constellation-boundary]  (axis cs:90.1756,    +25.009566) --  (axis cs:90.1905,    +26.009475) --  (axis cs:90.2057,    +27.009383) --  (axis cs:90.2211,    +28.009289) --  (axis cs:90.2289,    +28.509242) ;
  \draw[constellation-boundary]  (axis cs:118.258,    +33.181229) --  (axis cs:118.266,    +33.681187) --  (axis cs:118.282,    +34.681101) --  (axis cs:118.29,    +35.181058) ;
  \draw[constellation-boundary]  (axis cs:118.258,    +33.181229) --  (axis cs:119.005,    +33.173171) --  (axis cs:120.001,    +33.162515) --  (axis cs:120.997,    +33.151963) --  (axis cs:121.993,    +33.141516) --  (axis cs:122.989,    +33.131179) --  (axis cs:123.985,    +33.120954) --  (axis cs:124.98,    +33.110844) --  (axis cs:125.976,    +33.100854) --  (axis cs:126.971,    +33.090984) --  (axis cs:127.966,    +33.081240) --  (axis cs:128.961,    +33.071623) --  (axis cs:129.956,    +33.062136) --  (axis cs:130.951,    +33.052782) --  (axis cs:131.946,    +33.043565) --  (axis cs:132.94,    +33.034486) --  (axis cs:133.935,    +33.025549) --  (axis cs:134.93,    +33.016756) --  (axis cs:135.924,    +33.008109) --  (axis cs:136.918,    +32.999612) --  (axis cs:137.912,    +32.991267) --  (axis cs:138.906,    +32.983077) --  (axis cs:139.901,    +32.975043) --  (axis cs:140.894,    +32.967169) --  (axis cs:141.888,    +32.959456) --  (axis cs:142.882,    +32.951907) --  (axis cs:143.876,    +32.944525) --  (axis cs:144.869,    +32.937311) --  (axis cs:145.863,    +32.930267) --  (axis cs:146.856,    +32.923397) --  (axis cs:147.85,    +32.916700) --  (axis cs:148.843,    +32.910181) --  (axis cs:149.836,    +32.903840) --  (axis cs:150.084,    +32.902283) ;
  \draw[constellation-boundary]  (axis cs:150.042,    +27.902415) --  (axis cs:150.046,    +28.402402) --  (axis cs:150.055,    +29.402377) --  (axis cs:150.063,    +30.402351) --  (axis cs:150.071,    +31.402324) --  (axis cs:150.08,    +32.402297) --  (axis cs:150.084,    +32.902283) ;
  \draw[constellation-boundary]  (axis cs:150.042,    +27.902415) --  (axis cs:150.788,    +27.897806) --  (axis cs:151.782,    +27.891821) --  (axis cs:152.777,    +27.886020) --  (axis cs:153.771,    +27.880405) --  (axis cs:154.765,    +27.874979) --  (axis cs:155.759,    +27.869741) --  (axis cs:156.753,    +27.864694) --  (axis cs:157.747,    +27.859840) --  (axis cs:158.741,    +27.855180) --  (axis cs:159.238,    +27.852923) ;
  \draw[constellation-boundary]  (axis cs:159.224,    +25.352955) --  (axis cs:159.23,    +26.352942) --  (axis cs:159.236,    +27.352930) --  (axis cs:159.238,    +27.852923) ;
  \draw[constellation-boundary]  (axis cs:166.683,    +25.325050) --  (axis cs:166.686,    +26.325044) --  (axis cs:166.69,    +27.325039) --  (axis cs:166.694,    +28.325033) --  (axis cs:166.698,    +29.325027) --  (axis cs:166.702,    +30.325021) --  (axis cs:166.706,    +31.325015) --  (axis cs:166.71,    +32.325009) --  (axis cs:166.714,    +33.325003) ;
  \draw[constellation-boundary]  (axis cs:166.694,    +28.325033) --  (axis cs:167.688,    +28.322174) --  (axis cs:168.681,    +28.319521) --  (axis cs:169.675,    +28.317073) --  (axis cs:170.668,    +28.314832) --  (axis cs:171.662,    +28.312799) --  (axis cs:172.655,    +28.310973) --  (axis cs:173.648,    +28.309356) --  (axis cs:174.642,    +28.307948) --  (axis cs:175.635,    +28.306750) --  (axis cs:176.629,    +28.305761) --  (axis cs:177.622,    +28.304982) --  (axis cs:178.615,    +28.304414) --  (axis cs:179.609,    +28.304056) ;
  \draw[constellation-boundary]  (axis cs:179.608,    +25.304056) --  (axis cs:179.608,    +26.304056) --  (axis cs:179.609,    +27.304056) --  (axis cs:179.609,    +28.304056) ;
  \draw[constellation-boundary]  (axis cs:140.722,    +39.218824) --  (axis cs:140.97,    +39.216874) --  (axis cs:141.962,    +39.209173) --  (axis cs:142.954,    +39.201637) --  (axis cs:143.946,    +39.194266) --  (axis cs:144.938,    +39.187064) --  (axis cs:145.682,    +39.181774) ;
  \draw[constellation-boundary]  (axis cs:145.682,    +39.181774) --  (axis cs:145.685,    +39.431764) --  (axis cs:145.697,    +40.431722) --  (axis cs:145.709,    +41.431678) ;
  \draw[constellation-boundary]  (axis cs:154.359,    +39.377419) --  (axis cs:154.369,    +40.377394) --  (axis cs:154.378,    +41.377368) ;
  \draw[constellation-boundary]  (axis cs:154.359,    +39.377419) --  (axis cs:154.855,    +39.374738) --  (axis cs:155.846,    +39.369517) --  (axis cs:156.837,    +39.364487) --  (axis cs:157.827,    +39.359649) --  (axis cs:158.818,    +39.355004) --  (axis cs:159.809,    +39.350554) --  (axis cs:160.799,    +39.346300) --  (axis cs:161.79,    +39.342244) --  (axis cs:162.78,    +39.338386) --  (axis cs:163.523,    +39.335623) ;
  \draw[constellation-boundary]  (axis cs:163.489,    +33.335685) --  (axis cs:163.495,    +34.335675) --  (axis cs:163.5,    +35.335665) --  (axis cs:163.506,    +36.335655) --  (axis cs:163.511,    +37.335645) --  (axis cs:163.517,    +38.335634) --  (axis cs:163.523,    +39.335623) ;
  \draw[constellation-boundary]  (axis cs:163.489,    +33.335685) --  (axis cs:163.738,    +33.334787) --  (axis cs:164.73,    +33.331323) --  (axis cs:165.722,    +33.328062) --  (axis cs:166.714,    +33.325003) ;
  \draw[constellation-boundary]  (axis cs:128.44,    +46.577732) --  (axis cs:128.461,    +47.577630) --  (axis cs:128.483,    +48.577525) --  (axis cs:128.505,    +49.577416) --  (axis cs:128.529,    +50.577301) --  (axis cs:128.553,    +51.577182) --  (axis cs:128.579,    +52.577058) --  (axis cs:128.606,    +53.576928) --  (axis cs:128.634,    +54.576791) --  (axis cs:128.664,    +55.576647) --  (axis cs:128.695,    +56.576496) --  (axis cs:128.728,    +57.576337) --  (axis cs:128.762,    +58.576169) --  (axis cs:128.799,    +59.575991) ;
  \draw[constellation-boundary]  (axis cs:128.44,    +46.577732) --  (axis cs:129.184,    +46.570553) --  (axis cs:130.176,    +46.561095) --  (axis cs:131.168,    +46.551772) --  (axis cs:132.159,    +46.542584) --  (axis cs:133.151,    +46.533535) --  (axis cs:134.142,    +46.524627) --  (axis cs:135.133,    +46.515864) --  (axis cs:136.124,    +46.507248) --  (axis cs:137.114,    +46.498781) --  (axis cs:138.105,    +46.490466) --  (axis cs:139.095,    +46.482306) --  (axis cs:139.591,    +46.478284) ;
  \draw[constellation-boundary]  (axis cs:139.512,    +41.478600) --  (axis cs:140.008,    +41.474612) --  (axis cs:141,    +41.466755) --  (axis cs:141.992,    +41.459059) --  (axis cs:142.983,    +41.451527) --  (axis cs:143.975,    +41.444162) --  (axis cs:144.966,    +41.436964) --  (axis cs:145.957,    +41.429937) --  (axis cs:146.948,    +41.423083) --  (axis cs:147.939,    +41.416403) --  (axis cs:148.93,    +41.409899) --  (axis cs:149.921,    +41.403574) --  (axis cs:150.911,    +41.397429) --  (axis cs:151.902,    +41.391467) --  (axis cs:152.893,    +41.385688) --  (axis cs:153.883,    +41.380094) --  (axis cs:154.378,    +41.377368) ;
  \draw[constellation-boundary]  (axis cs:139.512,    +41.478600) --  (axis cs:139.527,    +42.478541) --  (axis cs:139.542,    +43.478480) --  (axis cs:139.558,    +44.478417) --  (axis cs:139.574,    +45.478352) --  (axis cs:139.591,    +46.478284) ;
  \draw[constellation-boundary]  (axis cs:40.4023,    +34.537508) --  (axis cs:40.9055,    +34.533615) --  (axis cs:41.9118,    +34.525708) --  (axis cs:42.6664,    +34.519672) ;
  \draw[constellation-boundary]  (axis cs:40.4023,    +34.537508) --  (axis cs:40.4138,    +35.537464) --  (axis cs:40.4256,    +36.537418) --  (axis cs:40.4346,    +37.287384) ;
  \draw[constellation-boundary]  (axis cs:22.8665,    +28.645431) --  (axis cs:22.8724,    +29.645417) --  (axis cs:22.8785,    +30.645404) --  (axis cs:22.8847,    +31.645390) --  (axis cs:22.891,    +32.645375) --  (axis cs:22.8975,    +33.645360) --  (axis cs:22.9041,    +34.645345) --  (axis cs:22.9109,    +35.645330) ;
  \draw[constellation-boundary]  (axis cs:22.8665,    +28.645431) --  (axis cs:23.7467,    +28.641362) --  (axis cs:24.7528,    +28.636527) --  (axis cs:25.7587,    +28.631498) --  (axis cs:26.7646,    +28.626275) ;
  



% Labels

% Phoenix too small to label

\node[constellation-label] at (axis cs:{85,40})  {\Huge \bfseries\textls{Auriga}};
\node[constellation-label] at (axis cs:{260,60}) {\huge \bfseries\textls{Draco}};
\node[constellation-label] at (axis cs:{175,74.5}) {\large \bfseries\textls{Draco}};

\node[constellation-label] at (axis cs:{240,80})  {\Large \bfseries\textls{Minor}};
\node[constellation-label] at (axis cs:{230,75}){\Large \bfseries\textls{Ursa}};
\node[constellation-label] at (axis cs:{115,50}) {\huge \bfseries\textls{Lynx}};

\node[constellation-label] at (axis cs:{135,66}) {\large \bfseries\textls{Ursa Major}};
\node[constellation-label,rotate=76] at (axis cs:{104,32}) {\bfseries\textls{Gemini}};
\node[constellation-label] at (axis cs:{91,70}) {\Large\bfseries\textls{Camelopardalis}};

\node[constellation-label] at (axis cs:{52,45}) {\huge\bfseries\textls{Perseus}};

\node[constellation-label] at (axis cs:{05,55}) {\Large\bfseries\textls{Cassiopeia}};
\node[constellation-label] at (axis cs:{330,71}) {\Large\bfseries\textls{Cepheus}};
\node[constellation-label,rotate=300] at (axis cs:{60,30.5}) {{Taurus}};
\node[constellation-label,rotate=-45] at (axis cs:{45,30.5}) {{Aries}};
\node[constellation-label,rotate=-30] at (axis cs:{30,32}) {{Triangulum}};
\node[constellation-label,rotate=-18] at (axis cs:{20,32.5}) {{Pisces}};
\node[constellation-label,rotate=15] at (axis cs:{345,32}) {{Pegasus}};

\node[constellation-label] at (axis cs:{337,45}) {\large\bfseries\textls{Lacerta}};
\node[constellation-label] at (axis cs:{12,41}) {\huge\bfseries\textls{Andromeda}};

\node[constellation-label] at (axis cs:{282,37}) {\LARGE\bfseries\textls{Lyra}};
\node[constellation-label] at (axis cs:{305,45}) {\huge\bfseries\textls{Cygnus}};

\node[constellation-label] at (axis cs:{255,41}){\Large \bfseries\textls{Hercules}};

\end{polaraxis}


% Nebulae and nebulae names



\begin{polaraxis}[rotate=90,name=NGC,at=(base.center),anchor=center,axis lines=none]

  \clip (0\tendegree,-6\tendegree) arc (270:90:6\tendegree)
  -- (2\tendegree,6\tendegree)  -- (2\tendegree,-6\tendegree)
   -- cycle ;


\node[Messier,pin={[pin distance=-0.8\onedegree,Messier-label]0:{M81}}] at (axis cs:148.900,69.067) {\tikz\pgfuseplotmark{GAL};}; %  6.9
\node[Messier,pin={[pin distance=-0.8\onedegree,Messier-label]0:{M82}}] at (axis cs:148.950,69.683) {\tikz\pgfuseplotmark{GAL};}; %  8.4
\node[Messier,pin={[pin distance=-0.8\onedegree,Messier-label]-90:{M52}}] at (axis cs:351.050,61.583) {\tikz\pgfuseplotmark{OC};}; %  6.9
\node[Messier,pin={[pin distance=-1\onedegree,Messier-label]90:{M103}}] at (axis cs:23.300,60.700) {\tikz\pgfuseplotmark{OC};}; %  7.4
\node[Messier,pin={[pin distance=-0.8\onedegree,Messier-label]90:{M102}}] at (axis cs:226.625,55.767) {\tikz\pgfuseplotmark{GAL};}; % 10.0
\node[Messier,pin={[pin distance=-0.8\onedegree,Messier-label]90:{M108}}] at (axis cs:167.875,55.667) {\tikz\pgfuseplotmark{GAL};}; % 10.1
\node[Messier,pin={[pin distance=-0.8\onedegree,Messier-label]90:{M101}}] at (axis cs:210.800,54.350) {\tikz\pgfuseplotmark{GAL};}; %  7.7
\node[Messier,pin={[pin distance=-0.8\onedegree,Messier-label]90:{M109}}] at (axis cs:179.400,53.383) {\tikz\pgfuseplotmark{GAL};}; %  9.8
\node[Messier,pin={[pin distance=-1\onedegree,Messier-label]-90:{M39}}] at (axis cs:323.050,48.433) {\tikz\pgfuseplotmark{OC};}; %  4.6
\node[Messier,pin={[pin distance=-0.8\onedegree,Messier-label]90:{M51}}] at (axis cs:202.475,47.200) {\tikz\pgfuseplotmark{GAL};}; %  8.4
\node[Messier,pin={[pin distance=-0.8\onedegree,Messier-label]90:{}}] at (axis cs:202.500,47.267) {\tikz\pgfuseplotmark{GAL};}; %  9.6
\node[Messier,pin={[pin distance=-0.8\onedegree,Messier-label]90:{M106}}] at (axis cs:184.750,47.300) {\tikz\pgfuseplotmark{GAL};}; %  8.3
\node[Messier,pin={[pin distance=-0.8\onedegree,Messier-label]90:{M92}}] at (axis cs:259.275,43.133) {\tikz\pgfuseplotmark{GC};}; %  6.5
\node[Messier,pin={[pin distance=-0.8\onedegree,Messier-label]90:{M34}}] at (axis cs:40.500,42.783) {\tikz\pgfuseplotmark{OC};}; %  5.2
\node[Messier,pin={[pin distance=-0.8\onedegree,Messier-label]90:{M63}}] at (axis cs:198.950,42.033) {\tikz\pgfuseplotmark{GAL};}; %  8.6
\node[Messier,pin={[pin distance=-1.8\onedegree,Messier-label]135:{M31}}] at (axis cs:10.675,41.267) {\tikz\pgfuseplotmark{GAL};}; %  3.5
\node[Messier,pin={[pin distance=-0.8\onedegree,Messier-label]90:{M94}}] at (axis cs:192.725,41.117) {\tikz\pgfuseplotmark{GAL};}; %  8.2
\node[Messier,pin={[pin distance=-1.8\onedegree,Messier-label]45:{M110}}] at (axis cs:10.100,41.683) {\tikz\pgfuseplotmark{GAL};}; %  8.0
\node[Messier,pin={[pin distance=-1\onedegree,Messier-label]-90:{M32}}] at (axis cs:10.675,40.867) {\tikz\pgfuseplotmark{GAL};}; %  8.2
\node[Messier,pin={[pin distance=-0.8\onedegree,Messier-label]180:{M29}}] at (axis cs:305.975,38.533) {\tikz\pgfuseplotmark{OC};}; %  6.6
\node[Messier,pin={[pin distance=-0.8\onedegree,Messier-label]90:{M13}}] at (axis cs:250.425,36.467) {\tikz\pgfuseplotmark{GC};}; %  5.9
\node[Messier,pin={[pin distance=-0.8\onedegree,Messier-label]90:{M38}}] at (axis cs:82.175,35.833) {\tikz\pgfuseplotmark{OC};}; %  6.4
\node[Messier,pin={[pin distance=-1\onedegree,Messier-label]180:{M36}}] at (axis cs:84.025,34.133) {\tikz\pgfuseplotmark{OC};}; %  6.0
\node[Messier,pin={[pin distance=-0.8\onedegree,Messier-label]0:{M57}}] at (axis cs:283.400,33.033) {\tikz\pgfuseplotmark{PN};}; %  9.0
\node[Messier,pin={[pin distance=-0.8\onedegree,Messier-label]90:{M37}}] at (axis cs:88.100,32.550) {\tikz\pgfuseplotmark{OC};}; %  5.6
\node[Messier,pin={[pin distance=-0.8\onedegree,Messier-label]90:{M33}}] at (axis cs:23.475,30.650) {\tikz\pgfuseplotmark{GAL};}; %  5.7
\node[Messier,pin={[pin distance=-0.8\onedegree,Messier-label]180:{M56}}] at (axis cs:289.150,30.183) {\tikz\pgfuseplotmark{GC};}; %  8.3
\node[Messier,pin={[pin distance=-0.8\onedegree,Messier-label]90:{M3}}] at (axis cs:205.550,28.383) {\tikz\pgfuseplotmark{GC};}; %  6.4


\node[NGC,pin={[pin distance=-1.2\onedegree,NGC-label]90:{I10}}] at (axis cs:5.100,59.300) {\tikz\pgfuseplotmark{GAL};}; % 10.3
\node[NGC,pin={[pin distance=-1.2\onedegree,NGC-label]90:{147}}] at (axis cs:8.300,48.500) {\tikz\pgfuseplotmark{GAL};}; %  9.3
\node[NGC,pin={[pin distance=-1.2\onedegree,NGC-label]90:{185}}] at (axis cs:9.750,48.333) {\tikz\pgfuseplotmark{GAL};}; %  9.2
\node[NGC,pin={[pin distance=-1.2\onedegree,NGC-label]90:{278}}] at (axis cs:13.025,47.550) {\tikz\pgfuseplotmark{GAL};}; % 10.9
\node[NGC,pin={[pin distance=-1.2\onedegree,NGC-label]90:{404}}] at (axis cs:17.350,35.717) {\tikz\pgfuseplotmark{GAL};}; % 10.1
\node[NGC,pin={[pin distance=-1.2\onedegree,NGC-label]90:{672}}] at (axis cs:26.975,27.433) {\tikz\pgfuseplotmark{GAL};}; % 10.8
\node[NGC,pin={[pin distance=-1.2\onedegree,NGC-label]90:{891}}] at (axis cs:35.650,42.350) {\tikz\pgfuseplotmark{GAL};}; % 10.0
\node[NGC,pin={[pin distance=-1.2\onedegree,NGC-label]0:{925}}] at (axis cs:36.825,33.583) {\tikz\pgfuseplotmark{GAL};}; % 10.0
\node[NGC,pin={[pin distance=-1.2\onedegree,NGC-label]90:{1023}}] at (axis cs:40.100,39.067) {\tikz\pgfuseplotmark{GAL};}; %  9.5
\node[NGC,pin={[pin distance=-1.2\onedegree,NGC-label]90:{I342}}] at (axis cs:56.700,68.100) {\tikz\pgfuseplotmark{GAL};}; %  9. 
\node[NGC,pin={[pin distance=-1.2\onedegree,NGC-label]90:{2146}}] at (axis cs:94.675,78.350) {\tikz\pgfuseplotmark{GAL};}; % 10.5
\node[NGC,pin={[pin distance=-1.2\onedegree,NGC-label]90:{2336}}] at (axis cs:111.775,80.183) {\tikz\pgfuseplotmark{GAL};}; % 10.5
\node[NGC,pin={[pin distance=-1.2\onedegree,NGC-label]90:{2366}}] at (axis cs:112.225,69.217) {\tikz\pgfuseplotmark{GAL};}; % 10.9
\node[NGC,pin={[pin distance=-1.2\onedegree,NGC-label]90:{2403}}] at (axis cs:114.225,65.600) {\tikz\pgfuseplotmark{GAL};}; %  8.4
\node[NGC,pin={[pin distance=-1.2\onedegree,NGC-label]90:{2683}}] at (axis cs:133.175,33.417) {\tikz\pgfuseplotmark{GAL};}; %  9.7
\node[NGC,pin={[pin distance=-1.2\onedegree,NGC-label]90:{2681}}] at (axis cs:133.375,51.317) {\tikz\pgfuseplotmark{GAL};}; % 10.3
\node[NGC,pin={[pin distance=-1.2\onedegree,NGC-label]90:{2655}}] at (axis cs:133.900,78.217) {\tikz\pgfuseplotmark{GAL};}; % 10.1
\node[NGC,pin={[pin distance=-1.2\onedegree,NGC-label]90:{2768}}] at (axis cs:137.900,60.033) {\tikz\pgfuseplotmark{GAL};}; % 10.0
\node[NGC,pin={[pin distance=-1.2\onedegree,NGC-label]-90:{2787}}] at (axis cs:139.825,69.200) {\tikz\pgfuseplotmark{GAL};}; % 10.8
\node[NGC,pin={[pin distance=-1.2\onedegree,NGC-label]90:{2841}}] at (axis cs:140.500,50.967) {\tikz\pgfuseplotmark{GAL};}; %  9.3
\node[NGC,pin={[pin distance=-1.2\onedegree,NGC-label]90:{2859}}] at (axis cs:141.075,34.517) {\tikz\pgfuseplotmark{GAL};}; % 10.7
\node[NGC,pin={[pin distance=-1.2\onedegree,NGC-label]90:{2976}}] at (axis cs:146.825,67.917) {\tikz\pgfuseplotmark{GAL};}; % 10.2
\node[NGC,pin={[pin distance=-1.2\onedegree,NGC-label]90:{2985}}] at (axis cs:147.600,72.283) {\tikz\pgfuseplotmark{GAL};}; % 10.5
\node[NGC,pin={[pin distance=-1.2\onedegree,NGC-label]90:{3079}}] at (axis cs:150.500,55.683) {\tikz\pgfuseplotmark{GAL};}; % 10.6
\node[NGC,pin={[pin distance=-1.2\onedegree,NGC-label]90:{3077}}] at (axis cs:150.825,68.733) {\tikz\pgfuseplotmark{GAL};}; %  9.9
\node[NGC,pin={[pin distance=-1.2\onedegree,NGC-label]90:{3147}}] at (axis cs:154.225,73.400) {\tikz\pgfuseplotmark{GAL};}; % 10.7
\node[NGC,pin={[pin distance=-1.2\onedegree,NGC-label]90:{3184}}] at (axis cs:154.575,41.417) {\tikz\pgfuseplotmark{GAL};}; %  9.8
\node[NGC,pin={[pin distance=-1.2\onedegree,NGC-label]90:{3198}}] at (axis cs:154.975,45.550) {\tikz\pgfuseplotmark{GAL};}; % 10.4
\node[NGC,pin={[pin distance=-1.2\onedegree,NGC-label]90:{3245}}] at (axis cs:156.825,28.500) {\tikz\pgfuseplotmark{GAL};}; % 10.8
\node[NGC,pin={[pin distance=-1.2\onedegree,NGC-label]90:{I2574}}] at (axis cs:157.100,68.417) {\tikz\pgfuseplotmark{GAL};}; % 10.6
\node[NGC,pin={[pin distance=-1.2\onedegree,NGC-label]90:{3310}}] at (axis cs:159.675,53.500) {\tikz\pgfuseplotmark{GAL};}; % 10.9
\node[NGC,pin={[pin distance=-1.2\onedegree,NGC-label]90:{3359}}] at (axis cs:161.650,63.217) {\tikz\pgfuseplotmark{GAL};}; % 10.5
\node[NGC,pin={[pin distance=-1.2\onedegree,NGC-label]90:{3414}}] at (axis cs:162.825,27.983) {\tikz\pgfuseplotmark{GAL};}; % 10.8
\node[NGC,pin={[pin distance=-1.2\onedegree,NGC-label]90:{3486}}] at (axis cs:165.100,28.967) {\tikz\pgfuseplotmark{GAL};}; % 10.3
\node[NGC,pin={[pin distance=-1.2\onedegree,NGC-label]90:{3610}}] at (axis cs:169.600,58.783) {\tikz\pgfuseplotmark{GAL};}; % 10.8
\node[NGC,pin={[pin distance=-1.2\onedegree,NGC-label]90:{3631}}] at (axis cs:170.250,53.167) {\tikz\pgfuseplotmark{GAL};}; % 10.4
\node[NGC,pin={[pin distance=-1.2\onedegree,NGC-label]90:{3665}}] at (axis cs:171.175,38.767) {\tikz\pgfuseplotmark{GAL};}; % 10.8
\node[NGC,pin={[pin distance=-1.2\onedegree,NGC-label]90:{3718}}] at (axis cs:173.150,53.067) {\tikz\pgfuseplotmark{GAL};}; % 10.5
\node[NGC,pin={[pin distance=-1.2\onedegree,NGC-label]90:{3726}}] at (axis cs:173.325,47.033) {\tikz\pgfuseplotmark{GAL};}; % 10.4
\node[NGC,pin={[pin distance=-1.2\onedegree,NGC-label]90:{3898}}] at (axis cs:177.300,56.083) {\tikz\pgfuseplotmark{GAL};}; % 10.8
\node[NGC,pin={[pin distance=-1.2\onedegree,NGC-label]90:{3938}}] at (axis cs:178.200,44.117) {\tikz\pgfuseplotmark{GAL};}; % 10.4
\node[NGC,pin={[pin distance=-1.2\onedegree,NGC-label]90:{3945}}] at (axis cs:178.300,60.683) {\tikz\pgfuseplotmark{GAL};}; % 10.6
\node[NGC,pin={[pin distance=-1.2\onedegree,NGC-label]90:{3953}}] at (axis cs:178.450,52.333) {\tikz\pgfuseplotmark{GAL};}; % 10.1
\node[NGC,pin={[pin distance=-1.2\onedegree,NGC-label]90:{3998}}] at (axis cs:179.475,55.450) {\tikz\pgfuseplotmark{GAL};}; % 10.6
\node[NGC,pin={[pin distance=-1.2\onedegree,NGC-label]90:{4036}}] at (axis cs:180.350,61.900) {\tikz\pgfuseplotmark{GAL};}; % 10.6
\node[NGC,pin={[pin distance=-1.2\onedegree,NGC-label]90:{4051}}] at (axis cs:180.800,44.533) {\tikz\pgfuseplotmark{GAL};}; % 10.3
\node[NGC,pin={[pin distance=-1.2\onedegree,NGC-label]90:{4088}}] at (axis cs:181.400,50.550) {\tikz\pgfuseplotmark{GAL};}; % 10.5
\node[NGC,pin={[pin distance=-1.2\onedegree,NGC-label]90:{4096}}] at (axis cs:181.500,47.483) {\tikz\pgfuseplotmark{GAL};}; % 10.6
\node[NGC,pin={[pin distance=-1.2\onedegree,NGC-label]90:{4111}}] at (axis cs:181.775,43.067) {\tikz\pgfuseplotmark{GAL};}; % 10.8
\node[NGC,pin={[pin distance=-1.2\onedegree,NGC-label]90:{4125}}] at (axis cs:182.025,65.183) {\tikz\pgfuseplotmark{GAL};}; %  9.8
\node[NGC,pin={[pin distance=-1.2\onedegree,NGC-label]90:{4151}}] at (axis cs:182.625,39.400) {\tikz\pgfuseplotmark{GAL};}; % 10.4
\node[NGC,pin={[pin distance=-1.2\onedegree,NGC-label]90:{4203}}] at (axis cs:183.775,33.200) {\tikz\pgfuseplotmark{GAL};}; % 10.7
\node[NGC,pin={[pin distance=-1.2\onedegree,NGC-label]90:{4214}}] at (axis cs:183.900,36.333) {\tikz\pgfuseplotmark{GAL};}; %  9.7
\node[NGC,pin={[pin distance=-1.2\onedegree,NGC-label]90:{4236}}] at (axis cs:184.175,69.467) {\tikz\pgfuseplotmark{GAL};}; %  9.7
\node[NGC,pin={[pin distance=-1.2\onedegree,NGC-label]90:{4244}}] at (axis cs:184.375,37.817) {\tikz\pgfuseplotmark{GAL};}; % 10.2
\node[NGC,pin={[pin distance=-1.2\onedegree,NGC-label]90:{4274}}] at (axis cs:184.950,29.617) {\tikz\pgfuseplotmark{GAL};}; % 10.4
\node[NGC,pin={[pin distance=-1.2\onedegree,NGC-label]90:{4278}}] at (axis cs:185.025,29.283) {\tikz\pgfuseplotmark{GAL};}; % 10.2
\node[NGC,pin={[pin distance=-1.2\onedegree,NGC-label]90:{4314}}] at (axis cs:185.650,29.883) {\tikz\pgfuseplotmark{GAL};}; % 10.5
\node[NGC,pin={[pin distance=-1.2\onedegree,NGC-label]90:{4395}}] at (axis cs:186.450,33.550) {\tikz\pgfuseplotmark{GAL};}; % 10.2
\node[NGC,pin={[pin distance=-1.2\onedegree,NGC-label]90:{4414}}] at (axis cs:186.600,31.217) {\tikz\pgfuseplotmark{GAL};}; % 10.3
\node[NGC,pin={[pin distance=-1.2\onedegree,NGC-label]90:{4449}}] at (axis cs:187.050,44.100) {\tikz\pgfuseplotmark{GAL};}; %  9.4
\node[NGC,pin={[pin distance=-1.2\onedegree,NGC-label]90:{4490}}] at (axis cs:187.650,41.633) {\tikz\pgfuseplotmark{GAL};}; %  9.8
\node[NGC,pin={[pin distance=-1.2\onedegree,NGC-label]90:{4559}}] at (axis cs:189,27.967) {\tikz\pgfuseplotmark{GAL};}; %  9.9
\node[NGC,pin={[pin distance=-1.2\onedegree,NGC-label]90:{4618}}] at (axis cs:190.375,41.150) {\tikz\pgfuseplotmark{GAL};}; % 10.8
\node[NGC,pin={[pin distance=-1.2\onedegree,NGC-label]90:{4631}}] at (axis cs:190.525,32.533) {\tikz\pgfuseplotmark{GAL};}; %  9.3
\node[NGC,pin={[pin distance=-1.2\onedegree,NGC-label]90:{4656}}] at (axis cs:191,32.167) {\tikz\pgfuseplotmark{GAL};}; % 10.4
\node[NGC,pin={[pin distance=-1.2\onedegree,NGC-label]90:{5005}}] at (axis cs:197.725,37.050) {\tikz\pgfuseplotmark{GAL};}; %  9.8
\node[NGC,pin={[pin distance=-1.2\onedegree,NGC-label]90:{5033}}] at (axis cs:198.350,36.600) {\tikz\pgfuseplotmark{GAL};}; % 10.1
\node[NGC,pin={[pin distance=-1.2\onedegree,NGC-label]90:{5322}}] at (axis cs:207.325,60.200) {\tikz\pgfuseplotmark{GAL};}; % 10.0
\node[NGC,pin={[pin distance=-1.2\onedegree,NGC-label]90:{5371}}] at (axis cs:208.925,40.467) {\tikz\pgfuseplotmark{GAL};}; % 10.8
\node[NGC,pin={[pin distance=-1.2\onedegree,NGC-label]90:{5474}}] at (axis cs:211.250,53.667) {\tikz\pgfuseplotmark{GAL};}; % 10.9
\node[NGC,pin={[pin distance=-1.2\onedegree,NGC-label]90:{5585}}] at (axis cs:214.950,56.733) {\tikz\pgfuseplotmark{GAL};}; % 10.9
\node[NGC,pin={[pin distance=-1.2\onedegree,NGC-label]90:{5676}}] at (axis cs:218.200,49.467) {\tikz\pgfuseplotmark{GAL};}; % 10.9
\node[NGC,pin={[pin distance=-1.2\onedegree,NGC-label]90:{5907}}] at (axis cs:228.975,56.317) {\tikz\pgfuseplotmark{GAL};}; % 10.4
\node[NGC,pin={[pin distance=-1.2\onedegree,NGC-label]90:{6503}}] at (axis cs:267.350,70.150) {\tikz\pgfuseplotmark{GAL};}; % 10.2
\node[NGC,pin={[pin distance=-1.2\onedegree,NGC-label]0:{6946}}] at (axis cs:308.700,60.150) {\tikz\pgfuseplotmark{GAL};}; %  8.9
\node[NGC,pin={[pin distance=-1.2\onedegree,NGC-label]90:{7217}}] at (axis cs:331.975,31.367) {\tikz\pgfuseplotmark{GAL};}; % 10.2
\node[NGC,pin={[pin distance=-1.2\onedegree,NGC-label]90:{7331}}] at (axis cs:339.275,34.417) {\tikz\pgfuseplotmark{GAL};}; %  9.5
\node[NGC,pin={[pin distance=-1.2\onedegree,NGC-label]90:{7457}}] at (axis cs:345.250,30.150) {\tikz\pgfuseplotmark{GAL};}; % 10.8
\node[NGC,pin={[pin distance=-1.2\onedegree,NGC-label]0:{7640}}] at (axis cs:350.525,40.850) {\tikz\pgfuseplotmark{GAL};}; % 10.9
\node[NGC,pin={[pin distance=-1.2\onedegree,NGC-label]90:{1514}}] at (axis cs:62.300,30.783) {\tikz\pgfuseplotmark{PN};}; % 10. 
\node[NGC,pin={[pin distance=-1.2\onedegree,NGC-label]90:{6543}}] at (axis cs:269.650,66.633) {\tikz\pgfuseplotmark{PN};}; %  9. 
\node[NGC,pin={[pin distance=-1.2\onedegree,NGC-label]-90:{6826}}] at (axis cs:296.200,50.517) {\tikz\pgfuseplotmark{PN};}; % 10. 
\node[NGC,pin={[pin distance=-1.2\onedegree,NGC-label]90:{7027}}] at (axis cs:316.775,42.233) {\tikz\pgfuseplotmark{PN};}; % 10. 
\node[NGC,pin={[pin distance=-1.2\onedegree,NGC-label]-90:{7662}}] at (axis cs:351.475,42.550) {\tikz\pgfuseplotmark{PN};}; %  9. 
\node[NGC,pin={[pin distance=-1.2\onedegree,NGC-label]90:{281}}] at (axis cs:13.200,56.617) {\tikz\pgfuseplotmark{CN};}; %  7. 
\node[NGC,pin={[pin distance=-1.2\onedegree,NGC-label]0:{I1805}}] at (axis cs:38.175,61.450) {\tikz\pgfuseplotmark{CN};}; %  6.5
\node[NGC,pin={[pin distance=-1.2\onedegree,NGC-label]90:{I1848}}] at (axis cs:42.800,60.433) {\tikz\pgfuseplotmark{CN};}; %  6.5
\node[NGC,pin={[pin distance=-1.2\onedegree,NGC-label]90:{I 348}}] at (axis cs:56.125,32.283) {\tikz\pgfuseplotmark{CN};}; %  7.3
\node[NGC,pin={[pin distance=-1.2\onedegree,NGC-label]90:{1624}}] at (axis cs:70.100,50.450) {\tikz\pgfuseplotmark{CN};}; % 10.4
\node[NGC,pin={[pin distance=-1.2\onedegree,NGC-label]90:{7023}}] at (axis cs:315.125,68.167) {\tikz\pgfuseplotmark{CN};}; %  7. 
\node[NGC,pin={[pin distance=-1.2\onedegree,NGC-label]90:{I1396}}] at (axis cs:324.775,57.500) {\tikz\pgfuseplotmark{CN};}; %  3.5
\node[NGC,pin={[pin distance=-1.2\onedegree,NGC-label]90:{I5146}}] at (axis cs:328.350,47.267) {\tikz\pgfuseplotmark{CN};}; %  7.2
\node[NGC,pin={[pin distance=-1.2\onedegree,NGC-label]90:{7380}}] at (axis cs:341.750,58.100) {\tikz\pgfuseplotmark{CN};}; %  7.2
\node[NGC,pin={[pin distance=-1.2\onedegree,NGC-label]0:{103}}] at (axis cs:6.325,61.350) {\tikz\pgfuseplotmark{OC};}; %  9.8
\node[NGC,pin={[pin distance=-1.2\onedegree,NGC-label]90:{129}}] at (axis cs:7.475,60.233) {\tikz\pgfuseplotmark{OC};}; %  6.5
\node[NGC,pin={[pin distance=-1.2\onedegree,NGC-label]0:{133}}] at (axis cs:7.800,63.367) {\tikz\pgfuseplotmark{OC};}; %  9. 
\node[NGC,pin={[pin distance=-1.2\onedegree,NGC-label]180:{146}}] at (axis cs:8.275,63.300) {\tikz\pgfuseplotmark{OC};}; %  9.1
\node[NGC,pin={[pin distance=-1.2\onedegree,NGC-label]180:{189}}] at (axis cs:9.900,61.067) {\tikz\pgfuseplotmark{OC};}; %  8.8
\node[NGC,pin={[pin distance=-1.2\onedegree,NGC-label]90:{225}}] at (axis cs:10.850,61.783) {\tikz\pgfuseplotmark{OC};}; %  7.0
\node[NGC,pin={[pin distance=-1.2\onedegree,NGC-label]-90:{188}}] at (axis cs:11,85.333) {\tikz\pgfuseplotmark{OC};}; %  8.1
\node[NGC,pin={[pin distance=-1.2\onedegree,NGC-label]90:{381}}] at (axis cs:17.075,61.583) {\tikz\pgfuseplotmark{OC};}; %  9. 
\node[NGC,pin={[pin distance=-1.2\onedegree,NGC-label]90:{436}}] at (axis cs:18.900,58.817) {\tikz\pgfuseplotmark{OC};}; %  8.8
\node[NGC,pin={[pin distance=-1.2\onedegree,NGC-label]0:{457}}] at (axis cs:19.775,58.333) {\tikz\pgfuseplotmark{OC};}; %  6.4
\node[NGC,pin={[pin distance=-1.2\onedegree,NGC-label]90:{559}}] at (axis cs:22.375,63.300) {\tikz\pgfuseplotmark{OC};}; %  9.5
\node[NGC,pin={[pin distance=-1.2\onedegree,NGC-label]90:{637}}] at (axis cs:25.725,64) {\tikz\pgfuseplotmark{OC};}; %  8.2
\node[NGC,pin={[pin distance=-1.2\onedegree,NGC-label]180:{654}}] at (axis cs:26.025,61.883) {\tikz\pgfuseplotmark{OC};}; %  6.5
\node[NGC,pin={[pin distance=-1.2\onedegree,NGC-label]180:{659}}] at (axis cs:26.050,60.700) {\tikz\pgfuseplotmark{OC};}; %  7.9
\node[NGC,pin={[pin distance=-1.2\onedegree,NGC-label]180:{663}}] at (axis cs:26.500,61.250) {\tikz\pgfuseplotmark{OC};}; %  7.1
\node[NGC,pin={[pin distance=-1.2\onedegree,NGC-label]90:{752}}] at (axis cs:29.450,37.683) {\tikz\pgfuseplotmark{OC};}; %  5.7
\node[NGC,pin={[pin distance=-1.2\onedegree,NGC-label]90:{744}}] at (axis cs:29.600,55.483) {\tikz\pgfuseplotmark{OC};}; %  7.9
\node[NGC,pin={[pin distance=-1.2\onedegree,NGC-label]-90:{869}}] at (axis cs:34.750,57.150) {\tikz\pgfuseplotmark{OC};}; %  4. 
\node[NGC,pin={[pin distance=-1.2\onedegree,NGC-label]90:{884}}] at (axis cs:35.600,57.117) {\tikz\pgfuseplotmark{OC};}; %  4. 
\node[NGC,pin={[pin distance=-1.2\onedegree,NGC-label]90:{956}}] at (axis cs:38.100,44.650) {\tikz\pgfuseplotmark{OC};}; %  9. 
\node[NGC,pin={[pin distance=-1.2\onedegree,NGC-label]0:{957}}] at (axis cs:38.400,57.533) {\tikz\pgfuseplotmark{OC};}; %  7.6
\node[NGC,pin={[pin distance=-1.2\onedegree,NGC-label]90:{1027}}] at (axis cs:40.675,61.550) {\tikz\pgfuseplotmark{OC};}; %  6.7
\node[NGC,pin={[pin distance=-1.2\onedegree,NGC-label]-90:{1245}}] at (axis cs:48.675,47.250) {\tikz\pgfuseplotmark{OC};}; %  8.4
\node[NGC,pin={[pin distance=-1.2\onedegree,NGC-label]90:{1342}}] at (axis cs:52.900,37.333) {\tikz\pgfuseplotmark{OC};}; %  6.7
\node[NGC,pin={[pin distance=-1.2\onedegree,NGC-label]90:{1444}}] at (axis cs:57.350,52.667) {\tikz\pgfuseplotmark{OC};}; %  6.6
\node[NGC,pin={[pin distance=-1.2\onedegree,NGC-label]90:{1496}}] at (axis cs:61.100,52.617) {\tikz\pgfuseplotmark{OC};}; % 10. 
\node[NGC,pin={[pin distance=-1.2\onedegree,NGC-label]90:{1502}}] at (axis cs:61.925,62.333) {\tikz\pgfuseplotmark{OC};}; %  5.7
\node[NGC,pin={[pin distance=-1.2\onedegree,NGC-label]90:{1513}}] at (axis cs:62.500,49.517) {\tikz\pgfuseplotmark{OC};}; %  8.4
\node[NGC,pin={[pin distance=-1.2\onedegree,NGC-label]90:{1528}}] at (axis cs:63.850,51.233) {\tikz\pgfuseplotmark{OC};}; %  6.4
\node[NGC,pin={[pin distance=-1.2\onedegree,NGC-label]90:{1545}}] at (axis cs:65.225,50.250) {\tikz\pgfuseplotmark{OC};}; %  6.2
\node[NGC,pin={[pin distance=-1.2\onedegree,NGC-label]90:{1582}}] at (axis cs:68,43.850) {\tikz\pgfuseplotmark{OC};}; %  7. 
\node[NGC,pin={[pin distance=-1.2\onedegree,NGC-label]90:{1605}}] at (axis cs:68.750,45.250) {\tikz\pgfuseplotmark{OC};}; % 10.7
\node[NGC,pin={[pin distance=-1.2\onedegree,NGC-label]90:{1664}}] at (axis cs:72.775,43.700) {\tikz\pgfuseplotmark{OC};}; %  7.6
\node[NGC,pin={[pin distance=-1.2\onedegree,NGC-label]90:{1724}}] at (axis cs:75.875,49.500) {\tikz\pgfuseplotmark{OC};}; % 10. 
\node[NGC,pin={[pin distance=-1.2\onedegree,NGC-label]90:{1778}}] at (axis cs:77.025,37.050) {\tikz\pgfuseplotmark{OC};}; %  7.7
\node[NGC,pin={[pin distance=-1.2\onedegree,NGC-label]90:{1857}}] at (axis cs:80.050,39.350) {\tikz\pgfuseplotmark{OC};}; %  7.0
\node[NGC,pin={[pin distance=-1.2\onedegree,NGC-label]90:{1893}}] at (axis cs:80.675,33.400) {\tikz\pgfuseplotmark{OC};}; %  7.5
\node[NGC,pin={[pin distance=-1.2\onedegree,NGC-label]-90:{1907}}] at (axis cs:82,35.317) {\tikz\pgfuseplotmark{OC};}; %  8.2
\node[NGC,pin={[pin distance=-1.2\onedegree,NGC-label]90:{2126}}] at (axis cs:90.750,49.900) {\tikz\pgfuseplotmark{OC};}; % 10. 
\node[NGC,pin={[pin distance=-1.2\onedegree,NGC-label]90:{2281}}] at (axis cs:102.325,41.067) {\tikz\pgfuseplotmark{OC};}; %  5.4
\node[NGC,pin={[pin distance=-1.2\onedegree,NGC-label]0:{2331}}] at (axis cs:106.800,27.350) {\tikz\pgfuseplotmark{OC};}; %  9. 
\node[NGC,pin={[pin distance=-1.2\onedegree,NGC-label]-90:{6791}}] at (axis cs:290.175,37.850) {\tikz\pgfuseplotmark{OC};}; %  9.5
\node[NGC,pin={[pin distance=-1.2\onedegree,NGC-label]90:{6811}}] at (axis cs:294.550,46.567) {\tikz\pgfuseplotmark{OC};}; %  6.8
\node[NGC,pin={[pin distance=-1.2\onedegree,NGC-label]90:{6819}}] at (axis cs:295.325,40.183) {\tikz\pgfuseplotmark{OC};}; %  7.3
\node[NGC,pin={[pin distance=-1.2\onedegree,NGC-label]90:{6834}}] at (axis cs:298.050,29.417) {\tikz\pgfuseplotmark{OC};}; %  7.8
\node[NGC,pin={[pin distance=-1.2\onedegree,NGC-label]90:{6866}}] at (axis cs:300.925,44) {\tikz\pgfuseplotmark{OC};}; %  7.6
\node[NGC,pin={[pin distance=-1.2\onedegree,NGC-label]0:{6871}}] at (axis cs:301.475,35.783) {\tikz\pgfuseplotmark{OC};}; %  5.2
\node[NGC,pin={[pin distance=-1.2\onedegree,NGC-label]-90:{6883}}] at (axis cs:302.825,35.850) {\tikz\pgfuseplotmark{OC};}; %  8. 
\node[NGC,pin={[pin distance=-1.2\onedegree,NGC-label]90:{I4996}}] at (axis cs:304.125,37.633) {\tikz\pgfuseplotmark{OC};}; %  7.3
\node[NGC,pin={[pin distance=-1.2\onedegree,NGC-label]90:{6910}}] at (axis cs:305.775,40.783) {\tikz\pgfuseplotmark{OC};}; %  7.4
\node[NGC,pin={[pin distance=-1.2\onedegree,NGC-label]90:{6939}}] at (axis cs:307.850,60.633) {\tikz\pgfuseplotmark{OC};}; %  7.8
\node[NGC,pin={[pin distance=-1.2\onedegree,NGC-label]90:{6940}}] at (axis cs:308.650,28.300) {\tikz\pgfuseplotmark{OC};}; %  6.3
\node[NGC,pin={[pin distance=-1.2\onedegree,NGC-label]-90:{6997}}] at (axis cs:314.125,44.633) {\tikz\pgfuseplotmark{OC};}; % 10. 
\node[NGC,pin={[pin distance=-1.2\onedegree,NGC-label]90:{7031}}] at (axis cs:316.825,50.833) {\tikz\pgfuseplotmark{OC};}; %  9.1
\node[NGC,pin={[pin distance=-1.2\onedegree,NGC-label]90:{7039}}] at (axis cs:317.800,45.650) {\tikz\pgfuseplotmark{OC};}; %  7.6
\node[NGC,pin={[pin distance=-1.2\onedegree,NGC-label]-90:{I1369}}] at (axis cs:318.025,47.733) {\tikz\pgfuseplotmark{OC};}; %  6.8
\node[NGC,pin={[pin distance=-1.2\onedegree,NGC-label]90:{7062}}] at (axis cs:320.800,46.383) {\tikz\pgfuseplotmark{OC};}; %  8.3
\node[NGC,pin={[pin distance=-1.2\onedegree,NGC-label]90:{7067}}] at (axis cs:321.050,48.017) {\tikz\pgfuseplotmark{OC};}; %  9.7
\node[NGC,pin={[pin distance=-1.2\onedegree,NGC-label]0:{7063}}] at (axis cs:321.100,36.500) {\tikz\pgfuseplotmark{OC};}; %  7.0
\node[NGC,pin={[pin distance=-1.2\onedegree,NGC-label]90:{7082}}] at (axis cs:322.350,47.083) {\tikz\pgfuseplotmark{OC};}; %  7.2
\node[NGC,pin={[pin distance=-1.2\onedegree,NGC-label]90:{7086}}] at (axis cs:322.625,51.583) {\tikz\pgfuseplotmark{OC};}; %  8.4
\node[NGC,pin={[pin distance=-1.2\onedegree,NGC-label]90:{7128}}] at (axis cs:326,53.717) {\tikz\pgfuseplotmark{OC};}; %  9.7
\node[NGC,pin={[pin distance=-1.2\onedegree,NGC-label]90:{7142}}] at (axis cs:326.475,65.800) {\tikz\pgfuseplotmark{OC};}; %  9.3
\node[NGC,pin={[pin distance=-1.2\onedegree,NGC-label]90:{7160}}] at (axis cs:328.425,62.600) {\tikz\pgfuseplotmark{OC};}; %  6.1
\node[NGC,pin={[pin distance=-1.2\onedegree,NGC-label]90:{7209}}] at (axis cs:331.300,46.500) {\tikz\pgfuseplotmark{OC};}; %  6.7
\node[NGC,pin={[pin distance=-1.2\onedegree,NGC-label]90:{7226}}] at (axis cs:332.625,55.417) {\tikz\pgfuseplotmark{OC};}; %  9.6
\node[NGC,pin={[pin distance=-1.2\onedegree,NGC-label]-90:{I1434}}] at (axis cs:332.625,52.833) {\tikz\pgfuseplotmark{OC};}; %  9. 
\node[NGC,pin={[pin distance=-1.2\onedegree,NGC-label]0:{7235}}] at (axis cs:333.150,57.283) {\tikz\pgfuseplotmark{OC};}; %  7.7
\node[NGC,pin={[pin distance=-1.2\onedegree,NGC-label]90:{7243}}] at (axis cs:333.825,49.883) {\tikz\pgfuseplotmark{OC};}; %  6.4
\node[NGC,pin={[pin distance=-1.2\onedegree,NGC-label]180:{7245}}] at (axis cs:333.825,54.333) {\tikz\pgfuseplotmark{OC};}; %  9.2
\node[NGC,pin={[pin distance=-1.2\onedegree,NGC-label]-90:{I1442}}] at (axis cs:334.125,54.050) {\tikz\pgfuseplotmark{OC};}; %  9.1
\node[NGC,pin={[pin distance=-1.2\onedegree,NGC-label]90:{7261}}] at (axis cs:335.100,58.083) {\tikz\pgfuseplotmark{OC};}; %  8.4
\node[NGC,pin={[pin distance=-1.2\onedegree,NGC-label]90:{7296}}] at (axis cs:337.050,52.283) {\tikz\pgfuseplotmark{OC};}; % 10. 
\node[NGC,pin={[pin distance=-1.2\onedegree,NGC-label]90:{7510}}] at (axis cs:347.875,60.567) {\tikz\pgfuseplotmark{OC};}; %  7.9
\node[NGC,pin={[pin distance=-1.2\onedegree,NGC-label]90:{7686}}] at (axis cs:352.550,49.133) {\tikz\pgfuseplotmark{OC};}; %  5.6
\node[NGC,pin={[pin distance=-1.2\onedegree,NGC-label]90:{7762}}] at (axis cs:357.450,68.033) {\tikz\pgfuseplotmark{OC};}; % 10. 
\node[NGC,pin={[pin distance=-1.2\onedegree,NGC-label]0:{7788/90}}] at (axis cs:359.175,61.400) {\tikz\pgfuseplotmark{OC};}; %  9. 
\node[NGC,pin={[pin distance=-1.2\onedegree,NGC-label]-90:{7789}}] at (axis cs:359.250,56.733) {\tikz\pgfuseplotmark{OC};}; %  6.7
%\node[NGC,pin={[pin distance=-1.2\onedegree,NGC-label]0:{7790}}] at (axis cs:359.600,61.217) {\tikz\pgfuseplotmark{OC};}; %  8.5
\node[NGC,pin={[pin distance=-1.2\onedegree,NGC-label]90:{2419}}] at (axis cs:114.525,38.883) {\tikz\pgfuseplotmark{GC};}; % 10.4
\node[NGC,pin={[pin distance=-1.2\onedegree,NGC-label]90:{5466}}] at (axis cs:211.375,28.533) {\tikz\pgfuseplotmark{GC};}; %  9.1
\node[NGC,pin={[pin distance=-1.2\onedegree,NGC-label]90:{6229}}] at (axis cs:251.750,47.533) {\tikz\pgfuseplotmark{GC};}; %  9.4
\end{polaraxis}

% Stars and star names
\input{./input/I_Stars.tex}
\begin{polaraxis}[rotate=90,name=stars,at=(base.center),anchor=center,axis lines=none]
\boldmath

  \clip (0\tendegree,-6\tendegree) arc (270:90:6\tendegree)
  -- (2\tendegree,6\tendegree)  -- (2\tendegree,-6\tendegree)
   -- cycle ;
\node[designation-label,anchor=south west]  at (axis cs:{0},{+89+30/ 60  })  {Polaris};   
\node[designation-label,anchor=south west]  at (axis cs:{18*15+36/4},{+39+46/ 60  })  {Vega};   
\node[designation-label,anchor=south west]  at (axis cs:{20*15+41/4},{+45+16/ 60  })  {Deneb};   
  
\node[designation-label,anchor=south west]  at (axis cs:{03*15+28/4},{+50+15/ 60  })  {Algenib};   
\node[designation-label,anchor=south west]  at (axis cs:{05*15+19/4},{+46+50/ 60  })  {Capella};   
\node[designation-label,anchor=south west]  at (axis cs:{06*15+01/4},{+45+16/ 60  })  {Menkalinan};   



\node[ecliptics-empty,pin={[pin distance=-0.4\onedegree,interest-label]-90:{NEP}}] at (axis cs:{18*15},{+66+33/ 60  }) {\pgfuseplotmark{+}} ; %  ecliptic pole
\node[ecliptics-empty,pin={[pin distance=-0.4\onedegree,interest-label]-90:{SEP}}] at (axis cs:{5*15},{-66-33/ 60  }) {\pgfuseplotmark{+}} ; %  ecliptic pole

\node[interest,pin={[pin distance=-0.4\onedegree,interest-label]90:{RR}}] at (axis cs:291.3663040022100,+42.7843592383900) {\pgfuseplotmark{+}} ; %  RR Lyr

\node[pin={[pin distance=-0.4\onedegree,Bayer]00:{$\boldsymbol\alpha$}}] at (axis cs:{2.097},{29.090}) {}; % And,  2.06 
\node[pin={[pin distance=-0.2\onedegree,Bayer]00:{$\boldsymbol\beta$}}] at (axis cs:{17.433},{35.620}) {}; % And,  2.08 
\node[pin={[pin distance=-0.4\onedegree,Bayer]90:{$\boldsymbol\gamma^1$}}] at (axis cs:{30.975},{42.330}) {}; % And,  2.17 
\node[pin={[pin distance=-0.6\onedegree,Bayer]00:{$\boldsymbol\chi$}}] at (axis cs:{24.838},{44.386}) {}; % And,  5.01 
\node[pin={[pin distance=-0.4\onedegree,Bayer]00:{$\boldsymbol\delta$}}] at (axis cs:{9.832},{30.861}) {}; % And,  3.27 
\node[pin={[pin distance=-0.6\onedegree,Bayer]00:{$\boldsymbol\epsilon$}}] at (axis cs:{9.639},{29.312}) {}; % And,  4.35 
\node[pin={[pin distance=-0.6\onedegree,Bayer]00:{$\boldsymbol\iota$}}] at (axis cs:{354.534},{43.268}) {}; % And,  4.29 
\node[pin={[pin distance=-0.6\onedegree,Bayer]180:{$\boldsymbol\kappa$}}] at (axis cs:{355.102},{44.334}) {}; % And,  4.15 
\node[pin={[pin distance=-0.6\onedegree,Bayer]00:{$\boldsymbol\lambda$}}] at (axis cs:{354.391},{46.458}) {}; % And,  3.87 
\node[pin={[pin distance=-0.4\onedegree,Bayer]00:{$\boldsymbol\mu$}}] at (axis cs:{14.188},{38.499}) {}; % And,  3.87 
\node[pin={[pin distance=-0.6\onedegree,Bayer]180:{$\boldsymbol\nu$}}] at (axis cs:{12.454},{41.079}) {}; % And,  4.54 
\node[pin={[pin distance=-0.6\onedegree,Bayer]90:{$\boldsymbol\omega$}}] at (axis cs:{21.914},{45.407}) {}; % And,  4.83 
\node[pin={[pin distance=-0.4\onedegree,Bayer]-90:{$\boldsymbol\omicron$}}] at (axis cs:{345.480},{42.326}) {}; % And,  3.64 
\node[pin={[pin distance=-0.4\onedegree,Bayer]00:{$\boldsymbol\pi$}}] at (axis cs:{9.220},{33.719}) {}; % And,  4.35 
\node[pin={[pin distance=-0.6\onedegree,Bayer]00:{$\boldsymbol\psi$}}] at (axis cs:{356.509},{46.420}) {}; % And,  4.98 
\node[pin={[pin distance=-0.6\onedegree,Bayer]00:{$\boldsymbol\rho$}}] at (axis cs:{5.280},{37.969}) {}; % And,  5.16 
\node[pin={[pin distance=-0.6\onedegree,Bayer]00:{$\boldsymbol\sigma$}}] at (axis cs:{4.582},{36.785}) {}; % And,  4.51 
\node[pin={[pin distance=-0.6\onedegree,Bayer]00:{$\boldsymbol\tau$}}] at (axis cs:{25.145},{40.577}) {}; % And,  4.97 
\node[pin={[pin distance=-0.6\onedegree,Bayer]00:{$\boldsymbol\upsilon$}}] at (axis cs:{24.199},{41.405}) {}; % And,  4.10 
\node[pin={[pin distance=-0.6\onedegree,Bayer]00:{$\boldsymbol\varphi$}}] at (axis cs:{17.376},{47.242}) {}; % And,  4.26 
\node[pin={[pin distance=-0.6\onedegree,Bayer]00:{$\boldsymbol\vartheta$}}] at (axis cs:{4.273},{38.681}) {}; % And,  4.62 
\node[pin={[pin distance=-0.6\onedegree,Bayer]00:{$\boldsymbol\xi$}}] at (axis cs:{20.585},{45.529}) {}; % And,  4.87 
\node[pin={[pin distance=-0.6\onedegree,Flaamsted]00:{10}}] at (axis cs:{349.968},{42.078}) {}; % And,  5.79 
\node[pin={[pin distance=-0.6\onedegree,Flaamsted]00:{11}}] at (axis cs:{349.874},{48.625}) {}; % And,  5.45 
\node[pin={[pin distance=-0.6\onedegree,Flaamsted]00:{12}}] at (axis cs:{350.222},{38.182}) {}; % And,  5.78 
\node[pin={[pin distance=-0.6\onedegree,Flaamsted]90:{13}}] at (axis cs:{351.781},{42.912}) {}; % And,  5.75 
\node[pin={[pin distance=-0.6\onedegree,Flaamsted]00:{14}}] at (axis cs:{352.823},{39.236}) {}; % And,  5.23 
\node[pin={[pin distance=-0.6\onedegree,Flaamsted]00:{15}}] at (axis cs:{353.656},{40.236}) {}; % And,  5.56 
\node[pin={[pin distance=-0.6\onedegree,Flaamsted]00:{18}}] at (axis cs:{354.785},{50.472}) {}; % And,  5.35 
\node[pin={[pin distance=-0.6\onedegree,Flaamsted]90:{ 2}}] at (axis cs:{345.652},{42.758}) {}; % And,  5.09 
\node[pin={[pin distance=-0.6\onedegree,Flaamsted]00:{22}}] at (axis cs:{2.580},{46.072}) {}; % And,  5.03 
\node[pin={[pin distance=-0.6\onedegree,Flaamsted]-90:{23}}] at (axis cs:{3.379},{41.035}) {}; % And,  5.72 
\node[pin={[pin distance=-0.6\onedegree,Flaamsted]90:{26}}] at (axis cs:{4.676},{43.791}) {}; % And,  6.11 
\node[pin={[pin distance=-0.6\onedegree,Flaamsted]00:{28}}] at (axis cs:{7.531},{29.752}) {}; % And,  5.22 
\node[pin={[pin distance=-0.6\onedegree,Flaamsted]90:{ 3}}] at (axis cs:{346.046},{50.052}) {}; % And,  4.65 
\node[pin={[pin distance=-0.6\onedegree,Flaamsted]-90:{32}}] at (axis cs:{10.280},{39.459}) {}; % And,  5.31 
\node[pin={[pin distance=-0.6\onedegree,Flaamsted]00:{39}}] at (axis cs:{15.726},{41.345}) {}; % And,  5.94 
\node[pin={[pin distance=-0.6\onedegree,Flaamsted]00:{ 4}}] at (axis cs:{346.914},{46.387}) {}; % And,  5.30 
\node[pin={[pin distance=-0.6\onedegree,Flaamsted]00:{41}}] at (axis cs:{17.004},{43.942}) {}; % And,  5.04 
\node[pin={[pin distance=-0.6\onedegree,Flaamsted]00:{44}}] at (axis cs:{17.578},{42.081}) {}; % And,  5.67 
\node[pin={[pin distance=-0.6\onedegree,Flaamsted]00:{45}}] at (axis cs:{17.793},{37.724}) {}; % And,  5.79 
\node[pin={[pin distance=-0.6\onedegree,Flaamsted]-90:{47}}] at (axis cs:{20.919},{37.715}) {}; % And,  5.59 
\node[pin={[pin distance=-0.6\onedegree,Flaamsted]00:{49}}] at (axis cs:{22.525},{47.007}) {}; % And,  5.26 
\node[pin={[pin distance=-0.6\onedegree,Flaamsted]00:{ 5}}] at (axis cs:{346.939},{49.296}) {}; % And,  5.69 
\node[pin={[pin distance=-0.3\onedegree,Flaamsted]00:{51}}] at (axis cs:{24.498},{48.628}) {}; % And,  3.59 
\node[pin={[pin distance=-0.6\onedegree,Flaamsted]00:{55}}] at (axis cs:{28.322},{40.730}) {}; % And,  5.40 
\node[pin={[pin distance=-0.6\onedegree,Flaamsted]00:{56}}] at (axis cs:{29.039},{37.252}) {}; % And,  5.69 
\node[pin={[pin distance=-0.6\onedegree,Flaamsted]00:{58}}] at (axis cs:{32.122},{37.859}) {}; % And,  4.79 
\node[pin={[pin distance=-0.6\onedegree,Flaamsted]00:{59}}] at (axis cs:{32.720},{39.039}) {}; % And,  6.08 
\node[pin={[pin distance=-0.6\onedegree,Flaamsted]00:{ 6}}] at (axis cs:{347.613},{43.544}) {}; % And,  5.91 
\node[pin={[pin distance=-0.6\onedegree,Flaamsted]00:{60}}] at (axis cs:{33.306},{44.231}) {}; % And,  4.83 
\node[pin={[pin distance=-0.6\onedegree,Flaamsted]00:{62}}] at (axis cs:{34.820},{47.380}) {}; % And,  5.32 
\node[pin={[pin distance=-0.6\onedegree,Flaamsted]-90:{63}}] at (axis cs:{35.243},{50.151}) {}; % And,  5.57 
\node[pin={[pin distance=-0.6\onedegree,Flaamsted]180:{64}}] at (axis cs:{36.104},{50.006}) {}; % And,  5.19 
\node[pin={[pin distance=-0.6\onedegree,Flaamsted]00:{65}}] at (axis cs:{36.406},{50.278}) {}; % And,  4.72 
\node[pin={[pin distance=-0.6\onedegree,Flaamsted]180:{66}}] at (axis cs:{36.966},{50.570}) {}; % And,  6.17 
\node[pin={[pin distance=-0.6\onedegree,Flaamsted]00:{ 7}}] at (axis cs:{348.138},{49.406}) {}; % And,  4.54 
\node[pin={[pin distance=-0.6\onedegree,Flaamsted]90:{ 8}}] at (axis cs:{349.436},{49.015}) {}; % And,  4.83 
\node[pin={[pin distance=-0.6\onedegree,Flaamsted]-90:{ 9}}] at (axis cs:{349.597},{41.774}) {}; % And,  5.99 
\node[pin={[pin distance=-0.6\onedegree,Flaamsted]00:{33}}] at (axis cs:{40.171},{27.061}) {}; % Ari,  5.30 
\node[pin={[pin distance=-0.6\onedegree,Flaamsted]00:{35}}] at (axis cs:{40.863},{27.707}) {}; % Ari,  4.65 
\node[pin={[pin distance=-0.6\onedegree,Flaamsted]00:{39}}] at (axis cs:{41.977},{29.247}) {}; % Ari,  4.52 
\node[pin={[pin distance=-0.6\onedegree,Flaamsted]00:{41}}] at (axis cs:{42.496},{27.260}) {}; % Ari,  3.62 
\node[pin={[pin distance=-0.6\onedegree,Flaamsted]00:{55}}] at (axis cs:{47.403},{29.077}) {}; % Ari,  5.75 
\node[pin={[pin distance=-0.6\onedegree,Flaamsted]00:{56}}] at (axis cs:{48.059},{27.257}) {}; % Ari,  5.78 
\node[pin={[pin distance=-0.6\onedegree,Flaamsted]00:{59}}] at (axis cs:{49.982},{27.071}) {}; % Ari,  5.92 
\node[pin={[pin distance=-0.6\onedegree,Flaamsted]00:{62}}] at (axis cs:{50.550},{27.607}) {}; % Ari,  5.55 
\node[pin={[pin distance=-0.1\onedegree,Bayer]00:{$\boldsymbol\alpha$}}] at (axis cs:{79.172},{45.998}) {}; % Aur,  0.08 
\node[pin={[pin distance=-0.1\onedegree,Bayer]-90:{$\boldsymbol\beta$}}] at (axis cs:{89.882},{44.947}) {}; % Aur,  1.90 
\node[pin={[pin distance=-0.6\onedegree,Bayer]00:{$\boldsymbol\chi$}}] at (axis cs:{83.182},{32.192}) {}; % Aur,  4.74 
\node[pin={[pin distance=-0.4\onedegree,Bayer]90:{$\boldsymbol\delta$}}] at (axis cs:{89.882},{54.285}) {}; % Aur,  3.73 
\node[pin={[pin distance=-0.4\onedegree,Bayer]00:{$\boldsymbol\epsilon$}}] at (axis cs:{75.492},{43.823}) {}; % Aur,  3.03 
\node[pin={[pin distance=-0.4\onedegree,Bayer]00:{$\boldsymbol\eta$}}] at (axis cs:{76.629},{41.234}) {}; % Aur,  3.17 
\node[pin={[pin distance=-0.2\onedegree,Bayer]00:{$\boldsymbol\iota$}}] at (axis cs:{74.248},{33.166}) {}; % Aur,  2.68 
%\node[pin={[pin distance=-0.6\onedegree,Bayer]00:{$\boldsymbol\kappa$}}] at (axis cs:{93.845},{29.498}) {}; % Aur,  4.33 
\node[pin={[pin distance=-0.6\onedegree,Bayer]00:{$\boldsymbol\lambda$}}] at (axis cs:{79.785},{40.099}) {}; % Aur,  4.69 
\node[pin={[pin distance=-0.4\onedegree,Bayer]-90:{$\boldsymbol\mu$}}] at (axis cs:{78.357},{38.484}) {}; % Aur,  4.83 
\node[pin={[pin distance=-0.4\onedegree,Bayer]90:{$\boldsymbol\nu$}}] at (axis cs:{87.872},{39.148}) {}; % Aur,  3.97 
\node[pin={[pin distance=-0.6\onedegree,Bayer]00:{$\boldsymbol\omega$}}] at (axis cs:{74.814},{37.890}) {}; % Aur,  4.99 
\node[pin={[pin distance=-0.6\onedegree,Bayer]00:{$\boldsymbol\omicron$}}] at (axis cs:{86.475},{49.826}) {}; % Aur,  5.46 
\node[pin={[pin distance=-0.4\onedegree,Bayer]-90:{$\boldsymbol\pi$}}] at (axis cs:{89.984},{45.937}) {}; % Aur,  4.33 
\node[pin={[pin distance=-0.6\onedegree,Bayer]180:{$\boldsymbol\psi^1$}}] at (axis cs:{96.225},{49.288}) {}; % Aur,  4.95 
\node[pin={[pin distance=-0.6\onedegree,Bayer]-90:{$\boldsymbol\psi^2$}}] at (axis cs:{99.833},{42.489}) {}; % Aur,  4.79 
\node[pin={[pin distance=-0.6\onedegree,Bayer]-90:{$\boldsymbol\psi^3$}}] at (axis cs:{99.705},{39.903}) {}; % Aur,  5.35 
\node[pin={[pin distance=-0.6\onedegree,Bayer]00:{$\boldsymbol\psi^4$}}] at (axis cs:{100.771},{44.524}) {}; % Aur,  5.03 
\node[pin={[pin distance=-0.6\onedegree,Bayer]00:{$\boldsymbol\psi^5$}}] at (axis cs:{101.685},{43.577}) {}; % Aur,  5.25 
\node[pin={[pin distance=-0.6\onedegree,Bayer]180:{$\boldsymbol\psi^6$}}] at (axis cs:{101.915},{48.789}) {}; % Aur,  5.22 
\node[pin={[pin distance=-0.6\onedegree,Bayer]00:{$\boldsymbol\psi^7$}}] at (axis cs:{102.691},{41.781}) {}; % Aur,  4.98 
\node[pin={[pin distance=-0.6\onedegree,Bayer]180:{$\boldsymbol\psi^8$}}] at (axis cs:{103.488},{38.505}) {}; % Aur,  6.45 
\node[pin={[pin distance=-0.6\onedegree,Bayer]-90:{$\boldsymbol\psi^9$}}] at (axis cs:{104.134},{46.274}) {}; % Aur,  5.85 
\node[pin={[pin distance=-0.6\onedegree,Bayer]00:{$\boldsymbol\rho$}}] at (axis cs:{80.452},{41.804}) {}; % Aur,  5.21 
\node[pin={[pin distance=-0.6\onedegree,Bayer]00:{$\boldsymbol\sigma$}}] at (axis cs:{81.163},{37.385}) {}; % Aur,  4.99 
\node[pin={[pin distance=-0.6\onedegree,Bayer]00:{$\boldsymbol\tau$}}] at (axis cs:{87.293},{39.181}) {}; % Aur,  4.52 
\node[pin={[pin distance=-0.6\onedegree,Bayer]00:{$\boldsymbol\upsilon$}}] at (axis cs:{87.760},{37.305}) {}; % Aur,  4.73 
\node[pin={[pin distance=-0.6\onedegree,Bayer]90:{$\boldsymbol\varphi$}}] at (axis cs:{81.912},{34.476}) {}; % Aur,  5.06 
\node[pin={[pin distance=-0.3\onedegree,Bayer]90:{$\boldsymbol\vartheta$}}] at (axis cs:{89.930},{37.212}) {}; % Aur,  2.65 
\node[pin={[pin distance=-0.6\onedegree,Bayer]180:{$\boldsymbol\xi$}}] at (axis cs:{88.712},{55.707}) {}; % Aur,  4.97 
\node[pin={[pin distance=-0.4\onedegree,Bayer]-90:{$\boldsymbol\zeta$}}] at (axis cs:{75.620},{41.076}) {}; % Aur,  3.76 
\node[pin={[pin distance=-0.6\onedegree,Flaamsted]-90:{14}}] at (axis cs:{78.852},{32.688}) {}; % Aur,  4.99 
\node[pin={[pin distance=-0.6\onedegree,Flaamsted]-90:{16}}] at (axis cs:{79.544},{33.372}) {}; % Aur,  4.54 
%\node[pin={[pin distance=-0.6\onedegree,Flaamsted]00:{17}}] at (axis cs:{79.579},{33.767}) {}; % Aur,  6.15 
\node[pin={[pin distance=-0.6\onedegree,Flaamsted]00:{18}}] at (axis cs:{79.849},{33.985}) {}; % Aur,  6.49 
\node[pin={[pin distance=-0.6\onedegree,Flaamsted]90:{19}}] at (axis cs:{80.004},{33.958}) {}; % Aur,  5.04 
\node[pin={[pin distance=-0.6\onedegree,Flaamsted]00:{ 2}}] at (axis cs:{73.158},{36.703}) {}; % Aur,  4.78 
\node[pin={[pin distance=-0.6\onedegree,Flaamsted]00:{22}}] at (axis cs:{80.845},{28.937}) {}; % Aur,  6.45 
\node[pin={[pin distance=-0.6\onedegree,Flaamsted]-90:{26}}] at (axis cs:{84.659},{30.492}) {}; % Aur,  5.41 
\node[pin={[pin distance=-0.6\onedegree,Flaamsted]00:{36}}] at (axis cs:{90.244},{47.902}) {}; % Aur,  5.72 
\node[pin={[pin distance=-0.6\onedegree,Flaamsted]180:{38}}] at (axis cs:{90.825},{42.911}) {}; % Aur,  6.09 
\node[pin={[pin distance=-0.6\onedegree,Flaamsted]00:{39}}] at (axis cs:{91.264},{42.981}) {}; % Aur,  5.90 
\node[pin={[pin distance=-0.6\onedegree,Flaamsted]00:{40}}] at (axis cs:{91.646},{38.482}) {}; % Aur,  5.35 
\node[pin={[pin distance=-0.6\onedegree,Flaamsted]00:{41}}] at (axis cs:{92.902},{48.711}) {}; % Aur,  6.17 
\node[pin={[pin distance=-0.6\onedegree,Flaamsted]00:{43}}] at (axis cs:{94.570},{46.360}) {}; % Aur,  6.33 
\node[pin={[pin distance=-0.6\onedegree,Flaamsted]180:{45}}] at (axis cs:{95.442},{53.452}) {}; % Aur,  5.34 
\node[pin={[pin distance=-0.6\onedegree,Flaamsted]00:{47}}] at (axis cs:{97.512},{46.686}) {}; % Aur,  5.87 
\node[pin={[pin distance=-0.6\onedegree,Flaamsted]00:{48}}] at (axis cs:{97.142},{30.493}) {}; % Aur,  5.75 
\node[pin={[pin distance=-0.6\onedegree,Flaamsted]00:{49}}] at (axis cs:{98.800},{28.022}) {}; % Aur,  5.27 
\node[pin={[pin distance=-0.6\onedegree,Flaamsted]90:{ 5}}] at (axis cs:{75.076},{39.395}) {}; % Aur,  5.98 
\node[pin={[pin distance=-0.6\onedegree,Flaamsted]90:{51}}] at (axis cs:{99.665},{39.391}) {}; % Aur,  5.70 
%\node[pin={[pin distance=-0.6\onedegree,Flaamsted]00:{53}}] at (axis cs:{99.596},{28.984}) {}; % Aur,  5.75 
\node[pin={[pin distance=-0.6\onedegree,Flaamsted]00:{54}}] at (axis cs:{99.888},{28.263}) {}; % Aur,  6.06 
\node[pin={[pin distance=-0.6\onedegree,Flaamsted]-90:{59}}] at (axis cs:{103.256},{38.869}) {}; % Aur,  6.10 
\node[pin={[pin distance=-0.6\onedegree,Flaamsted]-90:{ 6}}] at (axis cs:{75.097},{39.655}) {}; % Aur,  6.44 
\node[pin={[pin distance=-0.8\onedegree,Flaamsted]45:{60}}] at (axis cs:{103.306},{38.438}) {}; % Aur,  6.33 
\node[pin={[pin distance=-0.6\onedegree,Flaamsted]00:{62}}] at (axis cs:{104.762},{38.052}) {}; % Aur,  6.02 
\node[pin={[pin distance=-0.6\onedegree,Flaamsted]90:{63}}] at (axis cs:{107.914},{39.320}) {}; % Aur,  4.89 
\node[pin={[pin distance=-0.6\onedegree,Flaamsted]-90:{64}}] at (axis cs:{109.509},{40.883}) {}; % Aur,  5.87 
\node[pin={[pin distance=-0.6\onedegree,Flaamsted]00:{65}}] at (axis cs:{110.511},{36.760}) {}; % Aur,  5.13 
\node[pin={[pin distance=-0.6\onedegree,Flaamsted]00:{66}}] at (axis cs:{111.035},{40.672}) {}; % Aur,  5.19 
\node[pin={[pin distance=-0.6\onedegree,Flaamsted]00:{ 9}}] at (axis cs:{76.669},{51.597}) {}; % Aur,  4.98 
\node[pin={[pin distance=-0.6\onedegree,Bayer]00:{$\boldsymbol\beta$}}] at (axis cs:{225.487},{40.390}) {}; % Boo,  3.48 
\node[pin={[pin distance=-0.6\onedegree,Bayer]00:{$\boldsymbol\chi$}}] at (axis cs:{228.621},{29.164}) {}; % Boo,  5.28 
\node[pin={[pin distance=-0.6\onedegree,Bayer]00:{$\boldsymbol\delta$}}] at (axis cs:{228.876},{33.315}) {}; % Boo,  3.47 
\node[pin={[pin distance=-0.6\onedegree,Bayer]00:{$\boldsymbol\epsilon$}}] at (axis cs:{221.247},{27.074}) {}; % Boo,  2.50 
\node[pin={[pin distance=-0.6\onedegree,Bayer]00:{$\boldsymbol\gamma$}}] at (axis cs:{218.019},{38.308}) {}; % Boo,  3.04 
\node[pin={[pin distance=-0.6\onedegree,Bayer]00:{$\boldsymbol\iota$}}] at (axis cs:{214.041},{51.367}) {}; % Boo,  4.75 
\node[pin={[pin distance=-0.6\onedegree,Bayer]00:{$\boldsymbol\kappa^2$}}] at (axis cs:{213.371},{51.790}) {}; % Boo,  4.52 
\node[pin={[pin distance=-0.6\onedegree,Bayer]00:{$\boldsymbol\lambda$}}] at (axis cs:{214.096},{46.088}) {}; % Boo,  4.18 
\node[pin={[pin distance=-0.6\onedegree,Bayer]00:{$\boldsymbol\mu^1$}}] at (axis cs:{231.123},{37.377}) {}; % Boo,  4.30 
\node[pin={[pin distance=-0.6\onedegree,Bayer]00:{$\boldsymbol\nu^1$}}] at (axis cs:{232.732},{40.833}) {}; % Boo,  5.03 
\node[pin={[pin distance=-0.6\onedegree,Bayer]00:{$\boldsymbol\nu^2$}}] at (axis cs:{232.946},{40.899}) {}; % Boo,  5.00 
\node[pin={[pin distance=-0.6\onedegree,Bayer]00:{$\boldsymbol\rho$}}] at (axis cs:{217.957},{30.371}) {}; % Boo,  3.57 
\node[pin={[pin distance=-0.6\onedegree,Bayer]00:{$\boldsymbol\sigma$}}] at (axis cs:{218.670},{29.745}) {}; % Boo,  4.47 
\node[pin={[pin distance=-0.6\onedegree,Bayer]00:{$\boldsymbol\varphi$}}] at (axis cs:{234.457},{40.353}) {}; % Boo,  5.26 
\node[pin={[pin distance=-0.6\onedegree,Bayer]00:{$\boldsymbol\vartheta$}}] at (axis cs:{216.299},{51.851}) {}; % Boo,  4.05 
\node[pin={[pin distance=-0.6\onedegree,Flaamsted]00:{11}}] at (axis cs:{210.294},{27.387}) {}; % Boo,  6.23 
\node[pin={[pin distance=-0.6\onedegree,Flaamsted]00:{13}}] at (axis cs:{212.072},{49.458}) {}; % Boo,  5.26 
\node[pin={[pin distance=-0.6\onedegree,Flaamsted]00:{24}}] at (axis cs:{217.158},{49.845}) {}; % Boo,  5.58 
\node[pin={[pin distance=-0.6\onedegree,Flaamsted]00:{33}}] at (axis cs:{219.709},{44.404}) {}; % Boo,  5.40 
\node[pin={[pin distance=-0.6\onedegree,Flaamsted]00:{38}}] at (axis cs:{222.328},{46.116}) {}; % Boo,  5.76 
\node[pin={[pin distance=-0.6\onedegree,Flaamsted]00:{39}}] at (axis cs:{222.422},{48.721}) {}; % Boo,  5.68 
\node[pin={[pin distance=-0.6\onedegree,Flaamsted]00:{40}}] at (axis cs:{224.904},{39.265}) {}; % Boo,  5.64 
\node[pin={[pin distance=-0.6\onedegree,Flaamsted]00:{44}}] at (axis cs:{225.947},{47.654}) {}; % Boo,  4.83 
\node[pin={[pin distance=-0.6\onedegree,Flaamsted]00:{47}}] at (axis cs:{226.358},{48.151}) {}; % Boo,  5.59 
\node[pin={[pin distance=-0.6\onedegree,Flaamsted]00:{50}}] at (axis cs:{230.452},{32.934}) {}; % Boo,  5.38 
\node[pin={[pin distance=-0.6\onedegree,Flaamsted]00:{ 9}}] at (axis cs:{209.142},{27.492}) {}; % Boo,  5.01 
\node[pin={[pin distance=-0.6\onedegree,Bayer]00:{$\boldsymbol\alpha$}}] at (axis cs:{73.513},{66.343}) {}; % Cam,  4.31 
\node[pin={[pin distance=-0.4\onedegree,Bayer]-90:{$\boldsymbol\beta$}}] at (axis cs:{75.855},{60.442}) {}; % Cam,  4.03 
\node[pin={[pin distance=-0.6\onedegree,Bayer]00:{$\boldsymbol\gamma$}}] at (axis cs:{57.590},{71.332}) {}; % Cam,  4.62 
\node[pin={[pin distance=-0.6\onedegree,Flaamsted]00:{ 1}}] at (axis cs:{68.008},{53.911}) {}; % Cam,  5.77 
\node[pin={[pin distance=-0.6\onedegree,Flaamsted]00:{11}}] at (axis cs:{76.535},{58.972}) {}; % Cam,  5.23 
\node[pin={[pin distance=-0.6\onedegree,Flaamsted]00:{14}}] at (axis cs:{78.380},{62.691}) {}; % Cam,  6.49 
\node[pin={[pin distance=-0.6\onedegree,Flaamsted]00:{15}}] at (axis cs:{79.866},{58.117}) {}; % Cam,  6.13 
\node[pin={[pin distance=-0.6\onedegree,Flaamsted]00:{16}}] at (axis cs:{80.866},{57.544}) {}; % Cam,  5.25 
\node[pin={[pin distance=-0.6\onedegree,Flaamsted]00:{17}}] at (axis cs:{82.543},{63.067}) {}; % Cam,  5.44 
\node[pin={[pin distance=-0.6\onedegree,Flaamsted]00:{18}}] at (axis cs:{83.141},{57.221}) {}; % Cam,  6.46 
\node[pin={[pin distance=-0.6\onedegree,Flaamsted]00:{19}}] at (axis cs:{84.313},{64.155}) {}; % Cam,  6.17 
\node[pin={[pin distance=-0.6\onedegree,Flaamsted]90:{ 2}}] at (axis cs:{69.992},{53.473}) {}; % Cam,  5.37 
\node[pin={[pin distance=-0.6\onedegree,Flaamsted]00:{23}}] at (axis cs:{86.035},{61.476}) {}; % Cam,  6.16 
\node[pin={[pin distance=-0.6\onedegree,Flaamsted]00:{24}}] at (axis cs:{85.757},{56.581}) {}; % Cam,  6.05 
\node[pin={[pin distance=-0.6\onedegree,Flaamsted]00:{26}}] at (axis cs:{86.627},{56.115}) {}; % Cam,  5.93 
\node[pin={[pin distance=-0.6\onedegree,Flaamsted]90:{ 3}}] at (axis cs:{69.978},{53.079}) {}; % Cam,  5.08 
\node[pin={[pin distance=-0.6\onedegree,Flaamsted]00:{30}}] at (axis cs:{88.072},{58.964}) {}; % Cam,  6.14 
\node[pin={[pin distance=-0.6\onedegree,Flaamsted]00:{31}}] at (axis cs:{88.741},{59.888}) {}; % Cam,  5.20 
\node[pin={[pin distance=-0.6\onedegree,Flaamsted]00:{36}}] at (axis cs:{93.213},{65.718}) {}; % Cam,  5.35 
\node[pin={[pin distance=-0.6\onedegree,Flaamsted]00:{37}}] at (axis cs:{92.496},{58.936}) {}; % Cam,  5.35 
\node[pin={[pin distance=-0.6\onedegree,Flaamsted]00:{ 4}}] at (axis cs:{72.001},{56.757}) {}; % Cam,  5.28 
\node[pin={[pin distance=-0.6\onedegree,Flaamsted]00:{40}}] at (axis cs:{93.919},{59.999}) {}; % Cam,  5.35 
\node[pin={[pin distance=-0.6\onedegree,Flaamsted]00:{42}}] at (axis cs:{102.738},{67.572}) {}; % Cam,  5.14 
\node[pin={[pin distance=-0.6\onedegree,Flaamsted]00:{43}}] at (axis cs:{103.426},{68.888}) {}; % Cam,  5.11 
\node[pin={[pin distance=-0.6\onedegree,Flaamsted]00:{47}}] at (axis cs:{110.572},{59.902}) {}; % Cam,  6.37 
\node[pin={[pin distance=-0.6\onedegree,Flaamsted]00:{49}}] at (axis cs:{116.614},{62.830}) {}; % Cam,  6.47 
\node[pin={[pin distance=-0.6\onedegree,Flaamsted]00:{ 5}}] at (axis cs:{73.763},{55.259}) {}; % Cam,  5.53 
\node[pin={[pin distance=-0.6\onedegree,Flaamsted]00:{51}}] at (axis cs:{116.667},{65.456}) {}; % Cam,  5.94 
\node[pin={[pin distance=-0.6\onedegree,Flaamsted]00:{53}}] at (axis cs:{120.427},{60.324}) {}; % Cam,  6.02 
\node[pin={[pin distance=-0.4\onedegree,Flaamsted]00:{ 7}}] at (axis cs:{74.322},{53.752}) {}; % Cam,  4.46 
\node[pin={[pin distance=-0.6\onedegree,Flaamsted]90:{ 8}}] at (axis cs:{74.943},{53.155}) {}; % Cam,  6.09 
\node[pin={[pin distance=-0.4\onedegree,Bayer]00:{$\boldsymbol\alpha$}}] at (axis cs:{10.127},{56.537}) {}; % Cas,  2.25 
\node[pin={[pin distance=-0.4\onedegree,Bayer]00:{$\boldsymbol\beta$}}] at (axis cs:{2.295},{59.150}) {}; % Cas,  2.28 
\node[pin={[pin distance=-0.3\onedegree,Bayer]180:{$\boldsymbol\gamma$}}] at (axis cs:{14.177},{60.717}) {}; % Cas,  2.18 
\node[pin={[pin distance=-0.6\onedegree,Bayer]00:{$\boldsymbol\chi$}}] at (axis cs:{23.483},{59.232}) {}; % Cas,  4.69 
\node[pin={[pin distance=-0.3\onedegree,Bayer]90:{$\boldsymbol\delta$}}] at (axis cs:{21.454},{60.235}) {}; % Cas,  2.68 
\node[pin={[pin distance=-0.3\onedegree,Bayer]-90:{$\boldsymbol\epsilon$}}] at (axis cs:{28.599},{63.670}) {}; % Cas,  3.35 
\node[pin={[pin distance=-0.4\onedegree,Bayer]00:{$\boldsymbol\eta$}}] at (axis cs:{12.276},{57.815}) {}; % Cas,  3.45 
\node[pin={[pin distance=-0.6\onedegree,Bayer]00:{$\boldsymbol\iota$}}] at (axis cs:{37.266},{67.402}) {}; % Cas,  4.63 
\node[pin={[pin distance=-0.4\onedegree,Bayer]-90:{$\boldsymbol\kappa$}}] at (axis cs:{8.250},{62.932}) {}; % Cas,  4.19 
\node[pin={[pin distance=-0.6\onedegree,Bayer]-90:{$\boldsymbol\lambda$}}] at (axis cs:{7.943},{54.522}) {}; % Cas,  4.74 
\node[pin={[pin distance=-0.6\onedegree,Bayer]00:{$\boldsymbol\mu$}}] at (axis cs:{17.068},{54.920}) {}; % Cas,  5.17 
\node[pin={[pin distance=-0.6\onedegree,Bayer]00:{$\boldsymbol\nu$}}] at (axis cs:{12.208},{50.968}) {}; % Cas,  4.91 
\node[pin={[pin distance=-0.6\onedegree,Bayer]00:{$\boldsymbol\omega$}}] at (axis cs:{29.000},{68.685}) {}; % Cas,  4.98 
\node[pin={[pin distance=-0.4\onedegree,Bayer]90:{$\boldsymbol\omicron$}}] at (axis cs:{11.181},{48.284}) {}; % Cas,  4.48 
\node[pin={[pin distance=-0.6\onedegree,Bayer]00:{$\boldsymbol\pi$}}] at (axis cs:{10.867},{47.024}) {}; % Cas,  4.96 
\node[pin={[pin distance=-0.6\onedegree,Bayer]90:{$\boldsymbol\psi$}}] at (axis cs:{21.483},{68.130}) {}; % Cas,  4.72 
\node[pin={[pin distance=-0.6\onedegree,Bayer]00:{$\boldsymbol\rho$}}] at (axis cs:{358.596},{57.499}) {}; % Cas,  4.51 
\node[pin={[pin distance=-0.6\onedegree,Bayer]00:{$\boldsymbol\sigma$}}] at (axis cs:{359.752},{55.755}) {}; % Cas,  4.88 
\node[pin={[pin distance=-0.6\onedegree,Bayer]00:{$\boldsymbol\tau$}}] at (axis cs:{356.764},{58.652}) {}; % Cas,  4.88 
\node[pin={[pin distance=-0.6\onedegree,Bayer]00:{$\boldsymbol\upsilon^{1,2}$}}] at (axis cs:{13.751},{58.973}) {}; % Cas,  4.82 
%\node[pin={[pin distance=-0.6\onedegree,Bayer]00:{$\boldsymbol\upsilon^2$}}] at (axis cs:{14.166},{59.181}) {}; % Cas,  4.63 
\node[pin={[pin distance=-0.3\onedegree,Bayer]90:{$\boldsymbol\varphi$}}] at (axis cs:{20.020},{58.231}) {}; % Cas,  5.00 
\node[pin={[pin distance=-0.6\onedegree,Bayer]90:{$\boldsymbol\vartheta$}}] at (axis cs:{17.776},{55.150}) {}; % Cas,  4.34 
\node[pin={[pin distance=-0.6\onedegree,Bayer]00:{$\boldsymbol\xi$}}] at (axis cs:{10.516},{50.512}) {}; % Cas,  4.81 
\node[pin={[pin distance=-0.4\onedegree,Bayer]-90:{$\boldsymbol\zeta$}}] at (axis cs:{9.243},{53.897}) {}; % Cas,  3.68 
\node[pin={[pin distance=-0.6\onedegree,Flaamsted]00:{ 1}}] at (axis cs:{346.653},{59.420}) {}; % Cas,  4.85 
\node[pin={[pin distance=-0.6\onedegree,Flaamsted]00:{10}}] at (axis cs:{1.611},{64.196}) {}; % Cas,  5.57 
\node[pin={[pin distance=-0.6\onedegree,Flaamsted]00:{12}}] at (axis cs:{6.198},{61.831}) {}; % Cas,  5.39 
\node[pin={[pin distance=-0.6\onedegree,Flaamsted]-90:{13}}] at (axis cs:{7.855},{66.520}) {}; % Cas,  6.18 
\node[pin={[pin distance=-0.6\onedegree,Flaamsted]90:{16}}] at (axis cs:{8.604},{66.750}) {}; % Cas,  6.48 
\node[pin={[pin distance=-0.6\onedegree,Flaamsted]-90:{ 2}}] at (axis cs:{347.434},{59.332}) {}; % Cas,  5.69 
\node[pin={[pin distance=-0.6\onedegree,Flaamsted]90:{21}}] at (axis cs:{11.413},{74.988}) {}; % Cas,  5.65 
\node[pin={[pin distance=-0.6\onedegree,Flaamsted]-90:{23}}] at (axis cs:{11.942},{74.847}) {}; % Cas,  5.43 
\node[pin={[pin distance=-0.6\onedegree,Flaamsted]00:{31}}] at (axis cs:{17.664},{68.778}) {}; % Cas,  5.32 
\node[pin={[pin distance=-0.6\onedegree,Flaamsted]00:{32}}] at (axis cs:{17.923},{65.019}) {}; % Cas,  5.57 
\node[pin={[pin distance=-0.6\onedegree,Flaamsted]00:{35}}] at (axis cs:{20.272},{64.658}) {}; % Cas,  6.33 
\node[pin={[pin distance=-0.6\onedegree,Flaamsted]00:{38}}] at (axis cs:{22.807},{70.264}) {}; % Cas,  5.83 
\node[pin={[pin distance=-0.6\onedegree,Flaamsted]90:{ 4}}] at (axis cs:{351.209},{62.283}) {}; % Cas,  4.97 
\node[pin={[pin distance=-0.6\onedegree,Flaamsted]00:{40}}] at (axis cs:{24.629},{73.040}) {}; % Cas,  5.28 
\node[pin={[pin distance=-0.6\onedegree,Flaamsted]00:{42}}] at (axis cs:{25.733},{70.622}) {}; % Cas,  5.18 
\node[pin={[pin distance=-0.6\onedegree,Flaamsted]00:{43}}] at (axis cs:{25.586},{68.043}) {}; % Cas,  5.57 
\node[pin={[pin distance=-0.6\onedegree,Flaamsted]-90:{44}}] at (axis cs:{25.832},{60.551}) {}; % Cas,  5.78 
\node[pin={[pin distance=-0.6\onedegree,Flaamsted]180:{47}}] at (axis cs:{31.281},{77.281}) {}; % Cas,  5.27 
\node[pin={[pin distance=-0.6\onedegree,Flaamsted]00:{48}}] at (axis cs:{30.489},{70.907}) {}; % Cas,  4.52 
\node[pin={[pin distance=-0.6\onedegree,Flaamsted]180:{49}}] at (axis cs:{31.381},{76.115}) {}; % Cas,  5.22 
\node[pin={[pin distance=-0.4\onedegree,Flaamsted]90:{50}}] at (axis cs:{30.859},{72.421}) {}; % Cas,  3.95 
\node[pin={[pin distance=-0.6\onedegree,Flaamsted]90:{52}}] at (axis cs:{30.719},{64.901}) {}; % Cas,  6.00 
\node[pin={[pin distance=-0.6\onedegree,Flaamsted]180:{53}}] at (axis cs:{30.751},{64.390}) {}; % Cas,  5.62 
\node[pin={[pin distance=-0.6\onedegree,Flaamsted]00:{55}}] at (axis cs:{33.621},{66.524}) {}; % Cas,  6.06 
\node[pin={[pin distance=-0.6\onedegree,Flaamsted]00:{ 6}}] at (axis cs:{357.209},{62.214}) {}; % Cas,  5.54 
\node[pin={[pin distance=-0.6\onedegree,Flaamsted]-90:{ 9}}] at (axis cs:{1.057},{62.288}) {}; % Cas,  5.87 
\node[pin={[pin distance=-0.4\onedegree,Bayer]00:{$\boldsymbol\alpha$}}] at (axis cs:{319.645},{62.586}) {}; % Cep,  2.47 
\node[pin={[pin distance=-0.4\onedegree,Bayer]00:{$\boldsymbol\beta$}}] at (axis cs:{322.165},{70.561}) {}; % Cep,  3.23 
\node[pin={[pin distance=-0.6\onedegree,Bayer]00:{$\boldsymbol\delta$}}] at (axis cs:{337.293},{58.415}) {}; % Cep,  4.07 
\node[pin={[pin distance=-0.6\onedegree,Bayer]-90:{$\boldsymbol\epsilon$}}] at (axis cs:{333.759},{57.043}) {}; % Cep,  4.18 
\node[pin={[pin distance=-0.4\onedegree,Bayer]-90:{$\boldsymbol\eta$}}] at (axis cs:{311.322},{61.839}) {}; % Cep,  3.42 
\node[pin={[pin distance=-0.4\onedegree,Bayer]00:{$\boldsymbol\gamma$}}] at (axis cs:{354.837},{77.632}) {}; % Cep,  3.22 
\node[pin={[pin distance=-0.4\onedegree,Bayer]00:{$\boldsymbol\iota$}}] at (axis cs:{342.420},{66.200}) {}; % Cep,  3.50 
\node[pin={[pin distance=-0.6\onedegree,Bayer]00:{$\boldsymbol\kappa$}}] at (axis cs:{302.222},{77.711}) {}; % Cep,  4.40 
\node[pin={[pin distance=-0.6\onedegree,Bayer]00:{$\boldsymbol\lambda$}}] at (axis cs:{332.877},{59.414}) {}; % Cep,  5.09 
\node[pin={[pin distance=-0.4\onedegree,Bayer]-90:{$\boldsymbol\mu$}}] at (axis cs:{325.877},{58.780}) {}; % Cep,  4.02 
\node[pin={[pin distance=-0.6\onedegree,Bayer]00:{$\boldsymbol\nu$}}] at (axis cs:{326.362},{61.121}) {}; % Cep,  4.31 
\node[pin={[pin distance=-0.6\onedegree,Bayer]00:{$\boldsymbol\omicron$}}] at (axis cs:{349.656},{68.111}) {}; % Cep,  4.86 
\node[pin={[pin distance=-0.6\onedegree,Bayer]00:{$\boldsymbol\pi$}}] at (axis cs:{346.974},{75.387}) {}; % Cep,  4.42 
%\node[pin={[pin distance=-0.6\onedegree,Bayer]00:{$\boldsymbol\rho^1$}}] at (axis cs:{336.677},{78.786}) {}; % Cep,  5.84 
\node[pin={[pin distance=-0.6\onedegree,Bayer]00:{$\boldsymbol\rho^{1,2}$}}] at (axis cs:{337.471},{78.824}) {}; % Cep,  5.45 
\node[pin={[pin distance=-0.6\onedegree,Bayer]-90:{$\boldsymbol\vartheta$}}] at (axis cs:{307.395},{62.994}) {}; % Cep,  4.21 
\node[pin={[pin distance=-0.6\onedegree,Bayer]00:{$\boldsymbol\xi$}}] at (axis cs:{330.948},{64.628}) {}; % Cep,  4.40 
\node[pin={[pin distance=-0.6\onedegree,Bayer]00:{$\boldsymbol\zeta$}}] at (axis cs:{332.714},{58.201}) {}; % Cep,  3.34 
\node[pin={[pin distance=-0.6\onedegree,Flaamsted]00:{11}}] at (axis cs:{325.480},{71.311}) {}; % Cep,  4.54 
\node[pin={[pin distance=-0.6\onedegree,Flaamsted]00:{12}}] at (axis cs:{326.855},{60.693}) {}; % Cep,  5.52 
\node[pin={[pin distance=-0.6\onedegree,Flaamsted]90:{13}}] at (axis cs:{328.721},{56.611}) {}; % Cep,  5.80 
\node[pin={[pin distance=-0.6\onedegree,Flaamsted]00:{14}}] at (axis cs:{330.519},{58.000}) {}; % Cep,  5.55 
\node[pin={[pin distance=-0.6\onedegree,Flaamsted]00:{16}}] at (axis cs:{329.812},{73.180}) {}; % Cep,  5.04 
\node[pin={[pin distance=-0.6\onedegree,Flaamsted]180:{18}}] at (axis cs:{330.971},{63.120}) {}; % Cep,  5.29 
\node[pin={[pin distance=-0.6\onedegree,Flaamsted]-90:{19}}] at (axis cs:{331.287},{62.280}) {}; % Cep,  5.12 
\node[pin={[pin distance=-0.6\onedegree,Flaamsted]00:{20}}] at (axis cs:{331.252},{62.786}) {}; % Cep,  5.26 
\node[pin={[pin distance=-0.6\onedegree,Flaamsted]00:{24}}] at (axis cs:{332.452},{72.341}) {}; % Cep,  4.80 
\node[pin={[pin distance=-0.6\onedegree,Flaamsted]00:{25}}] at (axis cs:{334.553},{62.804}) {}; % Cep,  5.74 
\node[pin={[pin distance=-0.6\onedegree,Flaamsted]00:{26}}] at (axis cs:{336.772},{65.132}) {}; % Cep,  5.53 
\node[pin={[pin distance=-0.6\onedegree,Flaamsted]00:{30}}] at (axis cs:{339.663},{63.584}) {}; % Cep,  5.20 
\node[pin={[pin distance=-0.6\onedegree,Flaamsted]00:{31}}] at (axis cs:{338.942},{73.643}) {}; % Cep,  5.08 
\node[pin={[pin distance=-0.6\onedegree,Flaamsted]00:{ 4}}] at (axis cs:{310.796},{66.657}) {}; % Cep,  5.59 
\node[pin={[pin distance=-0.6\onedegree,Flaamsted]00:{ 6}}] at (axis cs:{319.843},{64.872}) {}; % Cep,  5.19 
\node[pin={[pin distance=-0.6\onedegree,Flaamsted]00:{ 7}}] at (axis cs:{321.942},{66.809}) {}; % Cep,  5.42 
\node[pin={[pin distance=-0.6\onedegree,Flaamsted]00:{ 9}}] at (axis cs:{324.480},{62.082}) {}; % Cep,  4.79 
\node[pin={[pin distance=-0.6\onedegree,Bayer]00:{$\boldsymbol\chi$}}] at (axis cs:{125.016},{27.218}) {}; % Cnc,  5.13 
\node[pin={[pin distance=-0.6\onedegree,Bayer]00:{$\boldsymbol\iota$}}] at (axis cs:{131.674},{28.760}) {}; % Cnc,  4.02 
\node[pin={[pin distance=-0.6\onedegree,Bayer]00:{$\boldsymbol\rho^1$}}] at (axis cs:{133.149},{28.331}) {}; % Cnc,  5.95 
\node[pin={[pin distance=-0.6\onedegree,Bayer]00:{$\boldsymbol\rho^2$}}] at (axis cs:{133.915},{27.927}) {}; % Cnc,  5.23 
\node[pin={[pin distance=-0.6\onedegree,Bayer]00:{$\boldsymbol\sigma^1$}}] at (axis cs:{133.144},{32.474}) {}; % Cnc,  5.67 
\node[pin={[pin distance=-0.6\onedegree,Bayer]00:{$\boldsymbol\sigma^2$}}] at (axis cs:{134.236},{32.910}) {}; % Cnc,  5.44 
\node[pin={[pin distance=-0.6\onedegree,Bayer]00:{$\boldsymbol\sigma^3$}}] at (axis cs:{134.886},{32.419}) {}; % Cnc,  5.23 
\node[pin={[pin distance=-0.6\onedegree,Bayer]00:{$\boldsymbol\tau$}}] at (axis cs:{137.000},{29.654}) {}; % Cnc,  5.42 
\node[pin={[pin distance=-0.6\onedegree,Bayer]00:{$\boldsymbol\varphi^1$}}] at (axis cs:{126.615},{27.893}) {}; % Cnc,  5.58 
\node[pin={[pin distance=-0.6\onedegree,Flaamsted]00:{15}}] at (axis cs:{123.287},{29.656}) {}; % Cnc,  5.62 
\node[pin={[pin distance=-0.6\onedegree,Flaamsted]00:{46}}] at (axis cs:{131.339},{30.698}) {}; % Cnc,  6.12 
\node[pin={[pin distance=-0.6\onedegree,Flaamsted]00:{53}}] at (axis cs:{133.119},{28.259}) {}; % Cnc,  6.29 
\node[pin={[pin distance=-0.6\onedegree,Flaamsted]00:{57}}] at (axis cs:{133.561},{30.579}) {}; % Cnc,  5.42 
\node[pin={[pin distance=-0.6\onedegree,Flaamsted]00:{61}}] at (axis cs:{134.494},{30.234}) {}; % Cnc,  6.25 
\node[pin={[pin distance=-0.6\onedegree,Flaamsted]00:{66}}] at (axis cs:{135.351},{32.252}) {}; % Cnc,  5.95 
\node[pin={[pin distance=-0.6\onedegree,Flaamsted]00:{67}}] at (axis cs:{135.453},{27.903}) {}; % Cnc,  6.06 
\node[pin={[pin distance=-0.6\onedegree,Bayer]00:{$\boldsymbol\beta$}}] at (axis cs:{197.968},{27.878}) {}; % Com,  4.25 
\node[pin={[pin distance=-0.6\onedegree,Bayer]00:{$\boldsymbol\gamma$}}] at (axis cs:{186.734},{28.268}) {}; % Com,  4.34 
\node[pin={[pin distance=-0.6\onedegree,Flaamsted]00:{14}}] at (axis cs:{186.600},{27.268}) {}; % Com,  4.92 
\node[pin={[pin distance=-0.6\onedegree,Flaamsted]00:{30}}] at (axis cs:{192.323},{27.552}) {}; % Com,  5.76 
\node[pin={[pin distance=-0.6\onedegree,Flaamsted]00:{31}}] at (axis cs:{192.925},{27.541}) {}; % Com,  4.93 
\node[pin={[pin distance=-0.6\onedegree,Flaamsted]00:{37}}] at (axis cs:{195.069},{30.785}) {}; % Com,  4.89 
\node[pin={[pin distance=-0.6\onedegree,Flaamsted]00:{41}}] at (axis cs:{196.795},{27.625}) {}; % Com,  4.78 
\node[pin={[pin distance=-0.6\onedegree,Flaamsted]00:{ 9}}] at (axis cs:{184.873},{28.157}) {}; % Com,  6.38 
\node[pin={[pin distance=-0.6\onedegree,Bayer]00:{$\boldsymbol\beta$}}] at (axis cs:{231.957},{29.106}) {}; % CrB,  3.66 
\node[pin={[pin distance=-0.6\onedegree,Bayer]00:{$\boldsymbol\eta$}}] at (axis cs:{230.801},{30.288}) {}; % CrB,  4.997
\node[pin={[pin distance=-0.6\onedegree,Bayer]00:{$\boldsymbol\iota$}}] at (axis cs:{240.361},{29.851}) {}; % CrB,  4.98 
\node[pin={[pin distance=-0.6\onedegree,Bayer]00:{$\boldsymbol\kappa$}}] at (axis cs:{237.808},{35.657}) {}; % CrB,  4.81 
\node[pin={[pin distance=-0.6\onedegree,Bayer]00:{$\boldsymbol\lambda$}}] at (axis cs:{238.948},{37.947}) {}; % CrB,  5.44 
\node[pin={[pin distance=-0.6\onedegree,Bayer]00:{$\boldsymbol\mu$}}] at (axis cs:{233.812},{39.010}) {}; % CrB,  5.14 
\node[pin={[pin distance=-0.6\onedegree,Bayer]00:{$\boldsymbol\nu^1$}}] at (axis cs:{245.589},{33.799}) {}; % CrB,  5.21 
\node[pin={[pin distance=-0.6\onedegree,Bayer]00:{$\boldsymbol\omicron$}}] at (axis cs:{230.036},{29.616}) {}; % CrB,  5.51 
\node[pin={[pin distance=-0.6\onedegree,Bayer]00:{$\boldsymbol\pi$}}] at (axis cs:{235.997},{32.516}) {}; % CrB,  5.57 
\node[pin={[pin distance=-0.6\onedegree,Bayer]00:{$\boldsymbol\rho$}}] at (axis cs:{240.261},{33.303}) {}; % CrB,  5.40 
\node[pin={[pin distance=-0.6\onedegree,Bayer]00:{$\boldsymbol\sigma$}}] at (axis cs:{243.670},{33.858}) {}; % CrB,  5.54 
\node[pin={[pin distance=-0.6\onedegree,Bayer]00:{$\boldsymbol\tau$}}] at (axis cs:{242.243},{36.491}) {}; % CrB,  4.73 
\node[pin={[pin distance=-0.6\onedegree,Bayer]00:{$\boldsymbol\upsilon$}}] at (axis cs:{244.187},{29.150}) {}; % CrB,  5.79 
\node[pin={[pin distance=-0.6\onedegree,Bayer]00:{$\boldsymbol\vartheta$}}] at (axis cs:{233.232},{31.359}) {}; % CrB,  4.16 
\node[pin={[pin distance=-0.6\onedegree,Bayer]00:{$\boldsymbol\xi$}}] at (axis cs:{245.524},{30.892}) {}; % CrB,  4.85 
\node[pin={[pin distance=-0.6\onedegree,Bayer]00:{$\boldsymbol\zeta^2$}}] at (axis cs:{234.844},{36.636}) {}; % CrB,  5.00 
\node[pin={[pin distance=-0.6\onedegree,Bayer]00:{$\boldsymbol\alpha^2$}}] at (axis cs:{194.007},{38.318}) {}; % CVn,  2.89 
\node[pin={[pin distance=-0.6\onedegree,Bayer]00:{$\boldsymbol\beta$}}] at (axis cs:{188.436},{41.357}) {}; % CVn,  4.25 
\node[pin={[pin distance=-0.6\onedegree,Flaamsted]00:{10}}] at (axis cs:{191.248},{39.279}) {}; % CVn,  5.96 
\node[pin={[pin distance=-0.6\onedegree,Flaamsted]00:{11}}] at (axis cs:{192.174},{48.467}) {}; % CVn,  6.25 
\node[pin={[pin distance=-0.6\onedegree,Flaamsted]00:{14}}] at (axis cs:{196.435},{35.799}) {}; % CVn,  5.20 
\node[pin={[pin distance=-0.6\onedegree,Flaamsted]00:{17}}] at (axis cs:{197.513},{38.499}) {}; % CVn,  5.93 
\node[pin={[pin distance=-0.6\onedegree,Flaamsted]00:{19}}] at (axis cs:{198.883},{40.855}) {}; % CVn,  5.78 
\node[pin={[pin distance=-0.6\onedegree,Flaamsted]00:{ 2}}] at (axis cs:{184.031},{40.660}) {}; % CVn,  5.67 
\node[pin={[pin distance=-0.6\onedegree,Flaamsted]00:{20}}] at (axis cs:{199.386},{40.573}) {}; % CVn,  4.71 
\node[pin={[pin distance=-0.6\onedegree,Flaamsted]00:{21}}] at (axis cs:{199.560},{49.682}) {}; % CVn,  5.15 
\node[pin={[pin distance=-0.6\onedegree,Flaamsted]00:{23}}] at (axis cs:{200.079},{40.151}) {}; % CVn,  5.60 
\node[pin={[pin distance=-0.6\onedegree,Flaamsted]00:{24}}] at (axis cs:{203.614},{49.016}) {}; % CVn,  4.67 
\node[pin={[pin distance=-0.6\onedegree,Flaamsted]00:{25}}] at (axis cs:{204.365},{36.295}) {}; % CVn,  4.91 
\node[pin={[pin distance=-0.6\onedegree,Flaamsted]00:{ 3}}] at (axis cs:{184.953},{48.984}) {}; % CVn,  5.29 
\node[pin={[pin distance=-0.6\onedegree,Flaamsted]00:{ 4}}] at (axis cs:{185.946},{42.543}) {}; % CVn,  6.03 
\node[pin={[pin distance=-0.6\onedegree,Flaamsted]00:{ 5}}] at (axis cs:{186.006},{51.562}) {}; % CVn,  4.76 
\node[pin={[pin distance=-0.6\onedegree,Flaamsted]00:{ 6}}] at (axis cs:{186.462},{39.019}) {}; % CVn,  5.02 
\node[pin={[pin distance=-0.6\onedegree,Flaamsted]00:{ 7}}] at (axis cs:{187.512},{51.536}) {}; % CVn,  6.22 
\node[pin={[pin distance=-0.6\onedegree,Flaamsted]00:{ 9}}] at (axis cs:{189.693},{40.875}) {}; % CVn,  6.35 
\node[pin={[pin distance=-0\onedegree,Bayer]-90:{$\boldsymbol\alpha$}}] at (axis cs:{310.358},{45.280}) {}; % Cyg,  1.33 
\node[pin={[pin distance=-0.4\onedegree,Bayer]00:{$\boldsymbol\epsilon$}}] at (axis cs:{311.553},{33.970}) {}; % Cyg,  2.49 
\node[pin={[pin distance=-0.4\onedegree,Bayer]00:{$\boldsymbol\gamma$}}] at (axis cs:{305.557},{40.257}) {}; % Cyg,  2.23 
\node[pin={[pin distance=-0.6\onedegree,Bayer]00:{$\boldsymbol\beta^1$}}] at (axis cs:{292.680},{27.960}) {}; % Cyg,  3.08 
\node[pin={[pin distance=-0.4\onedegree,Bayer]00:{$\boldsymbol\delta$}}] at (axis cs:{296.244},{45.131}) {}; % Cyg,  2.91 
\node[pin={[pin distance=-0.6\onedegree,Bayer]00:{$\boldsymbol\eta$}}] at (axis cs:{299.077},{35.083}) {}; % Cyg,  3.90 
\node[pin={[pin distance=-0.6\onedegree,Bayer]90:{$\boldsymbol\iota^1$}}] at (axis cs:{291.858},{52.320}) {}; % Cyg,  5.73 
\node[pin={[pin distance=-0.6\onedegree,Bayer]00:{$\boldsymbol\iota^2$}}] at (axis cs:{292.426},{51.730}) {}; % Cyg,  3.77 
\node[pin={[pin distance=-0.6\onedegree,Bayer]00:{$\boldsymbol\kappa$}}] at (axis cs:{289.276},{53.368}) {}; % Cyg,  3.80 
\node[pin={[pin distance=-0.6\onedegree,Bayer]00:{$\boldsymbol\lambda$}}] at (axis cs:{311.852},{36.491}) {}; % Cyg,  4.57 
\node[pin={[pin distance=-0.6\onedegree,Bayer]00:{$\boldsymbol\mu^1$}}] at (axis cs:{326.036},{28.742}) {}; % Cyg,  4.77 
\node[pin={[pin distance=-0.6\onedegree,Bayer]00:{$\boldsymbol\nu$}}] at (axis cs:{314.293},{41.167}) {}; % Cyg,  3.94 
\node[pin={[pin distance=-0.6\onedegree,Bayer]00:{$\boldsymbol\omega^{1,2}$}}] at (axis cs:{307.515},{48.952}) {}; % Cyg,  4.95 
%\node[pin={[pin distance=-0.6\onedegree,Bayer]00:{$\boldsymbol\omega^2$}}] at (axis cs:{307.828},{49.220}) {}; % Cyg,  5.45 
\node[pin={[pin distance=-0.6\onedegree,Bayer]00:{$\boldsymbol\pi^1$}}] at (axis cs:{325.524},{51.190}) {}; % Cyg,  4.68 
\node[pin={[pin distance=-0.6\onedegree,Bayer]00:{$\boldsymbol\pi^2$}}] at (axis cs:{326.698},{49.309}) {}; % Cyg,  4.25 
\node[pin={[pin distance=-0.6\onedegree,Bayer]00:{$\boldsymbol\psi$}}] at (axis cs:{298.907},{52.439}) {}; % Cyg,  4.91 
\node[pin={[pin distance=-0.6\onedegree,Bayer]00:{$\boldsymbol\rho$}}] at (axis cs:{323.495},{45.592}) {}; % Cyg,  3.99 
\node[pin={[pin distance=-0.6\onedegree,Bayer]00:{$\boldsymbol\sigma$}}] at (axis cs:{319.354},{39.394}) {}; % Cyg,  4.26 
\node[pin={[pin distance=-0.6\onedegree,Bayer]00:{$\boldsymbol\tau$}}] at (axis cs:{318.698},{38.045}) {}; % Cyg,  3.74 
\node[pin={[pin distance=-0.6\onedegree,Bayer]00:{$\boldsymbol\upsilon$}}] at (axis cs:{319.479},{34.897}) {}; % Cyg,  4.42 
\node[pin={[pin distance=-0.6\onedegree,Bayer]00:{$\boldsymbol\varphi$}}] at (axis cs:{294.844},{30.153}) {}; % Cyg,  4.68 
\node[pin={[pin distance=-0.6\onedegree,Bayer]00:{$\boldsymbol\vartheta$}}] at (axis cs:{294.111},{50.221}) {}; % Cyg,  4.50 
\node[pin={[pin distance=-0.4\onedegree,Bayer]00:{$\boldsymbol\xi$}}] at (axis cs:{316.233},{43.928}) {}; % Cyg,  3.71 
\node[pin={[pin distance=-0.4\onedegree,Bayer]180:{$\boldsymbol\zeta$}}] at (axis cs:{318.234},{30.227}) {}; % Cyg,  3.20 
\node[pin={[pin distance=-0.6\onedegree,Flaamsted]00:{11}}] at (axis cs:{293.951},{36.944}) {}; % Cyg,  6.04 
\node[pin={[pin distance=-0.6\onedegree,Flaamsted]00:{14}}] at (axis cs:{294.860},{42.818}) {}; % Cyg,  5.41 
\node[pin={[pin distance=-0.6\onedegree,Flaamsted]00:{15}}] at (axis cs:{296.069},{37.354}) {}; % Cyg,  4.89 
\node[pin={[pin distance=-0.6\onedegree,Flaamsted]00:{16}}] at (axis cs:{295.454},{50.525}) {}; % Cyg,  5.96 
\node[pin={[pin distance=-0.6\onedegree,Flaamsted]00:{17}}] at (axis cs:{296.607},{33.727}) {}; % Cyg,  5.00 
\node[pin={[pin distance=-0.6\onedegree,Flaamsted]00:{19}}] at (axis cs:{297.642},{38.722}) {}; % Cyg,  5.19 
\node[pin={[pin distance=-0.6\onedegree,Flaamsted]00:{ 2}}] at (axis cs:{291.032},{29.621}) {}; % Cyg,  4.99 
\node[pin={[pin distance=-0.6\onedegree,Flaamsted]00:{20}}] at (axis cs:{297.657},{52.988}) {}; % Cyg,  5.02 
\node[pin={[pin distance=-0.6\onedegree,Flaamsted]00:{22}}] at (axis cs:{298.966},{38.487}) {}; % Cyg,  4.95 
\node[pin={[pin distance=-0.6\onedegree,Flaamsted]00:{23}}] at (axis cs:{298.322},{57.523}) {}; % Cyg,  5.14 
\node[pin={[pin distance=-0.6\onedegree,Flaamsted]00:{25}}] at (axis cs:{299.980},{37.043}) {}; % Cyg,  5.15 
\node[pin={[pin distance=-0.6\onedegree,Flaamsted]00:{26}}] at (axis cs:{300.340},{50.105}) {}; % Cyg,  5.06 
\node[pin={[pin distance=-0.6\onedegree,Flaamsted]90:{27}}] at (axis cs:{301.591},{35.972}) {}; % Cyg,  5.39 
\node[pin={[pin distance=-0.6\onedegree,Flaamsted]00:{28}}] at (axis cs:{302.357},{36.840}) {}; % Cyg,  4.94 
\node[pin={[pin distance=-0.6\onedegree,Flaamsted]00:{29}}] at (axis cs:{303.633},{36.806}) {}; % Cyg,  4.95 
\node[pin={[pin distance=-0.4\onedegree,Flaamsted]00:{31}}] at (axis cs:{303.408},{46.741}) {}; % Cyg,  3.81 
\node[pin={[pin distance=-0.4\onedegree,Flaamsted]180:{32}}] at (axis cs:{303.868},{47.714}) {}; % Cyg,  4.00 
\node[pin={[pin distance=-0.4\onedegree,Flaamsted]00:{33}}] at (axis cs:{303.349},{56.568}) {}; % Cyg,  4.28 
\node[pin={[pin distance=-0.6\onedegree,Flaamsted]-90:{34}}] at (axis cs:{304.447},{38.033}) {}; % Cyg,  4.79 
\node[pin={[pin distance=-0.6\onedegree,Flaamsted]00:{35}}] at (axis cs:{304.663},{34.983}) {}; % Cyg,  5.18 
\node[pin={[pin distance=-0.6\onedegree,Flaamsted]00:{36}}] at (axis cs:{304.619},{37.000}) {}; % Cyg,  5.58 
\node[pin={[pin distance=-0.6\onedegree,Flaamsted]00:{39}}] at (axis cs:{305.965},{32.190}) {}; % Cyg,  4.43 
\node[pin={[pin distance=-0.6\onedegree,Flaamsted]-90:{ 4}}] at (axis cs:{291.538},{36.318}) {}; % Cyg,  5.18 
\node[pin={[pin distance=-0.6\onedegree,Flaamsted]00:{40}}] at (axis cs:{306.893},{38.440}) {}; % Cyg,  5.64 
\node[pin={[pin distance=-0.6\onedegree,Flaamsted]90:{41}}] at (axis cs:{307.349},{30.369}) {}; % Cyg,  4.01 
\node[pin={[pin distance=-0.6\onedegree,Flaamsted]00:{42}}] at (axis cs:{307.335},{36.455}) {}; % Cyg,  5.91 
\node[pin={[pin distance=-0.6\onedegree,Flaamsted]00:{43}}] at (axis cs:{306.759},{49.383}) {}; % Cyg,  5.73 
\node[pin={[pin distance=-0.6\onedegree,Flaamsted]180:{44}}] at (axis cs:{307.747},{36.936}) {}; % Cyg,  6.21 
\node[pin={[pin distance=-0.6\onedegree,Flaamsted]00:{47}}] at (axis cs:{308.476},{35.251}) {}; % Cyg,  4.65 
\node[pin={[pin distance=-0.6\onedegree,Flaamsted]00:{48}}] at (axis cs:{309.382},{31.572}) {}; % Cyg,  6.32 
\node[pin={[pin distance=-0.6\onedegree,Flaamsted]00:{49}}] at (axis cs:{310.261},{32.307}) {}; % Cyg,  5.51 
\node[pin={[pin distance=-0.6\onedegree,Flaamsted]00:{51}}] at (axis cs:{310.553},{50.340}) {}; % Cyg,  5.41 
\node[pin={[pin distance=-0.6\onedegree,Flaamsted]180:{52}}] at (axis cs:{311.416},{30.720}) {}; % Cyg,  4.23 
\node[pin={[pin distance=-0.6\onedegree,Flaamsted]-90:{55}}] at (axis cs:{312.235},{46.114}) {}; % Cyg,  4.86 
\node[pin={[pin distance=-0.6\onedegree,Flaamsted]00:{56}}] at (axis cs:{312.521},{44.059}) {}; % Cyg,  5.07 
\node[pin={[pin distance=-0.6\onedegree,Flaamsted]00:{57}}] at (axis cs:{313.311},{44.387}) {}; % Cyg,  4.79 
\node[pin={[pin distance=-0.6\onedegree,Flaamsted]00:{59}}] at (axis cs:{314.956},{47.521}) {}; % Cyg,  4.76 
\node[pin={[pin distance=-0.6\onedegree,Flaamsted]00:{60}}] at (axis cs:{315.296},{46.156}) {}; % Cyg,  5.43 
\node[pin={[pin distance=-0.6\onedegree,Flaamsted]00:{61}}] at (axis cs:{316.725},{38.749}) {}; % Cyg,  5.20 
\node[pin={[pin distance=-0.6\onedegree,Flaamsted]00:{63}}] at (axis cs:{316.650},{47.648}) {}; % Cyg,  4.53 
\node[pin={[pin distance=-0.6\onedegree,Flaamsted]00:{68}}] at (axis cs:{319.613},{43.946}) {}; % Cyg,  5.05 
\node[pin={[pin distance=-0.6\onedegree,Flaamsted]-90:{69}}] at (axis cs:{321.446},{36.667}) {}; % Cyg,  5.92 
\node[pin={[pin distance=-0.6\onedegree,Flaamsted]180:{70}}] at (axis cs:{321.839},{37.117}) {}; % Cyg,  5.30 
\node[pin={[pin distance=-0.6\onedegree,Flaamsted]00:{71}}] at (axis cs:{322.362},{46.540}) {}; % Cyg,  5.23 
\node[pin={[pin distance=-0.6\onedegree,Flaamsted]00:{72}}] at (axis cs:{323.694},{38.534}) {}; % Cyg,  4.87 
\node[pin={[pin distance=-0.6\onedegree,Flaamsted]00:{74}}] at (axis cs:{324.237},{40.413}) {}; % Cyg,  5.05 
\node[pin={[pin distance=-0.6\onedegree,Flaamsted]00:{75}}] at (axis cs:{325.046},{43.274}) {}; % Cyg,  5.10 
\node[pin={[pin distance=-0.6\onedegree,Flaamsted]-90:{76}}] at (axis cs:{325.393},{40.805}) {}; % Cyg,  6.08 
\node[pin={[pin distance=-0.6\onedegree,Flaamsted]90:{77}}] at (axis cs:{325.596},{41.077}) {}; % Cyg,  5.73 
\node[pin={[pin distance=-0.6\onedegree,Flaamsted]00:{79}}] at (axis cs:{325.857},{38.284}) {}; % Cyg,  5.70 
\node[pin={[pin distance=-0.6\onedegree,Flaamsted]00:{ 8}}] at (axis cs:{292.943},{34.453}) {}; % Cyg,  4.74 
\node[pin={[pin distance=-0.6\onedegree,Flaamsted]00:{ 9}}] at (axis cs:{293.712},{29.463}) {}; % Cyg,  5.41 
\node[pin={[pin distance=-0.4\onedegree,Bayer]180:{$\boldsymbol\gamma$}}] at (axis cs:{269.152},{51.489}) {}; % Dra,  2.23 
\node[pin={[pin distance=-0.6\onedegree,Bayer]00:{$\boldsymbol\alpha$}}] at (axis cs:{211.097},{64.376}) {}; % Dra,  3.65 
\node[pin={[pin distance=-0.4\onedegree,Bayer]00:{$\boldsymbol\beta$}}] at (axis cs:{262.608},{52.301}) {}; % Dra,  2.80 
\node[pin={[pin distance=-0.4\onedegree,Bayer]00:{$\boldsymbol\chi$}}] at (axis cs:{275.264},{72.733}) {}; % Dra,  3.57 
\node[pin={[pin distance=-0.4\onedegree,Bayer]00:{$\boldsymbol\delta$}}] at (axis cs:{288.139},{67.661}) {}; % Dra,  3.08 
\node[pin={[pin distance=-0.6\onedegree,Bayer]00:{$\boldsymbol\epsilon$}}] at (axis cs:{297.043},{70.268}) {}; % Dra,  3.84 
\node[pin={[pin distance=-0.4\onedegree,Bayer]00:{$\boldsymbol\eta$}}] at (axis cs:{245.998},{61.514}) {}; % Dra,  2.73 
\node[pin={[pin distance=-0.4\onedegree,Bayer]00:{$\boldsymbol\iota$}}] at (axis cs:{231.232},{58.966}) {}; % Dra,  3.30 
%\node[pin={[pin distance=-0.6\onedegree,Bayer]00:{$\boldsymbol\kappa$}}] at (axis cs:{188.371},{69.788}) {}; % Dra,  3.89 
\node[pin={[pin distance=-0.6\onedegree,Bayer]00:{$\boldsymbol\lambda$}}] at (axis cs:{172.851},{69.331}) {}; % Dra,  3.81 
\node[pin={[pin distance=-0.6\onedegree,Bayer]00:{$\boldsymbol\mu$}}] at (axis cs:{256.334},{54.470}) {}; % Dra,  5.55 
\node[pin={[pin distance=-0.6\onedegree,Bayer]00:{$\boldsymbol\nu^2$}}] at (axis cs:{263.067},{55.173}) {}; % Dra,  4.86 
\node[pin={[pin distance=-0.6\onedegree,Bayer]00:{$\boldsymbol\omega$}}] at (axis cs:{264.238},{68.758}) {}; % Dra,  4.78 
\node[pin={[pin distance=-0.6\onedegree,Bayer]00:{$\boldsymbol\omicron$}}] at (axis cs:{282.800},{59.388}) {}; % Dra,  4.64 
\node[pin={[pin distance=-0.6\onedegree,Bayer]00:{$\boldsymbol\pi$}}] at (axis cs:{290.167},{65.714}) {}; % Dra,  4.59 
\node[pin={[pin distance=-0.6\onedegree,Bayer]00:{$\boldsymbol\psi^1$}}] at (axis cs:{265.485},{72.149}) {}; % Dra,  4.57 
\node[pin={[pin distance=-0.6\onedegree,Bayer]00:{$\boldsymbol\psi^2$}}] at (axis cs:{268.796},{72.005}) {}; % Dra,  5.42 
\node[pin={[pin distance=-0.6\onedegree,Bayer]00:{$\boldsymbol\rho$}}] at (axis cs:{300.704},{67.873}) {}; % Dra,  4.51 
\node[pin={[pin distance=-0.6\onedegree,Bayer]00:{$\boldsymbol\sigma$}}] at (axis cs:{293.090},{69.661}) {}; % Dra,  4.67 
\node[pin={[pin distance=-0.6\onedegree,Bayer]00:{$\boldsymbol\tau$}}] at (axis cs:{288.888},{73.355}) {}; % Dra,  4.45 
\node[pin={[pin distance=-0.6\onedegree,Bayer]00:{$\boldsymbol\upsilon$}}] at (axis cs:{283.599},{71.297}) {}; % Dra,  4.82 
\node[pin={[pin distance=-0.6\onedegree,Bayer]00:{$\boldsymbol\varphi$}}] at (axis cs:{275.189},{71.338}) {}; % Dra,  4.23 
\node[pin={[pin distance=-0.6\onedegree,Bayer]00:{$\boldsymbol\vartheta$}}] at (axis cs:{240.472},{58.565}) {}; % Dra,  4.01 
\node[pin={[pin distance=-0.6\onedegree,Bayer]00:{$\boldsymbol\xi$}}] at (axis cs:{268.382},{56.873}) {}; % Dra,  3.74 
\node[pin={[pin distance=-0.4\onedegree,Bayer]180:{$\boldsymbol\zeta$}}] at (axis cs:{257.197},{65.714}) {}; % Dra,  3.18 
\node[pin={[pin distance=-0.6\onedegree,Flaamsted]00:{10}}] at (axis cs:{207.858},{64.723}) {}; % Dra,  4.61 
\node[pin={[pin distance=-0.6\onedegree,Flaamsted]00:{15}}] at (axis cs:{246.996},{68.768}) {}; % Dra,  4.97 
\node[pin={[pin distance=-0.6\onedegree,Flaamsted]00:{17}}] at (axis cs:{249.057},{52.924}) {}; % Dra,  5.08 
\node[pin={[pin distance=-0.6\onedegree,Flaamsted]00:{18}}] at (axis cs:{250.230},{64.589}) {}; % Dra,  4.84 
\node[pin={[pin distance=-0.6\onedegree,Flaamsted]180:{19}}] at (axis cs:{254.007},{65.135}) {}; % Dra,  4.88 
\node[pin={[pin distance=-0.6\onedegree,Flaamsted]00:{ 2}}] at (axis cs:{174.012},{69.323}) {}; % Dra,  5.20 
\node[pin={[pin distance=-0.6\onedegree,Flaamsted]00:{20}}] at (axis cs:{254.105},{65.039}) {}; % Dra,  6.44 
\node[pin={[pin distance=-0.6\onedegree,Flaamsted]00:{26}}] at (axis cs:{263.748},{61.874}) {}; % Dra,  5.25 
\node[pin={[pin distance=-0.6\onedegree,Flaamsted]00:{27}}] at (axis cs:{262.991},{68.135}) {}; % Dra,  5.08 
\node[pin={[pin distance=-0.6\onedegree,Flaamsted]00:{ 3}}] at (axis cs:{175.618},{66.745}) {}; % Dra,  5.31 
\node[pin={[pin distance=-0.6\onedegree,Flaamsted]180:{30}}] at (axis cs:{267.268},{50.781}) {}; % Dra,  5.02 
\node[pin={[pin distance=-0.6\onedegree,Flaamsted]00:{35}}] at (axis cs:{267.363},{76.963}) {}; % Dra,  5.03 
\node[pin={[pin distance=-0.6\onedegree,Flaamsted]00:{36}}] at (axis cs:{273.474},{64.397}) {}; % Dra,  5.00 
\node[pin={[pin distance=-0.6\onedegree,Flaamsted]00:{37}}] at (axis cs:{273.821},{68.756}) {}; % Dra,  5.96 
\node[pin={[pin distance=-0.6\onedegree,Flaamsted]00:{39}}] at (axis cs:{275.978},{58.801}) {}; % Dra,  5.04 
\node[pin={[pin distance=-0.6\onedegree,Flaamsted]00:{ 4}}] at (axis cs:{187.528},{69.201}) {}; % Dra,  5.03 
\node[pin={[pin distance=-0.6\onedegree,Flaamsted]00:{41}}] at (axis cs:{270.038},{80.004}) {}; % Dra,  5.68 
\node[pin={[pin distance=-0.6\onedegree,Flaamsted]00:{42}}] at (axis cs:{276.496},{65.563}) {}; % Dra,  4.83 
\node[pin={[pin distance=-0.6\onedegree,Flaamsted]00:{45}}] at (axis cs:{278.144},{57.045}) {}; % Dra,  4.78 
\node[pin={[pin distance=-0.6\onedegree,Flaamsted]00:{46}}] at (axis cs:{280.658},{55.539}) {}; % Dra,  5.03 
\node[pin={[pin distance=-0.6\onedegree,Flaamsted]00:{48}}] at (axis cs:{284.188},{57.815}) {}; % Dra,  5.67 
\node[pin={[pin distance=-0.6\onedegree,Flaamsted]00:{49}}] at (axis cs:{285.181},{55.658}) {}; % Dra,  5.52 
\node[pin={[pin distance=-0.6\onedegree,Flaamsted]00:{50}}] at (axis cs:{281.593},{75.434}) {}; % Dra,  5.37 
\node[pin={[pin distance=-0.6\onedegree,Flaamsted]00:{51}}] at (axis cs:{286.230},{53.396}) {}; % Dra,  5.41 
\node[pin={[pin distance=-0.6\onedegree,Flaamsted]00:{53}}] at (axis cs:{287.919},{56.859}) {}; % Dra,  5.14 
\node[pin={[pin distance=-0.6\onedegree,Flaamsted]180:{54}}] at (axis cs:{288.480},{57.705}) {}; % Dra,  5.00 
\node[pin={[pin distance=-0.6\onedegree,Flaamsted]00:{55}}] at (axis cs:{287.441},{65.978}) {}; % Dra,  6.27 
\node[pin={[pin distance=-0.6\onedegree,Flaamsted]00:{59}}] at (axis cs:{287.291},{76.560}) {}; % Dra,  5.12 
\node[pin={[pin distance=-0.6\onedegree,Flaamsted]-90:{ 6}}] at (axis cs:{188.683},{70.022}) {}; % Dra,  4.95 
\node[pin={[pin distance=-0.6\onedegree,Flaamsted]90:{64}}] at (axis cs:{300.369},{64.821}) {}; % Dra,  5.23 
\node[pin={[pin distance=-0.6\onedegree,Flaamsted]-90:{65}}] at (axis cs:{300.584},{64.634}) {}; % Dra,  6.26 
\node[pin={[pin distance=-0.6\onedegree,Flaamsted]180:{66}}] at (axis cs:{301.387},{61.995}) {}; % Dra,  5.41 
\node[pin={[pin distance=-0.6\onedegree,Flaamsted]180:{68}}] at (axis cs:{302.895},{62.078}) {}; % Dra,  5.71 
\node[pin={[pin distance=-0.6\onedegree,Flaamsted]00:{ 7}}] at (axis cs:{191.893},{66.790}) {}; % Dra,  5.43 
\node[pin={[pin distance=-0.6\onedegree,Flaamsted]180:{71}}] at (axis cs:{304.903},{62.257}) {}; % Dra,  5.72 
\node[pin={[pin distance=-0.6\onedegree,Flaamsted]00:{73}}] at (axis cs:{307.877},{74.955}) {}; % Dra,  5.19 
\node[pin={[pin distance=-0.6\onedegree,Flaamsted]00:{74}}] at (axis cs:{307.365},{81.091}) {}; % Dra,  5.97 
\node[pin={[pin distance=-0.6\onedegree,Flaamsted]90:{75}}] at (axis cs:{307.061},{81.423}) {}; % Dra,  5.37 
\node[pin={[pin distance=-0.6\onedegree,Flaamsted]00:{76}}] at (axis cs:{310.647},{82.531}) {}; % Dra,  5.75 
\node[pin={[pin distance=-0.6\onedegree,Flaamsted]00:{ 8}}] at (axis cs:{193.869},{65.438}) {}; % Dra,  5.23 
\node[pin={[pin distance=-0.6\onedegree,Flaamsted]00:{ 9}}] at (axis cs:{194.979},{66.597}) {}; % Dra,  5.37 
\node[pin={[pin distance=-0.4\onedegree,Bayer]00:{$\boldsymbol\alpha$}}] at (axis cs:{113.649},{31.888}) {}; % Gem,  1.58 
\node[pin={[pin distance=-0.4\onedegree,Bayer]00:{$\boldsymbol\beta$}}] at (axis cs:{116.329},{28.026}) {}; % Gem,  1.22 
\node[pin={[pin distance=-0.6\onedegree,Bayer]00:{$\boldsymbol\chi$}}] at (axis cs:{120.880},{27.794}) {}; % Gem,  4.94 
\node[pin={[pin distance=-0.6\onedegree,Bayer]00:{$\boldsymbol\iota$}}] at (axis cs:{111.432},{27.798}) {}; % Gem,  3.79 
\node[pin={[pin distance=-0.6\onedegree,Bayer]00:{$\boldsymbol\omicron$}}] at (axis cs:{114.791},{34.584}) {}; % Gem,  4.89 
\node[pin={[pin distance=-0.6\onedegree,Bayer]00:{$\boldsymbol\pi$}}] at (axis cs:{116.876},{33.415}) {}; % Gem,  5.14 
\node[pin={[pin distance=-0.6\onedegree,Bayer]00:{$\boldsymbol\rho$}}] at (axis cs:{112.278},{31.785}) {}; % Gem,  4.18 
\node[pin={[pin distance=-0.6\onedegree,Bayer]00:{$\boldsymbol\sigma$}}] at (axis cs:{115.828},{28.884}) {}; % Gem,  4.25 
\node[pin={[pin distance=-0.6\onedegree,Bayer]00:{$\boldsymbol\tau$}}] at (axis cs:{107.785},{30.245}) {}; % Gem,  4.39 
\node[pin={[pin distance=-0.6\onedegree,Bayer]00:{$\boldsymbol\vartheta$}}] at (axis cs:{103.197},{33.961}) {}; % Gem,  3.61 
\node[pin={[pin distance=-0.6\onedegree,Flaamsted]00:{25}}] at (axis cs:{100.337},{28.196}) {}; % Gem,  6.43 
\node[pin={[pin distance=-0.6\onedegree,Flaamsted]00:{28}}] at (axis cs:{101.189},{28.971}) {}; % Gem,  5.42 
\node[pin={[pin distance=-0.6\onedegree,Flaamsted]00:{53}}] at (axis cs:{108.988},{27.897}) {}; % Gem,  5.75 
\node[pin={[pin distance=-0.6\onedegree,Flaamsted]00:{59}}] at (axis cs:{111.139},{27.638}) {}; % Gem,  5.78 
\node[pin={[pin distance=-0.6\onedegree,Flaamsted]00:{64}}] at (axis cs:{112.335},{28.118}) {}; % Gem,  5.07 
\node[pin={[pin distance=-0.6\onedegree,Flaamsted]00:{65}}] at (axis cs:{112.453},{27.916}) {}; % Gem,  5.01 
\node[pin={[pin distance=-0.6\onedegree,Flaamsted]00:{70}}] at (axis cs:{114.637},{35.048}) {}; % Gem,  5.58 
\node[pin={[pin distance=-0.6\onedegree,Bayer]00:{$\boldsymbol\chi$}}] at (axis cs:{238.169},{42.451}) {}; % Her,  4.61 
\node[pin={[pin distance=-0.6\onedegree,Bayer]00:{$\boldsymbol\epsilon$}}] at (axis cs:{255.072},{30.926}) {}; % Her,  3.91 
\node[pin={[pin distance=-0.4\onedegree,Bayer]00:{$\boldsymbol\eta$}}] at (axis cs:{250.724},{38.922}) {}; % Her,  3.48 
\node[pin={[pin distance=-0.4\onedegree,Bayer]00:{$\boldsymbol\iota$}}] at (axis cs:{264.866},{46.006}) {}; % Her,  3.81 
\node[pin={[pin distance=-0.6\onedegree,Bayer]00:{$\boldsymbol\mu$}}] at (axis cs:{266.615},{27.721}) {}; % Her,  3.42 
\node[pin={[pin distance=-0.6\onedegree,Bayer]00:{$\boldsymbol\nu$}}] at (axis cs:{269.626},{30.189}) {}; % Her,  4.41 
\node[pin={[pin distance=-0.6\onedegree,Bayer]00:{$\boldsymbol\omicron$}}] at (axis cs:{271.886},{28.762}) {}; % Her,  3.84 
\node[pin={[pin distance=-0.4\onedegree,Bayer]00:{$\boldsymbol\pi$}}] at (axis cs:{258.762},{36.809}) {}; % Her,  3.14 
\node[pin={[pin distance=-0.6\onedegree,Bayer]00:{$\boldsymbol\rho$}}] at (axis cs:{260.921},{37.146}) {}; % Her,  4.51 
\node[pin={[pin distance=-0.6\onedegree,Bayer]00:{$\boldsymbol\sigma$}}] at (axis cs:{248.526},{42.437}) {}; % Her,  4.21 
\node[pin={[pin distance=-0.6\onedegree,Bayer]00:{$\boldsymbol\tau$}}] at (axis cs:{244.935},{46.313}) {}; % Her,  3.90 
\node[pin={[pin distance=-0.6\onedegree,Bayer]00:{$\boldsymbol\upsilon$}}] at (axis cs:{240.700},{46.037}) {}; % Her,  4.72 
\node[pin={[pin distance=-0.6\onedegree,Bayer]00:{$\boldsymbol\varphi$}}] at (axis cs:{242.192},{44.935}) {}; % Her,  4.24 
\node[pin={[pin distance=-0.4\onedegree,Bayer]00:{$\boldsymbol\vartheta$}}] at (axis cs:{269.063},{37.251}) {}; % Her,  3.84 
\node[pin={[pin distance=-0.6\onedegree,Bayer]00:{$\boldsymbol\xi$}}] at (axis cs:{269.441},{29.248}) {}; % Her,  3.70 
\node[pin={[pin distance=-0.4\onedegree,Bayer]-90:{$\boldsymbol\zeta$}}] at (axis cs:{250.322},{31.603}) {}; % Her,  2.85 
\node[pin={[pin distance=-0.6\onedegree,Flaamsted]00:{104}}] at (axis cs:{272.976},{31.405}) {}; % Her,  4.98 
\node[pin={[pin distance=-0.6\onedegree,Flaamsted]00:{107}}] at (axis cs:{275.254},{28.870}) {}; % Her,  5.13 
\node[pin={[pin distance=-0.6\onedegree,Flaamsted]00:{108}}] at (axis cs:{275.237},{29.859}) {}; % Her,  5.61 
\node[pin={[pin distance=-0.6\onedegree,Flaamsted]00:{ 2}}] at (axis cs:{238.658},{43.138}) {}; % Her,  5.36 
\node[pin={[pin distance=-0.6\onedegree,Flaamsted]00:{25}}] at (axis cs:{246.351},{37.394}) {}; % Her,  5.54 
\node[pin={[pin distance=-0.6\onedegree,Flaamsted]00:{30}}] at (axis cs:{247.161},{41.881}) {}; % Her,  4.90 
\node[pin={[pin distance=-0.6\onedegree,Flaamsted]00:{34}}] at (axis cs:{247.525},{48.961}) {}; % Her,  6.44 
\node[pin={[pin distance=-0.6\onedegree,Flaamsted]00:{ 4}}] at (axis cs:{238.877},{42.566}) {}; % Her,  5.75 
\node[pin={[pin distance=-0.6\onedegree,Flaamsted]00:{42}}] at (axis cs:{249.687},{48.928}) {}; % Her,  4.88 
\node[pin={[pin distance=-0.6\onedegree,Flaamsted]00:{50}}] at (axis cs:{252.662},{29.806}) {}; % Her,  5.72 
\node[pin={[pin distance=-0.6\onedegree,Flaamsted]90:{52}}] at (axis cs:{252.309},{45.983}) {}; % Her,  4.82 
\node[pin={[pin distance=-0.6\onedegree,Flaamsted]00:{53}}] at (axis cs:{253.242},{31.702}) {}; % Her,  5.34 
\node[pin={[pin distance=-0.6\onedegree,Flaamsted]00:{59}}] at (axis cs:{255.402},{33.568}) {}; % Her,  5.28 
\node[pin={[pin distance=-0.6\onedegree,Flaamsted]90:{61}}] at (axis cs:{255.876},{35.414}) {}; % Her,  6.27 
\node[pin={[pin distance=-0.6\onedegree,Flaamsted]00:{68}}] at (axis cs:{259.332},{33.100}) {}; % Her,  4.81 
\node[pin={[pin distance=-0.6\onedegree,Flaamsted]180:{69}}] at (axis cs:{259.418},{37.291}) {}; % Her,  4.63 
\node[pin={[pin distance=-0.6\onedegree,Flaamsted]00:{72}}] at (axis cs:{260.165},{32.468}) {}; % Her,  5.39 
\node[pin={[pin distance=-0.6\onedegree,Flaamsted]00:{74}}] at (axis cs:{260.088},{46.241}) {}; % Her,  5.50 
\node[pin={[pin distance=-0.6\onedegree,Flaamsted]00:{77}}] at (axis cs:{261.684},{48.260}) {}; % Her,  5.84 
\node[pin={[pin distance=-0.6\onedegree,Flaamsted]00:{78}}] at (axis cs:{262.957},{28.407}) {}; % Her,  5.66 
\node[pin={[pin distance=-0.6\onedegree,Flaamsted]00:{82}}] at (axis cs:{264.157},{48.586}) {}; % Her,  5.36 
\node[pin={[pin distance=-0.6\onedegree,Flaamsted]00:{90}}] at (axis cs:{268.325},{40.008}) {}; % Her,  5.17 
\node[pin={[pin distance=-0.6\onedegree,Flaamsted]180:{99}}] at (axis cs:{271.756},{30.562}) {}; % Her,  5.06 
\node[pin={[pin distance=-0.6\onedegree,Bayer]00:{$\boldsymbol\alpha$}}] at (axis cs:{337.823},{50.282}) {}; % Lac,  3.78 
\node[pin={[pin distance=-0.6\onedegree,Bayer]00:{$\boldsymbol\beta$}}] at (axis cs:{335.890},{52.229}) {}; % Lac,  4.43 
\node[pin={[pin distance=-0.4\onedegree,Flaamsted]90:{ 1}}] at (axis cs:{333.992},{37.748}) {}; % Lac,  4.12 
\node[pin={[pin distance=-0.6\onedegree,Flaamsted]00:{10}}] at (axis cs:{339.815},{39.050}) {}; % Lac,  4.89 
\node[pin={[pin distance=-0.6\onedegree,Flaamsted]00:{11}}] at (axis cs:{340.129},{44.276}) {}; % Lac,  4.50 
\node[pin={[pin distance=-0.6\onedegree,Flaamsted]90:{12}}] at (axis cs:{340.369},{40.225}) {}; % Lac,  5.24 
\node[pin={[pin distance=-0.6\onedegree,Flaamsted]90:{13}}] at (axis cs:{341.023},{41.819}) {}; % Lac,  5.11 
\node[pin={[pin distance=-0.6\onedegree,Flaamsted]00:{14}}] at (axis cs:{342.591},{41.953}) {}; % Lac,  5.93 
\node[pin={[pin distance=-0.6\onedegree,Flaamsted]00:{15}}] at (axis cs:{343.008},{43.312}) {}; % Lac,  4.94 
\node[pin={[pin distance=-0.6\onedegree,Flaamsted]00:{16}}] at (axis cs:{344.098},{41.604}) {}; % Lac,  5.60 
\node[pin={[pin distance=-0.6\onedegree,Flaamsted]00:{ 2}}] at (axis cs:{335.256},{46.537}) {}; % Lac,  4.56 
\node[pin={[pin distance=-0.6\onedegree,Flaamsted]00:{ 4}}] at (axis cs:{336.129},{49.476}) {}; % Lac,  4.60 
\node[pin={[pin distance=-0.4\onedegree,Flaamsted]00:{ 5}}] at (axis cs:{337.383},{47.707}) {}; % Lac,  4.35 
\node[pin={[pin distance=-0.6\onedegree,Flaamsted]00:{ 6}}] at (axis cs:{337.622},{43.123}) {}; % Lac,  4.53 
\node[pin={[pin distance=-0.6\onedegree,Flaamsted]00:{ 8}}] at (axis cs:{338.968},{39.634}) {}; % Lac,  5.71 
\node[pin={[pin distance=-0.6\onedegree,Flaamsted]00:{ 9}}] at (axis cs:{339.343},{51.545}) {}; % Lac,  4.65 
\node[pin={[pin distance=-0.6\onedegree,Flaamsted]00:{15}}] at (axis cs:{145.889},{29.974}) {}; % Leo,  5.64 
\node[pin={[pin distance=-0.6\onedegree,Bayer]00:{$\boldsymbol\beta$}}] at (axis cs:{156.971},{36.707}) {}; % LMi,  4.21 
\node[pin={[pin distance=-0.6\onedegree,Flaamsted]00:{10}}] at (axis cs:{143.556},{36.397}) {}; % LMi,  4.55 
\node[pin={[pin distance=-0.6\onedegree,Flaamsted]00:{11}}] at (axis cs:{143.915},{35.810}) {}; % LMi,  5.39 
\node[pin={[pin distance=-0.6\onedegree,Flaamsted]00:{13}}] at (axis cs:{145.678},{35.093}) {}; % LMi,  6.12 
\node[pin={[pin distance=-0.6\onedegree,Flaamsted]00:{19}}] at (axis cs:{149.421},{41.056}) {}; % LMi,  5.11 
\node[pin={[pin distance=-0.6\onedegree,Flaamsted]00:{20}}] at (axis cs:{150.253},{31.924}) {}; % LMi,  5.38 
\node[pin={[pin distance=-0.6\onedegree,Flaamsted]00:{21}}] at (axis cs:{151.857},{35.245}) {}; % LMi,  4.49 
\node[pin={[pin distance=-0.6\onedegree,Flaamsted]00:{22}}] at (axis cs:{153.776},{31.468}) {}; % LMi,  6.47 
\node[pin={[pin distance=-0.6\onedegree,Flaamsted]00:{23}}] at (axis cs:{154.060},{29.310}) {}; % LMi,  5.49 
\node[pin={[pin distance=-0.6\onedegree,Flaamsted]00:{24}}] at (axis cs:{154.117},{28.682}) {}; % LMi,  6.46 
\node[pin={[pin distance=-0.6\onedegree,Flaamsted]00:{27}}] at (axis cs:{155.776},{33.908}) {}; % LMi,  5.89 
\node[pin={[pin distance=-0.6\onedegree,Flaamsted]00:{28}}] at (axis cs:{156.036},{33.719}) {}; % LMi,  5.52 
\node[pin={[pin distance=-0.6\onedegree,Flaamsted]00:{30}}] at (axis cs:{156.478},{33.796}) {}; % LMi,  4.73 
\node[pin={[pin distance=-0.6\onedegree,Flaamsted]00:{32}}] at (axis cs:{157.527},{38.925}) {}; % LMi,  5.79 
\node[pin={[pin distance=-0.6\onedegree,Flaamsted]00:{33}}] at (axis cs:{157.964},{32.379}) {}; % LMi,  5.91 
\node[pin={[pin distance=-0.6\onedegree,Flaamsted]00:{34}}] at (axis cs:{158.379},{34.989}) {}; % LMi,  5.58 
\node[pin={[pin distance=-0.6\onedegree,Flaamsted]00:{35}}] at (axis cs:{159.089},{36.327}) {}; % LMi,  6.29 
\node[pin={[pin distance=-0.6\onedegree,Flaamsted]00:{37}}] at (axis cs:{159.680},{31.976}) {}; % LMi,  4.68 
\node[pin={[pin distance=-0.6\onedegree,Flaamsted]00:{38}}] at (axis cs:{159.782},{37.910}) {}; % LMi,  5.84 
\node[pin={[pin distance=-0.6\onedegree,Flaamsted]00:{42}}] at (axis cs:{161.466},{30.682}) {}; % LMi,  5.26 
\node[pin={[pin distance=-0.6\onedegree,Flaamsted]00:{43}}] at (axis cs:{162.238},{29.416}) {}; % LMi,  6.15 
\node[pin={[pin distance=-0.6\onedegree,Flaamsted]00:{44}}] at (axis cs:{162.474},{27.974}) {}; % LMi,  6.05 
\node[pin={[pin distance=-0.6\onedegree,Flaamsted]00:{46}}] at (axis cs:{163.328},{34.215}) {}; % LMi,  3.79 
\node[pin={[pin distance=-0.6\onedegree,Flaamsted]00:{ 7}}] at (axis cs:{142.680},{33.656}) {}; % LMi,  5.86 
\node[pin={[pin distance=-0.6\onedegree,Flaamsted]00:{ 8}}] at (axis cs:{142.885},{35.103}) {}; % LMi,  5.38 
\node[pin={[pin distance=-0.6\onedegree,Flaamsted]00:{ 9}}] at (axis cs:{143.377},{36.487}) {}; % LMi,  6.18 
\node[pin={[pin distance=-0.6\onedegree,Bayer]00:{$\boldsymbol\alpha$}}] at (axis cs:{140.264},{34.392}) {}; % Lyn,  3.13 
\node[pin={[pin distance=-0.6\onedegree,Flaamsted]90:{ 1}}] at (axis cs:{94.478},{61.515}) {}; % Lyn,  5.03 
\node[pin={[pin distance=-0.6\onedegree,Flaamsted]00:{11}}] at (axis cs:{99.410},{56.857}) {}; % Lyn,  5.88 
\node[pin={[pin distance=-0.6\onedegree,Flaamsted]00:{12}}] at (axis cs:{101.559},{59.442}) {}; % Lyn,  4.87 
\node[pin={[pin distance=-0.6\onedegree,Flaamsted]00:{13}}] at (axis cs:{101.706},{57.169}) {}; % Lyn,  5.34 
\node[pin={[pin distance=-0.6\onedegree,Flaamsted]00:{14}}] at (axis cs:{103.271},{59.448}) {}; % Lyn,  5.36 
\node[pin={[pin distance=-0.6\onedegree,Flaamsted]00:{15}}] at (axis cs:{104.319},{58.423}) {}; % Lyn,  4.36 
\node[pin={[pin distance=-0.6\onedegree,Flaamsted]00:{16}}] at (axis cs:{104.405},{45.094}) {}; % Lyn,  4.90 
\node[pin={[pin distance=-0.6\onedegree,Flaamsted]180:{18}}] at (axis cs:{108.979},{59.637}) {}; % Lyn,  5.20 
\node[pin={[pin distance=-0.6\onedegree,Flaamsted]00:{19}}] at (axis cs:{110.717},{55.281}) {}; % Lyn,  5.802
\node[pin={[pin distance=-0.6\onedegree,Flaamsted]180:{ 2}}] at (axis cs:{94.906},{59.011}) {}; % Lyn,  4.45 
\node[pin={[pin distance=-0.6\onedegree,Flaamsted]180:{21}}] at (axis cs:{111.679},{49.211}) {}; % Lyn,  4.62 
\node[pin={[pin distance=-0.6\onedegree,Flaamsted]00:{22}}] at (axis cs:{112.483},{49.672}) {}; % Lyn,  5.36 
\node[pin={[pin distance=-0.6\onedegree,Flaamsted]00:{23}}] at (axis cs:{115.206},{57.083}) {}; % Lyn,  6.08 
\node[pin={[pin distance=-0.6\onedegree,Flaamsted]00:{24}}] at (axis cs:{115.752},{58.710}) {}; % Lyn,  4.96 
\node[pin={[pin distance=-0.6\onedegree,Flaamsted]00:{25}}] at (axis cs:{118.622},{47.386}) {}; % Lyn,  6.25 
\node[pin={[pin distance=-0.6\onedegree,Flaamsted]00:{26}}] at (axis cs:{118.678},{47.564}) {}; % Lyn,  5.45 
\node[pin={[pin distance=-0.6\onedegree,Flaamsted]00:{27}}] at (axis cs:{122.114},{51.507}) {}; % Lyn,  4.80 
\node[pin={[pin distance=-0.6\onedegree,Flaamsted]00:{28}}] at (axis cs:{121.791},{43.260}) {}; % Lyn,  6.35 
\node[pin={[pin distance=-0.6\onedegree,Flaamsted]180:{29}}] at (axis cs:{124.460},{59.571}) {}; % Lyn,  5.64 
\node[pin={[pin distance=-0.6\onedegree,Flaamsted]00:{30}}] at (axis cs:{125.109},{57.743}) {}; % Lyn,  5.89 
\node[pin={[pin distance=-0.6\onedegree,Flaamsted]00:{31}}] at (axis cs:{125.709},{43.188}) {}; % Lyn,  4.24 
\node[pin={[pin distance=-0.6\onedegree,Flaamsted]00:{32}}] at (axis cs:{128.341},{36.436}) {}; % Lyn,  6.20 
\node[pin={[pin distance=-0.6\onedegree,Flaamsted]00:{33}}] at (axis cs:{128.683},{36.420}) {}; % Lyn,  5.76 
\node[pin={[pin distance=-0.6\onedegree,Flaamsted]00:{34}}] at (axis cs:{130.254},{45.834}) {}; % Lyn,  5.37 
\node[pin={[pin distance=-0.6\onedegree,Flaamsted]00:{35}}] at (axis cs:{132.987},{43.726}) {}; % Lyn,  5.15 
\node[pin={[pin distance=-0.6\onedegree,Flaamsted]00:{36}}] at (axis cs:{138.451},{43.218}) {}; % Lyn,  5.30 
\node[pin={[pin distance=-0.6\onedegree,Flaamsted]00:{38}}] at (axis cs:{139.711},{36.803}) {}; % Lyn,  3.82 
\node[pin={[pin distance=-0.6\onedegree,Flaamsted]00:{ 4}}] at (axis cs:{95.515},{59.372}) {}; % Lyn,  6.09 
\node[pin={[pin distance=-0.6\onedegree,Flaamsted]00:{42}}] at (axis cs:{144.591},{40.240}) {}; % Lyn,  5.29 
\node[pin={[pin distance=-0.6\onedegree,Flaamsted]00:{43}}] at (axis cs:{145.501},{39.758}) {}; % Lyn,  5.61 
\node[pin={[pin distance=-0.6\onedegree,Flaamsted]00:{ 5}}] at (axis cs:{96.704},{58.417}) {}; % Lyn,  5.19 
\node[pin={[pin distance=-0.6\onedegree,Flaamsted]00:{ 6}}] at (axis cs:{97.696},{58.162}) {}; % Lyn,  5.86 
\node[pin={[pin distance=-0.6\onedegree,Flaamsted]00:{ 7}}] at (axis cs:{98.637},{55.353}) {}; % Lyn,  6.45 
\node[pin={[pin distance=-0.6\onedegree,Flaamsted]180:{ 8}}] at (axis cs:{99.422},{61.481}) {}; % Lyn,  5.94 
\node[pin={[pin distance=-0.2\onedegree,Bayer]00:{$\boldsymbol\alpha$}}] at (axis cs:{279.235},{38.784}) {}; % Lyr,  0.03 
\node[pin={[pin distance=-0.6\onedegree,Bayer]180:{$\boldsymbol\beta$}}] at (axis cs:{282.520},{33.362}) {}; % Lyr,  3.52 
\node[pin={[pin distance=-0.6\onedegree,Bayer]00:{$\boldsymbol\delta^{1,2}$}}] at (axis cs:{283.431},{36.972}) {}; % Lyr,  5.58 
%\node[pin={[pin distance=-0.6\onedegree,Bayer]00:{$\boldsymbol\delta^2$}}] at (axis cs:{283.626},{36.898}) {}; % Lyr,  4.28 
\node[pin={[pin distance=-0.6\onedegree,Bayer]00:{$\boldsymbol\epsilon^1$}}] at (axis cs:{281.085},{39.670}) {}; % Lyr,  5.01 
\node[pin={[pin distance=-0.6\onedegree,Bayer]00:{$\boldsymbol\eta$}}] at (axis cs:{288.440},{39.146}) {}; % Lyr,  4.41 
\node[pin={[pin distance=-0.4\onedegree,Bayer]180:{$\boldsymbol\gamma$}}] at (axis cs:{284.736},{32.689}) {}; % Lyr,  3.25 
\node[pin={[pin distance=-0.6\onedegree,Bayer]00:{$\boldsymbol\iota$}}] at (axis cs:{286.826},{36.100}) {}; % Lyr,  5.26 
\node[pin={[pin distance=-0.6\onedegree,Bayer]00:{$\boldsymbol\kappa$}}] at (axis cs:{274.965},{36.064}) {}; % Lyr,  4.32 
\node[pin={[pin distance=-0.6\onedegree,Bayer]00:{$\boldsymbol\lambda$}}] at (axis cs:{285.003},{32.145}) {}; % Lyr,  4.94 
\node[pin={[pin distance=-0.6\onedegree,Bayer]90:{$\boldsymbol\mu$}}] at (axis cs:{276.057},{39.507}) {}; % Lyr,  5.12 
%\node[pin={[pin distance=-0.6\onedegree,Bayer]00:{$\boldsymbol\nu^1$}}] at (axis cs:{282.441},{32.813}) {}; % Lyr,  5.93 
\node[pin={[pin distance=-0.6\onedegree,Bayer]00:{$\boldsymbol\nu^{1,2}$}}] at (axis cs:{282.470},{32.551}) {}; % Lyr,  5.23 
\node[pin={[pin distance=-0.6\onedegree,Bayer]00:{$\boldsymbol\vartheta$}}] at (axis cs:{289.092},{38.134}) {}; % Lyr,  4.34 
\node[pin={[pin distance=-0.6\onedegree,Bayer]00:{$\boldsymbol\zeta^1$}}] at (axis cs:{281.193},{37.605}) {}; % Lyr,  4.34 
\node[pin={[pin distance=-0.6\onedegree,Flaamsted]00:{13}}] at (axis cs:{283.834},{43.946}) {}; % Lyr,  4.19 
\node[pin={[pin distance=-0.6\onedegree,Flaamsted]00:{16}}] at (axis cs:{285.360},{46.935}) {}; % Lyr,  5.01 
\node[pin={[pin distance=-0.6\onedegree,Flaamsted]00:{17}}] at (axis cs:{286.857},{32.502}) {}; % Lyr,  5.23 
\node[pin={[pin distance=-0.6\onedegree,Flaamsted]00:{19}}] at (axis cs:{287.942},{31.283}) {}; % Lyr,  5.94 
\node[pin={[pin distance=-0.4\onedegree,Bayer]00:{$\boldsymbol\beta$}}] at (axis cs:{345.944},{28.083}) {}; % Peg,  2.47 
\node[pin={[pin distance=-0.4\onedegree,Bayer]90:{$\boldsymbol\eta$}}] at (axis cs:{340.751},{30.221}) {}; % Peg,  2.94 
\node[pin={[pin distance=-0.6\onedegree,Bayer]00:{$\boldsymbol\omicron$}}] at (axis cs:{340.439},{29.308}) {}; % Peg,  4.80 
\node[pin={[pin distance=-0.6\onedegree,Bayer]90:{$\boldsymbol\pi^{1,2}$}}] at (axis cs:{332.307},{33.172}) {}; % Peg,  5.59 
%\node[pin={[pin distance=-0.6\onedegree,Bayer]00:{$\boldsymbol\pi^2$}}] at (axis cs:{332.497},{33.178}) {}; % Peg,  4.29 
\node[pin={[pin distance=-0.6\onedegree,Flaamsted]00:{14}}] at (axis cs:{327.461},{30.174}) {}; % Peg,  5.08 
\node[pin={[pin distance=-0.6\onedegree,Flaamsted]00:{15}}] at (axis cs:{328.125},{28.793}) {}; % Peg,  5.53 
\node[pin={[pin distance=-0.6\onedegree,Flaamsted]00:{23}}] at (axis cs:{331.394},{28.964}) {}; % Peg,  5.69 
\node[pin={[pin distance=-0.6\onedegree,Flaamsted]00:{32}}] at (axis cs:{335.331},{28.330}) {}; % Peg,  4.81 
\node[pin={[pin distance=-0.6\onedegree,Flaamsted]00:{38}}] at (axis cs:{337.508},{32.573}) {}; % Peg,  5.63 
\node[pin={[pin distance=-0.6\onedegree,Flaamsted]00:{63}}] at (axis cs:{350.206},{30.415}) {}; % Peg,  5.58 
\node[pin={[pin distance=-0.6\onedegree,Flaamsted]00:{64}}] at (axis cs:{350.479},{31.812}) {}; % Peg,  5.35 
\node[pin={[pin distance=-0.6\onedegree,Flaamsted]00:{67}}] at (axis cs:{351.212},{32.385}) {}; % Peg,  5.56 
\node[pin={[pin distance=-0.6\onedegree,Flaamsted]00:{72}}] at (axis cs:{353.488},{31.325}) {}; % Peg,  4.98 
\node[pin={[pin distance=-0.6\onedegree,Flaamsted]00:{73}}] at (axis cs:{353.659},{33.497}) {}; % Peg,  5.63 
\node[pin={[pin distance=-0.6\onedegree,Flaamsted]00:{78}}] at (axis cs:{355.998},{29.361}) {}; % Peg,  4.94 
\node[pin={[pin distance=-0.6\onedegree,Flaamsted]00:{79}}] at (axis cs:{357.414},{28.842}) {}; % Peg,  5.96 
\node[pin={[pin distance=-0.6\onedegree,Flaamsted]00:{85}}] at (axis cs:{0.542},{27.082}) {}; % Peg,  5.80 
\node[pin={[pin distance=-0.2\onedegree,Bayer]00:{$\boldsymbol\alpha$}}] at (axis cs:{51.081},{49.861}) {}; % Per,  1.81 
\node[pin={[pin distance=-0.2\onedegree,Bayer]00:{$\boldsymbol\beta$}}] at (axis cs:{47.042},{40.956}) {}; % Per,  2.11 
\node[pin={[pin distance=-0.4\onedegree,Bayer]00:{$\boldsymbol\delta$}}] at (axis cs:{55.731},{47.787}) {}; % Per,  3.02 
\node[pin={[pin distance=-0.4\onedegree,Bayer]00:{$\boldsymbol\epsilon$}}] at (axis cs:{59.463},{40.010}) {}; % Per,  2.91 
\node[pin={[pin distance=-0.4\onedegree,Bayer]00:{$\boldsymbol\eta$}}] at (axis cs:{42.674},{55.895}) {}; % Per,  3.76 
\node[pin={[pin distance=-0.4\onedegree,Bayer]00:{$\boldsymbol\gamma$}}] at (axis cs:{46.199},{53.506}) {}; % Per,  2.94 
\node[pin={[pin distance=-0.4\onedegree,Bayer]90:{$\boldsymbol\iota$}}] at (axis cs:{47.267},{49.613}) {}; % Per,  4.05 
\node[pin={[pin distance=-0.4\onedegree,Bayer]00:{$\boldsymbol\kappa$}}] at (axis cs:{47.374},{44.857}) {}; % Per,  3.79 
\node[pin={[pin distance=-0.4\onedegree,Bayer]00:{$\boldsymbol\lambda$}}] at (axis cs:{61.646},{50.351}) {}; % Per,  4.28 
\node[pin={[pin distance=-0.4\onedegree,Bayer]90:{$\boldsymbol\mu$}}] at (axis cs:{63.724},{48.409}) {}; % Per,  4.15 
\node[pin={[pin distance=-0.4\onedegree,Bayer]00:{$\boldsymbol\nu$}}] at (axis cs:{56.298},{42.578}) {}; % Per,  3.78 
\node[pin={[pin distance=-0.4\onedegree,Bayer]180:{$\boldsymbol\omega$}}] at (axis cs:{47.822},{39.611}) {}; % Per,  4.61 
\node[pin={[pin distance=-0.4\onedegree,Bayer]00:{$\boldsymbol\omicron$}}] at (axis cs:{56.080},{32.288}) {}; % Per,  3.86 
\node[pin={[pin distance=-0.4\onedegree,Bayer]00:{$\boldsymbol\pi$}}] at (axis cs:{44.690},{39.663}) {}; % Per,  4.69 
\node[pin={[pin distance=-0.4\onedegree,Bayer]00:{$\boldsymbol\psi$}}] at (axis cs:{54.122},{48.192}) {}; % Per,  4.32 
\node[pin={[pin distance=-0.4\onedegree,Bayer]00:{$\boldsymbol\rho$}}] at (axis cs:{46.294},{38.840}) {}; % Per,  3.41 
\node[pin={[pin distance=-0.4\onedegree,Bayer]-90:{$\boldsymbol\sigma$}}] at (axis cs:{52.644},{47.995}) {}; % Per,  4.34 
\node[pin={[pin distance=-0.4\onedegree,Bayer]00:{$\boldsymbol\tau$}}] at (axis cs:{43.564},{52.762}) {}; % Per,  3.94 
\node[pin={[pin distance=-0.4\onedegree,Bayer]90:{$\boldsymbol\varphi$}}] at (axis cs:{25.915},{50.689}) {}; % Per,  4.04 
\node[pin={[pin distance=-0.4\onedegree,Bayer]90:{$\boldsymbol\vartheta$}}] at (axis cs:{41.050},{49.228}) {}; % Per,  4.11 
\node[pin={[pin distance=-0.4\onedegree,Bayer]00:{$\boldsymbol\xi$}}] at (axis cs:{59.741},{35.791}) {}; % Per,  4.05 
\node[pin={[pin distance=-0.4\onedegree,Bayer]00:{$\boldsymbol\zeta$}}] at (axis cs:{58.533},{31.884}) {}; % Per,  2.88 
\node[pin={[pin distance=-0.6\onedegree,Flaamsted]180:{ 1}}] at (axis cs:{27.997},{55.147}) {}; % Per,  5.53 
\node[pin={[pin distance=-0.6\onedegree,Flaamsted]180:{10}}] at (axis cs:{36.317},{56.610}) {}; % Per,  6.27 
\node[pin={[pin distance=-0.6\onedegree,Flaamsted]90:{11}}] at (axis cs:{40.762},{55.106}) {}; % Per,  5.76 
\node[pin={[pin distance=-0.6\onedegree,Flaamsted]09:{12}}] at (axis cs:{40.562},{40.194}) {}; % Per,  4.90 
\node[pin={[pin distance=-0.6\onedegree,Flaamsted]00:{14}}] at (axis cs:{41.021},{44.297}) {}; % Per,  5.44 
\node[pin={[pin distance=-0.6\onedegree,Flaamsted]00:{16}}] at (axis cs:{42.646},{38.319}) {}; % Per,  4.22 
\node[pin={[pin distance=-0.6\onedegree,Flaamsted]00:{17}}] at (axis cs:{42.878},{35.060}) {}; % Per,  4.55 
\node[pin={[pin distance=-0.6\onedegree,Flaamsted]00:{ 2}}] at (axis cs:{28.039},{50.793}) {}; % Per,  5.70 
\node[pin={[pin distance=-0.6\onedegree,Flaamsted]00:{20}}] at (axis cs:{43.428},{38.337}) {}; % Per,  5.35 
\node[pin={[pin distance=-0.6\onedegree,Flaamsted]00:{21}}] at (axis cs:{44.322},{31.934}) {}; % Per,  5.10 
\node[pin={[pin distance=-0.6\onedegree,Flaamsted]00:{24}}] at (axis cs:{44.765},{35.183}) {}; % Per,  4.94 
%\node[pin={[pin distance=-0.6\onedegree,Flaamsted]00:{29}}] at (axis cs:{49.657},{50.222}) {}; % Per,  5.17 
\node[pin={[pin distance=-0.6\onedegree,Flaamsted]00:{ 3}}] at (axis cs:{29.640},{49.204}) {}; % Per,  5.71 
\node[pin={[pin distance=-0.6\onedegree,Flaamsted]00:{30}}] at (axis cs:{49.447},{44.025}) {}; % Per,  5.49 
\node[pin={[pin distance=-0.6\onedegree,Flaamsted]-90:{31}}] at (axis cs:{49.782},{50.095}) {}; % Per,  5.04 
\node[pin={[pin distance=-0.6\onedegree,Flaamsted]180:{32}}] at (axis cs:{50.361},{43.329}) {}; % Per,  4.95 
\node[pin={[pin distance=-0.6\onedegree,Flaamsted]90:{34}}] at (axis cs:{52.342},{49.509}) {}; % Per,  4.69 
\node[pin={[pin distance=-0.6\onedegree,Flaamsted]00:{36}}] at (axis cs:{53.109},{46.057}) {}; % Per,  5.31 
\node[pin={[pin distance=-0.6\onedegree,Flaamsted]180:{ 4}}] at (axis cs:{30.575},{54.487}) {}; % Per,  5.00 
\node[pin={[pin distance=-0.6\onedegree,Flaamsted]00:{40}}] at (axis cs:{55.594},{33.965}) {}; % Per,  4.98 
\node[pin={[pin distance=-0.6\onedegree,Flaamsted]00:{42}}] at (axis cs:{57.386},{33.091}) {}; % Per,  5.15 
\node[pin={[pin distance=-0.6\onedegree,Flaamsted]00:{43}}] at (axis cs:{59.152},{50.695}) {}; % Per,  5.28 
\node[pin={[pin distance=-0.6\onedegree,Flaamsted]00:{48}}] at (axis cs:{62.165},{47.712}) {}; % Per,  4.01 
\node[pin={[pin distance=-0.6\onedegree,Flaamsted]180:{49}}] at (axis cs:{62.064},{37.727}) {}; % Per,  6.07 
\node[pin={[pin distance=-0.6\onedegree,Flaamsted]00:{ 5}}] at (axis cs:{32.872},{57.645}) {}; % Per,  6.41 
\node[pin={[pin distance=-0.6\onedegree,Flaamsted]00:{50}}] at (axis cs:{62.153},{38.040}) {}; % Per,  5.52 
\node[pin={[pin distance=-0.6\onedegree,Flaamsted]00:{52}}] at (axis cs:{63.722},{40.484}) {}; % Per,  4.70 
\node[pin={[pin distance=-0.6\onedegree,Flaamsted]00:{53}}] at (axis cs:{65.388},{46.499}) {}; % Per,  4.82 
\node[pin={[pin distance=-0.6\onedegree,Flaamsted]00:{54}}] at (axis cs:{65.103},{34.567}) {}; % Per,  4.93 
\node[pin={[pin distance=-0.6\onedegree,Flaamsted]00:{55}}] at (axis cs:{66.121},{34.131}) {}; % Per,  5.73 
\node[pin={[pin distance=-0.6\onedegree,Flaamsted]180:{56}}] at (axis cs:{66.156},{33.960}) {}; % Per,  5.79 
\node[pin={[pin distance=-0.6\onedegree,Flaamsted]-90:{57}}] at (axis cs:{68.354},{43.064}) {}; % Per,  6.09 
\node[pin={[pin distance=-0.6\onedegree,Flaamsted]00:{58}}] at (axis cs:{69.173},{41.265}) {}; % Per,  4.25 
\node[pin={[pin distance=-0.6\onedegree,Flaamsted]00:{59}}] at (axis cs:{70.726},{43.365}) {}; % Per,  5.30 
\node[pin={[pin distance=-0.6\onedegree,Flaamsted]00:{ 7}}] at (axis cs:{34.519},{57.516}) {}; % Per,  5.99 
\node[pin={[pin distance=-0.6\onedegree,Flaamsted]90:{ 8}}] at (axis cs:{34.500},{57.900}) {}; % Per,  5.75 
\node[pin={[pin distance=-0.6\onedegree,Flaamsted]00:{ 9}}] at (axis cs:{35.589},{55.845}) {}; % Per,  5.20 
\node[pin={[pin distance=-0.6\onedegree,Bayer]00:{$\boldsymbol\sigma$}}] at (axis cs:{15.705},{31.804}) {}; % Psc,  5.50 
\node[pin={[pin distance=-0.6\onedegree,Bayer]90:{$\boldsymbol\tau$}}] at (axis cs:{17.915},{30.090}) {}; % Psc,  4.52 
\node[pin={[pin distance=-0.6\onedegree,Bayer]00:{$\boldsymbol\upsilon$}}] at (axis cs:{19.867},{27.264}) {}; % Psc,  4.75 
\node[pin={[pin distance=-0.6\onedegree,Flaamsted]00:{65}}] at (axis cs:{12.472},{27.710}) {}; % Psc,  5.55 
\node[pin={[pin distance=-0.6\onedegree,Flaamsted]00:{67}}] at (axis cs:{13.994},{27.209}) {}; % Psc,  6.08 
\node[pin={[pin distance=-0.6\onedegree,Flaamsted]00:{68}}] at (axis cs:{14.459},{28.992}) {}; % Psc,  5.43 
\node[pin={[pin distance=-0.6\onedegree,Flaamsted]-90:{78}}] at (axis cs:{17.006},{32.012}) {}; % Psc,  6.23 
\node[pin={[pin distance=-0.6\onedegree,Flaamsted]-90:{82}}] at (axis cs:{17.778},{31.425}) {}; % Psc,  5.16 
\node[pin={[pin distance=-0.6\onedegree,Flaamsted]00:{91}}] at (axis cs:{20.281},{28.738}) {}; % Psc,  5.22 
\node[pin={[pin distance=-0.4\onedegree,Bayer]00:{$\boldsymbol\beta$}}] at (axis cs:{81.573},{28.607}) {}; % Tau,  1.68 
\node[pin={[pin distance=-0.6\onedegree,Bayer]00:{$\boldsymbol\psi$}}] at (axis cs:{61.752},{29.001}) {}; % Tau,  5.22 
\node[pin={[pin distance=-0.6\onedegree,Bayer]00:{$\boldsymbol\varphi$}}] at (axis cs:{65.088},{27.351}) {}; % Tau,  4.95 
\node[pin={[pin distance=-0.6\onedegree,Flaamsted]00:{136}}] at (axis cs:{88.332},{27.612}) {}; % Tau,  4.56 
\node[pin={[pin distance=-0.6\onedegree,Flaamsted]00:{41}}] at (axis cs:{61.652},{27.600}) {}; % Tau,  5.18 
%\node[pin={[pin distance=-0.6\onedegree,Bayer]00:{$\boldsymbol\alpha$}}] at (axis cs:{28.270},{29.579}) {}; % Tri,  3.42 
\node[pin={[pin distance=-0.4\onedegree,Bayer]90:{$\boldsymbol\beta$}}] at (axis cs:{32.386},{34.987}) {}; % Tri,  3.02 
\node[pin={[pin distance=-0.4\onedegree,Bayer]90:{$\boldsymbol\delta$}}] at (axis cs:{34.263},{34.224}) {}; % Tri,  4.87 
\node[pin={[pin distance=-0.6\onedegree,Bayer]00:{$\boldsymbol\epsilon$}}] at (axis cs:{30.741},{33.284}) {}; % Tri,  5.52 
\node[pin={[pin distance=-0.4\onedegree,Bayer]180:{$\boldsymbol\gamma$}}] at (axis cs:{34.329},{33.847}) {}; % Tri,  4.01 
%\node[pin={[pin distance=-0.6\onedegree,Flaamsted]00:{10}}] at (axis cs:{34.737},{28.643}) {}; % Tri,  5.30 
\node[pin={[pin distance=-0.6\onedegree,Flaamsted]00:{11}}] at (axis cs:{36.866},{31.801}) {}; % Tri,  5.56 
%\node[pin={[pin distance=-0.6\onedegree,Flaamsted]00:{12}}] at (axis cs:{37.042},{29.669}) {}; % Tri,  5.29 
%\node[pin={[pin distance=-0.6\onedegree,Flaamsted]00:{13}}] at (axis cs:{37.202},{29.932}) {}; % Tri,  5.90 
\node[pin={[pin distance=-0.6\onedegree,Flaamsted]00:{14}}] at (axis cs:{38.026},{36.147}) {}; % Tri,  5.15 
\node[pin={[pin distance=-0.6\onedegree,Flaamsted]00:{15}}] at (axis cs:{38.945},{34.687}) {}; % Tri,  5.39 
\node[pin={[pin distance=-0.6\onedegree,Flaamsted]180:{ 5}}] at (axis cs:{32.854},{31.526}) {}; % Tri,  6.23 
\node[pin={[pin distance=-0.6\onedegree,Flaamsted]00:{ 6}}] at (axis cs:{33.093},{30.303}) {}; % Tri,  5.15 
\node[pin={[pin distance=-0.6\onedegree,Flaamsted]00:{ 7}}] at (axis cs:{33.985},{33.359}) {}; % Tri,  5.25 
\node[pin={[pin distance=-0.4\onedegree,Bayer]00:{$\boldsymbol\alpha$}}] at (axis cs:{165.932},{61.751}) {}; % UMa,  1.82 
\node[pin={[pin distance=-0.4\onedegree,Bayer]00:{$\boldsymbol\beta$}}] at (axis cs:{165.460},{56.382}) {}; % UMa,  2.35 
\node[pin={[pin distance=-0.4\onedegree,Bayer]00:{$\boldsymbol\epsilon$}}] at (axis cs:{193.507},{55.960}) {}; % UMa,  1.76 
\node[pin={[pin distance=-0.4\onedegree,Bayer]00:{$\boldsymbol\eta$}}] at (axis cs:{206.885},{49.313}) {}; % UMa,  1.86 
\node[pin={[pin distance=-0.4\onedegree,Bayer]00:{$\boldsymbol\gamma$}}] at (axis cs:{178.458},{53.695}) {}; % UMa,  2.43 
\node[pin={[pin distance=-0.4\onedegree,Bayer]00:{$\boldsymbol\zeta$}}] at (axis cs:{200.981},{54.925}) {}; % UMa,  2.22 
\node[pin={[pin distance=-0.6\onedegree,Bayer]00:{$\boldsymbol\chi$}}] at (axis cs:{176.513},{47.779}) {}; % UMa,  3.70 
\node[pin={[pin distance=-0.6\onedegree,Bayer]00:{$\boldsymbol\delta$}}] at (axis cs:{183.857},{57.032}) {}; % UMa,  3.30 
\node[pin={[pin distance=-0.6\onedegree,Bayer]00:{$\boldsymbol\iota$}}] at (axis cs:{134.802},{48.042}) {}; % UMa,  3.14 
\node[pin={[pin distance=-0.6\onedegree,Bayer]00:{$\boldsymbol\kappa$}}] at (axis cs:{135.906},{47.156}) {}; % UMa,  3.58 
\node[pin={[pin distance=-0.6\onedegree,Bayer]00:{$\boldsymbol\lambda$}}] at (axis cs:{154.274},{42.914}) {}; % UMa,  3.44 
\node[pin={[pin distance=-0.6\onedegree,Bayer]00:{$\boldsymbol\mu$}}] at (axis cs:{155.582},{41.499}) {}; % UMa,  3.05 
\node[pin={[pin distance=-0.6\onedegree,Bayer]00:{$\boldsymbol\nu$}}] at (axis cs:{169.620},{33.094}) {}; % UMa,  3.49 
\node[pin={[pin distance=-0.6\onedegree,Bayer]00:{$\boldsymbol\omega$}}] at (axis cs:{163.495},{43.190}) {}; % UMa,  4.67 
\node[pin={[pin distance=-0.6\onedegree,Bayer]00:{$\boldsymbol\omicron$}}] at (axis cs:{127.566},{60.718}) {}; % UMa,  3.36 
\node[pin={[pin distance=-0.6\onedegree,Bayer]00:{$\boldsymbol\pi^1$}}] at (axis cs:{129.799},{65.021}) {}; % UMa,  5.64 
\node[pin={[pin distance=-1\onedegree,Bayer]-135:{$\boldsymbol\pi^2$}}] at (axis cs:{130.053},{64.328}) {}; % UMa,  4.59 
\node[pin={[pin distance=-0.6\onedegree,Bayer]00:{$\boldsymbol\psi$}}] at (axis cs:{167.416},{44.498}) {}; % UMa,  3.00 
\node[pin={[pin distance=-0.6\onedegree,Bayer]-90:{$\boldsymbol\rho$}}] at (axis cs:{135.636},{67.629}) {}; % UMa,  4.77 
\node[pin={[pin distance=-0.7\onedegree,Bayer]90:{$\boldsymbol\sigma^{1,2}$}}] at (axis cs:{139.098},{66.873}) {}; % UMa,  5.14 
%\node[pin={[pin distance=-0.6\onedegree,Bayer]00:{$\boldsymbol\sigma^2$}}] at (axis cs:{137.598},{67.134}) {}; % UMa,  4.82 
\node[pin={[pin distance=-0.6\onedegree,Bayer]00:{$\boldsymbol\tau$}}] at (axis cs:{137.729},{63.513}) {}; % UMa,  4.64 
\node[pin={[pin distance=-0.6\onedegree,Bayer]00:{$\boldsymbol\upsilon$}}] at (axis cs:{147.747},{59.039}) {}; % UMa,  3.78 
\node[pin={[pin distance=-0.6\onedegree,Bayer]00:{$\boldsymbol\varphi$}}] at (axis cs:{148.027},{54.064}) {}; % UMa,  4.56 
\node[pin={[pin distance=-0.6\onedegree,Bayer]00:{$\boldsymbol\vartheta$}}] at (axis cs:{143.214},{51.677}) {}; % UMa,  3.18 
\node[pin={[pin distance=-0.6\onedegree,Bayer]00:{$\boldsymbol\xi$}}] at (axis cs:{169.546},{31.529}) {}; % UMa,  3.79 
\node[pin={[pin distance=-0.6\onedegree,Flaamsted]00:{15}}] at (axis cs:{137.218},{51.605}) {}; % UMa,  4.46 
\node[pin={[pin distance=-0.6\onedegree,Flaamsted]00:{16}}] at (axis cs:{138.586},{61.423}) {}; % UMa,  5.19 
\node[pin={[pin distance=-0.6\onedegree,Flaamsted]00:{17}}] at (axis cs:{138.957},{56.741}) {}; % UMa,  5.28 
\node[pin={[pin distance=-0.6\onedegree,Flaamsted]00:{18}}] at (axis cs:{139.047},{54.022}) {}; % UMa,  4.82 
\node[pin={[pin distance=-0.6\onedegree,Flaamsted]-90:{ 2}}] at (axis cs:{128.651},{65.145}) {}; % UMa,  5.46 
\node[pin={[pin distance=-0.6\onedegree,Flaamsted]-90:{22}}] at (axis cs:{143.723},{72.206}) {}; % UMa,  5.77 
\node[pin={[pin distance=-0.6\onedegree,Flaamsted]00:{23}}] at (axis cs:{142.882},{63.062}) {}; % UMa,  3.65 
\node[pin={[pin distance=-0.6\onedegree,Flaamsted]00:{24}}] at (axis cs:{143.620},{69.830}) {}; % UMa,  4.56 
\node[pin={[pin distance=-0.6\onedegree,Flaamsted]00:{26}}] at (axis cs:{143.706},{52.051}) {}; % UMa,  4.48 
\node[pin={[pin distance=-0.6\onedegree,Flaamsted]180:{27}}] at (axis cs:{145.738},{72.253}) {}; % UMa,  5.14 
\node[pin={[pin distance=-0.6\onedegree,Flaamsted]00:{31}}] at (axis cs:{148.929},{49.820}) {}; % UMa,  5.28 
\node[pin={[pin distance=-0.6\onedegree,Flaamsted]00:{32}}] at (axis cs:{154.508},{65.108}) {}; % UMa,  5.75 
\node[pin={[pin distance=-0.6\onedegree,Flaamsted]00:{35}}] at (axis cs:{157.477},{65.626}) {}; % UMa,  6.31 
\node[pin={[pin distance=-0.6\onedegree,Flaamsted]00:{36}}] at (axis cs:{157.657},{55.980}) {}; % UMa,  4.83 
\node[pin={[pin distance=-0.6\onedegree,Flaamsted]00:{37}}] at (axis cs:{158.790},{57.082}) {}; % UMa,  5.16 
\node[pin={[pin distance=-0.6\onedegree,Flaamsted]00:{38}}] at (axis cs:{160.486},{65.716}) {}; % UMa,  5.11 
\node[pin={[pin distance=-0.6\onedegree,Flaamsted]00:{39}}] at (axis cs:{160.931},{57.199}) {}; % UMa,  5.79 
\node[pin={[pin distance=-0.6\onedegree,Flaamsted]00:{41}}] at (axis cs:{161.594},{57.366}) {}; % UMa,  6.35 
\node[pin={[pin distance=-0.6\onedegree,Flaamsted]00:{42}}] at (axis cs:{162.849},{59.320}) {}; % UMa,  5.56 
\node[pin={[pin distance=-0.6\onedegree,Flaamsted]00:{43}}] at (axis cs:{162.796},{56.582}) {}; % UMa,  5.66 
\node[pin={[pin distance=-0.6\onedegree,Flaamsted]00:{44}}] at (axis cs:{163.394},{54.585}) {}; % UMa,  5.12 
\node[pin={[pin distance=-0.6\onedegree,Flaamsted]00:{46}}] at (axis cs:{163.935},{33.507}) {}; % UMa,  5.03 
\node[pin={[pin distance=-0.6\onedegree,Flaamsted]00:{47}}] at (axis cs:{164.867},{40.430}) {}; % UMa,  5.03 
\node[pin={[pin distance=-0.6\onedegree,Flaamsted]00:{49}}] at (axis cs:{165.210},{39.212}) {}; % UMa,  5.07 
\node[pin={[pin distance=-0.6\onedegree,Flaamsted]00:{ 5}}] at (axis cs:{133.344},{61.962}) {}; % UMa,  5.72 
\node[pin={[pin distance=-0.6\onedegree,Flaamsted]00:{51}}] at (axis cs:{166.130},{38.241}) {}; % UMa,  6.03 
\node[pin={[pin distance=-0.6\onedegree,Flaamsted]00:{55}}] at (axis cs:{169.783},{38.186}) {}; % UMa,  4.76 
\node[pin={[pin distance=-0.6\onedegree,Flaamsted]00:{56}}] at (axis cs:{170.707},{43.482}) {}; % UMa,  4.99 
\node[pin={[pin distance=-0.6\onedegree,Flaamsted]00:{57}}] at (axis cs:{172.267},{39.337}) {}; % UMa,  5.36 
\node[pin={[pin distance=-0.6\onedegree,Flaamsted]00:{58}}] at (axis cs:{172.630},{43.173}) {}; % UMa,  5.95 
\node[pin={[pin distance=-0.6\onedegree,Flaamsted]00:{59}}] at (axis cs:{174.586},{43.625}) {}; % UMa,  5.57 
\node[pin={[pin distance=-0.6\onedegree,Flaamsted]00:{ 6}}] at (axis cs:{134.156},{64.604}) {}; % UMa,  5.57 
\node[pin={[pin distance=-0.6\onedegree,Flaamsted]00:{60}}] at (axis cs:{174.640},{46.834}) {}; % UMa,  6.08 
\node[pin={[pin distance=-0.6\onedegree,Flaamsted]00:{61}}] at (axis cs:{175.263},{34.202}) {}; % UMa,  5.31 
\node[pin={[pin distance=-0.6\onedegree,Flaamsted]00:{62}}] at (axis cs:{175.393},{31.746}) {}; % UMa,  5.75 
\node[pin={[pin distance=-0.6\onedegree,Flaamsted]00:{66}}] at (axis cs:{178.993},{56.598}) {}; % UMa,  5.83 
\node[pin={[pin distance=-0.6\onedegree,Flaamsted]00:{67}}] at (axis cs:{180.528},{43.045}) {}; % UMa,  5.21 
\node[pin={[pin distance=-0.6\onedegree,Flaamsted]00:{68}}] at (axis cs:{182.937},{57.054}) {}; % UMa,  6.33 
\node[pin={[pin distance=-0.6\onedegree,Flaamsted]00:{70}}] at (axis cs:{185.212},{57.864}) {}; % UMa,  5.52 
\node[pin={[pin distance=-0.6\onedegree,Flaamsted]00:{71}}] at (axis cs:{186.263},{56.778}) {}; % UMa,  5.84 
\node[pin={[pin distance=-0.6\onedegree,Flaamsted]00:{73}}] at (axis cs:{186.896},{55.713}) {}; % UMa,  5.69 
\node[pin={[pin distance=-0.6\onedegree,Flaamsted]00:{74}}] at (axis cs:{187.489},{58.406}) {}; % UMa,  5.36 
\node[pin={[pin distance=-0.6\onedegree,Flaamsted]00:{75}}] at (axis cs:{187.518},{58.768}) {}; % UMa,  6.07 
\node[pin={[pin distance=-0.6\onedegree,Flaamsted]00:{76}}] at (axis cs:{190.391},{62.713}) {}; % UMa,  6.02 
\node[pin={[pin distance=-0.6\onedegree,Flaamsted]00:{78}}] at (axis cs:{195.182},{56.366}) {}; % UMa,  4.94 
\node[pin={[pin distance=-0.6\onedegree,Flaamsted]00:{80}}] at (axis cs:{201.306},{54.988}) {}; % UMa,  4.00 
\node[pin={[pin distance=-0.6\onedegree,Flaamsted]00:{81}}] at (axis cs:{203.530},{55.348}) {}; % UMa,  5.60 
\node[pin={[pin distance=-0.6\onedegree,Flaamsted]00:{82}}] at (axis cs:{204.877},{52.921}) {}; % UMa,  5.47 
\node[pin={[pin distance=-0.6\onedegree,Flaamsted]00:{83}}] at (axis cs:{205.184},{54.681}) {}; % UMa,  4.63 
\node[pin={[pin distance=-0.6\onedegree,Flaamsted]00:{84}}] at (axis cs:{206.649},{54.432}) {}; % UMa,  5.69 
\node[pin={[pin distance=-0.6\onedegree,Flaamsted]00:{86}}] at (axis cs:{208.463},{53.728}) {}; % UMa,  5.70 
\node[pin={[pin distance=-0.2\onedegree,Bayer]90:{$\boldsymbol\alpha$}}] at (axis cs:{37.955},{89.264}) {}; % UMi,  2.00 
\node[pin={[pin distance=-0.4\onedegree,Bayer]00:{$\boldsymbol\beta$}}] at (axis cs:{222.676},{74.155}) {}; % UMi,  2.06 
\node[pin={[pin distance=-0.6\onedegree,Bayer]90:{$\boldsymbol\delta$}}] at (axis cs:{263.054},{86.586}) {}; % UMi,  4.35 
\node[pin={[pin distance=-0.6\onedegree,Bayer]00:{$\boldsymbol\epsilon$}}] at (axis cs:{251.493},{82.037}) {}; % UMi,  4.22 
\node[pin={[pin distance=-0.6\onedegree,Bayer]00:{$\boldsymbol\eta$}}] at (axis cs:{244.376},{75.755}) {}; % UMi,  4.96 
\node[pin={[pin distance=-0.4\onedegree,Bayer]00:{$\boldsymbol\gamma$}}] at (axis cs:{230.182},{71.834}) {}; % UMi,  3.03 
\node[pin={[pin distance=-0.6\onedegree,Bayer]00:{$\boldsymbol\lambda$}}] at (axis cs:{259.235},{89.038}) {}; % UMi,  6.34 
\node[pin={[pin distance=-0.6\onedegree,Bayer]00:{$\boldsymbol\vartheta$}}] at (axis cs:{232.854},{77.349}) {}; % UMi,  4.99 
\node[pin={[pin distance=-0.6\onedegree,Bayer]00:{$\boldsymbol\zeta$}}] at (axis cs:{236.015},{77.794}) {}; % UMi,  4.29 
\node[pin={[pin distance=-0.6\onedegree,Flaamsted]90:{11}}] at (axis cs:{229.275},{71.824}) {}; % UMi,  5.01 
\node[pin={[pin distance=-0.6\onedegree,Flaamsted]90:{19}}] at (axis cs:{242.706},{75.877}) {}; % UMi,  5.48 
\node[pin={[pin distance=-0.6\onedegree,Flaamsted]00:{20}}] at (axis cs:{243.134},{75.211}) {}; % UMi,  6.35 
\node[pin={[pin distance=-0.6\onedegree,Flaamsted]180:{24}}] at (axis cs:{262.698},{86.968}) {}; % UMi,  5.77 
\node[pin={[pin distance=-0.6\onedegree,Flaamsted]00:{ 3}}] at (axis cs:{211.735},{74.594}) {}; % UMi,  6.43 
\node[pin={[pin distance=-0.6\onedegree,Flaamsted]00:{ 4}}] at (axis cs:{212.212},{77.547}) {}; % UMi,  4.80 
\node[pin={[pin distance=-0.6\onedegree,Flaamsted]00:{ 5}}] at (axis cs:{216.881},{75.696}) {}; % UMi,  4.24 
\node[pin={[pin distance=-0.6\onedegree,Flaamsted]00:{15}}] at (axis cs:{300.275},{27.754}) {}; % Vul,  4.66 
\node[pin={[pin distance=-0.6\onedegree,Flaamsted]00:{21}}] at (axis cs:{303.561},{28.695}) {}; % Vul,  5.20 
\node[pin={[pin distance=-0.6\onedegree,Flaamsted]00:{23}}] at (axis cs:{303.942},{27.814}) {}; % Vul,  4.52 
\node[pin={[pin distance=-0.6\onedegree,Flaamsted]00:{31}}] at (axis cs:{313.032},{27.097}) {}; % Vul,  4.57 
\node[pin={[pin distance=-0.6\onedegree,Flaamsted]00:{32}}] at (axis cs:{313.640},{28.057}) {}; % Vul,  5.00 
\node[pin={[pin distance=-0.6\onedegree,Flaamsted]00:{35}}] at (axis cs:{321.917},{27.608}) {}; % Vul,  5.39 
\end{polaraxis}


%
\end{tikzpicture}

\end{document}
