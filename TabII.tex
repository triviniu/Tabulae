%%%%%%%%%%%%%%%%%%%%%%%%%%%%%%%%%%%%%%%%%%%%%%%%%%%%%%%%%%%%%%%%%%%%%%
%
% rm TabulaI.pdf ; pdflatex -shell-escape  TabI.tex 
%
%%%%%%%%%%%%%%%%%%%%%%%%%%%%%%%%%%%%%%%%%%%%%%%%%%%%%%%%%%%%%%%%%%%%%%
\documentclass[10pt,landscape]{article}
\usepackage[margin=1cm,a3paper]{geometry}


\usepackage[margin=1cm,a3paper]{geometry}

\def\pgfsysdriver{pgfsys-pdftex.def}
\usepackage{pgfplots}
\usepgfplotslibrary{external} 
\usetikzlibrary{pgfplots.external}
\usetikzlibrary{calc}
\usetikzlibrary{shapes}
\usetikzlibrary{patterns}
\usetikzlibrary{plotmarks}
\usepgflibrary{arrows}
\usepgfplotslibrary{polar}
\tikzexternalize

\usepackage{times,latexsym,amssymb}
\usepackage{eulervm} % math font
\pagestyle{empty}

\usepackage{amsmath} 
\usepackage{nicefrac} 
%
% This is used because from 1.11 onwards the use of \pgfdeclareplotmark will
% introduce a shift of the stellar markers.
%
\pgfplotsset{compat=1.10}
%
% Use to increase spacing for constellation labels
\usepackage[letterspace=400]{microtype}
%
\usepackage{pagecolor}
\usepackage{natbib}
%


%\pagecolor{black}
\pagecolor{white}


\begin{document} 


%\input{./input/styles_dark.tex}
\input{./input/styles_po.tex}
\center

\pgfplotsset{xmin=0,xmax=360,ymin=90,ymax=20
,y coord trafo/.code=\pgfmathparse{90-#1}
,y coord inv trafo/.code=\pgfmathparse{90-#1}
}

\tikzsetnextfilename{TabulaII}


\begin{tikzpicture}

\begin{polaraxis}[rotate=270,name=base,axis y line=none,axis x line=none]\end{polaraxis}



% The Milky Way first, as it should not obstruct coordinate systems 


\begin{polaraxis}[rotate=270,name=constellations,at=(base.center),anchor=center,axis lines=none]

\clip (6\tendegree,0\tendegree) arc (0:90:6\tendegree) -- 
(-2\tendegree,6\tendegree) -- (-2\tendegree,-6\tendegree) -- (0\tendegree,-6\tendegree)
--  (0\tendegree,-6\tendegree) arc (270:359.9999:6\tendegree) -- cycle ;

\addplot[MW1] table[x index=0,y index=1] {./MW/IaII_MWbase.dat}  -- cycle ;
\addplot[MW0] table[x index=0,y index=1] {./MW/IaII_ol1_4.dat}  -- cycle ;

\addplot[MW2] table[x index=0,y index=1] {./MW/IaII_ol2_100.dat}  -- cycle ;
\addplot[MW2] table[x index=0,y index=1] {./MW/IaII_ol2_104.dat}  -- cycle ;
\addplot[MW2] table[x index=0,y index=1] {./MW/IaII_ol2_105.dat}  -- cycle ;
\addplot[MW2] table[x index=0,y index=1] {./MW/IaII_ol2_108.dat}  -- cycle ;
\addplot[MW2] table[x index=0,y index=1] {./MW/IaII_ol2_109.dat}  -- cycle ;
\addplot[MW2] table[x index=0,y index=1] {./MW/IaII_ol2_112.dat}  -- cycle ;
\addplot[MW2] table[x index=0,y index=1] {./MW/IaII_ol2_11.dat}  -- cycle ;
\addplot[MW2] table[x index=0,y index=1] {./MW/IaII_ol2_12.dat}  -- cycle ;
\addplot[MW2] table[x index=0,y index=1] {./MW/IaII_ol2_16.dat}  -- cycle ;
\addplot[MW2] table[x index=0,y index=1] {./MW/IaII_ol2_19.dat}  -- cycle ;
\addplot[MW2] table[x index=0,y index=1] {./MW/IaII_ol2_27.dat}  -- cycle ;
\addplot[MW2] table[x index=0,y index=1] {./MW/IaII_ol2_28.dat}  -- cycle ;
\addplot[MW2] table[x index=0,y index=1] {./MW/IaII_ol2_2.dat}  -- cycle ;
\addplot[MW2] table[x index=0,y index=1] {./MW/IaII_ol2_30.dat}  -- cycle ;
\addplot[MW2] table[x index=0,y index=1] {./MW/IaII_ol2_31.dat}  -- cycle ;
\addplot[MW2] table[x index=0,y index=1] {./MW/IaII_ol2_32.dat}  -- cycle ;
\addplot[MW2] table[x index=0,y index=1] {./MW/IaII_ol2_35.dat}  -- cycle ;
\addplot[MW2] table[x index=0,y index=1] {./MW/IaII_ol2_38.dat}  -- cycle ;
\addplot[MW2] table[x index=0,y index=1] {./MW/IaII_ol2_40.dat}  -- cycle ;
\addplot[MW2] table[x index=0,y index=1] {./MW/IaII_ol2_44.dat}  -- cycle ;
\addplot[MW2] table[x index=0,y index=1] {./MW/IaII_ol2_45.dat}  -- cycle ;
\addplot[MW2] table[x index=0,y index=1] {./MW/IaII_ol2_4.dat}  -- cycle ;
\addplot[MW2] table[x index=0,y index=1] {./MW/IaII_ol2_54.dat}  -- cycle ;
\addplot[MW2] table[x index=0,y index=1] {./MW/IaII_ol2_67.dat}  -- cycle ;
\addplot[MW2] table[x index=0,y index=1] {./MW/IaII_ol2_69.dat}  -- cycle ;
\addplot[MW2] table[x index=0,y index=1] {./MW/IaII_ol2_74.dat}  -- cycle ;
\addplot[MW2] table[x index=0,y index=1] {./MW/IaII_ol2_76.dat}  -- cycle ;
\addplot[MW2] table[x index=0,y index=1] {./MW/IaII_ol2_78.dat}  -- cycle ;
\addplot[MW2] table[x index=0,y index=1] {./MW/IaII_ol2_82.dat}  -- cycle ;
\addplot[MW2] table[x index=0,y index=1] {./MW/IaII_ol2_87.dat}  -- cycle ;
\addplot[MW2] table[x index=0,y index=1] {./MW/IaII_ol2_93.dat}  -- cycle ;
\addplot[MW2] table[x index=0,y index=1] {./MW/IaII_ol2_94.dat}  -- cycle ;
\addplot[MW2] table[x index=0,y index=1] {./MW/IaII_ol2_95.dat}  -- cycle ;
\addplot[MW2] table[x index=0,y index=1] {./MW/IaII_ol2_98.dat}  -- cycle ;

\addplot[MW1] table[x index=0,y index=1] {./MW/IaII_ol2_66.dat}  -- cycle ;

\end{polaraxis}



\input{./input/IaII_Grid.tex}
\input{./input/II_Axes.tex}
\input{./input/IaII_Frame.tex}
\input{./input/IaII_Legend.tex}

% The Galactic coordinate system
\input{./input/II_MWcoord.tex}


% Constellations
%
%
% Some are commented out because they are surrounded by already plotted borders
% The x-coordinate is in fractional hours, so must be times 15
%
\begin{polaraxis}[rotate=270,name=constellations,at=(base.center),anchor=center,axis lines=none]

\clip (6\tendegree,0\tendegree) arc (0:90:6\tendegree) -- 
(-2\tendegree,6\tendegree) -- (-2\tendegree,-6\tendegree) -- (0\tendegree,-6\tendegree)
--  (0\tendegree,-6\tendegree) arc (270:359.9999:6\tendegree) -- cycle ;

\draw[constellation-boundary]  (axis cs:344.465,    +35.168228) --  (axis cs:344.463,    +35.668224) --  (axis cs:344.457,    +36.668215) --  (axis cs:344.452,    +37.668206) --  (axis cs:344.447,    +38.668197) --  (axis cs:344.441,    +39.668187) --  (axis cs:344.435,    +40.668177) --  (axis cs:344.429,    +41.668167) --  (axis cs:344.423,    +42.668157) --  (axis cs:344.417,    +43.668146) --  (axis cs:344.41,    +44.668135) --  (axis cs:344.403,    +45.668123) --  (axis cs:344.396,    +46.668111) --  (axis cs:344.389,    +47.668099) --  (axis cs:344.381,    +48.668086) --  (axis cs:344.373,    +49.668073) --  (axis cs:344.365,    +50.668059) --  (axis cs:344.356,    +51.668044) --  (axis cs:344.347,    +52.668029) --  (axis cs:344.338,    +53.668013) --  (axis cs:344.328,    +54.667996) --  (axis cs:344.318,    +55.667978) --  (axis cs:344.307,    +56.667960) --  (axis cs:344.295,    +57.667940) --  (axis cs:344.283,    +58.667920) --  (axis cs:344.27,    +59.667898) --  (axis cs:344.269,    +59.751229) ;
  \draw[constellation-boundary]  (axis cs:344.343,    +53.168021) --  (axis cs:345.358,    +53.171359) --  (axis cs:346.374,    +53.174490) --  (axis cs:347.39,    +53.177412) --  (axis cs:348.405,    +53.180125) --  (axis cs:349.421,    +53.182628) --  (axis cs:350.437,    +53.184919) --  (axis cs:351.453,    +53.186999) ;
  \draw[constellation-boundary]  (axis cs:351.466,    +50.687011) --  (axis cs:351.461,    +51.687006) --  (axis cs:351.456,    +52.687001) --  (axis cs:351.453,    +53.186999) ;
  \draw[constellation-boundary]  (axis cs:351.466,    +50.687011) --  (axis cs:352.48,    +50.688875) --  (axis cs:353.495,    +50.690527) --  (axis cs:354.51,    +50.691965) --  (axis cs:355.271,    +50.692903) ;
  \draw[constellation-boundary]  (axis cs:355.276,    +48.692907) --  (axis cs:355.273,    +49.692905) --  (axis cs:355.271,    +50.692903) ;
  \draw[constellation-boundary]  (axis cs:355.276,    +48.692907) --  (axis cs:355.53,    +48.693192) --  (axis cs:356.543,    +48.694201) --  (axis cs:357.557,    +48.694996) --  (axis cs:358.571,    +48.695576) --  (axis cs:359.584,    +48.695941) --  (axis cs:360.598,    +48.696092) --  (axis cs:361.612,    +48.696027) --  (axis cs:362.626,    +48.695748) --  (axis cs:363.639,    +48.695253) --  (axis cs:364.146,    +48.694926) ;
  \draw[constellation-boundary]  (axis cs:364.143,    +46.694927) --  (axis cs:364.145,    +47.694926) --  (axis cs:364.146,    +48.694926) ;
  \draw[constellation-boundary]  (axis cs:364.143,    +46.694927) --  (axis cs:364.65,    +46.694546) ;
  \draw[constellation-boundary]  (axis cs:362.611,    +25.695750) --  (axis cs:362.612,    +26.695750) --  (axis cs:362.612,    +27.695750) --  (axis cs:362.613,    +28.695750) ;
  \draw[constellation-boundary]  (axis cs:361.606,    +28.696028) --  (axis cs:362.613,    +28.695750) ;
  \draw[constellation-boundary]  (axis cs:361.606,    +28.696028) --  (axis cs:361.606,    +29.696028) --  (axis cs:361.607,    +30.696028) --  (axis cs:361.607,    +31.696028) --  (axis cs:361.607,    +32.029361) ;
  \draw[constellation-boundary]  (axis cs:357.829,    +32.028500) --  (axis cs:358.584,    +32.028913) --  (axis cs:359.592,    +32.029276) --  (axis cs:360.599,    +32.029425) --  (axis cs:361.607,    +32.029361) ;
  \draw[constellation-boundary]  (axis cs:357.829,    +32.028500) --  (axis cs:357.828,    +32.695167) --  (axis cs:357.828,    +32.778500) ;
  \draw[constellation-boundary]  (axis cs:354.049,    +32.774639) --  (axis cs:354.553,    +32.775327) --  (axis cs:355.561,    +32.776543) --  (axis cs:356.568,    +32.777546) --  (axis cs:357.576,    +32.778336) --  (axis cs:357.828,    +32.778500) ;
  \draw[constellation-boundary]  (axis cs:354.049,    +32.774639) --  (axis cs:354.047,    +33.691305) --  (axis cs:354.045,    +34.691303) --  (axis cs:354.044,    +35.191302) ;
  \draw[constellation-boundary]  (axis cs:284.284,    +25.164180) --  (axis cs:284.27,    +26.164094) ;
  \draw[constellation-boundary]  (axis cs:203.969,    +25.360356) --  (axis cs:203.963,    +26.360342) --  (axis cs:203.957,    +27.360327) --  (axis cs:203.954,    +27.860320) ;
  \draw[constellation-boundary]  (axis cs:210.789,    +27.897658) --  (axis cs:210.785,    +28.397646) --  (axis cs:210.777,    +29.397621) --  (axis cs:210.771,    +30.147603) ;
  \draw[constellation-boundary]  (axis cs:210.771,    +30.147603) --  (axis cs:211.392,    +30.151432) --  (axis cs:211.889,    +30.154546) ;
  \draw[constellation-boundary]  (axis cs:211.889,    +30.154546) --  (axis cs:211.887,    +30.404539) --  (axis cs:211.878,    +31.404512) --  (axis cs:211.87,    +32.404484) --  (axis cs:211.861,    +33.404456) --  (axis cs:211.851,    +34.404427) --  (axis cs:211.842,    +35.404397) --  (axis cs:211.832,    +36.404367) --  (axis cs:211.823,    +37.404336) --  (axis cs:211.812,    +38.404304) --  (axis cs:211.802,    +39.404271) --  (axis cs:211.791,    +40.404237) --  (axis cs:211.78,    +41.404202) --  (axis cs:211.769,    +42.404166) --  (axis cs:211.757,    +43.404129) --  (axis cs:211.745,    +44.404091) --  (axis cs:211.732,    +45.404051) --  (axis cs:211.719,    +46.404010) --  (axis cs:211.706,    +47.403967) --  (axis cs:211.692,    +48.403923) --  (axis cs:211.677,    +49.403877) --  (axis cs:211.662,    +50.403829) --  (axis cs:211.646,    +51.403779) --  (axis cs:211.629,    +52.403726) --  (axis cs:211.612,    +53.403672) --  (axis cs:211.594,    +54.403614) --  (axis cs:211.584,    +54.903584) ;
  \draw[constellation-boundary]  (axis cs:211.584,    +54.903584) --  (axis cs:212.077,    +54.906716) --  (axis cs:213.062,    +54.913112) --  (axis cs:214.048,    +54.919685) --  (axis cs:215.033,    +54.926433) --  (axis cs:216.019,    +54.933353) --  (axis cs:217.005,    +54.940444) --  (axis cs:217.991,    +54.947704) --  (axis cs:218.977,    +54.955131) --  (axis cs:219.963,    +54.962721) --  (axis cs:220.95,    +54.970474) --  (axis cs:221.937,    +54.978387) --  (axis cs:222.924,    +54.986458) --  (axis cs:223.911,    +54.994684) --  (axis cs:224.898,    +55.003062) --  (axis cs:225.886,    +55.011591) --  (axis cs:226.874,    +55.020268) --  (axis cs:227.862,    +55.029090) --  (axis cs:228.85,    +55.038055) --  (axis cs:229.591,    +55.044871) ;
  \draw[constellation-boundary]  (axis cs:217.251,    +54.942243) --  (axis cs:217.24,    +55.442202) --  (axis cs:217.217,    +56.442117) --  (axis cs:217.192,    +57.442027) --  (axis cs:217.166,    +58.441932) --  (axis cs:217.139,    +59.441831) --  (axis cs:217.109,    +60.441724) --  (axis cs:217.078,    +61.441611) --  (axis cs:217.045,    +62.441489) ;
  \draw[constellation-boundary]  (axis cs:229.657,    +52.545177) --  (axis cs:229.632,    +53.545059) --  (axis cs:229.605,    +54.544935) --  (axis cs:229.591,    +55.044871) ;
  \draw[constellation-boundary]  (axis cs:229.657,    +52.545177) --  (axis cs:229.905,    +52.547469) --  (axis cs:230.894,    +52.556720) --  (axis cs:231.884,    +52.566107) --  (axis cs:232.874,    +52.575625) --  (axis cs:233.865,    +52.585272) --  (axis cs:234.855,    +52.595045) --  (axis cs:235.846,    +52.604942) --  (axis cs:236.837,    +52.614958) --  (axis cs:237.084,    +52.617481) ;
  \draw[constellation-boundary]  (axis cs:237.125,    +51.117686) --  (axis cs:237.868,    +51.125301) --  (axis cs:238.86,    +51.135554) --  (axis cs:239.852,    +51.145919) --  (axis cs:240.845,    +51.156392) --  (axis cs:241.837,    +51.166970) --  (axis cs:242.83,    +51.177649) --  (axis cs:243.823,    +51.188428) --  (axis cs:244.816,    +51.199302) --  (axis cs:245.809,    +51.210268) --  (axis cs:246.803,    +51.221324) --  (axis cs:247.797,    +51.232465) --  (axis cs:248.791,    +51.243688) --  (axis cs:249.785,    +51.254991) --  (axis cs:250.78,    +51.266369) --  (axis cs:251.775,    +51.277819) --  (axis cs:252.77,    +51.289338) --  (axis cs:253.765,    +51.300922) --  (axis cs:254.761,    +51.312568) --  (axis cs:255.757,    +51.324273) ;
  \draw[constellation-boundary]  (axis cs:237.365,    +39.618913) --  (axis cs:237.348,    +40.618825) --  (axis cs:237.33,    +41.618733) --  (axis cs:237.311,    +42.618638) --  (axis cs:237.292,    +43.618541) --  (axis cs:237.273,    +44.618440) --  (axis cs:237.252,    +45.618336) --  (axis cs:237.231,    +46.618228) --  (axis cs:237.209,    +47.618115) --  (axis cs:237.186,    +48.617999) --  (axis cs:237.162,    +49.617877) --  (axis cs:237.137,    +50.617751) --  (axis cs:237.112,    +51.617619) --  (axis cs:237.084,    +52.617481) ;
  \draw[constellation-boundary]  (axis cs:232.644,    +39.572120) --  (axis cs:233.141,    +39.576913) --  (axis cs:234.134,    +39.586595) --  (axis cs:235.128,    +39.596402) --  (axis cs:236.123,    +39.606332) --  (axis cs:237.117,    +39.616382) --  (axis cs:238.111,    +39.626550) --  (axis cs:239.106,    +39.636831) --  (axis cs:240.101,    +39.647223) --  (axis cs:241.095,    +39.657722) --  (axis cs:242.09,    +39.668326) --  (axis cs:243.086,    +39.679032) --  (axis cs:244.081,    +39.689835) --  (axis cs:245.076,    +39.700734) --  (axis cs:246.072,    +39.711724) ;
  \draw[constellation-boundary]  (axis cs:232.747,    +32.572617) --  (axis cs:232.733,    +33.572551) --  (axis cs:232.719,    +34.572484) --  (axis cs:232.705,    +35.572415) --  (axis cs:232.69,    +36.572344) --  (axis cs:232.675,    +37.572272) --  (axis cs:232.66,    +38.572197) --  (axis cs:232.644,    +39.572120) ;
  \draw[constellation-boundary]  (axis cs:229.016,    +32.537682) --  (axis cs:229.265,    +32.539951) --  (axis cs:230.259,    +32.549115) --  (axis cs:231.254,    +32.558415) --  (axis cs:232.249,    +32.567850) --  (axis cs:232.747,    +32.572617) ;
  \draw[constellation-boundary]  (axis cs:229.1,    +25.538063) --  (axis cs:229.088,    +26.538012) --  (axis cs:229.077,    +27.537959) --  (axis cs:229.065,    +28.537906) --  (axis cs:229.053,    +29.537852) --  (axis cs:229.041,    +30.537796) --  (axis cs:229.029,    +31.537740) --  (axis cs:229.016,    +32.537682) ;
  \draw[constellation-boundary]  (axis cs:227.606,    +25.524616) --  (axis cs:228.353,    +25.531300) --  (axis cs:229.349,    +25.540335) --  (axis cs:230.345,    +25.549510) --  (axis cs:231.341,    +25.558822) --  (axis cs:232.337,    +25.568269) --  (axis cs:233.334,    +25.577847) --  (axis cs:234.33,    +25.587553) --  (axis cs:235.326,    +25.597385) --  (axis cs:236.323,    +25.607339) --  (axis cs:237.32,    +25.617414) --  (axis cs:238.316,    +25.627604) --  (axis cs:239.313,    +25.637909) --  (axis cs:240.31,    +25.648323) --  (axis cs:241.307,    +25.658845) --  (axis cs:242.304,    +25.669471) --  (axis cs:243.302,    +25.680198) --  (axis cs:243.8,    +25.685599) ;
  \draw[constellation-boundary]  (axis cs:180.634,    +85.803908) --  (axis cs:181.468,    +85.803961) --  (axis cs:182.303,    +85.804191) --  (axis cs:183.137,    +85.804598) --  (axis cs:183.972,    +85.805181) --  (axis cs:184.806,    +85.805942) --  (axis cs:185.641,    +85.806878) --  (axis cs:186.476,    +85.807992) --  (axis cs:187.311,    +85.809281) --  (axis cs:188.146,    +85.810747) --  (axis cs:188.982,    +85.812388) --  (axis cs:189.818,    +85.814204) --  (axis cs:190.654,    +85.816196) --  (axis cs:191.49,    +85.818362) --  (axis cs:192.327,    +85.820702) --  (axis cs:193.164,    +85.823217) --  (axis cs:194.002,    +85.825905) --  (axis cs:194.84,    +85.828766) --  (axis cs:195.679,    +85.831799) --  (axis cs:196.518,    +85.835004) --  (axis cs:197.357,    +85.838381) --  (axis cs:198.197,    +85.841928) --  (axis cs:199.038,    +85.845645) --  (axis cs:199.88,    +85.849532) --  (axis cs:200.722,    +85.853587) --  (axis cs:201.564,    +85.857810) --  (axis cs:202.408,    +85.862201) --  (axis cs:203.252,    +85.866757) --  (axis cs:204.097,    +85.871480) --  (axis cs:204.943,    +85.876367) --  (axis cs:205.789,    +85.881417) --  (axis cs:206.637,    +85.886631) --  (axis cs:207.485,    +85.892006) --  (axis cs:208.334,    +85.897542) --  (axis cs:209.184,    +85.903237) --  (axis cs:210.036,    +85.909092) --  (axis cs:210.888,    +85.915104) --  (axis cs:211.741,    +85.921272) --  (axis cs:212.595,    +85.927596) --  (axis cs:213.023,    +85.930816) ;
  \draw[constellation-boundary]  (axis cs:216.783,    +79.444991) --  (axis cs:216.535,    +80.444059) --  (axis cs:216.23,    +81.442911) --  (axis cs:215.846,    +82.441461) --  (axis cs:215.345,    +83.439573) --  (axis cs:214.665,    +84.437011) --  (axis cs:213.689,    +85.433331) --  (axis cs:213.023,    +85.930816) ;
  \draw[constellation-boundary]  (axis cs:203.809,    +79.362941) --  (axis cs:204.044,    +79.364133) --  (axis cs:204.985,    +79.369019) --  (axis cs:205.926,    +79.374087) --  (axis cs:206.868,    +79.379336) --  (axis cs:207.81,    +79.384764) --  (axis cs:208.752,    +79.390370) --  (axis cs:209.695,    +79.396152) --  (axis cs:210.638,    +79.402110) --  (axis cs:211.582,    +79.408241) --  (axis cs:212.526,    +79.414544) --  (axis cs:213.471,    +79.421017) --  (axis cs:214.417,    +79.427659) --  (axis cs:215.363,    +79.434469) --  (axis cs:216.309,    +79.441443) --  (axis cs:216.783,    +79.444991) ;
  \draw[constellation-boundary]  (axis cs:204.157,    +76.363819) --  (axis cs:204.059,    +77.363572) --  (axis cs:203.945,    +78.363283) --  (axis cs:203.809,    +79.362941) ;
  \draw[constellation-boundary]  (axis cs:180.611,    +76.303908) --  (axis cs:181.561,    +76.303968) --  (axis cs:182.511,    +76.304230) --  (axis cs:183.461,    +76.304694) --  (axis cs:184.411,    +76.305358) --  (axis cs:185.362,    +76.306224) --  (axis cs:186.312,    +76.307290) --  (axis cs:187.262,    +76.308557) --  (axis cs:188.213,    +76.310024) --  (axis cs:189.163,    +76.311692) --  (axis cs:190.114,    +76.313558) --  (axis cs:191.064,    +76.315623) --  (axis cs:192.015,    +76.317887) --  (axis cs:192.966,    +76.320349) --  (axis cs:193.918,    +76.323008) --  (axis cs:194.869,    +76.325863) --  (axis cs:195.821,    +76.328914) --  (axis cs:196.772,    +76.332159) --  (axis cs:197.724,    +76.335599) --  (axis cs:198.677,    +76.339232) --  (axis cs:199.629,    +76.343057) --  (axis cs:200.582,    +76.347073) --  (axis cs:201.535,    +76.351280) --  (axis cs:202.488,    +76.355675) --  (axis cs:203.442,    +76.360259) --  (axis cs:204.157,    +76.363819) ;
  \draw[constellation-boundary]  (axis cs:181.594,    +33.303971) --  (axis cs:182.586,    +33.304245) --  (axis cs:183.578,    +33.304728) --  (axis cs:184.57,    +33.305422) --  (axis cs:185.562,    +33.306326) --  (axis cs:186.554,    +33.307439) ;
  \draw[constellation-boundary]  (axis cs:181.591,    +44.303971) --  (axis cs:182.579,    +44.304243) --  (axis cs:182.826,    +44.304344) ;
  \draw[constellation-boundary]  (axis cs:182.826,    +44.304344) --  (axis cs:182.826,    +45.304344) --  (axis cs:182.825,    +46.304344) --  (axis cs:182.824,    +47.304344) --  (axis cs:182.823,    +48.304344) --  (axis cs:182.822,    +49.304343) --  (axis cs:182.821,    +50.304343) --  (axis cs:182.82,    +51.304343) --  (axis cs:182.819,    +52.304343) ;
  \draw[constellation-boundary]  (axis cs:182.819,    +52.304343) --  (axis cs:183.557,    +52.304722) --  (axis cs:184.541,    +52.305410) --  (axis cs:185.525,    +52.306307) --  (axis cs:186.509,    +52.307412) --  (axis cs:187.494,    +52.308724) --  (axis cs:188.478,    +52.310243) --  (axis cs:189.462,    +52.311970) --  (axis cs:190.447,    +52.313903) --  (axis cs:191.431,    +52.316041) --  (axis cs:192.416,    +52.318385) --  (axis cs:193.4,    +52.320933) --  (axis cs:194.385,    +52.323685) --  (axis cs:195.37,    +52.326640) --  (axis cs:196.354,    +52.329796) --  (axis cs:197.339,    +52.333154) --  (axis cs:198.324,    +52.336713) --  (axis cs:199.309,    +52.340470) --  (axis cs:200.294,    +52.344425) --  (axis cs:201.279,    +52.348578) --  (axis cs:202.264,    +52.352926) --  (axis cs:203.25,    +52.357468) --  (axis cs:203.742,    +52.359812) ;
  \draw[constellation-boundary]  (axis cs:203.795,    +47.859938) --  (axis cs:204.289,    +47.862335) --  (axis cs:205.277,    +47.867273) --  (axis cs:206.264,    +47.872402) --  (axis cs:207.252,    +47.877719) --  (axis cs:208.24,    +47.883224) --  (axis cs:209.228,    +47.888915) --  (axis cs:210.216,    +47.894791) --  (axis cs:211.205,    +47.900849) --  (axis cs:211.699,    +47.903945) ;
  \draw[constellation-boundary]  (axis cs:203.795,    +47.859938) --  (axis cs:203.79,    +48.359926) --  (axis cs:203.779,    +49.359899) --  (axis cs:203.767,    +50.359871) --  (axis cs:203.755,    +51.359842) --  (axis cs:203.742,    +52.359812) ;
  \draw[constellation-boundary]  (axis cs:200.227,    +27.843782) --  (axis cs:200.475,    +27.844798) --  (axis cs:201.469,    +27.848987) --  (axis cs:202.463,    +27.853374) --  (axis cs:203.457,    +27.857956) --  (axis cs:204.451,    +27.862733) --  (axis cs:205.445,    +27.867702) --  (axis cs:206.439,    +27.872864) --  (axis cs:207.433,    +27.878215) --  (axis cs:208.427,    +27.883755) --  (axis cs:209.422,    +27.889482) --  (axis cs:210.416,    +27.895394) --  (axis cs:210.789,    +27.897658) ;
  \draw[constellation-boundary]  (axis cs:200.227,    +27.843782) --  (axis cs:200.224,    +28.343776) --  (axis cs:200.219,    +29.343765) --  (axis cs:200.213,    +30.343754) --  (axis cs:200.208,    +31.343743) ;
  \draw[constellation-boundary]  (axis cs:186.558,    +31.307441) --  (axis cs:187.55,    +31.308765) --  (axis cs:188.543,    +31.310297) --  (axis cs:189.536,    +31.312038) --  (axis cs:190.528,    +31.313987) --  (axis cs:191.521,    +31.316143) --  (axis cs:192.513,    +31.318506) --  (axis cs:193.506,    +31.321075) --  (axis cs:194.499,    +31.323850) --  (axis cs:195.492,    +31.326829) --  (axis cs:196.484,    +31.330012) --  (axis cs:197.477,    +31.333397) --  (axis cs:198.47,    +31.336984) --  (axis cs:199.463,    +31.340772) --  (axis cs:200.208,    +31.343743) ;
  \draw[constellation-boundary]  (axis cs:186.558,    +31.307441) --  (axis cs:186.556,    +32.307440) --  (axis cs:186.554,    +33.307439) ;
  \draw[constellation-boundary]  (axis cs:344.269,    +59.751229) --  (axis cs:345.289,    +59.754582) --  (axis cs:346.309,    +59.757727) --  (axis cs:347.329,    +59.760662) --  (axis cs:348.349,    +59.763387) --  (axis cs:348.86,    +59.764670) ;
  \draw[constellation-boundary]  (axis cs:348.86,    +59.764670) --  (axis cs:348.85,    +60.681326) --  (axis cs:348.84,    +61.681313) --  (axis cs:348.829,    +62.681299) --  (axis cs:348.816,    +63.681284) ;
  \draw[constellation-boundary]  (axis cs:348.816,    +63.681284) --  (axis cs:349.328,    +63.682518) --  (axis cs:350.352,    +63.684828) --  (axis cs:351.377,    +63.686925) --  (axis cs:352.401,    +63.688807) --  (axis cs:353.425,    +63.690474) --  (axis cs:354.449,    +63.691926) --  (axis cs:355.218,    +63.692873) ;
  \draw[constellation-boundary]  (axis cs:355.218,    +63.692873) --  (axis cs:355.212,    +64.692869) --  (axis cs:355.205,    +65.692865) --  (axis cs:355.198,    +66.692861) ;
  \draw[constellation-boundary]  (axis cs:355.198,    +66.692861) --  (axis cs:355.455,    +66.693151) --  (axis cs:356.483,    +66.694174) --  (axis cs:357.511,    +66.694980) --  (axis cs:358.539,    +66.695569) --  (axis cs:359.567,    +66.695939) --  (axis cs:360.595,    +66.696092) --  (axis cs:361.623,    +66.696026) --  (axis cs:362.652,    +66.695743) --  (axis cs:363.68,    +66.695241) --  (axis cs:364.708,    +66.694522) ;
  \draw[constellation-boundary]  (axis cs:300.573,    +59.851075) --  (axis cs:300.552,    +60.350965) --  (axis cs:300.508,    +61.350736) --  (axis cs:300.485,    +61.850615) ;
  \draw[constellation-boundary]  (axis cs:300.485,    +61.850615) --  (axis cs:301.497,    +61.861116) --  (axis cs:302.508,    +61.871508) --  (axis cs:303.52,    +61.881786) --  (axis cs:304.533,    +61.891949) --  (axis cs:305.545,    +61.901992) --  (axis cs:306.558,    +61.911913) --  (axis cs:306.812,    +61.914374) ;
  \draw[constellation-boundary]  (axis cs:306.812,    +61.914374) --  (axis cs:306.79,    +62.414266) --  (axis cs:306.743,    +63.414039) --  (axis cs:306.693,    +64.413796) --  (axis cs:306.639,    +65.413535) --  (axis cs:306.58,    +66.413252) --  (axis cs:306.517,    +67.412946) ;
  \draw[constellation-boundary]  (axis cs:306.517,    +67.412946) --  (axis cs:307.28,    +67.420309) --  (axis cs:308.298,    +67.430012) --  (axis cs:309.316,    +67.439585) --  (axis cs:310.334,    +67.449023) ;
  \draw[constellation-boundary]  (axis cs:310.334,    +67.449023) --  (axis cs:310.269,    +68.448723) --  (axis cs:310.197,    +69.448395) --  (axis cs:310.119,    +70.448035) --  (axis cs:310.033,    +71.447638) --  (axis cs:309.937,    +72.447197) --  (axis cs:309.83,    +73.446705) --  (axis cs:309.709,    +74.446151) --  (axis cs:309.573,    +75.445523) ;
  \draw[constellation-boundary]  (axis cs:301.873,    +75.370868) --  (axis cs:302.385,    +75.376056) --  (axis cs:303.41,    +75.386348) --  (axis cs:304.435,    +75.396523) --  (axis cs:305.462,    +75.406577) --  (axis cs:306.488,    +75.416507) --  (axis cs:307.516,    +75.426311) --  (axis cs:308.544,    +75.435984) --  (axis cs:309.573,    +75.445523) ;
  \draw[constellation-boundary]  (axis cs:301.873,    +75.370868) --  (axis cs:301.703,    +76.370001) --  (axis cs:301.506,    +77.369000) --  (axis cs:301.275,    +78.367830) --  (axis cs:301.002,    +79.366444) --  (axis cs:300.674,    +80.364773) ;
  \draw[constellation-boundary]  (axis cs:300.674,    +80.364773) --  (axis cs:301.192,    +80.370027) --  (axis cs:302.228,    +80.380450) --  (axis cs:303.266,    +80.390759) --  (axis cs:304.305,    +80.400949) --  (axis cs:305.345,    +80.411018) --  (axis cs:306.387,    +80.420962) --  (axis cs:307.429,    +80.430777) --  (axis cs:308.473,    +80.440460) --  (axis cs:309.517,    +80.450008) --  (axis cs:310.563,    +80.459417) --  (axis cs:311.609,    +80.468685) --  (axis cs:312.657,    +80.477807) --  (axis cs:313.706,    +80.486781) ;
  \draw[constellation-boundary]  (axis cs:313.706,    +80.486781) --  (axis cs:313.36,    +81.485314) --  (axis cs:312.923,    +82.483461) --  (axis cs:312.352,    +83.481043) --  (axis cs:311.577,    +84.477755) --  (axis cs:310.462,    +85.473021) --  (axis cs:308.721,    +86.465621) --  (axis cs:308.331,    +86.630629) ;
  \draw[constellation-boundary]  (axis cs:308.331,    +86.630629) --  (axis cs:309.459,    +86.640190) --  (axis cs:310.59,    +86.649601) --  (axis cs:311.724,    +86.658859) --  (axis cs:312.862,    +86.667959) --  (axis cs:314.004,    +86.676897) --  (axis cs:315.149,    +86.685669) --  (axis cs:316.297,    +86.694271) --  (axis cs:317.448,    +86.702699) --  (axis cs:318.603,    +86.710950) --  (axis cs:319.762,    +86.719019) --  (axis cs:320.923,    +86.726902) --  (axis cs:322.087,    +86.734596) --  (axis cs:323.255,    +86.742097) --  (axis cs:324.426,    +86.749401) --  (axis cs:325.599,    +86.756506) --  (axis cs:326.776,    +86.763407) --  (axis cs:327.955,    +86.770101) --  (axis cs:329.137,    +86.776584) --  (axis cs:330.322,    +86.782854) --  (axis cs:331.51,    +86.788907) --  (axis cs:332.7,    +86.794741) --  (axis cs:333.892,    +86.800351) --  (axis cs:335.087,    +86.805736) --  (axis cs:336.284,    +86.810892) --  (axis cs:337.484,    +86.815817) --  (axis cs:338.685,    +86.820509) --  (axis cs:339.889,    +86.824964) --  (axis cs:341.094,    +86.829181) --  (axis cs:342.302,    +86.833156) --  (axis cs:343.511,    +86.836889) ;
  \draw[constellation-boundary]  (axis cs:343.511,    +86.836889) --  (axis cs:342.404,    +87.668572) --  (axis cs:339.261,    +88.663876) ;
  \draw[constellation-boundary]  (axis cs:339.261,    +88.663876) --  (axis cs:340.749,    +88.668190) --  (axis cs:342.243,    +88.672206) --  (axis cs:343.742,    +88.675922) --  (axis cs:345.247,    +88.679333) --  (axis cs:346.757,    +88.682436) --  (axis cs:348.272,    +88.685228) --  (axis cs:349.79,    +88.687706) --  (axis cs:351.313,    +88.689867) --  (axis cs:352.838,    +88.691710) --  (axis cs:354.366,    +88.693233) --  (axis cs:355.896,    +88.694433) --  (axis cs:357.428,    +88.695310) --  (axis cs:358.961,    +88.695862) --  (axis cs:360.494,    +88.696090) --  (axis cs:362.028,    +88.695992) --  (axis cs:363.561,    +88.695569) ;
  \draw[constellation-boundary]  (axis cs:335.911,    +56.882568) --  (axis cs:336.165,    +56.883841) --  (axis cs:337.182,    +56.888807) --  (axis cs:338.199,    +56.893575) --  (axis cs:339.216,    +56.898144) --  (axis cs:340.234,    +56.902513) --  (axis cs:341.251,    +56.906680) --  (axis cs:342.269,    +56.910643) --  (axis cs:343.286,    +56.914402) --  (axis cs:344.304,    +56.917955) ;
  \draw[constellation-boundary]  (axis cs:335.931,    +55.632620) --  (axis cs:335.915,    +56.632579) --  (axis cs:335.911,    +56.882568) ;
  \draw[constellation-boundary]  (axis cs:333.138,    +55.617836) --  (axis cs:334.153,    +55.623381) --  (axis cs:335.169,    +55.628733) --  (axis cs:335.931,    +55.632620) ;
  \draw[constellation-boundary]  (axis cs:333.175,    +53.367939) --  (axis cs:333.171,    +53.617928) --  (axis cs:333.155,    +54.617883) --  (axis cs:333.138,    +55.617836) ;
  \draw[constellation-boundary]  (axis cs:330.639,    +53.353265) --  (axis cs:331.146,    +53.356294) --  (axis cs:332.16,    +53.362211) --  (axis cs:333.175,    +53.367939) ;
  \draw[constellation-boundary]  (axis cs:309.831,    +55.275325) --  (axis cs:310.842,    +55.284695) --  (axis cs:311.854,    +55.293926) --  (axis cs:312.865,    +55.303017) --  (axis cs:313.877,    +55.311964) --  (axis cs:314.889,    +55.320764) --  (axis cs:315.901,    +55.329415) --  (axis cs:316.914,    +55.337914) --  (axis cs:317.927,    +55.346258) --  (axis cs:318.939,    +55.354445) --  (axis cs:319.953,    +55.362472) --  (axis cs:320.966,    +55.370336) --  (axis cs:321.979,    +55.378035) --  (axis cs:322.993,    +55.385567) --  (axis cs:324.007,    +55.392929) --  (axis cs:325.021,    +55.400119) --  (axis cs:326.036,    +55.407135) --  (axis cs:327.05,    +55.413973) --  (axis cs:328.065,    +55.420633) --  (axis cs:329.08,    +55.427112) --  (axis cs:330.095,    +55.433408) --  (axis cs:330.602,    +55.436487) ;
  \draw[constellation-boundary]  (axis cs:309.831,    +55.275325) --  (axis cs:309.827,    +55.441970) --  (axis cs:309.797,    +56.441831) --  (axis cs:309.765,    +57.441684) --  (axis cs:309.732,    +58.441530) --  (axis cs:309.697,    +59.441366) --  (axis cs:309.66,    +60.441192) --  (axis cs:309.624,    +61.357690) ;
  \draw[constellation-boundary]  (axis cs:308.661,    +61.348638) --  (axis cs:309.624,    +61.357690) ;
  \draw[constellation-boundary]  (axis cs:308.717,    +59.932235) --  (axis cs:308.697,    +60.432145) --  (axis cs:308.661,    +61.348638) ;
  \draw[constellation-boundary]  (axis cs:181.596,    +28.303971) --  (axis cs:181.595,    +29.303971) --  (axis cs:181.595,    +30.303971) --  (axis cs:181.595,    +31.303971) --  (axis cs:181.595,    +32.303971) --  (axis cs:181.594,    +33.303971) --  (axis cs:181.594,    +34.303971) --  (axis cs:181.594,    +35.303971) --  (axis cs:181.594,    +36.303971) --  (axis cs:181.593,    +37.303971) --  (axis cs:181.593,    +38.303971) --  (axis cs:181.593,    +39.303971) --  (axis cs:181.593,    +40.303971) --  (axis cs:181.592,    +41.303971) --  (axis cs:181.592,    +42.303971) --  (axis cs:181.592,    +43.303971) --  (axis cs:181.591,    +44.303971) ;
  \draw[constellation-boundary]  (axis cs:246.28,    +26.712875) --  (axis cs:246.266,    +27.712798) --  (axis cs:246.252,    +28.712719) --  (axis cs:246.237,    +29.712638) --  (axis cs:246.222,    +30.712556) --  (axis cs:246.207,    +31.712472) --  (axis cs:246.192,    +32.712386) --  (axis cs:246.176,    +33.712298) --  (axis cs:246.159,    +34.712209) --  (axis cs:246.143,    +35.712117) --  (axis cs:246.126,    +36.712022) --  (axis cs:246.108,    +37.711925) --  (axis cs:246.09,    +38.711826) --  (axis cs:246.072,    +39.711724) ;
  \draw[constellation-boundary]  (axis cs:243.787,    +26.685525) --  (axis cs:244.285,    +26.690949) --  (axis cs:245.282,    +26.701867) --  (axis cs:246.28,    +26.712875) ;
  \draw[constellation-boundary]  (axis cs:243.8,    +25.685599) --  (axis cs:243.787,    +26.685525) ;
  \draw[constellation-boundary]  (axis cs:290.133,    +27.732407) --  (axis cs:290.258,    +27.733840) --  (axis cs:291.26,    +27.745259) --  (axis cs:292.262,    +27.756603) --  (axis cs:293.265,    +27.767870) --  (axis cs:294.267,    +27.779056) --  (axis cs:295.27,    +27.790158) --  (axis cs:296.272,    +27.801171) ;
  \draw[constellation-boundary]  (axis cs:290.095,    +30.232193) --  (axis cs:290.221,    +30.233626) --  (axis cs:291.223,    +30.245048) --  (axis cs:291.599,    +30.249312) ;
  \draw[constellation-boundary]  (axis cs:291.599,    +30.249312) --  (axis cs:291.583,    +31.249225) --  (axis cs:291.568,    +32.249136) --  (axis cs:291.552,    +33.249046) --  (axis cs:291.535,    +34.248953) --  (axis cs:291.519,    +35.248858) --  (axis cs:291.501,    +36.248760) --  (axis cs:291.493,    +36.748711) ;
  \draw[constellation-boundary]  (axis cs:291.493,    +36.748711) --  (axis cs:292.12,    +36.755799) ;
  \draw[constellation-boundary]  (axis cs:292.12,    +36.755799) --  (axis cs:292.111,    +37.255749) --  (axis cs:292.093,    +38.255647) --  (axis cs:292.074,    +39.255543) --  (axis cs:292.055,    +40.255435) --  (axis cs:292.035,    +41.255324) --  (axis cs:292.015,    +42.255210) --  (axis cs:291.994,    +43.255092) --  (axis cs:291.984,    +43.755031) ;
  \draw[constellation-boundary]  (axis cs:288.47,    +43.714937) --  (axis cs:288.972,    +43.720717) --  (axis cs:289.976,    +43.732226) --  (axis cs:290.98,    +43.743666) --  (axis cs:291.984,    +43.755031) ;
  \draw[constellation-boundary]  (axis cs:288.47,    +43.714937) --  (axis cs:288.459,    +44.214873) --  (axis cs:288.436,    +45.214741) --  (axis cs:288.413,    +46.214605) --  (axis cs:288.388,    +47.214463) --  (axis cs:288.375,    +47.714391) ;
  \draw[constellation-boundary]  (axis cs:287.121,    +47.699864) --  (axis cs:287.108,    +48.199789) --  (axis cs:287.081,    +49.199634) --  (axis cs:287.053,    +50.199473) --  (axis cs:287.024,    +51.199305) --  (axis cs:286.994,    +52.199129) --  (axis cs:286.962,    +53.198945) --  (axis cs:286.929,    +54.198753) --  (axis cs:286.895,    +55.198551) --  (axis cs:286.877,    +55.698446) ;
  \draw[constellation-boundary]  (axis cs:286.877,    +55.698446) --  (axis cs:287.63,    +55.707185) --  (axis cs:288.635,    +55.718782) --  (axis cs:289.641,    +55.730313) --  (axis cs:290.647,    +55.741776) --  (axis cs:291.653,    +55.753167) --  (axis cs:291.905,    +55.756003) ;
  \draw[constellation-boundary]  (axis cs:291.905,    +55.756003) --  (axis cs:291.887,    +56.255901) --  (axis cs:291.849,    +57.255690) --  (axis cs:291.809,    +58.255468) ;
  \draw[constellation-boundary]  (axis cs:291.809,    +58.255468) --  (axis cs:292.565,    +58.263952) --  (axis cs:293.572,    +58.275194) --  (axis cs:294.58,    +58.286354) --  (axis cs:295.588,    +58.297428) --  (axis cs:296.596,    +58.308412) --  (axis cs:297.1,    +58.313870) ;
  \draw[constellation-boundary]  (axis cs:297.1,    +58.313870) --  (axis cs:297.06,    +59.313653) --  (axis cs:297.039,    +59.813539) ;
  \draw[constellation-boundary]  (axis cs:297.039,    +59.813539) --  (axis cs:297.544,    +59.818976) --  (axis cs:298.553,    +59.829776) --  (axis cs:299.563,    +59.840477) --  (axis cs:300.573,    +59.851075) --  (axis cs:301.584,    +59.861567) --  (axis cs:302.595,    +59.871948) --  (axis cs:303.606,    +59.882217) --  (axis cs:304.617,    +59.892370) --  (axis cs:305.629,    +59.902403) --  (axis cs:306.641,    +59.912314) --  (axis cs:307.653,    +59.922099) --  (axis cs:308.666,    +59.931756) --  (axis cs:308.717,    +59.932235) ;
  \draw[constellation-boundary]  (axis cs:330.763,    +44.603636) --  (axis cs:330.751,    +45.603600) --  (axis cs:330.738,    +46.603562) --  (axis cs:330.725,    +47.603523) --  (axis cs:330.712,    +48.603483) --  (axis cs:330.697,    +49.603440) --  (axis cs:330.683,    +50.603396) --  (axis cs:330.668,    +51.603350) --  (axis cs:330.652,    +52.603302) --  (axis cs:330.635,    +53.603252) --  (axis cs:330.617,    +54.603199) --  (axis cs:330.602,    +55.436487) ;
  \draw[constellation-boundary]  (axis cs:329.879,    +44.598244) --  (axis cs:330.257,    +44.600572) --  (axis cs:330.763,    +44.603636) ;
  \draw[constellation-boundary]  (axis cs:329.882,    +44.348254) --  (axis cs:329.879,    +44.598244) ;
  \draw[constellation-boundary]  (axis cs:329.377,    +44.345110) --  (axis cs:329.882,    +44.348254) ;
  \draw[constellation-boundary]  (axis cs:329.461,    +36.595374) --  (axis cs:329.451,    +37.595343) --  (axis cs:329.441,    +38.595311) --  (axis cs:329.431,    +39.595279) --  (axis cs:329.42,    +40.595245) --  (axis cs:329.409,    +41.595211) --  (axis cs:329.397,    +42.595175) --  (axis cs:329.386,    +43.595138) --  (axis cs:329.377,    +44.345110) ;
  \draw[constellation-boundary]  (axis cs:327.32,    +36.581538) --  (axis cs:328.327,    +36.588151) --  (axis cs:329.335,    +36.594583) --  (axis cs:330.343,    +36.600833) --  (axis cs:331.35,    +36.606899) ;
  \draw[constellation-boundary]  (axis cs:327.395,    +28.581788) --  (axis cs:327.387,    +29.581759) --  (axis cs:327.378,    +30.581730) --  (axis cs:327.369,    +31.581699) --  (axis cs:327.359,    +32.581669) --  (axis cs:327.35,    +33.581637) --  (axis cs:327.34,    +34.581605) --  (axis cs:327.33,    +35.581572) --  (axis cs:327.32,    +36.581538) ;
  \draw[constellation-boundary]  (axis cs:315.084,    +28.487183) --  (axis cs:315.335,    +28.489352) --  (axis cs:316.34,    +28.497937) --  (axis cs:317.344,    +28.506370) --  (axis cs:318.349,    +28.514648) --  (axis cs:319.354,    +28.522770) --  (axis cs:320.359,    +28.530731) --  (axis cs:321.364,    +28.538530) --  (axis cs:322.369,    +28.546165) --  (axis cs:323.374,    +28.553632) --  (axis cs:324.379,    +28.560931) --  (axis cs:325.384,    +28.568058) --  (axis cs:326.39,    +28.575011) --  (axis cs:327.395,    +28.581788) ;
  \draw[constellation-boundary]  (axis cs:315.084,    +28.487183) --  (axis cs:315.073,    +29.487134) ;
  \draw[constellation-boundary]  (axis cs:296.251,    +29.301054) --  (axis cs:297.254,    +29.311978) --  (axis cs:298.257,    +29.322807) --  (axis cs:299.26,    +29.333538) --  (axis cs:300.263,    +29.344167) --  (axis cs:301.266,    +29.354692) --  (axis cs:302.27,    +29.365109) --  (axis cs:303.273,    +29.375415) --  (axis cs:304.277,    +29.385607) --  (axis cs:305.281,    +29.395681) --  (axis cs:306.285,    +29.405635) --  (axis cs:307.289,    +29.415465) --  (axis cs:308.293,    +29.425169) --  (axis cs:309.297,    +29.434743) --  (axis cs:310.301,    +29.444185) --  (axis cs:311.305,    +29.453491) --  (axis cs:312.31,    +29.462658) --  (axis cs:313.314,    +29.471685) --  (axis cs:314.319,    +29.480568) --  (axis cs:315.073,    +29.487134) ;
  \draw[constellation-boundary]  (axis cs:296.272,    +27.801171) --  (axis cs:296.265,    +28.301132) --  (axis cs:296.251,    +29.301054) ;
  \draw[constellation-boundary]  (axis cs:196.097,    +69.329372) --  (axis cs:196.07,    +70.329325) --  (axis cs:196.039,    +71.329275) --  (axis cs:196.005,    +72.329218) --  (axis cs:195.967,    +73.329156) --  (axis cs:195.924,    +74.329085) --  (axis cs:195.876,    +75.329005) --  (axis cs:195.821,    +76.328914) ;
  \draw[constellation-boundary]  (axis cs:196.097,    +69.329372) --  (axis cs:197.066,    +69.332676) --  (axis cs:198.035,    +69.336177) --  (axis cs:199.005,    +69.339874) --  (axis cs:199.974,    +69.343767) --  (axis cs:200.944,    +69.347854) --  (axis cs:201.913,    +69.352134) --  (axis cs:202.883,    +69.356606) --  (axis cs:203.854,    +69.361268) --  (axis cs:204.824,    +69.366120) --  (axis cs:205.794,    +69.371160) --  (axis cs:206.765,    +69.376386) --  (axis cs:207.736,    +69.381797) --  (axis cs:208.708,    +69.387392) --  (axis cs:209.679,    +69.393169) --  (axis cs:210.651,    +69.399126) ;
  \draw[constellation-boundary]  (axis cs:210.821,    +65.399654) --  (axis cs:210.783,    +66.399538) --  (axis cs:210.743,    +67.399412) --  (axis cs:210.699,    +68.399275) --  (axis cs:210.651,    +69.399126) ;
  \draw[constellation-boundary]  (axis cs:210.821,    +65.399654) --  (axis cs:211.798,    +65.405821) --  (axis cs:212.775,    +65.412167) --  (axis cs:213.752,    +65.418687) --  (axis cs:214.73,    +65.425382) --  (axis cs:215.708,    +65.432249) --  (axis cs:216.686,    +65.439286) --  (axis cs:217.665,    +65.446491) --  (axis cs:218.644,    +65.453862) --  (axis cs:219.623,    +65.461397) --  (axis cs:220.602,    +65.469094) --  (axis cs:221.582,    +65.476950) --  (axis cs:222.562,    +65.484964) --  (axis cs:223.542,    +65.493132) --  (axis cs:224.522,    +65.501453) --  (axis cs:225.503,    +65.509925) --  (axis cs:226.484,    +65.518544) --  (axis cs:227.466,    +65.527309) --  (axis cs:228.448,    +65.536217) --  (axis cs:229.43,    +65.545264) --  (axis cs:230.412,    +65.554450) --  (axis cs:231.395,    +65.563770) --  (axis cs:232.378,    +65.573223) --  (axis cs:233.361,    +65.582805) --  (axis cs:234.345,    +65.592514) --  (axis cs:235.33,    +65.602348) ;
  \draw[constellation-boundary]  (axis cs:235.33,    +65.602348) --  (axis cs:235.268,    +66.602040) --  (axis cs:235.202,    +67.601706) --  (axis cs:235.13,    +68.601343) --  (axis cs:235.051,    +69.600946) ;
  \draw[constellation-boundary]  (axis cs:235.051,    +69.600946) --  (axis cs:236.032,    +69.610866) --  (axis cs:237.013,    +69.620905) --  (axis cs:237.995,    +69.631059) --  (axis cs:238.978,    +69.641326) --  (axis cs:239.96,    +69.651703) --  (axis cs:240.944,    +69.662187) --  (axis cs:241.928,    +69.672775) --  (axis cs:242.912,    +69.683463) --  (axis cs:243.897,    +69.694250) --  (axis cs:244.882,    +69.705130) --  (axis cs:245.868,    +69.716102) --  (axis cs:246.854,    +69.727163) --  (axis cs:247.841,    +69.738308) ;
  \draw[constellation-boundary]  (axis cs:247.841,    +69.738308) --  (axis cs:247.742,    +70.737746) --  (axis cs:247.632,    +71.737125) --  (axis cs:247.511,    +72.736435) --  (axis cs:247.374,    +73.735663) --  (axis cs:247.221,    +74.734793) ;
  \draw[constellation-boundary]  (axis cs:247.221,    +74.734793) --  (axis cs:248.203,    +74.745969) --  (axis cs:249.186,    +74.757224) --  (axis cs:250.17,    +74.768557) --  (axis cs:251.155,    +74.779963) --  (axis cs:252.14,    +74.791439) --  (axis cs:253.126,    +74.802982) --  (axis cs:254.113,    +74.814589) --  (axis cs:255.1,    +74.826255) --  (axis cs:256.088,    +74.837979) --  (axis cs:257.077,    +74.849755) --  (axis cs:258.067,    +74.861582) --  (axis cs:259.057,    +74.873454) --  (axis cs:260.048,    +74.885370) --  (axis cs:261.04,    +74.897324) --  (axis cs:261.537,    +74.903315) ;
  \draw[constellation-boundary]  (axis cs:261.537,    +74.903315) --  (axis cs:261.346,    +75.902167) --  (axis cs:261.128,    +76.900848) --  (axis cs:260.874,    +77.899314) --  (axis cs:260.575,    +78.897510) --  (axis cs:260.218,    +79.895353) ;
  \draw[constellation-boundary]  (axis cs:260.218,    +79.895353) --  (axis cs:260.712,    +79.901329) --  (axis cs:261.7,    +79.913307) --  (axis cs:262.69,    +79.925319) --  (axis cs:263.681,    +79.937361) --  (axis cs:264.673,    +79.949429) --  (axis cs:265.666,    +79.961519) --  (axis cs:266.66,    +79.973628) --  (axis cs:267.656,    +79.985752) ;
  \draw[constellation-boundary]  (axis cs:267.656,    +79.985752) --  (axis cs:267.21,    +80.983041) --  (axis cs:266.655,    +81.979661) --  (axis cs:265.943,    +82.975328) --  (axis cs:264.997,    +83.969568) --  (axis cs:263.681,    +84.961539) --  (axis cs:261.722,    +85.949570) ;
  \draw[constellation-boundary]  (axis cs:261.722,    +85.949570) --  (axis cs:262.697,    +85.961583) --  (axis cs:263.674,    +85.973624) --  (axis cs:264.654,    +85.985691) --  (axis cs:265.636,    +85.997781) --  (axis cs:266.622,    +86.009889) --  (axis cs:267.611,    +86.022013) --  (axis cs:268.603,    +86.034148) --  (axis cs:269.597,    +86.046292) --  (axis cs:270.595,    +86.058439) --  (axis cs:271.596,    +86.070588) --  (axis cs:272.6,    +86.082734) --  (axis cs:273.607,    +86.094873) --  (axis cs:274.617,    +86.107001) --  (axis cs:275.631,    +86.119115) --  (axis cs:276.647,    +86.131211) --  (axis cs:277.667,    +86.143285) --  (axis cs:278.69,    +86.155333) --  (axis cs:279.717,    +86.167352) --  (axis cs:280.746,    +86.179336) --  (axis cs:281.779,    +86.191283) --  (axis cs:282.816,    +86.203188) --  (axis cs:283.855,    +86.215047) --  (axis cs:284.898,    +86.226856) --  (axis cs:285.945,    +86.238612) --  (axis cs:286.995,    +86.250309) --  (axis cs:288.048,    +86.261944) --  (axis cs:289.105,    +86.273513) --  (axis cs:290.165,    +86.285011) --  (axis cs:291.229,    +86.296434) --  (axis cs:292.296,    +86.307779) --  (axis cs:293.367,    +86.319041) --  (axis cs:294.441,    +86.330215) --  (axis cs:295.518,    +86.341298) --  (axis cs:296.6,    +86.352285) --  (axis cs:297.684,    +86.363172) --  (axis cs:298.772,    +86.373956) --  (axis cs:299.864,    +86.384630) --  (axis cs:300.959,    +86.395192) --  (axis cs:302.057,    +86.405638) --  (axis cs:303.159,    +86.415962) --  (axis cs:304.265,    +86.426161) --  (axis cs:305.374,    +86.436230) --  (axis cs:306.486,    +86.446166) --  (axis cs:307.602,    +86.455964) --  (axis cs:308.721,    +86.465621) ;
  \draw[constellation-boundary]  (axis cs:274.342,    +47.547602) --  (axis cs:274.329,    +48.047520) --  (axis cs:274.301,    +49.047353) --  (axis cs:274.273,    +50.047179) --  (axis cs:274.258,    +50.547089) ;
  \draw[constellation-boundary]  (axis cs:255.786,    +50.324447) --  (axis cs:256.783,    +50.336208) --  (axis cs:257.779,    +50.348020) --  (axis cs:258.776,    +50.359879) --  (axis cs:259.773,    +50.371783) --  (axis cs:260.77,    +50.383727) --  (axis cs:261.768,    +50.395708) --  (axis cs:262.765,    +50.407722) --  (axis cs:263.763,    +50.419766) --  (axis cs:264.762,    +50.431836) --  (axis cs:265.76,    +50.443928) --  (axis cs:266.759,    +50.456038) --  (axis cs:267.758,    +50.468164) --  (axis cs:268.757,    +50.480301) --  (axis cs:269.757,    +50.492445) --  (axis cs:270.756,    +50.504593) --  (axis cs:271.756,    +50.516741) --  (axis cs:272.757,    +50.528886) --  (axis cs:273.757,    +50.541024) --  (axis cs:274.258,    +50.547089) ;
  \draw[constellation-boundary]  (axis cs:255.786,    +50.324447) --  (axis cs:255.772,    +50.824361) --  (axis cs:255.757,    +51.324273) ;
  \draw[constellation-boundary]  (axis cs:203.574,    +62.359406) --  (axis cs:204.063,    +62.361782) --  (axis cs:205.042,    +62.366675) --  (axis cs:206.02,    +62.371757) --  (axis cs:206.999,    +62.377027) --  (axis cs:207.979,    +62.382484) --  (axis cs:208.958,    +62.388125) --  (axis cs:209.938,    +62.393949) --  (axis cs:210.917,    +62.399955) --  (axis cs:211.897,    +62.406140) --  (axis cs:212.877,    +62.412504) --  (axis cs:213.858,    +62.419043) --  (axis cs:214.838,    +62.425757) --  (axis cs:215.819,    +62.432643) --  (axis cs:216.8,    +62.439699) --  (axis cs:217.045,    +62.441489) ;
  \draw[constellation-boundary]  (axis cs:203.574,    +62.359406) --  (axis cs:203.551,    +63.359351) ;
  \draw[constellation-boundary]  (axis cs:181.582,    +63.303970) --  (axis cs:182.557,    +63.304239) --  (axis cs:183.533,    +63.304715) --  (axis cs:184.509,    +63.305398) --  (axis cs:185.485,    +63.306286) --  (axis cs:186.461,    +63.307382) --  (axis cs:187.436,    +63.308683) --  (axis cs:188.412,    +63.310189) --  (axis cs:189.388,    +63.311901) --  (axis cs:190.365,    +63.313817) --  (axis cs:191.341,    +63.315938) --  (axis cs:192.317,    +63.318262) --  (axis cs:193.293,    +63.320789) --  (axis cs:194.269,    +63.323517) --  (axis cs:195.246,    +63.326448) --  (axis cs:196.222,    +63.329578) --  (axis cs:197.199,    +63.332909) --  (axis cs:198.176,    +63.336438) --  (axis cs:199.153,    +63.340164) --  (axis cs:200.13,    +63.344088) --  (axis cs:201.107,    +63.348206) --  (axis cs:202.084,    +63.352520) --  (axis cs:203.062,    +63.357026) --  (axis cs:203.551,    +63.359351) ;
  \draw[constellation-boundary]  (axis cs:181.582,    +63.303970) --  (axis cs:181.581,    +64.303970) --  (axis cs:181.58,    +65.303970) --  (axis cs:181.579,    +65.803970) ;
  \draw[constellation-boundary]  (axis cs:180.606,    +65.803908) --  (axis cs:181.579,    +65.803970) ;
  \draw[constellation-boundary]  (axis cs:241.806,    +25.664146) ;
  \draw[constellation-boundary]  (axis cs:273.467,    +47.536987) --  (axis cs:273.842,    +47.541537) --  (axis cs:274.843,    +47.553663) --  (axis cs:275.844,    +47.565773) --  (axis cs:276.845,    +47.577865) --  (axis cs:277.846,    +47.589935) --  (axis cs:278.848,    +47.601978) --  (axis cs:279.85,    +47.613992) --  (axis cs:280.852,    +47.625972) --  (axis cs:281.855,    +47.637916) --  (axis cs:282.857,    +47.649818) --  (axis cs:283.86,    +47.661676) --  (axis cs:284.863,    +47.673486) --  (axis cs:285.866,    +47.685245) --  (axis cs:286.87,    +47.696948) --  (axis cs:287.874,    +47.708592) --  (axis cs:288.375,    +47.714391) ;
  \draw[constellation-boundary]  (axis cs:273.824,    +30.039155) --  (axis cs:273.808,    +31.039056) --  (axis cs:273.791,    +32.038955) --  (axis cs:273.774,    +33.038851) --  (axis cs:273.757,    +34.038745) --  (axis cs:273.739,    +35.038637) --  (axis cs:273.721,    +36.038526) --  (axis cs:273.702,    +37.038412) --  (axis cs:273.683,    +38.038295) --  (axis cs:273.663,    +39.038175) --  (axis cs:273.642,    +40.038051) --  (axis cs:273.621,    +41.037924) --  (axis cs:273.6,    +42.037793) --  (axis cs:273.577,    +43.037657) --  (axis cs:273.554,    +44.037517) --  (axis cs:273.53,    +45.037372) --  (axis cs:273.506,    +46.037223) --  (axis cs:273.48,    +47.037067) --  (axis cs:273.467,    +47.536987) ;
  \draw[constellation-boundary]  (axis cs:273.824,    +30.039155) --  (axis cs:274.199,    +30.043704) --  (axis cs:275.2,    +30.055824) --  (axis cs:276.2,    +30.067929) --  (axis cs:276.701,    +30.073974) ;
  \draw[constellation-boundary]  (axis cs:276.763,    +26.074349) --  (axis cs:276.748,    +27.074257) --  (axis cs:276.732,    +28.074165) --  (axis cs:276.717,    +29.074070) --  (axis cs:276.701,    +30.073974) ;
  \draw[constellation-boundary]  (axis cs:276.763,    +26.074349) --  (axis cs:277.263,    +26.080387) --  (axis cs:278.264,    +26.092446) --  (axis cs:279.264,    +26.104478) --  (axis cs:280.265,    +26.116478) --  (axis cs:281.266,    +26.128444) --  (axis cs:282.267,    +26.140371) --  (axis cs:283.269,    +26.152255) --  (axis cs:284.27,    +26.164094) ;
  \draw[constellation-boundary]  (axis cs:284.277,    +25.664137) --  (axis cs:285.278,    +25.675926) --  (axis cs:286.28,    +25.687662) --  (axis cs:287.281,    +25.699341) --  (axis cs:288.283,    +25.710960) --  (axis cs:289.285,    +25.722516) --  (axis cs:290.161,    +25.732571) ;
  \draw[constellation-boundary]  (axis cs:343.709,    +35.165608) --  (axis cs:344.465,    +35.168228) --  (axis cs:345.473,    +35.171543) --  (axis cs:346.482,    +35.174651) --  (axis cs:347.49,    +35.177552) --  (axis cs:348.498,    +35.180244) --  (axis cs:349.506,    +35.182728) --  (axis cs:350.515,    +35.185003) --  (axis cs:351.523,    +35.187067) --  (axis cs:352.532,    +35.188920) --  (axis cs:353.54,    +35.190561) --  (axis cs:354.044,    +35.191302) ;
  \draw[constellation-boundary]  (axis cs:343.709,    +35.165608) --  (axis cs:343.706,    +35.665603) ;
  \draw[constellation-boundary]  (axis cs:331.36,    +35.606926) --  (axis cs:332.367,    +35.612804) --  (axis cs:333.375,    +35.618494) --  (axis cs:334.382,    +35.623995) --  (axis cs:335.39,    +35.629304) --  (axis cs:336.398,    +35.634421) --  (axis cs:337.406,    +35.639342) --  (axis cs:338.414,    +35.644068) --  (axis cs:339.422,    +35.648596) --  (axis cs:340.43,    +35.652925) --  (axis cs:341.438,    +35.657053) --  (axis cs:342.446,    +35.660980) --  (axis cs:343.454,    +35.664704) --  (axis cs:343.706,    +35.665603) ;
  \draw[constellation-boundary]  (axis cs:331.36,    +35.606926) --  (axis cs:331.35,    +36.606899) ;
  \draw[constellation-boundary]  (axis cs:180.602,    +28.303908) --  (axis cs:181.596,    +28.303971) ;
  \draw[constellation-boundary]  (axis cs:227.606,    +25.524616) ;
  \draw[constellation-boundary]  (axis cs:290.161,    +25.732571) --  (axis cs:290.154,    +26.232531) --  (axis cs:290.14,    +27.232449) --  (axis cs:290.125,    +28.232365) --  (axis cs:290.11,    +29.232280) --  (axis cs:290.095,    +30.232193) ;
  \draw[constellation-boundary]  (axis cs:322.648,    +25.548153) --  (axis cs:322.639,    +26.548118) --  (axis cs:322.63,    +27.548083) --  (axis cs:322.62,    +28.548047) ;
  \draw[constellation-boundary]  (axis cs:-4.729,    +50.692903) ;
  \draw[constellation-boundary]  (axis cs:-4.724,    +48.692907) --  (axis cs:-4.727,    +49.692905) --  (axis cs:-4.729,    +50.692903) ;
  \draw[constellation-boundary]  (axis cs:-4.724,    +48.692907) --  (axis cs:-4.47,    +48.693192) --  (axis cs:-3.457,    +48.694201) --  (axis cs:-2.443,    +48.694996) --  (axis cs:-1.429,    +48.695576) --  (axis cs:-0.416,    +48.695941) --  (axis cs:0.59819,    +48.696092) --  (axis cs:1.61194,    +48.696027) --  (axis cs:2.62569,    +48.695748) --  (axis cs:3.63943,    +48.695253) --  (axis cs:4.1463,    +48.694926) ;
  \draw[constellation-boundary]  (axis cs:4.14321,    +46.694927) --  (axis cs:4.14472,    +47.694926) --  (axis cs:4.1463,    +48.694926) ;
  \draw[constellation-boundary]  (axis cs:4.14321,    +46.694927) --  (axis cs:4.64961,    +46.694546) --  (axis cs:5.66238,    +46.693623) --  (axis cs:6.67514,    +46.692487) --  (axis cs:7.68788,    +46.691136) --  (axis cs:8.70058,    +46.689573) --  (axis cs:9.71326,    +46.687797) --  (axis cs:10.7259,    +46.685809) --  (axis cs:11.7385,    +46.683609) --  (axis cs:12.751,    +46.681199) --  (axis cs:13.7636,    +46.678578) --  (axis cs:14.776,    +46.675749) ;
  \draw[constellation-boundary]  (axis cs:14.776,    +46.675749) --  (axis cs:14.7823,    +47.675740) --  (axis cs:14.7888,    +48.675730) ;
  \draw[constellation-boundary]  (axis cs:14.7888,    +48.675730) --  (axis cs:15.8021,    +48.672689) --  (axis cs:16.8153,    +48.669441) --  (axis cs:17.8285,    +48.665986) --  (axis cs:18.5883,    +48.663260) ;
  \draw[constellation-boundary]  (axis cs:18.5883,    +48.663260) --  (axis cs:18.5969,    +49.663245) --  (axis cs:18.6058,    +50.663228) ;
  \draw[constellation-boundary]  (axis cs:18.6058,    +50.663228) --  (axis cs:18.8593,    +50.662293) --  (axis cs:19.8733,    +50.658425) --  (axis cs:20.8872,    +50.654354) --  (axis cs:21.901,    +50.650081) --  (axis cs:22.9147,    +50.645607) --  (axis cs:23.9283,    +50.640934) --  (axis cs:24.9418,    +50.636064) --  (axis cs:25.9552,    +50.630998) --  (axis cs:26.9685,    +50.625737) ;
  \draw[constellation-boundary]  (axis cs:22.4079,    +50.647869) --  (axis cs:22.4191,    +51.647844) --  (axis cs:22.4308,    +52.647818) --  (axis cs:22.4431,    +53.647791) --  (axis cs:22.4559,    +54.647763) ;
  \draw[constellation-boundary]  (axis cs:26.9314,    +47.625835) --  (axis cs:26.9432,    +48.625803) --  (axis cs:26.9556,    +49.625771) --  (axis cs:26.9685,    +50.625737) ;
  \draw[constellation-boundary]  (axis cs:26.9314,    +47.625835) --  (axis cs:27.9432,    +47.620388) --  (axis cs:28.9549,    +47.614750) --  (axis cs:29.9665,    +47.608924) --  (axis cs:30.978,    +47.602910) --  (axis cs:31.9894,    +47.596710) --  (axis cs:32.6214,    +47.592743) ;
  \draw[constellation-boundary]  (axis cs:32.6214,    +47.592743) --  (axis cs:32.6356,    +48.592698) --  (axis cs:32.6504,    +49.592651) --  (axis cs:32.6658,    +50.592602) --  (axis cs:32.6737,    +51.092577) ;
  \draw[constellation-boundary]  (axis cs:32.6737,    +51.092577) --  (axis cs:33.0535,    +51.090158) --  (axis cs:34.0661,    +51.083585) --  (axis cs:35.0785,    +51.076831) --  (axis cs:36.0908,    +51.069901) --  (axis cs:37.1029,    +51.062795) --  (axis cs:38.1149,    +51.055516) --  (axis cs:39.1267,    +51.048067) --  (axis cs:39.8854,    +51.042369) ;
  \draw[constellation-boundary]  (axis cs:39.6793,    +37.293149) --  (axis cs:39.6823,    +37.543138) --  (axis cs:39.6945,    +38.543092) --  (axis cs:39.707,    +39.543044) --  (axis cs:39.7199,    +40.542995) --  (axis cs:39.7333,    +41.542945) --  (axis cs:39.747,    +42.542893) --  (axis cs:39.7612,    +43.542839) --  (axis cs:39.7758,    +44.542784) --  (axis cs:39.791,    +45.542727) --  (axis cs:39.8067,    +46.542667) --  (axis cs:39.823,    +47.542605) --  (axis cs:39.8399,    +48.542541) --  (axis cs:39.8576,    +49.542474) --  (axis cs:39.8759,    +50.542405) --  (axis cs:39.8854,    +51.042369) ;
  \draw[constellation-boundary]  (axis cs:31.8711,    +37.347078) --  (axis cs:32.8789,    +37.340717) --  (axis cs:33.8866,    +37.334175) --  (axis cs:34.8943,    +37.327454) --  (axis cs:35.9018,    +37.320556) --  (axis cs:36.9093,    +37.313483) --  (axis cs:37.9166,    +37.306237) --  (axis cs:38.9239,    +37.298822) --  (axis cs:39.931,    +37.291238) --  (axis cs:40.4346,    +37.287384) ;
  \draw[constellation-boundary]  (axis cs:31.8542,    +35.597131) --  (axis cs:31.8638,    +36.597101) --  (axis cs:31.8711,    +37.347078) ;
  \draw[constellation-boundary]  (axis cs:22.9109,    +35.645330) --  (axis cs:23.7928,    +35.641253) --  (axis cs:24.8008,    +35.636410) --  (axis cs:25.8086,    +35.631371) --  (axis cs:26.8164,    +35.626139) --  (axis cs:27.8241,    +35.620714) --  (axis cs:28.8317,    +35.615099) --  (axis cs:29.8393,    +35.609296) --  (axis cs:30.8468,    +35.603306) --  (axis cs:31.8542,    +35.597131) ;
  \draw[constellation-boundary]  (axis cs:12.443,    +33.681888) --  (axis cs:12.695,    +33.681269) --  (axis cs:13.7028,    +33.678660) --  (axis cs:14.7106,    +33.675843) --  (axis cs:15.7184,    +33.672819) --  (axis cs:16.7261,    +33.669589) --  (axis cs:17.7338,    +33.666153) --  (axis cs:18.7414,    +33.662512) --  (axis cs:19.749,    +33.658669) --  (axis cs:20.7566,    +33.654623) --  (axis cs:21.7641,    +33.650376) --  (axis cs:22.7715,    +33.645930) --  (axis cs:22.8975,    +33.645360) ;
  \draw[constellation-boundary]  (axis cs:12.4171,    +25.681919) --  (axis cs:12.4201,    +26.681916) --  (axis cs:12.4232,    +27.681912) --  (axis cs:12.4264,    +28.681908) --  (axis cs:12.4295,    +29.681904) --  (axis cs:12.4328,    +30.681900) --  (axis cs:12.4361,    +31.681896) --  (axis cs:12.4395,    +32.681892) --  (axis cs:12.443,    +33.681888) ;
  \draw[constellation-boundary]  (axis cs:2.61133,    +25.695750) --  (axis cs:2.6118,    +26.695750) --  (axis cs:2.61228,    +27.695750) --  (axis cs:2.61277,    +28.695750) ;
  \draw[constellation-boundary]  (axis cs:1.6062,    +28.696028) --  (axis cs:2.61277,    +28.695750) ;
  \draw[constellation-boundary]  (axis cs:1.6062,    +28.696028) --  (axis cs:1.60642,    +29.696028) --  (axis cs:1.60665,    +30.696028) --  (axis cs:1.60688,    +31.696028) --  (axis cs:1.60696,    +32.029361) ;
  \draw[constellation-boundary]  (axis cs:-2.171,    +32.028500) --  (axis cs:-1.416,    +32.028913) --  (axis cs:-0.408,    +32.029276) --  (axis cs:0.599433,    +32.029425) --  (axis cs:1.60696,    +32.029361) ;
  \draw[constellation-boundary]  (axis cs:-2.171,    +32.028500) --  (axis cs:-2.172,    +32.695167) --  (axis cs:-2.172,    +32.778500) ;
  \draw[constellation-boundary]  (axis cs:-4.439,    +32.776543) --  (axis cs:-3.432,    +32.777546) --  (axis cs:-2.424,    +32.778336) --  (axis cs:-2.172,    +32.778500) ;
  \draw[constellation-boundary]  (axis cs:26.7446,    +25.626328) --  (axis cs:26.7512,    +26.626311) --  (axis cs:26.7578,    +27.626293) --  (axis cs:26.7646,    +28.626275) ;
  \draw[constellation-boundary]  (axis cs:26.7446,    +25.626328) --  (axis cs:27.7497,    +25.620918) --  (axis cs:28.7548,    +25.615317) --  (axis cs:29.7599,    +25.609528) --  (axis cs:30.5136,    +25.605064) ;
  \draw[constellation-boundary]  (axis cs:30.5136,    +25.605064) --  (axis cs:30.5211,    +26.605042) --  (axis cs:30.5286,    +27.605019) --  (axis cs:30.5305,    +27.855014) ;
  \draw[constellation-boundary]  (axis cs:30.5305,    +27.855014) --  (axis cs:30.7819,    +27.853502) --  (axis cs:31.7874,    +27.847339) --  (axis cs:32.7928,    +27.840993) --  (axis cs:33.7981,    +27.834466) --  (axis cs:34.8034,    +27.827761) --  (axis cs:35.8086,    +27.820879) --  (axis cs:36.8137,    +27.813822) --  (axis cs:37.8188,    +27.806593) --  (axis cs:38.0701,    +27.804759) ;
  \draw[constellation-boundary]  (axis cs:38.0701,    +27.804759) --  (axis cs:38.0771,    +28.554734) --  (axis cs:38.0867,    +29.554699) --  (axis cs:38.0965,    +30.554663) --  (axis cs:38.1031,    +31.221305) ;
  \draw[constellation-boundary]  (axis cs:38.1031,    +31.221305) --  (axis cs:38.8574,    +31.215736) --  (axis cs:39.8631,    +31.208163) --  (axis cs:40.8687,    +31.200425) --  (axis cs:41.8742,    +31.192524) --  (axis cs:42.8797,    +31.184461) --  (axis cs:43.885,    +31.176241) --  (axis cs:44.8902,    +31.167865) --  (axis cs:45.8954,    +31.159336) --  (axis cs:46.9005,    +31.150657) --  (axis cs:47.9054,    +31.141829) --  (axis cs:48.9103,    +31.132857) --  (axis cs:49.9151,    +31.123742) --  (axis cs:50.9198,    +31.114488) --  (axis cs:51.9243,    +31.105098) --  (axis cs:52.9288,    +31.095573) --  (axis cs:53.9332,    +31.085918) --  (axis cs:54.9375,    +31.076136) --  (axis cs:55.9416,    +31.066228) --  (axis cs:56.9457,    +31.056199) --  (axis cs:57.9496,    +31.046051) --  (axis cs:58.9535,    +31.035788) --  (axis cs:59.9572,    +31.025412) --  (axis cs:60.9608,    +31.014928) --  (axis cs:61.9643,    +31.004337) --  (axis cs:62.9677,    +30.993644) --  (axis cs:63.971,    +30.982851) --  (axis cs:64.9741,    +30.971963) --  (axis cs:65.9772,    +30.960982) --  (axis cs:66.9801,    +30.949911) --  (axis cs:67.9829,    +30.938755) --  (axis cs:68.9856,    +30.927516) --  (axis cs:69.4869,    +30.921867) ;
  \draw[constellation-boundary]  (axis cs:42.6283,    +31.186492) --  (axis cs:42.632,    +31.519810) --  (axis cs:42.6432,    +32.519765) --  (axis cs:42.6547,    +33.519719) --  (axis cs:42.6664,    +34.519672) ;
  \draw[constellation-boundary]  (axis cs:52.3572,    +25.434014) --  (axis cs:52.369,    +26.433959) --  (axis cs:52.3809,    +27.433902) --  (axis cs:52.393,    +28.433845) --  (axis cs:52.4054,    +29.433786) --  (axis cs:52.418,    +30.433726) --  (axis cs:52.4266,    +31.100352) ;
  \draw[constellation-boundary]  (axis cs:69.5738,    +36.254710) --  (axis cs:70.0754,    +36.249038) --  (axis cs:71.0784,    +36.237638) --  (axis cs:72.0812,    +36.226166) --  (axis cs:72.4573,    +36.221847) ;
  \draw[constellation-boundary]  (axis cs:72.4573,    +36.221847) --  (axis cs:72.4752,    +37.221744) --  (axis cs:72.4935,    +38.221638) --  (axis cs:72.5124,    +39.221530) --  (axis cs:72.5319,    +40.221418) --  (axis cs:72.5519,    +41.221303) --  (axis cs:72.5725,    +42.221184) --  (axis cs:72.5939,    +43.221062) --  (axis cs:72.6159,    +44.220935) --  (axis cs:72.6386,    +45.220804) --  (axis cs:72.6622,    +46.220668) --  (axis cs:72.6867,    +47.220528) --  (axis cs:72.7121,    +48.220382) --  (axis cs:72.7385,    +49.220230) --  (axis cs:72.7661,    +50.220071) --  (axis cs:72.7947,    +51.219906) --  (axis cs:72.8247,    +52.219734) --  (axis cs:72.8402,    +52.719645) ;
  \draw[constellation-boundary]  (axis cs:77.4847,    +52.665551) --  (axis cs:77.5009,    +53.165456) --  (axis cs:77.5345,    +54.165258) --  (axis cs:77.5697,    +55.165051) --  (axis cs:77.6067,    +56.164833) ;
  \draw[constellation-boundary]  (axis cs:77.6067,    +56.164833) --  (axis cs:78.6106,    +56.152981) --  (axis cs:79.6142,    +56.141084) --  (axis cs:80.6175,    +56.129146) --  (axis cs:81.6205,    +56.117170) --  (axis cs:82.6231,    +56.105161) --  (axis cs:83.6255,    +56.093121) --  (axis cs:84.6275,    +56.081054) --  (axis cs:85.6292,    +56.068965) --  (axis cs:86.6306,    +56.056857) --  (axis cs:87.6317,    +56.044733) --  (axis cs:88.6325,    +56.032597) --  (axis cs:89.6329,    +56.020454) --  (axis cs:90.6331,    +56.008306) --  (axis cs:91.6329,    +55.996157) --  (axis cs:92.6324,    +55.984012) --  (axis cs:93.6316,    +55.971873) --  (axis cs:94.1311,    +55.965808) ;
  \draw[constellation-boundary]  (axis cs:94.0573,    +53.966255) --  (axis cs:94.5568,    +53.960192) --  (axis cs:95.5555,    +53.948077) --  (axis cs:96.5539,    +53.935979) --  (axis cs:97.552,    +53.923903) --  (axis cs:98.5498,    +53.911851) --  (axis cs:99.5473,    +53.899828) --  (axis cs:100.046,    +53.893829) ;
  \draw[constellation-boundary]  (axis cs:94.0573,    +53.966255) --  (axis cs:94.0933,    +54.966037) --  (axis cs:94.1311,    +55.965808) --  (axis cs:94.1708,    +56.965566) --  (axis cs:94.2128,    +57.965311) --  (axis cs:94.2572,    +58.965042) --  (axis cs:94.3042,    +59.964756) --  (axis cs:94.3542,    +60.964453) --  (axis cs:94.4074,    +61.964130) ;
  \draw[constellation-boundary]  (axis cs:99.9194,    +49.894589) --  (axis cs:99.949,    +50.894412) --  (axis cs:99.9799,    +51.894226) --  (axis cs:100.012,    +52.894032) --  (axis cs:100.046,    +53.893829) ;
  \draw[constellation-boundary]  (axis cs:99.9194,    +49.894589) --  (axis cs:100.418,    +49.888596) --  (axis cs:101.416,    +49.876636) --  (axis cs:102.413,    +49.864715) --  (axis cs:103.41,    +49.852837) --  (axis cs:104.406,    +49.841005) ;
  \draw[constellation-boundary]  (axis cs:104.265,    +44.341840) --  (axis cs:104.277,    +44.841771) --  (axis cs:104.301,    +45.841629) --  (axis cs:104.326,    +46.841482) --  (axis cs:104.352,    +47.841329) --  (axis cs:104.378,    +48.841170) --  (axis cs:104.406,    +49.841005) ;
  \draw[constellation-boundary]  (axis cs:104.265,    +44.341840) --  (axis cs:105.262,    +44.330051) --  (axis cs:106.259,    +44.318314) --  (axis cs:107.256,    +44.306633) --  (axis cs:108.252,    +44.295012) --  (axis cs:109.249,    +44.283455) --  (axis cs:110.245,    +44.271964) --  (axis cs:111.241,    +44.260544) --  (axis cs:112.236,    +44.249198) --  (axis cs:112.734,    +44.243553) ;
  \draw[constellation-boundary]  (axis cs:112.561,    +35.244534) --  (axis cs:112.569,    +35.744486) --  (axis cs:112.587,    +36.744388) --  (axis cs:112.604,    +37.744286) --  (axis cs:112.623,    +38.744183) --  (axis cs:112.642,    +39.744076) --  (axis cs:112.661,    +40.743966) --  (axis cs:112.681,    +41.743853) --  (axis cs:112.702,    +42.743736) --  (axis cs:112.723,    +43.743615) --  (axis cs:112.734,    +44.243553) ;
  \draw[constellation-boundary]  (axis cs:100.09,    +35.390565) --  (axis cs:101.089,    +35.378593) --  (axis cs:102.087,    +35.366659) --  (axis cs:103.085,    +35.354766) --  (axis cs:104.083,    +35.342919) --  (axis cs:105.081,    +35.331120) --  (axis cs:106.079,    +35.319373) --  (axis cs:107.076,    +35.307682) --  (axis cs:108.074,    +35.296050) --  (axis cs:109.071,    +35.284481) --  (axis cs:110.068,    +35.272978) --  (axis cs:111.065,    +35.261545) --  (axis cs:112.062,    +35.250186) --  (axis cs:113.059,    +35.238903) --  (axis cs:114.056,    +35.227700) --  (axis cs:115.052,    +35.216580) --  (axis cs:116.048,    +35.205547) --  (axis cs:117.045,    +35.194605) --  (axis cs:118.041,    +35.183755) --  (axis cs:118.29,    +35.181058) ;
  \draw[constellation-boundary]  (axis cs:99.9657,    +27.891313) --  (axis cs:99.9812,    +28.891220) --  (axis cs:99.997,    +29.891125) --  (axis cs:100.013,    +30.891028) --  (axis cs:100.03,    +31.890929) --  (axis cs:100.046,    +32.890828) --  (axis cs:100.064,    +33.890724) --  (axis cs:100.081,    +34.890618) --  (axis cs:100.09,    +35.390565) ;
  \draw[constellation-boundary]  (axis cs:90.2211,    +28.009289) --  (axis cs:90.9711,    +28.000178) --  (axis cs:91.9709,    +27.988030) --  (axis cs:92.9707,    +27.975886) --  (axis cs:93.9703,    +27.963751) --  (axis cs:94.9698,    +27.951627) --  (axis cs:95.9692,    +27.939519) --  (axis cs:96.9685,    +27.927429) --  (axis cs:97.9677,    +27.915363) --  (axis cs:98.9667,    +27.903323) --  (axis cs:99.9657,    +27.891313) ;
  \draw[constellation-boundary]  (axis cs:73.2125,    +28.712437) --  (axis cs:73.9639,    +28.703734) --  (axis cs:74.9657,    +28.692076) --  (axis cs:75.9674,    +28.680359) --  (axis cs:76.9689,    +28.668589) --  (axis cs:77.9704,    +28.656768) --  (axis cs:78.9717,    +28.644899) --  (axis cs:79.973,    +28.632987) --  (axis cs:80.9741,    +28.621035) --  (axis cs:81.9751,    +28.609047) --  (axis cs:82.976,    +28.597026) --  (axis cs:83.9767,    +28.584977) --  (axis cs:84.9774,    +28.572902) --  (axis cs:85.9779,    +28.560806) --  (axis cs:86.9783,    +28.548691) --  (axis cs:87.9787,    +28.536563) --  (axis cs:88.9788,    +28.524424) --  (axis cs:89.9789,    +28.512279) --  (axis cs:90.2289,    +28.509242) ;
  \draw[constellation-boundary]  (axis cs:73.2125,    +28.712437) --  (axis cs:73.22,    +29.212394) --  (axis cs:73.2353,    +30.212305) ;
  \draw[constellation-boundary]  (axis cs:69.4768,    +30.255257) --  (axis cs:69.492,    +31.255171) --  (axis cs:69.5077,    +32.255083) --  (axis cs:69.5236,    +33.254993) --  (axis cs:69.54,    +34.254901) --  (axis cs:69.5567,    +35.254806) --  (axis cs:69.5738,    +36.254710) ;
  \draw[constellation-boundary]  (axis cs:69.4768,    +30.255257) --  (axis cs:69.978,    +30.249589) --  (axis cs:70.9804,    +30.238197) --  (axis cs:71.9827,    +30.226732) --  (axis cs:72.9848,    +30.215199) --  (axis cs:73.2353,    +30.212305) ;
  \draw[constellation-boundary]  (axis cs:52.3131,    +52.936601) --  (axis cs:53.3229,    +52.927025) --  (axis cs:54.3325,    +52.917320) --  (axis cs:55.3419,    +52.907486) --  (axis cs:56.351,    +52.897529) --  (axis cs:57.3599,    +52.887451) --  (axis cs:58.3686,    +52.877255) --  (axis cs:59.377,    +52.866944) --  (axis cs:60.3852,    +52.856522) --  (axis cs:61.3931,    +52.845991) --  (axis cs:62.4007,    +52.835356) --  (axis cs:63.4082,    +52.824619) --  (axis cs:64.4153,    +52.813784) --  (axis cs:65.4223,    +52.802854) --  (axis cs:66.4289,    +52.791832) --  (axis cs:67.4353,    +52.780722) --  (axis cs:68.4415,    +52.769528) --  (axis cs:69.4473,    +52.758252) --  (axis cs:70.4529,    +52.746899) --  (axis cs:71.4583,    +52.735472) --  (axis cs:72.4634,    +52.723974) --  (axis cs:73.4682,    +52.712409) --  (axis cs:74.4727,    +52.700781) --  (axis cs:75.477,    +52.689093) --  (axis cs:76.481,    +52.677348) --  (axis cs:77.4847,    +52.665551) ;
  \draw[constellation-boundary]  (axis cs:52.3131,    +52.936601) --  (axis cs:52.3262,    +53.436539) --  (axis cs:52.3534,    +54.436411) --  (axis cs:52.3819,    +55.436277) ;
  \draw[constellation-boundary]  (axis cs:49.8539,    +55.459649) --  (axis cs:50.3597,    +55.455044) --  (axis cs:51.3709,    +55.445728) --  (axis cs:52.3819,    +55.436277) ;
  \draw[constellation-boundary]  (axis cs:49.8539,    +55.459649) --  (axis cs:49.8829,    +56.459518) --  (axis cs:49.9134,    +57.459380) ;
  \draw[constellation-boundary]  (axis cs:48.9009,    +57.468493) --  (axis cs:48.9326,    +58.468352) --  (axis cs:48.9663,    +59.468202) --  (axis cs:49.0019,    +60.468042) --  (axis cs:49.0398,    +61.467873) --  (axis cs:49.0803,    +62.467693) --  (axis cs:49.1235,    +63.467500) --  (axis cs:49.1699,    +64.467293) --  (axis cs:49.2198,    +65.467070) --  (axis cs:49.2736,    +66.466829) --  (axis cs:49.332,    +67.466569) --  (axis cs:49.3954,    +68.466285) ;
  \draw[constellation-boundary]  (axis cs:49.3954,    +68.466285) --  (axis cs:49.9055,    +68.461712) --  (axis cs:50.9254,    +68.452458) --  (axis cs:51.9448,    +68.443065) --  (axis cs:52.9638,    +68.433537) --  (axis cs:53.9824,    +68.423877) --  (axis cs:54.2369,    +68.421442) ;
  \draw[constellation-boundary]  (axis cs:54.2369,    +68.421442) --  (axis cs:54.3111,    +69.421087) --  (axis cs:54.3924,    +70.420697) --  (axis cs:54.4821,    +71.420267) --  (axis cs:54.5816,    +72.419790) --  (axis cs:54.6928,    +73.419258) --  (axis cs:54.8178,    +74.418659) --  (axis cs:54.9594,    +75.417980) --  (axis cs:55.1214,    +76.417204) --  (axis cs:55.3086,    +77.416307) ;
  \draw[constellation-boundary]  (axis cs:56.7262,    +77.402587) --  (axis cs:56.9487,    +78.401501) --  (axis cs:57.2125,    +79.400213) --  (axis cs:57.5305,    +80.398661) ;
  \draw[constellation-boundary]  (axis cs:57.5305,    +80.398661) --  (axis cs:57.9202,    +80.394843) --  (axis cs:58.9585,    +80.384582) --  (axis cs:59.9957,    +80.374204) --  (axis cs:61.0317,    +80.363713) --  (axis cs:62.0666,    +80.353113) --  (axis cs:63.1003,    +80.342408) --  (axis cs:64.1328,    +80.331602) --  (axis cs:65.1641,    +80.320697) --  (axis cs:66.1943,    +80.309697) --  (axis cs:67.2232,    +80.298607) --  (axis cs:68.251,    +80.287429) --  (axis cs:69.2775,    +80.276168) --  (axis cs:70.3029,    +80.264827) --  (axis cs:71.327,    +80.253410) --  (axis cs:72.3499,    +80.241920) --  (axis cs:73.3715,    +80.230362) --  (axis cs:74.392,    +80.218739) --  (axis cs:75.4112,    +80.207055) --  (axis cs:76.4292,    +80.195314) --  (axis cs:77.4459,    +80.183519) --  (axis cs:78.4615,    +80.171675) --  (axis cs:79.4758,    +80.159784) --  (axis cs:80.4888,    +80.147851) ;
  \draw[constellation-boundary]  (axis cs:80.4888,    +80.147851) --  (axis cs:80.936,    +81.145214) --  (axis cs:81.4956,    +82.141914) --  (axis cs:82.2165,    +83.137659) --  (axis cs:83.1807,    +84.131965) --  (axis cs:84.536,    +85.123951) ;
  \draw[constellation-boundary]  (axis cs:84.536,    +85.123951) --  (axis cs:85.5502,    +85.111864) --  (axis cs:86.5619,    +85.099757) --  (axis cs:87.571,    +85.087634) --  (axis cs:88.5777,    +85.075499) --  (axis cs:89.5818,    +85.063355) --  (axis cs:90.5835,    +85.051208) --  (axis cs:91.5827,    +85.039059) --  (axis cs:92.5795,    +85.026913) --  (axis cs:93.5739,    +85.014774) --  (axis cs:94.5659,    +85.002645) --  (axis cs:95.5555,    +84.990530) --  (axis cs:96.5428,    +84.978432) --  (axis cs:97.5277,    +84.966356) --  (axis cs:98.5103,    +84.954303) --  (axis cs:99.4907,    +84.942279) --  (axis cs:100.469,    +84.930285) --  (axis cs:101.445,    +84.918327) --  (axis cs:102.418,    +84.906407) --  (axis cs:103.389,    +84.894528) --  (axis cs:104.359,    +84.882694) --  (axis cs:105.326,    +84.870909) --  (axis cs:106.291,    +84.859174) --  (axis cs:107.253,    +84.847495) --  (axis cs:108.214,    +84.835873) --  (axis cs:109.173,    +84.824312) --  (axis cs:110.13,    +84.812814) --  (axis cs:111.084,    +84.801384) --  (axis cs:112.037,    +84.790024) --  (axis cs:112.987,    +84.778738) --  (axis cs:113.936,    +84.767527) --  (axis cs:114.883,    +84.756395) --  (axis cs:115.828,    +84.745345) --  (axis cs:116.771,    +84.734379) --  (axis cs:117.712,    +84.723502) --  (axis cs:118.652,    +84.712714) --  (axis cs:119.59,    +84.702019) --  (axis cs:120.526,    +84.691420) --  (axis cs:121.46,    +84.680920) --  (axis cs:122.392,    +84.670520) --  (axis cs:123.323,    +84.660224) --  (axis cs:124.253,    +84.650033) --  (axis cs:125.18,    +84.639952) --  (axis cs:126.106,    +84.629981) --  (axis cs:127.031,    +84.620124) --  (axis cs:127.954,    +84.610382) ;
  \draw[constellation-boundary]  (axis cs:127.954,    +84.610382) --  (axis cs:129.403,    +85.602792) --  (axis cs:131.692,    +86.590773) --  (axis cs:135.832,    +87.568922) ;
  \draw[constellation-boundary]  (axis cs:130.403,    +86.097546) --  (axis cs:131.288,    +86.088246) --  (axis cs:132.172,    +86.079068) --  (axis cs:133.054,    +86.070013) --  (axis cs:133.934,    +86.061083) --  (axis cs:134.812,    +86.052282) --  (axis cs:135.689,    +86.043609) --  (axis cs:136.564,    +86.035068) --  (axis cs:137.438,    +86.026659) --  (axis cs:138.31,    +86.018385) --  (axis cs:139.18,    +86.010247) --  (axis cs:140.049,    +86.002247) --  (axis cs:140.917,    +85.994387) --  (axis cs:141.783,    +85.986667) --  (axis cs:142.648,    +85.979090) --  (axis cs:143.511,    +85.971657) --  (axis cs:144.373,    +85.964370) --  (axis cs:145.234,    +85.957230) --  (axis cs:146.093,    +85.950238) --  (axis cs:146.951,    +85.943395) --  (axis cs:147.808,    +85.936704) --  (axis cs:148.664,    +85.930165) --  (axis cs:149.519,    +85.923779) --  (axis cs:150.373,    +85.917549) --  (axis cs:151.225,    +85.911474) --  (axis cs:152.077,    +85.905557) --  (axis cs:152.928,    +85.899797) --  (axis cs:153.777,    +85.894198) --  (axis cs:154.626,    +85.888758) --  (axis cs:155.474,    +85.883480) --  (axis cs:156.321,    +85.878364) --  (axis cs:157.167,    +85.873412) --  (axis cs:158.012,    +85.868624) --  (axis cs:158.856,    +85.864001) --  (axis cs:159.7,    +85.859544) --  (axis cs:160.543,    +85.855254) --  (axis cs:161.385,    +85.851131) --  (axis cs:162.227,    +85.847177) --  (axis cs:163.068,    +85.843392) --  (axis cs:163.909,    +85.839777) --  (axis cs:164.748,    +85.836332) --  (axis cs:165.588,    +85.833059) --  (axis cs:166.426,    +85.829957) --  (axis cs:167.265,    +85.827027) --  (axis cs:168.102,    +85.824270) --  (axis cs:168.94,    +85.821686) --  (axis cs:169.777,    +85.819276) --  (axis cs:170.613,    +85.817040) --  (axis cs:171.45,    +85.814979) --  (axis cs:172.286,    +85.813092) --  (axis cs:173.121,    +85.811381) --  (axis cs:173.957,    +85.809845) --  (axis cs:174.792,    +85.808486) --  (axis cs:175.627,    +85.807302) --  (axis cs:176.462,    +85.806295) --  (axis cs:177.296,    +85.805464) --  (axis cs:178.131,    +85.804809) --  (axis cs:178.965,    +85.804332) --  (axis cs:179.8,    +85.804031) ;
  \draw[constellation-boundary]  (axis cs:174.435,    +76.308421) --  (axis cs:174.91,    +76.307772) --  (axis cs:175.86,    +76.306626) --  (axis cs:176.81,    +76.305680) --  (axis cs:177.761,    +76.304935) --  (axis cs:178.711,    +76.304391) --  (axis cs:179.661,    +76.304049) ;
  \draw[constellation-boundary]  (axis cs:174.435,    +76.308421) --  (axis cs:174.462,    +77.308401) --  (axis cs:174.494,    +78.308379) --  (axis cs:174.532,    +79.308352) ;
  \draw[constellation-boundary]  (axis cs:162.819,    +79.340189) --  (axis cs:163.757,    +79.336533) --  (axis cs:164.695,    +79.333068) --  (axis cs:165.633,    +79.329793) --  (axis cs:166.57,    +79.326710) --  (axis cs:167.508,    +79.323820) --  (axis cs:168.445,    +79.321124) --  (axis cs:169.382,    +79.318621) --  (axis cs:170.318,    +79.316314) --  (axis cs:171.255,    +79.314201) --  (axis cs:172.191,    +79.312284) --  (axis cs:173.127,    +79.310564) --  (axis cs:174.064,    +79.309040) --  (axis cs:174.532,    +79.308352) ;
  \draw[constellation-boundary]  (axis cs:162.819,    +79.340189) --  (axis cs:162.947,    +80.339932) --  (axis cs:163.105,    +81.339616) ;
  \draw[constellation-boundary]  (axis cs:142.191,    +81.467773) --  (axis cs:142.659,    +81.464002) --  (axis cs:143.595,    +81.456577) --  (axis cs:144.53,    +81.449310) --  (axis cs:145.465,    +81.442203) --  (axis cs:146.398,    +81.435258) --  (axis cs:147.332,    +81.428475) --  (axis cs:148.264,    +81.421858) --  (axis cs:149.196,    +81.415408) --  (axis cs:150.127,    +81.409125) --  (axis cs:151.057,    +81.403013) --  (axis cs:151.987,    +81.397071) --  (axis cs:152.916,    +81.391303) --  (axis cs:153.845,    +81.385708) --  (axis cs:154.773,    +81.380289) --  (axis cs:155.701,    +81.375046) --  (axis cs:156.628,    +81.369982) --  (axis cs:157.555,    +81.365097) --  (axis cs:158.481,    +81.360393) --  (axis cs:159.407,    +81.355870) --  (axis cs:160.332,    +81.351530) --  (axis cs:161.257,    +81.347373) --  (axis cs:162.181,    +81.343402) --  (axis cs:163.105,    +81.339616) ;
  \draw[constellation-boundary]  (axis cs:140.615,    +72.974143) --  (axis cs:140.664,    +73.473947) --  (axis cs:140.77,    +74.473518) --  (axis cs:140.89,    +75.473031) --  (axis cs:141.028,    +76.472475) --  (axis cs:141.187,    +77.471831) --  (axis cs:141.374,    +78.471077) --  (axis cs:141.595,    +79.470183) --  (axis cs:141.862,    +80.469103) --  (axis cs:142.191,    +81.467773) ;
  \draw[constellation-boundary]  (axis cs:123.086,    +73.138381) --  (axis cs:123.575,    +73.133260) --  (axis cs:124.553,    +73.123101) --  (axis cs:125.531,    +73.113058) --  (axis cs:126.508,    +73.103133) --  (axis cs:127.484,    +73.093329) --  (axis cs:128.46,    +73.083649) --  (axis cs:129.435,    +73.074095) --  (axis cs:130.41,    +73.064669) --  (axis cs:131.384,    +73.055376) --  (axis cs:132.358,    +73.046217) --  (axis cs:133.331,    +73.037195) --  (axis cs:134.304,    +73.028312) --  (axis cs:135.276,    +73.019572) --  (axis cs:136.248,    +73.010976) --  (axis cs:137.219,    +73.002526) --  (axis cs:138.19,    +72.994226) --  (axis cs:139.16,    +72.986077) --  (axis cs:140.131,    +72.978082) --  (axis cs:141.1,    +72.970243) --  (axis cs:142.069,    +72.962562) --  (axis cs:143.038,    +72.955041) --  (axis cs:144.007,    +72.947683) --  (axis cs:144.975,    +72.940489) --  (axis cs:145.942,    +72.933461) --  (axis cs:146.909,    +72.926602) --  (axis cs:147.876,    +72.919913) --  (axis cs:148.843,    +72.913396) --  (axis cs:149.809,    +72.907053) --  (axis cs:150.775,    +72.900886) --  (axis cs:151.741,    +72.894895) --  (axis cs:152.706,    +72.889084) --  (axis cs:153.671,    +72.883453) --  (axis cs:154.636,    +72.878005) --  (axis cs:155.6,    +72.872740) --  (axis cs:156.564,    +72.867660) --  (axis cs:157.528,    +72.862766) --  (axis cs:158.492,    +72.858060) --  (axis cs:159.455,    +72.853543) --  (axis cs:160.418,    +72.849216) --  (axis cs:161.381,    +72.845080) --  (axis cs:162.344,    +72.841138) --  (axis cs:163.306,    +72.837388) --  (axis cs:164.268,    +72.833834) --  (axis cs:165.231,    +72.830475) --  (axis cs:166.193,    +72.827312) --  (axis cs:167.154,    +72.824347) --  (axis cs:168.116,    +72.821580) --  (axis cs:169.078,    +72.819012) --  (axis cs:170.039,    +72.816643) --  (axis cs:171,    +72.814475) --  (axis cs:171.961,    +72.812508) ;
  \draw[constellation-boundary]  (axis cs:122.129,    +59.643402) --  (axis cs:122.171,    +60.643180) --  (axis cs:122.216,    +61.642943) --  (axis cs:122.264,    +62.642691) --  (axis cs:122.316,    +63.642422) --  (axis cs:122.371,    +64.642132) --  (axis cs:122.43,    +65.641821) --  (axis cs:122.495,    +66.641484) --  (axis cs:122.564,    +67.641120) --  (axis cs:122.64,    +68.640723) --  (axis cs:122.723,    +69.640288) --  (axis cs:122.814,    +70.639811) --  (axis cs:122.914,    +71.639284) --  (axis cs:123.026,    +72.638699) --  (axis cs:123.086,    +73.138381) ;
  \draw[constellation-boundary]  (axis cs:107.753,    +59.803726) --  (axis cs:107.801,    +60.803448) --  (axis cs:107.852,    +61.803151) ;
  \draw[constellation-boundary]  (axis cs:107.753,    +59.803726) --  (axis cs:108.747,    +59.792136) --  (axis cs:109.74,    +59.780612) --  (axis cs:110.733,    +59.769155) --  (axis cs:111.726,    +59.757771) --  (axis cs:112.718,    +59.746461) --  (axis cs:113.71,    +59.735230) --  (axis cs:114.702,    +59.724081) --  (axis cs:115.693,    +59.713017) --  (axis cs:116.684,    +59.702042) --  (axis cs:117.675,    +59.691158) --  (axis cs:118.665,    +59.680369) --  (axis cs:119.655,    +59.669678) --  (axis cs:120.645,    +59.659089) --  (axis cs:121.634,    +59.648604) --  (axis cs:122.624,    +59.638227) --  (axis cs:123.612,    +59.627960) --  (axis cs:124.601,    +59.617806) --  (axis cs:125.589,    +59.607770) --  (axis cs:126.577,    +59.597853) --  (axis cs:127.565,    +59.588058) --  (axis cs:128.552,    +59.578389) --  (axis cs:128.799,    +59.575991) ;
  \draw[constellation-boundary]  (axis cs:94.4074,    +61.964130) --  (axis cs:94.9066,    +61.958069) --  (axis cs:95.9047,    +61.945960) --  (axis cs:96.9024,    +61.933869) --  (axis cs:97.8997,    +61.921801) --  (axis cs:98.8966,    +61.909759) --  (axis cs:99.8932,    +61.897747) --  (axis cs:100.889,    +61.885768) --  (axis cs:101.885,    +61.873826) --  (axis cs:102.88,    +61.861925) --  (axis cs:103.875,    +61.850067) --  (axis cs:104.87,    +61.838258) --  (axis cs:105.864,    +61.826500) --  (axis cs:106.858,    +61.814796) --  (axis cs:107.852,    +61.803151) ;
  \draw[constellation-boundary]  (axis cs:140.568,    +25.469429) --  (axis cs:140.578,    +26.469391) --  (axis cs:140.588,    +27.469352) --  (axis cs:140.598,    +28.469312) --  (axis cs:140.608,    +29.469271) --  (axis cs:140.619,    +30.469230) --  (axis cs:140.629,    +31.469187) --  (axis cs:140.64,    +32.469144) --  (axis cs:140.652,    +33.469100) --  (axis cs:140.663,    +34.469055) --  (axis cs:140.675,    +35.469009) --  (axis cs:140.687,    +36.468961) --  (axis cs:140.699,    +37.468913) --  (axis cs:140.712,    +38.468863) --  (axis cs:140.722,    +39.218824) ;
  \draw[constellation-boundary]  (axis cs:120.145,    +25.660426) --  (axis cs:120.158,    +26.660356) --  (axis cs:120.172,    +27.660285) ;
  \draw[constellation-boundary]  (axis cs:120.172,    +27.660285) --  (axis cs:120.919,    +27.652374) --  (axis cs:121.916,    +27.641919) ;
  \draw[constellation-boundary]  (axis cs:121.916,    +27.641919) --  (axis cs:121.929,    +28.641849) --  (axis cs:121.943,    +29.641778) --  (axis cs:121.957,    +30.641705) --  (axis cs:121.971,    +31.641630) --  (axis cs:121.986,    +32.641554) --  (axis cs:121.993,    +33.141516) ;
  \draw[constellation-boundary]  (axis cs:-4.782,    +63.692873) ;
  \draw[constellation-boundary]  (axis cs:-4.782,    +63.692873) --  (axis cs:-4.788,    +64.692869) --  (axis cs:-4.795,    +65.692865) --  (axis cs:-4.802,    +66.692861) ;
  \draw[constellation-boundary]  (axis cs:-4.802,    +66.692861) --  (axis cs:-4.545,    +66.693151) --  (axis cs:-3.517,    +66.694174) --  (axis cs:-2.489,    +66.694980) --  (axis cs:-1.461,    +66.695569) --  (axis cs:-0.433,    +66.695939) --  (axis cs:0.595318,    +66.696092) --  (axis cs:1.62345,    +66.696026) --  (axis cs:2.65157,    +66.695743) --  (axis cs:3.67968,    +66.695241) --  (axis cs:4.70775,    +66.694522) --  (axis cs:5.7358,    +66.693586) --  (axis cs:6.7638,    +66.692432) ;
  \draw[constellation-boundary]  (axis cs:6.7638,    +66.692432) --  (axis cs:6.77197,    +67.692427) --  (axis cs:6.78087,    +68.692422) --  (axis cs:6.79061,    +69.692416) --  (axis cs:6.8013,    +70.692409) --  (axis cs:6.81313,    +71.692402) --  (axis cs:6.82626,    +72.692394) --  (axis cs:6.84096,    +73.692385) --  (axis cs:6.85752,    +74.692375) --  (axis cs:6.87634,    +75.692363) --  (axis cs:6.89792,    +76.692350) --  (axis cs:6.92295,    +77.692335) ;
  \draw[constellation-boundary]  (axis cs:6.92295,    +77.692335) --  (axis cs:7.97818,    +77.690928) --  (axis cs:9.03327,    +77.689299) --  (axis cs:10.0882,    +77.687449) --  (axis cs:11.143,    +77.685378) --  (axis cs:12.1976,    +77.683087) --  (axis cs:13.2519,    +77.680577) --  (axis cs:14.306,    +77.677848) --  (axis cs:15.3599,    +77.674903) --  (axis cs:16.4135,    +77.671741) --  (axis cs:17.4668,    +77.668364) --  (axis cs:18.5198,    +77.664773) --  (axis cs:19.5725,    +77.660969) --  (axis cs:20.6249,    +77.656954) --  (axis cs:21.6768,    +77.652729) --  (axis cs:22.7284,    +77.648296) --  (axis cs:23.7796,    +77.643656) --  (axis cs:24.8304,    +77.638811) --  (axis cs:25.8807,    +77.633762) --  (axis cs:26.9306,    +77.628512) --  (axis cs:27.98,    +77.623062) --  (axis cs:29.0289,    +77.617414) --  (axis cs:30.0774,    +77.611570) --  (axis cs:31.1253,    +77.605532) --  (axis cs:32.1727,    +77.599303) --  (axis cs:33.2195,    +77.592884) --  (axis cs:34.2657,    +77.586278) --  (axis cs:35.3114,    +77.579486) --  (axis cs:36.3565,    +77.572512) --  (axis cs:37.401,    +77.565358) --  (axis cs:38.4448,    +77.558026) --  (axis cs:39.488,    +77.550519) --  (axis cs:40.5306,    +77.542839) --  (axis cs:41.5724,    +77.534989) --  (axis cs:42.6137,    +77.526972) --  (axis cs:43.6542,    +77.518791) --  (axis cs:44.694,    +77.510448) --  (axis cs:45.7331,    +77.501946) --  (axis cs:46.7714,    +77.493288) --  (axis cs:47.809,    +77.484477) --  (axis cs:48.8459,    +77.475516) --  (axis cs:49.882,    +77.466409) --  (axis cs:50.9174,    +77.457157) --  (axis cs:51.9519,    +77.447765) --  (axis cs:52.9857,    +77.438235) --  (axis cs:54.0186,    +77.428571) --  (axis cs:55.0507,    +77.418775) --  (axis cs:56.0821,    +77.408852) --  (axis cs:56.7262,    +77.402587) ;
  \draw[constellation-boundary]  (axis cs:38.7624,    +57.551291) --  (axis cs:39.2698,    +57.547534) --  (axis cs:40.2846,    +57.539892) --  (axis cs:41.2992,    +57.532084) --  (axis cs:42.3135,    +57.524113) --  (axis cs:43.3276,    +57.515981) --  (axis cs:44.3415,    +57.507690) --  (axis cs:45.3551,    +57.499244) --  (axis cs:46.3685,    +57.490644) --  (axis cs:47.3816,    +57.481894) --  (axis cs:48.3945,    +57.472997) --  (axis cs:49.4072,    +57.463954) --  (axis cs:49.9134,    +57.459380) ;
  \draw[constellation-boundary]  (axis cs:38.7624,    +57.551291) --  (axis cs:38.7887,    +58.551195) --  (axis cs:38.8024,    +59.051144) ;
  \draw[constellation-boundary]  (axis cs:30.7956,    +59.104599) --  (axis cs:31.1772,    +59.102308) --  (axis cs:32.1945,    +59.096072) --  (axis cs:33.2117,    +59.089652) --  (axis cs:34.2287,    +59.083050) --  (axis cs:35.2454,    +59.076267) --  (axis cs:36.262,    +59.069307) --  (axis cs:37.2783,    +59.062172) --  (axis cs:38.2944,    +59.054863) --  (axis cs:38.8024,    +59.051144) ;
  \draw[constellation-boundary]  (axis cs:30.7737,    +58.104665) --  (axis cs:30.7845,    +58.604632) --  (axis cs:30.7956,    +59.104599) ;
  \draw[constellation-boundary]  (axis cs:27.5952,    +58.122710) --  (axis cs:28.1039,    +58.119948) --  (axis cs:29.1211,    +58.114280) --  (axis cs:30.1381,    +58.108422) --  (axis cs:30.7737,    +58.104665) ;
  \draw[constellation-boundary]  (axis cs:27.5336,    +54.622876) --  (axis cs:27.5501,    +55.622832) --  (axis cs:27.5674,    +56.622785) --  (axis cs:27.5857,    +57.622736) --  (axis cs:27.5952,    +58.122710) ;
  \draw[constellation-boundary]  (axis cs:22.4559,    +54.647763) --  (axis cs:22.9638,    +54.645496) --  (axis cs:23.9796,    +54.640814) --  (axis cs:24.9952,    +54.635933) --  (axis cs:26.0107,    +54.630856) --  (axis cs:27.026,    +54.625584) --  (axis cs:27.5336,    +54.622876) ;
  \draw[constellation-boundary]  (axis cs:-4.104,    +88.694433) --  (axis cs:-2.572,    +88.695310) --  (axis cs:-1.039,    +88.695862) --  (axis cs:0.49436,    +88.696090) --  (axis cs:2.02799,    +88.695992) --  (axis cs:3.56121,    +88.695569) --  (axis cs:5.0935,    +88.694822) --  (axis cs:6.62436,    +88.693751) --  (axis cs:8.15327,    +88.692357) --  (axis cs:9.67974,    +88.690642) --  (axis cs:11.2033,    +88.688607) --  (axis cs:12.7234,    +88.686255) --  (axis cs:14.2396,    +88.683588) --  (axis cs:15.7515,    +88.680609) --  (axis cs:17.2586,    +88.677321) --  (axis cs:18.7604,    +88.673726) --  (axis cs:20.2566,    +88.669830) --  (axis cs:21.7468,    +88.665635) --  (axis cs:23.2306,    +88.661145) --  (axis cs:24.7075,    +88.656365) --  (axis cs:26.1773,    +88.651300) --  (axis cs:27.6397,    +88.645954) --  (axis cs:29.0943,    +88.640331) --  (axis cs:30.5409,    +88.634438) --  (axis cs:31.9791,    +88.628280) --  (axis cs:33.4088,    +88.621861) --  (axis cs:34.8297,    +88.615188) --  (axis cs:36.2416,    +88.608266) --  (axis cs:37.6443,    +88.601100) --  (axis cs:39.0376,    +88.593698) --  (axis cs:40.4215,    +88.586064) --  (axis cs:41.7957,    +88.578205) --  (axis cs:43.1602,    +88.570127) --  (axis cs:44.5148,    +88.561836) --  (axis cs:45.8595,    +88.553338) --  (axis cs:47.1943,    +88.544639) --  (axis cs:48.519,    +88.535746) --  (axis cs:49.8336,    +88.526665) --  (axis cs:51.1382,    +88.517401) --  (axis cs:52.4327,    +88.507962) --  (axis cs:53.7171,    +88.498353) --  (axis cs:54.9914,    +88.488580) --  (axis cs:56.2557,    +88.478650) --  (axis cs:57.5101,    +88.468569) --  (axis cs:58.7545,    +88.458342) --  (axis cs:59.989,    +88.447975) --  (axis cs:61.2137,    +88.437475) --  (axis cs:62.4286,    +88.426848) --  (axis cs:63.634,    +88.416098) --  (axis cs:64.8297,    +88.405232) --  (axis cs:66.016,    +88.394256) --  (axis cs:67.193,    +88.383175) --  (axis cs:68.3607,    +88.371994) --  (axis cs:69.5192,    +88.360719) --  (axis cs:70.6687,    +88.349355) --  (axis cs:71.8093,    +88.337907) --  (axis cs:72.9411,    +88.326381) --  (axis cs:74.0642,    +88.314782) --  (axis cs:75.1788,    +88.303114) --  (axis cs:76.285,    +88.291383) --  (axis cs:77.3829,    +88.279594) --  (axis cs:78.4726,    +88.267750) --  (axis cs:79.5543,    +88.255858) --  (axis cs:80.6281,    +88.243921) --  (axis cs:81.6942,    +88.231944) --  (axis cs:82.7526,    +88.219931) --  (axis cs:83.8035,    +88.207887) --  (axis cs:84.8471,    +88.195815) --  (axis cs:85.8834,    +88.183721) --  (axis cs:86.9127,    +88.171608) --  (axis cs:87.9349,    +88.159481) --  (axis cs:88.9504,    +88.147342) --  (axis cs:89.9591,    +88.135197) --  (axis cs:90.9612,    +88.123048) --  (axis cs:91.9569,    +88.110901) --  (axis cs:92.9462,    +88.098757) --  (axis cs:93.9294,    +88.086621) --  (axis cs:94.9064,    +88.074497) --  (axis cs:95.8775,    +88.062387) --  (axis cs:96.8428,    +88.050295) --  (axis cs:97.8023,    +88.038225) --  (axis cs:98.7562,    +88.026180) --  (axis cs:99.7046,    +88.014162) --  (axis cs:100.648,    +88.002176) --  (axis cs:101.585,    +87.990224) --  (axis cs:102.518,    +87.978308) --  (axis cs:103.445,    +87.966433) --  (axis cs:104.368,    +87.954601) --  (axis cs:105.286,    +87.942814) --  (axis cs:106.199,    +87.931076) --  (axis cs:107.107,    +87.919390) --  (axis cs:108.011,    +87.907757) --  (axis cs:108.91,    +87.896181) --  (axis cs:109.805,    +87.884664) --  (axis cs:110.696,    +87.873208) --  (axis cs:111.582,    +87.861817) --  (axis cs:112.465,    +87.850492) --  (axis cs:113.343,    +87.839236) --  (axis cs:114.218,    +87.828052) --  (axis cs:115.089,    +87.816941) --  (axis cs:115.955,    +87.805905) --  (axis cs:116.819,    +87.794948) --  (axis cs:117.679,    +87.784071) --  (axis cs:118.535,    +87.773276) --  (axis cs:119.388,    +87.762565) --  (axis cs:120.237,    +87.751941) --  (axis cs:121.083,    +87.741405) --  (axis cs:121.926,    +87.730959) --  (axis cs:122.766,    +87.720606) --  (axis cs:123.603,    +87.710346) --  (axis cs:124.437,    +87.700183) --  (axis cs:125.268,    +87.690117) --  (axis cs:126.096,    +87.680151) --  (axis cs:126.921,    +87.670286) --  (axis cs:127.744,    +87.660525) --  (axis cs:128.564,    +87.650867) --  (axis cs:129.381,    +87.641316) --  (axis cs:130.196,    +87.631873) --  (axis cs:131.008,    +87.622540) --  (axis cs:131.818,    +87.613317) --  (axis cs:132.625,    +87.604207) --  (axis cs:133.43,    +87.595211) --  (axis cs:134.233,    +87.586331) --  (axis cs:135.034,    +87.577567) --  (axis cs:135.832,    +87.568922) ;
  \draw[constellation-boundary]  (axis cs:171.849,    +65.812616) --  (axis cs:172.823,    +65.810828) --  (axis cs:173.796,    +65.809244) --  (axis cs:174.769,    +65.807865) --  (axis cs:175.742,    +65.806691) --  (axis cs:176.715,    +65.805722) --  (axis cs:177.688,    +65.804960) --  (axis cs:178.661,    +65.804403) --  (axis cs:179.633,    +65.804052) ;
  \draw[constellation-boundary]  (axis cs:171.849,    +65.812616) --  (axis cs:171.855,    +66.312611) --  (axis cs:171.868,    +67.312598) --  (axis cs:171.881,    +68.312585) --  (axis cs:171.896,    +69.312571) --  (axis cs:171.913,    +70.312555) --  (axis cs:171.931,    +71.312537) --  (axis cs:171.951,    +72.312518) --  (axis cs:171.961,    +72.812508) ;
  \draw[constellation-boundary]  (axis cs:90.1756,    +25.009566) --  (axis cs:90.1905,    +26.009475) --  (axis cs:90.2057,    +27.009383) --  (axis cs:90.2211,    +28.009289) --  (axis cs:90.2289,    +28.509242) ;
  \draw[constellation-boundary]  (axis cs:118.258,    +33.181229) --  (axis cs:118.266,    +33.681187) --  (axis cs:118.282,    +34.681101) --  (axis cs:118.29,    +35.181058) ;
  \draw[constellation-boundary]  (axis cs:118.258,    +33.181229) --  (axis cs:119.005,    +33.173171) --  (axis cs:120.001,    +33.162515) --  (axis cs:120.997,    +33.151963) --  (axis cs:121.993,    +33.141516) --  (axis cs:122.989,    +33.131179) --  (axis cs:123.985,    +33.120954) --  (axis cs:124.98,    +33.110844) --  (axis cs:125.976,    +33.100854) --  (axis cs:126.971,    +33.090984) --  (axis cs:127.966,    +33.081240) --  (axis cs:128.961,    +33.071623) --  (axis cs:129.956,    +33.062136) --  (axis cs:130.951,    +33.052782) --  (axis cs:131.946,    +33.043565) --  (axis cs:132.94,    +33.034486) --  (axis cs:133.935,    +33.025549) --  (axis cs:134.93,    +33.016756) --  (axis cs:135.924,    +33.008109) --  (axis cs:136.918,    +32.999612) --  (axis cs:137.912,    +32.991267) --  (axis cs:138.906,    +32.983077) --  (axis cs:139.901,    +32.975043) --  (axis cs:140.894,    +32.967169) --  (axis cs:141.888,    +32.959456) --  (axis cs:142.882,    +32.951907) --  (axis cs:143.876,    +32.944525) --  (axis cs:144.869,    +32.937311) --  (axis cs:145.863,    +32.930267) --  (axis cs:146.856,    +32.923397) --  (axis cs:147.85,    +32.916700) --  (axis cs:148.843,    +32.910181) --  (axis cs:149.836,    +32.903840) --  (axis cs:150.084,    +32.902283) ;
  \draw[constellation-boundary]  (axis cs:150.042,    +27.902415) --  (axis cs:150.046,    +28.402402) --  (axis cs:150.055,    +29.402377) --  (axis cs:150.063,    +30.402351) --  (axis cs:150.071,    +31.402324) --  (axis cs:150.08,    +32.402297) --  (axis cs:150.084,    +32.902283) ;
  \draw[constellation-boundary]  (axis cs:150.042,    +27.902415) --  (axis cs:150.788,    +27.897806) --  (axis cs:151.782,    +27.891821) --  (axis cs:152.777,    +27.886020) --  (axis cs:153.771,    +27.880405) --  (axis cs:154.765,    +27.874979) --  (axis cs:155.759,    +27.869741) --  (axis cs:156.753,    +27.864694) --  (axis cs:157.747,    +27.859840) --  (axis cs:158.741,    +27.855180) --  (axis cs:159.238,    +27.852923) ;
  \draw[constellation-boundary]  (axis cs:159.224,    +25.352955) --  (axis cs:159.23,    +26.352942) --  (axis cs:159.236,    +27.352930) --  (axis cs:159.238,    +27.852923) ;
  \draw[constellation-boundary]  (axis cs:166.683,    +25.325050) --  (axis cs:166.686,    +26.325044) --  (axis cs:166.69,    +27.325039) --  (axis cs:166.694,    +28.325033) --  (axis cs:166.698,    +29.325027) --  (axis cs:166.702,    +30.325021) --  (axis cs:166.706,    +31.325015) --  (axis cs:166.71,    +32.325009) --  (axis cs:166.714,    +33.325003) ;
  \draw[constellation-boundary]  (axis cs:166.694,    +28.325033) --  (axis cs:167.688,    +28.322174) --  (axis cs:168.681,    +28.319521) --  (axis cs:169.675,    +28.317073) --  (axis cs:170.668,    +28.314832) --  (axis cs:171.662,    +28.312799) --  (axis cs:172.655,    +28.310973) --  (axis cs:173.648,    +28.309356) --  (axis cs:174.642,    +28.307948) --  (axis cs:175.635,    +28.306750) --  (axis cs:176.629,    +28.305761) --  (axis cs:177.622,    +28.304982) --  (axis cs:178.615,    +28.304414) --  (axis cs:179.609,    +28.304056) ;
  \draw[constellation-boundary]  (axis cs:179.608,    +25.304056) --  (axis cs:179.608,    +26.304056) --  (axis cs:179.609,    +27.304056) --  (axis cs:179.609,    +28.304056) ;
  \draw[constellation-boundary]  (axis cs:140.722,    +39.218824) --  (axis cs:140.97,    +39.216874) --  (axis cs:141.962,    +39.209173) --  (axis cs:142.954,    +39.201637) --  (axis cs:143.946,    +39.194266) --  (axis cs:144.938,    +39.187064) --  (axis cs:145.682,    +39.181774) ;
  \draw[constellation-boundary]  (axis cs:145.682,    +39.181774) --  (axis cs:145.685,    +39.431764) --  (axis cs:145.697,    +40.431722) --  (axis cs:145.709,    +41.431678) ;
  \draw[constellation-boundary]  (axis cs:154.359,    +39.377419) --  (axis cs:154.369,    +40.377394) --  (axis cs:154.378,    +41.377368) ;
  \draw[constellation-boundary]  (axis cs:154.359,    +39.377419) --  (axis cs:154.855,    +39.374738) --  (axis cs:155.846,    +39.369517) --  (axis cs:156.837,    +39.364487) --  (axis cs:157.827,    +39.359649) --  (axis cs:158.818,    +39.355004) --  (axis cs:159.809,    +39.350554) --  (axis cs:160.799,    +39.346300) --  (axis cs:161.79,    +39.342244) --  (axis cs:162.78,    +39.338386) --  (axis cs:163.523,    +39.335623) ;
  \draw[constellation-boundary]  (axis cs:163.489,    +33.335685) --  (axis cs:163.495,    +34.335675) --  (axis cs:163.5,    +35.335665) --  (axis cs:163.506,    +36.335655) --  (axis cs:163.511,    +37.335645) --  (axis cs:163.517,    +38.335634) --  (axis cs:163.523,    +39.335623) ;
  \draw[constellation-boundary]  (axis cs:163.489,    +33.335685) --  (axis cs:163.738,    +33.334787) --  (axis cs:164.73,    +33.331323) --  (axis cs:165.722,    +33.328062) --  (axis cs:166.714,    +33.325003) ;
  \draw[constellation-boundary]  (axis cs:128.44,    +46.577732) --  (axis cs:128.461,    +47.577630) --  (axis cs:128.483,    +48.577525) --  (axis cs:128.505,    +49.577416) --  (axis cs:128.529,    +50.577301) --  (axis cs:128.553,    +51.577182) --  (axis cs:128.579,    +52.577058) --  (axis cs:128.606,    +53.576928) --  (axis cs:128.634,    +54.576791) --  (axis cs:128.664,    +55.576647) --  (axis cs:128.695,    +56.576496) --  (axis cs:128.728,    +57.576337) --  (axis cs:128.762,    +58.576169) --  (axis cs:128.799,    +59.575991) ;
  \draw[constellation-boundary]  (axis cs:128.44,    +46.577732) --  (axis cs:129.184,    +46.570553) --  (axis cs:130.176,    +46.561095) --  (axis cs:131.168,    +46.551772) --  (axis cs:132.159,    +46.542584) --  (axis cs:133.151,    +46.533535) --  (axis cs:134.142,    +46.524627) --  (axis cs:135.133,    +46.515864) --  (axis cs:136.124,    +46.507248) --  (axis cs:137.114,    +46.498781) --  (axis cs:138.105,    +46.490466) --  (axis cs:139.095,    +46.482306) --  (axis cs:139.591,    +46.478284) ;
  \draw[constellation-boundary]  (axis cs:139.512,    +41.478600) --  (axis cs:140.008,    +41.474612) --  (axis cs:141,    +41.466755) --  (axis cs:141.992,    +41.459059) --  (axis cs:142.983,    +41.451527) --  (axis cs:143.975,    +41.444162) --  (axis cs:144.966,    +41.436964) --  (axis cs:145.957,    +41.429937) --  (axis cs:146.948,    +41.423083) --  (axis cs:147.939,    +41.416403) --  (axis cs:148.93,    +41.409899) --  (axis cs:149.921,    +41.403574) --  (axis cs:150.911,    +41.397429) --  (axis cs:151.902,    +41.391467) --  (axis cs:152.893,    +41.385688) --  (axis cs:153.883,    +41.380094) --  (axis cs:154.378,    +41.377368) ;
  \draw[constellation-boundary]  (axis cs:139.512,    +41.478600) --  (axis cs:139.527,    +42.478541) --  (axis cs:139.542,    +43.478480) --  (axis cs:139.558,    +44.478417) --  (axis cs:139.574,    +45.478352) --  (axis cs:139.591,    +46.478284) ;
  \draw[constellation-boundary]  (axis cs:40.4023,    +34.537508) --  (axis cs:40.9055,    +34.533615) --  (axis cs:41.9118,    +34.525708) --  (axis cs:42.6664,    +34.519672) ;
  \draw[constellation-boundary]  (axis cs:40.4023,    +34.537508) --  (axis cs:40.4138,    +35.537464) --  (axis cs:40.4256,    +36.537418) --  (axis cs:40.4346,    +37.287384) ;
  \draw[constellation-boundary]  (axis cs:22.8665,    +28.645431) --  (axis cs:22.8724,    +29.645417) --  (axis cs:22.8785,    +30.645404) --  (axis cs:22.8847,    +31.645390) --  (axis cs:22.891,    +32.645375) --  (axis cs:22.8975,    +33.645360) --  (axis cs:22.9041,    +34.645345) --  (axis cs:22.9109,    +35.645330) ;
  \draw[constellation-boundary]  (axis cs:22.8665,    +28.645431) --  (axis cs:23.7467,    +28.641362) --  (axis cs:24.7528,    +28.636527) --  (axis cs:25.7587,    +28.631498) --  (axis cs:26.7646,    +28.626275) ;
  



% Labels

% Phoenix too small to label



\node[constellation-label] at (axis cs:{85,40})  {\Huge \bfseries\textls{Auriga}};
\node[constellation-label] at (axis cs:{260,60}) {\huge \bfseries\textls{Draco}};

\node[constellation-label] at (axis cs:{240,80}) {\Large \bfseries\textls{Ursa}};
  \node[constellation-label] at (axis cs:{230,75}) {\Large \bfseries\textls{Minor}};
\node[constellation-label] at (axis cs:{115,50}) {\huge \bfseries\textls{Lynx}};

\node[constellation-label] at (axis cs:{163,51}) {\Huge \bfseries\textls{Ursa}};
  \node[constellation-label] at (axis cs:{157,46}) {\Huge \bfseries\textls{Major}};
\node[constellation-label,rotate=25] at (axis cs:{150,37}) {\large\bfseries\textls{Leo Minor}};
\node[constellation-label] at (axis cs:{225,45}) {\huge\bfseries\textls{Bootes}};

\node[constellation-label] at (axis cs:{197,45}) {\Large\bfseries\textls{Canes}};
\node[constellation-label] at (axis cs:{195.5,40}) {\Large\bfseries\textls{Venatici}};

\node[constellation-label,rotate=35] at (axis cs:{145,32}) {\bfseries\textls{Leo}};
\node[constellation-label,rotate=51] at (axis cs:{129,32}) {\bfseries\textls{Cancer}};
\node[constellation-label,rotate=72] at (axis cs:{108,32}) {\bfseries\textls{Gemini}};
\node[constellation-label] at (axis cs:{91,70}) {\Large\bfseries\textls{Camelopardalis}};
\node[constellation-label,rotate=-14] at (axis cs:{194,30.7}) {\bfseries\textls{Coma}};
\node[constellation-label,rotate=-6] at (axis cs:{186,30.7}) {\bfseries\textls{Berenices}};

\node[constellation-label,rotate=0] at (axis cs:{242,35}) {\bfseries\textls{Corona}};
\node[constellation-label,rotate=0] at (axis cs:{237,35}) {\bfseries\textls{Borealis}};

\node[constellation-label] at (axis cs:{68,45}) {\large\bfseries\textls{Perseus}};
\node[constellation-label] at (axis cs:{30,72}) {\large\bfseries\textls{Cassiopeia}};
\node[constellation-label] at (axis cs:{330,71}) {\large\bfseries\textls{Cepheus}};
\node[constellation-label,rotate=-72] at (axis cs:{72,29}) {{Tau}};
\node[constellation-label] at (axis cs:{264,40}) {\LARGE\bfseries\textls{Hercules}};

\node[constellation-label] at (axis cs:{282,37}) {\LARGE\bfseries\textls{Lyra}};
\node[constellation-label] at (axis cs:{294,48.5}) {\Large\bfseries\textls{Cygnus}};

\end{polaraxis}


% Nebulae and nebulae names
\input{./input/II_Nebulae.tex}
% Stars and star names
\begin{polaraxis}[rotate=270,name=stars,at=(base.center),anchor=center,axis lines=none]

\clip (6\tendegree,0\tendegree) arc (0:90:6\tendegree) -- 
(-2\tendegree,6\tendegree) -- (-2\tendegree,-6\tendegree) -- (0\tendegree,-6\tendegree)
--  (0\tendegree,-6\tendegree) arc (270:359.9999:6\tendegree) -- cycle ;

\node[stars] at (axis cs:{0.129},{59.560}) {\tikz\pgfuseplotmark{m6cv};};
\node[stars] at (axis cs:{0.182},{45.253}) {\tikz\pgfuseplotmark{m6cv};};
\node[stars] at (axis cs:{0.330},{49.981}) {\tikz\pgfuseplotmark{m6c};};
\node[stars] at (axis cs:{0.404},{61.223}) {\tikz\pgfuseplotmark{m6av};};
\node[stars] at (axis cs:{0.433},{42.367}) {\tikz\pgfuseplotmark{m6c};};
\node[stars] at (axis cs:{0.542},{27.082}) {\tikz\pgfuseplotmark{m6av};};
\node[stars] at (axis cs:{0.650},{66.099}) {\tikz\pgfuseplotmark{m6ab};};
\node[stars] at (axis cs:{0.857},{63.640}) {\tikz\pgfuseplotmark{m6cv};};
\node[stars] at (axis cs:{0.965},{66.712}) {\tikz\pgfuseplotmark{m6cv};};
\node[stars] at (axis cs:{1.057},{62.288}) {\tikz\pgfuseplotmark{m6bb};};
\node[stars] at (axis cs:{1.152},{42.092}) {\tikz\pgfuseplotmark{m6bb};};
\node[stars] at (axis cs:{1.175},{67.166}) {\tikz\pgfuseplotmark{m6a};};
\node[stars] at (axis cs:{1.224},{34.660}) {\tikz\pgfuseplotmark{m6b};};
\node[stars] at (axis cs:{1.255},{27.675}) {\tikz\pgfuseplotmark{m6c};};
\node[stars] at (axis cs:{1.276},{61.314}) {\tikz\pgfuseplotmark{m6av};};
\node[stars] at (axis cs:{1.566},{58.437}) {\tikz\pgfuseplotmark{m6cvb};};
\node[stars] at (axis cs:{1.611},{64.196}) {\tikz\pgfuseplotmark{m6a};};
\node[stars] at (axis cs:{1.653},{29.021}) {\tikz\pgfuseplotmark{m6bvb};};
\node[stars] at (axis cs:{2.097},{29.090}) {\tikz\pgfuseplotmark{m2bv};};
\node[stars] at (axis cs:{2.137},{63.204}) {\tikz\pgfuseplotmark{m6c};};
\node[stars] at (axis cs:{2.171},{36.627}) {\tikz\pgfuseplotmark{m6c};};
\node[stars] at (axis cs:{2.251},{28.247}) {\tikz\pgfuseplotmark{m6c};};
\node[stars] at (axis cs:{2.295},{59.150}) {\tikz\pgfuseplotmark{m2cv};};
\node[stars] at (axis cs:{2.334},{79.714}) {\tikz\pgfuseplotmark{m6b};};
\node[stars] at (axis cs:{2.580},{46.072}) {\tikz\pgfuseplotmark{m5b};};
\node[stars] at (axis cs:{2.996},{48.152}) {\tikz\pgfuseplotmark{m6c};};
\node[stars] at (axis cs:{3.379},{41.035}) {\tikz\pgfuseplotmark{m6a};};
\node[stars] at (axis cs:{3.510},{33.206}) {\tikz\pgfuseplotmark{m6c};};
\node[stars] at (axis cs:{3.779},{31.536}) {\tikz\pgfuseplotmark{m6c};};
\node[stars] at (axis cs:{3.794},{27.283}) {\tikz\pgfuseplotmark{m6c};};
\node[stars] at (axis cs:{3.979},{61.000}) {\tikz\pgfuseplotmark{m6c};};
\node[stars] at (axis cs:{4.058},{76.951}) {\tikz\pgfuseplotmark{m6c};};
\node[stars] at (axis cs:{4.090},{43.595}) {\tikz\pgfuseplotmark{m6cb};};
\node[stars] at (axis cs:{4.225},{49.462}) {\tikz\pgfuseplotmark{m6cv};};
\node[stars] at (axis cs:{4.238},{61.533}) {\tikz\pgfuseplotmark{m6a};};
\node[stars] at (axis cs:{4.273},{38.681}) {\tikz\pgfuseplotmark{m5av};};
\node[stars] at (axis cs:{4.288},{47.947}) {\tikz\pgfuseplotmark{m6a};};
\node[stars] at (axis cs:{4.429},{51.433}) {\tikz\pgfuseplotmark{m6bv};};
\node[stars] at (axis cs:{4.582},{36.785}) {\tikz\pgfuseplotmark{m5av};};
\node[stars] at (axis cs:{4.659},{31.517}) {\tikz\pgfuseplotmark{m6b};};
\node[stars] at (axis cs:{4.676},{43.791}) {\tikz\pgfuseplotmark{m6bb};};
\node[stars] at (axis cs:{4.923},{40.729}) {\tikz\pgfuseplotmark{m6c};};
\node[stars] at (axis cs:{5.102},{30.936}) {\tikz\pgfuseplotmark{m6b};};
\node[stars] at (axis cs:{5.129},{48.969}) {\tikz\pgfuseplotmark{m6c};};
\node[stars] at (axis cs:{5.190},{32.911}) {\tikz\pgfuseplotmark{m6a};};
\node[stars] at (axis cs:{5.280},{37.969}) {\tikz\pgfuseplotmark{m5c};};
\node[stars] at (axis cs:{6.065},{52.020}) {\tikz\pgfuseplotmark{m6av};};
\node[stars] at (axis cs:{6.198},{61.831}) {\tikz\pgfuseplotmark{m5c};};
\node[stars] at (axis cs:{6.277},{53.047}) {\tikz\pgfuseplotmark{m6av};};
\node[stars] at (axis cs:{7.057},{44.394}) {\tikz\pgfuseplotmark{m5c};};
\node[stars] at (axis cs:{7.236},{36.899}) {\tikz\pgfuseplotmark{m6c};};
\node[stars] at (axis cs:{7.531},{29.752}) {\tikz\pgfuseplotmark{m5cv};};
\node[stars] at (axis cs:{7.583},{59.977}) {\tikz\pgfuseplotmark{m6b};};
\node[stars] at (axis cs:{7.729},{77.019}) {\tikz\pgfuseplotmark{m6c};};
\node[stars] at (axis cs:{7.855},{66.520}) {\tikz\pgfuseplotmark{m6cv};};
\node[stars] at (axis cs:{7.857},{33.581}) {\tikz\pgfuseplotmark{m6bb};};
\node[stars] at (axis cs:{7.921},{52.839}) {\tikz\pgfuseplotmark{m6a};};
\node[stars] at (axis cs:{7.943},{54.522}) {\tikz\pgfuseplotmark{m5a};};
\node[stars] at (axis cs:{8.205},{28.280}) {\tikz\pgfuseplotmark{m6c};};
\node[stars] at (axis cs:{8.250},{62.932}) {\tikz\pgfuseplotmark{m4cv};};
\node[stars] at (axis cs:{8.293},{54.895}) {\tikz\pgfuseplotmark{m6b};};
\node[stars] at (axis cs:{8.330},{70.982}) {\tikz\pgfuseplotmark{m6c};};
\node[stars] at (axis cs:{8.604},{66.750}) {\tikz\pgfuseplotmark{m6c};};
\node[stars] at (axis cs:{9.035},{54.168}) {\tikz\pgfuseplotmark{m5b};};
\node[stars] at (axis cs:{9.114},{60.326}) {\tikz\pgfuseplotmark{m6a};};
\node[stars] at (axis cs:{9.194},{44.489}) {\tikz\pgfuseplotmark{m5cv};};
\node[stars] at (axis cs:{9.220},{33.719}) {\tikz\pgfuseplotmark{m4cvb};};
\node[stars] at (axis cs:{9.243},{53.897}) {\tikz\pgfuseplotmark{m4av};};
\node[stars] at (axis cs:{9.338},{35.399}) {\tikz\pgfuseplotmark{m5c};};
\node[stars] at (axis cs:{9.639},{29.312}) {\tikz\pgfuseplotmark{m4c};};
\node[stars] at (axis cs:{9.791},{49.355}) {\tikz\pgfuseplotmark{m5c};};
\node[stars] at (axis cs:{9.832},{30.861}) {\tikz\pgfuseplotmark{m3c};};
\node[stars] at (axis cs:{9.947},{82.494}) {\tikz\pgfuseplotmark{m6c};};
\node[stars] at (axis cs:{10.127},{56.537}) {\tikz\pgfuseplotmark{m2c};};
\node[stars] at (axis cs:{10.280},{39.459}) {\tikz\pgfuseplotmark{m5c};};
\node[stars] at (axis cs:{10.514},{66.147}) {\tikz\pgfuseplotmark{m6a};};
\node[stars] at (axis cs:{10.516},{50.512}) {\tikz\pgfuseplotmark{m5av};};
\node[stars] at (axis cs:{10.629},{58.753}) {\tikz\pgfuseplotmark{m6c};};
\node[stars] at (axis cs:{10.867},{47.024}) {\tikz\pgfuseplotmark{m5bv};};
\node[stars] at (axis cs:{11.109},{47.864}) {\tikz\pgfuseplotmark{m6a};};
\node[stars] at (axis cs:{11.181},{48.284}) {\tikz\pgfuseplotmark{m4cv};};
\node[stars] at (axis cs:{11.322},{55.221}) {\tikz\pgfuseplotmark{m5cb};};
\node[stars] at (axis cs:{11.413},{74.988}) {\tikz\pgfuseplotmark{m6av};};
\node[stars] at (axis cs:{11.545},{44.861}) {\tikz\pgfuseplotmark{m6b};};
\node[stars] at (axis cs:{11.660},{69.325}) {\tikz\pgfuseplotmark{m6c};};
\node[stars] at (axis cs:{11.677},{59.574}) {\tikz\pgfuseplotmark{m6c};};
\node[stars] at (axis cs:{11.942},{74.847}) {\tikz\pgfuseplotmark{m5c};};
\node[stars] at (axis cs:{12.038},{72.674}) {\tikz\pgfuseplotmark{m6a};};
\node[stars] at (axis cs:{12.208},{50.968}) {\tikz\pgfuseplotmark{m5b};};
\node[stars] at (axis cs:{12.276},{57.815}) {\tikz\pgfuseplotmark{m3cvb};};
\node[stars] at (axis cs:{12.454},{41.079}) {\tikz\pgfuseplotmark{m5av};};
\node[stars] at (axis cs:{12.472},{27.710}) {\tikz\pgfuseplotmark{m6ab};};
\node[stars] at (axis cs:{12.576},{45.002}) {\tikz\pgfuseplotmark{m6bv};};
\node[stars] at (axis cs:{12.682},{64.247}) {\tikz\pgfuseplotmark{m5cv};};
\node[stars] at (axis cs:{12.739},{51.508}) {\tikz\pgfuseplotmark{m6cv};};
\node[stars] at (axis cs:{12.818},{61.805}) {\tikz\pgfuseplotmark{m6bv};};
\node[stars] at (axis cs:{12.891},{51.571}) {\tikz\pgfuseplotmark{m6c};};
\node[stars] at (axis cs:{13.267},{61.124}) {\tikz\pgfuseplotmark{m5a};};
\node[stars] at (axis cs:{13.368},{37.418}) {\tikz\pgfuseplotmark{m6b};};
\node[stars] at (axis cs:{13.448},{52.689}) {\tikz\pgfuseplotmark{m6cb};};
\node[stars] at (axis cs:{13.721},{83.707}) {\tikz\pgfuseplotmark{m6a};};
\node[stars] at (axis cs:{13.751},{58.973}) {\tikz\pgfuseplotmark{m5a};};
\node[stars] at (axis cs:{13.772},{48.678}) {\tikz\pgfuseplotmark{m6cv};};
\node[stars] at (axis cs:{13.994},{27.209}) {\tikz\pgfuseplotmark{m6b};};
\node[stars] at (axis cs:{14.054},{57.996}) {\tikz\pgfuseplotmark{m6c};};
\node[stars] at (axis cs:{14.166},{59.181}) {\tikz\pgfuseplotmark{m5a};};
\node[stars] at (axis cs:{14.177},{60.717}) {\tikz\pgfuseplotmark{m2cvb};};
\node[stars] at (axis cs:{14.188},{38.499}) {\tikz\pgfuseplotmark{m4b};};
\node[stars] at (axis cs:{14.196},{60.363}) {\tikz\pgfuseplotmark{m6a};};
\node[stars] at (axis cs:{14.230},{68.776}) {\tikz\pgfuseplotmark{m6c};};
\node[stars] at (axis cs:{14.332},{61.422}) {\tikz\pgfuseplotmark{m6c};};
\node[stars] at (axis cs:{14.415},{45.839}) {\tikz\pgfuseplotmark{m6b};};
\node[stars] at (axis cs:{14.459},{28.992}) {\tikz\pgfuseplotmark{m5c};};
\node[stars] at (axis cs:{14.559},{33.951}) {\tikz\pgfuseplotmark{m6b};};
\node[stars] at (axis cs:{14.629},{66.352}) {\tikz\pgfuseplotmark{m6b};};
\node[stars] at (axis cs:{15.015},{44.713}) {\tikz\pgfuseplotmark{m6bb};};
\node[stars] at (axis cs:{15.129},{70.983}) {\tikz\pgfuseplotmark{m6c};};
\node[stars] at (axis cs:{15.363},{49.544}) {\tikz\pgfuseplotmark{m6c};};
\node[stars] at (axis cs:{15.577},{51.035}) {\tikz\pgfuseplotmark{m6c};};
\node[stars] at (axis cs:{15.705},{31.804}) {\tikz\pgfuseplotmark{m5c};};
\node[stars] at (axis cs:{15.726},{41.345}) {\tikz\pgfuseplotmark{m6b};};
\node[stars] at (axis cs:{15.904},{61.075}) {\tikz\pgfuseplotmark{m6bb};};
\node[stars] at (axis cs:{16.010},{52.502}) {\tikz\pgfuseplotmark{m6b};};
\node[stars] at (axis cs:{16.081},{61.580}) {\tikz\pgfuseplotmark{m6a};};
\node[stars] at (axis cs:{16.115},{29.659}) {\tikz\pgfuseplotmark{m6c};};
\node[stars] at (axis cs:{16.547},{32.181}) {\tikz\pgfuseplotmark{m6c};};
\node[stars] at (axis cs:{16.751},{56.935}) {\tikz\pgfuseplotmark{m6c};};
\node[stars] at (axis cs:{16.789},{53.498}) {\tikz\pgfuseplotmark{m6c};};
\node[stars] at (axis cs:{17.004},{43.942}) {\tikz\pgfuseplotmark{m5b};};
\node[stars] at (axis cs:{17.006},{32.012}) {\tikz\pgfuseplotmark{m6c};};
\node[stars] at (axis cs:{17.068},{54.920}) {\tikz\pgfuseplotmark{m5cvb};};
\node[stars] at (axis cs:{17.139},{58.263}) {\tikz\pgfuseplotmark{m6a};};
\node[stars] at (axis cs:{17.187},{86.257}) {\tikz\pgfuseplotmark{m4c};};
\node[stars] at (axis cs:{17.301},{80.011}) {\tikz\pgfuseplotmark{m6c};};
\node[stars] at (axis cs:{17.376},{47.242}) {\tikz\pgfuseplotmark{m4cvb};};
\node[stars] at (axis cs:{17.433},{35.620}) {\tikz\pgfuseplotmark{m2bvb};};
\node[stars] at (axis cs:{17.578},{42.081}) {\tikz\pgfuseplotmark{m6a};};
\node[stars] at (axis cs:{17.664},{68.778}) {\tikz\pgfuseplotmark{m5c};};
\node[stars] at (axis cs:{17.776},{55.150}) {\tikz\pgfuseplotmark{m4cv};};
\node[stars] at (axis cs:{17.778},{31.425}) {\tikz\pgfuseplotmark{m5c};};
\node[stars] at (axis cs:{17.793},{37.724}) {\tikz\pgfuseplotmark{m6a};};
\node[stars] at (axis cs:{17.857},{64.203}) {\tikz\pgfuseplotmark{m6av};};
\node[stars] at (axis cs:{17.915},{30.090}) {\tikz\pgfuseplotmark{m5a};};
\node[stars] at (axis cs:{17.923},{65.019}) {\tikz\pgfuseplotmark{m6a};};
\node[stars] at (axis cs:{18.070},{79.674}) {\tikz\pgfuseplotmark{m6a};};
\node[stars] at (axis cs:{18.142},{45.337}) {\tikz\pgfuseplotmark{m6b};};
\node[stars] at (axis cs:{18.248},{30.064}) {\tikz\pgfuseplotmark{m6c};};
\node[stars] at (axis cs:{18.291},{61.706}) {\tikz\pgfuseplotmark{m6cvb};};
\node[stars] at (axis cs:{18.520},{28.529}) {\tikz\pgfuseplotmark{m6cv};};
\node[stars] at (axis cs:{19.050},{71.744}) {\tikz\pgfuseplotmark{m6bv};};
\node[stars] at (axis cs:{19.055},{87.145}) {\tikz\pgfuseplotmark{m6c};};
\node[stars] at (axis cs:{19.079},{33.114}) {\tikz\pgfuseplotmark{m6b};};
\node[stars] at (axis cs:{19.129},{79.910}) {\tikz\pgfuseplotmark{m6c};};
\node[stars] at (axis cs:{19.271},{44.902}) {\tikz\pgfuseplotmark{m6b};};
\node[stars] at (axis cs:{19.350},{31.744}) {\tikz\pgfuseplotmark{m6c};};
\node[stars] at (axis cs:{19.542},{47.420}) {\tikz\pgfuseplotmark{m6c};};
\node[stars] at (axis cs:{19.558},{57.803}) {\tikz\pgfuseplotmark{m6cv};};
\node[stars] at (axis cs:{19.696},{37.386}) {\tikz\pgfuseplotmark{m6cb};};
\node[stars] at (axis cs:{19.867},{27.264}) {\tikz\pgfuseplotmark{m5a};};
\node[stars] at (axis cs:{20.020},{58.231}) {\tikz\pgfuseplotmark{m5bb};};
\node[stars] at (axis cs:{20.081},{77.571}) {\tikz\pgfuseplotmark{m6c};};
\node[stars] at (axis cs:{20.272},{64.658}) {\tikz\pgfuseplotmark{m6cb};};
\node[stars] at (axis cs:{20.281},{28.738}) {\tikz\pgfuseplotmark{m5c};};
\node[stars] at (axis cs:{20.496},{76.239}) {\tikz\pgfuseplotmark{m6c};};
\node[stars] at (axis cs:{20.585},{45.529}) {\tikz\pgfuseplotmark{m5b};};
\node[stars] at (axis cs:{20.839},{58.143}) {\tikz\pgfuseplotmark{m6cb};};
\node[stars] at (axis cs:{20.906},{34.246}) {\tikz\pgfuseplotmark{m6c};};
\node[stars] at (axis cs:{20.919},{37.715}) {\tikz\pgfuseplotmark{m6a};};
\node[stars] at (axis cs:{20.945},{78.726}) {\tikz\pgfuseplotmark{m6bv};};
\node[stars] at (axis cs:{21.443},{70.980}) {\tikz\pgfuseplotmark{m6c};};
\node[stars] at (axis cs:{21.454},{60.235}) {\tikz\pgfuseplotmark{m3av};};
\node[stars] at (axis cs:{21.483},{68.130}) {\tikz\pgfuseplotmark{m5ab};};
\node[stars] at (axis cs:{21.537},{34.579}) {\tikz\pgfuseplotmark{m6c};};
\node[stars] at (axis cs:{21.578},{43.458}) {\tikz\pgfuseplotmark{m6b};};
\node[stars] at (axis cs:{21.776},{34.377}) {\tikz\pgfuseplotmark{m6c};};
\node[stars] at (axis cs:{21.861},{41.101}) {\tikz\pgfuseplotmark{m6c};};
\node[stars] at (axis cs:{21.914},{45.407}) {\tikz\pgfuseplotmark{m5ab};};
\node[stars] at (axis cs:{22.525},{47.007}) {\tikz\pgfuseplotmark{m5c};};
\node[stars] at (axis cs:{22.717},{66.098}) {\tikz\pgfuseplotmark{m6c};};
\node[stars] at (axis cs:{22.807},{70.264}) {\tikz\pgfuseplotmark{m6a};};
\node[stars] at (axis cs:{23.032},{34.800}) {\tikz\pgfuseplotmark{m6c};};
\node[stars] at (axis cs:{23.357},{58.327}) {\tikz\pgfuseplotmark{m6a};};
\node[stars] at (axis cs:{23.461},{89.015}) {\tikz\pgfuseplotmark{m6c};};
\node[stars] at (axis cs:{23.483},{59.232}) {\tikz\pgfuseplotmark{m5a};};
\node[stars] at (axis cs:{23.569},{37.237}) {\tikz\pgfuseplotmark{m6bv};};
\node[stars] at (axis cs:{23.969},{41.076}) {\tikz\pgfuseplotmark{m6c};};
\node[stars] at (axis cs:{24.113},{48.723}) {\tikz\pgfuseplotmark{m6cv};};
\node[stars] at (axis cs:{24.199},{41.405}) {\tikz\pgfuseplotmark{m4b};};
\node[stars] at (axis cs:{24.498},{48.628}) {\tikz\pgfuseplotmark{m4a};};
\node[stars] at (axis cs:{24.532},{57.977}) {\tikz\pgfuseplotmark{m6a};};
\node[stars] at (axis cs:{24.629},{73.040}) {\tikz\pgfuseplotmark{m5cv};};
\node[stars] at (axis cs:{24.633},{45.399}) {\tikz\pgfuseplotmark{m6cv};};
\node[stars] at (axis cs:{24.838},{44.386}) {\tikz\pgfuseplotmark{m5b};};
\node[stars] at (axis cs:{25.055},{53.868}) {\tikz\pgfuseplotmark{m6c};};
\node[stars] at (axis cs:{25.145},{40.577}) {\tikz\pgfuseplotmark{m5bv};};
\node[stars] at (axis cs:{25.165},{43.297}) {\tikz\pgfuseplotmark{m6a};};
\node[stars] at (axis cs:{25.413},{30.047}) {\tikz\pgfuseplotmark{m6b};};
\node[stars] at (axis cs:{25.446},{42.613}) {\tikz\pgfuseplotmark{m5b};};
\node[stars] at (axis cs:{25.515},{35.246}) {\tikz\pgfuseplotmark{m6a};};
\node[stars] at (axis cs:{25.574},{58.628}) {\tikz\pgfuseplotmark{m6cb};};
\node[stars] at (axis cs:{25.586},{68.043}) {\tikz\pgfuseplotmark{m6av};};
\node[stars] at (axis cs:{25.733},{70.622}) {\tikz\pgfuseplotmark{m5cv};};
\node[stars] at (axis cs:{25.743},{61.421}) {\tikz\pgfuseplotmark{m6c};};
\node[stars] at (axis cs:{25.819},{45.323}) {\tikz\pgfuseplotmark{m6c};};
\node[stars] at (axis cs:{25.832},{60.551}) {\tikz\pgfuseplotmark{m6ab};};
\node[stars] at (axis cs:{25.915},{50.689}) {\tikz\pgfuseplotmark{m4bv};};
\node[stars] at (axis cs:{25.959},{32.192}) {\tikz\pgfuseplotmark{m6c};};
\node[stars] at (axis cs:{26.075},{57.537}) {\tikz\pgfuseplotmark{m6cvb};};
\node[stars] at (axis cs:{26.111},{46.140}) {\tikz\pgfuseplotmark{m6c};};
\node[stars] at (axis cs:{26.192},{57.089}) {\tikz\pgfuseplotmark{m6c};};
\node[stars] at (axis cs:{26.937},{63.852}) {\tikz\pgfuseplotmark{m6a};};
\node[stars] at (axis cs:{26.950},{46.230}) {\tikz\pgfuseplotmark{m6c};};
\node[stars] at (axis cs:{27.162},{37.953}) {\tikz\pgfuseplotmark{m6b};};
\node[stars] at (axis cs:{27.173},{32.690}) {\tikz\pgfuseplotmark{m6a};};
\node[stars] at (axis cs:{27.315},{47.897}) {\tikz\pgfuseplotmark{m6cb};};
\node[stars] at (axis cs:{27.738},{51.933}) {\tikz\pgfuseplotmark{m6b};};
\node[stars] at (axis cs:{27.997},{55.147}) {\tikz\pgfuseplotmark{m6av};};
\node[stars] at (axis cs:{28.039},{50.793}) {\tikz\pgfuseplotmark{m6av};};
\node[stars] at (axis cs:{28.211},{51.475}) {\tikz\pgfuseplotmark{m6c};};
\node[stars] at (axis cs:{28.270},{29.579}) {\tikz\pgfuseplotmark{m3cvb};};
\node[stars] at (axis cs:{28.322},{40.730}) {\tikz\pgfuseplotmark{m5c};};
\node[stars] at (axis cs:{28.452},{55.597}) {\tikz\pgfuseplotmark{m6c};};
\node[stars] at (axis cs:{28.599},{63.670}) {\tikz\pgfuseplotmark{m3cv};};
\node[stars] at (axis cs:{28.724},{40.702}) {\tikz\pgfuseplotmark{m6cv};};
\node[stars] at (axis cs:{28.740},{37.128}) {\tikz\pgfuseplotmark{m6c};};
\node[stars] at (axis cs:{29.000},{68.685}) {\tikz\pgfuseplotmark{m5bv};};
\node[stars] at (axis cs:{29.039},{37.252}) {\tikz\pgfuseplotmark{m6ab};};
\node[stars] at (axis cs:{29.432},{27.804}) {\tikz\pgfuseplotmark{m6a};};
\node[stars] at (axis cs:{29.497},{47.095}) {\tikz\pgfuseplotmark{m6c};};
\node[stars] at (axis cs:{29.638},{61.698}) {\tikz\pgfuseplotmark{m6bv};};
\node[stars] at (axis cs:{29.640},{49.204}) {\tikz\pgfuseplotmark{m6a};};
\node[stars] at (axis cs:{29.908},{64.622}) {\tikz\pgfuseplotmark{m5c};};
\node[stars] at (axis cs:{30.489},{70.907}) {\tikz\pgfuseplotmark{m5a};};
\node[stars] at (axis cs:{30.575},{54.487}) {\tikz\pgfuseplotmark{m5b};};
\node[stars] at (axis cs:{30.719},{64.901}) {\tikz\pgfuseplotmark{m6b};};
\node[stars] at (axis cs:{30.738},{77.916}) {\tikz\pgfuseplotmark{m6b};};
\node[stars] at (axis cs:{30.741},{33.284}) {\tikz\pgfuseplotmark{m6a};};
\node[stars] at (axis cs:{30.751},{64.390}) {\tikz\pgfuseplotmark{m6a};};
\node[stars] at (axis cs:{30.794},{73.851}) {\tikz\pgfuseplotmark{m6cb};};
\node[stars] at (axis cs:{30.859},{72.421}) {\tikz\pgfuseplotmark{m4b};};
\node[stars] at (axis cs:{30.975},{42.330}) {\tikz\pgfuseplotmark{m2cb};};
\node[stars] at (axis cs:{31.281},{77.281}) {\tikz\pgfuseplotmark{m5c};};
\node[stars] at (axis cs:{31.381},{76.115}) {\tikz\pgfuseplotmark{m5c};};
\node[stars] at (axis cs:{32.122},{37.859}) {\tikz\pgfuseplotmark{m5a};};
\node[stars] at (axis cs:{32.140},{44.459}) {\tikz\pgfuseplotmark{m6c};};
\node[stars] at (axis cs:{32.169},{58.423}) {\tikz\pgfuseplotmark{m6av};};
\node[stars] at (axis cs:{32.355},{81.296}) {\tikz\pgfuseplotmark{m6b};};
\node[stars] at (axis cs:{32.386},{34.987}) {\tikz\pgfuseplotmark{m3b};};
\node[stars] at (axis cs:{32.532},{53.843}) {\tikz\pgfuseplotmark{m6c};};
\node[stars] at (axis cs:{32.720},{39.039}) {\tikz\pgfuseplotmark{m6bb};};
\node[stars] at (axis cs:{32.854},{31.526}) {\tikz\pgfuseplotmark{m6c};};
\node[stars] at (axis cs:{32.872},{57.645}) {\tikz\pgfuseplotmark{m6c};};
\node[stars] at (axis cs:{33.093},{30.303}) {\tikz\pgfuseplotmark{m5cvb};};
\node[stars] at (axis cs:{33.208},{79.691}) {\tikz\pgfuseplotmark{m6cvb};};
\node[stars] at (axis cs:{33.306},{44.231}) {\tikz\pgfuseplotmark{m5a};};
\node[stars] at (axis cs:{33.338},{74.028}) {\tikz\pgfuseplotmark{m6c};};
\node[stars] at (axis cs:{33.401},{51.066}) {\tikz\pgfuseplotmark{m5cv};};
\node[stars] at (axis cs:{33.423},{58.560}) {\tikz\pgfuseplotmark{m6c};};
\node[stars] at (axis cs:{33.510},{47.484}) {\tikz\pgfuseplotmark{m6b};};
\node[stars] at (axis cs:{33.621},{66.524}) {\tikz\pgfuseplotmark{m6b};};
\node[stars] at (axis cs:{33.658},{28.691}) {\tikz\pgfuseplotmark{m6c};};
\node[stars] at (axis cs:{33.985},{33.359}) {\tikz\pgfuseplotmark{m5c};};
\node[stars] at (axis cs:{33.991},{47.812}) {\tikz\pgfuseplotmark{m6c};};
\node[stars] at (axis cs:{34.190},{83.561}) {\tikz\pgfuseplotmark{m6c};};
\node[stars] at (axis cs:{34.263},{34.224}) {\tikz\pgfuseplotmark{m5b};};
\node[stars] at (axis cs:{34.329},{33.847}) {\tikz\pgfuseplotmark{m4b};};
\node[stars] at (axis cs:{34.500},{57.900}) {\tikz\pgfuseplotmark{m6a};};
\node[stars] at (axis cs:{34.519},{57.516}) {\tikz\pgfuseplotmark{m6bb};};
\node[stars] at (axis cs:{34.737},{28.643}) {\tikz\pgfuseplotmark{m5c};};
\node[stars] at (axis cs:{34.795},{46.472}) {\tikz\pgfuseplotmark{m6c};};
\node[stars] at (axis cs:{34.820},{47.380}) {\tikz\pgfuseplotmark{m5c};};
\node[stars] at (axis cs:{34.844},{48.955}) {\tikz\pgfuseplotmark{m6c};};
\node[stars] at (axis cs:{35.173},{47.311}) {\tikz\pgfuseplotmark{m6c};};
\node[stars] at (axis cs:{35.243},{50.151}) {\tikz\pgfuseplotmark{m6av};};
\node[stars] at (axis cs:{35.481},{57.243}) {\tikz\pgfuseplotmark{m6c};};
\node[stars] at (axis cs:{35.589},{55.845}) {\tikz\pgfuseplotmark{m5cv};};
\node[stars] at (axis cs:{35.710},{41.396}) {\tikz\pgfuseplotmark{m6ab};};
\node[stars] at (axis cs:{35.966},{55.365}) {\tikz\pgfuseplotmark{m6cv};};
\node[stars] at (axis cs:{36.104},{50.006}) {\tikz\pgfuseplotmark{m5c};};
\node[stars] at (axis cs:{36.317},{56.610}) {\tikz\pgfuseplotmark{m6cv};};
\node[stars] at (axis cs:{36.406},{50.278}) {\tikz\pgfuseplotmark{m5a};};
\node[stars] at (axis cs:{36.780},{27.012}) {\tikz\pgfuseplotmark{m6b};};
\node[stars] at (axis cs:{36.866},{31.801}) {\tikz\pgfuseplotmark{m6a};};
\node[stars] at (axis cs:{36.966},{50.570}) {\tikz\pgfuseplotmark{m6cv};};
\node[stars] at (axis cs:{37.042},{29.669}) {\tikz\pgfuseplotmark{m5c};};
\node[stars] at (axis cs:{37.202},{29.932}) {\tikz\pgfuseplotmark{m6b};};
\node[stars] at (axis cs:{37.266},{67.402}) {\tikz\pgfuseplotmark{m5avb};};
\node[stars] at (axis cs:{37.569},{33.834}) {\tikz\pgfuseplotmark{m6c};};
\node[stars] at (axis cs:{37.955},{89.264}) {\tikz\pgfuseplotmark{m2bv};};
\node[stars] at (axis cs:{38.026},{36.147}) {\tikz\pgfuseplotmark{m5c};};
\node[stars] at (axis cs:{38.219},{34.542}) {\tikz\pgfuseplotmark{m6a};};
\node[stars] at (axis cs:{38.866},{39.664}) {\tikz\pgfuseplotmark{m6c};};
\node[stars] at (axis cs:{38.911},{37.312}) {\tikz\pgfuseplotmark{m6a};};
\node[stars] at (axis cs:{38.945},{34.687}) {\tikz\pgfuseplotmark{m5cvb};};
\node[stars] at (axis cs:{39.179},{31.608}) {\tikz\pgfuseplotmark{m6b};};
\node[stars] at (axis cs:{39.238},{38.734}) {\tikz\pgfuseplotmark{m6b};};
\node[stars] at (axis cs:{39.277},{32.891}) {\tikz\pgfuseplotmark{m6c};};
\node[stars] at (axis cs:{39.400},{65.745}) {\tikz\pgfuseplotmark{m6a};};
\node[stars] at (axis cs:{39.508},{72.818}) {\tikz\pgfuseplotmark{m5c};};
\node[stars] at (axis cs:{39.574},{37.727}) {\tikz\pgfuseplotmark{m6c};};
\node[stars] at (axis cs:{39.616},{38.089}) {\tikz\pgfuseplotmark{m6c};};
\node[stars] at (axis cs:{40.171},{27.061}) {\tikz\pgfuseplotmark{m5c};};
\node[stars] at (axis cs:{40.562},{40.194}) {\tikz\pgfuseplotmark{m5bv};};
\node[stars] at (axis cs:{40.748},{53.526}) {\tikz\pgfuseplotmark{m6a};};
\node[stars] at (axis cs:{40.758},{48.265}) {\tikz\pgfuseplotmark{m6c};};
\node[stars] at (axis cs:{40.762},{55.106}) {\tikz\pgfuseplotmark{m6a};};
\node[stars] at (axis cs:{40.863},{27.707}) {\tikz\pgfuseplotmark{m5a};};
\node[stars] at (axis cs:{41.021},{44.297}) {\tikz\pgfuseplotmark{m5c};};
\node[stars] at (axis cs:{41.050},{49.228}) {\tikz\pgfuseplotmark{m4bvb};};
\node[stars] at (axis cs:{41.207},{67.824}) {\tikz\pgfuseplotmark{m6b};};
\node[stars] at (axis cs:{41.743},{35.984}) {\tikz\pgfuseplotmark{m6c};};
\node[stars] at (axis cs:{41.765},{35.555}) {\tikz\pgfuseplotmark{m6c};};
\node[stars] at (axis cs:{41.880},{44.272}) {\tikz\pgfuseplotmark{m6c};};
\node[stars] at (axis cs:{41.949},{81.448}) {\tikz\pgfuseplotmark{m6a};};
\node[stars] at (axis cs:{41.977},{29.247}) {\tikz\pgfuseplotmark{m5a};};
\node[stars] at (axis cs:{42.231},{69.634}) {\tikz\pgfuseplotmark{m6cv};};
\node[stars] at (axis cs:{42.362},{37.326}) {\tikz\pgfuseplotmark{m6c};};
\node[stars] at (axis cs:{42.378},{57.084}) {\tikz\pgfuseplotmark{m6cv};};
\node[stars] at (axis cs:{42.496},{27.260}) {\tikz\pgfuseplotmark{m4avb};};
\node[stars] at (axis cs:{42.646},{38.319}) {\tikz\pgfuseplotmark{m4cv};};
\node[stars] at (axis cs:{42.647},{36.948}) {\tikz\pgfuseplotmark{m6c};};
\node[stars] at (axis cs:{42.674},{55.895}) {\tikz\pgfuseplotmark{m4ab};};
\node[stars] at (axis cs:{42.878},{35.060}) {\tikz\pgfuseplotmark{m5av};};
\node[stars] at (axis cs:{42.924},{46.842}) {\tikz\pgfuseplotmark{m6b};};
\node[stars] at (axis cs:{42.941},{58.314}) {\tikz\pgfuseplotmark{m6cb};};
\node[stars] at (axis cs:{42.995},{68.888}) {\tikz\pgfuseplotmark{m6bv};};
\node[stars] at (axis cs:{43.217},{52.997}) {\tikz\pgfuseplotmark{m6cb};};
\node[stars] at (axis cs:{43.338},{48.570}) {\tikz\pgfuseplotmark{m6cv};};
\node[stars] at (axis cs:{43.428},{38.337}) {\tikz\pgfuseplotmark{m5cv};};
\node[stars] at (axis cs:{43.564},{52.762}) {\tikz\pgfuseplotmark{m4bvb};};
\node[stars] at (axis cs:{43.987},{61.521}) {\tikz\pgfuseplotmark{m6a};};
\node[stars] at (axis cs:{44.103},{64.332}) {\tikz\pgfuseplotmark{m6c};};
\node[stars] at (axis cs:{44.139},{47.164}) {\tikz\pgfuseplotmark{m6b};};
\node[stars] at (axis cs:{44.211},{51.261}) {\tikz\pgfuseplotmark{m6c};};
\node[stars] at (axis cs:{44.322},{31.934}) {\tikz\pgfuseplotmark{m5bv};};
\node[stars] at (axis cs:{44.510},{38.615}) {\tikz\pgfuseplotmark{m6b};};
\node[stars] at (axis cs:{44.690},{39.663}) {\tikz\pgfuseplotmark{m5a};};
\node[stars] at (axis cs:{44.765},{35.183}) {\tikz\pgfuseplotmark{m5b};};
\node[stars] at (axis cs:{44.916},{41.033}) {\tikz\pgfuseplotmark{m6bv};};
\node[stars] at (axis cs:{44.957},{47.221}) {\tikz\pgfuseplotmark{m5c};};
\node[stars] at (axis cs:{45.050},{38.131}) {\tikz\pgfuseplotmark{m6b};};
\node[stars] at (axis cs:{45.218},{52.352}) {\tikz\pgfuseplotmark{m5cb};};
\node[stars] at (axis cs:{45.876},{28.270}) {\tikz\pgfuseplotmark{m6c};};
\node[stars] at (axis cs:{45.986},{47.848}) {\tikz\pgfuseplotmark{m6cv};};
\node[stars] at (axis cs:{46.199},{53.506}) {\tikz\pgfuseplotmark{m3bv};};
\node[stars] at (axis cs:{46.294},{38.840}) {\tikz\pgfuseplotmark{m3cv};};
\node[stars] at (axis cs:{46.338},{40.582}) {\tikz\pgfuseplotmark{m6b};};
\node[stars] at (axis cs:{46.385},{56.706}) {\tikz\pgfuseplotmark{m5a};};
\node[stars] at (axis cs:{46.417},{56.069}) {\tikz\pgfuseplotmark{m6b};};
\node[stars] at (axis cs:{46.533},{79.419}) {\tikz\pgfuseplotmark{m6ab};};
\node[stars] at (axis cs:{46.829},{64.057}) {\tikz\pgfuseplotmark{m6b};};
\node[stars] at (axis cs:{46.947},{47.309}) {\tikz\pgfuseplotmark{m6cv};};
\node[stars] at (axis cs:{47.016},{52.213}) {\tikz\pgfuseplotmark{m6c};};
\node[stars] at (axis cs:{47.042},{40.956}) {\tikz\pgfuseplotmark{m2bvb};};
\node[stars] at (axis cs:{47.267},{49.613}) {\tikz\pgfuseplotmark{m4b};};
\node[stars] at (axis cs:{47.374},{44.857}) {\tikz\pgfuseplotmark{m4av};};
\node[stars] at (axis cs:{47.403},{29.077}) {\tikz\pgfuseplotmark{m6a};};
\node[stars] at (axis cs:{47.537},{27.820}) {\tikz\pgfuseplotmark{m6c};};
\node[stars] at (axis cs:{47.822},{39.611}) {\tikz\pgfuseplotmark{m5a};};
\node[stars] at (axis cs:{47.928},{81.471}) {\tikz\pgfuseplotmark{m6bv};};
\node[stars] at (axis cs:{47.984},{74.393}) {\tikz\pgfuseplotmark{m5a};};
\node[stars] at (axis cs:{48.040},{42.376}) {\tikz\pgfuseplotmark{m6c};};
\node[stars] at (axis cs:{48.059},{27.257}) {\tikz\pgfuseplotmark{m6av};};
\node[stars] at (axis cs:{48.110},{47.726}) {\tikz\pgfuseplotmark{m6cb};};
\node[stars] at (axis cs:{48.349},{48.177}) {\tikz\pgfuseplotmark{m6b};};
\node[stars] at (axis cs:{48.736},{42.504}) {\tikz\pgfuseplotmark{m6b};};
\node[stars] at (axis cs:{48.835},{30.557}) {\tikz\pgfuseplotmark{m6a};};
\node[stars] at (axis cs:{48.946},{32.857}) {\tikz\pgfuseplotmark{m6c};};
\node[stars] at (axis cs:{48.950},{57.141}) {\tikz\pgfuseplotmark{m6a};};
\node[stars] at (axis cs:{49.008},{34.688}) {\tikz\pgfuseplotmark{m6cv};};
\node[stars] at (axis cs:{49.019},{45.346}) {\tikz\pgfuseplotmark{m6c};};
\node[stars] at (axis cs:{49.051},{50.938}) {\tikz\pgfuseplotmark{m5b};};
\node[stars] at (axis cs:{49.147},{32.184}) {\tikz\pgfuseplotmark{m6bv};};
\node[stars] at (axis cs:{49.298},{40.483}) {\tikz\pgfuseplotmark{m6cb};};
\node[stars] at (axis cs:{49.381},{65.658}) {\tikz\pgfuseplotmark{m6c};};
\node[stars] at (axis cs:{49.417},{31.127}) {\tikz\pgfuseplotmark{m6c};};
\node[stars] at (axis cs:{49.441},{39.283}) {\tikz\pgfuseplotmark{m6b};};
\node[stars] at (axis cs:{49.447},{44.025}) {\tikz\pgfuseplotmark{m5c};};
\node[stars] at (axis cs:{49.657},{50.222}) {\tikz\pgfuseplotmark{m5c};};
\node[stars] at (axis cs:{49.683},{34.223}) {\tikz\pgfuseplotmark{m5a};};
\node[stars] at (axis cs:{49.782},{50.095}) {\tikz\pgfuseplotmark{m5b};};
\node[stars] at (axis cs:{49.982},{27.071}) {\tikz\pgfuseplotmark{m6b};};
\node[stars] at (axis cs:{49.997},{65.652}) {\tikz\pgfuseplotmark{m5av};};
\node[stars] at (axis cs:{50.082},{77.735}) {\tikz\pgfuseplotmark{m5cv};};
\node[stars] at (axis cs:{50.085},{29.048}) {\tikz\pgfuseplotmark{m4c};};
\node[stars] at (axis cs:{50.361},{43.329}) {\tikz\pgfuseplotmark{m5b};};
\node[stars] at (axis cs:{50.469},{49.071}) {\tikz\pgfuseplotmark{m6bb};};
\node[stars] at (axis cs:{50.550},{27.607}) {\tikz\pgfuseplotmark{m6a};};
\node[stars] at (axis cs:{50.805},{49.213}) {\tikz\pgfuseplotmark{m5cv};};
\node[stars] at (axis cs:{51.081},{49.861}) {\tikz\pgfuseplotmark{m2av};};
\node[stars] at (axis cs:{51.124},{33.536}) {\tikz\pgfuseplotmark{m6avb};};
\node[stars] at (axis cs:{51.169},{64.586}) {\tikz\pgfuseplotmark{m5cv};};
\node[stars] at (axis cs:{51.489},{49.121}) {\tikz\pgfuseplotmark{m6bv};};
\node[stars] at (axis cs:{51.647},{28.715}) {\tikz\pgfuseplotmark{m6cv};};
\node[stars] at (axis cs:{52.013},{49.063}) {\tikz\pgfuseplotmark{m5b};};
\node[stars] at (axis cs:{52.086},{33.807}) {\tikz\pgfuseplotmark{m6a};};
\node[stars] at (axis cs:{52.218},{49.848}) {\tikz\pgfuseplotmark{m6a};};
\node[stars] at (axis cs:{52.267},{59.940}) {\tikz\pgfuseplotmark{m4cvb};};
\node[stars] at (axis cs:{52.342},{49.509}) {\tikz\pgfuseplotmark{m5a};};
\node[stars] at (axis cs:{52.360},{46.938}) {\tikz\pgfuseplotmark{m6cb};};
\node[stars] at (axis cs:{52.478},{58.879}) {\tikz\pgfuseplotmark{m5av};};
\node[stars] at (axis cs:{52.501},{55.452}) {\tikz\pgfuseplotmark{m5bv};};
\node[stars] at (axis cs:{52.546},{59.366}) {\tikz\pgfuseplotmark{m6cb};};
\node[stars] at (axis cs:{52.581},{71.864}) {\tikz\pgfuseplotmark{m6cv};};
\node[stars] at (axis cs:{52.644},{47.995}) {\tikz\pgfuseplotmark{m4cv};};
\node[stars] at (axis cs:{52.654},{48.104}) {\tikz\pgfuseplotmark{m6a};};
\node[stars] at (axis cs:{52.872},{49.210}) {\tikz\pgfuseplotmark{m6cv};};
\node[stars] at (axis cs:{52.954},{49.401}) {\tikz\pgfuseplotmark{m6c};};
\node[stars] at (axis cs:{53.036},{48.023}) {\tikz\pgfuseplotmark{m5cv};};
\node[stars] at (axis cs:{53.084},{84.911}) {\tikz\pgfuseplotmark{m6a};};
\node[stars] at (axis cs:{53.109},{46.057}) {\tikz\pgfuseplotmark{m5cv};};
\node[stars] at (axis cs:{53.162},{44.856}) {\tikz\pgfuseplotmark{m6cv};};
\node[stars] at (axis cs:{53.167},{35.462}) {\tikz\pgfuseplotmark{m6b};};
\node[stars] at (axis cs:{53.384},{58.765}) {\tikz\pgfuseplotmark{m6cb};};
\node[stars] at (axis cs:{53.396},{39.899}) {\tikz\pgfuseplotmark{m6av};};
\node[stars] at (axis cs:{53.413},{54.975}) {\tikz\pgfuseplotmark{m6b};};
\node[stars] at (axis cs:{53.422},{57.869}) {\tikz\pgfuseplotmark{m6c};};
\node[stars] at (axis cs:{54.122},{48.192}) {\tikz\pgfuseplotmark{m4cv};};
\node[stars] at (axis cs:{54.501},{42.583}) {\tikz\pgfuseplotmark{m6c};};
\node[stars] at (axis cs:{54.582},{56.932}) {\tikz\pgfuseplotmark{m6c};};
\node[stars] at (axis cs:{54.855},{75.740}) {\tikz\pgfuseplotmark{m6c};};
\node[stars] at (axis cs:{55.283},{37.580}) {\tikz\pgfuseplotmark{m6av};};
\node[stars] at (axis cs:{55.539},{63.217}) {\tikz\pgfuseplotmark{m5bv};};
\node[stars] at (axis cs:{55.594},{33.965}) {\tikz\pgfuseplotmark{m5b};};
\node[stars] at (axis cs:{55.678},{59.969}) {\tikz\pgfuseplotmark{m6ab};};
\node[stars] at (axis cs:{55.731},{47.787}) {\tikz\pgfuseplotmark{m3bv};};
\node[stars] at (axis cs:{56.027},{48.524}) {\tikz\pgfuseplotmark{m6b};};
\node[stars] at (axis cs:{56.080},{32.288}) {\tikz\pgfuseplotmark{m4avb};};
\node[stars] at (axis cs:{56.131},{36.460}) {\tikz\pgfuseplotmark{m6a};};
\node[stars] at (axis cs:{56.170},{46.100}) {\tikz\pgfuseplotmark{m6b};};
\node[stars] at (axis cs:{56.298},{42.578}) {\tikz\pgfuseplotmark{m4ab};};
\node[stars] at (axis cs:{56.405},{38.676}) {\tikz\pgfuseplotmark{m6c};};
\node[stars] at (axis cs:{56.497},{45.682}) {\tikz\pgfuseplotmark{m6a};};
\node[stars] at (axis cs:{56.504},{67.202}) {\tikz\pgfuseplotmark{m6av};};
\node[stars] at (axis cs:{56.510},{63.345}) {\tikz\pgfuseplotmark{m5a};};
\node[stars] at (axis cs:{56.884},{55.922}) {\tikz\pgfuseplotmark{m6b};};
\node[stars] at (axis cs:{56.954},{32.195}) {\tikz\pgfuseplotmark{m6c};};
\node[stars] at (axis cs:{57.075},{50.737}) {\tikz\pgfuseplotmark{m6c};};
\node[stars] at (axis cs:{57.284},{43.963}) {\tikz\pgfuseplotmark{m6bv};};
\node[stars] at (axis cs:{57.307},{70.871}) {\tikz\pgfuseplotmark{m5c};};
\node[stars] at (axis cs:{57.332},{57.118}) {\tikz\pgfuseplotmark{m6cb};};
\node[stars] at (axis cs:{57.380},{65.526}) {\tikz\pgfuseplotmark{m4cv};};
\node[stars] at (axis cs:{57.386},{33.091}) {\tikz\pgfuseplotmark{m5cv};};
\node[stars] at (axis cs:{57.402},{63.297}) {\tikz\pgfuseplotmark{m6a};};
\node[stars] at (axis cs:{57.518},{44.968}) {\tikz\pgfuseplotmark{m6a};};
\node[stars] at (axis cs:{57.590},{71.332}) {\tikz\pgfuseplotmark{m5a};};
\node[stars] at (axis cs:{57.924},{68.507}) {\tikz\pgfuseplotmark{m6c};};
\node[stars] at (axis cs:{57.974},{34.359}) {\tikz\pgfuseplotmark{m6a};};
\node[stars] at (axis cs:{58.019},{31.169}) {\tikz\pgfuseplotmark{m6c};};
\node[stars] at (axis cs:{58.411},{48.650}) {\tikz\pgfuseplotmark{m6a};};
\node[stars] at (axis cs:{58.430},{57.975}) {\tikz\pgfuseplotmark{m6a};};
\node[stars] at (axis cs:{58.533},{31.884}) {\tikz\pgfuseplotmark{m3bvb};};
\node[stars] at (axis cs:{58.992},{47.871}) {\tikz\pgfuseplotmark{m5c};};
\node[stars] at (axis cs:{59.120},{35.081}) {\tikz\pgfuseplotmark{m5cv};};
\node[stars] at (axis cs:{59.126},{71.822}) {\tikz\pgfuseplotmark{m6c};};
\node[stars] at (axis cs:{59.152},{50.695}) {\tikz\pgfuseplotmark{m5c};};
\node[stars] at (axis cs:{59.285},{61.109}) {\tikz\pgfuseplotmark{m5bb};};
\node[stars] at (axis cs:{59.356},{63.072}) {\tikz\pgfuseplotmark{m5b};};
\node[stars] at (axis cs:{59.463},{40.010}) {\tikz\pgfuseplotmark{m3bv};};
\node[stars] at (axis cs:{59.622},{38.840}) {\tikz\pgfuseplotmark{m6c};};
\node[stars] at (axis cs:{59.741},{35.791}) {\tikz\pgfuseplotmark{m4bv};};
\node[stars] at (axis cs:{60.312},{36.989}) {\tikz\pgfuseplotmark{m6c};};
\node[stars] at (axis cs:{61.113},{59.155}) {\tikz\pgfuseplotmark{m5b};};
\node[stars] at (axis cs:{61.513},{68.680}) {\tikz\pgfuseplotmark{m6b};};
\node[stars] at (axis cs:{61.646},{50.351}) {\tikz\pgfuseplotmark{m4c};};
\node[stars] at (axis cs:{61.652},{27.600}) {\tikz\pgfuseplotmark{m5cv};};
\node[stars] at (axis cs:{61.653},{54.009}) {\tikz\pgfuseplotmark{m6c};};
\node[stars] at (axis cs:{61.662},{65.521}) {\tikz\pgfuseplotmark{m6c};};
\node[stars] at (axis cs:{61.752},{29.001}) {\tikz\pgfuseplotmark{m5c};};
\node[stars] at (axis cs:{61.795},{74.000}) {\tikz\pgfuseplotmark{m6c};};
\node[stars] at (axis cs:{62.064},{37.727}) {\tikz\pgfuseplotmark{m6b};};
\node[stars] at (axis cs:{62.153},{38.040}) {\tikz\pgfuseplotmark{m6av};};
\node[stars] at (axis cs:{62.165},{47.712}) {\tikz\pgfuseplotmark{m4bv};};
\node[stars] at (axis cs:{62.343},{54.829}) {\tikz\pgfuseplotmark{m6c};};
\node[stars] at (axis cs:{62.365},{59.908}) {\tikz\pgfuseplotmark{m6c};};
\node[stars] at (axis cs:{62.507},{86.626}) {\tikz\pgfuseplotmark{m6a};};
\node[stars] at (axis cs:{62.511},{80.698}) {\tikz\pgfuseplotmark{m5b};};
\node[stars] at (axis cs:{62.746},{33.587}) {\tikz\pgfuseplotmark{m6a};};
\node[stars] at (axis cs:{63.216},{68.501}) {\tikz\pgfuseplotmark{m6c};};
\node[stars] at (axis cs:{63.437},{72.126}) {\tikz\pgfuseplotmark{m6b};};
\node[stars] at (axis cs:{63.499},{37.967}) {\tikz\pgfuseplotmark{m6c};};
\node[stars] at (axis cs:{63.717},{37.543}) {\tikz\pgfuseplotmark{m6c};};
\node[stars] at (axis cs:{63.722},{40.484}) {\tikz\pgfuseplotmark{m5a};};
\node[stars] at (axis cs:{63.724},{48.409}) {\tikz\pgfuseplotmark{m4cv};};
\node[stars] at (axis cs:{63.757},{57.460}) {\tikz\pgfuseplotmark{m6bv};};
\node[stars] at (axis cs:{64.180},{53.612}) {\tikz\pgfuseplotmark{m5c};};
\node[stars] at (axis cs:{64.223},{61.850}) {\tikz\pgfuseplotmark{m6a};};
\node[stars] at (axis cs:{64.284},{57.860}) {\tikz\pgfuseplotmark{m6a};};
\node[stars] at (axis cs:{64.466},{80.536}) {\tikz\pgfuseplotmark{m6c};};
\node[stars] at (axis cs:{64.534},{42.141}) {\tikz\pgfuseplotmark{m6cv};};
\node[stars] at (axis cs:{64.561},{50.295}) {\tikz\pgfuseplotmark{m5av};};
\node[stars] at (axis cs:{64.805},{50.048}) {\tikz\pgfuseplotmark{m5c};};
\node[stars] at (axis cs:{65.041},{31.953}) {\tikz\pgfuseplotmark{m6c};};
\node[stars] at (axis cs:{65.048},{50.921}) {\tikz\pgfuseplotmark{m6av};};
\node[stars] at (axis cs:{65.060},{41.808}) {\tikz\pgfuseplotmark{m6b};};
\node[stars] at (axis cs:{65.088},{27.351}) {\tikz\pgfuseplotmark{m5bb};};
\node[stars] at (axis cs:{65.103},{34.567}) {\tikz\pgfuseplotmark{m5b};};
\node[stars] at (axis cs:{65.168},{65.140}) {\tikz\pgfuseplotmark{m5c};};
\node[stars] at (axis cs:{65.388},{46.499}) {\tikz\pgfuseplotmark{m5av};};
\node[stars] at (axis cs:{65.449},{60.736}) {\tikz\pgfuseplotmark{m5cv};};
\node[stars] at (axis cs:{65.466},{56.506}) {\tikz\pgfuseplotmark{m6b};};
\node[stars] at (axis cs:{65.741},{59.616}) {\tikz\pgfuseplotmark{m6cb};};
\node[stars] at (axis cs:{65.899},{42.428}) {\tikz\pgfuseplotmark{m6c};};
\node[stars] at (axis cs:{66.121},{34.131}) {\tikz\pgfuseplotmark{m6a};};
\node[stars] at (axis cs:{66.156},{33.960}) {\tikz\pgfuseplotmark{m6ab};};
\node[stars] at (axis cs:{66.526},{31.439}) {\tikz\pgfuseplotmark{m5c};};
\node[stars] at (axis cs:{66.754},{57.585}) {\tikz\pgfuseplotmark{m6c};};
\node[stars] at (axis cs:{66.762},{80.824}) {\tikz\pgfuseplotmark{m5c};};
\node[stars] at (axis cs:{67.055},{83.808}) {\tikz\pgfuseplotmark{m6av};};
\node[stars] at (axis cs:{67.217},{30.361}) {\tikz\pgfuseplotmark{m6cb};};
\node[stars] at (axis cs:{67.502},{83.340}) {\tikz\pgfuseplotmark{m5c};};
\node[stars] at (axis cs:{67.660},{32.458}) {\tikz\pgfuseplotmark{m6c};};
\node[stars] at (axis cs:{67.987},{36.743}) {\tikz\pgfuseplotmark{m6cv};};
\node[stars] at (axis cs:{68.008},{53.911}) {\tikz\pgfuseplotmark{m6avb};};
\node[stars] at (axis cs:{68.354},{43.064}) {\tikz\pgfuseplotmark{m6bb};};
\node[stars] at (axis cs:{68.378},{72.528}) {\tikz\pgfuseplotmark{m6b};};
\node[stars] at (axis cs:{68.658},{28.961}) {\tikz\pgfuseplotmark{m6b};};
\node[stars] at (axis cs:{69.101},{64.261}) {\tikz\pgfuseplotmark{m6b};};
\node[stars] at (axis cs:{69.173},{41.265}) {\tikz\pgfuseplotmark{m4c};};
\node[stars] at (axis cs:{69.978},{53.079}) {\tikz\pgfuseplotmark{m5bv};};
\node[stars] at (axis cs:{69.992},{53.473}) {\tikz\pgfuseplotmark{m5c};};
\node[stars] at (axis cs:{70.332},{28.615}) {\tikz\pgfuseplotmark{m6a};};
\node[stars] at (axis cs:{70.351},{48.301}) {\tikz\pgfuseplotmark{m6a};};
\node[stars] at (axis cs:{70.459},{38.280}) {\tikz\pgfuseplotmark{m6b};};
\node[stars] at (axis cs:{70.726},{43.365}) {\tikz\pgfuseplotmark{m5c};};
\node[stars] at (axis cs:{70.840},{49.974}) {\tikz\pgfuseplotmark{m6b};};
\node[stars] at (axis cs:{71.054},{40.786}) {\tikz\pgfuseplotmark{m6b};};
\node[stars] at (axis cs:{71.502},{76.611}) {\tikz\pgfuseplotmark{m6c};};
\node[stars] at (axis cs:{71.685},{40.312}) {\tikz\pgfuseplotmark{m6b};};
\node[stars] at (axis cs:{72.001},{56.757}) {\tikz\pgfuseplotmark{m5c};};
\node[stars] at (axis cs:{72.029},{55.603}) {\tikz\pgfuseplotmark{m6c};};
\node[stars] at (axis cs:{72.210},{75.941}) {\tikz\pgfuseplotmark{m6b};};
\node[stars] at (axis cs:{72.304},{31.437}) {\tikz\pgfuseplotmark{m6a};};
\node[stars] at (axis cs:{72.329},{32.588}) {\tikz\pgfuseplotmark{m6a};};
\node[stars] at (axis cs:{72.478},{37.488}) {\tikz\pgfuseplotmark{m5b};};
\node[stars] at (axis cs:{72.652},{70.941}) {\tikz\pgfuseplotmark{m6c};};
\node[stars] at (axis cs:{72.789},{48.741}) {\tikz\pgfuseplotmark{m6a};};
\node[stars] at (axis cs:{73.022},{63.505}) {\tikz\pgfuseplotmark{m5c};};
\node[stars] at (axis cs:{73.158},{36.703}) {\tikz\pgfuseplotmark{m5a};};
\node[stars] at (axis cs:{73.196},{27.897}) {\tikz\pgfuseplotmark{m6b};};
\node[stars] at (axis cs:{73.199},{42.587}) {\tikz\pgfuseplotmark{m6a};};
\node[stars] at (axis cs:{73.291},{52.841}) {\tikz\pgfuseplotmark{m6c};};
\node[stars] at (axis cs:{73.513},{66.343}) {\tikz\pgfuseplotmark{m4c};};
\node[stars] at (axis cs:{73.714},{44.061}) {\tikz\pgfuseplotmark{m6b};};
\node[stars] at (axis cs:{73.763},{55.259}) {\tikz\pgfuseplotmark{m6a};};
\node[stars] at (axis cs:{74.029},{52.870}) {\tikz\pgfuseplotmark{m6a};};
\node[stars] at (axis cs:{74.084},{36.169}) {\tikz\pgfuseplotmark{m6b};};
\node[stars] at (axis cs:{74.248},{33.166}) {\tikz\pgfuseplotmark{m3av};};
\node[stars] at (axis cs:{74.322},{53.752}) {\tikz\pgfuseplotmark{m4cv};};
\node[stars] at (axis cs:{74.814},{37.890}) {\tikz\pgfuseplotmark{m5bb};};
\node[stars] at (axis cs:{74.943},{53.155}) {\tikz\pgfuseplotmark{m6b};};
\node[stars] at (axis cs:{75.076},{39.395}) {\tikz\pgfuseplotmark{m6bb};};
\node[stars] at (axis cs:{75.086},{81.194}) {\tikz\pgfuseplotmark{m5b};};
\node[stars] at (axis cs:{75.097},{39.655}) {\tikz\pgfuseplotmark{m6c};};
\node[stars] at (axis cs:{75.400},{61.078}) {\tikz\pgfuseplotmark{m6b};};
\node[stars] at (axis cs:{75.492},{43.823}) {\tikz\pgfuseplotmark{m3bv};};
\node[stars] at (axis cs:{75.583},{74.269}) {\tikz\pgfuseplotmark{m6bv};};
\node[stars] at (axis cs:{75.620},{41.076}) {\tikz\pgfuseplotmark{m4av};};
\node[stars] at (axis cs:{75.710},{66.823}) {\tikz\pgfuseplotmark{m6c};};
\node[stars] at (axis cs:{75.828},{41.442}) {\tikz\pgfuseplotmark{m6bv};};
\node[stars] at (axis cs:{75.855},{60.442}) {\tikz\pgfuseplotmark{m4bvb};};
\node[stars] at (axis cs:{76.053},{73.764}) {\tikz\pgfuseplotmark{m6cv};};
\node[stars] at (axis cs:{76.061},{30.494}) {\tikz\pgfuseplotmark{m6c};};
\node[stars] at (axis cs:{76.165},{74.067}) {\tikz\pgfuseplotmark{m6b};};
\node[stars] at (axis cs:{76.535},{58.972}) {\tikz\pgfuseplotmark{m5cvb};};
\node[stars] at (axis cs:{76.624},{61.170}) {\tikz\pgfuseplotmark{m6b};};
\node[stars] at (axis cs:{76.629},{41.234}) {\tikz\pgfuseplotmark{m3cv};};
\node[stars] at (axis cs:{76.669},{51.597}) {\tikz\pgfuseplotmark{m5bvb};};
\node[stars] at (axis cs:{76.706},{43.175}) {\tikz\pgfuseplotmark{m6c};};
\node[stars] at (axis cs:{77.403},{69.639}) {\tikz\pgfuseplotmark{m6c};};
\node[stars] at (axis cs:{77.435},{64.919}) {\tikz\pgfuseplotmark{m6c};};
\node[stars] at (axis cs:{77.438},{28.030}) {\tikz\pgfuseplotmark{m6bvb};};
\node[stars] at (axis cs:{77.578},{37.302}) {\tikz\pgfuseplotmark{m6cb};};
\node[stars] at (axis cs:{77.679},{46.962}) {\tikz\pgfuseplotmark{m6a};};
\node[stars] at (axis cs:{77.910},{29.904}) {\tikz\pgfuseplotmark{m6c};};
\node[stars] at (axis cs:{78.094},{73.946}) {\tikz\pgfuseplotmark{m5cv};};
\node[stars] at (axis cs:{78.262},{61.850}) {\tikz\pgfuseplotmark{m6c};};
\node[stars] at (axis cs:{78.323},{37.337}) {\tikz\pgfuseplotmark{m6cb};};
\node[stars] at (axis cs:{78.357},{38.484}) {\tikz\pgfuseplotmark{m5a};};
\node[stars] at (axis cs:{78.380},{62.691}) {\tikz\pgfuseplotmark{m6c};};
\node[stars] at (axis cs:{78.648},{76.473}) {\tikz\pgfuseplotmark{m6cb};};
\node[stars] at (axis cs:{78.684},{53.214}) {\tikz\pgfuseplotmark{m6c};};
\node[stars] at (axis cs:{78.798},{59.406}) {\tikz\pgfuseplotmark{m6c};};
\node[stars] at (axis cs:{78.852},{32.688}) {\tikz\pgfuseplotmark{m5bvb};};
\node[stars] at (axis cs:{79.076},{34.312}) {\tikz\pgfuseplotmark{m6bvb};};
\node[stars] at (axis cs:{79.172},{45.998}) {\tikz\pgfuseplotmark{m1bvb};};
\node[stars] at (axis cs:{79.544},{33.372}) {\tikz\pgfuseplotmark{m5av};};
\node[stars] at (axis cs:{79.555},{73.268}) {\tikz\pgfuseplotmark{m6a};};
\node[stars] at (axis cs:{79.565},{42.792}) {\tikz\pgfuseplotmark{m6av};};
\node[stars] at (axis cs:{79.579},{33.767}) {\tikz\pgfuseplotmark{m6cv};};
\node[stars] at (axis cs:{79.668},{40.465}) {\tikz\pgfuseplotmark{m6c};};
\node[stars] at (axis cs:{79.750},{33.748}) {\tikz\pgfuseplotmark{m5cv};};
\node[stars] at (axis cs:{79.785},{40.099}) {\tikz\pgfuseplotmark{m5ab};};
\node[stars] at (axis cs:{79.849},{33.985}) {\tikz\pgfuseplotmark{m6c};};
\node[stars] at (axis cs:{79.866},{58.117}) {\tikz\pgfuseplotmark{m6bv};};
\node[stars] at (axis cs:{80.004},{33.958}) {\tikz\pgfuseplotmark{m5b};};
\node[stars] at (axis cs:{80.010},{44.425}) {\tikz\pgfuseplotmark{m6c};};
\node[stars] at (axis cs:{80.061},{41.086}) {\tikz\pgfuseplotmark{m5c};};
\node[stars] at (axis cs:{80.094},{62.654}) {\tikz\pgfuseplotmark{m6a};};
\node[stars] at (axis cs:{80.169},{66.747}) {\tikz\pgfuseplotmark{m6c};};
\node[stars] at (axis cs:{80.247},{27.957}) {\tikz\pgfuseplotmark{m6c};};
\node[stars] at (axis cs:{80.303},{29.570}) {\tikz\pgfuseplotmark{m6a};};
\node[stars] at (axis cs:{80.452},{41.804}) {\tikz\pgfuseplotmark{m5c};};
\node[stars] at (axis cs:{80.640},{79.231}) {\tikz\pgfuseplotmark{m5b};};
\node[stars] at (axis cs:{80.710},{41.029}) {\tikz\pgfuseplotmark{m6a};};
\node[stars] at (axis cs:{80.845},{28.937}) {\tikz\pgfuseplotmark{m6c};};
\node[stars] at (axis cs:{80.866},{57.544}) {\tikz\pgfuseplotmark{m5c};};
\node[stars] at (axis cs:{81.160},{31.224}) {\tikz\pgfuseplotmark{m6c};};
\node[stars] at (axis cs:{81.160},{31.143}) {\tikz\pgfuseplotmark{m6b};};
\node[stars] at (axis cs:{81.163},{37.385}) {\tikz\pgfuseplotmark{m5bb};};
\node[stars] at (axis cs:{81.187},{36.200}) {\tikz\pgfuseplotmark{m6c};};
\node[stars] at (axis cs:{81.248},{73.707}) {\tikz\pgfuseplotmark{m6c};};
\node[stars] at (axis cs:{81.573},{28.607}) {\tikz\pgfuseplotmark{m2ab};};
\node[stars] at (axis cs:{81.703},{34.392}) {\tikz\pgfuseplotmark{m6b};};
\node[stars] at (axis cs:{81.714},{33.262}) {\tikz\pgfuseplotmark{m6c};};
\node[stars] at (axis cs:{81.726},{35.457}) {\tikz\pgfuseplotmark{m6c};};
\node[stars] at (axis cs:{81.784},{30.208}) {\tikz\pgfuseplotmark{m6a};};
\node[stars] at (axis cs:{81.912},{34.476}) {\tikz\pgfuseplotmark{m5bb};};
\node[stars] at (axis cs:{82.004},{33.763}) {\tikz\pgfuseplotmark{m6cb};};
\node[stars] at (axis cs:{82.419},{29.186}) {\tikz\pgfuseplotmark{m6c};};
\node[stars] at (axis cs:{82.543},{63.067}) {\tikz\pgfuseplotmark{m5cv};};
\node[stars] at (axis cs:{82.688},{39.826}) {\tikz\pgfuseplotmark{m6cb};};
\node[stars] at (axis cs:{82.703},{41.462}) {\tikz\pgfuseplotmark{m6b};};
\node[stars] at (axis cs:{83.141},{57.221}) {\tikz\pgfuseplotmark{m6c};};
\node[stars] at (axis cs:{83.182},{32.192}) {\tikz\pgfuseplotmark{m5a};};
\node[stars] at (axis cs:{83.364},{32.801}) {\tikz\pgfuseplotmark{m6c};};
\node[stars] at (axis cs:{83.409},{34.725}) {\tikz\pgfuseplotmark{m6c};};
\node[stars] at (axis cs:{83.981},{27.662}) {\tikz\pgfuseplotmark{m6c};};
\node[stars] at (axis cs:{84.066},{47.715}) {\tikz\pgfuseplotmark{m6b};};
\node[stars] at (axis cs:{84.147},{54.428}) {\tikz\pgfuseplotmark{m6a};};
\node[stars] at (axis cs:{84.218},{40.182}) {\tikz\pgfuseplotmark{m6b};};
\node[stars] at (axis cs:{84.313},{64.155}) {\tikz\pgfuseplotmark{m6cb};};
\node[stars] at (axis cs:{84.317},{66.696}) {\tikz\pgfuseplotmark{m6c};};
\node[stars] at (axis cs:{84.441},{33.559}) {\tikz\pgfuseplotmark{m6c};};
\node[stars] at (axis cs:{84.659},{30.492}) {\tikz\pgfuseplotmark{m5cvb};};
\node[stars] at (axis cs:{84.786},{41.358}) {\tikz\pgfuseplotmark{m6c};};
\node[stars] at (axis cs:{84.826},{29.215}) {\tikz\pgfuseplotmark{m6bv};};
\node[stars] at (axis cs:{84.932},{75.044}) {\tikz\pgfuseplotmark{m6c};};
\node[stars] at (axis cs:{85.150},{31.358}) {\tikz\pgfuseplotmark{m6bv};};
\node[stars] at (axis cs:{85.175},{31.921}) {\tikz\pgfuseplotmark{m6cv};};
\node[stars] at (axis cs:{85.335},{53.481}) {\tikz\pgfuseplotmark{m6cb};};
\node[stars] at (axis cs:{85.337},{29.487}) {\tikz\pgfuseplotmark{m6cb};};
\node[stars] at (axis cs:{85.610},{65.697}) {\tikz\pgfuseplotmark{m6a};};
\node[stars] at (axis cs:{85.757},{56.581}) {\tikz\pgfuseplotmark{m6b};};
\node[stars] at (axis cs:{86.035},{61.476}) {\tikz\pgfuseplotmark{m6c};};
\node[stars] at (axis cs:{86.475},{49.826}) {\tikz\pgfuseplotmark{m5cv};};
\node[stars] at (axis cs:{86.627},{56.115}) {\tikz\pgfuseplotmark{m6bv};};
\node[stars] at (axis cs:{86.750},{51.521}) {\tikz\pgfuseplotmark{m6c};};
\node[stars] at (axis cs:{86.811},{42.527}) {\tikz\pgfuseplotmark{m6c};};
\node[stars] at (axis cs:{87.269},{62.808}) {\tikz\pgfuseplotmark{m6c};};
\node[stars] at (axis cs:{87.293},{39.181}) {\tikz\pgfuseplotmark{m5a};};
\node[stars] at (axis cs:{87.735},{51.514}) {\tikz\pgfuseplotmark{m6c};};
\node[stars] at (axis cs:{87.742},{27.968}) {\tikz\pgfuseplotmark{m6a};};
\node[stars] at (axis cs:{87.760},{37.305}) {\tikz\pgfuseplotmark{m5av};};
\node[stars] at (axis cs:{87.857},{32.125}) {\tikz\pgfuseplotmark{m6cv};};
\node[stars] at (axis cs:{87.872},{39.148}) {\tikz\pgfuseplotmark{m4b};};
\node[stars] at (axis cs:{88.072},{58.964}) {\tikz\pgfuseplotmark{m6c};};
\node[stars] at (axis cs:{88.165},{39.575}) {\tikz\pgfuseplotmark{m6c};};
\node[stars] at (axis cs:{88.167},{33.917}) {\tikz\pgfuseplotmark{m6bv};};
\node[stars] at (axis cs:{88.231},{68.471}) {\tikz\pgfuseplotmark{m6c};};
\node[stars] at (axis cs:{88.332},{27.612}) {\tikz\pgfuseplotmark{m5av};};
\node[stars] at (axis cs:{88.712},{55.707}) {\tikz\pgfuseplotmark{m5b};};
\node[stars] at (axis cs:{88.741},{59.888}) {\tikz\pgfuseplotmark{m5cv};};
\node[stars] at (axis cs:{88.746},{31.701}) {\tikz\pgfuseplotmark{m6b};};
\node[stars] at (axis cs:{89.141},{28.942}) {\tikz\pgfuseplotmark{m6c};};
\node[stars] at (axis cs:{89.396},{66.096}) {\tikz\pgfuseplotmark{m6c};};
\node[stars] at (axis cs:{89.841},{49.924}) {\tikz\pgfuseplotmark{m6b};};
\node[stars] at (axis cs:{89.882},{54.285}) {\tikz\pgfuseplotmark{m4ab};};
\node[stars] at (axis cs:{89.882},{44.947}) {\tikz\pgfuseplotmark{m2bv};};
\node[stars] at (axis cs:{89.930},{37.212}) {\tikz\pgfuseplotmark{m3av};};
\node[stars] at (axis cs:{89.941},{55.321}) {\tikz\pgfuseplotmark{m6c};};
\node[stars] at (axis cs:{89.950},{54.547}) {\tikz\pgfuseplotmark{m6b};};
\node[stars] at (axis cs:{89.984},{45.937}) {\tikz\pgfuseplotmark{m4cv};};
\node[stars] at (axis cs:{90.079},{44.592}) {\tikz\pgfuseplotmark{m6cb};};
\node[stars] at (axis cs:{90.244},{47.902}) {\tikz\pgfuseplotmark{m6av};};
\node[stars] at (axis cs:{90.252},{27.572}) {\tikz\pgfuseplotmark{m6b};};
\node[stars] at (axis cs:{90.292},{31.034}) {\tikz\pgfuseplotmark{m6c};};
\node[stars] at (axis cs:{90.334},{85.182}) {\tikz\pgfuseplotmark{m6b};};
\node[stars] at (axis cs:{90.429},{48.959}) {\tikz\pgfuseplotmark{m6b};};
\node[stars] at (axis cs:{90.703},{49.906}) {\tikz\pgfuseplotmark{m6b};};
\node[stars] at (axis cs:{90.723},{43.378}) {\tikz\pgfuseplotmark{m6c};};
\node[stars] at (axis cs:{90.730},{32.636}) {\tikz\pgfuseplotmark{m6c};};
\node[stars] at (axis cs:{90.768},{57.018}) {\tikz\pgfuseplotmark{m6c};};
\node[stars] at (axis cs:{90.825},{42.911}) {\tikz\pgfuseplotmark{m6b};};
\node[stars] at (axis cs:{91.121},{51.573}) {\tikz\pgfuseplotmark{m6cv};};
\node[stars] at (axis cs:{91.261},{37.964}) {\tikz\pgfuseplotmark{m6c};};
\node[stars] at (axis cs:{91.264},{42.981}) {\tikz\pgfuseplotmark{m6b};};
\node[stars] at (axis cs:{91.284},{59.393}) {\tikz\pgfuseplotmark{m6c};};
\node[stars] at (axis cs:{91.289},{75.586}) {\tikz\pgfuseplotmark{m6c};};
\node[stars] at (axis cs:{91.391},{33.599}) {\tikz\pgfuseplotmark{m6c};};
\node[stars] at (axis cs:{91.536},{35.388}) {\tikz\pgfuseplotmark{m6bb};};
\node[stars] at (axis cs:{91.594},{29.512}) {\tikz\pgfuseplotmark{m6bv};};
\node[stars] at (axis cs:{91.646},{38.482}) {\tikz\pgfuseplotmark{m5cv};};
\node[stars] at (axis cs:{91.663},{63.454}) {\tikz\pgfuseplotmark{m6c};};
\node[stars] at (axis cs:{91.861},{41.854}) {\tikz\pgfuseplotmark{m6c};};
\node[stars] at (axis cs:{92.096},{41.056}) {\tikz\pgfuseplotmark{m6c};};
\node[stars] at (axis cs:{92.496},{58.936}) {\tikz\pgfuseplotmark{m5c};};
\node[stars] at (axis cs:{92.902},{48.711}) {\tikz\pgfuseplotmark{m6cb};};
\node[stars] at (axis cs:{92.941},{52.647}) {\tikz\pgfuseplotmark{m6c};};
\node[stars] at (axis cs:{93.084},{32.693}) {\tikz\pgfuseplotmark{m6a};};
\node[stars] at (axis cs:{93.213},{65.718}) {\tikz\pgfuseplotmark{m5c};};
\node[stars] at (axis cs:{93.439},{51.172}) {\tikz\pgfuseplotmark{m6b};};
\node[stars] at (axis cs:{93.845},{29.498}) {\tikz\pgfuseplotmark{m4c};};
\node[stars] at (axis cs:{93.919},{59.999}) {\tikz\pgfuseplotmark{m5c};};
\node[stars] at (axis cs:{94.478},{61.515}) {\tikz\pgfuseplotmark{m5bv};};
\node[stars] at (axis cs:{94.570},{46.360}) {\tikz\pgfuseplotmark{m6c};};
\node[stars] at (axis cs:{94.712},{69.320}) {\tikz\pgfuseplotmark{m5a};};
\node[stars] at (axis cs:{94.906},{59.011}) {\tikz\pgfuseplotmark{m4cv};};
\node[stars] at (axis cs:{95.300},{29.541}) {\tikz\pgfuseplotmark{m6c};};
\node[stars] at (axis cs:{95.442},{53.452}) {\tikz\pgfuseplotmark{m5c};};
\node[stars] at (axis cs:{95.515},{59.372}) {\tikz\pgfuseplotmark{m6b};};
\node[stars] at (axis cs:{95.777},{56.976}) {\tikz\pgfuseplotmark{m6c};};
\node[stars] at (axis cs:{96.225},{49.288}) {\tikz\pgfuseplotmark{m5bv};};
\node[stars] at (axis cs:{96.608},{56.285}) {\tikz\pgfuseplotmark{m6av};};
\node[stars] at (axis cs:{96.704},{58.417}) {\tikz\pgfuseplotmark{m5cb};};
\node[stars] at (axis cs:{96.898},{32.563}) {\tikz\pgfuseplotmark{m6c};};
\node[stars] at (axis cs:{97.060},{70.535}) {\tikz\pgfuseplotmark{m6b};};
\node[stars] at (axis cs:{97.142},{30.493}) {\tikz\pgfuseplotmark{m6av};};
\node[stars] at (axis cs:{97.512},{46.686}) {\tikz\pgfuseplotmark{m6b};};
\node[stars] at (axis cs:{97.696},{58.162}) {\tikz\pgfuseplotmark{m6ab};};
\node[stars] at (axis cs:{98.113},{32.455}) {\tikz\pgfuseplotmark{m6av};};
\node[stars] at (axis cs:{98.637},{55.353}) {\tikz\pgfuseplotmark{m6cv};};
\node[stars] at (axis cs:{98.800},{28.022}) {\tikz\pgfuseplotmark{m5cv};};
\node[stars] at (axis cs:{99.137},{38.445}) {\tikz\pgfuseplotmark{m5cv};};
\node[stars] at (axis cs:{99.410},{56.857}) {\tikz\pgfuseplotmark{m6b};};
\node[stars] at (axis cs:{99.422},{61.481}) {\tikz\pgfuseplotmark{m6bb};};
\node[stars] at (axis cs:{99.479},{73.695}) {\tikz\pgfuseplotmark{m6c};};
\node[stars] at (axis cs:{99.596},{28.984}) {\tikz\pgfuseplotmark{m6a};};
\node[stars] at (axis cs:{99.665},{39.391}) {\tikz\pgfuseplotmark{m6a};};
\node[stars] at (axis cs:{99.705},{39.903}) {\tikz\pgfuseplotmark{m5c};};
\node[stars] at (axis cs:{99.833},{42.489}) {\tikz\pgfuseplotmark{m5a};};
\node[stars] at (axis cs:{99.888},{28.263}) {\tikz\pgfuseplotmark{m6bv};};
\node[stars] at (axis cs:{99.991},{44.014}) {\tikz\pgfuseplotmark{m6c};};
\node[stars] at (axis cs:{100.120},{77.996}) {\tikz\pgfuseplotmark{m6a};};
\node[stars] at (axis cs:{100.134},{71.749}) {\tikz\pgfuseplotmark{m6a};};
\node[stars] at (axis cs:{100.337},{28.196}) {\tikz\pgfuseplotmark{m6c};};
\node[stars] at (axis cs:{100.407},{35.932}) {\tikz\pgfuseplotmark{m6c};};
\node[stars] at (axis cs:{100.771},{44.524}) {\tikz\pgfuseplotmark{m5b};};
\node[stars] at (axis cs:{100.807},{37.147}) {\tikz\pgfuseplotmark{m6c};};
\node[stars] at (axis cs:{101.048},{53.296}) {\tikz\pgfuseplotmark{m6c};};
\node[stars] at (axis cs:{101.052},{36.109}) {\tikz\pgfuseplotmark{m6c};};
\node[stars] at (axis cs:{101.189},{28.971}) {\tikz\pgfuseplotmark{m5c};};
\node[stars] at (axis cs:{101.559},{59.442}) {\tikz\pgfuseplotmark{m5bb};};
\node[stars] at (axis cs:{101.559},{79.565}) {\tikz\pgfuseplotmark{m5c};};
\node[stars] at (axis cs:{101.685},{43.577}) {\tikz\pgfuseplotmark{m5cb};};
\node[stars] at (axis cs:{101.706},{57.169}) {\tikz\pgfuseplotmark{m5c};};
\node[stars] at (axis cs:{101.915},{48.789}) {\tikz\pgfuseplotmark{m5c};};
\node[stars] at (axis cs:{102.053},{55.704}) {\tikz\pgfuseplotmark{m6cb};};
\node[stars] at (axis cs:{102.422},{32.607}) {\tikz\pgfuseplotmark{m6av};};
\node[stars] at (axis cs:{102.691},{41.781}) {\tikz\pgfuseplotmark{m5b};};
\node[stars] at (axis cs:{102.738},{67.572}) {\tikz\pgfuseplotmark{m5c};};
\node[stars] at (axis cs:{103.197},{33.961}) {\tikz\pgfuseplotmark{m4a};};
\node[stars] at (axis cs:{103.256},{38.869}) {\tikz\pgfuseplotmark{m6bvb};};
\node[stars] at (axis cs:{103.265},{35.788}) {\tikz\pgfuseplotmark{m6b};};
\node[stars] at (axis cs:{103.271},{59.448}) {\tikz\pgfuseplotmark{m5c};};
\node[stars] at (axis cs:{103.282},{44.840}) {\tikz\pgfuseplotmark{m6c};};
\node[stars] at (axis cs:{103.306},{38.438}) {\tikz\pgfuseplotmark{m6c};};
\node[stars] at (axis cs:{103.426},{68.888}) {\tikz\pgfuseplotmark{m5b};};
\node[stars] at (axis cs:{103.488},{38.505}) {\tikz\pgfuseplotmark{m6c};};
\node[stars] at (axis cs:{103.811},{43.910}) {\tikz\pgfuseplotmark{m6cv};};
\node[stars] at (axis cs:{103.813},{45.826}) {\tikz\pgfuseplotmark{m6c};};
\node[stars] at (axis cs:{104.134},{46.274}) {\tikz\pgfuseplotmark{m6av};};
\node[stars] at (axis cs:{104.234},{46.705}) {\tikz\pgfuseplotmark{m6b};};
\node[stars] at (axis cs:{104.252},{33.681}) {\tikz\pgfuseplotmark{m6b};};
\node[stars] at (axis cs:{104.305},{57.563}) {\tikz\pgfuseplotmark{m6bv};};
\node[stars] at (axis cs:{104.319},{58.423}) {\tikz\pgfuseplotmark{m4c};};
\node[stars] at (axis cs:{104.405},{45.094}) {\tikz\pgfuseplotmark{m5bv};};
\node[stars] at (axis cs:{104.762},{38.052}) {\tikz\pgfuseplotmark{m6b};};
\node[stars] at (axis cs:{105.017},{76.977}) {\tikz\pgfuseplotmark{m5a};};
\node[stars] at (axis cs:{105.339},{70.808}) {\tikz\pgfuseplotmark{m6a};};
\node[stars] at (axis cs:{105.877},{29.337}) {\tikz\pgfuseplotmark{m6b};};
\node[stars] at (axis cs:{106.287},{47.775}) {\tikz\pgfuseplotmark{m6c};};
\node[stars] at (axis cs:{106.465},{70.732}) {\tikz\pgfuseplotmark{m6cv};};
\node[stars] at (axis cs:{106.505},{59.802}) {\tikz\pgfuseplotmark{m6c};};
\node[stars] at (axis cs:{106.548},{34.474}) {\tikz\pgfuseplotmark{m6a};};
\node[stars] at (axis cs:{106.843},{34.009}) {\tikz\pgfuseplotmark{m6b};};
\node[stars] at (axis cs:{106.854},{28.177}) {\tikz\pgfuseplotmark{m6c};};
\node[stars] at (axis cs:{107.055},{33.832}) {\tikz\pgfuseplotmark{m6c};};
\node[stars] at (axis cs:{107.151},{37.445}) {\tikz\pgfuseplotmark{m6c};};
\node[stars] at (axis cs:{107.785},{30.245}) {\tikz\pgfuseplotmark{m4cv};};
\node[stars] at (axis cs:{107.914},{39.320}) {\tikz\pgfuseplotmark{m5b};};
\node[stars] at (axis cs:{108.348},{51.429}) {\tikz\pgfuseplotmark{m5cv};};
\node[stars] at (axis cs:{108.492},{71.816}) {\tikz\pgfuseplotmark{m6c};};
\node[stars] at (axis cs:{108.959},{47.240}) {\tikz\pgfuseplotmark{m6a};};
\node[stars] at (axis cs:{108.979},{59.637}) {\tikz\pgfuseplotmark{m5c};};
\node[stars] at (axis cs:{108.988},{27.897}) {\tikz\pgfuseplotmark{m6av};};
\node[stars] at (axis cs:{109.390},{52.131}) {\tikz\pgfuseplotmark{m6b};};
\node[stars] at (axis cs:{109.428},{38.876}) {\tikz\pgfuseplotmark{m6c};};
\node[stars] at (axis cs:{109.509},{40.883}) {\tikz\pgfuseplotmark{m6b};};
\node[stars] at (axis cs:{109.517},{30.956}) {\tikz\pgfuseplotmark{m6c};};
\node[stars] at (axis cs:{109.633},{49.465}) {\tikz\pgfuseplotmark{m5b};};
\node[stars] at (axis cs:{110.263},{42.656}) {\tikz\pgfuseplotmark{m6c};};
\node[stars] at (axis cs:{110.323},{45.228}) {\tikz\pgfuseplotmark{m6a};};
\node[stars] at (axis cs:{110.511},{36.760}) {\tikz\pgfuseplotmark{m5b};};
\node[stars] at (axis cs:{110.556},{38.996}) {\tikz\pgfuseplotmark{m6c};};
\node[stars] at (axis cs:{110.572},{59.902}) {\tikz\pgfuseplotmark{m6cb};};
\node[stars] at (axis cs:{110.717},{55.281}) {\tikz\pgfuseplotmark{m6ab};};
\node[stars] at (axis cs:{111.035},{40.672}) {\tikz\pgfuseplotmark{m5c};};
\node[stars] at (axis cs:{111.139},{27.638}) {\tikz\pgfuseplotmark{m6a};};
\node[stars] at (axis cs:{111.238},{51.887}) {\tikz\pgfuseplotmark{m6a};};
\node[stars] at (axis cs:{111.342},{81.257}) {\tikz\pgfuseplotmark{m6cv};};
\node[stars] at (axis cs:{111.432},{27.798}) {\tikz\pgfuseplotmark{m4a};};
\node[stars] at (axis cs:{111.679},{49.211}) {\tikz\pgfuseplotmark{m5av};};
\node[stars] at (axis cs:{111.858},{66.331}) {\tikz\pgfuseplotmark{m6c};};
\node[stars] at (axis cs:{112.166},{27.553}) {\tikz\pgfuseplotmark{m6c};};
\node[stars] at (axis cs:{112.215},{48.184}) {\tikz\pgfuseplotmark{m6av};};
\node[stars] at (axis cs:{112.219},{46.314}) {\tikz\pgfuseplotmark{m6c};};
\node[stars] at (axis cs:{112.278},{31.785}) {\tikz\pgfuseplotmark{m4cvb};};
\node[stars] at (axis cs:{112.335},{28.118}) {\tikz\pgfuseplotmark{m5b};};
\node[stars] at (axis cs:{112.453},{27.916}) {\tikz\pgfuseplotmark{m5b};};
\node[stars] at (axis cs:{112.483},{49.672}) {\tikz\pgfuseplotmark{m5cb};};
\node[stars] at (axis cs:{112.720},{68.465}) {\tikz\pgfuseplotmark{m6a};};
\node[stars] at (axis cs:{112.769},{82.411}) {\tikz\pgfuseplotmark{m5bv};};
\node[stars] at (axis cs:{113.353},{37.188}) {\tikz\pgfuseplotmark{m6c};};
\node[stars] at (axis cs:{113.465},{51.314}) {\tikz\pgfuseplotmark{m6c};};
\node[stars] at (axis cs:{113.649},{31.888}) {\tikz\pgfuseplotmark{m2ab};};
\node[stars] at (axis cs:{113.787},{30.961}) {\tikz\pgfuseplotmark{m5c};};
\node[stars] at (axis cs:{114.132},{46.180}) {\tikz\pgfuseplotmark{m6a};};
\node[stars] at (axis cs:{114.196},{55.755}) {\tikz\pgfuseplotmark{m6b};};
\node[stars] at (axis cs:{114.324},{40.025}) {\tikz\pgfuseplotmark{m6c};};
\node[stars] at (axis cs:{114.474},{48.774}) {\tikz\pgfuseplotmark{m6bv};};
\node[stars] at (axis cs:{114.637},{35.048}) {\tikz\pgfuseplotmark{m6ab};};
\node[stars] at (axis cs:{114.791},{34.584}) {\tikz\pgfuseplotmark{m5b};};
\node[stars] at (axis cs:{114.975},{32.010}) {\tikz\pgfuseplotmark{m6c};};
\node[stars] at (axis cs:{115.061},{38.344}) {\tikz\pgfuseplotmark{m6a};};
\node[stars] at (axis cs:{115.127},{87.020}) {\tikz\pgfuseplotmark{m5bv};};
\node[stars] at (axis cs:{115.206},{57.083}) {\tikz\pgfuseplotmark{m6b};};
\node[stars] at (axis cs:{115.302},{48.131}) {\tikz\pgfuseplotmark{m6a};};
\node[stars] at (axis cs:{115.681},{34.000}) {\tikz\pgfuseplotmark{m6b};};
\node[stars] at (axis cs:{115.752},{58.710}) {\tikz\pgfuseplotmark{m5b};};
\node[stars] at (axis cs:{115.828},{28.884}) {\tikz\pgfuseplotmark{m4cv};};
\node[stars] at (axis cs:{116.017},{50.434}) {\tikz\pgfuseplotmark{m5c};};
\node[stars] at (axis cs:{116.329},{28.026}) {\tikz\pgfuseplotmark{m1cvb};};
\node[stars] at (axis cs:{116.614},{62.830}) {\tikz\pgfuseplotmark{m6cv};};
\node[stars] at (axis cs:{116.664},{37.517}) {\tikz\pgfuseplotmark{m5cv};};
\node[stars] at (axis cs:{116.667},{65.456}) {\tikz\pgfuseplotmark{m6b};};
\node[stars] at (axis cs:{116.876},{33.415}) {\tikz\pgfuseplotmark{m5cv};};
\node[stars] at (axis cs:{117.760},{33.234}) {\tikz\pgfuseplotmark{m6b};};
\node[stars] at (axis cs:{117.774},{54.129}) {\tikz\pgfuseplotmark{m6b};};
\node[stars] at (axis cs:{118.152},{55.209}) {\tikz\pgfuseplotmark{m6c};};
\node[stars] at (axis cs:{118.622},{47.386}) {\tikz\pgfuseplotmark{m6c};};
\node[stars] at (axis cs:{118.678},{47.564}) {\tikz\pgfuseplotmark{m5c};};
\node[stars] at (axis cs:{118.920},{35.412}) {\tikz\pgfuseplotmark{m6cv};};
\node[stars] at (axis cs:{119.569},{43.977}) {\tikz\pgfuseplotmark{m6c};};
\node[stars] at (axis cs:{120.049},{73.918}) {\tikz\pgfuseplotmark{m5c};};
\node[stars] at (axis cs:{120.337},{59.047}) {\tikz\pgfuseplotmark{m6a};};
\node[stars] at (axis cs:{120.427},{60.324}) {\tikz\pgfuseplotmark{m6bv};};
\node[stars] at (axis cs:{120.480},{35.413}) {\tikz\pgfuseplotmark{m6c};};
\node[stars] at (axis cs:{120.628},{63.090}) {\tikz\pgfuseplotmark{m6bb};};
\node[stars] at (axis cs:{120.649},{57.274}) {\tikz\pgfuseplotmark{m6cv};};
\node[stars] at (axis cs:{120.880},{27.794}) {\tikz\pgfuseplotmark{m5b};};
\node[stars] at (axis cs:{121.196},{79.479}) {\tikz\pgfuseplotmark{m5c};};
\node[stars] at (axis cs:{121.404},{27.530}) {\tikz\pgfuseplotmark{m6cvb};};
\node[stars] at (axis cs:{121.791},{43.260}) {\tikz\pgfuseplotmark{m6c};};
\node[stars] at (axis cs:{122.114},{51.507}) {\tikz\pgfuseplotmark{m5avb};};
\node[stars] at (axis cs:{122.346},{42.430}) {\tikz\pgfuseplotmark{m6c};};
\node[stars] at (axis cs:{122.397},{29.093}) {\tikz\pgfuseplotmark{m6c};};
\node[stars] at (axis cs:{122.516},{58.248}) {\tikz\pgfuseplotmark{m6bv};};
\node[stars] at (axis cs:{123.203},{68.474}) {\tikz\pgfuseplotmark{m5c};};
\node[stars] at (axis cs:{123.287},{29.656}) {\tikz\pgfuseplotmark{m6av};};
\node[stars] at (axis cs:{123.459},{56.452}) {\tikz\pgfuseplotmark{m6av};};
\node[stars] at (axis cs:{123.960},{60.380}) {\tikz\pgfuseplotmark{m6cb};};
\node[stars] at (axis cs:{124.225},{84.058}) {\tikz\pgfuseplotmark{m6c};};
\node[stars] at (axis cs:{124.460},{59.571}) {\tikz\pgfuseplotmark{m6a};};
\node[stars] at (axis cs:{124.566},{54.144}) {\tikz\pgfuseplotmark{m6cv};};
\node[stars] at (axis cs:{124.821},{62.507}) {\tikz\pgfuseplotmark{m6av};};
\node[stars] at (axis cs:{124.885},{75.757}) {\tikz\pgfuseplotmark{m6a};};
\node[stars] at (axis cs:{125.016},{27.218}) {\tikz\pgfuseplotmark{m5bv};};
\node[stars] at (axis cs:{125.109},{57.743}) {\tikz\pgfuseplotmark{m6b};};
\node[stars] at (axis cs:{125.168},{72.407}) {\tikz\pgfuseplotmark{m6bb};};
\node[stars] at (axis cs:{125.540},{74.820}) {\tikz\pgfuseplotmark{m6c};};
\node[stars] at (axis cs:{125.684},{60.631}) {\tikz\pgfuseplotmark{m6c};};
\node[stars] at (axis cs:{125.709},{43.188}) {\tikz\pgfuseplotmark{m4cv};};
\node[stars] at (axis cs:{125.952},{53.220}) {\tikz\pgfuseplotmark{m6av};};
\node[stars] at (axis cs:{126.138},{82.431}) {\tikz\pgfuseplotmark{m6c};};
\node[stars] at (axis cs:{126.178},{42.005}) {\tikz\pgfuseplotmark{m6bb};};
\node[stars] at (axis cs:{126.270},{35.011}) {\tikz\pgfuseplotmark{m6b};};
\node[stars] at (axis cs:{126.615},{27.893}) {\tikz\pgfuseplotmark{m6a};};
\node[stars] at (axis cs:{126.903},{45.653}) {\tikz\pgfuseplotmark{m6c};};
\node[stars] at (axis cs:{127.442},{67.297}) {\tikz\pgfuseplotmark{m6b};};
\node[stars] at (axis cs:{127.566},{60.718}) {\tikz\pgfuseplotmark{m3cv};};
\node[stars] at (axis cs:{127.833},{37.265}) {\tikz\pgfuseplotmark{m6c};};
\node[stars] at (axis cs:{128.139},{53.115}) {\tikz\pgfuseplotmark{m6cv};};
\node[stars] at (axis cs:{128.222},{69.320}) {\tikz\pgfuseplotmark{m6c};};
\node[stars] at (axis cs:{128.229},{38.016}) {\tikz\pgfuseplotmark{m6b};};
\node[stars] at (axis cs:{128.341},{36.436}) {\tikz\pgfuseplotmark{m6c};};
\node[stars] at (axis cs:{128.651},{65.145}) {\tikz\pgfuseplotmark{m5c};};
\node[stars] at (axis cs:{128.683},{36.420}) {\tikz\pgfuseplotmark{m6a};};
\node[stars] at (axis cs:{129.203},{74.724}) {\tikz\pgfuseplotmark{m6cb};};
\node[stars] at (axis cs:{129.579},{32.802}) {\tikz\pgfuseplotmark{m6b};};
\node[stars] at (axis cs:{129.592},{53.402}) {\tikz\pgfuseplotmark{m6av};};
\node[stars] at (axis cs:{129.750},{52.925}) {\tikz\pgfuseplotmark{m6c};};
\node[stars] at (axis cs:{129.792},{59.939}) {\tikz\pgfuseplotmark{m6c};};
\node[stars] at (axis cs:{129.799},{65.021}) {\tikz\pgfuseplotmark{m6av};};
\node[stars] at (axis cs:{129.823},{52.711}) {\tikz\pgfuseplotmark{m6b};};
\node[stars] at (axis cs:{129.928},{73.629}) {\tikz\pgfuseplotmark{m6c};};
\node[stars] at (axis cs:{130.053},{64.328}) {\tikz\pgfuseplotmark{m5a};};
\node[stars] at (axis cs:{130.076},{31.942}) {\tikz\pgfuseplotmark{m6c};};
\node[stars] at (axis cs:{130.254},{45.834}) {\tikz\pgfuseplotmark{m5c};};
\node[stars] at (axis cs:{130.751},{46.901}) {\tikz\pgfuseplotmark{m6c};};
\node[stars] at (axis cs:{131.339},{30.698}) {\tikz\pgfuseplotmark{m6b};};
\node[stars] at (axis cs:{131.674},{28.760}) {\tikz\pgfuseplotmark{m4bb};};
\node[stars] at (axis cs:{132.205},{66.708}) {\tikz\pgfuseplotmark{m6c};};
\node[stars] at (axis cs:{132.634},{33.285}) {\tikz\pgfuseplotmark{m6c};};
\node[stars] at (axis cs:{132.987},{43.726}) {\tikz\pgfuseplotmark{m5c};};
\node[stars] at (axis cs:{133.042},{42.003}) {\tikz\pgfuseplotmark{m6b};};
\node[stars] at (axis cs:{133.049},{45.313}) {\tikz\pgfuseplotmark{m6b};};
\node[stars] at (axis cs:{133.119},{28.259}) {\tikz\pgfuseplotmark{m6cvb};};
\node[stars] at (axis cs:{133.144},{32.474}) {\tikz\pgfuseplotmark{m6av};};
\node[stars] at (axis cs:{133.149},{28.331}) {\tikz\pgfuseplotmark{m6b};};
\node[stars] at (axis cs:{133.275},{59.056}) {\tikz\pgfuseplotmark{m6c};};
\node[stars] at (axis cs:{133.344},{61.962}) {\tikz\pgfuseplotmark{m6a};};
\node[stars] at (axis cs:{133.482},{35.538}) {\tikz\pgfuseplotmark{m6cv};};
\node[stars] at (axis cs:{133.561},{30.579}) {\tikz\pgfuseplotmark{m5cb};};
\node[stars] at (axis cs:{133.915},{27.927}) {\tikz\pgfuseplotmark{m5c};};
\node[stars] at (axis cs:{134.127},{40.201}) {\tikz\pgfuseplotmark{m6b};};
\node[stars] at (axis cs:{134.156},{64.604}) {\tikz\pgfuseplotmark{m6a};};
\node[stars] at (axis cs:{134.208},{45.631}) {\tikz\pgfuseplotmark{m6a};};
\node[stars] at (axis cs:{134.236},{32.910}) {\tikz\pgfuseplotmark{m5c};};
\node[stars] at (axis cs:{134.494},{30.234}) {\tikz\pgfuseplotmark{m6c};};
\node[stars] at (axis cs:{134.802},{48.042}) {\tikz\pgfuseplotmark{m3cv};};
\node[stars] at (axis cs:{134.886},{32.419}) {\tikz\pgfuseplotmark{m5cvb};};
\node[stars] at (axis cs:{135.128},{37.604}) {\tikz\pgfuseplotmark{m6c};};
\node[stars] at (axis cs:{135.160},{41.783}) {\tikz\pgfuseplotmark{m4b};};
\node[stars] at (axis cs:{135.351},{32.252}) {\tikz\pgfuseplotmark{m6bb};};
\node[stars] at (axis cs:{135.420},{39.713}) {\tikz\pgfuseplotmark{m6c};};
\node[stars] at (axis cs:{135.453},{27.903}) {\tikz\pgfuseplotmark{m6bvb};};
\node[stars] at (axis cs:{135.636},{67.629}) {\tikz\pgfuseplotmark{m5av};};
\node[stars] at (axis cs:{135.906},{47.156}) {\tikz\pgfuseplotmark{m4av};};
\node[stars] at (axis cs:{136.002},{54.284}) {\tikz\pgfuseplotmark{m6a};};
\node[stars] at (axis cs:{136.230},{32.377}) {\tikz\pgfuseplotmark{m6c};};
\node[stars] at (axis cs:{136.350},{48.530}) {\tikz\pgfuseplotmark{m5c};};
\node[stars] at (axis cs:{136.632},{38.452}) {\tikz\pgfuseplotmark{m5a};};
\node[stars] at (axis cs:{136.680},{59.344}) {\tikz\pgfuseplotmark{m6c};};
\node[stars] at (axis cs:{137.000},{29.654}) {\tikz\pgfuseplotmark{m5c};};
\node[stars] at (axis cs:{137.017},{32.540}) {\tikz\pgfuseplotmark{m6c};};
\node[stars] at (axis cs:{137.098},{66.873}) {\tikz\pgfuseplotmark{m5c};};
\node[stars] at (axis cs:{137.213},{33.882}) {\tikz\pgfuseplotmark{m6b};};
\node[stars] at (axis cs:{137.218},{51.605}) {\tikz\pgfuseplotmark{m4c};};
\node[stars] at (axis cs:{137.598},{67.134}) {\tikz\pgfuseplotmark{m5ab};};
\node[stars] at (axis cs:{137.662},{30.963}) {\tikz\pgfuseplotmark{m6bv};};
\node[stars] at (axis cs:{137.729},{63.513}) {\tikz\pgfuseplotmark{m5a};};
\node[stars] at (axis cs:{137.803},{80.829}) {\tikz\pgfuseplotmark{m6c};};
\node[stars] at (axis cs:{138.451},{43.218}) {\tikz\pgfuseplotmark{m5cv};};
\node[stars] at (axis cs:{138.586},{61.423}) {\tikz\pgfuseplotmark{m5c};};
\node[stars] at (axis cs:{138.809},{34.634}) {\tikz\pgfuseplotmark{m6b};};
\node[stars] at (axis cs:{138.839},{84.181}) {\tikz\pgfuseplotmark{m6c};};
\node[stars] at (axis cs:{138.957},{56.741}) {\tikz\pgfuseplotmark{m5cv};};
\node[stars] at (axis cs:{138.969},{72.946}) {\tikz\pgfuseplotmark{m6b};};
\node[stars] at (axis cs:{139.047},{54.022}) {\tikz\pgfuseplotmark{m5av};};
\node[stars] at (axis cs:{139.380},{46.817}) {\tikz\pgfuseplotmark{m6b};};
\node[stars] at (axis cs:{139.608},{35.364}) {\tikz\pgfuseplotmark{m6bb};};
\node[stars] at (axis cs:{139.711},{36.803}) {\tikz\pgfuseplotmark{m4ab};};
\node[stars] at (axis cs:{139.982},{74.016}) {\tikz\pgfuseplotmark{m6c};};
\node[stars] at (axis cs:{140.182},{51.266}) {\tikz\pgfuseplotmark{m6cb};};
\node[stars] at (axis cs:{140.247},{38.188}) {\tikz\pgfuseplotmark{m6bb};};
\node[stars] at (axis cs:{140.264},{34.392}) {\tikz\pgfuseplotmark{m3bv};};
\node[stars] at (axis cs:{140.348},{45.370}) {\tikz\pgfuseplotmark{m6c};};
\node[stars] at (axis cs:{140.363},{32.902}) {\tikz\pgfuseplotmark{m6c};};
\node[stars] at (axis cs:{140.430},{56.699}) {\tikz\pgfuseplotmark{m6av};};
\node[stars] at (axis cs:{141.232},{51.574}) {\tikz\pgfuseplotmark{m6cb};};
\node[stars] at (axis cs:{141.434},{63.941}) {\tikz\pgfuseplotmark{m6cv};};
\node[stars] at (axis cs:{141.965},{75.098}) {\tikz\pgfuseplotmark{m6c};};
\node[stars] at (axis cs:{142.167},{45.601}) {\tikz\pgfuseplotmark{m5cb};};
\node[stars] at (axis cs:{142.680},{33.656}) {\tikz\pgfuseplotmark{m6ab};};
\node[stars] at (axis cs:{142.882},{63.062}) {\tikz\pgfuseplotmark{m4av};};
\node[stars] at (axis cs:{142.885},{35.103}) {\tikz\pgfuseplotmark{m5cv};};
\node[stars] at (axis cs:{143.214},{51.677}) {\tikz\pgfuseplotmark{m3cv};};
\node[stars] at (axis cs:{143.297},{45.514}) {\tikz\pgfuseplotmark{m6c};};
\node[stars] at (axis cs:{143.377},{36.487}) {\tikz\pgfuseplotmark{m6c};};
\node[stars] at (axis cs:{143.556},{36.397}) {\tikz\pgfuseplotmark{m5av};};
\node[stars] at (axis cs:{143.620},{69.830}) {\tikz\pgfuseplotmark{m5av};};
\node[stars] at (axis cs:{143.706},{52.051}) {\tikz\pgfuseplotmark{m4c};};
\node[stars] at (axis cs:{143.723},{72.206}) {\tikz\pgfuseplotmark{m6av};};
\node[stars] at (axis cs:{143.766},{39.621}) {\tikz\pgfuseplotmark{m5a};};
\node[stars] at (axis cs:{143.915},{35.810}) {\tikz\pgfuseplotmark{m5cv};};
\node[stars] at (axis cs:{144.028},{74.319}) {\tikz\pgfuseplotmark{m6c};};
\node[stars] at (axis cs:{144.179},{31.162}) {\tikz\pgfuseplotmark{m6av};};
\node[stars] at (axis cs:{144.272},{81.326}) {\tikz\pgfuseplotmark{m4c};};
\node[stars] at (axis cs:{144.406},{60.213}) {\tikz\pgfuseplotmark{m6c};};
\node[stars] at (axis cs:{144.591},{40.240}) {\tikz\pgfuseplotmark{m5c};};
\node[stars] at (axis cs:{144.866},{67.272}) {\tikz\pgfuseplotmark{m6b};};
\node[stars] at (axis cs:{145.320},{55.866}) {\tikz\pgfuseplotmark{m6c};};
\node[stars] at (axis cs:{145.396},{31.278}) {\tikz\pgfuseplotmark{m6bv};};
\node[stars] at (axis cs:{145.501},{39.758}) {\tikz\pgfuseplotmark{m6a};};
\node[stars] at (axis cs:{145.562},{69.237}) {\tikz\pgfuseplotmark{m6a};};
\node[stars] at (axis cs:{145.678},{35.093}) {\tikz\pgfuseplotmark{m6bv};};
\node[stars] at (axis cs:{145.680},{48.431}) {\tikz\pgfuseplotmark{m6c};};
\node[stars] at (axis cs:{145.738},{72.253}) {\tikz\pgfuseplotmark{m5c};};
\node[stars] at (axis cs:{145.779},{54.363}) {\tikz\pgfuseplotmark{m6c};};
\node[stars] at (axis cs:{145.889},{29.974}) {\tikz\pgfuseplotmark{m6a};};
\node[stars] at (axis cs:{146.152},{64.984}) {\tikz\pgfuseplotmark{m6c};};
\node[stars] at (axis cs:{146.379},{78.135}) {\tikz\pgfuseplotmark{m6c};};
\node[stars] at (axis cs:{146.632},{57.128}) {\tikz\pgfuseplotmark{m5bv};};
\node[stars] at (axis cs:{146.825},{79.137}) {\tikz\pgfuseplotmark{m6b};};
\node[stars] at (axis cs:{147.147},{46.021}) {\tikz\pgfuseplotmark{m5b};};
\node[stars] at (axis cs:{147.598},{65.593}) {\tikz\pgfuseplotmark{m6c};};
\node[stars] at (axis cs:{147.747},{59.039}) {\tikz\pgfuseplotmark{m4av};};
\node[stars] at (axis cs:{148.027},{54.064}) {\tikz\pgfuseplotmark{m5a};};
\node[stars] at (axis cs:{148.764},{61.116}) {\tikz\pgfuseplotmark{m6c};};
\node[stars] at (axis cs:{148.929},{49.820}) {\tikz\pgfuseplotmark{m5c};};
\node[stars] at (axis cs:{149.307},{57.418}) {\tikz\pgfuseplotmark{m6b};};
\node[stars] at (axis cs:{149.421},{41.056}) {\tikz\pgfuseplotmark{m5b};};
\node[stars] at (axis cs:{149.487},{45.414}) {\tikz\pgfuseplotmark{m6c};};
\node[stars] at (axis cs:{149.595},{72.879}) {\tikz\pgfuseplotmark{m6a};};
\node[stars] at (axis cs:{149.608},{27.759}) {\tikz\pgfuseplotmark{m6c};};
\node[stars] at (axis cs:{149.901},{29.645}) {\tikz\pgfuseplotmark{m6a};};
\node[stars] at (axis cs:{149.965},{56.812}) {\tikz\pgfuseplotmark{m5c};};
\node[stars] at (axis cs:{150.253},{31.924}) {\tikz\pgfuseplotmark{m5c};};
\node[stars] at (axis cs:{151.151},{53.892}) {\tikz\pgfuseplotmark{m6a};};
\node[stars] at (axis cs:{151.293},{52.371}) {\tikz\pgfuseplotmark{m6c};};
\node[stars] at (axis cs:{151.857},{35.245}) {\tikz\pgfuseplotmark{m4cv};};
\node[stars] at (axis cs:{152.066},{31.604}) {\tikz\pgfuseplotmark{m6c};};
\node[stars] at (axis cs:{152.143},{83.918}) {\tikz\pgfuseplotmark{m6c};};
\node[stars] at (axis cs:{152.746},{40.661}) {\tikz\pgfuseplotmark{m6cv};};
\node[stars] at (axis cs:{152.803},{37.402}) {\tikz\pgfuseplotmark{m6av};};
\node[stars] at (axis cs:{153.457},{27.136}) {\tikz\pgfuseplotmark{m6b};};
\node[stars] at (axis cs:{153.776},{31.468}) {\tikz\pgfuseplotmark{m6c};};
\node[stars] at (axis cs:{153.782},{59.986}) {\tikz\pgfuseplotmark{m6cv};};
\node[stars] at (axis cs:{154.060},{29.310}) {\tikz\pgfuseplotmark{m5c};};
\node[stars] at (axis cs:{154.117},{28.682}) {\tikz\pgfuseplotmark{m6c};};
\node[stars] at (axis cs:{154.274},{42.914}) {\tikz\pgfuseplotmark{m3c};};
\node[stars] at (axis cs:{154.508},{65.108}) {\tikz\pgfuseplotmark{m6av};};
\node[stars] at (axis cs:{154.745},{46.761}) {\tikz\pgfuseplotmark{m6c};};
\node[stars] at (axis cs:{154.862},{48.397}) {\tikz\pgfuseplotmark{m6b};};
\node[stars] at (axis cs:{155.062},{53.779}) {\tikz\pgfuseplotmark{m6c};};
\node[stars] at (axis cs:{155.130},{54.217}) {\tikz\pgfuseplotmark{m6b};};
\node[stars] at (axis cs:{155.264},{68.747}) {\tikz\pgfuseplotmark{m6bv};};
\node[stars] at (axis cs:{155.544},{41.229}) {\tikz\pgfuseplotmark{m6av};};
\node[stars] at (axis cs:{155.582},{41.499}) {\tikz\pgfuseplotmark{m3bv};};
\node[stars] at (axis cs:{155.776},{33.908}) {\tikz\pgfuseplotmark{m6b};};
\node[stars] at (axis cs:{155.924},{29.616}) {\tikz\pgfuseplotmark{m6c};};
\node[stars] at (axis cs:{156.033},{65.566}) {\tikz\pgfuseplotmark{m5bv};};
\node[stars] at (axis cs:{156.036},{33.719}) {\tikz\pgfuseplotmark{m6a};};
\node[stars] at (axis cs:{156.437},{35.425}) {\tikz\pgfuseplotmark{m6c};};
\node[stars] at (axis cs:{156.478},{33.796}) {\tikz\pgfuseplotmark{m5a};};
\node[stars] at (axis cs:{156.867},{41.601}) {\tikz\pgfuseplotmark{m6b};};
\node[stars] at (axis cs:{156.971},{36.707}) {\tikz\pgfuseplotmark{m4c};};
\node[stars] at (axis cs:{157.016},{48.785}) {\tikz\pgfuseplotmark{m6c};};
\node[stars] at (axis cs:{157.152},{45.212}) {\tikz\pgfuseplotmark{m6c};};
\node[stars] at (axis cs:{157.423},{84.252}) {\tikz\pgfuseplotmark{m6a};};
\node[stars] at (axis cs:{157.477},{65.626}) {\tikz\pgfuseplotmark{m6c};};
\node[stars] at (axis cs:{157.527},{38.925}) {\tikz\pgfuseplotmark{m6a};};
\node[stars] at (axis cs:{157.611},{64.258}) {\tikz\pgfuseplotmark{m6b};};
\node[stars] at (axis cs:{157.657},{55.980}) {\tikz\pgfuseplotmark{m5ab};};
\node[stars] at (axis cs:{157.769},{82.558}) {\tikz\pgfuseplotmark{m5cv};};
\node[stars] at (axis cs:{157.964},{32.379}) {\tikz\pgfuseplotmark{m6b};};
\node[stars] at (axis cs:{158.308},{40.425}) {\tikz\pgfuseplotmark{m5a};};
\node[stars] at (axis cs:{158.379},{34.989}) {\tikz\pgfuseplotmark{m6a};};
\node[stars] at (axis cs:{158.773},{75.713}) {\tikz\pgfuseplotmark{m5a};};
\node[stars] at (axis cs:{158.790},{57.082}) {\tikz\pgfuseplotmark{m5c};};
\node[stars] at (axis cs:{159.008},{80.494}) {\tikz\pgfuseplotmark{m6c};};
\node[stars] at (axis cs:{159.089},{36.327}) {\tikz\pgfuseplotmark{m6c};};
\node[stars] at (axis cs:{159.468},{34.078}) {\tikz\pgfuseplotmark{m6c};};
\node[stars] at (axis cs:{159.680},{31.976}) {\tikz\pgfuseplotmark{m5a};};
\node[stars] at (axis cs:{159.774},{53.668}) {\tikz\pgfuseplotmark{m6a};};
\node[stars] at (axis cs:{159.782},{37.910}) {\tikz\pgfuseplotmark{m6a};};
\node[stars] at (axis cs:{160.451},{68.443}) {\tikz\pgfuseplotmark{m6a};};
\node[stars] at (axis cs:{160.486},{65.716}) {\tikz\pgfuseplotmark{m5bv};};
\node[stars] at (axis cs:{160.547},{31.697}) {\tikz\pgfuseplotmark{m6bv};};
\node[stars] at (axis cs:{160.767},{69.076}) {\tikz\pgfuseplotmark{m5b};};
\node[stars] at (axis cs:{160.887},{46.204}) {\tikz\pgfuseplotmark{m5cb};};
\node[stars] at (axis cs:{160.931},{57.199}) {\tikz\pgfuseplotmark{m6a};};
\node[stars] at (axis cs:{161.267},{67.411}) {\tikz\pgfuseplotmark{m6bv};};
\node[stars] at (axis cs:{161.466},{30.682}) {\tikz\pgfuseplotmark{m5cb};};
\node[stars] at (axis cs:{161.594},{57.366}) {\tikz\pgfuseplotmark{m6c};};
\node[stars] at (axis cs:{162.208},{65.132}) {\tikz\pgfuseplotmark{m6c};};
\node[stars] at (axis cs:{162.238},{29.416}) {\tikz\pgfuseplotmark{m6c};};
\node[stars] at (axis cs:{162.370},{63.810}) {\tikz\pgfuseplotmark{m6c};};
\node[stars] at (axis cs:{162.474},{27.974}) {\tikz\pgfuseplotmark{m6b};};
\node[stars] at (axis cs:{162.796},{56.582}) {\tikz\pgfuseplotmark{m6a};};
\node[stars] at (axis cs:{162.849},{59.320}) {\tikz\pgfuseplotmark{m6a};};
\node[stars] at (axis cs:{163.134},{52.504}) {\tikz\pgfuseplotmark{m6c};};
\node[stars] at (axis cs:{163.328},{34.215}) {\tikz\pgfuseplotmark{m4av};};
\node[stars] at (axis cs:{163.394},{54.585}) {\tikz\pgfuseplotmark{m5b};};
\node[stars] at (axis cs:{163.495},{43.190}) {\tikz\pgfuseplotmark{m5a};};
\node[stars] at (axis cs:{163.742},{34.035}) {\tikz\pgfuseplotmark{m6a};};
\node[stars] at (axis cs:{163.935},{33.507}) {\tikz\pgfuseplotmark{m5b};};
\node[stars] at (axis cs:{164.060},{42.008}) {\tikz\pgfuseplotmark{m6b};};
\node[stars] at (axis cs:{164.824},{51.882}) {\tikz\pgfuseplotmark{m6c};};
\node[stars] at (axis cs:{164.867},{40.430}) {\tikz\pgfuseplotmark{m5b};};
\node[stars] at (axis cs:{164.887},{36.093}) {\tikz\pgfuseplotmark{m6bv};};
\node[stars] at (axis cs:{164.987},{77.770}) {\tikz\pgfuseplotmark{m6c};};
\node[stars] at (axis cs:{165.061},{45.526}) {\tikz\pgfuseplotmark{m5c};};
\node[stars] at (axis cs:{165.086},{42.912}) {\tikz\pgfuseplotmark{m6b};};
\node[stars] at (axis cs:{165.106},{51.502}) {\tikz\pgfuseplotmark{m6c};};
\node[stars] at (axis cs:{165.210},{39.212}) {\tikz\pgfuseplotmark{m5b};};
\node[stars] at (axis cs:{165.274},{63.421}) {\tikz\pgfuseplotmark{m6c};};
\node[stars] at (axis cs:{165.460},{56.382}) {\tikz\pgfuseplotmark{m2cv};};
\node[stars] at (axis cs:{165.864},{70.031}) {\tikz\pgfuseplotmark{m6c};};
\node[stars] at (axis cs:{165.932},{61.751}) {\tikz\pgfuseplotmark{m2av};};
\node[stars] at (axis cs:{166.130},{38.241}) {\tikz\pgfuseplotmark{m6bb};};
\node[stars] at (axis cs:{167.330},{36.309}) {\tikz\pgfuseplotmark{m6av};};
\node[stars] at (axis cs:{167.410},{43.208}) {\tikz\pgfuseplotmark{m6bv};};
\node[stars] at (axis cs:{167.416},{67.210}) {\tikz\pgfuseplotmark{m6bv};};
\node[stars] at (axis cs:{167.416},{44.498}) {\tikz\pgfuseplotmark{m3b};};
\node[stars] at (axis cs:{168.046},{68.272}) {\tikz\pgfuseplotmark{m6cv};};
\node[stars] at (axis cs:{168.135},{35.814}) {\tikz\pgfuseplotmark{m6cb};};
\node[stars] at (axis cs:{168.417},{41.089}) {\tikz\pgfuseplotmark{m6cb};};
\node[stars] at (axis cs:{169.017},{52.773}) {\tikz\pgfuseplotmark{m6cb};};
\node[stars] at (axis cs:{169.174},{49.476}) {\tikz\pgfuseplotmark{m6b};};
\node[stars] at (axis cs:{169.546},{31.529}) {\tikz\pgfuseplotmark{m4avb};};
\node[stars] at (axis cs:{169.620},{33.094}) {\tikz\pgfuseplotmark{m3c};};
\node[stars] at (axis cs:{169.783},{38.186}) {\tikz\pgfuseplotmark{m5av};};
\node[stars] at (axis cs:{170.224},{67.101}) {\tikz\pgfuseplotmark{m6c};};
\node[stars] at (axis cs:{170.455},{57.075}) {\tikz\pgfuseplotmark{m6cv};};
\node[stars] at (axis cs:{170.707},{43.482}) {\tikz\pgfuseplotmark{m5b};};
\node[stars] at (axis cs:{170.714},{64.330}) {\tikz\pgfuseplotmark{m6b};};
\node[stars] at (axis cs:{171.488},{55.850}) {\tikz\pgfuseplotmark{m6a};};
\node[stars] at (axis cs:{171.606},{33.451}) {\tikz\pgfuseplotmark{m6c};};
\node[stars] at (axis cs:{172.267},{39.337}) {\tikz\pgfuseplotmark{m5cb};};
\node[stars] at (axis cs:{172.269},{61.778}) {\tikz\pgfuseplotmark{m6a};};
\node[stars] at (axis cs:{172.432},{56.738}) {\tikz\pgfuseplotmark{m6c};};
\node[stars] at (axis cs:{172.603},{46.657}) {\tikz\pgfuseplotmark{m6cv};};
\node[stars] at (axis cs:{172.630},{43.173}) {\tikz\pgfuseplotmark{m6b};};
\node[stars] at (axis cs:{172.721},{47.929}) {\tikz\pgfuseplotmark{m6c};};
\node[stars] at (axis cs:{172.851},{69.331}) {\tikz\pgfuseplotmark{m4av};};
\node[stars] at (axis cs:{172.961},{81.127}) {\tikz\pgfuseplotmark{m6bv};};
\node[stars] at (axis cs:{173.086},{61.082}) {\tikz\pgfuseplotmark{m5c};};
\node[stars] at (axis cs:{173.485},{36.815}) {\tikz\pgfuseplotmark{m6c};};
\node[stars] at (axis cs:{173.770},{54.785}) {\tikz\pgfuseplotmark{m6a};};
\node[stars] at (axis cs:{174.012},{69.323}) {\tikz\pgfuseplotmark{m5c};};
\node[stars] at (axis cs:{174.075},{27.781}) {\tikz\pgfuseplotmark{m6a};};
\node[stars] at (axis cs:{174.426},{77.595}) {\tikz\pgfuseplotmark{m6c};};
\node[stars] at (axis cs:{174.471},{50.618}) {\tikz\pgfuseplotmark{m6b};};
\node[stars] at (axis cs:{174.586},{43.625}) {\tikz\pgfuseplotmark{m6a};};
\node[stars] at (axis cs:{174.635},{33.626}) {\tikz\pgfuseplotmark{m6c};};
\node[stars] at (axis cs:{174.640},{46.834}) {\tikz\pgfuseplotmark{m6b};};
\node[stars] at (axis cs:{174.687},{45.108}) {\tikz\pgfuseplotmark{m6cb};};
\node[stars] at (axis cs:{175.114},{57.970}) {\tikz\pgfuseplotmark{m6cv};};
\node[stars] at (axis cs:{175.263},{34.202}) {\tikz\pgfuseplotmark{m5cv};};
\node[stars] at (axis cs:{175.393},{31.746}) {\tikz\pgfuseplotmark{m6ab};};
\node[stars] at (axis cs:{175.431},{55.172}) {\tikz\pgfuseplotmark{m6cv};};
\node[stars] at (axis cs:{175.618},{66.745}) {\tikz\pgfuseplotmark{m5c};};
\node[stars] at (axis cs:{176.513},{47.779}) {\tikz\pgfuseplotmark{m4av};};
\node[stars] at (axis cs:{176.732},{55.628}) {\tikz\pgfuseplotmark{m5c};};
\node[stars] at (axis cs:{177.424},{34.932}) {\tikz\pgfuseplotmark{m6a};};
\node[stars] at (axis cs:{177.790},{33.375}) {\tikz\pgfuseplotmark{m6cb};};
\node[stars] at (axis cs:{178.245},{37.719}) {\tikz\pgfuseplotmark{m6cv};};
\node[stars] at (axis cs:{178.458},{53.695}) {\tikz\pgfuseplotmark{m2cv};};
\node[stars] at (axis cs:{178.809},{36.756}) {\tikz\pgfuseplotmark{m6cv};};
\node[stars] at (axis cs:{178.993},{56.598}) {\tikz\pgfuseplotmark{m6a};};
\node[stars] at (axis cs:{179.222},{61.549}) {\tikz\pgfuseplotmark{m6c};};
\node[stars] at (axis cs:{179.530},{32.274}) {\tikz\pgfuseplotmark{m6cv};};
\node[stars] at (axis cs:{179.823},{33.167}) {\tikz\pgfuseplotmark{m6b};};
\node[stars] at (axis cs:{179.989},{34.035}) {\tikz\pgfuseplotmark{m6c};};
\node[stars] at (axis cs:{180.078},{80.853}) {\tikz\pgfuseplotmark{m6cv};};
\node[stars] at (axis cs:{180.414},{36.042}) {\tikz\pgfuseplotmark{m6a};};
\node[stars] at (axis cs:{180.528},{43.045}) {\tikz\pgfuseplotmark{m5cvb};};
\node[stars] at (axis cs:{181.117},{85.587}) {\tikz\pgfuseplotmark{m6c};};
\node[stars] at (axis cs:{181.313},{76.906}) {\tikz\pgfuseplotmark{m6a};};
\node[stars] at (axis cs:{181.415},{62.933}) {\tikz\pgfuseplotmark{m6cvb};};
\node[stars] at (axis cs:{182.446},{74.661}) {\tikz\pgfuseplotmark{m6c};};
\node[stars] at (axis cs:{182.692},{27.281}) {\tikz\pgfuseplotmark{m6bv};};
\node[stars] at (axis cs:{182.750},{81.710}) {\tikz\pgfuseplotmark{m6bb};};
\node[stars] at (axis cs:{182.937},{57.054}) {\tikz\pgfuseplotmark{m6c};};
\node[stars] at (axis cs:{183.004},{28.536}) {\tikz\pgfuseplotmark{m6c};};
\node[stars] at (axis cs:{183.050},{77.616}) {\tikz\pgfuseplotmark{m5cv};};
\node[stars] at (axis cs:{183.681},{53.435}) {\tikz\pgfuseplotmark{m6c};};
\node[stars] at (axis cs:{183.785},{70.200}) {\tikz\pgfuseplotmark{m6a};};
\node[stars] at (axis cs:{183.834},{87.700}) {\tikz\pgfuseplotmark{m6cv};};
\node[stars] at (axis cs:{183.857},{57.032}) {\tikz\pgfuseplotmark{m3cv};};
\node[stars] at (axis cs:{183.923},{72.551}) {\tikz\pgfuseplotmark{m6cv};};
\node[stars] at (axis cs:{184.031},{40.660}) {\tikz\pgfuseplotmark{m6ab};};
\node[stars] at (axis cs:{184.126},{33.061}) {\tikz\pgfuseplotmark{m5b};};
\node[stars] at (axis cs:{184.214},{86.436}) {\tikz\pgfuseplotmark{m6c};};
\node[stars] at (axis cs:{184.373},{53.191}) {\tikz\pgfuseplotmark{m6a};};
\node[stars] at (axis cs:{184.377},{28.937}) {\tikz\pgfuseplotmark{m6a};};
\node[stars] at (axis cs:{184.632},{30.249}) {\tikz\pgfuseplotmark{m6c};};
\node[stars] at (axis cs:{184.708},{75.160}) {\tikz\pgfuseplotmark{m5c};};
\node[stars] at (axis cs:{184.873},{28.157}) {\tikz\pgfuseplotmark{m6c};};
\node[stars] at (axis cs:{184.953},{48.984}) {\tikz\pgfuseplotmark{m5c};};
\node[stars] at (axis cs:{185.212},{57.864}) {\tikz\pgfuseplotmark{m6a};};
\node[stars] at (axis cs:{185.484},{47.182}) {\tikz\pgfuseplotmark{m6c};};
\node[stars] at (axis cs:{185.946},{42.543}) {\tikz\pgfuseplotmark{m6bv};};
\node[stars] at (axis cs:{186.006},{51.562}) {\tikz\pgfuseplotmark{m5av};};
\node[stars] at (axis cs:{186.263},{56.778}) {\tikz\pgfuseplotmark{m6a};};
\node[stars] at (axis cs:{186.277},{63.803}) {\tikz\pgfuseplotmark{m6c};};
\node[stars] at (axis cs:{186.462},{39.019}) {\tikz\pgfuseplotmark{m5b};};
\node[stars] at (axis cs:{186.600},{27.268}) {\tikz\pgfuseplotmark{m5b};};
\node[stars] at (axis cs:{186.601},{71.930}) {\tikz\pgfuseplotmark{m6c};};
\node[stars] at (axis cs:{186.734},{28.268}) {\tikz\pgfuseplotmark{m4c};};
\node[stars] at (axis cs:{186.896},{55.713}) {\tikz\pgfuseplotmark{m6a};};
\node[stars] at (axis cs:{187.489},{58.406}) {\tikz\pgfuseplotmark{m5cv};};
\node[stars] at (axis cs:{187.512},{51.536}) {\tikz\pgfuseplotmark{m6cb};};
\node[stars] at (axis cs:{187.518},{58.768}) {\tikz\pgfuseplotmark{m6b};};
\node[stars] at (axis cs:{187.528},{69.201}) {\tikz\pgfuseplotmark{m5bv};};
\node[stars] at (axis cs:{187.709},{53.076}) {\tikz\pgfuseplotmark{m6c};};
\node[stars] at (axis cs:{188.371},{69.788}) {\tikz\pgfuseplotmark{m4bv};};
\node[stars] at (axis cs:{188.412},{33.247}) {\tikz\pgfuseplotmark{m5c};};
\node[stars] at (axis cs:{188.436},{41.357}) {\tikz\pgfuseplotmark{m4cv};};
\node[stars] at (axis cs:{188.448},{33.385}) {\tikz\pgfuseplotmark{m6c};};
\node[stars] at (axis cs:{188.683},{70.022}) {\tikz\pgfuseplotmark{m5b};};
\node[stars] at (axis cs:{189.693},{40.875}) {\tikz\pgfuseplotmark{m6cv};};
\node[stars] at (axis cs:{189.820},{35.952}) {\tikz\pgfuseplotmark{m6cv};};
\node[stars] at (axis cs:{190.391},{62.713}) {\tikz\pgfuseplotmark{m6b};};
\node[stars] at (axis cs:{190.767},{61.155}) {\tikz\pgfuseplotmark{m6cv};};
\node[stars] at (axis cs:{191.108},{80.621}) {\tikz\pgfuseplotmark{m6c};};
\node[stars] at (axis cs:{191.113},{44.103}) {\tikz\pgfuseplotmark{m6c};};
\node[stars] at (axis cs:{191.248},{39.279}) {\tikz\pgfuseplotmark{m6b};};
\node[stars] at (axis cs:{191.283},{45.440}) {\tikz\pgfuseplotmark{m5cv};};
\node[stars] at (axis cs:{191.829},{62.781}) {\tikz\pgfuseplotmark{m6b};};
\node[stars] at (axis cs:{191.893},{66.790}) {\tikz\pgfuseplotmark{m5c};};
\node[stars] at (axis cs:{192.164},{60.320}) {\tikz\pgfuseplotmark{m6a};};
\node[stars] at (axis cs:{192.174},{48.467}) {\tikz\pgfuseplotmark{m6c};};
\node[stars] at (axis cs:{192.307},{83.413}) {\tikz\pgfuseplotmark{m5cb};};
\node[stars] at (axis cs:{192.323},{27.552}) {\tikz\pgfuseplotmark{m6a};};
\node[stars] at (axis cs:{192.545},{37.517}) {\tikz\pgfuseplotmark{m6bv};};
\node[stars] at (axis cs:{192.925},{27.541}) {\tikz\pgfuseplotmark{m5bv};};
\node[stars] at (axis cs:{193.507},{55.960}) {\tikz\pgfuseplotmark{m2av};};
\node[stars] at (axis cs:{193.555},{33.534}) {\tikz\pgfuseplotmark{m6c};};
\node[stars] at (axis cs:{193.736},{47.197}) {\tikz\pgfuseplotmark{m6av};};
\node[stars] at (axis cs:{193.869},{65.438}) {\tikz\pgfuseplotmark{m5cv};};
\node[stars] at (axis cs:{194.007},{38.318}) {\tikz\pgfuseplotmark{m3bvb};};
\node[stars] at (axis cs:{194.073},{54.099}) {\tikz\pgfuseplotmark{m6ab};};
\node[stars] at (axis cs:{194.108},{65.994}) {\tikz\pgfuseplotmark{m6cv};};
\node[stars] at (axis cs:{194.282},{46.177}) {\tikz\pgfuseplotmark{m6b};};
\node[stars] at (axis cs:{194.579},{64.925}) {\tikz\pgfuseplotmark{m6c};};
\node[stars] at (axis cs:{194.697},{75.472}) {\tikz\pgfuseplotmark{m6b};};
\node[stars] at (axis cs:{194.979},{66.597}) {\tikz\pgfuseplotmark{m5c};};
\node[stars] at (axis cs:{195.069},{30.785}) {\tikz\pgfuseplotmark{m5bv};};
\node[stars] at (axis cs:{195.182},{56.366}) {\tikz\pgfuseplotmark{m5bv};};
\node[stars] at (axis cs:{195.446},{63.610}) {\tikz\pgfuseplotmark{m6b};};
\node[stars] at (axis cs:{196.207},{73.025}) {\tikz\pgfuseplotmark{m6cb};};
\node[stars] at (axis cs:{196.435},{35.799}) {\tikz\pgfuseplotmark{m5cv};};
\node[stars] at (axis cs:{196.468},{45.269}) {\tikz\pgfuseplotmark{m6av};};
\node[stars] at (axis cs:{196.595},{62.042}) {\tikz\pgfuseplotmark{m6c};};
\node[stars] at (axis cs:{196.795},{27.625}) {\tikz\pgfuseplotmark{m5a};};
\node[stars] at (axis cs:{196.973},{27.556}) {\tikz\pgfuseplotmark{m6c};};
\node[stars] at (axis cs:{197.411},{37.423}) {\tikz\pgfuseplotmark{m6b};};
\node[stars] at (axis cs:{197.513},{38.499}) {\tikz\pgfuseplotmark{m6bvb};};
\node[stars] at (axis cs:{197.968},{27.878}) {\tikz\pgfuseplotmark{m4cv};};
\node[stars] at (axis cs:{198.106},{80.471}) {\tikz\pgfuseplotmark{m6c};};
\node[stars] at (axis cs:{198.429},{40.153}) {\tikz\pgfuseplotmark{m5b};};
\node[stars] at (axis cs:{198.883},{40.855}) {\tikz\pgfuseplotmark{m6a};};
\node[stars] at (axis cs:{199.119},{68.408}) {\tikz\pgfuseplotmark{m6c};};
\node[stars] at (axis cs:{199.386},{40.573}) {\tikz\pgfuseplotmark{m5av};};
\node[stars] at (axis cs:{199.560},{49.682}) {\tikz\pgfuseplotmark{m5cv};};
\node[stars] at (axis cs:{199.616},{34.098}) {\tikz\pgfuseplotmark{m6a};};
\node[stars] at (axis cs:{200.079},{40.151}) {\tikz\pgfuseplotmark{m6a};};
\node[stars] at (axis cs:{200.516},{43.903}) {\tikz\pgfuseplotmark{m6cv};};
\node[stars] at (axis cs:{200.975},{37.034}) {\tikz\pgfuseplotmark{m6bv};};
\node[stars] at (axis cs:{200.981},{54.925}) {\tikz\pgfuseplotmark{m2cb};};
\node[stars] at (axis cs:{201.306},{54.988}) {\tikz\pgfuseplotmark{m4bv};};
\node[stars] at (axis cs:{201.534},{72.391}) {\tikz\pgfuseplotmark{m6av};};
\node[stars] at (axis cs:{201.569},{46.028}) {\tikz\pgfuseplotmark{m6b};};
\node[stars] at (axis cs:{201.737},{78.644}) {\tikz\pgfuseplotmark{m6a};};
\node[stars] at (axis cs:{201.998},{52.745}) {\tikz\pgfuseplotmark{m6c};};
\node[stars] at (axis cs:{202.109},{40.729}) {\tikz\pgfuseplotmark{m6c};};
\node[stars] at (axis cs:{202.191},{50.719}) {\tikz\pgfuseplotmark{m6c};};
\node[stars] at (axis cs:{202.816},{42.106}) {\tikz\pgfuseplotmark{m6b};};
\node[stars] at (axis cs:{203.530},{55.348}) {\tikz\pgfuseplotmark{m6av};};
\node[stars] at (axis cs:{203.591},{38.789}) {\tikz\pgfuseplotmark{m6c};};
\node[stars] at (axis cs:{203.614},{49.016}) {\tikz\pgfuseplotmark{m5av};};
\node[stars] at (axis cs:{203.678},{76.546}) {\tikz\pgfuseplotmark{m6c};};
\node[stars] at (axis cs:{203.699},{37.182}) {\tikz\pgfuseplotmark{m5bv};};
\node[stars] at (axis cs:{204.166},{49.487}) {\tikz\pgfuseplotmark{m6c};};
\node[stars] at (axis cs:{204.296},{71.242}) {\tikz\pgfuseplotmark{m5c};};
\node[stars] at (axis cs:{204.365},{36.295}) {\tikz\pgfuseplotmark{m5bb};};
\node[stars] at (axis cs:{204.812},{51.804}) {\tikz\pgfuseplotmark{m6c};};
\node[stars] at (axis cs:{204.877},{52.921}) {\tikz\pgfuseplotmark{m5cv};};
\node[stars] at (axis cs:{205.065},{31.012}) {\tikz\pgfuseplotmark{m6c};};
\node[stars] at (axis cs:{205.089},{57.207}) {\tikz\pgfuseplotmark{m6cv};};
\node[stars] at (axis cs:{205.097},{50.519}) {\tikz\pgfuseplotmark{m6cv};};
\node[stars] at (axis cs:{205.163},{28.065}) {\tikz\pgfuseplotmark{m6c};};
\node[stars] at (axis cs:{205.184},{54.681}) {\tikz\pgfuseplotmark{m5av};};
\node[stars] at (axis cs:{205.375},{64.822}) {\tikz\pgfuseplotmark{m6a};};
\node[stars] at (axis cs:{205.596},{82.752}) {\tikz\pgfuseplotmark{m6b};};
\node[stars] at (axis cs:{205.620},{41.674}) {\tikz\pgfuseplotmark{m6c};};
\node[stars] at (axis cs:{205.663},{78.064}) {\tikz\pgfuseplotmark{m6b};};
\node[stars] at (axis cs:{205.681},{34.989}) {\tikz\pgfuseplotmark{m6b};};
\node[stars] at (axis cs:{205.978},{52.064}) {\tikz\pgfuseplotmark{m6b};};
\node[stars] at (axis cs:{206.305},{55.879}) {\tikz\pgfuseplotmark{m6c};};
\node[stars] at (axis cs:{206.556},{41.089}) {\tikz\pgfuseplotmark{m6b};};
\node[stars] at (axis cs:{206.579},{38.504}) {\tikz\pgfuseplotmark{m6b};};
\node[stars] at (axis cs:{206.649},{54.432}) {\tikz\pgfuseplotmark{m6av};};
\node[stars] at (axis cs:{206.749},{38.543}) {\tikz\pgfuseplotmark{m5cb};};
\node[stars] at (axis cs:{206.885},{49.313}) {\tikz\pgfuseplotmark{m2av};};
\node[stars] at (axis cs:{207.161},{31.190}) {\tikz\pgfuseplotmark{m6a};};
\node[stars] at (axis cs:{207.437},{36.633}) {\tikz\pgfuseplotmark{m6c};};
\node[stars] at (axis cs:{207.440},{61.489}) {\tikz\pgfuseplotmark{m6b};};
\node[stars] at (axis cs:{207.616},{58.539}) {\tikz\pgfuseplotmark{m6c};};
\node[stars] at (axis cs:{207.746},{68.315}) {\tikz\pgfuseplotmark{m6cb};};
\node[stars] at (axis cs:{207.788},{34.664}) {\tikz\pgfuseplotmark{m6bv};};
\node[stars] at (axis cs:{207.858},{64.723}) {\tikz\pgfuseplotmark{m5av};};
\node[stars] at (axis cs:{207.948},{34.444}) {\tikz\pgfuseplotmark{m5av};};
\node[stars] at (axis cs:{208.285},{40.338}) {\tikz\pgfuseplotmark{m6c};};
\node[stars] at (axis cs:{208.293},{28.648}) {\tikz\pgfuseplotmark{m6b};};
\node[stars] at (axis cs:{208.463},{53.728}) {\tikz\pgfuseplotmark{m6av};};
\node[stars] at (axis cs:{209.044},{32.032}) {\tikz\pgfuseplotmark{m6c};};
\node[stars] at (axis cs:{209.142},{27.492}) {\tikz\pgfuseplotmark{m5bv};};
\node[stars] at (axis cs:{209.384},{61.493}) {\tikz\pgfuseplotmark{m6c};};
\node[stars] at (axis cs:{210.294},{27.387}) {\tikz\pgfuseplotmark{m6c};};
\node[stars] at (axis cs:{210.461},{68.678}) {\tikz\pgfuseplotmark{m6c};};
\node[stars] at (axis cs:{210.551},{45.753}) {\tikz\pgfuseplotmark{m6c};};
\node[stars] at (axis cs:{210.749},{50.972}) {\tikz\pgfuseplotmark{m6c};};
\node[stars] at (axis cs:{211.097},{64.376}) {\tikz\pgfuseplotmark{m4av};};
\node[stars] at (axis cs:{211.735},{74.594}) {\tikz\pgfuseplotmark{m6c};};
\node[stars] at (axis cs:{211.982},{43.854}) {\tikz\pgfuseplotmark{m5cv};};
\node[stars] at (axis cs:{212.072},{49.458}) {\tikz\pgfuseplotmark{m5cv};};
\node[stars] at (axis cs:{212.192},{59.338}) {\tikz\pgfuseplotmark{m6c};};
\node[stars] at (axis cs:{212.212},{77.547}) {\tikz\pgfuseplotmark{m5a};};
\node[stars] at (axis cs:{212.813},{32.296}) {\tikz\pgfuseplotmark{m6b};};
\node[stars] at (axis cs:{213.017},{69.432}) {\tikz\pgfuseplotmark{m5cv};};
\node[stars] at (axis cs:{213.371},{51.790}) {\tikz\pgfuseplotmark{m5avb};};
\node[stars] at (axis cs:{213.598},{41.519}) {\tikz\pgfuseplotmark{m6c};};
\node[stars] at (axis cs:{214.041},{51.367}) {\tikz\pgfuseplotmark{m5avb};};
\node[stars] at (axis cs:{214.096},{46.088}) {\tikz\pgfuseplotmark{m4cv};};
\node[stars] at (axis cs:{214.101},{39.745}) {\tikz\pgfuseplotmark{m6c};};
\node[stars] at (axis cs:{214.338},{51.307}) {\tikz\pgfuseplotmark{m6c};};
\node[stars] at (axis cs:{214.455},{48.002}) {\tikz\pgfuseplotmark{m6c};};
\node[stars] at (axis cs:{214.499},{35.509}) {\tikz\pgfuseplotmark{m5a};};
\node[stars] at (axis cs:{214.733},{54.864}) {\tikz\pgfuseplotmark{m6c};};
\node[stars] at (axis cs:{214.949},{38.794}) {\tikz\pgfuseplotmark{m6c};};
\node[stars] at (axis cs:{215.036},{30.429}) {\tikz\pgfuseplotmark{m6cv};};
\node[stars] at (axis cs:{216.008},{27.413}) {\tikz\pgfuseplotmark{m6c};};
\node[stars] at (axis cs:{216.299},{51.851}) {\tikz\pgfuseplotmark{m4bv};};
\node[stars] at (axis cs:{216.371},{38.393}) {\tikz\pgfuseplotmark{m6c};};
\node[stars] at (axis cs:{216.881},{75.696}) {\tikz\pgfuseplotmark{m4cv};};
\node[stars] at (axis cs:{217.069},{36.197}) {\tikz\pgfuseplotmark{m6c};};
\node[stars] at (axis cs:{217.158},{49.845}) {\tikz\pgfuseplotmark{m6av};};
\node[stars] at (axis cs:{217.403},{41.796}) {\tikz\pgfuseplotmark{m6c};};
\node[stars] at (axis cs:{217.457},{31.791}) {\tikz\pgfuseplotmark{m6bv};};
\node[stars] at (axis cs:{217.692},{63.186}) {\tikz\pgfuseplotmark{m6b};};
\node[stars] at (axis cs:{217.928},{60.225}) {\tikz\pgfuseplotmark{m6cv};};
\node[stars] at (axis cs:{217.957},{30.371}) {\tikz\pgfuseplotmark{m4av};};
\node[stars] at (axis cs:{218.019},{38.308}) {\tikz\pgfuseplotmark{m3bv};};
\node[stars] at (axis cs:{218.129},{55.398}) {\tikz\pgfuseplotmark{m6a};};
\node[stars] at (axis cs:{218.334},{36.959}) {\tikz\pgfuseplotmark{m6cv};};
\node[stars] at (axis cs:{218.410},{79.660}) {\tikz\pgfuseplotmark{m6c};};
\node[stars] at (axis cs:{218.549},{32.534}) {\tikz\pgfuseplotmark{m6cv};};
\node[stars] at (axis cs:{218.566},{57.065}) {\tikz\pgfuseplotmark{m6c};};
\node[stars] at (axis cs:{218.660},{36.626}) {\tikz\pgfuseplotmark{m6b};};
\node[stars] at (axis cs:{218.665},{49.368}) {\tikz\pgfuseplotmark{m6av};};
\node[stars] at (axis cs:{218.670},{29.745}) {\tikz\pgfuseplotmark{m4cvb};};
\node[stars] at (axis cs:{219.552},{43.642}) {\tikz\pgfuseplotmark{m6a};};
\node[stars] at (axis cs:{219.563},{54.023}) {\tikz\pgfuseplotmark{m6a};};
\node[stars] at (axis cs:{219.709},{44.404}) {\tikz\pgfuseplotmark{m5c};};
\node[stars] at (axis cs:{220.514},{61.262}) {\tikz\pgfuseplotmark{m6cvb};};
\node[stars] at (axis cs:{220.935},{40.459}) {\tikz\pgfuseplotmark{m6a};};
\node[stars] at (axis cs:{221.247},{27.074}) {\tikz\pgfuseplotmark{m2cb};};
\node[stars] at (axis cs:{221.307},{32.788}) {\tikz\pgfuseplotmark{m6cv};};
\node[stars] at (axis cs:{222.278},{37.811}) {\tikz\pgfuseplotmark{m6c};};
\node[stars] at (axis cs:{222.328},{46.116}) {\tikz\pgfuseplotmark{m6a};};
\node[stars] at (axis cs:{222.385},{51.374}) {\tikz\pgfuseplotmark{m6cb};};
\node[stars] at (axis cs:{222.422},{48.721}) {\tikz\pgfuseplotmark{m6ab};};
\node[stars] at (axis cs:{222.493},{28.616}) {\tikz\pgfuseplotmark{m6a};};
\node[stars] at (axis cs:{222.585},{82.512}) {\tikz\pgfuseplotmark{m6a};};
\node[stars] at (axis cs:{222.624},{37.272}) {\tikz\pgfuseplotmark{m5c};};
\node[stars] at (axis cs:{222.676},{74.155}) {\tikz\pgfuseplotmark{m2bv};};
\node[stars] at (axis cs:{222.860},{59.294}) {\tikz\pgfuseplotmark{m5c};};
\node[stars] at (axis cs:{223.994},{32.300}) {\tikz\pgfuseplotmark{m6b};};
\node[stars] at (axis cs:{224.096},{49.628}) {\tikz\pgfuseplotmark{m6a};};
\node[stars] at (axis cs:{224.396},{65.932}) {\tikz\pgfuseplotmark{m5av};};
\node[stars] at (axis cs:{224.904},{39.265}) {\tikz\pgfuseplotmark{m6a};};
\node[stars] at (axis cs:{225.161},{47.277}) {\tikz\pgfuseplotmark{m6cvb};};
\node[stars] at (axis cs:{225.363},{60.204}) {\tikz\pgfuseplotmark{m6b};};
\node[stars] at (axis cs:{225.487},{40.390}) {\tikz\pgfuseplotmark{m3cv};};
\node[stars] at (axis cs:{225.775},{35.206}) {\tikz\pgfuseplotmark{m6a};};
\node[stars] at (axis cs:{225.947},{47.654}) {\tikz\pgfuseplotmark{m5av};};
\node[stars] at (axis cs:{225.991},{65.920}) {\tikz\pgfuseplotmark{m6c};};
\node[stars] at (axis cs:{226.358},{48.151}) {\tikz\pgfuseplotmark{m6a};};
\node[stars] at (axis cs:{226.570},{54.556}) {\tikz\pgfuseplotmark{m5c};};
\node[stars] at (axis cs:{226.646},{36.456}) {\tikz\pgfuseplotmark{m6c};};
\node[stars] at (axis cs:{227.081},{50.055}) {\tikz\pgfuseplotmark{m6c};};
\node[stars] at (axis cs:{227.685},{67.781}) {\tikz\pgfuseplotmark{m6c};};
\node[stars] at (axis cs:{228.398},{38.264}) {\tikz\pgfuseplotmark{m6c};};
\node[stars] at (axis cs:{228.525},{31.788}) {\tikz\pgfuseplotmark{m6bb};};
\node[stars] at (axis cs:{228.543},{42.171}) {\tikz\pgfuseplotmark{m6cv};};
\node[stars] at (axis cs:{228.621},{29.164}) {\tikz\pgfuseplotmark{m5c};};
\node[stars] at (axis cs:{228.660},{67.347}) {\tikz\pgfuseplotmark{m5c};};
\node[stars] at (axis cs:{228.876},{33.315}) {\tikz\pgfuseplotmark{m3cv};};
\node[stars] at (axis cs:{228.961},{50.938}) {\tikz\pgfuseplotmark{m6cb};};
\node[stars] at (axis cs:{229.275},{71.824}) {\tikz\pgfuseplotmark{m5b};};
\node[stars] at (axis cs:{229.875},{32.515}) {\tikz\pgfuseplotmark{m6c};};
\node[stars] at (axis cs:{230.021},{51.958}) {\tikz\pgfuseplotmark{m6av};};
\node[stars] at (axis cs:{230.036},{29.616}) {\tikz\pgfuseplotmark{m6av};};
\node[stars] at (axis cs:{230.174},{44.434}) {\tikz\pgfuseplotmark{m6c};};
\node[stars] at (axis cs:{230.182},{71.834}) {\tikz\pgfuseplotmark{m3bv};};
\node[stars] at (axis cs:{230.452},{32.934}) {\tikz\pgfuseplotmark{m5c};};
\node[stars] at (axis cs:{230.655},{62.047}) {\tikz\pgfuseplotmark{m6b};};
\node[stars] at (axis cs:{230.656},{39.581}) {\tikz\pgfuseplotmark{m6a};};
\node[stars] at (axis cs:{230.660},{63.341}) {\tikz\pgfuseplotmark{m6a};};
\node[stars] at (axis cs:{230.801},{30.288}) {\tikz\pgfuseplotmark{m5bv};};
\node[stars] at (axis cs:{231.021},{45.271}) {\tikz\pgfuseplotmark{m6b};};
\node[stars] at (axis cs:{231.123},{37.377}) {\tikz\pgfuseplotmark{m4cvb};};
\node[stars] at (axis cs:{231.232},{58.966}) {\tikz\pgfuseplotmark{m3cv};};
\node[stars] at (axis cs:{231.572},{34.336}) {\tikz\pgfuseplotmark{m5c};};
\node[stars] at (axis cs:{231.634},{54.020}) {\tikz\pgfuseplotmark{m6c};};
\node[stars] at (axis cs:{231.957},{29.106}) {\tikz\pgfuseplotmark{m4av};};
\node[stars] at (axis cs:{231.964},{60.670}) {\tikz\pgfuseplotmark{m6b};};
\node[stars] at (axis cs:{232.185},{47.201}) {\tikz\pgfuseplotmark{m6c};};
\node[stars] at (axis cs:{232.237},{55.195}) {\tikz\pgfuseplotmark{m6c};};
\node[stars] at (axis cs:{232.338},{62.099}) {\tikz\pgfuseplotmark{m6c};};
\node[stars] at (axis cs:{232.595},{31.286}) {\tikz\pgfuseplotmark{m6c};};
\node[stars] at (axis cs:{232.732},{40.833}) {\tikz\pgfuseplotmark{m5b};};
\node[stars] at (axis cs:{232.732},{64.209}) {\tikz\pgfuseplotmark{m6a};};
\node[stars] at (axis cs:{232.843},{36.616}) {\tikz\pgfuseplotmark{m6c};};
\node[stars] at (axis cs:{232.854},{77.349}) {\tikz\pgfuseplotmark{m5b};};
\node[stars] at (axis cs:{232.946},{40.899}) {\tikz\pgfuseplotmark{m5b};};
\node[stars] at (axis cs:{233.232},{31.359}) {\tikz\pgfuseplotmark{m4cv};};
\node[stars] at (axis cs:{233.812},{39.010}) {\tikz\pgfuseplotmark{m5cv};};
\node[stars] at (axis cs:{233.817},{53.922}) {\tikz\pgfuseplotmark{m6b};};
\node[stars] at (axis cs:{233.955},{38.374}) {\tikz\pgfuseplotmark{m6c};};
\node[stars] at (axis cs:{233.988},{54.630}) {\tikz\pgfuseplotmark{m6a};};
\node[stars] at (axis cs:{234.223},{29.991}) {\tikz\pgfuseplotmark{m6c};};
\node[stars] at (axis cs:{234.383},{54.509}) {\tikz\pgfuseplotmark{m6a};};
\node[stars] at (axis cs:{234.413},{69.283}) {\tikz\pgfuseplotmark{m6a};};
\node[stars] at (axis cs:{234.457},{40.353}) {\tikz\pgfuseplotmark{m5c};};
\node[stars] at (axis cs:{234.568},{46.798}) {\tikz\pgfuseplotmark{m6a};};
\node[stars] at (axis cs:{234.642},{50.423}) {\tikz\pgfuseplotmark{m6a};};
\node[stars] at (axis cs:{234.704},{34.675}) {\tikz\pgfuseplotmark{m6b};};
\node[stars] at (axis cs:{234.790},{57.924}) {\tikz\pgfuseplotmark{m6cv};};
\node[stars] at (axis cs:{234.844},{36.636}) {\tikz\pgfuseplotmark{m5bb};};
\node[stars] at (axis cs:{235.711},{52.361}) {\tikz\pgfuseplotmark{m5cv};};
\node[stars] at (axis cs:{235.997},{32.516}) {\tikz\pgfuseplotmark{m6a};};
\node[stars] at (axis cs:{236.015},{77.794}) {\tikz\pgfuseplotmark{m4c};};
\node[stars] at (axis cs:{236.645},{55.475}) {\tikz\pgfuseplotmark{m6b};};
\node[stars] at (axis cs:{236.667},{62.599}) {\tikz\pgfuseplotmark{m5c};};
\node[stars] at (axis cs:{236.908},{55.376}) {\tikz\pgfuseplotmark{m6bb};};
\node[stars] at (axis cs:{237.007},{31.736}) {\tikz\pgfuseplotmark{m6c};};
\node[stars] at (axis cs:{237.143},{28.157}) {\tikz\pgfuseplotmark{m6bv};};
\node[stars] at (axis cs:{237.808},{35.657}) {\tikz\pgfuseplotmark{m5a};};
\node[stars] at (axis cs:{238.069},{55.827}) {\tikz\pgfuseplotmark{m6a};};
\node[stars] at (axis cs:{238.169},{42.451}) {\tikz\pgfuseplotmark{m5av};};
\node[stars] at (axis cs:{238.658},{43.138}) {\tikz\pgfuseplotmark{m5cv};};
\node[stars] at (axis cs:{238.877},{42.566}) {\tikz\pgfuseplotmark{m6av};};
\node[stars] at (axis cs:{238.948},{37.947}) {\tikz\pgfuseplotmark{m5c};};
\node[stars] at (axis cs:{238.957},{58.912}) {\tikz\pgfuseplotmark{m6cv};};
\node[stars] at (axis cs:{239.374},{39.695}) {\tikz\pgfuseplotmark{m6c};};
\node[stars] at (axis cs:{239.448},{54.749}) {\tikz\pgfuseplotmark{m5bv};};
\node[stars] at (axis cs:{239.740},{36.644}) {\tikz\pgfuseplotmark{m6a};};
\node[stars] at (axis cs:{239.768},{49.881}) {\tikz\pgfuseplotmark{m6b};};
\node[stars] at (axis cs:{240.261},{33.303}) {\tikz\pgfuseplotmark{m5c};};
\node[stars] at (axis cs:{240.361},{29.851}) {\tikz\pgfuseplotmark{m5bv};};
\node[stars] at (axis cs:{240.472},{58.565}) {\tikz\pgfuseplotmark{m4b};};
\node[stars] at (axis cs:{240.523},{52.916}) {\tikz\pgfuseplotmark{m6b};};
\node[stars] at (axis cs:{240.663},{47.240}) {\tikz\pgfuseplotmark{m6cv};};
\node[stars] at (axis cs:{240.700},{46.037}) {\tikz\pgfuseplotmark{m5a};};
\node[stars] at (axis cs:{240.789},{59.411}) {\tikz\pgfuseplotmark{m6c};};
\node[stars] at (axis cs:{240.831},{36.632}) {\tikz\pgfuseplotmark{m6a};};
\node[stars] at (axis cs:{240.881},{76.794}) {\tikz\pgfuseplotmark{m6a};};
\node[stars] at (axis cs:{241.582},{67.810}) {\tikz\pgfuseplotmark{m5c};};
\node[stars] at (axis cs:{242.192},{44.935}) {\tikz\pgfuseplotmark{m4cv};};
\node[stars] at (axis cs:{242.243},{36.491}) {\tikz\pgfuseplotmark{m5av};};
\node[stars] at (axis cs:{242.262},{57.938}) {\tikz\pgfuseplotmark{m6c};};
\node[stars] at (axis cs:{242.358},{55.829}) {\tikz\pgfuseplotmark{m6c};};
\node[stars] at (axis cs:{242.706},{75.877}) {\tikz\pgfuseplotmark{m5c};};
\node[stars] at (axis cs:{242.915},{33.343}) {\tikz\pgfuseplotmark{m6cb};};
\node[stars] at (axis cs:{242.948},{42.374}) {\tikz\pgfuseplotmark{m6b};};
\node[stars] at (axis cs:{242.950},{36.425}) {\tikz\pgfuseplotmark{m6a};};
\node[stars] at (axis cs:{243.027},{39.055}) {\tikz\pgfuseplotmark{m6c};};
\node[stars] at (axis cs:{243.106},{67.144}) {\tikz\pgfuseplotmark{m6c};};
\node[stars] at (axis cs:{243.134},{75.211}) {\tikz\pgfuseplotmark{m6c};};
\node[stars] at (axis cs:{243.639},{73.395}) {\tikz\pgfuseplotmark{m6b};};
\node[stars] at (axis cs:{243.670},{33.858}) {\tikz\pgfuseplotmark{m6avb};};
\node[stars] at (axis cs:{243.947},{27.422}) {\tikz\pgfuseplotmark{m6c};};
\node[stars] at (axis cs:{244.187},{29.150}) {\tikz\pgfuseplotmark{m6avb};};
\node[stars] at (axis cs:{244.314},{59.755}) {\tikz\pgfuseplotmark{m5cv};};
\node[stars] at (axis cs:{244.376},{75.755}) {\tikz\pgfuseplotmark{m5b};};
\node[stars] at (axis cs:{244.541},{68.554}) {\tikz\pgfuseplotmark{m6c};};
\node[stars] at (axis cs:{244.797},{49.038}) {\tikz\pgfuseplotmark{m6b};};
\node[stars] at (axis cs:{244.935},{46.313}) {\tikz\pgfuseplotmark{m4bv};};
\node[stars] at (axis cs:{244.980},{39.708}) {\tikz\pgfuseplotmark{m5c};};
\node[stars] at (axis cs:{245.453},{69.109}) {\tikz\pgfuseplotmark{m5c};};
\node[stars] at (axis cs:{245.524},{30.892}) {\tikz\pgfuseplotmark{m5av};};
\node[stars] at (axis cs:{245.589},{33.799}) {\tikz\pgfuseplotmark{m5cvb};};
\node[stars] at (axis cs:{245.735},{32.333}) {\tikz\pgfuseplotmark{m6cb};};
\node[stars] at (axis cs:{245.946},{61.696}) {\tikz\pgfuseplotmark{m6av};};
\node[stars] at (axis cs:{245.998},{61.514}) {\tikz\pgfuseplotmark{m3av};};
\node[stars] at (axis cs:{246.106},{55.205}) {\tikz\pgfuseplotmark{m6av};};
\node[stars] at (axis cs:{246.351},{37.394}) {\tikz\pgfuseplotmark{m6a};};
\node[stars] at (axis cs:{246.430},{78.964}) {\tikz\pgfuseplotmark{m6a};};
\node[stars] at (axis cs:{246.996},{68.768}) {\tikz\pgfuseplotmark{m5b};};
\node[stars] at (axis cs:{247.161},{41.881}) {\tikz\pgfuseplotmark{m5bv};};
\node[stars] at (axis cs:{247.181},{51.408}) {\tikz\pgfuseplotmark{m6c};};
\node[stars] at (axis cs:{247.525},{48.961}) {\tikz\pgfuseplotmark{m6c};};
\node[stars] at (axis cs:{247.662},{77.446}) {\tikz\pgfuseplotmark{m6c};};
\node[stars] at (axis cs:{247.762},{35.225}) {\tikz\pgfuseplotmark{m6c};};
\node[stars] at (axis cs:{247.868},{72.612}) {\tikz\pgfuseplotmark{m6c};};
\node[stars] at (axis cs:{247.947},{45.598}) {\tikz\pgfuseplotmark{m6ab};};
\node[stars] at (axis cs:{248.107},{60.823}) {\tikz\pgfuseplotmark{m6b};};
\node[stars] at (axis cs:{248.526},{42.437}) {\tikz\pgfuseplotmark{m4cv};};
\node[stars] at (axis cs:{249.047},{46.613}) {\tikz\pgfuseplotmark{m6a};};
\node[stars] at (axis cs:{249.057},{52.924}) {\tikz\pgfuseplotmark{m5bb};};
\node[stars] at (axis cs:{249.229},{63.073}) {\tikz\pgfuseplotmark{m6c};};
\node[stars] at (axis cs:{249.470},{78.919}) {\tikz\pgfuseplotmark{m6c};};
\node[stars] at (axis cs:{249.502},{56.015}) {\tikz\pgfuseplotmark{m5c};};
\node[stars] at (axis cs:{249.687},{48.928}) {\tikz\pgfuseplotmark{m5bv};};
\node[stars] at (axis cs:{250.230},{64.589}) {\tikz\pgfuseplotmark{m5a};};
\node[stars] at (axis cs:{250.322},{31.603}) {\tikz\pgfuseplotmark{m3avb};};
\node[stars] at (axis cs:{250.616},{49.936}) {\tikz\pgfuseplotmark{m6c};};
\node[stars] at (axis cs:{250.724},{38.922}) {\tikz\pgfuseplotmark{m3cv};};
\node[stars] at (axis cs:{250.744},{55.690}) {\tikz\pgfuseplotmark{m6cv};};
\node[stars] at (axis cs:{250.776},{77.514}) {\tikz\pgfuseplotmark{m6b};};
\node[stars] at (axis cs:{250.965},{34.039}) {\tikz\pgfuseplotmark{m6b};};
\node[stars] at (axis cs:{251.299},{43.217}) {\tikz\pgfuseplotmark{m6b};};
\node[stars] at (axis cs:{251.324},{56.782}) {\tikz\pgfuseplotmark{m5a};};
\node[stars] at (axis cs:{251.493},{82.037}) {\tikz\pgfuseplotmark{m4cv};};
\node[stars] at (axis cs:{251.832},{42.239}) {\tikz\pgfuseplotmark{m6bv};};
\node[stars] at (axis cs:{252.309},{45.983}) {\tikz\pgfuseplotmark{m5avb};};
\node[stars] at (axis cs:{252.419},{43.430}) {\tikz\pgfuseplotmark{m6c};};
\node[stars] at (axis cs:{252.650},{41.896}) {\tikz\pgfuseplotmark{m6c};};
\node[stars] at (axis cs:{252.662},{29.806}) {\tikz\pgfuseplotmark{m6a};};
\node[stars] at (axis cs:{252.680},{32.554}) {\tikz\pgfuseplotmark{m6c};};
\node[stars] at (axis cs:{253.242},{31.702}) {\tikz\pgfuseplotmark{m5c};};
\node[stars] at (axis cs:{253.323},{47.417}) {\tikz\pgfuseplotmark{m6b};};
\node[stars] at (axis cs:{254.007},{65.135}) {\tikz\pgfuseplotmark{m5bv};};
\node[stars] at (axis cs:{254.070},{73.128}) {\tikz\pgfuseplotmark{m6c};};
\node[stars] at (axis cs:{254.105},{65.039}) {\tikz\pgfuseplotmark{m6cvb};};
\node[stars] at (axis cs:{254.459},{42.512}) {\tikz\pgfuseplotmark{m6c};};
\node[stars] at (axis cs:{254.761},{69.186}) {\tikz\pgfuseplotmark{m6c};};
\node[stars] at (axis cs:{254.840},{56.689}) {\tikz\pgfuseplotmark{m6b};};
\node[stars] at (axis cs:{255.072},{30.926}) {\tikz\pgfuseplotmark{m4b};};
\node[stars] at (axis cs:{255.321},{60.649}) {\tikz\pgfuseplotmark{m6c};};
\node[stars] at (axis cs:{255.402},{33.568}) {\tikz\pgfuseplotmark{m5cv};};
\node[stars] at (axis cs:{255.417},{75.297}) {\tikz\pgfuseplotmark{m6cv};};
\node[stars] at (axis cs:{255.566},{64.601}) {\tikz\pgfuseplotmark{m6b};};
\node[stars] at (axis cs:{255.571},{31.884}) {\tikz\pgfuseplotmark{m6c};};
\node[stars] at (axis cs:{255.876},{35.414}) {\tikz\pgfuseplotmark{m6cv};};
\node[stars] at (axis cs:{255.973},{34.790}) {\tikz\pgfuseplotmark{m6b};};
\node[stars] at (axis cs:{256.207},{48.804}) {\tikz\pgfuseplotmark{m6bv};};
\node[stars] at (axis cs:{256.271},{43.812}) {\tikz\pgfuseplotmark{m6c};};
\node[stars] at (axis cs:{256.334},{54.470}) {\tikz\pgfuseplotmark{m6ab};};
\node[stars] at (axis cs:{256.944},{40.516}) {\tikz\pgfuseplotmark{m6c};};
\node[stars] at (axis cs:{257.003},{31.206}) {\tikz\pgfuseplotmark{m6c};};
\node[stars] at (axis cs:{257.009},{35.935}) {\tikz\pgfuseplotmark{m5c};};
\node[stars] at (axis cs:{257.071},{50.842}) {\tikz\pgfuseplotmark{m6c};};
\node[stars] at (axis cs:{257.197},{65.714}) {\tikz\pgfuseplotmark{m3c};};
\node[stars] at (axis cs:{257.389},{40.777}) {\tikz\pgfuseplotmark{m5b};};
\node[stars] at (axis cs:{257.628},{52.409}) {\tikz\pgfuseplotmark{m6c};};
\node[stars] at (axis cs:{257.918},{49.746}) {\tikz\pgfuseplotmark{m6cb};};
\node[stars] at (axis cs:{258.136},{62.874}) {\tikz\pgfuseplotmark{m6a};};
\node[stars] at (axis cs:{258.762},{36.809}) {\tikz\pgfuseplotmark{m3cv};};
\node[stars] at (axis cs:{259.122},{60.671}) {\tikz\pgfuseplotmark{m6cv};};
\node[stars] at (axis cs:{259.235},{89.038}) {\tikz\pgfuseplotmark{m6cv};};
\node[stars] at (axis cs:{259.332},{33.100}) {\tikz\pgfuseplotmark{m5av};};
\node[stars] at (axis cs:{259.418},{37.291}) {\tikz\pgfuseplotmark{m5av};};
\node[stars] at (axis cs:{259.597},{38.811}) {\tikz\pgfuseplotmark{m6b};};
\node[stars] at (axis cs:{259.702},{28.823}) {\tikz\pgfuseplotmark{m6a};};
\node[stars] at (axis cs:{259.905},{80.136}) {\tikz\pgfuseplotmark{m6a};};
\node[stars] at (axis cs:{260.088},{46.241}) {\tikz\pgfuseplotmark{m5cv};};
\node[stars] at (axis cs:{260.141},{48.189}) {\tikz\pgfuseplotmark{m6c};};
\node[stars] at (axis cs:{260.165},{32.468}) {\tikz\pgfuseplotmark{m5cvb};};
\node[stars] at (axis cs:{260.380},{28.758}) {\tikz\pgfuseplotmark{m6c};};
\node[stars] at (axis cs:{260.432},{39.975}) {\tikz\pgfuseplotmark{m6av};};
\node[stars] at (axis cs:{260.439},{53.420}) {\tikz\pgfuseplotmark{m6a};};
\node[stars] at (axis cs:{260.921},{37.146}) {\tikz\pgfuseplotmark{m5avb};};
\node[stars] at (axis cs:{261.113},{36.952}) {\tikz\pgfuseplotmark{m6cb};};
\node[stars] at (axis cs:{261.250},{67.307}) {\tikz\pgfuseplotmark{m6c};};
\node[stars] at (axis cs:{261.350},{52.791}) {\tikz\pgfuseplotmark{m6c};};
\node[stars] at (axis cs:{261.422},{60.048}) {\tikz\pgfuseplotmark{m6a};};
\node[stars] at (axis cs:{261.684},{48.260}) {\tikz\pgfuseplotmark{m6a};};
\node[stars] at (axis cs:{261.692},{34.696}) {\tikz\pgfuseplotmark{m6b};};
\node[stars] at (axis cs:{262.608},{52.301}) {\tikz\pgfuseplotmark{m3ab};};
\node[stars] at (axis cs:{262.668},{38.882}) {\tikz\pgfuseplotmark{m6c};};
\node[stars] at (axis cs:{262.682},{57.877}) {\tikz\pgfuseplotmark{m6c};};
\node[stars] at (axis cs:{262.698},{86.968}) {\tikz\pgfuseplotmark{m6a};};
\node[stars] at (axis cs:{262.731},{31.158}) {\tikz\pgfuseplotmark{m6a};};
\node[stars] at (axis cs:{262.957},{28.407}) {\tikz\pgfuseplotmark{m6a};};
\node[stars] at (axis cs:{262.991},{68.135}) {\tikz\pgfuseplotmark{m5b};};
\node[stars] at (axis cs:{263.054},{86.586}) {\tikz\pgfuseplotmark{m4c};};
\node[stars] at (axis cs:{263.067},{55.173}) {\tikz\pgfuseplotmark{m5av};};
\node[stars] at (axis cs:{263.280},{41.243}) {\tikz\pgfuseplotmark{m6a};};
\node[stars] at (axis cs:{263.382},{57.559}) {\tikz\pgfuseplotmark{m6c};};
\node[stars] at (axis cs:{263.748},{61.874}) {\tikz\pgfuseplotmark{m5c};};
\node[stars] at (axis cs:{263.927},{37.302}) {\tikz\pgfuseplotmark{m6b};};
\node[stars] at (axis cs:{264.034},{28.185}) {\tikz\pgfuseplotmark{m6c};};
\node[stars] at (axis cs:{264.089},{27.566}) {\tikz\pgfuseplotmark{m6cv};};
\node[stars] at (axis cs:{264.153},{30.785}) {\tikz\pgfuseplotmark{m6b};};
\node[stars] at (axis cs:{264.157},{48.586}) {\tikz\pgfuseplotmark{m5c};};
\node[stars] at (axis cs:{264.166},{69.571}) {\tikz\pgfuseplotmark{m6c};};
\node[stars] at (axis cs:{264.238},{68.758}) {\tikz\pgfuseplotmark{m5a};};
\node[stars] at (axis cs:{264.287},{72.456}) {\tikz\pgfuseplotmark{m6b};};
\node[stars] at (axis cs:{264.866},{46.006}) {\tikz\pgfuseplotmark{m4av};};
\node[stars] at (axis cs:{264.990},{31.202}) {\tikz\pgfuseplotmark{m6b};};
\node[stars] at (axis cs:{265.156},{43.471}) {\tikz\pgfuseplotmark{m6c};};
\node[stars] at (axis cs:{265.172},{31.287}) {\tikz\pgfuseplotmark{m6cb};};
\node[stars] at (axis cs:{265.341},{51.818}) {\tikz\pgfuseplotmark{m6b};};
\node[stars] at (axis cs:{265.485},{72.149}) {\tikz\pgfuseplotmark{m5ab};};
\node[stars] at (axis cs:{265.773},{44.084}) {\tikz\pgfuseplotmark{m6c};};
\node[stars] at (axis cs:{265.997},{53.802}) {\tikz\pgfuseplotmark{m6a};};
\node[stars] at (axis cs:{266.418},{31.505}) {\tikz\pgfuseplotmark{m6c};};
\node[stars] at (axis cs:{266.615},{27.721}) {\tikz\pgfuseplotmark{m3cb};};
\node[stars] at (axis cs:{266.783},{47.612}) {\tikz\pgfuseplotmark{m6c};};
\node[stars] at (axis cs:{267.268},{50.781}) {\tikz\pgfuseplotmark{m5bv};};
\node[stars] at (axis cs:{267.363},{76.963}) {\tikz\pgfuseplotmark{m5b};};
\node[stars] at (axis cs:{267.542},{78.306}) {\tikz\pgfuseplotmark{m6cv};};
\node[stars] at (axis cs:{267.595},{29.322}) {\tikz\pgfuseplotmark{m6a};};
\node[stars] at (axis cs:{268.004},{46.643}) {\tikz\pgfuseplotmark{m6c};};
\node[stars] at (axis cs:{268.020},{39.982}) {\tikz\pgfuseplotmark{m6b};};
\node[stars] at (axis cs:{268.325},{40.008}) {\tikz\pgfuseplotmark{m5cvb};};
\node[stars] at (axis cs:{268.382},{56.873}) {\tikz\pgfuseplotmark{m4av};};
\node[stars] at (axis cs:{268.611},{75.171}) {\tikz\pgfuseplotmark{m6c};};
\node[stars] at (axis cs:{268.796},{72.005}) {\tikz\pgfuseplotmark{m5c};};
\node[stars] at (axis cs:{268.849},{55.971}) {\tikz\pgfuseplotmark{m6b};};
\node[stars] at (axis cs:{269.063},{37.251}) {\tikz\pgfuseplotmark{m4av};};
\node[stars] at (axis cs:{269.152},{51.489}) {\tikz\pgfuseplotmark{m2c};};
\node[stars] at (axis cs:{269.202},{45.351}) {\tikz\pgfuseplotmark{m6cv};};
\node[stars] at (axis cs:{269.441},{29.248}) {\tikz\pgfuseplotmark{m4av};};
\node[stars] at (axis cs:{269.626},{30.189}) {\tikz\pgfuseplotmark{m4cv};};
\node[stars] at (axis cs:{269.676},{36.288}) {\tikz\pgfuseplotmark{m6b};};
\node[stars] at (axis cs:{269.718},{45.476}) {\tikz\pgfuseplotmark{m6cv};};
\node[stars] at (axis cs:{269.984},{45.501}) {\tikz\pgfuseplotmark{m6a};};
\node[stars] at (axis cs:{270.038},{80.004}) {\tikz\pgfuseplotmark{m6ab};};
\node[stars] at (axis cs:{270.152},{33.213}) {\tikz\pgfuseplotmark{m6b};};
\node[stars] at (axis cs:{270.400},{33.311}) {\tikz\pgfuseplotmark{m6c};};
\node[stars] at (axis cs:{270.787},{48.464}) {\tikz\pgfuseplotmark{m6cb};};
\node[stars] at (axis cs:{271.253},{41.946}) {\tikz\pgfuseplotmark{m6c};};
\node[stars] at (axis cs:{271.457},{32.230}) {\tikz\pgfuseplotmark{m6a};};
\node[stars] at (axis cs:{271.723},{50.823}) {\tikz\pgfuseplotmark{m6c};};
\node[stars] at (axis cs:{271.756},{30.562}) {\tikz\pgfuseplotmark{m5bb};};
\node[stars] at (axis cs:{271.870},{43.462}) {\tikz\pgfuseplotmark{m5b};};
\node[stars] at (axis cs:{271.886},{28.762}) {\tikz\pgfuseplotmark{m4av};};
\node[stars] at (axis cs:{272.009},{36.401}) {\tikz\pgfuseplotmark{m5c};};
\node[stars] at (axis cs:{272.293},{30.469}) {\tikz\pgfuseplotmark{m6c};};
\node[stars] at (axis cs:{272.406},{38.458}) {\tikz\pgfuseplotmark{m6cv};};
\node[stars] at (axis cs:{272.496},{36.466}) {\tikz\pgfuseplotmark{m6a};};
\node[stars] at (axis cs:{272.632},{54.286}) {\tikz\pgfuseplotmark{m6b};};
\node[stars] at (axis cs:{272.938},{33.447}) {\tikz\pgfuseplotmark{m6bv};};
\node[stars] at (axis cs:{272.976},{31.405}) {\tikz\pgfuseplotmark{m5bv};};
\node[stars] at (axis cs:{273.178},{41.147}) {\tikz\pgfuseplotmark{m6c};};
\node[stars] at (axis cs:{273.270},{38.773}) {\tikz\pgfuseplotmark{m6b};};
\node[stars] at (axis cs:{273.474},{64.397}) {\tikz\pgfuseplotmark{m5b};};
\node[stars] at (axis cs:{273.821},{68.756}) {\tikz\pgfuseplotmark{m6b};};
\node[stars] at (axis cs:{273.885},{45.209}) {\tikz\pgfuseplotmark{m6c};};
\node[stars] at (axis cs:{273.912},{42.159}) {\tikz\pgfuseplotmark{m6a};};
\node[stars] at (axis cs:{274.278},{40.937}) {\tikz\pgfuseplotmark{m6b};};
\node[stars] at (axis cs:{274.965},{36.064}) {\tikz\pgfuseplotmark{m4cv};};
\node[stars] at (axis cs:{274.967},{29.666}) {\tikz\pgfuseplotmark{m6b};};
\node[stars] at (axis cs:{274.984},{51.348}) {\tikz\pgfuseplotmark{m6c};};
\node[stars] at (axis cs:{275.189},{71.338}) {\tikz\pgfuseplotmark{m4cv};};
\node[stars] at (axis cs:{275.237},{29.859}) {\tikz\pgfuseplotmark{m6a};};
\node[stars] at (axis cs:{275.254},{28.870}) {\tikz\pgfuseplotmark{m5b};};
\node[stars] at (axis cs:{275.264},{72.733}) {\tikz\pgfuseplotmark{m4avb};};
\node[stars] at (axis cs:{275.280},{49.725}) {\tikz\pgfuseplotmark{m6c};};
\node[stars] at (axis cs:{275.386},{49.122}) {\tikz\pgfuseplotmark{m5bv};};
\node[stars] at (axis cs:{275.950},{53.301}) {\tikz\pgfuseplotmark{m6c};};
\node[stars] at (axis cs:{275.978},{58.801}) {\tikz\pgfuseplotmark{m5bb};};
\node[stars] at (axis cs:{275.989},{38.739}) {\tikz\pgfuseplotmark{m6cv};};
\node[stars] at (axis cs:{276.039},{83.175}) {\tikz\pgfuseplotmark{m6c};};
\node[stars] at (axis cs:{276.057},{39.507}) {\tikz\pgfuseplotmark{m5b};};
\node[stars] at (axis cs:{276.244},{27.395}) {\tikz\pgfuseplotmark{m6cb};};
\node[stars] at (axis cs:{276.495},{29.829}) {\tikz\pgfuseplotmark{m6a};};
\node[stars] at (axis cs:{276.496},{65.563}) {\tikz\pgfuseplotmark{m5a};};
\node[stars] at (axis cs:{276.926},{59.549}) {\tikz\pgfuseplotmark{m6c};};
\node[stars] at (axis cs:{277.437},{77.547}) {\tikz\pgfuseplotmark{m6a};};
\node[stars] at (axis cs:{278.144},{57.045}) {\tikz\pgfuseplotmark{m5a};};
\node[stars] at (axis cs:{278.208},{30.554}) {\tikz\pgfuseplotmark{m5c};};
\node[stars] at (axis cs:{278.486},{52.353}) {\tikz\pgfuseplotmark{m5cb};};
\node[stars] at (axis cs:{278.806},{34.458}) {\tikz\pgfuseplotmark{m6b};};
\node[stars] at (axis cs:{279.055},{65.489}) {\tikz\pgfuseplotmark{m6b};};
\node[stars] at (axis cs:{279.156},{33.469}) {\tikz\pgfuseplotmark{m5cv};};
\node[stars] at (axis cs:{279.190},{43.222}) {\tikz\pgfuseplotmark{m6cv};};
\node[stars] at (axis cs:{279.235},{38.784}) {\tikz\pgfuseplotmark{m1bvb};};
\node[stars] at (axis cs:{279.390},{62.526}) {\tikz\pgfuseplotmark{m6a};};
\node[stars] at (axis cs:{279.527},{39.668}) {\tikz\pgfuseplotmark{m6cv};};
\node[stars] at (axis cs:{279.888},{40.935}) {\tikz\pgfuseplotmark{m6c};};
\node[stars] at (axis cs:{279.970},{52.196}) {\tikz\pgfuseplotmark{m6b};};
\node[stars] at (axis cs:{280.008},{30.849}) {\tikz\pgfuseplotmark{m6c};};
\node[stars] at (axis cs:{280.051},{38.367}) {\tikz\pgfuseplotmark{m6c};};
\node[stars] at (axis cs:{280.235},{62.749}) {\tikz\pgfuseplotmark{m6b};};
\node[stars] at (axis cs:{280.422},{31.618}) {\tikz\pgfuseplotmark{m6c};};
\node[stars] at (axis cs:{280.534},{34.746}) {\tikz\pgfuseplotmark{m6cb};};
\node[stars] at (axis cs:{280.658},{55.539}) {\tikz\pgfuseplotmark{m5bvb};};
\node[stars] at (axis cs:{280.793},{70.793}) {\tikz\pgfuseplotmark{m6c};};
\node[stars] at (axis cs:{280.819},{39.300}) {\tikz\pgfuseplotmark{m6cb};};
\node[stars] at (axis cs:{280.871},{53.872}) {\tikz\pgfuseplotmark{m6c};};
\node[stars] at (axis cs:{280.900},{36.556}) {\tikz\pgfuseplotmark{m6b};};
\node[stars] at (axis cs:{280.965},{31.926}) {\tikz\pgfuseplotmark{m6a};};
\node[stars] at (axis cs:{281.076},{61.048}) {\tikz\pgfuseplotmark{m6bb};};
\node[stars] at (axis cs:{281.085},{39.670}) {\tikz\pgfuseplotmark{m5bb};};
\node[stars] at (axis cs:{281.193},{37.605}) {\tikz\pgfuseplotmark{m4cvb};};
\node[stars] at (axis cs:{281.231},{54.897}) {\tikz\pgfuseplotmark{m6c};};
\node[stars] at (axis cs:{281.409},{79.943}) {\tikz\pgfuseplotmark{m6c};};
\node[stars] at (axis cs:{281.445},{74.086}) {\tikz\pgfuseplotmark{m5c};};
\node[stars] at (axis cs:{281.554},{41.442}) {\tikz\pgfuseplotmark{m6b};};
\node[stars] at (axis cs:{281.593},{75.434}) {\tikz\pgfuseplotmark{m5c};};
\node[stars] at (axis cs:{281.680},{52.988}) {\tikz\pgfuseplotmark{m6bv};};
\node[stars] at (axis cs:{281.989},{31.757}) {\tikz\pgfuseplotmark{m6b};};
\node[stars] at (axis cs:{282.067},{48.768}) {\tikz\pgfuseplotmark{m6b};};
\node[stars] at (axis cs:{282.441},{32.813}) {\tikz\pgfuseplotmark{m6bvb};};
\node[stars] at (axis cs:{282.470},{32.551}) {\tikz\pgfuseplotmark{m5c};};
\node[stars] at (axis cs:{282.520},{33.362}) {\tikz\pgfuseplotmark{m4avb};};
\node[stars] at (axis cs:{282.800},{59.388}) {\tikz\pgfuseplotmark{m5avb};};
\node[stars] at (axis cs:{282.896},{52.975}) {\tikz\pgfuseplotmark{m6a};};
\node[stars] at (axis cs:{282.900},{28.784}) {\tikz\pgfuseplotmark{m6c};};
\node[stars] at (axis cs:{282.902},{36.539}) {\tikz\pgfuseplotmark{m6b};};
\node[stars] at (axis cs:{283.030},{41.383}) {\tikz\pgfuseplotmark{m6c};};
\node[stars] at (axis cs:{283.306},{50.708}) {\tikz\pgfuseplotmark{m5b};};
\node[stars] at (axis cs:{283.431},{36.972}) {\tikz\pgfuseplotmark{m6av};};
\node[stars] at (axis cs:{283.443},{57.487}) {\tikz\pgfuseplotmark{m6c};};
\node[stars] at (axis cs:{283.555},{27.909}) {\tikz\pgfuseplotmark{m6a};};
\node[stars] at (axis cs:{283.599},{71.297}) {\tikz\pgfuseplotmark{m5a};};
\node[stars] at (axis cs:{283.626},{36.898}) {\tikz\pgfuseplotmark{m4cvb};};
\node[stars] at (axis cs:{283.696},{48.859}) {\tikz\pgfuseplotmark{m6a};};
\node[stars] at (axis cs:{283.717},{41.603}) {\tikz\pgfuseplotmark{m5c};};
\node[stars] at (axis cs:{283.719},{33.969}) {\tikz\pgfuseplotmark{m6bb};};
\node[stars] at (axis cs:{283.834},{43.946}) {\tikz\pgfuseplotmark{m4cv};};
\node[stars] at (axis cs:{284.107},{65.258}) {\tikz\pgfuseplotmark{m6a};};
\node[stars] at (axis cs:{284.188},{57.815}) {\tikz\pgfuseplotmark{m6a};};
\node[stars] at (axis cs:{284.257},{32.901}) {\tikz\pgfuseplotmark{m5cb};};
\node[stars] at (axis cs:{284.322},{62.397}) {\tikz\pgfuseplotmark{m6cb};};
\node[stars] at (axis cs:{284.694},{40.679}) {\tikz\pgfuseplotmark{m6cvb};};
\node[stars] at (axis cs:{284.736},{32.689}) {\tikz\pgfuseplotmark{m3cv};};
\node[stars] at (axis cs:{284.747},{50.809}) {\tikz\pgfuseplotmark{m6c};};
\node[stars] at (axis cs:{284.801},{39.217}) {\tikz\pgfuseplotmark{m6cv};};
\node[stars] at (axis cs:{285.003},{32.145}) {\tikz\pgfuseplotmark{m5bv};};
\node[stars] at (axis cs:{285.057},{50.533}) {\tikz\pgfuseplotmark{m5c};};
\node[stars] at (axis cs:{285.181},{55.658}) {\tikz\pgfuseplotmark{m6a};};
\node[stars] at (axis cs:{285.230},{33.802}) {\tikz\pgfuseplotmark{m6b};};
\node[stars] at (axis cs:{285.360},{46.935}) {\tikz\pgfuseplotmark{m5bv};};
\node[stars] at (axis cs:{285.452},{33.621}) {\tikz\pgfuseplotmark{m6c};};
\node[stars] at (axis cs:{285.529},{52.261}) {\tikz\pgfuseplotmark{m6cb};};
\node[stars] at (axis cs:{286.230},{53.396}) {\tikz\pgfuseplotmark{m5c};};
\node[stars] at (axis cs:{286.241},{31.744}) {\tikz\pgfuseplotmark{m6a};};
\node[stars] at (axis cs:{286.243},{30.733}) {\tikz\pgfuseplotmark{m6b};};
\node[stars] at (axis cs:{286.291},{49.923}) {\tikz\pgfuseplotmark{m6c};};
\node[stars] at (axis cs:{286.446},{29.922}) {\tikz\pgfuseplotmark{m6c};};
\node[stars] at (axis cs:{286.657},{28.628}) {\tikz\pgfuseplotmark{m6a};};
\node[stars] at (axis cs:{286.826},{36.100}) {\tikz\pgfuseplotmark{m5cv};};
\node[stars] at (axis cs:{286.857},{32.502}) {\tikz\pgfuseplotmark{m5cb};};
\node[stars] at (axis cs:{287.107},{52.426}) {\tikz\pgfuseplotmark{m6bv};};
\node[stars] at (axis cs:{287.291},{76.560}) {\tikz\pgfuseplotmark{m5bv};};
\node[stars] at (axis cs:{287.441},{65.978}) {\tikz\pgfuseplotmark{m6c};};
\node[stars] at (axis cs:{287.847},{40.429}) {\tikz\pgfuseplotmark{m6c};};
\node[stars] at (axis cs:{287.919},{56.859}) {\tikz\pgfuseplotmark{m5cv};};
\node[stars] at (axis cs:{287.942},{31.283}) {\tikz\pgfuseplotmark{m6bv};};
\node[stars] at (axis cs:{288.139},{67.661}) {\tikz\pgfuseplotmark{m3b};};
\node[stars] at (axis cs:{288.440},{39.146}) {\tikz\pgfuseplotmark{m4cv};};
\node[stars] at (axis cs:{288.480},{57.705}) {\tikz\pgfuseplotmark{m5b};};
\node[stars] at (axis cs:{288.830},{50.071}) {\tikz\pgfuseplotmark{m6cb};};
\node[stars] at (axis cs:{288.854},{30.526}) {\tikz\pgfuseplotmark{m6bv};};
\node[stars] at (axis cs:{288.888},{73.355}) {\tikz\pgfuseplotmark{m4c};};
\node[stars] at (axis cs:{288.998},{27.926}) {\tikz\pgfuseplotmark{m6cv};};
\node[stars] at (axis cs:{289.092},{38.134}) {\tikz\pgfuseplotmark{m4cv};};
\node[stars] at (axis cs:{289.214},{46.999}) {\tikz\pgfuseplotmark{m6b};};
\node[stars] at (axis cs:{289.276},{53.368}) {\tikz\pgfuseplotmark{m4av};};
\node[stars] at (axis cs:{289.658},{49.569}) {\tikz\pgfuseplotmark{m6c};};
\node[stars] at (axis cs:{289.755},{37.445}) {\tikz\pgfuseplotmark{m6c};};
\node[stars] at (axis cs:{289.902},{54.376}) {\tikz\pgfuseplotmark{m6c};};
\node[stars] at (axis cs:{289.912},{37.330}) {\tikz\pgfuseplotmark{m6c};};
\node[stars] at (axis cs:{290.067},{57.645}) {\tikz\pgfuseplotmark{m6b};};
\node[stars] at (axis cs:{290.138},{35.186}) {\tikz\pgfuseplotmark{m6cb};};
\node[stars] at (axis cs:{290.167},{65.714}) {\tikz\pgfuseplotmark{m5a};};
\node[stars] at (axis cs:{290.398},{76.559}) {\tikz\pgfuseplotmark{m6cv};};
\node[stars] at (axis cs:{290.418},{79.603}) {\tikz\pgfuseplotmark{m6b};};
\node[stars] at (axis cs:{290.639},{33.518}) {\tikz\pgfuseplotmark{m6b};};
\node[stars] at (axis cs:{290.892},{33.222}) {\tikz\pgfuseplotmark{m6c};};
\node[stars] at (axis cs:{290.985},{43.388}) {\tikz\pgfuseplotmark{m6a};};
\node[stars] at (axis cs:{291.025},{36.452}) {\tikz\pgfuseplotmark{m6c};};
\node[stars] at (axis cs:{291.032},{29.621}) {\tikz\pgfuseplotmark{m5b};};
\node[stars] at (axis cs:{291.138},{50.241}) {\tikz\pgfuseplotmark{m6cv};};
\node[stars] at (axis cs:{291.538},{36.318}) {\tikz\pgfuseplotmark{m5cv};};
\node[stars] at (axis cs:{291.610},{62.557}) {\tikz\pgfuseplotmark{m6c};};
\node[stars] at (axis cs:{291.858},{52.320}) {\tikz\pgfuseplotmark{m6a};};
\node[stars] at (axis cs:{291.902},{37.941}) {\tikz\pgfuseplotmark{m6cv};};
\node[stars] at (axis cs:{292.426},{51.730}) {\tikz\pgfuseplotmark{m4a};};
\node[stars] at (axis cs:{292.680},{27.960}) {\tikz\pgfuseplotmark{m3bvb};};
\node[stars] at (axis cs:{292.695},{36.228}) {\tikz\pgfuseplotmark{m6c};};
\node[stars] at (axis cs:{292.751},{70.989}) {\tikz\pgfuseplotmark{m6b};};
\node[stars] at (axis cs:{292.806},{55.732}) {\tikz\pgfuseplotmark{m6cv};};
\node[stars] at (axis cs:{292.831},{50.307}) {\tikz\pgfuseplotmark{m6a};};
\node[stars] at (axis cs:{292.943},{34.453}) {\tikz\pgfuseplotmark{m5a};};
\node[stars] at (axis cs:{293.090},{69.661}) {\tikz\pgfuseplotmark{m5av};};
\node[stars] at (axis cs:{293.292},{60.159}) {\tikz\pgfuseplotmark{m6cb};};
\node[stars] at (axis cs:{293.423},{49.262}) {\tikz\pgfuseplotmark{m6bv};};
\node[stars] at (axis cs:{293.582},{51.237}) {\tikz\pgfuseplotmark{m6av};};
\node[stars] at (axis cs:{293.672},{42.412}) {\tikz\pgfuseplotmark{m5c};};
\node[stars] at (axis cs:{293.693},{63.434}) {\tikz\pgfuseplotmark{m6c};};
\node[stars] at (axis cs:{293.712},{29.463}) {\tikz\pgfuseplotmark{m5cv};};
\node[stars] at (axis cs:{293.755},{52.502}) {\tikz\pgfuseplotmark{m6c};};
\node[stars] at (axis cs:{293.951},{36.944}) {\tikz\pgfuseplotmark{m6b};};
\node[stars] at (axis cs:{293.983},{50.239}) {\tikz\pgfuseplotmark{m6c};};
\node[stars] at (axis cs:{294.111},{50.221}) {\tikz\pgfuseplotmark{m4cb};};
\node[stars] at (axis cs:{294.158},{44.695}) {\tikz\pgfuseplotmark{m5c};};
\node[stars] at (axis cs:{294.290},{29.334}) {\tikz\pgfuseplotmark{m6c};};
\node[stars] at (axis cs:{294.420},{35.022}) {\tikz\pgfuseplotmark{m6c};};
\node[stars] at (axis cs:{294.486},{49.284}) {\tikz\pgfuseplotmark{m6cb};};
\node[stars] at (axis cs:{294.672},{54.974}) {\tikz\pgfuseplotmark{m6bv};};
\node[stars] at (axis cs:{294.844},{30.153}) {\tikz\pgfuseplotmark{m5a};};
\node[stars] at (axis cs:{294.860},{42.818}) {\tikz\pgfuseplotmark{m5cv};};
\node[stars] at (axis cs:{294.894},{45.957}) {\tikz\pgfuseplotmark{m6c};};
\node[stars] at (axis cs:{294.937},{33.979}) {\tikz\pgfuseplotmark{m6bb};};
\node[stars] at (axis cs:{295.055},{60.507}) {\tikz\pgfuseplotmark{m6cb};};
\node[stars] at (axis cs:{295.171},{43.078}) {\tikz\pgfuseplotmark{m6c};};
\node[stars] at (axis cs:{295.209},{45.525}) {\tikz\pgfuseplotmark{m5bv};};
\node[stars] at (axis cs:{295.454},{50.525}) {\tikz\pgfuseplotmark{m6bb};};
\node[stars] at (axis cs:{295.490},{40.254}) {\tikz\pgfuseplotmark{m6c};};
\node[stars] at (axis cs:{295.686},{32.427}) {\tikz\pgfuseplotmark{m6b};};
\node[stars] at (axis cs:{295.790},{30.678}) {\tikz\pgfuseplotmark{m6b};};
\node[stars] at (axis cs:{295.795},{38.672}) {\tikz\pgfuseplotmark{m6c};};
\node[stars] at (axis cs:{295.810},{58.016}) {\tikz\pgfuseplotmark{m6c};};
\node[stars] at (axis cs:{295.915},{57.043}) {\tikz\pgfuseplotmark{m6c};};
\node[stars] at (axis cs:{295.938},{41.773}) {\tikz\pgfuseplotmark{m6av};};
\node[stars] at (axis cs:{295.964},{34.163}) {\tikz\pgfuseplotmark{m6b};};
\node[stars] at (axis cs:{295.983},{27.135}) {\tikz\pgfuseplotmark{m6cv};};
\node[stars] at (axis cs:{296.069},{37.354}) {\tikz\pgfuseplotmark{m5b};};
\node[stars] at (axis cs:{296.077},{69.337}) {\tikz\pgfuseplotmark{m6b};};
\node[stars] at (axis cs:{296.159},{34.414}) {\tikz\pgfuseplotmark{m6cv};};
\node[stars] at (axis cs:{296.204},{40.717}) {\tikz\pgfuseplotmark{m6cv};};
\node[stars] at (axis cs:{296.244},{45.131}) {\tikz\pgfuseplotmark{m3bvb};};
\node[stars] at (axis cs:{296.415},{36.091}) {\tikz\pgfuseplotmark{m6cb};};
\node[stars] at (axis cs:{296.464},{35.013}) {\tikz\pgfuseplotmark{m6bvb};};
\node[stars] at (axis cs:{296.607},{33.727}) {\tikz\pgfuseplotmark{m5bb};};
\node[stars] at (axis cs:{296.646},{32.889}) {\tikz\pgfuseplotmark{m6cvb};};
\node[stars] at (axis cs:{296.686},{68.438}) {\tikz\pgfuseplotmark{m6cv};};
\node[stars] at (axis cs:{296.862},{47.907}) {\tikz\pgfuseplotmark{m6c};};
\node[stars] at (axis cs:{296.866},{38.407}) {\tikz\pgfuseplotmark{m6a};};
\node[stars] at (axis cs:{297.043},{70.268}) {\tikz\pgfuseplotmark{m4avb};};
\node[stars] at (axis cs:{297.211},{33.437}) {\tikz\pgfuseplotmark{m6cv};};
\node[stars] at (axis cs:{297.365},{38.710}) {\tikz\pgfuseplotmark{m6bvb};};
\node[stars] at (axis cs:{297.478},{28.440}) {\tikz\pgfuseplotmark{m6cv};};
\node[stars] at (axis cs:{297.483},{27.085}) {\tikz\pgfuseplotmark{m6c};};
\node[stars] at (axis cs:{297.642},{38.722}) {\tikz\pgfuseplotmark{m5cvb};};
\node[stars] at (axis cs:{297.656},{40.599}) {\tikz\pgfuseplotmark{m6av};};
\node[stars] at (axis cs:{297.657},{52.988}) {\tikz\pgfuseplotmark{m5bv};};
\node[stars] at (axis cs:{297.695},{37.826}) {\tikz\pgfuseplotmark{m6cv};};
\node[stars] at (axis cs:{297.831},{47.377}) {\tikz\pgfuseplotmark{m6cv};};
\node[stars] at (axis cs:{297.996},{47.027}) {\tikz\pgfuseplotmark{m6av};};
\node[stars] at (axis cs:{298.030},{47.932}) {\tikz\pgfuseplotmark{m6bv};};
\node[stars] at (axis cs:{298.068},{36.432}) {\tikz\pgfuseplotmark{m6b};};
\node[stars] at (axis cs:{298.255},{47.808}) {\tikz\pgfuseplotmark{m6cv};};
\node[stars] at (axis cs:{298.322},{57.523}) {\tikz\pgfuseplotmark{m5c};};
\node[stars] at (axis cs:{298.398},{59.709}) {\tikz\pgfuseplotmark{m6bb};};
\node[stars] at (axis cs:{298.701},{36.995}) {\tikz\pgfuseplotmark{m6a};};
\node[stars] at (axis cs:{298.842},{58.250}) {\tikz\pgfuseplotmark{m6b};};
\node[stars] at (axis cs:{298.907},{52.439}) {\tikz\pgfuseplotmark{m5bb};};
\node[stars] at (axis cs:{298.966},{38.487}) {\tikz\pgfuseplotmark{m5b};};
\node[stars] at (axis cs:{298.981},{58.846}) {\tikz\pgfuseplotmark{m5b};};
\node[stars] at (axis cs:{299.077},{35.083}) {\tikz\pgfuseplotmark{m4bvb};};
\node[stars] at (axis cs:{299.079},{56.687}) {\tikz\pgfuseplotmark{m6c};};
\node[stars] at (axis cs:{299.184},{36.251}) {\tikz\pgfuseplotmark{m6b};};
\node[stars] at (axis cs:{299.188},{50.902}) {\tikz\pgfuseplotmark{m6c};};
\node[stars] at (axis cs:{299.308},{40.368}) {\tikz\pgfuseplotmark{m5cb};};
\node[stars] at (axis cs:{299.484},{42.261}) {\tikz\pgfuseplotmark{m6cb};};
\node[stars] at (axis cs:{299.620},{63.534}) {\tikz\pgfuseplotmark{m6c};};
\node[stars] at (axis cs:{299.658},{30.984}) {\tikz\pgfuseplotmark{m6a};};
\node[stars] at (axis cs:{299.674},{70.367}) {\tikz\pgfuseplotmark{m6c};};
\node[stars] at (axis cs:{299.814},{52.056}) {\tikz\pgfuseplotmark{m6bv};};
\node[stars] at (axis cs:{299.835},{45.772}) {\tikz\pgfuseplotmark{m6b};};
\node[stars] at (axis cs:{299.980},{37.043}) {\tikz\pgfuseplotmark{m5cv};};
\node[stars] at (axis cs:{300.275},{27.754}) {\tikz\pgfuseplotmark{m5av};};
\node[stars] at (axis cs:{300.314},{37.099}) {\tikz\pgfuseplotmark{m6c};};
\node[stars] at (axis cs:{300.340},{50.105}) {\tikz\pgfuseplotmark{m5bb};};
\node[stars] at (axis cs:{300.369},{64.821}) {\tikz\pgfuseplotmark{m5c};};
\node[stars] at (axis cs:{300.584},{64.634}) {\tikz\pgfuseplotmark{m6cb};};
\node[stars] at (axis cs:{300.703},{31.959}) {\tikz\pgfuseplotmark{m6c};};
\node[stars] at (axis cs:{300.704},{67.873}) {\tikz\pgfuseplotmark{m5a};};
\node[stars] at (axis cs:{300.906},{29.897}) {\tikz\pgfuseplotmark{m6a};};
\node[stars] at (axis cs:{301.120},{48.229}) {\tikz\pgfuseplotmark{m6c};};
\node[stars] at (axis cs:{301.151},{32.219}) {\tikz\pgfuseplotmark{m6av};};
\node[stars] at (axis cs:{301.185},{63.890}) {\tikz\pgfuseplotmark{m6cb};};
\node[stars] at (axis cs:{301.222},{68.027}) {\tikz\pgfuseplotmark{m6cv};};
\node[stars] at (axis cs:{301.278},{51.839}) {\tikz\pgfuseplotmark{m6cv};};
\node[stars] at (axis cs:{301.291},{38.478}) {\tikz\pgfuseplotmark{m6c};};
\node[stars] at (axis cs:{301.340},{56.341}) {\tikz\pgfuseplotmark{m6c};};
\node[stars] at (axis cs:{301.387},{61.995}) {\tikz\pgfuseplotmark{m5c};};
\node[stars] at (axis cs:{301.558},{53.165}) {\tikz\pgfuseplotmark{m6ab};};
\node[stars] at (axis cs:{301.591},{35.972}) {\tikz\pgfuseplotmark{m5cv};};
\node[stars] at (axis cs:{301.923},{34.423}) {\tikz\pgfuseplotmark{m6c};};
\node[stars] at (axis cs:{302.222},{77.711}) {\tikz\pgfuseplotmark{m4cb};};
\node[stars] at (axis cs:{302.357},{36.840}) {\tikz\pgfuseplotmark{m5bv};};
\node[stars] at (axis cs:{302.895},{62.078}) {\tikz\pgfuseplotmark{m6a};};
\node[stars] at (axis cs:{303.132},{51.463}) {\tikz\pgfuseplotmark{m6b};};
\node[stars] at (axis cs:{303.349},{56.568}) {\tikz\pgfuseplotmark{m4cv};};
\node[stars] at (axis cs:{303.365},{60.640}) {\tikz\pgfuseplotmark{m6a};};
\node[stars] at (axis cs:{303.408},{46.741}) {\tikz\pgfuseplotmark{m4avb};};
\node[stars] at (axis cs:{303.428},{43.379}) {\tikz\pgfuseplotmark{m6b};};
\node[stars] at (axis cs:{303.520},{36.605}) {\tikz\pgfuseplotmark{m6cv};};
\node[stars] at (axis cs:{303.561},{28.695}) {\tikz\pgfuseplotmark{m5cv};};
\node[stars] at (axis cs:{303.633},{36.806}) {\tikz\pgfuseplotmark{m5bvb};};
\node[stars] at (axis cs:{303.849},{33.729}) {\tikz\pgfuseplotmark{m6a};};
\node[stars] at (axis cs:{303.868},{47.714}) {\tikz\pgfuseplotmark{m4bv};};
\node[stars] at (axis cs:{303.930},{50.233}) {\tikz\pgfuseplotmark{m6c};};
\node[stars] at (axis cs:{303.942},{27.814}) {\tikz\pgfuseplotmark{m5a};};
\node[stars] at (axis cs:{304.003},{45.579}) {\tikz\pgfuseplotmark{m6b};};
\node[stars] at (axis cs:{304.014},{38.898}) {\tikz\pgfuseplotmark{m6c};};
\node[stars] at (axis cs:{304.117},{37.056}) {\tikz\pgfuseplotmark{m6c};};
\node[stars] at (axis cs:{304.230},{40.365}) {\tikz\pgfuseplotmark{m5cv};};
\node[stars] at (axis cs:{304.371},{42.722}) {\tikz\pgfuseplotmark{m6c};};
\node[stars] at (axis cs:{304.381},{66.854}) {\tikz\pgfuseplotmark{m6b};};
\node[stars] at (axis cs:{304.381},{29.147}) {\tikz\pgfuseplotmark{m6c};};
\node[stars] at (axis cs:{304.447},{38.033}) {\tikz\pgfuseplotmark{m5av};};
\node[stars] at (axis cs:{304.529},{40.732}) {\tikz\pgfuseplotmark{m6avb};};
\node[stars] at (axis cs:{304.603},{55.397}) {\tikz\pgfuseplotmark{m6ab};};
\node[stars] at (axis cs:{304.619},{37.000}) {\tikz\pgfuseplotmark{m6a};};
\node[stars] at (axis cs:{304.663},{34.983}) {\tikz\pgfuseplotmark{m5cv};};
\node[stars] at (axis cs:{304.707},{46.322}) {\tikz\pgfuseplotmark{m6cv};};
\node[stars] at (axis cs:{304.740},{39.004}) {\tikz\pgfuseplotmark{m6cb};};
\node[stars] at (axis cs:{304.903},{62.257}) {\tikz\pgfuseplotmark{m6av};};
\node[stars] at (axis cs:{304.984},{46.837}) {\tikz\pgfuseplotmark{m6cv};};
\node[stars] at (axis cs:{305.025},{68.880}) {\tikz\pgfuseplotmark{m6av};};
\node[stars] at (axis cs:{305.063},{39.403}) {\tikz\pgfuseplotmark{m6cb};};
\node[stars] at (axis cs:{305.128},{53.595}) {\tikz\pgfuseplotmark{m6c};};
\node[stars] at (axis cs:{305.298},{63.980}) {\tikz\pgfuseplotmark{m6a};};
\node[stars] at (axis cs:{305.513},{41.132}) {\tikz\pgfuseplotmark{m6c};};
\node[stars] at (axis cs:{305.522},{45.795}) {\tikz\pgfuseplotmark{m6av};};
\node[stars] at (axis cs:{305.557},{40.257}) {\tikz\pgfuseplotmark{m2cvb};};
\node[stars] at (axis cs:{305.655},{31.265}) {\tikz\pgfuseplotmark{m6b};};
\node[stars] at (axis cs:{305.689},{41.026}) {\tikz\pgfuseplotmark{m6b};};
\node[stars] at (axis cs:{305.731},{42.984}) {\tikz\pgfuseplotmark{m6cb};};
\node[stars] at (axis cs:{305.935},{37.476}) {\tikz\pgfuseplotmark{m6bv};};
\node[stars] at (axis cs:{305.965},{32.190}) {\tikz\pgfuseplotmark{m4c};};
\node[stars] at (axis cs:{306.271},{59.600}) {\tikz\pgfuseplotmark{m6c};};
\node[stars] at (axis cs:{306.598},{56.639}) {\tikz\pgfuseplotmark{m6cb};};
\node[stars] at (axis cs:{306.759},{49.383}) {\tikz\pgfuseplotmark{m6av};};
\node[stars] at (axis cs:{306.782},{34.329}) {\tikz\pgfuseplotmark{m6c};};
\node[stars] at (axis cs:{306.893},{38.440}) {\tikz\pgfuseplotmark{m6a};};
\node[stars] at (axis cs:{307.061},{81.423}) {\tikz\pgfuseplotmark{m5cb};};
\node[stars] at (axis cs:{307.262},{83.625}) {\tikz\pgfuseplotmark{m6c};};
\node[stars] at (axis cs:{307.335},{36.455}) {\tikz\pgfuseplotmark{m6b};};
\node[stars] at (axis cs:{307.349},{30.369}) {\tikz\pgfuseplotmark{m4bv};};
\node[stars] at (axis cs:{307.363},{56.068}) {\tikz\pgfuseplotmark{m6b};};
\node[stars] at (axis cs:{307.365},{81.091}) {\tikz\pgfuseplotmark{m6bb};};
\node[stars] at (axis cs:{307.395},{62.994}) {\tikz\pgfuseplotmark{m4c};};
\node[stars] at (axis cs:{307.500},{45.928}) {\tikz\pgfuseplotmark{m6c};};
\node[stars] at (axis cs:{307.503},{72.531}) {\tikz\pgfuseplotmark{m6c};};
\node[stars] at (axis cs:{307.515},{48.952}) {\tikz\pgfuseplotmark{m5bv};};
\node[stars] at (axis cs:{307.747},{36.936}) {\tikz\pgfuseplotmark{m6cb};};
\node[stars] at (axis cs:{307.828},{49.220}) {\tikz\pgfuseplotmark{m5c};};
\node[stars] at (axis cs:{307.838},{52.309}) {\tikz\pgfuseplotmark{m6c};};
\node[stars] at (axis cs:{307.877},{74.955}) {\tikz\pgfuseplotmark{m5cv};};
\node[stars] at (axis cs:{307.901},{34.330}) {\tikz\pgfuseplotmark{m6c};};
\node[stars] at (axis cs:{307.944},{56.780}) {\tikz\pgfuseplotmark{m6c};};
\node[stars] at (axis cs:{308.476},{35.251}) {\tikz\pgfuseplotmark{m5av};};
\node[stars] at (axis cs:{308.479},{46.694}) {\tikz\pgfuseplotmark{m6avb};};
\node[stars] at (axis cs:{308.710},{51.854}) {\tikz\pgfuseplotmark{m6c};};
\node[stars] at (axis cs:{309.348},{38.328}) {\tikz\pgfuseplotmark{m6c};};
\node[stars] at (axis cs:{309.382},{31.572}) {\tikz\pgfuseplotmark{m6cb};};
\node[stars] at (axis cs:{309.748},{30.334}) {\tikz\pgfuseplotmark{m6a};};
\node[stars] at (axis cs:{309.751},{56.005}) {\tikz\pgfuseplotmark{m6c};};
\node[stars] at (axis cs:{309.889},{40.579}) {\tikz\pgfuseplotmark{m6bvb};};
\node[stars] at (axis cs:{310.013},{43.459}) {\tikz\pgfuseplotmark{m6b};};
\node[stars] at (axis cs:{310.075},{60.505}) {\tikz\pgfuseplotmark{m6b};};
\node[stars] at (axis cs:{310.151},{29.805}) {\tikz\pgfuseplotmark{m6b};};
\node[stars] at (axis cs:{310.261},{32.307}) {\tikz\pgfuseplotmark{m6ab};};
\node[stars] at (axis cs:{310.358},{45.280}) {\tikz\pgfuseplotmark{m1cv};};
\node[stars] at (axis cs:{310.485},{41.717}) {\tikz\pgfuseplotmark{m6a};};
\node[stars] at (axis cs:{310.553},{50.340}) {\tikz\pgfuseplotmark{m5c};};
\node[stars] at (axis cs:{310.647},{82.531}) {\tikz\pgfuseplotmark{m6a};};
\node[stars] at (axis cs:{310.665},{60.601}) {\tikz\pgfuseplotmark{m6c};};
\node[stars] at (axis cs:{310.796},{66.657}) {\tikz\pgfuseplotmark{m6a};};
\node[stars] at (axis cs:{310.807},{57.114}) {\tikz\pgfuseplotmark{m6cv};};
\node[stars] at (axis cs:{310.851},{35.588}) {\tikz\pgfuseplotmark{m6cv};};
\node[stars] at (axis cs:{311.092},{56.488}) {\tikz\pgfuseplotmark{m6bv};};
\node[stars] at (axis cs:{311.322},{61.839}) {\tikz\pgfuseplotmark{m3c};};
\node[stars] at (axis cs:{311.338},{57.580}) {\tikz\pgfuseplotmark{m5a};};
\node[stars] at (axis cs:{311.416},{30.720}) {\tikz\pgfuseplotmark{m4c};};
\node[stars] at (axis cs:{311.553},{33.970}) {\tikz\pgfuseplotmark{m2c};};
\node[stars] at (axis cs:{311.588},{52.995}) {\tikz\pgfuseplotmark{m6c};};
\node[stars] at (axis cs:{311.661},{46.531}) {\tikz\pgfuseplotmark{m6c};};
\node[stars] at (axis cs:{311.795},{34.374}) {\tikz\pgfuseplotmark{m5bv};};
\node[stars] at (axis cs:{311.837},{45.580}) {\tikz\pgfuseplotmark{m6cv};};
\node[stars] at (axis cs:{311.852},{36.491}) {\tikz\pgfuseplotmark{m5avb};};
\node[stars] at (axis cs:{311.889},{80.552}) {\tikz\pgfuseplotmark{m5c};};
\node[stars] at (axis cs:{311.955},{47.832}) {\tikz\pgfuseplotmark{m6a};};
\node[stars] at (axis cs:{311.971},{52.407}) {\tikz\pgfuseplotmark{m6c};};
\node[stars] at (axis cs:{312.178},{51.910}) {\tikz\pgfuseplotmark{m6cb};};
\node[stars] at (axis cs:{312.235},{46.114}) {\tikz\pgfuseplotmark{m5av};};
\node[stars] at (axis cs:{312.322},{64.042}) {\tikz\pgfuseplotmark{m6c};};
\node[stars] at (axis cs:{312.478},{46.661}) {\tikz\pgfuseplotmark{m6cv};};
\node[stars] at (axis cs:{312.521},{44.059}) {\tikz\pgfuseplotmark{m5b};};
\node[stars] at (axis cs:{312.868},{28.250}) {\tikz\pgfuseplotmark{m6av};};
\node[stars] at (axis cs:{313.002},{32.849}) {\tikz\pgfuseplotmark{m6c};};
\node[stars] at (axis cs:{313.032},{27.097}) {\tikz\pgfuseplotmark{m5av};};
\node[stars] at (axis cs:{313.281},{29.649}) {\tikz\pgfuseplotmark{m6c};};
\node[stars] at (axis cs:{313.311},{44.387}) {\tikz\pgfuseplotmark{m5av};};
\node[stars] at (axis cs:{313.327},{45.181}) {\tikz\pgfuseplotmark{m5cv};};
\node[stars] at (axis cs:{313.475},{33.438}) {\tikz\pgfuseplotmark{m5c};};
\node[stars] at (axis cs:{313.640},{28.057}) {\tikz\pgfuseplotmark{m5bv};};
\node[stars] at (axis cs:{313.685},{75.925}) {\tikz\pgfuseplotmark{m6b};};
\node[stars] at (axis cs:{313.958},{47.418}) {\tikz\pgfuseplotmark{m6av};};
\node[stars] at (axis cs:{314.071},{56.887}) {\tikz\pgfuseplotmark{m6c};};
\node[stars] at (axis cs:{314.106},{50.728}) {\tikz\pgfuseplotmark{m6a};};
\node[stars] at (axis cs:{314.108},{49.196}) {\tikz\pgfuseplotmark{m6b};};
\node[stars] at (axis cs:{314.145},{44.925}) {\tikz\pgfuseplotmark{m6b};};
\node[stars] at (axis cs:{314.293},{41.167}) {\tikz\pgfuseplotmark{m4b};};
\node[stars] at (axis cs:{314.581},{44.472}) {\tikz\pgfuseplotmark{m6a};};
\node[stars] at (axis cs:{314.625},{50.462}) {\tikz\pgfuseplotmark{m6ab};};
\node[stars] at (axis cs:{314.629},{41.940}) {\tikz\pgfuseplotmark{m6c};};
\node[stars] at (axis cs:{314.853},{42.324}) {\tikz\pgfuseplotmark{m6c};};
\node[stars] at (axis cs:{314.856},{59.439}) {\tikz\pgfuseplotmark{m6a};};
\node[stars] at (axis cs:{314.956},{47.521}) {\tikz\pgfuseplotmark{m5av};};
\node[stars] at (axis cs:{315.296},{46.156}) {\tikz\pgfuseplotmark{m5cv};};
\node[stars] at (axis cs:{315.304},{36.026}) {\tikz\pgfuseplotmark{m6b};};
\node[stars] at (axis cs:{315.538},{56.670}) {\tikz\pgfuseplotmark{m6ab};};
\node[stars] at (axis cs:{315.587},{39.509}) {\tikz\pgfuseplotmark{m6cb};};
\node[stars] at (axis cs:{315.601},{44.791}) {\tikz\pgfuseplotmark{m6cv};};
\node[stars] at (axis cs:{315.770},{38.657}) {\tikz\pgfuseplotmark{m6b};};
\node[stars] at (axis cs:{315.859},{50.352}) {\tikz\pgfuseplotmark{m6c};};
\node[stars] at (axis cs:{315.931},{46.863}) {\tikz\pgfuseplotmark{m6c};};
\node[stars] at (axis cs:{315.948},{53.286}) {\tikz\pgfuseplotmark{m6b};};
\node[stars] at (axis cs:{315.967},{41.628}) {\tikz\pgfuseplotmark{m6cb};};
\node[stars] at (axis cs:{316.233},{43.928}) {\tikz\pgfuseplotmark{m4av};};
\node[stars] at (axis cs:{316.372},{78.126}) {\tikz\pgfuseplotmark{m6b};};
\node[stars] at (axis cs:{316.597},{71.432}) {\tikz\pgfuseplotmark{m6b};};
\node[stars] at (axis cs:{316.626},{31.185}) {\tikz\pgfuseplotmark{m6av};};
\node[stars] at (axis cs:{316.650},{47.648}) {\tikz\pgfuseplotmark{m5av};};
\node[stars] at (axis cs:{316.725},{38.749}) {\tikz\pgfuseplotmark{m5cvb};};
\node[stars] at (axis cs:{317.162},{30.206}) {\tikz\pgfuseplotmark{m6avb};};
\node[stars] at (axis cs:{317.565},{53.563}) {\tikz\pgfuseplotmark{m6a};};
\node[stars] at (axis cs:{317.951},{59.987}) {\tikz\pgfuseplotmark{m6avb};};
\node[stars] at (axis cs:{318.234},{30.227}) {\tikz\pgfuseplotmark{m3c};};
\node[stars] at (axis cs:{318.340},{81.231}) {\tikz\pgfuseplotmark{m6b};};
\node[stars] at (axis cs:{318.360},{36.633}) {\tikz\pgfuseplotmark{m6b};};
\node[stars] at (axis cs:{318.427},{64.404}) {\tikz\pgfuseplotmark{m6cv};};
\node[stars] at (axis cs:{318.543},{29.901}) {\tikz\pgfuseplotmark{m6c};};
\node[stars] at (axis cs:{318.698},{38.045}) {\tikz\pgfuseplotmark{m4avb};};
\node[stars] at (axis cs:{318.905},{47.974}) {\tikz\pgfuseplotmark{m6c};};
\node[stars] at (axis cs:{318.927},{77.012}) {\tikz\pgfuseplotmark{m6bv};};
\node[stars] at (axis cs:{319.071},{44.238}) {\tikz\pgfuseplotmark{m6c};};
\node[stars] at (axis cs:{319.123},{42.252}) {\tikz\pgfuseplotmark{m6c};};
\node[stars] at (axis cs:{319.259},{53.997}) {\tikz\pgfuseplotmark{m6b};};
\node[stars] at (axis cs:{319.309},{55.798}) {\tikz\pgfuseplotmark{m6b};};
\node[stars] at (axis cs:{319.328},{58.611}) {\tikz\pgfuseplotmark{m6cv};};
\node[stars] at (axis cs:{319.347},{42.684}) {\tikz\pgfuseplotmark{m6c};};
\node[stars] at (axis cs:{319.354},{39.394}) {\tikz\pgfuseplotmark{m4cv};};
\node[stars] at (axis cs:{319.479},{34.897}) {\tikz\pgfuseplotmark{m4cvb};};
\node[stars] at (axis cs:{319.613},{43.946}) {\tikz\pgfuseplotmark{m5bv};};
\node[stars] at (axis cs:{319.635},{61.184}) {\tikz\pgfuseplotmark{m6cv};};
\node[stars] at (axis cs:{319.645},{62.586}) {\tikz\pgfuseplotmark{m2cvb};};
\node[stars] at (axis cs:{319.731},{41.040}) {\tikz\pgfuseplotmark{m6c};};
\node[stars] at (axis cs:{319.815},{58.623}) {\tikz\pgfuseplotmark{m6avb};};
\node[stars] at (axis cs:{319.842},{38.237}) {\tikz\pgfuseplotmark{m6bv};};
\node[stars] at (axis cs:{319.843},{64.872}) {\tikz\pgfuseplotmark{m5cv};};
\node[stars] at (axis cs:{319.870},{49.510}) {\tikz\pgfuseplotmark{m6a};};
\node[stars] at (axis cs:{320.139},{60.756}) {\tikz\pgfuseplotmark{m6b};};
\node[stars] at (axis cs:{320.256},{40.345}) {\tikz\pgfuseplotmark{m6c};};
\node[stars] at (axis cs:{320.341},{32.613}) {\tikz\pgfuseplotmark{m6b};};
\node[stars] at (axis cs:{320.502},{49.389}) {\tikz\pgfuseplotmark{m6a};};
\node[stars] at (axis cs:{320.675},{30.310}) {\tikz\pgfuseplotmark{m6b};};
\node[stars] at (axis cs:{320.952},{37.351}) {\tikz\pgfuseplotmark{m6cb};};
\node[stars] at (axis cs:{321.206},{80.525}) {\tikz\pgfuseplotmark{m6b};};
\node[stars] at (axis cs:{321.331},{46.714}) {\tikz\pgfuseplotmark{m6a};};
\node[stars] at (axis cs:{321.446},{36.667}) {\tikz\pgfuseplotmark{m6bvb};};
\node[stars] at (axis cs:{321.687},{52.898}) {\tikz\pgfuseplotmark{m6b};};
\node[stars] at (axis cs:{321.715},{48.835}) {\tikz\pgfuseplotmark{m5cv};};
\node[stars] at (axis cs:{321.839},{37.117}) {\tikz\pgfuseplotmark{m5c};};
\node[stars] at (axis cs:{321.855},{59.750}) {\tikz\pgfuseplotmark{m6cv};};
\node[stars] at (axis cs:{321.917},{27.608}) {\tikz\pgfuseplotmark{m5c};};
\node[stars] at (axis cs:{321.942},{66.809}) {\tikz\pgfuseplotmark{m5c};};
\node[stars] at (axis cs:{322.034},{32.225}) {\tikz\pgfuseplotmark{m6a};};
\node[stars] at (axis cs:{322.165},{70.561}) {\tikz\pgfuseplotmark{m3cv};};
\node[stars] at (axis cs:{322.220},{55.419}) {\tikz\pgfuseplotmark{m6c};};
\node[stars] at (axis cs:{322.362},{46.540}) {\tikz\pgfuseplotmark{m5c};};
\node[stars] at (axis cs:{322.585},{52.958}) {\tikz\pgfuseplotmark{m6b};};
\node[stars] at (axis cs:{322.747},{60.459}) {\tikz\pgfuseplotmark{m6a};};
\node[stars] at (axis cs:{322.864},{52.620}) {\tikz\pgfuseplotmark{m6c};};
\node[stars] at (axis cs:{323.236},{49.977}) {\tikz\pgfuseplotmark{m6a};};
\node[stars] at (axis cs:{323.325},{45.854}) {\tikz\pgfuseplotmark{m6cv};};
\node[stars] at (axis cs:{323.495},{45.592}) {\tikz\pgfuseplotmark{m4bv};};
\node[stars] at (axis cs:{323.614},{51.698}) {\tikz\pgfuseplotmark{m6c};};
\node[stars] at (axis cs:{323.694},{38.534}) {\tikz\pgfuseplotmark{m5b};};
\node[stars] at (axis cs:{323.830},{28.197}) {\tikz\pgfuseplotmark{m6c};};
\node[stars] at (axis cs:{323.858},{68.219}) {\tikz\pgfuseplotmark{m6cv};};
\node[stars] at (axis cs:{324.010},{45.374}) {\tikz\pgfuseplotmark{m6bv};};
\node[stars] at (axis cs:{324.058},{30.055}) {\tikz\pgfuseplotmark{m6cb};};
\node[stars] at (axis cs:{324.237},{40.413}) {\tikz\pgfuseplotmark{m5b};};
\node[stars] at (axis cs:{324.366},{44.696}) {\tikz\pgfuseplotmark{m6c};};
\node[stars] at (axis cs:{324.412},{54.042}) {\tikz\pgfuseplotmark{m6c};};
\node[stars] at (axis cs:{324.480},{62.082}) {\tikz\pgfuseplotmark{m5av};};
\node[stars] at (axis cs:{324.740},{57.489}) {\tikz\pgfuseplotmark{m6ab};};
\node[stars] at (axis cs:{325.046},{43.274}) {\tikz\pgfuseplotmark{m5b};};
\node[stars] at (axis cs:{325.180},{54.872}) {\tikz\pgfuseplotmark{m6c};};
\node[stars] at (axis cs:{325.393},{40.805}) {\tikz\pgfuseplotmark{m6bb};};
\node[stars] at (axis cs:{325.480},{71.311}) {\tikz\pgfuseplotmark{m5a};};
\node[stars] at (axis cs:{325.505},{35.510}) {\tikz\pgfuseplotmark{m6bv};};
\node[stars] at (axis cs:{325.524},{51.190}) {\tikz\pgfuseplotmark{m5a};};
\node[stars] at (axis cs:{325.535},{45.765}) {\tikz\pgfuseplotmark{m6cv};};
\node[stars] at (axis cs:{325.596},{41.077}) {\tikz\pgfuseplotmark{m6ab};};
\node[stars] at (axis cs:{325.662},{49.600}) {\tikz\pgfuseplotmark{m6b};};
\node[stars] at (axis cs:{325.689},{59.271}) {\tikz\pgfuseplotmark{m6b};};
\node[stars] at (axis cs:{325.767},{72.320}) {\tikz\pgfuseplotmark{m5c};};
\node[stars] at (axis cs:{325.777},{41.155}) {\tikz\pgfuseplotmark{m6av};};
\node[stars] at (axis cs:{325.857},{38.284}) {\tikz\pgfuseplotmark{m6ab};};
\node[stars] at (axis cs:{325.877},{58.780}) {\tikz\pgfuseplotmark{m4bv};};
\node[stars] at (axis cs:{326.036},{28.742}) {\tikz\pgfuseplotmark{m5ab};};
\node[stars] at (axis cs:{326.222},{62.460}) {\tikz\pgfuseplotmark{m6b};};
\node[stars] at (axis cs:{326.302},{52.268}) {\tikz\pgfuseplotmark{m6c};};
\node[stars] at (axis cs:{326.362},{61.121}) {\tikz\pgfuseplotmark{m4cv};};
\node[stars] at (axis cs:{326.435},{35.857}) {\tikz\pgfuseplotmark{m6c};};
\node[stars] at (axis cs:{326.698},{49.309}) {\tikz\pgfuseplotmark{m4c};};
\node[stars] at (axis cs:{326.754},{70.151}) {\tikz\pgfuseplotmark{m6c};};
\node[stars] at (axis cs:{326.855},{60.693}) {\tikz\pgfuseplotmark{m6a};};
\node[stars] at (axis cs:{327.035},{36.580}) {\tikz\pgfuseplotmark{m6c};};
\node[stars] at (axis cs:{327.122},{38.648}) {\tikz\pgfuseplotmark{m6b};};
\node[stars] at (axis cs:{327.284},{66.792}) {\tikz\pgfuseplotmark{m6cb};};
\node[stars] at (axis cs:{327.329},{61.273}) {\tikz\pgfuseplotmark{m6c};};
\node[stars] at (axis cs:{327.417},{41.149}) {\tikz\pgfuseplotmark{m6c};};
\node[stars] at (axis cs:{327.461},{30.174}) {\tikz\pgfuseplotmark{m5b};};
\node[stars] at (axis cs:{327.771},{39.537}) {\tikz\pgfuseplotmark{m6cv};};
\node[stars] at (axis cs:{327.905},{65.753}) {\tikz\pgfuseplotmark{m6cb};};
\node[stars] at (axis cs:{328.004},{55.797}) {\tikz\pgfuseplotmark{m6avb};};
\node[stars] at (axis cs:{328.054},{79.552}) {\tikz\pgfuseplotmark{m6c};};
\node[stars] at (axis cs:{328.125},{28.793}) {\tikz\pgfuseplotmark{m6a};};
\node[stars] at (axis cs:{328.611},{80.308}) {\tikz\pgfuseplotmark{m6cv};};
\node[stars] at (axis cs:{328.721},{56.611}) {\tikz\pgfuseplotmark{m6av};};
\node[stars] at (axis cs:{328.836},{61.542}) {\tikz\pgfuseplotmark{m6c};};
\node[stars] at (axis cs:{328.879},{65.321}) {\tikz\pgfuseplotmark{m6bvb};};
\node[stars] at (axis cs:{329.163},{63.625}) {\tikz\pgfuseplotmark{m5cv};};
\node[stars] at (axis cs:{329.296},{66.156}) {\tikz\pgfuseplotmark{m6c};};
\node[stars] at (axis cs:{329.463},{74.996}) {\tikz\pgfuseplotmark{m6cv};};
\node[stars] at (axis cs:{329.723},{62.698}) {\tikz\pgfuseplotmark{m6bv};};
\node[stars] at (axis cs:{329.812},{73.180}) {\tikz\pgfuseplotmark{m5b};};
\node[stars] at (axis cs:{330.111},{33.006}) {\tikz\pgfuseplotmark{m6c};};
\node[stars] at (axis cs:{330.461},{52.882}) {\tikz\pgfuseplotmark{m6a};};
\node[stars] at (axis cs:{330.519},{58.000}) {\tikz\pgfuseplotmark{m6av};};
\node[stars] at (axis cs:{330.736},{44.650}) {\tikz\pgfuseplotmark{m6av};};
\node[stars] at (axis cs:{330.948},{64.628}) {\tikz\pgfuseplotmark{m4cb};};
\node[stars] at (axis cs:{330.971},{63.120}) {\tikz\pgfuseplotmark{m5cv};};
\node[stars] at (axis cs:{331.144},{32.942}) {\tikz\pgfuseplotmark{m6c};};
\node[stars] at (axis cs:{331.252},{62.786}) {\tikz\pgfuseplotmark{m5cv};};
\node[stars] at (axis cs:{331.287},{62.280}) {\tikz\pgfuseplotmark{m5b};};
\node[stars] at (axis cs:{331.319},{46.745}) {\tikz\pgfuseplotmark{m6cv};};
\node[stars] at (axis cs:{331.394},{28.964}) {\tikz\pgfuseplotmark{m6a};};
\node[stars] at (axis cs:{331.463},{48.231}) {\tikz\pgfuseplotmark{m6cv};};
\node[stars] at (axis cs:{331.508},{45.014}) {\tikz\pgfuseplotmark{m5b};};
\node[stars] at (axis cs:{331.551},{45.248}) {\tikz\pgfuseplotmark{m6c};};
\node[stars] at (axis cs:{331.557},{56.343}) {\tikz\pgfuseplotmark{m6cv};};
\node[stars] at (axis cs:{331.790},{58.841}) {\tikz\pgfuseplotmark{m6c};};
\node[stars] at (axis cs:{331.857},{53.307}) {\tikz\pgfuseplotmark{m6cv};};
\node[stars] at (axis cs:{332.068},{49.797}) {\tikz\pgfuseplotmark{m6c};};
\node[stars] at (axis cs:{332.170},{45.742}) {\tikz\pgfuseplotmark{m6bv};};
\node[stars] at (axis cs:{332.307},{33.172}) {\tikz\pgfuseplotmark{m6ab};};
\node[stars] at (axis cs:{332.452},{72.341}) {\tikz\pgfuseplotmark{m5a};};
\node[stars] at (axis cs:{332.497},{33.178}) {\tikz\pgfuseplotmark{m4c};};
\node[stars] at (axis cs:{332.564},{72.111}) {\tikz\pgfuseplotmark{m6c};};
\node[stars] at (axis cs:{332.662},{70.132}) {\tikz\pgfuseplotmark{m6ab};};
\node[stars] at (axis cs:{332.714},{58.201}) {\tikz\pgfuseplotmark{m3cv};};
\node[stars] at (axis cs:{332.715},{30.553}) {\tikz\pgfuseplotmark{m6c};};
\node[stars] at (axis cs:{332.791},{50.823}) {\tikz\pgfuseplotmark{m5c};};
\node[stars] at (axis cs:{332.877},{59.414}) {\tikz\pgfuseplotmark{m5bv};};
\node[stars] at (axis cs:{332.953},{56.839}) {\tikz\pgfuseplotmark{m5c};};
\node[stars] at (axis cs:{332.987},{59.085}) {\tikz\pgfuseplotmark{m6c};};
\node[stars] at (axis cs:{333.008},{60.759}) {\tikz\pgfuseplotmark{m5c};};
\node[stars] at (axis cs:{333.094},{63.291}) {\tikz\pgfuseplotmark{m6av};};
\node[stars] at (axis cs:{333.199},{34.605}) {\tikz\pgfuseplotmark{m5c};};
\node[stars] at (axis cs:{333.220},{73.307}) {\tikz\pgfuseplotmark{m6bvb};};
\node[stars] at (axis cs:{333.294},{86.108}) {\tikz\pgfuseplotmark{m5c};};
\node[stars] at (axis cs:{333.411},{28.608}) {\tikz\pgfuseplotmark{m6b};};
\node[stars] at (axis cs:{333.455},{45.440}) {\tikz\pgfuseplotmark{m6a};};
\node[stars] at (axis cs:{333.457},{63.162}) {\tikz\pgfuseplotmark{m6b};};
\node[stars] at (axis cs:{333.470},{39.715}) {\tikz\pgfuseplotmark{m4cv};};
\node[stars] at (axis cs:{333.685},{42.954}) {\tikz\pgfuseplotmark{m6a};};
\node[stars] at (axis cs:{333.759},{57.043}) {\tikz\pgfuseplotmark{m4cv};};
\node[stars] at (axis cs:{333.992},{37.748}) {\tikz\pgfuseplotmark{m4b};};
\node[stars] at (axis cs:{334.111},{57.220}) {\tikz\pgfuseplotmark{m6b};};
\node[stars] at (axis cs:{334.124},{27.804}) {\tikz\pgfuseplotmark{m6c};};
\node[stars] at (axis cs:{334.368},{49.127}) {\tikz\pgfuseplotmark{m6c};};
\node[stars] at (axis cs:{334.553},{62.804}) {\tikz\pgfuseplotmark{m6a};};
\node[stars] at (axis cs:{334.734},{37.769}) {\tikz\pgfuseplotmark{m6cb};};
\node[stars] at (axis cs:{335.162},{58.410}) {\tikz\pgfuseplotmark{m6c};};
\node[stars] at (axis cs:{335.165},{50.981}) {\tikz\pgfuseplotmark{m6c};};
\node[stars] at (axis cs:{335.256},{46.537}) {\tikz\pgfuseplotmark{m5av};};
\node[stars] at (axis cs:{335.331},{28.330}) {\tikz\pgfuseplotmark{m5ab};};
\node[stars] at (axis cs:{335.462},{42.078}) {\tikz\pgfuseplotmark{m6cv};};
\node[stars] at (axis cs:{335.710},{36.659}) {\tikz\pgfuseplotmark{m6c};};
\node[stars] at (axis cs:{335.751},{57.285}) {\tikz\pgfuseplotmark{m6c};};
\node[stars] at (axis cs:{335.752},{62.419}) {\tikz\pgfuseplotmark{m6b};};
\node[stars] at (axis cs:{335.890},{52.229}) {\tikz\pgfuseplotmark{m4c};};
\node[stars] at (axis cs:{335.976},{38.573}) {\tikz\pgfuseplotmark{m6cv};};
\node[stars] at (axis cs:{336.129},{49.476}) {\tikz\pgfuseplotmark{m5a};};
\node[stars] at (axis cs:{336.503},{70.771}) {\tikz\pgfuseplotmark{m5c};};
\node[stars] at (axis cs:{336.677},{78.786}) {\tikz\pgfuseplotmark{m6a};};
\node[stars] at (axis cs:{336.690},{37.444}) {\tikz\pgfuseplotmark{m6cb};};
\node[stars] at (axis cs:{336.772},{65.132}) {\tikz\pgfuseplotmark{m6a};};
\node[stars] at (axis cs:{336.860},{39.810}) {\tikz\pgfuseplotmark{m6c};};
\node[stars] at (axis cs:{336.943},{31.840}) {\tikz\pgfuseplotmark{m6b};};
\node[stars] at (axis cs:{337.046},{27.019}) {\tikz\pgfuseplotmark{m6c};};
\node[stars] at (axis cs:{337.081},{64.085}) {\tikz\pgfuseplotmark{m6c};};
\node[stars] at (axis cs:{337.293},{58.415}) {\tikz\pgfuseplotmark{m4bvb};};
\node[stars] at (axis cs:{337.383},{47.707}) {\tikz\pgfuseplotmark{m4cv};};
\node[stars] at (axis cs:{337.471},{78.824}) {\tikz\pgfuseplotmark{m5c};};
\node[stars] at (axis cs:{337.508},{32.573}) {\tikz\pgfuseplotmark{m6a};};
\node[stars] at (axis cs:{337.527},{49.356}) {\tikz\pgfuseplotmark{m6cv};};
\node[stars] at (axis cs:{337.622},{43.123}) {\tikz\pgfuseplotmark{m5a};};
\node[stars] at (axis cs:{337.823},{50.282}) {\tikz\pgfuseplotmark{m4a};};
\node[stars] at (axis cs:{337.893},{29.543}) {\tikz\pgfuseplotmark{m6cv};};
\node[stars] at (axis cs:{338.068},{76.226}) {\tikz\pgfuseplotmark{m6a};};
\node[stars] at (axis cs:{338.078},{54.037}) {\tikz\pgfuseplotmark{m6c};};
\node[stars] at (axis cs:{338.110},{39.780}) {\tikz\pgfuseplotmark{m6bb};};
\node[stars] at (axis cs:{338.262},{69.913}) {\tikz\pgfuseplotmark{m6b};};
\node[stars] at (axis cs:{338.320},{70.374}) {\tikz\pgfuseplotmark{m6cb};};
\node[stars] at (axis cs:{338.419},{56.625}) {\tikz\pgfuseplotmark{m6av};};
\node[stars] at (axis cs:{338.942},{73.643}) {\tikz\pgfuseplotmark{m5b};};
\node[stars] at (axis cs:{338.967},{56.071}) {\tikz\pgfuseplotmark{m6c};};
\node[stars] at (axis cs:{338.968},{39.634}) {\tikz\pgfuseplotmark{m6avb};};
\node[stars] at (axis cs:{338.972},{50.071}) {\tikz\pgfuseplotmark{m6cv};};
\node[stars] at (axis cs:{339.033},{35.577}) {\tikz\pgfuseplotmark{m6b};};
\node[stars] at (axis cs:{339.203},{35.652}) {\tikz\pgfuseplotmark{m6c};};
\node[stars] at (axis cs:{339.304},{75.372}) {\tikz\pgfuseplotmark{m6a};};
\node[stars] at (axis cs:{339.343},{51.545}) {\tikz\pgfuseplotmark{m5a};};
\node[stars] at (axis cs:{339.573},{45.183}) {\tikz\pgfuseplotmark{m6c};};
\node[stars] at (axis cs:{339.658},{56.795}) {\tikz\pgfuseplotmark{m5cv};};
\node[stars] at (axis cs:{339.663},{63.584}) {\tikz\pgfuseplotmark{m5c};};
\node[stars] at (axis cs:{339.815},{39.050}) {\tikz\pgfuseplotmark{m5bv};};
\node[stars] at (axis cs:{339.893},{37.593}) {\tikz\pgfuseplotmark{m6b};};
\node[stars] at (axis cs:{340.077},{53.846}) {\tikz\pgfuseplotmark{m6b};};
\node[stars] at (axis cs:{340.129},{44.276}) {\tikz\pgfuseplotmark{m4c};};
\node[stars] at (axis cs:{340.369},{40.225}) {\tikz\pgfuseplotmark{m5cv};};
\node[stars] at (axis cs:{340.380},{30.966}) {\tikz\pgfuseplotmark{m6c};};
\node[stars] at (axis cs:{340.400},{41.549}) {\tikz\pgfuseplotmark{m6b};};
\node[stars] at (axis cs:{340.439},{29.308}) {\tikz\pgfuseplotmark{m5a};};
\node[stars] at (axis cs:{340.587},{53.909}) {\tikz\pgfuseplotmark{m6b};};
\node[stars] at (axis cs:{340.731},{37.803}) {\tikz\pgfuseplotmark{m6c};};
\node[stars] at (axis cs:{340.751},{30.221}) {\tikz\pgfuseplotmark{m3bvb};};
\node[stars] at (axis cs:{340.768},{47.169}) {\tikz\pgfuseplotmark{m6c};};
\node[stars] at (axis cs:{341.022},{39.465}) {\tikz\pgfuseplotmark{m6bb};};
\node[stars] at (axis cs:{341.023},{41.819}) {\tikz\pgfuseplotmark{m5b};};
\node[stars] at (axis cs:{341.205},{52.517}) {\tikz\pgfuseplotmark{m6c};};
\node[stars] at (axis cs:{341.221},{39.201}) {\tikz\pgfuseplotmark{m6c};};
\node[stars] at (axis cs:{341.265},{58.147}) {\tikz\pgfuseplotmark{m6c};};
\node[stars] at (axis cs:{341.394},{30.442}) {\tikz\pgfuseplotmark{m6cb};};
\node[stars] at (axis cs:{341.543},{44.546}) {\tikz\pgfuseplotmark{m6av};};
\node[stars] at (axis cs:{341.847},{58.483}) {\tikz\pgfuseplotmark{m6c};};
\node[stars] at (axis cs:{341.871},{83.154}) {\tikz\pgfuseplotmark{m5a};};
\node[stars] at (axis cs:{342.046},{37.417}) {\tikz\pgfuseplotmark{m6av};};
\node[stars] at (axis cs:{342.184},{62.938}) {\tikz\pgfuseplotmark{m6b};};
\node[stars] at (axis cs:{342.200},{54.415}) {\tikz\pgfuseplotmark{m6c};};
\node[stars] at (axis cs:{342.420},{66.200}) {\tikz\pgfuseplotmark{m3c};};
\node[stars] at (axis cs:{342.443},{55.903}) {\tikz\pgfuseplotmark{m5c};};
\node[stars] at (axis cs:{342.542},{50.677}) {\tikz\pgfuseplotmark{m6c};};
\node[stars] at (axis cs:{342.591},{41.953}) {\tikz\pgfuseplotmark{m6bv};};
\node[stars] at (axis cs:{342.763},{85.374}) {\tikz\pgfuseplotmark{m6b};};
\node[stars] at (axis cs:{342.844},{61.697}) {\tikz\pgfuseplotmark{m6ab};};
\node[stars] at (axis cs:{343.008},{43.312}) {\tikz\pgfuseplotmark{m5b};};
\node[stars] at (axis cs:{343.218},{50.412}) {\tikz\pgfuseplotmark{m6cv};};
\node[stars] at (axis cs:{343.266},{60.101}) {\tikz\pgfuseplotmark{m6b};};
\node[stars] at (axis cs:{343.297},{40.167}) {\tikz\pgfuseplotmark{m6cv};};
\node[stars] at (axis cs:{343.417},{44.749}) {\tikz\pgfuseplotmark{m6a};};
\node[stars] at (axis cs:{343.529},{40.377}) {\tikz\pgfuseplotmark{m6a};};
\node[stars] at (axis cs:{343.604},{84.346}) {\tikz\pgfuseplotmark{m5a};};
\node[stars] at (axis cs:{343.761},{37.077}) {\tikz\pgfuseplotmark{m6b};};
\node[stars] at (axis cs:{343.935},{36.351}) {\tikz\pgfuseplotmark{m6ab};};
\node[stars] at (axis cs:{344.098},{41.604}) {\tikz\pgfuseplotmark{m6avb};};
\node[stars] at (axis cs:{344.108},{49.734}) {\tikz\pgfuseplotmark{m5bv};};
\node[stars] at (axis cs:{344.269},{48.684}) {\tikz\pgfuseplotmark{m5cv};};
\node[stars] at (axis cs:{344.420},{39.309}) {\tikz\pgfuseplotmark{m6c};};
\node[stars] at (axis cs:{344.788},{59.814}) {\tikz\pgfuseplotmark{m6c};};
\node[stars] at (axis cs:{344.793},{52.654}) {\tikz\pgfuseplotmark{m6c};};
\node[stars] at (axis cs:{345.021},{56.945}) {\tikz\pgfuseplotmark{m5bv};};
\node[stars] at (axis cs:{345.378},{57.105}) {\tikz\pgfuseplotmark{m6cv};};
\node[stars] at (axis cs:{345.480},{42.326}) {\tikz\pgfuseplotmark{m4av};};
\node[stars] at (axis cs:{345.547},{44.573}) {\tikz\pgfuseplotmark{m6c};};
\node[stars] at (axis cs:{345.652},{42.758}) {\tikz\pgfuseplotmark{m5b};};
\node[stars] at (axis cs:{345.688},{44.059}) {\tikz\pgfuseplotmark{m6cvb};};
\node[stars] at (axis cs:{345.839},{58.564}) {\tikz\pgfuseplotmark{m6c};};
\node[stars] at (axis cs:{345.887},{67.209}) {\tikz\pgfuseplotmark{m5c};};
\node[stars] at (axis cs:{345.944},{28.083}) {\tikz\pgfuseplotmark{m2cv};};
\node[stars] at (axis cs:{346.046},{50.052}) {\tikz\pgfuseplotmark{m5a};};
\node[stars] at (axis cs:{346.653},{59.420}) {\tikz\pgfuseplotmark{m5a};};
\node[stars] at (axis cs:{346.771},{35.636}) {\tikz\pgfuseplotmark{m6cv};};
\node[stars] at (axis cs:{346.792},{52.816}) {\tikz\pgfuseplotmark{m6b};};
\node[stars] at (axis cs:{346.794},{59.727}) {\tikz\pgfuseplotmark{m6c};};
\node[stars] at (axis cs:{346.866},{32.825}) {\tikz\pgfuseplotmark{m6cvb};};
\node[stars] at (axis cs:{346.914},{46.387}) {\tikz\pgfuseplotmark{m5c};};
\node[stars] at (axis cs:{346.939},{49.296}) {\tikz\pgfuseplotmark{m6a};};
\node[stars] at (axis cs:{346.949},{63.633}) {\tikz\pgfuseplotmark{m6c};};
\node[stars] at (axis cs:{346.974},{75.387}) {\tikz\pgfuseplotmark{m4cb};};
\node[stars] at (axis cs:{346.988},{64.222}) {\tikz\pgfuseplotmark{m6c};};
\node[stars] at (axis cs:{347.434},{59.332}) {\tikz\pgfuseplotmark{m6avb};};
\node[stars] at (axis cs:{347.613},{43.544}) {\tikz\pgfuseplotmark{m6b};};
\node[stars] at (axis cs:{348.138},{49.406}) {\tikz\pgfuseplotmark{m5a};};
\node[stars] at (axis cs:{348.267},{29.441}) {\tikz\pgfuseplotmark{m6c};};
\node[stars] at (axis cs:{348.321},{57.168}) {\tikz\pgfuseplotmark{m6avb};};
\node[stars] at (axis cs:{348.560},{50.618}) {\tikz\pgfuseplotmark{m6c};};
\node[stars] at (axis cs:{348.590},{29.772}) {\tikz\pgfuseplotmark{m6cb};};
\node[stars] at (axis cs:{348.655},{74.231}) {\tikz\pgfuseplotmark{m6b};};
\node[stars] at (axis cs:{348.907},{70.888}) {\tikz\pgfuseplotmark{m6av};};
\node[stars] at (axis cs:{349.176},{53.213}) {\tikz\pgfuseplotmark{m6a};};
\node[stars] at (axis cs:{349.319},{45.164}) {\tikz\pgfuseplotmark{m6c};};
\node[stars] at (axis cs:{349.329},{75.299}) {\tikz\pgfuseplotmark{m6c};};
\node[stars] at (axis cs:{349.436},{49.015}) {\tikz\pgfuseplotmark{m5avb};};
\node[stars] at (axis cs:{349.597},{41.774}) {\tikz\pgfuseplotmark{m6bv};};
\node[stars] at (axis cs:{349.656},{68.111}) {\tikz\pgfuseplotmark{m5avb};};
\node[stars] at (axis cs:{349.864},{34.793}) {\tikz\pgfuseplotmark{m6c};};
\node[stars] at (axis cs:{349.874},{48.625}) {\tikz\pgfuseplotmark{m5c};};
\node[stars] at (axis cs:{349.923},{48.381}) {\tikz\pgfuseplotmark{m6cvb};};
\node[stars] at (axis cs:{349.968},{42.078}) {\tikz\pgfuseplotmark{m6a};};
\node[stars] at (axis cs:{350.060},{61.970}) {\tikz\pgfuseplotmark{m6cv};};
\node[stars] at (axis cs:{350.144},{62.213}) {\tikz\pgfuseplotmark{m6c};};
\node[stars] at (axis cs:{350.184},{44.116}) {\tikz\pgfuseplotmark{m6cb};};
\node[stars] at (axis cs:{350.206},{30.415}) {\tikz\pgfuseplotmark{m6a};};
\node[stars] at (axis cs:{350.222},{38.182}) {\tikz\pgfuseplotmark{m6ab};};
\node[stars] at (axis cs:{350.479},{31.812}) {\tikz\pgfuseplotmark{m5c};};
\node[stars] at (axis cs:{350.636},{60.133}) {\tikz\pgfuseplotmark{m6av};};
\node[stars] at (axis cs:{351.037},{41.613}) {\tikz\pgfuseplotmark{m6cv};};
\node[stars] at (axis cs:{351.209},{62.283}) {\tikz\pgfuseplotmark{m5bvb};};
\node[stars] at (axis cs:{351.212},{32.385}) {\tikz\pgfuseplotmark{m6a};};
\node[stars] at (axis cs:{351.754},{87.307}) {\tikz\pgfuseplotmark{m6a};};
\node[stars] at (axis cs:{351.781},{42.912}) {\tikz\pgfuseplotmark{m6av};};
\node[stars] at (axis cs:{351.819},{70.360}) {\tikz\pgfuseplotmark{m6a};};
\node[stars] at (axis cs:{352.508},{58.549}) {\tikz\pgfuseplotmark{m5bvb};};
\node[stars] at (axis cs:{352.531},{49.133}) {\tikz\pgfuseplotmark{m6c};};
\node[stars] at (axis cs:{352.665},{38.662}) {\tikz\pgfuseplotmark{m6bv};};
\node[stars] at (axis cs:{352.823},{39.236}) {\tikz\pgfuseplotmark{m5cv};};
\node[stars] at (axis cs:{352.929},{28.403}) {\tikz\pgfuseplotmark{m6cv};};
\node[stars] at (axis cs:{353.429},{45.058}) {\tikz\pgfuseplotmark{m6c};};
\node[stars] at (axis cs:{353.488},{31.325}) {\tikz\pgfuseplotmark{m5bv};};
\node[stars] at (axis cs:{353.656},{40.236}) {\tikz\pgfuseplotmark{m6av};};
\node[stars] at (axis cs:{353.659},{33.497}) {\tikz\pgfuseplotmark{m6a};};
\node[stars] at (axis cs:{353.695},{38.024}) {\tikz\pgfuseplotmark{m6c};};
\node[stars] at (axis cs:{353.746},{71.642}) {\tikz\pgfuseplotmark{m6a};};
\node[stars] at (axis cs:{354.127},{32.904}) {\tikz\pgfuseplotmark{m6c};};
\node[stars] at (axis cs:{354.384},{44.429}) {\tikz\pgfuseplotmark{m6a};};
\node[stars] at (axis cs:{354.391},{46.458}) {\tikz\pgfuseplotmark{m4bvb};};
\node[stars] at (axis cs:{354.534},{43.268}) {\tikz\pgfuseplotmark{m4cv};};
\node[stars] at (axis cs:{354.785},{50.472}) {\tikz\pgfuseplotmark{m5c};};
\node[stars] at (axis cs:{354.792},{75.293}) {\tikz\pgfuseplotmark{m6b};};
\node[stars] at (axis cs:{354.837},{77.632}) {\tikz\pgfuseplotmark{m3cv};};
\node[stars] at (axis cs:{354.838},{74.003}) {\tikz\pgfuseplotmark{m6b};};
\node[stars] at (axis cs:{355.102},{44.334}) {\tikz\pgfuseplotmark{m4cv};};
\node[stars] at (axis cs:{355.169},{36.721}) {\tikz\pgfuseplotmark{m6c};};
\node[stars] at (axis cs:{355.362},{49.512}) {\tikz\pgfuseplotmark{m6c};};
\node[stars] at (axis cs:{355.477},{57.260}) {\tikz\pgfuseplotmark{m6c};};
\node[stars] at (axis cs:{355.561},{44.992}) {\tikz\pgfuseplotmark{m6cv};};
\node[stars] at (axis cs:{355.631},{61.679}) {\tikz\pgfuseplotmark{m6c};};
\node[stars] at (axis cs:{355.771},{51.939}) {\tikz\pgfuseplotmark{m6c};};
\node[stars] at (axis cs:{355.998},{29.361}) {\tikz\pgfuseplotmark{m5b};};
\node[stars] at (axis cs:{356.202},{55.799}) {\tikz\pgfuseplotmark{m6c};};
\node[stars] at (axis cs:{356.509},{46.420}) {\tikz\pgfuseplotmark{m5b};};
\node[stars] at (axis cs:{356.653},{66.782}) {\tikz\pgfuseplotmark{m6b};};
\node[stars] at (axis cs:{356.758},{57.451}) {\tikz\pgfuseplotmark{m6av};};
\node[stars] at (axis cs:{356.764},{58.652}) {\tikz\pgfuseplotmark{m5bv};};
\node[stars] at (axis cs:{356.888},{46.832}) {\tikz\pgfuseplotmark{m6b};};
\node[stars] at (axis cs:{356.978},{67.807}) {\tikz\pgfuseplotmark{m5b};};
\node[stars] at (axis cs:{357.163},{64.876}) {\tikz\pgfuseplotmark{m6cvb};};
\node[stars] at (axis cs:{357.209},{62.214}) {\tikz\pgfuseplotmark{m6avb};};
\node[stars] at (axis cs:{357.225},{59.979}) {\tikz\pgfuseplotmark{m6c};};
\node[stars] at (axis cs:{357.300},{58.963}) {\tikz\pgfuseplotmark{m6c};};
\node[stars] at (axis cs:{357.414},{28.842}) {\tikz\pgfuseplotmark{m6b};};
\node[stars] at (axis cs:{357.421},{36.425}) {\tikz\pgfuseplotmark{m6bv};};
\node[stars] at (axis cs:{357.593},{51.622}) {\tikz\pgfuseplotmark{m6c};};
\node[stars] at (axis cs:{358.105},{75.544}) {\tikz\pgfuseplotmark{m6cb};};
\node[stars] at (axis cs:{358.596},{57.499}) {\tikz\pgfuseplotmark{m5av};};
\node[stars] at (axis cs:{358.889},{47.356}) {\tikz\pgfuseplotmark{m6b};};
\node[stars] at (axis cs:{358.891},{57.412}) {\tikz\pgfuseplotmark{m6bv};};
\node[stars] at (axis cs:{359.265},{42.658}) {\tikz\pgfuseplotmark{m6bv};};
\node[stars] at (axis cs:{359.285},{55.706}) {\tikz\pgfuseplotmark{m6av};};
\node[stars] at (axis cs:{359.390},{60.023}) {\tikz\pgfuseplotmark{m6c};};
\node[stars] at (axis cs:{359.752},{55.755}) {\tikz\pgfuseplotmark{m5bb};};
\node[stars] at (axis cs:{359.872},{33.724}) {\tikz\pgfuseplotmark{m6cb};};
\end{polaraxis}

\begin{polaraxis}[rotate=270,name=stars,at=(base.center),anchor=center,axis lines=none]
  \boldmath

  
\clip (6\tendegree,0\tendegree) arc (0:90:6\tendegree) -- 
(-2\tendegree,6\tendegree) -- (-2\tendegree,-6\tendegree) -- (0\tendegree,-6\tendegree)
--  (0\tendegree,-6\tendegree) arc (270:359.9999:6\tendegree) -- cycle ;

\node[designation-label,anchor=west]  at (axis cs:{05*15+00/4},{+89+15/ 60  })  {Polaris};   
\node[designation-label,anchor=south west]  at (axis cs:{03*15+24/4},{+49+51/ 60  })  {Algenib};   
\node[designation-label,anchor=south]  at (axis cs:{05*15+18/4},{+45+59/ 60  })  {Capella};   
\node[designation-label,anchor=south west]  at (axis cs:{05*15+26/4},{+28+36/ 60  })  {Alnath};   
\node[designation-label,anchor=south]  at (axis cs:{05*15+59/4},{+44+56/ 60  })  {Menkalinan};   
\node[designation-label,anchor=south east]  at (axis cs:{07*15+40/4},{+31+53/ 60  })  {Castor}; 
\node[designation-label,anchor=south west]  at (axis cs:{11*15+ 3/4},{+61+45/ 60  })  {Dubhe};   
\node[designation-label,anchor=south west]  at (axis cs:{12*15+54/4},{+55+57/ 60  })  {Alioth};   
\node[designation-label,anchor=south]  at (axis cs:{13*15+40/4},{+51+18/ 60  })  {Alkaid};   
\node[designation-label,anchor=north]  at (axis cs:{18*15+36/4},{+38+46/ 60  })  {Vega};   
\node[designation-label,anchor=south west]  at (axis cs:{20*15+41/4},{+45+16/ 60  })  {Deneb};   


\node[ecliptics-empty,pin={[pin distance=-0.4\onedegree,interest-label]-90:{NEP}}] at (axis cs:{18*15},{+66+33/ 60  }) {\pgfuseplotmark{+}} ; %  ecliptic pole
\node[ecliptics-empty,pin={[pin distance=-0.4\onedegree,interest-label]-90:{SEP}}] at (axis cs:{5*15},{-66-33/ 60  }) {\pgfuseplotmark{+}} ; %  ecliptic pole

\node[interest,pin={[pin distance=-0.4\onedegree,interest-label]180:{RR}}] at (axis cs:291.3663040022100,+42.7843592383900) {\pgfuseplotmark{+}} ; %  RR Lyr

\node[pin={[pin distance=-0.0\onedegree,Bayer]00:{$\boldsymbol\alpha$}}] at (axis cs:{79.172},{45.998}) {}; % Aur,  0.08 
\node[pin={[pin distance=-0.2\onedegree,Bayer]-90:{$\boldsymbol\beta$}}] at (axis cs:{89.882},{44.947}) {}; % Aur,  1.90 
\node[pin={[pin distance=-0.6\onedegree,Bayer]00:{$\boldsymbol\chi$}}] at (axis cs:{83.182},{32.192}) {}; % Aur,  4.74 
\node[pin={[pin distance=-0.4\onedegree,Bayer]90:{$\boldsymbol\delta$}}] at (axis cs:{89.882},{54.285}) {}; % Aur,  3.73 
\node[pin={[pin distance=-0.2\onedegree,Bayer]00:{$\boldsymbol\epsilon$}}] at (axis cs:{75.492},{43.823}) {}; % Aur,  3.03 
\node[pin={[pin distance=-0.2\onedegree,Bayer]-90:{$\boldsymbol\eta$}}] at (axis cs:{76.629},{41.234}) {}; % Aur,  3.17 
\node[pin={[pin distance=-0.2\onedegree,Bayer]00:{$\boldsymbol\iota$}}] at (axis cs:{74.248},{33.166}) {}; % Aur,  2.68 
\node[pin={[pin distance=-0.6\onedegree,Bayer]00:{$\boldsymbol\kappa$}}] at (axis cs:{93.845},{29.498}) {}; % Aur,  4.33 
\node[pin={[pin distance=-0.6\onedegree,Bayer]90:{$\boldsymbol\lambda$}}] at (axis cs:{79.785},{40.099}) {}; % Aur,  4.69 
\node[pin={[pin distance=-0.6\onedegree,Bayer]00:{$\boldsymbol\mu$}}] at (axis cs:{78.357},{38.484}) {}; % Aur,  4.83 
\node[pin={[pin distance=-0.6\onedegree,Bayer]00:{$\boldsymbol\nu$}}] at (axis cs:{87.872},{39.148}) {}; % Aur,  3.97 
\node[pin={[pin distance=-0.6\onedegree,Bayer]00:{$\boldsymbol\omega$}}] at (axis cs:{74.814},{37.890}) {}; % Aur,  4.99 
\node[pin={[pin distance=-0.6\onedegree,Bayer]00:{$\boldsymbol\omicron$}}] at (axis cs:{86.475},{49.826}) {}; % Aur,  5.46 
\node[pin={[pin distance=-0.3\onedegree,Bayer]90:{$\boldsymbol\pi$}}] at (axis cs:{89.984},{45.937}) {}; % Aur,  4.33 
\node[pin={[pin distance=-0.6\onedegree,Bayer]00:{$\boldsymbol\psi^1$}}] at (axis cs:{96.225},{49.288}) {}; % Aur,  4.95 
\node[pin={[pin distance=-0.6\onedegree,Bayer]90:{$\boldsymbol\psi^2$}}] at (axis cs:{99.833},{42.489}) {}; % Aur,  4.79 
\node[pin={[pin distance=-0.6\onedegree,Bayer]90:{$\boldsymbol\psi^3$}}] at (axis cs:{99.705},{39.903}) {}; % Aur,  5.35 
\node[pin={[pin distance=-0.6\onedegree,Bayer]00:{$\boldsymbol\psi^4$}}] at (axis cs:{100.771},{44.524}) {}; % Aur,  5.03 
\node[pin={[pin distance=-0.6\onedegree,Bayer]00:{$\boldsymbol\psi^5$}}] at (axis cs:{101.685},{43.577}) {}; % Aur,  5.25 
\node[pin={[pin distance=-0.6\onedegree,Bayer]00:{$\boldsymbol\psi^6$}}] at (axis cs:{101.915},{48.789}) {}; % Aur,  5.22 
\node[pin={[pin distance=-0.6\onedegree,Bayer]-90:{$\boldsymbol\psi^7$}}] at (axis cs:{102.691},{41.781}) {}; % Aur,  4.98 
\node[pin={[pin distance=-0.6\onedegree,Bayer]-90:{$\boldsymbol\psi^8$}}] at (axis cs:{103.488},{38.505}) {}; % Aur,  6.45 
\node[pin={[pin distance=-0.6\onedegree,Bayer]90:{$\boldsymbol\psi^9$}}] at (axis cs:{104.134},{46.274}) {}; % Aur,  5.85 
\node[pin={[pin distance=-0.6\onedegree,Bayer]180:{$\boldsymbol\rho$}}] at (axis cs:{80.452},{41.804}) {}; % Aur,  5.21 
\node[pin={[pin distance=-0.6\onedegree,Bayer]00:{$\boldsymbol\sigma$}}] at (axis cs:{81.163},{37.385}) {}; % Aur,  4.99 
\node[pin={[pin distance=-0.6\onedegree,Bayer]00:{$\boldsymbol\tau$}}] at (axis cs:{87.293},{39.181}) {}; % Aur,  4.52 
\node[pin={[pin distance=-0.6\onedegree,Bayer]00:{$\boldsymbol\upsilon$}}] at (axis cs:{87.760},{37.305}) {}; % Aur,  4.73 
\node[pin={[pin distance=-0.3\onedegree,Bayer]90:{$\boldsymbol\varphi$}}] at (axis cs:{81.912},{34.476}) {}; % Aur,  5.06 
\node[pin={[pin distance=-0.2\onedegree,Bayer]90:{$\boldsymbol\vartheta$}}] at (axis cs:{89.930},{37.212}) {}; % Aur,  2.65 
\node[pin={[pin distance=-0.6\onedegree,Bayer]00:{$\boldsymbol\xi$}}] at (axis cs:{88.712},{55.707}) {}; % Aur,  4.97 
\node[pin={[pin distance=-0.4\onedegree,Bayer]00:{$\boldsymbol\zeta$}}] at (axis cs:{75.620},{41.076}) {}; % Aur,  3.76 
\node[pin={[pin distance=-0.6\onedegree,Flaamsted]00:{14}}] at (axis cs:{78.852},{32.688}) {}; % Aur,  4.99 
\node[pin={[pin distance=-0.6\onedegree,Flaamsted]00:{16}}] at (axis cs:{79.544},{33.372}) {}; % Aur,  4.54 
\node[pin={[pin distance=-0.6\onedegree,Flaamsted]180:{17}}] at (axis cs:{79.579},{33.767}) {}; % Aur,  6.15 
\node[pin={[pin distance=-0.6\onedegree,Flaamsted]00:{18}}] at (axis cs:{79.849},{33.985}) {}; % Aur,  6.49 
\node[pin={[pin distance=-0.6\onedegree,Flaamsted]-90:{19}}] at (axis cs:{80.004},{33.958}) {}; % Aur,  5.04 
\node[pin={[pin distance=-0.6\onedegree,Flaamsted]00:{22}}] at (axis cs:{80.845},{28.937}) {}; % Aur,  6.45 
\node[pin={[pin distance=-0.6\onedegree,Flaamsted]90:{26}}] at (axis cs:{84.659},{30.492}) {}; % Aur,  5.41 
\node[pin={[pin distance=-0.6\onedegree,Flaamsted]00:{ 2}}] at (axis cs:{73.158},{36.703}) {}; % Aur,  4.78 
\node[pin={[pin distance=-0.6\onedegree,Flaamsted]00:{36}}] at (axis cs:{90.244},{47.902}) {}; % Aur,  5.72 
\node[pin={[pin distance=-0.6\onedegree,Flaamsted]00:{38}}] at (axis cs:{90.825},{42.911}) {}; % Aur,  6.09 
\node[pin={[pin distance=-0.6\onedegree,Flaamsted]-90:{39}}] at (axis cs:{91.264},{42.981}) {}; % Aur,  5.90 
\node[pin={[pin distance=-0.6\onedegree,Flaamsted]00:{40}}] at (axis cs:{91.646},{38.482}) {}; % Aur,  5.35 
\node[pin={[pin distance=-0.6\onedegree,Flaamsted]00:{41}}] at (axis cs:{92.902},{48.711}) {}; % Aur,  6.17 
\node[pin={[pin distance=-0.6\onedegree,Flaamsted]00:{43}}] at (axis cs:{94.570},{46.360}) {}; % Aur,  6.33 
\node[pin={[pin distance=-0.6\onedegree,Flaamsted]00:{45}}] at (axis cs:{95.442},{53.452}) {}; % Aur,  5.34 
\node[pin={[pin distance=-0.6\onedegree,Flaamsted]00:{47}}] at (axis cs:{97.512},{46.686}) {}; % Aur,  5.87 
\node[pin={[pin distance=-0.6\onedegree,Flaamsted]00:{48}}] at (axis cs:{97.142},{30.493}) {}; % Aur,  5.75 
\node[pin={[pin distance=-0.6\onedegree,Flaamsted]00:{49}}] at (axis cs:{98.800},{28.022}) {}; % Aur,  5.27 
\node[pin={[pin distance=-0.6\onedegree,Flaamsted]0:{51}}] at (axis cs:{99.665},{39.391}) {}; % Aur,  5.70 
\node[pin={[pin distance=-0.6\onedegree,Flaamsted]00:{53}}] at (axis cs:{99.596},{28.984}) {}; % Aur,  5.75 
\node[pin={[pin distance=-0.6\onedegree,Flaamsted]00:{54}}] at (axis cs:{99.888},{28.263}) {}; % Aur,  6.06 
\node[pin={[pin distance=-0.6\onedegree,Flaamsted]90:{59}}] at (axis cs:{103.256},{38.869}) {}; % Aur,  6.10 
\node[pin={[pin distance=-0.6\onedegree,Flaamsted]00:{ 5}}] at (axis cs:{75.076},{39.395}) {}; % Aur,  5.98 
\node[pin={[pin distance=-0.6\onedegree,Flaamsted]00:{60}}] at (axis cs:{103.306},{38.438}) {}; % Aur,  6.33 
\node[pin={[pin distance=-0.6\onedegree,Flaamsted]00:{62}}] at (axis cs:{104.762},{38.052}) {}; % Aur,  6.02 
\node[pin={[pin distance=-0.6\onedegree,Flaamsted]00:{63}}] at (axis cs:{107.914},{39.320}) {}; % Aur,  4.89 
\node[pin={[pin distance=-0.6\onedegree,Flaamsted]00:{64}}] at (axis cs:{109.509},{40.883}) {}; % Aur,  5.87 
\node[pin={[pin distance=-0.6\onedegree,Flaamsted]-90:{65}}] at (axis cs:{110.511},{36.760}) {}; % Aur,  5.13 
\node[pin={[pin distance=-0.6\onedegree,Flaamsted]90:{66}}] at (axis cs:{111.035},{40.672}) {}; % Aur,  5.19 
\node[pin={[pin distance=-0.6\onedegree,Flaamsted]180:{ 6}}] at (axis cs:{75.097},{39.655}) {}; % Aur,  6.44 
\node[pin={[pin distance=-0.4\onedegree,Flaamsted]00:{ 9}}] at (axis cs:{76.669},{51.597}) {}; % Aur,  4.98


\node[pin={[pin distance=-0.2\onedegree,Bayer]-90:{$\boldsymbol\alpha$}}] at (axis cs:{113.649},{31.888}) {}; % Gem,  1.58 
\node[pin={[pin distance=-0.4\onedegree,Bayer]-90:{$\boldsymbol\omicron$}}] at (axis cs:{114.791},{34.584}) {}; % Gem,  4.89 
\node[pin={[pin distance=-0.6\onedegree,Bayer]00:{$\boldsymbol\pi$}}] at (axis cs:{116.876},{33.415}) {}; % Gem,  5.14 
\node[pin={[pin distance=-0.4\onedegree,Bayer]00:{$\boldsymbol\rho$}}] at (axis cs:{112.278},{31.785}) {}; % Gem,  4.18 
\node[pin={[pin distance=-0.6\onedegree,Bayer]00:{$\boldsymbol\tau$}}] at (axis cs:{107.785},{30.245}) {}; % Gem,  4.39 
\node[pin={[pin distance=-0.4\onedegree,Bayer]00:{$\boldsymbol\vartheta$}}] at (axis cs:{103.197},{33.961}) {}; % Gem,  3.61 
\node[pin={[pin distance=-0.6\onedegree,Flaamsted]00:{70}}] at (axis cs:{114.637},{35.048}) {}; % Gem,  5.58 


 
\node[pin={[pin distance=-0.2\onedegree,Bayer]00:{$\boldsymbol\alpha$}}] at (axis cs:{140.264},{34.392}) {}; % Lyn,  3.13 
\node[pin={[pin distance=-0.6\onedegree,Flaamsted]00:{11}}] at (axis cs:{99.410},{56.857}) {}; % Lyn,  5.88 
\node[pin={[pin distance=-0.6\onedegree,Flaamsted]00:{12}}] at (axis cs:{101.559},{59.442}) {}; % Lyn,  4.87 
\node[pin={[pin distance=-0.6\onedegree,Flaamsted]00:{13}}] at (axis cs:{101.706},{57.169}) {}; % Lyn,  5.34 
\node[pin={[pin distance=-0.6\onedegree,Flaamsted]00:{14}}] at (axis cs:{103.271},{59.448}) {}; % Lyn,  5.36 
\node[pin={[pin distance=-0.4\onedegree,Flaamsted]-90:{15}}] at (axis cs:{104.319},{58.423}) {}; % Lyn,  4.36 
\node[pin={[pin distance=-0.6\onedegree,Flaamsted]-90:{16}}] at (axis cs:{104.405},{45.094}) {}; % Lyn,  4.90 
\node[pin={[pin distance=-0.6\onedegree,Flaamsted]00:{18}}] at (axis cs:{108.979},{59.637}) {}; % Lyn,  5.20 
\node[pin={[pin distance=-0.6\onedegree,Flaamsted]00:{19}}] at (axis cs:{110.717},{55.281}) {}; % Lyn,  5.802
\node[pin={[pin distance=-0.6\onedegree,Flaamsted]-90:{ 1}}] at (axis cs:{94.478},{61.515}) {}; % Lyn,  5.03 
\node[pin={[pin distance=-0.6\onedegree,Flaamsted]00:{21}}] at (axis cs:{111.679},{49.211}) {}; % Lyn,  4.62 
\node[pin={[pin distance=-0.6\onedegree,Flaamsted]180:{22}}] at (axis cs:{112.483},{49.672}) {}; % Lyn,  5.36 
\node[pin={[pin distance=-0.6\onedegree,Flaamsted]00:{23}}] at (axis cs:{115.206},{57.083}) {}; % Lyn,  6.08 
\node[pin={[pin distance=-0.6\onedegree,Flaamsted]00:{24}}] at (axis cs:{115.752},{58.710}) {}; % Lyn,  4.96 
\node[pin={[pin distance=-0.6\onedegree,Flaamsted]00:{25}}] at (axis cs:{118.622},{47.386}) {}; % Lyn,  6.25 
\node[pin={[pin distance=-0.6\onedegree,Flaamsted]90:{26}}] at (axis cs:{118.678},{47.564}) {}; % Lyn,  5.45 
\node[pin={[pin distance=-0.5\onedegree,Flaamsted]00:{27}}] at (axis cs:{122.114},{51.507}) {}; % Lyn,  4.80 
\node[pin={[pin distance=-0.6\onedegree,Flaamsted]00:{28}}] at (axis cs:{121.791},{43.260}) {}; % Lyn,  6.35 
\node[pin={[pin distance=-0.6\onedegree,Flaamsted]00:{29}}] at (axis cs:{124.460},{59.571}) {}; % Lyn,  5.64 
\node[pin={[pin distance=-0.4\onedegree,Flaamsted]00:{ 2}}] at (axis cs:{94.906},{59.011}) {}; % Lyn,  4.45 
\node[pin={[pin distance=-0.6\onedegree,Flaamsted]00:{30}}] at (axis cs:{125.109},{57.743}) {}; % Lyn,  5.89 
\node[pin={[pin distance=-0.4\onedegree,Flaamsted]00:{31}}] at (axis cs:{125.709},{43.188}) {}; % Lyn,  4.24 
\node[pin={[pin distance=-0.6\onedegree,Flaamsted]00:{32}}] at (axis cs:{128.341},{36.436}) {}; % Lyn,  6.20 
\node[pin={[pin distance=-0.6\onedegree,Flaamsted]-90:{33}}] at (axis cs:{128.683},{36.420}) {}; % Lyn,  5.76 
\node[pin={[pin distance=-0.6\onedegree,Flaamsted]-90:{34}}] at (axis cs:{130.254},{45.834}) {}; % Lyn,  5.37 
\node[pin={[pin distance=-0.6\onedegree,Flaamsted]00:{35}}] at (axis cs:{132.987},{43.726}) {}; % Lyn,  5.15 
\node[pin={[pin distance=-0.6\onedegree,Flaamsted]00:{36}}] at (axis cs:{138.451},{43.218}) {}; % Lyn,  5.30 
\node[pin={[pin distance=-0.3\onedegree,Flaamsted]00:{38}}] at (axis cs:{139.711},{36.803}) {}; % Lyn,  3.82 
\node[pin={[pin distance=-0.6\onedegree,Flaamsted]90:{42}}] at (axis cs:{144.591},{40.240}) {}; % Lyn,  5.29 
\node[pin={[pin distance=-0.6\onedegree,Flaamsted]00:{43}}] at (axis cs:{145.501},{39.758}) {}; % Lyn,  5.61 
\node[pin={[pin distance=-0.6\onedegree,Flaamsted]-90:{ 4}}] at (axis cs:{95.515},{59.372}) {}; % Lyn,  6.09 
\node[pin={[pin distance=-0.6\onedegree,Flaamsted]00:{ 5}}] at (axis cs:{96.704},{58.417}) {}; % Lyn,  5.19 
\node[pin={[pin distance=-0.6\onedegree,Flaamsted]00:{ 6}}] at (axis cs:{97.696},{58.162}) {}; % Lyn,  5.86 
\node[pin={[pin distance=-0.6\onedegree,Flaamsted]00:{ 7}}] at (axis cs:{98.637},{55.353}) {}; % Lyn,  6.45 
\node[pin={[pin distance=-0.6\onedegree,Flaamsted]00:{ 8}}] at (axis cs:{99.422},{61.481}) {}; % Lyn,  5.94 


\node[pin={[pin distance=-0.6\onedegree,Bayer]00:{$\boldsymbol\sigma^1$}}] at (axis cs:{133.144},{32.474}) {}; % Cnc,  5.67 
\node[pin={[pin distance=-0.6\onedegree,Bayer]00:{$\boldsymbol\sigma^2$}}] at (axis cs:{134.236},{32.910}) {}; % Cnc,  5.44 
\node[pin={[pin distance=-0.6\onedegree,Bayer]00:{$\boldsymbol\sigma^3$}}] at (axis cs:{134.886},{32.419}) {}; % Cnc,  5.23 
\node[pin={[pin distance=-0.6\onedegree,Flaamsted]00:{46}}] at (axis cs:{131.339},{30.698}) {}; % Cnc,  6.12 
\node[pin={[pin distance=-0.6\onedegree,Flaamsted]90:{57}}] at (axis cs:{133.561},{30.579}) {}; % Cnc,  5.42 
\node[pin={[pin distance=-0.6\onedegree,Flaamsted]180:{61}}] at (axis cs:{134.494},{30.234}) {}; % Cnc,  6.25 
\node[pin={[pin distance=-0.6\onedegree,Flaamsted]-90:{66}}] at (axis cs:{135.351},{32.252}) {}; % Cnc,  5.95 


\node[pin={[pin distance=-0.6\onedegree,Bayer]00:{$\boldsymbol\beta$}}] at (axis cs:{156.971},{36.707}) {}; % LMi,  4.21 
\node[pin={[pin distance=-0.6\onedegree,Flaamsted]00:{10}}] at (axis cs:{143.556},{36.397}) {}; % LMi,  4.55 
\node[pin={[pin distance=-0.6\onedegree,Flaamsted]00:{11}}] at (axis cs:{143.915},{35.810}) {}; % LMi,  5.39 
\node[pin={[pin distance=-0.6\onedegree,Flaamsted]00:{13}}] at (axis cs:{145.678},{35.093}) {}; % LMi,  6.12 
\node[pin={[pin distance=-0.6\onedegree,Flaamsted]00:{19}}] at (axis cs:{149.421},{41.056}) {}; % LMi,  5.11 
\node[pin={[pin distance=-0.6\onedegree,Flaamsted]180:{20}}] at (axis cs:{150.253},{31.924}) {}; % LMi,  5.38 
\node[pin={[pin distance=-0.4\onedegree,Flaamsted]00:{21}}] at (axis cs:{151.857},{35.245}) {}; % LMi,  4.49 
\node[pin={[pin distance=-0.6\onedegree,Flaamsted]00:{22}}] at (axis cs:{153.776},{31.468}) {}; % LMi,  6.47 
\node[pin={[pin distance=-0.6\onedegree,Flaamsted]90:{27}}] at (axis cs:{155.776},{33.908}) {}; % LMi,  5.89 
\node[pin={[pin distance=-0.6\onedegree,Flaamsted]00:{28}}] at (axis cs:{156.036},{33.719}) {}; % LMi,  5.52 
\node[pin={[pin distance=-0.4\onedegree,Flaamsted]-90:{30}}] at (axis cs:{156.478},{33.796}) {}; % LMi,  4.73 
\node[pin={[pin distance=-0.6\onedegree,Flaamsted]00:{32}}] at (axis cs:{157.527},{38.925}) {}; % LMi,  5.79 
\node[pin={[pin distance=-0.6\onedegree,Flaamsted]00:{33}}] at (axis cs:{157.964},{32.379}) {}; % LMi,  5.91 
\node[pin={[pin distance=-0.6\onedegree,Flaamsted]00:{34}}] at (axis cs:{158.379},{34.989}) {}; % LMi,  5.58 
\node[pin={[pin distance=-0.6\onedegree,Flaamsted]00:{35}}] at (axis cs:{159.089},{36.327}) {}; % LMi,  6.29 
\node[pin={[pin distance=-0.6\onedegree,Flaamsted]00:{37}}] at (axis cs:{159.680},{31.976}) {}; % LMi,  4.68 
\node[pin={[pin distance=-0.6\onedegree,Flaamsted]00:{38}}] at (axis cs:{159.782},{37.910}) {}; % LMi,  5.84 
\node[pin={[pin distance=-0.6\onedegree,Flaamsted]00:{42}}] at (axis cs:{161.466},{30.682}) {}; % LMi,  5.26 
\node[pin={[pin distance=-0.3\onedegree,Flaamsted]00:{46}}] at (axis cs:{163.328},{34.215}) {}; % LMi,  3.79 
\node[pin={[pin distance=-0.6\onedegree,Flaamsted]00:{ 7}}] at (axis cs:{142.680},{33.656}) {}; % LMi,  5.86 
\node[pin={[pin distance=-0.6\onedegree,Flaamsted]00:{ 8}}] at (axis cs:{142.885},{35.103}) {}; % LMi,  5.38 
\node[pin={[pin distance=-0.6\onedegree,Flaamsted]90:{ 9}}] at (axis cs:{143.377},{36.487}) {}; % LMi,  6.18 

\node[pin={[pin distance=-0.6\onedegree,Flaamsted]-90:{37}}] at (axis cs:{195.069},{30.785}) {}; % Com,  4.89 


\node[pin={[pin distance=-0.6\onedegree,Bayer]180:{$\boldsymbol\lambda$}}] at (axis cs:{61.646},{50.351}) {}; % Per,  4.28 
\node[pin={[pin distance=-0.6\onedegree,Bayer]00:{$\boldsymbol\mu$}}] at (axis cs:{63.724},{48.409}) {}; % Per,  4.15 
\node[pin={[pin distance=-0.6\onedegree,Flaamsted]00:{48}}] at (axis cs:{62.165},{47.712}) {}; % Per,  4.01 
\node[pin={[pin distance=-0.6\onedegree,Flaamsted]00:{49}}] at (axis cs:{62.064},{37.727}) {}; % Per,  6.07 
\node[pin={[pin distance=-0.6\onedegree,Flaamsted]00:{50}}] at (axis cs:{62.153},{38.040}) {}; % Per,  5.52 
\node[pin={[pin distance=-0.6\onedegree,Flaamsted]00:{52}}] at (axis cs:{63.722},{40.484}) {}; % Per,  4.70 
\node[pin={[pin distance=-0.6\onedegree,Flaamsted]00:{53}}] at (axis cs:{65.388},{46.499}) {}; % Per,  4.82 
\node[pin={[pin distance=-0.6\onedegree,Flaamsted]00:{54}}] at (axis cs:{65.103},{34.567}) {}; % Per,  4.93 
\node[pin={[pin distance=-0.6\onedegree,Flaamsted]00:{55}}] at (axis cs:{66.121},{34.131}) {}; % Per,  5.73 
\node[pin={[pin distance=-0.6\onedegree,Flaamsted]00:{56}}] at (axis cs:{66.156},{33.960}) {}; % Per,  5.79 
\node[pin={[pin distance=-0.6\onedegree,Flaamsted]00:{57}}] at (axis cs:{68.354},{43.064}) {}; % Per,  6.09 
\node[pin={[pin distance=-0.6\onedegree,Flaamsted]00:{58}}] at (axis cs:{69.173},{41.265}) {}; % Per,  4.25 
\node[pin={[pin distance=-0.6\onedegree,Flaamsted]00:{59}}] at (axis cs:{70.726},{43.365}) {}; % Per,  5.30 



\node[pin={[pin distance=-0.4\onedegree,Bayer]00:{$\boldsymbol\alpha$}}] at (axis cs:{73.513},{66.343}) {}; % Cam,  4.31 
\node[pin={[pin distance=-0.4\onedegree,Bayer]00:{$\boldsymbol\beta$}}] at (axis cs:{75.855},{60.442}) {}; % Cam,  4.03 
\node[pin={[pin distance=-0.6\onedegree,Bayer]180:{$\boldsymbol\gamma$}}] at (axis cs:{57.590},{71.332}) {}; % Cam,  4.62 
\node[pin={[pin distance=-0.6\onedegree,Flaamsted]00:{11}}] at (axis cs:{76.535},{58.972}) {}; % Cam,  5.23 
\node[pin={[pin distance=-0.6\onedegree,Flaamsted]180:{14}}] at (axis cs:{78.380},{62.691}) {}; % Cam,  6.49 
\node[pin={[pin distance=-0.6\onedegree,Flaamsted]90:{15}}] at (axis cs:{79.866},{58.117}) {}; % Cam,  6.13 
\node[pin={[pin distance=-0.6\onedegree,Flaamsted]00:{16}}] at (axis cs:{80.866},{57.544}) {}; % Cam,  5.25 
\node[pin={[pin distance=-0.6\onedegree,Flaamsted]00:{17}}] at (axis cs:{82.543},{63.067}) {}; % Cam,  5.44 
\node[pin={[pin distance=-0.6\onedegree,Flaamsted]00:{18}}] at (axis cs:{83.141},{57.221}) {}; % Cam,  6.46 
\node[pin={[pin distance=-0.6\onedegree,Flaamsted]00:{19}}] at (axis cs:{84.313},{64.155}) {}; % Cam,  6.17 
\node[pin={[pin distance=-0.6\onedegree,Flaamsted]00:{ 1}}] at (axis cs:{68.008},{53.911}) {}; % Cam,  5.77 
\node[pin={[pin distance=-0.6\onedegree,Flaamsted]00:{23}}] at (axis cs:{86.035},{61.476}) {}; % Cam,  6.16 
\node[pin={[pin distance=-0.6\onedegree,Flaamsted]180:{24}}] at (axis cs:{85.757},{56.581}) {}; % Cam,  6.05 
\node[pin={[pin distance=-0.6\onedegree,Flaamsted]180:{26}}] at (axis cs:{86.627},{56.115}) {}; % Cam,  5.93 
\node[pin={[pin distance=-0.6\onedegree,Flaamsted]-90:{ 2}}] at (axis cs:{69.992},{53.473}) {}; % Cam,  5.37 
\node[pin={[pin distance=-0.6\onedegree,Flaamsted]00:{30}}] at (axis cs:{88.072},{58.964}) {}; % Cam,  6.14 
\node[pin={[pin distance=-0.6\onedegree,Flaamsted]00:{31}}] at (axis cs:{88.741},{59.888}) {}; % Cam,  5.20 
\node[pin={[pin distance=-0.6\onedegree,Flaamsted]00:{36}}] at (axis cs:{93.213},{65.718}) {}; % Cam,  5.35 
\node[pin={[pin distance=-0.6\onedegree,Flaamsted]00:{37}}] at (axis cs:{92.496},{58.936}) {}; % Cam,  5.35 
\node[pin={[pin distance=-0.6\onedegree,Flaamsted]90:{ 3}}] at (axis cs:{69.978},{53.079}) {}; % Cam,  5.08 
\node[pin={[pin distance=-0.6\onedegree,Flaamsted]00:{40}}] at (axis cs:{93.919},{59.999}) {}; % Cam,  5.35 
\node[pin={[pin distance=-0.6\onedegree,Flaamsted]00:{42}}] at (axis cs:{102.738},{67.572}) {}; % Cam,  5.14 
\node[pin={[pin distance=-0.6\onedegree,Flaamsted]00:{43}}] at (axis cs:{103.426},{68.888}) {}; % Cam,  5.11 
\node[pin={[pin distance=-0.6\onedegree,Flaamsted]180:{47}}] at (axis cs:{110.572},{59.902}) {}; % Cam,  6.37 
\node[pin={[pin distance=-0.6\onedegree,Flaamsted]00:{49}}] at (axis cs:{116.614},{62.830}) {}; % Cam,  6.47 
\node[pin={[pin distance=-0.6\onedegree,Flaamsted]00:{ 4}}] at (axis cs:{72.001},{56.757}) {}; % Cam,  5.28 
\node[pin={[pin distance=-0.6\onedegree,Flaamsted]00:{51}}] at (axis cs:{116.667},{65.456}) {}; % Cam,  5.94 
\node[pin={[pin distance=-0.6\onedegree,Flaamsted]180:{53}}] at (axis cs:{120.427},{60.324}) {}; % Cam,  6.02 
\node[pin={[pin distance=-0.6\onedegree,Flaamsted]00:{ 5}}] at (axis cs:{73.763},{55.259}) {}; % Cam,  5.53 
\node[pin={[pin distance=-0.6\onedegree,Flaamsted]90:{ 7}}] at (axis cs:{74.322},{53.752}) {}; % Cam,  4.46 
\node[pin={[pin distance=-0.6\onedegree,Flaamsted]-90:{ 8}}] at (axis cs:{74.943},{53.155}) {}; % Cam,  6.09 


\node[pin={[pin distance=-0.6\onedegree,Bayer]00:{$\boldsymbol\iota$}}] at (axis cs:{37.266},{67.402}) {}; % Cas,  4.63
\node[pin={[pin distance=-0.6\onedegree,Bayer]90:{$\boldsymbol\omega$}}] at (axis cs:{29.000},{68.685}) {}; % Cas,  4.98 
\node[pin={[pin distance=-0.6\onedegree,Flaamsted]180:{21}}] at (axis cs:{11.413},{74.988}) {}; % Cas,  5.65 
\node[pin={[pin distance=-0.6\onedegree,Flaamsted]00:{23}}] at (axis cs:{11.942},{74.847}) {}; % Cas,  5.43 

\node[pin={[pin distance=-0.6\onedegree,Flaamsted]00:{31}}] at (axis cs:{17.664},{68.778}) {}; % Cas,  5.32 
\node[pin={[pin distance=-0.6\onedegree,Flaamsted]00:{32}}] at (axis cs:{17.923},{65.019}) {}; % Cas,  5.57 
\node[pin={[pin distance=-0.6\onedegree,Flaamsted]00:{35}}] at (axis cs:{20.272},{64.658}) {}; % Cas,  6.33 
\node[pin={[pin distance=-0.6\onedegree,Flaamsted]00:{38}}] at (axis cs:{22.807},{70.264}) {}; % Cas,  5.83 
\node[pin={[pin distance=-0.6\onedegree,Flaamsted]-90:{40}}] at (axis cs:{24.629},{73.040}) {}; % Cas,  5.28 
\node[pin={[pin distance=-0.6\onedegree,Flaamsted]00:{42}}] at (axis cs:{25.733},{70.622}) {}; % Cas,  5.18 
\node[pin={[pin distance=-0.6\onedegree,Flaamsted]00:{43}}] at (axis cs:{25.586},{68.043}) {}; % Cas,  5.57 
\node[pin={[pin distance=-0.6\onedegree,Flaamsted]00:{44}}] at (axis cs:{25.832},{60.551}) {}; % Cas,  5.78 
\node[pin={[pin distance=-0.6\onedegree,Flaamsted]00:{47}}] at (axis cs:{31.281},{77.281}) {}; % Cas,  5.27 
\node[pin={[pin distance=-0.6\onedegree,Flaamsted]00:{48}}] at (axis cs:{30.489},{70.907}) {}; % Cas,  4.52 
\node[pin={[pin distance=-0.6\onedegree,Flaamsted]00:{49}}] at (axis cs:{31.381},{76.115}) {}; % Cas,  5.22 
\node[pin={[pin distance=-0.4\onedegree,Flaamsted]-90:{50}}] at (axis cs:{30.859},{72.421}) {}; % Cas,  3.95 
\node[pin={[pin distance=-0.6\onedegree,Flaamsted]00:{52}}] at (axis cs:{30.719},{64.901}) {}; % Cas,  6.00 
\node[pin={[pin distance=-0.6\onedegree,Flaamsted]00:{53}}] at (axis cs:{30.751},{64.390}) {}; % Cas,  5.62 
\node[pin={[pin distance=-0.6\onedegree,Flaamsted]00:{55}}] at (axis cs:{33.621},{66.524}) {}; % Cas,  6.06 

\node[pin={[pin distance=-0.4\onedegree,Bayer]00:{$\boldsymbol\alpha$}}] at (axis cs:{319.645},{62.586}) {}; % Cep,  2.47 
\node[pin={[pin distance=-0.4\onedegree,Bayer]00:{$\boldsymbol\beta$}}] at (axis cs:{322.165},{70.561}) {}; % Cep,  3.23 
\node[pin={[pin distance=-0.6\onedegree,Bayer]00:{$\boldsymbol\delta$}}] at (axis cs:{337.293},{58.415}) {}; % Cep,  4.07 
\node[pin={[pin distance=-0.6\onedegree,Bayer]00:{$\boldsymbol\epsilon$}}] at (axis cs:{333.759},{57.043}) {}; % Cep,  4.18 
\node[pin={[pin distance=-0.4\onedegree,Bayer]00:{$\boldsymbol\eta$}}] at (axis cs:{311.322},{61.839}) {}; % Cep,  3.42 
\node[pin={[pin distance=-0.4\onedegree,Bayer]00:{$\boldsymbol\gamma$}}] at (axis cs:{354.837},{77.632}) {}; % Cep,  3.22 
\node[pin={[pin distance=-0.6\onedegree,Bayer]00:{$\boldsymbol\iota$}}] at (axis cs:{342.420},{66.200}) {}; % Cep,  3.50 
\node[pin={[pin distance=-0.6\onedegree,Bayer]00:{$\boldsymbol\kappa$}}] at (axis cs:{302.222},{77.711}) {}; % Cep,  4.40 
\node[pin={[pin distance=-0.6\onedegree,Bayer]00:{$\boldsymbol\lambda$}}] at (axis cs:{332.877},{59.414}) {}; % Cep,  5.09 
\node[pin={[pin distance=-0.6\onedegree,Bayer]00:{$\boldsymbol\mu$}}] at (axis cs:{325.877},{58.780}) {}; % Cep,  4.02 
\node[pin={[pin distance=-0.6\onedegree,Bayer]00:{$\boldsymbol\nu$}}] at (axis cs:{326.362},{61.121}) {}; % Cep,  4.31 
\node[pin={[pin distance=-0.6\onedegree,Bayer]00:{$\boldsymbol\omicron$}}] at (axis cs:{349.656},{68.111}) {}; % Cep,  4.86 
\node[pin={[pin distance=-0.4\onedegree,Bayer]-90:{$\boldsymbol\pi$}}] at (axis cs:{346.974},{75.387}) {}; % Cep,  4.42 
\node[pin={[pin distance=-0.6\onedegree,Bayer]180:{$\boldsymbol\rho^1$}}] at (axis cs:{336.677},{78.786}) {}; % Cep,  5.84 
\node[pin={[pin distance=-0.6\onedegree,Bayer]00:{$\boldsymbol\rho^2$}}] at (axis cs:{337.471},{78.824}) {}; % Cep,  5.45 
\node[pin={[pin distance=-0.6\onedegree,Bayer]00:{$\boldsymbol\vartheta$}}] at (axis cs:{307.395},{62.994}) {}; % Cep,  4.21 
\node[pin={[pin distance=-0.6\onedegree,Bayer]00:{$\boldsymbol\xi$}}] at (axis cs:{330.948},{64.628}) {}; % Cep,  4.40 
\node[pin={[pin distance=-0.6\onedegree,Bayer]00:{$\boldsymbol\zeta$}}] at (axis cs:{332.714},{58.201}) {}; % Cep,  3.34 
\node[pin={[pin distance=-0.6\onedegree,Flaamsted]00:{11}}] at (axis cs:{325.480},{71.311}) {}; % Cep,  4.54 
\node[pin={[pin distance=-0.6\onedegree,Flaamsted]00:{12}}] at (axis cs:{326.855},{60.693}) {}; % Cep,  5.52 
\node[pin={[pin distance=-0.6\onedegree,Flaamsted]00:{13}}] at (axis cs:{328.721},{56.611}) {}; % Cep,  5.80 
\node[pin={[pin distance=-0.6\onedegree,Flaamsted]00:{14}}] at (axis cs:{330.519},{58.000}) {}; % Cep,  5.55 
\node[pin={[pin distance=-0.6\onedegree,Flaamsted]00:{16}}] at (axis cs:{329.812},{73.180}) {}; % Cep,  5.04 
\node[pin={[pin distance=-0.6\onedegree,Flaamsted]00:{18}}] at (axis cs:{330.971},{63.120}) {}; % Cep,  5.29 
\node[pin={[pin distance=-0.6\onedegree,Flaamsted]00:{19}}] at (axis cs:{331.287},{62.280}) {}; % Cep,  5.12 
\node[pin={[pin distance=-0.6\onedegree,Flaamsted]00:{20}}] at (axis cs:{331.252},{62.786}) {}; % Cep,  5.26 
\node[pin={[pin distance=-0.6\onedegree,Flaamsted]00:{24}}] at (axis cs:{332.452},{72.341}) {}; % Cep,  4.80 
\node[pin={[pin distance=-0.6\onedegree,Flaamsted]00:{25}}] at (axis cs:{334.553},{62.804}) {}; % Cep,  5.74 
\node[pin={[pin distance=-0.6\onedegree,Flaamsted]00:{26}}] at (axis cs:{336.772},{65.132}) {}; % Cep,  5.53 
\node[pin={[pin distance=-0.6\onedegree,Flaamsted]00:{30}}] at (axis cs:{339.663},{63.584}) {}; % Cep,  5.20 
\node[pin={[pin distance=-0.6\onedegree,Flaamsted]-90:{31}}] at (axis cs:{338.942},{73.643}) {}; % Cep,  5.08 
\node[pin={[pin distance=-0.6\onedegree,Flaamsted]00:{ 4}}] at (axis cs:{310.796},{66.657}) {}; % Cep,  5.59 
\node[pin={[pin distance=-0.6\onedegree,Flaamsted]00:{ 6}}] at (axis cs:{319.843},{64.872}) {}; % Cep,  5.19 
\node[pin={[pin distance=-0.6\onedegree,Flaamsted]00:{ 7}}] at (axis cs:{321.942},{66.809}) {}; % Cep,  5.42 
\node[pin={[pin distance=-0.6\onedegree,Flaamsted]00:{ 9}}] at (axis cs:{324.480},{62.082}) {}; % Cep,  4.79 



\node[pin={[pin distance=-0.2\onedegree,Bayer]00:{$\boldsymbol\alpha$}}] at (axis cs:{165.932},{61.751}) {}; % UMa,  1.82 
\node[pin={[pin distance=-0.4\onedegree,Bayer]00:{$\boldsymbol\beta$}}] at (axis cs:{165.460},{56.382}) {}; % UMa,  2.35 
\node[pin={[pin distance=-0.2\onedegree,Bayer]00:{$\boldsymbol\epsilon$}}] at (axis cs:{193.507},{55.960}) {}; % UMa,  1.76 
\node[pin={[pin distance=-0.\onedegree,Bayer]00:{$\boldsymbol\eta$}}] at (axis cs:{206.885},{49.313}) {}; % UMa,  1.86 
\node[pin={[pin distance=-0.4\onedegree,Bayer]00:{$\boldsymbol\gamma$}}] at (axis cs:{178.458},{53.695}) {}; % UMa,  2.43 
\node[pin={[pin distance=-0.2\onedegree,Bayer]-90:{$\boldsymbol\zeta$}}] at (axis cs:{200.981},{54.925}) {}; % UMa,  2.22 
\node[pin={[pin distance=-0.4\onedegree,Bayer]00:{$\boldsymbol\chi$}}] at (axis cs:{176.513},{47.779}) {}; % UMa,  3.70 
\node[pin={[pin distance=-0.2\onedegree,Bayer]-90:{$\boldsymbol\delta$}}] at (axis cs:{183.857},{57.032}) {}; % UMa,  3.30 
\node[pin={[pin distance=-0.4\onedegree,Bayer]00:{$\boldsymbol\iota$}}] at (axis cs:{134.802},{48.042}) {}; % UMa,  3.14 
\node[pin={[pin distance=-0.4\onedegree,Bayer]00:{$\boldsymbol\kappa$}}] at (axis cs:{135.906},{47.156}) {}; % UMa,  3.58 
\node[pin={[pin distance=-0.4\onedegree,Bayer]00:{$\boldsymbol\lambda$}}] at (axis cs:{154.274},{42.914}) {}; % UMa,  3.44 
\node[pin={[pin distance=-0.2\onedegree,Bayer]180:{$\boldsymbol\mu$}}] at (axis cs:{155.582},{41.499}) {}; % UMa,  3.05 
\node[pin={[pin distance=-0.4\onedegree,Bayer]00:{$\boldsymbol\nu$}}] at (axis cs:{169.620},{33.094}) {}; % UMa,  3.49 
\node[pin={[pin distance=-0.6\onedegree,Bayer]00:{$\boldsymbol\omega$}}] at (axis cs:{163.495},{43.190}) {}; % UMa,  4.67 
\node[pin={[pin distance=-0.4\onedegree,Bayer]00:{$\boldsymbol\omicron$}}] at (axis cs:{127.566},{60.718}) {}; % UMa,  3.36 
\node[pin={[pin distance=-0.6\onedegree,Bayer]180:{$\boldsymbol\pi^1$}}] at (axis cs:{129.799},{65.021}) {}; % UMa,  5.64 
\node[pin={[pin distance=-0.6\onedegree,Bayer]00:{$\boldsymbol\pi^2$}}] at (axis cs:{130.053},{64.328}) {}; % UMa,  4.59 
\node[pin={[pin distance=-0.2\onedegree,Bayer]00:{$\boldsymbol\psi$}}] at (axis cs:{167.416},{44.498}) {}; % UMa,  3.00 
\node[pin={[pin distance=-0.6\onedegree,Bayer]00:{$\boldsymbol\rho$}}] at (axis cs:{135.636},{67.629}) {}; % UMa,  4.77 
\node[pin={[pin distance=-0.6\onedegree,Bayer]00:{$\boldsymbol\sigma^1$}}] at (axis cs:{137.098},{66.873}) {}; % UMa,  5.14 
\node[pin={[pin distance=-0.6\onedegree,Bayer]-90:{$\boldsymbol\sigma^2$}}] at (axis cs:{137.598},{67.134}) {}; % UMa,  4.82 
\node[pin={[pin distance=-0.6\onedegree,Bayer]00:{$\boldsymbol\tau$}}] at (axis cs:{137.729},{63.513}) {}; % UMa,  4.64 
\node[pin={[pin distance=-0.4\onedegree,Bayer]00:{$\boldsymbol\upsilon$}}] at (axis cs:{147.747},{59.039}) {}; % UMa,  3.78 
\node[pin={[pin distance=-0.6\onedegree,Bayer]00:{$\boldsymbol\varphi$}}] at (axis cs:{148.027},{54.064}) {}; % UMa,  4.56 
\node[pin={[pin distance=-0.4\onedegree,Bayer]00:{$\boldsymbol\vartheta$}}] at (axis cs:{143.214},{51.677}) {}; % UMa,  3.18 
\node[pin={[pin distance=-0.4\onedegree,Bayer]00:{$\boldsymbol\xi$}}] at (axis cs:{169.546},{31.529}) {}; % UMa,  3.79 
\node[pin={[pin distance=-0.6\onedegree,Flaamsted]00:{15}}] at (axis cs:{137.218},{51.605}) {}; % UMa,  4.46 
\node[pin={[pin distance=-0.6\onedegree,Flaamsted]00:{16}}] at (axis cs:{138.586},{61.423}) {}; % UMa,  5.19 
\node[pin={[pin distance=-0.6\onedegree,Flaamsted]00:{17}}] at (axis cs:{138.957},{56.741}) {}; % UMa,  5.28 
\node[pin={[pin distance=-0.6\onedegree,Flaamsted]00:{18}}] at (axis cs:{139.047},{54.022}) {}; % UMa,  4.82 
\node[pin={[pin distance=-0.6\onedegree,Flaamsted]00:{22}}] at (axis cs:{143.723},{72.206}) {}; % UMa,  5.77 
\node[pin={[pin distance=-0.4\onedegree,Flaamsted]00:{23}}] at (axis cs:{142.882},{63.062}) {}; % UMa,  3.65 
\node[pin={[pin distance=-0.6\onedegree,Flaamsted]00:{24}}] at (axis cs:{143.620},{69.830}) {}; % UMa,  4.56 
\node[pin={[pin distance=-0.4\onedegree,Flaamsted]180:{26}}] at (axis cs:{143.706},{52.051}) {}; % UMa,  4.48 
\node[pin={[pin distance=-0.6\onedegree,Flaamsted]90:{27}}] at (axis cs:{145.738},{72.253}) {}; % UMa,  5.14 
\node[pin={[pin distance=-0.6\onedegree,Flaamsted]00:{ 2}}] at (axis cs:{128.651},{65.145}) {}; % UMa,  5.46 
\node[pin={[pin distance=-0.6\onedegree,Flaamsted]00:{31}}] at (axis cs:{148.929},{49.820}) {}; % UMa,  5.28 
\node[pin={[pin distance=-0.6\onedegree,Flaamsted]00:{32}}] at (axis cs:{154.508},{65.108}) {}; % UMa,  5.75 
\node[pin={[pin distance=-0.6\onedegree,Flaamsted]-90:{35}}] at (axis cs:{157.477},{65.626}) {}; % UMa,  6.31 
\node[pin={[pin distance=-0.6\onedegree,Flaamsted]00:{36}}] at (axis cs:{157.657},{55.980}) {}; % UMa,  4.83 
\node[pin={[pin distance=-0.6\onedegree,Flaamsted]00:{37}}] at (axis cs:{158.790},{57.082}) {}; % UMa,  5.16 
\node[pin={[pin distance=-0.6\onedegree,Flaamsted]180:{38}}] at (axis cs:{160.486},{65.716}) {}; % UMa,  5.11 
\node[pin={[pin distance=-0.6\onedegree,Flaamsted]90:{39}}] at (axis cs:{160.931},{57.199}) {}; % UMa,  5.79 
\node[pin={[pin distance=-0.6\onedegree,Flaamsted]180:{41}}] at (axis cs:{161.594},{57.366}) {}; % UMa,  6.35 
\node[pin={[pin distance=-0.6\onedegree,Flaamsted]00:{42}}] at (axis cs:{162.849},{59.320}) {}; % UMa,  5.56 
\node[pin={[pin distance=-0.6\onedegree,Flaamsted]00:{43}}] at (axis cs:{162.796},{56.582}) {}; % UMa,  5.66 
\node[pin={[pin distance=-0.6\onedegree,Flaamsted]00:{44}}] at (axis cs:{163.394},{54.585}) {}; % UMa,  5.12 
\node[pin={[pin distance=-0.6\onedegree,Flaamsted]180:{46}}] at (axis cs:{163.935},{33.507}) {}; % UMa,  5.03 
\node[pin={[pin distance=-0.6\onedegree,Flaamsted]00:{47}}] at (axis cs:{164.867},{40.430}) {}; % UMa,  5.03 
\node[pin={[pin distance=-0.6\onedegree,Flaamsted]00:{49}}] at (axis cs:{165.210},{39.212}) {}; % UMa,  5.07 
\node[pin={[pin distance=-0.6\onedegree,Flaamsted]00:{51}}] at (axis cs:{166.130},{38.241}) {}; % UMa,  6.03 
\node[pin={[pin distance=-0.6\onedegree,Flaamsted]00:{55}}] at (axis cs:{169.783},{38.186}) {}; % UMa,  4.76 
\node[pin={[pin distance=-0.6\onedegree,Flaamsted]00:{56}}] at (axis cs:{170.707},{43.482}) {}; % UMa,  4.99 
\node[pin={[pin distance=-0.6\onedegree,Flaamsted]00:{57}}] at (axis cs:{172.267},{39.337}) {}; % UMa,  5.36 
\node[pin={[pin distance=-0.6\onedegree,Flaamsted]00:{58}}] at (axis cs:{172.630},{43.173}) {}; % UMa,  5.95 
\node[pin={[pin distance=-0.6\onedegree,Flaamsted]00:{59}}] at (axis cs:{174.586},{43.625}) {}; % UMa,  5.57 
\node[pin={[pin distance=-0.6\onedegree,Flaamsted]00:{ 5}}] at (axis cs:{133.344},{61.962}) {}; % UMa,  5.72 
\node[pin={[pin distance=-0.6\onedegree,Flaamsted]-90:{60}}] at (axis cs:{174.640},{46.834}) {}; % UMa,  6.08 
\node[pin={[pin distance=-0.6\onedegree,Flaamsted]00:{61}}] at (axis cs:{175.263},{34.202}) {}; % UMa,  5.31 
\node[pin={[pin distance=-0.6\onedegree,Flaamsted]00:{62}}] at (axis cs:{175.393},{31.746}) {}; % UMa,  5.75 
\node[pin={[pin distance=-0.6\onedegree,Flaamsted]90:{66}}] at (axis cs:{178.993},{56.598}) {}; % UMa,  5.83 
\node[pin={[pin distance=-0.6\onedegree,Flaamsted]-90:{67}}] at (axis cs:{180.528},{43.045}) {}; % UMa,  5.21 
\node[pin={[pin distance=-0.6\onedegree,Flaamsted]00:{68}}] at (axis cs:{182.937},{57.054}) {}; % UMa,  6.33 
\node[pin={[pin distance=-0.6\onedegree,Flaamsted]00:{ 6}}] at (axis cs:{134.156},{64.604}) {}; % UMa,  5.57 
\node[pin={[pin distance=-0.6\onedegree,Flaamsted]00:{70}}] at (axis cs:{185.212},{57.864}) {}; % UMa,  5.52 
\node[pin={[pin distance=-0.6\onedegree,Flaamsted]00:{71}}] at (axis cs:{186.263},{56.778}) {}; % UMa,  5.84 
\node[pin={[pin distance=-0.6\onedegree,Flaamsted]00:{73}}] at (axis cs:{186.896},{55.713}) {}; % UMa,  5.69 
\node[pin={[pin distance=-0.6\onedegree,Flaamsted]00:{74}}] at (axis cs:{187.489},{58.406}) {}; % UMa,  5.36 
\node[pin={[pin distance=-0.6\onedegree,Flaamsted]90:{75}}] at (axis cs:{187.518},{58.768}) {}; % UMa,  6.07 
\node[pin={[pin distance=-0.6\onedegree,Flaamsted]00:{76}}] at (axis cs:{190.391},{62.713}) {}; % UMa,  6.02 
\node[pin={[pin distance=-0.6\onedegree,Flaamsted]80:{78}}] at (axis cs:{195.182},{56.366}) {}; % UMa,  4.94 
\node[pin={[pin distance=-0.6\onedegree,Flaamsted]180:{80}}] at (axis cs:{201.306},{54.988}) {}; % UMa,  4.00 
\node[pin={[pin distance=-0.6\onedegree,Flaamsted]00:{81}}] at (axis cs:{203.530},{55.348}) {}; % UMa,  5.60 
\node[pin={[pin distance=-0.6\onedegree,Flaamsted]00:{82}}] at (axis cs:{204.877},{52.921}) {}; % UMa,  5.47 
\node[pin={[pin distance=-0.6\onedegree,Flaamsted]00:{83}}] at (axis cs:{205.184},{54.681}) {}; % UMa,  4.63 
\node[pin={[pin distance=-0.6\onedegree,Flaamsted]90:{84}}] at (axis cs:{206.649},{54.432}) {}; % UMa,  5.69 
\node[pin={[pin distance=-0.6\onedegree,Flaamsted]00:{86}}] at (axis cs:{208.463},{53.728}) {}; % UMa,  5.70 


\node[pin={[pin distance=-0.4\onedegree,Bayer]00:{$\boldsymbol\alpha^2$}}] at (axis cs:{194.007},{38.318}) {}; % CVn,  2.89 
\node[pin={[pin distance=-0.3\onedegree,Bayer]-90:{$\boldsymbol\beta$}}] at (axis cs:{188.436},{41.357}) {}; % CVn,  4.25 
\node[pin={[pin distance=-0.6\onedegree,Flaamsted]00:{10}}] at (axis cs:{191.248},{39.279}) {}; % CVn,  5.96 
\node[pin={[pin distance=-0.6\onedegree,Flaamsted]00:{11}}] at (axis cs:{192.174},{48.467}) {}; % CVn,  6.25 
\node[pin={[pin distance=-0.6\onedegree,Flaamsted]00:{14}}] at (axis cs:{196.435},{35.799}) {}; % CVn,  5.20 
\node[pin={[pin distance=-0.6\onedegree,Flaamsted]00:{17}}] at (axis cs:{197.513},{38.499}) {}; % CVn,  5.93 
\node[pin={[pin distance=-0.6\onedegree,Flaamsted]00:{19}}] at (axis cs:{198.883},{40.855}) {}; % CVn,  5.78 
\node[pin={[pin distance=-0.6\onedegree,Flaamsted]90:{20}}] at (axis cs:{199.386},{40.573}) {}; % CVn,  4.71 
\node[pin={[pin distance=-0.6\onedegree,Flaamsted]-90:{21}}] at (axis cs:{199.560},{49.682}) {}; % CVn,  5.15 
\node[pin={[pin distance=-0.6\onedegree,Flaamsted]180:{23}}] at (axis cs:{200.079},{40.151}) {}; % CVn,  5.60 
\node[pin={[pin distance=-0.6\onedegree,Flaamsted]00:{24}}] at (axis cs:{203.614},{49.016}) {}; % CVn,  4.67 
\node[pin={[pin distance=-0.6\onedegree,Flaamsted]00:{25}}] at (axis cs:{204.365},{36.295}) {}; % CVn,  4.91 
\node[pin={[pin distance=-0.6\onedegree,Flaamsted]00:{ 2}}] at (axis cs:{184.031},{40.660}) {}; % CVn,  5.67 
\node[pin={[pin distance=-0.6\onedegree,Flaamsted]00:{ 3}}] at (axis cs:{184.953},{48.984}) {}; % CVn,  5.29 
\node[pin={[pin distance=-0.6\onedegree,Flaamsted]00:{ 4}}] at (axis cs:{185.946},{42.543}) {}; % CVn,  6.03 
\node[pin={[pin distance=-0.4\onedegree,Flaamsted]00:{ 5}}] at (axis cs:{186.006},{51.562}) {}; % CVn,  4.76 
\node[pin={[pin distance=-0.6\onedegree,Flaamsted]00:{ 6}}] at (axis cs:{186.462},{39.019}) {}; % CVn,  5.02 
\node[pin={[pin distance=-0.6\onedegree,Flaamsted]00:{ 7}}] at (axis cs:{187.512},{51.536}) {}; % CVn,  6.22 
\node[pin={[pin distance=-0.6\onedegree,Flaamsted]00:{ 9}}] at (axis cs:{189.693},{40.875}) {}; % CVn,  6.35 


\node[pin={[pin distance=-0.3\onedegree,Bayer]00:{$\boldsymbol\beta$}}] at (axis cs:{225.487},{40.390}) {}; % Boo,  3.48 
\node[pin={[pin distance=-0.4\onedegree,Bayer]00:{$\boldsymbol\delta$}}] at (axis cs:{228.876},{33.315}) {}; % Boo,  3.47 
\node[pin={[pin distance=-0.4\onedegree,Bayer]00:{$\boldsymbol\gamma$}}] at (axis cs:{218.019},{38.308}) {}; % Boo,  3.04 
\node[pin={[pin distance=-0.6\onedegree,Bayer]180:{$\boldsymbol\iota$}}] at (axis cs:{214.041},{51.367}) {}; % Boo,  4.75 
\node[pin={[pin distance=-0.6\onedegree,Bayer]00:{$\boldsymbol\kappa^2$}}] at (axis cs:{213.371},{51.790}) {}; % Boo,  4.52 
\node[pin={[pin distance=-0.4\onedegree,Bayer]00:{$\boldsymbol\lambda$}}] at (axis cs:{214.096},{46.088}) {}; % Boo,  4.18 
\node[pin={[pin distance=-0.5\onedegree,Bayer]00:{$\boldsymbol\mu^1$}}] at (axis cs:{231.123},{37.377}) {}; % Boo,  4.30 
\node[pin={[pin distance=-0.6\onedegree,Bayer]-90:{$\boldsymbol\nu^1$}}] at (axis cs:{232.732},{40.833}) {}; % Boo,  5.03 
\node[pin={[pin distance=-0.6\onedegree,Bayer]90:{$\boldsymbol\nu^2$}}] at (axis cs:{232.946},{40.899}) {}; % Boo,  5.00 
\node[pin={[pin distance=-0.4\onedegree,Bayer]00:{$\boldsymbol\rho$}}] at (axis cs:{217.957},{30.371}) {}; % Boo,  3.57 
\node[pin={[pin distance=-0.4\onedegree,Bayer]90:{$\boldsymbol\varphi$}}] at (axis cs:{234.457},{40.353}) {}; % Boo,  5.26 
\node[pin={[pin distance=-0.4\onedegree,Bayer]00:{$\boldsymbol\vartheta$}}] at (axis cs:{216.299},{51.851}) {}; % Boo,  4.05 
\node[pin={[pin distance=-0.6\onedegree,Flaamsted]180:{13}}] at (axis cs:{212.072},{49.458}) {}; % Boo,  5.26 
\node[pin={[pin distance=-0.6\onedegree,Flaamsted]-90:{24}}] at (axis cs:{217.158},{49.845}) {}; % Boo,  5.58 
\node[pin={[pin distance=-0.6\onedegree,Flaamsted]00:{33}}] at (axis cs:{219.709},{44.404}) {}; % Boo,  5.40 
\node[pin={[pin distance=-0.6\onedegree,Flaamsted]-90:{38}}] at (axis cs:{222.328},{46.116}) {}; % Boo,  5.76 
\node[pin={[pin distance=-0.6\onedegree,Flaamsted]00:{39}}] at (axis cs:{222.422},{48.721}) {}; % Boo,  5.68 
\node[pin={[pin distance=-0.6\onedegree,Flaamsted]00:{40}}] at (axis cs:{224.904},{39.265}) {}; % Boo,  5.64 
\node[pin={[pin distance=-0.6\onedegree,Flaamsted]00:{44}}] at (axis cs:{225.947},{47.654}) {}; % Boo,  4.83 
\node[pin={[pin distance=-0.6\onedegree,Flaamsted]00:{47}}] at (axis cs:{226.358},{48.151}) {}; % Boo,  5.59 
\node[pin={[pin distance=-0.6\onedegree,Flaamsted]90:{50}}] at (axis cs:{230.452},{32.934}) {}; % Boo,  5.38 

\node[pin={[pin distance=-0.6\onedegree,Bayer]00:{$\boldsymbol\eta$}}] at (axis cs:{230.801},{30.288}) {}; % CrB,  4.997
\node[pin={[pin distance=-0.6\onedegree,Bayer]00:{$\boldsymbol\kappa$}}] at (axis cs:{237.808},{35.657}) {}; % CrB,  4.81 
\node[pin={[pin distance=-0.6\onedegree,Bayer]00:{$\boldsymbol\lambda$}}] at (axis cs:{238.948},{37.947}) {}; % CrB,  5.44 
\node[pin={[pin distance=-0.6\onedegree,Bayer]-90:{$\boldsymbol\mu$}}] at (axis cs:{233.812},{39.010}) {}; % CrB,  5.14 
\node[pin={[pin distance=-0.6\onedegree,Bayer]00:{$\boldsymbol\nu^1$}}] at (axis cs:{245.589},{33.799}) {}; % CrB,  5.21 
\node[pin={[pin distance=-0.6\onedegree,Bayer]00:{$\boldsymbol\pi$}}] at (axis cs:{235.997},{32.516}) {}; % CrB,  5.57 
\node[pin={[pin distance=-0.6\onedegree,Bayer]00:{$\boldsymbol\rho$}}] at (axis cs:{240.261},{33.303}) {}; % CrB,  5.40 
\node[pin={[pin distance=-0.6\onedegree,Bayer]00:{$\boldsymbol\sigma$}}] at (axis cs:{243.670},{33.858}) {}; % CrB,  5.54 
\node[pin={[pin distance=-0.6\onedegree,Bayer]00:{$\boldsymbol\tau$}}] at (axis cs:{242.243},{36.491}) {}; % CrB,  4.73 
\node[pin={[pin distance=-0.6\onedegree,Bayer]00:{$\boldsymbol\vartheta$}}] at (axis cs:{233.232},{31.359}) {}; % CrB,  4.16 
\node[pin={[pin distance=-0.6\onedegree,Bayer]00:{$\boldsymbol\xi$}}] at (axis cs:{245.524},{30.892}) {}; % CrB,  4.85 
\node[pin={[pin distance=-0.6\onedegree,Bayer]00:{$\boldsymbol\zeta^2$}}] at (axis cs:{234.844},{36.636}) {}; % CrB,  5.00 

\node[pin={[pin distance=-0.6\onedegree,Bayer]00:{$\boldsymbol\chi$}}] at (axis cs:{238.169},{42.451}) {}; % Her,  4.61 
\node[pin={[pin distance=-0.6\onedegree,Bayer]00:{$\boldsymbol\epsilon$}}] at (axis cs:{255.072},{30.926}) {}; % Her,  3.91 
\node[pin={[pin distance=-0.2\onedegree,Bayer]90:{$\boldsymbol\eta$}}] at (axis cs:{250.724},{38.922}) {}; % Her,  3.48 
\node[pin={[pin distance=-0.4\onedegree,Bayer]00:{$\boldsymbol\iota$}}] at (axis cs:{264.866},{46.006}) {}; % Her,  3.81 
\node[pin={[pin distance=-0.6\onedegree,Bayer]00:{$\boldsymbol\nu$}}] at (axis cs:{269.626},{30.189}) {}; % Her,  4.41 
\node[pin={[pin distance=-0.2\onedegree,Bayer]00:{$\boldsymbol\pi$}}] at (axis cs:{258.762},{36.809}) {}; % Her,  3.14 
\node[pin={[pin distance=-0.6\onedegree,Bayer]00:{$\boldsymbol\rho$}}] at (axis cs:{260.921},{37.146}) {}; % Her,  4.51 
\node[pin={[pin distance=-0.6\onedegree,Bayer]00:{$\boldsymbol\sigma$}}] at (axis cs:{248.526},{42.437}) {}; % Her,  4.21 
\node[pin={[pin distance=-0.6\onedegree,Bayer]00:{$\boldsymbol\tau$}}] at (axis cs:{244.935},{46.313}) {}; % Her,  3.90 
\node[pin={[pin distance=-0.6\onedegree,Bayer]00:{$\boldsymbol\upsilon$}}] at (axis cs:{240.700},{46.037}) {}; % Her,  4.72 
\node[pin={[pin distance=-0.6\onedegree,Bayer]00:{$\boldsymbol\varphi$}}] at (axis cs:{242.192},{44.935}) {}; % Her,  4.24 
\node[pin={[pin distance=-0.4\onedegree,Bayer]00:{$\boldsymbol\vartheta$}}] at (axis cs:{269.063},{37.251}) {}; % Her,  3.84 
\node[pin={[pin distance=-0.2\onedegree,Bayer]90:{$\boldsymbol\zeta$}}] at (axis cs:{250.322},{31.603}) {}; % Her,  2.85 
\node[pin={[pin distance=-0.6\onedegree,Flaamsted]00:{104}}] at (axis cs:{272.976},{31.405}) {}; % Her,  4.98 
\node[pin={[pin distance=-0.6\onedegree,Flaamsted]90:{25}}] at (axis cs:{246.351},{37.394}) {}; % Her,  5.54 
\node[pin={[pin distance=-0.6\onedegree,Flaamsted]00:{ 2}}] at (axis cs:{238.658},{43.138}) {}; % Her,  5.36 
\node[pin={[pin distance=-0.6\onedegree,Flaamsted]00:{30}}] at (axis cs:{247.161},{41.881}) {}; % Her,  4.90 
\node[pin={[pin distance=-0.6\onedegree,Flaamsted]00:{34}}] at (axis cs:{247.525},{48.961}) {}; % Her,  6.44 
\node[pin={[pin distance=-0.6\onedegree,Flaamsted]00:{42}}] at (axis cs:{249.687},{48.928}) {}; % Her,  4.88 
\node[pin={[pin distance=-0.6\onedegree,Flaamsted]180:{ 4}}] at (axis cs:{238.877},{42.566}) {}; % Her,  5.75 
\node[pin={[pin distance=-0.6\onedegree,Flaamsted]00:{52}}] at (axis cs:{252.309},{45.983}) {}; % Her,  4.82 
\node[pin={[pin distance=-0.6\onedegree,Flaamsted]00:{53}}] at (axis cs:{253.242},{31.702}) {}; % Her,  5.34 
\node[pin={[pin distance=-0.6\onedegree,Flaamsted]00:{59}}] at (axis cs:{255.402},{33.568}) {}; % Her,  5.28 
\node[pin={[pin distance=-0.6\onedegree,Flaamsted]00:{61}}] at (axis cs:{255.876},{35.414}) {}; % Her,  6.27 
\node[pin={[pin distance=-0.6\onedegree,Flaamsted]00:{68}}] at (axis cs:{259.332},{33.100}) {}; % Her,  4.81 
\node[pin={[pin distance=-0.6\onedegree,Flaamsted]00:{69}}] at (axis cs:{259.418},{37.291}) {}; % Her,  4.63 
\node[pin={[pin distance=-0.6\onedegree,Flaamsted]90:{72}}] at (axis cs:{260.165},{32.468}) {}; % Her,  5.39 
\node[pin={[pin distance=-0.6\onedegree,Flaamsted]90:{74}}] at (axis cs:{260.088},{46.241}) {}; % Her,  5.50 
\node[pin={[pin distance=-0.6\onedegree,Flaamsted]00:{77}}] at (axis cs:{261.684},{48.260}) {}; % Her,  5.84 
\node[pin={[pin distance=-0.6\onedegree,Flaamsted]00:{82}}] at (axis cs:{264.157},{48.586}) {}; % Her,  5.36 
\node[pin={[pin distance=-0.6\onedegree,Flaamsted]00:{90}}] at (axis cs:{268.325},{40.008}) {}; % Her,  5.17 
\node[pin={[pin distance=-0.6\onedegree,Flaamsted]00:{99}}] at (axis cs:{271.756},{30.562}) {}; % Her,  5.06 



\node[pin={[pin distance=-0.0\onedegree,Bayer]180:{$\boldsymbol\alpha$}}] at (axis cs:{279.235},{38.784}) {}; % Lyr,  0.03 
\node[pin={[pin distance=-0.4\onedegree,Bayer]00:{$\boldsymbol\beta$}}] at (axis cs:{282.520},{33.362}) {}; % Lyr,  3.52 
\node[pin={[pin distance=-0.6\onedegree,Bayer]00:{$\boldsymbol\delta^{1,2}$}}] at (axis cs:{283.431},{36.972}) {}; % Lyr,  5.58 
%\node[pin={[pin distance=-0.6\onedegree,Bayer]90:{$\boldsymbol\delta^2$}}] at (axis cs:{283.626},{36.898}) {}; % Lyr,  4.28 
\node[pin={[pin distance=-0.6\onedegree,Bayer]00:{$\boldsymbol\epsilon^1$}}] at (axis cs:{281.085},{39.670}) {}; % Lyr,  5.01 
\node[pin={[pin distance=-0.6\onedegree,Bayer]00:{$\boldsymbol\eta$}}] at (axis cs:{288.440},{39.146}) {}; % Lyr,  4.41 
\node[pin={[pin distance=-0.6\onedegree,Bayer]00:{$\boldsymbol\gamma$}}] at (axis cs:{284.736},{32.689}) {}; % Lyr,  3.25 
\node[pin={[pin distance=-0.6\onedegree,Bayer]00:{$\boldsymbol\iota$}}] at (axis cs:{286.826},{36.100}) {}; % Lyr,  5.26 
\node[pin={[pin distance=-0.6\onedegree,Bayer]00:{$\boldsymbol\kappa$}}] at (axis cs:{274.965},{36.064}) {}; % Lyr,  4.32 
\node[pin={[pin distance=-0.6\onedegree,Bayer]90:{$\boldsymbol\lambda$}}] at (axis cs:{285.003},{32.145}) {}; % Lyr,  4.94 
\node[pin={[pin distance=-0.4\onedegree,Bayer]90:{$\boldsymbol\mu$}}] at (axis cs:{276.057},{39.507}) {}; % Lyr,  5.12 
%\node[pin={[pin distance=-0.6\onedegree,Bayer]-90:{$\boldsymbol\nu^1$}}] at (axis cs:{282.441},{32.813}) {}; % Lyr,  5.93 
\node[pin={[pin distance=-0.6\onedegree,Bayer]-90:{$\boldsymbol\nu^{1,2}$}}] at (axis cs:{282.470},{32.551}) {}; % Lyr,  5.23 
\node[pin={[pin distance=-0.6\onedegree,Bayer]00:{$\boldsymbol\vartheta$}}] at (axis cs:{289.092},{38.134}) {}; % Lyr,  4.34 
\node[pin={[pin distance=-0.6\onedegree,Bayer]00:{$\boldsymbol\zeta^1$}}] at (axis cs:{281.193},{37.605}) {}; % Lyr,  4.34 
\node[pin={[pin distance=-0.4\onedegree,Flaamsted]00:{13}}] at (axis cs:{283.834},{43.946}) {}; % Lyr,  4.19 
\node[pin={[pin distance=-0.6\onedegree,Flaamsted]180:{16}}] at (axis cs:{285.360},{46.935}) {}; % Lyr,  5.01 
\node[pin={[pin distance=-0.6\onedegree,Flaamsted]00:{17}}] at (axis cs:{286.857},{32.502}) {}; % Lyr,  5.23 
\node[pin={[pin distance=-0.6\onedegree,Flaamsted]00:{19}}] at (axis cs:{287.942},{31.283}) {}; % Lyr,  5.94 
%%%%%%%%%%%%%%%%%%%%%%%%%%%%%%%%%%%%%%%%%%%%%%%%%%%%%%%%%%%%%%%%%%%%%%%%%%%%%%%%%%%%%
%%%%%%%%%%%%%%%%%%%%%%%%%%%%%%%%%%%%%%%%%%%%%%%%%%%%%%%%%%%%%%%%%%%%%%%%%%%%%%%%%%%%%
%%%%%%%%%%%%%%%%%%%%%%%%%%%%%%%%%%%%%%%%%%%%%%%%%%%%%%%%%%%%%%%%%%%%%%%%%%%%%%%%%%%%%

\node[pin={[pin distance=-.8\onedegree,Bayer]-45:{$\boldsymbol\delta$}}] at (axis cs:{296.244},{45.131}) {}; % Cyg,  2.91 
\node[pin={[pin distance=-0.6\onedegree,Bayer]00:{$\boldsymbol\iota^1$}}] at (axis cs:{291.858},{52.320}) {}; % Cyg,  5.73 
\node[pin={[pin distance=-0.4\onedegree,Bayer]00:{$\boldsymbol\iota^2$}}] at (axis cs:{292.426},{51.730}) {}; % Cyg,  3.77 
\node[pin={[pin distance=-0.6\onedegree,Bayer]00:{$\boldsymbol\psi$}}] at (axis cs:{298.907},{52.439}) {}; % Cyg,  4.91 
\node[pin={[pin distance=-0.5\onedegree,Bayer]-90:{$\boldsymbol\vartheta$}}] at (axis cs:{294.111},{50.221}) {}; % Cyg,  4.50 
\node[pin={[pin distance=-0.6\onedegree,Flaamsted]-90:{14}}] at (axis cs:{294.860},{42.818}) {}; % Cyg,  5.41 
\node[pin={[pin distance=-0.6\onedegree,Flaamsted]00:{20}}] at (axis cs:{297.657},{52.988}) {}; % Cyg,  5.02 
\node[pin={[pin distance=-0.6\onedegree,Flaamsted]90:{23}}] at (axis cs:{298.322},{57.523}) {}; % Cyg,  5.14 
\node[pin={[pin distance=-0.4\onedegree,Flaamsted]00:{33}}] at (axis cs:{303.349},{56.568}) {}; % Cyg,  4.28 

\node[pin={[pin distance=-0.3\onedegree,Bayer]00:{$\boldsymbol\gamma$}}] at (axis cs:{269.152},{51.489}) {}; % Dra,  2.23 
\node[pin={[pin distance=-0.4\onedegree,Bayer]180:{$\boldsymbol\alpha$}}] at (axis cs:{211.097},{64.376}) {}; % Dra,  3.65 
\node[pin={[pin distance=-0.3\onedegree,Bayer]00:{$\boldsymbol\beta$}}] at (axis cs:{262.608},{52.301}) {}; % Dra,  2.80 
\node[pin={[pin distance=-0.2\onedegree,Bayer]00:{$\boldsymbol\chi$}}] at (axis cs:{275.264},{72.733}) {}; % Dra,  3.57 
\node[pin={[pin distance=-0.2\onedegree,Bayer]00:{$\boldsymbol\delta$}}] at (axis cs:{288.139},{67.661}) {}; % Dra,  3.08 
\node[pin={[pin distance=-0.4\onedegree,Bayer]00:{$\boldsymbol\epsilon$}}] at (axis cs:{297.043},{70.268}) {}; % Dra,  3.84 
\node[pin={[pin distance=-0.2\onedegree,Bayer]00:{$\boldsymbol\eta$}}] at (axis cs:{245.998},{61.514}) {}; % Dra,  2.73 
\node[pin={[pin distance=-0.2\onedegree,Bayer]180:{$\boldsymbol\iota$}}] at (axis cs:{231.232},{58.966}) {}; % Dra,  3.30 
\node[pin={[pin distance=-0.4\onedegree,Bayer]00:{$\boldsymbol\kappa$}}] at (axis cs:{188.371},{69.788}) {}; % Dra,  3.89 
\node[pin={[pin distance=-0.4\onedegree,Bayer]-90:{$\boldsymbol\lambda$}}] at (axis cs:{172.851},{69.331}) {}; % Dra,  3.81 
\node[pin={[pin distance=-0.6\onedegree,Bayer]00:{$\boldsymbol\mu$}}] at (axis cs:{256.334},{54.470}) {}; % Dra,  5.55 
\node[pin={[pin distance=-0.6\onedegree,Bayer]90:{$\boldsymbol\nu^2$}}] at (axis cs:{263.067},{55.173}) {}; % Dra,  4.86 
\node[pin={[pin distance=-0.4\onedegree,Bayer]90:{$\boldsymbol\omega$}}] at (axis cs:{264.238},{68.758}) {}; % Dra,  4.78 
\node[pin={[pin distance=-0.4\onedegree,Bayer]180:{$\boldsymbol\omicron$}}] at (axis cs:{282.800},{59.388}) {}; % Dra,  4.64 
\node[pin={[pin distance=-0.6\onedegree,Bayer]00:{$\boldsymbol\pi$}}] at (axis cs:{290.167},{65.714}) {}; % Dra,  4.59 
\node[pin={[pin distance=-0.6\onedegree,Bayer]00:{$\boldsymbol\psi^1$}}] at (axis cs:{265.485},{72.149}) {}; % Dra,  4.57 
\node[pin={[pin distance=-0.6\onedegree,Bayer]00:{$\boldsymbol\psi^2$}}] at (axis cs:{268.796},{72.005}) {}; % Dra,  5.42 
\node[pin={[pin distance=-0.6\onedegree,Bayer]00:{$\boldsymbol\rho$}}] at (axis cs:{300.704},{67.873}) {}; % Dra,  4.51 
\node[pin={[pin distance=-0.6\onedegree,Bayer]90:{$\boldsymbol\sigma$}}] at (axis cs:{293.090},{69.661}) {}; % Dra,  4.67 
\node[pin={[pin distance=-0.5\onedegree,Bayer]180:{$\boldsymbol\tau$}}] at (axis cs:{288.888},{73.355}) {}; % Dra,  4.45 
\node[pin={[pin distance=-0.6\onedegree,Bayer]00:{$\boldsymbol\upsilon$}}] at (axis cs:{283.599},{71.297}) {}; % Dra,  4.82 
\node[pin={[pin distance=-0.6\onedegree,Bayer]00:{$\boldsymbol\varphi$}}] at (axis cs:{275.189},{71.338}) {}; % Dra,  4.23 
\node[pin={[pin distance=-0.4\onedegree,Bayer]90:{$\boldsymbol\vartheta$}}] at (axis cs:{240.472},{58.565}) {}; % Dra,  4.01 
\node[pin={[pin distance=-0.4\onedegree,Bayer]00:{$\boldsymbol\xi$}}] at (axis cs:{268.382},{56.873}) {}; % Dra,  3.74 
\node[pin={[pin distance=-0.6\onedegree,Bayer]00:{$\boldsymbol\zeta$}}] at (axis cs:{257.197},{65.714}) {}; % Dra,  3.18 
\node[pin={[pin distance=-0.6\onedegree,Flaamsted]00:{10}}] at (axis cs:{207.858},{64.723}) {}; % Dra,  4.61 
\node[pin={[pin distance=-0.6\onedegree,Flaamsted]00:{15}}] at (axis cs:{246.996},{68.768}) {}; % Dra,  4.97 
\node[pin={[pin distance=-0.6\onedegree,Flaamsted]00:{17}}] at (axis cs:{249.057},{52.924}) {}; % Dra,  5.08 
\node[pin={[pin distance=-0.6\onedegree,Flaamsted]90:{18}}] at (axis cs:{250.230},{64.589}) {}; % Dra,  4.84 
\node[pin={[pin distance=-0.6\onedegree,Flaamsted]00:{19}}] at (axis cs:{254.007},{65.135}) {}; % Dra,  4.88 
\node[pin={[pin distance=-0.6\onedegree,Flaamsted]-90:{20}}] at (axis cs:{254.105},{65.039}) {}; % Dra,  6.44 
\node[pin={[pin distance=-0.6\onedegree,Flaamsted]00:{26}}] at (axis cs:{263.748},{61.874}) {}; % Dra,  5.25 
\node[pin={[pin distance=-0.6\onedegree,Flaamsted]00:{27}}] at (axis cs:{262.991},{68.135}) {}; % Dra,  5.08 
\node[pin={[pin distance=-0.6\onedegree,Flaamsted]180:{ 2}}] at (axis cs:{174.012},{69.323}) {}; % Dra,  5.20 
\node[pin={[pin distance=-0.6\onedegree,Flaamsted]00:{30}}] at (axis cs:{267.268},{50.781}) {}; % Dra,  5.02 
\node[pin={[pin distance=-0.6\onedegree,Flaamsted]00:{35}}] at (axis cs:{267.363},{76.963}) {}; % Dra,  5.03 
\node[pin={[pin distance=-0.6\onedegree,Flaamsted]00:{36}}] at (axis cs:{273.474},{64.397}) {}; % Dra,  5.00 
\node[pin={[pin distance=-0.6\onedegree,Flaamsted]00:{37}}] at (axis cs:{273.821},{68.756}) {}; % Dra,  5.96 
\node[pin={[pin distance=-0.6\onedegree,Flaamsted]00:{39}}] at (axis cs:{275.978},{58.801}) {}; % Dra,  5.04 
\node[pin={[pin distance=-0.6\onedegree,Flaamsted]00:{ 3}}] at (axis cs:{175.618},{66.745}) {}; % Dra,  5.31 
\node[pin={[pin distance=-0.6\onedegree,Flaamsted]00:{41}}] at (axis cs:{270.038},{80.004}) {}; % Dra,  5.68 
\node[pin={[pin distance=-0.6\onedegree,Flaamsted]00:{42}}] at (axis cs:{276.496},{65.563}) {}; % Dra,  4.83 
\node[pin={[pin distance=-0.6\onedegree,Flaamsted]00:{45}}] at (axis cs:{278.144},{57.045}) {}; % Dra,  4.78 
\node[pin={[pin distance=-0.6\onedegree,Flaamsted]00:{46}}] at (axis cs:{280.658},{55.539}) {}; % Dra,  5.03 
\node[pin={[pin distance=-0.6\onedegree,Flaamsted]00:{48}}] at (axis cs:{284.188},{57.815}) {}; % Dra,  5.67 
\node[pin={[pin distance=-0.6\onedegree,Flaamsted]00:{49}}] at (axis cs:{285.181},{55.658}) {}; % Dra,  5.52 
\node[pin={[pin distance=-0.6\onedegree,Flaamsted]-90:{ 4}}] at (axis cs:{187.528},{69.201}) {}; % Dra,  5.03 
\node[pin={[pin distance=-0.6\onedegree,Flaamsted]00:{50}}] at (axis cs:{281.593},{75.434}) {}; % Dra,  5.37 
\node[pin={[pin distance=-0.6\onedegree,Flaamsted]00:{51}}] at (axis cs:{286.230},{53.396}) {}; % Dra,  5.41 
\node[pin={[pin distance=-0.6\onedegree,Flaamsted]90:{53}}] at (axis cs:{287.919},{56.859}) {}; % Dra,  5.14 
\node[pin={[pin distance=-0.6\onedegree,Flaamsted]00:{54}}] at (axis cs:{288.480},{57.705}) {}; % Dra,  5.00 
\node[pin={[pin distance=-0.6\onedegree,Flaamsted]00:{55}}] at (axis cs:{287.441},{65.978}) {}; % Dra,  6.27 
\node[pin={[pin distance=-0.6\onedegree,Flaamsted]00:{59}}] at (axis cs:{287.291},{76.560}) {}; % Dra,  5.12 
\node[pin={[pin distance=-0.6\onedegree,Flaamsted]00:{64}}] at (axis cs:{300.369},{64.821}) {}; % Dra,  5.23 
\node[pin={[pin distance=-0.6\onedegree,Flaamsted]90:{65}}] at (axis cs:{300.584},{64.634}) {}; % Dra,  6.26 
\node[pin={[pin distance=-0.6\onedegree,Flaamsted]00:{66}}] at (axis cs:{301.387},{61.995}) {}; % Dra,  5.41 
\node[pin={[pin distance=-0.6\onedegree,Flaamsted]00:{68}}] at (axis cs:{302.895},{62.078}) {}; % Dra,  5.71 
\node[pin={[pin distance=-0.6\onedegree,Flaamsted]90:{ 6}}] at (axis cs:{188.683},{70.022}) {}; % Dra,  4.95 
\node[pin={[pin distance=-0.6\onedegree,Flaamsted]00:{71}}] at (axis cs:{304.903},{62.257}) {}; % Dra,  5.72 
\node[pin={[pin distance=-0.6\onedegree,Flaamsted]180:{73}}] at (axis cs:{307.877},{74.955}) {}; % Dra,  5.19 
\node[pin={[pin distance=-0.6\onedegree,Flaamsted]00:{74}}] at (axis cs:{307.365},{81.091}) {}; % Dra,  5.97 
\node[pin={[pin distance=-0.6\onedegree,Flaamsted]-90:{75}}] at (axis cs:{307.061},{81.423}) {}; % Dra,  5.37 
\node[pin={[pin distance=-0.6\onedegree,Flaamsted]-90:{76}}] at (axis cs:{310.647},{82.531}) {}; % Dra,  5.75 
\node[pin={[pin distance=-0.6\onedegree,Flaamsted]00:{ 7}}] at (axis cs:{191.893},{66.790}) {}; % Dra,  5.43 
\node[pin={[pin distance=-0.6\onedegree,Flaamsted]00:{ 8}}] at (axis cs:{193.869},{65.438}) {}; % Dra,  5.23 
\node[pin={[pin distance=-0.6\onedegree,Flaamsted]00:{ 9}}] at (axis cs:{194.979},{66.597}) {}; % Dra,  5.37 


\node[pin={[pin distance=-0.2\onedegree,Bayer]180:{$\boldsymbol\alpha$}}] at (axis cs:{37.955},{89.264}) {}; % UMi,  2.00 
\node[pin={[pin distance=-0.3\onedegree,Bayer]180:{$\boldsymbol\beta$}}] at (axis cs:{222.676},{74.155}) {}; % UMi,  2.06 
\node[pin={[pin distance=-0.4\onedegree,Bayer]-90:{$\boldsymbol\delta$}}] at (axis cs:{263.054},{86.586}) {}; % UMi,  4.35 
\node[pin={[pin distance=-0.4\onedegree,Bayer]00:{$\boldsymbol\epsilon$}}] at (axis cs:{251.493},{82.037}) {}; % UMi,  4.22 
\node[pin={[pin distance=-0.4\onedegree,Bayer]90:{$\boldsymbol\eta$}}] at (axis cs:{244.376},{75.755}) {}; % UMi,  4.96 
\node[pin={[pin distance=-0.4\onedegree,Bayer]00:{$\boldsymbol\gamma$}}] at (axis cs:{230.182},{71.834}) {}; % UMi,  3.03 
\node[pin={[pin distance=-0.6\onedegree,Bayer]180:{$\boldsymbol\lambda$}}] at (axis cs:{259.235},{89.038}) {}; % UMi,  6.34 
\node[pin={[pin distance=-0.6\onedegree,Bayer]00:{$\boldsymbol\vartheta$}}] at (axis cs:{232.854},{77.349}) {}; % UMi,  4.99 
\node[pin={[pin distance=-0.6\onedegree,Bayer]00:{$\boldsymbol\zeta$}}] at (axis cs:{236.015},{77.794}) {}; % UMi,  4.29 
\node[pin={[pin distance=-0.6\onedegree,Flaamsted]-90:{11}}] at (axis cs:{229.275},{71.824}) {}; % UMi,  5.01 
\node[pin={[pin distance=-0.6\onedegree,Flaamsted]-90:{19}}] at (axis cs:{242.706},{75.877}) {}; % UMi,  5.48 
\node[pin={[pin distance=-0.6\onedegree,Flaamsted]-90:{20}}] at (axis cs:{243.134},{75.211}) {}; % UMi,  6.35 
\node[pin={[pin distance=-0.6\onedegree,Flaamsted]00:{24}}] at (axis cs:{262.698},{86.968}) {}; % UMi,  5.77 
\node[pin={[pin distance=-0.6\onedegree,Flaamsted]180:{ 3}}] at (axis cs:{211.735},{74.594}) {}; % UMi,  6.43 
\node[pin={[pin distance=-0.6\onedegree,Flaamsted]180:{ 4}}] at (axis cs:{212.212},{77.547}) {}; % UMi,  4.80 
\node[pin={[pin distance=-0.4\onedegree,Flaamsted]00:{ 5}}] at (axis cs:{216.881},{75.696}) {}; % UMi,  4.24 
\end{polaraxis}


%
\end{tikzpicture}

\end{document}
